%This work is licensed under the Creative Commons License Attribution 4.0 International (CC-BY 4.0)
%https://creativecommons.org/licenses/by/4.0/legalcode
\documentclass[rgb]{standalone}
\usepackage{tkz-euclide}
\usepackage{amsmath}
\definecolor{myorange}{hsb}{0.0833, 1, 0.8}
\definecolor{mygreen}{hsb}{0.3333, 1, 0.8}
\definecolor{myblue}{hsb}{0.5833, 1, 0.8}
\definecolor{mymagenta}{hsb}{0.8333, 1, 0.8}
\begin{document}
	\begin{tikzpicture}[scale=1.5, font=\Large]
	    \draw[ultra thick, -latex] (7.5,10.25) -- (7.5,9.25);
	    \draw[ultra thick, -latex] (7.5,8.25) -- (7.5,7.25);
	    \draw[ultra thick, -latex] (7.5,6.25) -- (7.5,5.25);
	    \draw[ultra thick, -latex] (7.5,4.25) -- (7.5,3.25);
	    \draw[ultra thick, -latex] (7.5,4.25) -- (1.25,3.25);
	    \draw[ultra thick, -latex] (7.5,1.25) -- (0,0.25);
	    \draw[ultra thick, -latex] (7.5,1.25) -- (5,0.25);
	    \draw[ultra thick, -latex] (7.5,1.25) -- (10,0.25);
	    \draw[ultra thick, -latex] (7.5,1.25) -- (15,0.25);
	    \draw[thick, rounded corners, fill=myblue, fill opacity=0.1] (3,10.25) rectangle ++(9,1);
	    \node[anchor=center] at (7.5,10.75) {Stammfunktion einer rationalen Funktion $R(x)$ bestimmen};
	    \draw[thick, rounded corners, fill, fill opacity=0.1] (2.25,8.25) rectangle ++(10.5,1);
	    \node[anchor=center] at (7.5,9) {Polynomdivision $R(x)=p(x)+r(x)$,};
	    \node[anchor=center] at (7.5,8.5) {wobei $p$ ein Polynom und $r$ eine echt gebrochen rationale Funktion sind};
	    \draw[thick, rounded corners, fill, fill opacity=0.1] (4.5,6.25) rectangle ++(6,1);
	    \node[anchor=center] at (7.5,6.75) {Nenner von $r$ vollst\"andig faktorisieren};
	    \draw[thick, rounded corners, fill, fill opacity=0.1] (5.5,4.25) rectangle ++(4,1);
	    \node[anchor=center] at (7.5,4.75) {Partialbruchzerlegung};
	    \draw[thick, rounded corners, fill, fill opacity=0.1] (4,1.25) rectangle ++(7,2);
	    \draw[thick, rounded corners, fill, fill opacity=0.1] (-1,1.25) rectangle ++(4.5,2);
	    \node[anchor=center] at (1.25,3) {Summanden vom Typ};
	    \node[anchor=center] at (1.25,2.25) {$\dfrac{1}{(x-a)^k}$};
	    \node[anchor=center] at (1.25,1.5) {Regel 3.1 anwenden};
	    \node[anchor=center] at (7.5,3) {Summanden vom Typ};
	    \node[anchor=center] at (7.5,2.25) {$\dfrac{rx+s}{(x^2+bx+c)^n}$};
	    \node[anchor=center] at (7.5,1.5) {Wie in Bemerkung 3.5 beschrieben aufspalten};
		\draw[thick, rounded corners, fill, fill opacity=0.1] (-2.25,-1.75) rectangle ++(4.5,2);
		\draw[thick, rounded corners, fill, fill opacity=0.1] (2.75,-1.75) rectangle ++(4.5,2);
		\draw[thick, rounded corners, fill, fill opacity=0.1] (7.75,-1.75) rectangle ++(4.5,2);
		\draw[thick, rounded corners, fill, fill opacity=0.1] (12.75,-1.75) rectangle ++(4.5,2);
		\node[anchor=center] at (0,0) {Summanden vom Typ};
		\node[anchor=center] at (0,-0.75) {$\dfrac{2x+b}{x^2+bx+c}$};
		\node[anchor=center] at (0,-1.5) {Theorem 3.2 Fall 1};
		\node[anchor=center] at (5,0) {Summanden vom Typ};
		\node[anchor=center] at (5,-0.75) {$\dfrac{1}{x^2+bx+c}$};
		\node[anchor=center] at (5,-1.5) {Theorem 3.2 Fall 2.1};
		\node[anchor=center] at (10,0) {Summanden vom Typ};
		\node[anchor=center] at (10,-0.75) {$\dfrac{2x+b}{(x^2+bx+c)^n},\quad n>1$};
		\node[anchor=center] at (10,-1.5) {Bemerkung 3.5};
		\node[anchor=center] at (15,0) {Summanden vom Typ};
		\node[anchor=center] at (15,-0.75) {$\dfrac{1}{(x^2+bx+c)^n},\quad n>1$};
		\node[anchor=center] at (15,-1.5) {Komplizierte Methoden n\"otig};
	\end{tikzpicture}
\end{document}