%$Id:  $
\documentclass{mumie.article}
%$Id$
\begin{metainfo}
  \name{
    \lang{de}{Gradmaß und Bogenmaß}
    \lang{en}{Degrees and radians}
  }
  \begin{description} 
 This work is licensed under the Creative Commons License Attribution 4.0 International (CC-BY 4.0)   
 https://creativecommons.org/licenses/by/4.0/legalcode 

    \lang{de}{Beschreibung}
    \lang{en}{Description}
  \end{description}
  \begin{components}
    \component{generic_image}{content/rwth/HM1/images/g_img_00_Videobutton_schwarz.meta.xml}{00_Videobutton_schwarz}
    \component{generic_image}{content/rwth/HM1/images/g_img_00_video_button_schwarz-blau.meta.xml}{00_video_button_schwarz-blau}
    \component{js_lib}{system/media/mathlets/GWTGenericVisualization.meta.xml}{mathlet1}
  \end{components}
  \begin{links}
    \link{generic_article}{content/rwth/HM1/T105_Trigonometrische_Funktionen/g_art_content_17_trigonometrie_im_dreieck.meta.xml}{content_17_trigonometrie_im_dreieck}
    \link{generic_problem}{content/rwth/HM1/T103_Polynomfunktionen/training/g_prb_problem7.meta.xml}{problem7}
    \link{generic_article}{content/rwth/HM1/T104_weitere_elementare_Funktionen/g_art_content_14_potenzregeln.meta.xml}{power-rules}
  \end{links}
  \creategeneric
\end{metainfo}
\begin{content}
\usepackage{mumie.ombplus}
\ombchapter{5}
\ombarticle{2}
\usepackage{mumie.genericvisualization}

\begin{visualizationwrapper}

\title{\lang{de}{Gradmaß und Bogenmaß}\lang{en}{Degrees and radians}}
 
\begin{block}[annotation]
  übungsinhalt  
\end{block}

\begin{block}[annotation]
  Im Ticket-System: \href{http://team.mumie.net/issues/9029}{Ticket 9029}\\
\end{block}

\begin{block}[info-box]
\tableofcontents
\end{block}

\section{\lang{de}{Gradma"s und Bogenma"s}\lang{en}{Degrees and radians}}\label{winkelmasse}
\lang{de}{
Um die Gr"o"se eines Winkels anzugeben, sind zwei verschiedene Ma"seinheiten
gebr"auchlich: Das \textit{Gradma"s} (gekennzeichnet durch ein hochgestelltes
Gradzeichen, z.B. $90^\circ$) und das
\textit{Bogenma"s} (ohne besondere Kennzeichnung). Festgelegt ist, dass der
Vollwinkel im Gradma"s $360^\circ$ betr"agt und im Bogenma"s $2\pi$, was
ungef"ahr $6,283$ ist.
}
\lang{en}{
There are two customary units used to state the magnitude of an angle: \textit{degrees} (denoted via 
the small degree circle, e.g. $90^\circ$) and \textit{radians} (with no special symbol). We define 
the perigon or full angle to have $360^\circ$ or $2\pi \approx 6.283$ radians.
}

\begin{tabs*}[\initialtab{0}]
\tab{\lang{de}{Erl"auterung zu dieser Festlegung}\lang{en}{Remark}}
%\begin{incremental}
%\step
\lang{de}{
Das Gradma"s $360^\circ$ f"ur den Vollwinkel hat historische Gr"unde und geht auf
die Babylonier zur"uck, das Bogenma"s von $2\pi$ ist der Umfang eines Kreises mit Radius $1$.
Daher kommt auch der Name "`Bogenma"s"'.
}
\lang{en}{
The degree measurement of $360^\circ$ for a full angle has its roots in the Babylonian Empire, and 
the radian measurement of $2\pi$ is the circumference of a circle with radius $1$. The unit is sometimes written "rad", i.e. $2\pi\text{ rad} = 360^\circ$; since the radian measurement is a 
ratio of the arc length to the radius, it is in fact a unitless quantity and the "rad" is normally omitted.
}
%\end{incremental}
\end{tabs*}
\lang{de}{Die Ma"se der anderen Winkel ergeben sich dann als Anteile aus dem Vollwinkel.}
\lang{en}{The measurement of other angles can be written as ratios of the full angle.}
% Insbesondere betr"agt die Gr"o"se eines 
% gestreckten Winkels \mbox{$\frac{1}{2}\cdot 360^\circ=180^\circ$} bzw.
% $\frac{1}{2}\cdot 2\pi=\pi$, und die Gr"o"se 
% eines rechten Winkels betr"agt \mbox{$\frac{1}{2}\cdot 180^\circ=90^\circ$} bzw.
% $\frac{1}{2}\cdot \pi=\frac{\pi}{2}$.\\

\begin{definition}[\lang{de}{Winkelbezeichnungen}\lang{en}{Names of angles}]

\lang{de}{
\begin{table}
\head
Winkel & im Gradma"s & im Bogenma"s \\
  \body
\textbf{Vollwinkel} & $360^\circ$ & $2\pi$ \\
\textbf{"uberstumpfer Winkel} & zwischen $180^\circ$ und $360^\circ$ & zwischen $\pi$ und $2\pi$ \\
\textbf{gestreckter Winkel} & $180^\circ$ & $\pi$ \\
\textbf{stumpfer Winkel} & zwischen $90^\circ$ und $180^\circ$ & zwischen  $\frac{\pi}{2}$ und $\pi$ \\
\textbf{rechter Winkel} & $90^\circ$ & $\frac{\pi}{2}$ \\
\textbf{spitzer Winkel} & zwischen $0^\circ$ und $90^\circ$ & zwischen $0$ und $\frac{\pi}{2}$
\end{table}
}
\lang{en}{
\begin{table}
\head 
Angle & in degrees & in radians \\
  \body
full angle & $360^\circ$ & $2\pi$ \\
reflex angle & between $180^\circ$ and $360^\circ$ & between $\pi$ and $2\pi$ \\
straight angle & $180^\circ$ & $\pi$ \\
obtuse angle & between $90^\circ$ and $180^\circ$ & between $\frac{\pi}{2}$ and $\pi$ \\
right angle & $90^\circ$ & $\frac{\pi}{2}$ \\
acute angle & between $0^\circ$ and $90^\circ$ & between $0$ and $\frac{\pi}{2}$
\end{table}
}

\end{definition}


%\begin{example}
%\begin{enumerate}
%\item \lang{de}{Wenn man einen gestreckten Winkel in drei gleich gro"se Teile teilt, erh"alt
%	man Winkel, deren Gr"o"sen $\frac{1}{3}$ der Gr"o"se des gestreckten Winkels
%	sind, also $\frac{1}{3}\cdot 180^\circ=60^\circ$ bzw. im Bogenma"s 
%	\mbox{$\frac{1}{3}\cdot \pi=\frac{\pi}{3}\approx 1,047$. } }
%	\lang{en}{If we divide a straight angle into three equal-sized parts, we get angles whose sizes are $\frac{1}{3}$ the size of a straight angle, i.e.
%	$\frac{1}{3}\cdot 180^\circ=60^\circ$ or  
%	\mbox{$\frac{1}{3}\cdot \pi=\frac{\pi}{3}\approx 1.047$ radians. } }
%\item \lang{de}{Wird der rechte Winkel halbiert, erh"alt man einen Winkel der Gr"o"se
%	$\frac{1}{2}\cdot 90^\circ=45^\circ$ bzw. im Bogenma"s  \mbox{$\frac{1}{2}\cdot 
%	\frac{\pi}{2}=\frac{\pi}{4}\approx 0,785$. } }
%	\lang{en}{If a right angle is halved, we get an angle of
%	$\frac{1}{2}\cdot 90^\circ=45^\circ$ or  \mbox{$\frac{1}{2}\cdot 
%	\frac{\pi}{2}=\frac{\pi}{4}\approx 0.785$ radians. } }
%\end{enumerate}
%\end{example}

\lang{de}{
Zur Umrechnung von Gradma"s in Bogenma"s und umgekehrt bezieht man sich immer
auf den Vollwinkel. Ein Winkel 
von zum Beispiel $36^\circ$ hat also einen Anteil von $\frac{36}{360}$ am
Vollwinkel, und betr"agt daher im Bogenma"s 
\mbox{$\frac{36}{360}\cdot 2\pi=\frac{2\pi}{10}=\frac{\pi}{5}$.} Allgemein erh"alt man
die folgenden Umrechnungsformeln:
}
\lang{en}{
In order to easily convert an angle from degrees to radians and vice versa, we can relate the angle 
to a full angle: an angle of $36^\circ$ represents the fraction $\frac{36}{360}$ of a full angle, 
which in radians is \mbox{$\frac{36}{360}\cdot 2\pi=\frac{2\pi}{10}=\frac{\pi}{5}$}. In general, we 
get the following transformation formulas:
}
\begin{rule}[\lang{de}{Umrechnung zwischen Grad- und Bogenmaß}
             \lang{en}{Changing between degrees and radians}]\label{umrechnung-gradmass-bogenmass}
  \lang{de}{
  \[ \text{Bogenma"s des Winkels} = \frac{\text{Gradma"s des
  Winkels}}{360^\circ}\cdot 2\pi = \frac{\text{Gradma"s des
  Winkels}}{180^\circ}\cdot \pi \] 
%\end{rule}
und umkehrt
%\begin{rule}
    \[ \text{Gradma"s des Winkels} = \frac{\text{Bogenma"s des Winkels}}{2\pi}\cdot
  360^\circ= \frac{\text{Bogenma"s des Winkels}}{\pi}\cdot
  180^\circ \]
  }
  \lang{en}{
  We have
  \[ \text{angle in radians} = \frac{\text{angle in degrees}}{360^\circ}\cdot 2\pi = \frac{\text{angle in degrees}}{180^\circ}\cdot \pi \] 
%\end{rule}
and similarly,
%\begin{rule}
    \[ \text{angle in degrees} = \frac{\text{angle in radians}}{2\pi}\cdot
  360^\circ= \frac{\text{angle in radians}}{\pi}\cdot
  180^\circ \]
  }  
\end{rule}

\begin{quickcheck}
  \type{input.number}
  \displayprecision{3}
  \correctorprecision{3}
  \begin{variables}
    \randint{num}{1}{6}
    \function[calculate]{deg}{num/6*180}
    \function[calculate]{rad}{deg/180*pi}
  \end{variables}
  \text{\lang{de}{
        Wie groß ist der Winkel $\var{deg}^\circ$ im Bogenmaß?
        Runden Sie Ihr Ergebnis auf drei Stellen hinter dem Komma.
        \\\\
        Antwort: \ansref
        }
        \lang{en}{
        Convert the angle $\var{deg}^\circ$ into radians. Round the result to three decimal points.
        \\\\
        Result: \ansref
        }
  }
  \explanation{\lang{de}{
               Bogenmaß = Gradmaß / 180 $\cdot$ $\pi$, also $\var{deg}/180\cdot\pi$, also, da 
               $\pi = 3.14159\ldots$, Bogenmaß gerundet auf drei Stellen = $\var{rad}$.
               }
               \lang{en}{
               Angle in radians = angle in degrees $/ 180$ $\cdot$ $\pi$, so 
               $\var{deg}/180\cdot\pi$, and as $\pi = 3.14159\ldots$, the angle in radians up to 
               three decimal points is $\var{rad}$.
               }
  }
  \begin{answer}
    \solution{rad}
  \end{answer}
\end{quickcheck}
\lang{de}{
Eine kurze Zusammenfassung der hier angesprochenen Inhalte, samt Beispielen, kann den folgenden Videos entnommen werden:\\
\floatright{
    \href{https://api.stream24.net/vod/getVideo.php?id=10962-2-10750&mode=iframe&speed=true}{\image[75]{00_video_button_schwarz-blau}}
    \href{https://www.hm-kompakt.de/video?watch=128}{\image[75]{00_Videobutton_schwarz}}
    }
}
\lang{en}{
\\
}
\end{visualizationwrapper}


\end{content}