%$Id:  $
\documentclass{mumie.article}
%$Id$
\begin{metainfo}
  \name{
    \lang{de}{Trigonometrie im Dreieck}
    \lang{en}{Introduction to trigonometry}
  }
  \begin{description} 
 This work is licensed under the Creative Commons License Attribution 4.0 International (CC-BY 4.0)   
 https://creativecommons.org/licenses/by/4.0/legalcode 

    \lang{de}{Beschreibung}
    \lang{en}{Description}
  \end{description}
  \begin{components}
    \component{generic_image}{content/rwth/HM1/images/g_tkz_T105_PythagorasProof.meta.xml}{T105_PythagorasProof}
    \component{generic_image}{content/rwth/HM1/images/g_tkz_T105_Triangle_E.meta.xml}{T105_Triangle_E}
    \component{generic_image}{content/rwth/HM1/images/g_tkz_T105_Triangle_D.meta.xml}{T105_Triangle_D}
    \component{generic_image}{content/rwth/HM1/images/g_tkz_T105_Triangle_C.meta.xml}{T105_Triangle_C}
    \component{generic_image}{content/rwth/HM1/images/g_tkz_T105_Triangle_B.meta.xml}{T105_Triangle_B}
    \component{generic_image}{content/rwth/HM1/images/g_tkz_T105_Pythagoras.meta.xml}{T105_Pythagoras}
    \component{generic_image}{content/rwth/HM1/images/g_tkz_T105_Triangle_A.meta.xml}{T105_Triangle_A}
    \component{generic_image}{content/rwth/HM1/images/g_img_00_Videobutton_schwarz.meta.xml}{00_Videobutton_schwarz}
    \component{generic_image}{content/rwth/HM1/images/g_img_00_video_button_schwarz-blau.meta.xml}{00_video_button_schwarz-blau}
    \component{js_lib}{system/media/mathlets/GWTGenericVisualization.meta.xml}{mathlet1}
  \end{components}
  \begin{links}
    \link{generic_article}{content/rwth/HM1/T105_Trigonometrische_Funktionen/g_art_content_18_grad_und_bogenmass.meta.xml}{link-gradmass}
    \link{generic_article}{content/rwth/HM1/T105_Trigonometrische_Funktionen/g_art_content_19_allgemeiner_sinus_cosinus.meta.xml}{link-sinus-reell}
    \link{generic_article}{content/rwth/HM1/T209_Potenzreihen/g_art_content_28_exponentialreihe.meta.xml}{trig-reihe}
  \end{links}
  \creategeneric
\end{metainfo}
\begin{content}
\usepackage{mumie.ombplus}
\ombchapter{5}
\ombarticle{1}
\usepackage{mumie.genericvisualization}

\begin{visualizationwrapper}

\title{\lang{de}{Trigonometrie im Dreieck}\lang{en}{Introduction to trigonometry}}
 
\begin{block}[annotation]
  übungsinhalt
  
\end{block}
\begin{block}[annotation]
  Im Ticket-System: \href{http://team.mumie.net/issues/9028}{Ticket 9028}\\
\end{block}

\begin{block}[info-box]
\tableofcontents
\end{block}

\newcommand{\grad}{\circ}

\section{\lang{de}{Einführung und Motivation}\lang{en}{Introduction and motivation}}
\lang{de}{
Typische trigonometrische Funktionen sind die aus der Schule bekannten Abbildungen Sinus, Kosinus und Tangens. 
Diese Funktionen können unter anderem herangezogen werden, um Winkel und Längen in Dreiecken ins Verhältnis zu 
setzen (vgl. Bedeutung des Wortes \emph{Trigonometrie} mit Herkunft aus dem Griechischen: trígonon ‚Dreieck‘ und métron ‚Maß‘). 
Zusätzlich zu diesen grundlegenden Zusammenhängen finden die trigonometrischen Funktionen häufig Anwendung in 
einer Vielzahl von wissenschaftlichen Fragestellungen, beispielsweise in der Konstruktion (Bestimmung von Winkeln 
und Längen), der Mechanik (Analyse von schwingfähigen Systemen) oder aber auch in der Elektrotechnik (Verwendung 
des Wechselstroms).\\
Im Rahmen des folgenden Kapitels wollen wir uns die Grundlagen der trigonometrischen Funktionen erarbeiten und erste Beispielprobleme lösen.
}
\lang{en}{
The main trigonometric functions are the sine, cosine and tangent functions familiar from school. 
These functions can be, amongst other uses, applied to relate the angles and side lengths of a 
triangle (the word \emph{trigonometry} comes from the Greek words trigonom 'triangle' and métron 
'measure'). Trigonometric functions appear in many applications in various fields, particularly in 
construction (for finding angles and lengths), mechanics (analysing oscillatory motion) and 
electronics (for AC power and signal processing). In this chapter we lay down the basic definitions 
and rules of trigenometry and solve some example problems.
}

\section{\lang{de}{Bezeichnungen im rechtwinkligen Dreieck und Satz des Pythagoras}
         \lang{en}{Notation and the Pythagorean theorem}}\label{bezeichnungen}
\lang{de}{
  Wir betrachten ein rechtwinkliges Dreieck $ABC$ mit
  rechtem Winkel bei $C$, und verwenden die in der folgenden Abbildung
  angegebenen Notationen.
  }
  \lang{en}{
  Throughout this section we will refer to the right-angled triangle $ABC$ with a right angle
  $\gamma$ at $C$, and will use the notation shown in the following image.
  }
\begin{figure}
 \image{T105_Triangle_A}
\caption{\lang{de}{Notation im rechtwinkligen Dreieck}
         \lang{en}{Notation for right-angled triangles}}
\end{figure}

\begin{definition}[\lang{de}{Bezeichnungen im rechtwinkligen Dreieck}
                   \lang{en}{Definitions for right-angled triangles}]\label{def:rechtw_Dreieck}
\lang{de}{Im rechtwinkligen Dreieck haben die Seiten besondere Namen:}
\lang{en}{The sides of a right-angled triangle have special names:}\\
\begin{itemize}
\item \lang{de}{
      Die \textbf{Hypotenuse} ist die dem rechten Winkel gegen"uberliegende Seite.
      }
      \lang{en}{
      the \textit{hypotenuse} is the side opposite the right angle,
      }
\item \lang{de}{
      Die \textbf{Katheten} sind die am rechten Winkel anliegenden Seiten.
      }
      \lang{en}{
      the \textit{legs} (or \textit{catheti}, singular: \textit{cathetus}) are the sides adjacent to 
      the right angle,
      }
\item \lang{de}{
      Die \textbf{Ankathete} von $\alpha$ ist die am Winkel $\alpha$ anliegende Kathete (also die 
      Seite $b$), entsprechend ist die Ankathete von $\beta$ die dem Winkel $\beta$ anliegende 
      Kathete (also die Seite $a$).
      }
      \lang{en}{
      side $a$ is \textit{opposite} the angle $\alpha$ and \textit{adjacent} to the angle $\beta$, 
      and similarly,
      }
\item \lang{de}{
      die \textbf{Gegenkathete} von $\alpha$ ist dem Winkel $\alpha$ gegen"uberliegende Kathete 
      (also die Seite $a$), entsprechend ist die Gegenkathete von $\beta$ die dem Winkel $\beta$ 
      gegen"uberliegende Kathete (also die Seite $b$).
      }
      \lang{en}{
      side $b$ is \textit{opposite} the angle $\beta$ and \textit{adjacent} to the angle $\alpha$.
      }
\end{itemize}
\end{definition}

 
\lang{de}{
Die Ankathete von $\beta$ ist also zugleich die Gegenkathete von $\alpha$, und die Gegenkathete von 
$\beta$ ist die Ankathete von $\alpha$.
\\\\
Ein wichtiger Satz, der die Seitenlängen eines rechtwinkligen Dreiecks miteinander in Verbindung bringt,
ist der Satz des Pythagoras.
}
\lang{en}{
The Pythaogrean theorem is important because it states a relationship between the side lengths of 
a right-angled triangle.
}

\label{satz-des-pythagoras}
\begin{theorem}[\lang{de}{Satz des Pythagoras}\lang{en}{Pythagorean theorem}]
%  \textbf{Satz des Pythagoras:}\\
  \lang{de}{
  Im rechtwinkligen Dreieck ist die Summe der Quadrate "uber den Katheten gleich
  dem Quadrat "uber der Hypotenuse.
  \\\\
  Mit den obigen Bezeichnungen gilt also
  }
  \lang{en}{
  In a right-angled triangle, the sum of the squares of both legs is equal to the square of the 
  hypotenuse.
  \\\\
  With the above notation,
  }
  \[ a^2+b^2=c^2.\]
\end{theorem}
\begin{figure}
  \image{T105_Pythagoras}
\caption{\lang{de}{Satz des Pythagoras}\lang{en}{The Pythagorean theorem}}
\end{figure}

% \begin{proof}[Beweis]
% \begin{showhide}
%  \lang{de}{Eine Möglichkeit, dies einzusehen, ist durch Betrachtung der folgenden Abbildung.}
%  \lang{en}{The simplest way to visualize this is by looking at the following image.}
%  \begin{figure} \image{T105_Pythagoras} \end{figure}
%  \lang{de}{Hier wird auf jede Seite eines Quadrats $PQRS$ der Seitenlänge $c$ das gegebene rechtwinklige Dreieck aufgesetzt.
%  Da das Dreieck rechtwinklig ist, bilden die Strecken der Länge $a$ und $b$, die sich bei einem der Eckpunkte $P$, $Q$, $R$ oder $S$ treffen,
%  eine Seite des roten Quadrats, da die Innnenwinkelsumme im Dreieck stets $180^\circ$ beträgt. Für den vorliegenden
%  Fall gilt foglich $\beta+90^\circ+\alpha=180^\circ$. Die gesamt Figur ist also ein Quadrat mit
%  Seitenlänge $a+b$. Da der Flächeninhalt eines Quadrats gerade das Quadrat einer Seitenlänge ist und der Flächeninhalt der rechtwinkligen Dreiecke
%  gerade $\frac{1}{2}ab$, ist daher $(a+b)^2$
%  (=Flächeninhalt des großen Quadrats) gleich groß wie $c^2+4\cdot (\frac{1}{2}ab)$ (=Flächeninhalt des kleinen 
%  Quadrats und der vier rechtwinkligen Dreiecke).\\
%  Aufgrund der ersten binomischen Formel erhält man noch $(a+b)^2=a^2+2ab+b^2$, und daher
%  \[   a^2+2ab+b^2= c^2+4\cdot \left(\frac{1}{2}ab\right)=c^2+2ab. \]
%  Subtraktion von $2ab$ auf beiden Seiten ergibt schließlich
%  \[ a^2+b^2= c^2.\]
%  }
% \end{showhide}
% \end{proof}
%

\begin{proof}[\lang{de}{Beweis}\lang{en}{Proof}]
 \begin{showhide}
  \lang{de}{Eine Möglichkeit, dies einzusehen, ist durch Betrachtung der folgenden Abbildung.}
  \lang{en}{The simplest way to visualize this is by looking at the following image.}
  \begin{figure} 
  \image{T105_PythagorasProof}
  \end{figure}
  \lang{de}{
  Hier wird auf jede Seite des (schwarzen) Quadrats $PQRS$ der Seitenlänge $c$ das gegebene rechtwinklige Dreieck 
  (in orange) aufgesetzt. Da das Dreieck rechtwinklig ist, bilden die Katheten $a$ und $b$ jeweils zweier benachbarter Dreiecke, 
  die sich in einem der Eckpunkte P, Q, R oder S treffen, eine Strecke der Länge $a+b$. Dies resultiert
  aus der Tatsache, dass die Summe der Innenwinkel eines Dreicks stets $180^\circ$ beträgt, im vorliegenden Fall also 
  $\alpha + 90^\circ + \beta=180^\circ$ ist. Die gesamte Figur ergibt daher ein neues Quadrat mit Seitenlänge $a+b$ und
  Flächeninhalt $(a+b)^2 $. Da sich die Fläche dieses Quadrats aber gerade aus den Flächen des kleineren 
  Quadrats (mit Flächeninhalt $c^2$) und der vier rechtwinkligen Dreiecke (mit Flächeninhalt von je $\frac{1}{2}ab$)
  zusammensetzt, ergibt sich die zu beweisende Formel
  }
  \lang{en}{
  The sides of the (black) square $PQRS$ have the same length $c$ as the hypotenuse of the triangle 
  which is superimposed on each side of the square (orange). Thanks to the right angle, the legs of 
  each triangle form a line of length $a+b$ with the leg of each neighbouring triangle where they 
  meet at the points $P$, $Q$, $R$ and $S$. In particular, the sum of the interior angles of a 
  triangle is $180^\circ$, so $\alpha + 90^\circ + \beta=180^\circ$, giving us the straight lines. 
  There is hence a new, larger square formed by the edges of the triangles (orange), with sides of 
  length $a+b$. As the area of this new square is simply the sum of the area of the smaller square 
  ($c^2$) and the areas of the triangles ($\frac{1}{2}ab$ each), we have the formula 
  }
  
  \begin{align*}
                   \;  & (a+b)^2 &= c^2+4\cdot \left(\frac{1}{2}ab\right) \quad &\vert 
                   \text{\lang{de}{1. binomische Formel}\lang{en}{expanding on both sides}}\\
   \Leftrightarrow \;  & a^2+2ab+b^2 \; &= c^2+2ab  &\vert -2ab \\
   \Leftrightarrow \;  & a^2+b^2 &= c^2  &
   \end{align*}
  
 \end{showhide}
\end{proof}


% \begin{center}
%     \lang{de}{\iframe[400][225][S]{https://www.stream24.net/vod/getVideo.php?id=10962-1-5533&mode=iframe}}
% \end{center}

\begin{quickcheckcontainer}
\randomquickcheckpool{1}{2}
\begin{quickcheck}
	\begin{variables}	
	% Pythagoräische Tripel:
	\randint{v}{1}{3}
	\randint{u0}{1}{2}
	\function[calculate]{u}{u0+v}
	\function[calculate]{a}{(u^2-v^2)/10}
	\function[calculate]{b}{(2*u*v)/10}
	\function[calculate]{c}{(u^2+v^2)/10}
	\function[calculate]{c2}{c^2}
	\end{variables}

	\type{input.number}	
	\field{real}
	\displayprecision{2}
	\correctorprecision{2}

	\text{\lang{de}{
        In einem rechtwinkligen Dreieck sind die Kathetenl"angen $\var{a}\text{ cm}$ und 
        $\var{b}\text{ cm}$ gegeben.\\
        Bestimmen Sie die L"ange der Hypotenuse des Dreiecks (in $\text{cm}$ und auf zwei Stellen 
        nach dem Komma gerundet).\\ \\
      	Die L"ange der Hypotenuse beträgt gerundet \ansref cm.
        }
        \lang{en}{
        Consider a right-angled triangle with legs of length $\var{a}\text{ cm}$ and 
        $\var{b}\text{ cm}$.\\
        Find the length of the hypotenuse of the triangle (in $\text{cm}$ and rounded to two decimal 
        places).\\\\
        The length of the hypotenuse is \ansref cm.
        }}

	\begin{answer}
		\solution{c}
	\end{answer}
	\explanation{\lang{de}{Nach dem Satz des Pythagoras ist die Hypotenusenlänge in cm gleich }
               \lang{en}{By the Pythagorean theorem, the length of the hypotenuse is }
                $\sqrt{\var{a}^2+\var{b}^2}=\sqrt{\var{c2}}=\var{c}$.}
	\end{quickcheck}

	
	\begin{quickcheck}
	\begin{variables}	
	% Pythagoräische Tripel:
	\randint{v}{1}{3}
	\randint{u0}{1}{2}
	\function[calculate]{u}{u0+v}
	\function[calculate]{a}{(u^2-v^2)/10}
	\function[calculate]{b}{(2*u*v)/10}
	\function[calculate]{c}{(u^2+v^2)/10}
	\function[calculate]{b2}{b^2}
	\end{variables}


	\type{input.number}	
	\field{real}
	\displayprecision{2}
	\correctorprecision{2}

	\text{\lang{de}{
        Im rechtwinkligen Dreieck $ABC$ sind die Kathetenlänge $\var{a}\text{ cm}$ und die 
        Hypotenusenlänge $\var{c}\text{ cm}$ gegeben.\\
    		Bestimmen Sie die L"ange der anderen Kathete des Dreiecks (in $\text{cm}$ und auf zwei 
        Stellen nach dem Komma gerundet).\\ \\
        Die L"ange der zweiten Kathete beträgt gerundet \ansref cm.
        }
        \lang{en}{
        Consider the right-angled triangle $ABC$ with a leg of length $\var{a}\text{ cm}$ and the 
        hypotenuse of length $\var{c}\text{ cm}$.\\
        Find the length of the other leg of the triangle (in $\text{cm}$ and rounded to two decimal 
        places).\\\\
        The length of the other leg of the triangle is \ansref cm.
        }}

	\begin{answer}
		\solution{b}
	\end{answer}
	\explanation{\lang{de}{Nach dem Satz des Pythagoras ist die zweite Kathetenlänge in cm gleich }
               \lang{en}{By the Pythagorean theorem, the length of the other leg of the triangle is }
               $\sqrt{\var{c}^2-\var{a}^2}=\sqrt{\var{b2}}=\var{b}$.}

	\end{quickcheck}
	
\end{quickcheckcontainer}
	



\section{\lang{de}{Sinus, Kosinus und Tangens}\lang{en}{Sine, cosine, and tangent}}\label{sinus}

\lang{de}{
Im rechtwinkligen Dreieck hat man auch besondere Zusammenh"ange zwischen den
Seitenl"angen und den Innenwinkeln, welche mithilfe der Winkelfunktionen 
Sinus, Kosinus und Tangens beschrieben werden.
}
\lang{en}{
In a right triangles there are special relationships between the side lengths and inner angles which 
can be described with the help of the sine, cosine, and tangent functions.
}

% Obwohl diese Funktionen zun"achst 
% "uber die Seitenverh"altnisse definiert werden, k"onnen sie auch ohne Kenntnis der
% Dreieckseiten berechnet werden 
% (vgl. \href{http://de.wikipedia.org/wiki/Sinus_und_Kosinus#Analytische_Definition}{Analytische 
% Definition von Sinus und Kosinus auf Wikipedia}).
% %Abschnitt \link{link2}{Trigonometrische Funktionen}
% %in \link{overview-el-functions}{Kapitel VI}).
% Damit liefern sie die M"oglichkeit, auch aus Kenntnis einer Seite und einem Winkel
% weitere Seiten des rechtwinkligen Dreiecks zu berechnen.

\begin{definition}[\lang{de}{Sinus, Kosinus und Tangens}
                   \lang{en}{Sine, cosine, and tangent}] \label{def-trigFkt}
\lang{de}{
\textbf{Sinus}, \textbf{Kosinus} und \textbf{Tangens} ordnen einem Winkel im rechtwinkligen
Dreieck die L"angenverh"altnisse der Katheten und Hypotenuse zu.
F"ur die Definition betrachtet man zun"achst ein rechtwinkliges
Dreieck $ABC$ wie in der Abbildung.
} 
\lang{en}{
Sine, cosine and tangent help us relate the side lengths of a triangle with its interior angles.
Consider first a right-angled triangle $ABC$ as in the figure.
}

\label{sin-cos-im-dreieck}
\begin{figure}
  \image{T105_Triangle_B}
\caption{\lang{de}{Rechtwinkliges Dreieck $ABC$}\lang{en}{Right-angled triangle $ABC$}}
\end{figure}

\lang{de}{
Man definiert dann \textbf{Sinus}, \textbf{Kosinus} und \textbf{Tangens} von $\alpha$ durch
}
\lang{en}{
We define the sine, cosine, and tangent of the angle $\alpha$ by:
}
  \begin{eqnarray*}
  \lang{de}{
    \sin(\alpha) &:=& \frac{\text{L"ange der Gegenkathete von }\alpha}
                           {\text{L"ange der Hypotenuse}}
                 =  \frac{a}{c}, \\
    \cos(\alpha) &:=& \frac{\text{L"ange der Ankathete von }\alpha}
                           {\text{L"ange der Hypotenuse}}
                 =  \frac{b}{c}, \\
    \tan(\alpha) &:=& \frac{\text{L"ange der Gegenkathete von }\alpha}
                           {\text{L"ange der Ankathete von }\alpha}
                 =  \frac{a}{b} = \frac{\sin(\alpha)}{\cos(\alpha)}.
    }
    \lang{en}{
    \sin(\alpha) &:=& \frac{a}{c}= \frac{\text{Length of the opposite leg of }\alpha}
                                        {\text{Length of the hypotenuse}} \\
    \cos(\alpha) &:=& \frac{b}{c}= \frac{\text{Length of the adjacent leg to }\alpha}
                                        {\text{Length of the hypotenuse}} \\
    \tan(\alpha) &:=& \frac{a}{b}= \frac{\text{Length of the opposite leg of }\alpha}
                                        {\text{Length of the adjacent leg to }\alpha} 
                                 = \frac{\sin(\alpha)}{\cos(\alpha)}
    }
  \end{eqnarray*}
  \lang{de}{Analoge Zusammenhänge gelten für den Winkel $\beta$.
  \\
  \floatright{\href{https://www.hm-kompakt.de/video?watch=127}{\image[75]{00_Videobutton_schwarz}}}\\\\
  }
  \lang{en}{Analogous definitions hold for the angle $\beta$.}
\end{definition}
%Eine Zusammenfassung der in Definition \ref{def-trigFkt} gegebenen Zusammenhänge, samt kurzen Rechenbeispielen, kann folgendem Video entnommen werden:\\

\begin{remark}
\lang{de}{
Die Werte $\sin(\alpha)$, $\cos(\alpha)$ und $\tan(\alpha)$ hängen nicht von der Gr"o"se des 
Dreiecks ab, sondern nur von der Winkelgr"o"se von $\alpha$.
}
\lang{en}{
The values $\sin(\alpha)$, $\cos(\alpha)$ and $\tan(\alpha)$ do not depend on the size of the 
triangle, but rather only on the angle $\alpha$.
}
\end{remark}


\begin{quickcheck}
		\field{rational}
		\type{input.number}

\begin{variables}
	% Pythagoräische Tripel:
	\randint{v}{1}{3}
	\randint{u0}{1}{2}
	\function[calculate]{u}{u0+v}
	\function[calculate]{a}{(u^2-v^2)}
	\function[calculate]{b}{(2*u*v)}
	\function[calculate]{c}{(u^2+v^2)}
\function[calculate]{ca}{b/c}
\function[calculate]{sa}{a/c}
\function[calculate]{ta}{a/b}
\end{variables}
\text{\lang{de}{
Im rechtwinkligen Dreieck $ABC$ sind die Kathetenl"angen $a=\var{a}\text{ cm}$
und $b=\var{b}\text{ cm}$ bekannt.\\
Dann sind $\cos(\alpha)$=\ansref, $\sin(\alpha)$=\ansref und $\tan(\alpha)$=\ansref (Die Ergebnisse können als Bruch angegeben werden).\\ 
}
\lang{en}{
Consider a right-angled triangle $ABC$ with leg lengths $a=\var{a}\text{ cm}$ and 
$b=\var{b}\text{ cm}$. \\
Then $\cos(\alpha)$=\ansref, $\sin(\alpha)$=\ansref, and $\tan(\alpha)$=\ansref (the results may be 
submitted as fractions).
}}


\begin{answer}
		\solution{ca}
	\end{answer}

	\begin{answer}
		\solution{sa}
	\end{answer}
	
	\begin{answer}
		\solution{ta}
	\end{answer}
\end{quickcheck}


\lang{de}{
Auf diese Weise sind also Sinus, Kosinus und Tangens f"ur alle spitzen Winkel (d.h. Winkel kleiner 
als $90^\grad$) definiert.\\
Betrachtet man statt $\alpha$ den Winkel $\beta$ und beachtet, dass wegen der Winkelsumme im Dreieck 
$\alpha+\beta+90^\circ=180^\circ\; \Leftrightarrow\; \beta=90^\grad-\alpha\;$ gilt, so erh"alt man 
die Gleichungen
}
\lang{en}{
In this way the sine, cosine, and tangent of an angle are defined for all acute angles (angles below 
$90^\grad$).\\
Consider the angle $\beta$ instead of the angle $\alpha$, and notice that as the sum of angles in a 
triangle is $180^\grad$, we have $\beta=90^\grad-\alpha$. Hence:
}
\begin{rule}\label{rule:sin_cos}
  \begin{eqnarray*}
    \lang{de}{\sin(90^\grad-\alpha) &=& \cos(\alpha), \\
    \cos(90^\grad-\alpha) &=& \sin(\alpha).}
    \lang{en}{\sin(90^\grad-\alpha) &=& \cos(\alpha) \\
    \cos(90^\grad-\alpha) &=& \sin(\alpha)} 
  \end{eqnarray*}

\end{rule}

\lang{de}{
Aus dem \lref{satz-des-pythagoras}{Satz des Pythagoras} erh"alt man noch einen Zusammenhang zwischen 
$\sin(\alpha)$ und $\cos(\alpha)$, n"amlich
}
\lang{en}{
From the \lref{satz-des-pythagoras}{Pythagorean theorem} we get another relationship between
$\sin(\alpha)$ and $\cos(\alpha)$, namely:
}

\begin{rule}\label{rule-trigon-pythagoras}
\[(\sin(\alpha))^2+(\cos(\alpha))^2=1\lang{de}{.} \]
\end{rule}
\begin{proof*}
\begin{showhide}
\lang{de}{Nach dem Satz des Pythagoras ist $a^2+b^2=c^2$. Daraus folgt}
\lang{en}{By the Pythagorean Theorem, $a^2+b^2=c^2$. It follows that}
\[(\sin(\alpha))^2+(\cos(\alpha))^2=\big(\frac{a}{c}\big)^2+\big(\frac{b}{c}\big)^2
=\frac{a^2+b^2}{c^2}=1. \]
\end{showhide}
\end{proof*}



\begin{example}
\label{right-angled-abb4}
  \begin{figure}
    \image{T105_Triangle_C}
	\caption{\lang{de}{Sinus und Kosinus von $45^\grad$.}\lang{en}{Sine and cosine of $45^\grad$.}}
  \end{figure}
  \begin{enumerate}
    \item \label{bsp-sinus-45-grad}
    \lang{de}{
    Betrachten wir ein rechtwinkliges Dreieck mit $\alpha=45^\grad$. Dann
    ist auch $\beta=90^\grad-\alpha=45^\grad$. Das heißt, die beiden Winkel sind
    gleich gro"s und damit sind auch die beiden Katheten gleich lang. Ein solches 
    Dreieck wird als \emph{\notion{gleichschenklig}} bezeichnet. Aus dem
    Satz des Pythagoras erh"alt man nun
    \[ c^2=a^2+b^2=2\cdot a^2 .\]
    Teilt man die Gleichung durch $a^2$ und zieht die Wurzel, erh"alt man somit
    \[ \frac{c}{a}=\sqrt{2}. \]
    Also gilt
    \[ \cos(45^\grad)=\sin(45^\grad)=
      \frac{a}{c}=\frac{1}{\sqrt{2}}=\frac{\sqrt{2}}{2}. \]
    }
    \lang{en}{
    Consider the right triangle with $\alpha=45^\grad$. Clearly $\beta$ is also 
    $\beta=90^\grad-\alpha=45^\grad$. Both angles are the same and hence both legs are also the 
    same length. From the Pythagorean theorem we get 
    \[ c^2=a^2+b^2=2\cdot a^2\]
    If we divide the equation by $a^2$ and then take the square root we get
    \[ \frac{c}{a}=\sqrt{2} \]
    and hence
    \[ \cos(45^\grad)=\sin(45^\grad)=
      \frac{a}{c}=\frac{1}{\sqrt{2}}=\frac{\sqrt{2}}{2} \]
    }
  \begin{figure}
    \image{T105_Triangle_D}
	\caption{\lang{de}{Sinus und Kosinus von $30^\grad$ und $60^\grad$.}
           \lang{en}{Sine and cosine of $30^\grad$ and $60^\grad$.}}
  \end{figure}
  \item \label{bsp-sinus-60-grad}\label{right-angled-abb5}
    \lang{de}{
    Betrachten wir ein rechtwinkliges Dreieck mit $\alpha=30^\grad$. Dann
    gilt unter Berücksichtigung der Innenwinkelsumme in Dreiecken
    $\beta=90^\grad-\alpha=60^\grad$. Spiegelt man das Dreieck an der Seite $b$,
    so erh"alt man ein Dreieck $ABB'$, dessen Innenwinkel s"amtlich
    $60^\grad$ betragen, das also gleichseitig ist. Daher gilt:
    \[
    a=\overline{BC}=\frac{1}{2}\overline{BB'}=\frac{1}{2}\overline{AB}=\frac{1}{2}\cdot
    c \]
    Somit erh"alt man
    \[ \sin(30^\grad)=\frac{a}{c}=\frac{1}{2}, \]
    sowie
    \[ \cos(60^\grad)=\sin(30^\grad)=\frac{1}{2}. \]
    $\sin(60^\grad)$ und $\cos(30^\grad)$ ergeben sich nach Regel 
    \ref{rule:sin_cos} als
    \[
    \sin(60^\grad)= \cos(30^\grad)= \sqrt{1-\sin(30^\grad)^2}=\sqrt{\frac{3}{4}}=\frac{\sqrt{3}}{2}.
    \]
    }
    \lang{en}{
    Consider a right triangle with $\alpha=30^\grad$, so $\beta=90^\grad-\alpha=60^\grad$. If we 
    mirror the triangle along the side $b$, we get a triangle $ABB'$ whose angles are all
    $60^\grad$ and thus which is an equilateral triangle. Hence we have:
    \[
    a=\overline{BC}=\frac{1}{2}\overline{BB'}=\frac{1}{2}\overline{AB}=\frac{1}{2}\cdot c. 
    \]
    Using this we get that
    \[ \sin(30^\grad)=\frac{a}{c}=\frac{1}{2} \]
    and
    \[ \cos(60^\grad)=\sin(30^\grad)=\frac{1}{2}. \]
    Rule \ref{rule-trigon-pythagoras} also gives us $\sin(60^\grad)$ and $\cos(30^\grad)$:
    \[ \sin(60^\grad)= \cos(30^\grad)= \sqrt{1-\sin(30^\grad)^2
      }=\sqrt{\frac{3}{4}}=\frac{\sqrt{3}}{2}\]
    }
  \end{enumerate}

\end{example}

%\textbf{\lang{de}{Anmerkung:}\lang{en}{Remark}} 
\begin{remark}
\lang{de}{
Sinus, Kosinus und Tangens eines Winkels lassen sich mit
Hilfe analytischer Methoden berechnen, ohne vorher die Seitenl"angen des Dreiecks
kennen zu m"ussen. Diese werden wir noch im % \link{trig-reihe}{zweiten Teil} 
\ref[trig-reihe][zweiten Teil]{sec:sinus-kosinus}
dieses Kurses kennenlernen.
%(vgl. \href{http://de.wikipedia.org/wiki/Sinus_und_Kosinus#Analytische_Definition}{Analytische 
%Definition von Sinus und Kosinus auf Wikipedia}).
%(vgl. Abschnitt \link{link2}{Trigonometrische Funktionen}
%in \link{overview-el-functions}{Kapitel VI}).
Diese Methoden verwendet zum Beispiel der Taschenrechner.\\
Ist beispielsweise einer der spitzen Winkel eines rechtwinkligen Dreiecks bekannt, kann man 
hieraus den Sinus oder den Kosinus bestimmen. Ist zusätzlich noch die Länge einer beliebigen 
Seite dieses Dreiecks gegeben, lässt sich mithilfe der Formeln aus Definition \ref{def-trigFkt} 
das Dreieck komplett bestimmen, beispielweise gilt
%Man kann daher Sinus, Kosinus und Tangens auch benutzen, um aus gegebenen
%%Winkeln und einer Seitenl"ange eines rechtwinkligen Dreiecks die anderen
%Seiten zu berechnen. Dazu m"ussen nur die Gleichungen aus der ersten
%Definition nach den entsprechenden Seiten aufgel"ost werden, also zum Beispiel
}
\lang{en}{
The sine, cosine, and tangent of an angle can be calculated with the help of analytical methods 
without worrying about the side lengths of triangles (see the 
\href{http://en.wikipedia.org/wiki/Trigonometric_functions#Series_definitions}{analytical 
definition of sine and cosine on Wikipedia}). 
Pocket calculators and most software based-calculators, like those found on your cell phone, utilize 
these analytical methods to calculate sine and cosine. We can also use sine, cosine, and tangent to 
determine the side lengths of a right triangle when one (non right) angle and one side length are 
given. To do so, we simply solve for the component in the above definition that we need to 
determine, e.g.
}

\[
\text{\lang{de}{L"ange der Ankathete von }\lang{en}{{Length of side adjacent to }}}\alpha 
= \cos(\alpha)\cdot \text{\lang{de}{L"ange der Hypotenuse}\lang{en}{Length of hypotenuse}}.
\]
\end{remark}

\begin{block}[warning]
\lang{de}{
Da au"ser dem Gradma"s auch das Bogenma"s als Ma"seinheit für Winkel gebräuchlich ist (s. \link{link-gradmass}{nächster Abschnitt}),
ist bei der Verwendung der trigonometrischen Funktionen auf dem Taschenrechner darauf zu achten, dass die richtige
Ma"seinheit (\textit{deg} für Gradma"s und \textit{rad} für Bogenma"s) eingestellt ist.
}
\lang{en}{
As both degrees and radians can be used as a unit for measuring angles (see the 
\link{link-gradmass}{next section}), be careful that your calculator is set to the correct unit 
(\textit{deg} for degrees and \textit{rad} for radians)!
}
\end{block}

\lang{de}{
Folgende Wertetabelle enthält die Werte der Sinus- und der Kosinusfunktion für spezielle Winkel:
}
\lang{en}{
The following table of values contains the values for sine and cosine for some useful angles:
}
\begin{center}
\begin{table} \label{tbl:trigData}
\head[ccccccccc]
\lang{de}{Winkel in Grad}\lang{en}{Angle in Degrees} &
  $\;\; 0^{\circ}\;\;$&$\;\; 30 ^{\circ}\;\;$ &$\;\; 45 ^{\circ}\;\;$&$\;\; 60 ^{\circ}\;\;$& 
    $\;90^{\circ} \;$\\
\lang{de}{Winkel in Bogenmaß}\lang{en}{Angle in Radians} &
  $0$& $\frac{\pi}{6}$ & $\frac{\pi}{4}$ & $\frac{\pi}{3}$ & 
    $\frac{\pi}{2}$\\
\body
$\sin(x)$& $0$&$\frac{1}{2}$&$\frac{\sqrt{2}}{2}$&$\frac{\sqrt{3}}{2}$&$1$\\
$\cos(x)$& $1$&$\frac{\sqrt{3}}{2}$&$\frac{\sqrt{2}}{2}$&$\frac{1}{2}$&$0$\\
\end{table}
\end{center}
\lang{de}{
Eine kurze Zusammenfassung der in Abschnitt \ref{sinus} angesprochenen Themen kann dem folgenden Video entnommen werden:\\
\floatright{\href{https://api.stream24.net/vod/getVideo.php?id=10962-2-10923&mode=iframe&speed=true}{\image[75]{00_video_button_schwarz-blau}}}\\
}
\lang{en}{
All of these values can be found without a calculator like in the example above.
}

\section{\lang{de}{Umkehrung von Sinus, Kosinus und Tangens}
         \lang{en}{Inverse of sine, cosine and tangent}}\label{sec:arc-functions}
\lang{de}{
Wir wollen uns im Rahmen des nachfolgenden Abschnitts mit den Umkehrungen von Sinus, Kosinus und 
Tangens beschäftigen und starten mit der folgenden Fragestellung: Angenommen in einem rechtwinkligen 
Dreieck sind die Seitenlängen $c=2$ und $b=1$ bekannt (siehe Abbildung). Wie groß ist der Winkel 
$\beta$?
}
\lang{en}{
In this section we define the inverse functions of sine, cosine and tangent, and how these can be 
used to find the size of an angle given two side lengths of a triangle. For example, suppose that 
we are given side lengths $c=2$ and $b=1$ for the figure below, and want to find the angle $\beta$.
}
\begin{figure}
 \image{T105_Triangle_E}
\end{figure}
\lang{de}{Nach Definition \ref{def-trigFkt} gilt}
\lang{en}{By definition \ref{def-trigFkt},}
\begin{equation*}
    \sin(\beta)=\frac{\text{\lang{de}{L"ange der Gegenkathete von }
                            \lang{en}{Length of the side opposite }}\beta}
                     {\text{\lang{de}{L"ange der Hypotenuse}
                            \lang{en}{Length of the hypotenuse}}}=\frac{b}{c}=\frac{1}{2}.
\end{equation*}
\lang{de}{
Wir haben eine Gleichung mit einer Unbekannten. Augenscheinlich sollte sich das Problem also lösen 
lassen. Wie lässt sich diese Gleichung umstellen?\\
Wir erkennen, dass wir die Umkehrung der trigonometrischen Ausdrücke (im vorliegen Beispiel die Umkehrung des Sinus) benötigen. 
Diese Umkehrfunktionen werden mit dem Präfix \textbf{Arcus} bezeichnet.
Die Umkehrung des Sinus lautet Arcus-Sinus, die des Kosinus Arcus-Kosinus und die des Tangens Arcus-Tangens. Für die Verwendung dieser Funktionen haben sich folgende Kurzschreibweisen etabliert:
}
\lang{en}{
This is an equation with an unknown $\beta$, which by visual inspection has a solution. How can we 
rearrange this equation for $\beta$? 
We recognise that we need an inverse for the trigonometric functions (sine in this case). To denote 
the inverse we use the prefix \textit{arc}:
}
\begin{itemize}
    \item \notion{\lang{de}{Arcus-Sinus}\lang{en}{Inverse sine}}: 
      $\quad \arcsin(x)$ \lang{de}{oder}\lang{en}{or} $\sin^{-1}(x)$
    \item \notion{\lang{de}{Arcus-Kosinus}\lang{en}{Inverse cosine}}: 
      $\arccos(x)$ \lang{de}{oder}\lang{en}{or} $\cos^{-1}(x)$
    \item \notion{\lang{de}{Arcus-Tangens}\lang{en}{Inverse tangent}}:
      $\arctan(x)$ \lang{de}{oder}\lang{en}{or} $\tan^{-1}(x)$
\end{itemize}
\lang{de}{
Die Arcus-Ausdrücke lassen sich also verwenden, um aus einem gegebenen Sinus-, Kosinus- oder Tangens-Wert wieder den zugrundeliegenden Winkel zu bekommen.\\
Mit diesem Wissen lässt sich unsere Ausgangsfragestellung nun beantworten:
}
\lang{en}{
The purpose of these inverse functions is by definition for finding the angle that has a given sine, cosine or tangent value. They can be used to rearrange the above equation:
}
\begin{align*}
    & \sin(\beta)&=\frac{1}{2} \\
    \Leftrightarrow &\quad \beta&=\arcsin\left(\frac{1}{2}\right)=30^\circ
\end{align*}
\lang{de}{Der Winkel $\beta$ des gegebenen Dreiecks beträgt also $30^\circ$.}
\lang{en}{Hence the angle $\beta$ of the triangle is $30^\circ$.}
%\begin{example}
%Da $\sin(60^\circ)=\frac{\sqrt{3}}{2}=\cos(30^\circ)$ ist (vgl. Beispiel \ref{bsp-sinus-60-grad}), gelten
%\[  \arcsin\big(\frac{\sqrt{3}}{2}\big) = 60^\circ \quad\text{und}\quad \arccos\big(\frac{\sqrt{3}}{2}\big) = 30^\circ. \]
%\end{example}
\begin{block}[warning]
\lang{de}{
Taschenrechner verwenden zumeist jeweils die zweite Notationsoption bei
der Darstellung der Arcus-Ausdrücke, sprich 
$\sin^{-1}(x)$, $\cos^{-1}(x)$ und $\tan^{-1}(x)$.
Zudem ist bei der Verwendung der Arcus-Ausdrücke natürlich
wieder darauf zu achten, dass die ''richtige''
Ma"seinheit (Gradmaß oder Bogenmaß) im Taschenrechner eingestellt wird.
}
\lang{en}{
Most calculators use the notation $\sin^{-1}(x)$, $\cos^{-1}(x)$ und $\tan^{-1}(x)$ for the inverse 
trigonometric functions. For inverse trigonometric functions it is still important to set the 
correct unit for angles in the calculator.
}
\end{block}



\begin{remark}
\lang{de}{
Im rechtwinkligen Dreieck liegen die Sinus- und Kosinuswerte im Intervall $[0;1]$.
Damit sind der Arcus-Sinus und der 
Arcus-Kosinus zunächst nur auf diesem Intervall definiert.\\
Die Tangenswerte erstrecken sich hingegen auf alle positiven 
reellen Zahlen,  
% weshalb die Werte des Arcus-Tangens im Intervall $(0;\infty)$ liegt.\\
% neu
weshalb der Arcus-Tangens hier zunächst nur auf dem Intervall $(0;\infty)$ definiert wird.
\\\\
Im Abschnitt \link{link-sinus-reell}{Allgemeine trigonometrische Funktionen} werden 
Sinus und Kosinus als reelle Funktionen
auf ganz $\R$ (d.h. für alle Winkel im Bogenmaß) definiert. Für diese erweiterten 
Funktionen vergrößert sich auch die Wertemenge, wodurch
die zugehörigen Arcus-Funktionen auch auf einem größeren Bereich definiert sind.
}
\lang{en}{
The sine and cosine values of interior angles of a right-angled triangle lie in the interval 
$[0;1]$. Inverse sine and inverse cosine are therefore only defined on this interval for now.\\
The tangent on the other hand can be any positive real number, so the inverse tangent function is 
defined on the interval $(0;\infty)$.
}
\end{remark}



\begin{quickcheck}
		\field{real}
		\displayprecision{3}
		\type{input.number}
		\begin{variables}
			\randint{w}{1}{80}
			\function[calculate]{s}{sin(w*pi/180)}
		\end{variables}

			\text{\lang{de}{
            Bestimmen Sie mit dem Taschenrechner den Winkel $\alpha$ zwischen $0^\circ$ und 
            $90^\circ$ (auf ganze Grad gerundet) mit 
            }
            \lang{en}{
            Using a calculator, calculate the angle $\alpha$ between $0^\circ$ and $90^\circ$, 
            rounded to the nearest degree, with 
            }
            $\sin(\alpha)=\var{s}$.\\ $\alpha\approx$\ansref${\phantom{|}}^\circ$.}

		\begin{answer}
			\solution{w}
		\end{answer}
	\end{quickcheck}
% 
% 
% 	\begin{genericGWTVisualization}[550][1000]{mathlet1}
% 		\begin{variables}
% 			\randint{randomA}{1}{2}
% 
% 			\point[editable]{P}{rational}{var(randomA),var(randomA)}
% 		\end{variables}
% 		\color{P}{BLUE}
% 		\label{P}{$\textcolor{BLUE}{P}$}
% 
% 		\begin{canvas}
% 			\plotSize{300}
% 			\plotLeft{-3}
% 			\plotRight{3}
% 			\plot[coordinateSystem]{P}
% 		\end{canvas}
% 		\text{Der Punkt hat die Koordinaten $(\var{P}[x],\var{P}[y])$.}
% 	    	\end{genericGWTVisualization}

\end{visualizationwrapper}


\end{content}