
%$Id:  $
\documentclass{mumie.article}
%$Id$
\begin{metainfo}
  \name{
    \lang{de}{Überblick: Trigonometrische Funktionen}
    \lang{en}{Overview: Trigonometric functions}
  }
  \begin{description} 
 This work is licensed under the Creative Commons License Attribution 4.0 International (CC-BY 4.0)   
 https://creativecommons.org/licenses/by/4.0/legalcode 

    \lang{de}{Beschreibung}
    \lang{en}{Description}
  \end{description}
  \begin{components}
  \end{components}
  \begin{links}
\link{generic_article}{content/rwth/HM1/T105_Trigonometrische_Funktionen/g_art_content_19_allgemeiner_sinus_cosinus.meta.xml}{content_19_allgemeiner_sinus_cosinus}
\link{generic_article}{content/rwth/HM1/T105_Trigonometrische_Funktionen/g_art_content_18_grad_und_bogenmass.meta.xml}{content_18_grad_und_bogenmass}
\link{generic_article}{content/rwth/HM1/T105_Trigonometrische_Funktionen/g_art_content_17_trigonometrie_im_dreieck.meta.xml}{content_17_trigonometrie_im_dreieck}
\end{links}
  \creategeneric
\end{metainfo}
\begin{content}
\begin{block}[annotation]
	Im Ticket-System: \href{https://team.mumie.net/issues/30144}{Ticket 30144}
\end{block}



\begin{block}[annotation]
Im Entstehen: Überblicksseite für Kapitel Trigonometrische Funktionen
\end{block}

\usepackage{mumie.ombplus}
\ombchapter{1}
\title{\lang{de}{Überblick: Trigonometrische Funktionen}\lang{en}{Overview: Trigonometric functions}}




\begin{block}[info-box]
\lang{de}{\strong{Inhalt}}
\lang{en}{\strong{Contents}}


\lang{de}{
    \begin{enumerate}%[arabic chapter-overview]
   \item[5.1] \link{content_17_trigonometrie_im_dreieck}{Trigonometrie im Dreieck}
   \item[5.2] \link{content_18_grad_und_bogenmass}{Grad- und Bogenmaß}
   \item[5.3] \link{content_19_allgemeiner_sinus_cosinus}{Trigonometrische Funktionen}
     \end{enumerate}
}
\lang{en}{
    \begin{enumerate}%[arabic chapter-overview]
   \item[5.1] \link{content_17_trigonometrie_im_dreieck}{Introduction to trigonometry}
   \item[5.2] \link{content_18_grad_und_bogenmass}{Degrees and radians}
   \item[5.3] \link{content_19_allgemeiner_sinus_cosinus}{Trigonometric functions}
     \end{enumerate}
} %lang

\end{block}

\begin{zusammenfassung}

\lang{de}{
Ausgehend von den Winkelfunktionen im rechtwinkligen Dreieck erschließen wir Sinus, Kosinus, Tangens 
und Kotangens, stellen ihre wichtigsten Eigenschaften zusammen und lernen die Arcus-Funktionen kennen.
\\
Dazu gehört auch der Wechsel vom Gradmaß in das Bogenmaß, bei dem $2\pi$ als Länge der Kreislinie mit 
Radius eins einen Vollkreis beschreibt.
\\
Läuft man beliebig oft vorwärts oder rückwärts an einem solchen Einheitskreis, so ermöglicht dies,
den Definitionsbereich der trigonometrischen Funktionen periodisch auf die gesamten reellen Zahlen auszudehnen. 
\\
An deren Graphen lesen wir ihre Nullstellen und Extremstellen ab.
Wir sehen, dass der Kosinus dem Sinus um eine viertel Periode voraus ist.
}
\lang{en}{
Using the side lengths and interior angles of a right-angled triangle, we define the sine, cosine 
and tangent of an angle, outline their most important properties and introduce their inverse 
functions, prefixed with \textit{arc-}.
\\
We show how to convert between degrees and radians, the two main units used for the magnitude of an 
angle. Radians describe the length of the arc described by an angle on a unit circle.
\\
The unit circle allows us to extend the domain of our definition of the sine, cosine and tangent functions to the set of real numbers.
\\
We consider the period, roots, critical points and branches of the trigonometric functions. We note 
that the sine and cosine functions have the same graphs, just shifted along the $x$-axis by a 
quarter of their period.
}


\end{zusammenfassung}

\begin{block}[info]\strong{\lang{de}{Lernziele}\lang{en}{Learning Goals}}
\begin{itemize}[square]
\item \lang{de}{
Sie berechnen die Winkelfunktionen $sin$, $cos$, $tan$, $cotan$ und ihre Umkehrfunktionen im 
rechtwinkligen Dreieck.
}
\lang{en}{
Being able to calculate values of the trigonometric functions $sin$, $cos$, $tan$, $cotan$ and their 
inverse functions in a right-angled triangle.
}
\item \lang{de}{Sie rechnen Grad- in Bogenmaß um und umgekehrt.}
      \lang{en}{Being able to convert between degrees and radians.}
\item \lang{de}{
Sie bestimmen die trigonometrischen Funktionen am Einheitskreis und als Funktionen der reellen 
Zahlen.
}
\lang{en}{
Knowing the definition of the trigonometric functions in the unit circle, over the real numbers.
}
\item \lang{de}{
Sie kennen wichtige Eigenschaften der trigonometrischen Funktionen, wie deren Periodizität, 
Verschiebungsrelationen, Nullstellen, Minimal- und Maximalstellen.
}
\lang{en}{
Knowing important properties of trigonometric functions, such as their periodicity, identities 
between them, minimal and maximal points, and roots.
}
\item \lang{de}{Sie erkennen die Graphen der trigonomerischen und der Arcus-Funktionen.}
      \lang{en}{Recognising the graphs of the trigonometric functions and their inverses functions.}
\end{itemize}
\end{block}




\end{content}
