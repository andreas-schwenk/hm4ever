\documentclass{mumie.element.exercise}
%$Id$
\begin{metainfo}
  \name{
    \lang{de}{Ü03: Bogen- und Gradmaß}
    \lang{en}{Exercise 3}
  }
  \begin{description} 
 This work is licensed under the Creative Commons License Attribution 4.0 International (CC-BY 4.0)   
 https://creativecommons.org/licenses/by/4.0/legalcode 

    \lang{de}{Hier die Beschreibung}
    \lang{en}{}
  \end{description}
  \begin{components}
  \end{components}
  \begin{links}
  \end{links}
  \creategeneric
\end{metainfo}
\begin{content}

\title{
\lang{de}{Ü03: Bogen- und Gradmaß}
}

\begin{block}[annotation]
  Im Ticket-System: \href{http://team.mumie.net/issues/9720}{Ticket 9720}
\end{block}

 
 \begin{itemize}
  \item[1a)] Gegeben sind folgende Winkel in Gradmaß. Bestimmen Sie jeweils das zugehörige Bogenmaß.
  \begin{itemize}
    \item[i)] $30^\circ = x\cdot \pi$
    \item[ii)] $50^\circ = x\cdot \pi$
    \item[iii)] $2^\circ = x\cdot \pi$
    \item[iv)] $120^\circ = x \cdot\pi$
  \end{itemize} 
  \item[1b)] Gegeben sind folgende Winkel in Bogenmaß. Bestimmen Sie jeweils das zugehörige Gradmaß.
   \begin{itemize}
    \item[i)] $3\pi/2=x^\circ$
    \item[ii)] $\pi/6=x^\circ$
    \item[iii)] $2\pi=x^\circ$
    \item[iv)] $\pi/8=x^\circ$
  \end{itemize}
  \item[2a)] Wandeln Sie die Gradzahlen $90^\circ,\, 180^\circ,\, 45^\circ,\, 30^\circ,\, 270^\circ$ und $1^\circ$ in Bogenmaß um und veranschaulichen Sie sich die Bogenmaße im Einheitskreis.
  \item[2b)]Wandeln Sie die folgenden Bogenmaß-Angaben in Gradzahlen um: $\pi,\, 2\pi,\, -\frac{\pi}{2},\, \frac{\pi}{6},\, \frac{\pi}{3},\, \frac{3}{4}\pi,\, 1$
 \end{itemize}
 
 \begin{tabs*}[\initialtab{0}\class{exercise}]
  \tab{\lang{de}{Antworten}}
    \begin{itemize}
        \item[1a)] $\frac{1}{6}\pi,\, \frac{5}{18}\pi,\, \frac{1}{90}\pi,\, \frac{2}{3}\pi$
        \item[1b)] $270^\circ,\, 30^\circ,\, 360^\circ,\, 180^\circ$
        \item[2a)] $\frac{\pi}{2},\, \pi,\, \frac{\pi}{4},\, \frac{\pi}{6},\, \frac{3}{2}\pi,\, \frac{\pi}{180}$
        \item[2b)] $180^\circ,\, 360^\circ,\, -90^\circ,\, 30^\circ,\, 60^\circ,\, 135^\circ,\, \frac{180^\circ}{\pi}$
    \end{itemize}
  
  \tab{\lang{de}{Lösung 1a)}}
   Es gilt die Formel: \[\text{Bogenmaß des Winkels}=\frac{\text{Gradmaß des Winkels}}{360^\circ}\cdot 2\pi=\frac{\text{Gradmaß des Winkels}}{180^\circ}\cdot \pi\]
   Damit ergibt sich
   \begin{table}[\class{items}]
   i) $30^\circ = \frac{1}{6} \pi$ \\
   ii) $50^\circ = \frac{5}{18} \pi$ \\
   iii) $2^\circ = \frac{1}{90} \pi$ \\
   iv) $120^\circ = \frac{2}{3} \pi$
   \end{table}
 
   
  \tab{\lang{de}{Lösung 1b)}}
   Es gilt die Formel: \[\text{Gradmaß des Winkels}=\frac{\text{Bogenmaß des Winkels}}{2\pi}\cdot 360^\circ=\frac{\text{Bogenmaß des Winkels}}{\pi}\cdot 180^\circ\]
   Damit ergibt sich
   \begin{table}[\class{items}]
   i) $3\pi/2=270^\circ$ \\
   ii) $\pi/6=30^\circ$ \\
   iii) $2\pi=360^\circ$ \\
   iv) $\pi/8=180^\circ/8=45^\circ/2=22,5^\circ$
   \end{table}
 
   \tab{\lang{de}{Lösungsvideo 2)}}	
    Die Lösungswege zu Aufgabe 2 sind dem folgenden Video zu entnehmen\youtubevideo[500][300]{UL1ehX-GJxE}\\
 
 
\end{tabs*}


\end{content}