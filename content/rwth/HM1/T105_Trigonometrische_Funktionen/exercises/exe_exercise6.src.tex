\documentclass{mumie.element.exercise}
%$Id$
\begin{metainfo}
  \name{
    \lang{de}{Ü06: Trigonometrische Gleichungen}
    \lang{en}{Exercise 6}
  }
  \begin{description} 
 This work is licensed under the Creative Commons License Attribution 4.0 International (CC-BY 4.0)   
 https://creativecommons.org/licenses/by/4.0/legalcode 

    \lang{de}{Hier die Beschreibung}
    \lang{en}{}
  \end{description}
  \begin{components}
\component{generic_image}{content/rwth/HM1/images/g_tkz_T105_Exercise06_C.meta.xml}{T105_Exercise06_C}
\component{generic_image}{content/rwth/HM1/images/g_tkz_T105_Exercise06_B.meta.xml}{T105_Exercise06_B}
\component{generic_image}{content/rwth/HM1/images/g_tkz_T105_Exercise06_A.meta.xml}{T105_Exercise06_A}
\end{components}
  \begin{links}
  \end{links}
  \creategeneric
\end{metainfo}
\begin{content}

\title{
\lang{de}{Ü06: Trigonometrische Gleichungen}
}

\begin{block}[annotation]
  Im Ticket-System: \href{http://team.mumie.net/issues/9724}{Ticket 9724}
\end{block}
Geben Sie alle Lösungen für $x$ im Intervall von $0$ bis $2\pi$ bzw. von $0^{\circ}$ bis $360^{\circ}$ an. 
%Bestimmen Sie mithilfe eines Taschenrechners folgende Werte. Geben Sie die Ergebnisse in Grad- und Bogenmaß an (mit Winkeln zwischen $0^\circ$ und $360^\circ$ bzw. 0 und $2\pi$) und runden Sie jeweils auf 2 Nachkommastellen genau.
 \begin{table}[\class{items}]
      a) $\cos(x)=\frac{\sqrt{2}}{2}$ & b) $\sin(x)=-\frac{1}{2}$ & c) $\tan(x)=\frac{\sqrt{3}}{3}$
 \end{table}
 
 \begin{tabs*}[\initialtab{0}\class{exercise}]
  \tab{Lösung a)}
    \begin{incremental}[\initialsteps{1}]
    \step Die Lösung ergibt sich zu
        \begin{align*}
            x_1&=\pi/4=45^{\circ}\\
            x_2&=7\pi/4=315^{\circ}.
        \end{align*}
    \step \textbf{Erklärung:} Wir benutzen die Arcus-Kosinus-Funktion des Taschenrechners und erhalten 
      $x_1=\arccos(\sqrt{2}/2)=\pi/4=45^{\circ}$. Die weiteren Lösungen ergeben sich 
      durch Ausnutzung der Symmetrie bzw. Periodizität der Kosinus-Funktion, d.h.
      \[
      \cos(\frac{\pi}{4})=\cos(-\frac{\pi}{4})=\cos(-\frac{\pi}{4}+2\pi),
      \]
      also
      $x_2=2\pi-\pi/4=7\pi/4=315^{\circ}$ (siehe Abbildung). \\
      \begin{figure}
          \image{T105_Exercise06_A}
      \end{figure} 
    \end{incremental}
  \tab{Lösung b)}
    \begin{incremental}[\initialsteps{1}]
    \step Die Lösung ergibt sich zu
        \begin{align*}
            x_1&=7\pi/6=210^{\circ}\\
            x_2&=11\pi/6=330^{\circ}.
        \end{align*}
    \step \textbf{Erklärung:} Wir benutzen die Arcus-Sinus-Funktion des Taschenrechners und erhalten 
      $x_0=\arcsin(-1/2)=-\pi/6=-30^{\circ}$. Dieser Wert liegt nicht im vorgegebenen Intervall und ist somit kein Teil der gesuchten Lösung.
      Die Lösungen des vorgegebenen Intervalls ergeben sich 
      durch Ausnutzung der Symmetrie bzw. Periodizität der Sinus-Funktion, d.h.
      \[
      \sin(-\frac{\pi}{6})=\cos(-\frac{\pi}{6}-\frac{\pi}{2})=\cos(\frac{\pi}{6}+\frac{\pi}{2})=\sin(\frac{\pi}{6}+\pi)
      \]
      und
      \[
      \sin(-\frac{\pi}{6})=\sin(-\frac{\pi}{6}+2\pi),
      \]
      also
      $x_1=\pi+\pi/6=7\pi/6=210^{\circ}$ und $x_2=2\pi-\pi/6=11\pi/6=330^{\circ}$ (siehe Abbildung).   \\
      \begin{figure}
          \image{T105_Exercise06_B}
      \end{figure} 
    \end{incremental}
  \tab{Lösung c)}
    \begin{incremental}[\initialsteps{1}]
    \step Die Lösung ergibt sich zu
        \begin{align*}
            x_1&=\pi/6=30^{\circ}\\
            x_2&=7\pi/6=210^{\circ}.
        \end{align*}
    \step \textbf{Erklärung:} Wir benutzen die Arcus-Tangens-Funktion des Taschenrechners und erhalten 
      $x_1=\arctan(\sqrt{3}/3)=\pi/6=30^{\circ}$. Die weiteren Lösungen ergeben sich 
      durch Ausnutzung der Symmetrie bzw. Periodizität der Tangens-Funktion
      $x_2=\pi+\pi/6=7\pi/6=210^{\circ}$ (siehe Abbildung).  \\ 
      \begin{figure}
          \image{T105_Exercise06_C}
      \end{figure}
    \end{incremental}
\end{tabs*}


\end{content}