\documentclass{mumie.element.exercise}
%$Id$
\begin{metainfo}
  \name{
    \lang{de}{Ü01: Geometrische Probleme}
    \lang{en}{Exercise 1}
  }
  \begin{description} 
 This work is licensed under the Creative Commons License Attribution 4.0 International (CC-BY 4.0)   
 https://creativecommons.org/licenses/by/4.0/legalcode 

    \lang{de}{Hier die Beschreibung}
    \lang{en}{}
  \end{description}
  \begin{components}
\component{generic_image}{content/rwth/HM1/images/g_tkz_T105_Exercise01_B.meta.xml}{T105_Exercise01_B}
\component{generic_image}{content/rwth/HM1/images/g_tkz_T105_Exercise01_A.meta.xml}{T105_Exercise01_A}
\end{components}
  \begin{links}
  \end{links}
  \creategeneric
\end{metainfo}
\begin{content}

\title{
\lang{de}{Ü01: Geometrische Probleme}
}

\begin{block}[annotation]
  Im Ticket-System: \href{http://team.mumie.net/issues/9718}{Ticket 9718}
\end{block}

 
 
\lang{de}{
\begin{itemize}
    \item[1)] In einem rechtwinkligen Dreieck sind die Kathetenlängen 4 cm und 5 cm gegeben. Bestimmen Sie die Länge der Hypotenuse des Dreiecks (in cm und auf eine Stelle nach dem Komma gerundet).
    \item[2)] Berechnen Sie die fehlenden Seitenlängen in den rechtwinklingen Dreiecken. Die Zeichnungen sind nicht maßstabsgetreu. Nutzen Sie einen Taschenrechner. Runden Sie ihre Ergebnisse auf eine Stelle nach dem Komma.\\
      \begin{figure}
        \image{T105_Exercise01_A}
      \end{figure}   
\end{itemize}
}


\begin{tabs*}[\initialtab{0}\class{exercise}]
  \tab{
  \lang{de}{Antworten}
  \lang{en}{Answers}
  }
  \begin{itemize}
    \item[1)] $\sqrt{41}\approx 6,40$
    \item[2)] 
        \begin{itemize}
            \item[a)] $x\approx 4,3$; $y=2,5$
            \item[b)] $x\approx 4,7$; $y\approx 3,6$
            \item[c)] $x\approx 2,3$; $y\approx 4,6$        
        \end{itemize}
  \end{itemize}
      
  \tab{
  \lang{de}{Lösung 1)}
  \lang{en}{}
  }
    Es bezeichne $x$ die Länge der Hypotenuse (in cm). Nach dem Satz des Pythagoras gilt
    \[
     x^2=4^2+5^2=16+25=41.
    \]
    Es gilt also $x=\sqrt{41}\approx 6,4$.\\
    \begin{figure}
        \image{T105_Exercise01_B}
    \end{figure}
  \tab{\lang{de}{Lösungsvideo 2)}}	
    Die Lösungswege zu Aufgabe 2 sind dem folgenden Video zu entnehmen \youtubevideo[500][300]{_Qw2rS_OJKU}\\
\end{tabs*}
\end{content}