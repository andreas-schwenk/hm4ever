\documentclass{mumie.element.exercise}
%$Id$
\begin{metainfo}
  \name{
    \lang{de}{Ü07: Funktionsgraphen}
    \lang{en}{Exercise 7}
  }
  \begin{description} 
 This work is licensed under the Creative Commons License Attribution 4.0 International (CC-BY 4.0)   
 https://creativecommons.org/licenses/by/4.0/legalcode 

    \lang{de}{Hier die Beschreibung}
    \lang{en}{}
  \end{description}
  \begin{components}
  \component{generic_image}{content/rwth/HM1/images/g_tkz_T105_Exercise07.meta.xml}{T105_Exercise07}
  \end{components}
  \begin{links}
\link{generic_article}{content/rwth/HM1/T105_Trigonometrische_Funktionen/g_art_content_19_allgemeiner_sinus_cosinus.meta.xml}{content_19_allgemeiner_sinus_cosinus}
\end{links}
  \creategeneric
\end{metainfo}
\begin{content}

\title{
\lang{de}{Ü07: Funktionsgraphen}
}

\begin{block}[annotation]
  Im Ticket-System: \href{http://team.mumie.net/issues/9725}{Ticket 9725}
\end{block}


\lang{de}{In der nachfolgenden Abbildung sind in orangener, blauer und grüner Farbe die Graphen von drei Winkelfunktionen dargestellt.}
\lang{en}{In \lref{Trigonometrische_Funktionen_Uebung4_1}{Figure 1} there are three graphs of trigonometric functions in orange, blue, and green.}
\\
\label{Trigonometrische_Funktionen_Uebung4_1}
\begin{figure} 
  \image{T105_Exercise07}
  %\lang{de}{\caption{Abb 1: Drei Winkelfunktionen}}
  %\lang{en}{\caption{Figure 1: Three Trigonometric Functions}}
\end{figure}

\begin{table}[\class{items}]
\nowrap{ $g(x)= \cos\left(\frac{5}{2}\pi-x\right)$}\\
\nowrap{ $h(x)= \sin\left(x+\frac{\pi}{2}\right)$}\\ 
\nowrap{ $k(x)= \tan(x+\pi)$}\\ 
  \end{table}
\lang{de}{Ordnen Sie den drei Graphen jeweils die zugehörige Funktionsvorschrift zu und begründen Sie das mit bekannten Beziehungen zwischen den trigonometrischen Funktionen.\\ }
\lang{en}{Match each of the three functions to their corresponding graph and rationalize your decision with the relationships between the trigonometric functions.\\ }
  			

\begin{tabs*}[\initialtab{0}\class{exercise}]
\tab{\lang{de}{Antworten}\lang{en}{Labels of the graphs}}
	\begin{table}[\class{items}]
	\textcolor{#00CC00}{$k(x)$} 
		\lang{de}{Der grüne Graph gehört zur Funktion $k(x)= \tan(x+\pi)$.}
		\lang{en}{The graph of the function $k(x) = \tan(x+\pi)$ is green.}		
		\\
	\textcolor{#0066CC}{$h(x)$} 
		\lang{de}{Der blaue Graph gehört zur Funktion $h(x)= \sin\left(x+\frac{\pi}{2}\right)$.} 
		\lang{en}{The graph of the function $h(x)= \sin\left(x+\frac{\pi}{2}\right)$ is blue.}
		\\
	\textcolor{#CC6600}{$g(x)$} 
		\lang{de}{Der orangene Graph gehört zur Funktion $g(x)=\cos\left(\frac{5}{2}\pi-x\right)$.} 
		\lang{en}{The graph of the function $g(x)=\cos\left(\frac{5}{2}\pi-x\right)$ is orange.}
		\\
	\end{table}

\tab{\lang{de}{Lösung grüner Graph}\lang{en}{Green Graph}}
	\begin{incremental}[\initialsteps{1}]
	\step 
	%Begin{cosh}
		\lang{de}{Da der Tangens 
        \ref[content_19_allgemeiner_sinus_cosinus][$\pi$-periodisch]{tan_eigenschaften} ist,
        gilt für die Funktion $k(x)= \tan(x+\pi) =\tan(x)$.}
		\lang{en}{Because tangens is $\pi$-periodic, we have: $k(x)= \tan(x+\pi) =\tan(x)$. }
	\step 
	%End{cosh}
		\lang{de}{Der grüne Graph ist der Graph des Tangens und damit der Graph von $k(x)$.}
		\lang{en}{The green graph is the graph of tangens and hence the graph of $k(x)$.}	
	\end{incremental}

\tab{\lang{de}{Lösung orangener Graph}\lang{en}{Red Graph}}
	\begin{incremental}[\initialsteps{1}]
	    \step 
		    \lang{de}{Da der Kosinus \ref[content_19_allgemeiner_sinus_cosinus][symmetrisch zur y-Achse]{sin_cos_symmetry}
            ist, gilt $g(x)= \cos\left(\frac{5}{2}\pi-x\right)=\cos\left(x-\frac{5}{2}\pi\right)$, also ist
		    der Kosinus um $\frac{5}{2}\pi=2\pi+\frac{\pi}{2}$ nach rechts verschoben.}
		    \lang{en}{Because cosine is an even function, $g(x)= \cos\left(\frac{5}{2}\pi-x\right)=\cos\left(x-\frac{5}{2}\pi\right)$, hence cosine has been shifted
		    $\frac{5}{2}\pi=2\pi+\frac{\pi}{2}$ units to the right.}
	    \step 
		    \lang{de}{Da die Kosinusfunktion \ref[content_19_allgemeiner_sinus_cosinus][$2\pi$-periodisch]{sin_cos_periodic} ist, 
            % ist das dasselbe, als verschiebe man nur um $\frac{\pi}{2}$ nach rechts,
            gilt
		    $\; \cos\left(x-\frac{5}{2}\pi\right)=\cos\left(x-\frac{\pi}{2}\right)$.}
		    \lang{en}{Because cosine is $2\pi$-periodic, this is the same as shifting by only $\frac{\pi}{2}$ to the right, $\cos\left(x-\frac{5}{2}\pi\right)=\cos\left(x-\frac{\pi}{2}\right)$.}	
	    \step 
		    \lang{de}{Nach der \ref[content_19_allgemeiner_sinus_cosinus][Regel zur Verschiebung zwischen Sinus und Kosinus]{sin_cos_verschiebung} gilt: 
                      $\cos\left(x-\frac{\pi}{2}\right)=\sin(x)$.}
		    \lang{en}{We have: $\cos\left(x-\frac{\pi}{2}\right)=\sin(x)$.}	
	    \step 
		    \lang{de}{Der rote Graph der Sinusfunktion ist deshalb der Graph von $g(x)$.}
		    \lang{en}{The red graph of sine is the graph of $g(x)$.}
	\end{incremental}

\tab{\lang{de}{Lösung blauer Graph}\lang{en}{Blue Graph}}
	\begin{incremental}[\initialsteps{1}]
	    \step 
		    \lang{de}{Die Funktion $h(x)= \sin\left(x+\frac{\pi}{2}\right)$ ist der um $\frac{\pi}{2}$ nach links verschobene Sinus.}
		    \lang{en}{The function $h(x)= \sin\left(x+\frac{\pi}{2}\right)$ is a sine function shifted $\frac{\pi}{2}$ to the left.}	
	    \step 
		    \lang{de}{Nach der \ref[content_19_allgemeiner_sinus_cosinus][Regel zur Verschiebung zwischen Sinus und Kosinus]{sin_cos_verschiebung} gilt: 
                        $\; \sin\left(x+\frac{\pi}{2}\right)=\cos(x)$.}
		    \lang{en}{We have: $\sin\left(x+\frac{\pi}{2}\right)=\cos(x)$.}	
	    \step 
		    \lang{de}{Der blaue Graph des Kosinus ist also der Graph von $h(x)$.}
		    \lang{en}{The blue graph of cosine is the graph of $h(x)$.}
	\end{incremental}

   %\tab{\lang{de}{Video: ähnliche Übungsaufgabe}}	
   %\youtubevideo[500][300]{g2AsEF2qxlo}\\

\end{tabs*}

\end{content}