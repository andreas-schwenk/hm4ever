\documentclass{mumie.element.exercise}
%$Id$
\begin{metainfo}
  \name{
    \lang{de}{Ü02: Geometrische Probleme}
    \lang{en}{Exercise 2}
  }
  \begin{description} 
 This work is licensed under the Creative Commons License Attribution 4.0 International (CC-BY 4.0)   
 https://creativecommons.org/licenses/by/4.0/legalcode 

    \lang{de}{Hier die Beschreibung}
    \lang{en}{}
  \end{description}
  \begin{components}
\component{generic_image}{content/rwth/HM1/images/g_tkz_T105_Exercise02.meta.xml}{T105_Exercise02}
\end{components}
  \begin{links}
  \end{links}
  \creategeneric
\end{metainfo}
\begin{content}

\title{
\lang{de}{Ü02: Geometrische Probleme}
}

\begin{block}[annotation]
  Im Ticket-System: \href{http://team.mumie.net/issues/9719}{Ticket 9719}
\end{block}

 
 Gegeben sei ein gleichschenkliges, rechtwinkliges Dreieck. Die Länge der Hypotenuse des Dreiecks betrage 4 cm. 
 Bestimmen Sie die Länge der Katheten (in cm und auf zwei Stellen nach dem Komma gerundet).
 
 \begin{tabs*}[\initialtab{0}\class{exercise}]
  \tab{
  \lang{de}{Antwort}
  \lang{en}{Answer}
  }
$2\sqrt{2}$ cm $\approx 2,83$ cm
  \tab{
  \lang{de}{Lösung}
  \lang{en}{}
  }
Unter Berücksichtigung, dass ein gleichschenkliges Dreieck vorliegt, 
die beiden Katheten also gleich lang sind, bezeichne $x$ die Länge der Katheten (in cm).
Nach dem Satz des Pythagoras gilt dann
  \[
   x^2+x^2=4^2 \iff 2x^2=16 \iff x^2=8.
  \]
Es gilt also $x=\sqrt{8}=2\sqrt{2} \approx 2,83$.  
\begin{figure}
    \image{T105_Exercise02}
\end{figure}

\end{tabs*}
\end{content}