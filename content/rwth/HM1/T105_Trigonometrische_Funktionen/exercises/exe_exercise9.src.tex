\documentclass{mumie.element.exercise}
%$Id$
\begin{metainfo}
  \name{
    \lang{en}{...}
    \lang{de}{Ü09: Trigonometrische Gleichungen}
  }
  \begin{description} 
 This work is licensed under the Creative Commons License Attribution 4.0 International (CC-BY 4.0)   
 https://creativecommons.org/licenses/by/4.0/legalcode 

    \lang{en}{...}
    \lang{de}{...}
  \end{description}
  \begin{components}
\component{generic_image}{content/rwth/HM1/images/g_tkz_T105_Exercise09_D.meta.xml}{T105_Exercise09_D}
\component{generic_image}{content/rwth/HM1/images/g_tkz_T105_Exercise09_C.meta.xml}{T105_Exercise09_C}
\component{generic_image}{content/rwth/HM1/images/g_tkz_T105_Exercise09_B.meta.xml}{T105_Exercise09_B}
\component{generic_image}{content/rwth/HM1/images/g_tkz_T105_Exercise09_A.meta.xml}{T105_Exercise09_A}
\end{components}
  \begin{links}
\link{generic_article}{content/rwth/HM1/T105_Trigonometrische_Funktionen/g_art_content_19_allgemeiner_sinus_cosinus.meta.xml}{content_19_allgemeiner_sinus_cosinus}
\end{links}
  \creategeneric
\end{metainfo}
\begin{content}
\begin{block}[annotation]
	Im Ticket-System: \href{https://team.mumie.net/issues/21864}{Ticket 21864}
\end{block}
    \title{Ü09: Trigonometrische Gleichungen}
    Geben Sie \textbf{alle} Lösungen $x\in \R$ für die folgenden Gleichungen an.
    \begin{itemize}
        \item[a)]$\sin(x)=\frac{1}{2}$,
        \item[b)]$\cos^2(x)=\frac{3}{4}$,
        \item[c)]$\sin(2x)=\frac{\sqrt{3}}{2}$,
        \item[d)]$\sin^2(x)=\frac{3}{2}\cdot\cos(x)$.
    \end{itemize}
    
    \begin{tabs*}[\initialtab{0}\class{exercise}]
      \tab{Antworten}
      a)\begin{align*}
                x_{1,k}&=\pi/6+2\pi\cdot k\\
                x_{2,k}&=5\pi/6+2\pi\cdot k,\quad k\in\mathbb{Z}.
            \end{align*}
      b)\begin{align*}
                x_{1,k}&=\pi/6+\pi\cdot k\\
                x_{2,k}&=5\pi/6+\pi\cdot k,\quad k\in\mathbb{Z}.
            \end{align*}
      c)\begin{align*}
                x_{1,k}&=\pi/6+\pi\cdot k\\
                x_{2,k}&=\pi/3+\pi\cdot k,\quad k\in\mathbb{Z}.
            \end{align*}
      d)\begin{align*}
                x_{1,k}&=\pi/3+2\pi\cdot k\\
                x_{2,k}&=5\pi/3+2\pi\cdot k,\quad k\in\mathbb{Z}.
            \end{align*}
      
      \tab{Lösung a)}
        \begin{incremental}[\initialsteps{1}] 
          \step Die gegebene Gleichung lässt sich mit Hilfe des Arcus-Sinus nach $x$ umstellen und kann mit dem Taschenrechner (TR) ausgewertet werden
            \begin{align*}
                x_{TR}=\arcsin(1/2)=\frac{\pi}{6}=30^{\circ}.
            \end{align*}
          \step Dies ist erst einmal nur eine Lösung. Weitere Lösungen ergeben sich durch die 
          \ref[content_19_allgemeiner_sinus_cosinus][Periodizität]{sin_cos_periodic}, also
          \[
          x_{1,k}=\frac{\pi}{6}+2\pi\cdot k
          \]
          für alle $k\in \Z$.
          \step Nun gibt es im Intervall $[0,2\pi]$ noch eine weitere Lösung, wie man anhand der untenstehenden Skizze erkennen kann.\\
          
          Nach der \ref[content_19_allgemeiner_sinus_cosinus][Regel zur Verschiebung zwischen Sinus und Kosinus]{sin_cos_verschiebung} 
          und der \ref[content_19_allgemeiner_sinus_cosinus][Symmetrie]{sin_cos_symmetry} des Kosinus gilt: 
          \[
      \sin(\frac{\pi}{6})=\cos(\frac{\pi}{6}-\frac{\pi}{2})=\cos(-\frac{\pi}{6}+\frac{\pi}{2})=\sin(-\frac{\pi}{6}+\pi)
      \]
      Mit der \ref[content_19_allgemeiner_sinus_cosinus][Periodizität]{sin_cos_periodic}
      ergeben sich die restlichen Lösungen
      \[
      x_{2,k}=5\pi/6+2\pi\cdot k,\quad k\in\mathbb{Z}.
      \]
            \begin{figure}
              \image{T105_Exercise09_A}
            \end{figure}
  
        \end{incremental}
        
      \tab{Lösung b)}
        \begin{incremental}[\initialsteps{1}]   
          \step Die Gleichung wird nach $x$ umgestellt (beachte $\pm$ beim Wurzelziehen) und mit dem Taschenrechner ausgewertet
            \begin{align*}
                               &\quad \cos^2(x)&\;=\frac{3}{4}\\
              \Leftrightarrow  &\quad |\cos(x)|&\;=\frac{\sqrt{3}}{2}\\
              \Leftrightarrow  &\quad  \cos(x)&\;=\pm\frac{\sqrt{3}}{2}\\
              \Leftarrow  &\quad x_{1,0}&\;=\arccos(\sqrt{3}/2)=\frac{\pi}{6}=30^{\circ} \\
              \text{und}  & \quad  x_{2,0}&\; =\arccos(-\sqrt{3}/2)=\frac{5\pi}{6}=150^{\circ}.
             \end{align*}
          \step Wir haben also zwei Lösungen schon gefunden. Nun sind aber im Intervall $[0;2\pi]$ insgesamt vier Lösungen zu erwarten. 
          Diese erhalten wir durch die
          \ref[content_19_allgemeiner_sinus_cosinus][Symmetrie-Eigenschaft]{sin_cos_symmetry} des Kosinus.
%          \[
%          \cos(x)=\cos(-x).
%          \]
          \step Es ist
          \[
          \cos(\frac{\pi}{6})=\cos(-\frac{\pi}{6}) \text{ sowie } \cos(\frac{5\pi}{6})=\cos(-\frac{5\pi}{6}).
          \]
          Nun bemerke, dass
          \[
          \frac{\pi}{6} - \pi = -\frac{5\pi}{6} \text{ sowie } -\frac{\pi}{6}+\pi = \frac{5\pi}{6}.
          \]
          Deshalb können wir die vier Lösungen im Intervall $[0;2\pi]$ und alle weiteren Lösungen auf $\R$ kompakt
          schreiben:
          \begin{align*}
                x_{1,k}&=\pi/6+\pi\cdot k\\
                x_{2,k}&=5\pi/6+\pi\cdot k,\quad k\in\mathbb{Z}.
            \end{align*}
            
          \begin{figure}
              \image{T105_Exercise09_B}
            \end{figure}
        \end{incremental}
        
      \tab{Lösung c)}
        \begin{incremental}[\initialsteps{1}]
          \step Wir starten mit der Substitutionsvorschrift $u=2x$, erhalten
            \begin{align*}
                \sin(u)=\sqrt{3}/2
            \end{align*}
          und berechnen zunächst die Lösung in $u$.
          \step Die Gleichung wird nach $u$ umgestellt und mit dem Taschenrechner ausgewertet
            \begin{align*}
                u_{1,0}=\arcsin(\sqrt{3}/2)=\frac{\pi}{3}=60^{\circ}.
            \end{align*}
          \step Weitere Lösungen sind wegen der 
          \ref[content_19_allgemeiner_sinus_cosinus][Periodizität]{sin_cos_periodic}
          \[
                          u_{1,k}=\frac{\pi}{3}+2\pi k, \, k \in \Z.
          \]
          \step Es sind aber noch weitere Lösungen zu erwarten.
          Nach der \ref[content_19_allgemeiner_sinus_cosinus][Regel zur Verschiebung zwischen Sinus und Kosinus]{sin_cos_verschiebung} 
          und der \ref[content_19_allgemeiner_sinus_cosinus][Symmetrie]{sin_cos_symmetry} des Kosinus gilt: 
%          \ref[content_19_allgemeiner_sinus_cosinus][Es ist]{sin_cos_verschiebung}
          \[
          \sin(\frac{\pi}{3})=\cos(\frac{\pi}{3}-\frac{\pi}{2})=\cos(-\frac{\pi}{3}+\frac{\pi}{2})=\sin(-\frac{\pi}{3}+\pi).
          \]
          Die weiteren Lösungen sind also
          \begin{align*}
                u_{2,k}=2\pi/3+2\pi\cdot k,\quad k\in\mathbb{Z}.
            \end{align*}
          \begin{figure}
              \image{T105_Exercise09_C}
            \end{figure}
     
          \step Die Lösung in $x$ ergibt sich durch Anwendung der Resubstitutionsvorschrift $x=u/2$
            \begin{align*}
                x_{1,k}&=u_{1,k}/2=\pi/6+\pi\cdot k\\
                x_{2,k}&=u_{2,k}/2=\pi/3+\pi\cdot k,\quad k\in\mathbb{Z}
            \end{align*}
        \end{incremental}
        
      \tab{Lösung d)}
        \begin{incremental}[\initialsteps{1}]
          \step In einem ersten Schritt wird die Aufgabenstellung unter Berücksichtigung des
          \ref[content_19_allgemeiner_sinus_cosinus][triginometrischen Pyhtagoras]{trig-pythagoras}
          $\; \sin^2(x)+\cos^2(x)=1$ bzw. $\sin^2(x)=1-\cos^2(x) \;$ 
          so umgeformt, dass nur noch Kosinus-Ausdrücke vorkommen:
            \begin{align*}
                \sin^2(x)&=3/2\cdot \cos(x)\\
				\Leftrightarrow 1-\cos^2(x)&=3/2\cdot \cos(x).
            \end{align*}
          \step Die neue Gleichung kann umgestellt werden:
		  \begin{align*}
			\cos^2(x)+\frac{3}{2}\cdot\cos(x)-1=0.
		  \end{align*}
          Wir substituieren $u=\cos(x)$
		  und mit Hilfe der pq-Formel lösen wir
		  \begin{align*}
			u_{1,2}&=-\frac{3}{4}\pm\sqrt{\frac{9}{16}+1}\\
			\Leftrightarrow u_1&=1/2\quad\text{und}\quad u_2=-2.			
		  \end{align*}
		  \step Wir haben gezeigt, dass die ursprüngliche Fragestellung $\sin^2(x)=3/2\cdot \cos(x)$ die gleichen Lösungen besitzt wie die zwei Teilprobleme $\cos(x)=1/2$ und $\cos(x)=-2$. Die Bestimmung der Lösungen der beiden Teilprobleme erfolgt analog zu den vorherigen Aufgabenstellungen. 
          \step Unter Berücksichtigung der \ref[content_19_allgemeiner_sinus_cosinus][Wertemenge des Kosinus]{def_trig1}
                $W_{\cos(x)}=[-1,1]$ wird zudem deutlich, dass die Gleichung $\cos(x)=-2$ keine Lösungen besitzt. Wir können uns auf die Lösung des Teilproblems $\cos(x)=1/2$ konzentrieren.
          \step Die erste Lösung wird mit Hilfe des Taschenrechners bestimmt
		    \begin{align*}
				x_{1,0}=\arccos(1/2)=\pi/3=60^{\circ}.
		    \end{align*}
          \step Die weiteren Lösungen können unter Ausnutzung der \ref[content_19_allgemeiner_sinus_cosinus][Symmetrie]{sin_cos_symmetry}
            bzw. \ref[content_19_allgemeiner_sinus_cosinus][Periodizität]{sin_cos_periodic}
            des Kosinus bestimmt werden.
            \begin{align*}
                x_{1,k}=\pi/3+2\pi\cdot k,\quad k\in\mathbb{Z}.
            \end{align*}
            Weiterhin
            \[\cos(\frac{\pi}{3})=\cos(-\frac{\pi}{3}) = \cos(-\frac{\pi}{3}+2\pi)=\cos(\frac{5\pi}{3}).\]
           Es folgt
            \begin{align*}
                x_{2,k}=5\pi/3+2\pi\cdot k,\quad k\in\mathbb{Z}.
            \end{align*}
            \begin{figure}
             \image{T105_Exercise09_D}
            \end{figure}
        \end{incremental}
    \end{tabs*}
\end{content}

