\documentclass{mumie.element.exercise}
%$Id$
\begin{metainfo}
  \name{
    \lang{en}{...}
    \lang{de}{Ü10: Geometrische Probleme}
  }
  \begin{description} 
 This work is licensed under the Creative Commons License Attribution 4.0 International (CC-BY 4.0)   
 https://creativecommons.org/licenses/by/4.0/legalcode 

    \lang{en}{...}
    \lang{de}{...}
  \end{description}
  \begin{components}
\component{generic_image}{content/rwth/HM1/images/g_tkz_T105_Exercise10_C.meta.xml}{T105_Exercise10_C}
\component{generic_image}{content/rwth/HM1/images/g_tkz_T105_Exercise10_D.meta.xml}{T105_Exercise10_D}
\component{generic_image}{content/rwth/HM1/images/g_tkz_T105_Exercise10_B.meta.xml}{T105_Exercise10_B}
\component{generic_image}{content/rwth/HM1/images/g_tkz_T105_Exercise10_A.meta.xml}{T105_Exercise10_A}
\end{components}
  \begin{links}
  \end{links}
  \creategeneric
\end{metainfo}
\begin{content}
\begin{block}[annotation]
	Im Ticket-System: \href{https://team.mumie.net/issues/21863}{Ticket 21863}
\end{block}
    \title{Ü10: Geometrische Probleme}
    Gegeben sind die folgenden geometrischen Körper:
    \begin{figure}
      \image{T105_Exercise10_A}
    \end{figure}
    Lösen Sie die folgenden Problemstellungen:
    \begin{itemize}
        \item[a)] Leiten sie eine allgemeine Formel zur Berechnung der Strecke $s$ in Abhängigkeit von Radius $r$ und Winkel $\alpha$ für den gegebenen Kreis her.  
        \item[b)] Berechnen Sie den Winkel $\alpha$ unter der Annahme, dass der Umfang der Kegelgrundfläche und dessen Höhe identisch sind.
        \item[c)] Berechnen Sie die Höhe $h$ der gegebenen quadratischen Pyramide (mit Hilfe eines Taschenrechners). Die Spitze befindet sich senkrecht über den Mittelpunkt der $1600\,m^2$ großen Grundfläche. Der eingezeichnete Winkel beträgt $\alpha=60^{\circ}$.
    \end{itemize}
    
    \begin{tabs*}[\initialtab{0}\class{exercise}]
      \tab{Antworten}
       a) $s=2r\cdot\sin(\frac{\alpha}{2})$,\\
       b) $\alpha=\arctan(2\pi)\approx80,96^{\circ}$,\\
       c) $h\approx48,99\,m$.
        
      \tab{Lösung a)}
        Im unten in grün dargestellten rechtwinkligen Dreieck gelten die bekannten trigonometrischen Zusammenhänge  
            \begin{align*}
              \sin(\frac{\alpha}{2})=\frac{s/2}{r} \quad\Leftrightarrow\quad s=2r\cdot\sin(\alpha/2)
            \end{align*}
            \begin{figure}
                 \image{T105_Exercise10_B}
            \end{figure}
   

      \tab{Lösung b)}
        Wir betrachten das unten in orange dargestellte rechtwinklige Dreieck. Der Radius $r$ der Kegelgrundfläche ergibt 
        sich aus dem Umfang $U$ durch $r=\frac{U}{2\pi}$. Unter Berücksichtigung der Vorgabe $U=h$ erhalten wir schließlich
            \begin{align*}
              \tan(\alpha)=\frac{h}{r}=\frac{h}{h/(2\pi)}=2\pi\quad\Leftrightarrow\quad \alpha=\arctan(2\pi).
            \end{align*}
            
            \textit{Hinweis:} Im letzten Schritt haben wir eine Äquivalenzumformung, da wir den Innenwinkel $\alpha$ eines 
            Dreiecks suchen, für den $0<\alpha<\frac{\pi}{2}$ gilt. Damit sind alle weiteren Lösungen ausgeschlossen.            
            \begin{figure}
                 \image{T105_Exercise10_C}
            \end{figure}          
    

      \tab{Lösung c)}
        \begin{incremental}[\initialsteps{1}]
            \step Aus der gegebenen Grundfläche wird zunächst die Grundseitenlänge $a$ berechnet:            
            \begin{align*}
              a=\sqrt{1600\,m^2}=40\,m.
            \end{align*}  
            \step Die Diagonale $d$ der Grundfläche ergibt sich nach Pythagoras aus          
            \begin{align*}
              d=\sqrt{a^2+a^2}\approx56,57\,m.
            \end{align*}
          \step Wir betrachten das unten in orange eingezeichnete rechtwinklige Dreieck und verwenden zur Berechnung der gesuchten Höhe $h$ 
                die bekannten trigonometrischen Zusammenhänge
            \begin{align*}
              \tan(\alpha)=\frac{h}{d/2}\quad\Leftrightarrow\quad h=\tan(\alpha)\cdot{d/2}.
            \end{align*}
            \begin{figure}
                 \image{T105_Exercise10_D}
            \end{figure}
            \step Durch Einsetzen von $\alpha=60^{\circ}$ und $d=\sqrt{a^2+a^2}\approx56,57\,m$ erhalten wir schließlich $h\approx48,99\,m$.
        
        \end{incremental}
    \end{tabs*}

\end{content}

