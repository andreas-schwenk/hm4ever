\documentclass{mumie.element.exercise}
%$Id$
\begin{metainfo}
  \name{
    \lang{de}{Ü08: Trigonometrische Gleichungen}
    \lang{en}{Exercise 8}
  }
  \begin{description} 
 This work is licensed under the Creative Commons License Attribution 4.0 International (CC-BY 4.0)   
 https://creativecommons.org/licenses/by/4.0/legalcode 

    \lang{de}{Hier die Beschreibung}
    \lang{en}{}
  \end{description}
  \begin{components}
\component{generic_image}{content/rwth/HM1/images/g_tkz_T105_Exercise08_B.meta.xml}{T105_Exercise08_B}
\component{generic_image}{content/rwth/HM1/images/g_tkz_T105_Exercise08_A.meta.xml}{T105_Exercise08_A}
\end{components}
  \begin{links}
  \end{links}
  \creategeneric
\end{metainfo}
\begin{content}

\title{
\lang{de}{Ü08: Trigonometrische Gleichungen}
}

\begin{block}[annotation]
  Im Ticket-System: \href{http://team.mumie.net/issues/9726}{Ticket 9726}
\end{block}

\begin{table}[\class{items}]
a) Bestimmen Sie alle Lösungen $x\in \R$ von
  \[
   \sin\left(2x- \frac{1}{3}\right)=0.
  \]
b) Bestimmen Sie alle Lösungen $x\in \R$ von
 \[
   \cot\left(\frac{1}{2}x+\frac{3\pi}{8}\right)=0.
  \]
 \end{table}


 \begin{tabs*}[\initialtab{0}\class{exercise}]
  \tab{
  \lang{de}{Lösung a)}
  }
\begin{incremental}[\initialsteps{1}]
\step Alle Nullstellen von $\sin y$ sind gegeben durch $y=k\pi$ mit $k\in\mathbb{Z}$.
\step Wir lösen also
    \[
     2x-\frac{1}{3}=k\pi\iff 2x=\frac{1}{3}+k\pi \iff x=\frac{1}{6}+\frac{1}{2}k\pi
    \]
 und erhalten also \[ \mathbb{L}=\{\frac{1}{6}+\frac{1}{2}k\pi \, | \, k \in \Z\}.\]
 \begin{figure}
    \image{T105_Exercise08_A}
 \end{figure}
\end{incremental}

  \tab{
  \lang{de}{Lösung b)}
  }
\begin{incremental}[\initialsteps{1}]
\step Alle Nullstellen von $\cot(y)$ sind gegeben durch $y=\pi/2+k\pi$ mit $k\in\mathbb{Z}$.
\step Wir lösen also
    \[
     \frac{1}{2}x+\frac{3\pi}{8}=\frac{\pi}{2}+k\pi\iff \frac{1}{2}x=-\frac{3\pi}{8}+\frac{\pi}{2}+k\pi \iff\frac{1}{2}x=\frac{\pi}{8}+k\pi \iff x=\frac{\pi}{4}+2k\pi.
    \]
\step Damit erhalten wir \[ \mathbb{L}=\{ \frac{\pi}{4}+2k\pi \, | \, k \in \Z\}\] als die gesuchte Lösungsmenge.\\
 \begin{figure}
    \image{T105_Exercise08_B}
 \end{figure}
\end{incremental}
\end{tabs*}


\end{content}