\documentclass{mumie.element.exercise}
%$Id$
\begin{metainfo}
  \name{
    \lang{de}{Ü04: Funktionsgraphen}
    \lang{en}{Exercise 4}
  }
  \begin{description} 
 This work is licensed under the Creative Commons License Attribution 4.0 International (CC-BY 4.0)   
 https://creativecommons.org/licenses/by/4.0/legalcode 

    \lang{de}{Hier die Beschreibung}
    \lang{en}{}
  \end{description}
  \begin{components}
  	\component{generic_image}{content/rwth/HM1/images/g_tkz_T105_Exercise04_A.meta.xml}{T105_Exercise04_A}
    \component{generic_image}{content/rwth/HM1/images/g_tkz_T105_Exercise04_B.meta.xml}{T105_Exercise04_B}
  \end{components}
  \begin{links}
  \end{links}
  \creategeneric
\end{metainfo}
\begin{content}

\title{\lang{de}{Ü04: Funktionsgraphen}}

\begin{block}[annotation]
  Im Ticket-System: \href{http://team.mumie.net/issues/9722}{Ticket 9722}
\end{block}
\begin{itemize}
    \item[1)] Zeichnen Sie die Funktionsgraphen zur Sinus- und Kosinus-Funktion und markieren sie darin die Koordinaten für die folgenden Stellen der x-Achse $0,\, \frac{\pi}{6},\, \frac{\pi}{4},\, \frac{\pi}{3},\, \frac{\pi}{2}$.
    \item[2)] \lang{de}{In der nachfolgenden Abbildung sind in orangener, grüner und blauer Farbe die Graphen von Winkelfunktionen dargestellt.}
                \lang{en}{In \lref{Trigonometrische_Funktionen_Uebung2_1}{Figure 1} there are three graphs of trigonometric functions in orange, blue and green.}
                \\
                \label{Trigonometrische_Funktionen_Uebung2_1}
                \begin{figure} 
                       \image{T105_Exercise04_A}                 
                \end{figure}
                \begin{table}[\class{items}]
                  \nowrap{ $g(x)= \sin(x)$} &
                  \nowrap{ $h(x)= \cos(x)$} &
                  \nowrap{ $j(x)= \tan(x)$} & 
                \end{table}
                \lang{de}{Ordnen Sie den drei Graphen jeweils die zugehörige Funktionsvorschrift zu.}
                \lang{en}{Match each of the three functions to their corresponding graph.}
\end{itemize}
 




\begin{tabs*}[\initialtab{0}\class{exercise}]
\tab{\lang{de}{Antworten}\lang{en}{Answers}}
	\lang{de}{\begin{table}[\class{items}]
        1)   \begin{figure}\image{T105_Exercise04_B}\end{figure}
	  	2a) Der orange Graph gehört zur Funktion \textcolor{#CC6600}{$g(x)= \sin(x)$}.\\ 	
        2b) Der blaue Graph gehört zur Funktion \textcolor{#0066CC}{$h(x)= \cos(x)$}.\\ 
	    2c) Der grüne Graph gehört zur Funktion \textcolor{#00CC00}{$j(x)=\tan(x)$}. \\ 
	    		
	\end{table}}
    
    \lang{en}{\begin{table}[\class{items}]
        1)   \begin{figure}\image{T105_Exercise04_B}\end{figure}
        2a) The graph of the function \textcolor{#CC6600}{$g(x)= \sin(x)\,$} is orange.\\
        2b) The graph of the function \textcolor{#0066CC}{$h(x) = \cos(x)\,$} is blue.\\
        2c) The graph of the function \textcolor{#00CC00}{$j(x)=\tan(x)\,$} is green. \\ 
    \end{table}}
    
\tab{\lang{de}{Lösungsvideo 1)}}	
   Die Lösung zu Aufgabe 1 ist dem folgenden Video zu entnehmen \youtubevideo[500][300]{tss5Y5x5kw8}\\
    
\tab{\lang{de}{Lösung 2) Orangener Graph}\lang{en}{Orange Graph}}
	\begin{incremental}[\initialsteps{1}]
	    \step 
		    \lang{de}{Die Funktion $g(x)= \sin(x)\,$ ist $2\pi$-periodisch.}
		    \lang{en}{The function $g(x) = \sin(x)\,$ is $2\pi$-periodic.}	
	    \step 
		    \lang{de}{Die Nullstellen liegen bei den ganzzahligen Vielfachen der Kreiszahl $\pi$.}
		    \lang{en}{The roots are at multiples of $\pi$.}	
	    \step 
		    \lang{de}{Alle Funktionswerte liegen zwischen $-1$ und $+1$.}
		    \lang{en}{The function takes on values between $-1$ and $+1$.}		
	    \step 
		    \lang{de}{Die Extremstellen sind die Nullstellen der $\cos$-Funktion, nämlich $x_k=(2k+1)\cdot\frac{\pi}{2}$ für $k \in Z$.}
		    \lang{en}{The extrema are the roots of cosine, namely $x_k=(2k+1)\cdot\frac{\pi}{2}$ , $k \in Z$. }		
	\end{incremental}

\tab{\lang{de}{Lösung 2) blauer Graph}\lang{en}{Blue Graph}}
	\begin{incremental}[\initialsteps{1}]
	    \step 
		    \lang{de}{Die Funktion $h(x)= \cos(x)\,$ ist $2\pi$-periodisch.}
		    \lang{en}{The function $h(x)= \cos(x)\,$ is $2\pi$-periodic.}		
	    \step 
		    \lang{de}{Die Nullstellen liegen bei $x_k=(2k+1)\cdot\frac{\pi}{2}=\frac{\pi}{2}+k\pi$ für $k \in Z$.}
		    \lang{en}{The roots are at points $x_k=(2k+1)\cdot\frac{\pi}{2}=\frac{\pi}{2}+k\pi$ , $k \in Z$.}	
	    \step 
		    \lang{de}{Alle Funktionswerte liegen zwischen $-1$ und $+1$.}
		    \lang{en}{The function takes on values between $-1$ and $+1$.}	
	    \step 
		    \lang{de}{Die Extremstellen sind die ganzzahligen Vielfachen der Kreiszahl $\pi$.}
		    \lang{en}{The extrema are at multiples of $\pi$.}	
	\end{incremental}
	
\tab{\lang{de}{Lösung 2) grüner Graph}\lang{en}{Green Graph}}
	\begin{incremental}[\initialsteps{1}]
	    \step 
		    \lang{de}{Die Funktion $j(x)= \tan(x)\,$ ist der Quotient $\sin(x)$ durch $\cos(x)$.}
		    \lang{en}{The function $j(x)= \tan(x)\,$ is the quotient $\sin(x)$ divided by $\cos(x)$.}	
	    %\step 
		    %\lang{de}{Daher ist der Tangens an den Nullstellen des Kosinus nicht definiert und hat dort senkrechte Asymptoten.}
		    %\lang{en}{Hence tangent is not defined at the roots of cosine and has vertical asymptotes at those points.}
		    %It's normally bad style to start sentences with "hence" in English, but because of the way this is revealed step-by-step it's not _horrible_ here, but should in general be avoided. 		
	    \step 
		    \lang{de}{Die Definitionslücken liegen folglich bei: $x_k=(2k+1)\cdot\frac{\pi}{2}=\frac{\pi}{2}+k\pi\,$ für $k\in \mathbb{Z}$.}
		    \lang{en}{The asymptotes are at: $x_k=(2k+1)\cdot\frac{\pi}{2}=\frac{\pi}{2}+k\pi\,$ , $k\in \mathbb{Z}$.}	
	    \step 
		    \lang{de}{Der Tangens hat die Nullstellen des Zählers $\sin(x)$ als Nullstellen.}
		    \lang{en}{Tangent has roots where the numerator $\sin(x)$ has roots.}	
	    \step 
		    \lang{de}{Er verläuft monoton steigend zwischen den senkrechten Asymptoten, er ist $\pi$-periodisch.}
		    \lang{en}{It is monotonically increasing between its vertical asymptotes and is $\pi$-periodic.}
	\end{incremental}
\end{tabs*}

\end{content}