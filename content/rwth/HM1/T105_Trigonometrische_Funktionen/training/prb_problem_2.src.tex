\documentclass{mumie.problem.gwtmathlet}
%$Id$
\begin{metainfo}
  \name{
    \lang{de}{A02: Bogen- und Gradmaß}
    \lang{en}{Problem 2}
  }
  \begin{description} 
 This work is licensed under the Creative Commons License Attribution 4.0 International (CC-BY 4.0)   
 https://creativecommons.org/licenses/by/4.0/legalcode 

    \lang{de}{}
    \lang{en}{}
  \end{description}
  \corrector{system/problem/GenericCorrector.meta.xml}
  \begin{components}
    \component{js_lib}{system/problem/GenericMathlet.meta.xml}{mathlet}
    \component{generic_image}{content/rwth/HM1/images/g_img_right-angled-abb1.meta.xml}{image1}
  \end{components}
  \begin{links}
  \end{links}
  \creategeneric
\end{metainfo}
\begin{content}
\usepackage{mumie.ombplus}
\usepackage{mumie.genericproblem}

\lang{de}{
	\title{A02: Bogen- und Gradmaß}
}
\lang{en}{
	\title{Problem 2}
	}

\begin{block}[annotation]
  Im Ticket-System: \href{http://team.mumie.net/issues/9727}{Ticket 9727}
\end{block}

\begin{problem}
\begin{variables}
	\randint{z1}{0}{360}
	\function{s1}{z1/180}
	
	\randint{z2}{1}{9}
	\randint{z3}{1}{9}
	\function{k}{z2/z3}
	\randadjustIf{z2,z3}{k > 2}
	\function{s2}{180*k}
\end{variables}
\begin{question}
    \explanation{Ausgehend von $\pi=180^\circ$ lässt sich diese Aufgabe über einen Dreisatz lösen.}
	\type{input.number}
	\field{rational}
\lang{de}{\text{Gegeben ist der folgende Winkel in Gradmaß. Bestimmen Sie den zugehörigen Winkel in Bogenmaß. 
Geben Sie das Ergebnis als gekürzten Bruch an.\\
$\var{z1}^\circ =$\ansref $\cdot \pi$}}
\begin{answer}
	\solution{s1}
\end{answer}
\end{question}


\begin{question}
    \explanation{Ausgehend von $\pi=180^\circ$ lässt sich diese Aufgabe über einen Dreisatz lösen.}
	\type{input.number}
	\field{rational}
\lang{de}{\text{Gegeben ist der folgende Winkel in Bogenmaß. Bestimmen Sie den zugehörigen Winkel in Gradmaß. 
Geben Sie das Ergebnis als gekürzten Bruch an.
$\var{k}\pi= x^\circ$}}
\begin{answer}
	\text{$x=$}
	\solution{s2}
\end{answer}
\end{question}

\end{problem}

\embedmathlet{mathlet}
%\embedapplet{applet}

\end{content}