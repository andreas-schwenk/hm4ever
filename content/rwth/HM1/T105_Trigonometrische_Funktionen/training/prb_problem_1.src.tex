\documentclass{mumie.problem.gwtmathlet}
%$Id$
\begin{metainfo}
  \name{
    \lang{de}{A01: Geometrische Probleme}
    \lang{en}{problem_1}
  }
  \begin{description} 
 This work is licensed under the Creative Commons License Attribution 4.0 International (CC-BY 4.0)   
 https://creativecommons.org/licenses/by/4.0/legalcode 

    \lang{de}{}
    \lang{en}{}
  \end{description}
  \corrector{system/problem/GenericCorrector.meta.xml}
  \begin{components}
    \component{js_lib}{system/problem/GenericMathlet.meta.xml}{mathlet}
  \end{components}
  \begin{links}
  \end{links}
  \creategeneric
\end{metainfo}
\begin{content}
\usepackage{mumie.ombplus}
\usepackage{mumie.genericproblem}


\lang{de}{
	\title{A01: Geometrische Probleme}
}
\lang{en}{
	\title{Problem 1}
}
\begin{block}[annotation]
  Im Ticket-System: \href{http://team.mumie.net/issues/9157}{Ticket 9157}
\end{block}

\begin{problem}
 	\begin{variables}
      		\randint{a}{1}{10}
      		\randint{b}{1}{10}
      		\function[calculate]{z}{a^2+b^2}
      		\function{zz}{sqrt(z)}
     \end{variables}
     \begin{question}
     \explanation{Zur Lösung der Aufgabe kann der Satz des Pythagoras herangezogen werden.}
    	\type{input.function}
    	\field{real}
		\lang{de}{\text{In einem rechtwinkligen Dreieck sind die Kathetenlängen $\var{a}$ cm und $\var{b}$ cm gegeben.
Bestimmen Sie die Länge der Hypotenuse des Dreiecks (in cm). (Geben Sie für die Wurzel sqrt() ein.)}}
		\begin{answer}
		    \text{Länge der Hypothenuse in cm:}
		    \solution{zz}
		    \inputAsFunction{x}{ax} 
            \checkFuncForZero{ax-zz}{-10}{10}{100}
		\end{answer}
     \end{question}
 \end{problem}


\embedmathlet{mathlet}
\end{content}