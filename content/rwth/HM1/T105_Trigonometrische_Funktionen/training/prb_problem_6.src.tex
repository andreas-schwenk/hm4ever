\documentclass{mumie.problem.gwtmathlet}
%$Id$
\begin{metainfo}
  \name{
    \lang{de}{A06: Funktionseigenschaften}
    \lang{en}{Problem 6}
  }
  \begin{description} 
 This work is licensed under the Creative Commons License Attribution 4.0 International (CC-BY 4.0)   
 https://creativecommons.org/licenses/by/4.0/legalcode 

    \lang{de}{}
    \lang{en}{}
  \end{description}
  \corrector{system/problem/GenericCorrector.meta.xml}
  \begin{components}
    \component{js_lib}{system/problem/GenericMathlet.meta.xml}{mathlet}
    \component{generic_image}{content/rwth/HM1/images/g_img_right-angled-abb1.meta.xml}{image1}
  \end{components}
  \begin{links}
  \end{links}
  \creategeneric
\end{metainfo}
\begin{content}
\usepackage{mumie.ombplus}
\usepackage{mumie.genericproblem}

\lang{de}{
	\title{A06: Funktionseigenschaften}
}
\lang{en}{
	\title{Problem 6}
	}

\begin{block}[annotation]
  Im Ticket-System: \href{http://team.mumie.net/issues/9731}{Ticket 9731}
\end{block}


\begin{problem}

	\begin{variables} % global variables for pool
		\randint{c}{0}{1}c*cos(x)
		\randint{k0}{-3}{3}
		\randint[Z]{k1}{-1}{1}
		\function[calculate]{k2}{2+k1}
		\function[calculate]{k3}{k2/4} % 1/4 or 3/4
	\end{variables}
		
		
%Frage 1 von 2
\begin{question}
\explanation{Berücksichtigen Sie in welchen Intervallen die Funktionswerte der trigonometrischen Funktionen größer bzw. kleiner als Null sind.\\
Für Sinus und Kosinus haben diese Intervalle eine Länge von $\pi$.}
	\begin{variables}
		\function[normalize]{f}{c*cos(x)+(1-c)*sin(x)} % c=1: cos, c=0: sin
		\function[calculate]{fpl0}{-c/2+2*k0}
		\function[normalize]{fpl}{fpl0*pi} % f positive, left boundary
		\function[calculate]{fpr0}{fpl0+1}
		\function[normalize]{fpr}{fpr0*pi} % f positive, right boundary
		\function[calculate]{fpp0}{fpl0+k3}
		\function[normalize]{fpp}{fpp0*pi} % f positive, point in interval
		\function[normalize]{g}{(1-c)*cos(x)+c*sin(x)} % c=0: cos, c=1: sin
		\function[calculate]{gnl0}{(1+c)/2+2*k0}
		\function[normalize]{gnl}{gnl0*pi} % g negative, left boundary
		\function[calculate]{gnr0}{gnl0+1}
		\function[normalize]{gnr}{gnr0*pi} % g negative, right boundary
		\function[calculate]{gnp0}{gnl0+k3}
		\function[normalize]{gnp}{gnp0*pi} % g negative, point in interval
	\end{variables}

    \type{input.function}
    \field{rational}
\lang{de}{
	\text{Geben Sie das größte Intervall {\Large I} ein, in dem gilt:\\
	$\var{f}\geq 0\;$ und $\,\var{fpp0}\pi\in ${\Large I}. $\qquad$ {\Large I}={\LARGE [}\ansref ; \ansref{\LARGE ]}.
	\\\\
	Geben Sie das größte Intervall {\Large J} ein, in dem gilt:\\
	$\var{g} < 0\;$ und $\,\var{gnp0}\pi\in ${\Large J}. $\qquad$ {\Large J}={\LARGE (}\ansref ; \ansref{\LARGE )}.\\
	(Verwenden Sie bitte rationale Vielfache von $\pi$, geben Sie $\pi$ als pi ein.)}
	\explanation{}
	}
\lang{en}{
	\text{Find the largest interval {\Large I} such that:\\
	$\var{f}\geq 0\;$ and $\,\var{fpp0}\pi\in ${\Large I}. $\qquad$ {\Large I}={\LARGE [}\ansref , \ansref{\LARGE ]}.
	\\\\
	Find the largest interval {\Large J} such that:\\
	$\var{g} < 0\;$ and $\,\var{gnp0}\pi\in ${\Large J}. $\qquad$ {\Large J}={\LARGE (}\ansref , \ansref{\LARGE )}.\\
	(When inputting your answer, use rational multiples of $\pi$, use pi for $\pi$.)}
		\explanation{}
	}
	\begin{answer}
		\solution{fpl}
		\checkAsFunction{x}{-10}{10}{100}
    \end{answer}
	\begin{answer}
		\solution{fpr}
		\checkAsFunction{x}{-10}{10}{100}
    \end{answer}
	\begin{answer}
		\solution{gnl}
		\checkAsFunction{x}{-10}{10}{100}
    \end{answer}
	\begin{answer}
		\solution{gnr}
		\checkAsFunction{x}{-10}{10}{100}
    \end{answer}

\end{question} % 1
     
%Frage 2 von 2
\begin{question}
\explanation{Berücksichtigen Sie in welchen Intervallen die Funktionswerte der trigonometrischen Funktionen größer bzw. kleiner als Null sind.\\
Für den Tangens haben diese Intervalle eine Länge von $\pi/2$.}
\begin{variables}
	\randint[Z]{k}{-7}{7}
	\function[normalize]{left}{k*pi}
    \function[calculate]{kpoint}{(k+1/4)}
	\function[normalize]{point}{kpoint*pi}
    \function[calculate]{kright}{(k+1/2)}
	\function[normalize]{right}{kright*pi}
\end{variables}
	\lang{de}{
		\text{Geben Sie das größte Intervall {\Large I} ein, in dem gilt:\\
	Der Tangens ist definiert, $\tan(x)\geq 0\;$ und $\,\var{kpoint}\pi\in ${\Large I}.\\
	{\Large I}={\LARGE [}\ansref ; \ansref{\LARGE )}.}
	\explanation{}
	}
	\lang{en}{
		\text{Find the largest interval {\Large I} such that:\\
	The function tangent is defined, $\tan(x)\geq 0\;$ and $\,\var{kpoint}\pi\in ${\Large I}.\\
	{\Large I}={\LARGE [}\ansref , \ansref{\LARGE )}.}
	\explanation{}
	}
	\type{input.function}
	\field{rational}
	
	\begin{answer}
		\solution{left}
		\checkAsFunction{x}{-10}{10}{100}
	\end{answer}
	
	\begin{answer}
		\solution{right}
		\checkAsFunction{x}{-10}{10}{100}
	\end{answer}

\end{question} % 2
  
\end{problem}


\embedmathlet{mathlet}

\end{content}