\documentclass{mumie.problem.gwtmathlet}
%$Id$
\begin{metainfo}
  \name{
    \lang{de}{A04: Funktionseigenschaften}
    \lang{en}{Problem 4}
  }
  \begin{description} 
 This work is licensed under the Creative Commons License Attribution 4.0 International (CC-BY 4.0)   
 https://creativecommons.org/licenses/by/4.0/legalcode 

    \lang{de}{}
    \lang{en}{}
  \end{description}
  \corrector{system/problem/GenericCorrector.meta.xml}
  \begin{components}
    \component{js_lib}{system/problem/GenericMathlet.meta.xml}{mathlet}
    \component{generic_image}{content/rwth/HM1/images/g_img_right-angled-abb1.meta.xml}{image1}
  \end{components}
  \begin{links}
\link{generic_article}{content/rwth/HM1/T105_Trigonometrische_Funktionen/g_art_content_19_allgemeiner_sinus_cosinus.meta.xml}{content_19_allgemeiner_sinus_cosinus}
\end{links}
  \creategeneric
\end{metainfo}
\begin{content}
\usepackage{mumie.ombplus}
\usepackage{mumie.genericproblem}

\lang{de}{
	\title{A04: Funktionseigenschaften}
}
\lang{en}{
	\title{Problem 4}
	}

\begin{block}[annotation]
  Im Ticket-System: \href{http://team.mumie.net/issues/9729}{Ticket 9729}
\end{block}


\begin{problem} 
\randomquestionpool{1}{1}
\randomquestionpool{2}{3}

%Frage 1 von 3
\begin{question}
%\explanation{Zur Beantwortung dieser Frage empfiehlt es sich die im Vorlesungsteil dargestellte Herleitung der trigonometrischen Funktionen im Einheitskreis parat zu haben.}
	\lang{de}{
		\text{Welche  Aussagen sind richtig? Kreuzen Sie bitte an.}
		\explanation{Prüfen Sie jede der Aussagen anhand der Regel für die Nullstellen 
        des Sinus und Kosinus aus dem Vorlesungsteil.}
	}
    \lang{en}{
    	\text{Are the following statements correct?\\
    	}
    	\explanation{}
    }

    \permutechoices{1}{8} 
    \type{mc.yesno} 
    \field{real}
        
	\begin{choice}     		
		\text{$\sin(x) = 0 \;\Leftrightarrow\;  x = k \pi,\; \; k  \in \mathbb{Z}$}
		\solution{true}
	\end{choice}
	
	\begin{choice}
		\text{$\sin(x) = 0 \;\Leftrightarrow\; x = k \frac{\pi}{2},\; \; k \in \mathbb{Z}$}
		\solution{false} 
	\end{choice}
	
	\begin{choice}
		\text{$\cos(x) = 0\;\Leftrightarrow\;  x = l \pi,\; \; l  \in \mathbb{Z}$}
		\solution{false} 
	\end{choice}
	
	\begin{choice}
		\text{$\cos(x) = 0\;\Leftrightarrow\;  x = (2l + 1) \frac{\pi}{2},\; \; l   \in \mathbb{Z}$}
		\solution{true}  
	\end{choice}

	\begin{choice}
		\text{$\sin(x) = 0\;\Leftrightarrow\;  x = \frac{\pi}{2}+k\pi,\; \; k  \in \mathbb{Z}$}
		\solution{false} 
	\end{choice}
	
	\begin{choice}
		\text{$\cos(x) = 0 \;\Leftrightarrow\;  x = \frac{\pi}{2}+2l\pi,\; \; l \in \mathbb{Z}$}
		\solution{false} 
	\end{choice}

	\begin{choice}
		\text{$\sin(x) = 0\;\Leftrightarrow\;  x = 2\pi k,\; \; k  \in \mathbb{Z}$}
		\solution{false} 
	\end{choice}
	
	\begin{choice}
		\text{$\cos(x) = 0 \;\Leftrightarrow\;  x = \frac{\pi}{2}+l\pi,\; \; l \in \mathbb{Z}$}
		\solution{true} 
	\end{choice}
    %\explanation[equalChoice(0???????)]{Schauen Sie sich noch einmal die Nullstellen der Sinus-Funktion an.}
    %\explanation[equalChoice(?1??????)]{Es gilt zum Beispiel $\sin(\frac{\pi}{2})=1$.}
    %\explanation[equalChoice(??1?1???)]{Haben Sie die Sinus- und Kosinusfunktion verwechselt?}
    %\explanation[equalChoice(???0???1)]{Es gilt $(2l + 1) \frac{\pi}{2}= \frac{\pi}{2}+ l \pi$. Hilft Ihnen das weiter?}
    %\explanation[equalChoice(?????11?)]{Sinus und Kosinus haben beide zwei Nullstellen im Intervall $[0, 2\pi)$. Es gilt zwar $x=2\pi k$ für ein $k\in\Z \, \Rightarrow \sin(x)=0$, aber nicht die Äquivalenz.}
    %\explanation[equalChoice(???????0)]{Schauen Sie sich noch einmal die Nullstellen der Kosinus-Funktion an.}
  \end{question}

%Frage 2 von 3
\begin{question}
%\explanation{Zur Beantwortung dieser Frage empfiehlt es sich die im Vorlesungsteil dargestellte Herleitung der trigonometrischen Funktionen im Einheitskreis parat zu haben.}
	\lang{de}{
    \text{Überführen Sie die gegebenen Ausdrücke in eine der folgenden Funktionen
        $\sin(x), \cos(x), -\sin(x)$ oder $-\cos(x)$. Jede
        Teilaufgabe hat eine eindeutige Lösung. Verwenden Sie keine Leerzeichen bei der Eingabe!
        %Drücken Sie die angegebene Funktion jeweils durch 
        %$\sin(x), \cos(x), -\sin(x)$ oder $-\cos(x)$ aus. 
        %Verwenden Sie hier keine Leerzeichen bei der Eingabe!
        } 
	    \explanation{} 
    }
    \lang{en}{
    	\text{Express each of the given functions through sin(x), cos(x), -sin(x), or -cos(x):}
		\explanation{Sine and cosine are copies of each other shifted by $\frac{\pi}{2}$.}
    }
	\begin{variables}
		\string{s1}{sin(x)}
		\string{s2}{-cos(x)}
		\string{s3}{-sin(x)}
		\string{s4}{cos(x)}
		\string{s5}{-cos(x)}
		\string{s6}{-sin(x)}
    \end{variables}
    \type{input.text}    
      
    \begin{answer}%1
        \text{$\cos(x - \frac{\pi}{2})=$}
        \solution{s1}
        \inputAsString{g1}
		\checkStringsForRelation{equal(s1,g1)}
        \explanation{Verwenden Sie die Regel zur Verschiebung zwischen Sinus und Kosinus.} 
    \end{answer}
    
    \begin{answer}%2
        \text{$\sin(x - \frac{\pi}{2})=$}
        \solution{s2}
        \inputAsString{g2}
		\checkStringsForRelation{equal(s2,g2)}
        \explanation{Verwenden Sie z.B. die Verschiebung zwischen Sinus und Kosinus 
        und das Symmetrieverhalten von Sinus und Kosinus.}
    \end{answer}
    
    \begin{answer}%3
        \text{$\cos(x+\frac{\pi}{2})=$}
        \solution{s3}
        \inputAsString{g3}
		\checkStringsForRelation{equal(s3,g3)}
        \explanation{Verwenden Sie z.B. die Verschiebung zwischen Sinus und Kosinus
        und das Symmetrieverhalten von Sinus und Kosinus.}
    \end{answer}
    
    \begin{answer}%4
        \text{$\sin(x + \frac{\pi}{2})=$}
        \solution{s4}
        \inputAsString{g4}
		\checkStringsForRelation{equal(s4,g4)}
        \explanation{Verwenden Sie die Regel zur Verschiebung zwischen Sinus und Kosinus.}
    \end{answer}
    
    \begin{answer}%5
        \text{$\cos(x - \pi)=$}
        \solution{s5}
        \inputAsString{g5}
		\checkStringsForRelation{equal(s5,g5)}
        \explanation{Verwenden Sie z.B. für $\cos((x-\frac{\pi}{2})-\frac{\pi}{2})\,$ mehrfach 
        die Verschiebung zwischen Sinus und Kosinus und das Symmetrieverhalten von Sinus und Kosinus.}
    \end{answer}
  
    \begin{answer}%6
        \text{$\sin(4\pi-x)=$}
        \solution{s6}
        \inputAsString{g6}
		\checkStringsForRelation{equal(s6,g6)}        
        \explanation{Verwenden Sie die Periodizität und die Symmetrie der Sinus-Funktion.}
    \end{answer}
\end{question}

%Frage 3 von 3
\begin{question}
%\explanation{Zur Beantwortung dieser Frage empfiehlt es sich die im Vorlesungsteil dargestellte Herleitung der trigonometrischen Funktionen im Einheitskreis parat zu haben.}
	\lang{de}{
		\text{Überführen Sie die gegebenen Ausdrücke in eine der folgenden Funktionen
        $\sin(x), \cos(x), -\sin(x)$ oder $-\cos(x)$. Jede
        Teilaufgabe hat eine eindeutige Lösung. Verwenden Sie keine Leerzeichen bei der Eingabe!} 
    }
    \lang{en}{
    	\text{Express each of the given functions through sin(x), cos(x), -sin(x), or -cos(x):}
		\explanation{Sine and cosine are copies of each other shifted by $\frac{\pi}{2}$.}
    }
	\begin{variables}
		\string{s1}{-sin(x)}
		\string{s2}{cos(x)}
		\string{s3}{sin(x)}
		\string{s4}{-cos(x)}
		\string{s5}{-sin(x)}
		\string{s6}{cos(x)}
    \end{variables}
    \type{input.text}    
      
    \begin{answer}%1
        \text{$\cos(x + \frac{\pi}{2})=$}
        \solution{s1}
        \inputAsString{g1}
		\checkStringsForRelation{equal(s1,g1)}
        \explanation{Verwenden Sie z.B. die Verschiebung zwischen Sinus und Kosinus
        und das Symmetrieverhalten von Sinus und Kosinus.}
    \end{answer}
    
    \begin{answer}%2
        \text{$\sin(x + \frac{\pi}{2})=$}
        \solution{s2}
        \inputAsString{g2}
		\checkStringsForRelation{equal(s2,g2)}
        \explanation{Verwenden Sie die Verschiebung zwischen Sinus und Kosinus.}
    \end{answer}
    
    \begin{answer}%3
        \text{$\cos(\frac{\pi}{2}-x)=$}
        \solution{s3}
        \inputAsString{g3}
		\checkStringsForRelation{equal(s3,g3)}
        \explanation{Verwenden Sie die Verschiebung zwischen Sinus und Kosinus}
    \end{answer}
    
    \begin{answer}%4
        \text{$\sin(x - \frac{\pi}{2})=$}
        \solution{s4}
        \inputAsString{g4}
		\checkStringsForRelation{equal(s4,g4)}
        \explanation{Verwenden Sie z.B. die Verschiebung zwischen Sinus und Kosinus
        und das Symmetrieverhalten von Sinus und Kosinus.}
    \end{answer}
    
    \begin{answer}%5
        \text{$\sin(x + \pi)=$}
        \solution{s5}
        \inputAsString{g5}
		\checkStringsForRelation{equal(s5,g5)}        
        \explanation{Verwenden Sie z.B. für $\sin((x+\frac{\pi}{2})+\frac{\pi}{2})\,$ mehrfach 
        die Verschiebung zwischen Sinus und Kosinus und das Symmetrieverhalten von Kosinus und Sinus.}
    \end{answer}
  
    \begin{answer}%6
        \text{$\cos(x - 2\pi)=$}
        \solution{s6}
        \inputAsString{g6}
		\checkStringsForRelation{equal(s6,g6)}
        \explanation{Verwenden die Periodizität der Kosinus-Funktion.}
    \end{answer}
\end{question}
 
%

\end{problem}



\embedmathlet{mathlet}

\end{content}