\documentclass{mumie.problem.gwtmathlet}
%$Id$
\begin{metainfo}
  \name{
    \lang{de}{A05: Funktionseigenschaften}
    \lang{en}{Problem 5}
  }
  \begin{description} 
 This work is licensed under the Creative Commons License Attribution 4.0 International (CC-BY 4.0)   
 https://creativecommons.org/licenses/by/4.0/legalcode 

    \lang{de}{}
    \lang{en}{}
  \end{description}
  \corrector{system/problem/GenericCorrector.meta.xml}
  \begin{components}
    \component{js_lib}{system/problem/GenericMathlet.meta.xml}{mathlet}
    \component{generic_image}{content/rwth/HM1/images/g_img_right-angled-abb1.meta.xml}{image1}
  \end{components}
  \begin{links}
  \end{links}
  \creategeneric
\end{metainfo}
\begin{content}
\usepackage{mumie.ombplus}
\usepackage{mumie.genericproblem}

\lang{de}{
	\title{A05: Funktionseigenschaften}
}
\lang{en}{
	\title{Problem 5}
	}

\begin{block}[annotation]
  Im Ticket-System: \href{http://team.mumie.net/issues/9730}{Ticket 9730}
\end{block}


\begin{problem}

% Frage 1 von 3
\begin{question}
        \explanation{Es gilt $\tan(x)=\frac{\sin(x)}{\cos(x)}$ und $\tan(-x)=-\tan(x)$, 
        zudem ist $\tan(x-{\pi}/{3})$ die um ${\pi}/{3}$ nach rechts verschobene Funktion $\tan(x)$.}
	    \begin{variables}
	    	\randint{c}{1}{6}%6 permutations
	  %  	\randint{b}{2}{4}
	    	\number{b}{3}
	    	\function[calculate]{g1}{dirac(c-1)}
	    	\function[calculate]{g2}{dirac(c-2)}
	    	\function[calculate]{g3}{dirac(c-3)}
	    	\function[calculate]{g4}{dirac(c-4)}
	    	\function[calculate]{g5}{dirac(c-5)}
	    	\function[calculate]{g6}{dirac(c-6)}
		    \function[normalize]{f1}{g1* tan(x) +(g2)* tan(x) +(g3)* tan(-x) +(g4)* tan(x-pi/b) +(g5)* tan(-x) +(g6)* tan(x-pi/b)}
		    \function[normalize]{f2}{(g1)* tan(-x) +(g2)* tan(x-pi/b) +(g3)* tan(x) +(g4)* tan(x) +(g5)* tan(x-pi/b) +(g6)* tan(-x)}
		    \function[normalize]{f3}{(g1)* tan(x-pi/b) +(g2)* tan(-x) +(g3)* tan(x-pi/b) +(g4)* tan(-x) +(g5)* tan(x) +(g6)* tan(x)}
		    \function{f10}{g1* tan(x) +g2* tan(x) +g3* tan(-x) +g4* tan(x-pi/b) +g5* tan(-x) +g6* tan(x-pi/b)}
		    \function{f20}{(g1)* tan(-x) +(g2)* tan(x-pi/b) +(g3)* tan(x) +(g4)* tan(x) +(g5)* tan(x-pi/b) +(g6)* tan(-x)}
		    \function{f30}{(g1)* tan(x-pi/b) +(g2)* tan(-x) +(g3)* tan(x-pi/b) +(g4)* tan(-x) +(g5)* tan(x) +(g6)* tan(x)}

	    \end{variables}

    \plotF{1}{f10} % % the function a1 is defined below in 'variables' in the usual way
    \plotFrom{1}{-3} % % and is plotted starting from 0.0
    \plotTo{1}{3} % % and ending in 1.0 , it's a quarter of a circle
    \plotColor{1}{blue} % % colored blu
    \plotF{2}{f20} % % the function a1 is defined below in 'variables' in the usual way
    \plotFrom{2}{-3} % % and is plotted starting from 0.0
    \plotTo{2}{3} % % and ending in 1.0 , it's a quarter of a circle
    \plotColor{2}{red} % % colored red
    \plotF{3}{f30} % % the function a1 is defined below in 'variables' in the usual way
    \plotFrom{3}{-3} % % and is plotted starting from 0.0
    \plotTo{3}{3} % % and ending in 1.0 , it's a quarter of a circle
    \plotColor{3}{green} % % colored magenta
    \plotLeft{-3} % % defines the canvas bound left
    \plotRight{3} % % and right
    \plotSize{400}
    \type{input.function}
    \field{rational}

\lang{de}{
		\text{Die Graphen von $\tan(x)$, $\,\tan(-x)$ und $\,\tan\left(x-\frac{\pi}{\var{b}}\right)$ sind unten abgebildet.\\
		(Benutzen Sie pi für die Eingabe von $\pi$.)}
		\explanation{}
}
\lang{en}{
		\text{The graphs of $\tan(x)$, $\,\tan(-x)$ and $\,\tan\left(x-\frac{\pi}{\var{b}}\right)$ are depicted below.\\
    	(Input $\pi$ as pi.)}
		\explanation{}
}
\begin{answer}
	\lang{de}{\text{Der \textcolor{blue}{blaue} Funktionsgraph gehört zu $f(x) = $ }}
	\lang{en}{\text{The \textcolor{blue}{blue} graph is the graph of $f(x) = $ }}
	\solution{f1}
	\checkAsFunction{x}{-10}{10}{100}
\end{answer}

\begin{answer}
	\lang{de}{\text{Der \textcolor{red}{rote} Funktionsgraph gehört zu $g(x) = $ }}
	\lang{en}{\text{The \textcolor{red}{red} graph is the graph of $g(x) = $ }}
	\solution{f2}
	\checkAsFunction{x}{-10}{10}{100}
\end{answer}

\begin{answer}
	\lang{de}{\text{Der \textcolor{green}{grüne} Funktionsgraph gehört zu $h(x) = $ }}
	\lang{en}{\text{The \textcolor{green}{green} graph is the graph of $h(x) = $ }}
	\solution{f3}
	\checkAsFunction{x}{-10}{10}{100}
\end{answer}
\end{question} %1

%Frage 2 von 3
\begin{question}
    \explanation{Der Tangens ist als $\tan(x)=\frac{\sin(x)}{\cos(x)}$ definiert. 
    Die Nullstellen von $\cos$ sind also die Definitionslücken des Tangens.}
    \lang{de}{
	    \text{Geben Sie alle Stellen im Intervall $[\var{aleft}\pi;\,\var{aright}\pi]$ an, 
		an denen die Funktion Tangens nicht definiert ist. \\
		(Benutzen Sie pi für $\pi$.)\\
		Tangens ist nicht definiert für $x\in${\LARGE $\{$}\ansref ; \ansref ; \ansref{\LARGE $\}$} . }
	    \explanation{}
	}
    \lang{en}{
    	\text{For which $x$-values on the interval $[\var{aleft}\pi,\,\var{aright}\pi]$ is the function tangent undefined? \\
		(Input $\pi$ as pi.)\\
		Tangent is not defined for $x\in${\LARGE $\{$}\ansref , \ansref , \ansref{\LARGE $\}$}. }
    	\explanation{}
    }
	\begin{variables}
		\randint[Z]{aleft}{-9}{6}
		\function[calculate]{aright}{aleft+3}
		\function[normalize]{xleft}{aleft*pi}
		\function[normalize]{xright}{aright*pi}
	    \function[calculate]{x01}{aleft+1/2}
	    \function[calculate]{x02}{aleft+3/2}
	    \function[calculate]{x03}{aleft+5/2}
	    \function{x1}{x01*pi}
	    \function{x2}{x02*pi}
	    \function{x3}{x03*pi}
	\end{variables}
    \type{input.function}
    \field{rational}
    \permuteAnswers{1, 2, 3}
    \begin{answer}
        \solution{x1}
        \checkAsFunction{x}{-10}{10}{100}
    \end{answer}
    \begin{answer}
        \solution{x2}
        \checkAsFunction{x}{-10}{10}{100}
    \end{answer}
    \begin{answer}
        \solution{x3}
        \checkAsFunction{x}{-10}{10}{100}
    \end{answer}

\end{question}

\begin{question} % question 3 of 3
\explanation{Berücksichtigen Sie die im ersten Vorlesungsabschnitt besprochene Tabelle für 
 wichtige Funktionswerte der trigonometischen Funktionen und nutzen Sie ihre Eigenschaften 
 (Periodizität, Symmetrie, Verschiebung). \\
 Beispiel: $\; \sin\left(\frac{-5}{4} \pi \right)=\sin\left(\frac{1}{4} \pi-2 \pi\right)=\sin\left(\frac{\pi}{4}\right)$
     }
\lang{en}{\text{Find the function values.\\
	Input the exact values. Use, e.g.$\;$ -sqrt(5)/3 for $-\frac{\sqrt{5}}{3}$.}
	\explanation{}}
\lang{de}{\text{Bestimmen Sie die Funktionswerte.\\
	Geben Sie die exakten Werte ein, z.B.$\;$ -sqrt(5)/3 für $-\frac{\sqrt{5}}{3}$.}
	\explanation{}}
\begin{variables}
	\randint{a00}{-1}{1}
	\randint[Z]{a01}{-1}{1}
	\randint{a02}{0}{1}
	\function[calculate]{s10}{a01*(-1)^(a02)}
	\function[calculate]{c10}{(-1)^(a02)}
	\function[calculate]{t10}{a01}
 \end{variables}
 \begin{pool}
\begin{variables}
	\number{b}{4}
	\function[calculate]{a1}{(2*a00 + a02 + a01/b)}
	\function{a}{a1*pi}
	\function{rs4}{sqrt(2)/2} % result sine, b=4
	\function{rc4}{sqrt(2)/2} % result cosine, b=4
	\function{rt4}{1} % result tan, b=4

	\function{s1}{s10*rs4} % sin(a)
	\function{c1}{c10*rc4} % cos(a)
	\function{t1}{t10} % tan(a)
\end{variables}

\begin{variables}
	\number{b}{6}
	\function[calculate]{a1}{(2*a00 + a02 + a01/b)}
	\function{a}{a1*pi}
	\function{rs6}{1/2} % result sine, b=6
	\function{rc6}{sqrt(3)/2} % result cosine, b=6
	\function{rt6}{sqrt(3)/3} % result tan, b=6

	\function{s1}{s10*rs6} % sin(a)
	\function{c1}{c10*rc6} % cos(a)
	\function{t1}{t10*rt6} % tan(a)
\end{variables}

\begin{variables}
	\number{b}{3}
	\function[calculate]{a1}{(2*a00 + a02 + a01/b)}
	\function{a}{a1*pi}
	\function{rs3}{sqrt(3)/2} % result sine, b=3
	\function{rc3}{1/2} % result cosine, b=3
	\function{rt3}{sqrt(3)} % result tan, b=3

	\function{s1}{s10*rs3} % sin(a)
	\function{c1}{c10*rc3} % cos(a)
	\function{t1}{t10*rt3} % tan(a)
\end{variables}

\end{pool}
\type{input.function}
\field{rational}

	\begin{answer}
		\text{$\sin\left(\var{a}\right) = $ }
		\solution{s1}
        \inputAsFunction{x}{i1}
		\checkAsFunction{x}{-5}{5}{10}
	\end{answer}

	\begin{answer}
		\text{$\cos\left(\var{a}\right) = $ }
		\solution{c1}
        \inputAsFunction{x}{i2}
		\checkAsFunction{x}{-5}{5}{10}
	\end{answer}

	\begin{answer}
		\text{$\tan\left(\var{a}\right) = $ }
		\solution{t1}
        \inputAsFunction{x}{i3}
		\checkFuncForZero{i3-(i1/i2)}{-5}{5}{10}  
        %\checkAsFunction{x}{-5}{5}{10} 
        \explanation[i3 != t1 ]{Der Tangens ergibt sich zu $\tan(x)=\sin(x)/\cos(x)$.}
	\end{answer}

\end{question} % 3

\end{problem}

\embedmathlet{mathlet}

\end{content}