%$Id:  $
\documentclass{mumie.article}
%$Id$
\begin{metainfo}
  \name{
    \lang{de}{Überblick: Reihen}
    \lang{en}{overview: }
  }
  \begin{description} 
 This work is licensed under the Creative Commons License Attribution 4.0 International (CC-BY 4.0)   
 https://creativecommons.org/licenses/by/4.0/legalcode 

    \lang{de}{Beschreibung}
    \lang{en}{}
  \end{description}
  \begin{components}
  \end{components}
  \begin{links}
\link{generic_article}{content/rwth/HM1/T208_Reihen/g_art_content_26_produkt_von_reihen.meta.xml}{content_26_produkt_von_reihen}
\link{generic_article}{content/rwth/HM1/T208_Reihen/g_art_content_25_konvergenz_kriterien.meta.xml}{content_25_konvergenz_kriterien}
\link{generic_article}{content/rwth/HM1/T208_Reihen/g_art_content_24_reihen_und_konvergenz.meta.xml}{content_24_reihen_und_konvergenz}
\end{links}
  \creategeneric
\end{metainfo}
\begin{content}
\begin{block}[annotation]
	Im Ticket-System: \href{https://team.mumie.net/issues/30129}{Ticket 30129}
\end{block}
\begin{block}[annotation]
Copy of : /home/mumie/checkin/content/rwth/HM1/T209_Potenzreihen/art_T209_overview.src.tex
\end{block}



\begin{block}[annotation]
Im Entstehen: Überblicksseite für Kapitel Reihen
\end{block}

\usepackage{mumie.ombplus}
\ombchapter{1}
\lang{de}{\title{Überblick: Reihen}}
\lang{en}{\title{}}



\begin{block}[info-box]
\lang{de}{\strong{Inhalt}}
\lang{en}{\strong{Contents}}


\lang{de}{
    \begin{enumerate}%[arabic chapter-overview]
   \item[8.1] \link{content_24_reihen_und_konvergenz}{Reihen und Konvergenz}
   \item[8.2] \link{content_25_konvergenz_kriterien}{Konvergenzkriterien}
   \item[8.3] \link{content_26_produkt_von_reihen}{Produkt von Reihen}
   \end{enumerate}
} %lang

\end{block}

\begin{zusammenfassung}

\lang{de}{Reihen sind besondere Folgen, bei denen das nächste Folgeglied dadurch entsteht, zu dem vorgehenden eine Zahl zu addieren.
Reihen sind somit Summen mit unendlich vielen Summanden. Allerdings kommt es bei unendlichen Summen, anders als bei endlichen, auf die Reihenfolge der Summanden an.

Eine Reihe konvergiert, wenn die oben beschriebene Folge ihrer Partialsummen das tut. Die harmonische Reihe divergiert, die geometrische konvergiert. 
Auch Dezimalbrüche sind Beispiele für konvergente Reihen, und damit zeigt man $0,\overline{9}=1$.
\\
Für den Spezialfall alternierender reeller Reihen gibt das Leibniz-Kriterium Auskunft über Konvergenz.
\\
Wie bei Folgen macht es keinen großen Unterschied, ob die Reihenglieder reelle oder komplexe Zahlen sind.

Absolute Konvergenz liegt vor, wenn die Reihe der Absolutbeträge konvergiert. Bei solchen Reihen kommt es wiederum nicht auf die Summationsreihenfolge an.
Mit dem Majorantenkriterium, dem Wurzel- und dem Quotientenkriterium haben wir viele Methoden, die bei der Untersuchung auf absolute Konvergenz helfen.


Zuletzt geben wir mit dem Cauchy-Produkt eine Möglichkeit, Reihen zu multiplizieren. 
Als Anwendung beweisen wir damit die Funktionalgleichung (Potenzgesetz) der Exponentialreihe. 
}


\end{zusammenfassung}

\begin{block}[info]\lang{de}{\strong{Lernziele}}
\lang{en}{\strong{Learning Goals}} 
\begin{itemize}[square]
\item \lang{de}{Sie kennen den Reihen-Begriff und den Unterschied zwischen bedingter und absoluter Konvergenz.}
\item \lang{de}{Sie kennen wichtige Reihen wie die harmonische, die geometrische und die Exponentialreihe und ggf. deren Grenzwert.}
\item \lang{de}{Sie kennen das Leibniz-Kriterium und wenden es auf Beispiele an.}
\item \lang{de}{Sie wissen, welches der Kriterien für absolute Konvergenz in konkreten Situationen anwendbar ist und wenden alle Kriterien sicher an, 
auch auf Reihen mit komplexen Gliedern.}
\item \lang{de}{Sie wenden die Grenzwertsätze auf Reihen an.}
\item \lang{de}{Sie wissen, dass das Cauchy-Produkt ein sinnvoller Begriff eines Reihenprodukts ist und kennen mindestens ein Beispiel.}
\end{itemize}
\end{block}




\end{content}
