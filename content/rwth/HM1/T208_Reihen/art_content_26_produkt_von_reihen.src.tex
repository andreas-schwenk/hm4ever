%$Id:  $
\documentclass{mumie.article}
%$Id$
\begin{metainfo}
  \name{
    \lang{de}{Produkt von Reihen}
    \lang{en}{}
  }
  \begin{description} 
 This work is licensed under the Creative Commons License Attribution 4.0 International (CC-BY 4.0)   
 https://creativecommons.org/licenses/by/4.0/legalcode 

    \lang{de}{Beschreibung}
    \lang{en}{}
  \end{description}
  \begin{components}
\component{generic_image}{content/rwth/HM1/images/g_img_00_Videobutton_schwarz.meta.xml}{00_Videobutton_schwarz}
\component{generic_image}{content/rwth/HM1/images/g_img_00_video_button_schwarz-blau.meta.xml}{00_video_button_schwarz-blau}
\end{components}
  \begin{links}
    \link{generic_article}{content/rwth/HM1/T201neu_Vollstaendige_Induktion/g_art_content_03_binomischer_lehrsatz.meta.xml}{content_03_binomischer_lehrsatz}
    \link{generic_article}{content/rwth/HM1/T201neu_Vollstaendige_Induktion/g_art_content_02_vollstaendige_induktion.meta.xml}{content_02_vollstaendige_induktion}
    \link{generic_article}{content/rwth/HM1/T208_Reihen/g_art_content_26_produkt_von_reihen.meta.xml}{content_26_produkt_von_reihen}
    \link{generic_article}{content/rwth/HM1/T104_weitere_elementare_Funktionen/g_art_content_15_exponentialfunktionen.meta.xml}{content_15_exponentialfunktionen}
    \link{generic_article}{content/rwth/HM1/T208_Reihen/g_art_content_24_reihen_und_konvergenz.meta.xml}{reihen-und-konv}
    %\link{generic_article}{content/rwth/HM1/T201_Vollstaendige_Induktion_wichtige_Ungleichungen/g_art_content_02_vollstaendige_induktion.meta.xml}{vollst-ind}
    \link{generic_article}{content/rwth/HM1/T208_Reihen/g_art_content_25_konvergenz_kriterien.meta.xml}{abs-konv}
    %\link{generic_article}{content/rwth/HM1/T201_Vollstaendige_Induktion_wichtige_Ungleichungen/g_art_content_03_binomialkoeffizienten.meta.xml}{binom-koeff}
  \end{links}
  \creategeneric
\end{metainfo}
\begin{content}
\usepackage{mumie.ombplus}
\ombchapter{8}
\ombarticle{3}




\lang{de}{\title{Umordnung und Produkt von Reihen}}
 
\begin{block}[annotation]
 Im Ticket-System: \href{http://team.mumie.net/issues/9683}{Ticket 9683}\\
\end{block}

\begin{block}[info-box]
\tableofcontents
\end{block}
\section{Umordnung von Reihen}\label{sec:reihen-umordnung}


In einer früheren \ref[reihen-und-konv][Bemerkung]{rem:reihenfolge-summation} haben wir bereits gesehen, 
dass es bei der Summation einer unendlichen Reihe anders als bei 
einer endlichen Summe auf die Reihenfolge der Summanden ankommt. 
Durch die Definition einer Reihe als Folge ihrer Partialsummen haben wir eine Reihenfolge festgelegt.
Eine Umordnung der natürlichen Zahlen ist eine
bijektive Abbildung $\sigma\colon \N\to\N$. Diese benutzen wir, um die Umordnung einer Reihe 
\[\sum_{n=1}^\infty a_n\mapsto \sum_{n=1}^\infty a_{\sigma(n)}\]
zu beschreiben. Bei nur bedingt konvergenten Reihen kann bei einer Umordnung alles passieren: 
Je nach Umordnung kann der Grenzwert derselbe bleiben oder sich ändern, aber die Reihe kann auch divergieren.


\begin{tabs*}[\initialtab{0}]
\tab{Unveränderter Grenzwert}
Es sei $\sum_{n=1}^\infty a_n$ eine konvergente komplexe Reihe, und es sei
 $\sigma:\N\to\N$ eine Umordnung, die die Zahlen $1,\ldots,N$ unter sich vertauscht, 
die anderen Zahlen aber fest lässt, d.\,h. $\sigma(n)=n$ für $n>N$. 
Dann konvergiert die umgeordnete Reihe $\sum_{n=1}^\infty a_{\sigma(n)}$  gegen denselben Grenzwert.
Denn wir können die beiden Reihen schreiben als
\[\sum_{n=1}^\infty a_n=\sum_{n=1}^N a_n+\sum_{n=N+1}^\infty a_n \text{ bzw.} 
\sum_{n=1}^\infty a_{\sigma(n)}=\sum_{n=1}^N a_{\sigma(n)}+\sum_{n=N+1}^\infty a_{\sigma(n)}.\]
Die jeweils vorderen Summen sind endlich, deshalb können wir sie beliebig umordnen. 
Sie sind also gleich. Die beiden hinteren Reihen sind gleich, 
weil dort stets $a_n=a_{\sigma(n)}$ gilt. Also haben die beiden Reihen denselben Grenzwert.
\tab{Divergenz}
Die \ref[reihen-und-konv][alternierende Reihe]{ex:alter-harm-reihe} $\sum_{k=1}^\infty \frac{(-1)^{k+1}}{k}$ ist bedingt konvergent.
Wir zeigen, dass es eine Umordnung $\sigma:\N\to\N$ gibt, sodass $\sum_{n=1}^\infty \frac{(-1)^{\sigma(k)+1}}{\sigma(k)}$ divergiert.
Betrachten wir dazu die Reihenglieder mit ungerader Ordnung $k=2^n+1,\ldots, 2^{n+1}-1$. 
Das sind $2^{n-1}$ Stück. Für ihre Teilsumme finden wir also
\[\frac{1}{2^n+1}+\frac{1}{2^n+3}+\ldots+\frac{1}{2^{n+1}-1}>2^{n-1}\cdot\frac{1}{2^{n+1}}=\frac{1}{4}.\]
Ordnen wir die alternierende Reihe um in
\[1-\frac{1}{2}+\frac{1}{3}-\frac{1}{4}+\left(\frac{1}{5}+\frac{1}{7}\right)-\frac{1}{6}+\left(\frac{1}{9}+\frac{1}{11}+\frac{1}{13}+\frac{1}{15}\right)-\frac{1}{8}+\ldots,\]
dann ist die so entstandene Partialsummenfolge, d.\,h. die umgeordnete Reihe, divergent.
\end{tabs*}

Für absolut konvergente Reihen hat man hingegen das folgende Ergebnis.
\begin{theorem}
Es sei $\sum_{n=1}^\infty a_n$ eine absolut konvergente komplexe Reihe mit Grenzwert $A$. 
Dann konvergiert jede ihrer Umordnungen ebenfalls gegen den Grenzwert $A$.\\
\floatright{\href{https://www.hm-kompakt.de/video?watch=325}{\image[75]{00_Videobutton_schwarz}}}\\\\
\end{theorem}
\begin{proof*}
\begin{incremental}[\initialsteps{0}]
\step
Weil die Reihe $\sum_{n=1}^\infty \vert a_n\vert$ konvergiert,
gibt es zu jedem $\epsilon >0$ eine natürliche Zahl $N$ so, dass gilt
\[  \sum_{n=N+1}^\infty {\vert a_n\vert} <\frac{\epsilon}{2}.\]
Unter Verwendung der \ref[abs-konv][Dreiecksungleichung für absolut konvergente Reihen]{ex:dreiecksungl-reihen}   folgt 
\[\vert A-\sum_{n=1}^{N}a_n\vert
=\vert \sum_{n=N+1}^\infty a_n\vert \leq \sum_{n=N+1}^\infty {\vert a_n\vert}<\frac{\epsilon}{2}.\]

\step
Es sei nun $\sigma:\N\to\N$ eine Umordnung. 
Wir wählen eine natürliche Zahl $M\geq N$ derart, dass die Menge $\{1, 2,\ldots N\}$ enthalten ist in der Menge
der Bilder $\{\sigma(1), \sigma(2), \ldots, \sigma(M)\}$.
Dann gilt für alle $m>M$ mit Hilfe der Dreiecksungleichung
\[\vert \sum_{n=1}^m a_{\sigma(n)} -A\vert\leq \vert \sum_{n=1}^m a_{\sigma(n)} -\sum_{n=1}^{N}a_n\vert+
\vert \sum_{n=1}^{N}a_n-A\vert.\]
Den hinteren Term haben wir bereits durch $\frac{\epsilon}{2}$ abgeschätzt.
Im vorderen Term tritt keines der Glieder $a_1,\ldots, a_N$ mehr auf. 
Deshalb gilt die Abschätzung
\[\vert \sum_{n=1}^m a_{\sigma(n)} -\sum_{n=1}^{N}a_n\vert\leq 
 \sum_{n=N+1}^\infty{\vert a_n\vert}<\frac{\epsilon}{2}.\]
 
\step 
Wir erhalten insgesamt
\[\vert \sum_{n=1}^m a_{\sigma(n)} -A\vert<\frac{\epsilon}{2}+\frac{\epsilon}{2}=\epsilon\]
für alle $m>M\geq N$. Da $\epsilon$ beliebig war, 
folgt die Konvergenz der Reihe $\sum_{n=1}^\infty a_{\sigma(n)}$ gegen $A$.
\end{incremental}
\end{proof*}

Umordnung von Reihen ist auch Thema des folgenden Videos:
\floatright{\href{https://api.stream24.net/vod/getVideo.php?id=10962-2-10912&mode=iframe&speed=true}{\image[75]{00_video_button_schwarz-blau}}}

\section{Cauchy-Produkt}


Hat man zwei konvergente Reihen $\sum_{k=0}^\infty a_k=A$ und $\sum_{k=0}^\infty b_k=B$ gegeben, möchte man
oft daraus eine Reihe $\sum_{k=0}^\infty c_k$ konstruieren, die gegen $A\cdot B$ konvergiert.

Das \emph{Cauchy-Produkt} ist eine solche Konstruktion. Sie liefert allerdings nur für absolut konvergente Reihen den
gewünschten Grenzwert! Insbesondere ist das Cauchy-Produkt nicht das Produkt der Partialsummenfolgen.


\begin{definition}[Cauchy-Produkt]\label{def:cauchy-prod}
Das  \notion{Cauchy-Produkt} zweier (komplexer) Reihen  $\sum_{k=0}^\infty a_k$ und $\sum_{k=0}^\infty b_k$ 
ist die Reihe $\sum_{k=0}^\infty c_k$, die gegeben ist durch
\[   c_k=\sum_{j=0}^k a_jb_{k-j}. \]
Ins besondere ist $c_0=a_0b_0$, $c_1=a_0b_1+a_1b_0$, $c_2=a_0b_2+a_1b_1+a_2b_0$, etc.. 
\end{definition}

\begin{remark}
Generell gilt: Man erhält das $k$-te Glied $c_k$, indem man alle Produkte $a_jb_l$ aufsummiert, bei denen die Summe der Indizes $j+l$ gleich $k$ ist.
Beginnen die Reihen nicht bei $0$, sondern bei anderen Zahlen, ändert sich entsprechend die Summationsmenge von $c_k$.
Das Cauchy-Produkt der Reihen $\sum_{k=1}^\infty a_k$ und $\sum_{k=2}^\infty b_k$ ist also $\sum_{k=3}^\infty c_k$  mit
\[ c_k=\sum_{j=1}^{k-2} a_jb_{k-j} \]
für $k\geq 3$. Die Reihe fängt hier bei $3$ an, da es keine Produkte $a_jb_l$ gibt, deren Indizes sich zu $0$, $1$ oder $2$ summieren.
\end{remark}


\begin{theorem}\label{thm:cauchy-prod}
Sind $\sum_{k=0}^\infty a_k=A$ und $\sum_{k=0}^\infty b_k=B$ zwei absolut konvergente Reihen 
und $\sum_{k=0}^\infty c_k$ ihr Cauchy-Produkt,
so ist auch $\sum_{k=0}^\infty c_k$ absolut konvergent und es gilt
\[ \sum_{k=0}^\infty c_k =A\cdot B. \]
\end{theorem}

\begin{block}[warning]
\begin{incremental}[\initialsteps{1}]
\step
Sind die Reihen $\sum_{k=0}^\infty a_k=A$ und $\sum_{k=0}^\infty b_k=B$ nur bedingt konvergent, so kann man zwar ihr
Cauchy-Produkt $\sum_{k=0}^\infty c_k$ wie oben definieren, doch kann diese Reihe divergent sein. Selbst wenn sie konvergent ist, muss ihr Grenzwert
nicht $A\cdot B$ sein.
\step

Als Beispiel betrachte man die Reihen $\sum_{k=1}^\infty a_k$ und $\sum_{k=1}^\infty b_k$ mit 
$a_k=b_k=\frac{(-1)^k}{\sqrt{k}}$.
Nach dem \ref[reihen-und-konv][Leibniz-Kriterium]{thm:leibnizkriterium} sind die beiden Reihen konvergent, 
da $(\frac{1}{\sqrt{k}})_{k\in \N}$ eine monoton fallende Nullfolge ist.
Für die Glieder $c_k$ ($k\geq 2$) des Cauchy-Produkts gilt:
\[   c_k=\sum_{j=1}^{k-1} \frac{(-1)^j}{\sqrt{j}}\cdot \frac{(-1)^{k-j}}{\sqrt{k-j}}
=(-1)^k\cdot \sum_{j=1}^{k-1}\frac{1}{\sqrt{j(k-j)}}. \]
Wegen der \ref[content_02_vollstaendige_induktion][Ungleichung zwischen geometrischem und arithmetischem Mittel]{sec:wichtige-beispiele} ist $\sqrt{j(k-j)}\leq \frac{j+(k-j)}{2}=\frac{k}{2}$ für alle $j=1,\ldots, k-1$, und damit gilt:
\begin{eqnarray*}  
{|c_k|} &=& \left|  (-1)^k\cdot \sum_{j=1}^{k-1}\frac{1}{\sqrt{j(k-j)}}\right| 
= \sum_{j=1}^{k-1}\frac{1}{\sqrt{j(k-j)}} \\
&\geq &  \sum_{j=1}^{k-1}\frac{\,1\,}{\frac{k}{2}}
= \sum_{j=1}^{k-1}\frac{2}{k}=\frac{(k-1)\cdot 2}{k}=2-\frac{2}{k}.
\end{eqnarray*}
 Wegen $\lim_{k\to \infty} (2-\frac{2}{k})=2$ bilden die $c_k$ also keine Nullfolge, und insbesondere ist die Reihe $\sum_{k=2}^\infty c_k$ nicht konvergent.
\end{incremental}
\end{block}

\begin{remark}
Die $n$-te Partialsumme des Cauchy-Produkts $\sum_{k=0}^\infty c_k$ ist \textbf{nicht} das Produkt der $n$-ten Partialsummen der Reihen
$\sum_{k=0}^\infty a_k$ und $\sum_{k=0}^\infty b_k$. Letzteres ist nämlich
\[ \left( \sum_{j=0}^n a_j \right)\cdot \left( \sum_{l=0}^n b_l \right) = \sum_{j=0}^n  \sum_{l=0}^n a_jb_l, \]
wohingegen für Ersteres gilt: 
\[  \sum_{k=0}^n c_k =\sum_{k=0}^n \sum_{j=0}^k a_jb_{k-j} = \sum_{j,l\geq 0; j+l\leq n} a_jb_l. \]
Bei der $n$-ten Partialsumme des Cauchy-Produkts werden also nur die Produkte $a_jb_l$ aufsummiert, bei denen $j+l\leq n$ ist, wohingegen beim
Produkt der $n$-ten Partialsummen alle $a_jb_l$ aufsummiert werden, bei denen $j\leq n$ und $l\leq n$ ist.

Dies bedeutet aber auch, dass im Grenzübergang $\lim_{n\to \infty} \left( \sum_{k=0}^n c_k \right)$ die gleichen Elemente aufsummiert werden wie bei
$ \lim_{n\to \infty}  \left( \sum_{j=0}^n a_j \right)\cdot \left( \sum_{l=0}^n b_l \right) $, jedoch in einer anderen Reihenfolge!

Bei absolut konvergenten Reihen kann die Summation beliebig vertauscht werden, ohne den Grenzwert zu ändern.
Bei bedingt konvergenten Reihen ist jedoch die Reihenfolge der Summation wichtig! 
(Vergleiche den \ref[content_26_produkt_von_reihen][vorhergehenden Abschnitt]{sec:reihen-umordnung}.)

Dies ist der Grund, weshalb das Cauchy-Produkt nur bei absolut konvergenten Reihen mit Sicherheit das richtige Ergebnis liefert.
\end{remark}


\begin{example}
Sei $q\in \C$ mit $0<|q|<1$. Wir betrachten die geometrische Reihe $\sum_{k=0}^\infty q^k$.
Diese ist absolut konvergent und ihr Grenzwert ist 
\[ \sum_{k=0}^\infty q^k =\frac{1}{1-q}. \]
Als Cauchy-Produkt dieser geometrischen Reihe mit sich selbst erhalten wir die Reihe $\sum_{k=0}^\infty c_k$ mit
\[ c_k= \sum_{j=0}^k q^j\cdot q^{k-j}=\sum_{j=0}^k q^k =(k+1)q^k\]
für alle $k\geq 0$, also die Reihe
\[  \sum_{k=0}^\infty  (k+1)q^k.\]
Nach obigem Satz konvergiert das Cauchy-Produkt gegen das Produkt der Grenzwerte, also ist
\[  \sum_{k=0}^\infty  (k+1)q^k =\left( \frac{1}{1-q}\right)^2=\frac{1}{(1-q)^2}. \]
\end{example}

\begin{example}[Funktionalgleichung der Exponentialreihe] \label{ex:funktionalgleichung-exp}
Seien $z$ und $w$ komplexe Zahlen. Wir betrachten die \ref[abs-konv][Exponentialreihen]{ex:exp-reihe} zu $z$ und $w$
\[ \exp(z)=\sum_{n=0}^\infty \frac{z^n}{n!}\quad \text{bzw.} \quad \exp(w)=\sum_{n=0}^\infty \frac{w^n}{n!}. \]
Beide sind absolut konvergent und ihr Cauchy-Produkt $\sum_{k=0}^\infty c_k$ ist gegeben durch
\begin{eqnarray*}
 c_k &=&  \sum_{j=0}^k \frac{z^j}{j!}\cdot \frac{w^{k-j}}{(k-j)!} =\frac{1}{k!}\cdot \sum_{j=0}^k \binom{k}{j} z^jw^{k-j}\\
  &=& \frac{1}{k!}(z+w)^k
   \end{eqnarray*}
nach der Definition der \link{content_03_binomischer_lehrsatz}{Binomialkoeffizienten} und der allgemeinen \ref[content_03_binomischer_lehrsatz][binomischen Formel]{thm:binom}.

Das Cauchy-Produkt ist also genau die Exponentialreihe zu $z+w$:
\[ \exp(z+w)=\sum_{k=0}^\infty \frac{(z+w)^k}{k!} .\]
Wir haben die sogenannte Funktionalgleichung $\exp(z+w)=\exp(z)\cdot\exp(w)$ für die Exponentialreihe gezeigt. 
Vergleiche mit dem \ref[content_15_exponentialfunktionen][Potenzgesetz für Exponentialfunktionen]{{functional_eqn_exp}}!
\end{example}

Der Abschnitt zum Cauchy-Produkt mit dem Beispiel der Exponentialreihe ist in folgendem Video zusammengefasst:
\floatright{\href{https://api.stream24.net/vod/getVideo.php?id=10962-2-10913&mode=iframe&speed=true}{\image[75]{00_video_button_schwarz-blau}}}\\


\begin{quickcheck}
 \type{input.number}
  \field{rational}
 
  \begin{variables}
   \randint{n}{2}{5}
   \randint{m}{2}{5}
   \function[calculate]{c0}{1}
   \function[calculate]{c1}{m+1/n}
   \function[calculate]{c2}{m^2/2+m/n+1/n^2}
  \end{variables}
  
\text{
Berechnen Sie die ersten drei Koeffizienten des Cauchy-Produkts der Reihen $\sum_{k=0}^\infty (\frac{1}{\var{n}})^k$ und $\sum_{j=0}^\infty\frac{\var{m}^j}{j!}$.

$c_0=$ \ansref ,
$c_1=$ \ansref ,
$c_2=$ \ansref .
}
\begin{answer}
\solution{c0}
\end{answer}
\begin{answer}
\solution{c1}
\end{answer}
\begin{answer}
\solution{c2}
\end{answer}
\end{quickcheck}

\end{content}