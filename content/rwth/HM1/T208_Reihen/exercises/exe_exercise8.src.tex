\documentclass{mumie.element.exercise}
%$Id$
\begin{metainfo}
  \name{
    \lang{de}{Ü08: Cauchy-Produkt}
    \lang{en}{Exercise 8}
  }
  \begin{description} 
 This work is licensed under the Creative Commons License Attribution 4.0 International (CC-BY 4.0)   
 https://creativecommons.org/licenses/by/4.0/legalcode 

    \lang{de}{}
    \lang{en}{}
  \end{description}
  \begin{components}
  \end{components}
  \begin{links}
\link{generic_article}{content/rwth/HM1/T208_Reihen/g_art_content_26_produkt_von_reihen.meta.xml}{content_26_produkt_von_reihen}
\link{generic_article}{content/rwth/HM1/T208_Reihen/g_art_content_25_konvergenz_kriterien.meta.xml}{content_25_konvergenz_kriterien}
\end{links}
  \creategeneric
\end{metainfo}
\begin{content}

\title{
\lang{de}{Ü08: Cauchy-Produkt}
\lang{en}{Exercise 8}
}
 \begin{block}[annotation]
	Im Ticket-System: \href{http://team.mumie.net/issues/9874}{Ticket 9874}
\end{block}
 
Es sei $q\in\C$ mit $|q|<1$. Bestimmen Sie das Cauchy-Produkt der Reihen
\[\sum_{k=0}^{\infty}{q^{k}}\quad\text{ und }\quad\sum_{j=0}^{\infty}{(-q)^{j}}\,.\]
 


\begin{tabs*}[\initialtab{0}\class{exercise}]
  \tab{
  \lang{de}{Lösung}
  }
Die Voraussetzung an $q$ garantiert uns, dass die beiden gegebenen 
\ref[content_25_konvergenz_kriterien][geometrischen Reihen]{ex:absolut-konvergenz} absolut 
konvergieren mit den Grenzwerten
\[
 \sum_{k=0}^{\infty}{q^{k}}=\frac{1}{1-q} \quad \text{und} \quad \sum_{j=0}^{\infty}{(-q)^{j}}=\frac{1}{1+q}.
\] 
Wir bestimmen nun das \ref[content_26_produkt_von_reihen][Cauchy-Produkt]{def:cauchy-prod}. 
Dieses hat die Form $\sum_{k=0}^{\infty}{c_{k}}$ mit 
\[c_{k}=\sum_{j=0}^{k}{(-1)^{j}q^{j}\cdot q^{k-j}}=q^{k}\cdot\sum_{j=0}^{k}{(-1)^{j}}=\begin{cases}
                                                                                                                                  q^{k},&\text{ falls }k\text{ gerade,}\\
0,&\text{ falls }k\text{ ungerade.}
                                                                                                                                 \end{cases}\,\]
Bemerkung: Daraus erhalten wir außerdem
\[\frac{1}{1-q^{2}}=\frac{1}{1-q}\cdot\frac{1}{1+q}=\sum_{k=0}^{\infty}{q^{2k}}\,.\]
\end{tabs*}

\end{content}