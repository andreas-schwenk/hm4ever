\documentclass{mumie.element.exercise}
%$Id$
\begin{metainfo}
  \name{
    \lang{de}{Ü01: Konvergenz}
    \lang{en}{Exercise 1}
  }
  \begin{description} 
 This work is licensed under the Creative Commons License Attribution 4.0 International (CC-BY 4.0)   
 https://creativecommons.org/licenses/by/4.0/legalcode 

    \lang{de}{}
    \lang{en}{}
  \end{description}
  \begin{components}
  \end{components}
  \begin{links}
\link{generic_article}{content/rwth/HM1/T208_Reihen/g_art_content_24_reihen_und_konvergenz.meta.xml}{content_24_reihen_und_konvergenz}
\end{links}
  \creategeneric
\end{metainfo}
\begin{content}

\title{
\lang{de}{Ü1: Konvergenz}
\lang{en}{Exercise 1}
}
 
 \begin{block}[annotation]
	Im Ticket-System: \href{http://team.mumie.net/issues/9867}{Ticket 9867}
\end{block}
 
\lang{de}{Zeigen Sie, dass die Reihe \[\sum_{k=0}^{\infty}{\frac{(-1)^{k}}{(2k+1)!}}\] konvergiert.}

\begin{tabs*}[\initialtab{0}\class{exercise}]
  \tab{
  \lang{de}{Lösung}
  }

  \begin{incremental}[\initialsteps{1}]
     \step \lang{de}{Die Folge $(a_{k})_{k\in\N_{0}}$ mit
\[a_k := \frac{1}{(2k+1)!}\]
ist eine (streng) monoton fallende Nullfolge, denn es gilt für jedes $k\in\N_{0}$
\[a_{k+1}=\frac{1}{2k+3} \, \frac{1}{2k+2}\,a_{k}<a_{k}\,. \] }
     \step \lang{de}{Nach dem \ref[content_24_reihen_und_konvergenz][Leibniz-Kriterium]{thm:leibnizkriterium} konvergiert somit die Reihe
\[\sum_{k=1}^{\infty}\frac{(-1)^k}{(2k+1) !}\,.\]
Daher konvergiert auch die Reihe \[\sum_{k=0}^{\infty}{\frac{(-1)^{k}}{(2k+1)!}}\,,\]
denn diese Reihe unterscheidet sich von der Vorherigen nur um den Summanden $a_{0}=1$.}
  \end{incremental}

\end{tabs*}

\end{content}