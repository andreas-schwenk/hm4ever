\documentclass{mumie.element.exercise}
%$Id$
\begin{metainfo}
  \name{
    \lang{de}{Ü07: Cauchy-Produkt}
    \lang{en}{Exercise 7}
  }
  \begin{description} 
 This work is licensed under the Creative Commons License Attribution 4.0 International (CC-BY 4.0)   
 https://creativecommons.org/licenses/by/4.0/legalcode 

    \lang{de}{}
    \lang{en}{}
  \end{description}
  \begin{components}
  \end{components}
  \begin{links}
\link{generic_article}{content/rwth/HM1/T208_Reihen/g_art_content_26_produkt_von_reihen.meta.xml}{content_26_produkt_von_reihen}
\link{generic_article}{content/rwth/HM1/T208_Reihen/g_art_content_25_konvergenz_kriterien.meta.xml}{content_25_konvergenz_kriterien}
\end{links}
  \creategeneric
\end{metainfo}
\begin{content}

\title{
\lang{de}{Ü07: Cauchy-Produkt}
\lang{en}{Exercise 7}
}
 \begin{block}[annotation]
	Im Ticket-System: \href{http://team.mumie.net/issues/9873}{Ticket 9873}
\end{block}
 
Bestimmen Sie das Cauchy-Produkt der beiden Reihen
\[\sum_{k=0}^{\infty}{\frac{z^{k}}{k!}}\quad\text{ und }\quad\sum_{j=0}^{\infty}{w^{j}},\]
wobei wir $z,w\in\C$ mit $|w|<1$ voraussetzen. 

Bestimmen Sie mit Hilfe des Ergebnisses eine Reihendarstellung
für die komplexe Zahl $\frac{\exp(z)}{1-z}. $ 
 


\begin{tabs*}[\initialtab{0}\class{exercise}]
  \tab{
  \lang{de}{Lösung}
  }
Die erste Reihe ist die \ref[content_25_konvergenz_kriterien][Exponentialreihe]{ex:exp-reihe}, 
die für jedes $z\in\C$ absolut konvergiert. 
Die zweite Reihe ist eine \ref[content_25_konvergenz_kriterien][geometrische Reihe]{ex:absolut-konvergenz} in $w$, 
welche wegen $|w|<1$ ebenfalls absolut konvergiert. 
Daher können wir das \ref[content_26_produkt_von_reihen][Cauchy-Produkt]{def:cauchy-prod} bilden,
das die Form 
\[\sum_{l=0}^{\infty}{c_{l}}\quad\text{ mit }\quad c_{l}=\sum_{m=0}^{l}{\frac{z^{m}}{m!}w^{l-m}}\]
hat. Mit
\[
 \exp(z)=\sum_{k=0}^{\infty}{\frac{z^{k}}{k!}} \quad \text{und} \quad \frac{1}{1-w}=\sum_{j=0}^{\infty}{w^{j}}
\] 
erhalten wir weiter
\[\frac{\exp(z)}{1-w}=\sum_{k=0}^{\infty}{\sum_{l=0}^{k}{\frac{z^{l}w^{k-l}}{l!}}}\,.\]
Falls speziell $z=w$ ist, so hat diese Reihe die Gestalt
\[\frac{\exp(z)}{1-z}=\sum_{k=0}^{\infty}{\left(\sum_{l=0}^{k}{\frac{1}{l!}}\right)z^{k}}\,.\] 

\end{tabs*}

\end{content}