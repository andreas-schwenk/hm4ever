\documentclass{mumie.element.exercise}
%$Id$
\begin{metainfo}
  \name{
    \lang{de}{Ü02: Reihenwert}
    \lang{en}{Exercise 2}
  }
  \begin{description} 
 This work is licensed under the Creative Commons License Attribution 4.0 International (CC-BY 4.0)   
 https://creativecommons.org/licenses/by/4.0/legalcode 

    \lang{de}{}
    \lang{en}{}
  \end{description}
  \begin{components}
  \end{components}
  \begin{links}
\link{generic_article}{content/rwth/HM1/T208_Reihen/g_art_content_24_reihen_und_konvergenz.meta.xml}{content_24_reihen_und_konvergenz}
\end{links}
  \creategeneric
\end{metainfo}
\begin{content}

\title{
\lang{de}{Ü02: Reihenwert}
\lang{en}{Exercise 2}
}
 
 \begin{block}[annotation]
	Im Ticket-System: \href{http://team.mumie.net/issues/9868}{Ticket 9868}
\end{block}
 
\lang{de}{Berechnen Sie jeweils den Wert der folgenden Reihen:}
\begin{enumerate}
\item 
\begin{table}[\class{items}]
\nowrap{a) $\displaystyle \ \sum_{k=0}^{\infty}{\left(\frac{3}{5}\right)^{k}}$} \\
\nowrap{b) $\displaystyle \ \sum_{k=0}^{\infty}{\left(-\frac{1}{2}\right)^{k}}$} \\
\nowrap{c) $\displaystyle \ \sum_{k=0}^{\infty}{\left[\left(\frac{1}{3}\right)^{k}-\left(\frac{1}{2}\right)^{k+1}\right]}$}
\end{table}
\item
\begin{table}[\class{items}]
\nowrap{a) $\displaystyle \ \sum_{k=0}^{\infty}{\left(\frac{1}{3}\right)^{k}}$} \\
\nowrap{b) $\displaystyle \ \sum_{n=0}^{\infty}{\frac{1}{4^n}}$} \\
\nowrap{c) $\displaystyle \ \sum_{k=1}^{\infty}0,8^k$} \\
\nowrap{d) $\displaystyle \ \sum_{m=2}^{\infty}{\left(\frac{1}{2}\right)^{m}}$}
\end{table}
\end{enumerate}

\begin{tabs*}[\initialtab{0}\class{exercise}]


\tab{Antworten}
\begin{enumerate}
\item
\begin{table}[\class{items}]
\nowrap{a) $\displaystyle \ \sum_{k=0}^{\infty}{\left(\frac{3}{5}\right)^{k}}=\frac{5}{2}$} \\
\nowrap{b) $\displaystyle \ \sum_{k=0}^{\infty}{\left(-\frac{1}{2}\right)^{k}}=\frac{2}{3}$} \\
\nowrap{c) $\displaystyle \ \sum_{k=0}^{\infty}{\left[\left(\frac{1}{3}\right)^{k}-\left(\frac{1}{2}\right)^{k+1}\right]}=\frac{1}{2}$}
\end{table}
\item
\begin{table}[\class{items}]
\nowrap{a) $\displaystyle \ \sum_{k=0}^{\infty}{\left(\frac{1}{3}\right)^{k}}=\frac{3}{2}$} \\
\nowrap{b) $\displaystyle \ \sum_{n=0}^{\infty}{\frac{1}{4^n}}=\frac{4}{3}$} \\
\nowrap{b) $\displaystyle \ \sum_{k=1}^{\infty}0,8^k=4$} \\
\nowrap{c) $\displaystyle \ \sum_{m=2}^{\infty}{\left(\frac{1}{2}\right)^{m}}=\frac{1}{2}$}
\end{table}
\end{enumerate}

  \tab{
  \lang{de}{Lösung 1.a)}
  }
\lang{de}{Die Reihe ist eine \ref[content_24_reihen_und_konvergenz][geometrische Reihe] {ex:konvergenz-geo-reihe} mit $q=3/5$. Somit können wir den Reihenwert berechnen mit folgender Formel:
\[\sum_{k=0}^{\infty}{\left(\frac{3}{5}\right)^{k}}=\frac{1}{1-q}=\frac{1}{1-\frac{3}{5}}=\frac{5}{2}\,.\]}

\tab{
\lang{de}{Lösung 1.b)}}
\lang{de}{Die Reihe ist eine \ref[content_24_reihen_und_konvergenz][geometrische Reihe] {ex:konvergenz-geo-reihe} mit $q=-1/2$. Somit können wir den Reihenwert berechnen mit folgender Formel:
\[\sum_{k=0}^{\infty}{\left(-\frac{1}{2}\right)^{k}}=\frac{1}{1-q}=\frac{1}{1+\frac{1}{2}}=\frac{2}{3}\,.\]}

\tab{
\lang{de}{Lösung 1.c)}}


  \begin{incremental}[\initialsteps{1}]
     \step \lang{de}{Die Reihen $\sum_{k=0}^{\infty}{\left(\frac{1}{3}\right)^{k}}$ und $\sum_{k=0}^{\infty}{\left(\frac{1}{2}\right)^{k}}$ konvergieren absolut als geometrische Reihen.}
     \step \lang{de}{Nach den \ref[content_24_reihen_und_konvergenz][Grenzwertregeln]{rul:grenzwert-regeln} konvergiert somit auch die Reihe
\[\sum_{k=0}^{\infty}{\left(\frac{1}{3}\right)^{k}}-\frac{1}{2}\sum_{k=0}^{\infty}{\left(\frac{1}{2}\right)^{k}}=\sum_{k=0}^{\infty}{\left[\left(\frac{1}{3}\right)^{k}-\left(\frac{1}{2}\right)^{k+1}\right]}\]
und hat den Wert
\[\frac{1}{1-\frac{1}{3}}-\frac{1}{2} \cdot \frac{1}{1-\frac{1}{2}}=\frac{3}{2}-1=\frac{1}{2}\,.\]}
  \end{incremental}

    \tab{\lang{de}{Lösungsvideo 2)}}	
    \youtubevideo[500][300]{eOQ_1K4078w}\\


\end{tabs*}

\end{content}