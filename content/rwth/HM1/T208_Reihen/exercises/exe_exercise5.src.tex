\documentclass{mumie.element.exercise}
%$Id$
\begin{metainfo}
  \name{
    \lang{de}{Ü05: Konvergenz}
    \lang{en}{Exercise 5}
  }
  \begin{description} 
 This work is licensed under the Creative Commons License Attribution 4.0 International (CC-BY 4.0)   
 https://creativecommons.org/licenses/by/4.0/legalcode 

    \lang{de}{}
    \lang{en}{}
  \end{description}
  \begin{components}
  \end{components}
  \begin{links}
\link{generic_article}{content/rwth/HM1/T205_Konvergenz_von_Folgen/g_art_content_16_konvergenzkriterien.meta.xml}{content_16_konvergenzkriterien}
\link{generic_article}{content/rwth/HM1/T208_Reihen/g_art_content_25_konvergenz_kriterien.meta.xml}{content_25_konvergenz_kriterien}
\end{links}
  \creategeneric
\end{metainfo}
\begin{content}

\title{
\lang{de}{Ü05: Konvergenz}
\lang{en}{Exercise 5}
}
 
 \begin{block}[annotation]
	Im Ticket-System: \href{http://team.mumie.net/issues/9871}{Ticket 9871}
\end{block}
 
\begin{table}[\class{items}]
\nowrap{a) Zeigen Sie die Konvergenz der Reihe $\sum_{k=0}^{\infty} \frac{4(2i)^k}{3^k}$ mit Hilfe des Wurzelkriteriums.} \\
\nowrap{b) Zeigen Sie die Divergenz der Reihe $\sum_{k=1}^{\infty} \frac{1}{k+\frac{1}{k}}$  mit Hilfe des Minorantenkriteriums.}
\end{table}

In welchen Fällen liegt absolute Konvergenz vor?


\begin{tabs*}[\initialtab{0}\class{exercise}]
  \tab{
  \lang{de}{Lösung a)}
  }
\lang{de}{Für $k \in \mathbb{N}$ ist $\left|\left( 2i \right)^{k}\right| = \left|2i\right|^{k} = 2^{k}$. Damit folgt
\begin{eqnarray*}
\lim_{k\rightarrow \infty }\sqrt[k]{\left| \frac{4(2i)^{k}}{3^{k}}\right| } &=&
\lim_{k\rightarrow \infty }\frac{\sqrt[k]{4}\sqrt[k]{2^{k}}}{\sqrt[k]{3^{k}}} \\
& = &\lim_{k\rightarrow \infty }\frac{2}{3}\sqrt[k]{4} \\
& = & \frac{2}{3}\lim_{k\rightarrow \infty }\sqrt[k]{4} \\
& \overset{\ref[content_16_konvergenzkriterien][\lim_{k\rightarrow \infty }\sqrt[k]{q}=1]{sec:wichtige-beispiele}}{=} & \frac{2}{3} < 1.
\end{eqnarray*}
Nach dem \ref[content_25_konvergenz_kriterien][Wurzelkriterium]{thm:wurzelkriterium}  konvergiert die Reihe $\sum_{k=0}^{\infty }\frac{4(2i)^{k}}{3^{k}}$ also (absolut).



\textbf{Bemerkung:} Man kann die Reihe als geometrische Reihe auffassen und erh\"{a}lt als Wert der Reihe
\[4 \cdot \sum_{k=0}^{\infty} \left( \frac{2i}{3} \right)^{k} = 4 \cdot \frac{1}{1 - \frac{2i}{3}} = \frac{36}{13}+\frac{24}{13}i.\]
}

\tab{
\lang{de}{Lösung b)}}
\lang{de}{Für $k \in \mathbb{N}$ ist $\frac{1}{k} \leq 1 \leq k$, also $k + \frac{1}{k} \leq k+k = 2k $. Damit folgt
\[
    \frac{1}{k+\frac{1}{k}}\geq \frac{1}{2} \cdot \frac{1}{k}
\]
für alle $k\in \mathbb{N}$. Da die harmonische Reihe $\sum_{k=1}^{\infty }\frac{1}{k}$ divergiert, 
folgt nach dem \ref[content_25_konvergenz_kriterien][Minorantenkriterium]{thm:majoranten-krit}  
die Divergenz der Reihe \[\sum_{k=1}^{\infty }\frac{1}{k+\frac{1}{k}}. \]
Eine  Reihe, die nicht konvergiert,  kann  erst recht nicht absolut konvergieren. 
}

\end{tabs*}

\end{content}