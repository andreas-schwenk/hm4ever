\documentclass{mumie.problem.gwtmathlet}
%$Id$
\begin{metainfo}
  \name{
    \lang{de}{A07: Konvergenz}
    \lang{en}{Problem 7}
  }
  \begin{description} 
 This work is licensed under the Creative Commons License Attribution 4.0 International (CC-BY 4.0)   
 https://creativecommons.org/licenses/by/4.0/legalcode 

    \lang{de}{}
    \lang{en}{}
  \end{description}
  \corrector{system/problem/GenericCorrector.meta.xml}
  \begin{components}
    \component{js_lib}{system/problem/GenericMathlet.meta.xml}{mathlet}
  \end{components}
  \begin{links}
  \end{links}
  \creategeneric
\end{metainfo}
\begin{content}
\begin{block}[annotation]
	Im Ticket-System: \href{https://team.mumie.net/issues/18131}{Ticket 18131}
\end{block}
\usepackage{mumie.genericproblem}
\lang{de}{\title{A07: Konvergenz}}
\lang{en}{\title{Problem 7}}
\begin{block}[annotation]
Aufgabenpool mit vier Aufgaben, aus dem zufällig ausgewählt wird.
\end{block}

\begin{problem}
\randomquestionpool{1}{4}
%% Aufgabe 1
  \begin{question}
    \type{mc.multiple}
    \text{Entscheiden Sie, welche der folgenden Reihen konvergieren.}
	%%
	\begin{choice}
        		\text{(a)  $\sum_{n=1}^{\infty} {\frac{(-1)^{n}n}{n+1}}$ }
		\solution{false}
        \explanation[NOT[equalChoice(00000)] AND equalChoice(1???0)]{
         (a) divergiert: Die Koeffizienten bilden keine Nullfolge.}
        	\end{choice}
	\begin{choice}
        \text{(b) $\sum_{n=1}^{\infty} {\frac{(-1)^{n+1}}{\sqrt{n}}}$  }
		\solution{true}
        \explanation[NOT[equalChoice(00000)] AND equalChoice(?0??0)]{(b) konvergiert: Die Koeffizienten bilden eine alternierende Nullfolge, 
            also  greift das Leibniz-Kriterium. }
     \end{choice}
	\begin{choice}
       	\text{(c) $\sum_{n=1}^{\infty} {\frac{(-1)^{n-1}}{\sqrt[n]{n}}}$}
		\solution{false}
        \explanation[NOT[equalChoice(00000)] AND equalChoice(??1?0)]{ (c) divergiert: 
            Die Koeffizienten bilden keine Nullfolge, weil $\lim_{n\to\infty} \sqrt[n]{n}=1$.}
     \end{choice}
	\begin{choice}
        \text{(d) $\sum_{n=1}^{\infty} {\frac{(-1)^{n^2}}{n^2}}$ }
		\solution{true}
        \explanation[NOT[equalChoice(00000)] AND equalChoice(???00)]{(d) konvergiert: Die Koeffizienten bilden eine alternierende Nullfolge, 
            also  greift das Leibniz-Kriterium. Alternativ: Majorantenkriterium.}
	\end{choice}
    \begin{choice}
     \text{(e) Keine der angegebenen Reihen konvergiert.}
     \solution{false}
     \explanation[equalChoice(????1)]{Nicht alle angegebenen Reihen divergieren. Prüfen Sie mit bekannten Konvergenzkriterien für Reihen.}
     \end{choice}
  \end{question}

%% Aufgabe 2
  \begin{question}
    \type{mc.multiple}
    \text{Entscheiden Sie, welche der folgenden Reihen konvergieren.}
	%%
	\begin{choice}
       	\text{(a) $\sum_{n=1}^{\infty} {\frac{(-1)^{n-1}}{\sqrt[n]{n}}}$}
		\solution{false}
        \explanation[NOT[equalChoice(00000)] AND equalChoice(1???0)]{ (a) divergiert: 
            Die Koeffizienten bilden keine Nullfolge, weil $\lim_{n\to\infty} \sqrt[n]{n}=1$.}
     \end{choice}
	\begin{choice}
        \text{(b) $\sum_{n=1}^{\infty} {\frac{(-1)^{n+1}}{\sqrt{n}}}$  }
		\solution{true}
        \explanation[NOT[equalChoice(00000)] AND equalChoice(?0??0)]{(b) konvergiert: Die Koeffizienten bilden eine alternierende Nullfolge, 
            also  greift das Leibniz-Kriterium.}
     \end{choice}
     \begin{choice}
        		\text{(c)  $\sum_{n=1}^{\infty} {\frac{(-1)^{n}n}{n+1}}$ }
		\solution{false}
        \explanation[NOT[equalChoice(00000)] AND equalChoice(??1?0)]{
         (c) divergiert: Die Koeffizienten bilden keine Nullfolge.}
     \end{choice}	
	\begin{choice}
        \text{(d) $\sum_{n=1}^{\infty} {\frac{(-1)^{n^2}}{n^2}}$ }
		\solution{true}
        \explanation[NOT[equalChoice(00000)] AND equalChoice(???00)]{(d) konvergiert: Die Koeffizienten bilden eine alternierende Nullfolge, 
            also  greift das Leibniz-Kriterium. Alternativ: Majorantenkriterium.}
	\end{choice}
     \begin{choice}
     \text{(e) Keine der angegebenen Reihen konvergiert.}
     \solution{false}
     \explanation[equalChoice(????1)]{Nicht alle angegebenen Reihen divergieren. Prüfen Sie mit bekannten Konvergenzkriterien für Reihen.}
     \end{choice}
  \end{question}
  
%% Aufgabe 3
  \begin{question}
    \type{mc.multiple}
    \text{Entscheiden Sie, welche der folgenden Reihen konvergieren.}
	%%
	\begin{choice}
        		\text{(a)  $\sum_{n=1}^{\infty} {\frac{(-1)^{n}n}{n+1}}$ }
		\solution{false}
        \explanation[NOT[equalChoice(00000)] AND equalChoice(1???0)]{
         (a) divergiert: Die Koeffizienten bilden keine Nullfolge.}
        	\end{choice}
	\begin{choice}
        \text{(b) $\sum_{n=1}^{\infty} {\frac{(-1)^{n^2}}{n^2}}$ }
		\solution{true}
        \explanation[NOT[equalChoice(00000)] AND equalChoice(?0??0)]{(b) konvergiert: Die Koeffizienten bilden eine alternierende Nullfolge, 
            also greift das Leibniz-Kriterium. Alternativ: Majorantenkriterium.}
	\end{choice}
	\begin{choice}
       	\text{(c) $\sum_{n=1}^{\infty} {\frac{(-1)^{n-1}}{\sqrt[n]{n}}}$}
		\solution{false}
        \explanation[NOT[equalChoice(00000)] AND equalChoice(??1?0)]{ (c) divergiert: 
            Die Koeffizienten bilden keine Nullfolge, weil $\lim_{n\to\infty} \sqrt[n]{n}=1$.}
     \end{choice}	
    \begin{choice}
        \text{(d) $\sum_{n=1}^{\infty} {\frac{(-1)^{n+1}}{\sqrt{n}}}$  }
		\solution{true}
        \explanation[NOT[equalChoice(00000)] AND equalChoice(???00)]{(d) konvergiert: Die Koeffizienten bilden eine alternierende Nullfolge, 
            also  greift das Leibniz-Kriterium.}
     \end{choice}
      \begin{choice}
     \text{(e) Keine der angegebenen Reihen konvergiert.}
     \solution{false}
     \explanation[equalChoice(????1)]{Nicht alle angegebenen Reihen divergieren. Prüfen Sie mit bekannten Konvergenzkriterien für Reihen.}
     \end{choice}
  \end{question}
  
  
%% Aufgabe 4
  \begin{question}
    \type{mc.multiple}
    \text{Entscheiden Sie, welche der folgenden Reihen konvergieren.}
	%%
    \begin{choice}
        \text{(a) $\sum_{n=1}^{\infty} {\frac{(-1)^{n^2}}{n^2}}$ }
		\solution{true}
        \explanation[NOT[equalChoice(00000)] AND equalChoice(0???0)]{(a) konvergiert: Die Koeffizienten bilden eine alternierende Nullfolge, 
            also  greift das Leibniz-Kriterium. Alternativ: Majorantenkriterium.}
	\end{choice}
	\begin{choice}
        		\text{(b)  $\sum_{n=1}^{\infty} {\frac{(-1)^{n}n}{n+1}}$ }
		\solution{false}
        \explanation[NOT[equalChoice(00000)] AND equalChoice(?1??0)]{
         (b) divergiert: Die Koeffizienten bilden keine Nullfolge.}
        	\end{choice}
	\begin{choice}
        \text{(c) $\sum_{n=1}^{\infty} {\frac{(-1)^{n+1}}{\sqrt{n}}}$  }
		\solution{true}
        \explanation[NOT[equalChoice(00000)] AND equalChoice(??0?0)]{(c) konvergiert: Die Koeffizienten bilden eine alternierende Nullfolge, 
            also  greift das Leibniz-Kriterium.}
     \end{choice}
	\begin{choice}
       	\text{(d) $\sum_{n=1}^{\infty} {\frac{(-1)^{n-1}}{\sqrt[n]{n}}}$}
		\solution{false}
        \explanation[NOT[equalChoice(00000)] AND equalChoice(???10)]{ (d) divergiert: 
            Die Koeffizienten bilden keine Nullfolge, weil $\lim_{n\to\infty} \sqrt[n]{n}=1$.}
     \end{choice}
     \begin{choice}
     \text{(e) Keine der angegebenen Reihen konvergiert.}
     \solution{false}
     \explanation[equalChoice(????1)]{Nicht alle angegebenen Reihen divergieren. Prüfen Sie mit bekannten Konvergenzkriterien für Reihen.}
     \end{choice}
  \end{question}
  
\end{problem}
\embedmathlet{mathlet}
\end{content}