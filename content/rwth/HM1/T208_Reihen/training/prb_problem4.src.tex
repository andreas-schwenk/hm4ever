\documentclass{mumie.problem.gwtmathlet}
%$Id$
\begin{metainfo}
  \name{
    \lang{de}{A05: Konvergenz}
    \lang{en}{Problem 5}
  }
  \begin{description} 
 This work is licensed under the Creative Commons License Attribution 4.0 International (CC-BY 4.0)   
 https://creativecommons.org/licenses/by/4.0/legalcode 

    \lang{de}{}
    \lang{en}{}
  \end{description}
  \corrector{system/problem/GenericCorrector.meta.xml}
  \begin{components}
    \component{js_lib}{system/problem/GenericMathlet.meta.xml}{mathlet}
  \end{components}
  \begin{links}
  \end{links}
  \creategeneric
\end{metainfo}
\begin{content}
\usepackage{mumie.genericproblem}
\lang{de}{\title{A05: Konvergenz}}
\lang{en}{\title{Problem 5}}
\begin{block}[annotation]
  Im Ticket-System: \href{http://team.mumie.net/issues/9879}{Ticket 9879}
\end{block}
\begin{block}[annotation]
Aufgabenpool mit vier Aufgaben, aus dem zufällig ausgewählt wird.
\end{block}

\begin{problem}
\randomquestionpool{1}{4}
%% Aufgabe 1
  \begin{question}
    \type{mc.multiple}
    \text{Gegeben sei die Reihe $\sum_{k=1}^{\infty}{a_{k}}$ und es gebe ein $N\in\mathbb{N}$ mit der Eigenschaft $a_{n}\neq 0$ für alle $n\geq N$. 
    Wir  betrachten die Folge $(b_{k})_{k\geq N}$, die gegeben ist durch die Folgeglieder $b_{k}:=\frac{a_{k+1}}{a_{k}}$. 
    Entscheiden Sie, welche der folgenden Aussagen wahr sind.}
	%\permutechoices{1}{4}
	\begin{choice}
        \text{(a) Ist $(b_{k})_{k\geq N}$ eine Nullfolge, so konvergiert die Reihe $\sum_{k=1}^{\infty}{a_{k}}$ absolut.}
		\solution{true}
        \explanation[equalChoice(0???)]{
        Aussage (a) ist wahr:
        Es gibt insbesondere ein $N_0$ so, dass $\vert b_k\vert<\frac{1}{2}=:\beta<1$ für alle $k>N_0$ gilt.
        Benutze dann das Quotientenkriterium. }
        	\end{choice}
	\begin{choice}
        \text{(b) Ist das Folgeglied $b_{N}\in (0,1)$ und $(b_{k})_{k\geq N}$ monoton fallend, so konvergiert die Reihe $\sum_{k=1}^{\infty}{a_{k}}$ absolut.}
		\solution{false}
        \explanation[equalChoice(?1??)]{Aussage (b) ist falsch: Hier könnten die $b_k$ negativ werden mit $\lim_{k\to\infty}b_k=-1$. 
        Betrachte zum Beispiel die leicht variierte alternierende harmonische Reihe 
        mit $a_k=\frac{(-1)^k}{k}$ für $k\geq 2$ und $a_1=1$ sowie $N=1$. }
     \end{choice}
	\begin{choice}
       	\text{(c) Aus der Eigenschaft $b_{k}\in (-1,1)$ für alle $k\geq N$ 
        ist keine Konvergenzaussage für die Reihe $\sum_{k=1}^{\infty}{a_{k}}$ möglich.}
		\solution{true}
        \explanation[equalChoice(??0?)]{Aussage (c) ist wahr: Betrachte zum Beispiel die harmonische Reihe.}
     \end{choice}
	\begin{choice}
        \text{(d) Falls $b_{k}\geq 1$ für alle bis auf endlich viele $k\in\N$ gilt, 
        so ist die Reihe $\sum_{k=1}^{\infty}{a_{k}}$ divergent.}
		\solution{true}
        \explanation[equalChoice(???0)]{Aussage (d) ist wahr: Die Bedingung  besagt, dass $(a_{k})_{k\in\N}$ keine Nullfolge ist.}
	\end{choice}
%    \explanation{
 %   "{Ist $b_{N}\in (0,1)$ und $(b_{k})_{k\geq N}$ monoton fallend, so konvergiert die Reihe absolut.}" ist falsch, 
 %   denn die $b_k$ könnten negativ werden mit $\lim_{k\to\infty}b_k=-1$. Betrachte zum Beispiel die leicht variierte alternierende harmonische Reihe 
 %   mit $a_k=\frac{(-1)^k}{k}$ für $k\geq 2$ und $a_1=1$ sowie $N=1$. \\
 %   Für $b_{k}\in (-1,1)$ für alle $k\geq N$ betrachte man etwa die harmonische Reihe. \\
 %   }
  \end{question}

%% Aufgabe 2
  \begin{question}
    \type{mc.multiple}
    \text{Gegeben sei die Reihe $\sum_{k=1}^{\infty}{a_{k}}$ und es gebe ein $N\in\mathbb{N}$ mit der Eigenschaft $a_{n}\neq 0$ für alle $n\geq N$. 
    Wir  betrachten die Folge $(b_{k})_{k\geq N}$, die gegeben ist durch die Folgeglieder $b_{k}:=\frac{a_{k+1}}{a_{k}}$. 
    Entscheiden Sie, welche der folgenden Aussagen wahr sind.}
	%\permutechoices{1}{4}
    \begin{choice}
        \text{(a) Ist das Folgeglied $b_{N}\in (0,1)$ und $(b_{k})_{k\geq N}$ monoton fallend, so konvergiert die Reihe $\sum_{k=1}^{\infty}{a_{k}}$ absolut.}
		\solution{false}
        \explanation[equalChoice(1???)]{Aussage (a) ist falsch: Hier könnten die $b_k$ negativ werden mit $\lim_{k\to\infty}b_k=-1$. 
        Betrachte zum Beispiel die leicht variierte alternierende harmonische Reihe 
        mit $a_k=\frac{(-1)^k}{k}$ für $k\geq 2$ und $a_1=1$ sowie $N=1$. }
     \end{choice}
	\begin{choice}
        		\text{(b) Ist $(b_{k})_{k\geq N}$ eine Nullfolge, so konvergiert die Reihe $\sum_{k=1}^{\infty}{a_{k}}$ absolut.}
		\solution{true}
        \explanation[equalChoice(?0??)]{
       Aussage (b) ist wahr:
        Es gibt insbesondere ein $N_0$ so, dass $\vert b_k\vert<\frac{1}{2}=:\beta<1$ für alle $k>N_0$ gilt.
        Benutze dann das Quotientenkriterium. }
        	\end{choice}
	\begin{choice}
        \text{(c) Falls $b_{k}\geq 1$ für alle bis auf endlich viele $k\in\N$ gilt, 
        so ist die Reihe $\sum_{k=1}^{\infty}{a_{k}}$ divergent.}
		\solution{true}
        \explanation[equalChoice(??0?)]{Aussage (c) ist wahr: Die Bedingung  besagt, dass $(a_{k})_{k\in\N}$ keine Nullfolge ist.}
	\end{choice}
	\begin{choice}
       	\text{(d) Aus der Eigenschaft $b_{k}\in (-1,1)$ für alle $k\geq N$ 
        ist keine Konvergenzaussage für die Reihe $\sum_{k=1}^{\infty}{a_{k}}$ möglich.}
		\solution{true}
        \explanation[equalChoice(???0)]{Aussage (d) ist wahr: Betrachte zum Beispiel die harmonische Reihe.}
     \end{choice}
	
%    \explanation{
 %   "{Ist $b_{N}\in (0,1)$ und $(b_{k})_{k\geq N}$ monoton fallend, so konvergiert die Reihe absolut.}" ist falsch, 
 %   denn die $b_k$ könnten negativ werden mit $\lim_{k\to\infty}b_k=-1$. Betrachte zum Beispiel die leicht variierte alternierende harmonische Reihe 
 %   mit $a_k=\frac{(-1)^k}{k}$ für $k\geq 2$ und $a_1=1$ sowie $N=1$. \\
 %   Für $b_{k}\in (-1,1)$ für alle $k\geq N$ betrachte man etwa die harmonische Reihe. \\
 %   }
  \end{question}
  
  %% Aufgabe 3
  \begin{question}
    \type{mc.multiple}
    \text{Gegeben sei die Reihe $\sum_{k=1}^{\infty}{a_{k}}$ und es gebe ein $N\in\mathbb{N}$ mit der Eigenschaft $a_{n}\neq 0$ für alle $n\geq N$. 
    Wir  betrachten die Folge $(b_{k})_{k\geq N}$, die gegeben ist durch die Folgeglieder $b_{k}:=\frac{a_{k+1}}{a_{k}}$. 
    Entscheiden Sie, welche der folgenden Aussagen wahr sind.}
	%\permutechoices{1}{4}
    \begin{choice}
        \text{(a) Falls $b_{k}\geq 1$ für alle bis auf endlich viele $k\in\N$ gilt, 
        so ist die Reihe $\sum_{k=1}^{\infty}{a_{k}}$ divergent.}
		\solution{true}
        \explanation[equalChoice(0???)]{Aussage (a) ist wahr: Die Bedingung  besagt, dass $(a_{k})_{k\in\N}$ keine Nullfolge ist.}
	\end{choice}
	\begin{choice}
        		\text{(b) Ist $(b_{k})_{k\geq N}$ eine Nullfolge, so konvergiert die Reihe $\sum_{k=1}^{\infty}{a_{k}}$ absolut.}
		\solution{true}
        \explanation[equalChoice(?0??)]{
        Aussage (b) ist wahr:
        Es gibt insbesondere ein $N_0$ so, dass $\vert b_k\vert<\frac{1}{2}=:\beta<1$ für alle $k>N_0$ gilt.
        Benutze dann das Quotientenkriterium. }
        	\end{choice}
	\begin{choice}
        \text{(c) Ist das Folgeglied $b_{N}\in (0,1)$ und $(b_{k})_{k\geq N}$ monoton fallend, so konvergiert die Reihe $\sum_{k=1}^{\infty}{a_{k}}$ absolut.}
		\solution{false}
        \explanation[equalChoice(??1?)]{Aussage (c) ist falsch: Hier könnten die $b_k$ negativ werden mit $\lim_{k\to\infty}b_k=-1$. 
        Betrachte zum Beispiel die leicht variierte alternierende harmonische Reihe 
        mit $a_k=\frac{(-1)^k}{k}$ für $k\geq 2$ und $a_1=1$ sowie $N=1$. }
     \end{choice}
	\begin{choice}
       	\text{(d) Aus der Eigenschaft $b_{k}\in (-1,1)$ für alle $k\geq N$ 
        ist keine Konvergenzaussage für die Reihe $\sum_{k=1}^{\infty}{a_{k}}$ möglich.}
		\solution{true}
        \explanation[equalChoice(???0)]{Aussage (d) ist wahr: Betrachte zum Beispiel die harmonische Reihe.}
     \end{choice}
	
%    \explanation{
 %   "{Ist $b_{N}\in (0,1)$ und $(b_{k})_{k\geq N}$ monoton fallend, so konvergiert die Reihe absolut.}" ist falsch, 
 %   denn die $b_k$ könnten negativ werden mit $\lim_{k\to\infty}b_k=-1$. Betrachte zum Beispiel die leicht variierte alternierende harmonische Reihe 
 %   mit $a_k=\frac{(-1)^k}{k}$ für $k\geq 2$ und $a_1=1$ sowie $N=1$. \\
 %   Für $b_{k}\in (-1,1)$ für alle $k\geq N$ betrachte man etwa die harmonische Reihe. \\
 %   }
  \end{question}
  
  %% Aufgabe 4
  \begin{question}
    \type{mc.multiple}
    \text{Gegeben sei die Reihe $\sum_{k=1}^{\infty}{a_{k}}$ und es gebe ein $N\in\mathbb{N}$ mit der Eigenschaft $a_{n}\neq 0$ für alle $n\geq N$. 
    Wir  betrachten die Folge $(b_{k})_{k\geq N}$, die gegeben ist durch die Folgeglieder $b_{k}:=\frac{a_{k+1}}{a_{k}}$. 
    Entscheiden Sie, welche der folgenden Aussagen wahr sind.}
    \begin{choice}
        \text{(a) Ist das Folgeglied $b_{N}\in (0,1)$ und $(b_{k})_{k\geq N}$ monoton fallend, so konvergiert die Reihe $\sum_{k=1}^{\infty}{a_{k}}$ absolut.}
		\solution{false}
        \explanation[equalChoice(1???)]{Aussage (a) ist falsch: Hier könnten die $b_k$ negativ werden mit $\lim_{k\to\infty}b_k=-1$. 
        Betrachte zum Beispiel die leicht variierte alternierende harmonische Reihe 
        mit $a_k=\frac{(-1)^k}{k}$ für $k\geq 2$ und $a_1=1$ sowie $N=1$. }
     \end{choice}    
    \begin{choice}
        \text{(b) Falls $b_{k}\geq 1$ für alle bis auf endlich viele $k\in\N$ gilt, 
        so ist die Reihe $\sum_{k=1}^{\infty}{a_{k}}$ divergent.}
		\solution{true}
        \explanation[equalChoice(?0??)]{Aussage (b) ist wahr: Die Bedingung  besagt, dass $(a_{k})_{k\in\N}$ keine Nullfolge ist.}
	\end{choice}    
	\begin{choice}
        		\text{(c) Ist $(b_{k})_{k\geq N}$ eine Nullfolge, so konvergiert die Reihe $\sum_{k=1}^{\infty}{a_{k}}$ absolut.}
		\solution{true}
        \explanation[equalChoice(??0?)]{
        Aussage (c) ist wahr:
        Es gibt insbesondere ein $N_0$ so, dass $\vert b_k\vert<\frac{1}{2}=:\beta<1$ für alle $k>N_0$ gilt.
        Benutze dann das Quotientenkriterium.}
        	\end{choice}     
	\begin{choice}
       	\text{(d) Aus der Eigenschaft $b_{k}\in (-1,1)$ für alle $k\geq N$ 
        ist keine Konvergenzaussage für die Reihe $\sum_{k=1}^{\infty}{a_{k}}$ möglich.}
		\solution{true}
        \explanation[equalChoice(???0)]{Aussage (d) ist wahr: Betrachte zum Beispiel die harmonische Reihe.}
     \end{choice}
  \end{question}
  
\end{problem}
\embedmathlet{mathlet}
\end{content}