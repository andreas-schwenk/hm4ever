\documentclass{mumie.problem.gwtmathlet}
%$Id$
\begin{metainfo}
  \name{
    \lang{de}{A06: Konvergenz}
    \lang{en}{Problem 6}
  }
  \begin{description} 
 This work is licensed under the Creative Commons License Attribution 4.0 International (CC-BY 4.0)   
 https://creativecommons.org/licenses/by/4.0/legalcode 

    \lang{de}{}
    \lang{en}{}
  \end{description}
  \corrector{system/problem/GenericCorrector.meta.xml}
  \begin{components}
    \component{js_lib}{system/problem/GenericMathlet.meta.xml}{mathlet}
  \end{components}
  \begin{links}
  \end{links}
  \creategeneric
\end{metainfo}
\begin{content}
\usepackage{mumie.genericproblem}
\lang{de}{\title{A06: Konvergenz}}
\lang{en}{\title{Problem 6}}
\begin{block}[annotation]
  Im Ticket-System: \href{http://team.mumie.net/issues/9880}{Ticket 9880}
\end{block}

\begin{block}[annotation]
Aufgabenpool mit vier Aufgaben, aus dem zufällig ausgewählt wird.
\end{block}

\begin{problem}
\randomquestionpool{1}{4}
% Aufgabe 1
  \begin{question}
    \type{mc.multiple}
    \text{Es sei $\sum_{k=1}^{\infty} a_{k}$ eine absolut konvergente Reihe. Entscheiden Sie, welche der folgenden Aussagen wahr sind.}
    %\permutechoices{1}{3}
	\begin{choice}
		\text{(a) Für jedes $z\in\mathbb{C}$ ist $\sum_{k=1}^{\infty}{a_{k}z^{k}}$ absolut konvergent.}
		\solution{false}
        \explanation[equalChoice(1??)]{Aussage (a) ist falsch. Gegenbeispiel: Wähle
            $a_{k}=(1/2)^{k}$ und $z=4$.     Das ergibt die divergente Reihe 
            $\sum_{k=1}^{\infty}{a_{k}z^{k}}=\sum_{k=1}^{\infty}{2^{k}}$.}
	\end{choice}
	\begin{choice}
		\text{(b) Falls die Folge $(c_{k})_{k\in\mathbb{N}}$ mit $c_{k}=a_{k}k!$ beschränkt ist, dann konvergiert die Reihe $\sum_{k=1}^{\infty}{a_{k}z^{k}}$ für jedes $z\in\mathbb{C}$ absolut.}
		\solution{true}
        \explanation[equalChoice(?0?)]{Aussage (b) ist wahr. Weil hier ein $C>0$ existiert, so dass 
            $|a_{k}|k!\leq C$ für alle $k\in\mathbb{N}$, erhalten 
            wir die Abschätzung $|a_{k}z^{k}|\leq C \cdot \frac{|z|^{k}}{k!}$ für alle $k\in\mathbb{N}$. 
            Also ist die Exponentialreihe eine konvergente Majorante der Reihe, und somit konvergiert diese  absolut.}
	\end{choice}
	\begin{choice}
		\text{(c) Die Reihe $\sum_{k=1}^{\infty}{a_{k}^{2}}$ ist ebenfalls absolut konvergent.}
		\solution{true}
        \explanation[equalChoice(??0)]{Aussage (c) ist wahr. Die absolute Konvergenz impliziert, dass $(a_{k})_{k\in\mathbb{N}}$ eine Nullfolge ist. 
        Insbesondere ist  $|a_{k}|<1$ für alle $k\geq k_{0}$ für $k_0$ groß genug.
        Dann gilt $|a_{k}|^{2}<|a_{k}|$ für alle $k\geq k_{0}$. 
        Aus dem Majorantenkriterium folgt, dass auch $\sum_{k=1}^{\infty}{a_{k}^{2}}$  absolut konvergiert. }
	\end{choice}
  \end{question}

% Aufgabe 2
  \begin{question}
    \type{mc.multiple}
    \text{Es sei $\sum_{k=1}^{\infty} a_{k}$ eine absolut konvergente Reihe. Entscheiden Sie, welche der folgenden Aussagen wahr sind.}
    %\permutechoices{1}{3}
    \begin{choice}
		\text{(a) Die Reihe $\sum_{k=1}^{\infty}{a_{k}^{2}}$ ist ebenfalls absolut konvergent.}
		\solution{true}
        \explanation[equalChoice(0??)]{Aussage (a) ist wahr. Die absolute Konvergenz impliziert, dass $(a_{k})_{k\in\mathbb{N}}$ eine Nullfolge ist. 
        Insbesondere ist  $|a_{k}|<1$ für alle $k\geq k_{0}$ für $k_0$ groß genug.
        Dann gilt $|a_{k}|^{2}<|a_{k}|$ für alle $k\geq k_{0}$. 
        Aus dem Majorantenkriterium folgt, dass auch $\sum_{k=1}^{\infty}{a_{k}^{2}}$  absolut konvergiert. }
	\end{choice}	
	\begin{choice}
		\text{(b) Falls die Folge $(c_{k})_{k\in\mathbb{N}}$ mit $c_{k}=a_{k}k!$ beschränkt ist, dann konvergiert die Reihe $\sum_{k=1}^{\infty}{a_{k}z^{k}}$ für jedes $z\in\mathbb{C}$ absolut.}
		\solution{true}
        \explanation[equalChoice(?0?)]{Aussage (b) ist wahr. Weil hier ein $C>0$ existiert, so dass 
            $|a_{k}|k!\leq C$ für alle $k\in\mathbb{N}$, erhalten 
            wir die Abschätzung $|a_{k}z^{k}|\leq C \cdot \frac{|z|^{k}}{k!}$ für alle $k\in\mathbb{N}$. 
            Also ist die Exponentialreihe eine konvergente Majorante der Reihe, und somit konvergiert diese  absolut.}
	\end{choice}
    \begin{choice}
		\text{(c) Für jedes $z\in\mathbb{C}$ ist $\sum_{k=1}^{\infty}{a_{k}z^{k}}$ absolut konvergent.}
		\solution{false}
        \explanation[equalChoice(??1)]{Aussage (c) ist falsch. Gegenbeispiel: Wähle
            $a_{k}=(1/2)^{k}$ und $z=4$.     Das ergibt die divergente Reihe 
            $\sum_{k=1}^{\infty}{a_{k}z^{k}}=\sum_{k=1}^{\infty}{2^{k}}$.}
	\end{choice}
	
  \end{question}


% Aufgabe 3
  \begin{question}
    \type{mc.multiple}
    \text{Es sei $\sum_{k=1}^{\infty} a_{k}$ eine absolut konvergente Reihe. Entscheiden Sie, welche der folgenden Aussagen wahr sind.}
    %\permutechoices{1}{3}
	\begin{choice}
		\text{(a) Für jedes $z\in\mathbb{C}$ ist $\sum_{k=1}^{\infty}{a_{k}z^{k}}$ absolut konvergent.}
		\solution{false}
        \explanation[equalChoice(1??)]{Aussage (a) ist falsch. Gegenbeispiel: Wähle
            $a_{k}=(1/2)^{k}$ und $z=4$.     Das ergibt die divergente Reihe 
            $\sum_{k=1}^{\infty}{a_{k}z^{k}}=\sum_{k=1}^{\infty}{2^{k}}$.}
	\end{choice}
    \begin{choice}
		\text{(b) Die Reihe $\sum_{k=1}^{\infty}{a_{k}^{2}}$ ist ebenfalls absolut konvergent.}
		\solution{true}
        \explanation[equalChoice(?0?)]{Aussage (b) ist wahr. Die absolute Konvergenz impliziert, dass $(a_{k})_{k\in\mathbb{N}}$ eine Nullfolge ist. 
        Insbesondere ist  $|a_{k}|<1$ für alle $k\geq k_{0}$ für $k_0$ groß genug.
        Dann gilt $|a_{k}|^{2}<|a_{k}|$ für alle $k\geq k_{0}$. 
        Aus dem Majorantenkriterium folgt, dass auch $\sum_{k=1}^{\infty}{a_{k}^{2}}$  absolut konvergiert. }
	\end{choice}
	\begin{choice}
		\text{(c) Falls die Folge $(c_{k})_{k\in\mathbb{N}}$ mit $c_{k}=a_{k}k!$ beschränkt ist, dann konvergiert die Reihe $\sum_{k=1}^{\infty}{a_{k}z^{k}}$ für jedes $z\in\mathbb{C}$ absolut.}
		\solution{true}
        \explanation[equalChoice(??0)]{Aussage (c) ist wahr. Weil hier ein $C>0$ existiert, so dass 
            $|a_{k}|k!\leq C$ für alle $k\in\mathbb{N}$, erhalten 
            wir die Abschätzung $|a_{k}z^{k}|\leq C \cdot \frac{|z|^{k}}{k!}$ für alle $k\in\mathbb{N}$. 
            Also ist die Exponentialreihe eine konvergente Majorante der Reihe, und somit konvergiert diese  absolut.}
	\end{choice}
	
  \end{question}


% Aufgabe 4
  \begin{question}
    \type{mc.multiple}
    \text{Es sei $\sum_{k=1}^{\infty} a_{k}$ eine absolut konvergente Reihe. Entscheiden Sie, welche der folgenden Aussagen wahr sind.}
    %\permutechoices{1}{3}
    \begin{choice}
		\text{(a) Die Reihe $\sum_{k=1}^{\infty}{a_{k}^{2}}$ ist ebenfalls absolut konvergent.}
		\solution{true}
        \explanation[equalChoice(0??)]{Aussage (a) ist wahr. Die absolute Konvergenz impliziert, dass $(a_{k})_{k\in\mathbb{N}}$ eine Nullfolge ist. 
        Insbesondere ist  $|a_{k}|<1$ für alle $k\geq k_{0}$ für $k_0$ groß genug.
        Dann gilt $|a_{k}|^{2}<|a_{k}|$ für alle $k\geq k_{0}$. 
        Aus dem Majorantenkriterium folgt, dass auch $\sum_{k=1}^{\infty}{a_{k}^{2}}$  absolut konvergiert. }
	\end{choice}
	\begin{choice}
		\text{(b) Für jedes $z\in\mathbb{C}$ ist $\sum_{k=1}^{\infty}{a_{k}z^{k}}$ absolut konvergent.}
		\solution{false}
        \explanation[equalChoice(?1?)]{Aussage (b) ist falsch. Gegenbeispiel: Wähle
            $a_{k}=(1/2)^{k}$ und $z=4$.     Das ergibt die divergente Reihe 
            $\sum_{k=1}^{\infty}{a_{k}z^{k}}=\sum_{k=1}^{\infty}{2^{k}}$.}
	\end{choice}
	\begin{choice}
		\text{(c) Falls die Folge $(c_{k})_{k\in\mathbb{N}}$ mit $c_{k}=a_{k}k!$ beschränkt ist, dann konvergiert die Reihe $\sum_{k=1}^{\infty}{a_{k}z^{k}}$ für jedes $z\in\mathbb{C}$ absolut.}
		\solution{true}
        \explanation[equalChoice(??0)]{Aussage (c) ist wahr. Weil hier ein $C>0$ existiert, so dass 
            $|a_{k}|k!\leq C$ für alle $k\in\mathbb{N}$, erhalten 
            wir die Abschätzung $|a_{k}z^{k}|\leq C \cdot \frac{|z|^{k}}{k!}$ für alle $k\in\mathbb{N}$. 
            Also ist die Exponentialreihe eine konvergente Majorante der Reihe, und somit konvergiert diese  absolut.}
	\end{choice}
	
  \end{question}

\end{problem}

\embedmathlet{mathlet}
\end{content}