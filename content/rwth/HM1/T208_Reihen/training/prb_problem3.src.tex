\documentclass{mumie.problem.gwtmathlet}
%$Id$
\begin{metainfo}
  \name{
    \lang{de}{A04: Konvergenz}
    \lang{en}{Problem 4}
  }
  \begin{description} 
 This work is licensed under the Creative Commons License Attribution 4.0 International (CC-BY 4.0)   
 https://creativecommons.org/licenses/by/4.0/legalcode 

    \lang{de}{}
    \lang{en}{}
  \end{description}
  \corrector{system/problem/GenericCorrector.meta.xml}
  \begin{components}
    \component{js_lib}{system/problem/GenericMathlet.meta.xml}{mathlet}
  \end{components}
  \begin{links}
  \end{links}
  \creategeneric
\end{metainfo}
\begin{content}
\usepackage{mumie.genericproblem}
\lang{de}{\title{A04: Konvergenz}}
\lang{en}{\title{Problem 4}}
\begin{block}[annotation]
  Im Ticket-System: \href{http://team.mumie.net/issues/9878}{Ticket 9878}
\end{block}
\begin{block}[annotation]
Aufgabenpool, aus dem zufällig eine ausgewählt wird.
\end{block}

\begin{problem}
%% Um permutechoices zu ersetzen (durch Anzahl n Permutationen, n<n!: 
%%in Kopien Reihenfolge manuell vertauscht und equalChoice angepasst)
  \randomquestionpool{1}{4}
%% Auswahl 1
  \begin{question}
    \type{mc.multiple}
    \text{Gegeben sei die Reihe $\sum_{k=1}^{\infty}{a_{k}}$ und es   existiere $\beta:=\lim_{k\to\infty}{\sqrt[k]{|a_{k}|}}$. \\
    
    Welche der folgenden Aussagen sind richtig?}
	%\permutechoices{1}{3}
	\begin{choice}
		\text{Aus $\beta=1$ folgt die bedingte Konvergenz der Reihe.}
		\solution{false}
        \explanation[equalChoice(1??)]{Das Wurzelkriterium liefert für $\beta=1$ keine Konvergenzaussage. 
        Ein Gegenbeispiel ist die harmonische Reihe.}
	\end{choice}
	\begin{choice}
		\text{Ist $\beta>1$, so ist die Reihe divergent.}
		\solution{true}
        \explanation[equalChoice(?0?)]{Ist $\beta>1$, so ist $a_k$ keine Nullfolge, die Reihe also divergent.}
	\end{choice}
	\begin{choice}
		\text{Ist $\beta<1$, so konvergiert die Reihe absolut.}
		\solution{true}
        \explanation[equalChoice(??0)]{Ist $\beta<1$, so folgt die absolute Konvergenz mit dem Wurzelkriterium.}
	\end{choice}
   % \explanation{Wir benutzen das Wurzelkriterium. \\
    %Es sei $\varepsilon>0$. Aus der Limesdefinition folgt, dass es ein $k_{0}\in\mathbb{N}$ gibt, so dass $|\sqrt[k]{|a_{k}|}-\beta|<\varepsilon$  für alle $k\geq k_{0}$ gilt. \\
    %Im Fall $\beta<1$ folgt, dass es ein $\beta'<1$ und ein $k_{0}'\in\mathbb{N}$ gibt, so dass $|\sqrt[k]{|a_{k}|}|<\beta'$ für alle $k\geq k_{0}'$ gilt. Also liefert uns das Wurzelkriterium die absolute Konvergenz in diesem Fall. \\
    %Im Fall $\beta>1$ verfährt man entsprechend. \\
    %Die harmonische Reihe ist ein Gegenbeispiel für die Aussage für $\beta=1$.}
    
  \end{question}

%% Auswahl 2
\begin{question}
    \type{mc.multiple}
    \text{Gegeben sei die Reihe $\sum_{k=1}^{\infty}{a_{k}}$ und es   existiere $\beta:=\lim_{k\to\infty}{\sqrt[k]{|a_{k}|}}$. \\
    
    Welche der folgenden Aussagen sind richtig?}
	%\permutechoices{1}{3}
	\begin{choice}
		\text{Ist $\beta<1$, so konvergiert die Reihe absolut.}
		\solution{true}
        \explanation[equalChoice(0??)]{Ist $\beta<1$, so folgt die absolute Konvergenz mit dem Wurzelkriterium.}
	\end{choice}
	\begin{choice}
		\text{Ist $\beta>1$, so ist die Reihe divergent.}
		\solution{true}
        \explanation[equalChoice(?0?)]{Ist $\beta>1$, so ist $a_k$ keine Nullfolge, die Reihe also divergent.}
	\end{choice}
    \begin{choice}
		\text{Aus $\beta=1$ folgt die bedingte Konvergenz der Reihe.}
		\solution{false}
        \explanation[equalChoice(??1)]{Das Wurzelkriterium liefert für $\beta=1$ keine Konvergenzaussage. 
        Ein Gegenbeispiel ist die harmonische Reihe.}
	\end{choice}
   % \explanation{Wir benutzen das Wurzelkriterium. \\
    %Es sei $\varepsilon>0$. Aus der Limesdefinition folgt, dass es ein $k_{0}\in\mathbb{N}$ gibt, so dass $|\sqrt[k]{|a_{k}|}-\beta|<\varepsilon$  für alle $k\geq k_{0}$ gilt. \\
    %Im Fall $\beta<1$ folgt, dass es ein $\beta'<1$ und ein $k_{0}'\in\mathbb{N}$ gibt, so dass $|\sqrt[k]{|a_{k}|}|<\beta'$ für alle $k\geq k_{0}'$ gilt. Also liefert uns das Wurzelkriterium die absolute Konvergenz in diesem Fall. \\
    %Im Fall $\beta>1$ verfährt man entsprechend. \\
    %Die harmonische Reihe ist ein Gegenbeispiel für die Aussage für $\beta=1$.}
    
  \end{question}
  
%% Auswahl 3
  \begin{question}
    \type{mc.multiple}
    \text{Gegeben sei die Reihe $\sum_{k=1}^{\infty}{a_{k}}$ und es   existiere $\beta:=\lim_{k\to\infty}{\sqrt[k]{|a_{k}|}}$. \\
    
    Welche der folgenden Aussagen sind richtig?}
	%\permutechoices{1}{3}
	\begin{choice}
		\text{Aus $\beta=1$ folgt die bedingte Konvergenz der Reihe.}
		\solution{false}
        \explanation[equalChoice(1??)]{Das Wurzelkriterium liefert für $\beta=1$ keine Konvergenzaussage. 
        Ein Gegenbeispiel ist die harmonische Reihe.}
	\end{choice}	
	\begin{choice}
		\text{Ist $\beta<1$, so konvergiert die Reihe absolut.}
		\solution{true}
        \explanation[equalChoice(?0?)]{Ist $\beta<1$, so folgt die absolute Konvergenz mit dem Wurzelkriterium.}
	\end{choice}
    \begin{choice}
		\text{Ist $\beta>1$, so ist die Reihe divergent.}
		\solution{true}
        \explanation[equalChoice(??0)]{Ist $\beta>1$, so ist $a_k$ keine Nullfolge, die Reihe also divergent.}
	\end{choice}
   % \explanation{Wir benutzen das Wurzelkriterium. \\
    %Es sei $\varepsilon>0$. Aus der Limesdefinition folgt, dass es ein $k_{0}\in\mathbb{N}$ gibt, so dass $|\sqrt[k]{|a_{k}|}-\beta|<\varepsilon$  für alle $k\geq k_{0}$ gilt. \\
    %Im Fall $\beta<1$ folgt, dass es ein $\beta'<1$ und ein $k_{0}'\in\mathbb{N}$ gibt, so dass $|\sqrt[k]{|a_{k}|}|<\beta'$ für alle $k\geq k_{0}'$ gilt. Also liefert uns das Wurzelkriterium die absolute Konvergenz in diesem Fall. \\
    %Im Fall $\beta>1$ verfährt man entsprechend. \\
    %Die harmonische Reihe ist ein Gegenbeispiel für die Aussage für $\beta=1$.}
    
  \end{question}
  
%% Auswahl 4
  \begin{question}
    \type{mc.multiple}
    \text{Gegeben sei die Reihe $\sum_{k=1}^{\infty}{a_{k}}$ und es   existiere $\beta:=\lim_{k\to\infty}{\sqrt[k]{|a_{k}|}}$. \\
    
    Welche der folgenden Aussagen sind richtig?}
	%\permutechoices{1}{3}
    \begin{choice}
		\text{Ist $\beta<1$, so konvergiert die Reihe absolut.}
		\solution{true}
        \explanation[equalChoice(0??)]{Ist $\beta<1$, so folgt die absolute Konvergenz mit dem Wurzelkriterium.}
	\end{choice}
	\begin{choice}
		\text{Aus $\beta=1$ folgt die bedingte Konvergenz der Reihe.}
		\solution{false}
        \explanation[equalChoice(?1?)]{Das Wurzelkriterium liefert für $\beta=1$ keine Konvergenzaussage. 
        Ein Gegenbeispiel ist die harmonische Reihe.}
	\end{choice}
	\begin{choice}
		\text{Ist $\beta>1$, so ist die Reihe divergent.}
		\solution{true}
        \explanation[equalChoice(??0)]{Ist $\beta>1$, so ist $a_k$ keine Nullfolge, die Reihe also divergent.}
	\end{choice}	
   % \explanation{Wir benutzen das Wurzelkriterium. \\
    %Es sei $\varepsilon>0$. Aus der Limesdefinition folgt, dass es ein $k_{0}\in\mathbb{N}$ gibt, so dass $|\sqrt[k]{|a_{k}|}-\beta|<\varepsilon$  für alle $k\geq k_{0}$ gilt. \\
    %Im Fall $\beta<1$ folgt, dass es ein $\beta'<1$ und ein $k_{0}'\in\mathbb{N}$ gibt, so dass $|\sqrt[k]{|a_{k}|}|<\beta'$ für alle $k\geq k_{0}'$ gilt. Also liefert uns das Wurzelkriterium die absolute Konvergenz in diesem Fall. \\
    %Im Fall $\beta>1$ verfährt man entsprechend. \\
    %Die harmonische Reihe ist ein Gegenbeispiel für die Aussage für $\beta=1$.}
    
  \end{question}
\end{problem}

\embedmathlet{mathlet}
\end{content}