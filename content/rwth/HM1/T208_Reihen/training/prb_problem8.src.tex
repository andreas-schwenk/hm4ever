\documentclass{mumie.problem.gwtmathlet}
%$Id$
\begin{metainfo}
  \name{
    \lang{de}{A09: spezielle Reihen}
    \lang{en}{Problem 9}
  }
  \begin{description} 
 This work is licensed under the Creative Commons License Attribution 4.0 International (CC-BY 4.0)   
 https://creativecommons.org/licenses/by/4.0/legalcode 

    \lang{de}{}
    \lang{en}{}
  \end{description}
  \corrector{system/problem/GenericCorrector.meta.xml}
  \begin{components}
    \component{js_lib}{system/problem/GenericMathlet.meta.xml}{mathlet}
  \end{components}
  \begin{links}
  \end{links}
  \creategeneric
\end{metainfo}
\begin{content}
\begin{block}[annotation]
	Im Ticket-System: \href{https://team.mumie.net/issues/18133}{Ticket 18133}
\end{block}
\usepackage{mumie.genericproblem}
\lang{de}{\title{A09: spezielle Reihen}}
\lang{en}{\title{Problem 9}}
%\begin{block}[annotation]
%  Im Ticket-System: \href{http://team.mumie.net/issues/}{Ticket }
%\end{block}
\begin{block}[annotation]
Aufgabenpool mit vier Aufgaben, aus dem zufällig ausgewählt wird.
\end{block}

\begin{problem}

\randomquestionpool{1}{4}

\begin{variables}
    \randint{a}{2}{9}
    \randint{k}{2}{9}
    \randint{b}{1}{20}
    \function[calculate]{c}{b*25}
    \function[calculate]{f}{a/(a-1)}
    \randint{g}{0}{2}
\end{variables}

%Aufgabe 1

\begin{question}
    \type{mc.multiple}
    \field{rational}
    \text{Entscheiden Sie, welche der folgenden Aussagen wahr sind.}
    \begin{choice}
       	\text{(a) $\sum_{k=\var{k}}^{\infty}{(-\var{a})^k}$ ist absolut konvergent.}
		\solution{false}
        \explanation[NOT[equalChoice(00000)] AND equalChoice(1???0)]{(a) ist falsch. 
            Die geometrische Reihe $\sum_{k=0}^\infty q^k$ ist divergent für 
            $| q| \geq 1$.}
     \end{choice}	
     %%
	\begin{choice}
        \text{(b) $\sum_{n=0}^{\infty}\frac{(-\var{c})^n}{n!}$ konvergiert absolut.}
		\solution{true}
        \explanation[NOT[equalChoice(00000)] AND equalChoice(?0??0)]{(b) stimmt. Das ist die Reihe $\exp(-\var{c})$. 
            Die  Exponentialreihe konvergiert überall absolut.}
     \end{choice}
     %%
	\begin{choice}
        \text{(c) Es gilt $\sum_{n\in\N_0}\left(\frac{1}{\var{a}}\right)^{n}=\var{f}$. }
		\solution{true}
        \explanation[NOT[equalChoice(00000)] AND equalChoice(??0?0)]{(c) stimmt. Geometrische Reihe! }
     \end{choice}
     %%
	\begin{choice}
        \text{(d) Es gilt $\sum_{k=1}^{\infty}{\left(-\frac{1}{2}\right)^k}=\var{g}$.}
		\solution{false}
        \explanation[NOT[equalChoice(00000)] AND equalChoice(???10)]{(d) ist falsch. 
            Als geometrische Reihe ohne $0$-tes Glied ist der Wert 
            $\frac{1}{1+\frac{1}{2}}-1=-\frac{1}{3}$.}
     \end{choice}
     \begin{choice}
     \text{(e) Keine der Aussagen ist wahr.}
     \solution{false}
     \explanation[equalChoice(????1)]{Nicht alle angegebenen Aussagen sind falsch. Erinnern Sie sich an bekannte Reihen.}
     \end{choice}
\end{question}

%Aufgabe 2\begin{question}
\begin{question}
    \type{mc.multiple}
    \field{rational}
    \text{Entscheiden Sie, welche der folgenden Aussagen wahr sind.}
	\begin{choice}
        \text{(a) Es gilt $\sum_{n\in\N_0}\left(\frac{1}{\var{a}}\right)^{n}=\var{f}$. }
		\solution{true}
        \explanation[NOT[equalChoice(00000)] AND equalChoice(0???0)]{(a) stimmt. Geometrische Reihe! }
     \end{choice}
     %%
	\begin{choice}
        \text{(b) Es gilt $\sum_{k=1}^{\infty}{\left(-\frac{1}{2}\right)^k}=\var{g}$.}
		\solution{false}
        \explanation[NOT[equalChoice(00000)] AND equalChoice(?1??0)]{(b) ist falsch. 
            Als geometrische Reihe ohne $0$-tes Glied ist der Wert 
            $\frac{1}{1+\frac{1}{2}}-1=-\frac{1}{3}$.}
     \end{choice}
     %%
	\begin{choice}
       	\text{(c) $\sum_{k=\var{k}}^{\infty}{(-\var{a})^k}$ ist absolut konvergent.}
		\solution{false}
        \explanation[NOT[equalChoice(00000)] AND equalChoice(??1?0)]{(c) ist falsch. 
            Die geometrische Reihe $\sum_{k=0}^\infty q^k$ ist divergent für 
            $| q| \geq 1$.}
     \end{choice}
     %%
     \begin{choice}
        \text{(d) $\sum_{n=0}^{\infty}\frac{(-\var{c})^n}{n!}$ konvergiert absolut.}
		\solution{true}
        \explanation[NOT[equalChoice(00000)] AND equalChoice(???00)]{(d) stimmt. Das ist die Reihe $\exp(-\var{c})$. 
            Die  Exponentialreihe konvergiert überall absolut.}
     \end{choice}
	\begin{choice}
     \text{(e) Keine der Aussagen ist wahr.}
     \solution{false}
     \explanation[equalChoice(????1)]{Nicht alle angegebenen Aussagen sind falsch. Erinnern Sie sich an bekannte Reihen.}
     \end{choice}
\end{question}

%Aufgabe 3
\begin{question}
    \type{mc.multiple}
    \field{rational}
    \text{Entscheiden Sie, welche der folgenden Aussagen wahr sind.}
    \begin{choice}
        \text{(a) Es gilt $\sum_{k=1}^{\infty}{\left(-\frac{1}{2}\right)^k}=\var{g}$.}
		\solution{false}
        \explanation[NOT[equalChoice(00000)] AND equalChoice(1???0)]{(a) ist falsch. 
            Als geometrische Reihe ohne $0$-tes Glied ist der Wert 
            $\frac{1}{1+\frac{1}{2}}-1=-\frac{1}{3}$.}
     \end{choice}
     %%
	\begin{choice}
        \text{(b) Es gilt $\sum_{n\in\N_0}\left(\frac{1}{\var{a}}\right)^{n}=\var{f}$. }
		\solution{true}
        \explanation[NOT[equalChoice(00000)] AND equalChoice(?0??0)]{(b) stimmt. Geometrische Reihe! }
     \end{choice}
     %%
	\begin{choice}
        \text{(c) $\sum_{n=0}^{\infty}\frac{(-\var{c})^n}{n!}$ konvergiert absolut.}
		\solution{true}
        \explanation[NOT[equalChoice(00000)] AND equalChoice(??0?0)]{(c) stimmt. Das ist die Reihe $\exp(-\var{c})$. 
            Die  Exponentialreihe konvergiert überall absolut.}
     \end{choice}
     %%
	\begin{choice}
       	\text{(d) $\sum_{k=\var{k}}^{\infty}{(-\var{a})^k}$ ist absolut konvergent.}
		\solution{false}
        \explanation[NOT[equalChoice(00000)] AND equalChoice(???10)]{(d) ist falsch. 
            Die geometrische Reihe $\sum_{k=0}^\infty q^k$ ist divergent für 
            $| q| \geq 1$.}
     \end{choice}
     %%
	\begin{choice}
     \text{(e) Keine der Aussagen ist wahr.}
     \solution{false}
     \explanation[equalChoice(????1)]{Nicht alle angegebenen Aussagen sind falsch. Erinnern Sie sich an bekannte Reihen.}
     \end{choice}
\end{question}


%Aufgabe 4

\begin{question}
    \type{mc.multiple}
    \field{rational}
    \text{Entscheiden Sie, welche der folgenden Aussagen wahr sind.}
	\begin{choice}
        \text{(a) Es gilt $\sum_{n\in\N_0}\left(\frac{1}{\var{a}}\right)^{n}=\var{f}$. }
		\solution{true}
        \explanation[NOT[equalChoice(00000)] AND equalChoice(0???0)]{(a) stimmt. Geometrische Reihe! }
     \end{choice}
     %%
	\begin{choice}
        \text{(b) $\sum_{n=0}^{\infty}\frac{(-\var{c})^n}{n!}$ konvergiert absolut.}
		\solution{true}
        \explanation[NOT[equalChoice(00000)] AND equalChoice(?0??0)]{(b) stimmt. Das ist die Reihe $\exp(-\var{c})$. 
            Die  Exponentialreihe konvergiert überall absolut.}
     \end{choice}
     %%
	\begin{choice}
       	\text{(c) $\sum_{k=\var{k}}^{\infty}{(-\var{a})^k}$ ist absolut konvergent.}
		\solution{false}
        \explanation[NOT[equalChoice(00000)] AND equalChoice(??1?0)]{(c) ist falsch. 
            Die geometrische Reihe $\sum_{k=0}^\infty q^k$ ist divergent für 
            $| q| \geq 1$.}
     \end{choice}
     %%
	\begin{choice}
        \text{(d) Es gilt $\sum_{k=1}^{\infty}{\left(-\frac{1}{2}\right)^k}=\var{g}$.}
		\solution{false}
        \explanation[NOT[equalChoice(00000)] AND equalChoice(???10)]{(d) ist falsch. 
            Als geometrische Reihe ohne $0$-tes Glied ist der Wert 
            $\frac{1}{1+\frac{1}{2}}-1=-\frac{1}{3}$.}
     \end{choice}
     \begin{choice}
     \text{(e) Keine der Aussagen ist wahr.}
     \solution{false}
     \explanation[equalChoice(????1)]{Nicht alle angegebenen Aussagen sind falsch. Erinnern Sie sich an bekannte Reihen.}
     \end{choice}
\end{question}

\end{problem}
\embedmathlet{mathlet}
\end{content}