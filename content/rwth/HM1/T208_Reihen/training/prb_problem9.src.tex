\documentclass{mumie.problem.gwtmathlet}
%$Id$
\begin{metainfo}
  \name{
    \lang{de}{Aufgabe 10}
    \lang{en}{Problem 10}
  }
  \begin{description} 
 This work is licensed under the Creative Commons License Attribution 4.0 International (CC-BY 4.0)   
 https://creativecommons.org/licenses/by/4.0/legalcode 

    \lang{de}{...}
    \lang{en}{...}
  \end{description}
  \corrector{system/problem/GenericCorrector.meta.xml}
  \begin{components}
    \component{js_lib}{system/problem/GenericMathlet.meta.xml}{gwtmathlet}
  \end{components}
  \begin{links}
  \end{links}
  \creategeneric
\end{metainfo}
\begin{content}
\begin{block}[annotation]
	Im Ticket-System: \href{https://team.mumie.net/issues/19873}{Ticket 19873}
\end{block}
\begin{block}[annotation]
	Aufgabe für den Aufgaben-Pool. Taucht nicht mehr auf dem Testserver auf und wird nicht in den Kurs eingespielt.
\end{block}
\usepackage{mumie.genericproblem}




\lang{de}{\title{Aufgabe 10}}
\lang{en}{\title{Problem 10}}
     %\lang{de}{Wir betrachten die Reihe $s=\sum_{k=1}^{\infty}\frac{1}{4k^2-1}$.}
     \begin{problem}
               \begin{question}
                \text{Für diese Reihe ist die Konvergenz mit Hilfe einer konvergenten Majorante 
                leicht zu zeigen. Wählen Sie die richtige aus.}
                \type{mc.unique}    
                    \explanation[equalChoice(1)]{$\frac{1}{4k^2+1}<\frac{1}{4k^2-1}$, die Reihe $\sum_{k=1}^{\infty} a_k$ kann also keine Majorante sein.}
                    \explanation[equalChoice(3)]{$\frac{1}{4k^2}<\frac{1}{4k^2-1}$, die Reihe $\sum_{k=1}^{\infty} c_k$ kann also keine Majorante sein.}
                    \explanation[equalChoice(4)]{$\sum_{k=1}^{\infty } \frac{1}{k}$ ist divergent, kann also keine konvergente Majorante sein.}
                    \begin{choice}
                        \text{$a_k=\frac{1}{4k^2+1}$}
                        \solution{false}
                    \end{choice}
                    \begin{choice}
                        \text{$b_k=\frac{1}{k^2}$}
                        \solution{true}
                    \end{choice}
                    \begin{choice}
                        \text{$c_k=\frac{1}{4k^2}$}
                        \solution{false}
                    \end{choice} 
                    \begin{choice}
                        \text{$d_k=\frac{1}{k}$}
                        \solution{false}
                    \end{choice}
          \end{question} 
          
          \begin{question}
            \begin{variables}
                \function{pbz}{1/2 (1/(2x-1)-1/(2x+1))}
            \end{variables}
            \type{input.function}
            \field{rational}
            \text{Führen Sie eine Partialbruchzerlegung des Terms $\frac{1}{4x^2-1}$ mit Hilfe 
            der dritten binomischen Formel durch.}
            \begin{answer}
                \text{$\frac{1}{4x^2-1}=$}
                \solution{pbz}
                \inputAsFunction{x}{k}
                \checkAsFunction{x}{-10}{10}{100}
                \checkStringsForRelation{count(/,k)>1}
                \explanation[edited]{$4x^2-1=(2x-1)(2x+1)$, \\ also:
                $\frac{1}{4x^2-1}=\frac{A}{2x-1}+\frac{B}{2x+1}=
                \frac{A(2x+1)}{(2x-1)(2x+1)}+\frac{B(2x-1)}{(2x+1)(2x-1)},$ \\ folglich
                $1=(A+B)2x+A-B$, also  $A=\frac{1}{2}$ und $B=-\frac{1}{2}$.
                }
          \end{answer} 
          \end{question} 
            
            
          \begin{question}
            \type{input.number}
            \begin{variables}
                \number{s}{1/2}
            \end{variables}
            \text{Berechnen Sie den Wert s der Reihe. Benutzen Sie hierbei die Partialbruchzerlegung,
            um günstig Terme zu addieren, und verschieben Sie den Summationsindex.}
            \begin{answer}
               \text{$s=$}
               \solution{s}
               \explanation[edited]{Es ist 
               \begin{eqnarray*}s_n&=&\sum_{k=1}^{n}\frac{1}{4k^2-1}=\sum_{k=1}^{n}\frac{1}{(2k-1)(2k+1)}=
               \sum_{k=1}^{n}\frac{1}{2}\left[\frac{1}{(2k-1)}-\frac{1}{(2k+1)}\right]\\
               &=&
               \frac{1}{2}\left[\sum_{k=1}^{n}\frac{1}{2k-1}-\sum_{k=1}^{n}\frac{1}{2k+1}\right]=
               \frac{1}{2}\left[\sum_{k=0}^{n-1}\frac{1}{2k+1}-\sum_{k=1}^{n}\frac{1}{2k+1}\right]\\
               &=&
               \frac{1}{2}-\frac{1}{2}\frac{1}{2n+1},\end{eqnarray*} und somit ist $\lim_{n\to\infty}s_n=\frac{1}{2}$.
               }
            \end{answer}
            
          \end{question}
          
         
     
          
     \end{problem}


     

\embedmathlet{gwtmathlet}

\end{content}