\documentclass{mumie.problem.gwtmathlet}
%$Id$
\begin{metainfo}
  \name{
    \lang{de}{A03: Textaufgabe}
    \lang{en}{input numbers}
  }
  \begin{description} 
 This work is licensed under the Creative Commons License Attribution 4.0 International (CC-BY 4.0)   
 https://creativecommons.org/licenses/by/4.0/legalcode 

    \lang{de}{Die Beschreibung}
    \lang{en}{}
  \end{description}
  \corrector{system/problem/GenericCorrector.meta.xml}
  \begin{components}
    \component{js_lib}{system/problem/GenericMathlet.meta.xml}{gwtmathlet}
  \end{components}
  \begin{links}
  \end{links}
  \creategeneric
\end{metainfo}
\begin{content}
\begin{block}[annotation]
	Im Ticket-System: \href{https://team.mumie.net/issues/20150}{Ticket 20150}
\end{block}
\begin{block}[annotation]
	Im Ticket-System: \href{https://team.mumie.net/issues/18866}{Ticket 18866}
\end{block}
\usepackage{mumie.ombplus}
\usepackage{mumie.genericproblem}

\lang{de}{\title{A03: Textaufgabe}}
\lang{en}{\title{Problem 3}}

\begin{problem}

\randomquestionpool{1}{2}

  \begin{question}
		\lang{de}{
			\text{In einem Hörsaal befinden sich $\var{n}$ Stühle in der ersten Reihe. In jeder weiteren Reihe verringert sich die Anzahl
            um $\var{b}$ Stühle.\\\\
            a) Wie viele Stühle befinden sich in der $9.$ Reihe?\\\\
            b) Wie viele Stühle befinden sich in den ersten $9$ Reihen?\\\\
            Es befinden sich \ansref Stühle in der $9.$ Reihe. In den ersten $9$ Reihen stehen insgesamt \ansref Stühle.}}
		
		\explanation{Beschreiben Sie das Problem in a) mit Hilfe einer Folge und das Problem von b) mit Hilfe einer Reihe.}
		\type{input.number}
		\field{rational}
		

    \precision[false]{3}
    \displayprecision{3}
    \correctorprecision{3}
		
		\begin{variables}
			\randint{n}{78}{98}
            \randint{b}{3}{5}
			\function[calculate]{loes}{n-8*b}
            \function[calculate]{loes1}{(2*n-8*b)*4.5}
		\end{variables}
		
		\begin{answer}
			\solution{loes}
		\end{answer}
		
        \begin{answer}
			\solution{loes1}
		\end{answer}
		
        
	\end{question}

	\begin{question}
		\lang{de}{
			\text{Ein Turm wird gebaut, indem Würfel übereinander gestapelt werden. Dabei hat jeder weitere Würfel die halbe 
            Kantenlänge des darunter liegenden Würfels. Der erste Würfel hat eine Kantenlänge von $\var{a}$ Metern, der zweite Würfel von
            $\var{b}$ Metern und so weiter.\\\\
            a) Welche Höhe hat der Turm, der aus $\var{c}$ Würfeln besteht?\\\\
            b) Welche Höhe würde der Turm haben, wenn unendlich viele Würfel übereinandergesetzt würden?\\\\
            
            Der Turm aus $\var{c}$ Würfeln hat eine Höhe von \ansref Metern. Die Höhe des Turms, der aus unendlich vielen Würfeln 
            zusammengebaut wird, hätte eine Höhe von \ansref Metern.}}
		
		\explanation{Überlegen Sie sich den Zusammenhang mit der geometrischen Reihe. Verwenden Sie insbesondere auch die geometrische Summenformel.}
		\type{input.number}
		\field{rational}
		

    \precision[false]{3}
    \displayprecision{3}
    \correctorprecision{3}
		
		\begin{variables}
			\drawFromSet{a}{2,3,4,5,6,7,8}
            \drawFromSet{c}{8,9,10,11,12,13,14}
			\function[calculate]{b}{0.5*a}
			\function[calculate]{loes1}{2*a*(1-(1/2)^c)}
            \function[calculate]{loes2}{2*a}
		\end{variables}
		
		\begin{answer}
			\solution{loes1}
		\end{answer}
        
        	
		\begin{answer}
			\solution{loes2}
		\end{answer}
		
	\end{question}
	
\end{problem}

\embedmathlet{gwtmathlet}


\end{content}