
%$Id:  $
\documentclass{mumie.article}
%$Id$
\begin{metainfo}
  \name{
    \lang{de}{Überblick: Vektorrechnung}
    \lang{en}{Overview: Vectors}
  }
  \begin{description} 
 This work is licensed under the Creative Commons License Attribution 4.0 International (CC-BY 4.0)   
 https://creativecommons.org/licenses/by/4.0/legalcode 

    \lang{de}{Beschreibung}
    \lang{en}{Description}
  \end{description}
  \begin{components}
  \end{components}
  \begin{links}
\link{generic_article}{content/rwth/HM1/T108_Vektorrechnung/g_art_content_30_basen_eigenschaften.meta.xml}{content_30_basen_eigenschaften}
\link{generic_article}{content/rwth/HM1/T108_Vektorrechnung/g_art_content_29_linearkombination.meta.xml}{content_29_linearkombination}
\link{generic_article}{content/rwth/HM1/T108_Vektorrechnung/g_art_content_27_vektoren.meta.xml}{content_27_vektoren}
\end{links}
  \creategeneric
\end{metainfo}
\begin{content}
\begin{block}[annotation]
	Im Ticket-System: \href{https://team.mumie.net/issues/30141}{Ticket 30141}
\end{block}
\begin{block}[annotation]
Copy of : /home/mumie/checkin/content/rwth/HM1/T109_Skalar-_und_Vektorprodukt/art_T109_overview.src.tex
\end{block}




\begin{block}[annotation]
Im Entstehen: Überblicksseite für Kapitel Vektorrechnung
\end{block}

\usepackage{mumie.ombplus}
\ombchapter{1}
\title{\lang{de}{Überblick: Vektorrechnung}\lang{en}{Overview: Vectors}}



\begin{block}[info-box]
\lang{de}{\strong{Inhalt}}
\lang{en}{\strong{Contents}}


\lang{de}{
    \begin{enumerate}%[arabic chapter-overview]
   \item[8.1] \link{content_27_vektoren}{Vektoren im Anschauungsraum}
   \item[8.2] \link{content_29_linearkombination}{Rechenregeln für Vektoren und Linearkombinationen}
   \item[8.3] \link{content_30_basen_eigenschaften}{Lineare Unabhängigkeit von Vektoren, 
                                                    Erzeugendensysteme und Basen}
     \end{enumerate}
}
\lang{en}{
    \begin{enumerate}%[arabic chapter-overview]
   \item[8.1] \link{content_27_vektoren}{Vectors in space}
   \item[8.2] \link{content_29_linearkombination}{Rules for vectors and linear combinations}
   \item[8.3] \link{content_30_basen_eigenschaften}{Generating sets, linear independence and bases}
     \end{enumerate}
} %lang

\end{block}

\begin{zusammenfassung}
\lang{de}{
In diesem Kapitel führen wir Größen ein, die durch eine Länge und eine Richtung gekennzeichnet sind. Wir bezeichnen sie als Vektoren.
\\\\
Als Basisoperationen lernen wir die Vektoraddition und die Skalarmultiplikation kennen.
\\\\
Über die Linearkombination von Vektoren stellen wir fest, dass sich aus geeigneten vorgegebenen 
Vektoren alle weiteren Vektoren darstellen lassen. Das führt uns zu Erzeugendensystemen. Welche 
minimale Anzahl von Vektoren ist dafür nötig? Wir befassen uns mit dem Begriff der linearen 
(Un-)Abhängigkeit.
\\\\
Am Ende des Kapitels stoßen wir auf eine spezielle Klasse von Erzeugendensystemen: die Basen.
}
\lang{en}{
In this chapter we introduce objects that are charactarised by their length (magnitude) and their 
direction. These are called vectors.
\\\\
We define two operations for vectors, vector addition and scalar multiplication, and learn some of 
the associated rules.
\\\\
We define linear combinations of vectors, and determine that certain sets of vectors have the 
property that every vector can be written as a linear combination of these. These are called 
generating sets, and we investigate the minimal number of elements in such a set, leading us to the 
definition of linear (in-)dependence.
\\\\
At the end of the chapter we use linear independence to define a special type of generating set, 
a basis.
}

\end{zusammenfassung}
\begin{block}[info]\lang{de}{\strong{Lernziele}}
                   \lang{en}{\strong{Learning Goals}} 
\begin{itemize}[square]
\item \lang{de}{Sie betrachten Vektoren und Operationen auf Vektoren im euklidischen Anschauungsraum.}
      \lang{en}{Understanding vectors and vector operations in Euclidean space.}
\item \lang{de}{Sie berechnen Vektorterme, die Summen und Vielfache von Vektoren enthalten.}
      \lang{en}{Being able to evaluate sums and multiples of vectors.}
%\item \lang{de}{Sie erkennen welche Vektoren in Operationen kompatibel zueinander sind.}
\item \lang{de}{Sie untersuchen gegebene Vektoren auf lineare Abhängigkeit.}
      \lang{en}{Being able to determine whether a given set of vectors is linearly dependent.}
\item \lang{de}{
      Sie entscheiden, ob eine gegebene Menge von Vektoren ein Erzeugendensystem oder gar eine Basis 
      darstellt.
      }
      \lang{en}{
      Being able to determine whether a given set of vectors is a generating set or even a basis.
      }
\end{itemize}
\end{block}


\end{content}
