%$Id:  $
\documentclass{mumie.article}
%$Id$
\begin{metainfo}
  \name{
    \lang{de}{Rechenregeln und Linearkombination}
    \lang{en}{Rules for vectors and linear combinations}
  }
  \begin{description} 
 This work is licensed under the Creative Commons License Attribution 4.0 International (CC-BY 4.0)   
 https://creativecommons.org/licenses/by/4.0/legalcode 

    \lang{de}{Beschreibung}
    \lang{en}{Description}
  \end{description}
  \begin{components}
    \component{generic_image}{content/rwth/HM1/images/g_img_00_Videobutton_schwarz.meta.xml}{00_Videobutton_schwarz}
    \component{js_lib}{system/media/mathlets/GWTGenericVisualization.meta.xml}{mathlet1} 
  \end{components}
  \begin{links}
    %\link{generic_article}{content/rwth/HM1/T101neu_Elementare_Rechengrundlagen/g_art_content_02_rechengrundlagen_terme.meta.xml}{content_02_rechengrundlagen_terme}
    \link{generic_article}{content/rwth/HM1/T403a_Vektorraum/g_art_content_10a_vektorraum.meta.xml}{Allg_Vektorraum}  % Link zur Zielstelle
    \link{generic_article}{content/rwth/HM1/T108_Vektorrechnung/g_art_content_27_vektoren.meta.xml}{Vektor_Addition}  % Link zur Zielstelle
  \end{links}
  \creategeneric
\end{metainfo}
\begin{content}
\usepackage{mumie.ombplus}
\ombchapter{8}
\ombarticle{2}
\usepackage{mumie.genericvisualization}

\begin{visualizationwrapper}

\title{\lang{de}{Rechenregeln für Vektoren und Linearkombinationen}
       \lang{en}{Rules for vectors and linear combinations}}

\begin{block}[annotation]
  Inhalt: Rechenregeln für Vektoren (Ergänzung: Vektorraumdefinition), Def. von Linearkombinationen

\end{block}
\begin{block}[annotation]
  Im Ticket-System: \href{http://team.mumie.net/issues/9046}{Ticket 9046}\\
\end{block}

\begin{block}[info-box]
\tableofcontents
\end{block}

\section{\lang{de}{Rechenregeln für Vektoren}
         \lang{en}{Rules for vectors}} \label{Vektorraum_R_n}

\lang{de}{
Die Addition und die skalare Multiplikation von Vektoren im $\R^n$ lassen sich auch kombinieren.
\\\\
Für Umformungen gibt es mehrere Regeln, die sich aus den Rechenregeln für reelle Zahlen herleiten lassen.
}
\lang{en}{
Addition and scalar multiplication of vectors in $\R^n$ can also be combined together.
\\\\
The following rules apply to vectors in any $\R^n$, and are some of the so-called vector space axioms.
}



\begin{rule}[\lang{de}{Rechnenregeln für Vektoren}\lang{en}{Rules for calculating with vectors}]
\lang{de}{
Wir bezeichnen mit $\vectorspace{V}$ die Menge aller Vektoren im $\R^n$. Dann gilt f"ur die Addition 
dieser Vektoren:
}
\lang{en}{
For any natural $n$ we call the set of all vectors in $\R^n$ a vector space . Then for addition of 
vectors, we have
}
\begin{enumerate}
    \item \lang{de}{
          Das Kommutativgesetz: $ \vec{u} + \vec{v} = \vec{v} + \vec{u}$ f\"ur alle 
          $\vec{u},\vec{v} \in \vectorspace{V}$
          }
          \lang{en}{
          Commutativity: $ \vec{a} + \vec{b} = \vec{b} + \vec{a}$ for all 
          $\vec{a},\vec{b} \in \vectorspace{V}$,
          }
    \item \lang{de}{
          Das Assoziativgesetz: 
          \nowrap{$ \vec{u} + (\vec{v} + \vec{w}) = (\vec{u} + \vec{v}) + \vec{w}$ f\"ur alle 
          $\vec{u},\vec{v},\vec{w} \in \vectorspace{V}$}
          }
          \lang{en}{
          Associativity: 
          \nowrap{$ \vec{u} + (\vec{v} + \vec{w}) = (\vec{u} + \vec{v}) + \vec{w}$ for all
          $\vec{u},\vec{v},\vec{w} \in \vectorspace{V}$,}
          }
    \item \lang{de}{
          Es gibt einen Vektor, den man zu jedem Vektor $\vec{v}$ addieren kann, ohne dass sich 
          $\vec{v}$ ver\"andert.\\
          Dies ist der Nullvektor $\vec{0} \in \vectorspace{V}$, für den gilt 
          $ \vec{v} + \vec{0} = \vec{0} + \vec{v} = \vec{v}$ f\"ur alle 
          $\vec{v} \in \vectorspace{V} $.
          }
          \lang{en}{
          There exists a zero vector $\vec{0} \in \vectorspace{V}$ satisfying 
          $ \vec{v} + \vec{0} = \vec{0} + \vec{v} = \vec{v}$ for all 
          $\vec{v} \in \vectorspace{V} $.
          }
\end{enumerate}

\lang{de}{F"ur die Multiplikation mit Skalaren gilt:}
\lang{en}{For multiplication by a scalar we have:}
\begin{enumerate}
    \item \lang{de}{
          \nowrap{$ \alpha (\beta \vec{v}) = (\alpha \beta) \vec{v}$ f\"ur alle 
          $\vec{v} \in \vectorspace{V} $ und $ \alpha, \beta \in \R $}
          }
          \lang{en}{
          \nowrap{$ \alpha (\beta \vec{v}) = (\alpha \beta) \vec{v}$ for all 
          $\vec{v} \in \vectorspace{V} $ und $ \alpha, \beta \in \R $,}
          }
    \item \lang{de}{
          Es gibt einen Skalar $ \alpha \in \R $, den man mit jedem Vektor $\vec{v}$ multiplizieren 
          kann, so dass dieser unver\"andert bleibt. Dies ist $\alpha =1 $ und es gilt 
          \nowrap{$ 1 \cdot \vec{v} = \vec{v}$ f\"ur alle $\vec{v} \in \vectorspace{V} $}
          }
          \lang{en}{
          There exists a scalar $ 1 \in \R $ satisfying 
          \nowrap{$ 1 \cdot \vec{v} = \vec{v}$ for all $\vec{v} \in \vectorspace{V} $.}
          }
\end{enumerate}
\lang{de}{Zwischen Addition und Multiplikation mit Skalaren gelten die Distributivgesetze:}
\lang{en}{Between addition and multiplication by a scalar we have distributivity laws:}
\begin{enumerate}
    \item \lang{de}{
          \nowrap{$ \alpha (\vec{u} + \vec{v}) = \alpha \vec{u} + \alpha \vec{v}$ f\"ur alle 
          $\vec{u}, \vec{v} \in \vectorspace{V}$ und $\alpha \in \R$}
          }
          \lang{en}{
          \nowrap{$ \alpha (\vec{u} + \vec{v}) = \alpha \vec{u} + \alpha \vec{v}$ for all 
          $\vec{u}, \vec{v} \in \vectorspace{V}$ and $\alpha \in \R$,}
          }
    \item \lang{de}{
          \nowrap{$ (\alpha + \beta) \vec{v} = \alpha \vec{v} + \beta \vec{v}$ f\"ur alle 
          $\vec{v} \in \vectorspace{V}$ und $\alpha , \beta \in \R $}
          }
          \lang{en}{
          \nowrap{$ (\alpha + \beta) \vec{v} = \alpha \vec{v} + \beta \vec{v}$ for all 
          $\vec{v} \in \vectorspace{V}$ and $\alpha , \beta \in \R $.}
          }
\end{enumerate}
\end{rule}

\begin{remark} \label{supp:allg-vektorraum}
  \lang{de}{
  In der Mathematik bezeichnet man eine Menge $\vectorspace{V}$, auf der eine Addition $+$ und eine 
  Skalarmultiplikation $\cdot$ mit reellen Zahlen definiert ist, sodass die obigen Regeln erfüllt sind, 
  als einen \emph{reellen Vektorraum}. Die Elemente aus $\vectorspace{V}$ nennt man dann \emph{Vektoren}.
  \\    
%
%%% Video Hoever
%       
    \floatright{\href{https://www.hm-kompakt.de/video?watch=705}{\image[75]{00_Videobutton_schwarz}}}\\\\
  }
  \lang{en}{
  A set $\vectorspace{V}$ which has both an addition $+$ and a scalar multiplication $\cdot$ by real 
  numbers defined on it, satisfying the above rules, is called a \emph{real vector space}. A 
  \emph{vector} is then defined as an element of a real vector space $\vectorspace{V}$.
  }
\end{remark} 

\lang{de}{
In diesem Kapitel betrachten wir ausschließlich den Vektorraum $\R^n$. Dieser ist ein Spezialfall eines 
allgemeineren Vektorraum-Begriffs, der später in \ref[Allg_Vektorraum][Teil 3b, Abschnitt 4.1]{def:Allg_VR}
behandelt wird.
}
\lang{en}{
In this chapter we exclusively deal with the vector space $\R^n$. This is actually only a special 
case of a vector space, which is defined more generally 
\ref[Allg_Vektorraum][in a later chapter]{def:Allg_VR}.
}

\begin{example}    
    \begin{alignat}{3}
     & &3\cdot \left( \begin{pmatrix}2\\ 3\end{pmatrix}+ \begin{pmatrix}-1\\ 4\end{pmatrix}\right)
      - 3\cdot \begin{pmatrix}2\\ 3\end{pmatrix}    
      &&& \vert \, \text{\lang{de}{1. Distributivgesetz (Skalar $3$ ausklammern)}
                         \lang{en}{1st distributivity law (factoring the scalar $3$)}} \\
      &&&&&\\
     &=&3\cdot \left(\begin{pmatrix}2\\ 3\end{pmatrix}+ \begin{pmatrix}-1\\ 4\end{pmatrix}
      - \begin{pmatrix}2\\ 3\end{pmatrix} \right)   
      &&& \vert \, \text{\lang{de}{Kommutativgesetz (innerhalb der Klammer)}
                         \lang{en}{Commutativity (within the parentheses)}} \\
      &&&&&\\
     &=&3\cdot \left(\begin{pmatrix}-1\\ 4\end{pmatrix} + \;
        \begin{pmatrix}2\\ 3\end{pmatrix} - \begin{pmatrix}2\\ 3\end{pmatrix} \right) 
      &&& \vert \, \text{\lang{de}{1. Distributivgesetz}
                         \lang{en}{1st distributivity law}} \\
      &&&&&\\
     &=&3\cdot \begin{pmatrix}-1\\ 4\end{pmatrix} + 
        3\cdot \underbrace{\left(\begin{pmatrix}2\\ 3\end{pmatrix} - \begin{pmatrix}2\\ 3\end{pmatrix}\right)}_{=\begin{pmatrix}0\\ 0\end{pmatrix}} 
       &&&\\
       &&&&&\\
     &=& 3\cdot \begin{pmatrix}-1\\ 4\end{pmatrix} + 0 =\begin{pmatrix}-3\\ 12\end{pmatrix} &&&
    \end{alignat}
\end{example}

\begin{remark}                                       
    \begin{enumerate}
        \item \lang{de}{
            Wie in \ref[Vektor_Addition][Abschnitt 8.1.1 zu Vektoren]{Gegenvektor}  %  (Link zur Bem nach Def 1.2 in 8.1 Vektoren im Anschauungsraum
            einleitend bemerkt, gibt es zu jedem Vektor $\vec{v}$ einen sogenannten \emph{Gegenvektor}
            \ $-\vec{v}$. Dieser ist definiert durch \ $-\vec{v}=(-1)\cdot \vec{v}$.
            }
            \lang{en}{
            As was noted in the ref[Vektor_Addition][previous section]{Gegenvektor}, for each vector 
            $\vec{v}$ there is a negative vector $-\vec{v}$. This is defined by 
            $-\vec{v}=(-1)\cdot \vec{v}$.
            }

        \item \lang{de}{
            Damit definiert man nun, wie schon in 
            \ref[Vektor_Addition][Abschnitt 8.1.3, Rechnen mit Vektoren]{Bem_Vektor_Addition} erwähnt, 
            %  (Link zu Bemerkung 1.11 in 8.1 Vektoren im Anschauungsraum) 
            eine Subtraktion von Vektoren durch
            }
            \lang{en}{
            As in the \ref[Vektor_Addition][previous section]{Bem_Vektor_Addition}, we define 
            subtraction of vectors as 
            }
            \[ \vec{v}-\vec{w} = \vec{v}+(-\vec{w})= \vec{v}+(-1)\cdot \vec{w}. \]

        \item \lang{de}{
            Insbesondere gilt offenbar $\vec{v}-\vec{v}=\vec{v}+(-\vec{v})=\vec{0}$.\\ 
            Das bedeutet, zu jedem Vektor $ \vec{v} \in \vectorspace{V}$ gibt es einen sogenannten 
            \emph{inversen Vektor}, nämlich den Gegenvektor $(-\vec{v}) \in \vectorspace{V}$.
            }
            \lang{en}{
            In particular we have $\vec{v}-\vec{v}=\vec{v}+(-\vec{v})=\vec{0}$.\\
            This means that for every vector $ \vec{v} \in \vectorspace{V}$, the vector 
            $ -\vec{v} \in \vectorspace{V}$ is its so-called \emph{additive inverse}.
            }

        \item \lang{de}{
            Die Regeln zur skalaren Multiplikation und das zweite Distributivgesetz sorgen dafür, dass man mit 
            Vielfachen von gleichen Vektoren die Skalare wie gewohnt zusammenfassen kann. Zum Beispiel gilt für jeden Vektor
            $\vec{v}$ im $\R^n$, die Gleichheit $3\cdot \vec{v}- 5\cdot \vec{v}=-2\cdot \vec{v}$ 
            }
            \lang{en}{
            The rules for scalar multiplication and the second distributivity law ensure that 
            multiples of a vector can be expressed as expected using scalars. For example, for every 
            vector $\vec{v}$ in $\R^n$ we have $3\cdot \vec{v}- 5\cdot \vec{v}=-2\cdot \vec{v}$.
            }
            
        \item \lang{de}{
              Für die Multiplikation mit Skalaren folgt durch Anwendung des 
              Kommutativgesetzes für die Multiplikation $\R$
            %\ref[content_02_rechengrundlagen_terme][Kommutativgesetzes für die Multiplikation $\R$]{rule:rechengesetze}
              }
              \lang{en}{
              By commutativity of the multiplication of real numbers, 
              }
             \[ \alpha (\beta \vec{v}) = (\alpha \beta) \vec{v}=(\beta \alpha) \vec{v} =\beta (\alpha \vec{v})\]
    \end{enumerate}
\end{remark}

\begin{example}

        \[ 3\cdot \begin{pmatrix}2\\ -1\\ 3\end{pmatrix} - 5\cdot \begin{pmatrix}2\\ -1\\ 3\end{pmatrix}
        =  3\cdot \begin{pmatrix}2\\ -1\\ 3\end{pmatrix} + (-5)\cdot \begin{pmatrix}2\\ -1\\ 3\end{pmatrix}
        =         \begin{pmatrix}6\\ -3\\ 9\end{pmatrix}+         \begin{pmatrix}-10\\ 5\\ -15\end{pmatrix}
        =\begin{pmatrix}-4\\ 2\\ -6\end{pmatrix}\]
\lang{de}{und}\lang{en}{and}        
        \[ 3\cdot \begin{pmatrix}2\\ -1\\ 3\end{pmatrix} -5\cdot \begin{pmatrix}2\\ -1\\ 3\end{pmatrix}
        =(3-5)\cdot \begin{pmatrix}2\\ -1\\ 3\end{pmatrix}=-2\cdot \begin{pmatrix}2\\ -1\\ 3\end{pmatrix}
        =\begin{pmatrix}-4\\ 2\\ -6\end{pmatrix}\]

\end{example}
\begin{quickcheckcontainer}
\randomquickcheckpool{1}{1}
\begin{quickcheck}
        \field{rational}
        \type{input.number}
        \begin{variables}
%			\randint{a}{1}{3}
            \randint{a}{2}{4}
            \function[calculate]{a1}{a+1}
            \randint{q1}{-5}{5}
            \randint{q2}{-5}{5}
            \function[calculate]{s1}{a*q1}
            \function[calculate]{s2}{a*q2}

        \end{variables}

            \text{\lang{de}{
            Bestimmen Sie den Vektor $\vec{v}$ im $\R^2$, für den
            }
            \lang{en}{
            Determine the vector $\vec{v}$ in $\R^2$ such that
            }
            \[  \var{a}\cdot \left( \vec{v}+ \begin{pmatrix} \var{q1}\\ \var{q2}\end{pmatrix}\right) 
            = \var{a1}\cdot \vec{v} \]
            \lang{de}{gilt.}
            \lang{en}{holds.}
            \begin{table}[\class{no-padding}]
            \rowspan[l][m]{2} \lang{de}{Der Vektor ist}\lang{en}{The vector is} 
            $\vec{v}=\left(\begin{matrix} \\ \\ \end{matrix}\right.$ &  
            \ansref & \rowspan[l][m]{2} $\left.\begin{matrix} \\ \\ \end{matrix}\right)$. & \\ 
            \ansref & 
            \end{table}		
            }
        \begin{answer}
            \solution{s1}
        \end{answer}
        \begin{answer}
            \solution{s2}
        \end{answer}
        \explanation{\lang{de}{
        Mit den obigen Rechenregeln ist die linke Seite gleich\\
        $\var{a}\cdot \vec{v}+ \var{a}\cdot \begin{pmatrix} \var{q1}\\ \var{q2}\end{pmatrix}
        = \var{a}\cdot \vec{v} + \begin{pmatrix} \var{s1}\\ \var{s2}\end{pmatrix}. $\\
        Zieht man dann auf beiden Seiten der Gleichung $\var{a}\cdot \vec{v}$ ab, erhält man\\
        $\begin{pmatrix} \var{s1}\\ \var{s2}\end{pmatrix}
        = \var{a1}\cdot \vec{v}-\var{a}\cdot \vec{v}
        =(\var{a1} - \var{a})\cdot \vec{v}=\vec{v}. $
        }
        \lang{en}{
        Using the above rules for vectors, the left-hand side is immediately\\
        $\var{a}\cdot \vec{v}+ \var{a}\cdot \begin{pmatrix} \var{q1}\\ \var{q2}\end{pmatrix}
        = \var{a}\cdot \vec{v} + \begin{pmatrix} \var{s1}\\ \var{s2}\end{pmatrix}. $\\
        Subtracting $\var{a}\cdot \vec{v}$ from both sides of the equation yields\\
        $\begin{pmatrix} \var{s1}\\ \var{s2}\end{pmatrix}
        = \var{a1}\cdot \vec{v}-\var{a}\cdot \vec{v}
        =(\var{a1} - \var{a})\cdot \vec{v}=\vec{v}. $
        }}

    \end{quickcheck}
\end{quickcheckcontainer}
%
%%%%%%%%%%%%%%%%%%%%%%%%%%%%%%%%%%%%%%%%%%%%%%%%%%%%%%%%%%%%%%%%%%%%%%%%%%%%%%%%%
%   In Zusammenhang mit dem Video-Einbaus wurde diese Bemerkung vorverlegt !
%%%%%%%%%%%%%%%%%%%%%%%%%%%%%%%%%%%%%%%%%%%%%%%%%%%%%%%%%%%%%%%%%%%%%%%%%%%%%%%%%
%
%  \begin{supplement}[Allgemeiner Vektorraum]\label{supp:allg-vektorraum}
%  \begin{remark}
%    In der Mathematik wird ein \emph{reeller Vektorraum} definiert als eine Menge $V$, auf der eine Addition $+$
%    und eine Skalarmultiplikation $\cdot$ mit reellen Zahlen definiert ist, sodass die obigen Regeln erfüllt sind.
%    Die Elemente aus ${V}$ nennt man dann \emph{Vektoren}.
%    \\
%    \\
%    Die in diesem Kapitel betrachteten Vektorräume der Vektoren im $\R^n$ sind Spezialfälle eines allgemeineren Begriffs
%    des Vektorraums. Dieser wird in \ref[Allg_Vektorraum][Teil 3b, Abschnitt 4.1]{def:Allg_VR}
%    behandelt.
%  \end{remark}
%  \end{supplement}

\section{\lang{de}{Linearkombinationen}\lang{en}{Linear combinations}}\label{sec:lin-comb}

  \begin{definition}[\lang{de}{Linearkombination}\lang{en}{Linear combination}]\label{def:linearcomb}
    \lang{de}{Seien $\vec{v}_1, \ldots , \vec{v}_k$ Vektoren im $\R^n$.\\\\ Ein Vektor der Form}
    \lang{en}{Let  $\vec{v}_1, \ldots , \vec{v}_k$ be vectors in $\R^n$.\\\\ A vector of the form}
    \begin{displaymath}
      \alpha_1\, \vec{v}_1 + \ldots +\alpha_k\, \vec{v}_k = \sum_{i=1}^k \alpha_i\, \vec{v}_i
    \end{displaymath}
    \lang{de}{
    mit % Koeffizienten
    $\alpha_1, \ldots , \alpha_k \in  \R$ 
    hei"st \notion{Linearkombination} von $\vec{v}_1,  \ldots,
    \vec{v}_k$. \\
    \\ Die reellen Zahlen $\alpha_1, \ldots , \alpha_k $ nennt man dabei auch die \notion{Koeffizienten} der Linearkombination.
    }
    \lang{en}{
    with coefficients $\alpha_1, \ldots , \alpha_k \in \R$ is called a \notion{linear combination} of 
    $\vec{v}_1, \ldots , \vec{v}_k$.
    }
  \end{definition}

\begin{remark}
\lang{de}{Eine Linearkombination ist eine \emph{endliche} Summe von Vektoren.}
\lang{en}{A linear combination is a \emph{finite} sum of vectors.}
\end{remark}

%
%%% Video Hoever
%
\lang{de}{
Das folgende Video veranschaulicht die Definition der \emph{Linearkombination} an einem Beispiel im 2-dimensionalen
Koordinatensystem und visualisiert, wie man einen beliebigen Vektor des $\R^2$ mit einer geeigneten Linearkombination
darstellen kann.
\\
    \floatright{\href{https://www.hm-kompakt.de/video?watch=706}{\image[75]{00_Videobutton_schwarz}}}\\\\
}
\lang{en}{}

\begin{example} \label{ex:linearkombinationen}
%
%%% Video Hoever
%
\lang{de}{
\begin{enumerate}[alph]
  \item Wir lösen zunächst die Abschlussfrage aus dem vorherigen Video: 
   $\quad$ \floatright{\href{https://www.hm-kompakt.de/video?watch=706lsg}{\image[75]{00_Videobutton_schwarz}}}\\\\
%
  \item Linearkombinationen der Vektoren $\begin{pmatrix} 2\\ -1\end{pmatrix}$, $\begin{pmatrix} -1\\ 3\end{pmatrix}$
und $\begin{pmatrix} 1\\ 0\end{pmatrix}$ sind zum 
Beispiel
\[ \frac{1}{2}\cdot \begin{pmatrix} 2\\ -1\end{pmatrix} + \frac{1}{2}\cdot \begin{pmatrix} -1\\ 3\end{pmatrix}
+ \frac{1}{2} \cdot \begin{pmatrix} 1\\ 0\end{pmatrix}
= \begin{pmatrix} 1 \\ 1\end{pmatrix},\]
\[ (-1) \cdot \begin{pmatrix} 2\\ -1\end{pmatrix} + 0\cdot \begin{pmatrix} -1\\ 3\end{pmatrix}
+ 3 \cdot \begin{pmatrix} 1\\ 0\end{pmatrix}
= \begin{pmatrix} 1 \\ 1\end{pmatrix}\]
und
\[ 3\cdot \begin{pmatrix} 2\\ -1\end{pmatrix} + 1\cdot \begin{pmatrix} -1\\ 3\end{pmatrix}
-5 \cdot \begin{pmatrix} 1\\ 0\end{pmatrix}
= \begin{pmatrix} 0\\ 0\end{pmatrix}.\]
%
\end{enumerate}
}
\lang{en}{
Some example linear combinations of the vectors $\begin{pmatrix} 2\\ -1\end{pmatrix}$, 
$\begin{pmatrix} -1\\ 3\end{pmatrix}$ and $\begin{pmatrix} 1\\ 0\end{pmatrix}$ are
\[ \frac{1}{2}\cdot \begin{pmatrix} 2\\ -1\end{pmatrix} + 
     \frac{1}{2}\cdot \begin{pmatrix} -1\\ 3\end{pmatrix} + 
       \frac{1}{2} \cdot \begin{pmatrix} 1\\ 0\end{pmatrix} =
          \begin{pmatrix} 1 \\ 1\end{pmatrix},\]
\[ (-1) \cdot \begin{pmatrix} 2\\ -1\end{pmatrix} + 
     0\cdot \begin{pmatrix} -1\\ 3\end{pmatrix} + 
       3 \cdot \begin{pmatrix} 1\\ 0\end{pmatrix} = 
         \begin{pmatrix} 1 \\ 1\end{pmatrix}\]
and
\[ 3\cdot \begin{pmatrix} 2\\ -1\end{pmatrix} + 
     1\cdot \begin{pmatrix} -1\\ 3\end{pmatrix} - 
       5 \cdot \begin{pmatrix} 1\\ 0\end{pmatrix} = 
         \begin{pmatrix} 0\\ 0\end{pmatrix}.\]
}
\end{example}


\begin{quickcheck}
        \field{rational}
        \type{input.number}
        \begin{variables}
            \randint{q1}{-5}{5}
            \randint[Z]{q2}{-5}{5}
            \randint[Z]{r1}{-5}{5}
            \randint[Z]{r2}{-5}{5}
            \randint[Z]{s1}{-5}{5}
            \number{s2}{0}

            \function[calculate]{c1}{q2/r2}
            \function[calculate]{c2}{(q1-r1*c1)/s1}

            % für explanation
            \function[normalize]{k1}{r1*a+s1*b}
            \function[calculate]{z}{c1*r1}		

        \end{variables}

            \text{\lang{de}{
            Schreiben Sie den Vektor $\vec{w}=\begin{pmatrix} \var{q1}\\ \var{q2}\end{pmatrix}$ als 
            Linearkombination der Vektoren $\vec{v}_1=\begin{pmatrix} \var{r1}\\ \var{r2}\end{pmatrix}$
            und $\vec{v}_2=\begin{pmatrix} \var{s1}\\ \var{s2}\end{pmatrix}$.\\
            Es ist
            $\begin{pmatrix} \var{q1}\\ \var{q2}\end{pmatrix}=$\ansref $\cdot \begin{pmatrix} \var{r1}\\ \var{r2}\end{pmatrix}
            + $\ansref $\cdot \begin{pmatrix} \var{s1}\\ \var{s2}\end{pmatrix}$.
            }
            \lang{en}{
            Write the vector $\vec{w}=\begin{pmatrix} \var{q1}\\ \var{q2}\end{pmatrix}$ as a linear
            combination of the vectors $\vec{v}_1=\begin{pmatrix} \var{r1}\\ \var{r2}\end{pmatrix}$ 
            and $\vec{v}_2=\begin{pmatrix} \var{s1}\\ \var{s2}\end{pmatrix}$.\\
            It is 
            $\begin{pmatrix} \var{q1}\\ \var{q2}\end{pmatrix}=$\ansref $\cdot \begin{pmatrix} \var{r1}\\ \var{r2}\end{pmatrix}
            + $\ansref $\cdot \begin{pmatrix} \var{s1}\\ \var{s2}\end{pmatrix}$.
            }}

        \begin{answer}
            \solution{c1}
        \end{answer}
        \begin{answer}
            \solution{c2}
        \end{answer}
        \explanation{\lang{de}{
        Es sind reelle Zahlen $a$ und $b$ zu finden, so dass\\
        $ \begin{pmatrix} \var{q1}\\ \var{q2}\end{pmatrix} = 
          a \cdot \begin{pmatrix} \var{r1}\\ \var{r2}\end{pmatrix} + 
            b \cdot \begin{pmatrix} \var{s1}\\ \var{s2}\end{pmatrix}. $\\
        Also dass\\
        $ \begin{pmatrix} \var{q1}\\ \var{q2}\end{pmatrix} = 
          \begin{pmatrix} \var{k1}\\ \var{r2}a \end{pmatrix}. $\\
        Aus der zweiten Komponente erhält man $a=\frac{\var{q2}}{\var{r2}}=\var{c1}$, und damit aus 
        der ersten\\
        $b=\frac{\var{q1}-(\var{r1}a)}{\var{s1}}=\frac{\var{q1}-(\var{z})}{\var{s1}}=\var{c2}.$
        }
        \lang{en}{
        We are searching for real numbers $a$ and $b$ such that\\
        $ \begin{pmatrix} \var{q1}\\ \var{q2}\end{pmatrix} = 
          a \cdot \begin{pmatrix} \var{r1}\\ \var{r2}\end{pmatrix} + 
            b \cdot \begin{pmatrix} \var{s1}\\ \var{s2}\end{pmatrix}. $\\
        That is, such that\\
        $ \begin{pmatrix} \var{q1}\\ \var{q2}\end{pmatrix} = 
          \begin{pmatrix} \var{k1}\\ \var{r2}a \end{pmatrix}. $\\
        From the equality of the second components, $a=\frac{\var{q2}}{\var{r2}}=\var{c1}$, and 
        hence from the equality of the first components,\\
        $b=\frac{\var{q1}-(\var{r1}a)}{\var{s1}}=\frac{\var{q1}-(\var{z})}{\var{s1}}=\var{c2}.$
        }}

    \end{quickcheck}


\begin{example}
\lang{de}{
\begin{enumerate}[alph]
  \item Jeder Vektor im $\R^3$ ist eine Linearkombination der sogenannten \emph{Standardvektoren}
    \[  \vec{e}_1=\begin{pmatrix} 1\\ 0\\ 0\end{pmatrix},\quad \vec{e}_2=\begin{pmatrix} 0\\ 1\\ 0\end{pmatrix}
    \quad \text{und}\quad \vec{e}_3=\begin{pmatrix} 0\\ 0\\ 1\end{pmatrix}. \]
    Für den Vektor $\vec{v}=\begin{pmatrix} 5\\ -2\\ 4\end{pmatrix}$ gilt beispielsweise
    \[ \begin{pmatrix} 5\\ -2\\ 4\end{pmatrix} = 5\cdot \begin{pmatrix} 1\\ 0\\ 0\end{pmatrix}
    + (-2)\cdot \begin{pmatrix} 0\\ 1\\ 0\end{pmatrix}+4\cdot \begin{pmatrix} 0\\ 0\\ 1\end{pmatrix}
    = 5\cdot \vec{e}_1 -2\cdot \vec{e}_2+4\cdot \vec{e}_3.\]
    \\
    Die Koeffizienten der Linearkombination mit den Standardvektoren $\vec{e}_1, \vec{e}_2 , \vec{e}_3$ entsprechen dabei genau
    den Komponenten des Vektors $\vec{v}$, der durch $\vec{e}_1, \vec{e}_2 , \vec{e}_3$ linear kombiniert wird.  
%
%%% Video Hoever
%
  \item Ein weiteres Beispiel für eine Linearkombination im 3-dimensionalen Raum: 
   $\quad$ \floatright{\href{https://www.hm-kompakt.de/video?watch=708}{\image[75]{00_Videobutton_schwarz}}}\\\\
%
\end{enumerate}
}
\lang{en}{
Every vector in $\R^3$ is a linear combination of the so-called \emph{standard unit vectors}
\[  \vec{e}_1=\begin{pmatrix} 1\\ 0\\ 0\end{pmatrix},
      \quad \vec{e}_2=\begin{pmatrix} 0\\ 1\\ 0\end{pmatrix}\quad \text{and}
        \quad \vec{e}_3=\begin{pmatrix} 0\\ 0\\ 1\end{pmatrix}. \]
For example, the vector $\vec{v}=\begin{pmatrix} 5\\ -2\\ 4\end{pmatrix}$ can be written as
\[ \begin{pmatrix} 5\\ -2\\ 4\end{pmatrix} = 
     5\cdot \begin{pmatrix} 1\\ 0\\ 0\end{pmatrix} + 
       (-2)\cdot \begin{pmatrix} 0\\ 1\\ 0\end{pmatrix} + 
         4\cdot \begin{pmatrix} 0\\ 0\\ 1\end{pmatrix} = 
           5\cdot \vec{e}_1 -2\cdot \vec{e}_2+4\cdot \vec{e}_3.\]
\\
Note that the coefficients of the linear combination of the unit vectors $\vec{e}_1, \vec{e}_2 , 
\vec{e}_3$ are precisely the components of the vectors $\vec{v}$.
}
\end{example}

\end{visualizationwrapper}


\end{content}