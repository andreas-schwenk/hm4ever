\documentclass{mumie.element.exercise}
%$Id$
\begin{metainfo}
  \name{
    \lang{de}{Ü07: Erzeugendensysteme/Basen}
    \lang{en}{}
  }
  \begin{description} 
 This work is licensed under the Creative Commons License Attribution 4.0 International (CC-BY 4.0)   
 https://creativecommons.org/licenses/by/4.0/legalcode 

    \lang{de}{Hier die Beschreibung}
    \lang{en}{}
  \end{description}
  \begin{components}
  \end{components}
  \begin{links}
  \end{links}
  \creategeneric
\end{metainfo}
\begin{content}
\begin{block}[annotation]
	Im Ticket-System: \href{http://team.mumie.net/issues/9464}{Ticket 9464}
\end{block}

\title{
\lang{de}{Ü07: Erzeugendensysteme/Basen}
}

\begin{enumerate}
% Video
  \item \lang{de}{Machen Sie sich anschaulich klar, ob die folgende Mengen ein Erzeugendensystem
                bzw. sogar eine Basis des $\R^2$ bilden.}
    \begin{table}[\class{items}]
       \nowrap{a) $\left\{\begin{pmatrix}1\\3\end{pmatrix}, \begin{pmatrix}0\\1\end{pmatrix}  \right\}$} 
     & \nowrap{b) $\left\{\begin{pmatrix}2\\1\end{pmatrix} \right\}$} 
     & \nowrap{c) $\left\{\begin{pmatrix}3\\1\end{pmatrix}, \begin{pmatrix}1\\3\end{pmatrix}  \right\}$} \\
       \nowrap{d) $\left\{\begin{pmatrix}3\\1\end{pmatrix}, \begin{pmatrix}1\\3\end{pmatrix}, \begin{pmatrix}1\\1 \end{pmatrix} \right\}$}
     & \nowrap{e) $\left\{\begin{pmatrix}2\\-1\end{pmatrix}, \begin{pmatrix}-4\\2\end{pmatrix} \right\}$}
     &\\
    \end{table}
%    
  \item \lang{de}{Bilden folgende Mengen eine Basis des $\R^3$?}
    \begin{table}[\class{items}]
      \nowrap{a) $\left\{\begin{pmatrix}1\\1\\0\end{pmatrix}, \begin{pmatrix}2\\-1\\3\end{pmatrix}  \right\} \qquad$} 
     & \nowrap{b) $\left\{\begin{pmatrix}1\\0\\1\end{pmatrix}, \begin{pmatrix}3\\-1\\4\end{pmatrix}, \begin{pmatrix}0\\2\\3\end{pmatrix} \right\}\qquad$}
     & \nowrap{c) $\left\{\begin{pmatrix}2\\-1\\4\end{pmatrix}, \begin{pmatrix}3\\0\\5\end{pmatrix}, \begin{pmatrix}-6\\3\\-12\end{pmatrix}\right\}$}
    \end{table}
\end{enumerate}

\begin{tabs*}[\initialtab{0}\class{exercise}]
  \tab{
  \lang{de}{Antwort}
  }
\begin{table}[\class{items}]

       \nowrap{1 a) ist eine Basis} 
     & \nowrap{1 b) ist keine Basis, auch kein Erzeugendensystem} \\
       \nowrap{1 c) ist eine Basis} 
     & \nowrap{1 d) ist ein Erzeugendensystem, aber keine Basis} \\
       \nowrap{1 e) ist kein Erzeugendensystem (da linear abhängig)}
     & \nowrap{2 a) ist keine Basis} \\
       \nowrap{2 b) ist eine Basis}
     & \nowrap{2 c) ist keine Basis}
    
\end{table}
% Video
  \tab{\lang{de}{Lösungsvideo 1 a) - e)}}
    \youtubevideo[500][300]{dfrSsYiq39Q}\\
  
  \tab{
  \lang{de}{Lösung 2 a)}}
  
  \begin{incremental}[\initialsteps{1}]
    \step 
    \lang{de}{Da die Dimension des $\R^3$ gleich $3$ ist, benötigt man mindestens
	$3$ Vektoren um jeden Vektor des $\R^3$ erzeugen zu können. Da die 
	Menge nur aus zwei Elementen besteht, kann sie kein Erzeugendensystem
	des $\R^3$ bilden und ist damit auch keine Basis des $\R^3$. }
  \end{incremental}

  \tab{
  \lang{de}{Lösung 2 b)}
  }
  \begin{incremental}[\initialsteps{1}]
    \step \lang{de}{Da die Menge aus $3$ Elementen besteht, müssen wir diese nur auf lineare
	Unabhängigkeit überprüfen.}
    \step \lang{de}{Dazu seien $a,b,c\in \R$ und wir setzen an }
	\[\begin{pmatrix}0\\0\\0\end{pmatrix} \overset{!}{=} a\cdot\begin{pmatrix}1\\0\\1\end{pmatrix}+b\cdot \begin{pmatrix}3\\-1\\4\end{pmatrix} +c\cdot \begin{pmatrix}0\\2\\3\end{pmatrix} = \begin{pmatrix}a+3b\\-b+2c\\a+4b+3c\end{pmatrix}. \]
    \step \lang{de}{Aus der ersten Zeile ergibt sich $a=-3b$ und aus der zweiten Zeile ergibt sich
	$\frac{1}{2}b=c$. Einsetzen in die dritte Zeile liefert $ -3b+4b+ \frac{3}{2}b = 0$, was
	äquivalent dazu ist, dass $\frac{5}{2}b=0$ gilt und somit $b=0$. 
	Wenn $b=0$, dann gilt auch $a=0$ und $c=0$, also sind die Vektoren
	linear unabhängig und bilden somit eine Basis des $\R^3$. }
  \end{incremental}

  \tab{
  \lang{de}{Lösung zu 2 c)}
  }
  \begin{incremental}[\initialsteps{1}]
    \step \lang{de}{Wie man sehen kann, entspricht der dritte Vektor genau dem $(-3)-$fachen des ersten
	Vektors. Damit sind die beiden Vektoren linear abhängig und es folgt sofort, dass die Menge keine 
    Basis des $\R^3$ ist. 
    \\
    Erkennt man das nicht sofort, geht man analog zu Teilaufgabe b) vor und überprüft die drei Vektoren auf 
    lineare Unabhängigkeit.}
    \step \lang{de}{Wir setzen für $a,b,c\in \R$ }
	\[\begin{pmatrix}0\\0\\0\end{pmatrix} \overset{!}{=} a\cdot\begin{pmatrix}2\\-1\\4\end{pmatrix}+b\cdot \begin{pmatrix}3\\0\\5\end{pmatrix} +c\cdot \begin{pmatrix}-6\\3\\-12 \end{pmatrix} = \begin{pmatrix}2a+3b-6c\\-a+3c\\4a+5b-12c\end{pmatrix}. \]
    \step \lang{de}{Aus der zweiten Zeile ergibt sich, dass $a=3c$ ist. Ersetzen wir nun in der ersten Zeile $a$ durch $3c$,
    so erhalten wir $2 \cdot 3c +3b -6c=0$, also $b=0$.
    Setzt man nun $b=0$ und $a=3c$ in die dritte Gleichung ein, so ergibt sich:
    \begin{align*}
    4a+5b-12c &\,=0 \\
    \Leftrightarrow \quad 4\cdot 3c+5\cdot 0-12c &\,=0\\
    \Leftrightarrow \quad \quad \quad\quad \quad \quad\quad \quad0&\,=0
    
    \end{align*}
    Es folgt also, dass die Gleichung
    \[\begin{pmatrix}0\\0\\0\end{pmatrix} \overset{!}{=} 3c\cdot\begin{pmatrix}2\\-1\\4\end{pmatrix}+0\cdot \begin{pmatrix}3\\0\\5\end{pmatrix} +c\cdot \begin{pmatrix}-6\\3\\-12 \end{pmatrix} \]
	für alle $c \in \R$ erfüllt ist. Es gibt also unendlich viele Lösungen des Gleichungssystems. Somit ist hiermit auch rechnerisch gezeigt, dass die Vektoren linear 
    abhängig sind und somit keine Basis des $\R^3$ bilden. }
  \end{incremental}

\end{tabs*}

\end{content}