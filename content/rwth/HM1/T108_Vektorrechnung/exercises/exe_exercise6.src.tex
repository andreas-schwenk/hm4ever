\documentclass{mumie.element.exercise}
%$Id$
\begin{metainfo}
  \name{
    \lang{de}{Ü06: Erzeugendensysteme}
    \lang{en}{}
  }
  \begin{description} 
 This work is licensed under the Creative Commons License Attribution 4.0 International (CC-BY 4.0)   
 https://creativecommons.org/licenses/by/4.0/legalcode 

    \lang{de}{Hier die Beschreibung}
    \lang{en}{}
  \end{description}
  \begin{components}
  \end{components}
  \begin{links}
  \end{links}
  \creategeneric
\end{metainfo}
\begin{content}
\begin{block}[annotation]
	Im Ticket-System: \href{http://team.mumie.net/issues/9457}{Ticket 9457}
\end{block}

\title{
\lang{de}{Ü06: Erzeugendensysteme}
}
 
\lang{de}{Bilden die folgenden Mengen ein Erzeugendensystem des $\R^3$?}
\begin{table}[\class{items}]
  \nowrap{a) $\left\{ \begin{pmatrix}1\\0\\1\end{pmatrix}, \begin{pmatrix}2\\0\\1\end{pmatrix}, \begin{pmatrix}5\\0\\0\end{pmatrix}  \right\}$} \\
  \nowrap{b) $\left\{ \begin{pmatrix}2\\-1\\2\end{pmatrix}, \begin{pmatrix}0\\1\\1\end{pmatrix}, \begin{pmatrix}2\\1\\0\end{pmatrix}  \right\}$}
\end{table}

\begin{tabs*}[\initialtab{0}\class{exercise}]
  \tab{
  \lang{de}{Antwort}
  }
\begin{table}[\class{items}]

    \nowrap{a) Nein} \\
    \nowrap{b) Ja} 
  \end{table}

  \tab{
  \lang{de}{Lösung a)}}
  
  \begin{incremental}[\initialsteps{1}]
    \step 
    \lang{de}{In der zweiten Komponente jedes in der Menge enthaltenen Vektors
	steht eine Null. Also kann durch Linearkombination der Vektoren auch
	nie ein Eintrag ungleich Null an zweiter Stelle erzeugt werden. Damit bilden 
	die Vektoren kein Erzeugendensystem. Zur Verdeutlichung: Beispielsweise
	kann der Vektor \begin{align*}\begin{pmatrix}1\\1\\1\end{pmatrix} \end{align*} nicht aus den angegebenen 
	Vektoren erzeugt werden.}
    
  \end{incremental}

  \tab{
  \lang{de}{Lösung b)}
  }
  \begin{incremental}[\initialsteps{1}]
    \step \lang{de}{Da die Menge aus $3$ Elementen besteht und die Dimension des $\R^3$ 
	auch $3$ ist, genügt es zu testen, ob die Vektoren linear unabhänig sind.}
    \step \lang{de}{Dazu seien $a,b,c\in \R$ und wir setzen an}
	\[\begin{pmatrix}0\\0\\0\end{pmatrix} \overset{!}{=} a\cdot \begin{pmatrix}2\\-1\\2\end{pmatrix} + b\cdot \begin{pmatrix}0\\1\\1\end{pmatrix} + c \cdot \begin{pmatrix}2\\1\\0\end{pmatrix} = \begin{pmatrix}2a+2c\\-a+b+c\\2a+b\end{pmatrix}.\]
    \step \lang{de}{Aus der ersten Zeile ergibt sich $-a=c$ und aus der dritten Zeile
 	erhalten wir $b=-2a$. Setzen wir diese Bedingungen in die zweite
 	Zeile ein, so erhalten wir $ -a -2a-a = 0$, was gleichbedeutend mit
 	$a=0$ ist. Wenn aber $a=0$ gilt, so müssen auch $b=0$ und $c=0$ 
 	gelten. Das bedeutet genau, dass die Vektoren linear unabhängig sind
 	und sie bilden damit ein Erzeugendensystem.}
  \end{incremental}

\end{tabs*}

\end{content}