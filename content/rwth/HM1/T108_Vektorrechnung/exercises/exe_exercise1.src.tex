\documentclass{mumie.element.exercise}
%$Id$
\begin{metainfo}
  \name{
    \lang{de}{Ü01: Vektoren/Pfeilklassen}
    \lang{en}{Exercise 1}
  }
  \begin{description} 
 This work is licensed under the Creative Commons License Attribution 4.0 International (CC-BY 4.0)   
 https://creativecommons.org/licenses/by/4.0/legalcode 

    \lang{de}{Hier die Beschreibung}
    \lang{en}{}
  \end{description}
  \begin{components}
    \component{generic_image}{content/rwth/HM1/images/g_tkz_T108_Exercise01_A.meta.xml}{T108_Exercise01_A}
    \component{generic_image}{content/rwth/HM1/images/g_tkz_T108_Exercise01_B.meta.xml}{T108_Exercise01_B}  
  \end{components}
  \begin{links}
  \end{links}
  \creategeneric
\end{metainfo}
\begin{content}
\title{
	\lang{de}{Ü01: Vektoren/Pfeilklassen}
  	\lang{en}{Exercise 1}
}

\begin{block}[annotation]
	Vektoren und Pfeilklassen
\end{block}
\begin{block}[annotation]
	Im Ticket-System: \href{http://team.mumie.net/issues/9109}{Ticket 9109}
\end{block}

\begin{block}[exercise]
	\lang{de}{Welche der folgenden Pfeile gehören zur gleichen Pfeilklasse? \\%
	Wie viele verschiedene Vektoren sind abgebildet?}	
	\lang{en}{Which of the following arrows belong to the same arrow class? \\
	How many different vectors are depicted in the image below?}
	\begin{figure}
		\image{T108_Exercise01_A}
	\end{figure}
\end{block}

\begin{tabs*}[\initialtab{0}\class{exercise}]
	\tab{\lang{de}{Klasseneinteilung}\lang{en}{Classification}}	
  		\lang{de}{Zwei Pfeile gehören zur gleichen Pfeilklasse, wenn sie durch Parallelverschiebung ineinander überführbar sind.
		Dies ist genau dann der Fall, wenn sie in die gleiche Richtung zeigen und die gleiche Länge besitzen. Damit erhalten wir folgende Klasseneinteilung:}
		\lang{en}{Two arrows belong to the same arrow class if they can be put overtop of each other by parallel shifts. 
		This is the case if they both point in the same direction and have the same length, hence we get the following classification:}
		\begin{figure}
			\image{T108_Exercise01_B}
		\end{figure}
		
  	\tab{\lang{de}{Anzahl der Vektoren}\lang{en}{Number of Vectors}}
  		\lang{de}{Anhand der Einteilung in Pfeilklassen sind also sechs verschiedene Vektoren abgebildet.}
  		\lang{en}{With the help of the division into arrow classes, there are six different vectors depicted.}
		\begin{figure}
			\image{T108_Exercise01_B}
		\end{figure}
\end{tabs*}
\end{content}