\documentclass{mumie.element.exercise}
%$Id$
\begin{metainfo}
  \name{
    \lang{de}{Ü04: Vektorrechnung}
    \lang{en}{}
  }
  \begin{description} 
 This work is licensed under the Creative Commons License Attribution 4.0 International (CC-BY 4.0)   
 https://creativecommons.org/licenses/by/4.0/legalcode 

    \lang{de}{Hier die Beschreibung}
    \lang{en}{}
  \end{description}
  \begin{components}
      \component{generic_image}{content/rwth/HM1/images/g_tkz_T108_Exercise04.meta.xml}{T108_Exercise04}
  \end{components}
  \begin{links}
  \end{links}
  \creategeneric
\end{metainfo}
\begin{content}

\begin{block}[annotation]
	Im Ticket-System: \href{http://team.mumie.net/issues/9455}{Ticket 9455}
\end{block}

\title{
\lang{de}{Ü04: Vektorrechnung}
}
 
\begin{table}[\class{items}]
\nowrap{\lang{de}{a) Berechnen Sie die folgenden Vektoren. Kürzen Sie auftretende Brüche vollständig.}
  \begin{table}[\class{items}]
     \nowrap{i) $\quad \begin{pmatrix}3\\-1\\4\end{pmatrix}-\begin{pmatrix}2\\0\\1\end{pmatrix}+ 5\cdot \begin{pmatrix}4\\1\\-1\end{pmatrix}$} \\
     \nowrap{ii) $\quad 4\cdot \left( \begin{pmatrix}2\\1\\-3\end{pmatrix}+\begin{pmatrix}-3\\2\\9\end{pmatrix} \right)  $ }\\
     \nowrap{iii) $\quad\begin{pmatrix}\frac{3}{5}\\1\end{pmatrix}+\begin{pmatrix}\frac{3}{15}\\3\end{pmatrix}-\begin{pmatrix}\frac{1}{10}\\5\end{pmatrix}$} 
  \end{table} 
}
\nowrap{\lang{de}{b) Bestimmen Sie einen Vektor $\overset{\rightarrow}{v}$ im $\R^3$ für den gilt:}
%  \begin{align*}
  \begin{table}[\class{items}]
\nowrap{$\quad 
  2 \cdot \left( \overset{\rightarrow}{v} + \begin{pmatrix}-\frac{1}{2}\\1\\-\frac{3}{2}\end{pmatrix} \right)  = \overset{\rightarrow}{v} 
$}
  \end{table} 
%  \end{align*}
}
\\

\nowrap{\lang{de}{c) Gegeben seien die zwei Punkte $P=(3; 0; 4)$ und $Q=(-2;-2;-4).$\newline
  Bestimmen Sie den Punkt $M \in \R^3,$ der auf der Strecke $PQ$ liegt und diese 
  im Verh\"altnis $4:5$ teilt.}
}

\end{table}
\\

\begin{tabs*}[\initialtab{0}\class{exercise}]
  \tab{
  \lang{de}{Antwort}
  }
\begin{table}[\class{items}]

    \nowrap{a) \begin{table}[\class{items}]
  		\nowrap{i) $\begin{pmatrix}21\\4\\-2\end{pmatrix}$} \\
    	\nowrap{ii) $\begin{pmatrix}-4\\12\\24\end{pmatrix}$ }\\
    	\nowrap{iii) $\begin{pmatrix}\frac{7}{10}\\-1\end{pmatrix}$} 
				\end{table} }\\
				
     \nowrap{b) $\begin{pmatrix}1\\-2\\3\end{pmatrix}$}  \\
     \\

     \nowrap{c) $\displaystyle \left( \frac{7}{9}; -\frac{8}{9}; \frac{4}{9} \right )$}
   
  \end{table}

  \tab{
  \lang{de}{Lösung a i)}}
  
  \begin{incremental}[\initialsteps{1}]
    \step 
    \lang{de}{Wir berechnen von links nach rechts zunächst die Summe der ersten beiden Vektoren. 
    Gleichzeitig nutzen wir im ersten Schritt die Definition der skalaren Multiplikation aus. 
    Wir addieren dann die verbleibenden beiden Vektoren und erhalten}
    \begin{align*}
    \begin{pmatrix}3\\-1\\4\end{pmatrix}-\begin{pmatrix}2\\0\\1\end{pmatrix}+ 5\cdot \begin{pmatrix}4\\1\\-1\end{pmatrix}
    &= \begin{pmatrix}1\\-1\\3\end{pmatrix} + \begin{pmatrix}20\\5\\-5\end{pmatrix} \\
    &= \begin{pmatrix}21\\4\\-2\end{pmatrix}. 
    \end{align*}
  \end{incremental}
  
  \tab{
  \lang{de}{Lösung a ii)}}
  
  \begin{incremental}[\initialsteps{1}]
  	\step 
  	\lang{de}{Mit Hilfe des Distributivgesetzes lösen wir zunächst die Klammern auf und nutzen die Definition der skalaren Multiplikation. 
  	Im zweiten Schritt addieren wir die beiden Vektoren komponentenweise. 
  	Es ergibt sich}
  	\begin{align*}
  	4\cdot \left( \begin{pmatrix}2\\1\\-3\end{pmatrix}+\begin{pmatrix}-3\\2\\9\end{pmatrix} \right)
  	&= \begin{pmatrix}8\\4\\-12\end{pmatrix}+ \begin{pmatrix}-12\\8\\36\end{pmatrix} \\
  	&= \begin{pmatrix}-4\\12\\24\end{pmatrix}. 
  	\end{align*}
  	\step     
    Alternativ kann auch zuerst die Summe der beiden Vektoren in der Klammer durch komponentenweise Addition berechnet werden.
    Dann wird im 2. Schritt der Summenvektor skalar mit dem Faktor 4 multipliziert.
    \[
  	  4 \cdot \left( \begin{pmatrix}2\\1\\-3\end{pmatrix}+\begin{pmatrix}-3\\2\\9\end{pmatrix} \right)
  	= 4 \cdot \begin{pmatrix}-1\\3\\6\end{pmatrix}
  	= \begin{pmatrix}-4\\12\\24\end{pmatrix}.     
    \]
   
  \end{incremental}
  \tab{
  \lang{de}{Lösung a iii)}}
  
  \begin{incremental}[\initialsteps{1}]
 	\step 
  	\lang{de}{Wir bringen in der ersten Komponente zunächst alle Brüche auf den kleinsten gemeinsamen Nenner $30$. 
  	Dann addieren wir jeweils die Komponenten der Vektoren. 
  	Anschließen kürzen wir das Ergebnis der ersten Komponente mit $3$ und erhalten somit das Ergebnis als}
  	\begin{align*}
  	\begin{pmatrix}\frac{3}{5}\\1\end{pmatrix}+\begin{pmatrix}\frac{3}{15}\\3\end{pmatrix}-\begin{pmatrix}\frac{1}{10}\\5\end{pmatrix} 
  	&= \begin{pmatrix}\frac{18}{30}\\1\end{pmatrix} + \begin{pmatrix}\frac{6}{30}\\3\end{pmatrix} - \begin{pmatrix}\frac{3}{30}\\5\end{pmatrix}\\
  	&= \begin{pmatrix}\frac{21}{30}\\-1\end{pmatrix} \\
  	&=\begin{pmatrix}\frac{7}{10}\\-1\end{pmatrix}. 
  	\end{align*}
  \end{incremental}
   
  \tab{
  \lang{de}{Lösung b)}
  }
  \begin{incremental}[\initialsteps{1}]
    
    \step \lang{de}{Anwenden des Distributivgesetzes liefert zunächst die Gleichung in der Form}
	\[ 2\overset{\rightarrow}{v} + \begin{pmatrix}-1\\2\\-3\end{pmatrix}  = \overset{\rightarrow}{v}. \]
	\step \lang{de}{Wir ziehen dann auf beiden Seiten $1\cdot \overset{\rightarrow}{v}$ ab und erhalten }
	\[\overset{\rightarrow}{v} + \begin{pmatrix}-1\\2\\-3\end{pmatrix} =0. \]
 	\step \lang{de}{Dies ist äquivalent dazu, dass }
 	\[\overset{\rightarrow}{v} = - \begin{pmatrix}-1\\2\\-3\end{pmatrix} = \begin{pmatrix}1\\-2\\3\end{pmatrix}. \]
  \end{incremental}
   
  \tab{
  \lang{de}{Lösung c)}
  }
  \begin{incremental}[\initialsteps{1}]
    
    \step \lang{de}{Zur Lösung dieser Aufgabe betrachten wir die Verbindungsvektoren $\overrightarrow{PQ}$ von $P$ nach $Q$ und $\overrightarrow{PM}$ von $P$ nach $M$. 
    Da der Punkt $M$ auf der Strecke $PQ$ liegt, zeigen die beiden Vektoren $\overrightarrow{PQ}$ und $\overrightarrow{PM}$ in die gleiche Richtung.
    Aus der Tatsache, dass der Punkt M die Strecke Strecke $PQ$ im Verh\"altnis $4:5$ teilt, k\"onnen wir zudem das L\"angenverh\"altnis der 
    Vektoren $\overrightarrow{PM}$ und $\overrightarrow{PQ}$ ableiten.} 
    
    \begin{figure}
    \image{T108_Exercise04}
    \end{figure}
    
    \step \lang{de}{Wir nutzen also die Skalierungseigenschaft von Vektoren und bestimmen den Vektor $\overrightarrow{PM}$ aus $\overrightarrow{PQ}$ durch:}    
    \[\overrightarrow{PM} = \frac{4}{4+5} \overrightarrow{PQ}.\] 
    
       
    \step \lang{de}{Nun berechnen wir den Ortsvektor zum Punkt $M$ aus dem Ortsvektor zu $P$ und dem Verbindungsvektor von $P$ nach $M$:}
    \begin{align*}
  	\overrightarrow{0M} &= \overrightarrow{0P} + \overrightarrow{PM} = \overrightarrow{0P} + \frac{4}{9} \overrightarrow{PQ} \\
   	&= \begin{pmatrix} 3 \\0 \\ 4\end{pmatrix} + \frac{4}{9} \cdot \begin{pmatrix} -2-3\\-2-0\\-4-4 \end{pmatrix} \\
   	&= \begin{pmatrix} 3 \\0 \\ 4\end{pmatrix} + \frac{4}{9} \cdot \begin{pmatrix} -5\\-2\\-8 \end{pmatrix} \\
   	&= \begin{pmatrix} \frac{7}{9}\\ -\frac{8}{9}\\ \frac{4}{9} \end{pmatrix}  
   	\end{align*}
    
    \lang{de}{ $M= \left( \frac{7}{9}; -\frac{8}{9}; \frac{4}{9} \right )$ ist der gesuchte Punkt, der die Strecke $PQ$ im Verh\"altnis $4:5$ teilt.}

    \end{incremental}
% 
% \tab{\lang{de}{Video: ähnliche Übungsaufgabe}}
%  \youtubevideo[500][300]{eFqtOvChdh8}\\
%
\end{tabs*}

\end{content}