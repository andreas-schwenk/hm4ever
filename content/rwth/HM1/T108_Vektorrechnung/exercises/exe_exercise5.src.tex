\documentclass{mumie.element.exercise}
%$Id$
\begin{metainfo}
  \name{
    \lang{de}{Ü05: Lineare Abhängigkeit}
    \lang{en}{}
  }
  \begin{description} 
 This work is licensed under the Creative Commons License Attribution 4.0 International (CC-BY 4.0)   
 https://creativecommons.org/licenses/by/4.0/legalcode 

    \lang{de}{Hier die Beschreibung}
    \lang{en}{}
  \end{description}
  \begin{components}
  \end{components}
  \begin{links}
    \link{generic_article}{content/rwth/HM1/T108_Vektorrechnung/g_art_content_30_basen_eigenschaften.meta.xml}{3.2_Lineare_Abhängigkeit}
    \end{links}
  \creategeneric
\end{metainfo}
\begin{content}



\begin{block}[annotation]
	Im Ticket-System: \href{http://team.mumie.net/issues/9456}{Ticket 9456}
\end{block}

\title{
\lang{de}{Ü05: Lineare Abhängigkeit}
}
 
\lang{de}{Untersuchen Sie jeweils,
ob die gegebenen Vektoren linear abh{\"a}ngig
oder linear unabh{\"a}ngig sind.}

\begin{table}[\class{items}]
  \nowrap{a) $\begin{pmatrix}3\\1\\-2\end{pmatrix},
			  \begin{pmatrix}-1\\3\\-2\end{pmatrix}$} 
 & \nowrap{d) $\begin{pmatrix}1\\0\\-2\end{pmatrix},
              \begin{pmatrix}0\\1\\3\end{pmatrix},
              \begin{pmatrix}1\\1\\1\end{pmatrix}$} \\
%
   \nowrap{b) $\begin{pmatrix}1\\0\\2\end{pmatrix}, 
               \begin{pmatrix}0\\0\\0\end{pmatrix}$} 
 &  \nowrap{e) $\begin{pmatrix}1\\0\\-2\end{pmatrix},
                \begin{pmatrix}0\\1\\3\end{pmatrix},
                \begin{pmatrix}0\\1\\1\end{pmatrix}$} \\  
%                
   \nowrap{c) $\begin{pmatrix}2\\4\\-2\end{pmatrix}, 
               \begin{pmatrix}-3\\-6\\3\end{pmatrix}$} 
 &  \nowrap{f) $\begin{pmatrix}1\\0\\-1\end{pmatrix},
                \begin{pmatrix}3\\1\\-2\end{pmatrix},
                \begin{pmatrix}0\\2\\c\end{pmatrix}$
                (für welche Werte von c?)}            
\end{table}

\begin{tabs*}[\initialtab{0}\class{exercise}]
  \tab{
  \lang{de}{Antwort}
  }
\begin{table}[\class{items}]

    \nowrap{a) linear unabhängig} \\
    \nowrap{b) linear abhängig} \\
    \nowrap{c) linear abhängig} \\
    \nowrap{d) linear abhängig}  \\
    \nowrap{e) linear unabhängig}  \\    
    \nowrap{f) linear abhängig nur für $c=2$, also 
               linear unabhängig für alle $c \neq 2$  }
  \end{table}

  \tab{
  \lang{de}{Lösung a)}}
  
  \begin{incremental}[\initialsteps{1}]
    \step 
    \lang{de}{ Zwei Vektoren sind linear abhängig, wenn sie Vielfache voneinander sind.
 	Vergleicht man die erste Komponente beider Vektoren müsste man den ersten
 	Vektor mit $-\frac{1}{3}$ multiplizieren um die erste Komponente
 	des zweiten Vektors zu erhalten. In der zweiten Komponente liefert 
 	die Multiplikation mit $-\frac{1}{3}$ aber nicht die zweite Komponente des anderen
 	Vektors. Somit sind die Vektoren linear unabhängig.}
  \end{incremental}

  \tab{
  \lang{de}{Lösung b)}
  }
  \begin{incremental}[\initialsteps{1}]
     \step \lang{de}{Wenn der Nullvektor in einer Menge von Vektoren ist, die auf lineare
	Unabhängigkeit überprüft werden sollen, so sind die Vektoren stets 
	linear abhängig, da man einfach die anderen Vektoren mit dem Skalar $0$
	multiplizieren muss um den Nullvektor zu erhalten.}
     
  \end{incremental}

  \tab{
  \lang{de}{Lösung c)}
  }
  \begin{incremental}[\initialsteps{1}]
    \step \lang{de}{Zwei Vektoren sind Vektoren sind linear abhängig, wenn sie Vielfache voneinander
	sind. Multipliziert man hier den ersten Vektor mit $-\frac{3}{2}$, so
	erhält man den zweiten Vektor. Es liegt also eine lineare Abhängigkeit vor.}
    
    
  \end{incremental}

 \tab{
  \lang{de}{Lösung d)}
  }
  \begin{incremental}[\initialsteps{1}]
    \step \lang{de}{Drei Vektoren sind linear abhängig, wenn sich einer der Vektoren als
	Linearkombination der anderen beiden darstellen lässt. Da}
	\begin{align*}\begin{pmatrix}1\\0\\-2\end{pmatrix}+\begin{pmatrix}0\\1\\3\end{pmatrix} = \begin{pmatrix}1\\1\\1\end{pmatrix}, \end{align*}
	\lang{de}{sind die Vektoren linear abhängig.}
    
 \end{incremental}

  \tab{
  \lang{de}{Lösung e)}
  }
  \begin{incremental}[\initialsteps{1}]
    \step \lang{de}{Nach dem \ref[3.2_Lineare_Abhängigkeit][Satz über lineare Unabhängigkeit]{Satz_Lin_Unabh} sind die drei Vektoren 
    \[\begin{pmatrix}1\\0\\-2\end{pmatrix},
    \begin{pmatrix}0\\1\\3\end{pmatrix},
    \begin{pmatrix}0\\1\\1\end{pmatrix}\]    
    genau dann linear unabh"angig sind, wenn die Gleichung 
    \[ r_1\begin{pmatrix}1\\0\\-2\end{pmatrix} +r_2\begin{pmatrix}0\\1\\3\end{pmatrix} +r_3\begin{pmatrix}0\\1\\1\end{pmatrix}
    =\begin{pmatrix}0\\0\\0\end{pmatrix} \, (*)\]
    nur f"ur $r_1=r_2=r_3=0$ erf"ullt ist.}  

    \step \lang{de}{Die obige Gleichung (*)
%    \[ r_1\begin{pmatrix}1\\0\\-2\end{pmatrix} +r_2\begin{pmatrix}0\\1\\3\end{pmatrix} +r_3\begin{pmatrix}0\\1\\1\end{pmatrix}
%    =\begin{pmatrix}0\\0\\0\end{pmatrix} \]
    liefert das lineare Gleichungssystem}
    \begin{align*}
     r_1 &+0r_2 &+0r_3 &= 0 \\
    0r_1 &+1r_2 &+1r_3 &= 0 \\
    -2r_1 &+3r_2&+1r_3 &= 0
    \end{align*}

    \step \lang{de}{Aus der 1. Gleichung folgt sofort, dass $r_1=0$ ist. Aus der 2. Gleichung folgt $r_3=-r_2$. Beides eingesetzt in die 3. Gleichung ergibt:
     \[-2 \cdot 0 +3r_2 +1\cdot(-r_2) = 2r_2=0.\] Also ist auch $r_2=0$ und somit auch $r_3=-r_2=0. $ \\
     Damit ist die lineare Unabhängigkeit der drei Vektoren nachgewiesen.}
        
  \end{incremental}

  \tab{\lang{de}{Lösungsvideo f)}}
    \youtubevideo[500][300]{877Q2txoOtk}\\


\end{tabs*}

\end{content}