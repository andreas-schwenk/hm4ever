\documentclass{mumie.element.exercise}
%$Id$
\begin{metainfo}
  \name{
    \lang{de}{Ü02: Verbindungsvektoren}
    \lang{en}{}
  }
  \begin{description} 
 This work is licensed under the Creative Commons License Attribution 4.0 International (CC-BY 4.0)   
 https://creativecommons.org/licenses/by/4.0/legalcode 

    \lang{de}{Hier die Beschreibung}
    \lang{en}{}
  \end{description}
  \begin{components}
  \end{components}
  \begin{links}
\link{generic_article}{content/rwth/HM1/T108_Vektorrechnung/g_art_content_27_vektoren.meta.xml}{content_27_vektoren}
\end{links}
  \creategeneric
\end{metainfo}
\begin{content}

\title{
\lang{de}{Übung 2}
}
\begin{block}[annotation]
	Vektoren durch zwei Punkte 
\end{block}
\begin{block}[annotation]
	Im Ticket-System: \href{http://team.mumie.net/issues/9453}{Ticket 9453}
\end{block}
%
% Aufgabenstellung
% 
 \begin{enumerate}[alph]
% Video 
    \item Zeichnen Sie die Punkte $\,P=(3;1), Q=(1;-2)\,$ und $\,S=(-2;3)$ und
          die zugehörigen Orstvektoren $\,\vec{p}, \vec{q}\,$ und $\,\vec{s}.$
    \item Berechnen Sie $\,\vec{p}+\vec{q}\,$ und $\,\vec{p}-\vec{s}.$
    \item Welcher Vektor führt von  $\,P\,$ zu $\,S$, welcher von $\,Q\,$ zu $\,P.$   
    \item Bestimmen und zeichnen Sie $\,2\cdot\vec{p}, -\frac{1}{2}\cdot\vec{p}\,$ 
          und $\,2\cdot (\vec{p}+\vec{q}).$
    \item Wie erhält man den Punkt $\,T$, der genau zwischen $\,P\,$ und $\,Q\,$ liegt?   
%
    \item \lang{de}{Berechnen Sie zu den Punkten $A = (1;4;-2)$, $B = (2;1;1)$ und $C = (-1;-1;2)$ die Vektoren}
          \lang{en}{Given the points $A = (1,4,-2)$, $B = (2,1,1)$, and $C = (-1,-1,2)$, calculate:}	
          \[\vec{AB},\quad \vec{AC}\quad \text{\lang{de}{und}} \text{\lang{en}{and}} \quad \vec{BC}.\]

\end{enumerate}

\begin{tabs*}[\initialtab{0}\class{exercise}]
%        
     \tab{\lang{de}{Antworten}}
       \begin{enumerate}[alph]
          \item -
          \item $\; \vec{p}+\vec{q} =\begin{pmatrix}4\\-1\end{pmatrix}\;$ und $\; \vec{p}-\vec{s}=\begin{pmatrix}5\\-2\end{pmatrix},$
          
          \item $\; \vec{PS}=\begin{pmatrix}-5\\2\end{pmatrix}\;$ und $\; \vec{QP}=\begin{pmatrix}2\\3\end{pmatrix},$ 
          
          \item $2\vec{p}=\begin{pmatrix}6\\2\end{pmatrix},$ $\; -\frac{1}{2}\vec{p}=\begin{pmatrix}-1,5\\-0,5\end{pmatrix},$ 
                $\; 2(\vec{p}+\vec{q})=\begin{pmatrix}8\\-2\end{pmatrix},$
                
          \item $\; T=\begin{pmatrix}2\\-0,5\end{pmatrix},$  
 
          \item $\; \vec{AB}= \begin{pmatrix} 1\\-3\\3\end{pmatrix},\quad 
                 \vec{AC}= \begin{pmatrix} -2\\-5\\4\end{pmatrix},\quad  
                 \vec{BC}= \begin{pmatrix} -3\\-2\\1\end{pmatrix}.$

      \end{enumerate}

%

% Video         
     \tab{\lang{de}{Lösungsvideo a) - e)}}
        \youtubevideo[500][300]{6ljJQTnIyOA}\\
%
	\tab{Lösung f)}		
	  	\lang{de}{Die Berechnung der Vektoren $\vec{AB},\; \vec{AC}\;$ und $\; \vec{BC}\;$
                  erfolgt nach dem \ref[content_27_vektoren][Satz über Verbindungsvektoren]{thm:verbindungsvektor}:}
%	  	\lang{en}{The vector $\vec{AB}$ is given by:}		
		\[\vec{AB} = \vec{OB}-\vec{OA}= \begin{pmatrix} 2\\1\\1\end{pmatrix}-\begin{pmatrix} 1\\4\\-2\end{pmatrix} 
         = \begin{pmatrix} 2-1\\1-4\\1-(-2)\end{pmatrix} = \begin{pmatrix} 1\\-3\\3\end{pmatrix}.\]
%  	\tab{$\vec{AC}$}			
%		\lang{de}{Der Vektor $\vec{AC}$ wird nach dem \ref[content_27_vektoren][Satz über Verbindungsvektoren]{thm:verbindungsvektor}
%        berechnet durch}
%  		\lang{en}{The vector $\vec{AC}$ is given by:}
  		\[\vec{AC} = \vec{OC}-\vec{OA}= \begin{pmatrix} -1-1\\-1-4\\2-(-2)\end{pmatrix} = \begin{pmatrix} -2\\-5\\4\end{pmatrix}.\]
%  	\tab{$\vec{BC}$}
%  		\lang{de}{Der Vektor $\vec{BC}$ wird nach dem \ref[content_27_vektoren][Satz über Verbindungsvektoren]{thm:verbindungsvektor}
%        berechnet durch}
%  		\lang{en}{The vector $\vec{BC}$ is given by:}	
		\[\vec{BC} = \vec{OC}-\vec{OB}= \begin{pmatrix} -1-2\\-1-1\\2-1\end{pmatrix} = \begin{pmatrix} -3\\-2\\1\end{pmatrix}.\]
              
\end{tabs*}

\end{content}