%$Id:  $
\documentclass{mumie.article}
%$Id$
\begin{metainfo}
  \name{
    \lang{de}{Vektoren im Anschauungsraum}
    \lang{en}{Vectors in space}
  }
  \begin{description} 
 This work is licensed under the Creative Commons License Attribution 4.0 International (CC-BY 4.0)   
 https://creativecommons.org/licenses/by/4.0/legalcode 

    \lang{de}{Beschreibung}
    \lang{en}{Description}
  \end{description}
  \begin{components}
\component{generic_image}{content/rwth/HM1/images/g_tkz_T108_Example.meta.xml}{T108_Example}
\component{generic_image}{content/rwth/HM1/images/g_tkz_T108_CollinearVectors.meta.xml}{T108_CollinearVectors}
\component{generic_image}{content/rwth/HM1/images/g_tkz_T108_VectorMultiplication.meta.xml}{T108_VectorMultiplication}
\component{generic_image}{content/rwth/HM1/images/g_tkz_T108_VectorDifference.meta.xml}{T108_VectorDifference}
\component{generic_image}{content/rwth/HM1/images/g_tkz_T108_VectorCommutativity.meta.xml}{T108_VectorCommutativity}
\component{generic_image}{content/rwth/HM1/images/g_tkz_T108_VectorAddition.meta.xml}{T108_VectorAddition}
\component{generic_image}{content/rwth/HM1/images/g_tkz_T108_LocationVector.meta.xml}{T108_LocationVector}
\component{generic_image}{content/rwth/HM1/images/g_tkz_T108_Vectors.meta.xml}{T108_Vectors}
\component{generic_image}{content/rwth/HM1/images/g_tkz_T108_Vector3D.meta.xml}{T108_Vector3D}
\component{generic_image}{content/rwth/HM1/images/g_tkz_T108_Vector2D.meta.xml}{T108_Vector2D}
\component{generic_image}{content/rwth/HM1/images/g_img_00_Videobutton_schwarz.meta.xml}{00_Videobutton_schwarz}
  \end{components}
  \begin{links}
    %\link{generic_article}{content/rwth/HM1/T101neu_Elementare_Rechengrundlagen/g_art_content_01_zahlenmengen.meta.xml}{content_01_zahlenmengen}
    \link{generic_article}{content/rwth/HM1/T109_Skalar-_und_Vektorprodukt/g_art_content_31_skalarprodukt.meta.xml}{link-skalarprodukt}
  \end{links}
  \creategeneric
\end{metainfo}
\begin{content}
\usepackage{mumie.ombplus}
\ombchapter{8}
\ombarticle{1}
\usepackage{mumie.genericvisualization}

\begin{visualizationwrapper}

\title{\lang{de}{Vektoren im Anschauungsraum}\lang{en}{Vectors in space}}
 
\begin{block}[annotation]
  übungsinhalt
  
\end{block}
\begin{block}[annotation]
  Im Ticket-System: \href{http://team.mumie.net/issues/9044}{Ticket 9044}\\
\end{block}

\begin{block}[info-box]
\tableofcontents
\end{block}

\section{\lang{de}{Vektoren}\lang{en}{Vectors in space}}
\label{sec:1_1_Vektoren}

\lang{de}{
Viele physikalische Größen sind nicht nur durch eine Maßzahl festgelegt, sondern auch durch ihre 
Richtung im Raum. In diesem Kapitel wollen wir uns solchen gerichteten Größen, sogenannten Vektoren, 
widmen.
\\\\
Wir betrachten hierzu die Vektoren im $n$-dimensionalen \emph{euklidischen Anschauungsraum}, der 
gegeben ist durch die Menge der Punkte des $\R^n:=\{ (x_1;\ldots; x_n) \mid x_i\in \R\}.\;$ 
% Änderung/Ergänzung aufgrund Ticket #11:
Die Menge $\,\R^n\,$ ist dabei definiert als das $n-$fache kartesisches Produkt%\ref[content_01_zahlenmengen][kartesische Produkt ]{def:mengenoperationen} 
der Menge $\R$, die Vektoren sind also die \emph{n-Tupel} $\, (x_1;\ldots; x_n).$
\\\\
In den Beispielen beschränken wir uns dabei in der Regel auf den anschaulichen Fall des zwei- oder 
drei-dimensionalen Raumes $\R^2$ bzw. $\R^3$. 
}

\lang{en}{
Many physical quantities are not only determined by their magnitude, rather also by the their 
direction in space. In this chapter we will focus on learning about these directional quantities: 
so-called vectors.
\\\\
For this we consider the vectors in $n$-dimensional \emph{Euclidean space}, which is given by the set 
$\R^n:=\{ (x_1;\ldots; x_n) \mid x_i\in \R\}.\;$ The set $\,\R^n\,$ is therefore defined as the 
$n$-fold Cartesian product of the set $\,\R^n$, with the vectors being \emph{$n$-tuples} 
$\, (x_1;\ldots; x_n).$
\\\\
In our examples we will restrict ourselves to the two- and three-dimensional spaces $\R^2$ and 
$\R^3$. 
}




\begin{example}
\begin{table}[\class{item}]
$\underline{\text{f"ur } n=2}$    & &
$\underline{\text{f"ur } n=3}$ \\ 
 \lang{de}{die Ebene}\lang{en}{the plane} && 
   \lang{de}{der $3$-dimensionale Raum}\lang{en}{$3$-dimensional space} \\ 
\begin{center} \image{T108_Vector2D}\end{center} && \begin{center} \image{T108_Vector3D}\end{center}
 \end{table}
\end{example}

\begin{definition}[Vektor] \label{Gegenvektor}  % Ziel des Links aus Bem. 2.2 (8.2 Rechenregeln und Linearkombination)
\lang{de}{
Ein \emph{Vektor} $\vec{v}$ im $n$-dimensionalen Raum ist eine Gr"o"se, die durch eine \emph{L"ange} 
und eine \emph{Richtung} im $\R^n$ gekennzeichnet ist. \\
Dargestellt werden Vektoren durch \emph{Pfeile} (gerichtete Strecken) im $\R^n$, deren Spitzen die 
Richtung anzeigen. Die Länge des Vektors ist eine reelle Zahl $\geq 0$ und entspricht der Länge der 
gerichteten Strecken.
}
\lang{en}{
A \emph{vector} $\vec{v}$ in $n$-dimensional space is an object characterised by its magnitude and 
its direction in $\R^n$.\\
Vectors are represented on a graph by \emph{arrows} of a given length in $\R^n$, representing the 
vector's positive real magnitude, and pointing in the direction of the vector.
}
\end{definition}

\lang{de}{
Ein Vektor der Länge Null heißt \emph{Nullvektor}.
\\\\
Ist $\vec{v}$ ein Vektor, so bezeichnen wir mit $-\vec{v}$ denjenigen Vektor, 
der dieselbe Länge wie $\vec{v}$ besitzt, jedoch genau in die entgegengesetzte Richtung zeigt. 
Der Vektor $-\vec{v}$ heißt \emph{Gegenvektor } 
\ zum Vektor $\vec{v}$. 
In der Literatur wird ein Vektor $\vec{v}$ auch oft fettgedruckt geschrieben, also mit $\mathbf{v}$ bezeichnet.
}
\lang{en}{
Ein Vektor der Länge Null heißt \emph{Nullvektor}.
\\\\
If $\vec{v}$ is a vector, then we denote by $-\vec{v}$ the vector that has the same length as 
$\vec{v}$ but that points in the opposite direction of $\vec{v}$. This is often called the negative 
of the vector $\vec{v}$. In the literature, often $\mathbf{v}$ is used instead of $\vec{v}$ to 
denote a vector.
} %This has no formal name in English.

\begin{remark} \label{rem:pfeilklasse}
\lang{de}{
Die Lage eines Vektors $\vec{v}$ im $\R^n$ ist nicht eindeutig.
Alle Pfeile, die in Richtung und Länge übereinstimmen, stellen denselben Vektor dar, 
unabhängig von ihrem Anfangs- und Endpunkt im $\R^n$. Man spricht auch von \emph{Pfeilklassen}.
}
\lang{en}{
An arrow in $\R^n$ does not uniquely represent a vector. Every arrow that points in the same 
direction and has the same length represents the same vector, regardless of its position in $\R^n$. 
That is to say, a vector does not by default have a position assosciated with it, only a length and 
direction.
}
\end{remark}

\begin{example}
\begin{center}
\image{T108_Vectors}
\end{center}  

\lang{de}{
Die drei orangen Pfeile gehören zur selben Pfeilklasse, da sie die gleiche L"ange und die gleiche 
Richtung haben. Sie stellen also denselben Vektor dar. Ebenso stellen die drei blauen Pfeile 
denselben Vektor dar. Der grüne Pfeil stellt einen anderen Vektor als die blauen Pfeile dar, da er in 
die entgegengesetzte Richtung geht.
}
\lang{en}{
The three orange arrows represent the same vector, as they have the same length and direction. 
Similarly, the three blue arrows also represent one vector. The green arrow represents a different 
vector to the blue arrows, as it points in the opposite direction.
}
\end{example}

\begin{definition}[\lang{de}{Ortsvektoren und Verbindungsvektoren}
                   \lang{en}{Position vectors}]\label{def:orts_verb_vec}
\lang{de}{
Ist $P$ ein Punkt im $\R^n$, so nennt man den Vektor, der durch den Pfeil mit 
Anfang im Nullpunkt $O=(0;\ldots; 0)$ und Ende beim Punkt $P$ dargestellt wird,
den \emph{Ortsvektor von $P$} und bezeichnet ihn mit $\overrightarrow{OP}$ oder kurz mit $\vec{p}$.\\
\\
Sind $Q$ und $R$ Punkte im $\R^n$, dann ist 
der \emph{Verbindungsvektor von $Q$ nach $R$} der Vektor, der durch den
Pfeil mit Anfang bei $Q$ und Ende bei $R$ dargestellt wird. Dieser wird mit
$\overrightarrow{QR}$ bezeichnet.
}
\lang{en}{
Let $P$ be a point in $\R^n$. Then we call the vector represented by an arrow from the origin 
$O=(0;\ldots; 0)$ to the point $P$ the \emph{position vector of $P$} and denote it by 
$\overrightarrow{OP}$ or $\vec{p}$.
\\\\
Let $Q$ and $R$ be points in $\R^n$. Then we denote the vector represented by an arrow from $Q$ to 
$R$ by $\overrightarrow{QR}$.
}
\end{definition}


\begin{example}
\begin{center}
\image{T108_LocationVector}
\end{center} 

\lang{de}{
Ortsvektor $\vec{p}=\overrightarrow{OP}$ von $P$ und Verbindungsvektor $\overrightarrow{QR}$.
}
\lang{en}{
The position vector $\vec{p}=\overrightarrow{OP}$ of $P$ and the vector $\overrightarrow{QR}$ from 
$Q$ to $R$.
}
\end{example}

\begin{quickcheck}
		\field{rational}
		\type{input.number}
		\begin{variables}
			\number{s1}{1}
			\number{s2}{2}
		\end{variables}
		
			\text{\lang{de}{
      In obigem Beispiel haben die Punkte $Q$ und $R$ die Koordinaten $Q=(\frac{1}{2};1)$ und
			$R=(\frac{3}{2};3)$. Welche Koordinaten muss der Punkt $S$ haben, damit der Ortsvektor von $S$
			gleich dem Verbindungsvektor $\overrightarrow{QR}$ ist?\\
			$S=($\ansref;\ansref$)$.
      }
      \lang{en}{
      In the above example, the points $Q$ and $R$ have the coordinates $Q=(\frac{1}{2};1)$ and 
      $R=(\frac{3}{2};3)$. Which coordinates must the point $S$ have for the position vector of $S$ 
      to be the same as the vector $\overrightarrow{QR}$?\\
      $S=($\ansref;\ansref$)$.
      }}
		
		\begin{answer}
			\solution{s1}
		\end{answer}
		\begin{answer}
			\solution{s2}
		\end{answer}
		\explanation{\lang{de}{
    Damit die Vektoren dieselben sind, müssen die darstellenden Pfeile in die gleiche Richtung gehen
		und gleich lang sein. Der Pfeil zum Verbindungsvektor $\overrightarrow{QR}$ geht von $Q=(\frac{1}{2};1)$ nach
		$R=(\frac{3}{2};3)$, also $1$ Einheit in $x$-Richtung und $2$ Einheiten in $y$-Richtung. Daher muss auch der Pfeil zum
		Ortsvektor von $S$ eine Einheit in $x$-Richtung und $2$ Einheiten in $y$-Richtung gehen. Da der Pfeil zum
		Ortsvektor im Ursprung startet, muss also das Ende $S$ die Koordinaten $(1;2)$ haben.		
		}
    \lang{en}{
    So that the vectors are equal, their magnitudes and directions must be equal. The arrow of the 
    vector $\overrightarrow{QR}$ goes from $Q=(\frac{1}{2};1)$ to	$R=(\frac{3}{2};3)$, so it goes 
    $1$ unit in the positive $x$-direction and $2$ units in the positive $y$-direction. The same 
    arrow starting at the origin must of course end at the coordinates $(1;2)$.
    }}
	\end{quickcheck}

%%%%%%%%%%%  ab hier Umstellung in Zusammenhang mit Einbau der Videos %%%%%%%%%%%%%%%%%%%%%%%%%%%%%%%%%

\lang{de}{
Wir haben die Koordinaten des Verbindungsvektors hier im Beispiel aus der Anschauung hergeleitet.
Verbindungsvektoren im $\R^n$ lassen sich aber auch berechnen, wie wir später in Satz \ref{thm:verbindungsvektor} 
sehen werden. Hierzu führen wir zunächst die Komponentendarstellung von Vektoren im $\R^n$ ein und zeigen, 
wie man hiermit mit Vektoren rechnen kann.
}
\lang{en}{
In the above example we represent vectors using arrows on a graph. They can also be represented by 
tuples, as we will show now, and then vectors between points can be found using theorem 
\ref{thm:verbindungsvektor}. 
}

\section{\lang{de}{Komponentendarstellung von Vektoren}\lang{en}{Component representation of vectors}}\label{sec:kompon_spalt_darst}

\lang{de}{
Jeder Vektor $\vec{v}$ im $\R^n$ entspricht dem Ortsvektor von genau einem Punkt 
$P=(p_1;\ldots; p_n)$.  
Wir schreiben daher diesen Vektor auch als
\[ \vec{v}=\overrightarrow{OP}=\left( \begin{smallmatrix} p_1 \\ p_2 \\ \vdots \\ p_n \end{smallmatrix} \right). \]
Man verwendet die Spaltenschreibweise für Vektoren, um diese von Punkten im $\R^n$ besser unterscheiden zu können.
\\\\
% \begin{remark}
Der Nullvektor $\vec{0}=\left( \begin{smallmatrix} 0 \\ 0 \\ \vdots \\ 0 \end{smallmatrix} \right)$ 
(= Ortsvektor von $O=(0; \ldots; 0)$) wird oft auch einfach als $0$, also ohne Vektorpfeil, geschrieben.
% \end{remark}
}
\lang{en}{
Every vector $\vec{v}$ in $\R^n$ is the position vector of precisely the point $P=(p_1;\ldots; p_n)$. 
We therefore write this vector as 
\[ \vec{v}=\overrightarrow{OP}=\left( \begin{smallmatrix} p_1 \\ p_2 \\ \vdots \\ p_n \end{smallmatrix} \right). \]
This vertical 'component' or tuple representation of the vector distinguishes it from the point. 
\\\\
% \begin{remark}
The zero vector $\vec{0}=\left( \begin{smallmatrix} 0 \\ 0 \\ \vdots \\ 0 \end{smallmatrix} \right)$  
(= position vector of $O=(0; \ldots; 0)$) is often simply denoted by $0$.
% \end{remark}
}

\section{\lang{de}{Rechnen mit Vektoren}\lang{en}{Calculating with vectors}}\label{sec:rechnen}

\lang{de}{
Zwei Vektoren k"onnen addiert werden und ein Vektor kann mit einer 
reellen Zahl multipliziert werden. Das Ergebnis ist jeweils ein neuer Vektor.
}
\lang{en}{
Two vectors can be added together, and a vector can be multiplied by a real number. The result of 
these operations is always a vector.
}

\begin{definition}[\lang{de}{Vektoraddition}\lang{en}{Vector addition}]\label{def:vec-addition}
\lang{de}{
Die Summe $\vec{v}+\vec{w}$ zweier Vektoren $\vec{v}$ und $\vec{w}$ ist gegeben durch Aneinanderhängen 
der zwei Vektoren:
}
\lang{en}{
The sum $\vec{v}+\vec{w}$ of two vectors $\vec{v}$ and $\vec{w}$ is represented by joining together 
their two arrows in space.
}

\begin{center}
\image{T108_VectorAddition}
\end{center} 

\lang{de}{In Komponentendarstellung ist diese Summe durch komponentenweise Addition gegeben:}
\lang{en}{The sum corresponds to componentwise addition of the two tuples:}
\[ \left( \begin{smallmatrix} v_1 \\ v_2 \\ \vdots \\ v_n \end{smallmatrix} \right)+
\left( \begin{smallmatrix} w_1 \\ w_2 \\ \vdots \\ w_n \end{smallmatrix} \right)=
\left( \begin{smallmatrix} v_1+w_1 \\ v_2+w_2 \\ \vdots \\ v_n+w_n \end{smallmatrix} \right) \]
\end{definition}

%%%%%%%%%%%%%%% Ziel des Links aus Bemerkung 2.2, Abschnitt 8.2 Rechenregeln und Linearkombination
\begin{remark} \label{Bem_Vektor_Addition}
    \lang{de}{
    Anhand der folgenden Skizze erkennt man auch, dass $\vec{v}+\vec{w} = \vec{w}+\vec{v}$ gilt.     
    \\
     Die Addition zweier Vektoren ist also \emph{kommutativ}, d.h. es kommt beim Addieren nicht auf die Reihenfolge an.
     Wichtig hingegen ist die Eigenschaft (gemäß Bem. \ref{rem:pfeilklasse}), dass es sich bei Vektoren um \emph{Pfeilklassen} 
     und nicht um \emph{Pfeile} mit festem Anfangs- und Endpunkt handelt.
     }          
    \lang{en}{
    With the help of the following sketch, we can see that the addition of two vectors 
    $\vec{v}+\vec{w} = \vec{w}+\vec{v}$ is \emph{commutative}, i.e. the order in which you add
    two vectors does not matter. The orange and blue arrows can be connected together in either order 
    to obtain the same green arrow.
    }
    \begin{center}
\image{T108_VectorCommutativity}
\end{center} 
    \\
    \lang{de}{
    Mit Hilfe des Gegenvektors können wir auch Differenzen von Vektoren bilden: \\
    Sind $\vec{v}$ und $\vec{w}$ Vektoren, so ist
    $\vec{v}-\vec{w}$ der Vektor gegeben durch $\vec{v} + (-\vec{w})$, wobei $-\vec{w}$ der Gegenvektor zu $\vec{w}$ ist.
    }
    \lang{en}{
    Using the negative of a vector we can also calculate the difference between two vectors: if 
    $\vec{v}$ and $\vec{w}$ are two vectors, then $\vec{v}-\vec{w}$ is the vector given by 
    $\vec{v} + (-\vec{w})$, where $-\vec{w}$ is simply the vector pointing in the opposite direction 
    of $\vec{w}$.
    }
	\\
    \begin{center}
\image{T108_VectorDifference}
\end{center} 
    \\     
    \lang{de}{F\"ur jeden Vektor $\vec{v}$ gilt $\vec{v} - \vec{v} = \vec{v} + (-\vec{v}) = \vec{0}$.}
    \lang{en}{We have $\vec{v} - \vec{v} = \vec{v} + (-\vec{v}) = \vec{0}$ for any vector $\vec{v}$.}       
\end{remark}

%
%%% Video Hoever
%
\lang{de}{
Im folgenden Video wird die Definition von Vektoren sowie die Vektoraddition im 
2-dimensionalen reellen Raum veranschaulicht:
\\
\floatright{\href{https://www.hm-kompakt.de/video?watch=700}{\image[75]{00_Videobutton_schwarz}}}\\\\
}
\lang{en}{}

\begin{example}
\lang{de}{
  \begin{enumerate}[alph]
  \item Wir lösen die Aufgabe aus dem vorstehenden Video: 
   $\quad$ \floatright{\href{https://www.hm-kompakt.de/video?watch=701lsg}{\image[75]{00_Videobutton_schwarz}}}\\\\
%
  \item Wir berechnen nun die Summe zweier Vektoren im 3-dimensionalen 
    \\   
    Für $\vec{v} = \begin{pmatrix} 2\\3\\1\end{pmatrix}$
    und
    $\vec{w} = \begin{pmatrix} 4\\1\\-2\end{pmatrix}$ 
    gilt
	\[\vec{v} + \vec{w} = \begin{pmatrix} 2 + 4 \\ 3 + 1 \\ 1 + (-2) \end{pmatrix} = \begin{pmatrix} 6 \\ 4\\ -1\end{pmatrix}
    \]
    und
    \[\vec{w} + \vec{v} = \begin{pmatrix} 4 + 2 \\ 1 + 3 \\ (-2) + 1 \end{pmatrix} = \begin{pmatrix} 6 \\ 4\\ -1\end{pmatrix}.
    \]
    Die Differenz der beiden Vektoren ist gegeben durch
    \[\vec{v}-\vec{w} =\vec{v} + (-\vec{w})
	= \begin{pmatrix} 2\\3\\1\end{pmatrix} + \left(-\begin{pmatrix} 4\\1\\-2\end{pmatrix}\right) 
	% Alternativen zu \left und \right?
	= \begin{pmatrix} 2\\3\\1\end{pmatrix} + \begin{pmatrix} -4\\-1\\2\end{pmatrix}
	= \begin{pmatrix} -2\\2\\3\end{pmatrix}. \]
  \end{enumerate}
}
\lang{en}{
We now calculate the sum and then the difference of two vectors in $3$-dimensions.
\\   
Let $\vec{v} = \begin{pmatrix} 2\\3\\1\end{pmatrix}$ and 
$\vec{w} = \begin{pmatrix} 4\\1\\-2\end{pmatrix}$. 
Then
\[
\vec{v} + \vec{w} = \begin{pmatrix} 2 + 4 \\ 3 + 1 \\ 1 + (-2) \end{pmatrix} = \begin{pmatrix} 6 \\ 4\\ -1\end{pmatrix}
\]
and
\[
\vec{w} + \vec{v} = \begin{pmatrix} 4 + 2 \\ 1 + 3 \\ (-2) + 1 \end{pmatrix} = \begin{pmatrix} 6 \\ 4\\ -1\end{pmatrix}.
\]
The difference of the two vectors is given by
    \[\vec{v}-\vec{w} =\vec{v} + (-\vec{w})
	= \begin{pmatrix} 2\\3\\1\end{pmatrix} + \left(-\begin{pmatrix} 4\\1\\-2\end{pmatrix}\right) 
	% Alternativen zu \left und \right?
	= \begin{pmatrix} 2\\3\\1\end{pmatrix} + \begin{pmatrix} -4\\-1\\2\end{pmatrix}
	= \begin{pmatrix} -2\\2\\3\end{pmatrix}. \]
}
\end{example}

\begin{definition}[\lang{de}{Skalarmultiplikation}\lang{en}{Scalar multiplication}]\label{def:scalar_mult}
    \lang{de}{
    Die Multiplikation $\lambda \cdot \vec{v}$ eines Vektors $\vec{v}$ mit einer reellen Zahl $\lambda \in \R$ (einem sogenannten
    \emph{Skalar}) \emph{streckt} oder \emph{verkürzt} die Länge des Vektors um den Faktor $|\lambda|$. Ist $\lambda<0$,
    wird zusätzlich die Richtung des Vektors umgekehrt, $\lambda \cdot \vec{v}$ zeigt in diesem Fall folglich in Richtung $-\vec{v}$:
    }
    \lang{en}{
    The multiple $\lambda \cdot \vec{v}$ of a vector $\vec{v}$ by a real number 
    $\lambda \in \R$ (a so-called \emph{scalar}) is the vector $\vec{v}$, \emph{stretched} or 
    \emph{shortened} by a factor of $|\lambda|$. If $\lambda<0$, the direction of the vector is 
    reversed, so it is in the direction of the negative vector $-\vec{v}$:
    }

    \begin{center}
    \image{T108_VectorMultiplication}
    \end{center}

    \lang{de}{
    In Komponentendarstellung ist diese Multiplikation durch komponentenweise Multiplikation gegeben:
    }
    \lang{en}{
    Multiplication by a scalar corresponds to componentwise multiplication by the scalar:
    }
    \[ \lambda \cdot \left( \begin{smallmatrix} v_1 \\ v_2 \\ \vdots \\ v_n \end{smallmatrix} \right)=
    \left( \begin{smallmatrix} \lambda v_1 \\ \lambda v_2 \\ \vdots \\ \lambda v_n \end{smallmatrix} 
    \right)\]
%
%%% Video Hoever
%  
\lang{de}{
\floatright{\href{https://www.hm-kompakt.de/video?watch=701}{\image[75]{00_Videobutton_schwarz}}}\\\\
}
\lang{en}{}
\end{definition}
\lang{de}{
Die skalare Multiplikation eines Vektors $\vec{v}$ mit dem Skalar $\lambda =1$ lässt den Vektor unverändert.
Multipliziert man $\vec{v}$ mit $\lambda =-1$, erhält man den Gegenvektor $-\vec{v}$.
}
\lang{en}{
Scalar multiplication of a vector $\vec{v}$ by the scalar $\lambda =1$ leaves the vecor unchanged. 
Multiplying $\vec{v}$ by $\lambda =-1$, we obtain the negative vector $-\vec{v}$.
}


\begin{block}[warning]
    \lang{de}{
    Es gibt auch ein \link{link-skalarprodukt}{\emph{Skalarprodukt }} zwischen zwei Vektoren. 
  	Die Multiplikation mit Skalaren (auch als \emph{skalare Multiplikation} bezeichnet) darf nicht 
    mit dem Skalarprodukt verwechselt werden.
    }
  	\lang{en}{
    There is also a so-called \emph{scalar product}, or \emph{dot product} between two vectors. This 
    is not to be confused with \emph{scalar multiplication} as described above.
    }
\end{block}

\begin{example}
    \lang{de}{Wir berechnen zwei Skalierungen des Vektors }
    \lang{en}{Let us calculate two scalar multiples of the vector }
    $\vec{v} = \begin{pmatrix} 2\\1\\-2\end{pmatrix}$.\\ 
    \begin{itemize}
        \item 
            \lang{de}{Für $\lambda=2$ gilt:}
            \lang{en}{by the scalar $\lambda=2$:}
            \[2\cdot \vec{v} = 2\cdot \begin{pmatrix} 2\\1\\-2\end{pmatrix} = \begin{pmatrix} 4\\2\\-4\end{pmatrix}\]
        \item
            \lang{de}{Weiterhin gilt für $\lambda=-3$:}
            \lang{en}{by the scalar $\lambda=-3$:}
            	\[-3\cdot \vec{v} = -3 \cdot\begin{pmatrix} 2\\1\\-2\end{pmatrix} = \begin{pmatrix} -6\\-3\\6\end{pmatrix}.\]
    \end{itemize}
\end{example}

\begin{quickcheck}
		\field{rational}
		\type{input.number}
		\begin{variables}
			\randint{q1}{-5}{5}
			\randint{q2}{-5}{5}
			\randint{q3}{-5}{5}
			\randint{r1}{-5}{5}
			\randint{r2}{-5}{5}
			\randint{r3}{-5}{5}
			\randint[Z]{c}{-2}{2}
			\function[calculate]{s1}{c*r1+q1}
			\function[calculate]{s2}{c*r2+q2}
			\function[calculate]{s3}{c*r3+q3}
		\end{variables}
		
			\text{\lang{de}{Berechnen Sie den folgenden Vektor:}
            \lang{en}{Evaluate the following vector expression:}\\
			\begin{table}[\class{no-padding}]
			\rowspan[l][m]{3} $\begin{pmatrix}\var{q1}\\ \var{q2}\\ \var{q3} \end{pmatrix} 
			+ (\var{c}) \cdot \begin{pmatrix}\var{r1}\\ \var{r2}\\ \var{r3} \end{pmatrix} = 
			\left(\begin{matrix} \\ \\ \\ \end{matrix}\right.$ & 
			\ansref & \rowspan[l][m]{3} $\left.\begin{matrix} \\ \\ \\ \end{matrix}\right)$. & \\ 
			\ansref & \\ 
			\ansref & 
			\end{table}
			}
		
		\begin{answer}
			\solution{s1}
		\end{answer}
		\begin{answer}
			\solution{s2}
		\end{answer}
		\begin{answer}
			\solution{s3}
		\end{answer}
		\explanation{\lang{de}{Die Berechnung erfolgt komponentenweise.}
                 \lang{en}{Both operations are componentwise.}}
	\end{quickcheck}

%
%%% Video Hoever
% 
\lang{de}{
Unter Verwendung der Vektoraddition und der Skalarmultiplikation können wir nun auch Verbindungsvektoren zwischen
zwei Punkten im $\R^n$ berechnen. Dies wird im Video anschaulich erläutert: 
\\
\floatright{\href{https://www.hm-kompakt.de/video?watch=702}{\image[75]{00_Videobutton_schwarz}}}\\\\
}
%
\lang{en}{Now we can calculate the vector between two points in $\R^n$.}


\begin{theorem}[\lang{de}{Verbindungsvektoren}
                \lang{en}{Vector between two points}] \label{thm:verbindungsvektor}
\lang{de}{
Sind $Q=(q_1;\ldots; q_n)$ und $R=(r_1;\ldots; r_n)$ Punkte im $\R^n$, so gilt f"ur den 
Verbindungsvektor von $Q$ \emph{(= Anfangspunkt)}  und $R$ \emph{(= Endpunkt)}:
}
\lang{en}{
Let $Q=(q_1;\ldots; q_n)$ and $R=(r_1;\ldots; r_n)$ be points in $\R^n$. Then the vector from $Q$ 
to $R$ is:
}
\[ \overrightarrow{QR}=\overrightarrow{OR}-\overrightarrow{OQ}
  = \left(\begin{smallmatrix} r_1 \\ r_2 \\ \vdots \\ r_n \end{smallmatrix} \right) - \left(\begin{smallmatrix} q_1 \\ q_2 \\ \vdots \\ q_n \end{smallmatrix} \right)
  = \left(\begin{smallmatrix} r_1-q_1 \\ r_2-q_2 \\ \vdots \\ r_n-q_n \end{smallmatrix} \right). \]

\lang{de}{Man rechnet: "`Endpunkt minus Anfangspunkt"'.}
\lang{en}{We calculate the 'end point' minus the 'starting point'.}

\end{theorem}

\begin{example}
\lang{de}{
  \begin{enumerate}[alph]
  \item \\
    \begin{center}
    \image{T108_Example}
    \end{center} 
    In diesem Beispiel sind $Q=(1; 1)$ und $R=(4; 3)$.\\
    Also:
    \[ \overrightarrow{QR}=\left(\begin{smallmatrix} 4-1 \\ 3-1\end{smallmatrix} \right)=\left(\begin{smallmatrix} 3 \\ 2\end{smallmatrix} \right) \]
%
%%% Video Hoever
%   
  \item Als weiteres Beispiel sehen wir hier die Lösung zur Aufgabe aus dem vorigen Video: 
   $\quad$ \floatright{\href{https://www.hm-kompakt.de/video?watch=702lsg}{\image[75]{00_Videobutton_schwarz}}}\\\\
%
  \end{enumerate}
}
\lang{en}{
    \begin{center}
    \image{T108_Example}
    \end{center} 
    In this example we have $Q=(1; 1)$ and $R=(4; 3)$.\\
    Hence:
    \[ \overrightarrow{QR}=\left(\begin{smallmatrix} 4-1 \\ 3-1\end{smallmatrix} \right)=\left(\begin{smallmatrix} 3 \\ 2\end{smallmatrix} \right) \]
}
\end{example}

\begin{quickcheck}
		\field{rational}
		\type{input.number}
		\begin{variables}
			\randint{q1}{-5}{5}
			\randint{q2}{-5}{5}
			\randint{q3}{-5}{5}
			\randint{r1}{-5}{5}
			\randint{r2}{-5}{5}
			\randint{r3}{-5}{5}
			\function[calculate]{s1}{r1-q1}
			\function[calculate]{s2}{r2-q2}
			\function[calculate]{s3}{r3-q3}
		\end{variables}
		
			\text{\lang{de}{
      Bestimmen Sie den Verbindungsvektor $\overrightarrow{QR}$ von
			$Q=(\var{q1};\var{q2})$ nach $R=(\var{r1};\var{r2})$.
      }
      \lang{en}{
      Determine the vector $\overrightarrow{QR}$ from $Q=(\var{q1};\var{q2})$ to $R=(\var{r1};\var{r2})$.
      }\\
			\begin{table}[\class{no-padding}]
			\rowspan[l][m]{2} 
      \lang{de}{Der Vektor ist}\lang{en}{The vector is} 
      $\overrightarrow{QR}=\left(\begin{matrix} \\ \\ \end{matrix}\right.$ &  
			\ansref & \rowspan[l][m]{2} $\left.\begin{matrix} \\ \\ \end{matrix}\right)$. & \\ 
			\ansref & 
			\end{table}
      }
		
		\begin{answer}
			\solution{s1}
		\end{answer}
		\begin{answer}
			\solution{s2}
		\end{answer}
% 		\begin{answer}
% 			\solution{s3}
% 		\end{answer}
		\explanation{\lang{de}{
    Die Berechnung folgt nach der Regel \glqq{}Endpunkt minus Anfangspunkt\grqq{}, d.h. die erste 
		Komponente ist $\var{r1}-(\var{q1})=\var{s1}$ und die zweite ist $\var{r2}-(\var{q2})=\var{s2}$.	
		}
    \lang{en}{
    The calculation follows from the 'end point' minus 'start point' rule, so the first component is 
    $\var{r1}-(\var{q1})=\var{s1}$ and the second is $\var{r2}-(\var{q2})=\var{s2}$.
    }}
	\end{quickcheck}


%%%%%%%%%%%%%%%%%%%%%%%%%%%%%%%%%%%%%%%%%%%%%%%%%%%%%%%%%%%%
%
% Nachfolgendes Beispiels durch Beipielaufgabe 4c) ersetzt
%
%%%%%%%%%%%%%%%%%%%%%%%%%%%%%%%%%%%%%%%%%%%%%%%%%%%%%%%%%%%%
%
% \begin{example}
% Die Skalierungseigenschaft kann auch benutzt werden, um Verbindungsvektoren in bestimmten Situationen zu berechnen:
%
% Seien $P$ und $Q$ zwei Punkte im $\R^n$ und $M$ ein Punkt auf der Strecke $PQ$, der die Strecke im Verhältnis $r:s$ teilt.
% Dann geht der Verbindungsvektor $\overrightarrow{PM}$ in die gleiche Richtung wie der Verbindungsvektor $\overrightarrow{PQ}$,
% aber die Längen liegen im Verhältnis $r: (r+s)$. Also gilt für die Verbindungsvektoren
% \[ \overrightarrow{PM}=\frac{r}{r+s} \overrightarrow{PQ}.\]
% \image[300]{teilpunkt}
% \end{example}
%
%%%%%%%%%%%%%%%%%%%%%%%%%%%%%%%

\section{\lang{de}{Kollinearität von Vektoren}\lang{en}{Parallel vectors}}

%%%%%%%%%%%  Ende der Umstellung in Zusammenhang mit Einbau der Videos %%%%%%%%%%%%%%%%%%%%%%%%%%%%%%%%%

\lang{de}{
Mit Hilfe der skalaren Multiplikation können wir untersuchen, ob zwei Vektoren parallel zueinander 
sind.
}
\lang{en}{
With the help of scalar multiplication we can investigate whether two vectors are parallel to each 
other.
}

\begin{definition}[\lang{de}{Kollinearität von Vektoren}\lang{en}{Parallel vectors}]\label{def:kollinear}
    \lang{de}{
    Zwei Vektoren $\vec{v}$ und $\vec{w}$, von denen keiner gleich dem Nullvektor ist, heißen 
    \emph{kollinear}, wenn es ein $\lambda\in\R$ gibt, so dass
    }
    \lang{en}{
    Two non-zero vectors $\vec{v}$ and $\vec{w}$ are called \emph{parallel} if there exists a 
    $\lambda\in\R$ such that:
    }
  	\[\lambda\cdot\vec{v} = \vec{w}\]
    \lang{de}{
    gilt. Ist dabei $\lambda>0$, zeigen $\vec{v}$ und $\vec{w}$ in die gleiche Richtung, so heißen 
    $\vec{v}$ und $\vec{w}$ auch \emph{parallel}, andernfalls nennt man sie \emph{antiparallel}.
    }
    \lang{en}{
    holds. In particular, if it holds with $\lambda<0$, then $\vec{v}$ and $\vec{w}$ go in opposite 
    directions, and are called \emph{antiparallel}.
    }
\end{definition}

\lang{de}{
Zwei vom Nullvektor verschiedene Vektoren sind also genau dann kollinear, wenn einer ein Vielfaches 
des anderen ist.
}
\lang{en}{
Often two vectors are called parallel or \emph{collinear} if one is a multiple of the other, 
regardless of sign.
}
\begin{center}
\image{T108_CollinearVectors}
\end{center} 

\lang{de}{
In der Abbildung sind alle drei Vektoren $\vec{u}$, $\vec{v}$ und $\vec{w}$ paarweise kollinear. 
Die Vektoren $\vec{u}$ und $\vec{v}$ sind zudem parallel, wohingegen der Vektor $\vec{w}$ 
antiparallel sowohl zu $\vec{u}$, also auch zu $\vec{v}$ ist.
}
\emph{en}{
In the illustration, the three vectors $\vec{u}$, $\vec{v}$ and $\vec{w}$ are pairwise parallel to 
each other. In particular, the vectors $\vec{u}$ and $\vec{v}$ are parallel, and $\vec{w}$ is 
antiparallel to both $\vec{u}$ and $\vec{v}$.
}

\begin{example}
	\lang{de}{Gegeben seien die drei Vektoren}
  \lang{en}{Consider the three vectors}  
    $\vec{u} = \begin{pmatrix} 2 \\-1\\3\end{pmatrix}$, $\vec{v} = \begin{pmatrix} -4 \\2\\-6\end{pmatrix}$
    \lang{de}{ und }
    \lang{en}{and}
    $\vec{w} = \begin{pmatrix} 4 \\-1\\5\end{pmatrix}$\lang{de}{.}
    \lang{en}{.}\\
    \lang{de}{Es sind $\vec{u}$ und $\vec{v}$ kollinear (und antiparallel), denn es gilt}
    \lang{en}{$\vec{u}$ and $\vec{v}$ are (anti)parallel, as}
	\[(-2)\cdot\vec{u} = \vec{v}.\]
    \lang{de}{
    Aber $\vec{u}$ ist nicht kollinear zu $\vec{w}$, denn $\lambda \cdot\vec{u} = \vec{w}$ würde 
    bedeuten
    }
    \lang{en}{
    However, $\vec{u}$ is not parallel to $\vec{w}$ because $\lambda \cdot\vec{u} = \vec{w}$ would 
    give us
    }
	\[\lambda \cdot \begin{pmatrix} 2 \\-1\\3\end{pmatrix} = \begin{pmatrix} 4 \\-1\\5\end{pmatrix},\]
    \lang{de}{d.h. die drei Gleichungen}
    \lang{en}{i.e. that the three equations}
	\begin{align*}
		2\lambda & = 4 \\
		-1\lambda & = -1 \\
		3\lambda & = 5
	\end{align*}
    \lang{de}{
    müssten alle von demselben $\lambda$ erfüllt sein. Die erste Gleichung liefert jedoch 
    $\lambda = 2$, die zweite Gleichung $\lambda = 1$ und die dritte Gleichung 
    $\lambda = \frac{5}{3}$.
    }
  	\lang{en}{
    all need to be satisifed by the same $\lambda$. This cannot however be the case.	The first 
    equation gives us that $\lambda = 2$, the second that $\lambda = 1$, and the third equation 
    $\lambda = \frac{5}{3}$.
    }	
\end{example}


\end{visualizationwrapper}


\end{content}