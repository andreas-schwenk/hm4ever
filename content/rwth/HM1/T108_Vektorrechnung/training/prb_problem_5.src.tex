\documentclass{mumie.problem.gwtmathlet}
%$Id$
\begin{metainfo}
  \name{
    \lang{de}{A05: Lineare Abhängigkeit}
    \lang{en}{input numbers}
  }
  \begin{description} 
 This work is licensed under the Creative Commons License Attribution 4.0 International (CC-BY 4.0)   
 https://creativecommons.org/licenses/by/4.0/legalcode 

    \lang{de}{Die Beschreibung}
    \lang{en}{}
  \end{description}
  \corrector{system/problem/GenericCorrector.meta.xml}
  \begin{components}
    \component{js_lib}{system/problem/GenericMathlet.meta.xml}{mathlet}
  \end{components}
  \begin{links}
  \end{links}
  \creategeneric
\end{metainfo}
\begin{content}
\usepackage{mumie.genericproblem}
\usepackage{mumie.ombplus}

\lang{de}{\title{A05: Lineare Abhängigkeit}}
\lang{en}{\title{Problem 5}}


\begin{block}[annotation]
	Im Ticket-System: \href{http://team.mumie.net/issues/9510}{Ticket 9510}
\end{block}

\begin{problem}


	\begin{question}
 		\lang{de}{
			\text{Ist die folgende Menge im $\R^3$ linear abhängig?\\
			$ \left\{ \begin{pmatrix} \var{a1} \\ \var{a2} \\ \var{a3}\end{pmatrix},
			\begin{pmatrix} \var{b1} \\ \var{b2} \\ \var{b3}\end{pmatrix},
			\begin{pmatrix} \var{c1} \\ \var{c2} \\ \var{c3}\end{pmatrix} \right\} $}

% Erklärung abhängig vom Wert der Steuerungsvariablen \var{w}:
%
            \explanation[w=0]{ 
            $ \lambda_1 \cdot \begin{pmatrix} \var{a1} \\ \var{a2} \\ \var{a3}\end{pmatrix}
             + \lambda_2 \cdot \begin{pmatrix} \var{b1} \\ \var{b2} \\ \var{b3}\end{pmatrix}
             + \lambda_3 \cdot \begin{pmatrix} \var{c1} \\ \var{c2} \\ \var{c3}\end{pmatrix} 
             = \begin{pmatrix} 0 \\ 0 \\ 0 \end{pmatrix} $            
            \\ hat zum Beispiel die L\"osung 
            $\lambda_1=\var{mu}$, $\lambda_2=\var{lambda}$ und $\lambda_3=-1.$  % falls w=0
            Die drei Vektoren sind also linear abhängig.
           }
            \explanation[w=1]{ 
            $ \lambda_1 \cdot \begin{pmatrix} \var{a1} \\ \var{a2} \\ \var{a3}\end{pmatrix}
             + \lambda_2 \cdot \begin{pmatrix} \var{b1} \\ \var{b2} \\ \var{b3}\end{pmatrix}
             + \lambda_3 \cdot \begin{pmatrix} \var{c1} \\ \var{c2} \\ \var{c3}\end{pmatrix} 
             = \begin{pmatrix} 0 \\ 0 \\ 0 \end{pmatrix} $            
            \\ hat nur die L\"osung 
            $\lambda_1=\lambda_2=\lambda_3 = 0. $                               % falls w=1
            Die drei Vektoren sind also linear unabhängig.
           }
 		}
		
		\field{rational}
		
		
		\begin{variables}
			\randint{a1}{-5}{5}
			\randint{a2}{-6}{6}
			\randint{a3}{-6}{6}
			\randint[Z]{b1}{-5}{5}
			\randint{b2}{-6}{6}
			\randint{b3}{-6}{6}
			\randadjustIf{a1,a2,a3}{a1=0 and a2=0 and a3=0}
			\randadjustIf{b1,b2,b3}{b1=0 and b2=0 and b3=0}
			
			\randint{mu}{-2}{2}
			\randint[Z]{lambda}{-2}{2}
            \randint{w}{0}{1}
            \randint{h}{1}{6}
			\function[calculate]{c1}{mu*a1+lambda*b1+h*w}
			\function[calculate]{c2}{mu*a2+lambda*b2}
			\function[calculate]{c3}{mu*a3+lambda*b3}
            
            \function[calculate]{det}{a1*b2*c3+a2*b3*c1+a3*b1*c2-a3*b2*c1-a2*b1*c3-a1*b3*c2}
					
			
		\end{variables}

		\type{mc.yesno}

        \begin{choice}
        \text{}
        \solution{compute}
        \iscorrect{det}{=}{0}
    \end{choice}
    

	\end{question}
	

	
	
\end{problem}

\embedmathlet{mathlet}

\end{content}