\documentclass{mumie.problem.gwtmathlet}
%$Id$
\begin{metainfo}
  \name{
    \lang{de}{A06: Erzeugendensysteme}
    \lang{en}{input numbers}
  }
  \begin{description} 
 This work is licensed under the Creative Commons License Attribution 4.0 International (CC-BY 4.0)   
 https://creativecommons.org/licenses/by/4.0/legalcode 

    \lang{de}{Die Beschreibung}
    \lang{en}{}
  \end{description}
  \corrector{system/problem/GenericCorrector.meta.xml}
  \begin{components}
    \component{js_lib}{system/problem/GenericMathlet.meta.xml}{mathlet}
  \end{components}
  \begin{links}
  \end{links}
  \creategeneric
\end{metainfo}
\begin{content}

\usepackage{mumie.genericproblem}
\usepackage{mumie.ombplus}

\lang{de}{\title{A06: Erzeugendensysteme}}
\lang{en}{\title{Problem 6}}


\begin{block}[annotation]
	Im Ticket-System: \href{http://team.mumie.net/issues/9521}{Ticket 9521}
\end{block}


\begin{problem}
\randomquestionpool{1}{2}  
%
%   Fall 1 liefert 3 linear abhängige Vektoren, die also kein Erzeugendensystem des R^3 bilden
%
	\begin{question}
		\lang{de}{
			\text{Bildet die folgende Menge ein Erzeugendensystem im $\R^3$ ?\\
			$ \left\{ \begin{pmatrix} \var{a1} \\\var{a2} \\ \var{a3}\end{pmatrix},
			\begin{pmatrix} \var{b1} \\\var{b2} \\ \var{b3}\end{pmatrix},
			\begin{pmatrix} \var{c1} \\\var{c2} \\ \var{c3}\end{pmatrix} \right\} $}
            \explanation{Die 3 Vektoren sind linear abhängig, können also kein Erzeugendensystem des $\R^3$ bilden.}
		}
		\lang{en}{}
		
		\type{mc.yesno}
		\field{rational}
		
    	\precision[false]{3}
    	\displayprecision{3}
    	\correctorprecision{3}
		
		\begin{variables}
			\randint{a1}{-20}{20}
			\randint{a2}{-20}{20}
			\randint{a3}{-20}{20}
			
			\randint{b1}{-20}{20}
			\randint{b2}{-20}{20}
			\randint{b3}{-20}{20}
			
 			\randint[Z]{kmu}{-3}{3}                         
			\randint{klambda}{-3}{3}
			\randint{k}{1}{2}
			\function[calculate]{mu}{kmu/k}                     
			\function[calculate]{lambda}{klambda/k}
%
%       Zur Vermeidung von Dezimalzahlen bzw. Brüchen vereinfacht:
%
% 			\randint[Z]{mu}{-3}{3}                         
% 			\randint{lambda}{-3}{3}

			\function[calculate]{c1}{mu*a1+lambda*b1}           % Der Vektor (c_1, c_2, c_3) ist eine Linearkombination aus 
			\function[calculate]{c2}{mu*a2+lambda*b2}           % (a_1, a_2, a_3) und (b_1, b_2, b_3), folglich sind die drei
			\function[calculate]{c3}{mu*a3+lambda*b3}           % Vektoren linear abhängig und bilden kein Erzeugendensystem.
                                                          
			\randadjustIf{a1,a2,a3}{a1=0 and a2=0 and a3=0}     % Ausschluss, dass (a_1, a_2, a_3) Nullvektor ist
			\randadjustIf{b1,b2,b3}{b1=0 and b2=0 and b3=0}     % Ausschluss, dass (b_1, b_2, b_3) Nullvektor ist
 			
		\end{variables}
		
		\begin{choice}
		   % Test:        
 		   % \text{mu $=\var{mu}$ und lambda $=\var{lambda}$}
            \text{}
			\solution{false}
		\end{choice}

	\end{question}
	
	

%
%   Fall 2 liefert 3 linear unabhängige Vektoren, die somit ein Erzeugendensystem des R^3 bilden
%	
\begin{question}
		\lang{de}{
			\text{Bildet die folgende Menge ein Erzeugendensystem im $\R^3$ ?\\
			$ \left\{ \begin{pmatrix} \var{x1} \\\var{x2} \\ \var{x3}\end{pmatrix},
			\begin{pmatrix} \var{y1} \\\var{y2} \\ \var{y3}\end{pmatrix},
			\begin{pmatrix} \var{z1} \\\var{z2} \\ \var{z3}\end{pmatrix} \right\} $}
			\explanation{Die 3 Vektoren sind linear unabhängig und bilden somit ein Erzeugendensystem des $\R^3.$}
		}
		\lang{en}{}
		
		\type{mc.yesno}
		\field{real}
		
    	\precision[false]{3}
    	\displayprecision{3}
    	\correctorprecision{3}
		
		\begin{variables}
			\randint{x1}{-15}{15}
			\randint{x2}{-15}{15}
			\randint{x3}{-15}{15}
			\randint{y1}{-15}{15}
			\randint{y2}{-15}{15}
			\randint{y3}{-15}{15}
			\randint{z1}{-15}{15}
			\randint{z2}{-15}{15}
			\randint{z3}{-15}{15}
            
           % Test:
           % \function[calculate]{det}{x1*y2*z3-x1*y3*z2-x2*y1*z3+x2*y3*z1+x3*y1*z2-x3*y2*z1} 
            
			\randadjustIf{x1,x2,x3,y1,y2,y3,z1,z2,z3}{x1*y2*z3-x1*y3*z2-x2*y1*z3+x2*y3*z1+x3*y1*z2-x3*y2*z1=0}
            
                                                         % durch die randadjustIf-Abfrage wird Det = 0 ausgeschlossen. 
                                                         % Damit ist sichergestellt, dass die drei erzeugten Vektoren 
                                                         % linear unabhängig sind. Sie bilden also ein Erzeugendensystem.     
		\end{variables}         

		\begin{choice} 
           % Test:        
 		   % \text{Determinante $=\var{det}$}  
             \text{}
			\solution{true}          
		\end{choice}
	\end{question}
	
	
	
	
	
\end{problem}

\embedmathlet{mathlet}


\end{content}