\documentclass{mumie.problem.gwtmathlet}
%$Id$
\begin{metainfo}
  \name{
    \lang{de}{A01: Vektoren/Pfeilklassen}
    \lang{en}{problem_1}
  }
  \begin{description} 
 This work is licensed under the Creative Commons License Attribution 4.0 International (CC-BY 4.0)   
 https://creativecommons.org/licenses/by/4.0/legalcode 

    \lang{de}{Vergleichen von Zahlen und einfache Addition}
    \lang{en}{}
  \end{description}
  \corrector{system/problem/GenericCorrector.meta.xml}
 \begin{components}
    \component{generic_image}{content/rwth/HM1/images/g_tkz_T108_Problem01_C.meta.xml}{T108_Problem01_C}
    \component{generic_image}{content/rwth/HM1/images/g_tkz_T108_Problem01_B.meta.xml}{T108_Problem01_B}
    \component{generic_image}{content/rwth/HM1/images/g_tkz_T108_Problem01_A.meta.xml}{T108_Problem01_A}
    \component{js_lib}{system/problem/GenericMathlet.meta.xml}{mathlet}
  \end{components}
  \begin{links}
  \end{links}
  \creategeneric
\end{metainfo}
\begin{content}
\usepackage{mumie.ombplus}
\usepackage{mumie.genericproblem}


\lang{de}{
	\title{A01: Vektoren/Pfeilklassen}
}
\lang{en}{
	\title{Problem 1}
}

\begin{block}[annotation]
  Im Ticket-System: \href{http://team.mumie.net/issues/9465}{Ticket 9465}
\end{block}
\begin{problem}

\randomquestionpool{1}{3}

%Question 1 of 3	
\begin{question}
	\lang{de}{ 
      	\text{Wie viele verschiedene Vektoren sind hier abgebildet?\\  	
      	\begin{figure}
        \image{T108_Problem01_A}
        \end{figure}}
    	\explanation{Pfeile gleicher Länge und Richtung beschreiben den gleichen Vektor.}
	}
    \lang{en}{
    	\text{How many different vectors are depicted here?\\
    	\begin{figure}
        \image{T108_Problem01_A}
        \end{figure}}
    	\explanation{Arrows of the same length and direction describe the same vector.}
    }
	\begin{variables}
		\number{anzahl}{4}
	\end{variables}
	\type{input.number}
	\begin{answer}
  		\lang{de}{\text{Anzahl Vektoren: }}
  		\lang{en}{\text{Number of Vectors:}}
		\solution{anzahl}
	\end{answer}
\end{question}

%Question 2 of 3
\begin{question}
	\lang{de}{ 
      	\text{Wie viele verschiedene Vektoren sind hier abgebildet?\\  	
      	\begin{figure}
        \image{T108_Problem01_B}
        \end{figure}}
    	\explanation{Pfeile gleicher Länge und Richtung beschreiben den gleichen Vektor.}
    }
    \lang{en}{
    	\text{How many different vectors are depicted here?\\
    	\begin{figure}
        \image{T108_Problem01_B}
        \end{figure}}
    	\explanation{Arrows of the same length and direction describe the same vector.}
    }
	\begin{variables}
		\number{anzahl}{5}
	\end{variables}
	\type{input.number}
	\begin{answer}
  		\lang{de}{\text{Anzahl Vektoren: }}
  		\lang{en}{\text{Number of Vectors:}}
		\solution{anzahl}
	\end{answer}
\end{question}

%Question 3 of 3
\begin{question}
	\lang{de}{ 
      	\text{Wie viele verschiedene Vektoren sind hier abgebildet?\\
      	\begin{figure}
        \image{T108_Problem01_C}
        \end{figure}}
    	\explanation{Pfeile gleicher Länge und Richtung beschreiben den gleichen Vektor.}
    }
    \lang{en}{
    	\text{How many different vectors are depicted here?\\
    	\begin{figure}
        \image{T108_Problem01_C}
        \end{figure}}
    	\explanation{Arrows of the same length and direction describe the same vector.}
    }	
    \begin{variables}
		\number{anzahl}{6}
    \end{variables}
    \type{input.number}
	\begin{answer}
  		\lang{de}{\text{Anzahl Vektoren: }}
  		\lang{en}{\text{Number of Vectors:}}
		\solution{anzahl}
	\end{answer}
\end{question}

\end{problem}

\embedmathlet{mathlet}
\end{content}