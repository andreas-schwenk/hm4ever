\documentclass{mumie.problem.gwtmathlet}
%$Id$
\begin{metainfo}
  \name{
    \lang{de}{A07: Basen}
    \lang{en}{input numbers}
  }
  \begin{description} 
 This work is licensed under the Creative Commons License Attribution 4.0 International (CC-BY 4.0)   
 https://creativecommons.org/licenses/by/4.0/legalcode 

    \lang{de}{Die Beschreibung}
    \lang{en}{}
  \end{description}
  \corrector{system/problem/GenericCorrector.meta.xml}
  \begin{components}
    \component{js_lib}{system/problem/GenericMathlet.meta.xml}{mathlet}
  \end{components}
  \begin{links}
  \end{links}
  \creategeneric
\end{metainfo}
\begin{content}
\usepackage{mumie.genericproblem}
\usepackage{mumie.ombplus}

\lang{de}{\title{A07: Basen}}
\lang{en}{\title{Problem 7}}


\begin{block}[annotation]
	Im Ticket-System: \href{http://team.mumie.net/issues/9522}{Ticket 9522}
\end{block}


\begin{problem}
\randomquestionpool{1}{4}
\randomquestionpool{5}{6}

	\begin{question}
		\lang{de}{
			\text{Bildet die folgende Menge eine Basis des $\R^2$ ?\\
			$ \left\{ \begin{pmatrix} \var{a1} \\\var{a2}\end{pmatrix} \right\}$}
            \explanation{Eine Basis des $\R^2$ muss 2 Elemente haben.}
		}
		\lang{en}{}
		
		\type{mc.yesno}
		\field{rational}
		
%    	\precision[false]{1}
%    	\displayprecision{1}
%    	\correctorprecision{1}
		
		\begin{variables}
			\randint[Z]{a1}{-30}{30}
			\randint[Z]{a2}{-30}{30}

			
		\end{variables}
		
		\begin{choice}
			\text{}
			\solution{false}
		\end{choice}

	\end{question}
	
	
	

    \begin{question}
		\lang{de}{
			\text{Bildet die folgende Menge eine Basis des $\R^2$ ?\\
			$ \left\{ \begin{pmatrix} \var{b1} \\\var{b2}\end{pmatrix},
			\begin{pmatrix} \var{c1} \\\var{c2} \end{pmatrix},
			\begin{pmatrix} \var{d1} \\\var{d2} \end{pmatrix} \right\} $}
			\explanation{Drei Vektoren im $\R^2$ können nicht linear unabhängig sein.}
		}
		\lang{en}{}
		
		\type{mc.yesno}
		\field{rational}
		
%    	\precision[false]{1}    % Die Variable wird mit 1 Nachkommastelle definiert
%    	\displayprecision{1}    % Die Variable wird mit 1 Nachkommastelle angezeigt
%    	\correctorprecision{1}  % Die Variable wird mit 1 Nachkommastelle korrigiert
		
		\begin{variables}
			\randint[Z]{b1}{-20}{20}
			\randint{b2}{-20}{20}
			
			\randint{k}{1}{2}
			\randint{klambda}{-3}{3}
			\randint{kmu}{-3}{3}
			\randadjustIf{klambda,kmu}{klambda=kmu}
			\randint{l}{0}{1}
			
			\function{lambda}{klambda/k+l}          
			\function{mu}{kmu/k+l}                  
			
			\function[normalize]{c1}{lambda*b1}    
			\function[normalize]{c2}{lambda*b2}
    
			\function[normalize]{d1}{mu*b1}
			\function[normalize]{d2}{mu*b2}			

		\end{variables}

		\begin{choice}
			\text{}
			\solution{false}
		\end{choice}
	\end{question}
	
	\begin{question}
		\lang{de}{
			\text{Bildet die folgende Menge eine Basis des $\R^2$ ?\\
			$ \left\{ \begin{pmatrix} \var{m1} \\\var{m2} \end{pmatrix},
			\begin{pmatrix} \var{n1} \\\var{n2} \end{pmatrix} \right\} $}
			\explanation{Die beiden Vektoren sind linear abhängig.}
		}
		\lang{en}{}
		
		\type{mc.yesno}
		\field{rational}
		
%    	\precision[false]{3}
%    	\displayprecision{3}
%    	\correctorprecision{3}
		
		\begin{variables}
			\randint[Z]{m1}{-25}{25}
			\randint{m2}{-25}{25}
			
			\randint{alpha}{-3}{3}                
			
			\function[normalize]{n1}{alpha*m1}    
			\function[normalize]{n2}{alpha*m2}

		
		\end{variables}

		\begin{choice}
			\text{}
			\solution{false}
		\end{choice}
	\end{question}

	\begin{question}
		\lang{de}{
			\text{Bildet die folgende Menge eine Basis des $\R^2$ ?\\
			$ \left\{ \begin{pmatrix} \var{e1} \\\var{e2} \end{pmatrix},
			\begin{pmatrix} \var{f1} \\\var{f2} \end{pmatrix} \right\} $}
			\explanation{Die beiden Vektoren sind linear unabhängig.}
		}
		\lang{en}{}
		
		\type{mc.yesno}
		\field{real}
		
    	\precision[false]{3}
    	\displayprecision{3}
    	\correctorprecision{3}
		
		\begin{variables}
			\randint{e1}{-25}{25}
			\randint{e2}{-25}{25}
			
			\randint{f1}{-25}{25}
			\randint{f2}{-25}{25}
			
			\randadjustIf{e1,e2,f1,f2}{e1*f2-e2*f1=0}
			
		
		\end{variables}

		\begin{choice}
			\text{}
			\solution{true}
		\end{choice}
	\end{question}
	
	\begin{question}
		\lang{de}{
			\text{Bildet die folgende Menge eine Basis des $\R^3$ ?\\
			$ \left\{ \begin{pmatrix} \var{g1} \\\var{g2} \\\var{g3} \end{pmatrix},
            \begin{pmatrix} \var{h1} \\\var{h2} \\\var{h3} \end{pmatrix},
			\begin{pmatrix} \var{i1} \\\var{i2} \\\var{i3} \end{pmatrix} \right\} $}
			\explanation{Die drei Vektoren sind linear unabhängig.}
		}
		\lang{en}{}
		
		\type{mc.yesno}
		\field{real}
		
    	\precision[false]{3}
    	\displayprecision{3}
    	\correctorprecision{3}
		
		\begin{variables}
			\randint{g1}{-15}{15}
			\randint{g2}{-15}{15}
			\randint{g3}{-15}{15}
			\randint{h1}{-15}{15}
			\randint{h2}{-15}{15}
			\randint{h3}{-15}{15}
			\randint{i1}{-15}{15}
			\randint{i2}{-15}{15}
			\randint{i3}{-15}{15}
            
           % Test:
           % \function[calculate]{det}{g1*h2*i3-g1*h3*i2-g2*h1*i3+g2*h3*i1+g3*h1*i2-g3*h2*i1} 
            
			\randadjustIf{g1,g2,g3,h1,h2,h3,i1,i2,i3}{g1*h2*i3-g1*h3*i2-g2*h1*i3+g2*h3*i1+g3*h1*i2-g3*h2*i1=0}
            
		\end{variables}

		\begin{choice}
			\text{}
			\solution{true}
		\end{choice}
	\end{question}
		
	\begin{question}
		\lang{de}{
			\text{Bildet die folgende Menge eine Basis des $\R^3$ ?\\
			$ \left\{ \begin{pmatrix} \var{j1} \\\var{j2} \\\var{j3} \end{pmatrix},
            \begin{pmatrix} \var{k1} \\\var{k2} \\\var{k3} \end{pmatrix},
			\begin{pmatrix} \var{l1} \\\var{l2} \\\var{l3} \end{pmatrix} \right\} $}
			\explanation{Die drei Vektoren sind linear abhängig.}
		}
		\lang{en}{}
		
		\type{mc.yesno}
		\field{rational}
		
    	\precision[false]{3}
    	\displayprecision{3}
    	\correctorprecision{3}
		
		\begin{variables}
			\randint{j1}{-20}{20}
			\randint{j2}{-20}{20}
			\randint{j3}{-20}{20}
			
			\randint{k1}{-20}{20}
			\randint{k2}{-20}{20}
			\randint{k3}{-20}{20}
			
 			\randint[Z]{rs}{-3}{3}                         
			\randint{rt}{-3}{3}
			\randint{r}{1}{2}
			\function[calculate]{s}{rs/r}                     
			\function[calculate]{t}{rt/r}

			\function[calculate]{l1}{s*j1+t*k1}           % Der Vektor (l_1, l_2, l_3) ist eine Linearkombination aus 
			\function[calculate]{l2}{s*j2+t*k2}           % (j_1, j_2, j_3) und (k_1, k_2, k_3), folglich sind die drei
			\function[calculate]{l3}{s*j3+t*k3}           % Vektoren linear abhängig und bilden kein Erzeugendensystem.
                                                          
			\randadjustIf{j1,j2,j3}{j1=0 and j2=0 and j3=0}     % Ausschluss, dass (j_1, j_2, j_3) Nullvektor ist
			\randadjustIf{k1,k2,k3}{k1=0 and k2=0 and k3=0}     % Ausschluss, dass (k_1, k_2, k_3) Nullvektor ist
 			
		\end{variables}

		\begin{choice}
			\text{}
			\solution{false}
		\end{choice}
	\end{question}
	
	
	
	
\end{problem}

\embedmathlet{mathlet}


\end{content}