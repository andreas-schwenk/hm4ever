%$Id:  $
\documentclass{mumie.article}
%$Id$
\begin{metainfo}
  \name{
    \lang{de}{Erzeugendensysteme und Basen}
    \lang{en}{Generating sets, linear independence and bases}
  }
  \begin{description} 
 This work is licensed under the Creative Commons License Attribution 4.0 International (CC-BY 4.0)   
 https://creativecommons.org/licenses/by/4.0/legalcode 

    \lang{de}{Beschreibung}
    \lang{en}{Description}
  \end{description}
  \begin{components}
    \component{generic_image}{content/rwth/HM1/images/g_tkz_T108_LinearDependence.meta.xml}{T108_LinearDependence}
    \component{generic_image}{content/rwth/HM1/images/g_img_00_Videobutton_schwarz.meta.xml}{00_Videobutton_schwarz}
    \component{js_lib}{system/media/mathlets/GWTGenericVisualization.meta.xml}{mathlet1}
  \end{components}
  \begin{links}
    \link{generic_article}{content/rwth/HM1/T108_Vektorrechnung/g_art_content_30_basen_eigenschaften.meta.xml}{Erzeugendensystem}
% alt  \link{generic_article}{content/rwth/HM1/T111_Matrizen,_lineare_Gleichungssysteme/g_art_content_40_lineare_gleichungssysteme.meta.xml}{lgs}
% content_40_ ... ist in T112neu_ ...
    \link{generic_article}{content/rwth/HM1/T112neu_Lineare_Gleichungssysteme/g_art_content_40_lineare_gleichungssysteme.meta.xml}{lgs}
    \link{generic_article}{content/rwth/HM1/T108_Vektorrechnung/g_art_content_29_linearkombination.meta.xml}{lin-comb}
  \end{links}
  \creategeneric
\end{metainfo}
\begin{content}
\usepackage{mumie.ombplus}
\ombchapter{8}
\ombarticle{3}
\usepackage{mumie.genericvisualization}

\begin{visualizationwrapper}

\title{\lang{de}{Lineare Unabhängigkeit von Vektoren, Erzeugendensysteme und Basen}
       \lang{en}{Generating sets, linear independence and bases}}
 
\begin{block}[annotation]
  übungsinhalt
  
\end{block}
\begin{block}[annotation]
  Im Ticket-System: \href{http://team.mumie.net/issues/9047}{Ticket 9047}\\
\end{block}

\begin{block}[info-box]
\tableofcontents
\end{block}

\lang{de}{
Im Zusammenhang mit \ref[lin-comb][Linearkombinationen]{sec:lin-comb}
ergeben sich drei wichtige Begriffe:
Erzeugendensysteme, lineare Unabhängigkeit und Basen.
}
\lang{en}{
There are several important definitions that arise from the definition of a 
\ref[lin-comb][linear combination]{sec:lin-comb} of vectors: generating sets, linear independence 
and bases.
}

\section{\lang{de}{Erzeugendensysteme}\lang{en}{Generating sets}}  \label{def_ES}

%
%%% Video Hoever
%
\lang{de}{
In den \ref[lin-comb][Beispielen zu Linearkombinationen]{ex:linearkombinationen} haben wir festgestellt, dass
aus geeigneten fest vorgegebenen Vektoren des $\R^3$ alle weiteren Vektoren aus dem $\R^2$ als Linearkombination
darstellen lassen. Dies führt uns zu dem Begriff des \emph{Erzeugendensystems.}
\\
\floatright{\href{https://www.hm-kompakt.de/video?watch=709}{\image[75]{00_Videobutton_schwarz}}}\\\\
}
\lang{en}{
In the \ref[lin-comb][final example]{ex:linearkombinationen} of the previous section we established 
that any vector in $\R^3$ is a linear combination of a specific given set of vectors.
}
%

\begin{definition}[\lang{de}{Erzeugendensystem}\lang{en}{Generating system}]
\lang{de}{
Eine Menge von Vektoren $\{ \vec{v}_1;\ldots;\vec{v}_l \}$ im $\R^n$ hei"st \emph{Erzeugendensystem 
des $\R^n$}, wenn jeder Vektor im $\R^n$ sich als Linearkombination der Vektoren 
$\vec{v}_1,\ldots,\vec{v}_l$ schreiben lässt.\\
}
\lang{en}{
A set of vectors $\{ \vec{v}_1;\ldots;\vec{v}_l \}$ in $\R^n$ is called a \emph{generating set of 
$\R^n$} if every vector in $\R^n$ can be written as a linear combination of the vectors 
$\vec{v}_1,\ldots,\vec{v}_l$.
}
\end{definition}

\lang{de}{
Dabei gilt stets $l \geq n$, wie später in \ref{ES_Basis} (Theorem zu Basen) noch gezeigt wird.
}
\lang{en}{
It is necessary that $l \geq n$, as is shown later in theorem \ref{ES_Basis}.
}

\begin{example}
\begin{tabs*}
\tab{\lang{de}{Standardvektoren}\lang{en}{Standard unit vectors}}
\lang{de}{Die Standardvektoren im $\R^n$}
\lang{en}{The standard unit vectors in $\R^n$}
\[ \vec{e}_1=\begin{pmatrix} 1\\ 0\\ \vdots \\ 0\end{pmatrix},
   \quad \vec{e}_2=\begin{pmatrix} 0\\ 1\\ \vdots \\ 0\end{pmatrix},
     \quad \ldots, \quad \vec{e}_n=\begin{pmatrix} 0\\ \vdots \\ 0\\ 1\end{pmatrix} \]
\lang{de}{
bilden ein Erzeugendensystem des $\R^n$, da jeder Vektor 
$\begin{pmatrix} v_1\\ \vdots \\ v_n\end{pmatrix}$ geschrieben werden kann als
}
\lang{en}{
\emph{generate} $\R^n$, that is, the set of the standard unit vectors is a generating set of $\R^n$, 
as every vector $\begin{pmatrix} v_1\\ \vdots \\ v_n\end{pmatrix}$ can be written as
}
\[ \begin{pmatrix} v_1\\ \vdots \\ v_n\end{pmatrix} = v_1\cdot \vec{e}_1 +v_2\cdot \vec{e}_2 +\ldots +v_n\cdot \vec{e}_n. \]
\lang{de}{
Es lässt sich also jeder Vektor $\vec{v} \in \R^n$ als Linearkombination aus den Standardvektoren 
\emph{erzeugen}.
}
\lang{en}{
Hence every vector $\vec{v} \in \R^n$ is \emph{generated} by the standard unit vectors.
}
\tab{\lang{de}{3 Vektoren im $\R^3$}\lang{en}{Three vectors in $\R^3$}}
\lang{de}{Die Vektoren}
\lang{en}{The vectors}
\[
\vec{v}_1 \, = \, \begin{pmatrix}
1 \\ 2 \\ 0
\end{pmatrix}, 
\vec{v}_2 \, = \, \begin{pmatrix}
2 \\ 3 \\ 0
\end{pmatrix}, 
\vec{v}_3 \, = \, \begin{pmatrix}
3 \\ 4 \\ 0
\end{pmatrix}
\]
\lang{de}{
bilden kein Erzeugendensystem des $\R^3$, da bei allen die 3. Komponente Null ist.
\\
Demzufolge lässt sich zum Beispiel der Vektor
}
\lang{en}{
do not generate $\R^3$, as their third component is always zero.
\\
Hence, for example the vector
}
$ \begin{pmatrix}
0 \\ 0 \\ 1 \end{pmatrix} $
\lang{de}{
nicht aus $\vec{v}_1$, $\vec{v}_2$ und $\vec{v}_3$ erzeugen, denn es gibt keine
rellen Zahlen $r_1,r_2,r_3$, für die gilt: 
}
\lang{en}{
cannot be written as a linear combination of $\vec{v}_1$, $\vec{v}_2$ and $\vec{v}_3$, as there are 
no real numbers $r_1,r_2,r_3$ such that:
}
\[1=r_1 \cdot 0 + r_2 \cdot 0 + r_3 \cdot 0. \]

\lang{de}{Folglich gibt es auch keine Linearkombination}
\lang{en}{Thus there are no $r_1,r_2,r_3$ such that:}
 
\[ \begin{pmatrix}
0 \\ 0 \\ 1 \end{pmatrix}= r_1 \cdot \begin{pmatrix}1 \\ 2 \\ 0\end{pmatrix}
+ r_2\cdot \begin{pmatrix}2 \\ 3 \\ 0\end{pmatrix}
+ r_3 \cdot \begin{pmatrix}3 \\ 4 \\ 0\end{pmatrix}. \]

 
\tab{\lang{de}{3 Vektoren im $\R^2$}\lang{en}{Three vectors in $\R^2$}}
\lang{de}{
Die Vektoren $\begin{pmatrix} 2\\ -1\end{pmatrix}$, $\begin{pmatrix} -1\\ 3\end{pmatrix}$
und $\begin{pmatrix} 1\\ 0\end{pmatrix}$ bilden ein Erzeugendensystem des $\R^2$, denn jeder Vektor 
$\begin{pmatrix} v_1\\ v_2\end{pmatrix}$ lässt sich darstellen als
}
\lang{en}{
The vectors $\begin{pmatrix} 2\\ -1\end{pmatrix}$, $\begin{pmatrix} -1\\ 3\end{pmatrix}$ and 
$\begin{pmatrix} 1\\ 0\end{pmatrix}$ generate $\R^2$, as every vector 
$\begin{pmatrix} v_1\\ v_2\end{pmatrix}$ can be written as
}
\[ \begin{pmatrix} v_1\\ v_2\end{pmatrix} = -v_2\cdot \begin{pmatrix} 2\\ -1\end{pmatrix}
+ 0 \cdot \begin{pmatrix} -1\\ 3\end{pmatrix}
+ (v_1+2v_2) \cdot \begin{pmatrix} 1\\ 0\end{pmatrix}. \]
\end{tabs*}
\end{example}

\begin{remark}
\lang{de}{
Um im Allgemeinen zu entscheiden, ob eine gegebene Menge von Vektoren $\{ \vec{v}_1;\ldots;\vec{v}_l \}$ im $\R^n$
ein Erzeugendensystem ist, muss für jeden beliebigen Vektor $\vec{w}$ untersucht werden, ob die Gleichung 
\[ r_1\vec{v}_1+\ldots + r_l\vec{v}_l =\vec{w} \]
Lösungen $r_1, \ldots,r_l \in \R$ hat. \\
Ein allgemeines Verfahren, solche linearen Gleichungssysteme (hier mit $n$ Gleichungen 
und $l$ Variablen $r_1,\ldots, r_l$) zu lösen, wird im Abschnitt \link{lgs}{Lineare Gleichungssysteme} erklärt.\\
Im $\R^2$ oder $\R^3$ lassen sich Lösungen oft auch \glqq{}direkt\grqq{} finden.
}
\lang{en}{
In general, to decide whether a given set of vectors $\{ \vec{v}_1;\ldots;\vec{v}_l \}$ in $\R^n$ 
is a generating set, we must check whether for any vector $\vec{w}$ the equation 
\[ r_1\vec{v}_1+\ldots + r_l\vec{v}_l =\vec{w} \]
has solutions $r_1, \ldots,r_l \in \R$. \\
A general strategy for solving such systems of equations (here containing $n$ equations, one per 
component, and $l$ variables $r_1,\ldots, r_l$) is described in the section on \link{lgs}{systems 
of linear equations}.\\
In $\R^2$ or $\R^3$ we can often 'directly' find solutions.
}
\end{remark}

\begin{example}
\lang{de}{Wir untersuchen, ob die Vektoren}
\lang{en}{We will determine whether the vectors}
$
\vec{v}_1 \, = \, \begin{pmatrix}
1 \\ 2 \\ 3
\end{pmatrix}, 
\vec{v}_2 \, = \, \begin{pmatrix}
2 \\ 3 \\ 5       
\end{pmatrix}, 
\vec{v}_3 \, = \, \begin{pmatrix}
3 \\ 4 \\ 7
\end{pmatrix}
$
\lang{de}{
ein Erzeugenensystem des $\R^3$ bilden.
\\
Würde ${ \vec{v}_1,\vec{v}_2,\vec{v}_3 }$ ein Erzeugendensystem des $\R^3$ bilden, 
so m\"usste sich zum Beispiel der Vektor $\vec{w} = \begin{pmatrix} 0 \\ 0 \\ 1 \end{pmatrix} 
\in \R^3$ als als Linearkombination aus $\vec{v}_1$, $\vec{v}_2$ und $\vec{v}_3$ darstellen lassen, 
d.h. es g\"abe reelle Zahlen $r_1,r_2,r_3$, sodass gilt:
}
\lang{en}{
form a generating set of $\R^3$.
\\
If ${ \vec{v}_1,\vec{v}_2,\vec{v}_3 }$ were a generating set of $\R^3$, then for example the vector 
$\vec{w} = \begin{pmatrix} 0 \\ 0 \\ 1 \end{pmatrix} \in \R^3$ could be written as a linear 
combination of $\vec{v}_1$, $\vec{v}_2$ and $\vec{v}_3$, that is, there would exist real numbers 
$r_1,r_2,r_3$ such that:
}
 
\[ \begin{pmatrix}
  0 \\ 0 \\ 1 \end{pmatrix}= r_1 \cdot \begin{pmatrix}1 \\ 2 \\ 3\end{pmatrix}
  + r_2\cdot \begin{pmatrix}2 \\ 3 \\ 5\end{pmatrix}
  + r_3 \cdot \begin{pmatrix}3 \\ 4 \\ 7\end{pmatrix}.
\] 

\lang{de}{
Lösen wir diese Gleichung nun komponentenweise auf, dann gilt für die ersten beiden Komponeten:
}
\lang{en}{
Let us solve this equation componentwise. The first two rows read:
}
\[ r_1+2r_2+3r_3=0 \]
\lang{de}{und}
\lang{en}{and}
\[2r_1+3r_2+4r_3=0 \] 
\lang{de}{folglich auch}
\lang{en}{and added together give}

\[ (1+2) \cdot r_1+(2+3) \cdot r_2+(3+4) \cdot r_3=3r_1+5r_2+7r_3=0 \]

\lang{de}{Dies widerspricht jedoch der Gleichung der 3. Komponente:}
\lang{en}{This contradicts the equation in the third components:}

\[3r_1+5r_2+7r_3=1.\]

\lang{de}{
Der Vektor $\vec{w}$ lässt sich also nicht als Linearkombination der Vektoren 
$\vec{v}_1$, $\vec{v}_2$ und $\vec{v}_3$ darstellen und somit bilden diese 
auch kein Erzeugendensystem des $\R^3$.
}
\lang{en}{
Therefore the vector $\vec{w}$ cannot be written as a linear combination of the vectors $\vec{v}_1$, 
$\vec{v}_2$ and $\vec{v}_3$, and these do not form a generating set of $\R^3$.
}

\end{example}

 
\begin{quickcheck}
		\field{rational}
		\type{input.number}
		\begin{variables}
			\randint{k}{1}{2}	% Zufallsvariable zum Vertauschen:
			\function[calculate]{k1}{2-k}  % "Dirac"-funktionen
			\function[calculate]{k2}{k-1}
			\function[calculate]{nk}{3-k}  % gerade die andere Option


			\randint[Z]{r}{1}{4}
			\randint[Z]{s}{2}{4}
			\randint[Z]{t}{2}{4}
			\randint[Z]{a1}{-5}{5}
			\randint[Z]{a2}{-5}{5}
			\randint[Z]{c1}{-5}{5}
			\randint[Z]{c2}{-5}{5}

			\function[calculate]{b1}{-k2*r*a1}
			\function[calculate]{b2}{-r*a2}
			\function[calculate]{f1}{t*a1}
			\function[calculate]{f2}{t*a2}
			\function[calculate]{d1}{k1*s*c1}
			\function[calculate]{d2}{s*c2}
			
			\function[calculate]{v}{k2*a2/a1+k1*c2/c1}
		\end{variables}
		
			\text{\lang{de}{
            Welche der folgenden Mengen von Vektoren bildet ein Erzeugendensystem des $\R^2$?}
            }
            \lang{en}{
            Which of the following sets of vectors is a generating set of $\R^2$?
            }\\

            \begin{choices}{multiple}

                \begin{choice}
                  \text{$\{ \begin{pmatrix} \var{a1} \\ \var{a2} \end{pmatrix}; 
                        \begin{pmatrix} \var{b1} \\ \var{b2} \end{pmatrix}; 
                        \begin{pmatrix} \var{f1} \\ \var{f2} \end{pmatrix} \}$}
                  \solution[compute]{k=1}
                \end{choice}
                \begin{choice}
                  \text{$\{ \begin{pmatrix} \var{c1} \\ \var{c2} \end{pmatrix}; 
				        \begin{pmatrix} \var{d1}\\ \var{d2} \end{pmatrix} \}.$}
                  \solution[compute]{k=2}
                \end{choice}

             \end{choices}


		\explanation{\lang{de}{
    Als Linearkombination der Vektoren der $\var{nk}$. Menge lassen sich nur die Vektoren darstellen,
		bei denen die zweite Komponente, das $\var{v}$-fache der ersten Komponente ist.
    }
    \lang{en}{
    Set $\var{nk}$ only generates those vectors whose second component is $\var{v}$-times the first 
    component.
    }}
     
\end{quickcheck}

%%%%% alte Version %%%%%%%%%%%%%%%%%%%%%%%%%%%%%%%%%%%%%%%%%%%%%%%%%%%%%%%%%%%%%%%%%%%%%%%%%%%%%%%%%%%%%%%%%%%%%%%%
%
% \begin{quickcheck}
%		\field{rational}
%		\type{input.number}
%		\begin{variables}
%			\randint{k}{1}{2}	% Zufallsvariable zum Vertauschen:
%			\function[calculate]{k1}{2-k}  % "Dirac"-funktionen
%			\function[calculate]{k2}{k-1}
%			\function[calculate]{nk}{3-k}  % gerade die andere Option
%
%
%			\randint[Z]{r}{1}{4}
%			\randint[Z]{s}{2}{4}
%			\randint[Z]{t}{2}{4}
%			\randint[Z]{a1}{-5}{5}
%			\randint[Z]{a2}{-5}{5}
%			\randint[Z]{c1}{-5}{5}
%			\randint[Z]{c2}{-5}{5}
%
%			\function[calculate]{b1}{-k2*r*a1}
%			\function[calculate]{b2}{-r*a2}
%			\function[calculate]{f1}{t*a1}
%			\function[calculate]{f2}{t*a2}
%			\function[calculate]{d1}{k1*s*c1}
%			\function[calculate]{d2}{s*c2}
%			
%			\function[calculate]{v}{k2*a2/a1+k1*c2/c1}
%		\end{variables}
%		
%			\text{Welche der folgenden Mengen von Vektoren bildet ein Erzeugendensystem des $\R^2$?\\
%               
%				1) $\{ \begin{pmatrix} \var{a1} \\ \var{a2} \end{pmatrix}; 
%				\begin{pmatrix} \var{b1} \\ \var{b2} \end{pmatrix}; 
%				\begin{pmatrix} \var{f1} \\ \var{f2} \end{pmatrix} \}$ \\
%				2) $\{ \begin{pmatrix} \var{c1} \\ \var{c2} \end{pmatrix}; 
%				\begin{pmatrix} \var{d1}\\ \var{d2} \end{pmatrix} \}$.\\
%				Die Menge mit der Nummer \ansref.}
%		
%		\begin{answer}
%			\solution{k}
%		\end{answer}
%		\explanation{Als Linearkombination der Vektoren der Menge $\var{nk}$ lassen sich nur die Vektoren darstellen,
%		bei denen die zweite Komponente, das $\var{v}$-fache der ersten Komponente ist.}
%    
%\end{quickcheck}
%
%%%%%%%%%%%%%%%%%%%%%%%%%%%%%%%%%%%%%%%%%%%%%%%%%%%%%%%%%%%%%%%%%%%%%%%%%%%%%%%%%%%%%%%%%%%%%%%%%%%%%%%%%%%%%%%%%%%%%%

\section{\lang{de}{Lineare Unabhängigkeit}\lang{en}{Linear independence}} \label{def_lin_unabh}

%
%%% Video Hoever
%
\lang{de}{
Wir werden nun die im vorherigen Abschnitt eingeführten Erzeugendensystem genauer betrachten, um sie hinsichtlich
der erforderlichen Anzahl von Vektoren zu optimieren. Dies führt uns zu den Begriffen der \emph{Linearen Abhängigkeit}
und der \emph{Linearen Unabhängigkeit}.
\\
\floatright{\href{https://www.hm-kompakt.de/video?watch=710}{\image[75]{00_Videobutton_schwarz}}}\\\\
}
%
\lang{en}{
Now we will define \emph{linear dependence} and \emph{linear independence} in order to optimise the 
number of vectors needed in a generating set.
}

\begin{definition}[\lang{de}{Lineare (Un-)Abhängigkeit}\lang{en}{Linear (in-)dependence}]
\lang{de}{
Ein Vektor $\vec{v}$ hei"st \emph{linear abh"angig von Vektoren} $\vec{w}_1,\ldots,\vec{w}_k$, wenn er als Linearkombination
der Vektoren $\vec{w}_1,\ldots,\vec{w}_k$ dargestellt werden kann. 
Andernfalls hei"st $\vec{v}$ \emph{linear unabh"angig von} $\vec{w}_1,\ldots,\vec{w}_k$.
\\\\
Eine Menge von Vektoren $\{ \vec{v}_1;\ldots;\vec{v}_l\}$ hei"st \emph{linear abh"angig}, wenn mindestens einer der Vektoren
der Menge von den anderen linear abh"angig ist, und sie hei"st \emph{linear unabh"angig}, wenn keiner der Vektoren von 
den anderen linear abh"angig ist.
}
\lang{en}{
A vector $\vec{v}$ is called \emph{linearly dependent} on the vectors $\vec{w}_1,\ldots,\vec{w}_k$ 
if it can be written as a linear combination of $\vec{w}_1,\ldots,\vec{w}_k$. 
Otherwise $\vec{v}$ is called \emph{linearly independent} from $\vec{w}_1,\ldots,\vec{w}_k$.
\\\\
A set of vectors $\{ \vec{v}_1;\ldots;\vec{v}_l\}$ is called \emph{linearly dependent} if at least 
one of its vectors is linearly dependent on the others, and \emph{linearly independent} if none of 
its vectors is linearly dependent on the others.
}
\end{definition} 

\begin{quickcheck}
		\field{rational}
		\type{input.number}
		\begin{variables}
			\randint{k}{1}{3}	% Zufallsvariable zum Vertauschen:
			\function[calculate]{k1}{(2-k)*(3-k)/2}  % "Dirac"-funktionen
			\function[calculate]{k2}{(k-1)*(3-k)}
			\function[calculate]{k3}{(k-1)*(k-2)/2}

			\randint[Z]{r}{1}{4}
			\randint[Z]{s}{2}{4}
			\randint[Z]{t}{2}{4}
			\randint[Z]{d1}{-4}{4}
			\randint[Z]{d2}{-4}{4}
			\randint[Z]{c1}{-4}{4}
			\randint[Z]{c2}{-4}{4}

			\function[calculate]{a1}{r*c1}
			\function[calculate]{a2}{s*d1}
			\function[calculate]{a3}{r*c2+s*d2+k1}
			\function[calculate]{b1}{s*c1}
			\function[calculate]{b2}{t*d1}
			\function[calculate]{b3}{s*c2+t*d2-k2}
			\function[calculate]{f1}{t*c1}
			\function[calculate]{f2}{r*d1}
			\function[calculate]{f3}{t*c2+r*d2-2*k3}
			
			\function[calculate]{g1}{k1*a1+k2*b1+k3*f1}
			\function[calculate]{g2}{k1*a2+k2*b2+k3*f2}
			\function[calculate]{g3}{k1*a3+k2*b3+k3*f3}
			\function[calculate]{gg3}{k1*(a3-k1)+k2*(b3+k2)+k3*(f3+2*k3)}
			

		\end{variables}
		
			\text{\lang{de}{
      Welcher der folgenden Vektoren ist von den Vektoren 
      $\vec{w}_1=\begin{pmatrix} \var{c1} \\ 0\\ \var{c2} \end{pmatrix}$ und 
			$\vec{w}_2=\begin{pmatrix} 0\\ \var{d1}\\ \var{d2} \end{pmatrix}$ linear unabhängig?
      }
      \lang{en}{
      Which of the following vectors is linearly independent from the vectors 
      $\vec{w}_1=\begin{pmatrix} \var{c1} \\ 0\\ \var{c2} \end{pmatrix}$ and 
      $\vec{w}_2=\begin{pmatrix} 0\\ \var{d1}\\ \var{d2} \end{pmatrix}$? 
      }}\\
                                 
                \begin{choices}{unique}
               %  1)
                    \begin{choice}
                      \text{$\begin{pmatrix} \var{a1} \\ \var{a2}\\ \var{a3} \end{pmatrix}\quad $}
                      \solution[compute]{k=1}
                    \end{choice}
                    
                %  2)
                    \begin{choice}
                      \text{$\begin{pmatrix} \var{b1} \\ \var{b2}\\ \var{b3} \end{pmatrix}\quad $}
                      \solution[compute]{k=2}
                    \end{choice}
                    
                %  3)                    
                    \begin{choice}
                      \text{$\begin{pmatrix} \var{f1} \\ \var{f2} \\ \var{f3} \end{pmatrix}$}
                      \solution[compute]{k=3}
                    \end{choice}

                 \end{choices}
                
           
            
%%%%% alte Version %%%%%%%%%%%%%%%%%%%%%%%%%%%%%%%%%%%%%%%%%%%%%%%%%%%%%%%%%%%%%%%%%%%%%%%%%%%%%%%%%%%%%%%%%%%%%%%%
%                           
%				1) $\begin{pmatrix} \var{a1} \\ \var{a2}\\ \var{a3} \end{pmatrix}\quad $ 
%				2)  $\begin{pmatrix} \var{b1} \\ \var{b2}\\ \var{b3} \end{pmatrix}\quad $
%				3)  $\begin{pmatrix} \var{f1} \\ \var{f2} \\ \var{f3} \end{pmatrix}$\\
%				Die Vektoren mit den Nummern \ansref. (Geben Sie alle Nummern ohne Leerzeichen ein.)}
%		
%		\begin{answer}
%			\solution{k}
%		\end{answer}
%
%%%%%%%%%%%%%%%%%%%%%%%%%%%%%%%%%%%%%%%%%%%%%%%%%%%%%%%%%%%%%%%%%%%%%%%%%%%%%%%%%%%%%%%%%%%%%%%%%%%%%%%%%%%%%%%%%%
        
		\explanation{\lang{de}{
    Ein Vektor ist genau dann linear abhängig von $\,\vec{w}_1$ und $\vec{w}_2$, wenn man ihn als 
    Linearkombination dieser beiden Vektoren darstellen kann. Ist dies nicht möglich, ist er linear 
    unabhängig von $\,\vec{w}_1$ und $\vec{w}_2$.\\
		Versucht man also die Vektoren als $\, r\cdot \vec{w}_1+s\cdot \vec{w}_2$ darzustellen, erhält 
    man aus den ersten beiden	Komponenten die folgenden (einzig möglichen) Werte für $r$ und $s:$
    \begin{itemize}
        \item[im 1. Fall] $r=\var{r}, s=\var{s},$
        \item[im 2. Fall] $r=\var{s}, s=\var{t} \;$ und
    		\item[im 3. Fall] $r=\var{t}, s=\var{r}.$ 
    \end{itemize}    
    Im $\var{k}$-ten Fall liefert diese Linearkombination dann aber der Vektor 
		$\begin{pmatrix} \var{g1} \\ \var{g2}\\ \var{gg3} \end{pmatrix}$, der ungleich  
		$\begin{pmatrix} \var{g1} \\ \var{g2}\\ \var{g3} \end{pmatrix}\,$ ist. Deshalb ist 
    $\begin{pmatrix} \var{g1} \\ \var{g2}\\ \var{g3} \end{pmatrix}\,$  linear unabhängig von 
    $\vec{w}_1$ und $\vec{w}_2$. Bei den anderen beiden Vektoren hat man auf diese Weise eine 
    passende Linearkombination gefunden, d.h. diese sind linear abhängig von $\vec{w}_1$ und 
    $\vec{w}_2$.
      }
    \lang{en}{
    A vector is linearly dependent on $\,\vec{w}_1$ and $\vec{w}_2$ if and only if it can be 
    written as a linear combination of these two vectors. If this is not possible, then it is 
    linearly independent from $\,\vec{w}_1$ and $\vec{w}_2$.\\
    Hence if we try to write each vector as $\, r\cdot \vec{w}_1+s\cdot \vec{w}_2$ using only the 
    first two components, we obtain
    \begin{itemize}
        \item[for the first vector,] $r=\var{r}, s=\var{s},$
        \item[for the second vector,] $r=\var{s}, s=\var{t} \;$ and
    		\item[for the third vector,] $r=\var{t}, s=\var{r}.$ 
    \end{itemize}
    In the first case, this linear combination yields the vector 
    $\begin{pmatrix} \var{g1} \\ \var{g2}\\ \var{gg3} \end{pmatrix}$, which is not equal to 
		$\begin{pmatrix} \var{g1} \\ \var{g2}\\ \var{g3} \end{pmatrix}\,$. Therefore 
    $\begin{pmatrix} \var{g1} \\ \var{g2}\\ \var{g3} \end{pmatrix}\,$  is linearly independent from
    $\vec{w}_1$ and $\vec{w}_2$. The other two linear combinations do in fact yield the other two 
    vectors, so each of these two is linearly dependent on $\vec{w}_1$ and $\vec{w}_2$.
    }}
	\end{quickcheck}


\lang{de}{
Um herauszufinden, ob mehrere Vektoren $\vec{v}_1,\ldots,\vec{v}_l$ linear unabh"angig sind, müsste man also untersuchen,
ob $\vec{v}_1$ als Linearkombination von $\vec{v}_2,\ldots,\vec{v}_l$ geschrieben werden kann, oder ob
$\vec{v}_2$ als Linearkombination von $\vec{v}_1,\vec{v}_3, \ldots,\vec{v}_l$ geschrieben werden kann, etc.
\\
Einfacher geht es mit Hilfe des folgenden Satzes:
}
\lang{en}{
To determine whether multiple vectors $\vec{v}_1,\ldots,\vec{v}_l$ are linearly independent, 
we can check if $\vec{v}_1$ can be written as a linear combination of $\vec{v}_2,\ldots,\vec{v}_l$, 
if $\vec{v}_2$ can be written as a linear combination of $\vec{v}_1,\vec{v}_3, \ldots,\vec{v}_l$, 
etc.
\\
The next theorem makes this a simpler task:
}

\begin{theorem} \label{Satz_Lin_Unabh}
\lang{de}{
Vektoren $\vec{v}_1,\ldots,\vec{v}_l$ sind genau dann linear unabh"angig, wenn die Gleichung
}
\lang{en}{
The vectors $\vec{v}_1,\ldots,\vec{v}_l$ are linearly independent if and only if 
}
\[ r_1\vec{v}_1+\ldots +r_l\vec{v}_l=\vec{0} \]
\lang{de}{nur f"ur $r_1=\ldots=r_l=0$ erf"ullt ist.}
\lang{en}{is only satisfied by $r_1=\ldots=r_l=0$.}
\end{theorem}

\begin{example} \label{mehrere_vektoren}
\begin{tabs*}
\tab{\lang{de}{Standardvektoren}\lang{en}{Standard unit vectors}}
\lang{de}{Die Standardvektoren}
\lang{en}{The standard unit vectors}
\[ \vec{e}_1=\begin{pmatrix} 1\\ 0\\ \vdots \\ 0\end{pmatrix},\quad \vec{e}_2=\begin{pmatrix} 0\\ 1\\ \vdots \\ 0\end{pmatrix}
,\quad \ldots, \quad \vec{e}_n=\begin{pmatrix} 0\\ \vdots \\ 0\\ 1\end{pmatrix} \]
\lang{de}{sind linear unabhängig. Es ist nämlich}
\lang{en}{are linearly independent, since}
\[ r_1\vec{e}_1+r_2\vec{e}_2+\ldots +r_n \vec{e}_n=\begin{pmatrix} r_1\\ r_2\\ \vdots \\ r_n\end{pmatrix} \]
\lang{de}{nur dann der Nullvektor, wenn $r_1=r_2= \ldots =r_n=0$ gilt.}
\lang{en}{is only the zero vector if $r_1=r_2= \ldots =r_n=0$}

\tab{\lang{de}{3 Vektoren im $\R^3$}\lang{en}{Three vectors in $\R^3$}} 
\lang{de}{Seien}
\lang{en}{Let}
$
\vec{v}_1 \, = \, \begin{pmatrix}
1 \\ 2 \\ 3
\end{pmatrix}, 
\vec{v}_2 \, = \, \begin{pmatrix}
2 \\ 3 \\ 5
\end{pmatrix}, 
\vec{v}_3 \, = \, \begin{pmatrix}
3 \\ 4 \\ 7
\end{pmatrix}
$
\lang{de}{
gegeben.\\
Um zu entscheiden, ob $\vec{v}_1$, $\vec{v}_2$, $\vec{v}_3$ linear abh{\"a}ngig sind, muss man
also herausfinden, ob die Gleichung $r_1\vec{v}_1+r_2\vec{v}_2+r_3\vec{v}_3=\vec{0}$ nur durch 
$r_1=r_2=r_3=0$ gel"ost wird.
\\\\
Wegen
}
\lang{en}{
\\
To determine whether $\vec{v}_1$, $\vec{v}_2$, $\vec{v}_3$ are linearly dependent, we check whether 
the only solution of the equation $r_1\vec{v}_1+r_2\vec{v}_2+r_3\vec{v}_3=\vec{0}$ is 
$r_1=r_2=r_3=0$.
\\\\
As
}
\[
r_1 \cdot \begin{pmatrix} 1 \\ 2 \\ 3 \end{pmatrix}
+ r_2 \begin{pmatrix} 2 \\ 3 \\ 5 \end{pmatrix} 
+r_3 \begin{pmatrix} 3 \\ 4 \\ 7 \end{pmatrix}
=\begin{pmatrix} r_1+2r_2+3r_3\\ 2r_1+3r_2+4r_3 \\ 3r_1+5r_2+7r_3 \end{pmatrix}
\]
\lang{de}{ist also das System von Gleichungen}
\lang{en}{we must solve the linear system}
\begin{align*}
r_1 &+2r_2 &+3r_3 &= 0 \\
2r_1 &+3r_2 &+4r_3 &= 0 \\
3r_1 &+5r_2&+7r_3 &= 0
\end{align*}
\lang{de}{
zu lösen. Wie das im Allgemeinen gemacht wird, wird im Abschnitt \link{lgs}{Lineare Gleichungssysteme} erklärt.
\\\\
Das Gleichungssystem hier hat unendlich viele Lösungen. Zum Beispiel lösen $r_1=1$, $r_2=-2$ und $r_3=1$ 
alle drei Gleichungen. Die drei Vektoren sind daher linear abhängig.
\\\\
In diesem Fall ist sogar jeder einzelne der Vektoren von den anderen beiden linear abhängig, wie man durch Umstellen
der eben gefundenen Gleichung $1\cdot \vec{v}_1 -2\cdot \vec{v}_2 +1\cdot  \vec{v}_3=\vec{0}$ sehen kann:
}
\lang{en}{
Solving a linear system is covered in a \link{lgs}{previous chapter}.
\\\\
The system here has infinitely many solutions. For example, $r_1=1$, $r_2=-2$ and $r_3=1$ solves all 
three equations. The three vectors are thus linearly dependent.
\\\\
In this case, each of the vectors is linearly dependent on the other two, as can be seen simply by 
rearranging $1\cdot \vec{v}_1 -2\cdot \vec{v}_2 +1\cdot \vec{v}_3=\vec{0}$:
}
\begin{eqnarray*}
\vec{v}_1 &=& 2\cdot \vec{v}_2 -1\cdot  \vec{v}_3 \\
\vec{v}_2 &=& \frac{1}{2}( \vec{v}_1+\vec{v}_3) \\
\vec{v}_3 &=& 2\cdot \vec{v}_2 -1\cdot  \vec{v}_1
\end{eqnarray*}

\tab{\lang{de}{3 Vektoren im $\R^2$}\lang{en}{Three vectors in $\R^2$}}
\lang{de}{
Im \ref[lin-comb][Beispiel zu Linearkombinationen]{ex:linearkombinationen} hatten wir die Vektoren \\
$\begin{pmatrix} 2\\ -1\end{pmatrix}$, $\begin{pmatrix} -1\\ 3\end{pmatrix}$
und $\begin{pmatrix} 1\\ 0\end{pmatrix}$ betrachtet und gesehen, dass
}
\lang{en}{
In \ref[lin-comb][an example from the previous section]{ex:linearkombinationen} we considered the 
vectors \\
$\begin{pmatrix} 2\\ -1\end{pmatrix}$, $\begin{pmatrix} -1\\ 3\end{pmatrix}$ and 
$\begin{pmatrix} 1\\ 0\end{pmatrix}$, and saw that
}
\[ 3\cdot \begin{pmatrix} 2\\ -1\end{pmatrix} + 1\cdot \begin{pmatrix} -1\\ 3\end{pmatrix}
  -5 \cdot \begin{pmatrix} 1\\ 0\end{pmatrix}
  = \begin{pmatrix} 0\\ 0\end{pmatrix}. \]
\lang{de}{
Folglich lässt sich jeder der drei Vektoren aus den jeweils anderen beiden Vektoren erzeugen, 
beispielsweise ist
}
\lang{en}{
Therefore each of the three vectors can be written as a linear combination of the others, i.e.
}
\[ \begin{pmatrix} -1\\ 3\end{pmatrix} 
  = 5 \cdot \begin{pmatrix} 1\\ 0\end{pmatrix}
  - 3\cdot \begin{pmatrix} 2\\ -1\end{pmatrix}. \]

\begin{center}
\image{T108_LinearDependence}
\end{center}

\lang{de}{Diese drei Vektoren sind also linear abhängig.}
\lang{en}{Hence these three vectors are linearly dependent.}

\tab{\lang{de}{2 Vektoren im $\R^2$}\lang{en}{Two vectors in $\R^2$}}
\lang{de}{
Die zwei Vektoren $\begin{pmatrix} 2\\ -1\end{pmatrix}$ und $\begin{pmatrix} 1\\ 0\end{pmatrix}$ 
allein betrachtet sind hingegen linear unabhängig, denn aus
}
\lang{en}{
The two vectors $\begin{pmatrix} 2\\ -1\end{pmatrix}$ and $\begin{pmatrix} 1\\ 0\end{pmatrix}$ 
are linearly independent, since to satisfy
}
\[ r_1\cdot \begin{pmatrix} 2\\ -1\end{pmatrix} + r_2 \cdot \begin{pmatrix} 1\\ 0\end{pmatrix}
= \begin{pmatrix} 0\\ 0\end{pmatrix}, \]
\lang{de}{kann man direkt ablesen, dass $ r_1=0 $ und folglich auch $ r_2=0 $ sein muss.}
\lang{en}{we must clearly have $ r_1=0 $ and it also follows that $ r_2=0 $}
\end{tabs*}
\end{example}

\begin{quickcheckcontainer}
\randomquickcheckpool{1}{2}
    \begin{quickcheck}
		\field{rational}
		\type{input.number}
		\begin{variables}
			\number{k}{1}
%			\randint{k}{1}{2}	% Zufallsvariable zum Vertauschen:
			\function[calculate]{k1}{(2-k)}  % "Dirac"-funktionen
			\function[calculate]{k2}{(k-1)}
			\function[calculate]{nk}{3-k}  % gerade die andere Option

			\randint[Z]{r}{1}{4}
			\randint[Z]{s}{2}{4}
			\randint[Z]{t}{2}{4}
			\randint[Z]{a1}{-4}{4}
			\randint[Z]{a2}{-4}{4}
			\randint[Z]{a3}{-4}{4}

			\randint[Z]{d1}{-4}{4}
			\randint[Z]{d2}{-4}{4}
			\randint[Z]{c1}{-4}{4}
			\randint[Z]{c2}{-4}{4}
            \randint[Z]{h}{1}{3}

			\function[calculate]{b1}{s*a1}
			\function[calculate]{b2}{s*a2}
			\function[calculate]{b3}{s*a3-h*k2}
			\function[calculate]{f1}{t*c1}
			\function[calculate]{f2}{r*d1}
			\function[calculate]{f3}{t*c2+r*d2-h*k1}
			

		\end{variables}
		
			\text{\lang{de}{Welche der folgenden Mengen von Vektoren ist linear abhängig?}
            \lang{en}{Which of the following sets of vectors is linearly independent?}}\\
            
            \begin{choices}{multiple}

                \begin{choice}
                  \text{$\{ \begin{pmatrix} \var{a1} \\ \var{a2}\\ \var{a3} \end{pmatrix};
				        \begin{pmatrix} \var{b1} \\ \var{b2}\\ \var{b3} \end{pmatrix} \}$}
                  \solution[compute]{k=1}
                \end{choice}
                \begin{choice}
                  \text{$\{ \begin{pmatrix} \var{c1} \\ 0\\ \var{c2} \end{pmatrix};
                        \begin{pmatrix} 0 \\ \var{d1}\\ \var{d2} \end{pmatrix};
                        \begin{pmatrix} \var{f1} \\ \var{f2}\\ \var{f3} \end{pmatrix} \}$}
                  \solution[compute]{k=2}
                \end{choice}

             \end{choices}

		\explanation{\lang{de}{
    Die Menge $\{ \begin{pmatrix} \var{a1} \\ \var{a2}\\ \var{a3} \end{pmatrix}; 
    \begin{pmatrix} \var{b1} \\ \var{b2}\\ \var{b3} \end{pmatrix} \}$ 
    ist linear abhängig, denn es gilt
    $\begin{pmatrix} \var{b1} \\ \var{b2}\\ \var{b3} \end{pmatrix} =\var{s}\cdot
     \begin{pmatrix} \var{a1} \\ \var{a2}\\ \var{a3} \end{pmatrix}.$\\
    Die Menge $\{ \begin{pmatrix} \var{c1} \\ 0\\ \var{c2} \end{pmatrix};
    \begin{pmatrix} 0 \\ \var{d1}\\ \var{d2} \end{pmatrix};
    \begin{pmatrix} \var{f1} \\ \var{f2}\\ \var{f3} \end{pmatrix} \}$ ist linear unabhängig.
    }
    \lang{en}{
    The set $\{ \begin{pmatrix} \var{a1} \\ \var{a2}\\ \var{a3} \end{pmatrix}; 
                \begin{pmatrix} \var{b1} \\ \var{b2}\\ \var{b3} \end{pmatrix} \}$ 
    is linearly dependent, as
    $\begin{pmatrix} \var{b1} \\ \var{b2}\\ \var{b3} \end{pmatrix} = \var{s}\cdot
     \begin{pmatrix} \var{a1} \\ \var{a2}\\ \var{a3} \end{pmatrix}.$\\
    The set $\{ \begin{pmatrix} \var{c1} \\ 0\\ \var{c2} \end{pmatrix};
    \begin{pmatrix} 0 \\ \var{d1}\\ \var{d2} \end{pmatrix};
    \begin{pmatrix} \var{f1} \\ \var{f2}\\ \var{f3} \end{pmatrix} \}$ is linearly independent.
    }}
            
	\end{quickcheck}
    
    \begin{quickcheck}
		\field{rational}
		\type{input.number}
		\begin{variables}
			\number{k}{2}
%			\randint{k}{1}{2}	% Zufallsvariable zum Vertauschen:
			\function[calculate]{k1}{(2-k)}  % "Dirac"-funktionen
			\function[calculate]{k2}{(k-1)}
			\function[calculate]{nk}{3-k}  % gerade die andere Option

			\randint[Z]{r}{1}{4}
			\randint[Z]{s}{2}{4}
			\randint[Z]{t}{2}{4}
			\randint[Z]{a1}{-4}{4}
			\randint[Z]{a2}{-4}{4}
			\randint[Z]{a3}{-4}{4}

			\randint[Z]{d1}{-4}{4}
			\randint[Z]{d2}{-4}{4}
			\randint[Z]{c1}{-4}{4}
			\randint[Z]{c2}{-4}{4}
            \randint[Z]{h}{1}{3}

			\function[calculate]{b1}{s*a1}
			\function[calculate]{b2}{s*a2}
			\function[calculate]{b3}{s*a3-h*k2}
			\function[calculate]{f1}{t*c1}
			\function[calculate]{f2}{r*d1}
			\function[calculate]{f3}{t*c2+r*d2-h*k1}
			

		\end{variables}
		
			\text{Welche der folgenden Mengen von Vektoren ist linear abhängig?}\\
            
            \begin{choices}{multiple}

                \begin{choice}
                  \text{$\{ \begin{pmatrix} \var{a1} \\ \var{a2}\\ \var{a3} \end{pmatrix};
				        \begin{pmatrix} \var{b1} \\ \var{b2}\\ \var{b3} \end{pmatrix} \}$}
                  \solution[compute]{k=1}
                \end{choice}
                \begin{choice}
                  \text{$\{ \begin{pmatrix} \var{c1} \\ 0\\ \var{c2} \end{pmatrix};
                        \begin{pmatrix} 0 \\ \var{d1}\\ \var{d2} \end{pmatrix};
                        \begin{pmatrix} \var{f1} \\ \var{f2}\\ \var{f3} \end{pmatrix} \}$}
                  \solution[compute]{k=2}
                \end{choice}

             \end{choices}

		\explanation{Die Menge $\{ \begin{pmatrix} \var{c1} \\ 0\\ \var{c2} \end{pmatrix};
                        \begin{pmatrix} 0 \\ \var{d1}\\ \var{d2} \end{pmatrix};
                        \begin{pmatrix} \var{f1} \\ \var{f2}\\ \var{f3} \end{pmatrix} \}$ ist linear abhängig, denn es gilt
                     \[
                     \begin{pmatrix} \var{f1} \\ \var{f2}\\ \var{f3} \end{pmatrix} =\var{t}\cdot
                     \begin{pmatrix} \var{c1} \\ 0\\ \var{c2} \end{pmatrix} + \var{r}\cdot
                     \begin{pmatrix} 0 \\ \var{d1}\\ \var{d2} \end{pmatrix}
                     \]
                     Die Menge $\{ \begin{pmatrix} \var{a1} \\ \var{a2}\\ \var{a3} \end{pmatrix};
				     \begin{pmatrix} \var{b1} \\ \var{b2}\\ \var{b3} \end{pmatrix} \}$ ist linear unabhängig.
                     }
            
	\end{quickcheck}
    
\end{quickcheckcontainer}

%%%%% alte Version %%%%%%%%%%%%%%%%%%%%%%%%%%%%%%%%%%%%%%%%%%%%%%%%%%%%%%%%%%%%%%%%%%%%%%%%%%%%%%%%%%%%%%%%%%%%%%%%
%
%  \begin{quickcheck}
%          \field{rational}
%          \type{input.number}
%          \begin{variables}
%              \randint{k}{1}{2}	% Zufallsvariable zum Vertauschen:
%              \function[calculate]{k1}{(2-k)}  % "Dirac"-funktionen
%              \function[calculate]{k2}{(k-1)}
%
%              \randint[Z]{r}{1}{4}
%              \randint[Z]{s}{2}{4}
%              \randint[Z]{t}{2}{4}
%              \randint[Z]{a1}{-4}{4}
%              \randint[Z]{a2}{-4}{4}
%              \randint[Z]{a3}{-4}{4}
%
%              \randint[Z]{d1}{-4}{4}
%              \randint[Z]{d2}{-4}{4}
%              \randint[Z]{c1}{-4}{4}
%              \randint[Z]{c2}{-4}{4}
%              \randint[Z]{h}{1}{3}
%
%              \function[calculate]{b1}{s*a1}
%              \function[calculate]{b2}{s*a2}
%              \function[calculate]{b3}{s*a3-h*k2}
%              \function[calculate]{f1}{t*c1}
%              \function[calculate]{f2}{r*d1}
%              \function[calculate]{f3}{t*c2+r*d2-h*k1}
%
%
%          \end{variables}
%
%             \text{Welche der folgenden Mengen von Vektoren ist linear abhängig?\\
%                  1) $\{ \begin{pmatrix} \var{a1} \\ \var{a2}\\ \var{a3} \end{pmatrix};
%                  \begin{pmatrix} \var{b1} \\ \var{b2}\\ \var{b3} \end{pmatrix} \}$\\ 
%                  2)  $\{ \begin{pmatrix} \var{c1} \\ 0\\ \var{c2} \end{pmatrix};
%                  \begin{pmatrix} 0 \\ \var{d1}\\ \var{d2} \end{pmatrix};
%                  \begin{pmatrix} \var{f1} \\ \var{f2}\\ \var{f3} \end{pmatrix} \}$\\
%                  Die Menge mit der Nummer \ansref. }
%
%          \begin{answer}
%              \solution{k}
%          \end{answer}
%	\end{quickcheck}
%
%%%%%%%%%%%%%%%%%%%%%%%%%%%%%%%%%%%%%%%%%%%%%%%%%%%%%%%%%%%%%%%%%%%%%%%%%%%%%%%%%%%%%%%%%%%%%%%%%%%%%%%%%%%%%%%%%%%

\section{\lang{de}{Basis}\lang{en}{Bases}} \label{def_basis}

%
%%% Video Hoever
%
\lang{de}{
Wir betrachten nun spezielle Erzeugendensysteme des $\R^n$, nämlich solche, die ausschließlich aus 
\emph{linear unabhängigen} Vektoren bestehen. Dies führt uns schließlich zum Begriff der \emph{Basis}.
\\
\floatright{\href{https://www.hm-kompakt.de/video?watch=711}{\image[75]{00_Videobutton_schwarz}}}\\\\
}
%
\lang{en}{
We now consider special generating sets of $\R^n$, namely those which are \emph{linearly 
independent}. This leads us to the definition of a \emph{basis}.
}

\begin{definition}[\lang{de}{Basis}\lang{en}{Basis}]
\lang{de}{
Eine Menge von Vektoren $\{ \vec{v}_1;\ldots;\vec{v}_l \}$ im $\R^n$ hei"st \emph{Basis des $\R^n$},
wenn jeder Vektor im $\R^n$ sich eindeutig (d.h. auf genau eine Weise) als Linearkombination der 
Vektoren $\vec{v}_1,\ldots,\vec{v}_l$ schreiben lässt.
}
\lang{en}{
A set of vectors $\{ \vec{v}_1;\ldots;\vec{v}_l \}$ in $\R^n$ is called a \emph{basis of $\R^n$} 
if every vector in $\R^n$ can be uniquely written as a linear combination of the vectors 
$\vec{v}_1,\ldots,\vec{v}_l$.
}
\end{definition}

\begin{remark}
\begin{enumerate}
\item \lang{de}{
      Die Darstellung als Linearkombination der Vektoren einer Basis ist auf \textbf{genau} eine 
      Weise möglich, wohingegen es bei einem Erzeugendensystem nur \textbf{mindestens} eine 
      Möglichkeit geben muss.
      }
      \lang{en}{
      The representation of a vector as a linear combination of vectors in a basis is 
      \textbf{unique}, unlike for a general generating set, where each vector has \textbf{at least} 
      one representation.
      }
\item \lang{de}{Jede Basis ist also auch ein Erzeugendensystem.}
      \lang{en}{Every basis is a generating set.}
\item \lang{de}{
      Ist $\{ \vec{v}_1;\ldots;\vec{v}_l \}$ eine Basis des $\R^n$, so lässt sich auch der 
      Nullvektor auf genau eine Weise als Linearkombination darstellen. Da 
      $0\cdot \vec{v}_1+\ldots+ 0\cdot \vec{v}_l$ eine solche ist, ist sie auch die einzige. Dies 
      bedeutet, dass die Vektoren einer Basis linear unabhängig sind.
      }
      \lang{en}{
      If $\{ \vec{v}_1;\ldots;\vec{v}_l \}$ is a basis of $\R^n$, then even the zero vector is 
      uniquely expressed as a linear combination. Hence $0\cdot \vec{v}_1+\ldots+ 0\cdot \vec{v}_l$ 
      is the only linear combination of basis vectors that equals zero, and the vectors in a basis 
      are linearly independent.
      }
\end{enumerate}
\end{remark}

\lang{de}{Zur vorigen Bemerkung gilt auch die Umkehrung. Genauer}
\lang{en}{The converse of the final remark above also holds. That is:}

\begin{theorem}
\lang{de}{
Eine Menge von Vektoren $\{ \vec{v}_1;\ldots;\vec{v}_l \}$ ist genau dann eine Basis des $\R^n$,
wenn sie ein Erzeugendensystem ist und linear unabhängig ist.
}
\lang{en}{
A set of  vectors $\{ \vec{v}_1;\ldots;\vec{v}_l \}$ is a basis of $\R^n$ if and only if it is a 
linearly independent generating set.
}
\end{theorem}

\begin{example}
\begin{tabs*}
\tab{\lang{de}{Standardvektoren}\lang{en}{Standard unit vectors}}
\lang{de}{
Wie schon gesehen, bilden die Standardvektoren $\vec{e}_1,\ldots, \vec{e}_n $ 
ein Erzeugendensystem des $ \R^n $, und sie sind linear unabhängig. Also bilden 
sie eine Basis des $\R^n $.
}
\lang{en}{
As already established, the standard vectors $\vec{e}_1,\ldots, \vec{e}_n $ form a generating set of 
$ \R^n $, and are linearly independent. They therefore form a basis of $\R^n $.
}
\tab{\lang{de}{3 Vektoren im $\R^3 $}\lang{en}{Three vectors in $\R^3 $}}
\lang{de}{Die drei Vektoren}
\lang{en}{The three vectors}
$
\vec{v}_1 \, = \, \begin{pmatrix}
1 \\ 2 \\ 3
\end{pmatrix}, 
\vec{v}_2 \, = \, \begin{pmatrix}
2 \\ 3 \\ 5
\end{pmatrix}$ und $ 
\vec{v}_3 \, = \, \begin{pmatrix}
3 \\ 4 \\ 7
\end{pmatrix}
$
\lang{de}{
bilden weder ein Erzeugendensystem, noch sind sie linear unabhängig. Insbesondere bilden sie keine Basis.
}
\lang{en}{
Neither form a generating set, nor are linearly independent. In particular, they do not form a basis.
}
\tab{\lang{de}{3 Vektoren im $\R^2$}\lang{en}{Three vectors in $\R^2$}}
\lang{de}{
Die Vektoren $\begin{pmatrix} 2\\ -1\end{pmatrix}$, $\begin{pmatrix} -1\\ 3\end{pmatrix}$
und $\begin{pmatrix} 1\\ 0\end{pmatrix}$ bilden zwar ein Erzeugendensystem des $\R^2$, aber sie sind nicht
linear unabhängig. Im \ref[lin-comb][Beispiel zu Linearkombinationen]{ex:linearkombinationen} hatten wir auch gesehen,
dass sich der Vektor $\begin{pmatrix} 1 \\ 1\end{pmatrix}$ auf zwei verschiedene Arten als Linearkombination
dieser Vektoren schreiben lässt.\\
Sie bilden daher keine Basis des $\R^2$.
}
\lang{en}{
The vectors $\begin{pmatrix} 2\\ -1\end{pmatrix}$, $\begin{pmatrix} -1\\ 3\end{pmatrix}$ 
and $\begin{pmatrix} 1\\ 0\end{pmatrix}$ form a generating set of $\R^2$, however they are not 
linearly independent. As an alternative reason, in the 
\ref[lin-comb][previous section]{ex:linearkombinationen} we saw that the vector 
$\begin{pmatrix} 1 \\ 1\end{pmatrix}$ can be written in two different ways as a linear combination 
of these vectors.\\
They therefore do not form a basis of $\R^2$.
}
\end{tabs*}
\end{example}

\lang{de}{
Der folgende Satz macht die Entscheidung, ob eine gegebene Menge von Vektoren eine Basis ist, 
leichter.
}
\lang{en}{
The following theorem makes it easier to determine whether a given set of vectors is a basis.
}

\begin{theorem}  \label{ES_Basis} 
\begin{enumerate}
\item \lang{de}{
      Jede Basis des $\R^n$ hat gleich viele Elemente. Insbesondere, da 
      $\{ \vec{e}_1;\ldots; \vec{e}_n\}$ eine Basis ist, besitzt jede Basis des $\R^n$ genau $n$ 
      Elemente.
      }
      \lang{en}{
      Every basis of $\R^n$ has the same number of elements. In particular, as 
      $\{ \vec{e}_1;\ldots; \vec{e}_n\}$ is a basis, every basis of $\R^n$ contains exactly $n$ 
      vectors.
      }
\item \lang{de}{
      Für eine Menge $M$ von $n$ Vektoren im $\R^n$ sind gleichbedeutend:
      \begin{itemize}
      \item $M$ ist eine Basis,
      \item $M$ ist ein Erzeugendensystem,
      \item die Vektoren von $M$ sind linear unabhängig. 
      \end{itemize}
      }
      \lang{en}{
      For a set $M$ conntaining $n$ vectors in $\R^n$, the following are equivalent:
      \begin{itemize}
      \item $M$ is a basis,
      \item $M$ is a generating set,
      \item the vectors in $M$ are linearly independent. 
      \end{itemize}
      }
\item \lang{de}{
      Eine Menge $M$ von Vektoren im $\R^n$ mit mehr als $n$ Elementen ist stets linear abhängig.
      }
      \lang{en}{
      A set $M$ of vectors in $\R^n$ with more than $n$ elements is always linearly dependent.
      }
\item \lang{de}{
      Eine Menge $M$ von Vektoren im $\R^n$ mit weniger als $n$ Elementen ist nie ein 
      Erzeugendensystem des $\R^n$.
      }
      \lang{en}{
      A set $M$ of vectors in $\R^n$ with fewer than $n$ elements is never a generating set of 
      $\R^n$.
      }
\end{enumerate}
\end{theorem}

\begin{definition}[\lang{de}{Dimension}\lang{en}{Dimension}]\label{def:dimension}
\lang{de}{
Die Anzahl der Elemente einer Basis eines Vektorraums (und damit jeder Basis eines Vektorraums) 
nennt man die \emph{Dimension} des Vektorraumes.
}
\lang{en}{
The number of elements in a basis of a vector space (and hence every basis of that vector space) 
is called the \emph{dimension} of the vector space.
}
\end{definition}

\begin{example}
\begin{tabs*}
\tab{\lang{de}{3 Vektoren im $\R^3$}\lang{en}{Three vectors in $\R^3$}}
\lang{de}{In den vorigen Beispielen hatten wir die drei Vektoren}
\lang{en}{In the previous examples we considered the three vectors}
$
\vec{v}_1 \, = \, \begin{pmatrix}
1 \\ 2 \\ 3
\end{pmatrix}, 
\vec{v}_2 \, = \, \begin{pmatrix}
2 \\ 3 \\ 5
\end{pmatrix}$ und $ 
\vec{v}_3 \, = \, \begin{pmatrix}
3 \\ 4 \\ 7
\end{pmatrix}$
\lang{de}{
im $\R^3$. \\
Im früheren \ref[Erzeugendensystem][Beispiel zu 3 Vektoren im $\R^3$]{mehrere_vektoren} 
haben wir gezeigt, dass die 3 Vektoren linear abhängig sind. Nach dem vorausgegangenen Satz 
folgt dann direkt, dass sie auch kein Erzeugendensystem bilden.
}
\lang{en}{
in $\R^3$. \\
In the \ref[Erzeugendensystem][previous example for three vectors in $\R^3$]{mehrere_vektoren} we 
showed that these three vectors are linearly dependent. By the previous statement we immediately 
have that they do not form a basis.
}

\tab{\lang{de}{3 Vektoren im $\R^2$}\lang{en}{Three vectors in $\R^2$}}
\lang{de}{
Die Vektoren $\begin{pmatrix} 2\\ -1\end{pmatrix}$, $\begin{pmatrix} -1\\ 3\end{pmatrix}$
und $\begin{pmatrix} 1\\ 0\end{pmatrix}$ im $\R^2$ können nach obigem Satz gar nicht linear unabhängig sein,
was im früheren \ref[Erzeugendensystem][Beispiel zu 3 Vektoren im $\R^2$]{mehrere_vektoren} auch bestätigt wurde.
}
\lang{en}{
The vectors $\begin{pmatrix} 2\\ -1\end{pmatrix}$, $\begin{pmatrix} -1\\ 3\end{pmatrix}$ and 
$\begin{pmatrix} 1\\ 0\end{pmatrix}$ in $\R^2$ cannot be linearly independent, by the above theorem. 
We also showed this in the 
\ref[Erzeugendensystem][previous example for three vectors in $\R^2$]{mehrere_vektoren}.
}

\tab{\lang{de}{2 Vektoren im $\R^2$}\lang{en}{Two vectors in $\R^2$}}
\lang{de}{
Im früheren \ref[Erzeugendensystem][Beispiel zu 2 Vektoren im $\R^2$]{mehrere_vektoren}
haben wir gezeigt, dass die Vektoren $\begin{pmatrix} 2\\ -1\end{pmatrix}$ und 
$\begin{pmatrix} 1\\ 0\end{pmatrix}$ linear unabhängig sind. Nach obigem Satz bilden sie somit eine 
Basis des $\R^2$.
}
\lang{en}{
In the \ref[Erzeugendensystem][previous example for two vectors in $\R^2$]{mehrere_vektoren} we 
showed that the vectors $\begin{pmatrix} 2\\ -1\end{pmatrix}$ and 
$\begin{pmatrix} 1\\ 0\end{pmatrix}$ are linearly independent. By the above theorem, we immediately 
have that they are a basis of $\R^2$.
}

\end{tabs*}
\end{example}


\begin{quickcheckcontainer}
\randomquickcheckpool{1}{2}
\begin{quickcheck}
		\field{rational}
		\type{input.number}
		\begin{variables}
			\number{k}{2}
%			\randint{k}{1}{2}	% Zufallsvariable zum Vertauschen:
			\function[calculate]{k1}{(2-k)}  % "Dirac"-funktionen
			\function[calculate]{k2}{(k-1)}

			\randint[Z]{r}{1}{4}
			\randint[Z]{s}{2}{4}
			\randint[Z]{t}{2}{4}

% (Ticket #10)  	
%			\randint[Z]{d1}{-4}{4}
%			\randint[Z]{d2}{-4}{4}
%			\randint[Z]{c1}{-4}{4}
%			\randint[Z]{c2}{-4}{4}

			\randint[Z]{d1}{1}{5}
			\randint[Z]{d2}{1}{5}
			\randint[Z]{c1}{1}{5}
			\randint[Z]{c2}{1}{5}

			\function[calculate]{f1}{t*c1}
			\function[calculate]{f2}{r*d1}
			\function[calculate]{f3}{t*c2+r*d2+s*k2}
			
			\function[calculate]{z}{s*k2}
			

		\end{variables}
		
			\text{\lang{de}{Ist die folgende Menge von Vektoren eine Basis des $\R^3$?}
        \lang{en}{Is the foollowing set of vectors a basis of $\R^3$?}\\
				$\{ \begin{pmatrix} \var{c1} \\ 0\\ \var{c2} \end{pmatrix};
				\begin{pmatrix} \var{f1} \\ \var{f2}\\ \var{f3} \end{pmatrix};
				\begin{pmatrix} 0 \\ \var{d1}\\ \var{d2} \end{pmatrix} \}$}\\
                
                                
                \begin{choices}{unique}

                    \begin{choice}
                      \text{\lang{de}{ja}\lang{en}{yes}}
                      \solution[compute]{k2=1}
                    \end{choice}
                    \begin{choice}
                      \text{\lang{de}{nein}\lang{en}{no}}
                      \solution[compute]{k2=0}
                    \end{choice}

                 \end{choices}
                
                
%%%%% alte Version %%%%%%%%%%%%%%%%%%%%%%%%%%%%%%%%%%%%%%%%%%%%%%%%%%%%%%%%%%%%%%%%%%%%%%%%%%%%%%%%%%%%%%%%%%%%%%%%
%                
%
%				\ansref (Antwort $1$ für ja, $0$ für nein). 
%		
%		\begin{answer}
%			\solution{k2}
%		\end{answer}
%
%%%%%%%%%%%%%%%%%%%%%%%%%%%%%%%%%%%%%%%%%%%%%%%%%%%%%%%%%%%%%%%%%%%%%%%%%%%%%%%%%%%%%%%%%%%%%%%%%%%%%%%%%%%%%%%%%%%

		\explanation{\lang{de}{
    Da die Menge aus drei Elementen besteht, ist nur zu testen, ob sie linear unabhängig ist.\\
		Mit dem Ansatz:
    }
    \lang{en}{
    As the set contains three elements, we can simply check whether it is linearly independent.\\
    From the equation:
    }\\
		$ \begin{pmatrix} 0 \\ 0\\ 0 \end{pmatrix}=r_1\cdot \begin{pmatrix} \var{c1} \\ 0\\ \var{c2} \end{pmatrix}
		+r_2\cdot \begin{pmatrix} \var{f1} \\ \var{f2}\\ \var{f3} \end{pmatrix}
		+r_3\cdot \begin{pmatrix} 0 \\ \var{d1}\\ \var{d2} \end{pmatrix}=
% (Ticket #10) 	
% mit Klammern:    		\begin{pmatrix} \var{c1}r_1+(\var{f1})r_2 \\ \var{f2}r_2+(\var{d1})r_3\\ \var{c2}r_1+(\var{f3})r_2+(\var{d2})r_3 \end{pmatrix} \]
        \begin{pmatrix} \var{c1}r_1+\var{f1}r_2 \\ \var{f2}r_2+\var{d1}r_3\\ \var{c2}r_1+\var{f3}r_2+\var{d2}r_3 \end{pmatrix} $\\
    \lang{de}{
		erhält man aus der ersten Zeile $r_1=-\var{t}r_2$ und aus der zweiten Zeile $r_3=-\var{r}r_2$. 
    Durch Einsetzen in die dritte Zeile, erhält man
    }
    \lang{en}{
    the first row yields $r_1=-\var{t}r_2$ and the second row yields $r_3=-\var{r}r_2$. By 
    substituting these into the third row, we obtain
    }\\
% (Ticket #10) 	
% mit Klammern:    		\[ 0= \var{c2}\cdot (-\var{t}r_2)+(\var{f3})r_2+(\var{d2})\cdot (-\var{r}r_2)=\var{z}r_2. \]
    $ 0= \var{c2}\cdot (-\var{t}r_2)+\var{f3}r_2+\var{d2}\cdot (-\var{r}r_2)=\var{z}r_2. $\\
    \lang{de}{
		Also bekommt man aus der dritten Zeile die Bedingung $r_2=0$ und mit den anderen Gleichungen 
    auch $r_1=r_3=0$, weshalb dies die einzige Lösung ist. Die Vektoren sind also linear unabhängig.
		}
    \lang{en}{
    Hence the third row yields $r_2=0$ and thus $r_1=r_3=0$, and this is the only solution. The 
    vectors are therefore linearly independent, and form a basis.
    }}
	\end{quickcheck}

\begin{quickcheck}
		\field{rational}
		\type{input.number}
		\begin{variables}
			\number{k}{1}
%			\randint{k}{1}{2}	% Zufallsvariable zum Vertauschen:
			\function[calculate]{k1}{(2-k)}  % "Dirac"-funktionen
			\function[calculate]{k2}{(k-1)}

			\randint[Z]{r}{1}{4}
			\randint[Z]{s}{2}{4}
			\randint[Z]{t}{2}{4}

% (Ticket #10)  	
%			\randint[Z]{d1}{-4}{4}
%			\randint[Z]{d2}{-4}{4}
%			\randint[Z]{c1}{-4}{4}
%			\randint[Z]{c2}{-4}{4}

			\randint[Z]{d1}{1}{5}
			\randint[Z]{d2}{1}{5}
			\randint[Z]{c1}{1}{5}
			\randint[Z]{c2}{1}{5}

			\function[calculate]{f1}{t*c1}
			\function[calculate]{f2}{r*d1}
			\function[calculate]{f3}{t*c2+r*d2+s*k2}
			
			\function[calculate]{z}{s*k2}
			

		\end{variables}
		
			\text{\lang{de}{Ist die folgende Menge von Vektoren eine Basis des $\R^3$?}
            \lang{en}{Is the following set of vectors a basis of $\R^3$?}\\
				$\{ \begin{pmatrix} \var{c1} \\ 0\\ \var{c2} \end{pmatrix};
				\begin{pmatrix} \var{f1} \\ \var{f2}\\ \var{f3} \end{pmatrix};
				\begin{pmatrix} 0 \\ \var{d1}\\ \var{d2} \end{pmatrix} \}$}\\
                
                \begin{choices}{unique}

                    \begin{choice}
                      \text{\lang{de}{ja}\lang{en}{yes}}
                      \solution[compute]{k2=1}
                    \end{choice}
                    \begin{choice}
                      \text{\lang{de}{nein}\lang{en}{no}}
                      \solution[compute]{k2=0}
                    \end{choice}

                 \end{choices}
                
                
%%%%% alte Version %%%%%%%%%%%%%%%%%%%%%%%%%%%%%%%%%%%%%%%%%%%%%%%%%%%%%%%%%%%%%%%%%%%%%%%%%%%%%%%%%%%%%%%%%%%%%%%%
%                
%				\ansref (Antwort $1$ für ja, $0$ für nein). }
%	
%		\begin{answer}
%			\solution{k2}
%		\end{answer}
%
%%%%%%%%%%%%%%%%%%%%%%%%%%%%%%%%%%%%%%%%%%%%%%%%%%%%%%%%%%%%%%%%%%%%%%%%%%%%%%%%%%%%%%%%%%%%%%%%%%%%%%%%%%%%%%%%%%

		\explanation{\lang{de}{
    Da die Menge aus drei Elementen besteht, ist nur zu testen, ob sie linear unabhängig ist.\\
		Mit dem Ansatz:
    }
    \lang{en}{
    As the set contains three elements, we can simply check whether it is linearly independent.\\
    From the equation:
    }\\
		$ \begin{pmatrix} 0 \\ 0\\ 0 \end{pmatrix}=r_1\cdot \begin{pmatrix} \var{c1} \\ 0\\ \var{c2} \end{pmatrix}
		+r_2\cdot \begin{pmatrix} \var{f1} \\ \var{f2}\\ \var{f3} \end{pmatrix}
		+r_3\cdot \begin{pmatrix} 0 \\ \var{d1}\\ \var{d2} \end{pmatrix}=
% (Ticket #10) 	
% mit Klammern:    \begin{pmatrix} \var{c1}r_1+(\var{f1})r_2 \\ \var{f2}r_2+(\var{d1})r_3\\ \var{c2}r_1+(\var{f3})r_2+(\var{d2})r_3 \end{pmatrix}
        \begin{pmatrix} \var{c1}r_1+\var{f1}r_2 \\ \var{f2}r_2+\var{d1}r_3\\ \var{c2}r_1+\var{f3}r_2+\var{d2}r_3 \end{pmatrix}        
    $\\
    \lang{de}{
		erhält man aus der ersten Zeile $r_1=-\var{t}r_2$ und aus der zweiten Zeile $r_3=-\var{r}r_2$. 
    Durch Einsetzen in die dritte Zeile, erhält man die Gleichung
    }
    \lang{en}{
    the first row yields $r_1=-\var{t}r_2$ and the second row yields $r_3=-\var{r}r_2$. By 
    substituting these into the third row, we obtain
    }\\
% (Ticket #10)	
% mit Klammern:    \[ 0= \var{c2}\cdot (-\var{t}r_2)+(\var{f3})r_2+(\var{d2})\cdot (-\var{r}r_2)=0\cdot r_2.\]
		$ 0= \var{c2}\cdot (-\var{t}r_2)+\var{f3}r_2+\var{d2}\cdot (-\var{r}r_2)=0\cdot r_2.$\\
    \lang{de}{
		Diese ist für beliebiges $r_2$ erfüllt, weshalb dann zum Beispiel die Wahl von $r_2=1$, 
    $r_1=-\var{t}$ und $r_3=-\var{r}$ eine nichttriviale Linearkombination für den Nullvektor 
    ergibt. Die Vektoren sind daher linear abhängig.
    }
    \lang{en}{
    This is satisfied by any $r_2$, so for example setting $r_2=1$, $r_1=-\var{t}$ and 
    $r_3=-\var{r}$ yields a non-trivial linear combination representing the zero vector. The vectors 
    are therefore linearly dependent, and do not form a basis.
		}}
	\end{quickcheck}
\end{quickcheckcontainer}


\end{visualizationwrapper}


\end{content}