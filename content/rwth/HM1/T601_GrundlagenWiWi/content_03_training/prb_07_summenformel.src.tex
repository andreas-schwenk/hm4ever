\documentclass{mumie.problem.gwtmathlet}
%$Id$
\begin{metainfo}
  \name{
    \lang{en}{...}
    \lang{de}{A07: Gauß'sche Summenformel}
    \lang{zh}{...}
    \lang{fr}{...}
  }
  \begin{description} 
 This work is licensed under the Creative Commons License Attribution 4.0 International (CC-BY 4.0)   
 https://creativecommons.org/licenses/by/4.0/legalcode 

    \lang{en}{...}
    \lang{de}{...}
    \lang{zh}{...}
    \lang{fr}{...}
  \end{description}
  \corrector{system/problem/GenericCorrector.meta.xml}
  \begin{components}
    \component{js_lib}{system/problem/GenericMathlet.meta.xml}{gwtmathlet}
  \end{components}
  \begin{links}
  \end{links}
  \creategeneric
\end{metainfo}
\begin{content}
\lang{de}{\title{A07: Gauß'sche Summenformel}}
\lang{en}{\title{A07:  Gauss' summation formula}}
\begin{block}[annotation]
	Im Ticket-System: \href{https://team.mumie.net/issues/23822}{Ticket 23822}
\end{block}


\usepackage{mumie.genericproblem}

\begin{problem}
\begin{variables}
        \drawFromSet{tp}{80,90,100,110,120}
        \drawFromSet{ws}{20,30,40,50}
        \drawFromSet{m}{10,15,20,25}
        \function[calculate]{sum}{m*5*tp+(m-1)*m*ws/2}
        \function{lewo}{5*tp+(m-1)ws}
        \function[calculate]{mm}{m-1}
\end{variables}

\begin{question}
       \type{input.number} 
     \begin{answer}
        \lang{de}{\text{Ein Betrieb erhält den Auftrag, $\var{sum}$ Pullover herzustellen. In der ersten
        Woche werden täglich (5 Arbeitstage) $\var{tp}$ produziert. Bis zur Erfüllung des 
        Auftrages ist die Produktion um $\var{ws}$  je Woche zu steigern. \\Nach wieviel Wochen $m$ ist 
        der Auftrag erfüllt?}}
        \lang{en}{\text{A factory receives an order to produce $\var{sum}$ jumpers. In the first
        week (5 working days), $\var{tp}$ are produced daily. Until the 
        order is fulfilled, production is to be increased by $\var{ws}$ per week. \\After how many weeks $m$ is 
        the order fulfilled?}}
        \solution{m}
        \lang{de}{\explanation{Es muss eine arithmetische Summe gebildet werden: 
        $\sum_{k=0}^{m-1}5\cdot\var{tp}+k\cdot \var{ws}=m\cdot 5 \cdot\var{tp}+\frac{(m-1)m}{2}\var{ws}$. Dieser Term wird gleichgesetzt
        mit der erforderlichen Anzahl an Pullovern $\var{sum}$. Die positive Lösung dieser quadratischen Gleichung
        liefert die Anzahl der Wochen. 
        %$m=\var{m}$.
        }}
        \lang{en}{\explanation{An arithmetic sum must be formed: 
        $\sum_{k=0}^{m-1}5\cdot\var{tp}+k\cdot \var{ws}=m\cdot 5 \cdot\var{tp}+\frac{(m-1)m}{2}\var{ws}$. This term is set equal to
        the required number of jumpers $\var{sum}$. The positive solution of this quadratic equation
        gives the number of weeks. 
        %$m=\var{m}$.
        }}
     
     \end{answer}
\end{question}

\begin{question}
    \type{input.number}
     \begin{answer}
        \lang{de}{\text{Wieviel Stück werden in der letzten Woche hergestellt?}}
        \lang{en}{\text{How many were produced in the last week?}}
        \solution{lewo}
        \lang{de}{\explanation{Bei $\var{m}$ Wochen findet die Produktionssteigerung einmal weniger statt. 
        %Die Stückzahl
        %der letzten Woche ist damit $5\cdot \var{tp}+\var{mm}\cdot \var{ws}$.
        }}
        \lang{en}{\explanation{For $\var{m}$ weeks, the production increase takes place one time less often. 
        %The number of pieces
        %of last week is therefore $5\cdot \var{tp}+\var{mm}\cdot \var{ws}$.
        }}
     
     \end{answer}
\end{question}
\end{problem}

\embedmathlet{gwtmathlet}

\end{content}
