\documentclass{mumie.problem.gwtmathlet}
%$Id$
\begin{metainfo}
  \name{
    \lang{en}{...}
    \lang{de}{A05: arithmetische Folge}
    \lang{zh}{...}
    \lang{fr}{...}
  }
  \begin{description} 
 This work is licensed under the Creative Commons License Attribution 4.0 International (CC-BY 4.0)   
 https://creativecommons.org/licenses/by/4.0/legalcode 

    \lang{en}{...}
    \lang{de}{...}
    \lang{zh}{...}
    \lang{fr}{...}
  \end{description}
  \corrector{system/problem/GenericCorrector.meta.xml}
  \begin{components}
    \component{js_lib}{system/problem/GenericMathlet.meta.xml}{gwtmathlet}
  \end{components}
  \begin{links}
  \end{links}
  \creategeneric
\end{metainfo}
\begin{content}
\lang{de}{\title{A05: arithmetische Folge}}
\lang{en}{\title{A05: arithmetic sequence}}
\begin{block}[annotation]
	Im Ticket-System: \href{https://team.mumie.net/issues/24039}{Ticket 24039}
\end{block}

\usepackage{mumie.genericproblem}

     \begin{problem}
        \begin{variables}
            \randint{a}{2}{5}
            \function[calculate]{k0}{a*7000}
            \function[calculate]{A}{a*1200}
            \function{k1}{a*(7000-1*1200)}
            \function{k2}{a*(7000-2*1200)}
            \function{k3}{a*(7000-3*1200)}
            \function{k4}{a*(7000-4*1200)}
            \function{k5}{a*(7000-5*1200)}
            \function{p1}{a*120000/k0}
            \function{p2}{a*120000/k1}
            \function{p3}{a*120000/k2}
            \function{p4}{a*120000/k3}
            \function{p5}{a*120000/k4}
        \end{variables}
          \begin{question}
                \type{input.number}
               \lang{de}{\text{Ein Copy-Shop schafft $\var{a}$ neue Laserdrucker an. Ein neuer Laserdrucker hat einen Anschaffungswert
               von $7000$ \euro. Die Nutzungsdauer betrage $m=5$ Jahre. Danach rechnet man mit
               einem Restwert (Wiederverkaufswert) von $1000$ \euro je Laserdrucker.\\
               Erstellen Sie einen Plan für eine lineare Abschreibung:
               \begin{table}
     \head
     n&$K_n$&$A_n$&$p_n$ (in \%)
     \body
     \textbf{0}&\var{k0}     &-      &-\\
     \textbf{1}&\ansref&\ansref&\ansref\\
     \textbf{2}&\ansref&\ansref&\ansref\\
     \textbf{3}&\ansref&\ansref&\ansref\\
     \textbf{4}&\ansref&\ansref&\ansref\\
     \textbf{5}&\ansref&\ansref&\ansref\\
     \end{table}
               }}
               \lang{en}{\text{A copy shop acquires $\var{a}$ new laser printers. A new laser printer has a purchase price of
               $7000$ \euro. Its useful life is $m=5$ years. After that one expects
               a residual value (resale value) of $1000$ \euro per laser printer.
               Draw up a schedule for straight-line depreciation:
               \begin{table}
               \head
     n&$K_n$&$A_n$&$p_n$ (in \%)
     \body
     \textbf{0}&\var{k0}     &-      &-\\
     \textbf{1}&\ansref&\ansref&\ansref\\
     \textbf{2}&\ansref&\ansref&\ansref\\
     \textbf{3}&\ansref&\ansref&\ansref\\
     \textbf{4}&\ansref&\ansref&\ansref\\
     \textbf{5}&\ansref&\ansref&\ansref\\
     \end{table}
               }}
               
               
           %1.
               \begin{answer}
               \solution{k1}
               \end{answer}
               
               \begin{answer}
               \solution{A}
               \end{answer}
               
               \begin{answer}
               \solution{p1}
               \end{answer}
            %2.
               \begin{answer}
               \solution{k2}
               \end{answer}
               
               \begin{answer}
               \solution{A}
               \end{answer}
               
               \begin{answer}
               \solution{p2}
               \end{answer}
            %3.
               \begin{answer}
               \solution{k3}
               \end{answer}
               
               \begin{answer}
               \solution{A}
               \end{answer}
               
               \begin{answer}
               \solution{p3}
               \end{answer}
            %4.
               \begin{answer}
               \solution{k4}
               \end{answer}
               
               \begin{answer}
               \solution{A}
               \end{answer}
               
               \begin{answer}
               \solution{p4}
               \end{answer}
            %5.
               \begin{answer}
               \solution{k5}
               \end{answer}
               
               \begin{answer}
               \solution{A}
               \end{answer}
               
               \begin{answer}
               \solution{p5}
               \end{answer} 

               \lang{de}{
               \explanation{$A_n=A=\frac{(7000-1000)\cdot\var{a}}{5}$ und \\
               $p_n=\frac{A}{K_{n-1}} \cdot 100$.
               }}
               \lang{en}{
               \explanation{$A_n=A=\frac{(7000-1000)\cdot\var{a}}{5}$ and \\
               $p_n=\frac{A}{K_{n-1}} \cdot 100$.
               }}
          
          \end{question} 
          
     \end{problem}

\embedmathlet{gwtmathlet}

\end{content}
