\documentclass{mumie.problem.gwtmathlet}
%$Id$
\begin{metainfo}
  \name{
    \lang{en}{...}
    \lang{de}{A06: arithmetische Reihe der Restbuchwerte}
    \lang{zh}{...}
    \lang{fr}{...}
  }
  \begin{description} 
 This work is licensed under the Creative Commons License Attribution 4.0 International (CC-BY 4.0)   
 https://creativecommons.org/licenses/by/4.0/legalcode 

    \lang{en}{...}
    \lang{de}{...}
    \lang{zh}{...}
    \lang{fr}{...}
  \end{description}
  \corrector{system/problem/GenericCorrector.meta.xml}
  \begin{components}
    \component{js_lib}{system/problem/GenericMathlet.meta.xml}{gwtmathlet}
  \end{components}
  \begin{links}
  \end{links}
  \creategeneric
\end{metainfo}
\begin{content}
\lang{de}{\title{A06: arithmetische Reihe der Restbuchwerte}}
\lang{en}{\title{A06: arithmetic series of residual book values}}
\begin{block}[annotation]
	Im Ticket-System: \href{https://team.mumie.net/issues/23654}{Ticket 23654}
\end{block}

\usepackage{mumie.genericproblem}
% $A_1<\frac{2(K_0-K_m)}{m}\iff m<\frac{2(K_0-K_m)}{A_1}$
     \begin{problem}
        \begin{variables}
            \randint{a}{1}{4}
            \randint{b}{1}{3}
            \drawFromSet{p}{11,12,13,14,15,16}
            \function[calculate]{k0}{a*300000}
            \function[calculate]{km}{a*b*10000}
            \function{m}{2*(k0-km)/(p*0.01*k0)}
            \function[calculate]{mtrunc}{floor(m)+1}
        \end{variables}
          \begin{question}
               \type{input.number} 
               \lang{de}{\explanation{Hier sollte man die Abschätzung 
               $\frac{K_0-K_m}{m}<A_1<\frac{2(K_0-K_m)}{m}$ und $A_1=\frac{p}{100}\cdot K_0$ benutzen.
                Der rechte Teil der Ungleichung wird dann zu:
                $m<\frac{2(K_0-K_m)}{A_1}=\frac{200(K_0-K_m)}{p\cdot K_0}$}}
               \lang{en}{\explanation{Here one should use the bounds
               $\frac{K_0-K_m}{m}<A_1<\frac{2(K_0-K_m)}{m}$ and $A_1=\frac{p}{100}\cdot K_0$.
                The right-hand part of the inequality then becomes:
                $m<\frac{2(K_0-K_m)}{A_1}=\frac{200(K_0-K_m)}{p\cdot K_0}$}}
                \lang{de}{\text{In spätestens wieviel Jahren ist eine Fabrikeinrichtung mit einem 
               Anschaffungswert von $\var{k0}$ \euro bei geometrisch-degressiver Abschreibung
               mit einem Abschreibungsprozentsatz von $p=\var{p}$ \% auf einen Restwert
               von $\var{km}$ \euro abgeschrieben?\\
               $m<$ \ansref Jahre
               }}
               \lang{en}{\text{In how many years will a piece of factory equipment with an 
               acquisition value of $\var{k0}$ \euro have depreciated to a residual value of
               at most $\var{km}$ \euro, assuming geometric degressive depreciation
               with a depreciation rate of $p=\var{p}$ \%? \\
               $m<$ \ansref years
               }}
               
               \begin{answer}
                    \solution{mtrunc}
               \end{answer}          
          \end{question}     
     \end{problem}

\embedmathlet{gwtmathlet}

\end{content}
