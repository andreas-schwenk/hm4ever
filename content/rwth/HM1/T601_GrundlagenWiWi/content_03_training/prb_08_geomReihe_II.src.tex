\documentclass{mumie.problem.gwtmathlet}
%$Id$
\begin{metainfo}
  \name{
    \lang{en}{...}
    \lang{de}{A08: geometrische Reihe II}
    \lang{zh}{...}
    \lang{fr}{...}
  }
  \begin{description} 
 This work is licensed under the Creative Commons License Attribution 4.0 International (CC-BY 4.0)   
 https://creativecommons.org/licenses/by/4.0/legalcode 

    \lang{en}{...}
    \lang{de}{...}
    \lang{zh}{...}
    \lang{fr}{...}
  \end{description}
  \corrector{system/problem/GenericCorrector.meta.xml}
  \begin{components}
    \component{js_lib}{system/problem/GenericMathlet.meta.xml}{gwtmathlet}
  \end{components}
  \begin{links}
  \end{links}
  \creategeneric
\end{metainfo}
\begin{content}
\lang{de}{\title{A08: geometrische Reihe II}}
\lang{en}{\title{A08: geometric series II}}
\begin{block}[annotation]
	Im Ticket-System: \href{https://team.mumie.net/issues/24040}{Ticket 24040}
\end{block}


\usepackage{mumie.genericproblem}

     \begin{problem}
          \begin{variables}
                \drawFromSet{p}{1.1,1.2,1.3,1.4,1.5,1.6,1.7,1.8,1.9}
                \drawFromSet{k}{1100,1200,1300,1400,1500,1600,1700,1800,1900}
                \function{f}{k*(1-(1+p/100)^12)/(1-(1+p/100))} 
                \function{f1}{floor(f)}
          \end{variables}
          \begin{question}
                \lang{de}{\explanation{Hier handelt es sich um eine geometrische Reihe: 
                $s_{12}=\var{k}\cdot\sum_{k=0}^{11}q^k$
                %=\var{k}\frac{1-q^{12}}{1-q}$
                mit
                $q=(1+\frac{\var{p}}{100})$.}}
                \lang{en}{\explanation{This is a geometric series: 
                $s_{12}=\var{k}\cdot\sum_{k=0}^{11}q^k$
                %=\var{k}\frac{1-q^{12}}{1-q}$
                with
                $q=(1+\frac{\var{p}}{100})$.}}
                \type{input.number}
                \lang{de}{\text{In einem Betrieb soll innerhalb eines Jahres die Arbeitsproduktivität
                monatlich um \var{p} \% gesteigert werden. Im Januar betrug die Produktion \var{k}
                Stück.\\
                Wie hoch ist die gesamte Jahresproduktion $P$?\\
                \textit{Bitte runden Sie auf eine ganze Zahl.}\\
                $P=$\ansref}}
                \lang{en}{\text{A company's production increases over one year 
                by \var{p} \% per month. In January the production is \var{k}
                units.\\
                What will the total annual production $P$ be?\\
                \textit{Please round to a whole number.}\\
                 $P=$\ansref}}
               \begin{answer}
                    \solution{f1}
               \end{answer}          
          \end{question}     
     \end{problem}

\embedmathlet{gwtmathlet}

\end{content}
