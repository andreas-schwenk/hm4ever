\documentclass{mumie.element.exercise}
%$Id$
\begin{metainfo}
  \name{
     \lang{de}{Ü01: Mengenoperationen}
     \lang{en}{Ü01: Set operations}
     }
  \begin{description} 
 This work is licensed under the Creative Commons License Attribution 4.0 International (CC-BY 4.0)   
 https://creativecommons.org/licenses/by/4.0/legalcode 

    \lang{de}{Ü01: Mengenoperationen}
   \end{description}
  \begin{components}
    \component{generic_image}{content/rwth/HM1/images/g_tkz_T601_01_Exercise01.meta.xml}{T601_01_Exercise01}
  \end{components}
  \begin{links}
    \link{generic_article}{content/rwth/HM1/T101neu_Elementare_Rechengrundlagen/g_art_content_01_zahlenmengen.meta.xml}{content_01_zahlenmengen}
  \end{links}
  \creategeneric
\end{metainfo}

\begin{content}
  \title{
    \lang{de}{Ü01: Mengenoperationen}
     \lang{en}{Exercise 1: Set operations}
  }
\begin{block}[annotation]
	Im Ticket-System: \href{https://team.mumie.net/issues/23723}{Ticket 23723}
\end{block}
\begin{block}[annotation]
Kopie: hm4mint/T101_neu_elementare_Rechengrundlagen/exercise 1

Im Ticket-System: \href{https://team.mumie.net/issues/21971}{Ticket 21971}
\end{block}


 \begin{block}[annotation]
   Übung zu Mengenoperationen, Mengendarstellungen und Zahlenmengen    
\end{block}


 

 \lang{de}{Von der Menge    
  \[ N= \{ 1; 2; \ldots; 10 \} \]
  werden die Teilmengen
  \[ A= \{ 1; 3; 5; 7; 8; 9 \} \quad \text{und} \quad B=\{ x \in N \mid x\text{ ist durch}\, 3 \, \text{teilbar} \} \]
%    \[ A= \{ 1; 3; 5; 7; 8; 9 \} \quad \text{und} \quad B=\{ 3; 4; 6; 8; 10\} \]
  betrachtet.}
  \lang{en}{ In the set    
  \[ N= \{ 1, 2, \ldots, 10 \} \]
  consider the subsets
  \[ A= \{ 1, 3, 5, 7, 8, 9 \} \quad \text{and} \quad B=\{ x \in N \mid x\text{ is divisible by}\, 3\} \]
%    \[ A= \{ 1; 3; 5; 7; 8; 9 \} \quad \text{und} \quad B=\{ 3; 4; 6; 8; 10\} \]
  }
  \begin{enumerate}[alph]
  \item  \lang{de}{Schreiben Sie die Menge $\,B\,$ in eine aufzählende Darstellungform um und
        bestimmen Sie anschließend die Schnittmenge und die Vereinigungsmenge der 
        Mengen  $\,A\,$ und $\,B.$} 
        \lang{en}{Rewrite the set $\,B\,$ by listing its elements and
        then determine the intersection and union of the sets $\,A\,$ and $\,B.$. }
  \item  \lang{de}{Bestimmen Sie die Komplemente der Teilmengen $A$, $B$, $A\cup B$ und 
        $A\cap B$ in $N$. Welche Beziehung besteht zwischen diesen Komplementen?}
        \lang{en}{Determine the complements of the subsets $A$, $B$, $A\cup B$ and 
        $A\cap B$ in $N$. What is the relation between these complements?}
  \end{enumerate}
  
  \begin{tabs*}[\initialtab{0}\class{exercise}]
    \tab{ \lang{de}{Antworten}
          \lang{en}{Answers}}
      \begin{enumerate}[alph]
        \item \lang{de}{$B=\{ 3; 6; 9\}, \;$ 
              $A\cap B=\{3; 9 \}\;$ und 
              $\; A\cup B=\{ 1; 3; 5; 6; 7; 8; 9 \}.$}
              \lang{en}{$B=\{ 3, 6, 9\}, \;$ 
              $A\cap B=\{3, 9 \}\;$ und 
              $\; A\cup B=\{ 1, 3, 5, 6, 7, 8, 9 \}.$}
        \item \lang{de}{Die Komplemente sind:}
              \lang{en}{The set complements are:}
              \lang{de}{
          \begin{eqnarray*}
              \complement_N(A) &=& \{2; 4; 6; 10\}, \\
              \complement_N(B) &=& \{1; 2; 4; 5; 7; 8; 10 \}, \\
              \complement_N(A\cup B) &=& \{2; 4; 10 \}, \\
              \complement_N(A\cap B) &=& \{1; 2; 4; 5; 6; 7; 8; 10 \}.}
              \lang{en}{
          \begin{eqnarray*}
              \complement_N(A) &=& \{2, 4, 6, 10\}, \\
              \complement_N(B) &=& \{1, 2, 4, 5, 7, 8, 10 \}, \\
              \complement_N(A\cup B) &=& \{2, 4, 10 \}, \\
              \complement_N(A\cap B) &=& \{1, 2, 4, 5, 6, 7, 8, 10 \}.
          \end{eqnarray*}}
    
        \item \lang{de}{Das Komplement $\complement_N(A\cup B)$ ist gerade der Durchschnitt von
              $\complement_N(A)$ und $\complement_N(B)$ und das Komplement 
              $\complement_N(A\cap B)$ ist die Vereinigung von $\complement_N(A)$ 
              und $\complement_N(B)$, d.\,h. 
              \begin{eqnarray*}
                \complement_N(A\cup B) &=& \complement_N(A) \cap \complement_N(B), \\
                \complement_N(A\cap B) &=& \complement_N(A) \cup \complement_N(B).
              \end{eqnarray*}}
              \lang{en}{The complement $\complement_N(A\cup B)$ is the intersection of
              $\complement_N(A)$ and $\complement_N(B)$ and the complement 
              $\complement_N(A\cap B)$ is the union of $\complement_N(A)$ 
              and $\complement_N(B)$, i.e. 
              \begin{eqnarray*}
                \complement_N(A\cup B) &=& \complement_N(A) \cap \complement_N(B), \\
                \complement_N(A\cap B) &=& \complement_N(A) \cup \complement_N(B).
              \end{eqnarray*}}

      \end{enumerate}

    \tab{\lang{de}{Lösung a)}
         \lang{en}{Solution a)}
         }
     \begin{incremental}[\initialsteps{1}]
      \step 
       \lang{de}{Die Menge $N$ enthält alle \emph{natürlichen Zahlen} von $1$ bis $10$. Darunter 
        sind $3, 6\,$ und $\, 9\,$ die Zahlen, die durch $\,3\,$ teilbar sind. Daher
        ist \[B=\{ 3; 6; 9\}.\]}
        \lang{en}{The set $N$ contains all \emph{natural numbers} from $1$ to $10$. Among them 
        $3, 6\,$ and $\, 9\,$ are the numbers divisible by $\,3$. Therefore,
        \[B=\{ 3, 6, 9\}.\]}

       \step
        \lang{de}{Für den Durchschnitt und die Vereinigung der Mengen $A$ und $B$ gilt}
        \lang{en}{The intersection and the union of the sets $A$ and $B$ are}
        \lang{de}{
        \begin{eqnarray*} 
         A\cap B &=\{x\mid x\in A \text{ und } x\in B\} &= \{3; 9 \}, \\
         A\cup B &=\{x\mid x\in A \text{ oder } x\in B\} &= \{ 1; 3; 5; 6; 7; 8; 9 \}.
        \end{eqnarray*}}
        \lang{en}{
        \begin{eqnarray*} 
         A\cap B &=\{x\mid x\in A \text{ and } x\in B\} &= \{3, 9 \}, \\
         A\cup B &=\{x\mid x\in A \text{ or } x\in B\} &= \{ 1, 3, 5, 6, 7, 8, 9 \}.
        \end{eqnarray*}}
     \end{incremental}
     
    \tab{\lang{de}{Lösung b)}
        \lang{en}{Solution b)}}
     \begin{incremental}[\initialsteps{1}]
      \step 
       \lang{de}{Das Komplement einer
        Teilmenge besteht aus genau den Elementen, die in der umgebenden Obermenge 
        liegen, aber nicht in der Teilmenge selbst. Daher sind}
        \lang{en}{The complement of a
        subset consists of exactly those elements that lie in the surrounding superset 
        but not in the subset itself. Therefore}
      
        \lang{de}{
       \begin{align*}
         \complement_N(A) &\;=\;& N \setminus \{ 1; 3; 5; 7; 8; 9 \} &\;=\;&  \{2; 4; 6; 10\},\\
         \complement_N(B) &\;=\;& N \setminus \{ 3; 6; 9\} &\;=\;& \{1; 2; 4; 5; 7; 8; 10\},\\
         \complement_N(A\cap B) &\;=\;& N \setminus \{3; 9\} &\;=\;& \{1; 2; 4; 5; 6; 7; 8; 10 \} \quad \text{und}\\
         \complement_N(A\cup B) &\;=\;& N \setminus \{1; 3; 4; 5; 6; 7; 8; 9; 10\}&\;=\;& \{2; 4; 10 \}.
        \end{align*}}
        \lang{en}{
       \begin{align*}
         \complement_N(A) &\;=\;& N \setminus \{ 1, 3, 5, 7, 8, 9 \} &\;=\;&  \{2, 4, 6, 10\},\\
         \complement_N(B) &\;=\;& N \setminus \{ 3, 6, 9\} &\;=\;& \{1, 2, 4, 5, 7, 8, 10\},\\
         \complement_N(A\cap B) &\;=\;& N \setminus \{3, 9\} &\;=\;& \{1, 2, 4, 5, 6, 7, 8, 10 \} \quad \text{and}\\
         \complement_N(A\cup B) &\;=\;& N \setminus \{1, 3, 4, 5, 6, 7, 8, 9, 10\}&\;=\;& \{2, 4, 10 \}.
        \end{align*}}
        
    \step
       \lang{de}{Wenn man sich die Komplemente genau anschaut, kann man erkennen, dass 
       $\complement_N(A\cup B)$ genau der Durchschnitt von  $\complement_N(A)$ 
       und $\complement_N(B)$ ist und $\complement_N(A\cap B)$ genau deren Vereinigung
       ist, denn}
       \lang{en}{If you look closely at the complements, you will see that 
       $\complement_N(A\cup B)$ is exactly the intersection of $\complement_N(A)$ 
       and $\complement_N(B)$ and that $\complement_N(A\cap B)$ is exactly their union, because}

      \lang{de}{
    \begin{align*}
      \complement_N(A) \cap \complement_N(B) = \{2; 4; 6; 10\} \cap \{1; 2; 4; 5; 7; 8; 10\} &\,=\,& \{2; 4; 10 \} =  \complement_N(A\cup B) \quad \text{und}\\
      \complement_N(A) \cup \complement_N(B) = \{2; 4; 6; 10\} \cup \{1; 2; 4; 5; 7; 8; 10\} &\,=\,& \{1; 2; 4; 5; 6; 7; 8; 10 \} =  \complement_N(A\cap B) .
    \end{align*}}
    \lang{en}{
    \begin{align*}
      \complement_N(A) \cap \complement_N(B) = \{2, 4, 6, 10\} \cap \{1, 2, 4, 5, 7, 8, 10\} &\,=\,& \{2, 4, 10 \} =  \complement_N(A\cup B) \quad \text{and}\\
      \complement_N(A) \cup \complement_N(B) = \{2, 4, 6, 10\} \cup \{1, 2, 4, 5, 7, 8, 10\} &\,=\,& \{1, 2, 4, 5, 6, 7, 8, 10 \} =  \complement_N(A\cap B) .
    \end{align*}}
    
    \step
    \lang{de}{Dies ist nicht nur in dem vorliegenden Beispiel der Fall, sondern sogar allgemein, wie man sich 
    mit Hilfe eines Mengendiagramms klar machen kann:}
    \lang{en}{This is not only true for the example above. It holds in general, as one can see with the help of a Venn diagram:}
    \begin{center}
      \image{T601_01_Exercise01}
    \end{center}

    \lang{de}{$\complement_N(A)$ ist der Bereich in $N$ ohne $A$ und $\complement_N(B)$ ist der Bereich 
    in $N$ ohne $B$.
    Der Durchschnitt beider Bereiche ist genau der Bereich in $N$, der außerhalb von $A$ und außerhalb 
    von $B$ liegt,
    also das Komplement von $A\cup B$.\\
    Die Vereinigung beider Bereiche ist der Bereich in $N$, der außerhalb von $A$ oder außerhalb 
    von $B$ liegt, der also nicht von beiden Bereichen gleichzeitig bedeckt ist. Das ist also gleich 
    dem Komplement von $A\cap B$.}
    \lang{en}{$\complement_N(A)$ is the region in $N$ outside of $A$ and $\complement_N(B)$ is the region 
    in $N$ outside of $B$.
    The intersection of both regions is exactly the area in $N$ that lies outside of $A$ and outside 
    of $B$, which is the complement of $A\cup B$\\
    The union of both areas is the region in $N$ that lies either outside of $A$ or outside of 
    of $B$, i.e. that is not covered by both $A$ and $B$ at the same time. This is therefore equal to 
    the complement of $A\cap B$.}

   \end{incremental}
  \end{tabs*}

\end{content}

