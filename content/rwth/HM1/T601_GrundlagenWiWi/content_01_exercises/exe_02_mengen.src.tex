\documentclass{mumie.element.exercise}
%$Id$
\begin{metainfo}
  \name{
     \lang{de}{Ü02: Intervalle}
     \lang{en}{Ü02: Intervals}
  }
  \begin{description} 
 This work is licensed under the Creative Commons License Attribution 4.0 International (CC-BY 4.0)   
 https://creativecommons.org/licenses/by/4.0/legalcode 

    \lang{de}{Dies ist eine Übung zu Intervallen, Anordnungen}
    \lang{en}{This is an exercise about intervals and orderings}
  \end{description}
  \begin{components}
    \component{generic_image}{content/rwth/HM1/images/g_tkz_T601_01_Exercise02_G.meta.xml}{T601_01_Exercise02_G}
    \component{generic_image}{content/rwth/HM1/images/g_tkz_T601_01_Exercise02_E.meta.xml}{T601_01_Exercise02_E}
    \component{generic_image}{content/rwth/HM1/images/g_tkz_T601_01_Exercise02_D.meta.xml}{T601_01_Exercise02_D}
    \component{generic_image}{content/rwth/HM1/images/g_tkz_T601_01_Exercise02_F.meta.xml}{T601_01_Exercise02_F}
    \component{generic_image}{content/rwth/HM1/images/g_tkz_T601_01_Exercise02_C.meta.xml}{T601_01_Exercise02_C}
    \component{generic_image}{content/rwth/HM1/images/g_tkz_T601_01_Exercise02_A.meta.xml}{T601_01_Exercise02_A}
  \end{components}
  \begin{links}
    \link{generic_article}{content/rwth/HM1/T101neu_Elementare_Rechengrundlagen/g_art_content_02_rechengrundlagen_terme.meta.xml}{content_02_rechengrundlagen_terme}
    \link{generic_article}{content/rwth/HM1/T101neu_Elementare_Rechengrundlagen/g_art_content_01_zahlenmengen.meta.xml}{content_01_zahlenmengen}
  \end{links}
  \creategeneric
\end{metainfo}
\begin{content}
\title{\lang{de}{Ü02: Intervalle} \lang{en}{Exercise 2: Intervals}}

\begin{block}[annotation]
	Im Ticket-System: \href{https://team.mumie.net/issues/23724}{Ticket 23724}
\end{block}

\begin{block}[annotation]
Kopie: hm4mint/T101_neu_Elementare_Rechengraundlagen/exercise 2

Im Ticket-System: \href{https://team.mumie.net/issues/21972}{Ticket 21972}
\end{block}

  \begin{block}[annotation]
%
   Übung zu Intervallen, Anordnungen und Beträgen, Betragstermen und Wurzeltermen

  \end{block}


%
% Aufgabenstellung
%

\begin{enumerate}[alph] 

  \item \lang{de}{Was ist der Unterschied zwischen der Menge} \lang{en}{What is the difference between the set} $\,\{2\lang{de}{;}\lang{en}{,} 5\}\,$ \lang{de}{und dem Intervall} \lang{en}{and the interval} $\,[2\lang{de}{;}\lang{en}{,} 5)$?

  \item \lang{de}{Schreiben Sie die folgende Menge in Intervallschreibweise  
    und zeichnen Sie sie auf einem Zahlenstrahl ein:} \lang{en}{Write the following set in interval notation and draw it on a number line:} $A=\{x \in \R \mid \abs{x-1} \leq 1 \}$     
     
   \item \lang{de}{Ist der Durchschnitt zweier Intervalle wieder ein Intervall, sofern er nicht leer ist?
         Wie verhält es sich mit der Vereinigung zweier Intervalle?}
         \lang{en}{Is the intersection of two intervals always an interval, assuming it is non-empty?
         What about the union of two intervals?}
           
%
% Lösungen
%
  
\end{enumerate}  

  \begin{tabs*}[\initialtab{0}\class{exercise}]

    \tab{ \lang{de}{Lösung a)} \lang{en}{Solution a)}}
        \lang{de}{Die Menge} \lang{en}{The set} $\, \{2\lang{de}{;}\lang{en}{,} 5\} \,$ \lang{de}{besteht nur aus zwei Zahlen, nämlich der Zahl $2$ und der Zahl $5$.} \lang{en}{contains only two numbers: $2$ and $5$.} \\
        
        \lang{de}{Das Intervall} \lang{en}{The interval} $\,[2\lang{de}{;}\lang{en}{,} 5)\,$ \lang{de}{entspricht definitionsgemäß
        der Menge} \lang{en}{is defined as the set} $\, \{\,x\in\R\,|\,2 \leq x <5 \,\}\,$ \lang{de}{und enthält daher \emph{alle} reellen Zahlen zwischen
        $2$ und $5$ einschließlich der Zahl $2$, jedoch ohne die $5$.} \lang{en}{and it contains \emph{all} real numbers between 
        $2$ und $5$, including $2$, but excluding $5$.}
     
    \tab{\lang{de}{Lösung b)} \lang{en}{Solution b)}}
      \begin{incremental}[\initialsteps{1}]
       \step 
\lang{de}{Wir betrachten die Menge} \lang{en}{Consider the set} $\;A=\{x \in \R \mid \abs{x-1} \leq 1 \}.$ \\        
\lang{de}{Wir können $A$ als die Menge aller reellen Zahlen deuten, die von der Zahl $1$ den Abstand $1$ haben. 
        Dies liefert die Intervalldarstellung} \lang{en}{We can interpret $A$ as the set of all real numbers which are separated from $1$ by a distance of one. 
        This corresponds to the interval description}
        $\; A=[0\lang{de}{;}\lang{en}{,}2] \;$ \lang{de}{und folgende Darstellung auf dem Zahlenstrahl:} \lang{en}{and the following representation on the number line:} \\        
\\

        \begin{center}
           \image{T601_01_Exercise02_A} 
        \end{center}

    \end{incremental}

    \tab{\lang{de}{Lösung c)} \lang{en}{Solution c)}} 
     \begin{incremental}[\initialsteps{1}]
      \step 
        \lang{de}{Betrachten wir zwei beliebige Intervalle} \lang{en}{Consider two arbitrary intervals} $\, [a\lang{de}{;}\lang{en}{,}b] \,$ \lang{de}{und} \lang{en}{and} $\, [c\lang{de}{;}\lang{en}{,}d] \,$
        \lang{de}{mit} \lang{en}{, where} $\, a, b, c, d \in \R$, $a \leq b$ \lang{de}{und} \lang{en}{and} $c \leq d$. \lang{de}{Außerdem müssen} \lang{en}{Without loss of generality, we may assume that}
        $\, d \geq a\,$ \lang{de}{und} \lang{en}{and} $\, b \geq c $ \lang{de}{sein, da andernfalls} \lang{en}{as otherwise} 
        $\, [a\lang{de}{;}\lang{en}{,}b] \,$ \lang{de}{entweder links oder rechts von} \lang{en}{would be completely to the left or right of} $\, [c\lang{de}{;}\lang{en}{,}d]$ \lang{de}{ läge (und dann wäre 
        der Durchschnitt der Intervalle die leere Menge).} \lang{en}{, making the intersection of the intervals empty.}
        
        \\
         \begin{center}
            \image{T601_01_Exercise02_C}
        \end{center}
        \\
               
      \step
        \lang{de}{Unter diesen Bedingungen gibt es vier verschiedene Fälle zu betrachten.
        Dabei fixieren wir gedanklich das Intervall $\, [a;b] \,$ auf dem 
        Zahlenstrahl und untersuchen in Relation hierzu die verschiedenen Lagen
        des Intervalls $\, [c;d].$}
        \lang{en}{With these assumptions, we have four cases to consider. We fix the interval $\, [a,b] \,$ on the number line 
        and we consider the possible locations of the interval $\, [c,d]\,$ relative to it.}
        \\
        \\
  
        \begin{enumerate}
        
        \item[\lang{de}{1.Fall:} \lang{en}{Case 1:}] $\,c\,$ \lang{de}{liegt außerhalb des Intervalls} \lang{en}{is outside of the interval} $\, [a\lang{de}{;}\lang{en}{,}b],$ \lang{de}{also} \lang{en}{i.e.} $\, c < a,$\\
                       \lang{de}{und} \lang{en}{and} $\,d\,$ \lang{de}{liegt innerhalb, also} \lang{en}{is inside the interval, i.e.} $\, a \leq d \leq b \,$ \\
 
                      \\
                       \begin{center}
                          \image{T601_01_Exercise02_F}
                      \end{center}
                      \\
                      \lang{de}{\[
                        \Rightarrow \quad [c;d] \cap [a;b] = [a;d] 
                        \quad \text{ und } \quad [c;d] \cup [a;b] = [c;b]
                      \]}
                      \lang{en}{\[
                        \Rightarrow \quad [c,d] \cap [a,b] = [a,d] 
                        \quad \text{ and } \quad [c,d] \cup [a,b] = [c,b]
                      \]
                      }
        \end{enumerate}
    \step 
        \begin{enumerate}
        \item[\lang{de}{2.Fall:} \lang{en}{Case 2:}] $\,c\,$ \lang{de}{liegt außerhalb des Intervalls} \lang{en}{is outside of the interval} $\, [a\lang{de}{;}\lang{en}{,}b],$ \lang{de}{also} \lang{en}{i.e.} $\, c < a,$\\
                       \lang{de}{und} \lang{en}{and} $\,d\,$ \lang{de}{liegt ebenfalls außerhalb, also} \lang{en}{is also outside of the interval, i.e.} $\, d > b \,$ \\
 
                      \\
                       \begin{center}
                          \image{T601_01_Exercise02_E}
                      \end{center}
                      \\
                      \lang{de}{\[
                        \Rightarrow \quad [c;d] \cap [a;b] = [a;b] 
                        \quad \text{ und } \quad [c;d] \cup [a;b] = [c;d]
                      \]}
                      \lang{en}{\[
                        \Rightarrow \quad [c,d] \cap [a,b] = [a,b] 
                        \quad \text{ and } \quad [c,d] \cup [a,b] = [c,d]
                      \]}
        \end{enumerate}
    \step 
        \begin{enumerate}
       
        \item[\lang{de}{3.Fall:} \lang{en}{Case 3:}] $\,c\,$ \lang{de}{und} \lang{en}{and} $\,d\,$ \lang{de}{liegen beide innerhalb des Intervalls} \lang{en}{are both within the interval} $\, [a\lang{de}{;}\lang{en}{,}b],$ \\
                       \lang{de}{also} \lang{en}{i.e.} $\, a \leq c \leq b \,$ \lang{de}{und} \lang{en}{and} $\, a \leq d \leq b \,$ \\
                      
                      \\
                       \begin{center}
                          \image{T601_01_Exercise02_D}
                      \end{center}
                      \\
                      \lang{de}{\[
                        \Rightarrow \quad [c;d] \cap [a;b] = [c;d] 
                        \quad \text{ und } \quad [c;d] \cup [a;b] = [a;b]
                      \]}
                      \lang{en}{\[
                        \Rightarrow \quad [c,d] \cap [a,b] = [c,d] 
                        \quad \text{ and } \quad [c,d] \cup [a,b] = [a,b]
                      \]}
        \end{enumerate}
    \step 
        \begin{enumerate}

        \item[\lang{de}{4.Fall:} \lang{en}{Case 4:}] $\,c\,$ \lang{de}{liegt innerhalb des Intervalls} \lang{en}{is in the interval} $\, [a\lang{de}{;}\lang{en}{,}b],$ \lang{de}{also} \lang{en}{i.e.} $\, a \leq c \leq b,$\\
                       \lang{de}{und} \lang{en}{and} $\,d\,$ \lang{de}{liegt außerhalb, also} \lang{en}{is outside of the interval, i.e.} $\, d > b \,$ \\
 
                      \\
                       \begin{center}
                          \image{T601_01_Exercise02_G}
                      \end{center}
                      \\
                      \lang{de}{\[
                        \Rightarrow \quad [c;d] \cap [a;b] = [c;b]
                        \quad \text{ und } \quad [c;d] \cup [a;b] = [a;d]
                      \]}
                      \lang{en}{\[
                        \Rightarrow \quad [c,d] \cap [a,b] = [c,b]
                        \quad \text{ and } \quad [c,d] \cup [a,b] = [a,d]
                      \]}

                               
         \end{enumerate}
         \lang{de}{Sowohl der Durchschnitt, als auch die Vereinigung der Intervalle 
         $\, [a;b] \,$ und $\, [c;d] \,$ ergeben also stets wieder ein Intervall,
         sofern ihr Schnitt nicht leer ist.}
         \lang{en}{Therefore, both the intersection and the union of the intervals 
         $\, [a,b] \,$ und $\, [c,d] \,$ are again intervals, as long as the intersection is non-empty.}

     \step
        \lang{de}{
        Falls der Schnitt der Intervalle $\, [a;b] \,$ und $\, [c;d] \,$ leer ist,
        kann die Vereinigung der beiden Intervalle, aufgrund der \textit{"`Lücke"'} 
        zwischen $\,d\,$ und $\,a\,$ oder zwischen $\,b\,$ und $\,c\,$ auf dem
        Zahlenstrahl, \emph{kein} Intervall sein.}
        \lang{en}{If the intersection of $\, [a,b] \,$ and $\, [c,d] \,$ is empty, then the union of both intervals \emph{cannot} be an interval,
        due to the \textit{"gap"} between $\,d\,$ and $\,a\,$ or between $\,b\,$ and $\,c\,$, respectively.}

   \end{incremental}

  \end{tabs*}
  
\end{content}

