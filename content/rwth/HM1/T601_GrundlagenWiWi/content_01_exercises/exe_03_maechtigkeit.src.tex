\documentclass{mumie.element.exercise}
%$Id$
\begin{metainfo}
  \name{
    \lang{de}{Ü03: Mächtigkeit}
    \lang{en}{Ü03: Cardinality}
  }
  \begin{description} 
 This work is licensed under the Creative Commons License Attribution 4.0 International (CC-BY 4.0)   
 https://creativecommons.org/licenses/by/4.0/legalcode 

    \lang{de}{Mächtigkeit von Mengen}
    \lang{en}{Cardinality of sets}
  \end{description}
  \begin{components}
  \end{components}
  \begin{links}
  \end{links}
  \creategeneric
\end{metainfo}
\begin{content}
\title{
  \lang{de}{Ü03: Mächtigkeit}
  \lang{en}{Exercise 3: Cardinality}
}


\begin{block}[annotation]
	Im Ticket-System: \href{https://team.mumie.net/issues/23725}{Ticket 23725}
\end{block}
\begin{block}[annotation]
	Kopie: hm4mint/T204_Abbildungen_und_Funktionen/exercise 8
    
    Im Ticket-System: \href{https://team.mumie.net/issues/22806}{Ticket 22806}
\end{block}

\usepackage{mumie.ombplus}



  \lang{de}{Bestimmen Sie die Mächtigkeit folgender Mengen.}
  \lang{en}{Determine the cardinality of the following sets.}
  
  \begin{table}[\class{items}]
    \nowrap{a) \lang{de}{Die Lösungsmenge der Gleichung} \lang{en}{The solution set of the equation} $(x-3)^2 (x+91)^3 (x+17)=0$},\\
    \nowrap{b) $\{x\in \N $ $|$ $ x $ \lang{de}{ist keine Quadratzahl} \lang{en}{is not a square}$ \}$ },\\
    \nowrap{c) \lang{de}{$\{3;8;18;\frac{36}{3};\frac{44}{11};\frac{54}{18}\}.$} \lang{en}{$\{3,8,18,\frac{36}{3},\frac{44}{11},\frac{54}{18}\}.$}}
  \end{table}
  

  
  \begin{tabs*}[\initialtab{0}\class{exercise}]
    \tab{
      \lang{de}{Antwort} \lang{en}{Answers}} 
      a) $3$,\\
      b) $\infty$,\\
      c) $5$.    
    
    

    \tab{
      \lang{de}{Lösung a)} \lang{en}{Solution a)}}
    
      \lang{de}{Die Gleichung hat die folgenden Lösungen, die sich aus der Linearfaktozerlegung 
      direkt ablesen lassen:}
      \lang{en}{The solutions of the equation can be read off of the factorization directly:}
      \[ 3,-91,-17.\]
      \lang{de}{Die Lösungsmenge der Gleichung hat also 3 Elemente.}
      \lang{en}{The solution set therefore has 3 elements.}
    
    \tab{
      \lang{de}{Lösung b)} \lang{en}{Solution b)}}
      \lang{de}{$\{x\in \N $ $|$ $ x $ ist keine Quadratzahl$ \}$ ist die Menge aller natürlichen Zahlen bis auf 
       Quadratzahlen. Das heißt
      \[ \{x\in \N\, | \, x \text{ ist keine Quadratzahl} \}=\{2;3;5;6;7;8;10;11;...\}, \]
      was eine unendliche Menge ist.}
      \lang{en}{$\{x\in \N $ $|$ $ x $ is not a square$ \}$ is the set of all natural numbers other than squares:
      \[ \{x\in \N\, | \, x \text{ is not a square} \}=\{2,3,5,6,7,8,10,11,...\}. \]
      This set is infinite.}
      
    \tab{
      \lang{de}{Lösung c)} \lang{en}{Solution c)}}  
      \lang{de}{Die Menge \[\{3;8;18;\frac{36}{3};\frac{44}{11};\frac{54}{18}\}=\{3;8;18;12;4;3\}\]
      hat 5 Elemente, da die wiederholten Elemente nur einmal gezählt werden.}

      \lang{en}{The set \[\{3,8,18,\frac{36}{3},\frac{44}{11},\frac{54}{18}\}=\{3,8,18,12,4,3\}\]
      has 5 elements, since the repeated elements are only counted once.}
      
  \end{tabs*}
\end{content}