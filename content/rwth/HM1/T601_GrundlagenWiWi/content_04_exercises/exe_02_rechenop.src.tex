\documentclass{mumie.element.exercise}
%$Id$
\begin{metainfo}
  \name{
    \lang{de}{Ü02: Verknüpfungen von Funktionen}
    \lang{en}{Ü02: Operations on functions}
  }
  \begin{description} 
 This work is licensed under the Creative Commons License Attribution 4.0 International (CC-BY 4.0)   
 https://creativecommons.org/licenses/by/4.0/legalcode 

    \lang{de}{Hier die Beschreibung}
    \lang{en}{}
  \end{description}
  \begin{components}
  \end{components}
  \begin{links}
  \end{links}
  \creategeneric
\end{metainfo}
\begin{content}
\title{
\lang{de}{Ü02: Verknüpfungen von Funktionen}
\lang{en}{Exercise 2: Operations on functions}
}
\begin{block}[annotation]
	Im Ticket-System: \href{https://team.mumie.net/issues/23909}{Ticket 23909}
\end{block}

\begin{block}[annotation]

Kopie: hm4mint/T204_Abbildungen_und_Funktionen/exercise 6

  Im Ticket-System: \href{http://team.mumie.net/issues/9810}{Ticket 9810}
\end{block}


\lang{de}{Berechnen Sie für 
\begin{align*}
 f: & \R\to \R, \ f(x) = x^2+x, \\
 g: & \R\setminus\{0\} \to \R, \ g(x) = x^3 - \frac{1}{x}
\end{align*}
die Ausdrücke }

\lang{en}{Let 
\begin{align*}
 f: & \R\to \R, \ f(x) = x^2+x, \\
 g: & \R\setminus\{0\} \to \R, \ g(x) = x^3 - \frac{1}{x}.
\end{align*}
Calculate the expressions }

\begin{table}[\class{items}]
  \nowrap{a) $(f+g)(x)$,} &
  \nowrap{b) $(f-g)(x)$,} &
  \nowrap{c) $2\cdot f(x)$,} &
  \nowrap{d) $(f\cdot g)(x)$}
\end{table}
\lang{de}{und geben Sie die Funktion an (mit Definitionsbereich und Zielbereich).}
\lang{en}{and express the result as a function (including its domain and codomain).}

\begin{tabs*}[\initialtab{0}\class{exercise}]
  \tab{\lang{de}{Antwort}\lang{en}{Answer}} 
  
  \begin{table}[\class{items}]
  a) $f+g :\R\setminus\{0\} \to \R , \ (f+g)(x) = x^2+x+x^3-\frac{1}{x}$ \\
  b) $f-g :\R\setminus\{0\} \to \R , \ (f-g)(x) = x^2+x-x^3+ \frac{1}{x}$ \\
  c) $2\cdot f: \R \to \R , \ (2f)(x) = 2x^2+2x$ \\
  d) $f\cdot g: \R\setminus\{0\} \to \R ,  \ (f \cdot g)(x) = x^5 +x^4 -x-1 .$
  \end{table}
  
	\tab{\lang{de}{Lösung a)} \lang{en}{Solution a)}}
\lang{de}{Nach Definition ist die Summe der Funktionen dadurch bestimmt, dass man die Summe der
Funktionswerte berechnet, also ergibt sich
\[f+g :\R\setminus\{0\} \to \R , (f+g)(x) = f(x) +g(x)= x^2+x+x^3-\frac{1}{x} .\]
Der Definitionsbereich ist der Schnitt der Definitionsbereiche von $f$ und $g$.}
\lang{en}{By definition, the sum of the functions is given by summing the respective function values.
We find 
\[f+g :\R\setminus\{0\} \to \R , (f+g)(x) = f(x) +g(x)= x^2+x+x^3-\frac{1}{x} .\]
The domain is the intersection of the domains of $f$ and $g$.
}
	
    \tab{\lang{de}{Lösung b)} \lang{en}{Solution b)}}
	\[ f-g :\R\setminus\{0\} \to \R , (f-g)(x) = f(x)- g(x) = x^2+x-x^3+ \frac{1}{x} .\]
\lang{de}{Der Definitionsbereich ist der Schnitt der Definitionsbereiche von $f$ und $g$.}
\lang{en}{The domain is the intersection of the domains of $f$ and $g$.}
	
	\tab{\lang{de}{Lösung c)} \lang{en}{Solution c)}}
	\[ 2\cdot f: \R \to \R , (2f)(x) = 2\cdot(x^2+x)= 2x^2+2x. \]
\lang{de}{Der Definitionsbereich ist gleich dem der Funktion $f$.}
\lang{en}{The domain equals the domain of the function $f$.}

	\tab{\lang{de}{Lösung d)} \lang{en}{Solution d)}}
	\[ f\cdot g: \R\setminus\{0\} \to \R , (f\cdot g)(x) = f(x) \cdot g(x) = (x^2+x)(x^3-\frac{1}{x})= x^5 +x^4 -x-1 . \]
\lang{de}{Der Definitionsbereich ist der Schnitt der Definitionsbereiche von $f$ und $g$.}
\lang{en}{The domain is the intersection of the domains of $f$ and $g$.}
	
\end{tabs*}
\end{content}