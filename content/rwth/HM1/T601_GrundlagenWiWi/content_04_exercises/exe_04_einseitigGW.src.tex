\documentclass{mumie.element.exercise}
%$Id$
\begin{metainfo}
  \name{
    \lang{de}{Ü04: rechts- und linksseitige Grenzwerte}
    \lang{en}{Exercise 4: Right- and left-handed limits}
  }
  \begin{description} 
 This work is licensed under the Creative Commons License Attribution 4.0 International (CC-BY 4.0)   
 https://creativecommons.org/licenses/by/4.0/legalcode 

    \lang{de}{Hier die Beschreibung}
    \lang{en}{}
  \end{description}
  \begin{components}
  \end{components}
  \begin{links}
  \end{links}
  \creategeneric
\end{metainfo}
\begin{content}

\title{\lang{de}{Ü04: rechts- und linksseitige Grenzwerte}
    \lang{en}{Exercise 4: Right- and left-handed limits}}
\begin{block}[annotation]
	Im Ticket-System: \href{https://team.mumie.net/issues/23910}{Ticket 23910}
\end{block}

\begin{block}[annotation]
Kopie: hm4mint/T210_Stetigkeit/exercise 3

Im Ticket-System: \href{http://team.mumie.net/issues/9987}{Ticket 9987}
\end{block}
 
\lang{de}{Gegeben sei die Funktionsvorschrift $\displaystyle{f(x)=\frac{x^2-9}{x^2+x-12}}$.
\begin{enumerate}[a)]
\item a) Bestimmen Sie den maximalen Definitionsbereich.
\item b) Bestimmen Sie den links- und rechtsseitigen Grenzwert an den Nullstellen des Nenners.
\end{enumerate}}

\lang{en}{Given the function rule $\displaystyle{f(x)=\frac{x^2-9}{x^2+x-12}}$.
\begin{enumerate}[a)]
\item a) Determine the maximal domain.
\item b) Determine the left-handed and right-handed limits at the zeros of the denominator.
\end{enumerate}}

\begin{tabs*}[\initialtab{0}\class{exercise}]

  \tab{
  \lang{de}{Antwort } \lang{en}{Answer}}
 
\begin{enumerate}[a)]
 \item a) \lang{de}{Es ist} $D_{f}=\R\setminus\{-4;3\}$.
 \item b) \lang{de}{Es gilt}
 \begin{align*}
  &\lim_{x\searrow -4} f(x) = - \infty, \\
  &\lim_{x\nearrow -4} f(x) = \infty,\\
  &\lim_{x\searrow 3} f(x) = \frac{6}{7},\\
  &\lim_{x\nearrow 3} f(x) =\frac{6}{7}.
 \end{align*}

\end{enumerate}

\tab{
  \lang{de}{Lösung a}
  \lang{en}{Solution a)}}
  
  \lang{de}{Da $f$ eine gebrochen rationale Funktion ist, ist der Definitionsbereich gleich $\R$ ohne die Nullstellen des Nenners. Wir bestimmen daher die Nullstellen des         
       Nennerpolynoms mit der pq-Formel und erhalten:}
  \lang{en}{Since $f$ is a quotient of polynomials, the maximal domain is $\R$ with the zeros of the denominators removed.
  We find the zeros of the denominator polynomial with the quadratic formula:}
       \begin{align*}
        x_{1,2} &= -\frac{1}{2} \pm \sqrt{\frac{1}{4} + 12} & \\
                 &= -\frac{1}{2} \pm \sqrt{\frac{49}{4} } & \\
                 &= -\frac{1}{2} \pm {\frac{7}{2} }, & \\
       \end{align*}
       \lang{de}{also sind die Nullstellen $x_1 = -\frac{1}{2} + {\frac{7}{2} } = 3 $ und  $x_2 = -\frac{1}{2} - {\frac{7}{2}} = -4$.  
       Es ist folglich $D_{f}=\R\setminus\{-4;3\}$.}
       \lang{en}{so the zeros are $x_1 = -\frac{1}{2} + {\frac{7}{2} } = 3 $ and  $x_2 = -\frac{1}{2} - {\frac{7}{2}} = -4$.  
       It follows that $D_{f}=\R\setminus\{-4,3\}$.}
 \tab{
  \lang{de}{Lösung b}
  \lang{en}{Solution b)}}
  
  \lang{de}{Die Funktion $f$ vereinfachen wir, indem wir Zähler und Nenner in Linearfaktoren zerlegen. 
  Die Nullstellen des Nenners kennen wir schon aus Teil a) und im Zähler können wir die 3. binomische Formel anwenden:}
  \lang{en}{The function $f$ can be simplified by factoring the numerator and denominator into linear polynomials.
  The zeros of the denominator were found in (a), and we can use the third binomial formula to find
  the zeros of the numerator:}
       \begin{align}
        f(x)=\frac{x^2-9}{x^2+x-12}=\frac{(x-3)(x+3)}{(x-3)(x+4)}=\frac{x+3}{x+4}.  \label{2}
       \end{align}
       \lang{de}{Wir bestimmen nun die links- und rechtsseitigen Grenzwerte an den Stellen $-4$ und $3$:
        Es gilt }
        \lang{en}{Now we find the left-handed and right-handed limits at the points $-4$ and $3$. We have:}
  \[
   \lim_{x\nearrow-4}(x+4)=\lim_{x\searrow-4}(x+4)=0
  \]
 \lang{de}{und} \lang{en}{and} 
  \[
   \lim_{x\nearrow-4}(x+3)=\lim_{x\searrow -4}(x+3)=-1<0.
  \]
 \lang{de}{Da $x+4<0$ für $x<-4$ sowie $x+4>0$ für $x>-4$ gilt, erhalten wir }
 \lang{en}{Since $x+4<0$ for $x < -4$, and $x+4>0$ for $x > -4$, we obtain}
              \begin{align*}
               &\lim_{x\searrow -4}f(x) \underset{(1)}{=}\lim_{x\searrow -4} \frac{x+3}{x+4} = -\infty,\\
               &\lim_{x\nearrow -4}f(x) \underset{(1)}{=}\lim_{x\nearrow -4} \frac{x+3}{x+4} = +\infty.
              \end{align*}
       \lang{de}{Weiter gilt} \lang{en}{Also,}
	\begin{align*}
	 &\lim_{x\searrow 3}(x+3)=6=\lim_{x\nearrow 3}(x+3),\\
	 &\lim_{x\searrow 3}(x+4)=7=\lim_{x\nearrow 3}(x+4),
	\end{align*}
       \lang{de}{also} \lang{en}{and therefore}
	\[
	  \lim_{x\searrow 3}f(x) \underset{(1)}{=}\lim_{x\searrow 3} \frac{x+3}{x+4} = \frac{6}{7} = \lim_{x\nearrow 3}f(x).
	\] 
 
     \tab{\lang{de}{Videos: ähnliche Übungsaufgaben}
     \lang{en}{Videos: Similar exercises}}	
    \youtubevideo[500][300]{7FNmcJ3Vbtk}\\
 
 
\end{tabs*}

\end{content}