\documentclass{mumie.element.exercise}
%$Id$
\begin{metainfo}
  \name{
    \lang{de}{Ü01: Abbildungen}
    \lang{en}{Ü01: Functions}
  }
  \begin{description} 
 This work is licensed under the Creative Commons License Attribution 4.0 International (CC-BY 4.0)   
 https://creativecommons.org/licenses/by/4.0/legalcode 

    \lang{de}{Hier die Beschreibung}
    \lang{en}{}
  \end{description}
  \begin{components}
   \component{generic_image}{content/rwth/HM1/images/g_tkz_T601_04_Exercise01.meta.xml}{T601_04_Exercise01}
  \end{components}
  \begin{links}
  \end{links}
  \creategeneric
\end{metainfo}
\begin{content}
\title{\lang{de}{Ü01: Abbildungen}
    \lang{en}{Exercise 1: Functions}}
\begin{block}[annotation]
	Im Ticket-System: \href{https://team.mumie.net/issues/23831}{Ticket 23831}
\end{block}

\begin{block}[annotation]
Kopie: hm4mint/T204_Abbildungen_und_Funktionen/exercise 2

  Im Ticket-System: \href{http://team.mumie.net/issues/9806}{Ticket 9806}
\end{block}

\begin{center}
\image{T601_04_Exercise01}
\end{center}

\lang{de}{Welche Funktionen sind in der Graphik dargestellt? Welche dieser Funktionen lassen sich verketten?
Bestimmen Sie auch alle möglichen verketteten Funktionen.}

\lang{en}{What are the functions represented by this image? Which of these functions can be composed?
Determine all possible compositions of these functions.

}

\begin{tabs*}[\initialtab{0}\class{exercise}]
  \tab{\lang{de}{Antwort} \lang{en}{Answer}} 
  \lang{de}{Die Mengen $M$ (blau), $N$ (rot), $P$ (gelb) und $R$ (gr"un) sind gegeben durch
  \[ M=\{ 0;-1;3;7\},\quad N=\{0;1;2;4\},\quad P=\{-3;2;4\}\quad \text{und}\quad  R=\{1;9\}.\]}
  \lang{en}{The sets $M$ (blue), $N$ (red), $P$ (yellow) und $R$ (green) are
  \[ M=\{ 0,-1,3,7\},\quad N=\{0,1,2,4\},\quad P=\{-3,2,4\}\quad \text{and}\quad  R=\{1,9\}.\]}
\begin{itemize}
\item \lang{de}{Die durchgezogenen Pfeile stellen eine Funktion $T_1:M\to N$ dar mit}
\lang{en}{The solid arrows represent the function $T_1 : M \to N$ with values}
	\[ T_1(0)=1, \, T_1(-1)=4, \, T_1(7)=4, \, T_1(3)=2. \]
	\item \lang{de}{Die gestrichelten Pfeile stellen eine Funktion $T_2:P\to M$ dar mit}
 \lang{en}{The dashed arrows represent the function $T_2 : M \to N$ with values}
	\[ T_2(2)=-1, \, T_2(4)=3, \, T_2(-3)=0. \]
	\item \lang{de}{Die gepunkteten Pfeile stellen eine Funktion $T_3:P\to R$ dar mit}
 \lang{en}{The dotted arrows represent the function $T_3 : P \to R$ with values}
	\[ T_3(2)=9, \, T_3(4)=1, \, T_3(-3)=9. \]
	\item \lang{de}{Die Strichpunkt-Pfeile stellen eine Funktion $T_4:R\to N$ dar mit}
 \lang{en}{The arrows with alternating dots and dashes represent the function $T_4:R \to N$ with values}
	\lang{de}{\[ T_4(1)=2\, \text{ und }\, T_4(9)=1. \]}
 \lang{en}{\[ T_4(1)=2\, \text{ and }\, T_4(9)=1. \]}
\end{itemize}
\lang{de}{Die Funktionen $T_2:P\to M$ und $T_1:M\to N$ lassen sich zu einer Funktion $T_1\circ T_2:P\to N$
zusammensetzen, und die Funktionen $T_3:P\to R$ und $T_4:R\to N$ lassen sich zu einer Funktion 
$T_4\circ T_3:P\to N$ verketten.}
\lang{en}{The functions $T_2:P\to M$ and $T_1:M\to N$ can be composed to a function $T_1\circ T_2:P\to N$,
and the functions $T_3:P\to R$ and $T_4:R\to N$ can be composed to a function $T_4\circ T_3:P\to N$.}

\lang{de}{Hierbei sind die Zuordnungen für die Funktion $T_1\circ T_2:P\to N$ gegeben als
\[ 2\mapsto 4,\,  4\mapsto 2 \text{ und }-3\mapsto 1. \]
Die Zuordnungen für die Funktion $T_4\circ T_3:P\to N$ sind gegeben als
\[ 2\mapsto 1,\,  4\mapsto 2 \text{ und }-3\mapsto 1. \]}
\lang{en}{Here, the function $T_1\circ T_2:P\to N$ is given by the values
\[ 2\mapsto 4,\,  4\mapsto 2 \text{ and }-3\mapsto 1. \]
The function $T_4\circ T_3:P\to N$ is given by the values
\[ 2\mapsto 1,\,  4\mapsto 2 \text{ and }-3\mapsto 1. \]
}


   \tab{\lang{de}{Lösung zu den Funktionen}
   \lang{en}{Solution: the functions}}
  
  \begin{incremental}[\initialsteps{1}]
  \step \lang{de}{Definitionsbereich der Funktionen ist stets der Bereich, in dem die Pfeile starten, und 
  Zielbereich ist der Bereich, in dem die Pfeile enden.}
  \lang{en}{The domain of any of the functions is the region where the arrows begin.
  The codomain is the region where the arrows end.}
  \step \lang{de}{Die durchgezogenen Pfeile stellen daher eine Zuordnung $T_1:M\to N$ dar, welche 
  in der Tat eine Funktion ist, weil bei jedem Element in $M$ genau ein Pfeil startet, der in $N$ endet.\\
  Ebenso stellen die gestrichelten Pfeile eine Funktion $T_2:P\to M$ dar, die
  gepunkteten Pfeile eine Funktion $T_3:P\to R$ und die Strichpunkt-Pfeile eine 
  Funktion $T_4:R\to N$.}
  \lang{en}{Therefore, the solid lines represent a function $T_1:M\to N$.
  This indeed defines a function, because every element of $M$ is the start of exactly one arrow
  that ends in $N$. Simimlarly, the dashed lines define a function $T_2 : P \to M$,
  the dotted lines define a function $T_3:P\to R$, and the alternating lines
  define a function $T_4:R\to N$.
  }
  \step \lang{de}{Um angeben zu können, auf welche Elemente aus den Zielbereichen die Elemente aus den
  Definitionsbereichen abgebildet werden, muss man lediglich den Pfeilen folgen. Man sieht daher,
  dass für die Funktion $T_1:M\to N$ gilt:}
  \lang{en}{To determine how the elements of the domain are mapped to elements
  of the codomain, we simply follow the arrows. For example, under the function $T_1:M\to N$,}
  \[ 0\mapsto 1, \, -1\mapsto 4, \, 7\mapsto 4, \, 3\mapsto 2. \]
  \lang{de}{Oder anders ausgedrückt:} \lang{en}{In other words:}
  \[ T_1(0)=1, \, T_1(-1)=4, \, T_1(7)=4, \, T_1(3)=2. \]
  \step \lang{de}{Für die anderen Funktionen erhält man entsprechend:}
  \lang{en}{For the other functions, we find:}
  \[ T_2(2)=-1, \, T_2(4)=3, \, T_2(-3)=0, \]
  \[ T_3(2)=9, \, T_3(4)=1, \, T_3(-3)=9 \]
  \lang{de}{sowie} \lang{en}{and}
  \lang{de}{\[ T_4(1)=2\, \text{ und }\, T_4(9)=1. \]}
  \lang{en}{\[ T_4(1)=2\, \text{ and }\, T_4(9)=1. \]}
  \end{incremental}

   \tab{\lang{de}{Lösung zur Verkettung der Funktionen}
   \lang{en}{Solution: Compositions}}
  
  \begin{incremental}[\initialsteps{1}]
  \step \lang{de}{Zwei Funktionen lassen sich genau dann verketten, wenn der Zielbereich der einen
  Funktion gleich dem Definitionsbereich der anderen Funktion ist. Oder in der Graphik 
  ausgedrückt: Wenn dort, wo die einen Pfeile enden, andere Pfeile starten.}
  \lang{en}{Two functions can be composed if and only if the codomain of the first function
  equals the domain of the second. In terms of the figure: if and only if, at every point where an arrow ends,
  another arrow begins.
  }
  \step \lang{de}{Im vorliegenden Beispiel lassen sich also die Funktionen $T_2:P\to M$ und $T_1:M\to N$
  zusammensetzen, sowie die Funktionen $T_3:P\to R$ und $T_4:R\to N$. Beide Verkettungen
  ergeben dann Funktionen von $P$ nach $N$.}
  \lang{en}{In the example above, the functions $T_2:P\to M$ and $T_1:M\to N$ can be composed,
  and the functions $T_3:P\to R$ and $T_4:R\to N$ can also be composed.
  Both compositions then define functions from $P$ to $N$.}
  \step \lang{de}{Die Funktion $T_1\circ T_2:P\to N$ erfüllt dann:}
  \lang{en}{The function $T_1 \circ T_2:P\to N$ satisfies:}
  \[ (T_1\circ T_2)(2)=T_1(T_2(2))=T_1(-1)=4, \]
  \lang{de}{d.\,h. $2$ wird auf $4$ abgebildet. Entsprechend sind}
  \lang{en}{i.e. $2$ is mapped to $4$. Similarly,}
  \begin{align*}
  (T_1\circ T_2)(4) &=& T_1(T_2(4))=T_1(3)=2, \\
  (T_1\circ T_2)(-3) &=& T_1(T_2(-3))=T_1(0)=1.
  \end{align*}
  \lang{de}{Für die Funktion $T_4\circ T_3:P\to N$ erhält man}
  \lang{en}{For the function $T_4\circ T_3:P\to N$, we find}
  \begin{align*}
  (T_4\circ T_3)(2) &=& T_4(T_3(2))=T_4(9)=1,\\
  (T_4\circ T_3)(4) &=& T_4(T_3(4))=T_4(1)=2,\\
  (T_4\circ T_3)(-3) &=& T_4(T_3(-3))=T_4(9)=1.  
  \end{align*}
  \end{incremental}


\end{tabs*}
\end{content}