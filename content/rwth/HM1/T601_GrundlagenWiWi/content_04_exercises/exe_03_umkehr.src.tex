\documentclass{mumie.element.exercise}
%$Id$
\begin{metainfo}
  \name{
    \lang{de}{Ü03: Umkehrfunktion}
    \lang{en}{Exercise 3: Inverse function}
  }
  \begin{description} 
 This work is licensed under the Creative Commons License Attribution 4.0 International (CC-BY 4.0)   
 https://creativecommons.org/licenses/by/4.0/legalcode 

    \lang{de}{Hier die Beschreibung}
    \lang{en}{}
  \end{description}
  \begin{components}
  \end{components}
  \begin{links}
  \end{links}
  \creategeneric
\end{metainfo}
\begin{content}
\title{\lang{de}{Ü03: Umkehrfunktion}
    \lang{en}{Exercise 3: Inverse function}}
\begin{block}[annotation]
	Im Ticket-System: \href{https://team.mumie.net/issues/23830}{Ticket 23830}
\end{block}
\begin{block}[annotation]
Kopie: hm4mint/T204_Abbildungen_und_Funktionen/exercise 5

  Im Ticket-System: \href{http://team.mumie.net/issues/9809}{Ticket 9809}
\end{block}


\lang{de}{Bestimmen Sie die Wertemenge der Funktionen und bilden Sie die Umkehrfunktion.}
\lang{en}{Determine the range of the functions and find the inverse function.}

\begin{table}[\class{items}]
  \nowrap{a) $\, f:\R\to \R, \; f(x) = 2x+1,$} \\
  \nowrap{b) $\, g:\left.\left[\frac{3}{4},\infty\right.\right) \to \R, \; g(x) = 2x^2-3x+1,$}\\
  \nowrap{c) $\, h: \R\setminus\{1\} \to \R, \; h(x) = 3- \frac{1}{x-1}.$}
\end{table}

\begin{tabs*}[\initialtab{0}\class{exercise}]
  \tab{\lang{de}{Antwort}\lang{en}{Answer}} 
  
  \begin{table}[\class{items}]
  a) $\, W_f =\R$, & $f^{-1}:\R\to\R, \; f^{-1}(y) = \frac{y-1}{2},$ \\
  b) $\, W_g = \left.\left[-\frac{1}{8},\infty\right. \right), \;$ & 
     $g^{-1}:\left.\left[-\frac{1}{8},\infty\right. \right) \to 
     \left.\left[\frac{3}{4},\infty\right.\right) , \; g^{-1}(y) = \sqrt{\frac{1}{2}(y+\frac{1}{8})}+\frac{3}{4}$,\\
  c) $\, W_h= \R\setminus\{3\}, \; $ & $h^{-1}:  \R\setminus\{3\} \to \R\setminus\{1\}, 
    \; h^{-1}(y) = \frac{1}{3-y}+1$.
  \end{table}
  
	\tab{\lang{de}{Lösung a)} \lang{en}{Solution a)}}
    \lang{de}{$f$ ist surjektiv, denn für jedes $y\in\R$ ist $f(x)=y \,$
    äquivalent zu $\, 2x+1=y$, was wir nach $x$ auflösen können. 
    Wir erhalten dann $\,x= \frac{y-1}{2}$, womit wir direkt die 
    Umkehrfunktion bestimmt haben. }
    \lang{en}{$f$ is onto, because for each $y \in \R$, the equation $f(x)=y\,$
    is equivalent to $\, 2x+1=y$ and can be solved for $x$. We obtain
    $\,x = \frac{y-1}{2}$, which yields the inverse function.}

	\tab{\lang{de}{Lösung b)} \lang{en}{Solution b)}}
	\begin{incremental}[\initialsteps{1}]
  	\step 
		\lang{de}{Der Graph der Funktion $g$ ist eine nach oben geöffnete Parabel. Wir bestimmen die Scheitelpunktsform}
  \lang{en}{The graph of the function $g$ is a parabola that opens upwards. We write the equation of the parabola in vertex form:}
\begin{align*}
2x^2-3x+1 &= 2(x^2-\frac{3}{2}x)+1 \\
&= 2(x^2-\frac{3}{2}x +(\frac{3}{4})^2-(\frac{3}{4})^2 ) +1 \\
&= 2 (( x- \frac{3}{4})^2-\frac{9}{16})+1 \\
&= 2(x-\frac{3}{4})^2 - \frac{1}{8}.
\end{align*}
\lang{de}{Damit können wir den Scheitelpunkt $S=\left(\frac{3}{4}, - \frac{1}{8} \right)$ ablesen und erhalten, da die Parabel nach oben geöffnet ist, die Wertemenge $W_g= \left.\left[-\frac{1}{8},\infty\right. \right)$.}
\lang{en}{We can read off of the equation that the parabola has vertex $S=\left(\frac{3}{4}, - \frac{1}{8} \right)$, that it opens upwards, and that its range is $W_g= \left.\left[-\frac{1}{8},\infty\right. \right)$.}
	\step
		\lang{de}{Um die Umkehrfunktion zu finden, setzen wir $y=g(x)$ und lösen nach x auf:}
  \lang{en}{To find the inverse function, we set $y=g(x)$ and solve for $x$:}
\begin{align*}
&&\quad y = 2(x-\frac{3}{4})^2 - \frac{1}{8} \\
&\Leftrightarrow &\quad  y+ \frac{1}{8}= 2(x-\frac{3}{4})^2 \\
&\Leftrightarrow &\quad  \frac{1}{2} (  y+ \frac{1}{8}) = (x-\frac{3}{4})^2 \\
&\Rightarrow     &\quad  \sqrt{\frac{1}{2} (  y+ \frac{1}{8})} = x-\frac{3}{4} \\
&\Leftrightarrow &\quad  x = \sqrt{\frac{1}{2} (  y+ \frac{1}{8})} +\frac{3}{4},
\end{align*}
\lang{de}{womit wir die Umkehrfunktion direkt erhalten haben.}
\lang{en}{yielding the inverse function directly.}
		\end{incremental}
	
	\tab{\lang{de}{Lösung c)} \lang{en}{Solution c)}}
    \lang{de}{Der Wertebereich ergibt sich zu $\, W_h= \R\setminus\{3\}$, da der Bruch $\frac{1}{x-1}$ jeden Wert 
    außer $0$ annehmen kann.\\ }
    \lang{en}{The range is $\, W_h= \R\setminus\{3\}$, because the fraction $\frac{1}{x-1}$ takes
    every value other than $0$.\\ }
	\lang{de}{Die Umkehrfunktion ergibt sich durch Umformen der gegebenen Funktion nach $x$:}
 \lang{en}{The inverse function is obtained by solving the equation $y=h(x)$ for $x$:}
  \begin{align*}
   3-\frac{1}{x-1}=y \quad &\Leftrightarrow &\quad 3(x-1)-1=y(x-1)\\ 
                      &\Leftrightarrow &\quad 3x-3-1-yx=-y\\ 
                      &\Leftrightarrow &\quad (3-y)x=3-y+1\\ 
                      &\Leftrightarrow &\quad x=1+\frac{1}{3-y}.
  \end{align*}

	
\end{tabs*}
\end{content}