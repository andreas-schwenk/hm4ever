\documentclass{mumie.problem.gwtmathlet}
%$Id$
\begin{metainfo}
  \name{
    \lang{de}{A02: Summen- und Produktzeichen}
    \lang{en}{sum and product signs}
  }
  \begin{description} 
 This work is licensed under the Creative Commons License Attribution 4.0 International (CC-BY 4.0)   
 https://creativecommons.org/licenses/by/4.0/legalcode 

    \lang{de}{Die Beschreibung}
    \lang{en}{}
  \end{description}
  \corrector{system/problem/GenericCorrector.meta.xml}
  \begin{components}
    \component{js_lib}{system/problem/GenericMathlet.meta.xml}{mathlet}
  \end{components}
  \begin{links}
  \end{links}
  \creategeneric
\end{metainfo}
\begin{content}
\lang{de}{\title{A02: Summen- und Produktzeichen}}
\lang{en}{\title{A02: sum and product signs}}

\begin{block}[annotation]
	Im Ticket-System: \href{https://team.mumie.net/issues/23784}{Ticket 23784}
\end{block}

\begin{block}[annotation]
Kopie: hm4mint/T101neu_Elementare_Rechengraundlagen/training 5

Im Ticket-System: \href{https://team.mumie.net/issues/22287}{Ticket 22287}
\end{block}

\usepackage{mumie.genericproblem}


\begin{problem}
%
\randomquestionpool{1}{2}

%       
%Q1
	\begin{question}
	\lang{de}{\text{Berechnen Sie die folgenden Zahlenwerte und vereinfachen Sie Ihr Ergebnis 
        so weit wie möglich. (Fassen sie also vollständig zusammen!)\\
        \\
        
        $\sum_{k=0}^\var{n} \var{qk} =$\ansref                              %Q1.1
        $\qquad \sum_{l=\var{nb}}^\var{mb} \var{zb} = $\ansref  \\          %Q1.2
        \\
        
        $\prod_{j=2}^\var{nc} \left( \frac{1}{j+\var{mc}} \right) = $ \ansref       %Q1.3
        $\qquad \prod_{k=1}^\var{md} \frac{k+\var{id}}{k+\var{id1}} = $ \ansref     %Q1.4
        }}
    \lang{en}{\text{Calculate the following numerical values and simplify your result 
        as far as possible. (Simplify it completely!)\\
        \\
        
        $\sum_{k=0}^\var{n} \var{qk} =$\ansref                             %Q1.1
        $\qquad \sum_{l=\var{nb}}^\var{mb} \var{zb} = $\ansref  \\         %Q1.2
        \\
        
        $\prod_{j=2}^\var{nc} \left( \frac{1}{j+\var{mc}} \right) = $ \ansref     %Q1.3
        $\qquad \prod_{k=1}^\var{md} \frac{k+\var{id}}{k+\var{id1}} = $ \ansref   %Q1.4
        }}
	
	\begin{variables}
    %Q1.1 
    	\randint{qh}{-6}{6}
		\randadjustIf{qh}{qh=0 OR qh=2}
		\function[calculate]{q}{qh/2}
		\randint{n}{3}{8}
		\function[calculate]{f}{(q^(n+1)-1)/(q-1)}
		\function{qk}{q^k}

    %Q1.2
        \randint{nb}{1}{5}
        \randint{mb}{10}{20}
        \randint{zb}{2}{8}
        \function[calculate]{fb}{zb * (mb-nb+1)}
        
    %Q1.3
        \randint{nc}{2}{6}
        \randint{mc}{1}{5}	
        \function[calculate]{fc}{(mc+1)!/(mc+nc)!}
        
    %Q1.4        
        \randint{md}{3}{7}
        \randint{id}{0}{4}
        \function[calculate]{id1}{id+1}
        \function[calculate]{fd}{(1+id)/(md+1+id)}
    
	\end{variables}
    
	\type{input.number} 
    \field{rational}
        
    %Q1.1	
		\begin{answer}
%			\text{ $\sum_{k=0}^\var{n} \var{qk} =$}
			\solution{f}
		\end{answer}
    
    %Q1.2		
		\begin{answer}
%			\text{$\sum_{l=\var{nb}}^\var{mb} \var{zb} = $}
			\solution{fb}
		\end{answer}
        
    %Q1.3
    	\begin{answer}
%			\text{$\prod_{j=2}^\var{nc} \left( \frac{1}{j+\var{mc}} \right) = $}
			\solution{fc}
		\end{answer}
        
    %Q1.4		
		\begin{answer}
%			\text{$\prod_{k=1}^\var{md} \frac{k+\var{id}}{k+\var{id1}} = $}
			\solution{fd}
		\end{answer}
	
	\end{question}

%
%Q2	
	\begin{question}
	\lang{de}{\text{Berechnen Sie die folgenden Zahlenwerte und vereinfachen Sie Ihr Ergebnis 
        so weit wie möglich. (Fassen sie also vollständig zusammen!)\\
        \\
        
        $\sum_{j=1}^\var{na}  \var{ma} \cdot j \cdot (j+\var{za}) = $\ansref        %Q2.1
        $\qquad \sum_{k=0}^\var{n} \var{qk} =$\ansref   \\                          %Q2.2
        \\
        
        $\prod_{j=2}^\var{nc} \left( 1 - \frac{2}{j^2+j} \right) = $ \ansref        %Q2.3
        $\qquad \; \prod_{k=1}^\var{md} \frac{k+\var{id}}{k+\var{id1}} = $ \ansref     %Q2.4 
        }}
    \lang{en}{\text{Calculate the following numerical values and simplify your result 
        as far as possible. (Simplify it completely!)\\
        \\
        
        $\sum_{j=1}^\var{na}  \var{ma} \cdot j \cdot (j+\var{za}) = $\ansref     %Q2.1
        $\qquad \sum_{k=0}^\var{n} \var{qk} =$\ansref \\                          %Q2.2
        \\
        
        $\prod_{j=2}^\var{nc} \left( 1 - \frac{2}{j^2+j} \right) = $ \ansref       %Q2.3
        $\qquad \; \prod_{k=1}^\var{md} \frac{k+\var{id}}{k+\var{id1}} = $ \ansref %Q2.4 
        }}
	
    \begin{variables}
    
    %Q2.1
        \randint{ma}{2}{9}
        \randint{na}{1}{5}
        \randint{za}{1}{3}								
        \function[calculate]{fa}{na*(na+1)*1/2 * ( (2*na+1)*1/3*ma + ma*za )}
        
    %Q2.2
		\randint{q}{-6}{6}
		\randadjustIf{q}{q=0}
        \randadjustIf{q}{q=1} %hat andere Lösung
		\randint{n}{0}{2}
		\function[calculate]{f}{(q^(n+1)-1)/(q-1)}       
		\function{qk}{q^k}
                
    %Q2.3
        \randint{nc}{2}{6}
        \function{fc2}{ 1/3*(1+2/nc)}
        
    %Q2.4        
        \randint{md}{3}{7}
        \randint{id}{0}{4}
        \function[calculate]{id1}{id+1}
        \function[calculate]{fd}{(1+id)/(md+1+id)}
			
	\end{variables}
	
    \type{input.number} 
    \field{rational}

    %Q2.1
		\begin{answer}
%			\text{$\sum_{j=1}^\var{na}  \var{ma} \cdot j \cdot (j+\var{za}) = $}
			\solution{fa}
		\end{answer}
       
   
    %Q2.2	
		\begin{answer}
%			\text{ $\sum_{k=0}^\var{n} \var{qk} =$}
			\solution{f}
		\end{answer}
     
     
    %Q2.3
    	\begin{answer}
%			\text{$\prod_{j=2}^\var{nc} \left( 1 - \frac{2}{j^2+j} \right) = $}
			\solution{fc2}
		\end{answer}
        
    %Q2.4		
		\begin{answer}
%			\text{$\prod_{k=1}^\var{md} \frac{k+\var{id}}{k+\var{id1}} = $}
			\solution{fd}
		\end{answer}

	\end{question}

%

\end{problem}

\embedmathlet{mathlet}


\end{content}