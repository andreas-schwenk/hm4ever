\documentclass{mumie.problem.gwtmathlet}
%$Id$
\begin{metainfo}
 \name{
  \lang{de}{A03: Anwendung einfacher Gleichungen}
  }
  \begin{description} 
 This work is licensed under the Creative Commons License Attribution 4.0 International (CC-BY 4.0)   
 https://creativecommons.org/licenses/by/4.0/legalcode 

    \lang{de}{...}
  \end{description}
  \corrector{system/problem/GenericCorrector.meta.xml}
  \begin{components}
    \component{js_lib}{system/problem/GenericMathlet.meta.xml}{gwtmathlet}
  \end{components}
  \begin{links}
  \end{links}
  \creategeneric
\end{metainfo}

\begin{content}
\lang{de}{\title{A03: Anwendung einfacher Gleichungen}}
\lang{en}{\title{A03: application of simple equations}}
\begin{block}[annotation]
	Im Ticket-System: \href{https://team.mumie.net/issues/23785}{Ticket 23785}
\end{block}

\begin{block}[annotation]
Kopie: hm4mint/T101_neu_Elementare_Rechengrundlagen/training 11

Im Ticket-System: \href{https://team.mumie.net/issues/22336}{Ticket 22336}
\end{block}

\begin{block}[annotation]
	Anwendungen zu linearen und quadratischen GLeichungen
\end{block}

\usepackage{mumie.genericproblem}


\begin{problem}

%
  \randomquestionpool{1}{6}
%

% Question 1

\begin{question}

\begin{variables}
	\randint{a}{20}{50}
	\randint{b}{1}{30}
	\randadjustIf{a}{b>=a}
	
	\function[calculate, 2]{sc}{(a+b)/5}	
    \function[calculate, 2]{sb}{2*sc}
	\function[calculate, 2]{sl}{sb-b}
\end{variables}

	\type{input.number}
	\field{real}
    \precision{2}
    \lang{de}{\text{
	    Die Geschwister Ben, Carl und Laura gehen zusammen einkaufen. Sie geben zusammen $\var{a}$ € aus.
        Ben gibt dabei doppelt soviel aus wie Carl und Laura gibt $\var{b}$ € weniger aus als Ben. 
	    Welchen Betrag gibt jeder einzelne von ihnen aus?\\
        \\
        Ben gibt \ansref €, $\,$ Carl \ansref € $\,$ und Laura \ansref € aus.}}
    \lang{en}{\text{
       The siblings Ben, Carl and Laura go shopping together. They spend $\var{a}$ € together.
        Ben spends twice as much as Carl and Laura spends $\var{b}$ € less than Ben. 
	    What amount does each of them spend?\\
        \\
        Ben spends \ansref €, $\,$ Carl spends \ansref € $\,$ and Laura spends \ansref €.}}
  
  \lang{de}{\explanation{Zu Lösen ist die lineare Gleichung $x+ \frac{1}{2} x+ (x-\var{b})=\var{a}$.}}
  \lang{en}{\explanation{We need to solve the linear equation $x+ \frac{1}{2} x+ (x-\var{b})=\var{a}$. }}
    
    \begin{answer}
	    \solution{sb}
    \end{answer}
    \begin{answer}
	    \solution{sc}
    \end{answer}
    \begin{answer}
	    \solution{sl}
    \end{answer}

\end{question}

% Question 2

\begin{question}

\begin{variables}
	\randint{a}{50}{150}
	\randint{b}{3}{5}
	
	\function[calculate, normalize]{s}{2*a*b/(b-2)}	

\end{variables}

	\type{input.number}
	\field{rational}
  \lang{de}{\text{
	    Marc und Lea machen eine dreitägige Radtour. Am ersten Tag legen sie $50$ \% der Gesamtstrecke 
        zurück. Am zweiten Tag schaffen sie $\var{a}$. Für den letzten Tag bleibt dann noch ein $\var{b}$-tel 
        der Gesamtstrecke zu fahren. Wie lang ist die gesamte Strecke?\\
        \\
        Insgesamt fahren die beiden an den drei Tagen $\,$\ansref km.}}
    \lang{en}{\text{
     Marc and Lea go on a three-day bike tour. On the first day they cover $50$ \% of the total distance. 
        On the second day they manage $\var{a}$. On the last day they have to cycle $1/\var{b}$
        of the total distance. How long is the total distance?\\
        \\
        In total, the two of them cycle $\,$\ansref km over the three days.}}
        
    \lang{de}{\explanation{Zu Lösen ist die lineare Gleichung $\; \frac{1}{2} x+ \var{a} + \frac{1}{\var{b}} x = x $.}}
    \lang{en}{\explanation{We need to solve the linear equation $\; \frac{1}{2} x+ \var{a} + \frac{1}{\var{b}} x = x $. }}
    
    \begin{answer}
	    \solution{s}
    \end{answer}

\end{question}

% Question 3

\begin{question}

\begin{variables}
	\randint{a}{60}{120}
    \randint{d}{10}{30}
	\randint{b}{5}{15}
    \randadjustIf{b}{b>=d}
 
	\function[calculate, normalize]{cd}{d-b}    % Alter von David vor b Jahren
    \function[calculate, normalize]{co}{a-d-b}	% Alter der Großmutter vor b Jahren
	\function[calculate, normalize]{c}{co/cd}
	\function[calculate, normalize]{sd}{d}
	\function[calculate, normalize]{so}{a-d}	    

\end{variables}

	\type{input.number}
	\field{rational}
   \lang{de}{\text{
	    David und seine Großmutter sind zusammen $\var{a}$ Jahre alt. Vor $\var{b}$ Jahren war seine Großmutter
        $\var{c}$ mal so alt wie David. Wie alt sind die beiden heute?\\
        \\
        David ist \ansref Jahre alt und seine Großmutter ist \ansref .}}
    \lang{en}{\text{
	    David and his grandmother are $\var{a}$ years old together. $\var{b}$ years ago, his grandmother was
        $\var{c}$ times as old as David. How old are they today?\\
        \\
        David is \ansref  years old and his grandmother is \ansref years old.}}
        
    \lang{de}{\explanation{Zu Lösen ist die lineare Gleichung $\; \var{c} (x-\var{b}) = (\var{a}-x)-\var{b}$.}}
    \lang{en}{\explanation{We need to solve the linear equation $\; \var{c} (x-\var{b}) = (\var{a}-x)-\var{b}$.}}
    
    \begin{answer}
	    \solution{sd}
    \end{answer}
    \begin{answer}
	    \solution{so}
    \end{answer}

\end{question}

% Question 4

\begin{question}

\begin{variables}
	\randint{a}{10}{15}                                 % Geschwindikeit Marie
    \drawFromSet{d}{5,15}
    \function[calculate, normalize]{b}{a+d}             % Geschwindikeit Mutter 
 	\function[calculate, normalize]{sh}{1/4 * a/(b-a)}  % Fahrtzeit der Mutter in h
	\function[calculate, normalize]{sm}{sh * 60}	    % Fahrtzeit der Mutter in min   

\end{variables}

	\type{input.number}
	\field{rational}
   \lang{de}{\text{
	    Marie fährt mit ihrem Fahrrad mit (konstant) $\var{a}$ km/h zur Schule. Als ihre Mutter 
        bemerkt, dass Marie ihren Sportbeutel vergessen hat, fährt sie mit einem e-Bike hinterher, 
        um Marie den Sportbeutel zu bringen. Die Mutter bricht genau $15$ min später auf und fährt mit
        einer (konstanten) Geschwindigkeit von $\var{b}$ km/h.  
        Wie lange braucht sie, bis sie ihre Tochter eingeholt hat? (Geben Sie die Zeit in Minuten an.)\\
        \\
        Es dauert $\,$\ansref min, bis die Mutter Marie eingeholt hat.}}
    \lang{en}{\text{
    Marie rides her bicycle to school at (constant) $\var{a}$ km/h. When her mother notices
    that Marie has forgotten her sports bag, she rides behind her on an e-bike to bring Marie her gym bag. 
    The mother sets off exactly $15$ min later and rides at a (constant) speed of $\var{b}$ km/h.  
    How long does it take her to catch up with her daughter? (Give the time in minutes.)\\
       \\It takes $\,$\ansref min for the mother to catch up with Marie.
    }}
        
    \lang{de}{\explanation{Zu Lösen ist die lineare Gleichung 
                $\; (\frac{1}{4}$h $\,+ x\,$h $) \cdot \var{a} \,$km/h $\,= x\,$h $\,\cdot \var{b} \,$km/h.\\ 
               }}
    \lang{en}{\explanation{We need to solve the linear equation
              $\; (\frac{1}{4}$h $\,+ x\,$h $) \cdot \var{a} \,$km/h $\,= x\,$h $\,\cdot \var{b} \,$km/h.\\
              }}
    
    \begin{answer}
	    \solution{sm}
    \end{answer}

\end{question}

% Question 5

\begin{question}

\begin{variables}
	\randint[Z]{a}{-50}{50}
	\randint[Z]{b}{-50}{50}
%    \randadjustIf{b}{b<a}

    \function[calculate, normalize]{sum}{a+b}
    \function[calculate, normalize]{sn}{-a-b}
 	\function[calculate, normalize]{p}{a*b} 

\end{variables}

	\type{input.number}
	\field{rational}
    \lang{de}{\text{
	    Zerlegen Sie die Zahl $\var{sum}$ in zwei Summanden, deren Produkt $\var{p}$ ergibt. \\
        \\
        Die gesuchten Summanden sind $x_1=$\ansref und $x_2=$\ansref .}}
      \lang{en}{\text{ Decompose the number $\var{sum}$ into two summands whose product is $\var{p}$. \\
        \\
        The summands we are looking for are $x_1=$\ansref and $x_2=$\ansref .}}
        
   \lang{de}{\explanation[sum<0 AND p<0]
                {Zu Lösen ist die quadratische Gleichung $x^2 + \var{sn} x + (\var{p}) =0$.\\
                Für die Lösungen $x_1$ und $x_2$ gilt nach dem Satz von Viëta 
                $\var{sum}=(x_1+x_2)\,$ und $\var{p}=x_1 \cdot x_2$.
                }}
    \lang{en}{\explanation[sum<0 AND p<0]
              {We need to solve the quadratic equation $x^2 + \var{sn} x + (\var{p}) =0$.
                According to Vieta's theorem, the solutions $x_1$ and $x_2$ satisfy $\var{sum}=(x_1+x_2)\,$ und $\var{p}=x_1 \cdot x_2$.
              }}
   \lang{de}{\explanation[sum<0 AND p>=0]
                {Zu Lösen ist die quadratische Gleichung $x^2 + \var{sn} x + \var{p} =0$.\\
                Für die Lösungen $x_1$ und $x_2$ gilt nach dem Satz von Viëta 
                $\var{sum}=(x_1+x_2)\,$ und $\var{p}=x_1 \cdot x_2$.
                }}
   \lang{en}{\explanation[sum<0 AND p<0]
              {We need to solve the quadratic equation $x^2 + \var{sn} x + \var{p} =0$.
                
                According to Vieta's theorem, the solutions $x_1$ and $x_2$ satisfy $\var{sum}=(x_1+x_2)\,$ und $\var{p}=x_1 \cdot x_2$.
              }}
      
    \lang{de}{\explanation[sum>=0 AND p<0]
                {Zu Lösen ist die quadratische Gleichung $x^2 - \var{sum} x + (\var{p}) =0$.\\
                Für die Lösungen $x_1$ und $x_2$ gilt nach dem Satz von Viëta 
                $\var{sum}=(x_1+x_2)\,$ und $\var{p}=x_1 \cdot x_2$.
                }}   
   \lang{en}{\explanation[sum<0 AND p<0]
              {We need to solve the quadratic equation $x^2 - \var{sum} x + (\var{p}) =0$.
                
                According to Vieta's theorem, the solutions $x_1$ and $x_2$ satisfy $\var{sum}=(x_1+x_2)\,$ und $\var{p}=x_1 \cdot x_2$.
              }}
    \lang{de}{\explanation[sum>=0 AND p>=0]
                {Zu Lösen ist die quadratische Gleichung $x^2 - \var{sum} x + \var{p} =0$.\\
                Für die Lösungen $x_1$ und $x_2$ gilt nach dem Satz von Viëta 
                $\var{sum}=(x_1+x_2)\,$ und $\var{p}=x_1 \cdot x_2$.
                }}
     \lang{en}{\explanation[sum<0 AND p<0]
              {We need to solve the quadratic equation $x^2 - \var{sum} x + \var{p} =0$.
                
                According to Vieta's theorem, the solutions $x_1$ and $x_2$ satisfy $\var{sum}=(x_1+x_2)\,$ und $\var{p}=x_1 \cdot x_2$.
              }}
                
                
%
   \permuteAnswers{1, 2}
%   
    \begin{answer}
	    \solution{a}
    \end{answer}
    \begin{answer}
	    \solution{b}
    \end{answer}

\end{question}

% Question 6

\begin{question}

\begin{variables}
	\randint[Z]{a}{0}{35}
	\randint[Z]{b}{1}{10}

    \function[calculate, normalize]{s}{2*(2*a+b)}
 	\function[calculate, normalize]{f}{a*(a+b)} 

\end{variables}

	\type{input.number}
	\field{rational}
    \lang{de}{\text{
	    Ein rechteckiges Grundstück mit einer Fläche von $\var{f}$ qm ist von einem
        Zaun umgeben. Die längere Seite des Grundstücks ist $\var{b}$ m länger
        als die kürzere Seite. Wie lang ist der Zaun?  \\
        \\
        Der Zaun hat eine Länge von \ansref m.}}
    \lang{en}{\text{A rectangular plot of land with an area of $\var{f}$ sqm is surrounded by a
        fence. The longer side of the plot is $\var{b}$ m longer
        than the shorter side. What is the length of the fence?  \\
        \\
        The fence has a length of \ansref m.}}
        
   \lang{de}{\explanation{Die positive Lösung der quadratischen Gleichung $x^2 + \var{b} x - \var{f} =0$ liefert 
                die Länge der kürzeren Grundstücksseite.
                }}
    \lang{en}{\explanation{The positive solution of the quadratic equation $x^2 + \var{b} x - \var{f} =0$ yields 
                the length of the shorter side of the plot.}}
%   
    \begin{answer}
	    \solution{s}
    \end{answer}

\end{question}


\end{problem}

\embedmathlet{gwtmathlet}

\end{content}
