\documentclass{mumie.summary}
%$Id$
\begin{metainfo}
  \name{
    \lang{de}{quiz-summary}
    \lang{en}{}
  }
  \begin{description} 
 This work is licensed under the Creative Commons License Attribution 4.0 International (CC-BY 4.0)   
 https://creativecommons.org/licenses/by/4.0/legalcode 

    \lang{de}{Beschreibung}
    \lang{en}{}
  \end{description}
  \begin{components}
  \end{components}
  \begin{links}
%     \link{generic_pdf_document}{content/rwth/HM1/pdf/g_pdf_Musterloesung.meta.xml}{Musterloesung}
%     \link{generic_pdf_document}{content/rwth/HM1/pdf/g_pdf_Anrechnungen.meta.xml}{Anrechnungen}
%     \link{generic_pdf_document}{content/rwth/HM1/pdf/g_pdf_Altklausur20190718.meta.xml}{Altklausur20190718}
%     \link{generic_pdf_document}{content/rwth/HM1/pdf/g_pdf_Altklausur202104.meta.xml}{Altklausur202104}
%     \link{generic_pdf_document}{content/rwth/HM1/pdf/g_pdf_Probeklausur.meta.xml}{Probeklausur}
  \end{links}
  \creategeneric
\end{metainfo}
\begin{content}
\begin{block}[annotation]
	Im Ticket-System: \href{https://team.mumie.net/issues/29262}{Ticket 29262}
\end{block}
\begin{block}[annotation]
Copy of : org/rwth/HM1/sum_HM_1_summary.src.tex
\end{block}


  \title{
  	\lang{de}{Mathematik für Wirtschaftswissenschaftler}
  }
  
\tableofcontents

\textbf{Willkommen im Kurs!}

Hier finden Sie hier alles Wissenswerte zu Aufbau und Inhalten des Kurses. 

\textbf{Für wen ist dieser Kurs gemacht?}

Der WiWiMath.nrw ist ein Einführungskurs in die Wirtschaftsmathematik. 
Er deckt in etwa die Mathematik des ersten Semester eines BWL-Studiums ab.\\
\textbf{Für Schüler:innen und Studierende und einfach für Interessierte:} Dieser Kurs eignet sich für das Selbststudium vor Studienbeginn oder während der ersten Fachsemester. \\
Für die einzelnen Teilnehmer ist der Kurs kostenlos.\\
Eine Klausur inklusive Scheinerwerb ist in Planung und wird realisiert, 
wenn genügend Hochschulen diesen Schein als Prüfungsleistung anerkennen.\\
\textbf{Für Dozent:innen:} Der Kurs kann und darf auch von Dozent:innen im Ganzen oder in Teilen in Lehrveranstaltungen genutzt werden, 
z.B. über ein CMS-Plugin.

\textbf{Von wem ist der Kurs gemacht?}

Der Kurs ist entstanden im HM4mint-Projekt, einem Kooperationsprojekt von 17 NRW-Hochschulen. 
Im Grunde eigenständig, ergaben sich Inspirationen aus den Veranstaltungen in Wirtschaftsmathematik von Prof. E. Cramer, RWTH Aachen,
und Prof. R. Voller, HS Niederrhein.

\section{Aufbau des Kurses}

Der Kurs umfasst aktuell die vier Teile \textbf{Grundlagen}, \textbf{Elementare Finanzmathematik},
\textbf{Differentialrechnung} und \textbf{Integralrechnung} von Funktionen mit einer Variablen. In jedem Teil 
werden Sie Inhalte finden, die Ihnen schon aus der Schule bekannt vorkommen werden, wir werden diese aber 
auch schnell vertiefen und neue Begriffe und Methoden kennenlernen.

Jeder Teil des Kurses besteht aus mehreren Vorlesungstexten, zu denen jeweils 
Beispielaufgaben und Trainingsaufgaben zur Verfügung stehen. 
\begin{itemize}
\item \textbf{Vorlesungstexte}: Alle neuen Begriffe und Techniken werden hier vorgestellt und erklärt mit Beispielen
und Verständnis-Checks. Wo mathematische Beweise den Rahmen des Kurses übersteigen, wird für Interessierte auf den 
HM4MINT verlinkt.

\item \textbf{Beispielaufgaben − So wird's gemacht}:
Jeder Themenblock verfügt über Aufgabenstellungen mit vollständiger Musterlösung zum Üben.
\item \textbf{Trainingsaufgaben}: Hier sind Sie am Zug! Lösen Sie die Aufgabe und geben Sie Ihre Lösung in die 
Eingabefelder ein. Sie erhalten direkt ein Feedback zu Ihrer Lösung. 
\item\textbf{Zwischenprüfungen} In diesen werden die vorgestellten und geübten Themen überprüft. 
    Zur Teilnahme an der Präsenz-Klausur (in Planung) wird die Bearbeitung dieser Aufgaben  vorausgesetzt.
\end{itemize}

\section{Inhalte des Kurses}
Folgende Themen werden in den einzelnen Teilen behandelt. 
\begin{itemize}
\item \textbf{Grundlagen}: Mengen, Terme, Rechenoperationen, reelle Folgen, Reihen, Grenzwerte, Funktionen, Stetigkeit.
\item \textbf{Elementare Finanzmathematik}: Zinseszins, unterjährige Verzinsung, Rentenbar- und Rentenendwerte, Ratentilgung, Annuitätentilgung.
\item \textbf{Differentialrechnung}: Ableitungsbegriff, einseitige Ableitung, Ableitungsregeln, Zusammenhang zu Monotonie und Krümmung, 
Kurvendiskussion, Elastizität, Taylor-Entwicklung, Regeln von L'Hospital.
\item \textbf{Integralrechnung}: Stammfunktionen, bestimmtes Integral, Hauptsatz der Differential- und Integralrechnung, 
Grundintegrale, partielle Integration, Substitution, uneigentliche Integrale.
\end{itemize}


\section{Empfehlungen für die Bearbeitung dieses Kurses}

\begin{tabs*}
\tab{Regelmäßig}
Arbeiten Sie regelmäßig über einen längeren Zeitraum, indem Sie sich beispielsweise Zeit nehmen, einen der 
Vorlesungstexte komplett und gründlich durchzuarbeiten. 

Genießen die Vorteile eines Online-Kurses, 
dass Sie zu jeder Zeit von jedem Ort, so lange Sie wollen daran arbeiten können. \emph{Seien Sie sich aber der 
Gefahren bewusst:} Es ist anspruchsvoll, alleine über eine lange Zeit konstant und regelmäßig in einem 
Online-Kurs zu arbeiten. Eine gute Möglichkeit ist es, diesen Kurs mit einer Gruppe von Kommilitonen gemeinsam zu erarbeiten.

\tab{Wenig Taschenrechner}
Die Verlockung, einen Taschenrechner für die Aufgaben zu nutzen, ist groß. Insbesondere wenn leistungsfähige Taschenrechner direkt 
mehrere Rechnungen in einem Schritt ausführen können. Betrügen Sie sich jedoch nicht selbst. 
Versuchen Sie, die Aufgaben so weit wie möglich ohne Taschenrechner zu lösen und ihn nur für gerundete Werte für das Endergebnis zu benutzen. 
%Wert für das Endergebnis auszurechnen. 
\tab{Inhalte bauen aufeinander auf}
Die Inhalte des Kurses bauen aufeinander auf. Es ist darum empfehlenswert, am Anfang zu beginnen. Studieren Sie zuerst die Texte, dann  die vorgerechneten Beispielaufgaben. 
Bearbeiten Sie dann die Trainingsaufgaben so lange, bis Sie sich sicher fühlen, und 
lösen Sie erst dann die Zwischenprüfungen. 
\tab{Liste für Prüfungsvorbereitung}
Die Texte sind meistens leicht zu lesen. Sie sollten sie jedoch so genau und tief verstehen, 
dass Sie nach der Lektüre jedes Artikels eine kurze Zusammenfassung schreiben können. 
Notieren Sie sich dabei die zwei wichtigsten Begriffe.  Die so entstehende Liste kann eine 
nützliche Hilfe bei der Prüfungvorbereitung sein. 

\end{tabs*}

Übrigens: In der Mathematik ist es normal, dass sich einem Sachverhalte manchmal erst nach mehrmaligem Durcharbeiten erschließen. 
Lassen Sie sich also nicht entmutigen, sondern lesen Sie noch einmal, machen Sie sich Notizen und Skizzen. 
Erklären Sie sich selbst oder anderen, was Sie bereits verstehen und wo genau die Schwierigkeiten liegen. So helfen Sie Ihrem Gehirn, nach und nach ein neues Konzept zu spiegeln.

\section{Informationen zur Eingabe von Formeln}
\begin{showhide}%[\class{fancy}
\begin{itemize}
\item Gewöhnlich steht in den Aufgabenstellungen bereits eine Anleitung, wenn Sie bestimmte Ausdrücke eingeben können oder wie Sie neue Zeilen o.\,Ä. eingeben können. Halten Sie sich an diese Anleitungen, nur so können Ihre Lösungsangaben korrekt korrigiert werden.
Sie benötigen lediglich eine normale Laptop-Tastatur und keine Sonderzeichen.

\item Grundsätzlich können Sie in ein Eingabefeld alle möglichen mathematischen Terme eingeben. Es kommt eine Fehlermeldung, wenn bestimmte mathematische Operationen
oder Ausdrücke nicht erlaubt sind. 

\item Kürzen Sie Brüche immer vollständig oder geben Sie rationale Zahlen gekürzt an. Ansonsten kann die Lösung auch als falsch angezeigt werden. In diesem Fall erhalten Sie aber nach Abgabe einen Hinweis.

\item Wird bei einer Lösung eine Menge erwartet, dann geben Sie jedes Element der Menge einzeln an und klicken Sie auf '+', um neue Elemente der Menge hinzuzufügen.
Bei Intervallen können Sie durch Mausklick auf die Intervallgrenzen auch ändern, ob die Intervallgrenzen dazugehören oder nicht.

\item Bei Grenzwerten benötigen Sie häufiger das $\infty$-Zeichen. Dazu geben Sie "infinity" in das Eingabefeld ein. In der Vorschau sollte dann $\infty$ erscheinen.

\item Grundsätzlich wird $e$ als die eulersche Zahl aufgefasst. Die irrationale Zahl $\pi$ geben Sie mit 'pi' ein und die Quadratwurzel $\sqrt{x}$ können Sie mit 'sqrt(x)' eingeben.

\end{itemize}

\end{showhide}


\end{content}
