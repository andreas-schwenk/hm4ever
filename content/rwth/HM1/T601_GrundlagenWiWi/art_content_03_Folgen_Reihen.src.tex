%$Id:  $
\documentclass{mumie.article}
%$Id$
\begin{metainfo}
  \name{
    \lang{de}{Folgen und Reihen}
    \lang{en}{Sequences and series}
  }
  \begin{description} 
 This work is licensed under the Creative Commons License Attribution 4.0 International (CC-BY 4.0)   
 https://creativecommons.org/licenses/by/4.0/legalcode 

    \lang{de}{Beschreibung}
    \lang{en}{Description}
  \end{description}
  \begin{components}
    \component{generic_image}{content/rwth/HM1/images/g_tkz_T601_Inventory.meta.xml}{T601_Inventory}
    \component{generic_image}{content/rwth/HM1/images/g_img_00_Videobutton_schwarz.meta.xml}{00_Videobutton_schwarz}
    \component{js_lib}{system/media/mathlets/GWTGenericVisualization.meta.xml}{mathlet1}
  \end{components}
  \begin{links}
    \link{generic_article}{content/rwth/HM1/T208_Reihen/g_art_content_25_konvergenz_kriterien.meta.xml}{content_25_konvergenz_kriterien}
    \link{generic_article}{content/rwth/HM1/T603_Differentialrechnung/g_art_content_11_AbleitungSaetze.meta.xml}{content_11_AbleitungSaetze}
    \link{generic_article}{content/rwth/HM1/T201neu_Vollstaendige_Induktion/g_art_content_02_vollstaendige_induktion.meta.xml}{content_02_vollstaendige_induktion}
    \link{generic_article}{content/rwth/HM1/T209_Potenzreihen/g_art_content_27_konvergenzradius.meta.xml}{content_27_konvergenzradius}
    \link{generic_article}{content/rwth/HM1/T208_Reihen/g_art_content_26_produkt_von_reihen.meta.xml}{content_26_produkt_von_reihen}
    \link{generic_article}{content/rwth/HM1/T205_Konvergenz_von_Folgen/g_art_content_14_konvergenz.meta.xml}{konvergenz}
    \link{generic_article}{content/rwth/HM1/T205_Konvergenz_von_Folgen/g_art_content_15_monotone_konvergenz.meta.xml}{monot-konv}
    \link{generic_article}{content/rwth/HM1/T207_Intervall_Schachtelung/g_art_content_23_intervallschachtelung.meta.xml}{intervallschachtelung}
  \end{links}
  \creategeneric
\end{metainfo}
\begin{content}
\usepackage{mumie.ombplus}
\usepackage{mumie.genericvisualization}
\ombchapter{1}
\ombarticle{3}

\title{
\lang{de}{Folgen und Reihen}
\lang{en}{Sequences and series}
}
 
\begin{block}[annotation]
  
  
\end{block}
\begin{block}[annotation]
Im Ticket-System: \href{https://team.mumie.net/issues/22667}{Ticket 22667}
\end{block}

\begin{block}[info-box]
\tableofcontents
\end{block}


\section{
\lang{de}{Folgen und ihre Wertemengen}
\lang{en}{Sequences and their sets of values}
}

\lang{de}{In diesem Kapitel befassen wir uns mit (reellen) Folgen und später auch Reihen. Folgen treten in vielfältigen Anwendungen auf und 
erlauben in vielen Fällen ein systematisches Vorgehen, wenn etwas in einer Abfolge von Schritten passiert. }
\lang{en}{
In this chapter, we will consider (real) sequences and, later, series.
Sequences appear in many applications and allow one to work
systematically with events that take place in successive steps.
}

\begin{example}
  \begin{tabs*}[\initialtab{0}]
    \tab{\lang{de}{Lagerhaltung 1a} \lang{en}{Inventory 1a}}
 \lang{de}{ Ein Logistikunternehmen lagert unzählige Materialien und Stoffe in seinen Hallen und auf dem
 Außengelände der Firma. Dabei führen sie digital ausführlich Buch darüber, wo genau welche und wie viele 
  Materialien gelagert werden. Intern bezeichnet das Unternehmen diesen Vorgang als Lagerhaltung.\\
  
 In den nächsten Tagen soll ihr Bestand erweitert werden, weshalb das Unternehmen eine große
 Warenanlieferung eines Produkts erwartet. Dafür muss eine neue Lagerstätte vorbereitet werden.
 Die zuständigen Mitarbeiter fertigen große Regalsysteme, die bei Bedarf nach oben hin um weitere
 Fächer ausgebaut werden können. Die weitere Planung beschränkt sich auf eines dieser Regalsysteme 
 - wie nachfolgend dargestellt.\\ }
 \lang{en}{
  A logistics company stores countless materials both in its buildings and outside
  on company property. They keep extensive records on the location and quantity of
  the stored materials. Internally, the company calls this activity taking inventory.

  In the next few days, the stock of materials will be increased. Therefore, the company
  expects a large delivery of a product. A new warehouse must be prepared.
  The employees in charge of that are building large shelving systems that can be
  expanded upwards with further shelves if necessary.
  The planning below is concerned with one of these shelving systems,
  as represented in the following image.
 }
 
\begin{center}
\image{T601_Inventory}
\end{center}

\lang{de}{Aus der Masse eines jeden Kartons, der in das Regalsystem eingelagert werden soll, ermitteln die Mitarbeiter
die höchstzulässige Lagermenge an Kartons für das unterste Fach: $1024$ Stück. Sie entscheiden sich dafür, 
in ein jeweils darüber liegendes Fach $50$ Kartons weniger einräumen zu wollen als in das darunter liegende Fach.
Denn es ist bekannt, dass die Gesamtmasse aller Fächer, die über einem Fach liegen, auf die Stabilität des
Fachs einwirkt. In das System zur Lagerhaltung notieren sie folgendes:\\ }
\lang{en}{
Based on the weight of the boxes used in the shelving system, the employees
determine the maximum permissible number of boxes in the lowest shelf: $1024$ boxes.
They decide that $50$ fewer boxes should be stored on any given shelf than
the shelf below it. This is because the stability of each shelf is impacted
by the total weight of the shelves lying above it.
In the inventory system they record the following:\\
}

\lang{de}{ Lagerstätte: Außenbereich, $C3$, Regalsystem $A10$:
    \begin{itemize}
        \item Fach Nummer 0: Produkt $A$, $1024$ Stück
        \item Fach Nummer 1: Produkt $A$, $974$ Stück
        \item Fach Nummer 2: Produkt $A$, $924$ Stück
        \item Fach Nummer 3: Produkt $A$, $874$ Stück
    \end{itemize}
}
\lang{en}{ Facility: Outside area, $C3$, Shelf system $A10$:
    \begin{itemize}
        \item Shelf 0: Product $A$, $1024$ Boxes
        \item Shelf 1: Product $A$, $974$ Boxes
        \item Shelf 2: Product $A$, $924$ Boxes
        \item Shelf 3: Product $A$, $874$ Boxes
    \end{itemize}
}

\lang{de}{Wir sehen hier also eine Durchnummerierung der Fächer, wobei jedem Fach mit der Nummer $n$ eine bestimmte
Anzahl an Produkten der Sorte $A$ in Abhängigkeit der Fachnummer $n$ zugeordnet wird. \\\\
Steht $A(n)$ für die Anzahl an Produkten in Fach $n$, so erhalten wir sofort die Anzahl der Produkte durch die folgende
Berechnungsformel:\\\\
$\quad\quad\quad\quad\quad\quad\quad\quad A(n)=1024-50\cdot n$\\\\
Die Aufstellung derartiger Zuordnungen geschieht mit Hilfe von (arithmetischen) \notion{Folgen}.}
\lang{en}{
Here we see an enumeration of the shelves, such that the shelf numbered $n$
contains a certain number of boxes of product $A$ depending on the shelf number $n$. \\\\
If $A(n)$ denotes the number of boxes in shelf $n$, then we immediately obtain the number of boxes
by the following formula:\\\\
$\quad\quad\quad\quad\quad\quad\quad\quad A(n)=1024-50\cdot n$\\\\
Relations of this sort can be described by means of (arithmetic) \notion{sequences}.
}

 \tab{\lang{de}{Lagerhaltung 1b} \lang{en}{Inventory 1b}}
 
 \lang{de}{Kurz vor der Warenanlieferung bekommt das Unternehmen eine unangekündigte Kontrolle durch einen Gutachter.
 Dieser schaut sich das neue Regalsystem an und bemängelt Folgendes:\\
 
 \glqq Bei dieser Art Regalsystem, darf die Gesamtmasse der eingelagerten Produkte aller Fächer, die über
 einem bestimmten Fach liegen, die Masse der eingelagerten Produkte dieses (unteren) Fachs nicht überschreiten.\grqq \\
 
 Der zuständige Mitarbeiter schlägt daraufhin vor, ein jedes Fach mit der halben Masse des darunter liegenden Fachs
 zu füllen. Dadurch kommt er zu folgender Lagerhaltung:\\ }

 \lang{en}{
  Just before the delivery, the company receives an unannounced inspection by an expert
  who examines the new shelving system and raises the following complaints:\\

  "In this type of shelving system, the total weight of the products stored on the shelves
  lying over a given shelf may not exceed the weight of the products on the (lower) shelf."

 The employee in charge then proposes to fill every shelf to half of the weight of
 the shelf beneath it. This leads to the following layout:
 }
 
  \lang{de}{Lagerstätte: Außenbereich, $C3$, Regalsystem $A10$:
    \begin{itemize}
        \item Fach Nummer 0: Produkt $A$, $1024$ Stück
        \item Fach Nummer 1: Produkt $A$, $512$ Stück
        \item Fach Nummer 2: Produkt $A$, $256$ Stück
        \item Fach Nummer 3: Produkt $A$, $128$ Stück
    \end{itemize}}
  \lang{en}{Facility: Outside, $C3$, Shelf system $A10$:
    \begin{itemize}
        \item Shelf 0: Product $A$, $1024$ Boxes
        \item Shelf 1: Product $A$, $512$ Boxes
        \item Shelf 2: Product $A$, $256$ Boxes
        \item Shelf 3: Product $A$, $128$ Boxes
    \end{itemize}}
 
\lang{de}{Auch hier findet eine Durchnummerierung der Fächer statt, wobei jedem Fach mit der Nummer $n$ eine bestimmte
Anzahl an Produkten der Sorte $A$ in Abhängigkeit der Fachnummer $n$ zugeordnet wird.\\\\
Steht $A(n)$ für die Anzahl an Produkten in Fach $n$, so erhalten wir sofort die Anzahl der Produkte durch die folgende
Berechnungsformel:\\\\
$\quad\quad\quad\quad\quad\quad\quad\quad A(n)=1024\cdot (\frac{1}{2})^n$\\\\
Die Aufstellung derartiger Zuordnungen geschieht mit Hilfe von \notion{geometrischen Folgen}.\\

Wir werden im Verlauf des Kapitels sehen,  
dass 
\[
1024 \cdot \frac{1}{2} + 1024 \cdot \frac{1}{4} + \ldots = 1024 \cdot \sum_{k=1}^infty \left( \frac{1}{2} \right)^k 
\]
gegen $1024$ konvergiert. Der Vorschlag des Mitarbeiters erfüllt also die Forderung des Gutachters.}

\lang{en}{
Here we also find an enumeration of the shelves, such that the shelf numbered $n$ contains a certain
number of boxes of product $A$ depending on the number $n$.\\\\
If $A(n)$ stands for the number of boxes on shelf $n$, then we immediately obtain the number of boxes of product
by the following formula:\\\\
$\quad\quad\quad\quad\quad\quad\quad\quad A(n)=1024\cdot (\frac{1}{2})^n$\\\\
Relations of this sort can be described by means of \notion{geometric sequences}.\\

Later in this chapter, we will see that
\[
1024 \cdot \frac{1}{2} + 1024 \cdot \frac{1}{4} + \ldots = 1024 \cdot \sum_{k=1}^infty \left( \frac{1}{2} \right)^k 
\]
converges to $1024$. Therefore, the employee's suggestion meets the inspector's requirements.
}

    \end{tabs*}
\end{example}

\begin{definition}
\lang{de}{Eine reelle \notion{Folge} ist eine (meist unendliche) Liste von reellen Zahlen, die durchnummeriert sind. Für die Folge $a$ schreiben wir auch $(a_n)_{n \in \N}$ 
oder $(a_n)$ und meinen damit die Liste von Zahlen 
\[ (a_n) = a_1, \ a_2, \ a_3, \ \ldots \] 
Die einzelnen Zahlen der Liste nennt man \notion{Glieder} der Folge. 
Die \notion{Wertemenge} einer Folge $(a_n)_{n \in \N}$ ist die Menge der Glieder $\{ a_n \,|\, n \in \N\}$.}
\lang{en}{A real \notion{sequence} is a (generally infinite) list of
real numbers that are enumerated. The sequence $a$ is also written
$(a_n)_{n \in \N}$ or $(a_n)$ and stands for the list of numbers
\[ (a_n) = a_1, \ a_2, \ a_3, \ \ldots \]
The individual numbers in the list are called \notion{terms} of
the sequence. The \notion{set of values} of a sequence $(a_n)_{n \in \N}$ 
is the set of its terms $\{ a_n \,|\, n \in \N\}$.}

\end{definition}

\lang{de}{Da wir in diesem Themenblock nur reelle Folgen behandeln, werden wir oft das Wort \emph{reell} weglassen
und nur von \emph{Folgen} reden.

Häufig beginnt die Nummerierung einer Folge auch nicht bei $1$, sondern bei $0$ oder einer anderen ganzen Zahl. 
Dies machen wir kenntlich, indem wir etwa $(a_n)_{n \geq 0}$ schreiben. }

\lang{en}{
Since we will only consider real sequences in this module, we will often omit the word \emph{real}
and simply refer to these as \emph{sequences}.
}

\begin{example}\label{ex:22}
\begin{enumerate}
\item \lang{de}{\emph{Konstante Folgen:} Für $c\in \R$ ist die Folge $(a_n)_{n\in \N}$ mit $a_n=c$ für alle $n$
eine sogenannte konstante Folge. Es gilt also 
\[a_1 = c, \ a_2 = c, \ \ldots \]
Die Wertemenge ist lediglich $\{c\}$.}
\lang{en}{\emph{Constant sequences:} For $c \in \R$, the sequence
$(a_n)_{n \in \N}$ with $a_n = c$ for all $n$ is called a constant
sequence. We have
\[a_1 = c, \ a_2 = c, \ \ldots \]
The set of values is just $\{c\}$.}
\item \lang{de}{\emph{Arithmetische Folgen:} Für feste Werte $c,d\in \R$ nennt man die Folge $(a_n)_{n\geq 0}$ mit
\begin{align*}  
           & a_n=c+nd \quad \text{für alle } n\in \Nzero  \\
         (\text{d.h.} \quad  & a_0 = c, \ a_1 = c+d, \ a_2 = c+2d, \ a_3 = c+3d, \ \ldots )
\end{align*}
eine \emph{arithmetische Folge}. Arithmetische Folgen sind dadurch gekennzeichnet, dass die Differenz
aufeinanderfolgender Glieder der Folge stets dieselbe ist, hier $a_{n+1}-a_n=d$ für alle $n\in \Nzero$.\\
Konstante Folgen sind spezielle arithmetische Folgen mit $d=0$.}
\lang{en}{\emph{Arithmetic sequences:} For fixed values $c,d \in \R$, the sequence
$(a_n)_{n \geq 0}$ with
\begin{align*}  
           & a_n=c+nd \quad \text{for all } n\in \Nzero  \\
         (\text{i.e.} \quad  & a_0 = c, \ a_1 = c+d, \ a_2 = c+2d, \ a_3 = c+3d, \ \ldots )
\end{align*}
is called an \emph{arithmetic sequences}. Arithmetic sequences are characterized by the property
that the difference between any two consecutive terms of the sequence is the same;
here, $a_{n+1} - a_n = d$ for all $n \in \Nzero$.
Constant sequences are arithmetic sequences with $d=0$.
}
\item \lang{de}{\emph{Geometrische Folgen:} Für $u,q\in \R^*$ nennt man die Folge $(a_n)_{n\geq 0}$ mit
\begin{align*}  & a_n=u\cdot q^n \quad \text{für alle } n\in \Nzero \\
      (\text{d.h.} \quad & a_0 = u, \ a_1 = uq, \ a_2 = uq^2, \ a_3 = u q^3, \ \ldots) \end{align*}
eine \emph{geometrische Folge}. Geometrische Folgen sind dadurch gekennzeichnet, dass der Quotient
aufeinanderfolgender Glieder der Folge stets derselbe ist, hier $\frac{a_{n+1}}{a_n}=q$ für alle $n\in \Nzero$.\\
Konstante Folgen sind spezielle geometrische Folgen mit $q=1$.}
\lang{en}{\emph{Geometric sequences:} For $u,q\in \R^*$, the sequence $(a_n)_{n \geq 0}$ with
\begin{align*}  & a_n=u\cdot q^n \quad \text{for all } n\in \Nzero \\
      (\text{i.e.} \quad & a_0 = u, \ a_1 = uq, \ a_2 = uq^2, \ a_3 = u q^3, \ \ldots) \end{align*}
is called a \emph{geometric sequence}. Geometric sequences are characterized by the property
that the ratio of any two consecutive terms of the sequence is the same;
here, $\frac{a_{n+1}}{a_n} = q$ for all $n\in \Nzero$.
Constant sequences are geometric sequences with $q=1$.
}
\item \label{ex:sqrt-2} \lang{de}{Die Folge $(a_n)_{n\geq 1}$ mit $a_1=2$ und  
$a_{n+1}=\frac{a_n}{2}+\frac{1}{a_n}$ für alle $n \in \N$ ist eine rekursiv definierte Folge.
Die ersten Glieder der Folge sind also $a_1=2$, sowie
\begin{align*} a_2 &= \frac{a_1}{2}+\frac{1}{a_1} &=\frac{2}{2}+\frac{1}{2} &=\frac{3}{2} &=1,5\, , \\
a_3 &= \frac{a_2}{2}+\frac{1}{a_2} &=\frac{3}{4}+\frac{2}{3}&=\frac{17}{12}&=1,41\bar{6}\, , \\
a_4 &= \frac{a_3}{2}+\frac{1}{a_3}&=\frac{17}{24}+\frac{12}{17}&=\frac{577}{408} &\approx 1,4142157\, , \\
a_5 &= \frac{a_4}{2}+\frac{1}{a_4}&=\frac{577}{816}+\frac{408}{577}
&=\frac{665857}{470832} &\approx 1,414213562375\, .
\end{align*}}
\lang{en}{The sequence $(a_n)_{n\geq 1}$ with $a_1=2$ and
$a_{n+1}=\frac{a_n}{2}+\frac{1}{a_n}$ for all $n \in \N$ is a
recursively defined sequence. The first terms of the sequence are $a_1=2$ and
\begin{align*} a_2 &= \frac{a_1}{2}+\frac{1}{a_1} &=\frac{2}{2}+\frac{1}{2} &=\frac{3}{2} &=1.5\, , \\
a_3 &= \frac{a_2}{2}+\frac{1}{a_2} &=\frac{3}{4}+\frac{2}{3}&=\frac{17}{12}&=1.41\bar{6}\, , \\
a_4 &= \frac{a_3}{2}+\frac{1}{a_3}&=\frac{17}{24}+\frac{12}{17}&=\frac{577}{408} &\approx 1.4142157\, , \\
a_5 &= \frac{a_4}{2}+\frac{1}{a_4}&=\frac{577}{816}+\frac{408}{577}
&=\frac{665857}{470832} &\approx 1.414213562375\, .
\end{align*}
}


\item \lang{de}{Arithmetische Folgen können auch rekursiv definiert werden durch
$a_0=c$ und $a_{n+1}=a_n+d$ für alle $n\in \Nzero$, wobei wieder $c,d\in \R$ fest gewählte Zahlen 
sind.\\
Ebenso kann für $u,q\in \R^*$ die geometrische Folge $(a_n)_{n\geq 0}=(uq^n)_{n\geq 0}$ 
rekursiv definiert werden durch $a_0=u$ und $a_{n+1}=qa_n$.}
\lang{en}{Arithmetic sequences can also be defined recursively,
by specifying $a_0=c$ and $a_{n+1}=a_n+d$ for all $n\in \Nzero$,
where $c,d \in \R$ are any fixed values as before.
Similarly, for $u,q \in \R^*$, the geometric sequence $(a_n)_{n\geq 0}=(uq^n)_{n\geq 0}$ 
can be defined recursively by specifying $a_0=u$ and $a_{n+1}=qa_n$.
}

\end{enumerate}
\end{example}

\begin{quickcheckcontainer}
\randomquickcheckpool{1}{1}
\begin{quickcheck}
\type{input.number}
		\field{rational}
            
       	\begin{variables}
			\randint{x2}{2}{4}	
            \randint{a}{1}{4}   						
			\function[normalize]{afo}{(1/x2)^n}
            \function[normalize]{loes0}{1*a}
            \function[normalize]{loes1}{a+1/x2}
            \function[normalize]{loes2}{a+1/x2+1/x2^2}
                                         
		\end{variables}
	
		\lang{de}{
			\text{
            %\audio[1,0.5,0.8,1.25,1.5]{Folgendefinition_Kurztest_neu}\\
            Geben Sie die ersten drei Glieder der rekursiv definierten Folge 
            $(a_n)_{n\in\N}$ mit $a_1=\var{a}$ und $a_{n+1}= a_n + \var{afo}$ für $n\geq 1$ an.
		\\\\
        Geben Sie die Werte exakt und so weit gekürzt wie möglich ein.\\\\
        
        Die ersten drei Glieder der Folge lauten: $a_1=$\ansref, $a_2=$\ansref und $a_3=$\ansref.
 		}
       		}
    \lang{en}{
			\text{
            %\audio[1,0.5,0.8,1.25,1.5]{Folgendefinition_Kurztest_neu}\\
            Find the first three terms of the recursively defined sequence 
            $(a_n)_{n\in\N}$ with $a_1=\var{a}$ and $a_{n+1}= a_n + \var{afo}$ for $n\geq 1$.
		\\\\
        Enter the values exactly, simplified as much as possible.\\\\
        
        The first three terms of the sequence are: $a_1=$\ansref, $a_2=$\ansref and $a_3=$\ansref.
 		}
       		}
		
    	\begin{answer}
		\solution{loes0}
		\end{answer}
            
		\begin{answer}
		\solution{loes1}
		\end{answer}
        
        \begin{answer}
		\solution{loes2}
		\end{answer}
     \lang{de}{
        \explanation{Verwenden Sie das jeweils zuvor berechnete Glied der Folge, um das nächste zu bestimmen.} }
    \lang{en}{
        \explanation{Use the previous term in the sequence to compute the one following it.}
    }
	\end{quickcheck}
\end{quickcheckcontainer}

\lang{de}{
Um Folgen zu verstehen, werden wir sie auf bestimmte Eigenschaften untersuchen. Meistens
wird es darum gehen, ob die Werte der Folge eine bestimmte Tendenz aufweisen (z.\,B. alle steigen)
oder die Werte in einem  beschränkten Bereich enthalten sind. 
}
\lang{en}{
To understand sequences, we will consider various properties they can have.
These will mostly have to do with whether the values of the sequence
display any kind of trend (for example, if they are increasing) or whether they are contained in a bounded region.
}

\begin{definition}
\lang{de}{
Eine Folge $(a_n)_{n\in \N}$ heißt ...

\begin{itemize}
\item ... \notion{nach oben beschränkt}, wenn es eine reelle Zahl $C$ gibt, die größer als alle Glieder der Folge ist.

\item ... \notion{nach unten beschränkt}, wenn es eine reelle Zahl $c$ gibt, die kleiner als alle Glieder der Folge ist.

\item ... \notion{beschränkt}, wenn sie sowohl nach oben als auch nach unten beschränkt ist.

\item ... \notion{monoton wachsend}, wenn $a_1 \leq a_2 \leq a_3 \leq \ldots$ gilt.

\item ... \notion{streng monoton wachsend}, wenn $a_1 < a_2 < a_3 < \ldots$ gilt.

\item ... \notion{monoton fallend}, wenn $a_1 \geq a_2 \geq a_3 \geq \ldots$ gilt.

\item ... \notion{streng monoton fallend}, wenn $a_1 > a_2 > a_3 > \ldots$ gilt.
 

\end{itemize}
}
\lang{en}{
A sequence $(a_n)_{n\in \N}$ is called ...

\begin{itemize}
\item ... \notion{bounded from above} if there exists a real number $C$ that is greater than every term of the sequence.

\item ... \notion{bounded from below} if there exists a real number $c$ that is less than every term of the sequence.

\item ... \notion{bounded} if it is both bounded from above and from below.

\item ... \notion{monotonically increasing} if $a_1 \leq a_2 \leq a_3 \leq \ldots$.

\item ... \notion{strictly monotonically increasing} if $a_1 < a_2 < a_3 < \ldots$.

\item ... \notion{monotonically decreasing} if $a_1 \geq a_2 \geq a_3 \geq \ldots$.

\item ... \notion{strictly monotonically decreasing} if $a_1 > a_2 > a_3 > \ldots$.
 

\end{itemize}
}
\end{definition}


% Die beiden Begriffe der \emph{Monotonie} und \emph{Beschränktheit}~ werden im folgenden Video anhand eines Beispiels erläutert:
% \begin{center}
% \href{https://www.hm-kompakt.de/video?watch=301}{\image[75]{00_Videobutton_schwarz}}\\
% \end{center}


\begin{example}\label{ex:monotone-folgen}

\begin{tabs*}[\initialtab{0}] 
  
\tab{\lang{de}{Monotonie} \lang{en}{Monotonicity}}
\lang{de}{
\begin{enumerate}
\item \lang{de}{Konstante Folgen sind sowohl monoton wachsend 
als auch monoton fallend.}
\lang{en}{Constant sequence are both monotonically increasing
and monotonically decreasing.}
\item \lang{de}{Die Folge $(a_n)_{n\in \N}=((-1)^n)_{n\in \N}$ ist weder monoton wachsend noch monoton fallend, denn sie
nimmt immer abwechselnd die Werte $1$ und $-1$ an.}
\lang{en}{The sequence $(a_n)_{n\in \N}=((-1)^n)_{n\in \N}$ is neither monotonically increasing
nor monotonically decreasing, as it alternats between the values $1$ and $-1$.}
\item  \lang{de}{Arithmetische Folgen $(c+dn)_{n\geq 0}$ sind streng monoton wachsend, wenn $d>0$ ist, und streng monoton
fallend, wenn $d<0$ ist, denn für alle $n\in \Nzero$ ist 
\begin{align*} 
a_{n+1} &=a_n+d >a_n,& \ \text{ falls } \ d>0,\\ 
a_{n+1} &=a_n+d <a_n,& \ \text{ falls } \ d<0.
\end{align*}}
\lang{en}{Arithmetic sequences $(c+dn)_{n\geq 0}$ are strictly monotonically increasing if $d>0$ and strictly
monotonically decreasing if $d < 0$, because for every $n \in \Nzero$, we have
\begin{align*} 
a_{n+1} &=a_n+d >a_n,& \ \text{ falls } \ d>0,\\ 
a_{n+1} &=a_n+d <a_n,& \ \text{ falls } \ d<0.
\end{align*}}
\item \lang{de}{Bei  $(a_n)_{n\geq 0}=(uq^n)_{n\geq 0}$ hängt die Monotonie sowohl von 
$q$ als auch von $u$ ab.\\
Ist $q<0$, so ist die Folge nicht monoton, denn die Glieder der Folge sind abwechselnd positiv und
negativ (vgl. 2. Beispiel, wo $u=1$ und $q=-1$ ist).

Für $q>0$ haben alle $a_n$ das gleiche Vorzeichen wie $u$. Wir erhalten
folgende $5$ Fälle:
\begin{itemize}
\item $u>0$, $q>1$: Dann ist $a_n>0$ und $a_{n+1}=qa_n>a_n$ für alle $n\geq 0$. Also ist die 
Folge streng monoton wachsend.
\item $u>0$, $0<q<1$: Dann ist $a_n>0$ und $a_{n+1}=qa_n<a_n$ für alle $n\geq 0$. Also ist die 
Folge streng monoton fallend.
\item $u<0$, $q>1$: Dann ist $a_n<0$ und $a_{n+1}=qa_n<a_n$ für alle $n\geq 0$. Also ist die 
Folge streng monoton fallend.
\item $u<0$, $0<q<1$: Dann ist $a_n<0$ und $a_{n+1}=qa_n>a_n$ für alle $n\geq 0$. Also ist die 
Folge streng monoton wachsend.
\item  $q=1$: Dann ist die Folge konstant (s.\,o.).
\end{itemize}
}
\lang{en}{Whether  $(a_n)_{n\geq 0}=(uq^n)_{n\geq 0}$ is monotonic
depends on both of $q$ and $u$.
If $q<0$, then the sequence is not monotonic, as the terms of
the sequence alternate between positive and negative values
(compare Example 2, where $u=1$ and $q=-1$).

If $q>0$, then all $a_n$ have the same sign as $u$.
We have $5$ cases to distinguish:
\begin{itemize}
\item $u>0$, $q>1$: Here, $a_n>0$ and $a_{n+1}=qa_n>a_n$ for all $n\geq 0$,
so the sequence is strictly monotonically increasing.
\item $u>0$, $0<q<1$: Here, $a_n>0$ and $a_{n+1}=qa_n<a_n$ for all $n\geq 0$,
so the sequence is strictly monotonically decreasing.
\item $u<0$, $q>1$: Here, $a_n<0$ and $a_{n+1}=qa_n<a_n$ for all $n\geq 0$,
so the sequence is strictly monotonically decreasing.
\item $u<0$, $0<q<1$: Here, $a_n<0$ and $a_{n+1}=qa_n>a_n$ for all $n\geq 0$,
so the sequence is strictly monotonically increasing.
\item  $q=1$: This is a constant sequence (see above).
\end{itemize}
}

\end{enumerate}
}
\tab{\lang{de}{Beschränktheit} \lang{en}{Boundedness}}
\lang{de}{
\begin{enumerate}
\item Konstante Folgen sind beschränkt. 
\item Die Folge $(a_n)_{n\in \N}=((-1)^n)_{n\in \N}$ hat als Wertemenge $\{-1;1\}$. Sie ist also auch beschränkt.
\item  Arithmetische Folgen, die nicht konstant sind, sind zwar streng monoton fallend oder wachsend und daher in eine Richtung beschränkt, aber nicht in die andere. 
\item Bei geometrischen Folgen hängt die Frage nach der Beschränktheit von $u$ und $q$ ab. \\
Ist $|q|\leq 1$ so ist die Folge beschränkt, denn $|a_n|=|uq^n|=|u|\cdot |q|^n \leq |u|$ für alle $n\in\Nzero$, 
weshalb die Folge durch $|u|+1$ nach oben beschränkt ist. Nach unten ist sie z.\,B. durch $-(|u|+1)$ beschränkt. \\
Für $q>1$ ist die Folge zwar monoton und daher in eine Richtung beschränkt, aber nicht in die andere. Und für $q<-1$
ist die Folge weder nach oben noch nach unten beschränkt. 
\end{enumerate}
}
\lang{en}{
\begin{enumerate}
\item Constant sequences are bounded. 
\item The sequence $(a_n)_{n\in \N}=((-1)^n)_{n\in \N}$ has the set of values $\{-1,1\}$ so it is also bounded.
\item  Arithmetic sequences that are not constant are strictly monotonic
and therefore bounded in one direction but not in the other.
\item For geometric sequences, boundedness depends on $q$. \\
If $|q| \leq 1$, then the sequence is bounded because $|a_n|=|uq^n| = |u|\cdot |q|^n \leq |u|$
for every $n \in \Nzero$, so the sequence is bounded from above
by $|u|+1$ and from below by $-|u|-1$.\\
For $q>1$, the sequence is monotonic and therefore bounded in one direction, but not in the other.
If $q<-1$, then the sequence is neither bounded from above nor from below.
\end{enumerate}
}
\end{tabs*}
\end{example}


\section{\lang{de}{Folgen und Konvergenz} \lang{en}{Sequences and convergence}}
\lang{de}{Dadurch, dass Folgen meist unendlich sind, stellt sich die Frage, ob sich die Folge \glqq langfristig einem Zielwert\grqq annähert. Dies führt auf den
Begriff des Grenzwerts. Eine Folge mit Grenzwert heißt \emph{konvergent}.}

\lang{en}{
The fact that sequences are generally infinite raises the question of whether 
a given sequence tends to a target value "in the long run".
This leads to the concept of a limit. A sequence with a limit is called \emph{convergent}.
}

\begin{definition}\label{def:folgenkonvergent}
\lang{de}{Man sagt, dass eine Folge reeller Zahlen $(a_n)_{n\in \N}$ gegen eine Zahl $a$ \notion{konvergiert}, wenn f"ur jeden noch so 
kleinen Abstand $\epsilon > 0$ eine natürliche Zahl $N$ existiert, sodass ab $a_N$ alle weiteren Glieder der Folge weniger als  
$\epsilon$ von $a$ entfernt sind. Mathematisch exakt heißt das
\[ | a_n - a | < \epsilon  \ \text{für alle } \ n \geq N.\]
Die Zahl $a$ nennt man \notion{Limes} oder \notion{Grenzwert} der 
Folge  $(a_n)_{n\in \N}$ und schreibt
\[  \lim_{n\to\infty} a_n =a. \]}
\lang{en}{A sequence of real numbers $(a_n)_{n\in \N}$ is said to
\notion{converge} to a number $a$ if, for every arbitrarily small
distance $\epsilon > 0$, there exists a natural number $N$ such that
all terms of the sequence after $a_N$ are at a distance of less
than $\epsilon$ from $a$. This is written mathematically as
\[ | a_n - a | < \epsilon  \ \text{for all } \ n \geq N.\]
The number $a$ is called the \notion{limit} of the sequence $(a_n)_{n\in \N}$,
written \[  \lim_{n\to\infty} a_n =a. \]
}

\lang{de}{Eine Folge, die gegen $0$ konvergiert, nennt man \notion{Nullfolge}. Eine Folge, die nicht konvergiert, 
nennt man \notion{divergent}. }
\lang{en}{
A sequence that converges to $0$ is also called a \notion{null sequence}.
A sequence that does not converge is called \notion{divergent}.
}

\lang{de}{
Gibt es für jedes $C >0$ eine natürliche Zahl $N$, sodass $x_n > C$ für alle $n \geq N$ gilt, dann schreiben wir
auch $\lim_{n \to \infty} x_n = \infty$. Genauso schreiben wir $\lim_{n \to \infty} x_n = -\infty$, wenn 
$x_n < -C$ für alle $n \geq N$ gilt. In diesen beiden Fällen nennen wir die 
Folge \notion{bestimmt divergent}.}
\lang{en}{
If, for every $C > 0$, there exists a natural number $N$ such that
$x_n > C$ for all $n \ge N$, then we write $\lim_{n \to \infty} x_n = \infty$.
Similarly, we write $\lim_{n \to \infty} x_n = -\infty$ if $x_n < -C$ for all $n \geq N$.
In both of these cases, the sequence is called \notion{properly divergent}.
}
\end{definition}

%\image{konvergenz1}

\begin{example}\label{ex:einfache-folgen}
\begin{enumerate}
\item \lang{de}{Für $c\in \R$ konvergiert die konstante Folge $(a_n)_{n\in \N}$ mit $a_n=c$ für alle $n\in \N$ 
gegen die Zahl $c$. Man kann nämlich für jedes $\epsilon>0$ die Zahl $N=1$ wählen und erhält für alle
$n\geq N=1$:
\[  | a_n -c|=|c-c|=0<\epsilon, \]
wie gefordert.}
\lang{en}{For any $c\in \R$, the constant sequence $(a_n)_{n\in \N}$ with $a_n=c$ for all $n \in \N$
converges to $c$. Namely, for every $\epsilon>0$ we can choose $N=1$ to find
\[  | a_n -c|=|c-c|=0<\epsilon \]
for all $n\geq N=1$.
}
\item \lang{de}{Die Folge $(a_n)_{n\in \N}$ mit $a_n=(-1)^n$ für alle $n\in \N$ konvergiert nicht, ist also
divergent. Wäre nämlich $a$ ein Grenzwert und z.\,B. $\epsilon = 0,01$, so müssten die Glieder der Folge 
ab einem $a_N$ alle im Intervall $[a-0,01; a + 0,01]$ liegen. Dieses Intervall ist aber sehr schmal
und insbesondere zu klein, um sowohl die Zahl $1$ als auch die Zahl $-1$ zu umfassen. }
\lang{en}{The sequence $(a_n)_{n\in \N}$ with $a_n=(-1)^n$ for all $n \in \N$ does not converge,
i.e. it is divergent. Namely, if $a$ were a limit and $\epsilon=0.01$ (for example)
then all of the terms of the sequence beginning with some $a_N$ would have to
lie in the interval $[a-0,01; a + 0,01]$. However, this interval is too narrow
to contain both the number $1$ and the number $-1$.
}
\item \lang{de}{\emph{Harmonische Folge:} Die Folge $(a_n)_{n\in \N}$ mit $a_n=\frac{1}{n}$ ist eine Nullfolge, d.\,h. $\lim_{n\to \infty} \frac{1}{n}=0$.\\ 
Zu $\epsilon>0$ kann man nämlich eine Zahl $N>\frac{1}{\epsilon}$ wählen und 
dann gilt für alle $n\geq N$:
\[ |a_n-0|=|\frac{1}{n}|=\frac{1}{n}<\frac{1}{N}<\epsilon. \]}
\lang{en}{\emph{Harmonic sequence:} The sequence $(a_n)_{n\in \N}$ with $a_n = \frac{1}{n}$ is a null sequence,
i.e. $\lim_{n\to \infty} \frac{1}{n}=0$.\\
Namely, for any $\epsilon>0$, we can choose a number $N>\frac{1}{\epsilon}$,
and then
\[ |a_n-0|=|\frac{1}{n}|=\frac{1}{n}<\frac{1}{N}<\epsilon\]
for every $n\geq N$.
}
\item \lang{de}{Die \emph{Eulersche Zahl} $e = 2,71828\ldots$ kann als Grenzwert der Folge $(a_n)$ mit $a_n = (1+\frac{1}{n})^n$ definiert werden. }
\lang{en}{\emph{Euler's number} $e = 2.71828\ldots$ can be defined
as the limit of the sequence $(a_n)_{n\in \N}$ with $a_n = (1+\frac{1}{n})^n$.}
\item \lang{de}{Nicht-konstante arithmetische Folgen und geometrische Folgen $(uq^n)_{n\in \Nzero}$ mit $|q|>1$ konvergieren
nicht, denn die Abstände zwischen den Gliedern der Folge werden nicht kleiner und können sich so nicht einer festen Zahl annähern. Die Folge ist insbesondere nicht 
beschränkt. }
\lang{en}{Nonconstant arithmetic sequences and geometric sequences $(uq^n)_{n\in \Nzero}$ with $|q|>1$
do not converge, as the distance between consecutive terms of the sequence does not decrease and in particular
do not tend to a fixed number. These sequences are not even bounded.}

\lang{de}{Geometrische Folgen $(uq^n)_{n\in \Nzero}$ mit $|q|<1$ sind Nullfolgen, denn die Potenzen einer Zahl, deren Betrag kleiner $1$ ist, nähern sich dem Wert $0$ 
immer weiter an. (Für $q = 0,5$ ist etwa $q^2 = 0,25$, $q^3 = 0,125$, $q^4 = 0,0625$, $\ldots$)}
\lang{en}{Geometric sequences $(uq^n)_{n\in \Nzero}$ with $|q|<1$ are null sequences, 
because the powers of a number whose absolute value is less than 1 tend to $0$.
(For $q=0.5$, for example, $q^2 = 0.25$, $q^3 = 0.125$, $q^4 = 0.0625$, $\ldots$)}
\end{enumerate}
\end{example}

\lang{de}{Das folgende Theorem ist zwar eher theoretischer Natur, aber kann in einigen Fällen benutzt werden, um die Existenz eines Grenzwertes auszuschließen. 
}
\lang{en}{
Although the following theorem is of a rather theoretical nature,
it can used in some cases to rule out the existence of a limit.
}

\begin{theorem}\label{thm:eindeutigkeitgw}
\begin{enumerate}
\item \lang{de}{Eine Folge besitzt höchstens einen Grenzwert. Das heißt, wenn $(a_n)_{n\in \N}$ konvergiert, ist der
Grenzwert eindeutig bestimmt.}
\lang{en}{
A sequence can have at most one limit. That is, if $(a_n)_{n\in \N}$ converges,
then its limit is unique.
}
\item \lang{de}{Eine konvergente Folge ist beschränkt.}
\lang{en}{Every convergent sequence is bounded.}
\end{enumerate}
\end{theorem}

\begin{proof*}
\lang{de}{Der Beweis des Theorems kann bei Interesse im Hauptkurs \ref[konvergenz][hier]{thm:eindeutigkeitgw} nachgelesen werden.}
\lang{en}{The interested reader can find the proof of this theorem
\ref[konvergenz][here]{thm:eindeutigkeitgw} in the main course.}
\end{proof*}


\lang{de}{
Um Grenzwerte konkret zu berechnen, sind die folgenden Regeln sehr hilfreich.
}

\lang{en}{
The following rules are very useful for computing limits in practice.
}

\begin{rule}[\lang{de}{Grenzwertregeln} \lang{en}{Limit rules}]\label{grenzwertregeln}
\lang{de}{Es seien  $(a_n)_{n\in \N}$, $(b_n)_{n\in \N}$ konvergente Folgen mit
$\lim_{n\to\infty} a_n =a$ und $\lim_{n\to\infty} b_n =b$. Dann gilt für alle $\alpha, \beta \in \R$:}
\lang{en}{
Let $(a_n)_{n\in \N}$, $(b_n)_{n\in \N}$ be convergent sequences with
$\lim_{n\to\infty} a_n =a$ and $\lim_{n\to\infty} b_n =b$.
For any $\alpha, \beta \in \R$:
}
\begin{enumerate}
\item \lang{de}{Die Folge $( \alpha a_n + \beta b_n )_{n\in \N}$ konvergiert und \[ \lim_{n\to\infty} ( \alpha a_n + \beta b_n )
= \alpha \cdot a + \beta \cdot b. \]}
\lang{en}{The sequence $( \alpha a_n + \beta b_n )_{n\in \N}$ converges and \[ \lim_{n\to\infty} ( \alpha a_n + \beta b_n )
= \alpha \cdot a + \beta \cdot b. \]}
\item \lang{de}{Die Folge $( a_n \cdot b_n )_{n\in \N}$ konvergiert und \[\lim_{n\to\infty} ( a_n \cdot b_n ) = a \cdot b. \]}
\lang{en}{The sequence $( a_n \cdot b_n )_{n\in \N}$ converges and \[\lim_{n\to\infty} ( a_n \cdot b_n ) = a \cdot b. \]}
\item \lang{de}{Falls $b_n \neq 0$ für alle $n \in \N$ und $b \neq 0$, so konvergiert die Folge $( \frac{a_n}{b_n} )_{n\in \N}$ und
\[  \lim_{n\to\infty} ( \frac{a_n}{b_n} )=\frac{a}{b}.\]}
\lang{en}{If $b_n \neq 0$ for all $n \in \N$ and $b \neq 0$, then the sequence $( \frac{a_n}{b_n} )_{n\in \N}$ converges and
\[  \lim_{n\to\infty} ( \frac{a_n}{b_n} )=\frac{a}{b}.\]}
\item \lang{de}{Gilt $a_n\leq b_n$ für alle $n\in \N$, dann gilt auch 
für die Grenzwerte
\[ \lim_{n\to \infty}  a_n \leq \lim_{n\to \infty} b_n.\]}
\lang{en}{If $a_n\leq b_n$ for all $n\in \N$, then this also
carries over to the limits:
\[ \lim_{n\to \infty}  a_n \leq \lim_{n\to \infty} b_n.\]}
\end{enumerate}
\end{rule}


\begin{proof*}
\lang{de}{Der Beweis der 1., 2. und 3. Aussage kann im Hauptkurs \ref[konvergenz][hier]{grenzwertregeln} nachgelesen werden. Es wird dort mit Hilfe
der Rechenregeln für Beträge die Definition von Konvergenz nachgewiesen. Der Beweis der 4. Aussage findet sich 
\ref[konvergenz][hier]{rule:grenzwerte-vergleichen}.}
\lang{en}{
The proofs of the 1st, 2nd and 3rd claims can be found 
\ref[konvergenz][here]{grenzwertregeln} in the main course. 
The proofs apply the rules for arithmetic with absolute values
to verify the definition of convergence. 
The proof of the 4th claim can be found
\ref[konvergenz][here]{rule:grenzwerte-vergleichen}.}
\end{proof*}

\begin{quickcheckcontainer}
\randomquickcheckpool{1}{1}
\begin{quickcheck}
		\field{real}
		\type{input.number}
		\begin{variables}
			\randint{a}{2}{6}
			\randint{b}{2}{8}
			\randint{c}{2}{6}
			\randint{d}{3}{6}
		    \function{ff}{a/n+b)}
            \function{gg}{c+(1/d)^n}
            \function{loes}{b/c}
		\end{variables}


         \lang{de}{\text{Bestimmen Sie den Grenzwert $a$ der konvergenten Folge
         $(a_n)_{n\in\N}$ definiert durch $a_n:=
         \frac{\frac{\var{a}}{n}+\var{b}}{\var{c}+(\frac{1}{\var{d}})^n}$\\\\
         
         Der Grenzwert von $(a_n)_{n\in\N}$ lautet: $a=$\ansref}}
         \lang{en}{\text{Find the limit $a$ of the convergence sequence
         $(a_n)_{n\in\N}$ defined by $a_n:=
         \frac{\frac{\var{a}}{n}+\var{b}}{\var{c}+(\frac{1}{\var{d}})^n}$\\\\
         
         The limit of $(a_n)_{n\in\N}$ is: $a=$\ansref}}
			

		\begin{answer}
        	\solution{loes}
	    \end{answer}
		\lang{de}{\explanation{Betrachten Sie die Grenzwerte der einzelnen konvergenten Folgen $p_{n}=
        \frac{\var{a}}{n}$, $q_{n}=p_{n}+\var{b}$, $r_{n}=(\frac{1}{\var{d}})^n$ 
        sowie $s_{n}=\var{c}+r_{n}$ und verwenden Sie dann die Grenzwertregeln. } }
        \lang{en}{
      \explanation{Consider the limits of the convergent sequences
      $\frac{\var{a}}{n}$, $q_{n}=p_{n}+\var{b}$, $r_{n}=(\frac{1}{\var{d}})^n$ 
      and $s_{n}=\var{c}+r_{n}$ separately, and then use the limit rules.}
        }
	\end{quickcheck}
\end{quickcheckcontainer}



\begin{example}
\begin{enumerate}
\item \lang{de}{Der Grenzwert der Folge $(a_n)_{n\in\N}$ definiert durch $a_n=\frac{1}{n^2}$
lässt sich mit Hilfe der Grenzwertregeln  bestimmen:
\[ \lim_{n\to \infty} \frac{1}{n^2}= (\lim_{n\to \infty} \frac{1}{n})\cdot (\lim_{n\to \infty} \frac{1}{n})=0\cdot 0=0. \]
Allgemeiner gilt sogar für alle $k\in\N$
\[\lim_{n\to \infty} \frac{1}{n^k}=(\lim_{n\to \infty} \frac{1}{n})^k =0^k=0.\]}
\lang{en}{The limit of the sequence $(a_n)_{n\in\N}$ defined by $a_n=\frac{1}{n^2}$
can be determined using the limit rules:
\[ \lim_{n\to \infty} \frac{1}{n^2}= (\lim_{n\to \infty} \frac{1}{n})\cdot (\lim_{n\to \infty} \frac{1}{n})=0\cdot 0=0. \]
More generally, for any $k\in\N$,
\[\lim_{n\to \infty} \frac{1}{n^k}=(\lim_{n\to \infty} \frac{1}{n})^k =0^k=0.\]}
\item \lang{de}{Für die Folge $(a_n)_{n\in \N}$ mit $a_n=\frac{2n^2-3n+1}{n^2+3n-3}$ lassen sich die Grenzwertregeln nicht direkt anwenden,
da die Folgen $(n^2)_{n\in \N}$ und $(n)_{n\in \N}$ nicht konvergieren. Wenn wir den Bruch jedoch durch die höchste Potenz
$n^2$ kürzen, erhalten wir 
\[ a_n=\frac{2n^2-3n+1}{n^2+3n-3}=\frac{2-3\cdot \frac{1}{n}+\frac{1}{n^2}}{1+3\cdot \frac{1}{n}-3\cdot \frac{1}{n^2}}.\]
Für den Zähler gilt
\[ \lim_{n\to \infty} \left(2-3\cdot \frac{1}{n}+\frac{1}{n^2}\right)= \lim_{n\to \infty} 2 -3\cdot\lim_{n\to \infty}\frac{1}{n}+
 \lim_{n\to \infty} \frac{1}{n^2} =2-3\cdot 0+0 =2, \]
und für den Nenner
\[ \lim_{n\to \infty} \left(1+3\cdot \frac{1}{n}-3\cdot \frac{1}{n^2} \right)=1+3\cdot 0 -3\cdot 0=1.\]
Man kann also die Regel für Quotienten anwenden und bekommt:
\[ \lim_{n\to \infty} a_n =\lim_{n\to \infty} \frac{2-3\cdot \frac{1}{n}+\frac{1}{n^2}}{1+3\cdot \frac{1}{n}-3\cdot \frac{1}{n^2}}
= \frac{2}{1}=2. \]}
\lang{en}{
The limit rules cannot be applied directly to the sequence $(a_n)_{n\in \N}$ with $a_n=\frac{2n^2-3n+1}{n^2+3n-3}$
because the sequences $(n^2)_{n\in \N}$ and $(n)_{n\in \N}$ do not converge.
However, if we divide both the numerator and denominator of the fraction
by the greatest power $n^2$, then we obtain
\[ a_n=\frac{2n^2-3n+1}{n^2+3n-3}=\frac{2-3\cdot \frac{1}{n}+\frac{1}{n^2}}{1+3\cdot \frac{1}{n}-3\cdot \frac{1}{n^2}}.\]
For the numerator, we have
\[ \lim_{n\to \infty} \left(2-3\cdot \frac{1}{n}+\frac{1}{n^2}\right)= \lim_{n\to \infty} 2 -3\cdot\lim_{n\to \infty}\frac{1}{n}+
 \lim_{n\to \infty} \frac{1}{n^2} =2-3\cdot 0+0 =2, \]
 and for the denominator,
 \[ \lim_{n\to \infty} \left(1+3\cdot \frac{1}{n}-3\cdot \frac{1}{n^2} \right)=1+3\cdot 0 -3\cdot 0=1.\]
Now we can use the limit rule for quotients to obtain
\[ \lim_{n\to \infty} a_n =\lim_{n\to \infty} \frac{2-3\cdot \frac{1}{n}+\frac{1}{n^2}}{1+3\cdot \frac{1}{n}-3\cdot \frac{1}{n^2}}
= \frac{2}{1}=2. \]
}
\item \lang{de}{Um den Grenzwert für die Folge $(a_n)_{n\in \N}$ mit $a_n=\frac{2n^2-3n+1}{n^3+2n^2-4}$ zu berechnen, geht man wie im vorigen
Beispiel vor und kürzt den Bruch zunächst durch die höchste Potenz des Nenners, hier also $n^3$:
\[ \lim_{n\to \infty} a_n =\lim_{n\to \infty} \frac{2n^2-3n+1}{n^3+2n^2-4}
=\lim_{n\to \infty}\frac{2\cdot \frac{1}{n} -3\cdot \frac{1}{n^2}+\frac{1}{n^3}}{1+2\cdot \frac{1}{n}-4\cdot \frac{1}{n^3}}
=\frac{2\cdot 0-3\cdot 0+0}{1+2\cdot 0-4\cdot 0}=\frac{0}{1}=0.\]
Die Folge ist also eine Nullfolge.}
\lang{en}{
To the determine the limit of the sequence $(a_n)_{n\in \N}$ with $a_n=\frac{2n^2-3n+1}{n^3+2n^2-4}$,
we emulate the previous example and divide both the numerator and denominator
by the highest power of the denominator, which in this case is $n^3$:
\[ \lim_{n\to \infty} a_n =\lim_{n\to \infty} \frac{2n^2-3n+1}{n^3+2n^2-4}
=\lim_{n\to \infty}\frac{2\cdot \frac{1}{n} -3\cdot \frac{1}{n^2}+\frac{1}{n^3}}{1+2\cdot \frac{1}{n}-4\cdot \frac{1}{n^3}}
=\frac{2\cdot 0-3\cdot 0+0}{1+2\cdot 0-4\cdot 0}=\frac{0}{1}=0.\]
Hence, this sequence is a null sequence.}
\item \lang{de}{Folgen von obigem Typ, bei denen die höchste Potenz im Zähler größer als die höchste Potenz im Nenner ist, konvergieren nicht,
wie wir am Beispiel $a_n=\frac{2n^4-3n+1}{n^3+2n^2-4}$ erläutern:\\
Zunächst kürzen wir wieder durch die höchste Potenz des Nenners, hier $n^3$, und erhalten
$a_n=\frac{2\cdot n -3\cdot \frac{1}{n^2}+\frac{1}{n^3}}{1+2\cdot \frac{1}{n}-4\cdot \frac{1}{n^3}}$.
Die Folge der Nenner $b_n=1+2\cdot \frac{1}{n}-4\cdot \frac{1}{n^3}$ ist nun konvergent gegen $1$.\\
Wäre die Folge $(a_n)_{n\in \N}$ konvergent, so wäre auch die Folge $(a_nb_n)_{n\in \N}$ konvergent, aber wegen
$a_nb_n=2\cdot n -3\cdot \frac{1}{n^2}+\frac{1}{n^3}>2\cdot n -3$ ist dies eine unbeschränkte Folge im Widerspruch zur Konvergenz.
}
\lang{en}{
Sequences of the above sort in which the highest power of $n$ is greater
than the highest power in the denominator do not converge,
as we will demonstrate with the example $a_n=\frac{2n^4-3n+1}{n^3+2n^2-4}$. \\
First, we divide by the highest power in the denominator as before,
i.e. $n^3$, to obtain
$a_n=\frac{2\cdot n -3\cdot \frac{1}{n^2}+\frac{1}{n^3}}{1+2\cdot \frac{1}{n}-4\cdot \frac{1}{n^3}}$.
The sequence of denominators $b_n=1+2\cdot \frac{1}{n}-4\cdot \frac{1}{n^3}$
converges to $1$. If the sequence $(a_n)_{n\in \N}$ were convergent,
then the sequence $(a_nb_n)_{n\in \N}$ would also converge, but
$a_nb_n=2\cdot n -3\cdot \frac{1}{n^2}+\frac{1}{n^3}>2\cdot n -3$
shows that the latter sequence is unbounded and in particular divergent.
}
\item \lang{de}{Die Folge $(a_n)_{n\in\mathbb{N}}=n^7\cdot 0,5^n$ konvergiert und ist eine Nullfolge. Mit den Regeln von de l'Hospital 
im Teil Differentialrechnung \ref[content_11_AbleitungSaetze][werden wir noch lernen]{thm:lhospital}, wie wir solche Grenzwerte bequem berechnen können. Ein elementarer Beweis 
findet sich im \ref[konvergenz][Hauptkurs]{thm:exp-schlaegt-pol}. }
\lang{en}{
The sequence $(a_n)_{n\in\mathbb{N}}=n^7\cdot 0.5^n$ converges and it is
a null sequence. In the section on differential calculus,
\ref[content_11_AbleitungSaetze][we will learn]{thm:lhospital} how to
calculate limits of this sort using L'Hôpital's rule.
An elementary proof may be found in the \ref[konvergenz][main course]{thm:exp-schlaegt-pol}.
}
\item \lang{de}{Die Folge $(a_n)_{n\in\mathbb{N}}=\frac{2^n}{n^7}$ konvergiert nicht, denn der Kehrwert 
ist $\frac{1}{a_n} = n^7 \cdot (0,5)^n$ und konvergiert gegen $0$ (siehe oben).

Generell gilt: Eine Folge kann nicht konvergieren, wenn ihr Kehrwert eine Nullfolge ist.}

\lang{en}{
The sequence $(a_n)_{n\in \N} =\frac{2^n}{n^7}$ does not converge, because the
reciprocal is $\frac{1}{a_n} = n^7 \cdot (0.5)^n$ which converges to $0$
(see above).

In general: a sequence does not converge if its reciprocals form a null sequence.
}



\end{enumerate}
\end{example}



\begin{block}[info]
\lang{de}{Bis wir im Teil Differentialrechnung weitere Regeln für Grenzwerte lernen, nutzen wir die folgende \textbf{Merkregel}:
\[\text{Exponentielles Wachstum setzt sich durch!}\]}

\lang{en}{Until we learn about further rules for limits in the
differential calculus sections, we will use the following \textbf{rule of thumb}:
\[\text{Exponential growth dominates!}\]
}
\end{block}



\section{\lang{de}{Reihen} \lang{en}{Series}}


\lang{de}{Eine Reihe ist eine besondere Folge. Man bildet sie aus einer anderen Folge, indem man deren Glieder aufsummiert.
}
\lang{en}{
A series is a particular type of sequence. Series are formed from sequences
by summing up their terms.
}
\begin{example}\label{ex:motivation-reihen}
\begin{tabs*}[\initialtab{0}]
\tab{\lang{de}{Kraftstoffverbrauch} \lang{en}{Fuel usage}}
\lang{de}{Ein Unternehmen hat eine neue Maschine angeschafft.
An jedem Tag wird die Menge des an dem Tag verbrauchten Kraftstoffes notiert.
Wir erhalten eine Folge $(a_n)_{n\in\N}$, die den Kraftstoffverbrauch der Maschine
am Tag $n$ seit der Anschaffung angibt.
Nun möchte der Geschäftsleiter wissen, 
wie viel Kraftstoff die Maschine im ersten Monat, im ersten Jahr oder allgemein nach Tag $n$ insgesamt verbraucht hat.

Um den Verbrauch nach Tag $n$ zu erhalten, müssen wir den jeweiligen Verbrauch $a_1, \ldots, a_n$ der ersten $n$ Tage aufsummieren.
Wir erhalten dann die Folge $(s_n)_{n\in \N}$ des Gesamtverbrauchs nach $n$ Tagen, die dann gegeben ist durch
\[
s_n = \sum_{k=1}^n a_k.
\]
Die Folge $(s_n)_{n\in \N}$ nennen wir dann auch Folge der Partialsummen oder auch \emph{Reihe}.
}
\lang{en}{
A company has bought a new machine.
Each day, the amount of fuel consumed by the machine is recorded.
We obtain a sequence $(a_n)_{n\in \N}$ that describes the fuel
used by the machine on the $n$-th day following its purchase.
The manager would like to know how much fule the machine has consumed
in the first month, the first year, or more generally by day $n$.\\

To determine the fuel used by day $n$, we must sum up the fuel usage
$a_1,...,a_n$ for the first $n$ days. We obtain the sequence $(s_n)_{n \in \N}$
describing the cumulative fuel consumption after $n$ days,
which is given by
\[
s_n = \sum_{k=1}^n a_k.
\]
The sequence $(s_n)_{n\in \N}$ is called the sequence of partial sums or \emph{series}.
}
\tab{\lang{de}{Geldschöpfung} \lang{en}{Money creation}}
\lang{de}{[Das folgende Beispiel zeigt eine traditionelle Sichtweise der Geldschöpfung mit der Bank als
Vermittlerin zwischen Sparern und Kreditnehmern. Diese Sicht ist nur bedingt richtig, aber als Modell verbreitet.]
}
\lang{en}{
[The following example illustrates a traditional understanding of
money creation with the bank as an intermediary between depositors
and borrowers. This point of view is only partially correct
but it is widespread as a model.
]
}

\lang{de}{Herr Müller zahlt seine Ersparnisse von 10.000 Euro auf ein Bankkonto ein. Die Bank muss von dieser Einlage 
10\% als Mindestreserve an die Notenbank transferieren, was Herrn Müller aber egal ist und er auch 
nicht mitkriegt. Am gleichen Tag vergibt die Bank einen Kredit von 9000 Euro 
an einen Koch, der dadurch eine neue Küche finanziert. }
\lang{en}{Mr. Mueller deposits his savings of 10,000 euros into a
bank account. The bank must transfer at least 10\% of the deposit
as a reserve to the central bank, which is fine with Mr. Mueller
and which he does not even notice.
On the same day, the bank extends a loan of 9000 euros to a cook
who uses it to finance a new kitchen.
}

\lang{de}{Über diverse Geschäfte gelangen die 9000 Euro an einen anderen Kunden der Bank, Frau Wagner, die diese wiederum
auf ihr Konto einzahlt. Wieder gehen 10\%, also diesmal 900 Euro, als Mindestreserve an die Notenbank. Am gleichen 
Tag vergibt die Bank einen Kredit über 8100 an einen anderen Kunden.}
\lang{en}{
Through a series of transactions, those 9000 euros make their way to another
customer of the bank, Ms. Wagner, who in turn deposits them into her
account. Again, 10\%, which in this case is 900 euros, is sent to the
central bank as a minimal reserve. On the same day, the bank
extends a loan of 8100 euros to a new customer.
}

\lang{de}{Obwohl es anfangs nur die 10.000 Euro von Herrn Müller gab, sind auf den Konten der Bank jetzt 19.000 Euro 
verbucht. Hier ist neues (Buch-)Geld entstanden.}
\lang{en}{
Although there were only Mr. Mueller's 10,000 euros in the beginning,
there are now 19,000 euros in the accounts of the bank.
New (broad) money has been created.
}

\lang{de}{Wenn die 8100 Euro, die die Bank als Kredit vergeben hat, über Umwege wieder auf ein Konto eingezahlt werden, 
erhöht sich die Geldmenge sogar noch weiter. Es stellt sich die Frage, ob am Ende unendlich viel Geld verbucht ist. }

\lang{en}{If the 8100 euros that the bank has lent out
make their way back into the bank's accounts then the amount of money
will continue to increase.
This raises the question of whether an infinite amount of money
will have been deposited in the end.
}

\lang{de}{Die eingezahlten Beträge bilden eine geometrische Folge $(a_n)$ mit $a_n = 10.000 \cdot (0,9)^n$, weil 10\% Mindestreserve
 jeweils abgehen. Die Gesamtsumme nach \glqq unendlich vielen\grqq Schritten ist 
 \[
 \sum_{n=0}^\infty 10.000 \cdot (0,9)^n = 10.000 \sum_{n=0}^\infty (0,9)^n.
\]
Wir werden mit Hilfe der geometrischen Reihe sehen, dass dieser Wert endlich ist, nämlich 100.000 Euro.}

\lang{en}{
The deposited sums form a geometric sequence $(a_n)$ with $a_n = 10,000 \cdot (0.9)^n$,
since a 10\% reserve is deducted each time. The total sum
after "infinitely many" steps is
 \[
 \sum_{n=0}^\infty 10,000 \cdot (0.9)^n = 10,000 \sum_{n=0}^\infty (0.9)^n.
\]
We will use the geometric series to see that this value is finite;
in fact, it is 100,000 euros.
}

\end{tabs*}
\end{example}
%Ende erstes Motivationsbeispiel

\begin{definition}\label{def:reihe}
\lang{de}{Gegeben sei eine Folge $( a_k )_{k\in \N}$. Dann nennen wir 
\[   s_n=a_1+a_2+\ldots+a_n=\sum_{i=1}^n a_i \]
die $n$-te \notion{Partialsumme}. Es ist also $s_1=a_1$, $s_2=a_1+a_2$, $s_3=a_1+a_2+a_3$, etc.}
\lang{en}{Given the sequence $(a_k)_{k \in \N}$, we call
\[   s_n=a_1+a_2+\ldots+a_n=\sum_{i=1}^n a_i \]
the $n$-th \notion{partial sum}. Thus $s_1=a_1$, $s_2=a_1+a_2$, $s_3=a_1+a_2+a_3$, etc.
}

\lang{de}{Eine \notion{Reihe} ist eine Folge von Partialsummen $(s_n)_{n \in \N}$.
Wir schreiben dafür abkürzend  auch
\[ \sum_{i=1}^\infty a_i. \]
Wir beachten aber, dass diese Schreibweise nichts darüber aussagt, ob der Grenzwert der Partialsummen existiert.
}
\lang{en}{
A \notion{series} is a sequence of partial sums $(s_n)_{n \in \N}$.
We abbreviate this with \[ \sum_{i=1}^\infty a_i. \]
Note, however, that this notation does not imply anything
as to whether or not the limit of the partial sums exists.
}
\end{definition}
\lang{de}{Wie auch schon bei Folgen muss der Start-Index nicht immer $n=1$ sein, sondern kann eine 
beliebige ganze Zahl sein.}
\lang{en}{Similarly to sequences, the index need not always start at $n=1$; it can start at any integer.}



\lang{de}{Bisher ist die Schreibweise $\sum_{k=1}^\infty a_k$ für eine Reihe lediglich ein Symbol, da man unendlich viele Zahlen
nicht aufsummieren kann. Wenn sich die Summe aber einem Grenzwert immer weiter annähert, je mehr Summanden man addiert hat, 
dann können wir diesen Grenzwert als Wert der unendlichen Summe nehmen.

Mathematisch exakter:}

\lang{en}{
So far, the notation $\sum_{k=1}^{\infty} a_k$ for a series is merely symbolic,
as it is not possible to sum infinitely many numbers.
However, if the sum approaches a limit as one includes more and more
summands, then we can view this limit as the value of the infinite sum.

More rigorously,
} 
\begin{definition}
\lang{de}{Eine Reihe $\sum_{k=1}^\infty a_k$ heißt \notion{konvergent}, wenn die Folge $(s_n)_{n\in \N}$ ihrer endlichen Partialsummen 
$s_n=\sum_{k=1}^n a_k$ gegen eine reelle Zahl $S$ konvergiert. Wir schreiben dann auch
\[  \sum_{k=1}^\infty a_k= S. \] 

Ist die Reihe $\sum_{k=1}^\infty a_k$ nicht konvergent, so heißt sie \notion{divergent}. 

Eine Reihe $\sum_{k=1}^\infty a_k$ heißt \notion{absolut konvergent}, 
wenn die Reihe der Absolutbeträge $\sum_{k=1}^\infty |a_k|$ konvergiert.
}
\lang{en}{
A series $\sum_{k=1}^\infty a_k$ is called \notion{convergent}
if the sequence $(s_n)_{n \in \N}$ of its finite partial sums
$s_n = \sum_{k=1}^n a_k$ converges to a real number $S$.
In this case, we write
\[  \sum_{k=1}^\infty a_k= S. \] 

If the series $\sum_{k=1}^\infty a_k$ does not converge
then we call it \notion{divergent}.

A series $\sum_{k=1}^\infty a_k$ is called \notion{absolutely convergent}
if the series of absolute values $\sum_{k=1}^\infty |a_k|$ converges.
}
\end{definition}

\lang{de}{In der Praxis haben wir meistens mit den folgenden beiden Beispielen zu tun, die wir daher etwas ausführlicher besprechen.
}
\lang{en}{
In practice, we will usually encounter one of two
examples below, which we therefore consider in more detail.
}

\begin{example}[\lang{de}{Arithmetische Reihe} \lang{en}{Arithmetic series}]\label{ex:arithmetischeReihe}
\lang{de}{Es sei eine arithmetische Folge $(a_k)_{k \geq 0}$ mit $a_k = c + d\cdot k$ gegeben. Wenn wir zu dieser speziellen 
Folge die Partialsummen $(s_n)_{n \geq 0}$ bilden, erhalten wir nach umsortieren
\begin{align*}
s_n &= (c + 0 \cdot d) + (c + 1 \cdot d) + \ldots + (c + n \cdot d) \\
    &= c + c + \ldots + c + d + 2 \cdot d + \ldots + n \cdot d \\
    &= (n+1) \cdot c + d \cdot (1 + 2 + \ldots + n).
\end{align*}
Die Zahlen von $1$ bis $n$ ergeben aufsummiert den Wert $\frac{n(n+1)}{2}$, wie im Hauptkurs
\ref[content_02_vollstaendige_induktion][hier]{ex:vollst-ind} bewiesen wird. Wenn wir dies einsetzen, lautet die Formel für $s_n$
\begin{align*}
s_n = \sum_{k=0}^n \left(c + d \cdot k \right)= (n+1) \cdot c + d \cdot \frac{n(n+1)}{2}
\end{align*}
und dieser Ausdruck wird auch als \emph{arithmetische Summe} oder \emph{arithmetische Reihe} bezeichnet.
Durch weiteres umformen können wir dies auch umschreiben zu 
\begin{align*}
s_n = (n+1) \cdot c + d \cdot \frac{n(n+1)}{2} = \frac{n+1}{2} \cdot \left( 2c + n \cdot d  \right)
 = \frac{n+1}{2} \cdot \left( a_0  + a_n  \right).
\end{align*}
Mit dieser Formel lassen sich arithmetische Folgen schnell und einfach aufsummieren. 
Wegen $\lim_{n \to \infty} s_n = \pm \infty$ konvergiert die unendliche Reihe $\sum_{k=0}^\infty \left( c + k \cdot d \right)$ 
jedoch nie.}

\lang{en}{
Suppose we aregiven an arithmetic sequence $(a_k)_{k\geq 0}$ with $a_k=c+d \cdot k$.
When we form the partial sums $(s_n)_{n \geq 0}$ of this particular sequence,
then after rearranging we find
\begin{align*}
s_n &= (c + 0 \cdot d) + (c + 1 \cdot d) + \ldots + (c + n \cdot d) \\
    &= c + c + \ldots + c + d + 2 \cdot d + \ldots + n \cdot d \\
    &= (n+1) \cdot c + d \cdot (1 + 2 + \ldots + n).
\end{align*}
Adding the numbers $1$ to $n$ together yields the value $\frac{n(n+1)}{2}$,
as proved \ref[content_02_vollstaendige_induktion][here]{ex:vollst-ind}
in the main course. After plugging this in, the formula for $s_n$
becomes
\begin{align*}
s_n = \sum_{k=0}^n \left(c + d \cdot k \right)= (n+1) \cdot c + d \cdot \frac{n(n+1)}{2}.
\end{align*}
This expression is also called an \emph{arithmetic sum} or \emph{arithmetic series}.
By a further rearrangement, we can rewrite this as
\begin{align*}
s_n = (n+1) \cdot c + d \cdot \frac{n(n+1)}{2} = \frac{n+1}{2} \cdot \left( 2c + n \cdot d  \right)
 = \frac{n+1}{2} \cdot \left( a_0  + a_n  \right).
\end{align*}
This formula makes it possible to sum up arithmetic sequences 
quickly and easily. However, since $\lim_{n \to \infty} s_n = \pm \infty$,
the infinite series $\sum_{k=0}^\infty (c+k \cdot d)$ never converges.
}
\end{example}


\begin{example}[\lang{de}{Geometrische Reihe} \lang{en}{Geometric series}]\label{ex:konvergenz-geo-reihe}
\lang{de}{Es sei $q\in \R$. Die \emph{geometrische Reihe} 
\[\sum_{k=0}^\infty q^k\]
ist eine Reihe, die sich dadurch auszeichnet, dass eine geometrische Folge $(q^k)_{k\geq 0}$ summiert wird.
Die Folge der Partialsummen $(s_n)_{n \geq 0}$ ist die Folge der geometrischen Summen
$s_n:=\sum_{k=0}^n q^k$. Für diese Partialsummen gibt es eine Formel, mit der die Summe schnell berechnet werden kann:
\[
s_n = 1 + q + q^2 + \ldots + q^n = \frac{1-q^{n+1}}{1-q},
\]
wenn $q \neq 1$ ist. Ein Beweis dieser Formel findet sich im \ref[content_02_vollstaendige_induktion][Hauptkurs]{rule:geom-summe}. Wir untersuchen
nun das Konvergenzverhalten der geometrischen Reihe.
\begin{enumerate}
\item Für $q=1$ ist $s_n=\sum_{k=0}^n 1^k=n+1$. Diese Folge
ist bestimmt divergent gegen $\infty$, also ist in diesem Fall
\[ \sum_{k=0}^\infty 1^k =\infty.\] Dies ist auch der einzige Fall, in dem die geometrische Summenformel nicht 
angewendet werden kann, um die Partialsummen auszurechnen. 
\item Für $|q|<1$ ist nach der geometrischen Summenformel
\[ s_n=\sum_{k=0}^n q^k=\frac{1-q^{n+1}}{1-q} \]
und $\lim_{n\to \infty} q^{n+1}=0$. 
Daher ist nach den Grenzwertregeln
\[ \lim_{n\to \infty} s_n=\lim_{n\to \infty}\frac{1-q^{n+1}}{1-q}
=\frac{1}{1-q}.\]
Die Reihe ist also konvergent und
\[ \sum_{k=0}^\infty q^k =\frac{1}{1-q}.\]
\item Falls $|q|\geq 1$, aber $q\neq 1$, ist immer noch 
\[ s_n=\sum_{k=0}^n q^k=\frac{1-q^{n+1}}{1-q}. \]
In diesem Fall konvergiert die Folge $(q^{n+1})_{n\geq 0}$ jedoch nicht, weshalb auch die Folge $(s_n)_{n\geq 0}$ nicht konvergiert.

Die Reihe $\sum_{k=0}^\infty q^k$ konvergiert also hier nicht.
\end{enumerate}
}
\lang{en}{
Let $q\in \R$. The \emph{geometric series} 
\[\sum_{k=0}^\infty q^k\]
is a series obtained by summing a geometric sequence $(q^k)_{k\geq 0}$.
The sequence $(s_n)_{n \geq 0}$ of partial sums is the sequence of geometric sums
$s_n:=\sum_{k=0}^n q^k$. 
For these partial sums, there is a formula with which the sum can be calculated quickly:
\[
s_n = 1 + q + q^2 + \ldots + q^n = \frac{1-q^{n+1}}{1-q},
\]
 A proof of this formula is given in \ref[content_02_vollstaendige_induktion][the main course]{rule:geom-summe}.
We will now investigate when the geometric series converges.

\begin{enumerate}
\item If $q=1$, then $s_n=\sum_{k=0}^n 1^k=n+1$. 
This sequence diverges properly to $\infty$, so
\[ \sum_{k=0}^\infty 1^k =\infty.\] 
This is also the only case in which the formula for geometric sums
cannot be applied to compute the partial sums.
\item If $|q|<1$, then the formula for geometric sums yields
\[ s_n=\sum_{k=0}^n q^k=\frac{1-q^{n+1}}{1-q} \]
and $\lim_{n\to \infty} q^{n+1}=0$. 
By the limit rules,
\[ \lim_{n\to \infty} s_n=\lim_{n\to \infty}\frac{1-q^{n+1}}{1-q}
=\frac{1}{1-q}.\]
So this series converges, and
\[ \sum_{k=0}^\infty q^k =\frac{1}{1-q}.\]
\item If $|q|\geq 1$ but $q\neq 1$, then
\[ s_n=\sum_{k=0}^n q^k=\frac{1-q^{n+1}}{1-q}. \]
In this case, the sequence $(q^{n+1})_{n\geq 0}$
does not converge, so the sequence $(s_n)_{n\geq 0}$
does not converge either.

Therefore, the series $\sum_{k=0}^\infty q^k$ does not converge
in this case.
\end{enumerate}
}
\end{example}

\begin{block}[warning]\label{rem:reihenfolge-summation}
\lang{de}{Bei endlichen Summen ist es egal, in welcher Reihenfolge man ihre Summanden aufaddiert.
Bei \glqq{}unendlichen Summen\grqq{} ist das nicht mehr der Fall. Wir müssen daher Summanden
in der vorgegebenen Reihenfolge addieren. 

Wenn eine Reihe absolut konvergiert, darf man die Summationsreihenfolge beliebig ändern.}

\lang{en}{
For finite sums, the order in which the summands are added together does not matter.
That is no longer the case for "infinite sums".
We have to add the summands together in the specified order.

If a series converges absolutely, then the order of summation can be changed arbitrarily.
}
\end{block}

\lang{de}{Aus den Grenzwertregeln für Folgen bekommt man direkt auch Grenzwertregeln für Reihen.}
\lang{en}{The limit rules for sequences immediately yield
limit rules for series.}

\begin{rule}[\lang{de}{Grenzwertregeln} \lang{en}{limit rules}]\label{rul:grenzwert-regeln}
\lang{de}{Es seien  $\sum_{k=1}^\infty a_k=A$ und $\sum_{k=1}^\infty b_k=B$ konvergente Reihen. Dann gilt für alle $\alpha, \beta \in \R$:

Die Reihe $\sum_{k=1}^\infty ( \alpha a_k + \beta b_k)$ konvergiert und es ist
\[ \sum_{k=1}^\infty ( \alpha a_k + \beta b_k)
= \alpha \cdot A + \beta \cdot B. \]}
\lang{en}{
Suppose $\sum_{k=1}^\infty a_k=A$ und $\sum_{k=1}^\infty b_k=B$ are convergent series.
For any $\alpha, \beta \in \R$, the series $\sum_{k=1}^\infty ( \alpha a_k + \beta b_k)$ converges and
\[ \sum_{k=1}^\infty ( \alpha a_k + \beta b_k)
= \alpha \cdot A + \beta \cdot B. \]
}
\end{rule}



\begin{quickcheck}
  \type{input.number}
  \field{rational}
 
  \begin{variables}
   \randint{n}{2}{10}
   \function[calculate]{s}{n/(n-1)}
   \function[calculate]{t}{s-1}
  \end{variables}

   \lang{de}{
  \text{
    Bestimmen Sie die Grenzwerte der Reihen
    $\sum_{k=0}^\infty (\frac{1}{\var{n}})^k$ und $\sum_{k=1}^\infty (\frac{1}{\var{n}})^k$.
   
   $\sum_{k=0}^\infty (\frac{1}{\var{n}})^k=$\ansref
   
   $\sum_{k=1}^\infty (\frac{1}{\var{n}})^k=$\ansref
   }}
   \lang{en}{
  \text{
    Find the limits of the series
    $\sum_{k=0}^\infty (\frac{1}{\var{n}})^k$ and $\sum_{k=1}^\infty (\frac{1}{\var{n}})^k$.
   
   $\sum_{k=0}^\infty (\frac{1}{\var{n}})^k=$\ansref
   
   $\sum_{k=1}^\infty (\frac{1}{\var{n}})^k=$\ansref
   }}
  
 
  \begin{answer}
  %\text{$\sum_{k=0}^\infty (\frac{1}{\var{n}})^k=$}
    \solution{s}
  \end{answer}
  \begin{answer}
  %\text{$\sum_{k=1}^\infty (\frac{1}{\var{n}})^k=$}
    \solution{t}
  \end{answer}
\end{quickcheck}


\section{\lang{de}{Konvergenzkriterien für Reihen} \lang{en}{Convergence tests for series}}

\lang{de}{Nachdem wir den Begriff der Konvergenz für Reihen definiert haben und auch wissen, dass die Grenzwertsätze gelten, stehen wir aber 
immer noch vor dem Problem, dass wir nur sehr wenige konvergente Reihen kennen. Um eine Reihe auf Konvergenz zu untersuchen, gibt es
verschiedene Kriterien, die wir hier kurz vorstellen. Viele dieser Kriterien nutzen den Vergleich zu anderen Reihen, von denen 
man schon weiß, dass sie konvergieren oder divergieren. Wir starten daher mit Beispielen.}

\lang{en}{
Although we have now defined the notion of convergence for series and we know that
the limit rules hold, we still have the problem that we know only very few examples of convergent series.
To test whether a series converges, there are several different criteria that can be used,
and we will quickly introduce them here. Many of these tests involve comparing with a different series
for which we already know that it converges or diverges. Therefore, we begin with some examples.
}

\begin{example}\label{ex:erste-reihen}
\begin{tabs*}[\initialtab{0}]
\tab{$\sum_{k=1}^\infty \frac{1}{k(k+1)}=1$}
\lang{de}{Für die Reihe $\sum_{k=1}^\infty \frac{1}{k(k+1)}$ sind die Partialsummen gegeben durch $s_n=\sum_{k=1}^n \frac{1}{k(k+1)}$.
Durch die Umformung
\[ \frac{1}{k(k+1)}=\frac{1+k-k}{k(k+1)}=
\frac{1+k}{k(k+1)}-\frac{k}{k(k+1)}=\frac{1}{k}-\frac{1}{k+1}\]
für jedes $k\in \N$ lässt sich $s_n$ einfacher schreiben als
\begin{eqnarray*}
 s_n &=& \sum_{k=1}^n \frac{1}{k(k+1)}=\sum_{k=1}^n (\frac{1}{k}-\frac{1}{k+1}) \\
 &=& \sum_{k=1}^n \frac{1}{k}-\sum_{k=1}^n \frac{1}{k+1} 
 = \sum_{k=1}^n \frac{1}{k} -\sum_{j=2}^{n+1} \frac{1}{j} \\
 &=& 1 -\frac{1}{n+1},
\end{eqnarray*}
wobei im vorletzten Schritt eine Indexverschiebung bei der Summe durchgeführt wurde.
Es folgt
\[ \sum_{k=1}^\infty \frac{1}{k(k+1)}=\lim_{n\to \infty} s_n=\lim_{n\to \infty} \left(1 -\frac{1}{n+1}\right) =1\]
und die Reihe ist also konvergent.}
\lang{en}{The series $\sum_{k=1}^\infty \frac{1}{k(k+1)}$ has the partial sums $s_n=\sum_{k=1}^n \frac{1}{k(k+1)}$.
By substituting
\[ \frac{1}{k(k+1)}=\frac{1+k-k}{k(k+1)}=
\frac{1+k}{k(k+1)}-\frac{k}{k(k+1)}=\frac{1}{k}-\frac{1}{k+1}\]
for every $k\in \N$, we can rearrange $s_n$ to the simpler sum
\begin{eqnarray*}
 s_n &=& \sum_{k=1}^n \frac{1}{k(k+1)}=\sum_{k=1}^n (\frac{1}{k}-\frac{1}{k+1}) \\
 &=& \sum_{k=1}^n \frac{1}{k}-\sum_{k=1}^n \frac{1}{k+1} 
 = \sum_{k=1}^n \frac{1}{k} -\sum_{j=2}^{n+1} \frac{1}{j} \\
 &=& 1 -\frac{1}{n+1},
\end{eqnarray*}
where the second-to-last step involves shifting an index in the sum.
Therefore,
\[ \sum_{k=1}^\infty \frac{1}{k(k+1)}=\lim_{n\to \infty} s_n=\lim_{n\to \infty} \left(1 -\frac{1}{n+1}\right) =1\]
so the series converges.}
\tab{\lang{de}{Harmonische Reihe $\sum_{k=1}^\infty \frac{1}{k}=\infty$}
\lang{en}{Harmonic series  $\sum_{k=1}^\infty \frac{1}{k}=\infty$}
}
\lang{de}{Die harmonische Reihe $\sum_{k=1}^\infty \frac{1}{k}$ 
entsteht durch Summation der harmonischen Folge.
Sie ist ein zentrales Beispiel für eine divergente Reihe.

Da alle Summanden $\frac{1}{k}$ positiv sind, ist die Folge der
 Partialsummen $s_n=\sum_{k=1}^n \frac{1}{k}$ streng monoton wachsend. 
 Um zu sehen, dass die Reihe nicht konvergiert, betrachten wir
 die Partialsummen $s_n$, wenn $n=2^m$  ($m\in \N$) eine Zweierpotenz ist. Man kann mit der Beweismethode der 
\emph{vollständigen Induktion} zeigen, dass
 \[ s_{2^m} \geq 1+m\cdot \frac{1}{2} \]
 für alle $m \in \mathbb{N}$ gilt. Wir verzichten hier auf den Beweis, aber wir können aus der Ungleichung folgern, 
dass die Teilfolge $(s_{2^m})_{m\in \N}$ nicht beschränkt ist. Und daher wächst auch die Folge $(s_{n})_{n\in \N}$ unbeschränkt.}
\lang{en}{
The harmonic series $\sum_{k=1}^\infty \frac{1}{k}$ arises by
summing the harmonic sequence. It is a central example of a divergent series.

Since all summands $\frac{1}{k}$ are positive, the sequence 
$s_n=\sum_{k=1}^n \frac{1}{k}$ of partial sums is strictly monotonically increasing.
To see that the series does not converge, we consider the partial sums
$s_n$ for powers of two, $n=2^m$ ($m \in \N$).
Using the proof method of \emph{complete induction}, one can show
that \[ s_{2^m} \geq 1+m\cdot \frac{1}{2} \] holds for all $m \in \mathbb{N}$.
We omit the proof here, but the inequality implies that the
subsequence $(s_{2^m})_{m\in \N}$ is unbounded.
Therefore, the full sequence $(s_n)_{n\in \N}$ is also unbounded.
}
\end{tabs*}
\end{example}

\lang{de}{Es folgt nun eine Auflistung verschiedener Konvergenzkriterien für Reihen. Mit einigen Kriterien kann nur die Konvergenz oder nur die
Divergenz nachgewiesen werden und es hängt auch immer von der gegebenen Reihe ab, welches Kriterium gewählt werden sollte. }

\lang{en}{We will now list a variety of convergence tests for series.
Some tests can only be used to prove convergence, some can only prove
divergence, and the choice of which test to use always
depends on the given series.}

\begin{theorem}[\lang{de}{Kriterien für Konvergenz} \lang{en}{Convergence tests}]
\lang{de}{Wir betrachten die Reihe $\sum_{k=1}^\infty a_k$ und die Folge der Summanden $(a_k)$. }
\lang{en}{Consider the series $\sum_{k=1}^\infty a_k$ and the sequence of summands $(a_k)$.}
\begin{itemize}
\item \lang{de}{Wenn die Reihe $\sum_{k=1}^\infty |a_k|$ konvergiert, dann konvergiert auch $\sum_{k=1}^\infty a_k$. 
(Aus absoluter Konvergenz folgt Konvergenz.)}
\lang{en}{If the series $\sum_{k=1}^\infty |a_k|$ converges, then the series $\sum_{k=1}^\infty a_k$ also converges.
(Absolute convergence implies convergence.)
}

\item \lang{de}{Wenn $(a_k)_{k\geq 1}$ keine Nullfolge ist, so ist $\sum_{k=1}^\infty a_k$ divergent.}
\lang{en}{If $(a_k)_{k\geq 1}$ is not a null sequence,
then $\sum_{k=1}^\infty a_k$ diverges.}

\item \lang{de}{Ist $(a_k)_{k\geq 1}$ eine monoton fallende Nullfolge ($a_1\geq a_2 \geq a_3 \geq \ldots \geq 0$), so ist die 
\emph{alternierende} Reihe
\[ \sum_{k=1}^\infty (-1)^k a_k \]
konvergent. Dies ist das \emph{Leibniz-Kriterium}.}
\lang{en}{
If $(a_k)_{k\geq 1}$ is a monotonically decreasing null sequence ($a_1\geq a_2 \geq a_3 \geq \ldots \geq 0$),
then the \emph{alternating} series
\[ \sum_{k=1}^\infty (-1)^k a_k \]
converges. This is the \emph{alternating series test}, or \emph{Leibniz test}.
}

\item \lang{de}{Ist $\sum_{k=1}^\infty c_k$ eine konvergente Reihe und gilt
  $|a_k|\leq c_k$ für alle $k\geq 1$, so ist die Reihe $\sum_{k=1}^\infty a_k$ absolut konvergent. Dies ist das 
  \emph{Majorantenkriterium}.}
  \lang{en}{ If $\sum_{k=1}^\infty c_k$ is a convergent series and
  $|a_k|\leq c_k$ for all $k\geq 1$, then the series $\sum_{k=1}^\infty a_k$ converges absolutely.
  This is the \emph{comparison test}.
  }
\item \lang{de}{Ist $\sum_{k=1}^\infty d_k$ eine divergente Reihe und gilt
$a_k\geq d_k \geq 0$ für alle $k\geq 1$, 
so ist die Reihe $\sum_{k=1}^\infty a_k$ divergent. Dies ist das \emph{Minorantenkriterium}.
}
\lang{en}{If $\sum_{k=1}^\infty d_k$ is a divergent series and
  $a_k\geq d_k \geq 0$ for all $k\geq 1$, then the series $\sum_{k=1}^\infty a_k$ diverges.
  This is also known as the \emph{comparison test}.
}
\item \lang{de}{Wenn $a_k \neq 0$ für alle $k \in \N$ gilt und der Grenzwert
\[ g:=  \lim_{k \to \infty}|\frac{a_{k+1}}{a_k}|\]
existiert, dann besagt das \emph{Quotientenkriterium}: \\
Im Fall $g<1$ ist die Reihe $\sum_{k=1}^\infty a_k$ absolut konvergent. \\
Im Fall $g>1$ ist die Reihe $\sum_{k=1}^\infty a_k$ divergent. \\
Im Fall $g=1$ können wir mit diesem Kriterium keine Aussage treffen.}
\lang{en}{If $a_k \neq 0$ for all $k \in \N$, and if the limit
\[ g:=  \lim_{k \to \infty}|\frac{a_{k+1}}{a_k}|\]
exists, then the \emph{ratio test} states the following: \\
If $g<1$, then the series $\sum_{k=1}^\infty a_k$ converges absolutely. \\
If $g>1$, then the series $\sum_{k=1}^\infty a_k$ diverges. \\
When $g=1$, the ratio test does not apply.
}
\end{itemize}


\end{theorem}

\begin{block}[warning]
\begin{itemize}

\item \lang{de}{Die Bedingung, dass $(a_k)_{k\geq 1}$ eine Nullfolge ist, ist \emph{nicht} ausreichend dafür, dass die Reihe  $\sum_{k=1}^\infty a_k$ konvergiert,
wie man anhand der harmonischen Reihe gut sehen kann.}
\lang{en}{The assumption that $(a_k)_{k\geq 1}$ is a null sequence
is \emph{not} sufficient for the series $\sum_{k=1}^\infty a_k$ to
converge, as one can see from the example of the harmonic series.}

\item \lang{de}{Ist keine der obigen Bedingungen erfüllt, so lässt sich auf diesem Weg keine Aussage zur Konvergenz treffen.}
\lang{en}{When none of the conditions above are satisfied,
we cannot use this fact to deduce anything about the convergence
of the series.}
\end{itemize}
\end{block}


\begin{example}\label{ex:absolut-konvergenz}
\begin{tabs*}
\tab{\lang{de}{Minorantenkriterium} \lang{en}{Comparison test}}
\lang{de}{Wir betrachten die Reihe $\sum_{k=1}^\infty \frac{1}{\sqrt{k}}$. 

Für jede natürliche Zahl $k \in \N$ gilt 
\[
\sqrt{k} \leq k \quad \text{bzw.} \quad \frac{1}{\sqrt{k}} \geq \frac{1}{k}.
\]
Wir wissen, dass die harmonische Reihe $\sum_{k=1}^\infty \frac{1}{k}$ divergiert. Nach dem Minorantenkriterium divergiert daher 
auch die Reihe $\sum_{k=1}^\infty \frac{1}{\sqrt{k}}$.
}
\lang{en}{
Consider the series $\sum_{k=1}^\infty \frac{1}{\sqrt{k}}$.

For any natural number $k\in \N$, we have
\[
\sqrt{k} \leq k \quad \text{and therefore} \quad \frac{1}{\sqrt{k}} \geq \frac{1}{k}.
\]
We know that the harmonic series $\sum_{k=1}^\infty \frac{1}{k}$ diverges.
By the comparison test, the series $\sum_{k=1}^\infty \frac{1}{\sqrt{k}}$ also diverges.
}


 \tab{\lang{de}{Alternierende harmonische Reihe}
 \lang{en}{Alternating harmonic series}
 }
\lang{de}{Die alternierende harmonische Reihe ist die Reihe
\[  \sum_{k=1}^\infty \frac{(-1)^{k+1}}{k}=1-\frac{1}{2}+\frac{1}{3}-\frac{1}{4}+\ldots \]
Nach dem Leibniz-Kriterium ist diese Reihe  konvergent. Die Reihe der Absolutbeträge ist die harmonische Reihe $\sum_{k=1}^\infty \frac{1}{k} $, welche nicht konvergiert.\\
Die alternierende harmonische Reihe ist also konvergent, aber nicht absolut konvergent.}
\lang{en}{
The alternating harmonic series is the series
\[  \sum_{k=1}^\infty \frac{(-1)^{k+1}}{k}=1-\frac{1}{2}+\frac{1}{3}-\frac{1}{4}+\ldots \]
The alternating series test implies that this series converges. The series of absolute values is the
harmonic series $\sum_{k=1}^\infty \frac{1}{k} $, which does not converge.
So the alternating harmonic series is convergent but not absolutely convergent.
}

\tab{\lang{de}{Majorantenkriterium} \lang{en}{Comparison test II}}
\begin{incremental}[\initialsteps{1}]
\step
\lang{de}{Wir zeigen mit dem Majorantenkriterium, dass die Reihe 
$\sum_{k=1}^\infty \frac{1}{k^2}=\frac{1}{1^2}+ \frac{1}{2^2}+\ldots\quad$ absolut konvergiert.}
\lang{en}{We will use the comparison test to show that the series
$\sum_{k=1}^\infty \frac{1}{k^2}=\frac{1}{1^2}+ \frac{1}{2^2}+\ldots\quad$ converges absolutely.}

\\
\step
\lang{de}{Für alle $k\in \N$ ist $k^2 > k$, also ist auch
\[ 2k^2 > k^2 + k \quad \text{bzw.} \quad k^2 > \frac{1}{2} k(k+1). \]
Für die Kehrwerte gilt damit 
\[
\frac{1}{k^2} < \frac{2}{k(k+1)}
\]
für alle $k \in \N$. Wir haben in Beispiel \ref{ex:erste-reihen} gesehen, dass die Reihe $\sum_{k=1}^\infty \frac{1}{k(k+1)}$
konvergiert. Nach den Grenzwertregeln konvergiert auch die Reihe 
\[
\sum_{k=1}^\infty \frac{2}{k(k+1)} = 2 \sum_{k=1}^\infty \frac{1}{k(k+1)}.
\]
Nach dem Majorantenkriterium ist also die Reihe $\sum_{k=1}^\infty \frac{1}{k^2}$ absolut konvergent.
}
\lang{en}{
For every $k\in \N$, we have $k^2 \geq k$, so
\[ 2k^2 > k^2 + k \quad \text{bzw.} \quad k^2 > \frac{1}{2} k(k+1). \]
This implies that the reciprocals satisfy
\[
\frac{1}{k^2} < \frac{2}{k(k+1)}
\]
for all $k \in \N$. In Example \ref{ex:erste-reihen} we saw that the series
$\sum_{k=1}^\infty \frac{1}{k(k+1)}$ converges.
By the limit rules, the series
\[
\sum_{k=1}^\infty \frac{2}{k(k+1)} = 2 \sum_{k=1}^\infty \frac{1}{k(k+1)}
\]
also converges. The comparison test shows that $\sum_{k=1}^\infty \frac{1}{k^2}$ converges absolutely.
}
\end{incremental}
\end{tabs*}
\end{example}

\begin{example}[\lang{de}{Exponentialreihe} \lang{en}{Exponential series}]
\lang{de}{Für jede beliebige reelle (oder sogar komplexe) Zahl $z$ ist die Reihe
\[ \exp(z)=\sum_{n=0}^\infty \frac{z^n}{n!} \]
absolut konvergent. Mit $n!$ bezeichnen wir hier, wie im letzten Kapitel eingeführt, das Produkt $1 \cdot 2 \cdots n$ der Zahlen 
von $1$ bis $n$.

Für $z=0$ ist die Reihe absolut konvergent, weil die Reihe $1+ 0 +0 + \ldots$
ist und nur einen Summanden ungleich null besitzt. (Der erste Summand lautet zwar  $0^0$, wird aber 
als $1$ gewertet.)

Für $z\neq 0$ benutzen wir das Quotientenkriterium und 
betrachten den Quotienten
\begin{eqnarray*}
 |\frac{a_{k+1}}{a_k}| &=& |\frac{\frac{z^{k+1}}{(k+1)!}}{\frac{z^{k}}{k!}}|=
\frac{|z|^{k+1}\cdot k!}{(k+1)!\cdot |z|^k}  \\
&=& \frac{|z|\cdot k!}{(k+1)\cdot k!\cdot 1}= \frac{|z|}{k+1}.
\end{eqnarray*}
Der Grenzwert hiervon ist 
\[  \lim_{k \to \infty} \ |\frac{a_{k+1}}{a_k}| =\lim_{k \to \infty} \ \frac{|z|}{k+1} = 0, \]
da $|z|$ eine konstante Zahl ist. Nach dem Quotientenkriterium ist die Reihe also absolut konvergent.

In der Tat kann man zeigen, dass der Grenzwert der Exponentialreihe dem Wert der $e$-Funktion $\exp(z)=e^z$ entspricht. }
\lang{en}{For any real (or even complex) number $z$, the series
\[ \exp(z)=\sum_{n=0}^\infty \frac{z^n}{n!} \]
converges absolutely. As in the last chapter, $n!$ stands for the product $1 \cdot 2 \cdots n$ of the numbers from $1$ to $n$.

For $z=0$, the series converges absolutely, because the series
is $1+ 0 +0 + \ldots$ with only one nonzero summand.
(The first summand is $0^0$, but it is defined to be $1$.)

For $z\neq 0$, we use the ratio test and consider the quotients
\begin{eqnarray*}
 |\frac{a_{k+1}}{a_k}| &=& |\frac{\frac{z^{k+1}}{(k+1)!}}{\frac{z^{k}}{k!}}|=
\frac{|z|^{k+1}\cdot k!}{(k+1)!\cdot |z|^k}  \\
&=& \frac{|z|\cdot k!}{(k+1)\cdot k!\cdot 1}= \frac{|z|}{k+1}.
\end{eqnarray*}
This tends to the limit
\[  \lim_{k \to \infty} \ |\frac{a_{k+1}}{a_k}| =\lim_{k \to \infty} \ \frac{|z|}{k+1} = 0, \]
as $|z|$ is a constant number. By the ratio test, the series converges absolutely.

One can show that the limit of the exponential series is exactly the value of the $\exp$-function $\exp(z)=e^z$.
}
\end{example}


\begin{quickcheck}
  \type{input.number}
  \field{rational}
 
  \begin{variables}
   \randint{n}{2}{6}
   \randint{m}{7}{11}
   \function[calculate]{s}{n/(n-1)}
   \function[calculate]{t}{m/(m+1)}
   \function[calculate]{r}{s-t}
  \end{variables}

 \lang{de}{
  \text{
    Bestimmen Sie den Grenzwert der Reihe
    $\sum_{k=0}^\infty ((\frac{1}{\var{n}})^k-(-\frac{1}{\var{m}})^k)$.
   
   Antwort: \ansref
    
   }}
  \lang{en}{
  \text{
    Find the limit of the series
    $\sum_{k=0}^\infty ((\frac{1}{\var{n}})^k-(-\frac{1}{\var{m}})^k)$.
   
   Answer: \ansref
    
   }}
 
  \begin{answer}
  \solution{r}
  \end{answer} 
  
\end{quickcheck}




\end{content}