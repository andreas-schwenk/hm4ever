
%$Id:  $
\documentclass{mumie.article}
%$Id$
\begin{metainfo}
  \name{
    \lang{de}{Funktionen}
%
    \lang{en}{Functions}
  }
  \begin{description} 
 This work is licensed under the Creative Commons License Attribution 4.0 International (CC-BY 4.0)   
 https://creativecommons.org/licenses/by/4.0/legalcode 

    \lang{de}{Beschreibung}
    \lang{en}{Description}
  \end{description}
  \begin{components}
    \component{generic_image}{content/rwth/HM1/images/g_tkz_T601_Signum.meta.xml}{T601_Signum}
    \component{generic_image}{content/rwth/HM1/images/g_tkz_T601_Graph_B.meta.xml}{T601_Graph_B}
    \component{generic_image}{content/rwth/HM1/images/g_tkz_T601_Graph_A.meta.xml}{T601_Graph_A}
    \component{generic_image}{content/rwth/HM1/images/g_tkz_T601_FunctionGraph.meta.xml}{T601_FunctionGraph}
    \component{generic_image}{content/rwth/HM1/images/g_tkz_T601_Map_F.meta.xml}{T601_Map_F}
    \component{generic_image}{content/rwth/HM1/images/g_tkz_T601_Map_E.meta.xml}{T601_Map_E}
    \component{generic_image}{content/rwth/HM1/images/g_tkz_T601_Map_D.meta.xml}{T601_Map_D}
    \component{generic_image}{content/rwth/HM1/images/g_tkz_T601_Map_C.meta.xml}{T601_Map_C}
    \component{generic_image}{content/rwth/HM1/images/g_tkz_T601_Map_B.meta.xml}{T601_Map_B}
    \component{generic_image}{content/rwth/HM1/images/g_tkz_T601_Map_A.meta.xml}{T601_Map_A}
    \component{generic_image}{content/rwth/HM1/images/g_img_00_Videobutton_schwarz.meta.xml}{00_Videobutton_schwarz}
    \component{generic_image}{content/rwth/HM1/images/g_img_00_Videobutton_blau.meta.xml}{00_Videobutton_blau}
    \component{js_lib}{system/media/mathlets/GWTGenericVisualization.meta.xml}{mathlet1}
  \end{components}
  \begin{links}
    \link{generic_article}{content/rwth/HM1/T101neu_Elementare_Rechengrundlagen/g_art_content_05_loesen_gleichungen_und_lgs.meta.xml}{content_05_loesen_gleichungen_und_lgs}
    \link{generic_article}{content/rwth/HM1/T102neu_Einfache_Reelle_Funktionen/g_art_content_06_funktionsbegriff_und_lineare_funktionen.meta.xml}{content_06_funktionsbegriff_und_lineare_funktionen}
    \link{generic_article}{content/rwth/HM1/T102neu_Einfache_Reelle_Funktionen/g_art_content_07_geradenformen.meta.xml}{content_07_geradenformen}
    \link{generic_article}{content/rwth/HM1/T102neu_Einfache_Reelle_Funktionen/g_art_content_08_quadratische_funktionen.meta.xml}{content_08_quadratische_funktionen}
    \link{generic_article}{content/rwth/HM1/T103_Polynomfunktionen/g_art_content_09_polynome.meta.xml}{content_09_polynome}
    \link{generic_article}{content/rwth/HM1/T403a_Vektorraum/g_art_content_10b_lineare_abb.meta.xml}{content_10b_lineare_abb}
    \link{generic_article}{content/rwth/HM1/T204_Abbildungen_und_Funktionen/g_art_content_10_abbildungen_verkettung.meta.xml}{content_10_abbildungen_verkettung}
\end{links}
  \creategeneric
\end{metainfo}

\begin{content}


\begin{block}[annotation]
	Im Ticket-System: \href{https://team.mumie.net/issues/22668}{Ticket 22668}
\end{block}

\usepackage{mumie.ombplus}
\ombchapter{1}
\ombarticle{4}

\usepackage{mumie.genericvisualization}

\begin{visualizationwrapper}

\title{
\lang{de}{Der Funktionsbegriff}
\lang{en}{Functions}
}


\begin{block}[info-box]
\tableofcontents
\end{block}


\section{\lang{de}{Funktionen} \lang{en}{Functions}}\label{sec:funktion} 

\lang{de}{Bevor wir uns mit den wichtigsten Funktionen und ihren Eigenschaften befassen, definieren wir zunächst, 
was wir generell unter einer Funktion verstehen und welche Begriffe wir im Zusammenhang
mit Funktionen verwenden werden.}
\lang{en}{
Before we begin to study the most important functions and their properties,
we will clarify what is meant by a function in general, as well as some
terminology that is used when working with functions.
}

\lang{de}{Eine Funktion ist dabei ein wichtiges Werkzeug, um verschiedene Größen (oder allgemeiner
Mengen) in Beziehung zueinander zu setzen. }
\lang{en}{
Functions are an important tool for describing relationships between
different quantities (or, more generally, sets).
}


  \begin{definition}[\lang{de}{Funktion} \lang{en}{Function}] \label{def:funktion}
    \lang{de}{Es seien $D$ und $Z$ zwei Mengen.}
    \lang{en}{Let $D$ and $Z$ be two sets.}
    \begin{itemize}
    
    \item \lang{de}{Eine \notion{Funktion} $f: D \rightarrow Z$ ist eine Vorschrift, die jedem 
    Element $x$ aus $D$ genau ein Element $y$ der Menge $Z$ zuordnet. }
      \lang{en}{ A \notion{function} $f: D \rightarrow Z$ is a rule 
    that assigns an element $y$ of the set $Z$ to every element $x$ of $D$.
}
    \item 
    \lang{de}{Die Menge $D$ heißt in diesem Fall \notion{Definitionsbereich} von $f$. 
    }
    \lang{en}{The set $D$ is called the \notion{domain} of $f$.}
    \item 
    \lang{de}{Die Menge $Z$ heißt \notion{Zielbereich} von $f$.}
    \lang{en}{The set $Z$ is called the \notion{codomain} of $f$.}
    
    \item 
    \lang{de}{Für den zugeordneten Funktionswert $y$ schreiben wir auch \notion{$f(x)$}. Statt 
    $y = f(x)$ schreiben wir manchmal auch $x \mapsto f(x)$. Die Variable $x$ nennen wir auch \notion{Argument}.
    }
    \lang{en}{
    The assigned value $y$ is written \notion{$f(x)$}. Instead of
    $y = f(x)$, we sometimes write $x \mapsto f(x)$.
    The variable $x$ is also called the \notion{argument}.
    }
    
    \item 
    \lang{de}{Die Menge aller Funktionswerte von $f$ 
    \[ 
      W_f := \{f(x) \mid x\in D\} \subseteq Z 
	\]
    ist eine Teilmenge des Zielbereichs von $f$ und wird als \notion{{Bildmenge}}
    oder \notion{{Wertemenge}} von $f$ bezeichnet. Sind die Funktionswerte reelle Zahlen, so
    nennen wir die Funktion $f$ auch \notion{reellwertig}.}
    \lang{en}{The set of values of $f$,
    \[ 
      W_f := \{f(x) \mid x\in D\} \subseteq Z 
	\]
    is a subset of the codomain $f$ and is called the \notion{{image}}
    or \notion{{range}} of $f$. If the values of $f$ are real, then
    we call $f$ \notion{real-valued}.}
    
    \end{itemize}
	 \end{definition}

    \lang{de}{
     Eine Funktion besteht also aus folgenden Angaben: Was in die Funktion eingesetzt werden darf (Definitionsbereich), was
     herauskommen kann (Zielmenge) und wie die Zuordnung geschieht (Funktionsgleichung). 
     }
    \lang{en}{
      Altogether, a function consists of the following data:
      what can be plugged into the function (its domain), what can come out
      of it (its codomain), and how the function acts (function rule).
    }

     \lang{de}{
     Häufig geben wir der Einfachheit halber den Definitionsbereich und die Zielmenge nicht an und schreiben einfach die 
     Funktionsgleichung $f(x) = \ldots$, um eine Funktion anzugeben. Es ist jedoch wichtig, einen Definitionsbereich anzugeben, 
     wenn nicht jede Zahl in die Funktion eingesetzt werden darf/soll. 
    }
    \lang{en}{
    For the sake of simplicity, we will often omit the domain and codomain
    and simply write an equation $f(x)=...$ when defining a function.
    However, it is important to specify a domain if there are numbers that
    cannot or should not be plugged into the function.
    }

   

\begin{example}\label{example:bereiche}
\begin{enumerate}

\item \lang{de}{Für $M=\{0;1;3\}$ und $N=\{1;2\}$ und die Funktion $f$ mit
\[ f(0) = 1, \ f(1) = 1, \ f(3) = 2\]
können wir die Situation folgendermaßen veranschaulichen:}
\lang{en}{
Let $M=\{0,1,3\}$ and $N=\{1,2\}$ and consider the function $f$
with \[ f(0) = 1, \ f(1) = 1, \ f(3) = 2.\]
This situation can be visualized as follows:
}

\begin{center}
\image{T601_Map_A}
\end{center}

\lang{de}{Bei Funktionen muss von jedem Element des Definitionsbereichs genau ein Pfeil starten.}
\lang{en}{
Functions have the property that every element in the domain
is the beginning of exactly one arrow.
}

\item 
\lang{de}{Das folgende Beispiel ist \notion{keine Funktion}: $f$ sei definiert durch $f(0) = 1, \ f(0) =2, \ f(1)=2$.}
\lang{en}{
The following example is \notion{not a function}: let $f$ be defined
by $f(0) = 1, \ f(0) =2, \ f(1)=2$.
}

\begin{center}
\image{T601_Map_B}
\end{center}
\lang{de}{Für die Zahl $3$, 
die im Definitionsbereich liegt, existiert kein Funktionswert. Außerdem ist der Funktionswert $f(0)$ nicht 
eindeutig - es darf immer nur genau einen Wert geben! Beides ist bei Funktionen nicht zulässig.}
\lang{en}{
The number $3$, which lies in the domain, is not mapped to any value.
In addition, the value of $f(0)$ is not unique - there can only be
exactly one value! Neither of these is allowed for a function.
}

\item 
\lang{de}{Die Funktion $f:\R\to \R,\, f(x)= x^2$ hat als Definitionsbereich die Menge $\,D_f=\R$, da jede reelle 
Zahl eingesetzt werden kann. Die Wertemenge besteht jedoch nur aus den nicht-negativen reellen Zahlen, 
denn für jede relle Zahl $\, x\,$ ist $\, x^2 \geq 0$. Deshalb gilt für die Wertemenge $W_f=[0,\infty)$.}
\lang{en}{
The function $f:\R\to \R,\, f(x)=x^2$ has the set $\,D_f=\R$ as its domain, as
any real number can be plugged into it. The range, however, consists only of the nonnegative realnumbers,
because for every real number $\, x\,$ we have $\, x^2 \geq 0$. The range is therefore $W_f=[0,\infty)$.
}

\item 
\lang{de}{Die Funktion mit der Funktionsgleichung $g(x) = \frac{1}{1-x}$ hat als Definitionsbereich die
Menge $\,D_g=\R\setminus \{1\}$ (\glqq $\R$ ohne $1$\grqq), da die Division durch $0$ nicht erlaubt ist. Die Wertemenge 
enthält in diesem Beispiel alle reellen Zahlen außer $\,0$, denn \glqq $1$ geteilt durch eine Zahl\grqq\ kann kann nicht
$0$ sein. Die Wertemenge ist folglich $\, W_g=\R\setminus \{0\}$.}
\lang{en}{
The function defined by the equation $g(x) = \frac{1}{1-x}$ has the set
$\, D_g = \R \setminus \{1\}$ ("$\R$ without $1$") as its domain, because
dividing by $0$ is not allowed. In this case, the range consists of all
real numbers other than $\, 0$, because "1 over a number" can never be $0$.
So the range is $W_g = \R\setminus \{0\}$.}

\end{enumerate}
\end{example}


\begin{quickcheck}
		\field{rational}
		\type{input.number}
		\begin{variables}
			\randint[Z]{d}{-5}{5}
			\randint{c}{-4}{4}
			\randint[Z]{b}{1}{4}
			\function[calculate]{a}{b*sign(d)*sign(c)}
			\function[calculate]{y}{d}
		    \function[normalize]{f}{a*x+d}
		    \function{g}{c-x}		    
		\end{variables}

      \lang{de}{
			\text{Welche reelle Zahl liegt nicht im maximalen Definitionsbereich der Funktion
			$f(x)=\frac{\var{f}}{\var{g}}$?\\ Die Zahl \ansref.}
       }
     \lang{en}{
			\text{Which real number does not lie in the maximal domain of
			$f(x)=\frac{\var{f}}{\var{g}}$?\\ The number \ansref.}
       }
		
		\begin{answer}
			\solution{c}
		\end{answer}
    \lang{de}{
		\explanation{Die einzige Zahl, die in die Vorschrift nicht eingesetzt werden darf, ist die, für die
		der Nenner $\var{g}$ gleich $0$ wird, hier also $\,x=\var{c}$.}}
    \lang{en}{
    \explanation{The only number that cannot be plugged into this expression
    is the number that makes the denominator $\var{g}$ equal $0$, i.e. $\,x=\var{c}$.    
    }}
	\end{quickcheck}

    \lang{de}{
    Zusammenhänge zwischen verschiedenen Größen werden oft in einer \emph{(Werte-) Tabelle}
    oder \emph{graphisch} in einem zweidimensionalen Koordinatensystem dargestellt.  
    Eine \emph{Wertetabelle} ist dabei eine Tabelle mit zwei Zeilen (oder Spalten), in die einzelne
    Werte des Definitionsbereichs und die dazugehörigen Funktionswerte eingetragen werden. 
    }
    \lang{en}{
    Relationships between different quantities can often be represented by
    a \emph{table of values}, or \emph{graphically} in a two-dimensional
    coordinate system. By a \emph{table of values}, we mean a table
    with two rows (or columns), in which the values in the domain
    and the associated function values are to be entered.
    }
    \begin{table}[\align{c} \cellaligns{ccccccc}]
        $\;\mathbf{x}\;$ && $\,x_1\,$ & $\,x_2\,$ & $\,x_3\,$ & $\,x_4\,$ & ...\\
        $\;\mathbf{f(x)}\;$ && $\,y_1\,$ & $\,y_2\,$ & $\,y_3\,$ & $\,y_4\,$ & ...\\
     \end{table}

   \lang{de}{Sie dient in der Regel zur Erstellung der \emph{graphischen} Darstellung der Funktion im
   zweidimensionalen Koordinatensystem.}
   \lang{en}{This is mostly used when creating the \emph{graphical}
    representation of a function in the two-dimensional
    coordinate system.
   }

    \begin{center}
        \image{T601_FunctionGraph}
    \end{center}

 \lang{de}{Die waagerechte Achse des Koordinatensystems heißt \emph{x-Achse} oder auch \notion{\emph{Abszisse}} 
 und die senkrechte Achse ist die \emph{y-Achse} oder auch die \notion{\emph{Ordinate}}. 
}
\lang{en}{The horizontal axis of the coordinate system is called
the \emph{x-axis} or the \notion{\emph{abscissa}} and the vertical
axis is called the \emph{y-axis} or the \notion{\emph{ordinate}}.
}

\begin{definition} \label{def:graph}
  	\lang{de}{
    Als \notion{\emph{Funktionsgraph}} oder kurz als \notion{\emph{Graph}} einer Funktion 
    $f:D\to \R$ bezeichnet man die Menge aller geordneten Paare $\,(x;f(x))$, also 
    \[ \text{Graph}(f):=\{ (x;y) \mid x\in D, y=f(x) \}=\{ (x;f(x)) \mid x\in D \} \subseteq \R^2.\] 
    }
    \lang{en}{
    The \notion{\emph{graph}} of a function $f:D\to \R$
    is the set of all ordered pairs $\,(x;f(x))$, i.e. 
    \[ \text{Graph}(f):=\{ (x,y) \mid x\in D, y=f(x) \}=\{ (x,f(x)) \mid x\in D \} \subseteq \R^2.\] 
    }
\end{definition}   

\lang{de}{
Übertragen wir die Punkte der Menge \notion{Graph}$(f)\,$ in ein Koordinatensystem, mit
dem Argument $\,x\,$ als Wert auf der \emph{x-Achse} und dem zugehörigen Funktionswert $\,f(x)\,$
als Wert auf der \emph{y-Achse}, so erhalten wir die graphische Darstellung der Funktion $f$.
}
\lang{en}{
By transferring the points of the set \notion{Graph}$(f)\,$ to a
coordinate system with the argument $x$ as the value on the \emph{x-axis}
and the associated function value $f(x)$ as the value on the \emph{y-axis},
we obtain a graphical representation of the function $f$.
}

\begin{example}
  \lang{de}{
  Wir schauen uns die graphische Darstellung für unsere Funktionen $f$ und $g$ 
  aus Beispiel \ref{example:bereiche} an:}
  \lang{en}{
  Let us look at the graphical representations of the functions $f$ and $g$
  that we considered in Example \ref{example:bereiche}:
  }
  
  \begin{tabs*}[\initialtab{1}\class{example}]
   \tab{$f(x)=x^2$}  
          \lang{de}{Der Funktionsgraph der Funktion $f:\R\to \R,\, x\mapsto x^2$ wird beschrieben durch die Menge 
           \[ \text{Graph}(f) =\{ (x;x^2) \mid x\in \R \}\]
           
          oder beispielhaft durch die Wertetabelle
          \begin{table}[\align{c} \cellaligns{ccrrrrr}]
              $\;\mathbf{x}\;$ && $-2$ & $\textcolor{#0066CC}{-1}$ & $\,\textcolor{#0066CC}{0}\,$ & $\,1\,$ & $\,\textcolor{#0066CC}{2}\,$ \\
              $\;\mathbf{f(x)}\;$ && $\,4\,$ & $\,\textcolor{#0066CC}{1}\,$ & $\,\textcolor{#0066CC}{0}\,$ & $\,1\,$ & $\,\textcolor{#0066CC}{4}\,$ \\
          \end{table} 
          und hat die folgende Darstellung im zweidimensionalen Koordinatensystem:
            \begin{center}
               \image{T601_Graph_A}
            \end{center}
          }
          \lang{en}{The graph of the function $f:\R\to \R,\, x\mapsto x^2$ can be described by the set 
           \[ \text{Graph}(f) =\{ (x,x^2) \mid x\in \R \}\]
          or, for example, by the table of values
          \begin{table}[\align{c} \cellaligns{ccrrrrr}]
              $\;\mathbf{x}\;$ && $-2$ & $\textcolor{#0066CC}{-1}$ & $\,\textcolor{#0066CC}{0}\,$ & $\,1\,$ & $\,\textcolor{#0066CC}{2}\,$ \\
              $\;\mathbf{f(x)}\;$ && $\,4\,$ & $\,\textcolor{#0066CC}{1}\,$ & $\,\textcolor{#0066CC}{0}\,$ & $\,1\,$ & $\,\textcolor{#0066CC}{4}\,$ \\
          \end{table} 
          and it has the following representation in the two-dimensional
          coordinate system:
            \begin{center}
               \image{T601_Graph_A}
            \end{center}
          }


   \tab{$g(x)=\frac{1}{1-x}$}
        \lang{de}{Der Funktionsgraph von $\,g:\R\setminus \{1\}\to \R\setminus \{0\},\, x\mapsto \frac{1}{1-x}\,$ wird beschrieben durch die Menge 
        \[ \text{Graph}(g) =\{\displaystyle \Big(x;\frac{1}{1-x}\Big) \mid x\in \R \setminus \{1\} \} \]
           
         und hat die folgende Darstellung im zweidimensionalen Koordinatensystem:

            \begin{center}
              \image{T601_Graph_B}
            \end{center}
        }
        \lang{en}{
          The graph of $\,g:\R\setminus \{1\}\to \R\setminus \{0\},\, x\mapsto \frac{1}{1-x}\,$
          can be described by the set
          \[ \text{Graph}(g) =\{\displaystyle \Big(x;\frac{1}{1-x}\Big) \mid x\in \R \setminus \{1\} \} \]
        and it has the following representation in the two-dimensional
        coordinate system:
          \begin{center}
            \image{T601_Graph_B}
          \end{center}
        }
  \end{tabs*}
\end{example}



\section{\lang{de}{Rechenoperationen mit Funktionen}
\lang{en}{Arithmetic with functions}}

\lang{de}{Wenn verschiedene Funktionen mit gleichem Definitionsbereich gegeben sind, können wir auch die gewöhnlichen
Rechenoperationen auf diese anwenden.}
\lang{en}{
If different functions have the same domain then we can apply the usual
operations of arithmetic to them.
}

\begin{definition}
\lang{de}{Es seien $f$ und $g$ zwei reellwertige Funktionen mit gleichem Definitionsbereich.  Dann 
können wir die Funktionen addieren, subtrahieren, multiplizieren oder auch dividieren. \\
Es gilt:
}
\lang{en}{
Let $f$ and $g$ be two real-valued functions with the same domain.
Then we can define the operations of addition, subtraction, multiplication and division by
}
\begin{itemize}
\item $\left( f + g \right) (x) = f(x) + g(x)$,

\item $\left( f - g \right) (x) = f(x) - g(x)$,

\item $\left( f \cdot g \right) (x) = f(x) \cdot g(x)$,

\item 
\lang{de}{$\left( \frac{f}{g} \right) (x) = \frac{f(x)}{g(x)}$, wobei hier $g(x) \neq 0$ gelten muss.}
\lang{en}{$\left( \frac{f}{g} \right) (x) = \frac{f(x)}{g(x)}$, if $g(x) \neq 0$.}

\end{itemize}
\lang{de}{Man sagt auch, dass die Rechenoperationen \emph{punktweise} definiert sind.}
\lang{en}{The arithmetic operations are said to be defined \emph{pointwise}.}

\end{definition}

\begin{example}
\lang{de}{Bei der Produktion eines Gutes entstehen Kosten in Abhängigkeit der produzierten Menge $x$ gemäß der Kostenfunktion $K(x) = 3x^2 + 5x$. 
Die Preis-Absatz-Funktion lautet $p(x) = x^2 + 18x - 1$. \\

Wir bestimmen die Erlösfunktion $E(x)$ als Produkt von Preis-Absatz-Funktion mit der Menge $x$:\\
}
\lang{en}{
The production of a good entails costs that depend on the produced quantity $x$
according to the cost function $K(x) = 3x^2 + 5x$.
The price-demand function is $p(x) = x^2 + 18x - 1$. \\

We will calculate the revenue function $E(x)$ as the product of the price-demand function and the quantity $x$:\\ }
\begin{align*}
E(x) &\ =\ & p(x) \cdot x \\
 &\ =\ & (x^2 + 18x - 1)\cdot x \\
 &\ =\ & x^3 +18x^2 - x.
\end{align*}

\lang{de}{Mit Hilfe der Erlösfunktion lässt sich nun die Gewinnfunktion $G(x)$ aus der Subtraktion von Erlös und Kosten bestimmen:
}
\lang{en}{
The profit function can now be determined as the difference between
the revenue and cost functions: \\
}
\begin{align*}
G(x) &\ =\ & E(x) - K(x)\\
 &\ =\ & (x^3 +18x^2 - x) - (3x^2 + 5x) \\
 &\ =\ & x^3 + 15x^2 - 6x.
\end{align*}

\lang{de}{Zur Beurteilung der Wirtschaftlichkeit ist es üblich, die Produktivitätsfunktion $P(x)$ als Quotient aus Erlös und Kosten zu betrachten.
}
\lang{en}{
When determining profitability, one usually considers the productivity function $P(x)$, which is 
defined as the ratio of revenue to cost:
}
\begin{align*}
P(x) &\ =\ & \frac{E(x)}{K(x)}\\
 &\ =\ & \frac{x^3 +18x^2 - x}{3x^2 + 5x}.
\end{align*}

\lang{de}{Zuletzt bestimmen wir den mathematischen und ökonomisch sinnvollen Definitionsbereich der Produktivitätsfunktion.\\
Hierzu muss der Nenner der Funktion ungleich Null sein.\\
Wir setzen den Nenner gleich Null, um die nicht definierten Werte zu bestimmen:}
\lang{en}{
Finally, we will determine the mathematically and
economically meaningful domains of the productivity function.
First of all, the denominator of the function must be nonzero. \\
We set the denominator equal to zero to find the values for which the function is undefined:
}

\lang{de}{\begin{align*}
0 &\ =\ & 3x^2 + 5x \\
\Leftrightarrow 0 &\ =\ & x\cdot(3x+5)\\
\Leftrightarrow 0 &\ =\ & x_1 \text{ oder } 0 \ =\  3x_2 +5 \\
\Leftrightarrow 0 &\ =\ & x_1 \text{ oder } -\frac{5}{3} \ =\ x_2
\end{align*}}
\lang{en}{\begin{align*}
0 &\ =\ & 3x^2 + 5x \\
\Leftrightarrow 0 &\ =\ & x\cdot(3x+5)\\
\Leftrightarrow 0 &\ =\ & x_1 \text{ or } 0 \ =\  3x_2 +5 \\
\Leftrightarrow 0 &\ =\ & x_1 \text{ or } -\frac{5}{3} \ =\ x_2
\end{align*}}
\lang{de}{Der mathematische Definitionsbereich von $P(x)$ ist $\,D_{P_m}=\R\setminus \{-\frac{5}{3};0\}$.\\
Ökonomisch gesehen muss die Menge $x$ jedoch stets größer oder gleich $0$ sein, wobei hier $x=0$ mathematisch nicht definiert ist. Also
$\,D_{P_o}=\R^+$.}

\lang{en}{
The mathematical domain of $P(x)$ is $\,D_{P_m}=\R\setminus \{-\frac{5}{3},0\}$.\\
From the economist's point of view, however, the quantity $x$ must always be greater than or equal to $0$.
Here, the function is mathematically undefined at $x=0$. So
$\,D_{P_o}=\R^+$.}



\end{example}

\lang{de}{Eine weitere wichtige Operation ist die \emph{Komposition} von Funktionen, mit der man eine ganze Funktion 
sinngemäß in eine andere Funktion einsetzt. }
\lang{en}{
Another important operation is composition of functions, in which
an entire function is essentially plugged into another function.
}

\begin{definition}[\lang{de}{Komposition} \lang{en}{Composition}]\label{kompositionv}
\lang{de}{Sind $T:L\to M$ und $S:M\to N$ Funktionen, dann ist die \emph{\notion{Komposition} $V= S\circ T$}
eine Funktion mit Definitionsbereich $L$ und Zielmenge $N$ und
\[   V(x) =  \left(S\circ T \right)(x) = S(T(x)). \]}
\lang{en}{
Suppose $T:L\to M$ and $S:M\to N$ are functions. The \emph{composition} $V= S\circ T$
is a function with domain $L$ and codomain $N$ defined by
\[   V(x) =  \left(S\circ T \right)(x) = S(T(x)). \]
}
\end{definition}
\lang{de}{Man berechnet also erst den Funktionswert unter der Funktion $T$ und setzt diesen dann in die Funktion $S$ ein.
Daher ist es wichtig, dass die Funktionswerte von $T$ im Definitionsbereich von $S$ liegen.}
\lang{en}{
That is, the value of the function $T$ is first calculated, then
plugged into the function $S$. It is therefore important that the
values of $T$ lie in the domain of $S$.
}

\lang{de}{Statt Komposition spricht man auch häufig von \emph{Verkettung} oder \emph{Hintereinanderausführung} der Funktionen.}



\begin{example}\label{ex:kompositionen}
\begin{enumerate}
\item \lang{de}{Es seien $f:\R\to \R, x\mapsto x^2$ und $g:\R\to \R, x\mapsto x+1$, dann sind
\[   (g\circ f)(x)=g(f(x))=g(x^2)=x^2+1 \]
 und 
\[  (f\circ g)(x)=f(g(x))=f(x+1)=(x+1)^2. \]
für alle reellen Zahlen $x$.}
\lang{en}{
Let $f:\R\to \R, x\mapsto x^2$ and $g:\R\to \R, x\mapsto x+1$. Then
\[   (g\circ f)(x)=g(f(x))=g(x^2)=x^2+1 \]
 and 
\[  (f\circ g)(x)=f(g(x))=f(x+1)=(x+1)^2 \]
for all real numbers $x$.
}
\item \lang{de}{Für $L=\{0; 2\}$, $M=\{0;1;3\}$ und $N=\{1;2\}$ mit den Funktionen $T:L\to M$ gegeben durch
\[ T(0) = 3,\quad T(2) = 0 \]
und $S:M\to N$ gegeben durch
\[ S(0) = 1,\quad S(1) = 1,\quad S(3) = 2  \]
ist die verkettete Funktion gegeben durch
\[ (S \circ T)(0)=  S(T(0))=S(3)=2,\quad (S \circ T)(2) = S(T(2))=S(0)=1. \]
Grafisch sieht dies folgendermaßen aus:}
\lang{en}{
Let $L=\{0, 2\}$, $M=\{0,1,3\}$ and $N=\{1,2\}$ with the functions
$T : L\to M$ given by \[ T(0) = 3,\quad T(2) = 0 \]
and $S:M\to N$ given by \[ S(0) = 1,\quad S(1) = 1,\quad S(3) = 2.\]
The composite function is given by
\[ (S \circ T)(0)=  S(T(0))=S(3)=2,\quad (S \circ T)(2) = S(T(2))=S(0)=1. \]
This is represented by the following diagram:
}

\begin{center}
\image{T601_Map_C}
\end{center}

\end{enumerate}
\end{example}

\begin{quickcheck}
    \begin{variables}
      		\randint{a}{-3}{3}
      		\randint[Z]{b}{-2}{2}
      		\randint{c}{1}{4}
      		\randint[Z]{d}{-2}{2}
      		\function[normalize]{f}{x^2+a*x+b}
      		\function[normalize]{g}{c*x+d}
      		\function[normalize]{fg}{(c*x+d)^2+a*(c*x+d)+b}
      		\function[normalize]{gf}{c*(x^2+a*x+b)+d}
      \end{variables}

      \type{input.function}
      \field{real}
      \lang{de}{\text{Bestimmen Sie die Komposition $f\circ g$ der Abbildungen\\
      $f:\R\to \R, x\mapsto \var{f}$ und $g:\R\to \R, x\mapsto \var{g}$.\\ $(f\circ g)(x)=$\ansref}
       }
       \lang{en}{\text{Compute the composition $f\circ g$ of the maps \\
      $f:\R\to \R, x\mapsto \var{f}$ and $g:\R\to \R, x\mapsto \var{g}$.\\ $(f\circ g)(x)=$\ansref}
       }
      \begin{answer}
          \solution{fg}
          \checkAsFunction{x}{-10}{10}{10}
      \end{answer}
\end{quickcheck}



\section{\lang{de}{Eigenschaften von Funktionen}\lang{en}{Properties of functions}}\label{Nullstellen}

\lang{de}{Die folgenden Eigenschaften benutzen wir, um festzustellen, wann eine Umkehrfunktion existiert.
}
\lang{en}{
The following properties will be used to decide when an inverse function exists.
}
\begin{definition}\label{def:functionprops}
\begin{enumerate}
\item \lang{de}{Eine Funktion heißt \notion{injektiv}, wenn jedes Element der Zielmenge
\textbf{\textit{höchstens einmal} als Funktionswert angenommen wird}.
\begin{figure}
\image{T601_Map_D}
\caption{Injektivität: Bei jedem Element endet \textbf{höchstens ein} Pfeil.}
\end{figure}}
\lang{en}{A function is called \notion{injective} or \notion{one-to-one}
if every element of the codomain occurs \textbf{\textit{at most once}}
as a function value.
\begin{figure}
\image{T601_Map_D}
\caption{Injective: Every element is the end of \textbf{at most one} arrow.}
\end{figure}}

\item \lang{de}{Eine Funktion heißt \notion{surjektiv}, wenn jedes Element der Zielmenge
\textbf{\textit{mindestens einmal} als Funktionswert angenommen wird}.
\begin{figure}
\image{T601_Map_E}
\caption{Surjektivität: Bei jedem Element endet \textbf{mindestens ein} Pfeil.}
\end{figure}
}
\lang{en}{A function is called \notion{surjective} or \notion{onto}
if every element of the codomain occurs \textbf{\textit{at least once}
as a function value}.
\begin{figure}
\image{T601_Map_E}
\caption{Surjective: Every element is the end of \textbf{at least one} arrow.}
\end{figure}
}
\item \lang{de}{Eine Funktion heißt \notion{bijektiv} oder \notion{umkehrbar}, wenn jedes Element in der Zielmenge
\textbf{\textit{genau einmal} als Funktionswert angenommen wird}.
\begin{figure}
\image{T601_Map_F}
\caption{Bijektivität: Bei jedem Element endet \textbf{genau ein} Pfeil.}
\end{figure}
}
\lang{en}{A function is called \notion{bijective} or \notion{invertible}
if every element of the codomain occurs \textbf{\textit{exactly once}} as a function value.
\begin{figure}
\image{T601_Map_F}
\caption{Bijective: Every element is the end of \textbf{exactly one} arrow.}
\end{figure}
}
\end{enumerate}

\lang{de}{Ist eine Funktion $f$ bijektiv, dann können wir die Zuordnung der Funktion $f$ rückgängig machen. Die Funktion, die 
das erreicht, nennen wir \notion{Umkehrfunktion} und schreiben dafür $f^{-1}$. Per Definition gilt $f^{-1}(f(x)) = x$ und $f(f^{-1}(y)) = y$.}
\lang{en}{
If a function $f$ is bijective, then the mapping induced by $f$ can be reversed.
The function that does this is called the \notion{inverse function},
written $f^{-1}$. By definition, $f^{-1}(f(x)) = x$ and $f(f^{-1}(y)) = y$.
}
\end{definition}

\lang{de}{Es bleibt natürlich noch die Frage, wie man die Umkehrfunktion berechnen kann (wenn die Funktion bijektiv ist). 
Ist die Funktion wie oben durch Pfeile zwischen Elementen dargestellt, dann ist die Umkehrfunktion dadurch gegeben, dass man 
die Richtung der Pfeile umdreht. (Anfang wird zu Pfeilspitze und umgekehrt.) Rechnerisch schreiben wir am einfachsten 
$y = f(x)$ und lösen nach $x$ auf, wie im folgenden Beispiel.}

\lang{en}{
Of course, we are still left with the question of how to calculate the inverse function
(assuming the function is bijective).
If the function is represented by arrows between elements as above, then the inverse function
is given by reversing the direction of the arrows.
(The beginning becomes the end of the arrow and vice versa.)
Algebraically, it is simplest to write $y=f(x)$ and solve for $x$
as in the following example.
}

\begin{example}

 \begin{tabs*}[\initialtab{1}\class{example}]
   \tab{$f(x)=2+\frac{1}{x}$}  
\lang{de}{Wir betrachten die Funktion $f:\R^*\to \R, f(x) = 2+\frac{1}{x}$.\\

Wir schreiben $y =  2+\frac{1}{x}$ und versuchen, die Gleichung nach $x$ aufzulösen. Wenn dies mit 
Äquivalenzumformungen gelingt, dann haben wir rechnerisch die Umkehrfunktion ermittelt. Es gilt:
\begin{align*}
 & \quad \quad y &=  2+\frac{1}{x} \\
 &\Leftrightarrow \ y -2 &= \frac{1}{x} \\
 &\Leftrightarrow \quad x &= \frac{1}{y-2}
\end{align*}
Die letzte Umformung ist nur gültig, wenn $y \neq 2$ gilt, denn sonst würden wir durch $0$ teilen.
Da die Funktion $\frac{1}{x}$ aber den Wert $0$ nicht annimmt, nimmt die Funktion $f(x)$ den Wert $2$ 
nicht an. Der Fall $y =2$ kann daher gar nicht auftreten. 

Durch Umbenennung der Variablen ($x$ und $y$ vertauschen) erhalten wir $f^{-1}$, also
\[ f^{-1}(x)=\frac{1}{x-2}. \]
 }
 \lang{en}{
Consider the function $f:\R^*\to \R, f(x) = 2+\frac{1}{x}$.\\

We write $y =  2+\frac{1}{x}$ and attempt to solve the equation for $x$.
If the equation can be rearranged in this way then we will have found the inverse function algebraically.
We have:
\begin{align*}
 & \quad \quad y &=  2+\frac{1}{x} \\
 &\Leftrightarrow \ y -2 &= \frac{1}{x} \\
 &\Leftrightarrow \quad x &= \frac{1}{y-2}
\end{align*}
The final rearrangement is only valid if $y \neq 2$, as we would otherwise be dividing by $0$.
Since the function $\frac{1}{x}$ never takes the value $0$, $f(x)$ never takes the value $2$,
so the case that $y=2$ never arises.

After renaming the variables (swapping $x$ and $y$), we obtain $f^{-1}$, i.e.
\[ f^{-1}(x)=\frac{1}{x-2}. \]
 }
            
    \tab{\lang{de}{Preis-Absatz-Funktion} \lang{en}{Price-demand function}}
    \lang{de}{Die Preis-Absatz-Funktion $p(x) = 5 + \frac{8}{x-4}$ beschreibt den Preis $p$ in Abhängigkeit der Menge $x$. \\
    Wir wollen in diesem Beispiel jedoch gerne herausfinden, wie sich die Menge $x$ in Abhängigkeit des Preises $p$ verhält. 
    Dazu benöigten wir die Funktion $x(p)$. Zur Bestimmung der Funktion $x(p)$ benötigen wir die Umkehrfunktion von $p(x)$:\\
    
    \begin{align*}
    & \quad \quad p &=  5+\frac{8}{x-4} \\
    &\Leftrightarrow \ p - 5 &= \frac{8}{x-4} \\
    &\Leftrightarrow \ x-4 &= \frac{8}{p-5}\\
    &\Leftrightarrow \quad x &= \frac{8}{p-5}+4\\
    &\rightarrow \quad x(p) &= \frac{8}{p-5}+4
    \end{align*}
    
    Die Funktion $x(p)$ ist nur gültig, wenn $p \neq 5$ gilt, denn sonst würden wir durch $0$ teilen.\\
    Zudem sollten wir hier den Definitionsbereich nicht nur mathematisch, sondern auch ökonomisch betrachten:\\
    Ökonomisch gesehen muss sowohl der Preis als auch die Menge größer $0$ sein. \\
    
     \begin{table}[\align{c} \cellaligns{ccccccc}]
        $\;\mathbf{p}\;$ && $\,0\,$ & $\,1\,$ & $\,2\,$ & $\,3\,$ & $\,4\,$ &$\,5\,$ & $\,6\,$ & ... \\
        $\;\mathbf{x(p)}\;$ && $\,2,4\,$ & $\,2\,$ & $\,1,33\,$ & $\,0\,$ & $\,-4\,$ &$\,n.D.\,$ & $\,12\,$ &$\,>0\,$\\
     \end{table}
    Die Tabelle zeigt, dass aus ökonomischer Sicht neben $p \neq 5$ auch $p \neq 4$ gilt, denn wenn der Preis $p=4$ ist, ergibt die Funktion $x(p)$ eine negative Menge.
    Mathematisch gesehen liefern negative Mengen kein Problem, aus ökonomischer Sicht sind diese jedoch nicht möglich. \\
    }
    \lang{en}{The price-demand function $p(x) = 5 + \frac{8}{x-4}$ describes the price $p$ as a function of quantity ($x$). \\
    In this example, however, we want to find out how the quantity $x$ behaves as a function of price ($p$). 
    For this, we need the function $x(p)$. 
    To determine the function $x(p)$, we need the inverse function of $p(x)$:\\
    
    \begin{align*}
    & \quad \quad p &=  5+\frac{8}{x-4} \\
    &\Leftrightarrow \ p - 5 &= \frac{8}{x-4} \\
    &\Leftrightarrow \ x-4 &= \frac{8}{p-5}\\
    &\Leftrightarrow \quad x &= \frac{8}{p-5}+4\\
    &\rightarrow \quad x(p) &= \frac{8}{p-5}+4
    \end{align*}
    
    The function $x(p)$ is only defined if $p \neq 5$,
    as we would otherwise be dividing by $0$.\\
    In addition, the domain should not only be mathematically meaningful;
    we also need to consider its economic meaning.
    From the point of view of economics,
    both the price and the quantity must be greater than $0$. \\
    
     \begin{table}[\align{c} \cellaligns{ccccccc}]
        $\;\mathbf{p}\;$ && $\,0\,$ & $\,1\,$ & $\,2\,$ & $\,3\,$ & $\,4\,$ &$\,5\,$ & $\,6\,$ & ... \\
        $\;\mathbf{x(p)}\;$ && $\,2.4\,$ & $\,2\,$ & $\,1.33\,$ & $\,0\,$ & $\,-4\,$ &$\,N/A\,$ & $\,12\,$ &$\,>0\,$\\
     \end{table}
    The table shows that $p \neq 4$ must hold in addition to $p \neq 5$,
    because the price $p=4$ corresponds to a negative quantity $x(p)$.
    A negative quantity does not lead to any mathematical problems
    but it is not economically meaningful.
    }
  \end{tabs*}
\end{example}

\begin{quickcheck}
    \begin{variables}
      		\randint[Z]{a}{-10}{10}
      		\randint[Z]{b}{-5}{5}
      		\randint[Z]{c}{10}{20}
      		
      		\function[normalize]{px}{(b/(x+a))-c}
      		
      		\function[normalize]{xp}{b/(p+c)-a}
      		
      \end{variables}

      \type{input.function}
      \field{real}
      \lang{de}{
      \text{Bestimmen Sie die Funktion $x(p)$ bei gegebener Preis-Absatz-Funktion $p(x) = \var{px}$\\
      $x(p)=$\ansref .}}
      \lang{en}{
      \text{Determine the function $x(p)$, given the price-demand function $p(x) = \var{px}$.\\
      $x(p)=$\ansref .}}
   
      \begin{answer}
          \solution{xp}
          \checkAsFunction{x}{-10}{10}{10}
      \end{answer}
      \lang{de}{
      \explanation{Bilden Sie die Umkehrfunktion von $p(x)$.}}
      \lang{en}{
      \explanation{Find the inverse function of $p(x)$.}}
\end{quickcheck}





\lang{de}{Graphisch kann man die Umkehrfunktion auch folgendermaßen ermitteln:}
\lang{en}{The inverse function can be found graphically as follows:}
\begin{rule}
\lang{de}{Man erhält den Graphen von $f^{-1}$ aus dem Graphen von $f$ durch Spiegelung an der Diagonalen $y=x$.  }
\lang{en}{
The graph of $f^{-1}$ is obtained from the graph of $f$ by reflecting across the diagonal $y=x$.
}
\end{rule}

\lang{de}{Die nachstehende Visualisierung zeigt das oben berechnete Beispiel $f(x)=2+\frac{1}{x}$ graphisch interpretiert.\\ }
\lang{en}{
The visualization below shows the example $f(x)=2+\frac{1}{x}$ from above, interpreted graphically.
}
\begin{genericGWTVisualization}[550][800]{mathlet1}
		\begin{variables}
			\function{f}{rational}{2+1/x}
			\function{g}{rational}{1/(x-2)}
			\function{di}{rational}{x}

% 			\function{axe}{rational}{0}
% 			\pointOnCurve[1/10,6]{xa}{rational}{axe}{3/2}
%			\number{a}{rational}{var(xa)[x]}

			\number[editable]{a}{rational}{1}
			\number{a0}{rational}{var(a)}
			\number{fa}{rational}{2+1/var(a)}
			
			\point{Ax}{rational}{var(a0),0}
			\point{Ag}{rational}{var(a0),var(fa)}
			\point{Ay}{rational}{0,var(fa)}
			\segment{v1}{rational}{var(Ax),var(Ag)}
			\segment{h1}{rational}{var(Ag),var(Ay)}

			\point{By}{rational}{0,var(a0)}
			\point{Bg}{rational}{var(fa),var(a0)}
			\point{Bx}{rational}{var(fa),0}
			\segment{v2}{rational}{var(Bx),var(Bg)}
			\segment{h2}{rational}{var(Bg),var(By)}
			
			
			
		\end{variables}
		\color{f}{#0066CC}
		\color{g}{#CC6600}
		\color{di}{LIGHT_GRAY}
		\color{v1}{GRAY}
		\color{h1}{GRAY}
		\color{Ag}{GRAY}
		\label{Ag}{$A$}
		%\color{Ax}{GRAY}
		%\label{Ax}{@2d[$a$]}
		
% 		\color{xa}{GRAY}
% 		\label{xa}{@2d[$a$]}

		\color{v2}{GRAY}
		\color{h2}{GRAY}
		\color{Bg}{GRAY}
		\label{Bg}{$B$}
		
		\begin{canvas}
			\plotSize{450}
			\plotLeft{-4}
			\plotRight{6}
			\plot[coordinateSystem]{di,v1,h1,v2,h2,f,g, Ax, Ag, Bg }
		\end{canvas}
    \lang{de}{
		\text{Der Graph der Funktion $\textcolor{#0066CC}{f(x)=2+\frac{1}{x}}$
        und der Umkehrfunktion $\textcolor{#CC6600}{f^{-1}(x)=\frac{1}{x-2}}$.\\\\
		\IFELSE{var(a)=0}{Für $x=\var{a}$ ist die Funktion $f$ nicht definiert! Damit nimmt die Umkehrfunktion $f^{-1}$ diesen Wert nicht an,
        d.h. $\var{a0}$ liegt nicht in der Wertemenge von $f^{-1}$.}{Für $x=\var{a}$ ist $f(\var{a0})=\var{fa}$, also ist $A=(\var{a0}; \var{fa})$ auf dem Graphen von $f$.\\
		Nach Definition der Umkehrfunktion ist: $f^{-1}(\var{fa})=\var{a0}$, also $B=(\var{fa}; \var{a0})$ auf dem Graphen von $f^{-1}$.\\
		Man erhält den Graphen von $f^{-1}$ durch Spiegelung des Graphen von $f$ an der Diagonalen $y=x\;$ (hellgrau eingezeichnet).}		
		}}
  \lang{en}{
		\text{The graph of the function $\textcolor{#0066CC}{f(x)=2+\frac{1}{x}}$
        and its inverse function $\textcolor{#CC6600}{f^{-1}(x)=\frac{1}{x-2}}$.\\\\
		\IFELSE{var(a)=0}{At $x=\var{a}$, the function $f$ is not defined! Therefore, the inverse function $f^{-1}$
        never takes that value, i.e. $\var{a0}$ does not lie in the range of $f^{-1}$.}{At $x=\var{a}$, we have $f(\var{a0})=\var{fa}$, so $A=(\var{a0}; \var{fa})$ lies on the graph of $f$.\\
		The definition of the inverse function implies $f^{-1}(\var{fa})=\var{a0}$, so $B=(\var{fa}; \var{a0})$ lies on the graph of $f^{-1}$.\\
		The graph of $f^{-1}$ is obtained by reflecting $f$ across the diagonal $y=x\;$ (colored gray).}
		}}
	   \end{genericGWTVisualization}
\\ \\
\lang{de}{Weitere Eigenschaften von Funktionen sind ähnlich wie die gleichlautenden Eigenschaften für Folgen definiert.
}
\lang{en}{
Several further properties of functions are defined similarly to
the properties of sequences of the same name.
}

\begin{definition}
\lang{de}{Eine reellwertige Funktion $f$ heißt ...

\begin{itemize}
\item ... \notion{nach oben beschränkt}, wenn es eine reelle Zahl $C$ gibt, die größer als alle Funktionswerte ist.

\item ... \notion{nach unten beschränkt}, wenn es eine reelle Zahl $c$ gibt, die kleiner als alle Funktionswerte ist.

\item ... \notion{beschränkt}, wenn sie sowohl nach oben als auch nach unten beschränkt ist.

\item ... \notion{monoton wachsend} bzw. \notion{streng monoton wachsend}, wenn für $x_1 < x_2$ stets $f(x_1) \leq f(x_2)$ bzw. $f(x_1) < f(x_2)$ gilt.

\item ... \notion{monoton fallend} bzw. \notion{streng monoton fallend}, wenn für  $x_1 < x_2$ stets $f(x_1) \geq f(x_2)$ bzw. $f(x_1) > f(x_2)$ gilt.

\end{itemize}
}
\lang{en}{
A real-valued function $f$ is called
\begin{itemize}
\item ... \notion{bounded from above} if there exists a real number
$C$ that is greater than every value of $f$.

\item ... \notion{bounded from below} if there exists a real number
$c$ that is less than every value of $f$.

\item ... \notion{bounded} if it is bounded both from above and
from below.

\item ... \notion{monotonically increasing} or \notion{strictly monotonically increasing}
if $x_1 < x_2$ always implies $f(x_1) \leq f(x_2)$ or $f(x_1) < f(x_2)$, respectively.

\item ... \notion{monotonically decreasing} or \notion{strictly monotonically decreasing}
if $x_1 < x_2$ always implies $f(x_1) \geq f(x_2)$ or $f(x_1) > f(x_2)$, respectively.
\end{itemize}
}
\end{definition}

\lang{de}{Zu guter letzt erwähnen wir hier noch besondere Punkte auf dem Funktionsgraphen: die \emph{Nullstellen}. 
}
\lang{en}{
Finally, we will mention a distinguished type of point on the
graph of a function: its \emph{zeros}.
}

\begin{definition}
\lang{de}{Ist $f$ eine Funktion und $x_0$ ein Element des Definitionsbereichs mit $f(x_0) = 0$, dann nennen wir
$x_0$ eine \notion{Nullstelle} von $f$.}
\lang{en}{
Let $f$ be a function. An element $x_0$ of the domain with $f(x_0) = 0$
is called a \notion{zero} of $f$.
}
\end{definition}
\lang{de}{Weitere besondere Punkte wie \emph{Extrempunkte} und \emph{Wendepunkte} werden wir im Teil \emph{Differentialrechnung} 
untersuchen. Tatsächlich werden wir dabei dann im Wesentlichen die Nullstellen der ersten oder 
zweiten Ableitung suchen.}
\lang{en}{
Other distinguished points such as \emph{extreme points} and
\emph{inflection points} will be studied in the section on
\emph{differential calculus}. This essentially amounts to looking
for the zeros of the first or second derivative.
}


\section{\lang{de}{Grenzwerte und Stetigkeit} \lang{en}{Limits and continuity}}
\lang{de}{Typischerweise betrachten wir Funktionen, die auf ganz $\R$ oder zumindest 
Teilintervallen davon definiert sind. Häufig interessieren wir uns dafür, wie 
sich die Funktionswerte für immer größer werdende oder immer kleiner werdende Funktionswerte
entwickeln - oder wie sie sich in der Nähe eines bestimmten $x$-Wertes verhalten.
Um dieses \glqq in der Nähe\grqq\ mathematisch korrekt zu formulieren, greifen wir 
auf den Grenzwertbegriff von Folgen zurück. }

\lang{en}{
Typically, we consider functions that are defined on all of $\R$
or at least on subintervals of it. We are often interested in the way
the function behaves for increasing or decreasing arguments - 
or how it behaves near a particular value of $x$. To formulate "near"
in a mathematically precise way, we use the notion of the limit
of a sequence.
}

\begin{definition}\label{def:funktionsgrenzwert}
\lang{de}{Es sei $f: D\to \R$ eine Funktion. Weiter sei $x_0 \in \R$ oder $x_0 = \pm \infty$. 
}
\lang{en}{
Let $f: D\to \R$ be a function and let $x_0 \in \R$ or $x_0 = \pm \infty$.
}
\begin{itemize}
\item  \lang{de}{Es sei $(x_n)$ eine Folge mit Folgengliedern 
in $D \setminus \{ x_0 \}$ und $\lim_{n \to \infty} x_n = x_0$. Wenn für jede solche Folge $(x_n)$ der 
Grenzwert $\lim_{n \to \infty} f(x_n)$ immer den gleichen Wert $L$ liefert, dann schreiben wir 
\[
\lim_{x \to x_0} f(x_0) = L
\]
und nennen diesen Wert den \notion{Grenzwert von $f$ an $x_0$}.}
\lang{en}{
Let $(x_n)$ be a sequence whose terms lie in $D \backslash \{x_0\}$
such that $\lim_{n \to \infty} x_n = x_0$. If, for every such sequence
$(x_n)$, the limit $\lim_{n \to \infty} f(x_n)$ has the same
value $L$, then we write
\[
\lim_{x \to x_0} f(x_0) = L
\]
and call this value the \notion{limit of $f$ at $x_0$}.
}

\item  \lang{de}{Es sei $(x_n)$ eine Folge mit Folgengliedern 
in $D$, $\lim_{n \to \infty} x_n = x_0$ und $x_n < x_0$ (bzw. $x_n > x_0$). Wenn für jede solche Folge $(x_n)$ der 
Grenzwert $\lim_{n \to \infty} f(x_n)$ immer den gleichen Wert $L$ liefert, dann schreiben wir 
\[
\lim_{x \nearrow x_0} f(x_0) = L \quad \left( \text{bzw.} \ \lim_{x \searrow x_0} f(x_0) = L\right)
\]
und nennen diesen Wert \notion{linksseitigen} (bzw. \notion{rechtsseitigen}) Grenzwert von $f$ an $x_0$.
}
\lang{en}{
Let $(x_n)$ be a sequence whose terms lie in $D$, with
$\lim_{n \to \infty} x_n = x_0$ and $x_n < x_0$ (or $x_n > x_0$).
If, for every such sequence $(x_n)$, the limit $\lim_{n \to \infty} f(x_n)$
has the same value $L$, then we write
\[
\lim_{x \nearrow x_0} f(x_0) = L \quad \left( \text{or} \ \lim_{x \searrow x_0} f(x_0) = L\right)
\]
and call this value the \notion{left-hand} (or \notion{right-hand})
limit of $f$ at $x_0$, respectively.
}

\end{itemize}
\lang{de}{Existiert keine Folge $(x_n)$ mit den geforderten Eigenschaften, so kann der Grenzwert von $f$ nicht gebildet werden.
}
\lang{en}{
If no sequence $(x_n)$ with these properties exists,
then the limit of $f$ does not exist.
}
\end{definition}

\lang{de}{Dass man hier \emph{alle} Folgen mit Grenzwert $x_0$ betrachtet, liegt daran, dass der Funktionengrenzwert unabhängig davon 
sein soll, wie man sich $x_0$ nähert.}
\lang{en}{
We consider \emph{all} sequences with limit $x_0$ because the
limit of the function is supposed to be independent of the way
we approach $x_0$.
}

\lang{de}{Mit Hilfe dieser Grenzwerte von Funktionen können wir eine weitere wichtige Eigenschaft, die Funktionen 
haben können, definieren: }
\lang{en}{
Using limits of functions, we can define another important
property that functions may have:
}

\begin{definition}[\lang{de}{Stetigkeit} \lang{en}{Continuity}]
\begin{itemize}
\item \lang{de}{Eine Funktion $f$ heißt \notion{stetig an $x_0$}, wenn  
\[
\lim_{x \to x_0} f(x) = f(x_0)
\]
gilt, also wenn der Grenzwert existiert und mit dem Funktionswert übereinstimmt.}
\lang{en}{
A function $f$ is called \notion{continuous at $x_0$}, if
\[
\lim_{x \to x_0} f(x) = f(x_0);
\]
that is, the limit exists and equals the value of the function there.
}
\item \lang{de}{Eine Funktion heißt \notion{stetig}, wenn sie für jedes $x_0$ des Definitionsbereichs stetig ist. 
}
\lang{en}{
A function is \notion{continuous} if it is continuous at every
point $x_0$ of the domain.
}

\end{itemize}
\end{definition}

\lang{de}{In der Praxis können wir Stetigkeit nachweisen, indem wir linksseitigen und rechtsseitigen Grenzwert
einzeln berechnen und jeweils mit dem Funktionswert vergleichen.}
\lang{en}{
In practice, we can check continuity by computing the left-hand
and right-hand limits separately and comparing them with the
value of the function.
}

\lang{de}{Anschaulich bedeutet Stetigkeit, dass die Funktion dort, wo sie definiert ist, keine Sprünge macht. 
}
\lang{en}{
Intuitively, a function being continuous means that it does not
jump anywhere where it is defined.
}

\begin{example}
\begin{tabs*}[\initialtab{1}\class{example}]
   \tab{$\frac{x^2-4}{x-2}$}

\lang{de}{Wir betrachten die Funktion 
\[ f:\R\setminus { \{2\} }\to \R, x\mapsto \frac{x^2-4}{x-2} \]
und $x_0=2\notin D$.}
\lang{en}{
Consider the function
\[ f:\R\setminus { \{2\} }\to \R, x\mapsto \frac{x^2-4}{x-2} \]
and $x_0=2\notin D$.
}

\lang{de}{Dann gilt für jede Folge $(x_n)_{n \in \N}$ mit $x_n\neq 2$ und
$\lim_{n\to \infty} x_n=2$:
\[ \lim_{n\to \infty} f(x_n)=\lim_{n\to \infty} \frac{x_n^2-4}{x_n-2}
=\lim_{n\to \infty} \frac{(x_n-2)(x_n+2)}{x_n-2}
=\lim_{n\to \infty} x_n+2 =2+2=4.\]

Also ist $\lim_{x\to 2} f(x)=4$.}

\lang{en}{
For any sequence $(x_n)_{n \in \N}$ with $x_n\neq 2$ and
$\lim_{n\to \infty} x_n=2$, we have
\[ \lim_{n\to \infty} f(x_n)=\lim_{n\to \infty} \frac{x_n^2-4}{x_n-2}
=\lim_{n\to \infty} \frac{(x_n-2)(x_n+2)}{x_n-2}
=\lim_{n\to \infty} x_n+2 =2+2=4.\]

Therefore, $\lim_{x\to 2} f(x)=4$.}

\tab{$\frac{x^2-4}{|x-2|}$}
\lang{de}{Wir betrachten die Funktion 
\[ f:\R\setminus { \{2\} }\to \R, x\mapsto \frac{x^2-4}{|x-2|} \]
und $x_0=2\notin D$.}
\lang{en}{
Consider the function
\[ f:\R\setminus { \{2\} }\to \R, x\mapsto \frac{x^2-4}{|x-2|} \]
and $x_0=2\notin D$.}

\lang{de}{Ist $(x_n)$ eine Folge mit $x_n> 2$ für alle $n \in \N$ und
$\lim_{n\to \infty} x_n=2$, so gilt:
\[ \lim_{n\to \infty} f(x_n)=\lim_{n\to \infty} \frac{x_n^2-4}{|x_n-2|}
=\lim_{n\to \infty} \frac{(x_n-2)(x_n+2)}{x_n-2}
=\lim_{n\to \infty} x_n+2 =2+2=4.\]}
\lang{en}{
For any sequence $(x_n)$ with $x_n> 2$ for all $n \in \N$ and
$\lim_{n\to \infty} x_n=2$, we have:
\[ \lim_{n\to \infty} f(x_n)=\lim_{n\to \infty} \frac{x_n^2-4}{|x_n-2|}
=\lim_{n\to \infty} \frac{(x_n-2)(x_n+2)}{x_n-2}
=\lim_{n\to \infty} x_n+2 =2+2=4.\]
}

\lang{de}{Ist jedoch $(x_n)$ eine Folge mit $x_n< 2$ und
$\lim_{n\to \infty} x_n=2$, so gilt:
\[ \lim_{n\to \infty} f(x_n)=\lim_{n\to \infty} \frac{x_n^2-4}{|x_n-2|}
=\lim_{n\to \infty} \frac{(x_n-2)(x_n+2)}{-(x_n-2)}
=\lim_{n\to \infty} -(x_n+2) =-(2+2)=-4.\]}
\lang{en}{
On the other hand, if $(x_n)$ is a sequence with $x_n < 2$ and
$\lim_{n\to \infty} x_n=2$, then
\[ \lim_{n\to \infty} f(x_n)=\lim_{n\to \infty} \frac{x_n^2-4}{|x_n-2|}
=\lim_{n\to \infty} \frac{(x_n-2)(x_n+2)}{-(x_n-2)}
=\lim_{n\to \infty} -(x_n+2) =-(2+2)=-4.\]
}

\lang{de}{Damit haben wir $\lim_{x \searrow 2} f(x) = 4$ und $\lim_{x \nearrow 2} f(x) = -4$. Die Funktion ist trotzdem 
stetig, denn an $x_0 = 2$ ist sie gar nicht definiert.}
\lang{en}{
It follows that $\lim_{x \searrow 2} f(x) = 4$ and $\lim_{x \nearrow 2} f(x) = -4$.
Even so, this function is continuous, because it is not even
defined at $x_0 = 2$.
}

\tab{sgn$(x)$}
\lang{de}{Wir betrachten die Signumfunktion 
\[
\text{sgn}(x) = \left\{ 
   \begin{matrix}
   -1,  &\text{ falls }x <0{,}\\
   0, &\text{ falls } x = 0 \text{,} \\
   1, &\text{ falls } x > 0 \text{.} 
\end{matrix} 
 \right.
\]}
\lang{en}{
Consider the sign function,
\[
\text{sgn}(x) = \left\{ 
   \begin{matrix}
   -1,  &\text{ if }x <0{,}\\
   0, &\text{ if } x = 0 \text{,} \\
   1, &\text{ if } x > 0 \text{.} 
\end{matrix} 
 \right.
\]}
\begin{center}
\image{T601_Signum}
\end{center}
\lang{de}{Für positive Zahlen ist die Funktion konstant 1, für negative Zahlen konstant -1. 
Entsprechend erhalten wir bei Näherung an $0$ die Grenzwerte
\begin{align*}
 &\lim_{t \nearrow 0} \text{sgn}(x) = \lim_{t \nearrow 0} -1 = -1, \\
 &\lim_{t \searrow 0} \text{sgn}(x) = \lim_{t \searrow 0} 1 = 1. 
\end{align*}
Die beiden Werte stimmen nicht überein und stimmen auch nicht mit dem Funktionswert 
sgn$(0)=0$ überein. Die Signumfunktion ist somit nicht stetig an der Stelle $x_0 = 0$. 
An allen anderen Stellen ist sie stetig. }

\lang{en}{
At positive numbers, the function is the constant 1, and at negative numbers it is the constant -1.
This means that we have the limits
\begin{align*}
 &\lim_{t \nearrow 0} \text{sgn}(x) = \lim_{t \nearrow 0} -1 = -1, \\
 &\lim_{t \searrow 0} \text{sgn}(x) = \lim_{t \searrow 0} 1 = 1,
\end{align*}
as we approach zero.
These values do not agree, and they also do not equal the value
sgn$(0)=0$. Therefore, the sign function is not continuous at the point $x_0 = 0$.
It is continuous at all other points.
}
\end{tabs*}
\end{example}


\lang{de}{Die meisten Funktionen, die wir aus der Schule kennen, sind stetig:}
\lang{en}{Most functions that we know from school are continuous:}
\begin{rule}
\begin{itemize}
\item \lang{de}{Die folgenden Funktionen sind Beispiele für stetige Funktionen:}
\lang{en}{The following functions are examples of continuous functions:}
$x^n$, $|x|$, $a^x$, $\ln(x)$, $\sin(x)$, $\cos(x)$.
\item \lang{de}{Die Summe, Differenz, das Produkt, der Quotient, die Verkettung von stetigen Funktionen 
ist wieder stetig. }
\lang{en}{
The sum, difference, product, quotient or composition of continous
functions is continuous.
}
\item \lang{de}{Insbesondere sind alle 
ganzrationalen Funktionen der Form $f(x) = a_n x^n + \ldots + a_1 x + a_0$ stetig.}
\lang{en}{
In particular, all polynomial functions of the form
$f(x) = a_n x^n + \ldots + a_1 x + a_0$ are continuous.
}
\end{itemize}
\end{rule}

\end{visualizationwrapper}

\end{content}
