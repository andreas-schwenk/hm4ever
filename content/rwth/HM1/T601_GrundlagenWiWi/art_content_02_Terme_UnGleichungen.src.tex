%$Id:  $
\documentclass{mumie.article}
%$Id$
\begin{metainfo}
  \name{
    \lang{de}{Elementares Rechnen und Termumformungen}
    \lang{en}{Elementary arithmetic and rearranging expressions}
  }
  \begin{description} 
 This work is licensed under the Creative Commons License Attribution 4.0 International (CC-BY 4.0)   
 https://creativecommons.org/licenses/by/4.0/legalcode 

    \lang{de}{Beschreibung}
    \lang{en}{Description}
  \end{description}
  \begin{components}
  \end{components}
  \begin{links}
    \link{generic_article}{content/rwth/HM1/T101neu_Elementare_Rechengrundlagen/g_art_content_05_loesen_gleichungen_und_lgs.meta.xml}{content_05_loesen_gleichungen_und_lgs}
    \link{generic_article}{content/rwth/HM1/T104_weitere_elementare_Funktionen/g_art_content_13_wurzelfunktionen.meta.xml}{wurzelfunktionen}
    \link{generic_article}{content/rwth/HM1/T101neu_Elementare_Rechengrundlagen/g_art_content_03_bruchrechnung.meta.xml}{bruchrechnung}
    \link{generic_article}{content/rwth/HM1/T104_weitere_elementare_Funktionen/g_art_content_14_potenzregeln.meta.xml}{potenzregeln}
  \end{links}
  \creategeneric
\end{metainfo}
\begin{content}

\usepackage{mumie.ombplus}
\ombchapter{1}
\ombarticle{2}

\usepackage{mumie.genericvisualization}

\begin{visualizationwrapper}

\title{
  \lang{de}{Elementares Rechnen und Termumformungen}
  \lang{en}{Elementary arithmetic and rearranging expressions}
}

 
\begin{block}[annotation]

\end{block}


\begin{block}[annotation]
	Im Ticket-System: \href{https://team.mumie.net/issues/22661}{Ticket 22661}
\end{block}

\begin{block}[info-box]
\tableofcontents
\end{block}



\section {\lang{de}{Grundrechenarten und Terme} \lang{en}{Elementary operations and expressions}} \label{sec:grundrechenarten}

\lang{de}{Nachdem wir im letzten Kapitel die Zahlenmengen eingeführt haben, 
rufen wir uns nun ins Gedächtnis, wie wir mit Zahlen oder Termen 
rechnen können. 

Um Missverständnisse zu vermeiden, wiederholen wir hier kurz einige Fachbegriffe. }

\lang{en}{Having introduced number sets in the last chapter,
we will now recall how to carry out arithmetic with numbers and expressions.

To avoid any ambiguity, we will first review some terminology.}

\begin{definition}[\lang{de}{Grundrechenarten} \lang{en}{Elementary operations}] \label{def:grundrechenarten}
\\
\lang{de}{
 \begin{table}[\class{layout} \cellaligns{lccl} \cellvaligns{tttt}]
    \notion{Addition} &$\;(\mathbf{+})\quad$ &$ 5+8  =13 \quad$  
     & "`$5+8 $"' sowie das Ergebnis $13$ bezeichnet man als \notion{Summe,} 
     die Zahlen $5$ und $8$ nennt man \notion{Summanden.}\\
&&&\\  
    \notion{Subtraktion} &$\;(\mathbf{-})\quad$ &$ 12-4  =8 \quad$  
    & "`$12 - 4 $"' sowie das Ergebnis $8$ bezeichnet man als \notion{Differenz,} 
    die Zahl $12$ nennt man \notion{Minuend} und die $4$ ist der \notion{Subtrahend.} \\
    
    \notion{Multiplikation} &$\; (\mathbf{\cdot})\quad$ &$ 3\cdot 6 =18 \quad$  
    & "`$3\cdot 6 $"' sowie das Ergebnis $18$ bezeichnet man als \notion{Produkt,}
    die Zahlen $3$ und $6$ nennt man \notion{Faktoren}.\\
    
    \notion{Division} &$\; (\mathbf{:})\quad$ &$ 12:2 = \frac{12}{2}=6  \quad $  

    & Die $12$ nennt man \notion{Dividend}, die $2$ \notion{Divisor}. Schreibt man $\frac{12}{2}$, bezeichnet man die 
    $12$ als \notion{Zähler}, die $2$ als \notion{Nenner} und nennt das Ganze einen \notion{Bruch}.  
    Diesen bezeichnet man dann auch, ebenso wie "`$12:2 $"' und das Ergebnis $6$, als \notion{Quotient.}
    
 \end{table}}

 \lang{en}{
 \begin{table}[\class{layout} \cellaligns{lccl} \cellvaligns{tttt}]
    \notion{Addition} &$\;(\mathbf{+})\quad$ &$ 5+8  =13 \quad$  
     & Both "$5+8 $" and the result $13$ are called the \notion{sum;} 
     the numbers $5$ and $8$ are called \notion{summands.}\\
&&&\\  
    \notion{Subtraction} &$\;(\mathbf{-})\quad$ &$ 12-4  =8 \quad$  
    & Both "$12 - 4 $" and the result $8$ are called the \notion{difference;} 
    the number $12$ is called the \notion{minuend} and $4$ is called the \notion{subtrahend.} \\
    
    \notion{Multiplication} &$\; (\mathbf{\cdot})\quad$ &$ 3\cdot 6 =18 \quad$  
    & Both "$3\cdot 6 $" and the result $18$ are called the \notion{product;}
    the numbers $3$ and $6$ are called \notion{factors}.\\
    
    \notion{Division} &$\; (\mathbf{/})\quad$ &$ 12/ 2 = \frac{12}{2}=6  \quad $  

    & $12$ is called the \notion{dividend} and $2$ the \notion{divisor}. When we write $\frac{12}{2}$, 
    $12$ is called the \notion{numerator} and $2$ the \notion{denominator}, and the entirety is called a \notion{fraction}.  
    This, along with "$12/ 2 $" and the result $6$, is also called the \notion{quotient.}
    
 \end{table}}
\end{definition}



\lang{de}{Ausgehend von  den Grundrechenarten können wir weitere mathematische Ausdrücke definieren: 
  
} 
\lang{en}{We can now define more general mathematical expressions
based on the elementary operations}
 
\begin{definition}[\lang{de}{Potenzen und Wurzeln} \lang{en}{Powers and roots}] \label{def:potenz}
 \lang{de}{Für beliebige reelle Zahlen $a$ und natürliche Zahlen $n$ gilt:
      
  Die ($n$-te)  \notion{\emph{Potenz}} $a^n$ bezeichnet das $n$-fache Produkt einer reellen Zahl $a$ mit sich 
  selbst, d.\,h. \[a^n := \underbrace{a\cdot a\cdot \ldots \cdot a}_{\text{$n$-mal}}.\]            
  Dabei nennt man $a$  die \notion{\emph{Basis}} und $\,n$ den \notion{\emph{Exponenten}} von $a^n$. Im 
  Spezialfall $n=2$ sprechen wir auch vom \notion{\emph{Quadrat}} von $a$. 
  
  Die \notion{\emph{n-te Wurzel}} $\sqrt[n]{a}$ definieren wir als  
  Zahl, die $\,n\,$-mal mit sich selbst multipliziert $\,a\,$ ergibt, d.\,h. deren 
  \emph{n-te Potenz} gleich $a$ ist. 
  
  Im Spezialfall $n=2$ sprechen wir von der \notion{\emph{Quadratwurzel}} oder einfach \notion{\emph{Wurzel}}
  aus $a$ und schreiben $\, \sqrt{a}\, $ statt $\, \sqrt[2]{a}.\;$}
  \lang{en}{ For any real number $a$ and natural number $n$:

  The ($n$-th) \notion{\emph{power}} $a^n$ is the product of $a$ with itself $n$ times, i.e.
  \[a^n := \underbrace{a\cdot a\cdot \ldots \cdot a}_{\text{$n$ times}}.\]
  Here, $a$ is called the \notion{\emph{base}} and $\,n$ the \notion{\emph{exponent}}
  in $a^n$. In the special case $n=2$, we call $a^2$ the \notion{\emph{square}} of $a$.

  The \notion{\emph{n-th root}} $\sqrt[n]{a}$ is the number that can be multiplied by
  itself $n$ times to yield $a$, i.e. the number whose \emph{n-th power} equals $a$.

  In the special case $n=2$, we call this the \notion{\emph{square root}}
  of $a$ and write $\, \sqrt{a}\, $ rather than $\, \sqrt[2]{a}.\;$
  }
    
\end{definition}

\lang{de}{Eine ($n$-te) Wurzel existiert nicht immer, z.\,B. gibt es keine reelle Zahl, deren Quadrat negativ ist. Daher 
besitzen negative Zahlen keine Quadratwurzeln. Für positive relle Zahlen existiert hingegen immer eine eindeutige 
$n$-te Wurzel und  definitionsgemäß gilt $\ (\sqrt[n]{a})^n = a.$}
\lang{en}{$n$-th roots do not always exist. For example, there is no
real number whose square is negative, so negative numbers do not
have square roots. For positive real numbers, on the other hand,
there always exists a unique $n$-th root, and $(\sqrt[n]{a})^n = a$ holds
by definition.}

\lang{de}{Wenn die Basis positiv ist, kann der Exponent auch negativ oder gebrochen (oder sogar eine 
beliebige reelle Zahl) sein. Die ersten solchen Fälle führen wir in den folgenden 
Potenzgesetzen mit auf:}
\lang{en}{If the base is positive, then the exponent can also be
negative or fractional (or even an arbitrary real number).
The first such cases are included in the following
exponentiation rules:}

\begin{rule}[\lang{de}{Potenzgesetze} \lang{en}{Exponentiation rules}]\label{rule:potenzgesetze}
\lang{de}{F\"ur positive relle Zahlen $a,b$ und beliebige Exponenten $r, s$ gilt:}
\lang{en}{For positive real numbers $a,b$ and any exponents $r,s$:}
\begin{itemize}
\item $a^1 = a, \quad a^0 = 1, \quad a^{-1} = \frac{1}{a},$
\item $a^r\cdot a^s=a^{r+s}, \quad$
\item $a^r\cdot b^r=(a\cdot b)^{r},\quad \left(\frac{a}{b}\right)^r = \frac{a^r}{b^r},$
\item $(a^r)^s=a^{r\cdot s},\quad$
\item $ \sqrt[r]{a} = a^{\frac{1}{r}}, \quad a^{\frac{s}{r}} = \sqrt[r]{a^s},$
\item $\sqrt[r]{a} \cdot \sqrt[r]{b} = \sqrt[r]{a \cdot b}.$ 

\end{itemize}

\end{rule}

\begin{example} 
\lang{de}{
  \begin{itemize}
      \item $2^5=2 \cdot 2 \cdot 2 \cdot 2 \cdot 2=32$
    \item $2^4 \cdot 2^2=\underbrace{2 \cdot 2 \cdot 2 \cdot 2}_{\text{$4$-mal}} \cdot \underbrace{2 \cdot 2}_{\text{$2$-mal}}
       = \underbrace{2 \cdot 2 \cdot 2 \cdot 2 \cdot 2 \cdot 2}_{\text{$6$-mal}}= 2^6$
    \item $5^2 \cdot 2^2=\underbrace{5 \cdot 5 }_{\text{$2$-mal}} \cdot \underbrace{2 \cdot 2}_{\text{$2$-mal}}
       = \underbrace{(5 \cdot 2) \cdot (5 \cdot 2) }_{\text{$2$-mal}}= (5 \cdot 2)^2=10^2$
    \item  $\sqrt[5]{32}=\sqrt[5]{(2^5)}=2$
    \item $3^{-2} = (3^{-1})^2 = \left(\frac{1}{3}\right)^2 = \frac{1}{9}$
  \end{itemize}
}
\lang{en}{
  \begin{itemize}
      \item $2^5=2 \cdot 2 \cdot 2 \cdot 2 \cdot 2=32$
    \item $2^4 \cdot 2^2=\underbrace{2 \cdot 2 \cdot 2 \cdot 2}_{\text{$4$ times}} \cdot \underbrace{2 \cdot 2}_{\text{$2$ times}}
       = \underbrace{2 \cdot 2 \cdot 2 \cdot 2 \cdot 2 \cdot 2}_{\text{$6$ times}}= 2^6$
    \item $5^2 \cdot 2^2=\underbrace{5 \cdot 5 }_{\text{$2$ times}} \cdot \underbrace{2 \cdot 2}_{\text{$2$ times}}
       = \underbrace{(5 \cdot 2) \cdot (5 \cdot 2) }_{\text{$2$ times}}= (5 \cdot 2)^2=10^2$
    \item  $\sqrt[5]{32}=\sqrt[5]{(2^5)}=2$
    \item $3^{-2} = (3^{-1})^2 = \left(\frac{1}{3}\right)^2 = \frac{1}{9}$
  \end{itemize}
}
\end{example}



\lang{de}{H\"aufig werden wir Ausdr"ucken begegnen, die zwei oder mehr Rechenoperationen enthalten und 
in denen zudem neben Zahlen auch noch \glqq Buchstaben\grqq\ (\emph{Variablen} oder \emph{Funktionen}) auftreten. 
Einen solchen Ausdruck bezeichnen wir als \emph{Term.}
}
\lang{en}{We will often encounter combinations of two or more operations
that also involve "letters" (\emph{variables}) or \emph{functions}.
These will be called \emph{expressions.}
}

\begin{definition}[\lang{de}{Terme} \lang{en}{Expression}]
\lang{de}{\notion{\emph{Terme}} sind sinnvolle Ausdrücke, die Zahlen, Variablen, Klammern und verschiedene Rechenoperationen 
enthalten können.}
\lang{en}{An \notion{\emph{expression}} is a meaningful combination
of numbers, variables, parentheses and mathematical operations.
}
\end{definition}

\begin{example} \label{ex:terme}   
\lang{de}{Die folgenden Ausdr"ucke sind Beispiele für Terme:}
\lang{en}{The following are examples of expressions:}

\begin{itemize}    

  \item $2\cdot 4+5\cdot (3+2)  $\\  

  \item $15 \cdot b-\frac{1}{4}b^4   $\\

  \item $(2x-1)\cdot y^6 - 2x \cdot (y^2)^3 $\\

  \item $\displaystyle{\frac{a^2-b^2}{a-b}} $ 

\end{itemize}
 
\end{example}

\lang{de}{Wenn es \glqq nur\grqq\ um das Ausrechnen von Werten geht, können wir die folgende Regel 
benutzen, um das korrekte Ergebnis zu erhalten. }
\lang{en}{If we only want to compute the value of an expression
then we can use the following rules to find the correct result.}

\begin{rule}[\lang{de}{Reihenfolge beim Rechnen} \lang{en}{Order of operations}]\label{rule:punkt-vor-strich}

\lang{de}{Bei der Berechnung von Termen ist die folgende Reihenfolge zu beachten:
\begin{enumerate}
	\item Ausdrücke in Klammern berechnen \textit{(die innersten Klammern zuerst)}
 	\item Potenzen \textit{(mit Klammer von innen nach au"sen}, ansonsten \textit{von oben nach unten)}
	\item Multiplikationen und Divisionen \textit{(von links nach rechts)}
	\item Additionen und Subtraktionen \textit{(von links nach rechts)}
\end{enumerate} 
}

\lang{en}{When evaluating an expression, use the following order of operations:
\begin{enumerate}
  \item Compute expressions in parentheses \textit{(beginning with the innermost parentheses)}
  \item Exponents (\textit{beginning with the innermost parentheses} if any, otherwise \textit{from top to bottom})
  \item Multiplication and division \textit{(from left to right)}
  \item Addition and subtraction \textit{(from left to right)}
\end{enumerate}
}

\\ \lang{de}{Insgesamt gilt also: \notion{Klammer- vor Potenz- vor Punkt- vor Strichrechnung.}    
        }
  \lang{en}{Altogether: \notion{Parentheses, Exponents, Multiplication, Division, Addition, Subtraction}}

\end{rule}


\begin{quickcheckcontainer}

\randomquickcheckpool{1}{4}  % vier verschiedene Terme mit randomisierten Zahlen
%1
\begin{quickcheck}
		\type{input.number}
		\begin{variables}
			\randint[Z]{a}{-1}{5}
			\randint[Z]{b}{1}{3}
			\randint[Z]{c}{2}{4}
			\randint[Z]{d}{1}{5}
		    \function{f}{a-b*c+d}
		    \function[calculate]{loes}{a-b*c+d}
		\end{variables}
		\lang{de}{
			\text{Berechnen Sie den folgenden Term:\\
				  $\var{a}-\var{b}\cdot \var{c}+\var{d}=$\ansref}}
      \lang{en}{
      \text{Calculate the following expression:\\
          $\var{a}-\var{b}\cdot \var{c}+\var{d}=$\ansref}}
		
		
		\begin{answer}
			\solution{loes}
		\end{answer}
	\end{quickcheck}
%2
\begin{quickcheck}
		\type{input.number}
		\begin{variables}
			\randint[Z]{a}{-1}{5}
			\randint[Z]{b}{1}{3}
			\randint[Z]{c}{2}{4}
			\randint[Z]{d}{1}{5}
		    \function[calculate]{loes}{a-b*(c+d)}
		\end{variables}
  		\lang{de}{
			\text{Berechnen Sie den folgenden Term:\\
				  $\var{a}-\var{b}\cdot (\var{c}+\var{d})=$\ansref}}
      \lang{en}{
      \text{Calculate the following expression:\\
          $\var{a}-\var{b}\cdot (\var{c}+\var{d})=$\ansref}}
		
		\begin{answer}
			\solution{loes}
		\end{answer}
	\end{quickcheck}
%3
\begin{quickcheck}
		\type{input.number}
		\begin{variables}
			\randint[Z]{a}{-1}{5}
			\randint[Z]{b}{1}{3}
			\randint[Z]{c}{2}{4}
			\randint[Z]{d}{1}{5}
		    \function[calculate]{loes}{(a-b)*c+d}
		\end{variables}

      \lang{de}{
			\text{Berechnen Sie den folgenden Term:\\
				  $(\var{a}-\var{b})\cdot \var{c}+\var{d}=$\ansref}}
      \lang{en}{
      \text{Calculate the following expression:\\
      $(\var{a}-\var{b})\cdot \var{c}+\var{d}=$\ansref}}

		
		\begin{answer}
			\solution{loes}
		\end{answer}
	\end{quickcheck}
%4
\begin{quickcheck}
		\type{input.number}
		\begin{variables}
			\randint[Z]{a}{-1}{5}
			\randint[Z]{b}{1}{3}
			\randint[Z]{c}{2}{4}
			\randint[Z]{d}{1}{5}
		    \function[calculate]{loes}{a-(b*c+d)}
		\end{variables}

      \lang{de}{
			\text{Berechnen Sie den folgenden Term:\\
				  $\var{a}-(\var{b}\cdot \var{c}+\var{d})=$\ansref}}
      \lang{en}{
      \text{Calculate the following expression:\\
				  $\var{a}-(\var{b}\cdot \var{c}+\var{d})=$\ansref}}

			
		\begin{answer}
			\solution{loes}
		\end{answer}
	\end{quickcheck}

\end{quickcheckcontainer}


\lang{de}{Wenn Variablen auftreten, können Terme nicht so einfach berechnet werden. Deshalb ist man häufig bestrebt, sie so weit wie möglich zu vereinfachen. 
Dabei werden Termumformungen stets durch ein Gleichheitszeichen \glqq{}$=$\grqq{} gekennzeichnet.}

\lang{en}{
Expressions involving variables cannot be calculated as easily.
Therefore, one generally tries to simplify them as much as possible.
Rearrangements of an expression are always labelled with an equality sign "$=$".
}


\begin{rule}[\lang{de}{Rechengesetze} \lang{en}{Laws of algebra}]\label{rule:rechengesetze}
\lang{de}{F\"ur die Addition und die Multiplikation reeller Zahlen $a,b$ und $c \;$ gelten: \\

\\	\begin{table}[\class{layout}  \cellaligns{lccc}]
		
		& \notion{Kommutativgesetz:} 
        &  $\quad a+b =b+a \quad$ & und & 
           $\quad a \cdot b = b \cdot a$ \\
	
	\end{table}       
\\	\begin{table}[\class{layout}  \cellaligns{lccc}]
	
    	& \notion{Assoziativgesetz:} 
        &  $\quad (a+b)+c =a+(b+c)\quad$  & und & 
           $\quad (a\cdot b)\cdot c = a\cdot (b\cdot c)$ \\ 
	
	\end{table}        
\\	\begin{table}[\class{layout}  \cellaligns{lccc}]
	
        & \notion{Distributivgesetz:}
		&  $\quad a \cdot (b + c) = a \cdot b+ a \cdot c \quad$ & und &
           $\quad (b + c)\cdot a = b\cdot a +c\cdot a $ \\ 		
	
	\end{table}
	}
 \lang{en}{Addition and multiplication of real numbers $a,b$ and $c \;$ satisfy the following laws: \\

\\	\begin{table}[\class{layout}  \cellaligns{lccc}]
		
		& \notion{Commutative law:} 
        &  $\quad a+b =b+a \quad$ & and & 
           $\quad a \cdot b = b \cdot a$ \\
	
	\end{table}       
\\	\begin{table}[\class{layout}  \cellaligns{lccc}]
	
    	& \notion{Associative law:} 
        &  $\quad (a+b)+c =a+(b+c)\quad$  & and & 
           $\quad (a\cdot b)\cdot c = a\cdot (b\cdot c)$ \\ 
	
	\end{table}        
\\	\begin{table}[\class{layout}  \cellaligns{lccc}]
	
        & \notion{Distributive law:}
		&  $\quad a \cdot (b + c) = a \cdot b+ a \cdot c \quad$ & and &
           $\quad (b + c)\cdot a = b\cdot a +c\cdot a $ \\ 		
	
	\end{table}}
    
    \lang{de}{Aus dem Distributivgesetz folgt, wie man Klammern ausmultipliziert:}
    \lang{en}{The distributive law shows how to multiply brackets:}
      \[ (a + b) \cdot (c + d) = a \cdot (c + d) + b \cdot (c + d) 
              = a \cdot c + a \cdot d + b \cdot c + b \cdot  d \]
  
\end{rule}

\begin{block}[warning]
\lang{de}{Die Rechengesetze gelten \notion{nicht} für Subtraktion und Division. 

Was aber möglich ist: Subtraktion und Division können als Addition und Multiplikation geschrieben werden:}
\lang{en}{
The above laws do \notion{not} hold for subtraction and division.
However, subtraction and division can be rewritten as addition and multiplication; for example,
}
\[
2-3 = 2 + (-3) = (-3) + 2, \quad \quad \frac{6}{12} = 6 \cdot \frac{1}{12} = \frac{1}{12} \cdot 6.
\]
\end{block}


\begin{quickcheck}
		\type{input.number}
		\begin{variables}
			\randint[Z]{a}{-1}{5}
			\randint[Z]{b}{1}{3}
			\randint[Z]{c}{2}{4}
			\randint[Z]{d}{1}{5}
            \randint[Z]{g}{1}{5}
		    
			\function[calculate]{loes1}{a*c}
			\function[calculate]{loes2}{d-b*c}
            \function[calculate]{loes3}{g*c-d}
		\end{variables}

      \lang{de}{
			\text{Vereinfachen Sie den folgenden Term m"oglichst weit: \\
			$(\var{a} - \var{b} x + \var{g} y)\cdot \var{c}+\var{d}\cdot (x-y)=$\ansref + \ansref $\cdot x + $\ansref $\cdot y$}}
		\lang{en}{
			\text{Simplify the following expression as much as possible: \\
			$(\var{a} - \var{b} x + \var{g} y)\cdot \var{c}+\var{d}\cdot (x-y)=$\ansref + \ansref $\cdot x + $\ansref $\cdot y$}}
		
		\begin{answer}
			\solution{loes1}
		\end{answer}
		\begin{answer}
			\solution{loes2}
		\end{answer}
        \begin{answer}
			\solution{loes3}
		\end{answer}
\end{quickcheck}


\lang{de}{Mit Hilfe des Distributivgesetzes können wir 
auch die \emph{binomischen Formeln} herleiten. Diese helfen dabei, schnell 
auszumultiplizieren oder - rückwärts angewandt - Terme zu faktorisieren.}

\lang{en}{
We can also use the distributive law to derive the \emph{binomial formulas}.
These are useful for quickly expanding expressions or - when used in reverse - factoring them.
}

\begin{rule}[\lang{de}{Binomische Formeln} \lang{en}{Binomial formulas}]\label{rule:binomische_formeln}

\lang{de}{Für alle reellen Zahlen $a$ und $b$ gelten:}
\lang{en}{For all real numbers $a$ and $b$:}

\lang{de}{
\\	\begin{table}[\class{layout} \cellaligns{llcl}]
		
		& \notion{1. binomische Formel:} &$\quad (a+b)^2 $&$ =a^2+2ab+b^2 $\\
             

		& \notion{2. binomische Formel:} &$\quad(a-b)^2 $&$ =a^2-2ab+b^2$ \\ 
        

        & \notion{3. binomische Formel:} &$\quad (a+b)(a-b) \; $&$ =a^2-b^2 $  \\
		
	
	\end{table}
}
\lang{en}{
\\	\begin{table}[\class{layout} \cellaligns{llcl}]
		
		& \notion{First binomial formula:} &$\quad (a+b)^2 $&$ =a^2+2ab+b^2 $\\
             

		& \notion{Second binomial formula:} &$\quad(a-b)^2 $&$ =a^2-2ab+b^2$ \\ 
        

        & \notion{Third binomial formula:} &$\quad (a+b)(a-b) \; $&$ =a^2-b^2 $  \\
		
	
	\end{table}
}
    
\end{rule}
 
\begin{tabs*}[\initialtab{0}]
\tab{\lang{de}{Herleitung} \lang{en}{Derivation}}\label{ausmultipliziert}

\lang{de}{Nach \lref{def:potenz}{Definition der Potenz} ist 

\[(a+b)^2 = (a+b)(a+b).\]

Multipliziert man diesen Term nach dem Distributivgesetz aus, so erhält man

\[ (a+b)(a+b) = a \cdot a + a \cdot b + b \cdot a + b \cdot  b =a^2 + 2ab + b^2.\]

Ersetzt man nun $b$ durch $-b$, folgt direkt
\[(a-b)^2=(a+ (-b))^2 = a^2+2a(-b)+(-b)^2=a^2-2ab+b^2 \]

und durch Ausmultiplizieren von $(a+b)(a+ (-b))$ erhält man entsprechend

\[ a \cdot a + a \cdot (-b) + b \cdot a + b \cdot  (-b) =a^2 -ab + ab - b^2=a^2-b^2.\]
}

\lang{en}{ By \lref{def:potenz}{definition},

\[(a+b)^2 = (a+b)(a+b).\]

When we multiply out this expression using the distributive law, we find

\[ (a+b)(a+b) = a \cdot a + a \cdot b + b \cdot a + b \cdot  b =a^2 + 2ab + b^2.\]

By replacing $b$ with $-b$, we immediately obtain
\[(a-b)^2=(a+ (-b))^2 = a^2+2a(-b)+(-b)^2=a^2-2ab+b^2 \]

while multiplying out $(a+b)(a+ (-b))$ leads to
\[ a \cdot a + a \cdot (-b) + b \cdot a + b \cdot  (-b) =a^2 -ab + ab - b^2=a^2-b^2.\]
}

\end{tabs*}

% Beispiele

\begin{example}
  \begin{itemize}
  \item $39^2=(40-1)^2=40^2-2\cdot 40\cdot 1+1^2=1600-80+1=1521$
  \item $(2x+3)^2=(2x)^2+2\cdot 2x\cdot 3+3^2=4x^2+12x+9$
  \item $(x+1)^2-(x-1)^2=(x^2+2x+1)-(x^2-2x+1)$\\
        $\phantom{(x+1)^2-(x-1)^2}=x^2+2x+1-x^2+2x-1$\\
        $\phantom{(x+1)^2-(x-1)^2}=2x+2x=4x$
  \item $(x+y+4)^2=((x+y)+4)^2$\\
        $\phantom{(x+y+4)^2}=(x+y)^2+2\cdot(x+y)\cdot 4+4^2$\\
        $\phantom{(x+y+4)^2}=x^2+2xy+y^2+8(x+y)+16$\\
        $\phantom{(x+y+4)^2}=x^2+2xy+y^2+8x+8y+16$
  \item $(2x+4)(2x-4)=(2x)^2-4^2=4x^2-16$
    \item $4x^2+4x+1=(2x)^2+2\cdot 2x\cdot 1+1^2=(2x+1)^2$ \\
  \end{itemize}

\end{example}
%%%%%%%%%%%%%%%%%%%%%%%%%%%%%555%%%%%%%%%%
\begin{quickcheckcontainer}

\randomquickcheckpool{1}{3}  
%1
\begin{quickcheck}
		\type{input.number}
		\begin{variables}
			\randint[Z]{a}{1}{4}
			\randint[Z]{b}{1}{5}

            \function[calculate]{t1}{2*a*b}
            \function[calculate]{t2}{a^2}
			\function[calculate]{loes1}{a}
			\function[calculate]{loes2}{b}
            \function[calculate]{loes3}{b^2}
		\end{variables}

       \lang{de}{
			\text{Vervollständigen Sie die folgende binomische Formel mit passenden natürlichen Zahlen: \\
			$($ \ansref $ \cdot x +$ \ansref $)^2= \var{t2} \cdot x^2 + \var{t1} \cdot x +$ \ansref}}

   \lang{en}{
			\text{Complete the following binomial formula using an appropriate natural number: \\
			$($ \ansref $ \cdot x +$ \ansref $)^2= \var{t2} \cdot x^2 + \var{t1} \cdot x +$ \ansref}}
		
		
		\begin{answer}
			\solution{loes1}
		\end{answer}
		\begin{answer}
			\solution{loes2}
		\end{answer}
        \begin{answer}
			\solution{loes3}
		\end{answer}
\end{quickcheck}

%2
\begin{quickcheck}
		\type{input.number}
		\begin{variables}
			\randint[Z]{a}{1}{4}
			\randint[Z]{b}{-5}{-1}

			\function[calculate]{t1}{-2*a*b}
			\function[calculate]{loes1}{b}
			\function[calculate]{loes2}{a^2}
            \function[calculate]{loes3}{b^2}
		\end{variables}

       \lang{de}{
			\text{Vervollständigen Sie die folgende binomische Formel: \\
			$(\var{a} +$ \ansref $\cdot y)^2=$ \ansref $ - \var{t1} \cdot y +$ \ansref $\cdot y^2$}}

   \lang{en}{
			\text{Complete the following binomial formula: \\
			$(\var{a} +$ \ansref $\cdot y)^2=$ \ansref $ - \var{t1} \cdot y +$ \ansref $\cdot y^2$}}
		
		
		\begin{answer}
			\solution{loes1}
		\end{answer}
		\begin{answer}
			\solution{loes2}
		\end{answer}
        \begin{answer}
			\solution{loes3}
		\end{answer}
\end{quickcheck}

%3
\begin{quickcheck}
		\type{input.number}
		\begin{variables}
			\randint[Z]{a}{1}{4}
			\randint[Z]{b}{1}{5}

            \function[calculate]{t1}{b^2}
            \function[calculate]{t2}{a^2}
			\function[calculate]{loes1}{b}
			\function{loes2}{a*z}
            \function[calculate]{loes3}{b}
		\end{variables}

     \lang{de}{
			\text{Vervollständigen Sie die folgende binomische Formel mit passenden natürlichen Zahlen: \\
			$( \var{a} z +$ \ansref $) \cdot ($\ansref$-$\ansref$) = \var{t2} z^2 - \var{t1} $}}
   
     \lang{en}{
			\text{Complete the following binomial formula using appropriate natural numbers: \\
			$( \var{a} z +$ \ansref $) \cdot ($\ansref$-$\ansref$) = \var{t2} z^2 - \var{t1} $}}
				
		\begin{answer}
			\solution{loes1}
		\end{answer}
		\begin{answer}
			\solution{loes2}
		\end{answer}
        \begin{answer}
			\solution{loes3}
		\end{answer}
\end{quickcheck}

\end{quickcheckcontainer}
%%%%%%%%%%%%%%%%%%%%%%%%%%%%%555%%%%%%%%%%


\section{\lang{de}{Einfache Gleichungen und Ungleichungen} \lang{en}{Simple equations and inequalities}}


\lang{de}{Meist wollen wir den Wert einer Variable ausrechnen, die in einer Gleichung oder Ungleichung 
auftritt. In vielen Fällen können wir dies mit \emph{Äquivalenzumformungen} bewerkstelligen. Diese werden
durch Äquivalenzpfeile \glqq $\Leftrightarrow$\grqq\ kenntlich gemacht.} 

\lang{en}{
We often want to compute the value of a variable that appears in
an equation or an inequality. In many cases, we can do this by \emph{rearranging} 
it to an equivalent statement. This is denoted by equivalence arrows "$\Leftrightarrow$".
}

\lang{de}{Die folgenden Umformungen sind Äquivalenzumformungen:}
\lang{en}{The following rearrangements yield equivalent statements:}

\begin{block}[info]
\lang{de}{1. Seiten vertauschen:}
\lang{en}{1) Switching sides:}
\begin{align*}
&&\quad					3x+12	&\;=	4		&\\
&\Leftrightarrow&\quad  	4		&\;=	3x+12	&
\end{align*}
\lang{de}{Bei Ungleichungen muss das Vergleichszeichen \glqq{}umgedreht\grqq{} werden:}
\lang{en}{In inequalities, the direction of the inequality symbol must be reversed:}
\begin{align*}
&&\quad					3x+12	&\;\leq	4		&\\
&\Leftrightarrow&\quad		4	&\;\geq	3x+12	&
\end{align*}

\lang{de}{2. Gleiche Zahl addieren oder subtrahieren:}
\lang{en}{2) Adding or subtracting by the same number:}
\begin{align*}
&&\quad					    	3x+12		&\;=4		&  		&\quad\vert  -3& \\
&\Leftrightarrow&\quad		3x+12-3		&\;=4-3	&		&&\\
&\Leftrightarrow&\quad		3x+9		&\;=1	&		&&
\end{align*}

\lang{de}{3. Mit gleicher Zahl $\neq0$ multiplizieren oder dadurch dividieren:}
\lang{en}{3) Multiplying or dividing by the same nonzero number:}
\begin{align*}
&&\quad			    		3x +12			&\;=4				&			&\quad\vert  \lang{de}{:}\lang{en}{/}(-3)&\\
&\Leftrightarrow&\quad	-x-4			&\;=-\frac{4}{3}	&			& &
\end{align*}
\lang{de}{Bei Ungleichungen muss das Vergleichszeichen \glqq{}umgedreht\grqq{} werden, sofern mit einer negativen
Zahl multipliziert oder durch eine negative Zahl dividiert wird:}
\lang{en}{When we multiply or divide an inequality by a negative
number, the inequality symbol must be reversed:}

\begin{align*}
&&\quad			    		3x +12			&\;<4				&		&\quad\vert  \lang{de}{\cdot}\lang{en}{/}(-\frac{1}{3})&\\
&\Leftrightarrow&\quad	-x-4				&\;>-\frac{4}{3}&			&  &\\
\end{align*}
\end{block}

\lang{de}{Äquivalenzumformungen besitzen die Eigenschaft, dass sie zwar die mathematischen Formeln verändern, aber 
die neue Aussage \glqq gleichwertig\grqq zur vorherigen ist. Mit \glqq gleichwertig\grqq ist gemeint, dass die äquivalenten Gleichungen
(oder Ungleichungen) die gleichen Lösungen besitzen. }
\lang{en}{These rearrangements yield a new statement
that is "equivalent" to the original one, even though the
mathematical formulas in it are different. "Equivalence" means that equivalent equations or inequalities
have the same solutions.}

\begin{example}
\begin{enumerate}[alph]
\item \lang{de}{Wir bestimmen die Lösung der folgenden Gleichung mit Hilfe von Äquivalenzumformungen: }
\lang{en}{We will solve the following equation by rearranging it:}
\begin{align*}
&&\quad					5(3x-5)+2x	&=7(3-2x)+16		&		&&\\
&\Leftrightarrow&\quad 	15x-25+2x	&=21-14x+16			&		&&\\
&\Leftrightarrow&\quad 	17x-25		&=37-14x			&		&\vert +14x&\\
&\Leftrightarrow&\quad 	31x-25		&=37				& 		&\vert +25&\\
&\Leftrightarrow&\quad 	31x			&=62				&		&\vert \lang{de}{:}\lang{en}{/}31&\\
&\Leftrightarrow&\quad 	x			&=\frac{62}{31}		&		&&\\
&\Leftrightarrow&\quad 	x			&=2 				&		&&
\end{align*}


\lang{de}{Die Lösung $\, x=2 \,$ können wir nun direkt ablesen und die
\emph{"`Lösungsmenge'"} angeben:}
\lang{en}{We can now read off the solution $\, x=2 \,$
and write out the \emph{solution set}:}
 \[\mathbb{L}=\{\,x\in\R\,|\,x = 2 \,\}=\{2\}.	\]
 
 \item \lang{de}{Wir bestimmen die Lösungsmenge $\mathbb{L}$ der folgenden Ungleichung: }
 \lang{en}{We will determine the solution set $\mathbb{L}$ of the following inequality:}
\begin{align*}
&&\quad					5x-4	    &\geq 2x+2  		&		&\quad\vert -2x+4&\\
&\Leftrightarrow&\quad 	3x      	&\geq 6 			&		&\quad\vert  \lang{de}{:}\lang{en}{/} 3&\\
&\Leftrightarrow&\quad 	x			&\geq 2 			&		&&
\end{align*}

\lang{de}{Hieraus lässt sich ablesen, dass alle reellen Zahlen,
die größer oder gleich $2$ sind, die Ungleichung lösen. Die Lösung der Ungleichung wird daher 
auch dargestellt als die \emph{"`Lösungsmenge'"}}
\lang{en}{
From this we can see that all real numbers that are greater than or
equal to $2$ solve the inequality. The solution of the inequality
can therefore be represented by the \emph{solution set}
}
 \[\mathbb{L}=\{\,x\in\R\,|\,x \geq 2 \,\}=[2\lang{de}{;}\lang{en}{,}\infty).	\]
 
\end{enumerate}

\end{example}


\begin{block}[warning]\label{proposition.warning.1}
\lang{de}{Potenzieren (insbesondere Quadrieren) und Wurzelziehen sind im Allgemeinen keine Äquivalenzumformungen!}
\lang{en}{Exponentiation (including squaring) and taking roots
do not generally yield equivalent propositions!}
\end{block}

\begin{quickcheck}
		\field{rational}
		\type{input.number}
		\begin{variables}
			\randint{a}{2}{4}
			\randint{b}{1}{5}
			\randint{c}{-2}{2}
			\randint{d}{2}{5}
			\function[calculate]{m}{a+d}
			\function[calculate]{n}{c+b}
			\function[calculate]{loes}{(b+c)/(a+d)}	
		\end{variables}
		
%			\text{Die Gleichung $\var{a}x-\var{b}=\var{c}-\var{d}x$ ist äquivalent zur Gleichung\\
%			 $x= $\ansref.}
            \lang{de}{
            \text{Lösen Sie die folgende Gleichung nach $x$ auf. \\
            $\var{a}x-\var{b}=\var{c}-\var{d}x \quad\Leftrightarrow\quad x= $ \ansref.}}

            \lang{en}{
            \text{Solve the following equation for $x$. \\
            $\var{a}x-\var{b}=\var{c}-\var{d}x \quad\Leftrightarrow\quad x= $ \ansref.}}
		\begin{answer}
			\solution{loes}
		\end{answer}
		\explanation{\lang{de}{Mittels Äquivalenzumformungen erhalten wir: } \lang{en}{We rearrange the equation as follows: }
    \begin{align*}
		  & \var{a}x-\var{b} &=\var{c}-\var{d}x &\quad & \vert +\var{d}x \\
		  \Leftrightarrow&\quad \var{a}x+\var{d}x-\var{b} &=\var{c} &\quad & \vert +\var{b} \\
		  \Leftrightarrow&\quad  \var{m}x &= \var{n} &\quad & \vert : \var{m} \\
		  \Leftrightarrow&\quad x&= \frac{\var{n}}{\var{m}} && %  =\var{loes} &&
		\end{align*}
		}
		
		
	\end{quickcheck}


\begin{quickcheck}
  \type{input.number}
  \field{rational}
  \begin{variables}
    \randint{num1}{2}{3}
    \randint{b1}{4}{5}
    \randint{c1}{6}{9}
    \function[calculate]{f1}{(c1-b1)/num1}
    \randint{b2}{2}{5}
    \function[calculate]{b2m}{-b2}
    \function[calculate]{b22}{b2*b2}

  \end{variables}
  \lang{de}{\text{Lösen Sie die folgenden Ungleichungen nach $x$ auf. \\
  $\var{num1}\,x+\var{b1}<\var{c1}\quad\Leftrightarrow\quad x<$ \ansref , \\
%  $\frac{x}{\var{b2m}} +\var{b2}\geq 0 \quad\Leftrightarrow\quad x \leq$ \ansref.}}
  $\displaystyle\Big(-\frac{x}{\var{b2}}\Big) +\var{b2}\geq 0 \quad\Leftrightarrow\quad x \leq$ \ansref.}}
  \lang{en}{\text{Solve the following inequalities for $x$. \\
  $\var{num1}\,x+\var{b1}<\var{c1}\quad\Leftrightarrow\quad x<$ \ansref , \\
%  $\frac{x}{\var{b2m}} +\var{b2}\geq 0 \quad\Leftrightarrow\quad x \leq$ \ansref.}}
  $\displaystyle\Big(-\frac{x}{\var{b2}}\Big) +\var{b2}\geq 0 \quad\Leftrightarrow\quad x \leq$ \ansref.}}
   \begin{answer}
    \solution{f1}
  \end{answer}
  \begin{answer}
    \solution{b22}
  \end{answer}
\end{quickcheck}


\lang{de}{
Die Gleichungen bzw. Ungleichungen, die wir bisher betrachtet haben, waren \emph{linear}, 
d.\,h. dass $1$ der höchste Exponent der Variablen $x$ ist. Eine Gleichung der Form $ax^2+bx=c$  
mit $a, b, c \in \R$ und $a \neq 0$ ist demnach nicht linear,
da die höchste Potenz der Variablen hier $x^2$ ist. Eine solche Gleichung 
bezeichnet man als \emph{quadratische Gleichung}. Für diese haben wir eine Lösungsformel.
}
\lang{en}{
The equations and inequalities that we have considered so far have
been \emph{linear}; that is, $1$ was the greatest exponent of the
variable $x$. Accordingly, an equation of the form $ax^2+bx=c$ with
$a, b, c \in \R$ and $a \neq 0$ is not linear, as the highest power
in the variable here is $x^2$. Such equations are called
\emph{quadratic equations}.
For these equations, we have the following formula.
}

\begin{rule}[\lang{de}{pq-Formel} \lang{en}{Quadratic formula}] \label{rule:pqFormel}

    \lang{de}{Die Lösungen einer quadratischen Gleichung $\;x^2+px+q=0\;$ sind gegeben durch
    \[x_{1,2}=-\frac{p}{2}{}\pm{}\sqrt{\Big{(}\frac{p}{2}\Big{)}^2-q}.\]
 
    Dabei gibt der Term unter der Wurzel, die sogenannte \emph{\notion{Diskriminante}} 
    \[D:=\Big{(}\frac{p}{2}\Big{)}^2-q, \]   
    Aufschluss über die Anzahl der Lösungen der quadratischen Gleichung. 
    Man unterscheidet die folgenden drei Fälle:}

    \lang{en}{The solutions of the quadratic equation $\;x^2+px+q=0\;$ are given by
    
    \[x_{1,2}=-\frac{p}{2}{}\pm{}\sqrt{\Big{(}\frac{p}{2}\Big{)}^2-q}.\]
    The term in the radical, called the \emph{\notion{discriminant}},
    \[D:=\Big{(}\frac{p}{2}\Big{)}^2-q, \]  
    yields information about the number of solutions of the quadratic equation.
    We distinguish the following three cases:
    }
    
    \begin{align*}
    1. \quad D &\,<\,& 0 \quad \Rightarrow \quad &\mathbb{L}= \emptyset .\\
    2. \quad D &\,=\,& 0 \quad \Rightarrow \quad & \mathbb{L}=\{-\frac{p}{2}\}.\\
    3. \quad D &\,>\,& 0 \quad \Rightarrow \quad &\mathbb{L}=\{-\frac{p}{2}-\sqrt{\Big{(}\frac{p}{2}\Big{)}^2-q}\lang{de}{;}\lang{en}{,} \: -\frac{p}{2}+\sqrt{\Big{(}\frac{p}{2}\Big{)}^2-q}\}. \\
    \end{align*}

\end{rule} 

\begin{proof*}
\lang{de}{Der Beweis der pq-Formel benutzt das Verfahren \emph{quadratische Ergänzung}. Dies wird im Hauptkurs \ref[content_05_loesen_gleichungen_und_lgs][hier]{alg:quadr_erg} ausgeführt.}
\lang{en}{The proof of the quadratic formula uses the method
of \emph{completing the square}. This is explained in detail
\ref[content_05_loesen_gleichungen_und_lgs][here]{alg:quadr_erg}
in the main course.}
\end{proof*}
\lang{de}{Wenn der Koeffizient (der Vorfaktor) von $x^2$ nicht den Wert $1$ hat, müssen wir die Gleichung
zunächst noch durch diese Zahl teilen.}
\lang{en}{If the coefficient of $x^2$ (i.e. the leaading coefficient)
is different from $1$, then we have to divide by this number first.}
\begin{example} \label{ex:pqFormel}
    \lang{de}{Wir berechnen die Lösung der quadratischen Gleichung $\; 2x^2-12x+10=0 \;$ mit Hilfe
    der \emph{pq-Formel}. Da die Gleichung noch nicht in der Form ist, die die pq-Formel 
    verlangt, teilen wir die Gleichung auf beiden Seiten durch $2$ und erhalten die Gleichung}
    \lang{en}{We will solve the quadratic equation $\; 2x^2-12x+10=0 \;$
    by means of the \emph{quadratic formula}. Since this equation is not
    in the form demanded by the quadratic formula, we first divide both
    sides of the equation by $2$ to get the equation}
    \[
    x^2-6x+5 = 0.
    \]
    \lang{de}{Nach der pq-Formel sind die Lösungen }
    \lang{en}{According to the quadratic formula, the solutions are}
    \begin{align*}
    x_{1,2}&\;=&\; 3{}\pm{}\sqrt{9-5} = 3 \pm 2,
    \end{align*}
    \lang{de}{also ist die Lösungsmenge}
    \lang{en}{so the solution set is} $\mathbb{L} =\{1\lang{de}{;}\lang{en}{,}\;5\}.$
\end{example}

\begin{quickcheckcontainer}
\randomquickcheckpool{1}{1}
\begin{quickcheck}
		\field{rational}
		\type{input.number}
		\begin{variables}
			\randint[Z]{a}{-3}{3}
			\randint[Z]{b}{2}{6}
			\randint{c}{-4}{4}
		    \function[normalize]{f}{a*x^2+b*x+c}
			\function[calculate]{p}{b/a}
			\function[calculate]{q}{c/a}
			\function[calculate]{l1}{-p/2}
			\function[calculate]{l2}{(p/2)^2}
			\function[calculate]{l3}{q}	
            \randadjustIf{c}{b^2-4*a*c<0}
		\end{variables}
  		\lang{de}{
			\text{Die pq-Formel für Lösungen der quadratischen Gleichung $\var{f}=0$ lautet\\
			 $x_{1,2}=$\ansref $\pm \sqrt{}($\ansref $-$\ansref $)$.}}

    \lang{en}{
			\text{The quadratic formula for the solutions of the quadratic equation $\var{f}=0$ is \\
			 $x_{1,2}=$\ansref $\pm \sqrt{}($\ansref $-$\ansref $)$.}}
		
		\begin{answer}
			\solution{l1}
		\end{answer}
		\begin{answer}
			\solution{l2}
		\end{answer}
		\begin{answer}
			\solution{l3}
		\end{answer}

	\end{quickcheck}
\end{quickcheckcontainer}


%%%%%%%%%%%%%%%%%%%%%%%%%%%%%%%%%%%%%%%%%%%

\section{\lang{de}{Schreibweisen für Summen und Produkte} \lang{en}{Notation for sums and products}}

\lang{de}{Für Summen mit vielen Summanden ist es unpraktisch, jeden einzelnen Summanden aufzuschreiben. 
Wir brauchen also eine abkürzende Schreibweise für solche Summen. Gleiches gilt für Produkte mit 
vielen Faktoren.}

\lang{en}{For sums with many summands, it is not practical to
write every single summand out. We need a shorthand notation for
sums of this kind. The same is true for products with many factors. 

}


\begin{definition}[\lang{de}{Summen- und Produktzeichen} \lang{en}{Sum and produuct symbols}] \label{def:summenzeichen}
 \lang{de}{F"ur ganzzahlige $m$ und $n$ mit $\, m \leq n$ und beliebige Werte $a_m, a_{m+1}, \ldots, a_n$
           definieren wir die Summe}
    \lang{en}{For integers $m$ and $n$ with $\, m \leq n$ and any values $a_m, a_{m+1}, \ldots, a_n$,
    we define the sum
    }
          
  \[\quad \sum^n_{k=m}{a_k}\,\coloneq\,a_m+a_{m+1}+\ldots+a_{n-1}+a_n \]
  \lang{de}{und das Produkt}
  \lang{en}{and the product}
  \[\quad \prod^n_{k=m}{a_k}\coloneq a_m\cdot a_{m+1}\cdot a_{m+2}\cdot\ldots \cdot a_{n-1}\cdot a_n.\]
  

\lang{de}{Wir nennen $k$ den \emph{\notion{Laufindex}}, $m$ die 
\emph{\notion{untere}} und $n$ die \emph{\notion{obere Grenze}}.}
\lang{en}{We call $k$ the \emph{\notion{index}}, $m$ the \emph{\notion{lower}}
and $n$ the \emph{\notion{upper bound}}.}

\end{definition}

\lang{de}{Das Symbol $\Sigma$ für die Summe ist der griechische Großbuchstabe \emph{Sigma} und das Symbol $\Pi$ für das Produkt ist 
der griechische Großbuchstabe \emph{Pi}. %In der Praxis wird das Summenzeichen häufiger gebraucht als das Produktzeichen. 
}
\lang{en}{The summation symbol $\Sigma$ is the Greek capital letter \emph{Sigma} and the
product symbol $\Pi$ is the Greek capital letter \emph{Pi}.
}

\begin{remark}
  \begin{itemize}[bullet]

\item 
    \lang{de}{Im Spezialfall $\,n=m\,$ gilt: $\quad \sum^m_{k=m}{a_k}=a_m \quad \text{und} \quad \prod^m_{k=m}{a_k}=a_m$.}
    \lang{en}{In the special case that $\,n=m\,$, we have: $\quad \sum^m_{k=m}{a_k}=a_m \quad \text{and} \quad \prod^m_{k=m}{a_k}=a_m$.}


\item 
    \lang{de}{Ist die untere Grenze größer als die obere Grenze, erhalten 
    wir die \notion{leere Summe} bzw. das \notion{leere Produkt}. Der Wert der leeren Summe ist definiert als $0$, 
    der Wert des leeren Produkts ist definiert als $1$.}
    \lang{en}{If the lower bound is greater than the upper bound, the result
    is the \notion{empty sum} or \notion{empty product}, respecively.
    The value of the empty sum is defined to be $0$, and the value of
    the empty product is defined to be $1$.
    }
    
 
  \item
    \lang{de}{Als Laufindex kann, anstelle von $k, \,$ jeder beliebige Buchstabe verwendet werden (der noch nicht 
    anderweitig verwendet wird).
     "Ublicherweise werden Buchstaben aus der Mitte des Alphabets gew"ahlt, wie z.\,B.
     $j,\;k,\;l,\;\ldots\;$ \\
     Entsprechend schreibt man}
     \lang{en}{Any letter (that does not already appear in a different context)
     can be used as an index in place of $k$.
     Generally, one uses letters from the middle of the alphabet,
     such as $j,\;k,\;l,\;\ldots\;$. \\
     Thus, we write}
     
     $\, \sum^n_{k=m}{a_k}=\sum^n_{j=m}{a_j}=\sum^n_{l=m}{a_l}=\ldots$
     

\item 
  \lang{de}{Durch eine \emph{Indexverschiebung} kann man die Darstellung einer Summe (oder eines Produkts) verändern, ohne 
  dass sich der Wert der Summe (bzw. des Produkts) verändert. 
  Hierbei werden die Grenzen um eine feste Zahl $\textcolor{#CC6600}{+z} \; (z \in \Z)\,$ 
  verschoben, während gleichzeitig der Laufindex um $\textcolor{#CC6600}{-z} \,$ verändert wird:}
  \lang{en}{By \emph{shifting the index}, the representation of a sum
  (or product) can be changed without changing its value.
  Here, the bounds are shifted by a fixed number $\textcolor{#CC6600}{+z} \; (z \in \Z)\,$ 
  while the index is simultaneously shifted by $\textcolor{#CC6600}{-z} \,$:}

  \[
  \sum^n_{k=m}{a_k} = \sum^{n \textcolor{#CC6600}{+z}}_{k=m \textcolor{#CC6600}{+z}}{a_{k\textcolor{#CC6600}{-z}}}
  \]
 
  \end{itemize}
  
  
\end{remark}

\begin{example}
    
  \begin{itemize}

    \item $\displaystyle \sum^5_{k=1} k^2 =1^2 + 2^2+ 3^2 +4^2 +5^2 = 55$

    \item $\displaystyle \sum^20_{j=18}{s}=s+s+s = 3s$ \\
          \lang{de}{Man beachte, dass hier $a_j=s\,$ unabhängig von $j$ 
          und somit konstant vom Wert $s$ ist. }
          \lang{en}{Notice here that $a_j=s\,$ is independent of $j$
          and therefore has the constant value $s$.}

    \item  $\displaystyle \sum^5_{k=1}{\frac{(-1)^{k+1}}{k}}=\frac{1}{1}-\frac{1}{2}+\frac{1}{3}-\frac{1}{4}+\frac{1}{5} = \frac{47}{60}$


    \item $\displaystyle \prod^5_{k=1}{k^2}=1^2\cdot 2^2\cdot 3^2 \cdot 4^2\cdot 5^2 \; = 14400$   \\\\    

    \item $\displaystyle \prod^20_{j=18}{p}=p\cdot p\cdot p=p^3 $ \\     
    \lang{de}{Man beachte, dass hier $a_j=p\,$ unabhängig von $j$ 
          und somit konstant vom Wert $p$ ist. }
    \lang{en}{Notice here that $a_j=p\,$ is independent of $j$
    and therefore has the constant value $p$.}

    \item  $\displaystyle \prod^5_{k=1}{\frac{(-1)^{k+1}}{k}}=\frac{1}{1}\cdot \frac{(-1)}{2}\cdot \frac{1}{3}\cdot \frac{(-1)}{4}\cdot \frac{1}{5} = \frac{1}{120}$ \\

    \item    
 $\displaystyle \sum^4_{k=0}{2^{k}} =\sum^{4\textcolor{#CC6600}{+1}}_{k=0\textcolor{#CC6600}{+1}}{2^{k\textcolor{#CC6600}{-1}}} 
 = \sum^5_{k=1}{2^{k-1}} = 31$   

\end{itemize}


\end{example}

\lang{de}{Eine weitere Kurznotation für Produkte ist die \emph{Fakultät}. Mit ihrer Hilfe kann 
zugleich der \emph{Binomialkoeffizient} definiert werden. }
\lang{en}{Another shorthand notation for products is the \emph{factorial}.
Moreover, we can use the factorial to define the \emph{binomial coefficients}.}

\begin{definition}
\begin{itemize}
\item \lang{de}{Es sei $n \in \N$. Wir schreiben kurz $n!$ für das Produkt der Zahlen
von $1$ bis $n$, also
\[
n! := \prod_{k=1}^n k = 1 \cdot 2 \cdots n
\]
und definieren außerdem $0! := 1$. Ausgesprochen wird $n!$ als \glqq n \emph{\notion{Fakultät}}\grqq.}
\lang{en}{Let $n \in \N.$ We write $n!$ for the product of the numbers from $1$ to $n$, i.e.
\[
n! := \prod_{k=1}^n k = 1 \cdot 2 \cdots n
\]
and we define $0! := 1$. $n!$ is pronounced "n factorial".
}
\item \lang{de}{Es seien $n, k \in \N_0$ und $n \geq k$. Dann ist der \emph{\notion{Binomialkoeffizent}} definiert durch 
\[ 
\binom{n}{k} := \frac{n!}{k! \cdot (n-k)!}
\]
und wird \glqq n über k\grqq ausgesprochen.}
\lang{en}{Let $n, k \in \N_0$ with $n \geq k$. The \emph{\notion{binomial coefficient}} is defined by
\[ 
\binom{n}{k} := \frac{n!}{k! \cdot (n-k)!}
\]
and is pronounced "n choose k".
}
\end{itemize}
\end{definition}

\lang{de}{Alternativ kann die Fakultät auch durch die Regeln $0! = 1$ und $n! = (n-1)!$ rekursiv definiert werden.

Fakultät und Binomialkoeffizient spielen eine große Rolle in der Kombinatorik. Wir benötigen sie im 
Rahmen dieses Kurses jedoch nur als Kurzschreibweise für bestimmte Zahlenwerte. So können wir 
nun beispielsweise die binomischen Formeln verallgemeinern zum binomischen Lehrsatz:}

\lang{en}{Alternatively, the factorial can also be defined recursively
via the rules $0! = 1$ and $n! = n \cdot (n-1)!$.

The factorial and binomial coefficients play a major role in combinatorics.
In this course, we will only need them as a shorthand notation for certain numbers.
For example, we can use them to generalize the binomial formulas
to the binomial theorem:
}

\begin{theorem}[\lang{de}{Binomischer Lehrsatz} \lang{en}{Binomial theorem}]\label{thm:binom}
\lang{de}{F"ur alle  $n\in\Nzero$ und alle reellen Zahlen $a,\, b$ gilt die
folgende Gleichung:}
\lang{en}{For all $n\in\Nzero$ and all real numbers $a,\, b$, the following
equation holds:}

\[
\quad(a+b)^n=\sum^n_{k=0}\,\binom{n}{k}\; a^{n-k}\,b^k.
\]
\lang{de}{Hierbei werden $a^0$, $b^0$ und $(a+b)^0$ stets gleich $1$ gesetzt, selbst wenn $a$, $b$ oder $a+b$ gleich $0$ sind.}
\lang{en}{Here, we define $a^0$, $b^0$ and $(a+b)^0$ to equal $1$,
even when $a$, $b$ or $a+b$ is $0$.}
\end{theorem}

\begin{quickcheck}
		\type{input.number}
		\begin{variables}
			\randint[Z]{a}{-1}{5}
			\randint[Z]{b}{-2}{2}
			\randint[Z]{j0}{1}{3}
			\randint{i}{3}{5}
			\function[calculate]{j1}{j0+i}
			\function[calculate]{jj}{j0+1}
		    \function{w1}{a*j0^b}
		    \function{w2}{a*jj^b}
		    \function{w3}{a*j1^b}
		\end{variables}

      \lang{de}{
			\text{Wofür steht die Summe $\displaystyle \sum^{\var{j1}}_{j=\var{j0}} {\var{a}\cdot j^{\var{b}}}$?\\
				  \ansref $+$\ansref$+\ldots +$\ansref}}

      \lang{en}{
			\text{What sum does the notation $\displaystyle \sum^{\var{j1}}_{j=\var{j0}} {\var{a}\cdot j^{\var{b}}}$ represent?\\
				  \ansref $+$\ansref$+\ldots +$\ansref}}
      
			
		\begin{answer}
			\solution{w1}
		\end{answer}
			
		\begin{answer}
			\solution{w2}
		\end{answer}
			
		\begin{answer}
			\solution{w3}
		\end{answer}
	\end{quickcheck}


  
\end{visualizationwrapper}

\end{content}