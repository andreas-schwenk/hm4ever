\documentclass{mumie.element.exercise}
%$Id$
\begin{metainfo}
  \name{
    \lang{de}{Ü02: Konvergenz von Folgen}
    \lang{en}{Ü02: Convergence of sequences}
  }
  \begin{description} 
 This work is licensed under the Creative Commons License Attribution 4.0 International (CC-BY 4.0)   
 https://creativecommons.org/licenses/by/4.0/legalcode 

    \lang{de}{Hier die Beschreibung}
    \lang{en}{}
  \end{description}
  \begin{components}
  \end{components}
  \begin{links}
\link{generic_article}{content/rwth/HM1/T206_Folgen_II/g_art_content_19_bestimmte_divergenz.meta.xml}{content_19_bestimmte_divergenz}
\link{generic_article}{content/rwth/HM1/T205_Konvergenz_von_Folgen/g_art_content_14_konvergenz.meta.xml}{content_14_konvergenz}
\end{links}
  \creategeneric
\end{metainfo}
\begin{content}
\title{\lang{de}{Ü02: Konvergenz von Folgen}
    \lang{en}{Exercise 2: Convergence of sequences}}

\begin{block}[annotation]
	Im Ticket-System: \href{https://team.mumie.net/issues/23799}{Ticket 23799}
\end{block}

\begin{block}[annotation]
Kopie: hm4mint/T205_Konvergenz_von_Folgen/exerciae 4

Im Ticket-System: \href{http://team.mumie.net/issues/9856}{Ticket 9856}
\end{block}
 
% Erweitere BA4 durch Teil d und e: Merkregel exp>polyn (Satz und Bemerkung/Merkregel in VL ergänzen)
 
\lang{de}{\begin{enumerate}[(a)]
 \item a) Untersuchen Sie die Folge $(a_{n})_{n\in\N}$ mit $a_{n}=\frac{n^{2}+3}{(n+2)^{2}}$ auf Konvergenz und bestimmen Sie gegebenenfalls den Grenzwert.
 \item b) Untersuchen Sie die Folge $(b_{n})_{n\in\N}$ mit $b_{n}=\frac{n+\frac{1}{n}}{n^{3}+3n+2}$ auf Konvergenz und bestimmen Sie gegebenenfalls den Grenzwert.
 \item c) Untersuchen Sie die Folge $(c_{n})_{n\in\N}$ mit $c_{n}=\frac{n^{3}+n^{2}+2n+1}{n^{2}+4n+5}$ auf Konvergenz und bestimmen Sie gegebenenfalls den Grenzwert.
 \item d) Untersuchen Sie die Folge $(d_{n})_{n\in\N}$ mit $d_{n}=\frac{n^{3}+2^nn}{3^n}$ auf Konvergenz und bestimmen Sie gegebenenfalls den Grenzwert.
 \item e) Untersuchen Sie die Folge $(e_{n})_{n\in\N}$ mit $e_{n}=\frac{1-2^n}{n^4+2}$ auf Konvergenz und bestimmen Sie gegebenenfalls den Grenzwert.
\end{enumerate}}

\lang{en}{\begin{enumerate}[(a)]
 \item a) Decide whether the sequence $(a_{n})_{n\in\N}$ with $a_{n}=\frac{n^{2}+3}{(n+2)^{2}}$ converges. If it converges, find its limit.
 \item b) Decide whether the sequence $(b_{n})_{n\in\N}$ with $b_{n}=\frac{n+\frac{1}{n}}{n^{3}+3n+2}$ converges. If it converges, find its limit.
 \item c) Decide whether the sequence $(c_{n})_{n\in\N}$ with $c_{n}=\frac{n^{3}+n^{2}+2n+1}{n^{2}+4n+5}$ converges. If it converges, find its limit.
 \item d) Decide whether the sequence $(d_{n})_{n\in\N}$ with $d_{n}=\frac{n^{3}+2^nn}{3^n}$ converges. If it converges, find its limit.
 \item e) Decide whether the sequence $(e_{n})_{n\in\N}$ with $e_{n}=\frac{1-2^n}{n^4+2}$ converges. If it converges, find its limit.
\end{enumerate}}

\begin{tabs*}[\initialtab{0}\class{exercise}]
  \tab{
  \lang{de}{Antworten}
  \lang{en}{Answers}
  }


\begin{table}[\class{items}]
a) \lang{de}{$a_n$ konvergiert gegen den Grenzwert $1$.} \lang{en}{$a_n$ converges with limit $1$.}\\
b) \lang{de}{$b_n$ konvergiert gegen den Grenzwert $0$.} \lang{en}{$b_n$ converges with limit $0$.}\\
c) \lang{de}{$c_n$ divergiert.} \lang{en}{$c_n$ diverges.}\\
d) \lang{de}{$d_n$ konvergiert gegen den Grenzwert $0$.} \lang{en}{$d_n$ converges with limit $1$.}\\
e) \lang{de}{$e_n$ divergiert.} \lang{en}{$e_n$ diverges.} 
\end{table}
  \tab{
  \lang{de}{Lösung a}
  \lang{en}{Solution a}}
  
  \begin{incremental}[\initialsteps{1}]
    \step 
    \lang{de}{Durch Ausklammern der Potenz $n^{2}$ in Zähler und Nenner und anschließendes Kürzen erhalten wir für jedes $n\in\N$}
    \lang{en}{By factoring out $n^2$ from the numerator and the denominator and simplifying the result, we find, for every $n\in\N$,}
	\[a_{n}=\frac{n^{2}+3}{n^{2}+4n+4}=\frac{n^{2}}{n^{2}}\cdot \frac{1+\frac{3}{n^2}}{1+\frac{4}{n}+\frac{4}{n^{2}}}=\frac{1+\frac{3}{n^2}}{1+\frac{4}{n}+\frac{4}{n^{2}}}\,.\]
     
    \step \lang{de}{Nach den Grenzwertsätzen ist diese Folge also konvergent und der Grenzwert ergibt sich als Quotient der Grenzwerte der Terme in Zähler und Nenner. Wir erhalten also 
	\[\lim_{n\to\infty}{a_{n}}=\frac{\lim_{n\to\infty}{(1+\frac{3}{n^2}})}{\lim_{n\to\infty}{(1+\frac{4}{n}+\frac{4}{n^{2}}})}=\frac{1}{1}=1\,.\]}
 \lang{en}{The limit theorems imply that this sequence converges and that the limit can be computed as the quotient of the limits of the terms in the numerator and denominator. We obtain the limit 
	\[\lim_{n\to\infty}{a_{n}}=\frac{\lim_{n\to\infty}{(1+\frac{3}{n^2}})}{\lim_{n\to\infty}{(1+\frac{4}{n}+\frac{4}{n^{2}}})}=\frac{1}{1}=1\,.\]}
  \end{incremental}

  \tab{
  \lang{de}{Lösung b}
  \lang{en}{Solution b}
  }
  \begin{incremental}[\initialsteps{1}]
     \step \lang{de}{Durch Ausklammern der Potenz $n^{3}$ in Zähler und Nenner und anschließendes Kürzen erhalten wir für jedes $n\in\N$}
     \lang{en}{By factoring out $n^3$ from the numerator and the denominator and simplifying the result, we find, for every $n\in\N$,}
	\[b_{n}=\frac{n^{3}}{n^{3}}\cdot \frac{\frac{1}{n^{2}}+\frac{1}{n^{4}}}{1+\frac{3}{n^{2}}+\frac{2}{n^{3}}}=\frac{\frac{1}{n^{2}}+\frac{1}{n^{4}}}{1+\frac{3}{n^{2}}+\frac{2}{n^{3}}}\,.\]
     \step \lang{de}{Nach den Grenzwertsätzen ist diese Folge also konvergent und der Grenzwert ergibt sich als Quotient der Grenzwerte der Terme in Zähler und Nenner. Wir erhalten also}
\lang{en}{The limit theorems imply that this sequence converges and that the limit can be computed as the quotient of the limits of the terms in the numerator and denominator. We obtain the limit}
 \[\lim_{n\to\infty}{b_{n}}=\frac{\lim_{n\to\infty}{(\frac{1}{n^{2}}+\frac{1}{n^{4}})}}{\lim_{n\to\infty}{(1+\frac{3}{n^{2}}+\frac{2}{n^{3}})}}=\frac{0}{1}=0\,.\]
    
  \end{incremental}

  \tab{
  \lang{de}{Lösung c}
  \lang{en}{Solution c}
  }
  
   \lang{de}{Die höchste Potenz von $n$ im Zählerterm ist $n^{3}$, während die höchste im Nennerterm auftretende Potenz $n^{2}$ ist. 
   Also ist die Folge divergent.}
   \lang{en}{The largest power of $n$ in the numerator is $n^3$, while the largest power in the denominator is $n^2$.
   Therefore, the sequence diverges.}
    
  \tab{
  \lang{de}{Lösung d}
  \lang{en}{Solution d}
  }
   \lang{de}{Wir spalten den Bruch auf:}
   \lang{en}{We split up the fraction:}
   \[\frac{n^2+2^nn}{3^n}=n^2\cdot(\frac{1}{3})^n+(\frac{2}{3})^n\cdot n\xrightarrow{n\to\infty} 0 + 0 =0,\]
   \lang{de}{da sich die exponentielle Abnahme gegenüber dem Wachstum von $n^2$ und $n$ durchsetzt.
   }
   \lang{en}{since the exponential decay outweighs the growth of $n^2$ and $n$.}
    
  
   \tab{
  \lang{de}{Lösung e}
  \lang{en}{Solution e}
  }
    \begin{incremental}[\initialsteps{1}]
   \step\lang{de}{Der Zähler strebt gegen $-\infty$, das polynomiale Wachstum des Nenners kann das nicht bremsen, so dass die Folge $e_n$ divergiert.
   (Exponentielles Wachstum setzt sich durch.)
   }
   \lang{en}{The numerator tends toward $-\infty$, and the polynomial growth of the denominator cannot stop it. The sequence $e_n$ diverges.
   (Exponential growth dominates.)
   }
   \end{incremental}

    \tab{\lang{de}{Videos: ähnliche Übungsaufgaben} \lang{en}{Videos: similar exercises}}	
    \youtubevideo[500][300]{f8KXBEqj1pg}\\\\
    \youtubevideo[500][300]{Om-E0R3AV2M}\\


\end{tabs*}


\end{content}