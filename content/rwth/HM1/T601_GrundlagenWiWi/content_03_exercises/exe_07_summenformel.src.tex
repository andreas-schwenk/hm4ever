\documentclass{mumie.element.exercise}
%$Id$
\begin{metainfo}
  \name{
    \lang{en}{Ü07: Gauss' summation formula}
    \lang{de}{Ü07: Gauß'sche Summenformel}
    \lang{zh}{...}
    \lang{fr}{...}
  }
  \begin{description} 
 This work is licensed under the Creative Commons License Attribution 4.0 International (CC-BY 4.0)   
 https://creativecommons.org/licenses/by/4.0/legalcode 

    \lang{en}{...}
    \lang{de}{...}
    \lang{zh}{...}
    \lang{fr}{...}
  \end{description}
  \begin{components}
  \end{components}
  \begin{links}
  \end{links}
  \creategeneric
\end{metainfo}
\begin{content}
\title{\lang{en}{Exercise 7: Gauss' summation formula}
    \lang{de}{Ü07: Gauß'sche Summenformel}}
\begin{block}[annotation]
	Im Ticket-System: \href{https://team.mumie.net/issues/23823}{Ticket 23823}
\end{block}

\lang{de}{a) Berechnen Sie die Summe der ersten hundert natürlichen Zahlen.

b) Berechnen Sie die Summe der ersten hundert geraden natürlichen Zahlen.}

\lang{en}{a) Calculate the sum of the first $100$ natural numbers.

b) Calculate the sum of the first $100$ even natural numbers.}

  \begin{tabs*}[\initialtab{0}\class{exercise}]
    \tab{
      \lang{en}{Answer}
      \lang{de}{Antwort}
      \lang{zh}{...}
      \lang{fr}{...}
    }
    \begin{incremental}[\initialsteps{1}]
      \step
        \lang{en}{a) $ \ \sum_{i=1}^{100}i=5050$
        
        b) $ \ \sum_{i=1}^{100}2i=2\cdot\sum_{i=1}^{100}i=2\cdot 5050=10100 $
        }
        \lang{de}{a) $ \ \sum_{i=1}^{100}i=5050$
        
        b) $ \ \sum_{i=1}^{100}2i=2\cdot\sum_{i=1}^{100}i=2\cdot 5050=10100 $
        }
        \lang{zh}{...}
        \lang{fr}{...}
      
    \end{incremental}
    \tab{
      \lang{en}{Solution a)}
      \lang{de}{Lösung a)}
      \lang{zh}{...}
      \lang{fr}{...}
    }
    \begin{incremental}[\initialsteps{1}]
      \step
        \lang{en}{
        As a student, Carl Friedrich Gauss was asked to find the sum of the first $100$ numbers.
        He said: $1+100=101$ and $2+99=101$ and so on; there are 50 such pairs.
        Therefore the result must be $\sum_{i=1}^{100} i = 50 \cdot 101 = 5050$.

        Generally, $\sum_{i=1}^n i = \frac{n}{2}(n+1) = \frac{n(n+1)}{2}.$ This is
        known as \notion{Gauss' summation formula}.
        }
        \lang{de}{
        Carl Friedrich Gauss sollte als Schüler die ersten hundert Zahlen addieren.
        
        Er sagte: $1+100=101$ und $2+99=101$ usw.; davon gibt es 50 Pärchen. Also ist das
        Ergebnis: $\sum_{i=1}^{100}i=50\cdot 101=5050$.
        
        Es gilt: $\sum_{i=1}^{n}i=\frac{n}{2}(n+1)=\frac{n(n+1)}{2}$.
        
        Dies nennt man auch die \notion{Gauss'sche Summenformel.}
        
        
        
        }
        \lang{zh}{...}
        \lang{fr}{...}
      \step
        \lang{en}{This formula can be proved rigorously with the principle of \notion{complete induction}.}
        \lang{de}{Diese Formel kann allgemeingültig mit Hilfe der \notion{vollständigen Induktion} 
        bewiesen werden.
        }
        \lang{zh}{...}
        \lang{fr}{...}
    \end{incremental}
    \tab{
      \lang{en}{Solution b)}
      \lang{de}{Lösung b)}
      \lang{zh}{...}
      \lang{fr}{...}
    }
    \begin{incremental}[\initialsteps{1}]
      \step
        \lang{en}{
        Using $\sum_{i=1}^{n}i=\frac{n(n+1)}{2}$, we have

        $\sum_{i=1}^{n}c\cdot i=c\cdot\sum_{i=1}^{n}i=c\cdot \frac{n(n+1)}{2}$. Here, we took
        out a factor of $c$ from each summand. Therefore,
        
        $\sum_{i=1}^{100}2i=2\cdot\sum_{i=1}^{100}i=2\cdot 5050=10100 .$
        
       
        }
        \lang{de}{
        Mit  $\sum_{i=1}^{n}i=\frac{n(n+1)}{2}$ gilt auch
        
        $\sum_{i=1}^{n}c\cdot i=c\cdot\sum_{i=1}^{n}i=c\cdot \frac{n(n+1)}{2}$. Der Faktor c
        wurde aus allen Summanden ausgeklammert. Also
        
        $\sum_{i=1}^{100}2i=2\cdot\sum_{i=1}^{100}i=2\cdot 5050=10100 .$
        
       
        
        }
        \lang{zh}{...}
        \lang{fr}{...}
      
    \end{incremental} 
  \end{tabs*}


\end{content}

