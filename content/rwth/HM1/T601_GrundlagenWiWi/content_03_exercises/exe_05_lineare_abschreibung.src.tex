\documentclass{mumie.element.exercise}
%$Id$
\begin{metainfo}
  \name{
    \lang{en}{Ü05: Straight-line depreciation}
    \lang{de}{Ü05: Lineare Abschreibung}
    \lang{zh}{...}
    \lang{fr}{...}
  }
  \begin{description} 
 This work is licensed under the Creative Commons License Attribution 4.0 International (CC-BY 4.0)   
 https://creativecommons.org/licenses/by/4.0/legalcode 

    \lang{en}{...}
    \lang{de}{...}
    \lang{zh}{...}
    \lang{fr}{...}
  \end{description}
  \begin{components}
  \end{components}
  \begin{links}
  \end{links}
  \creategeneric
\end{metainfo}
\begin{content}
\title{\lang{en}{Exercise 5: Straight-line depreciation}
    \lang{de}{Ü05: Lineare Abschreibung}}
\begin{block}[annotation]
	Im Ticket-System: \href{https://team.mumie.net/issues/24038}{Ticket 24038}
\end{block}



\lang{de}{Als Anwendung einer \notion{arithmetischen Folge} berechnen wir hier die Folge der Restbuchwerte 
bei der \notion{linearen Abschreibung}.

Bei der Einkommensteuererklärung kann ein PC als Arbeitsmaterial im Rahmen der Werbungskosten
abgesetzt werden. Erstellen Sie einen Plan für eine lineare Abschreibung.

Der Neuwert des PCs beträgt $K_0=1599,00$ \euro. Nach einer Nutzungsdauer von 
$m=3$  Jahren beträgt der Restwert $K_m=0$ \euro. }

\lang{en}{As an application of \notion{arithmetic sequences}, we will calculate the book value of a depreciable asset using \notion{straight-line depreciation}.

A PC can be declared as a business-related tax deductible expense. Set up a plan for its straight-line depreciation.

The purchase price of the PC is $K_0 = 1599.00$ \euro. After $m=3$ years, the remaining value is $K_m=0$ \euro.
}

  \begin{tabs*}[\initialtab{0}\class{exercise}]
    \tab{
      \lang{en}{Solution}
      \lang{de}{Lösung}
      \lang{zh}{...}
      \lang{fr}{...}
    }
    \begin{incremental}[\initialsteps{1}]
      \step
        \lang{en}{\notion{Book value at year-end:}\\
        $K_0=1599$\\
        $K_1=1066$\\
        $K_2=533$\\
        $K_3=0$
        
        
        \textbf{Depreciation schedule}
  \begin{table}[\cellaligns{crrcr}]
    \head[\cellaligns{lllll}]
    n & $K_n$ & A & $p'_n$ & $p_n$
    \body
    0 & 1599 &  &  & \\
    1 & 1066 & 533 & 33,3  & 33,33\\
    2 & 533 & 533 & 33,3 & 50\\
    3 & 0   &  533 & 33,3 & 100
    
  \end{table}
  
        }
        \lang{de}{
        \notion{Folge der Restbuchwerte:}\\
        $K_0=1599$\\
        $K_1=1066$\\
        $K_2=533$\\
        $K_3=0$
        
        
        \textbf{Abschreibungsplan}
  \begin{table}[\cellaligns{crrcr}]
    \head[\cellaligns{lllll}]
    n & $K_n$ & A & $p'_n$ & $p_n$
    \body
    0 & 1599 &  &  & \\
    1 & 1066 & 533 & 33,3  & 33,33\\
    2 & 533 & 533 & 33,3 & 50\\
    3 & 0   &  533 & 33,3 & 100
    
  \end{table}
        
        }
        \lang{zh}{...}
        \lang{fr}{...}
      
    \end{incremental}
    \tab{
      \lang{en}{Explanation}
      \lang{de}{Erklärung}
      \lang{zh}{...}
      \lang{fr}{...}
    }
    \begin{incremental}[\initialsteps{1}]
      \step
        \lang{en}{Depreciation describes the loss of value of a tangible asset
        over its useful life span. In practice, one of two variants is usually used to compute it:
        straight-line depreciation or reducing-balance depreciation.

        In straight-line depreciation, the \textit{accumulated depreciation = original cost - book value} is
        distributed evenly throughout the life of the asset:
        \begin{table}
        $K_n$ & Book value after n years\\
        $A_n=K_{n-1}-K_n=d$ &  Yearly depreciation expense \\
        $p_n=\frac{A_n}{K_{n-1}}\cdot 100$ & Depreciation rate in the n-th year\\
        $p'_n=\frac{A_n}{K_0}\cdot 100$ & Accumulated depreciation rate
        \end{table}
        }
        \lang{de}{Die Abschreibung beschreibt den Wertverlust 
        von Gebrauchsgegenständen im Verlauf ihrer (wirtschaftlichen) Nutzungsdauer. Zur Anwendung
        kommen zumeist zwei Varianten: die lineare Abschreibung oder die arithmetisch-degressive
        Abschreibung.
        
       Bei der linearen Abschreibung wird der \textit{Wertverlust = Anschaffungswert - Restwert} auf die gesamte 
        Nutzungsdauer gleichmäßig verteilt: 
        \begin{table}
        $K_n$ & Restbuchwert nach n Jahren\\
        $A_n=K_{n-1}-K_n=d$ & jährlicher Abschreibungsbetrag \\
        $p_n=\frac{A_n}{K_{n-1}}\cdot 100$ & Abschreibungsprozentsatz in der n-ten Periode vom Restwert der vorhergehenden Periode\\
        $p'_n=\frac{A_n}{K_0}\cdot 100$ & Abschreibungsprozentsatz vom Neuwert
        \end{table}
 
        
        }
        \lang{zh}{...}
        \lang{fr}{...}
      
    \end{incremental}
    \tab{
      \lang{en}{Calculation}
      \lang{de}{Rechnung}
      \lang{zh}{...}
      \lang{fr}{...}
    }
    \begin{incremental}[\initialsteps{1}]
      \step
        \lang{en}{ $K_0=1599$,\\
        $m=3$,\\
        $A_n=d=\frac{1599}{3}=533$,
        \begin{eqnarray*}
        p_1&=&\frac{533}{1599}\cdot 100=33.33\%,\\
        p_2&=&\frac{533}{1066}\cdot 100=50\%,\\
        p_3&=&\frac{533}{533}\cdot 100=100\%,\\
        p'_n&=&\frac{533}{1599}\cdot 100=33.33\%.
        \end{eqnarray*}}
        \lang{de}{
        $K_0=1599$,\\
        $m=3$,\\
        $A_n=d=\frac{1599}{3}=533$,
        \begin{eqnarray*}
        p_1&=&\frac{533}{1599}\cdot 100=33,33\%,\\
        p_2&=&\frac{533}{1066}\cdot 100=50\%,\\
        p_3&=&\frac{533}{533}\cdot 100=100\%,\\
        p'_n&=&\frac{533}{1599}\cdot 100=33,33\%.
        \end{eqnarray*}
        
        }
        \lang{zh}{...}
        \lang{fr}{...}
     
    \end{incremental}
  \end{tabs*}



\end{content}

