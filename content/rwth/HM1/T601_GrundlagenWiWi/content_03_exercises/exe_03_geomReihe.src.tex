\documentclass{mumie.element.exercise}
%$Id$
\begin{metainfo}
  \name{
    \lang{de}{Ü03: geometrische Reihe}
    \lang{en}{Ü03: geometric series}
  }
  \begin{description} 
 This work is licensed under the Creative Commons License Attribution 4.0 International (CC-BY 4.0)   
 https://creativecommons.org/licenses/by/4.0/legalcode 

    \lang{de}{}
    \lang{en}{}
  \end{description}
  \begin{components}
  \end{components}
  \begin{links}
\link{generic_article}{content/rwth/HM1/T208_Reihen/g_art_content_24_reihen_und_konvergenz.meta.xml}{content_24_reihen_und_konvergenz}
\end{links}
  \creategeneric
\end{metainfo}
\begin{content}
\title{\lang{de}{Ü03: geometrische Reihe}
    \lang{en}{Exercise 3: geometric series}}
\begin{block}[annotation]
	Im Ticket-System: \href{https://team.mumie.net/issues/23810}{Ticket 23810}
\end{block}

 \begin{block}[annotation]
 Kopie: hm4mint/T208_Reihen/exercise 2
 
Im Ticket-System: \href{http://team.mumie.net/issues/9868}{Ticket 9868}
\end{block}
 
\lang{de}{Berechnen Sie jeweils den Wert der folgenden Reihen:}
\lang{en}{Calculate the following series:}
\begin{table}[\class{items}]
\nowrap{a) $\displaystyle \ \sum_{k=0}^{\infty}{\left(\frac{3}{5}\right)^{k}}$} \\
\nowrap{b) $\displaystyle \ \sum_{k=0}^{\infty}{\left(-\frac{1}{2}\right)^{k}}$} \\
\nowrap{c) $\displaystyle \ \sum_{k=0}^{\infty}{\left[\left(\frac{1}{3}\right)^{k}-\left(\frac{1}{2}\right)^{k+1}\right]}$}
\end{table}


\begin{tabs*}[\initialtab{0}\class{exercise}]

\tab{\lang{de}{    Antworten    } \lang{en}{Answers}}
    \begin{enumerate}[a)]
\item a) $ \ \sum_{k=0}^{\infty}{\left(\frac{3}{5}\right)^{k}}=\frac{5}{2}\,.$
\item b) $ \ \sum_{k=0}^{\infty}{\left(-\frac{1}{2}\right)^{k}}=\frac{1}{1-q}=\frac{1}{1+\frac{1}{2}}=\frac{2}{3}\,.$
\item c) $ \ \sum_{k=0}^{\infty}{\left[\left(\frac{1}{3}\right)^{k}-\left(\frac{1}{2}\right)^{k+1}\right]}=\frac{1}{2}\,.$
\end{enumerate} 


  \tab{
  \lang{de}{Lösung a)} \lang{en}{Solution a}
  }
\lang{de}{Die Reihe ist eine geometrische Reihe mit $q=3/5$. Somit können wir den Reihenwert berechnen mit folgender Formel:}
\lang{en}{This series is a geometric series with $q=3/5$. We can calculate the series using the formula}
\[\sum_{k=0}^{\infty}{\left(\frac{3}{5}\right)^{k}}=\frac{1}{1-q}=\frac{1}{1-\frac{3}{5}}=\frac{5}{2}\,.\]

\tab{
\lang{de}{Lösung b)} \lang{en}{Solution b}}
\lang{de}{Die Reihe ist eine geometrische Reihe mit $q=-1/2$. Somit können wir den Reihenwert berechnen mit folgender Formel:}

\lang{en}{This series is a geometric series with $q=-1/2$. We can calculate the series using the formula}
\[\sum_{k=0}^{\infty}{\left(-\frac{1}{2}\right)^{k}}=\frac{1}{1-q}=\frac{1}{1+\frac{1}{2}}=\frac{2}{3}\,.\]

\tab{
\lang{de}{Lösung c)} \lang{en}{Solution c}}


  \begin{incremental}[\initialsteps{1}]
     \step \lang{de}{Die Reihen $\sum_{k=0}^{\infty}{\left(\frac{1}{3}\right)^{k}}$ und $\sum_{k=0}^{\infty}{\left(\frac{1}{2}\right)^{k}}$ konvergieren absolut als geometrische Reihen.}
     \lang{en}{The series $\sum_{k=0}^{\infty}{\left(\frac{1}{3}\right)^{k}}$ and $\sum_{k=0}^{\infty}{\left(\frac{1}{2}\right)^{k}}$ are absolutely convergent geometric series.}
     \step \lang{de}{Nach den Grenzwertregeln konvergiert somit auch die Reihe}
     \lang{en}{By the limit theorems, the series}
\[\sum_{k=0}^{\infty}{\left(\frac{1}{3}\right)^{k}}-\frac{1}{2}\sum_{k=0}^{\infty}{\left(\frac{1}{2}\right)^{k}}=\sum_{k=0}^{\infty}{\left[\left(\frac{1}{3}\right)^{k}-\left(\frac{1}{2}\right)^{k+1}\right]}\]
\lang{de}{und hat den Wert} \lang{en}{also converges and has the value}
\[\frac{1}{1-\frac{1}{3}}-\frac{1}{2} \cdot \frac{1}{1-\frac{1}{2}}=\frac{3}{2}-1=\frac{1}{2}\,.\]
  \end{incremental}

    \tab{\lang{de}{Video: ähnliche Übungsaufgabe} \lang{en}{Video: similar exercise}}	
    \youtubevideo[500][300]{eOQ_1K4078w}\\


\end{tabs*}

\end{content}