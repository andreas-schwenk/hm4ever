\documentclass{mumie.element.exercise}
%$Id$
\begin{metainfo}
  \name{
    \lang{de}{Ü04: Konvergenzkriterien}
    \lang{en}{Ü04: Convergence tests}
  }
  \begin{description} 
 This work is licensed under the Creative Commons License Attribution 4.0 International (CC-BY 4.0)   
 https://creativecommons.org/licenses/by/4.0/legalcode 

    \lang{de}{}
    \lang{en}{}
  \end{description}
  \begin{components}
  \end{components}
  \begin{links}
\link{generic_article}{content/rwth/HM1/T208_Reihen/g_art_content_25_konvergenz_kriterien.meta.xml}{content_25_konvergenz_kriterien}
\end{links}
  \creategeneric
\end{metainfo}
\begin{content}
\title{\lang{de}{Ü04: Konvergenzkriterien}
    \lang{en}{Exercise 4: Convergence tests}}
\begin{block}[annotation]
	Im Ticket-System: \href{https://team.mumie.net/issues/23821}{Ticket 23821}
\end{block}

 \begin{block}[annotation]
 Kopie: hm4mint/T208_Reihen/exercise 4
 
Im Ticket-System: \href{http://team.mumie.net/issues/9870}{Ticket 9870}
\end{block}
 
\begin{table}[\class{items}]
\nowrap{a) \lang{de}{Zeigen Sie die Konvergenz der Reihe $\sum_{k=1}^{\infty} \frac{1}{k(k+1)(k+2)}$ 
mit Hilfe des Majorantenkriteriums.}
\lang{en}{Show that the series $\sum_{k=1}^{\infty} \frac{1}{k(k+1)(k+2)}$ converges using the comparison test.}} \\
\nowrap{b) \lang{de}{Können Sie die Konvergenz der Reihe $\sum_{k=1}^{\infty} \frac{1}{k^2}$ 
mit Hilfe des Quotientenkriteriums zeigen?} \lang{en}{Can the ratio test be used to show that the series $\sum_{k=1}^{\infty} \frac{1}{k^2}$ converges?}}
\end{table}


\begin{tabs*}[\initialtab{0}\class{exercise}]
  \tab{
  \lang{de}{Lösung a)}
  \lang{en}{Solution a)}
  }
  \begin{incremental}[\initialsteps{1}]
  \step
\lang{de}{Zunächst schätzen wir ab. Es gilt $k+1 > k$ und $k+2>k$. Damit ist
\[
\frac{1}{k+1} < \frac{1}{k} \text{ und } \frac{1}{k+2}<\frac{1}{k}.
\]}

\lang{en}{First we bound the terms from above. We have $k+1 > k$ and $k+2>k$. Therefore,
\[
\frac{1}{k+1} < \frac{1}{k} \text{ and } \frac{1}{k+2}<\frac{1}{k}.
\]}

\step \lang{de}{Hiermit ergibt sich} \lang{en}{This implies}
\[
    0<\frac{1}{k(k+1)(k+2)}<\frac{1}{k^{3}}<\frac{1}{k^{2}} \]
\lang{de}{für alle $k\in\N$.} \lang{en}{for all $k\in\N$.}
\step \lang{de}{Also folgt aus der absoluten Konvergenz der Reihe $\sum_{k=1}^{\infty }\frac{1}{k^{2}}$ (in der Vorlesung gezeigt) 
nach dem Majorantenkriterium die Konvergenz der Reihe}
\lang{en}{By the comparison test, the absolute convergence of the series $\sum_{k=1}^{\infty }\frac{1}{k^{2}}$ (shown in lecture)
yields the convergence of the series}
$\sum_{k=1}^{\infty }\frac{1}{k(k+1)(k+2)}$.
\end{incremental}
\tab{
\lang{de}{Lösung b)}
\lang{en}{Solution b)}}
\lang{de}{Wir berechnen den folgenden Quotienten:}
\lang{en}{We compute the following ratios:}
\[
\frac{\frac{1}{(k+1)^{2}}}{\frac{1}{k^{2}}}=\frac{k^{2}}{(k+1)^{2}}=\frac{%
k^{2}}{k^{2}+2k+1}=\frac{1}{1+2\frac{1}{k}+\frac{1}{k^{2}}}
\]
\lang{de}{Also strebt der Quotient nach den Grenzwerts\"{a}tzen gegen 1 (alle
auftretenden Nenner sind von 0 verschieden). Deshalb kann man mit dem
Quotientenkriterium keine Aussage \"{u}ber die Konvergenz der
Reihe $\sum_{k=1}^{\infty }\frac{1}{k^{2}}$ treffen.}
\lang{en}{By the limit theorems, the ratio tends to 1 (all denominators are nonzero).
Therefore, we cannot use the ratio test to determine whether or not the series
$\sum_{k=1}^{\infty} \frac{1}{k^2}$ converges.}

\lang{de}{\textbf{Bemerkung:} Das bedeutet \textbf{nicht}, dass man überhaupt keine Aussage über die Konvergenz machen kann,
sondern nur, dass das Quotientenkriterium keine Aussage liefert. 
Wir haben in der Vorlesung gesehen, dass diese Reihe konvergiert.}
\lang{en}{\textbf{Remark:} This does \textbf{not} mean that the convergence of the series cannot be determined at all,
only that the ratio test does not determine it.
We saw in the lecture that this series converges.}


\end{tabs*}

\end{content}