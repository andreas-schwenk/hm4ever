\documentclass{mumie.element.exercise}
%$Id$
\begin{metainfo}
  \name{
    \lang{en}{Ü06: Reducing-balance depreciation}
    \lang{de}{Ü06: degressive Abschreibung}
    \lang{zh}{...}
    \lang{fr}{...}
  }
  \begin{description} 
 This work is licensed under the Creative Commons License Attribution 4.0 International (CC-BY 4.0)   
 https://creativecommons.org/licenses/by/4.0/legalcode 

    \lang{en}{...}
    \lang{de}{...}
    \lang{zh}{...}
    \lang{fr}{...}
  \end{description}
  \begin{components}
  \end{components}
  \begin{links}
  \end{links}
  \creategeneric
\end{metainfo}
\begin{content}
\title{\lang{de}{Ü06: degressive Abschreibung}
\lang{en}{Exercise 6: Reducing-balance depreciation}}
\begin{block}[annotation]
	Im Ticket-System: \href{https://team.mumie.net/issues/23655}{Ticket 23655}
\end{block}


\lang{de}{Eine Webmaschine mit Anschaffungskosten von $100000$ \euro soll in 20 Jahren auf den Restwert 
(Schrottwert) von $5000$ \euro (arithmetisch)-degressiv abgeschrieben werden. Der erste Abschreibungsbetrag soll 
$9000$ \euro betragen. 

Bestimmen Sie die Folge der Abschreibungsbeträge und erstellen Sie einen Abschreibungsplan.
}
\lang{en}{A loom with an initial purchase cost of $100000$ \euro must be depreciated to its salvage value of
$5000$ \euro after 20 years by the reducing-balance method. The first depreciation expense will be $9000$ \euro.

Compute the sequence of depreciation expenses and set up a depreciation schedule.

}


  \begin{tabs*}[\initialtab{0}\class{exercise}]
    \tab{
      \lang{en}{Solution}
      \lang{de}{Lösung}
      \lang{zh}{...}
      \lang{fr}{...}
    }
    \begin{incremental}[\initialsteps{1}]
      \step
        \lang{en}{\notion{Sequence of depreciation expenses ($d=-447.37$ \euro):}\\
        $A_1=9000$\\
        $A_2=8552.63$\\
        ...\\
        $A_{19}=947.34$\\
        $A_{20}=499.97$
        
        
        \textbf{Depreciation schedule}
  \begin{table}[\cellaligns{crrcr}]
    \head[\cellaligns{lllll}]
    n & $A_n$ & $K_n$ & $p'_n$ & $p_n$
    \body
    0  &      & 100000.00 &  & \\
    1  & 9000.00 & 91000.00 & 9.00  & 9.00\\
    2  & 8552.63 & 82447.40 & 8.55 &9.40 \\
    ...&...&...&...&...\\
    19 & 947.34   &  5499.97 & 0.96 & 14.83\\
    20 & 499.97   &  5000.00 &0.50 &9.09
  \end{table}}
        \lang{de}{
        
        \notion{Folge der Abschreibungsbeträge ($d=-447,37$ \euro):}\\
        $A_1=9000$\\
        $A_2=8552,63$\\
        ...\\
        $A_{19}=947,34$\\
        $A_{20}=499,97$
        
        
        \textbf{Abschreibungsplan}
  \begin{table}[\cellaligns{crrcr}]
    \head[\cellaligns{lllll}]
    n & $A_n$ & $K_n$ & $p'_n$ & $p_n$
    \body
    0  &      & 100000,00 &  & \\
    1  & 9000,00 & 91000,00 & 9,00  & 9,00\\
    2  & 8552,63 & 82447,40 & 8,55 &9,40 \\
    ...&...&...&...&...\\
    19 & 947,34   &  5499,97 & 0,96 & 14,83\\
    20 & 499,97   &  5000,00 &0,50 &9,09
  \end{table}
        
        
        }
        \lang{zh}{...}
        \lang{fr}{...}
     
    \end{incremental}
    \tab{
      \lang{en}{Explanation}
      \lang{de}{Erklärung}
      \lang{zh}{...}
      \lang{fr}{...}
    }
    \begin{incremental}[\initialsteps{1}]
      \step
        \lang{en}{In the \notion{reducing-balance method}, the depreciation costs $A_n$
        decrease by a constant value $d$ each year. 
        
        In other words, the \textit{depreciation costs} form an \textit{arithmetic sequence}:\\
        \begin{eqnarray}
        A_n&=&A_1+(n-1)d \text{ with }d<0\\
        \end{eqnarray}
        
        The book value of the asset is therefore the sum of an arithmetic series. We use 
        Gauss' summation formula $\sum_{i=1}^ni=\frac{n(n+1)}{2}$ and obtain:
        \begin{eqnarray*}
        K_0-K_n&=&\sum_{i=1}^nA_i\\
        &=&\sum_{i=1}^n\left(A_1+(i-1)d\right)\\
        &=&nA_1+\sum_{i=1}^n\left(id-d\right)\\
        &=&nA_1-nd+\sum_{i=1}^nid\\
        &=&nA_1-\frac{2nd}{2}+d\sum_{i=1}^ni\\
        &=&nA_1-\frac{2nd}{2}+d\frac{n(n+1)}{2}\\
        &=&n\left(A_1 + (n-1)\frac{d}{2}\right).
        \end{eqnarray*}
        In order to reach a specified remaining value $K_m$ after m years, we must have:
        \begin{eqnarray}
        K_0-K_m&=&mA_1+m(m-1)\frac{d}{2}\\
        \iff d&=&2\frac{K_0-K_m-mA_1}{m(m-1)}
        \end{eqnarray}}
        \lang{de}{Bei der \notion{arithmetisch-degressiven Abschreibung} wird festgelegt, dass die
        Abschreibungsbeträge $A_n$ um einen konstanten Wert $d$ (pro Jahr) fallen sollen. 
        
        Jetzt bilden die \textit{Abschreibungsbeträge} eine \textit{arithmetische Folge}:\\
        \begin{eqnarray}
        A_n&=&A_1+(n-1)d \text{ mit }d<0\\
        \end{eqnarray}
        
        Also verhalten sich die (Rest-)Buchwerte wie eine arithmetische Reihe. Wir benutzen die
        Gauss'sche Summenformel $\sum_{i=1}^ni=\frac{n(n+1)}{2}$ und es gilt dann:
        \begin{eqnarray*}
        K_0-K_n&=&\sum_{i=1}^nA_i\\
        &=&\sum_{i=1}^n\left(A_1+(i-1)d\right)\\
        &=&nA_1+\sum_{i=1}^n\left(id-d\right)\\
        &=&nA_1-nd+\sum_{i=1}^nid\\
        &=&nA_1-\frac{2nd}{2}+d\sum_{i=1}^ni\\
        &=&nA_1-\frac{2nd}{2}+d\frac{n(n+1)}{2}\\
        &=&n\left(A_1 + (n-1)\frac{d}{2}\right)
        \end{eqnarray*}
        Damit nach m Jahren ein bestimmter Restwert $K_m$ erreicht wird, muss gelten:
        \begin{eqnarray}
        K_0-K_m&=&mA_1+m(m-1)\frac{d}{2}\\
        \iff d&=&2\frac{K_0-K_m-mA_1}{m(m-1)}
        \end{eqnarray}
        }
        \step
        \lang{en}{Since $A_n$ is a monotonically decreasing sequence,
        \[K_0-K_m<mA_1\iff \frac{K_0-K_m}{m}<A_1.\]
        
        Equation $(1)$ implies: 
        \[(m-1)d=A_m-A_1>-A_1,\]
        because all depreciation costs $A_i$ are positive. Equation $(2)$ becomes:
        \[\frac{2(K_0-K_m)}{m}=2A_1+(m-1)d>A_1.\]
        The initial depreciation cost is therefore bounded by
        \[\frac{K_0-K_m}{m}<A_1<\frac{2(K_0-K_m)}{m}.\]
        The depreciation rate is given by:
        \begin{enumerate}
        \item
        $p_1=\frac{100A_1}{K_0}$,
        \item Depreciation rate for $n>1$:
        $p_n=\frac{A_n}{K_{n-1}}100=\frac{A_1+(n-1)d}{K_0-(n-1)A_1-\frac{(n-2)(n-1)}{2}d}$
        \item
        $p'_n=\frac{A_n}{K_0}100=\frac{A_1+(n-1)d}{K_0}100$
        \end{enumerate}
        }
        \lang{de}{
        Da $A_n$ eine monoton fallende Folge ist, gilt:
        \[K_0-K_m<mA_1\iff \frac{K_0-K_m}{m}<A_1\]
        
        Aus Gleichung $(1)$ folgt: 
        \[(m-1)d=A_m-A_1>-A_1,\]
        da alle Abschreibungsbeträge $A_i>0$. Gleichung$(2)$ wird damit zu:
        \[\frac{2(K_0-K_m)}{m}=2A_1+(m-1)d>A_1\]
        Um Abschätzungen machen zu können ergibt sich also:
        \[\frac{K_0-K_m}{m}<A_1<\frac{2(K_0-K_m)}{m}\]
        
        Für die Abschreibungsprozentsätze ergibt sich:
        \begin{enumerate}
        \item
        $p_1=\frac{100A_1}{K_0}$, Abschreibungsprozentsatz
        \item für $n>1$:
        $p_n=\frac{A_n}{K_{n-1}}100=\frac{A_1+(n-1)d}{K_0-(n-1)A_1-\frac{(n-2)(n-1)}{2}d}$
        \item
        $p'_n=\frac{A_n}{K_0}100=\frac{A_1+(n-1)d}{K_0}100$
        \end{enumerate}
        
        
        }
        \lang{zh}{...}
        \lang{fr}{...}
      
    \end{incremental}
    
    \tab{
      \lang{en}{Calculation}
      \lang{de}{Rechnung}
      \lang{zh}{...}
      \lang{fr}{...}
    }
    \begin{incremental}[\initialsteps{1}]
      \step
        \lang{en}{
        \[d=2\frac{(K_0-K_m)-mA_1}{m(m-1)}=2\frac{95000-20\cdot 9000}{20\cdot19}=-447.37\text { \euro},\] 
        \[\frac{K_0-K_{20}}{20}=4750<9000<9500=\frac{2(K_0-K_{20})}{20}\]
        }
        \lang{de}{
        \[d=2\frac{(K_0-K_m)-mA_1}{m(m-1)}=2\frac{95000-20\cdot 9000}{20\cdot19}=-447,37\text { \euro},\] 
        \[\frac{K_0-K_{20}}{20}=4750<9000<9500=\frac{2(K_0-K_{20})}{20}\]
        
        
        
        }
        \lang{zh}{...}
        \lang{fr}{...}
     
    \end{incremental}
    
  \end{tabs*}



\end{content}

