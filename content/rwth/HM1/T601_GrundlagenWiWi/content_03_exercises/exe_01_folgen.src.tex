\documentclass{mumie.element.exercise}
%$Id$
\begin{metainfo}
  \name{
    \lang{de}{Ü01: Folgen}
    \lang{en}{Ü01: Sequences}
  }
  \begin{description} 
 This work is licensed under the Creative Commons License Attribution 4.0 International (CC-BY 4.0)   
 https://creativecommons.org/licenses/by/4.0/legalcode 

    \lang{de}{Hier die Beschreibung}
    \lang{en}{}
  \end{description}
  \begin{components}
\component{generic_sound}{content/rwth/HM1/T205_Konvergenz_von_Folgen/Audios/g_snd_Beispielaugabe_1_a.meta.xml}{Beispielaugabe_1_a}
\end{components}
  \begin{links}
\link{generic_article}{content/rwth/HM1/T208_Reihen/g_art_content_24_reihen_und_konvergenz.meta.xml}{content_24_reihen_und_konvergenz}
\link{generic_article}{content/rwth/HM1/T205_Konvergenz_von_Folgen/g_art_content_13_reelle_folgen.meta.xml}{content_13_reelle_folgen}
\end{links}
  \creategeneric
\end{metainfo}
\begin{content}
\title{
    \lang{de}{Ü01: Folgen}
    \lang{en}{Exercise 1: Sequences}
  }
\begin{block}[annotation]
	Im Ticket-System: \href{https://team.mumie.net/issues/23814}{Ticket 23814}
\end{block}

\begin{block}[annotation]
Kopie: hm4mint/T205_Konvergenz_von_Folgen/exercise 1

Im Ticket-System: \href{http://team.mumie.net/issues/9853}{Ticket 9853}
\end{block}
 
\lang{de}{\begin{enumerate}[(a)]
 \item a) Gegeben sei die Folge $(a_{n})_{n\in\N}$ mit $a_{n}=n^{2}-n$ für alle $n\in\N$. Untersuchen Sie diese auf Monotonie und Beschränktheit.
 \item b) Gegeben sei die Folge $(a_{n})_{n\geq 1}$ durch  $a_{n}=2\cdot(-\frac{1}{2})^{n}$ für alle $n\in\N$. Untersuchen Sie diese auf Monotonie und Beschränktheit.
\end{enumerate}}

\lang{en}{\begin{enumerate}[(a)]
 \item a) Let $(a_{n})_{n\in\N}$ be the sequence with $a_{n}=n^{2}-n$ for all $n\in\N$. Determine whether this sequence is monotone and/or bounded.
 \item b) Define a sequence $(a_{n})_{n\geq 1}$ by  $a_{n}=2\cdot(-\frac{1}{2})^{n}$ for all $n\in\N$. Determine whether this sequence is monotone and/or bounded.
\end{enumerate}}

\begin{tabs*}[\initialtab{0}\class{exercise}]

  \tab{
  \lang{de}{Lösung a} \lang{en}{Solution a}}
  
  \begin{incremental}[\initialsteps{1}]
    \step 
    \lang{de}{\audio[1,0.5,0.8,1.25,1.5]{Beispielaugabe_1_a}\\Eine reelle Folge ist genau dann monoton steigend, wenn $a_{n+1}-a_{n}\geq 0$ für alle $n\in\N$ gilt. In unserem Fall haben wir also den Ausdruck
	\[a_{n+1}-a_{n}=(n+1)^{2}-(n+1)-(n^{2}-n)=(n+1)^{2}-n^{2}-1\]
	für alle $n\in\N$ zu untersuchen.}
 \lang{en}{A sequence of real numbers is monotone increasing if and only if $a_{n+1}-a_{n}\geq 0$ for all $n\in\N$.
 In our case, we need to consider the expression
	\[a_{n+1}-a_{n}=(n+1)^{2}-(n+1)-(n^{2}-n)=(n+1)^{2}-n^{2}-1\]
 for $n\in\N$.}
     
    \step \lang{de}{Nach der ersten binomischen Formel gilt nun
	\[(n+1)^{2}-n^{2}-1=n^{2}+2n+1-n^{2}-1=2n\geq 0\quad\text{für alle }n\in\N\,.\]
	Also haben wir sogar gezeigt, dass $a_{n+1}>a_{n}$ für alle $n\in\N$ gilt. 
	Daher ist die Folge streng monoton steigend und als solche natürlich auch nach unten beschränkt durch $a_{1}=0$.}
 \lang{en}{By the first binomial formula,
 \[(n+1)^{2}-n^{2}-1=n^{2}+2n+1-n^{2}-1=2n\geq 0\quad\text{for all }n\in\N\,.\]
 We have actually shown that $a_{n+1} > a_n$ for all $n\in\N$.
 The sequence is therefore strictly monotone increasing. As such, it is also bounded from below by $a_{1}=0$.
 }
    \step \lang{de}{Weiter gilt für jedes $n\in\N$ mit $n\geq 2$
	\[a_{n}=n(n-1)\geq n\,.\]
	Dies zeigt, dass für jedes $M >0$ ein $N\in\N$ existiert mit $a_{n}> M$ für alle $n\geq N$. Also ist die Folge nach oben unbeschränkt. }

 \lang{en}{For each $n\in\N$ with $n\geq 2$,
	\[a_{n}=n(n-1)\geq n\,.\]
	This implies that for every $M >0$, there exists an $N\in\N$ such that $a_{n}> M$ for all $n\geq N$. The sequence is therefore not bounded from above. }
  \end{incremental}

  \tab{
  \lang{de}{Lösung b}
  \lang{en}{Solution b}
  }
  \begin{incremental}[\initialsteps{1}]
    \step \lang{de}{ Bei der Folge handelt es sich um eine geometrische Folge mit $q<0$, die nicht monoton ist, da für jedes $n\in\N$ die 
    Folgenglieder $a_{n}$ und $a_{n+1}$ unterschiedliche Vorzeichen haben. 
     Weil $|q|=\frac{1}{2}$ gilt, wird der Betrag $|a_n|$ mit wachsendem $n$ immer kleiner. Somit ist die Wertemenge der Folge nach oben durch $|a_1|=2$ und nach unten durch $-|a_1|=-2$ beschränkt. 
     Damit ist die Folge insgesamt beschränkt, aber nicht monoton.}
     \lang{en}{This is a geometric sequence with parameter $q<0$. The sequence is not monotone,
     because the terms $a_n$ and $a_{n+1}$ have differing signs for each $n\in\N$.
     Since $|q|=\frac{1}{2}$, the absolute value $|a_n|$ decreases as $n$ grows. Therefore, the sequence is bounded from above by $|a_1|=2$ and from below by $-|a_1|=-2$.}
    
    
  \end{incremental}

\end{tabs*}

\end{content}