\documentclass{mumie.element.exercise}
%$Id$
\begin{metainfo}
  \name{
    \lang{en}{...}
    \lang{de}{Ü03: Exponentialgleichungen}
  }
  \begin{description} 
 This work is licensed under the Creative Commons License Attribution 4.0 International (CC-BY 4.0)   
 https://creativecommons.org/licenses/by/4.0/legalcode 

    \lang{en}{...}
    \lang{de}{...}
  \end{description}
  \begin{components}
  \end{components}
  \begin{links}
  \end{links}
  \creategeneric
\end{metainfo}
\begin{content}
\title{
\lang{de}{Ü03: Exponentialgleichungen}
\lang{en}{Exercise 3: Exponential equations}
}
\begin{block}[annotation]
    Kopie aus omb+
\end{block}
\begin{block}[annotation]
	Im Ticket-System: \href{https://team.mumie.net/issues/24056}{Ticket 24056}
\end{block}

\lang{de}{Bestimmen Sie alle Lösungen der folgenden Gleichungen.}
\lang{en}{Find all solutions of the following equations.}
\lang{fr}{Trouvez toutes les solutions des équations suivantes.}
\lang{zh}{请确定下列方程所有的解。}

\begin{table}[\class{items}]
	\nowrap{a) $e^{2x}-6e^{x}=-9$} & \nowrap{b) $e^{2x}+e^{x}=20$  } \\
	\nowrap{c) $e^{2x}-3e^{x}=-2$}  & \nowrap{d) $(e^{2x}-2e^{x})(e^{x}-3)=0$  } \\
\end{table}

%%%%%%%%%%%%%%%%%%%%%%%%%%%%%%%%%%

\begin{tabs*}[\initialtab{0}\class{exercise}]								
\tab{\lang{de}{Antwort}\lang{en}{Answer}\lang{fr}{Réponse} \lang{zh}{答案}}

	\begin{table}[\class{items}]
		\nowrap{a) $\mathbb{L} = \{ \ln 3 \}$}     & \nowrap{b) $\mathbb{L} = \{ \ln 4 \} $} \\
		\nowrap{c) $\mathbb{L} = \{ 0; \, \ln 2 \} $} & \nowrap{d) $\mathbb{L} = \{ \ln 2; \, \ln 3 \} $} \\ 
	\end{table}
 
\tab{\lang{de}{L"osung a)}\lang{en}{Solution a)}\lang{fr}{Solution pour a)} \lang{zh}{解答 a)}}

	\begin{incremental}[\initialsteps{1}] 										
	\step 
		\lang{de}{Man versuche den Ansatz $u=e^{x}$. Wendet man diese Substitution auf die Gleichung an, so erhalten wir }
        \lang{en}{We use the Ansatz $u=e^{x}$. Substituting this into the equation results in:}
        \lang{fr}{On utilise l'approche par $u=e^{x}$. En substituant ceci dans l'équation on a :}
        \lang{zh}{使用算式 $u=e^{x}$ 。将其代入方程后可得}
          \[
          \lang{de}{u^{2}-6u=-9\,.}
          \lang{en}{u^{2}-6u=-9\,}
          \lang{fr}{u^{2}-6u=-9\,}
          \lang{zh}{u^{2}-6u=-9\,}
          \]
        
        
	\step
	    \lang{de}{Man bestimmt alle positiven Lösungen dieser quadratischen Gleichung. Gemäß der zweiten binomischen 
            Formel ergibt sich die Umformungskette}
        \lang{en}{We need to find all positive solutions of this quadratic equation. 
        The second binomial formula can be used to give the following transformations:}
        \lang{fr}{Nous devons trouver toutes les solutions positives de cette équation quadratique. 
        La seconde formule du binôme peut être utilisée pour nous donner les transformations suivantes :}
        \lang{zh}{我们需要确定这个一元二次方程所有的正数解。 
        根据完全平方公式(第二个二项式公式)可得如下变换:}
	    \[
		\lang{de}{u^{2}-6u=-9\Leftrightarrow u^{2}-6u+9=0 \Leftrightarrow (u-3)^{2}=0 \Leftrightarrow u=3\,.}
		\lang{en}{u^{2}-6u=-9\Leftrightarrow u^{2}-6u+9=0 \Leftrightarrow (u-3)^{2}=0 \Leftrightarrow u=3\,}
        \lang{fr}{u^{2}-6u=-9\Leftrightarrow u^{2}-6u+9=0 \Leftrightarrow (u-3)^{2}=0 \Leftrightarrow u=3\,}
        \lang{zh}{u^{2}-6u=-9\Leftrightarrow u^{2}-6u+9=0 \Leftrightarrow (u-3)^{2}=0 \Leftrightarrow u=3\,}
		\]
	\step 
	    \lang{de}{Daraus erhalten wir die Gleichung $3=e^{x}$ und damit ist $x=\ln 3$ die einzige Lösung der ursprünglichen
            Gleichung.}
        \lang{en}{From the above, we get the equation $3=e^{x}$ and hence $x=\ln 3$ is 
        the only solution to the original equation.}
        \lang{fr}{A partir de ce qu'il y a ci-dessus, on obtient l'équation $3=e^{x}$ et donc $x=\ln 3$ 
        est la seule solution de l'équation de départ.}
        \lang{zh}{由此可得方程 $3=e^{x}$ ,因此 $x=\ln 3$ 是 
        是原方程的唯一解。}
        
	\end{incremental}     

\tab{\lang{de}{L"osung b)}\lang{en}{Solution b)}\lang{fr}{Solution pour b)} \lang{zh}{解答 b)}}

	\begin{incremental}[\initialsteps{1}] 										
      \step 
          \lang{de}{Man macht den Ansatz $u=e^{x}$.}
          \lang{en}{ Use the Ansatz $u=e^{x}$.}
          \lang{fr}{Utilisez l'approche $u=e^{x}$.}
          \lang{zh}{使用算式 $u=e^{x}$ 。}
      \step
          \lang{de}{Substituiert man dies in obige Gleichung, so erhält man die quadratische Gleichung $u^{2}+u-20=0$.}
          \lang{en}{Substitute the Ansatz into the above equation, giving the quadratic equation $u^{2}+u-20=0$. }
          \lang{fr}{Substituer ceci dans l'équation ci-dessus donne l'équation quadratique $u^{2}+u-20=0$. }
          \lang{zh}{将这个算式代入上述方程中,可得一元二次方程 $u^{2}+u-20=0$ 。 }
      \step 
          \lang{de}{Mit dem Satz von Vieta oder der $p$-$q$-Formel sieht man, dass $u^{2}+u-20=(u-4)(u+5)$ gilt.}
          \lang{en}{Using Vieta's theorem or the $p$-$q$ formula, we can see that $u^{2}+u-20=(u-4)(u+5)$.}
          \lang{fr}{En utilisant le théorème de Viète ou la formule $p$-$q$, on peut voir que $u^{2}+u-20=(u-4)(u+5)$.}
          \lang{zh}{使用韦达定理或者求根公式($p$-$q$ 公式) 可以看出 $u^{2}+u-20=(u-4)(u+5)$ 。}
      \step
          \lang{de}{Da man $u=e^{x}>0$ gesetzt hat, betrachtet man nur positive Lösungen $u$ der Gleichung.}
          \lang{en}{Because we substituted $u=e^{x}>0$, only the positive solutions in $u$ should be considered.}
          \lang{fr}{Puisque l'on a substitué $u=e^{x}>0$, seules les solutions positives $u$ doivent être considérées.}
          \lang{zh}{因为 $u=e^{x}>0$ , 所以我们仅考虑 $u$ 的正数解。}
      \step
          \lang{de}{Die einzige positive Lösung der Gleichung ist $u=4$. Daraus erhalten wir $x=\ln u=\ln 4$ als eindeutige 
            Lösung der ursprünglichen Gleichung.}
          \lang{en}{The only positive solution of the equation is $u=4$. 
        From this, we get $x=\ln u=\ln 4$ as the unique solution of the original equation.}
        \lang{fr}{La seule solution positive de l'équation est $u=4$. 
        De cela, on obtient $x=\ln u=\ln 4$ comme unique solution de l'équation de départ.}
        \lang{zh}{方程唯一的正数解是 $u=4$ 。 
        由此可得 $x=\ln u=\ln 4$ 是原方程的唯一解。}
	\end{incremental}


\tab{\lang{de}{L"osung c)}\lang{en}{Solution c)}\lang{fr}{Solution pour c)} \lang{zh}{解答 c)}}
	\begin{incremental}[\initialsteps{1}] 										
	\step 
	    \lang{de}{Man macht den Ansatz $u=e^{x}$.}
        \lang{en}{Use the Ansatz $u=e^{x}$.}
        \lang{fr}{Utilisez l'approche $u=e^{x}$.}
        \lang{zh}{使用算式 $u=e^{x}$ 。}
	\step
	    \lang{de}{Wendet man diese Substitution auf die Gleichung an, so erhalten wir
          \[
          u^{2}-3u=-2 \Leftrightarrow u^{2}-3u+2=0 \,.
          \]}
        \lang{en}{Substitute the Ansatz into the equation and obtain:
          \[
          u^{2}-3u=-2 \Leftrightarrow u^{2}-3u+2=0 \,
          \]}
          
         \lang{fr}{Substituez ceci dans l'équation et obtenez :
          \[
          u^{2}-3u=-2 \Leftrightarrow u^{2}-3u+2=0 \,
          \]}
          \lang{zh}{将这个算式代入方程可得:
          \[
          u^{2}-3u=-2 \Leftrightarrow u^{2}-3u+2=0 \,
          \]}
	\step
	    \lang{de}{Man bestimmt alle positiven Lösungen dieser quadratischen Gleichung mittels der $p$-$q$-Formel.}
        \lang{en}{All of the positive solutions of this quadratic equation can be determined with the help of the $p$-$q$ formula.}
        \lang{fr}{Toutes les solutions positives de cette équation quadratique peuvent être déterminées avec l'aide de la formule $p$-$q$.}
        \lang{zh}{这个一元二次方程所有的正数解可以通过求根公式($p$-$q$ 公式)得出。}
	\step
		\begin{eqnarray*}
			& u_{1,2} = & -\frac{p}{2} \pm \frac{1}{2}\sqrt{p^2-4q}, \\
			& u_{1} = & -\frac{-3}{2} + \frac{1}{2}\sqrt{(-3)^2-4\cdot 2} =  \frac{3}{2} + \frac{1}{2} = 2, \\
			& u_{2} = & -\frac{-3}{2} - \frac{1}{2}\sqrt{(-3)^2-4\cdot 2} =  \frac{3}{2} - \frac{1}{2} = 1 \\ 
		\end{eqnarray*}
	\step 
	    \lang{de}{Da beide Lösungen positiv sind, erhält man zwei Gleichungen}
        \lang{en}{Since both solutions are positive, we get two equations:}
        \lang{fr}{Puisque les deux solutions sont positives, on obtient deux équations :}
        \lang{zh}{因为两个解都是正数,所以得到两个方程:}
		\begin{eqnarray*}
			& & e^{x}=2 \lang{de}{}\lang{en}{}\lang{fr}{ \text{ ou } }\lang{zh}{\text{ 或 }} e^{x}=1 \\
			& \Leftrightarrow & x=\ln 2 \lang{de}{}\lang{en}{}\lang{fr}{ \text{ ou } }\lang{zh}{\text{ 或 }} x=\ln 1 = 0 \\ 
		\end{eqnarray*}
	\step
	    \lang{de}{Die Lösungsmenge der ursprünglichen Gleichung ist also $\mathbb{L} = \{ 0; \, \ln 2 \}$.}
        \lang{en}{The solution set of the original equation is $\mathbb{L} = \{ 0, \, \ln 2 \}$.}
        \lang{fr}{L'ensemble solution de l'équation de départ est $\mathbb{L} = \{ 0, \, \ln 2 \}$.}
        \lang{zh}{因此,原方程的解集是 $\mathbb{L} = \{ 0, \, \ln 2 \}$ 。}
	\end{incremental}     

 
\tab{\lang{de}{L"osung d)}\lang{en}{Solution d)}\lang{fr}{Solution pour d)} \lang{zh}{解答 d)}}
	\begin{incremental}[\initialsteps{1}] 										
      \step 
          \lang{de}{Man macht den Ansatz $u=e^{x}$.}
          \lang{en}{Use the Ansatz $u=e^{x}$.}
          \lang{fr}{Utilisez l'approche $u=e^{x}$.}
          \lang{zh}{使用算式 $u=e^{x}$。}
      \step
          \lang{de}{Wendet man diese Substitution auf die Gleichung an, so erhalten wir
            \[(u^{2}-2u)(u-3)=0 \Leftrightarrow u(u-2)(u-3)=0 \,.\] }
          \lang{en}{Substitute the Ansatz into the equation and obtain:
		    \[(u^{2}-2u)(u-3)=0 \Leftrightarrow u(u-2)(u-3)=0 \,\]}
            \lang{fr}{Substituez ceci dans l'équation et obtenez :
		    \[(u^{2}-2u)(u-3)=0 \Leftrightarrow u(u-2)(u-3)=0 \,\]}
            \lang{zh}{将这个算式代入方程可得
		    \[(u^{2}-2u)(u-3)=0 \Leftrightarrow u(u-2)(u-3)=0 \,\]}
      \step
          \lang{de}{Die Lösungen dieser Gleichung sind $u_1=0$, $u_2=2$ und $u_3=3$.}
          \lang{en}{The solutions to this equation are $u_1=0$, $u_2=2$, and $u_3=3$.}
          \lang{fr}{Les solutions de cette équation sont $u_1=0$, $u_2=2$, et $u_3=3$.}
          \lang{zh}{这个方程的解是 $u_1=0$, $u_2=2$ 和 $u_3=3$ 。}
      \step
          \lang{de}{Nach dem Ansatz $u=e^{x}>0$ ist $u$ strikt positiv und die Lösung 
            $u_1=0$ entfällt also aus der Betrachtung.}
          \lang{en}{According to this Ansatz $u=e^{x}>0$, $u$ is strictly positive, 
        and the solution $u_1=0$ is not a valid one.}
        \lang{fr}{Selon l'approche $u=e^{x}>0$, $u$ est strictement positif, 
        et la solution $u_1=0$ n'est pas une solution valide.}
        \lang{zh}{根据算式 $u=e^{x}>0$ ,$u$ 是正数,所以可排除解 $u_1=0$ 。}
      \step 
          \lang{de}{Für die beiden anderen Lösungen bekommt man
            \begin{eqnarray*}
                & & e^{x} = 2 \Leftrightarrow x = \ln 2, \\
                & & e^{x} = 3 \Leftrightarrow x = \ln 3. \\ 
            \end{eqnarray*}}
          \lang{en}{For the other two solutions, we get:
            \begin{eqnarray}
                & & e^{x} = 2 \Leftrightarrow x = \ln 2 \\
                & & e^{x} = 3 \Leftrightarrow x = \ln 3 \\ 
            \end{eqnarray}}
            \lang{fr}{Pour les deux autres solutions, on obtient :
            \begin{eqnarray*}
                & & e^{x} = 2 \Leftrightarrow x = \ln 2 \\
                & & e^{x} = 3 \Leftrightarrow x = \ln 3 \\ 
            \end{eqnarray*}}
          \lang{zh}{对其余两个方程可得:
            \begin{eqnarray*}
                & & e^{x} = 2 \Leftrightarrow x = \ln 2 \\
                & & e^{x} = 3 \Leftrightarrow x = \ln 3 \\ 
            \end{eqnarray*}}
      \step
          \lang{de}{Die Lösungsmenge der ursprünglichen Gleichung ist also $\mathbb{L} = \{ \ln 2; \, \ln 3 \}$.}
          \lang{en}{The solution set of the original equation is $\mathbb{L} = \{ \ln 2, \, \ln 3 \}$.}
          \lang{fr}{L'ensemble solution de l'équation de départ est $\mathbb{L} = \{ \ln 2, \, \ln 3 \}$.}
          \lang{zh}{原方程的解集是 $\mathbb{L} = \{ \ln 2, \, \ln 3 \}$ 。}
      \step
         \lang{de}{Ein anderer Lösungsweg: Ausklammern von $e^x$ aus der linken Klammer ergibt
            \[ e^x(e^x-2)(e^x-3)=0. \]}
         \lang{en}{Another solution method: we factor out an $e^x$ from the left hand side. This gives:
		    \[ e^x(e^x-2)(e^x-3)=0 \]}
            \lang{fr}{Autre méthode : on peut factoriser par $e^x$ dans le terme de gauche. Cela donne :
		    \[ e^x(e^x-2)(e^x-3)=0 \]}
         \lang{zh}{另一种解法:从左边分解出 $e^x$ 可得
		    \[ e^x(e^x-2)(e^x-3)=0 \]}
         \lang{de}{Da $e^x>0$ kann das nur gelten, wenn $(e^x-2)=0$ oder $(e^x-3)=0$. Also erhält man die Lösungen
            \begin{eqnarray*}
                & & e^{x} = 2 \Leftrightarrow x_1 = \ln 2, \\
                & & e^{x} = 3 \Leftrightarrow x_2 = \ln 3. \\ 
            \end{eqnarray*}}
         \lang{en}{Because $e^x>0$, this can only be true if $(e^x-2)=0$ or $(e^x-3)=0$. The solutions are thus:
            \begin{eqnarray}
                & & e^{x} = 2 \Leftrightarrow x_1 = \ln 2 \\
                & & e^{x} = 3 \Leftrightarrow x_2 = \ln 3 \\ 
            \end{eqnarray}}
            \lang{fr}{Puisque $e^x>0$, cela est seulement vraie si $(e^x-2)=0$ ou $(e^x-3)=0$. Les solutions sont donc :
            \begin{eqnarray*}
                & & e^{x} = 2 \Leftrightarrow x_1 = \ln 2 \\
                & & e^{x} = 3 \Leftrightarrow x_2 = \ln 3 \\ 
            \end{eqnarray*}}
            \lang{zh}{因为 $e^x>0$ 恒成立,所以只有当 $(e^x-2)=0$ 或 $(e^x-3)=0$ 时原方程成立。因此解是
            \begin{eqnarray*}
                & & e^{x} = 2 \Leftrightarrow x_1 = \ln 2 \\
                & & e^{x} = 3 \Leftrightarrow x_2 = \ln 3 \\ 
            \end{eqnarray*}}
	\end{incremental}     
\end{tabs*}  


\end{content}

