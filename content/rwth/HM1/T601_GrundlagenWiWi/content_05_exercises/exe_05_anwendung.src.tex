\documentclass{mumie.element.exercise}
%$Id$
\begin{metainfo}
  \name{
    \lang{en}{Ü05: Solving for exponent or base}
    \lang{de}{Ü05: Auflösen nach Exponent oder Basis }
  }
  \begin{description} 
 This work is licensed under the Creative Commons License Attribution 4.0 International (CC-BY 4.0)   
 https://creativecommons.org/licenses/by/4.0/legalcode 

    \lang{en}{...}
    \lang{de}{...}
  \end{description}
  \begin{components}
  \end{components}
  \begin{links}
  \end{links}
  \creategeneric
\end{metainfo}
\begin{content}
\title{\lang{en}{Exercise 5: Solving for exponent or base}
    \lang{de}{Ü05: Auflösen nach Exponent oder Basis }}
\begin{block}[annotation]
	Im Ticket-System: \href{https://team.mumie.net/issues/24055}{Ticket 24055}
\end{block}

\begin{itemize}
\item[a)]
\lang{de}{Ein Kapital ist auf $K_{11}=20763,51$ \euro angewachsen. Es war $n=11$ Jahre angelegt zu einem
Zinssatz von $3$ \%. Wie hoch war das Startkapital $K_0$?}
\lang{en}{An sum of money grows in value to $K_{11}=20763.51$ \euro. It was invested for $n=11$ years
at an interest rate of $3$ \%. What was its original value $K_0$?
}

\item[b)]
\lang{de}{Das Kapital $K_0$ ist nun in 11 Jahren nur auf $K_{11}=16735,03$ \euro angewachsen.
Zu welchen Zinssatz war es angelegt?}
\lang{en}{The sum with starting amount $K_0$ instead grows to $K_{11} = 16735.03$ \euro after $11$ years.
What was its interest rate?}
\item[c)]
\lang{de}{Wieviel Jahre $n$ muss das Startkapital $K_0=15000$ \euro angelegt werden, um bei einer Verzinsung 
von $1$ \% auch auf $K_{11}=20763,51$ \euro zu kommen?}
\lang{en}{For how many years must the starting sum $K_0=15000$ \euro be invested at an interest rate of $1$\%
in order to grow to $K_{11}=20763.51$\euro?}
\end{itemize}
  \begin{tabs*}[\initialtab{0}\class{exercise}]
    \tab{
      \lang{en}{Solution}
      \lang{de}{Lösung}
      \lang{zh}{...}
      \lang{fr}{...}
    }
    \begin{incremental}[\initialsteps{1}]
      \step
        \lang{en}{
        \begin{itemize}
        \item[a)]
        $K_0=15000$ \euro
        \item[b)]
        $1$ \%        
        \item[c)]
        $n=32.7$ years
        \end{itemize}
        }
        \lang{de}{
        \begin{itemize}
        \item[a)]
        $K_0=15000$ \euro
        \item[b)]
        $1$ \%        
        \item[c)]
        $n=32,7$ Jahre
        \end{itemize}
        }
        \lang{zh}{...}
        \lang{fr}{...}
     
    \end{incremental}
    \tab{
      \lang{en}{Calculations}
      \lang{de}{Rechnung}
      \lang{zh}{...}
      \lang{fr}{...}
    }
    \begin{incremental}[\initialsteps{1}]
      \step
        \lang{en}{
        \begin{itemize}
        \item[]
        The value grows exponentially, following the formula
        \[K_n=K_0(1+\frac{p}{100})^n,\]
        which will be discussed in more detail in the next chapter.
        \item[a)]
        After a simple division, we find:
        
        \begin{eqnarray*}
        K_0&=&\frac{K_n}{(1+\frac{p}{100})^n}\\
        &=&\frac{20763.51}{1.03^{11}}\\
        &=&15000
        \end{eqnarray*}

        \item[b)]
        The interest rate appears in the base of the equation.
        To extract $p$, we must take n-th roots:
        \begin{align*}
        \ & \sqrt[n]{\frac{K_n}{K_0}}&=&1+\frac{p}{100}\\
        \Leftrightarrow & \frac{p}{100}&=&\sqrt[n]{\frac{K_n}{K_0}}-1\\
        \Leftrightarrow & \frac{p}{100}&=&\sqrt[11]{\frac{16735.03}{15000}}-1\\
        \Leftrightarrow & \frac{p}{100}&=&1 \text{\%}
        \end{align*}

        \item[c)]
        The time $n$ appears in the exponent of the equation. 
        To extract $n$, we must take logarithms.
        Here we use $\ln$, but $\log$ to any other base can also be used.
        \begin{align*}
        \ & K_n&=&K_0(1+\frac{p}{100})^n\\
        \Leftrightarrow & 20763.51&=&15000\cdot 1.01^n\\
        \Leftrightarrow & \frac{20763.51}{15000}&=&1.01^n \\
        \Leftrightarrow & \ln\frac{20763.51}{15000}&=&\ln1.01^n \\
        \Leftrightarrow & \ln\frac{20763.51}{15000}&=&n\cdot\ln1.01\\
        \Leftrightarrow & n&=&32.67
        \end{align*}
        The calculation is simplest using the logarithm with respect to base 1.01:
        \begin{align*}
        \ & \log_{1,01}\frac{20763.51}{15000}&=&\log_{1.01}1.01^n \\
        \Leftrightarrow & 32.67&=&n\cdot \log_{1.01}.\\
        \Leftrightarrow & 32.67&=&n\cdot 1\\
        \Leftrightarrow & n&=&32.67
        \end{align*}
        \end{itemize}
        }
        \lang{de}{
        \begin{itemize}
        \item[]
        Das Kapital wächst exponentiell nach der Formel 
        \[K_n=K_0(1+\frac{p}{100})^n,\]
        die wir im nächsten Kapitel ausführlich besprechen werden. 
        \item[a)]
        Durch einfache Division erhalten wir:
        
        \begin{eqnarray*}
        K_0&=&\frac{K_n}{(1+\frac{p}{100})^n}\\
        &=&\frac{20763,51}{1,03^{11}}\\
        &=&15000
        \end{eqnarray*}
        \item[b)]
        Der Zinssatz steht in der "Basiszeile" der Gleichung. Um $p$ auszurechnen,
        müssen wir die n-te Wurzel ziehen:
        \begin{align*}
        \ & \sqrt[n]{\frac{K_n}{K_0}}&=&1+\frac{p}{100}\\
        \Leftrightarrow & \frac{p}{100}&=&\sqrt[n]{\frac{K_n}{K_0}}-1\\
        \Leftrightarrow & \frac{p}{100}&=&\sqrt[11]{\frac{16735,03}{15000}}-1\\
        \Leftrightarrow & \frac{p}{100}&=&1 \text{\%}
        \end{align*}
        \item[c)]
        Die Laufzeit $n$ steht im Exponenten der Gleichung. Um $n$ auszurechnen,
        müssen wir logarithmieren. Hier wird mit dem $\ln$ gerechnet, aber $\log$ zu einer
        anderen Basis geht ebenso:
        \begin{align*}
        \ & K_n&=&K_0(1+\frac{p}{100})^n\\
        \Leftrightarrow & 20763,51&=&15000\cdot 1,01^n\\
        \Leftrightarrow & \frac{20763,51}{15000}&=&1,01^n \\
        \Leftrightarrow & \ln\frac{20763,51}{15000}&=&\ln1,01^n \\
        \Leftrightarrow & \ln\frac{20763,51}{15000}&=&n\cdot\ln1,01\\
        \Leftrightarrow & n&=&32,67
        \end{align*}
       Am einfachsten geht es, indem man den Logarithmus zur Basis 1,01 nimmt:
        \begin{align*}
        \ & \log_{1,01}\frac{20763,51}{15000}&=&\log_{1,01}1,01^n \\
        \Leftrightarrow & 32,67&=&n\cdot \log_{1,01}1,01\\
        \Leftrightarrow & 32,67&=&n\cdot 1\\
        \Leftrightarrow & n&=&32,67
        \end{align*}
        \end{itemize}
        
        
        }
        \lang{zh}{...}
        \lang{fr}{...}
    \end{incremental}
  \end{tabs*}



\end{content}

