\documentclass{mumie.element.exercise}
%$Id$
\begin{metainfo}
  \name{
    \lang{en}{...}
    \lang{de}{Ü04: Logarithmusgleichungen}
  }
  \begin{description} 
 This work is licensed under the Creative Commons License Attribution 4.0 International (CC-BY 4.0)   
 https://creativecommons.org/licenses/by/4.0/legalcode 

    \lang{en}{...}
    \lang{de}{...}
  \end{description}
  \begin{components}
  \end{components}
  \begin{links}
  \end{links}
  \creategeneric
\end{metainfo}
\begin{content}
\title{
\lang{de}{Ü04: Logarithmusgleichungen}
\lang{en}{Exercise 4: Logarithm equations}
}
\begin{block}[annotation]
Im Ticket-System: \href{https://team.mumie.net/issues/24057}{Ticket 24057}
\end{block}

\begin{block}[annotation]
	Übungsaufgabe zur L"osung von Logarithmusgleichungen					 
\end{block}

\begin{block}[annotation]
Kopie aus omb+	   
\end{block}

\lang{de}{Bestimmen Sie alle Lösungen der folgenden beiden Gleichungen.}
\lang{en}{Find all of the solutions of the following two equations.}
\lang{fr}{Trouvez toutes les solutions des deux équations suivantes.}
\lang{zh}{请确定下列方程所有的解。}					
\begin{table}[\class{items}]
	\nowrap{a) $\ln(x^{2}-7x+11)=0$} & \\
	\nowrap{b) $\ln(e^{x}-5e)=1$} & \\ 
\end{table}

\begin{table}[\class{items}]
	\nowrap{
%begin-cosh
	\lang{de}{c) Für welche $x\in\mathbb{R}$ ist $\lg (4-x^2)$ definiert?}
%end-cosh
	\lang{en}{c) For which values of $x\in\mathbb{R}$ is $\lg (4-x^2)$ defined?}
    \lang{fr}{c) Pour quelles valeurs de $x\in\mathbb{R}$ $\lg (4-x^2)$ est-il défini ?}
    \lang{zh}{c) 函数 $\lg (4-x^2)$ 在 $x\in\mathbb{R}$ 中的定义域是什么?}}   & \\
\end{table}

%%%%%%%%%%%%%%%%%%%%%%%%%%%%%%%%%%%%%%%%


\begin{tabs*}[\initialtab{0}\class{exercise}]								
\tab{\lang{de}{Antwort}\lang{en}{Answer}\lang{fr}{Réponse} \lang{zh}{答案}}

	\lang{de}{
        \begin{table}[\class{items}]
          \nowrap{a) $x_1=2$ und  $x_2=5$} & \\
          \nowrap{b) $x=\ln 6 + 1$} & \\
          \nowrap{c) $x\in\mathbb{R}$ mit  $-2 < x$ und  $x < 2$, d.h.   $\,x\in (-2;2)\,$} & \\ 
		\end{table}}
    \lang{en}{
      \begin{table}[\class{items}]
          \nowrap{a) $x_1=2$ and  $x_2=5$} & \\
          \nowrap{b) $x=\ln 6 + 1$} & \\
          \nowrap{c) $x\in\mathbb{R}$ where  $-2 < x$ and  $x < 2$, i.e.   $\,x\in (-2,2)\,$} & \\ 
      \end{table}}
    \lang{fr}{
      \begin{table}[\class{items}]
          \nowrap{a) $x_1=2$ et  $x_2=5$} & \\
          \nowrap{b) $x=\ln 6 + 1$} & \\
          \nowrap{c) $x\in\mathbb{R}$ où  $-2 < x$ et  $x < 2$, c'est-à-dire   $\,x\in (-2,2)\,$} & \\ 
      \end{table}}
  \lang{zh}{
      \begin{table}[\class{items}]
          \nowrap{a) $x_1=2$ 和  $x_2=5$} & \\
          \nowrap{b) $x=\ln 6 + 1$} & \\
          \nowrap{c) $x\in\mathbb{R}$  并且 $-2 < x$ ,$x < 2$ ,即  $\,x\in (-2,2)\,$} & \\ 
      \end{table}}
\tab{\lang{de}{L"osung a)}\lang{en}{Solution a)}\lang{fr}{Solution pour a)} \lang{zh}{解答 a)}}

	\begin{incremental}[\initialsteps{1}] 										
	\step
	    \lang{de}{Es gilt $\ln y=0$ genau dann, wenn $y=1$. Also ist die Gleichung äquivalent zu $x^{2}-7x+11=1$.}
        \lang{en}{$\ln y=0$ if and only if $y=1$. The original equation is equivalent to $x^{2}-7x+11=1$.}
        \lang{fr}{$\ln y=0$ si et seulement si $y=1$. L'équation de départ est équivalente à $x^{2}-7x+11=1$.}
        \lang{zh}{$\ln y=0$ 当且仅当 $y=1$ 。原方程等价于 $x^{2}-7x+11=1$ 。}
	\step
	    \lang{de}{Man löst nun diese quadratische Gleichung. Wegen $(-2)\cdot (-5)=10$ und $-2-5=-7$ erhalten wir aus dem Satz von Vieta
	 	    (oder mit der $p$-$q$-Formel)
            \[
	 	        x^{2}-7x+11=1\Leftrightarrow x^{2}-7x+10=0 \Leftrightarrow (x-2)(x-5)=0 \Leftrightarrow x=2 \text{  oder    } x=5\,.
	 	    \] 
            }
        \lang{en}{
          Solve this quadratic equation. Because $(-2)\cdot (-5)=10$ and $-2-5=-7$, using Vieta's second theorem (or with the $p$-$q$ formula) we get
          \[
	 	      x^{2}-7x+11=1\Leftrightarrow x^{2}-7x+10=0 \Leftrightarrow (x-2)(x-5)=0 \Leftrightarrow x=2 \text{  or    } x=5\,.
	 	  \] 
        }
        \lang{fr}{
          Résolvez cette équation quadratique. Puisque $(-2)\cdot (-5)=10$ et $-2-5=-7$, 
          en utilisant le second théorème de Viète (ou avec la formule $p$-$q$) on obtient :
          \[
	 	      x^{2}-7x+11=1\Leftrightarrow x^{2}-7x+10=0 \Leftrightarrow (x-2)(x-5)=0 \Leftrightarrow x=2 \text{ ou  } x=5\,.
	 	  \] 
        }
        \lang{zh}{
          解二次方程。因为 $(-2)\cdot (-5)=10$ 以及 $-2-5=-7$ ,利用韦达定理 (或者求根公式,即 $p$-$q$ 公式) 可得
          \[
	 	    x^{2}-7x+11=1\Leftrightarrow x^{2}-7x+10=0 \Leftrightarrow (x-2)(x-5)=0 \Leftrightarrow x=2  \text{ 或 }  x=5\,
	 	  \] 
        }
        
	 	
	\step
	    \lang{de}{Daher sind die Lösungen der Gleichung $x_1=2$ und $x_2=5$.}
        \lang{en}{The solutions of the equation are $x_1=2$ and $x_2=5$.}
        \lang{fr}{Les solutions de l'équation sont $x_1=2$ et $x_2=5$.}
        \lang{zh}{方程的解为 $x_1=2$ 和 $x_2=5$ 。}
        
	\end{incremental}
 
 
\tab{\lang{de}{L"osung b)}\lang{en}{Solution b)}\lang{fr}{Solution pour b)} \lang{zh}{解答 b)}}

	\begin{incremental}[\initialsteps{1}] 										
	\step
		\lang{de}{Nach Definition der Logarithmusfunktion gilt}
        \lang{en}{By the definition of the logarithm:}
        \lang{fr}{Par définition du logarithme :}
        \lang{zh}{根据对数函数的定义}
        
	 	\[
	 	1 = \ln(e^{x}-5e) \Leftrightarrow e^{1} = e^{x}-5e
	 	\]
	\step
	    \lang{de}{Man forme nun die zweite Gleichung mittels der Rechenregel für Potenzen um.}
        \lang{en}{The second equation can now be transformed with the help of the rules for manipulating powers.}
        \lang{fr}{La seconde équation peut maintenant être transformée avec l'aide des règles de manipulations des puissances.}
        \lang{zh}{利用指数运算法则将第二个方程变形。}
		\begin{eqnarray*}
			&& e = e^{x}-5e \\
			& \Leftrightarrow & 6e = e^{x} \\
			& \Leftrightarrow & 6 = e^{x-1}\lang{de}{.}\lang{en}{}\lang{fr}{} \lang{zh}{} \\
		\end{eqnarray*}
	\step
	    \lang{de}{Man wende jetzt den natürlichen Logarithmus auf beide Seiten an.}
        \lang{en}{Take the natural log of both sides:}
        \lang{fr}{Prenez le logarithme népérien de chaque côté :}
        \lang{zh}{两边同时取自然对数,可得:}
		\begin{eqnarray*}
			& \Leftrightarrow & 6 = e^{x-1} \\
			& \Leftrightarrow & \ln 6 = \ln (e^{x-1}) \\
			& \Leftrightarrow & \ln 6 = x-1 \\
			& \Leftrightarrow & \ln (6) +1= x \\
		\end{eqnarray*}
	\step
	    \lang{de}{Daher ist $x=\ln (6) +1$ die einzige Lösung der Gleichung.}
        \lang{en}{ $x=\ln(6)+1$ is the only solution to the equation.}
        \lang{fr}{ $x=\ln(6)+1$ est la seule solution de l'équation.}
        \lang{zh}{所以,$x=\ln(6)+1$ 是方程的唯一解。}
	\end{incremental}


\tab{\lang{de}{L"osung c)}\lang{en}{Solution c)}\lang{fr}{Solution pour c)} \lang{zh}{解答 c)}}

	\begin{incremental}[\initialsteps{1}] 										
	\step
		\lang{de}{Der dekadische Logarithmus $\lg(y)$ ist nur für $y>0$ definiert.}
        \lang{en}{The base-10 logarithm $\lg(y)$ is only defined if $y>0$.}
        \lang{fr}{Le logarithme en base $10$, $\lg(y)$ est seulement défini pour $y>0$.}
        \lang{zh}{以 10 为底的对数 $\lg(y)$ 仅在 $y>0$ 时有定义。}
	\step
	    \lang{de}{Das gesuchte $x$ muss also $4-x^2>0$ erfüllen.}
        \lang{en}{The desired $x$ needs to satisfy $4-x^2>0$.}
        \lang{fr}{Le $x$ cherché doit satisfaire $4-x^2>0$.}
        \lang{zh}{所求的 $x$ 需要满足 $4-x^2>0$ 。}
	\step
	    \lang{de}{Man löst diese quadratische Ungleichung, indem man den quadratischen Term in Normalform bringt 
	 	    und die Nullstellen bestimmt.}
        \lang{en}{Solve the above quadratic inequality by putting the quadratic term into normal form:}
        \lang{fr}{Résolvez l'inéquation quadratique ci-dessus en mettant le terme quadratique dans sa forme normale :}
        \lang{zh}{通过将二次项化为标准形式来解二次不等式 ,从而确定其零点。 }
        
        \begin{align*}
		 	\lang{de}{& & 4-x^2 > 0 \qquad | \cdot(-1)\\
		 	& \Leftrightarrow & x^2 - 4 < 0 \,.}
		 	\lang{en}{& & 4-x^2 > 0 \qquad | \cdot(-1)\\
		 	& \Leftrightarrow & x^2 - 4 < 0 \,}
            \lang{fr}{& & 4-x^2 > 0 \qquad | \cdot(-1)\\
		 	& \Leftrightarrow & x^2 - 4 < 0 \,}
            \lang{zh}{& & 4-x^2 > 0 \qquad | \cdot(-1)\\
		 	& \Leftrightarrow & x^2 - 4 < 0 \,}
		\end{align*}
	\step
		\lang{de}{Die Nullstellen von $\,x^2-4\,=0$ sind $\,x_1=-2\,$ und $\,x_2=2$. Der Term ist zwischen den Nullstellen negativ, 
		    also für alle $x\in\mathbb{R}$ mit $-2 < x$ und $x < 2$.}
        \lang{en}{The roots of $\,x^2-4\,=0$ are $\,x_1=-2\,$ and $\,x_2=2$. Between the roots this term is negative, i.e. it is negative for all 
            $x\in\mathbb{R}$ if $-2 < x$ and $x < 2$.}
        \lang{fr}{Les racines de $\,x^2-4\,=0$ sont $\,x_1=-2\,$ et $\,x_2=2$. Entre les racines ce terme est négatif, c'est-à-dire qu'il est négatif pour tout 
            $x\in\mathbb{R}$ si $-2 < x$ et $x < 2$.}
        \lang{zh}{ $\,x^2-4\,=0$ 的零点是 $\,x_1=-2\,$ 和 $\,x_2=2$ 。此式在两个零点内为负,即对满足 $-2 < x$ 和 $x < 2$ 的所有 $x\in\mathbb{R}$ 都成立。}
    \step
		\lang{de}{$\lg (4-x^2)$ ist also für alle $\,x\in (-2;2)\,$ definiert.}
        \lang{en}{$\lg (4-x^2)$ is defined for all $\,x\in (-2,2)\,$.}
        \lang{fr}{$\lg (4-x^2)$ est défini pour tout $\,x\in (-2,2)\,$.}
        \lang{zh}{$\lg (4-x^2)$ 对于所有的 $\,x\in (-2,2)\,$ 都有定义。}
	\end{incremental}

\end{tabs*}  

\end{content}

