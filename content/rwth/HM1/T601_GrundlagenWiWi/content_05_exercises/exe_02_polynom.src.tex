\documentclass{mumie.element.exercise}
%$Id$
\begin{metainfo}
  \name{
    \lang{en}{}
    \lang{de}{Ü02: Nullstellen von Polynomen}
  }
  \begin{description} 
 This work is licensed under the Creative Commons License Attribution 4.0 International (CC-BY 4.0)   
 https://creativecommons.org/licenses/by/4.0/legalcode 

    \lang{en}{...}
    \lang{de}{...}
  \end{description}
  \begin{components}
  \end{components}
  \begin{links}
  \end{links}
  \creategeneric
\end{metainfo}
\begin{content}
\title{
\lang{de}{Ü02: Nullstellen von Polynomen}
\lang{en}{Exercise 2: Zeros of polynomials}
}

\begin{block}[annotation]
	Kopie aus dem omb+
\end{block}

\begin{block}[annotation]
	Im Ticket-System: \href{https://team.mumie.net/issues/24058}{Ticket 24058}
\end{block}

\lang{de}{Faktorisieren Sie die folgenden Funktionen und geben Sie ihre Nullstellen an.}
\lang{en}{Factor the following functions and find their roots.}
\lang{fr}{Factorisez les fonctions suivantes et déterminez leurs racines.}
\lang{zh}{请对下列函数进行因式分解并且确定它们的零点。}
\begin{table}[\class{items}]
	\nowrap{a) $f(x)=x^3-4x$}
	& \nowrap{b) $f(x)=2x^2-2$} \\
	\nowrap{c) $f(x)=x^4-2x^3+x^2$}
	& \nowrap{d) $f(x)=(2x^2+4x+2)^2$}
\end{table}


\begin{tabs*}[\initialtab{0}\class{exercise}]
\tab{\lang{de}{Antwort}\lang{en}{Answer}\lang{fr}{Réponse} \lang{zh}{答案}}
	\lang{de}{
      \begin{table}[\class{items}]
          \nowrap{a) $f(x)=x(x-2)(x+2)$; Nullstellen $x_1=0$, $x_2=2$, $x_3=-2$.} \\
          \nowrap{b) $f(x)=2(x-1)(x+1)$; Nullstellen  $x_1=1$, $x_2=-1$.} \\
          \nowrap{c) $f(x)=x^2(x-1)^2$; Nullstellen  $x_1=0$, $x_2=1$.} \\
          \nowrap{d) $f(x)=4(x+1)^4$; Nullstellen $x=-1$.}
      \end{table}}
    \lang{en}{
      \begin{table}[\class{items}]
          \nowrap{a) $f(x)=x(x-2)(x+2)$ with roots: $x_1=0$, $x_2=2$, and $x_3=-2$.} \\
          \nowrap{b) $f(x)=2(x-1)(x+1)$ with roots:  $x_1=1$ and $x_2=-1$.} \\
          \nowrap{c) $f(x)=x^2(x-1)^2$ with roots:  $x_1=0$ and $x_2=1$.} \\
          \nowrap{d) $f(x)=4(x+1)^4$ with roots: $x=-1$.}
      \end{table}}
    \lang{fr}{
      \begin{table}[\class{items}]
          \nowrap{a) $f(x)=x(x-2)(x+2)$ avec les racines : $x_1=0$, $x_2=2$, et $x_3=-2$.} \\
          \nowrap{b) $f(x)=2(x-1)(x+1)$ avec les racines :  $x_1=1$ et $x_2=-1$.} \\
          \nowrap{c) $f(x)=x^2(x-1)^2$ avec les racines :  $x_1=0$ et $x_2=1$.} \\
          \nowrap{d) $f(x)=4(x+1)^4$ avec la racine : $x=-1$.}
      \end{table}}
 \lang{zh}{
      \begin{table}[\class{items}]
          \nowrap{a) $f(x)=x(x-2)(x+2)$ ; 零点为 $x_1=0$, $x_2=2$, 和 $x_3=-2$ 。} \\
          \nowrap{b) $f(x)=2(x-1)(x+1)$ ; 零点为 $x_1=1$ 和 $x_2=-1$ 。} \\
          \nowrap{c) $f(x)=x^2(x-1)^2$ ; 零点为  $x_1=0$ 和 $x_2=1$ 。} \\
          \nowrap{d) $f(x)=4(x+1)^4$ ; 零点为 $x=-1$ 。}
      \end{table}}
\tab{\lang{de}{Lösung a)}\lang{en}{Solution a)}\lang{fr}{Solution pour a)} \lang{zh}{解答 a)}}
	\begin{incremental}[\initialsteps{1}]
	\step
		\lang{de}{Man klammere zuerst $x$ aus und wende dann die dritte binomische Formel an.}
        \lang{en}{First, factor out $x$ and then use the third binomial formula.}
        \lang{fr}{D'abord on factorise par $x$, puis on utilise la troisième identité remarquable :}
        \lang{zh}{首先分解出 $x$ 并且使用平方差公式(第三个二项式公式)。}
	\step
		\begin{eqnarray*}
			&& f(x) = x^3-4x = x(x^2-4) = x(x-2)(x+2) \\ 
		\end{eqnarray*}
	\step
		\lang{de}{$f(x)$ hat folglich die Nullstellen $x_1=0$, $x_2=2$, $x_3=-2$.}
        \lang{en}{$f(x)$ has roots at $x_1=0$, $x_2=2$, and $x_3=-2$.}
        \lang{fr}{$f$ admet trois racines : $x_1=0$, $x_2=2$, et $x_3=-2$.}
        \lang{zh}{$f(x)$ 在 $x_1=0$ , $x_2=2$ 和 $x_3=-2$ 处有零点。}
	\end{incremental}
  
  
\tab{\lang{de}{Lösung b)}\lang{en}{Solution b)}\lang{fr}{Solution pour b)} \lang{zh}{解答 b)}}
	\begin{incremental}[\initialsteps{1}]
	\step
		\lang{de}{Ausklammern von $2$ und Anwendung der dritten binomischen Formel ergibt}
        \lang{en}{Factor out $2$ and use the third binomial formula to give:}
        \lang{fr}{On factorise par $2$, puis on utilise la troisième identité remarquable afin d'obtenir :}
        \lang{zh}{分解出 $2$ 并且使用平方差公式(第三个二项式公式)可得:}
	\step
		\begin{eqnarray*}
			&& f(x) = 2x^2-2 = 2(x^2-1) = 2(x-1)(x+1) \\ 
		\end{eqnarray*}
	\step
		\lang{de}{Die Nullstellen von $f(x)$ sind $x_1=1$ und $x_2=-1$.}
        \lang{en}{The roots of $f(x)$ are $x_1=1$ and $x_2=-1$.}
        \lang{fr}{Les racines de $f$ sont : $x_1=1$ et $x_2=-1$.}
        \lang{zh}{$f(x)$ 的零点是 $x_1=1$ 和 $x_2=-1$ 。}
	\end{incremental}


\tab{\lang{de}{Lösung c)}\lang{en}{Solution c)}\lang{fr}{Solution pour c)} \lang{zh}{解答 c)}}
	\begin{incremental}[\initialsteps{1}]
	\step
		\lang{de}{Man klammere zuerst $x^2$ aus und wende dann die zweite binomische Formel an:}
        \lang{en}{First, factor out $x^2$ and then use the second binomial formula:}
        \lang{fr}{D'abord on factorise par $x^2$, puis on utilise la seconde identité remarquable :}
        \lang{zh}{首先分解出 $x^2$ 并且使用完全平方公式(第二个二项式公式):}
	\step
		\begin{eqnarray*}
			&& f(x) = x^4-2x^3+x^2 = x^2(x^2-2x+1) = x^2(x-1)^2 \\ 
		\end{eqnarray*}
	\step
		\lang{de}{Die Nullstellen von $f(x)$ sind also $x_1=0$ und $x_2=1$.}
        \lang{en}{The roots of $f(x)$ are thus $x_1=0$ and $x_2=1$.}
        \lang{fr}{Les racines de $f$ sont : $x_1=0$ et $x_2=1$.}
        \lang{zh}{$f(x)$ 的零点是 $x_1=0$ 和 $x_2=1$ 。}
	\end{incremental}


\tab{\lang{de}{Lösung d)}\lang{en}{Solution d)}\lang{fr}{Solution pour d)} \lang{zh}{解答 d)}}
	\begin{incremental}[\initialsteps{1}]
	\step
		\lang{de}{Man erinnere sich an die erste binomische Formel und die Rechenregeln f\"{u}r Potenzen:}
	    \lang{en}{Recall the first binomial formula and the rules for calculating with powers:}
        \lang{fr}{On factorise par $2$ et on utilise la première identité remarquable et les règles pour calculer avec des puissances :}
        \lang{zh}{回顾完全平方公式(第一个二项式公式)以及幂的计算法则:}
    \step
		\begin{eqnarray*}
			f(x) &=& (2x^2+4x+2)^2 \\
			&=& (2\,(x^2+2x+1)\,)^2 \\
			&=& 4(x^2+2x+1)^2 \\
			&=& 4((x+1)^2)^2 \\
			&=& 4(x+1)^4 \\ 
		\end{eqnarray*}
	\step
		\lang{de}{Damit hat $f(x)$ eine Nullstelle bei $x=-1$.}
        \lang{en}{$f(x)$ has a root at $x=-1$.}
        \lang{fr}{$f$ admet une racine : $x=-1$.}
        \lang{zh}{因此 $f(x)$ 的零点在 $x=-1$ 处。}
	\end{incremental}

\end{tabs*}






\end{content}

