\documentclass{mumie.element.exercise}
%$Id$
\begin{metainfo}
  \name{
    \lang{en}{Ü01: Equation of a line}
    \lang{de}{Ü01: Geradengleichung}
  }
  \begin{description} 
 This work is licensed under the Creative Commons License Attribution 4.0 International (CC-BY 4.0)   
 https://creativecommons.org/licenses/by/4.0/legalcode 

    \lang{en}{...}
    \lang{de}{...}
  \end{description}
  \begin{components}
  \end{components}
  \begin{links}
  \end{links}
  \creategeneric
\end{metainfo}
\begin{content}
\title{\lang{en}{Exercise 1: Equation of a line}
    \lang{de}{Ü01: Geradengleichung}}
\begin{block}[annotation]
	Im Ticket-System: \href{https://team.mumie.net/issues/24054}{Ticket 24054}
\end{block}
\begin{block}[annotation]
Kopie: hm4mint/T102_neu_Einfache_Reelle_Funktionen/exercise 4

	Im Ticket-System: \href{https://team.mumie.net/issues/21995}{Ticket 21995}
\end{block}


    \lang{de}{Bestimmen Sie die Gerade mit der Steigung $m=-2$, die 
    durch den Punkt $P=(1;9)$ verläuft.  }
    \lang{en}{Find the line with slope $m=-2$ that passes through the
    point $P=(1,9)$.}

\begin{tabs*}[\initialtab{0}\class{exercise}]
  		\tab{\lang{de}{Lösung} \lang{en}{Solution}}	

\lang{de}{Die Gerade ist durch den Punkt $P$ und die Steigung $m$ eindeutig festgelegt. 
Die Bestimmung der Geradengleichung erfolgt durch die Formel
\[y=m \cdot x+(y_{P}-m \cdot x_{P})\quad,\quad \text{ wobei }P=(x_{P};y_{P}).\]
Einsetzen der Werte liefert uns die gewünschte Darstellung der gesuchten Gerade
\[y=(-2)\cdot x+(9-(-2)\cdot 1)=-2x+11\,.\]}
\lang{en}{The line is uniquely determined by the point $P$ and the slope $m$.
The equation of the line can be found with the formula
\[y=m \cdot x+(y_{P}-m \cdot x_{P})\quad,\quad \text{ where }P=(x_{P},y_{P}).\]
Substituting in the given values yields the equation of the line
\[y=(-2)\cdot x+(9-(-2)\cdot 1)=-2x+11\,.\]}

\tab{\lang{de}{Video: ähnliche Übungsaufgabe} \lang{en}{Video: Similar exercises}}	
\youtubevideo[500][300]{UbXskYhBsM0}\\
  
\end{tabs*}

\end{content}

