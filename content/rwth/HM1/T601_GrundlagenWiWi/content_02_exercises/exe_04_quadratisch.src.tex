\documentclass{mumie.element.exercise}
%$Id$
\begin{metainfo}
  \name{
    \lang{de}{Ü04: quadratische Gleichungen}
  }
  \begin{description} 
 This work is licensed under the Creative Commons License Attribution 4.0 International (CC-BY 4.0)   
 https://creativecommons.org/licenses/by/4.0/legalcode 

    \lang{de}{Lösen quadratischer Gleichungen}
  \end{description}
  \begin{components}
  \end{components}
  \begin{links}
\link{generic_article}{content/rwth/HM1/T101neu_Elementare_Rechengrundlagen/g_art_content_05_loesen_gleichungen_und_lgs.meta.xml}{content_05_loesen_gleichungen_und_lgs}
\end{links}
  \creategeneric
\end{metainfo}
\begin{content}
\begin{block}[annotation]
	Im Ticket-System: \href{https://team.mumie.net/issues/23789}{Ticket 23789}
\end{block}
\begin{block}[annotation]
Kopie: hm4mint/T101neu_Elementare_Rechengrundlagen/exercise 12

Im Ticket-System: \href{https://team.mumie.net/issues/21982}{Ticket 21982}
\end{block}


  \title{
    \lang{de}{Ü04: quadratische Gleichungen}
    \lang{en}{Exercise 4: quadratic equations}
  }
 
%
% Aufgabenstellung
%
\begin{enumerate}[alph]
  \item \lang{de}{Lösen Sie die folgenden Gleichungen.} \lang{en}{Solve the following equations.}
        
    \begin{table}[\class{items}]
     \nowrap{i.   $\;  2x^2+8x-10=32 \quad$}      &
     \nowrap{ii.   $\;  -24-x=6x^2+23x+12$}          
    \end{table}
%
     \item \lang{de}{Die Quadrate zweier Zahlen, die auf dem Zahlenstrahl den gleichen
        Abstand zu der Zahl 20 haben, haben die Summe 832. Stellen Sie eine Gleichung
        auf und berechnen Sie diese Zahlen.}
        \lang{en}{The squares of two numbers that are equally distant from 20 on the number line sum to 832.
        Set up an equation for these numbers and compute them.}
\end{enumerate}
%
% Lösungen
%
  \begin{tabs*}[\initialtab{0}\class{exercise}]
  
    \tab{\lang{de}{Antworten} \lang{en}{Answers}}
      \begin{enumerate}[alph]
        \item 
        \lang{de}{
         \begin{table}[\class{items}]
           \nowrap{i.   $\; \mathbb{L}= \{ -7;3 \} \quad$} &
           \nowrap{ii.  $\; \mathbb{L}= \emptyset $} \\
        \end{table}
        }
        \lang{en}{
         \begin{table}[\class{items}]
           \nowrap{i.   $\; \mathbb{L}= \{ -7,3 \} \quad$} &
           \nowrap{ii.  $\; \mathbb{L}= \emptyset $} \\
        \end{table}
        }
        
        \item \lang{de}{Die gesuchten Zahlen sind 16 und 24.} \lang{en}{The numbers are 16 and 24.}

      \end{enumerate}

     
    \tab{\lang{de}{Lösung} \lang{en}{Solution} a) i.} 
     \begin{incremental}[\initialsteps{1}]
      \step 
        \lang{de}{Zuerst wird die Gleichung normiert, indem wir sie durch den Leitkoeffizienten 
        $\,2 \,$ teilen.}
        \lang{en}{First, we find an equivalent monic equation by dividing the given equation by the leading coefficient $\, 2.$}
        \[
        \begin{mtable}[\cellaligns{crcll}]
                         &  2x^2 + 8x - 10 &\,=\,& 32   & \vert :2\\
         \Leftrightarrow &\qquad  x^2 + 4x - 5 &\,=\,& 16  \quad &
        \end{mtable}
        \]

        
    \step  \lang{de}{Lösung mit pq-Formel:} \lang{en}{Use the quadratic formula:}
\\
    
        \lang{de}{Hierzu muss die Gleichung jedoch zunächst in einem weiteren Schritt noch
        in ihre Normalform gebracht werden.}
        \lang{en}{Before we can use the quadratic formula, we have to rewrite the equation in normal form.}
        \[
        \begin{mtable}[\cellaligns{crcll}]
                         &          x^2 + 4x - 5 &\,=\,& 16  \quad & \vert -16 \\
         \Leftrightarrow &\qquad    x^2 + 4x - 21 &\,=\,& 0  &
        \end{mtable}
        \]              
        \lang{de}{Aus der Normalform können die Parameter $\,p=4\,$ und $\,q=-21\,$ dann direkt
        abgelesen und in die pq-Formel eingesetzt werden.}
        \lang{en}{Now the parameters $\,p=4\,$ and $\,q=-21\,$ can be read off of the equation directly
        and plugged into the quadratic formula.}
        \lang{de}{
        \[
        \begin{mtable}[\cellaligns{crcll}]
             &  x_{1,2} &\,=\,&-\frac{4}{2} \pm \sqrt{\big(\frac{4}{2}\big)^2-(-21)} & \\
         \Leftrightarrow &\qquad    x_{1,2} &\,=\,&-2 \pm \sqrt{25} & \\
         \Leftrightarrow &\qquad    x_1=-2 -5=-7  &\,\text{oder}\,& x_2=-2+5=3  &
        \end{mtable}
        \]}
        \lang{en}{
        \[
        \begin{mtable}[\cellaligns{crcll}]
             &  x_{1,2} &\,=\,&-\frac{4}{2} \pm \sqrt{\big(\frac{4}{2}\big)^2-(-21)} & \\
         \Leftrightarrow &\qquad    x_{1,2} &\,=\,&-2 \pm \sqrt{25} & \\
         \Leftrightarrow &\qquad    x_1=-2 -5=-7  &\,\text{or}\,& x_2=-2+5=3  &
        \end{mtable}
        \]}
        \lang{de}{Die Lösungen sind $\;x_1=-7 \;$ und $\; x_2=3.$}
        \lang{en}{The solutions are $\;x_1=-7\;$ and $\; x_2 = 3.$}

   \end{incremental}

    \tab{\lang{de}{Lösung} \lang{en}{Solution} a) ii.} 
     \begin{incremental}[\initialsteps{1}]
      \step 
        \lang{de}{Zuerst bringen wir die Gleichung durch geeignete Äquivalenzumformungen in ihre
        Normalform.}
        \lang{en}{First, we rewrite the equation in an equivalent normal form.}
        \[
        \begin{mtable}[\cellaligns{crcll}]
                         &  -24-x &\,=\,& 6x^2+23x+12     & \vert +24+x\\
         \Leftrightarrow &\qquad  0 &\,=\,& 6x^2+24x+36   & \vert :6 \\
         \Leftrightarrow &\qquad  0 &\,=\,& x^2+4x+6    &
        \end{mtable}
        \]

    \step  \lang{de}{Lösung mit pq-Formel:} \lang{en}{Solve using the quadratic formula:}
\\
       
        \lang{de}{Aus der Normalform lesen wir die Parameter $\,p=4\,$ und $\,q=6\,$ ab und 
        setzen diese in die pq-Formel ein.}
        \lang{en}{Using the normal form, we read off the parameters $\,p = 4\,$ and $\,q=6\,$ and substitute them in the quadratic formula.}
        \[
        \begin{mtable}[\cellaligns{crcll}]
             &  x_{1,2} &\,=\,&-\frac{4}{2} \pm \sqrt{\big(\frac{4}{2}\big)^2-6)} & \\
         \Leftrightarrow &\qquad    x_{1,2} &\,=\,&-2 \pm \sqrt{-2} & 
        \end{mtable}
        \]
        \lang{de}{Die Diskriminante ist $-2 <0$, folglich hat die Gleichung keine Lösung.}
        \lang{en}{The discriminant is $-2 < 0$ and therefore the equation has no solution.}

   \end{incremental}


    \tab{Lösung b) } 
        \lang{de}{Gesucht sind zwei Zahlen, die zur Zahl $20$ den gleichen Abstand $a$ haben
        und deren Quadrate summiert die Zahl $\,832\,$ ergeben. 
        Dies führt zu folgender quadratischer Gleichung:}
        \lang{en}{We are looking for two numbers, each with the same distance $a$ to $20$,
        and the sum of whose squares is $\,832\,$. This leads to the following quadratic equation:}
       
        \[
        \begin{mtable}[\cellaligns{crcll}]
                         &  (20-a)^2 + (20+a)^2 &\,=\,& 832 & \\
         \Leftrightarrow &\qquad  400-40a+a^2 + 400+40a+a^2 &\,=\,& 832 &\\
         \Leftrightarrow &\qquad  800 +2 a^2  &\,=\,& 832   & \vert -800\\
         \Leftrightarrow &\qquad  2 a^2  &\,=\,& 32         & \vert :2\\
         \Leftrightarrow & \qquad   a^2  &\,=\,& 16         &\vert \sqrt{...} \\
         \Leftrightarrow & \qquad   a   &\,=\,& \pm 4  &
        \end{mtable}
        \]
 
        \lang{de}{Die quadratische Gleichung hat also zwei Lösungen für den Abstand
        der gesuchten Zahlen zur $20$. Nur $a=4$ ist eine sinnvolle Lösung, 
        da Abstände nicht negativ sein können. Damit sind die gesuchten Zahlen
        folglich:}
        \lang{en}{The quadratic equation for the distance of the two numbers to $20$ has two solutions.
        The only reasonable solution is $a=4$, as the distance cannot be negative.
        The two numbers are:}
        \[
          \begin{mtable}[\cellaligns{cccc}]
            &  z_1=20-4 = 16,    & &  \\       
            &  z_2=20+4 = 24.    & &  
           \end{mtable}
        \]
        
  \end{tabs*} 
\end{content}

