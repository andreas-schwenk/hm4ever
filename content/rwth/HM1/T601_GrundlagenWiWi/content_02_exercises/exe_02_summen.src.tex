\documentclass{mumie.element.exercise}
%$Id$
\begin{metainfo}
  \name{
    \lang{de}{Ü02: Summen- und Produktzeichen}
    \lang{en}{Exercise 2: Sum and product symbols}
  }
  \begin{description} 
 This work is licensed under the Creative Commons License Attribution 4.0 International (CC-BY 4.0)   
 https://creativecommons.org/licenses/by/4.0/legalcode 

    \lang{de}{Summen- und Produktformeln}
    \lang{en}{Sum and product formulae}
  \end{description}
  \begin{components}
  \end{components}
  \begin{links}
    \link{generic_article}{content/rwth/HM1/T101neu_Elementare_Rechengrundlagen/g_art_content_02_rechengrundlagen_terme.meta.xml}{content_02_rechengrundlagen_terme}
  \end{links}
  \creategeneric
\end{metainfo}
\begin{content}
\title{\lang{de}{Ü02: Summen- und Produktzeichen} \lang{en}{Exercise 2: Sum and product symbols}}
\begin{block}[annotation]
	Im Ticket-System: \href{https://team.mumie.net/issues/23787}{Ticket 23787}
\end{block}


\begin{block}[annotation]
Kopie: hm4mint/T101_neu_Elementare_Rechengrundlagen

Im Ticket-System: \href{https://team.mumie.net/issues/21975}{Ticket 21975}
\end{block}

\begin{block}[annotation]
  Übung zur Berechnung von Summenformeln und Herleitung von Produktformeln 
     
\end{block}

\usepackage{mumie.ombplus}


\begin{enumerate}
  \item \lang{de}{Geben Sie jeweils den Zahlenwert der folgenden Summen an:}
        \lang{en}{Compute the following sums:}
  \begin{table}[\class{items}]

      \nowrap{a) $\, \displaystyle\sum_{j=1}^3 j (j + 1)$} & 
      \nowrap{b) $\, \displaystyle\sum_{n=1}^1 (3 n + 7)$} \\
      \nowrap{c) $\, \displaystyle\sum_{k=1}^{101} 3$} &
      \nowrap{d) $\, \displaystyle\sum_{k=1}^5 \left( \frac{1}{k} - \frac{1}{k + 1} \right)$}
    \end{table}

  \item \lang{de}{Schreiben Sie die folgenden Produkte mit Hilfe des Produktzeichens:}
        \lang{en}{Write the following products using the product symbol:}
  \begin{table}[\class{items}]

      \nowrap{a) $\, 4 \cdot 7 \cdot 10 \cdot 13$} \\ 
      \nowrap{b) $\, \frac{2}{3} \cdot \frac{4}{5} \cdot \frac{6}{7} \cdot \frac{8}{9}$} \\
      \nowrap{c) $\, 9 \cdot 16 \cdot 25 \cdot 36 \cdot 49$}
    \end{table}
 \end{enumerate} 

  

\begin{tabs*}[\initialtab{0}\class{exercise}]
  \tab{
  \lang{de}{Antworten}
  \lang{en}{Answers}
  }
\begin{table}[\class{items}]

    \nowrap{1.a) $\; 20, \qquad$ b) $\; 10, \qquad$ c) $\; 303, \qquad$ d)  $\; \frac{5}{6}$}  \\
    
    \nowrap{2.a) $\, \displaystyle\prod_{j=1}^4 (3 j + 1)$} \\ 
    \nowrap{2.b) $\, \displaystyle\prod_{j=1}^4 \frac{2 j}{2 j + 1}$} \\
    \nowrap{2.c) $\, \displaystyle\prod_{j=3}^7 \, j^2$}

  \end{table}
\lang{de}{\textbf{Bemerkung:}
Die Lösungen für das Produktzeichen (2.a)-(2.c) sind nicht eindeutig. Es könnte z.\,B. eine \emph{Indexverschiebung} vorgenommen werden, 
die den Bereich des Laufindex verändert.
}
\lang{en}{\textbf{Note:}
The solutions (2.a)-(2.c) involving the product symbol are not unique. For example, an \emph{index shift} could be used to change the range of the running index.
}
  \tab{
  \lang{de}{Lösung 1.a)}
  \lang{en}{Solution 1.a)}
  }
  
  \begin{incremental}[\initialsteps{1}]
    \step 
    \lang{de}{Die Summe läuft über $j=1, 2, 3$. Das heißt, wir setzen diese Werte für $j$ ein und schreiben zunächst die Summe aus, um sie anschließend zu berechnen:}
    \lang{en}{The sum runs through $j=1, 2, 3$. This means that we plug in these values for $j$ and write out the sum in order to compute it:}
    \step
    \begin{equation*}\sum_{j=1}^3 j(j+1) = 1(1+1)+2(2+1)+3(3+1) = 2+ 2\cdot 3 + 3\cdot 4 = 2+6+12 =20.
    \end{equation*} 
    
  \end{incremental}
  

  \tab{
  \lang{de}{Lösung 1.b)}
  \lang{en}{Solution 1.b)}
  }
  
  \begin{incremental}[\initialsteps{1}]
    \step 
    \lang{de}{Die Summe läuft von $n=1$ bis $1$. Das bedeutet, dass sie nur aus einem einzigen Summanden besteht, den wir durch Einsetzen von $n=1$ erhalten.}
    \lang{en}{The sum runs from $n=1$ to $1$. This means that it contains only a single summand, which we obtain by setting $n=1$.}
     \step
     \lang{de}{Also}
     \lang{en}{So}
    
    \begin{equation*}
     \sum_{n=1}^1 (3n +7) = 3\cdot 1 +7 = 10. 
    \end{equation*}
    
    
  \end{incremental}
  
 
 \tab{
  \lang{de}{Lösung 1.c)}
  \lang{en}{Solution 1.c)}
  }
  
  \begin{incremental}[\initialsteps{1}]
    \step 
    \lang{de}{Der Laufindex $k$, der von $1$ bis $101$ läuft, kommt in der Summe nicht als Variable vor, d.\,h. jeder Summand ist gleich, nämlich $3$. }
    \lang{en}{The index $k$ runs from $1$ to $101$ but does not appear in the sum as a variable. This means that all summands are the same value $3$.}
     
     \step
     \lang{de}{Damit erhält man
     
    \begin{equation*}
       \sum\limits_{k=1}^{101} 3 = \underbrace{3+3+\cdots+3}_{101-\text{mal}} = 101\cdot 3 = 303.
    \end{equation*}
}

\lang{en}{Therefore,
     
    \begin{equation*}
       \sum\limits_{k=1}^{101} 3 = \underbrace{3+3+\cdots+3}_{101 \, \text{times}} = 101\cdot 3 = 303.
    \end{equation*}
}
\end{incremental}

\tab{
  \lang{de}{Lösung 1.d)}
  \lang{en}{Solution 1.d)}
  }
  
  \begin{incremental}[\initialsteps{1}]
    \step 
    \lang{de}{Setzen wir in die Summanden die Werte $k=1, \ldots , 5$ ein, so erhalten wir}
    \lang{en}{When we plug the values $k = 1, \ldots, 5$ into the summands, we obtain}
     
    
  \begin{equation*}
  	\sum_{k=1}^5 \left(\frac{1}{k}-\frac{1}{k+1}\right)= \left(\frac{1}{1}  \textcolor{#00CC00}{ - \frac{1}{2}}  \right) +
  	\left( \textcolor{#00CC00}{\frac{1}{2}}  \textcolor{#0066CC}{- \frac{1}{3}} \right) +
  	\left( \textcolor{#0066CC}{\frac{1}{3}}  \textcolor{#CC6600}{- \frac{1}{4}} \right) +
    \left( \textcolor{#CC6600}{\frac{1}{4}}  \textcolor{#CC00CC}{- \frac{1}{5}} \right)+
	\left( \textcolor{#CC00CC}{\frac{1}{5}} - \frac{1}{6} \right) = 1-\frac{1}{6} =\frac{5}{6}.
  \end{equation*}

	\lang{de}{Hinweis: Nebeneinanderstehende Summanden heben sich gegenseitig weg, sodass nur der erste
	und der letzte Summand übrig bleiben. Summen dieser Form nennt man \textit{Teleskopsummen}.}
 \lang{en}{Note: any two consecutive summands cancel each other, such that only the first and the last summand remain in the sum.
 Sums of this form are called \textit{telescoping sums}.}
	
    
    \lang{de}{Es gibt aber noch einen alternativen Weg, diese Summe zu berechnen.}
    \lang{en}{There is an alternative way to compute this sum, however.}
    \step
    \lang{de}{Nach dem Kommutativgesetz
    und dem Assoziativgesetz können wir die Summe in zwei Summen aufteilen,nämlich}
    \lang{en}{Using the commutative and associative axioms, we can split this sum into two parts:}
  \begin{equation*}
  	\sum_{k=1}^5 \left(\frac{1}{k}-\frac{1}{k+1}\right)= 
    \left(\sum_{k=1}^5 \frac{1}{k}\right) - \left(\sum_{k=1}^5\frac{1}{k+1}\right).
  \end{equation*} 
  \step
    \lang{de}{Dann ziehen wir den ersten Summanden aus der ersten Summe heraus und verschieben 
    den Index um 1. Das ergibt}
    \lang{en}{Then we pull the first summand out of the first sum and shift the index by 1. This yields}
  \begin{equation*}
  	\sum_{k=1}^5 \frac{1}{k}
    = \frac{1}{1} + \sum_{k=2}^5 \frac{1}{k}    
    = 1 + \sum_{k=1}^4 \frac{1}{k+1}.
  \end{equation*} 
   \lang{de}{Bei der zweiten Summe ziehen wir den letzten Summanden heraus und erhalten
  }
  \lang{en}{In the second sum, we pull out the final summand and obtain}
  \begin{equation*}
  	\sum_{k=1}^5\frac{1}{k+1}
    = \left(\sum_{k=1}^4\frac{1}{k+1}\right) + \frac{1}{5+1}  
    = \sum_{k=1}^4 \frac{1}{k+1} + \frac{1}{6}.
  \end{equation*} 
  
  \step \lang{de}{Insgesamt führt dies zu folgender Rechnung}
  \lang{en}{Altogether, this leads to the following calculation:}
  \begin{eqnarray*}
  	    \sum_{k=1}^5 \left(\frac{1}{k}-\frac{1}{k+1}\right)
    &=&  \left(\sum_{k=1}^5 \frac{1}{k}\right) - \left(\sum_{k=1}^5\frac{1}{k+1}\right)\\
    &=& \left(1 + \sum_{k=1}^4 \frac{1}{k+1}\right) - \left(\sum_{k=1}^4 \frac{1}{k+1} + \frac{1}{6}\right)\\
    &=& 1- \frac{1}{6} \; + \cancel{\left( \sum_{k=1}^4 \frac{1}{k+1}\right)} -\cancel{\left( \sum_{k=1}^4 \frac{1}{k+1}\right)}\\
    &=& \frac{5}{6}.
  \end{eqnarray*} 
  
    
\end{incremental}

%%%%%%%% Teil 2.

  \tab{
  \lang{de}{Lösung 2.a)}
  \lang{en}{Solution 2.a)}
  }
  
  \begin{incremental}[\initialsteps{1}]
    \step 
    \lang{de}{Wir stellen fest, dass die Differenz zweier aufeinanderfolgender Faktoren stets $3$ ist.
	Außerdem erhalten wir die Zahl $4$ als $1+3$, analog dann $7=1+2\cdot 3$ usw.}
    \lang{en}{We observe that the difference between two consecutive factors is always $3$.
    Also, we can write $4$ as $1+3$, so that $7=1+2 \cdot 3$ and so on.}
    \step
    \lang{de}{Damit ist eine mögliche Schreibweise für das angegebene Produkt}
    \lang{en}{Therefore, one possible way to write the given product is}
    \begin{equation*}
 4\cdot 7\cdot 10\cdot 13= \prod_{j=1}^4 (3 j + 1).
\end{equation*}
    
  \end{incremental}
  

  \tab{
  \lang{de}{Lösung 2.b)}
  \lang{en}{Solution 2.b)}
  }
  
  \begin{incremental}[\initialsteps{1}]
    \step 
    \lang{de}{In den Zählern werden die ersten vier geraden natürlichen Zahlen aufmultipliziert, welche in der Form $2\cdot n$ für eine 
	natürliche Zahl $n$ geschrieben werden können. So ist beispielsweise $2=2\cdot 1$ und $4=2\cdot 2$ usw. 
	In den Nennern werden vier ungerade Zahlen aufmultipliziert, welche in der Form $2\cdot m +1$ (für eine natürliche Zahl $m$)
	geschrieben werden können. Es ist zum Beispiel $3=2\cdot 1 +1$ und $5=2\cdot 2 +1$.}
    \lang{en}{In the numerators, we have the first four even natural numbers, which can be written in the form $2 \cdot n$ 
    for a natural number $n$. For example, $2 = 2 \cdot 1$ and $4 = 2 \cdot 2$ and so on.
    In the denominators, we are multiplying four odd numbers, which can be written in the form $2 \cdot m + 1$ for a natural number $m$.
    For example, $3 = 2 \cdot 1 + 1$ and $5 = 2 \cdot 2 + 1$.}
     \step
     \lang{de}{Damit erhalten wir eine mögliche Schreibweise für das vorliegende Produkt durch}
    \lang{en}{Therefore, one way to write the given product is}
    \begin{equation*}
	\prod_{j=1}^4 \frac{2 j}{2 j + 1}.
	\end{equation*}
    
    
  \end{incremental}
  
 
 \tab{
  \lang{de}{Lösung 2.c)}
  \lang{en}{Solution 2.c)}
  }
  
  \begin{incremental}[\initialsteps{1}]
    \step 
    \lang{de}{Es liegt ein Produkt von \textit{Quadratzahlen} vor, also von Zahlen der Form $a^2$ für natürliche
	Zahlen $a$. Beginnend beim Quadrat der Zahl $3$ multipliziert man aufsteigend bis zum
	Quadrat der Zahl $7$ alle zwischenliegenden Quadratzahlen auf. }
 \lang{en}{We are given a product of \textit{square numbers}, i.e. numbers of the form $a^2$ with a natural number $a$.
 Beginning with the square of $3$, we multiply all square numbers up to the square of $7$.}

     
     \step
     \lang{de}{Damit ist}
     \lang{en}{Therefore,}
    \begin{equation*}
 	    9 \cdot 16 \cdot 25 \cdot 36 \cdot 49 = \prod_{j=3}^7 \, j^2.
	\end{equation*}
    
    \lang{de}{Diese Lösung ist aber nicht eindeutig.}
    \lang{en}{This solution is not unique, though.}
    \step
    \lang{de}{Durch eine sogenannte \emph{Indexverschiebung} kann der Bereich des Laufindex
    verändert werden. Man erhält zum Beispiel}
    \lang{en}{By \emph{shifting the index}, the range of the running index can be changed. For example, we have}
    \begin{equation*}
 	    9 \cdot 16 \cdot 25 \cdot 36 \cdot 49 = \prod_{j=1}^5 \, (j+2)^2.
	\end{equation*}

    
\end{incremental}

\end{tabs*}
\end{content}