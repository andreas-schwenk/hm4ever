\documentclass{mumie.element.exercise}
%$Id$
\begin{metainfo}
  \name{
    \lang{de}{Ü03: lineare Gleichungen}
    \lang{en}{Ü03: linear equations}
  }
  \begin{description} 
 This work is licensed under the Creative Commons License Attribution 4.0 International (CC-BY 4.0)   
 https://creativecommons.org/licenses/by/4.0/legalcode 

    \lang{de}{Lösen linearer Gleichungen}
    \lang{en}{Solving linear equations}
  \end{description}
  \begin{components}
  \end{components}
  \begin{links}
\link{generic_article}{content/rwth/HM1/T101neu_Elementare_Rechengrundlagen/g_art_content_02_rechengrundlagen_terme.meta.xml}{content_02_rechengrundlagen_terme}
\link{generic_article}{content/rwth/HM1/T101neu_Elementare_Rechengrundlagen/g_art_content_05_loesen_gleichungen_und_lgs.meta.xml}{content_05_loesen_gleichungen_und_lgs}
\end{links}
  \creategeneric
\end{metainfo}
\begin{content}
\title{\lang{de}{Ü03: lineare Gleichungen} \lang{en}{Exercise 3: linear equations}}
\begin{block}[annotation]
	Im Ticket-System: \href{https://team.mumie.net/issues/23788}{Ticket 23788}
\end{block}


\begin{block}[annotation]
Kopie: hm4mint/T101_neu_Elementare_Rechengrundlagen/exercise 11

Im Ticket-System: \href{https://team.mumie.net/issues/21981}{Ticket 21981}
\end{block}

 \begin{block}[annotation]
%
   Übung zu linearen Gleichungen und deren Lösung
   \\
   Teile dieser Aufgabe entstammen aus dem OMB+ \\
   (Kapitel II Gleichungen in einer Unbekannten) 
%
\end{block}

 
%
% Aufgabenstellung
%
\begin{enumerate}[alph]
%
  \item \lang{de}{Prüfen Sie, ob es sich bei den folgenden Gleichungen um  
        lineare Gleichungen
        der Form $bx+c=0$ mit $b,c \in \R$ und $b \neq 0$ handelt und bestimmen Sie ihre Lösungsmenge.
        }
        \lang{en}{Decide whether the following equations are 
        linear equations
        of the form $bx+c=0$ with $b, c \in \R$ and $b \neq 0$, and determine their solutions.
        }
    \begin{table}[\class{items}]
     \nowrap{i.  $\; 8x+17=81$}  &                              
     \nowrap{ii. $\; -7(-x-3)=(-6+21x)\cdot \frac{1}{3} $}\\    
     \nowrap{iii.$\; \big((x+3)2+4\big)5-20=40$}&               
     \nowrap{iv. $\; -2-2x=-(x+1)^2+(x+1)(x-1)$} \\                     
    \end{table}
%
  \item \lang{de}{Stellen Sie Gleichungen zu den Textaufgaben auf und lösen Sie diese.}
        \lang{en}{Write equations for these word problems and solve them.}
    \lang{de}{
    \begin{enumerate}[roman]
      \item Eine Wandergruppe wandert 2 Tage. Am 1. Tag legen sie ein Viertel 
        der Gesamtstrecke und zusätzlich 2 km zurück. Am 2. Tag legen sie die 
        Hälfte der Gesamtstrecke und zusätzlich 1 km zurück.\\
        Wie lang ist die zurückgelegte Gesamtstrecke?\\

        \item Vier Freunde fahren zusammen in Urlaub. Sie teilen sich die 
        Benzinkosten für die Fahrt. Für die Hinfahrt zahlt jeder 2,40 €. 
        Auf der Rückfahrt sind sie nur noch zu dritt, es fallen aber dieselben 
        Benzinkosten an wie für die Hinfahrt. Wie viel kostet die 
        Rückfahrt pro Person?
      \end{enumerate}
    }
  \lang{en}{
    \begin{enumerate}[roman]
        \item A hiking group hikes for 2 days. On the first day, they cover 2 km more than a quarter of the entire distance.
        On the second day, they cover 1 km more than half of the entire distance.
        What distance did they hike?
        
      
        \item Four friends are driving for a holiday and splitting the cost of fuel.
        On the way there, each of them has to pay 2.40 €.
        On the return trip, only three of them drive together, while the total fuel cost remains the same.
        How much does the return trip cost per person?
    \end{enumerate}
}

\end{enumerate}
%
% Lösungen
%
  \begin{tabs*}[\initialtab{0}\class{exercise}]
  
    \tab{\lang{de}{Antworten} \lang{en}{Answers}}
      \begin{enumerate}[alph]
        \item 
         \begin{enumerate}[roman]
          \item $\mathbb{L}= \{ 8 \}$
          \item \lang{de}{Die Gleichung lässt sich nicht in die Form $bx+c=0$ mit $b \neq 0$ umformen und
                wird für keine reelle Zahl $x$ zu einer wahren Aussage. Es gilt daher }
                \lang{en}{The equation can not be written in the form $bx+c=0$ with $b \neq 0$,
                and is not satisfied by any real number $x$. Therefore,}
                $\mathbb{L}= \emptyset.$
          \item $\mathbb{L}= \{ 1 \}$
          \item \lang{de}{Die Gleichung lässt sich nicht in die Form $bx+c=0$ mit $b \neq 0$ umformen, ist
                aber unabhängig von $x$ immer erfüllt. Es gilt daher $\mathbb{L}= \R.$}
                \lang{en}{The equation can not be written in the form $bx+c=0$ with $b \neq 0$,
                and is always satisfied regardless of $x$. Therefore, $\mathbb{L}=\R.$}
         \end{enumerate}
        
        \item 
        \lang{de}{
         \begin{enumerate}[roman]
            \item Die Gesamtstrecke ist $12$ Kilometer lang.        
            \item Für die Rückfahrt zahlt jeder 3,20 €.
          \end{enumerate}
          }
          \lang{en}{
          \begin{enumerate}[roman]
            \item The hiking distance is $12$ kilometers.        
            \item Each has to pay 3.20 €.
          \end{enumerate}
          }
     \end{enumerate}

     
    \tab{\lang{de}{Lösung} \lang{en}{Solution} a) i.} 
     \begin{incremental}[\initialsteps{1}]
      \step 
        \lang{de}{Subtrahiert man auf beiden Seiten der Gleichung $81$, 
        erhält man $8x-64=0$, also eine lineare Gleichung gemäß 
        der gewünschten Form. Der Leitkoeffizient ist hier $b=8 \neq 0.$}
        \lang{en}{When we subtract $81$ from both sides of the equation, we obtain
        $8x-64=0$, which is a linear equation of the desired form. The leading coefficient here is $b=8 \neq 0$.}
      \step
        \lang{de}{Wir lösen die Gleichung weiter nach $x$ auf:}
        \lang{en}{We solve the equation for $x$:}
        \[
        \begin{mtable}[\cellaligns{crcll}]
            & 8x-64  &\,=\,& 0 \quad &\vert +64 \\
         \Leftrightarrow &\qquad 8x  &\,=\,& 64 \quad &\vert :8 \\
         \Leftrightarrow & \qquad x &\,=\,& 8  &
        \end{mtable}
        \]

      \lang{de}{Die Lösung ist also} \lang{en}{The solution is} $x = 8$.
 
  \end{incremental}
   
    \tab{\lang{de}{Lösung} \lang{en}{Solution} a) ii.} 

        \lang{de}{Wir lösen zunächst die Klammern rechts und links des Gleichheitszeichens 
        nach dem Distributivgesetz auf:}
        \lang{en}{We clear the parentheses to the right and left of the equals sign using the distributive axiom:}
        \[
        \begin{mtable}[\cellaligns{crcll}]
                         &       -7(-x-3) &\,=\,& (-6+21x)\cdot \frac{1}{3} &\\
         \Leftrightarrow &\qquad 7x + 21 &\,=\,& -2 + 7x & \vert +2\\
         \Leftrightarrow &\qquad  7x + 23 &\,=\,&  7x & \vert -7x\\
         \Leftrightarrow &\qquad    23 &\,=\,& 0 &
         \end{mtable}
        \]
        \lang{de}{Die Äquivalenzumformung der Gleichung führt zu der falschen Aussage $\,23=0$.
        Das bedeutet, dass auch die Ausgangsgleichung für keine reelle Zahl $x$ lösbar
        ist. Daher ist die Lösungsmenge $\mathbb{L}= \emptyset.$}
        \lang{en}{This shows that the equation is equivalent to the false proposition $\,23=0$.
        This equation is not satisfied by any real number $x$. Therefore, the solution set is $\mathbb{L} = \emptyset$.}
        
   
    \tab{\lang{de}{Lösung} \lang{en}{Solution} a) iii.} 
     \begin{incremental}[\initialsteps{1}]
      \step \lang{de}{Zuerst werden die Klammerterme von innen nach außen aufgelöst:}
      \lang{en}{First we clear the parentheses, beginning with the innermost pair:}
        \[
        \begin{mtable}[\cellaligns{crcll}]
                         &       \big((x+3)2+4\big)\cdot 5-20 &\,=\,& 40 &\\
         \Leftrightarrow &\qquad \big(2x+6+4\big)\cdot 5-20 &\,=\,& 40 &\\
         \Leftrightarrow &\qquad \big(2x+10\big)\cdot 5-20 &\,=\,& 40 &\\
         \Leftrightarrow &\qquad 10x+50-20 &\,=\,& 40 &\\
         \Leftrightarrow &\qquad 10x+30 &\,=\,& 40 &
         \end{mtable}
        \]
        \lang{de}{Subtrahiert man auf beiden Seiten der Gleichung die Zahl $40$, so erhält man
        mit \[10x-10=0 \] eine lineare Gleichung der gewünschten Form. Der 
        Leitkoeffizient ist $b=10 \neq 0.$}
        \lang{en}{After subtracting $40$ from both sides of the equation, we obtain \[10x-10=0,\]
        which is an equation of the desired form. The leading coefficient is $b=10 \neq 0.$
  }
      
     \step \lang{de}{Anschließend wird die Variable $x$ isoliert, d.h. die Gleichung wird
        nach $x$ aufgelöst:}
        \lang{en}{Finally, we isolate the variable $x$ to solve the equation:}
        \[
        \begin{mtable}[\cellaligns{crcll}]
                         &\qquad 10x+30 &\,=\,& 40 &\qquad \vert -30\\
         \Leftrightarrow &\qquad   10x &\,=\,& 10 &\qquad \vert :10\\
         \Leftrightarrow &\qquad     x &\,=\,& 1 &
         \end{mtable}
        \]
        \lang{de}{Die Lösung ist also} \lang{en}{The solution is} $x = 1$.
     \end{incremental}
   
    \tab{\lang{de}{Lösung} \lang{en}{Solution} a) iv.} 
         
        \lang{de}{Um zu prüfen, ob es sich bei der vorliegenden Gleichung um eine lineare 
        Gleichung handelt, wenden wir zunächst die binomischen Formeln an. Für den ersten Summanden verwenden wir 
        die 1. binomische Formel und für den 2. Summanden die 3. binomische Formel:}
        \lang{en}{To check whether the given equation is a linear equation, we first use the binomial formulas.
        For the first summand, we use the first binomial formula, and for the second summand, we use the third binomial formula:}
        \[
        \begin{mtable}[\cellaligns{crcll}]
                         &\qquad -2-2x &\,=\,& -(x+1)^2+(x+1)(x-1) &\\
         \Leftrightarrow &\qquad -2-2x &\,=\,&  -(x^2+2x+1)+(x^2-1)&\\
         \end{mtable}
        \]
        \lang{de}{Anschließend lösen wir die Klammern auf und sortieren die Summanden nach 
        den Potenzen von $x$:}
        \lang{en}{Then we clear the parentheses and sort the summands according to the exponent of $x$:}
        \[
        \begin{mtable}[\cellaligns{crcll}]
                         &\qquad -2-2x &\,=\,& -(x^2+2x+1)+(x^2-1)  &\\
         \Leftrightarrow &\qquad -2-2x &\,=\,&  -x^2 + x^2 -2x -1-1 &\\
         \Leftrightarrow &\qquad -2-2x &\,=\,&  -2x -2              &\vert +2x+2 \\
         \Leftrightarrow &\qquad 0 \cdot x &\,=\,& 0 &  
        \end{mtable}
        \]

        \lang{de}{Da sich die Summanden mit $x^2$ aufheben, ist die Gleichung zwar \textit{linear}, 
        aber mit Leitkoeffizient $\,b=0.\,$ Sie entspricht also nicht der gewünschten Form
        und ist daher auch nicht eindeutig lösbar.\\ }
        \lang{en}{Since the summands with $x^2$ cancel, the equation is \textit{linear},
        but with leading coefficient $\,b=0.\,$ It is therefore not of the desired form and also does not have a unique solution.\\ }
        
        
        \lang{de}{Im Gegensatz zu Teilaufgabe ii. ist das Ergebnis der Äquivalenzumformung hier jedoch
        eine wahre Aussage, nämlich $\,0=0$. Das bedeutet, dass die Ausgangsgleichung 
        für jede beliebige reelle Zahl $x$ ebenfalls wahr ist. Somit ist die Lösungsmenge $\mathbb{L}= \R.$}
        \lang{en}{Unlike part (ii) of this exercise, the equivalent equation here is a true statement, namely $\,0=0$.
        This means that the original equation is also true, for every arbitrary real number $x$.
        Therefore, the solution set is $\mathbb{L} = \R.$}
        
   
   
    \tab{\lang{de}{Lösung} \lang{en}{Solution} b) i.} 
     \begin{incremental}[\initialsteps{1}]
      \step 
        \lang{de}{Es wird nach der Gesamtstrecke gefragt, daher lassen sich die einzelnen 
        Abschnitte pro Tag in einer Gleichung über die Gesamtstrecke $x$ zusammenfassen:}
        \lang{en}{Since we are being asked for the total distance, we add the segments covered each day to an equation for the total distance $x$:}
      \step 
        \[
        \begin{mtable}[\cellaligns{crcll}]
            &(\frac{1}{4}x+2)+(\frac{1}{2}x+1) &\,=\,& x &\\
         \Leftrightarrow &\qquad \frac{3}{4}x+3    &\,=\,& x \quad &\vert -\frac{3}{4}x\\
         \Leftrightarrow & \qquad 3 &\,=\,& \frac{1}{4}x \quad &\vert \cdot 4\\
         \Leftrightarrow & \qquad 12 &\,=\,& x &
        \end{mtable}
        \]
        \lang{de}{Die Gesamtstrecke beträgt also 12 Kilometer.}
        \lang{en}{The total distance is 12 kilometers.}

   \end{incremental}
   
    \tab{\lang{de}{Lösung} \lang{en}{Solution} b) ii.} 
        \lang{de}{Gesucht sind die Kosten $k$ pro Person für die Rückfahrt, wenn die 
        Gesamtkosten für die Rückfahrt den Gesamtkosten für die Hinfahrt 
        entsprechen.}
        \lang{en}{We are looking for the cost $k$ per person for the return trip,
        given that the total cost is the same in both directions.}
        
        \[
        \begin{mtable}[\cellaligns{crcll}]
                         &  3\cdot k &\,=\,& 4 \cdot 2,4  & \vert :3\\
         \Leftrightarrow &\qquad  k &\,=\,& \displaystyle{\frac{9,6}{3}} \quad &\\
         \Leftrightarrow & \qquad k &\,=\,& 3,2 \quad &
        \end{mtable}
        \]
 
        \lang{de}{Für die Rückfahrt zahlt jeder der drei Freunde 3,20 €.}
        \lang{en}{Each of the three friends pays 3.20 € for the return trip.}
   
  \end{tabs*}


\end{content}

