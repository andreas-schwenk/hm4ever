\documentclass{mumie.element.exercise}
%$Id$
\begin{metainfo}
  \name{
    \lang{de}{Ü01: Terme}
    \lang{en}{Exercise 1}
  }
  \begin{description} 
 This work is licensed under the Creative Commons License Attribution 4.0 International (CC-BY 4.0)   
 https://creativecommons.org/licenses/by/4.0/legalcode 

    \lang{de}{Terme vereinfachen}
    \lang{en}{}
  \end{description}
  \begin{components}
  \end{components}
  \begin{links}
  \link{generic_article}{content/rwth/HM1/T101neu_Elementare_Rechengrundlagen/g_art_content_02_rechengrundlagen_terme.meta.xml}{content_02_rechengrundlagen_terme}
  \end{links}
  \creategeneric
\end{metainfo}
\begin{content}
\title{\lang{de}{Ü01: Terme} \lang{en}{Ü01: Terms}}
\begin{block}[annotation]
	Im Ticket-System: \href{https://team.mumie.net/issues/23790}{Ticket 23790}
\end{block}

\begin{block}[annotation]
Kopie: hm4mint/T101_neu_Elementare_Rechengrundlagen/exercise 4

Im Ticket-System: \href{https://team.mumie.net/issues/21974}{Ticket 21974}
\end{block}

\begin{block}[annotation]
  Übung zur Vereinfachung von Termen unter Anwendung 
  der Kommutativ-, Assoziativ- und Distributivgesetze
  sowie der binomischen Formeln
     
\end{block}

\usepackage{mumie.ombplus}



\lang{de}{Vereinfachen Sie die folgenden Terme so weit wie möglich:}
\lang{en}{Simplify the following terms as much as possible:}
\begin{table}[\class{items}]
  \nowrap{a) $\, 2x(1-x)-(2-x)$}\\
   \nowrap{b) $\, (x-1)(2-x)-(3x+x^2)+x\cdot 3$}\\
   \nowrap{c) $\, (5+x)(5-x)+(3-x)^2$}\\
   \nowrap{d) $\, (x+y)^2-(x-1)^2-(y-1)^2$}
\end{table}

\begin{tabs*}[\initialtab{0}\class{exercise}]
  \tab{
  \lang{de}{Antwort}
  \lang{en}{Answer}
  }

  \lang{de}{
\begin{table}[\class{items}]

    \nowrap{a) $\, -2x^2+3x-2$} \\ \nowrap{b) $\ -2x^2+3x-2$} \\
    \nowrap{c) $\, -6x+34$} \\  \nowrap{d) $\ 2(xy+x+y-1)\quad \text{oder} \quad 2xy+2x+2y-2$}  \\
\end{table}
}
\lang{en}{
\begin{table}[\class{items}]

    \nowrap{a) $\, -2x^2+3x-2$} \\ \nowrap{b) $\ -2x^2+3x-2$} \\
    \nowrap{c) $\, -6x+34$} \\  \nowrap{d) $\ 2(xy+x+y-1)\quad \text{or} \quad 2xy+2x+2y-2$}  \\
\end{table}
}

  \tab{
  \lang{de}{Lösung a)}
  \lang{en}{Solution a)}
  }
  
  \begin{incremental}[\initialsteps{1}]
    \step \lang{de}{Zunächst werden alle Terme gemäß dem 
    Distributivgesetz
    ausmultipliziert:}

    \[ 2x(1-x)-(2-x)=2x\cdot 1-2x\cdot x -2-(-x)=2x-2x^2-2+x.\]
    \step \lang{de}{Dann werden die Summanden nach $x$-Potenzen sortiert und zusammengefasst:}
    \lang{en}{The summands can be sorted by their powers of $x$ and added:}
    \[ 2x-2x^2-2+x=-2x^2+2x+x-2=-2x^2+3x-2. \]

  \end{incremental}

  \tab{
  \lang{de}{Lösung b)}
  \lang{en}{Solution b)}
  }
  
  \begin{incremental}[\initialsteps{1}]
    \step \lang{de}{Zunächst werden alle Terme gemäß dem 
        Distributivgesetz
        ausmultipliziert:}
    \lang{en}{First, all terms can be multiplied out with the distributive law:}
    \begin{eqnarray*}
    (x-1)(2-x)-(3x+x^2)+x\cdot 3&=&x\cdot (2-x)-1\cdot (2-x)-3x-x^2+3x\\
     &=&x\cdot 2-x\cdot x-2-(-x)-3x-x^2+3x \\
    &=&2x-x^2-2+x-3x-x^2+3x. \end{eqnarray*}
 	\step \lang{de}{Dann werden die Summanden nach $x$-Potenzen sortiert und zusammengefasst:}
 	\lang{en}{Then the summands can be sorted by their powers of $x$ and added:}
 	\[ 2x-x^2-2+x-3x-x^2+3x=-x^2-x^2+2x+x-3x+3x-2=-2x^2+3x-2. \]
  \end{incremental}

\tab{
  \lang{de}{Lösung c)}
  \lang{en}{Solution c)}
  }
  
%  \begin{incremental}[\initialsteps{1}]
%    \step 
\lang{de}{
        Unter Verwendung der zweiten und dritten 
        binomischen Formel
        vereinfacht sich der Term zu
        }
\lang{en}{
        Using the second and third binomial formula, this term simplifies to
}
        
    \begin{eqnarray*}
    (5+x)(5-x)+(3-x)^2 &=& 5^2-x^2 + 3^2-2\cdot 3\cdot x+x^2 \\
    &=& 25-x^2+9-6x+x^2 =-6x+34 .
    \end{eqnarray*}
 
%  \end{incremental}

\tab{
  \lang{de}{Lösung d)}
  \lang{en}{Solution d)}
  }
  
% \begin{incremental}[\initialsteps{1}]
%    \step 
\lang{de}{
        Unter Verwendung der ersten und zweiten 
        binomischen Formel
        vereinfacht sich der Term zu
        }
\lang{en}{
        Using the first and second binomial formula, this term simplifies to
}
    \begin{eqnarray*}
    (x+y)^2-(x-1)^2-(y-1)^2&=& x^2+2xy+y^2 - (x^2-2x+1) - (y^2-2y+1) \\
    &=& x^2+2xy+y^2-x^2+2x-1-y^2+2y-1\\
    &=& x^2-x^2+2xy+2x+y^2-y^2+2y-1-1\\
    &=& 2xy+2x+2y-2 .
    \end{eqnarray*}
    \lang{de}{
    Im letzten Term kann man noch den Faktor $\,2\,$ nach dem Distributivgesetz 
    ausklammern und erhält
    }
    \lang{en}{In the last term, we can factor out $\,2\,$ using the distributive law to obtain

    }
    \[
    2(xy+x+y-1) .
    \]
    \lang{de}{Welcher der beiden Ausdrücke nun einfacher ist, ist Ansichtssache.}
    \lang{en}{Which of these expressions is simpler is a matter of opinion.}
%  \end{incremental}


  %... other tabs

\end{tabs*}
\end{content}