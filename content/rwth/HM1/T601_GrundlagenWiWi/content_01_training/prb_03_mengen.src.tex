\documentclass{mumie.problem.gwtmathlet}
%$Id$
\begin{metainfo}
 \name{
  \lang{de}{A03: Intervalle}
  }
  \begin{description} 
 This work is licensed under the Creative Commons License Attribution 4.0 International (CC-BY 4.0)   
 https://creativecommons.org/licenses/by/4.0/legalcode 

    \lang{de}{...}
  \end{description}
  \corrector{system/problem/GenericCorrector.meta.xml}
  \begin{components}
    \component{js_lib}{system/problem/GenericMathlet.meta.xml}{gwtmathlet}
  \end{components}
  \begin{links}
  \end{links}
  \creategeneric
\end{metainfo}

\begin{content}
\lang{de}{
	\title{A03: Intervalle}
}
\lang{en}{
\title {A03: intervals}
}
\begin{block}[annotation]
	Im Ticket-System: \href{https://team.mumie.net/issues/23732}{Ticket 23732}
\end{block}

\begin{block}[annotation]
Kopie: hm4mint/T101_neu_Elementare_Rechengraundlagen/training 3

Im Ticket-System: \href{https://team.mumie.net/issues/22285}{Ticket 22285}
\end{block}

\begin{block}[annotation]
	Intervalle, Ordnungen, Brüche, Beträge
\end{block}

\usepackage{mumie.genericproblem}



\begin{problem}

% Q1
  \begin{question}
   \lang {de}{\text{In welchem der folgenden Intervalle liegt die Zahl $\var{a}$? \\
        W"ahlen Sie alle richtigen Antworten aus.}}
   \lang {en}{\text {In which of the following intervals does the number $\var{a}$ lie? \\
        Choose all the correct answers.}}    

    \explanation{}

    \permutechoices{1}{3}

    \type{mc.multiple}
    \field{rational}

    \begin{variables}
      %\randrat{a}{9}{20}{2}{6}
      \randint{a1}{9}{20}
      \randint{a2}{2}{6}
      \randint{c}{3}{7}
      \randint{d}{-2}{4}
      \randint{f}{0}{1}
      \function[calculate]{l1}{floor(a-c)}
      \function[calculate]{r1}{floor(a+d)}
      \function[calculate]{l2}{floor(a)-f+1}
      \function[calculate]{r2}{floor(a)+c}
      \function[calculate]{l3}{floor(a)-2}
      \function[calculate]{r3}{floor(a+d+0.5)+1}
      \function[normalize]{a}{a1/a2}
      \function[calculate]{t}{floor(a)}
      \randadjustIf{a2}{a=t}                        % make sure that 'a' is not an integer
      \randadjustIf{c,d}{a>r1 AND a<=l2 AND a>=r3}  % make sure there is a true solution
    \end{variables}

    \begin{choice}
      \lang{de}{\text{$[\var{l1};\var{r1}]$}}
      \lang{en}{\text{$[\var{l1},\var{r1}]$}}
      \solution{compute}
    \iscorrect{a}{<=}{r1}
    \end{choice}
    \begin{choice}
      \lang{de}{\text{$(\var{l2};\var{r2}]$}}
      \lang{en}{\text{$(\var{l2},\var{r2}]$}}
      \solution{compute}
      \iscorrect{a}{>}{l2}
    \end{choice}
    \begin{choice}
      \lang{de}{\text{$[\var{l3};\var{r3})$}}
      \lang{en}{\text{$[\var{l3},\var{r3})$}}
      \solution{compute}
      \iscorrect{a}{<}{r3}
    \end{choice}
  \end{question}
%
% Q2
%

  \begin{question}
  
    \type{input.interval}
    \field{rational}

    \lang {de} {\text{Schreiben Sie die folgende Menge in Intervallschreibweise 
        bzw. als Vereinigung von Intervallen und prüfen Sie, ob es sich 
        dabei um offene, halboffene oder geschlossene Intervalle handelt. 
        Ändern Sie diese ggf. durch anklicken.
        (Schreiben Sie "`\textit{infinity}"' für $\infty$.) \\
        }}
    \lang {en}{\text {Write the following sets in interval notation 
        or as a union of intervals and check whether they are open, half-open or closed intervals. 
       If necessary, change them by clicking on them.
        (Write "\textit{infinity}" for $\infty$.) \\ }}  
%        
    \begin{variables}
      \randint{a1}{1}{6}
      \randint{a2}{2}{7}
      \randadjustIf{a2}{a1 >= a2}
      \function[normalize]{a}{(a1/a2)^2}

      \randint{b1}{2}{11}
      \randint{b2}{1}{8}
      \randadjustIf{b2}{b1 < b2}
      \function[normalize]{b}{b1/b2}
      \function{f}{(x-b)^2}

      \randint{c1}{1}{10}
      \randint{c2}{1}{10}
      \function[normalize]{c}{(c1/c2)^2}
      
      \function[normalize]{l1}{-c1/c2}
      \function[normalize]{r1}{c1/c2}

      \function[normalize]{l2}{-(a1/a2)+b}
      \function[normalize]{r2}{(a1/a2)+b}
    \end{variables}
%
    \begin{answer}
          \text{$\{x \in \R \, \vert \, x \geq \var{c}\}=$}    
          \allowIntervalUnionsForInput
          \lang{de}{\solution{[c;infinity)}}
          \lang{en}{\solution{[c;infinity)}}
          \lang{de}{\explanation{Prüfen Sie die Intervallgrenzen.}}
          \lang{en}{\explanation{Check the interval bounds.}}
    \end{answer}
\\

    \begin{answer}
          \lang{de}{\text{$\{x \in \R \, \vert \, x < \var{a} \text{ oder } x > \var{b} \}=$}}
          \lang{en}{\text{$\{x \in \R \, \vert \, x < \var{a} \text{ or } x > \var{b} \}=$}}
          \allowIntervalUnionsForInput
          \lang{de}{\solution{(-infinity;a),(b;infinity)}}
          \lang{en}{\solution{(-infinity;a),(b;infinity)}}
          \lang{de}{\explanation{Sie kommen hier nicht mit einem einzigen Intervall aus.}}
          \lang{en}{\explanation{A single interval will not be enough.}}
    \end{answer}
%    

    \begin{answer}
          \text{$\{x \in \R \, \vert \, |x| \leq \var{r1}\}=$}    
          \allowIntervalUnionsForInput
          \lang{de}{\solution{[l1;r1]}}
          \lang{en}{\solution{[l1;r1]}}
          \lang{de}{\explanation{Prüfen Sie die Intervallgrenzen und beachten Sie
          den Betrag.}}
          \lang{en}{\explanation{Check the interval bounds, paying attention
          to the absolute value.}}
    \end{answer}

  \end{question}


%
%
\end{problem}

\embedmathlet{gwtmathlet}

\end{content}
