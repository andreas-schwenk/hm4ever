\documentclass{mumie.problem.gwtmathlet}
%$Id$
\begin{metainfo}
  \name{
    \lang{de}{A03: Verknüpfungen von Funktionen}
    \lang{en}{Problem 3}
  }
  \begin{description} 
 This work is licensed under the Creative Commons License Attribution 4.0 International (CC-BY 4.0)   
 https://creativecommons.org/licenses/by/4.0/legalcode 

    \lang{de}{Beschreibung}
    \lang{en}{}
  \end{description}
  \corrector{system/problem/GenericCorrector.meta.xml}
  \begin{components}
    \component{js_lib}{system/problem/GenericMathlet.meta.xml}{mathlet}
  \end{components}
  \begin{links}
  \end{links}
  \creategeneric
\end{metainfo}
\begin{content}
\lang{de}{\title{A03: Verknüpfungen von Funktionen}}
\lang{en}{\title{A03: Operations on functions}}
\begin{block}[annotation]
	Im Ticket-System: \href{https://team.mumie.net/issues/23827}{Ticket 23827}
\end{block}
\usepackage{mumie.genericproblem}

\begin{block}[annotation]
Kopie: hm4mint/T204_Abbildungen_und_Funktionen/training 6

Im Ticket-System: \href{http://team.mumie.net/issues/9817}{Ticket 9817}
\end{block}
\begin{problem}
    \begin{variables}
      \randint[Z]{a}{-9}{9}
      \randint[Z]{c}{-9}{9}
      \randint{b}{2}{4}
      \randint[Z]{q}{-9}{9}
      \randint{z}{-5}{5}
      \randint{d}{-10}{10}
      \function[normalize]{f}{a*x^b +c*x +d}
      \function[normalize]{g}{q*x^b + a/(x-z)}
      \function[normalize]{l}{(a+q)*x^b+c*x+a/(x-z)+d}
      \function[normalize]{k}{a*q*x^(2*b) + (a^2*x^b +a*c*x+a*d)/(x-z) +q*c*x^(b+1) +d*q*x^b}
	\end{variables} 
    
    \begin{question}
% QS  \explanation{Führen Sie eine Addition bzw. Multiplikation der gegebenen Funktionen durch.}
      \type{input.function}
      \field{rational}
      \lang{de}{\text{Berechnen Sie $(f+g)(x)$ und $(f\cdot g)(x)$ für 
      \begin{align*}
       f : & \R \to \R, \ f(x) =  \var{f} , \\
       g: & \R\setminus\{\var{z} \} \to \R , \ g(x) =  \var{g}.
      \end{align*}
      Kürzen Sie, falls möglich, vollständig und geben Sie nur die Vorschrift der gesuchten Funktionen ein. }}
      \lang{en}{\text{Calculate $(f+g)(x)$ and $(f\cdot g)(x)$ for 
      \begin{align*}
       f : & \R \to \R, \ f(x) = \var{f} , \\
         g: & \R\setminus\{\var{z} \} \to \R , \ g(x) =  \var{g}.
      \end{align*}
      Simplify completely, if possible, and enter only the rule of the functions in question.}}
      \begin{answer}
          \text{$(f+g)(x)=$}
          \solution{l}
          \checkAsFunction{x}{-10}{10}{100}
% QS neu          
          \lang{de}{\explanation{Überprüfen Sie die Addition der gegebenen Funktionen.}}
          \lang{en}{\explanation{Check the sum of the given functions.}}
% QS  
      \end{answer}
      \begin{answer}
          \text{$(f \cdot g)(x)=$}
          \solution{k}
          \checkAsFunction{x}{-20}{-10}{100}
% QS neu           
          \lang{de}{\explanation{Überprüfen Sie die Multiplikation der gegebenen Funktionen.}}
          \lang{en}{\explanation{Check the product of the given functions.}}
      
% QS           
      \end{answer}
	\end{question}
    
    \begin{question}
      \lang{de}{\explanation{Der maximale Definitionsbereich ergibt sich aus der Schnittmenge der Definitionsbereiche der Funktionen $f$ und $g$.}}
      \lang{en}{\explanation{The maximum domain of definition is given by the intersection of the domains of the functions $f$ and $g$.}}
      \lang{de}{\text{Was ist der maximale Definitionsbereich, auf dem sowohl $f+g$ als auch $f \cdot g$ definiert sind? Kreuzen Sie die richtige Antwort an.}}
     \lang{en}{\text{What is the maximum domain of definition on which both $f+g$ and $f \cdot g$ are defined? Tick the correct answer.}}
      \type{mc.unique}
      \permuteAnswers{1}{2}
      \begin{choice}
        \text{$\R$}
        \solution{false}
      \end{choice}
      \begin{choice}
          \text{$\R\setminus \{ \var{z}\}$}
          \solution{true}
      \end{choice}             
    \end{question}

\end{problem}

\embedmathlet{mathlet}
\end{content}