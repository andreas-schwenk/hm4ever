\documentclass{mumie.problem.gwtmathlet}
%$Id$
\begin{metainfo}
  \name{
    \lang{en}{}
    \lang{de}{A05: Funktionengrenzwerte}
  }
  \begin{description} 
 This work is licensed under the Creative Commons License Attribution 4.0 International (CC-BY 4.0)   
 https://creativecommons.org/licenses/by/4.0/legalcode 

    \lang{en}{...}
    \lang{de}{...}
  \end{description}
  \corrector{system/problem/GenericCorrector.meta.xml}
  \begin{components}
    \component{js_lib}{system/problem/GenericMathlet.meta.xml}{gwtmathlet}
  \end{components}
  \begin{links}
  \end{links}
  \creategeneric
\end{metainfo}
\begin{content}
\lang{de}{\title{A05: Funktionengrenzwerte}}
\lang{en}{\title{A05: Limits of functions}}
\begin{block}[annotation]
	Im Ticket-System: \href{https://team.mumie.net/issues/23829}{Ticket 23829}
\end{block}
\begin{block}[annotation]
Kopie: hm4mint/T210_stetigkeit/training 5

Im Ticket-System: \href{https://team.mumie.net/issues/18755}{Ticket 18755}
\end{block}
\usepackage{mumie.genericproblem}



\begin{problem}
    \begin{question}
        \type{input.generic}
        \begin{variables}
            \randint{a}{3}{5}
            \function[normalize]{f}{|3*(a-x)|/(a-x)}
            \function{g}{a-x}
            \function[normalize]{ga}{3*g}
            \function{sol1}{3}
            \function{sol2}{-3} 
            \string{s}{nein} 
        \end{variables}
        \lang{de}{
            \text{Untersuchen Sie die Funktion $f:\R\to\R$ mit $f(x)=\begin{cases}
                    \var{f}, &\text{für }x\neq\var{a},\\
                    0, &\text{für }x=\var{a}.
                    \end{cases}$}
        }
        \lang{en}{\text{Consider the function $f:\R\to\R$ with $f(x)=\begin{cases}
                    \var{f}, &\text{for }x\neq\var{a},\\
                    0, &\text{for }x=\var{a}.
                    \end{cases}$}}
        \begin{answer}
            \type{input.number}
            \text{$\lim_{x\to 0} f(x)=$}
            \solution{sol1}
            \explanation[edited]{$f(0)=3$}
        \end{answer}
        
        \begin{answer}
            \type{input.number}
            \text{$\lim_{x\to \infty} f(x)=$}
            \solution{sol2}
            \lang{de}{\explanation[edited]{$\var{ga}=\var{a} \cdot (\var{g})$. Mit $\var{g}$ kürzen, den negativen Nenner
            berücksichtigen, dann ist $\lim_{x\to \infty} f(x)=-3$}}
            \lang{en}{\explanation[edited]{$\var{ga}=\var{a} \cdot (\var{g})$. After dividing by $\var{g}$, keeping the negative denominator in mind,
            we find $\lim_{x\to \infty} f(x)=-3$.}}
        \end{answer}
        
        \begin{answer}
            \type{input.number}
            \text{$\lim_{x\searrow \var{a}} f(x)=$}
            \solution{sol2}
            \lang{de}{\explanation[edited]{$\var{ga}=\var{a} \cdot (\var{g})$. Nenner ist negativ, mit $\var{g}$ kürzen, dann ist
            $\lim_{x\searrow \var{a}} f(x)=-3$}}
             \lang{en}{\explanation[edited]{$\var{ga}=\var{a} \cdot (\var{g})$. The denominator is negative. After dividing by $\var{g}$, we find
            $\lim_{x\searrow \var{a}} f(x)=-3$.}}
            
        \end{answer}
        
        \begin{answer}
            \type{input.number}
            \text{$\lim_{x\nearrow \var{a}} f(x)=$}
            \solution{sol1}
            \lang{de}{\explanation[edited]{$\var{ga}=\var{a} \cdot (\var{g})$. Mit $\var{g}$ kürzen, dann ist $\lim_{x\nearrow \var{a}} f(x)=3$}}
            \lang{en}{\explanation[edited]{$\var{ga}=\var{a} \cdot (\var{g})$. Divide by $\var{g}$ to find $\lim_{x\nearrow \var{a}} f(x)=3$.}}
        \end{answer}
        
        \begin{answer}
            \type{mc.yesno}
            \lang{de}{
                \text{Existiert der folgende Grenzwert? }}
            \lang{en}{\text{Does the following limit exist? }}
            \begin{choice}
                \text{$\lim_{x\to \var{a}} f(x)$}
                \solution{false}
                \lang{de}{\explanation[edited]{Rechts- und linksseitiger Grenzwert müssen für die 
                Existenz des Grenzwertes identisch sein und mit dem Funktionswert übereinstimmen.}}
              \lang{en}{\explanation[edited]{The right-hand and left-hand limits must be equal for the 
                limit to exist, and they must also coincide with the function value.}}
            \end{choice}
        \end{answer}
        
        
     \end{question}
\end{problem}
            
\embedmathlet{gwtmathlet}

\end{content}
