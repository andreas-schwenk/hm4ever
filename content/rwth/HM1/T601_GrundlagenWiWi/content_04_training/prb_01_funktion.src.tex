\documentclass{mumie.problem.gwtmathlet}
%$Id$
\begin{metainfo}
  \name{
    \lang{de}{A01: Definitionsbereich und Wertebereich von Funktionen}
    \lang{en}{}
  }
  \begin{description} 
 This work is licensed under the Creative Commons License Attribution 4.0 International (CC-BY 4.0)   
 https://creativecommons.org/licenses/by/4.0/legalcode 

    \lang{de}{Beschreibung}
    \lang{en}{}
  \end{description}
  \corrector{system/problem/GenericCorrector.meta.xml}
  \begin{components}
    \component{generic_image}{content/rwth/HM1/images/g_tkz_T601_04_Problem01_B.meta.xml}{T601_04_Problem01_B}
    \component{generic_image}{content/rwth/HM1/images/g_tkz_T601_04_Problem01_C.meta.xml}{T601_04_Problem01_C}
    \component{generic_image}{content/rwth/HM1/images/g_tkz_T601_04_Problem01_A.meta.xml}{T601_04_Problem01_A}
    \component{js_lib}{system/problem/GenericMathlet.meta.xml}{mathlet}
  \end{components}
  \begin{links}
  \end{links}
  \creategeneric
\end{metainfo}
\begin{content}
\lang{de}{\title{A01: Definitionsbereich und Wertebereich von Funktionen}}
\lang{en}{\title{A01: Domain and range of a function}}
\begin{block}[annotation]
	Im Ticket-System: \href{https://team.mumie.net/issues/23825}{Ticket 23825}
\end{block}

\usepackage{mumie.genericproblem}

\begin{block}[annotation]
Kopie: hm4mint/T204_Abbildungen_und_Funktionen/training 2

Im Ticket-System: \href{http://team.mumie.net/issues/9813}{Ticket 9813}
\end{block}
\begin{problem}

\randomquestionpool{1}{3}
  \begin{question}
 \lang{de}{\text{\begin{figure}\image{T601_04_Problem01_A}\end{figure}
    Welche der obigen Graphiken stellt eine Funktion dar?
    }}
  \lang{en}{\text{\begin{figure}\image{T601_04_Problem01_A}\end{figure}
    Which of the above diagrams represents a function?
    }}
  \lang{de}{\explanation{Eine Funktion ist eine Zuordnung zwischen zwei Mengen, 
    die jedem Element des Definitionsbereich genau ein Element des Zielbereichs 
    zuordnet. Graphisch gesehen bedeutet das, dass eine Funktion genau dann vorliegt, 
    wenn von allen Elementen des Definitionsbereichs genau ein Pfeil in den 
    Zielbereich führt. Beachten Sie auch unbedingt die Richtung der Pfeile.}}
  \lang{en}{\explanation{A function is an assignment between two sets 
    that assigns to each element of the domain exactly one element of the codomain. In graphical terms, this means that we have a function
    if and only if exactly one arrow from every element of the domain leads to the 
    codomain. It is also important to consider the direction of the arrows.}}
   	
      \type{mc.yesno}
       \field{real}
      \precision{3}
      \begin{variables}
       \end{variables}
 		\begin{choice}
      \lang{de}{\text{erste Graphik}}
      \lang{en}{\text{first diagram}}
      \solution{false}
		\end{choice}
 		\begin{choice}
      \lang{de}{\text{zweite Graphik}}
      \lang{en}{\text{second diagram}}
      \solution{true}
		\end{choice}
 		\begin{choice}
      \lang{de}{\text{dritte Graphik}}
      \lang{en}{\text{third diagram}}
      \solution{true}
		\end{choice}
       %\explanation[equalChoice(??0)]{Schauen Sie sich noch einmal die dritte Graphik an und nehmen Sie die Menge $N$ als Definitionsbereich und $M$ als Zielbereich.}
  \end{question} 

\begin{question}
  \lang{de}{\text{\begin{figure}\image{T601_04_Problem01_C}\end{figure}
   Die Graphik stellt eine Funktion dar. Geben Sie Definitionsbereich, 
   Zielbereich und Wertemenge der Funktion an.
    }}
    \lang{en}{\text{\begin{figure}\image{T601_04_Problem01_C}\end{figure}
       This diagram represents a function. Find the domain, 
   codomain and range of the function.}}
    
  %\explanation{Der Definitionsbereich besteht aus allen Elementen des Definitionsbereichs. Der Zielbereich besteht aus allen Elementen des Zielbereichs. Die Wertemenge besteht aus allen Elementen des Zielbereichs, die tatsächlich zugeordnet worden sind.}
   	
      \type{input.finite-number-set}
       \field{real}
      \precision{3}
      \begin{variables}
      \number{m1}{-1}
      \number{m2}{0}
      \number{m3}{2}
      \number{m4}{3}
      \number{m5}{7}
      \number{n1}{0}
      \number{n2}{1}
      \number{n3}{2}
      \number{n4}{4}
      \number{n5}{9}           
       \end{variables}
 		\begin{answer}
       \lang{de}{\text{Definitionsbereich}}
       \lang{en}{\text{Domain}}
      \solution{m1,m2,m3,m4,m5}
      \lang{de}{\explanation{Der Definitionsbereich besteht aus allen Elementen der Menge $M$.}}
      \lang{en}{\explanation{The domain consists of all elements of the set $M$. }}
		\end{answer}
 		\begin{answer}
      \lang{de}{\text{Zielbereich}}
      \lang{en}{\text{Codomain}}
      \solution{n1,n2,n3,n4,n5}
      \lang{de}{\explanation{Der Zielbereich besteht aus allen Elementen der Menge $N$.}}
      \lang{en}{\explanation{The codomain consists of all elements of the set $N$. }}
		\end{answer}
 		\begin{answer}
      \lang{de}{\text{Wertemenge}}
      \lang{en}{\text{Range}}
      \solution{n4}
      \lang{de}{\explanation{Die Wertemenge besteht aus allen Elementen des Zielbereichs, die auch tatsächlich getroffen werden.}}
      \lang{en}{\explanation{The range consists of all elements of the codomain that are actually hit.}}
		\end{answer}

  \end{question} 
\begin{question}
   \lang{de}{\text{\begin{figure}\image{T601_04_Problem01_B}\end{figure}
   Die Graphik stellt eine Funktion dar. Geben Sie Definitionsbereich, Zielbereich und
   Wertemenge der Funktion an.
    }}
    \lang{en}{\text{\begin{figure}\image{T601_04_Problem01_B}\end{figure}
     This diagram represents a function. Give the domain, codomain and range of the function.}}
  %\explanation{Der Definitionsbereich besteht aus allen Elementen des Definitionsbereichs. Der Zielbereich besteht aus allen Elementen des Zielbereichs. Die Wertemenge besteht aus allen Elementen des Zielbereichs, die tatsächlich zugeordnet worden sind.}
   	
      \type{input.finite-number-set}
       \field{real}
      \precision{3}
      \begin{variables}
      \number{m1}{0}
      \number{m2}{3}
      \number{n2}{1}
      \number{n3}{2}
      \number{n4}{4}
       \end{variables}
 		\begin{answer}
     \lang{de}{\text{Definitionsbereich}}
     \lang{en}{\text{Domain}}
      \solution{n2,n3,n4}
     \lang{de}{\explanation{Es ist eine Funktion mit Definitionsbereich $N$ abgebildet.}}
     \lang{en}{\explanation{The function represented here has domain $N$.}}
		\end{answer}
 		\begin{answer}
      \lang{de}{\text{Zielbereich}}
      \lang{en}{\text{Codomain}}
      \solution{m1,m2}
      \lang{de}{\explanation{Der Zielbereich besteht hier aus allen Elementen der Menge $M$.}}
      \lang{en}{\explanation{The codomain here consists of all elements of the set $M$.}}
		\end{answer}
 		\begin{answer}
      \lang{de}{\text{Wertemenge}}
      \lang{en}{\text{Range}}
      \solution{m1,m2}
      \lang{de}{\explanation{Die Wertemenge besteht aus allen Elementen des Zielbereichs, die auch tatsächlich getroffen werden.}}
      \lang{en}{\explanation{The range consists of all elements of the target range that are actually hit.}}
		\end{answer}

  \end{question} 
\end{problem}

\embedmathlet{mathlet}
\end{content}