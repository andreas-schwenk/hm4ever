\documentclass{mumie.problem.gwtmathlet}
%$Id$
\begin{metainfo}
  \name{
    \lang{en}{...}
    \lang{de}{A04: Stückzahl und Zeitdauer}
  }
  \begin{description} 
 This work is licensed under the Creative Commons License Attribution 4.0 International (CC-BY 4.0)   
 https://creativecommons.org/licenses/by/4.0/legalcode 

    \lang{en}{...}
    \lang{de}{...}
  \end{description}
  \corrector{system/problem/GenericCorrector.meta.xml}
  \begin{components}
    \component{js_lib}{system/problem/GenericMathlet.meta.xml}{gwtmathlet}
  \end{components}
  \begin{links}
  \end{links}
  \creategeneric
\end{metainfo}
\begin{content}
\lang{de}{\title{A04: Stückzahl und Zeitdauer}}
\lang{en}{\title{A04: Quantity and duration}}
\begin{block}[annotation]
	Im Ticket-System: \href{https://team.mumie.net/issues/23791}{Ticket 23791}
\end{block}
\usepackage{mumie.genericproblem}

\begin{problem}
    
 
  \randomquestionpool{1}{5}
  %\randomquestionpool{3}{3}
  %\permutequestions
 
  \begin{question} % Question 1 
        \begin{variables}
            \function[calculate]{sol}{1400*(1,015)^11}
            \function[calculate]{sol2}{floor(sol)}
            \string{s}{September}
            \string{j}{Juni}
            \string{o}{Oktober}
        \end{variables}
        \lang{de}{\text{In einem Betrieb soll innerhalb eines Jahres die Arbeitsproduktivität
        monatlich um 1,5 \% gesteigert werden. Im Januar betrug die Produktion 1400
        Stück.}}
        \lang{en}{\text{In a company, labour productivity is to increase by 1.5 \% per month over one year.
        In January the production was 1400 pieces.}}
        
  \lang{de}{\explanation{Stückzahl im Dezember: $1400\cdot(1,015)^{11}$; 
  es sind nur 11 Steigerungen.\\ Bedenken Sie: Errechnen Sie z.B. 1 Monat, dann landen Sie im Februar.
  %\\Monat: $n=\log_{1,015}\frac{1577}{1400}$; es beginnt mit dem Januar,
  %so dass nach 8 Monaten der September erreicht ist.
  }}
  \lang{en}{\explanation{Number of units produced in December: $1400\cdot(1.015)^{11}$; 
  there are only 11 increments.\\ Remember: if you calculate e.g. 1 month, then you end up in February.
  %\\month: $n=\log_{1,015}\frac{1577}{1400}$; it starts with January,
  %so after 8 months you reach September.
  }}
  \begin{answer}
        \type{input.number}
        \lang{de}{\text{Welche Produktion wird im Monat Dezember erreicht?\\
        \textit{(Runden Sie bitte auf eine ganze Zahl.)}\\
        \\produzierte Stückzahl im Dezember =}}
        \lang{en}{\text{What will the production in the month of December be?\\
        \textit{(Please round to a whole number.)}\\
        \\Number of pieces produced in December =}}
        \solution{sol2}
  \end{answer}
  
  \begin{answer}
    \type{input.text}
    \lang{de}{\text{\\ In welchem Monat werden 1577 Stück produziert?\\
    \textit{(Geben Sie den Monat bitte als Wort ein.)}\\
    \\Monat =}}
    \lang{en}{\text{\\ In which month will 1577 pieces be produced?\\
    \textit{(Please enter the month as a word)}\\
    \\Month =}}
    \inputAsString{t}
    \solution{s}
    \checkStringsForRelation{equal(s,t)}
  \end{answer}
  
  \end{question}
  
   \begin{question} % Question 2 
        \begin{variables}
            \function[calculate]{sol}{1500*(1,014)^11}
            \function[calculate]{sol2}{floor(sol)}
            \string{j}{Juni}
        \end{variables}
        \lang{de}{\text{In einem Betrieb soll innerhalb eines Jahres die Arbeitsproduktivität
        monatlich um 1,4 \% gesteigert werden. Im Januar betrug die Produktion 1500
        Stück.}}
        \lang{en}{\text{In a company, labour productivity is to increase by 1.4 \% per month over one year.
        In January the production was 1500
        pieces.}}
        
  \lang{de}{\explanation{Stückzahl im Dezember: $1500\cdot(1,014)^{11}$; 
  es sind nur 11 Steigerungen.\\ Bedenken Sie: Errechnen Sie z.B. 1 Monat, dann landen Sie im Februar.
  %\\Monat: $n=\log_{1,014}\frac{1608}{1500}$; es beginnt mit dem Januar,
  %so dass nach 5 Monaten der Juni erreicht ist.
  }}
  \lang{en}{\explanation{Number of pieces in December: $1500\cdot(1.014)^{11}$; 
  there are only 11 increments.\\ Remember: if you calculate 1 month, for example, then you end up in February.
  %\\Month: $n=\log_{1,014}\frac{1608}{1500}$; it starts with January,
  %so after 5 months you reach June.
  }}
  \begin{answer}
        \type{input.number}
        \lang{de}{\text{Welche Produktion wird im Monat Dezember erreicht?\\
        \textit{(Runden Sie bitte auf eine ganze Zahl.)}\\
        \\produzierte Stückzahl im Dezember =}}
        \lang{en}{\text{What will the production in the month of December be?\\
        \textit{(Please round to a whole number.)}\\
        \\Number of pieces produced in December =}}
        \solution{sol2}
  \end{answer}
  
  \begin{answer}
    \type{input.text}
    \lang{de}{\text{\\ In welchem Monat werden 1608 Stück produziert?\\
    \textit{(Geben Sie den Monat bitte als Wort ein.)}\\
    \\Monat =}}
    \lang{en}{\text{\\ In which month will 1608 pieces be produced?\\
    \textit{(Please enter the month as a word)}\\
    \\Month =}}
    \inputAsString{t}
    \solution{j}
    \checkStringsForRelation{equal(j,t)}
  \end{answer}
  
  \end{question}
  
   \begin{question} % Question 3 
        \begin{variables}
            \function[calculate]{sol}{1300*(1,016)^11}
            \function[calculate]{sol2}{floor(sol)}
            \string{s}{September}
        \end{variables}
        \lang{de}{\text{In einem Betrieb soll innerhalb eines Jahres die Arbeitsproduktivität
        monatlich um 1,6 \% gesteigert werden. Im Januar betrug die Produktion 1300
        Stück.}}
        \lang{en}{\text{In a company, labour productivity is to increase by 1.6 \% per month over one year.
        In January the production was 1300
        pieces.}}
        
  \lang{de}{\explanation{Stückzahl im Dezember: $1300\cdot(1,016)^{11}$; 
  es sind nur 11 Steigerungen.\\ Bedenken Sie: Errechnen Sie z.B. 1 Monat, dann landen Sie im Februar.
  %\\Monat: $n=\log_{1,016}\frac{1476}{1300}$; es beginnt mit dem Januar,
  %so dass nach 8 Monaten der September erreicht ist.
  }}
  \lang{en}{\explanation{Number of pieces in December: $1300\cdot(1.016)^{11}$;
  there are only 11 increments.\\ Remember: if you calculate 1 month, for example, then you end up in February.
  %\\Month: $n=\log_{1,016}\frac{1476}{1300}$; it starts with January,
  %so that after 8 months you reach September.
  }}
  \begin{answer}
        \type{input.number}
        \lang{de}{\text{Welche Produktion wird im Monat Dezember erreicht?\\
        \textit{(Runden Sie bitte auf eine ganze Zahl.)}\\
        \\produzierte Stückzahl im Dezember =}}
        \lang{en}{\text{What will the production in the month of December be?\\
        \textit{(Please round to a whole number.)}\\
        \\Number of pieces produced in December =}}
        \solution{sol2}
  \end{answer}
  
  \begin{answer}
    \type{input.text}
    \lang{de}{\text{\\ In welchem Monat werden 1476 Stück produziert?\\
    \textit{(Geben Sie den Monat bitte als Wort ein.)}\\
    \\Monat =}}
    \lang{en}{\text{\\In which month will 1476 pieces be produced?\\
    \textit{(Please enter the month as a word)}\\
    \\Month =}}
    \inputAsString{t}
    \solution{s}
    \checkStringsForRelation{equal(s,t)}
  \end{answer}
  
  \end{question}
  
   \begin{question} % Question 4 
        \begin{variables}
            \function[calculate]{sol}{1600*(1,017)^11}
            \function[calculate]{sol2}{floor(sol)}
            \string{s}{September}
        \end{variables}
        \lang{de}{\text{In einem Betrieb soll innerhalb eines Jahres die Arbeitsproduktivität
        monatlich um 1,7 \% gesteigert werden. Im Januar betrug die Produktion 1600
        Stück.}}
        \lang{en}{\text{In a company, labour productivity is to increase by 1.7 \% per month over one year.
        In January the production was 1600
        pieces.}}
        
  \lang{de}{\explanation{Stückzahl im Dezember: $1600\cdot(1,017)^{11}$; 
  es sind nur 11 Steigerungen.\\ Bedenken Sie: Errechnen Sie z.B. 1 Monat, dann landen Sie im Februar.
  %\\Monat: $n=\log_{1,017}\frac{1831}{1600}$; es beginnt mit dem Januar,
  %so dass nach 8 Monaten der September erreicht ist.
  }}
   \lang{en}{\explanation{Number of pieces in December: $1600\cdot(1.017)^{11}$; 
  there are only 11 increments.\\ Remember: if you calculate 1 month, for example, then you end up in February.
  %\\Month: $n=\log_{1,017}\frac{1831}{1600}$; it starts with January,
  %so that after 8 months you reach September.
  }}
  \begin{answer}
        \type{input.number}
        \lang{de}{\text{Welche Produktion wird im Monat Dezember erreicht?\\
        \textit{(Runden Sie bitte auf eine ganze Zahl.)}\\
        \\produzierte Stückzahl im Dezember =}}
         \lang{en}{\text{What will the production in the month of December be?\\
        \textit{(Please round to a whole number.)}\\
        \\Number of pieces produced in December =}}
        \solution{sol2}
  \end{answer}
  
  \begin{answer}
    \type{input.text}
    \lang{de}{\text{\\ In welchem Monat werden 1831 Stück produziert?\\
    \textit{(Geben Sie den Monat bitte als Wort ein.)}\\
    \\Monat =}}
     \lang{en}{\text{\\ In which month will 1831 pieces be produced?\\
    \textit{(Please enter the month as a word)}\\
    \\Month =}}
    \inputAsString{t}
    \solution{s}
    \checkStringsForRelation{equal(s,t)}
  \end{answer}
  
  \end{question}
  
   \begin{question} % Question 5 
        \begin{variables}
            \function[calculate]{sol}{1200*(1,012)^11}
            \function[calculate]{sol2}{floor(sol)}
            \string{o}{Oktober}
        \end{variables}
        \lang{de}{\text{In einem Betrieb soll innerhalb eines Jahres die Arbeitsproduktivität
        monatlich um 1,2 \% gesteigert werden. Im Januar betrug die Produktion 1200
        Stück.}}
        \lang{en}{\text{In a company, labour productivity is to increase by 1.2 \% per month over one year.
        In January the production was 1200
        pieces.}}
        
  \lang{de}{\explanation{Stückzahl im Dezember: $1200\cdot(1,012)^{11}$; 
  es sind nur 11 Steigerungen.\\ Bedenken Sie: Errechnen Sie z.B. 1 Monat, dann landen Sie im Februar.
  %\\Monat: $n=\log_{1,012}\frac{1336}{1200}$; es beginnt mit dem Januar,
  %so dass nach 9 Monaten der Oktober erreicht ist.
  }}
  \lang{en}{\explanation{Number of pieces in December: $1200\cdot(1,012)^{11}$; 
  there are only 11 increments.\\ Remember: if you calculate 1 month, for example, then you end up in February.
  %\\Month: $n=\log_{1,012}\frac{1336}{1200}$; it starts with January,
  %so that after 9 months you reach Oktober.
  }}
  \begin{answer}
        \type{input.number}
        \lang{de}{\text{Welche Produktion wird im Monat Dezember erreicht?\\
        \textit{(Runden Sie bitte auf eine ganze Zahl.)}\\
        \\produzierte Stückzahl im Dezember =}}
         \lang{en}{\text{What will the production in the month of December be?\\
        \textit{(Please round to a whole number.)}\\
        \\Number of pieces produced in December =}}
        \solution{sol2}
  \end{answer}
  
  \begin{answer}
    \type{input.text}
    \lang{de}{\text{ \\In welchem Monat werden 1476 Stück produziert?\\
    \textit{(Geben Sie den Monat bitte als Wort ein.)}\\
    \\Monat =}}
     \lang{en}{\text{\\In which month will 1476 pieces be produced?\\
    \textit{(Please enter the month as a word)}\\
    \\Month =}}
    \inputAsString{t}
    \solution{o}
    \checkStringsForRelation{equal(o,t)}
  \end{answer}
  
  \end{question}
\end{problem}

\embedmathlet{gwtmathlet}

\end{content}
