\documentclass{mumie.problem.gwtmathlet}
%$Id$
\begin{metainfo}
  \name{
    \lang{en}{...}
    \lang{de}{A05: Verhalten von Polynomen}
  }
  \begin{description} 
 This work is licensed under the Creative Commons License Attribution 4.0 International (CC-BY 4.0)   
 https://creativecommons.org/licenses/by/4.0/legalcode 

    \lang{en}{...}
    \lang{de}{...}
  \end{description}
  \corrector{system/problem/GenericCorrector.meta.xml}
  \begin{components}
    \component{js_lib}{system/problem/GenericMathlet.meta.xml}{gwtmathlet}
  \end{components}
  \begin{links}
  \end{links}
  \creategeneric
\end{metainfo}
\begin{content}
\lang{de}{\title{A05: Verhalten von Polynomen}}
\lang{en}{\title{A05: Behaviour of polynomials}}
\begin{block}[annotation]
	Im Ticket-System: \href{https://team.mumie.net/issues/24059}{Ticket 24059}
\end{block}
\begin{block}[annotation]
Kopie: hm4mint/T103_Polynomfunktionen/training 2

	Im Ticket-System: \href{http://team.mumie.net/issues/9337}{Ticket 9337}
\end{block}

\usepackage{mumie.genericproblem}

\begin{problem}

\begin{variables}
    \randint[Z]{n1}{-150}{150}
	\randint{n2}{-150}{150}
    \randint{n3}{-150}{150}
    \randint{n4}{-15}{15}
    \randint[Z]{n5}{-3}{3}
    \randint{n0}{-150}{150}
    \randadjustIf{n1}{n1 = n2 OR n1 = n3 OR n1 = n4 OR n1 = n5}
    \randadjustIf{n2}{n2 = n3 OR n2 = n4 OR n2 = n5}
    \randadjustIf{n3}{n3 = n4 OR n3 = n5}
    \randadjustIf{n4}{n4 = n5}
    \function{f}{n5*x^5 + n4*x^4 +n3*x^3 + n2 * x^2 + n1*x + n0}
    \function{f1}{n5*x^5}
    \function{f2}{n1*x+n0}
    
    
    \randint[Z]{n1_1}{-150}{150}
  	\randint{n2_1}{-5}{5}
   	\randint{n3_1}{-15}{15}
   	\randint[Z]{n4_1}{-3}{3}
   	\randint{n0_1}{-50}{50}
    \randadjustIf{n1_1}{n1_1 = n2_1 OR n1_1 = n3_1 OR n1_1 = n4_1}
   	\randadjustIf{n2_1}{n2_1 = n3_1 OR n2_1 = n4_1 }
	\randadjustIf{n3_1}{n3_1 = n4_1}

    \function{g}{n4_1*x^4 +n3_1*x^3 + n2_1 * x^2 + n1_1*x + n0_1}
    \function{g1}{n4_1*x^4}
    \function{g2}{n1_1*x+n0_1}
    
\end{variables}

%Frage 1 von 3
\begin{question}
	\type{input.function}
	\field{real} 
	\lang{de}{
		\text{Wie verläuft das Polynom 
        $p_1(x) = \var{f}$\\
		für 
        $x \rightarrow \pm \infty$ und 
        $x \rightarrow 0$? \\
	    Fügen Sie den entsprechenden führenden Term ein. } 
	    \explanation{Für 
        $x \rightarrow \pm \infty$ verläuft  der Graph  wie\\
	    der des Terms 
        $a_nx^n$, der höchsten Potenz.\\
	    Für 
        $x \rightarrow 0$ hängt das Verhalten der Polynomfunktion von\\
	   	den Summanden mit den niedrigen Potenzen ab.}
   	}
    \lang{en}{
    	\text{How does the polynomial 
        $p_1(x) = \var{f}$ behave as 
        $x \rightarrow \pm \infty$ and as 
        $x \rightarrow 0$?\\
    	Input the leading term.}
    	\explanation{As 
        $x \rightarrow \pm \infty$ the graph behaves like the term with 
        the highest power, 
        $a_nx^n$.\\
    	As 
        $x \rightarrow 0$ the behaviour of the polynomial depends on 
        the summands with the smallest powers.}
    }
    \lang{fr}{
    	\text{Comment se comporte le polynôme 
        $p_1(x) = \var{f}$  quand 
        $x \rightarrow \pm \infty$ et quand 
        $x \rightarrow 0$ ?\\
    	}
    	\explanation{Quand
        $x \rightarrow \pm \infty$, le polynôme se comporte comme le terme de 
        plus haut degré, 
        $a_nx^n$.\\
    	Quand 
        $x \rightarrow 0$, le comportement du polynôme dépend du terme de plus bas degré.}
    }
     \lang{zh}{
    	\text{请分别输入当 
        $x \rightarrow \pm \infty$ 时和当 
        $x \rightarrow 0$ 时,多项式 
        $p_1(x) = \var{f}$ 的变化情况。\\
    	只输入决定项。}
    	\explanation{当 
        $x \rightarrow \pm \infty$ 时,图像的特性取决于最高次项 
        $a_nx^n$ 。\\
    	当 
        $x \rightarrow 0$ 时,多项式函数的特性取决于最低次项,即常数项。}
    }  
    \begin{answer}
    \lang{de}{\text{Für 
    $x \rightarrow \pm \infty$ verläuft das Polynom wie
    $f_{\infty}(x)= $}}
    \lang{en}{\text{As 
    $x \rightarrow \pm \infty$ the polynomial behaves like 
    $f_{\infty}(x)= $ }}
    \lang{fr}{\text{Quand
    $x \rightarrow \pm \infty$, le polynôme se comporte comme
    $f_{\infty}(x)= $ }}
    \lang{zh}{\text{当 
    $x \rightarrow \pm \infty$ 时,多项式的特性取决于 
    $f_{\infty}(x)= $ }}
    \solution{f1}
    \checkAsFunction{x}{-10}{10}{100}
    \end{answer}
    
    \begin{answer}
    \lang{de}{\text{Für 
    $x \rightarrow 0$ verläuft das Polynom wie 
    $f_0(x)= $}}
    \lang{en}{\text{As 
    $x \rightarrow 0$ the polynomial behaves like
    $f_0(x)= $}}
    \lang{fr}{\text{Quand 
    $x \rightarrow 0$, le polynôme se comporte comme 
    $f_0(x)= $}}
    \lang{zh}{\text{当 
    $x \rightarrow 0$ 时,多项式的特性取决于 
    $f_0(x)= $}}
    \solution{f2}
    \checkAsFunction{x}{-10}{10}{100}
    \end{answer}
\end{question}

%Frage 2 von 3
\begin{question}
	\type{input.function}
    \field{real} 
    \lang{de}{
	    \text{Wie verläuft das Polynom 
        $p_2(x) = \var{g}$\\
	    für 
        $x \rightarrow \pm \infty$ und 
        $x \rightarrow 0$? \\
	    Fügen Sie den entsprechenden Funktionsterm ein. }
	    \explanation{Für 
        $x \rightarrow \pm \infty$ verläuft  der Graph  wie\\
	    der des Terms 
        $a_nx^n$, der höchsten Potenz.\\
	    Für 
        $x \rightarrow 0$ hängt das Verhalten der Polynomfunktion von\\
	    den Summanden mit den niedrigen Potenze ab.}
    }
    \lang{en}{
    	\text{How does the polynomial 
        $p_2(x) = \var{g}$ behave as 
        $x \rightarrow \pm \infty$ and as 
        $x \rightarrow 0$?\\
    	Input the corresponding term of the function.}
    	\explanation{As 
        $x \rightarrow \pm \infty$ the graph 
        behaves like the term with the highest power,
        $a_nx^n$.\\
    	As 
        $x \rightarrow 0$ the behaviour of the polynomial 
        depends on the summands with the smallest power.}
    }
    \lang{fr}{
    	\text{Comment se comporte le polynôme
        $p_2(x) = \var{g}$  quand 
        $x \rightarrow \pm \infty$ et quand
        $x \rightarrow 0$ ?\\
    	}
    	\explanation{Quand 
        $x \rightarrow \pm \infty$, le polynôme se comporte comme le terme de 
        plus haut degré, 
        $a_nx^n$.\\
    	Quand
        $x \rightarrow 0$, le comportement du polynôme dépend du terme de plus bas degré.}
        }
    \lang{zh}{
    	\text{请分别输入当 
        $x \rightarrow \pm \infty$ 时和当 
        $x \rightarrow 0$ 时,多项式 
        $p_2(x) = \var{g}$ 的变化情况。\\
    	只输入决定项。}
    	\explanation{当 
        $x \rightarrow \pm \infty$ 时,图像的特性取决于最高次项 
        $a_nx^n$ 。\\
    	当 
        $x \rightarrow 0$ 时,多项式的特性取决于最低次项,即常数项。}
    }  
    \begin{answer}
    \lang{de}{\text{Für 
    $x \rightarrow \pm \infty$ verläuft das Polynom wie 
    $f_{\infty}(x)= $}}
    \lang{en}{\text{As 
    $x \rightarrow \pm \infty$ the polynomial behaves like 
    $f_{\infty}(x)= $ }}
    \lang{fr}{\text{Quand 
    $x \rightarrow \pm \infty$, le polynôme se comporte comme 
    $f_{\infty}(x)= $ }}
    \lang{zh}{\text{当 
    $x \rightarrow \pm \infty$ 时,多项式的特性取决于
    $f_{\infty}(x)= $ }}
    \solution{g1}
    \checkAsFunction{x}{-10}{10}{100}
    \end{answer}
    
    \begin{answer}
    \lang{de}{\text{Für 
    $x \rightarrow 0$ verläuft das Polynom wie 
    $f_0(x)= $}}
    \lang{en}{\text{As 
    $x \rightarrow 0$ the polynomial behaves like 
    $f_0(x)= $}}
    \lang{fr}{\text{Quand 
    $x \rightarrow 0$, le polynôme se comporte comme 
    $f_0(x)= $}}
    \lang{zh}{\text{当 
    $x \rightarrow 0$ 时,多项式的特性取决于 
    $f_0(x)= $}}
    \solution{g2}
    \checkAsFunction{x}{-10}{10}{100}
    \end{answer}
    
\end{question}

%Frage 3 von 3
\begin{question}    
	\type{mc.multiple}

    \lang{de}{
	    \text{Ordnen Sie die Polynomgleichungen den Graphen zu.\\
		Kreuzen Sie alle richtigen Antworten an.\\
		Verkleinern Sie die Graphen mit der Maus,\\
	 	so dass Sie einen größeren Ausschnitt sehen können.}
	 	\explanation{Vergleichen Sie den Verlauf der Polynome und der Graphen \\
	 	für 
        $x \rightarrow \pm \infty$ und 
        $x \rightarrow 0$.}
 	}
  	\lang{en}{
  		\text{Match each of the polynomial equations with their graphs and select the correct statements below. Choose "Yes" for those that are correct, "No" otherwise.\\
  		Zoom in using the mouse wheel to get a better view of the graphs.}
  		\explanation{As 
        $x \rightarrow \pm \infty$ the graph behaves like the term with the highest power,
        $a_nx^n$.\\
    	As 
        $x \rightarrow 0$ the behaviour of the polynomial depends on the summand with the smallest power.}
  	}
    \lang{fr}{
  		\text{Associez à chacun des polynômes 
        un graphe, et sélectionnez les propositions correctes ci-dessous. \\
  		Zoomez en utilisant la molette de la souris pour avoir une meilleure vue sur les graphes.}
  		\explanation{Quand 
        $x \rightarrow \pm \infty$, le polynôme se comporte comme le terme de plus haut degré,
        $a_nx^n$.\\
    	Quand 
        $x \rightarrow 0$, le comportement du polynôme dépend du terme de plus bas degré.}
  	}
  	\lang{zh}{
  		\text{请匹配下列多项式方程及其图像。选出所有正确的答案。\\
        通过鼠标滚轮改变图像大小,可以部分放大以便于观察。}
  		\explanation{当 
        $x \rightarrow \pm \infty$ 时和当 
        $x \rightarrow 0$ 时,对比多项式和图像的曲线。}
  	}
  	\plotF{1}{f} % % the function a1 is defined below in 'variables' in the usual way
    \plotFrom{1}{-50} % % and is plotted starting from 0.0
    \plotTo{1}{50} % % and ending in 1.0 , it's a quarter of a circle
    \plotColor{1}{blue} % % colored blu
    \plotF{2}{g} % % the function a1 is defined below in 'variables' in the usual way
    \plotFrom{2}{-50} % % and is plotted starting from 0.0
    \plotTo{2}{50} % % nd ending in 1.0 , it's a quarter of a circle
    \plotColor{2}{red}
    \plotLeft{-1.0} % % defines the canvas bound left
    \plotRight{1.0} % % and right
    \plotSize{400} 
   
  	
 	\permutechoices{1}{4}
 
	\begin{choice}
	    \lang{de}{\text{Das Polynom 
        $p_1(x) = \var{f}$ hat den blauen Funktionsgraphen.}}
	    \lang{en}{\text{The blue graph is the polynomial 
        $p_1(x) = \var{f}$.}}
        \lang{fr}{\text{Le graphe bleu est le polynôme
        $p_1(x) = \var{f}$.}}
        \lang{zh}{\text{多项式 
        $p_1(x) = \var{f}$ 对应蓝色的函数图像。}}
	  	\solution{true}    
    \end{choice}
    
    \begin{choice}
	    \lang{de}{\text{Das Polynom 
        $p_1(x) = \var{f}$ hat den roten Funktionsgraphen.}}
	    \lang{en}{\text{The red graph is the polynomial 
        $p_1(x) = \var{f}$.}}
        \lang{fr}{\text{Le graphe rouge est le polynôme
        $p_1(x) = \var{f}$.}}
        \lang{zh}{\text{多项式 
        $p_1(x) = \var{f}$ 对应红色的函数图像。}}
	  	\solution{false}
	\end{choice}
  		
  	\begin{choice}
	    \lang{de}{\text{Das Polynom 
        $p_2 (x) = \var{g}$ hat den blauen Funktionsgraphen.}}
	    \lang{en}{\text{The blue graph is the polynomial
        $p_2 (x) = \var{g}$.}}
        \lang{fr}{\text{Le graphe bleu est le polynôme 
        $p_2 (x) = \var{g}$.}}
        \lang{zh}{\text{多项式 
        $p_2 (x) = \var{g}$ 对应蓝色的函数图像。}}
	 	\solution{false}
  	\end{choice}
        
    \begin{choice}
	    \lang{de}{\text{Das Polynom 
        $p_2(x) = \var{g}$ hat den roten Funktionsgraphen.}}
	    \lang{en}{\text{The red graph is the polynomial 
        $p_2(x) = \var{g}$.}}
        \lang{fr}{\text{Le graphe rouge est le polynôme
        $p_2(x) = \var{g}$.}}
        \lang{zh}{\text{多项式 
        $p_2(x) = \var{g}$ 对应红色的函数图像。}}
	  	\solution{true}
    \end{choice}
\end{question}

\end{problem}
\embedmathlet{gwtmathlet}

\end{content}
