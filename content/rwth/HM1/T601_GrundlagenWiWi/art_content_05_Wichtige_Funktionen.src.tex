
%$Id:  $
\documentclass{mumie.article}
%$Id$
\begin{metainfo}
  \name{
    \lang{de}{Wichtige Funktionen}

    \lang{en}{Important functions}
  }
  \begin{description} 
 This work is licensed under the Creative Commons License Attribution 4.0 International (CC-BY 4.0)   
 https://creativecommons.org/licenses/by/4.0/legalcode 

    \lang{de}{Beschreibung}
    \lang{en}{}
  \end{description}
  \begin{components}
    \component{generic_image}{content/rwth/HM1/images/g_tkz_T601_SineCosine_B.meta.xml}{T601_SineCosine_B}
    \component{generic_image}{content/rwth/HM1/images/g_tkz_T601_Triangle_A.meta.xml}{T601_Triangle_A}
    \component{generic_image}{content/rwth/HM1/images/g_tkz_T601_Logarithms_B.meta.xml}{T601_Logarithms_B}
    \component{generic_image}{content/rwth/HM1/images/g_tkz_T601_Exponentials_B.meta.xml}{T601_Exponentials_B}
    \component{generic_image}{content/rwth/HM1/images/g_tkz_T601_Exponentials_A.meta.xml}{T601_Exponentials_A}
    \component{generic_image}{content/rwth/HM1/images/g_tkz_T601_LinearFunction.meta.xml}{T601_LinearFunction}
    \component{generic_image}{content/rwth/HM1/images/g_img_00_Videobutton_schwarz.meta.xml}{00_Videobutton_schwarz}
    \component{generic_image}{content/rwth/HM1/images/g_img_00_Videobutton_blau.meta.xml}{00_Videobutton_blau}
    \component{js_lib}{system/media/mathlets/GWTGenericVisualization.meta.xml}{mathlet1}
  \end{components}
  \begin{links}
    \link{generic_article}{content/rwth/HM1/T601_GrundlagenWiWi/g_art_content_04_Funktionsbegriff.meta.xml}{content_04_Funktionsbegriff}
    \link{generic_article}{content/rwth/HM1/T101neu_Elementare_Rechengrundlagen/g_art_content_05_loesen_gleichungen_und_lgs.meta.xml}{content_05_loesen_gleichungen_und_lgs}
    \link{generic_article}{content/rwth/HM1/T102neu_Einfache_Reelle_Funktionen/g_art_content_06_funktionsbegriff_und_lineare_funktionen.meta.xml}{content_06_funktionsbegriff_und_lineare_funktionen}
    \link{generic_article}{content/rwth/HM1/T102neu_Einfache_Reelle_Funktionen/g_art_content_07_geradenformen.meta.xml}{content_07_geradenformen}
    \link{generic_article}{content/rwth/HM1/T102neu_Einfache_Reelle_Funktionen/g_art_content_08_quadratische_funktionen.meta.xml}{content_08_quadratische_funktionen}
    \link{generic_article}{content/rwth/HM1/T103_Polynomfunktionen/g_art_content_09_polynome.meta.xml}{content_09_polynome}
    \link{generic_article}{content/rwth/HM1/T403a_Vektorraum/g_art_content_10b_lineare_abb.meta.xml}{content_10b_lineare_abb}
    \link{generic_article}{content/rwth/HM1/T204_Abbildungen_und_Funktionen/g_art_content_10_abbildungen_verkettung.meta.xml}{content_10_abbildungen_verkettung}
\end{links}
  \creategeneric
\end{metainfo}


\begin{content}
\title{
  \lang{de}{Wichtige Funktionen}
  \lang{en}{Important functions}
}
\usepackage{mumie.ombplus}
\ombchapter{1}
\ombarticle{5}
\usepackage{mumie.genericvisualization}



\begin{block}[annotation]
	Im Ticket-System: \href{https://team.mumie.net/issues/22666}{Ticket 22666}
\end{block}

\tableofcontents
\begin{visualizationwrapper}
\lang{de}{Wir stellen in diesem Kapitel die am häufigsten gebrauchten Funktionen vor. }
\lang{en}{
This chapter introduces the most commonly used functions.
}
\section{\lang{de}{Lineare Funktionen} \lang{en}{Linear functions}} \label{sec:linear}
\lang{de}{Viele Zusammenhänge lassen sich (zumindest in kleinen Intervallen) gut durch lineare Funktionen bzw. Geraden 
beschreiben. Da Geraden auch relativ leicht zu bestimmen und aufzustellen sind, gehören sie mit zu den wichtigsten
Funktionen. }
\lang{en}{
Many relationships can be represented well (at least over short intervals)
by linear functions, i.e. lines. Since lines are relatively easy to determine and to describe,
they are some of the most important functions.
}

\begin{definition}[\lang{de}{Lineare Funktion/Gerade} \lang{en}{Linear functions / Lines}] \label{def:linear_func}   
    \lang{de}{Eine Funktion $g:\R\to \R\,$ der Form
    \begin{align*}
		g(x)=mx+b 
	\end{align*}
   
	mit beliebigen, aber festen reellen Zahlen $m$ und $b$ hei\"st \notion{\emph{lineare Funktion.}} % oder \notion{\emph{Gerade}}. 
    }
    \lang{en}{
A function $g:\R\to \R\,$ of the form
\begin{align*}
		g(x)=mx+b 
	\end{align*}
for fixed real numbers $m$ and $b$ is called a \notion{\emph{linear function.}}  % or a \notion{\emph{line}}.
    }

    \lang{de}{
    Der Graph einer linearen Funktion ist eine 
    \notion{\emph{Gerade}}. Der Parameter \emph{$m$} beschreibt dabei die \notion{\emph{Steigung}} und
    \emph{$b$} den \notion{\emph{Ordinaten-}} bzw. \notion{\emph{$y$-Achsenabschnitt}}.
  	}
   \lang{en}{
    The graph of a linear function is a
    \notion{\emph{line}}. Here, the parameter \emph{$m$} stands for the \notion{\emph{slope}} and
    \emph{$b$} for the \emph{$y$-intercept}.
  	}
   
\end{definition}   

\lang{de}{Die Parameter $m$ und $b$ besitzen folgende anschauliche Bedeutung:}
\lang{en}{The parameters $m$ and $b$ have the following descriptive meaning:}

    \lang{de}{Der $y$-Achsenabschnitt $b$ gibt an, dass die Gerade $g$ die $y$-Achse
    im Punkt $(0;b)$ schneidet.}
    \lang{en}{
    The $y$-intercept being $b$ indicates that the line $g$ intersects
    the $y$-axis in the point $(0;b)$.
    }
    \begin{itemize}
    \item \lang{de}{Ist $b=0$, so verläuft die Gerade durch den \textit{Ursprung} 
    $(0;0)$ des Koordinatensystems (es ist $g(x)=mx$).}
    \lang{en}{
    If $b=0$, then the line runs through the \textit{origin}
    $(0;0)$ of the coordinate system (because $g(x)=mx$).
    }
    \item \lang{de}{ Ist $b>0$, so schneidet die Gerade die $y$-Achse oberhalb 
    der $x$-Achse.}
    \lang{en}{
    If $b>0$, then the line intersects the $y$-axis
    above the $x$-axis.
    }
    \item \lang{de}{Ist $b<0$, so schneidet die Gerade die $y$-Achse unterhalb 
    der $x$-Achse.}
    \lang{en}{
    If $b<0$, then the line intersects the $y$-axis
    below the $x$-axis.
    }
    \end{itemize}
    \lang{de}{Die Steigung $m$ gibt an, um wie viel die Gerade steigt, wenn man $x$  um $1$ 
    erh"oht.}
    \lang{en}{
    The slope $m$ indicates by how much the line increases in height
    if $x$ increases by $1$.
    }
    \begin{itemize}
    \item \lang{de}{Ist $m=0$, so ist die Gerade \emph{parallel} zur $x$-Achse (es ist $g(x)=b$).}
    \lang{en}{If $m=0$, then the line is \emph{parallel} to the $x$-axis (we have $g(x)=b$).}
    \item \lang{de}{Ist $m>0$, so \emph{steigt} die Gerade (und zwar mit wachsendem $m$  immer \emph{steiler}.}
    \lang{en}{}If $m>0$, then the line is \emph{increasing} (and for greater $m$ it increases more \emph{steeply}).

\item \lang{de}{Ist $m<0$, so \emph{f"allt} die Gerade.}
\lang{en}{If $m<0$, then the line is \emph{decreasing}.}
\end{itemize}

\lang{de}{Die Gerade in der folgenden Abbildung besitzt eine negative Steigung und einen positiven Ordinatenabschnitt.
}
\lang{en}{
The line in the image below has a negative slope and a positive $y$-intercept.
}
    \begin{center}
        \image{T601_LinearFunction}
    \end{center}

\lang{de}{Wenn zwei Punkte der Gerade gegeben sind, kann die Gerade, die durch beide Punkte geht, leicht aufgestellt werden.
}
\lang{en}{
Given two points, the line that passes through
both of them can be computed in a straightforward way.
}

\begin{rule}[\lang{de}{Aufstellen der Geradengleichung} \lang{en}{Equation of a line}]\label{rule:geradengleichung}
\lang{de}{Es seien zwei Punkte $P_1(x_1;y_1)$ und $P_2(x_2;y_2)$ mit unterschiedlichen $x$-Koordinaten gegeben. 
Die Gerade durch $P_1$ und $P_2$ hat die Steigung
\[
m = \frac{y_2 - y_1}{x_2 - x_1}.
\]
Den Ordinatenabschnitt $b$ können wir ausrechnen, indem wir den Wert von $m$ und die Koordinaten eines Punktes von oben 
in die Gleichung $y = mx + b$ einsetzen und nach $b$ auflösen. Als allgemeine Formel ergibt sich 
\[
b = y_1 - x_1 \cdot \frac{y_2-y_1}{x_2 - x_1}.
\]}
\lang{en}{
Suppose we are given two points $P_1(x_1;y_1)$ and $P_2(x_2;y_2)$ with distinct $x$-coordinates.
The line passing through $P_1$ and $P_2$ has slope
\[
m = \frac{y_2 - y_1}{x_2 - x_1}.
\]
The $y$-intercept $b$ can be calculated by plugging in the value of $m$ and the coordinates of either point
above in the equation $y = mx + b$, and then solving for $b$. This yields the general formula
\[
b = y_1 - x_1 \cdot \frac{y_2-y_1}{x_2 - x_1}.
\]
}
\end{rule}
\lang{de}{Häufig werden mehrere Geraden zu \emph{stückweise linearen Funktionen} zusammengesetzt - wie im 
folgenden Beispiel. }
\lang{en}{
Often, multiple lines are combined to \emph{piecewise linear functions},
as in the following example.
}

\begin{example}
\lang{de}{Ein Händler verkauft Aktenordner für 1,80 € pro Stück. Bei der Abnahme von mindestens 50 Ordnern 
sinkt der Stückpreis auf 1,60 €. Die Kosten für $x$ Ordner (in Euro) betragen also 
\[
K(x) = \left\{ 
   \begin{matrix}
   1,80 \cdot x, &\text{ falls }x <50{,}\\
1,60 \cdot x, &\text{ falls } x \geq 50 \text{.}
\end{matrix} 
 \right.
\]}
\lang{en}{
A vendor is selling folders for 1.80 € each.
For orders of at least 50 folders, the price per unit
is reduced to 1.60 €. The cost of buying $x$ folders is then
\[
K(x) = \left\{ 
   \begin{matrix}
   1.80 \cdot x, &\text{ if }x <50{,}\\
1.60 \cdot x, &\text{ if } x \geq 50 \text{.}
\end{matrix} 
 \right.
\]
}


\lang{de}{Ein anderer Händler übernimmt diese Preise, aber gewährt den reduzierten Preis von 1,60 € nicht für 
die ersten 49 Ordner, sondern erst ab dem 50. Ordner. Die Kosten betragen in diesem Fall
\[
K(x) = \left\{ 
   \begin{matrix}
   1,80 \cdot x, &\text{ falls }x <50{,}\\
1,80 \cdot 49 + 1,60 \cdot (x - 49), &\text{ falls } x \geq 50 \text{.}
\end{matrix} 
 \right.
\]}
\lang{en}{
A second vendor uses the above prices but only
offers the reduced price of 1.60 € for all folders after the 50th,
not for the first 49. In this case, the cost is
\[
K(x) = \left\{ 
   \begin{matrix}
   1.80 \cdot x, &\text{ if }x <50{,}\\
1.80 \cdot 49 + 1.60 \cdot (x - 49), &\text{ if } x \geq 50 \text{.}
\end{matrix} 
 \right.
\]
}


\lang{de}{Bei einer Bestellung von über 50 Ordnern setzen sich die Kosten hier zusammen aus den Kosten für die 
ersten 49 Ordner und den ermäßigten Kosten für jeden folgenden Ordner.}
\lang{en}{
For orders of more than 50 folders, the price above consists of the
price of the first 49 folders and the reduced price for all of the
other folders.
}


\end{example}

\begin{quickcheckcontainer}
\randomquickcheckpool{1}{2}
\randomquickcheckpool{3}{3}
\begin{quickcheck}
		\field{rational}
		\type{input.number}
		\begin{variables}
			\randint[Z]{a}{-5}{5}
			\randint[Z]{b}{1}{2}
			\randint{c}{-4}{4}
			\randint[Z]{d}{3}{5}
			\function[calculate]{m}{a/b}
			\function[calculate]{y}{c/d}
		    \function[normalize]{f}{m*x+y}
		\end{variables}

      \lang{de}{
			\text{Für die lineare Funktion $f(x)=\var{f}$ hat die zugehörige Gerade
			 die Steigung \ansref und den $y$-Achsenabschnitt \ansref.}}
      \lang{en}{
      \text{Given the linear function $f(x)=\var{f}$, the associated line
      has slope \ansref and $y$-intercept \ansref.}
      }
		
		\begin{answer}
			\solution{m}
		\end{answer}
		\begin{answer}
			\solution{y}
		\end{answer}
		\explanation{
    \lang{de}{Die Steigung ist der Koeffizient von $x$, also der Faktor, mit dem $x$ multipliziert wird.\\
		Der $y$-Achsenabschnitt ist die Zahl, die hinzuaddiert wird, d.\,h. der Wert von $f$ an der Stelle $0$.}
    \lang{en}{
    The slope is the coefficient of $x$, or the factor that $x$ is multiplied by.
    The $y$-intercept is the number that is added on, i.e. the value of $f(x)$ at $x=0$.
    }
    }
	\end{quickcheck}

\begin{quickcheck}
		\field{rational}
		\type{input.number}
		\begin{variables}
			\randint[Z]{a}{-5}{5}
			\randint[Z]{b}{2}{4}
			\randint{c}{-4}{4}
			\function[calculate]{m}{a}
			\function[calculate]{y}{c/b}
		    \function[normalize]{f0}{a*x+c/b}
		    \function{f}{f0}
		\end{variables}

      \lang{de}{
			\text{Für die lineare Funktion $f(x)=\var{f}$ hat die zugehörige Gerade
			 die Steigung \ansref und den $y$-Achsenabschnitt \ansref.}}
      \lang{en}{
        \text{Given the linear function $f(x)=\var{f}$, the associated line
        has slope \ansref and $y$-intercept \ansref.}}
		
		\begin{answer}
			\solution{m}
		\end{answer}
		\begin{answer}
			\solution{y}
		\end{answer}
		\explanation{
    \lang{de}{Die Steigung ist der Koeffizient von $x$, also der Faktor, mit dem $x$ multipliziert wird.\\
		Der $y$-Achsenabschnitt ist die Zahl, die hinzuaddiert wird, d.\,h. der Wert von $f$ an der Stelle $0$.}
    \lang{en}{
    The slope is the coefficient of $x$, or the factor that $x$ is multiplied by.
    The $y$-intercept is the number that is added on, i.e. the value of $f(x)$ at $x=0$.
    }
    }
  \end{quickcheck}

\begin{quickcheck}
		\field{rational}
		\type{input.function}
		\begin{variables}
			\randint[Z]{a}{-5}{5}
			\randint[Z]{b}{1}{4}
			\randint{c}{-4}{4}
			\randint[Z]{d}{1}{4}
			\function[calculate]{m}{a/b}
			\function[calculate]{y}{c/d}
		    \function[normalize]{f}{m*x+y}
		\end{variables}

      \lang{de}{
			\text{Für die Gerade mit Steigung $\var{m}$ und $y$-Achsenabschnitt $\var{y}$
			 ist die zugehörige Funktionsgleichung $y=$\ansref.}}
      \lang{en}{
      \text{The line with slope $\var{m}$ and $y$-intercept $\var{y}$
      has the equation $y=$\ansref.}
      }
    
		
		\begin{answer}
			\solution{f}
			\checkAsFunction{x}{-10}{10}{100}
		\end{answer}
		\explanation{\lang{de}{Die Funktionsgleichung ergibt sich als $y=mx+b$, wobei $m$ die Steigung und $b$ der $y$-Achsenabschnitt ist.}
      \lang{en}{The equation of the line is given by $y=mx+b$, where $m$ is the slope and $b$ the $y$-intercept.}}
	\end{quickcheck}

\end{quickcheckcontainer}

\section{\lang{de}{Polynomfunktionen} \lang{en}{Polynomial functions}}

\lang{de}{Die linearen Funktionen gehören zu den allgemeineren Polynomfunktionen, wo auch höhere Potenzen 
der Funktionsvariablen auftreten dürfen.}
\lang{en}{
Linear functions are a special case of polynomial functions.
In the latter, higher powers of the variable are also allowed to appear.
}


\begin{definition}[\lang{de}{Polynome} \lang{en}{Polynomials}] \label{polynomial}
\lang{de}{Funktionen der Form
\begin{equation*}
p(x) = a_n\, x^n + a_{n-1} \, x^{n-1} + a_{n-2} \, x^{n-2} + \ldots + a_1 \, x + a_0
\end{equation*}
mit Koeffizienten $a_0,\, a_1,\,\ldots, a_{n-1},\, a_n \in \mathbb{R}$ heißen 
\notion{\emph{Polynomfunktionen.}} 

Der größte Exponent der $x$-Potenzen in obiger Darstellung heißt \notion{Grad} der Polynomfunktion, wobei der
Koeffizient der betreffenden Potenz nicht null sein darf. 

Gilt für den Koeffizienten der höchsten Potenz $a_n \neq 0$, 
so sagen wir etwa, dass $p(x)$ eine Polynomfunktion 
\notion{von Grad n} ist. Auch gebräuchlich ist die Bezeichnung \notion{ganzrationale Funktion
$n$. Grades}. Die Zahl $a_n$, die als Vorfaktor bei der höchsten Potenz von $x$ steht, bezeichnet man als \textbf{Leitkoeffizienten}.}

\lang{en}{
A function of the form
\begin{equation*}
p(x) = a_n\, x^n + a_{n-1} \, x^{n-1} + a_{n-2} \, x^{n-2} + \ldots + a_1 \, x + a_0
\end{equation*}
with coefficients $a_0,\, a_1,\,\ldots, a_{n-1},\, a_n \in \mathbb{R}$
is called a \notion{\emph{polynomial function.}}

The highest power of $x$ in the above representation
for which the corresponding coefficient is nonzero
is called the \notion{degree} of the polynomial function.

If $a_n \neq 0$, then we call $p(x)$ a polynomial
of \notion{degree n}. The coefficient $a_n$ of 
the highest power of $x$ is called the
\notion{leading coefficient}.
}

\end{definition}

\lang{de}{Geraden sind nach dieser Definition Polynomfunktionen vom Grad 1 (wenn Steigung $m \neq 0$ gilt)
oder vom Grad 0 (wenn $m = 0$ gilt).}

\lang{en}{
By this definition, lines are polynomial functions of
degree $1$ (if their slope $m$ is nonzero) or of
degree $0$ (if $m=0$).
}.

\begin{example}
 \lang{de}{Polynome  unterschiedlichen Grades:}
 \lang{en}{Some polynomials of various degrees:}
\begin{align*}
p_0(x) &= -2 &&\quad\text{\lang{de}{Grad}\lang{en}{degree}} = 0,\\
p_1(x) &= 3x-2 &&\quad\text{\lang{de}{Grad}\lang{en}{degree}} = 1,\\
p_2(x) &= (x+2)^2 - 2 &&\quad\text{\lang{de}{Grad}\lang{en}{degree}} = 2,\\
p_3(x) &= -3x^3 + 2x^2 + 3x-2 &&\quad\text{\lang{de}{Grad}\lang{en}{degree}} = 3,\\
p_4(x) &= (x-1)\cdot(x-6)\cdot(x+3)\cdot(x-4) &&\quad\text{\lang{de}{Grad}\lang{en}{degree}} = 4.
\end{align*}

\lang{de}{Die Polynome $p_2$ und $p_4$ können wie in der Definition angegeben hingeschrieben werden, in dem 
man die Klammern ausmultipliziert. Dann ist z.\,B. 
\begin{equation*}
p_2(x) = (x+2)^2 - 2  = (x^2 + 4x + 4) - 2 = x^2 + 4x + 2.
\end{equation*}
Die Darstellung von $p_4(x)$ heißt \emph{Linearfaktorzerlegung}. Durch sie lassen sich sofort die Nullstellen der Funktion ablesen: 1, 6, -3, 4.
}
\lang{en}{
The polynomials $p_2$ and $p_4$ can be put in the form
of the definition by multiplying out the expressions in
parentheses. For example,
\begin{equation*}
p_2(x) = (x+2)^2 - 2  = (x^2 + 4x + 4) - 2 = x^2 + 4x + 2.
\end{equation*}
The representation of $p_4(x)$ is called its
\emph{linear factorization}. It allows us to read off
the zeros of the function directly: 1, 6, -3, 4.
}



\lang{de}{In der folgenden Visualisierung können Sie beliebige Polynomfunktionen mit bis zu sechs Summanden 
anzeigen lassen. }

\lang{en}{
In the visualization below, you can display any
polynomial involving at most six summands.
}



 \begin{genericGWTVisualization}[280][300]{mathlet1}
\lang{de}{\title{Polynom}}
\lang{en}{\title{Polynomials}}

	\begin{variables}

	\function[editable]{P5}{rational}{x^5}
	\function[editable]{P4}{rational}{0*x^4}
	\function[editable]{P3}{rational}{-4*x^3}
	\function[editable]{P2}{rational}{x^2}
	\function[editable]{P1}{rational}{3*x}
	\function[editable]{P0}{rational}{-2}

	\function{P5x}{rational}{var(P5)}
	\function{P4x}{rational}{var(P4)}
	\function{P3x}{rational}{var(P3)}
	\function{P2x}{rational}{var(P2)}
	\function{P1x}{rational}{var(P1)}
	\function{P0x}{rational}{var(P0)}

	\function{P}{rational}{var(P5)+var(P4)+var(P3)+var(P2)+var(P1)+var(P0)}
	\end{variables}


	\color{P}{#0066CC}

	\begin{canvas}
	\plotSize{380}
	\plotLeft{-5}
	\plotRight{5}
	\plot[coordinateSystem]{P}
	\end{canvas}
	\lang{de}{\text{Fügen Sie hier die Summanden (Vorfaktor mit $x$-Potenz) ein.}}
  \lang{en}{\text{Enter the summands (coefficients and powers of $x$) here.}}
	\text{$p(x) = \var{P5}+ (\var{P4})+ (\var{P3})+ (\var{P2})+ (\var{P1})+ (\var{P0})$
	\phantom{}}
\text{$p(x) =  \var{P}$} 
\end{genericGWTVisualization}
\end{example}
\lang{de}{Für das asymptotische Verhalten von Polynomfunktionen gilt die folgende Regel:}
\lang{en}{The asymptotic behavior of a polynomial function
is governed by the following rule:}
\begin{rule}
\lang{de}{
Für $x \to \pm \infty$ hängt das Verhalten der
     Polynomfunktion $p(x) = a_n x^n + \ldots + a_1 x + a_0$ (mit $a_n \neq 0$) 
     vom      Summanden mit der größten Potenz ab: 
     \[
     \lim_{x \to \pm\infty} \left(a_n x^n + \ldots + a_1 x + a_0 \right)= \lim_{x \to \pm\infty} a_n x^n.
     \]
}
\lang{en}{
The behavior of the polynomial $p(x) = a_n x^n + \ldots + a_1 x + a_0$ (with $a_n \neq 0$)
as $x \to \pm \infty$ depends only on the summand with the highest power of $x$:
      \[
     \lim_{x \to \pm\infty} \left(a_n x^n + \ldots + a_1 x + a_0 \right)= \lim_{x \to \pm\infty} a_n x^n.
     \]
}
  \lang{de}{
   Weiterhin gilt 
   \[\lim_{x \to \infty} x^n = +\infty, \quad
   \lim_{x \to -\infty} x^n = \left\{ 
   \begin{matrix}
   +\infty, &\text{ falls } n \text{ gerade, }\\
-\infty, &\text{ falls } n \text{ ungerade.}
\end{matrix} 
 \right.
   \]
  }
  \lang{en}{
   Furthermore, we have 
   \[\lim_{x \to \infty} x^n = +\infty, \quad
   \lim_{x \to -\infty} x^n = \left\{ 
   \begin{matrix}
   +\infty, &\text{ if } n \text{ is even, }\\
-\infty, &\text{ if } n \text{ is odd.}
\end{matrix} 
 \right.
   \]
  }
  
\end{rule}

\lang{de}{Für das Verhalten von Polynomfunktionen in der Nähe von $0$ schaut man sich hingegen nur die Potenzen mit den 
niedrigsten Exponenten an. Meistens lässt man in der Funktionsgleichung dazu alle $x$-Potenzen weg, deren Exponent 
größer als $1$ ist. }
\lang{en}{
On the other hand, to determine the behavior of a polynomial function as it approaches zero,
we only need to consider the terms involving the smallest powers of the variable.
This is generally done by omitting all terms in the defining equation
that involve a power of $x$ greater than $1$.
}

\begin{quickcheck}
		\field{rational}
		\type{input.function}
		\begin{variables}
			\randint{p1}{0}{1}
			\randint{p2}{0}{1}
			\randint{p3}{0}{2}
			\randint[Z]{a}{-5}{5}
			\randint{b1}{-3}{4}
			\randint{b2}{-3}{4}
			\randint[Z]{c}{-4}{4}
			\randint{d}{1}{4}
			\function[normalize]{ht}{p1*a*x^6+(1-p1)*a*x^5}  %ax^5 oder ax^6
			\function[normalize]{mt}{(2-p3)*b1*x^4+p3*b2*x^3-p3*p2*(b1+b2)*x^2}  %x^4 und x^3-terme und evtl. auch x^2, wenn nt keinen hat.
			\function[normalize]{nt}{(1-p2)*c*x^2+p2*c*x+d}   %cx^2+d oder cx+d
		    \function[normalize,sort]{f}{ht+mt+nt}
		    \function[normalize]{gr}{p1*6+(1-p1)*5}
		    \function[normalize]{lc}{a}
		\end{variables}

      \lang{de}{
			\text{Die Polynomfunktion $f(x)=\var{f}$ hat\\ 
			Grad \ansref und Leitkoeffizient \ansref.}
			\text{Für $x\to \pm \infty$ verhält sie sich wie die Funktion \ansref.}
            \explanation{Für den Grenzwert gegen $ \pm \infty$ benötigen Sie den Summanden mit dem höchsten Grad.}
      }
      \lang{en}{
      \text{The polynomial $f(x)=\var{f}$ \\
      is of degree \ansref and has leading coefficient \ansref.
      }
      \text{In the limit $x\to \pm \infty$, it behaves like the function \ansref.}
      \explanation{For the limit $x\to \pm \infty$, we need to consider the summand of the highest degree.}
      }      
		
		\begin{answer}
			\solution{gr}
			\checkAsFunction{x}{-2}{2}{100}
		\end{answer}
		\begin{answer}
			\solution{lc}
			\checkAsFunction{x}{-2}{2}{100}
		\end{answer}
		\begin{answer}
			\solution{ht}
			\checkAsFunction{x}{-2}{2}{100}
		\end{answer}
	\end{quickcheck}
	

\lang{de}{Über die Nullstellen von Polynomfunktionen lässt sich Folgendes aussagen:}
\lang{en}{
The following can be said about the zeros of polynomial functions:
}
\begin{theorem}
\begin{enumerate}
\item \lang{de}{Eine Polynomfunktion $p$ vom Grad $n$ besitzt höchstens $n$ verschiedene reelle Nullstellen.}
\lang{en}{
A polynomial $p$ of degree $n$ has at most $n$ distinct real zeros.
}
      
\item \lang{de}{Ist $c$ eine Nullstelle von $p$, so gibt es eine Polynomfunktion $q$ vom Grad $n-1$ mit
~\\
\begin{center}$p(x)=q(x)\cdot (x-c) \quad \text{für alle }x\in \R.$\end{center}
}
\lang{en}{If $c$ is a zero of $p$, then there exists a polynomial $q$ of degree $n-1$ with
~\\
\begin{center}$p(x)=q(x)\cdot (x-c) \quad \text{for all }x\in \R.$\end{center}
}
\end{enumerate}
%\floatright{\href{https://www.hm-kompakt.de/video?watch=117}{}\image[75]{00_Videobutton_schwarz}}\\\\
\end{theorem}


\section{
\lang{de}{Exponentialfunktion und Logarithmus}
\lang{en}{Exponential function and logarithm}
}

\lang{de}{Bisher haben wir nur Potenzen der Form $x^n$ betrachtet, bei denen die Funktionsvariable die Basis der Potenz ist. Die Funktionen $2^x, 10^x, e^x$ 
(wobei $e$ die {Eulersche Zahl} ist) sind Beispiele von "`Exponentialfunktionen"',
denn die Variable steht im Exponenten.}
\lang{en}{
Thus far, we have only considered powers of the form $x^n$ in which
the variable is the base. The functions $2^x$, $10^x$
and $e^x$ (where $e$ is the {Euler number}) are examples
of "exponential functions" because the variable
appears in the exponent.
}

\begin{definition}[
\lang{de}{Exponentialfunktion}
\lang{en}{Exponential function}]
   \lang{de}{Die Funktionen}
   \lang{en}{The functions}
     \begin{equation*}
    f(x) = a^x
      \end{equation*}
    \lang{de}{mit positiver Basis $a > 0$ heißen \notion{\emph{Exponentialfunktionen}}. Sie sind für alle reellen Zahlen $x$ definiert: $D_f = \mathbb{R}$. 
    
    Die Zahl $a$ heißt \notion{Basis}.}

    \lang{en}{for a positive base $a > 0$ are called \notion{\emph{exponential functions}}. They are defined for all real numbers $x$: $D_f = \mathbb{R}$.

    The number $a$ is called the \notion{base}.}
    
   \lang{de}{Die Exponentialfunktion mit der Basis $e$ heißt \notion{\emph{natürliche Exponentialfunktion}} oder kurz \notion{\emph{$e$-Funktion}}}
   \lang{en}{The exponential function to the base $e$ is called the \notion{\emph{natural exponential function}}, or \notion{\emph{exp-function}} for short.}
     \begin{equation*}
    f(x) = e^x = \exp(x).
      \end{equation*}
      \lang{de}{Dabei ist die \emph{Eulersche Zahl}}
      \lang{en}{Here, the \emph{Euler number}}
     \begin{equation*} e \approx 2\lang{de}{,}\lang{en}{.}718281828459...\end{equation*}
     \lang{de}{der Wert der Exponentialreihe $\sum_{n=0}^\infty \frac{1}{n!}$. Sie ist eine irrationale Zahl.}
      \lang{en}{is the value of the exponential series $\sum_{n=0}^\infty \frac{1}{n!}$. This is an irrational number.}
   % \floatright{\href{https://www.hm-kompakt.de/video?watch=141}{\image[75]{00_Videobutton_schwarz}}}\\~
\end{definition}

  \lang{de}{Die natürliche Exponentialfunktion ist eine der wichtigsten
  Funktionen der Mathematik und Naturwissenschaften. Sie
  hat die ganz besondere Eigenschaft, dass sie mit ihrer Ableitung
  (siehe Abschnitt {Ableitung}) übereinstimmt, das heißt, dass $(e^x)' = e^x$
  gilt.}
  \lang{en}{
  The natural exponential function is one of the most important
  functions in mathematics and the natural sciences.
  It has the very special property of coinciding with
  its own derivative (see the section on {derivatives}); that is, $(e^x)' = e^x$.9
  }


\lang{de}{Alle Exponentialfunktionen haben die Eigenschaft $f(0)=a^0=1$. Der Verlauf der Exponentialfunktionen
ist von der Basis $a$ abhängig und kann in zwei Fälle unterteilt werden: $0 < a < 1$ und $a > 1$. (Der Fall $a = 1$ wird meist 
ausgeschlossen, da die Funktion $f(x) = 1^x = 1$ konstant ist.) 
}
\lang{en}{
All exponential functions have the property that $f(0) = a^0 = 1$.
The shape of the function depends on the base $a$ and
falls into two cases: $0 < a < 1$ or $a > 1$.
(The case $a=1$ is usually excluded because the
function $f(x) = 1^x = 1$ is constant.)
}

\lang{de}{In der folgenden Abbildung
ist der Zusammenhang von Basiswert und Verlauf der Exponentialfunktionen ersichtlich.
}
\lang{en}{
The relationship between the base and the shape of the
function can be seen in the figure below.
}


\begin{table}[\class{item}]
\image{T601_Exponentials_A} & \image{T601_Exponentials_B}
\end{table}


\begin{example}
   \begin{tabs*}[\initialtab{1}\class{example}]
 \tab{$f(x) = a^x, \; a > 1$ }
   \begin{table}
  \head
  $f(x) = a^x$ & $a > 1, \;\; a \in \mathbb{R}$ 
  \body
  $D_f = \mathbb{R}$ & \lang{de}{Definitionsbereich ist die Menge aller reellen Zahlen.}\lang{en}{The domain is the set of all real numbers.} \\
  $W_f = \mathbb{R}^+$  & \lang{de}{Wertemenge ist die Menge der positiven reellen Zahlen.}\lang{en}{The range is the set of all positive real numbers.} \\
  $f(x) \neq 0$& \lang{de}{Die Funktionen haben keine Nullstellen.}\lang{en}{The functions have no zeros.}\\
  $f(x_1) < f(x_2)$ \lang{de}{für}\lang{en}{for} $x_1 < x_2$&  \lang{de}{Die Funktionen sind streng monoton wachsend.}\lang{en}{The functions are strictly monotonically increasing.}\\
  \lang{de}{Die Steigung von $f$ nimmt ständig zu.}\lang{en}{The slope is increasing}  & \lang{de}{Der Graph beschreibt eine Linkskurve.}\lang{en}{The graph is concave up} \\
  $\lim_{x \to -\infty}f(x) = 0, \ \lim_{x \to \infty}f(x) = \infty$& \lang{de}{Die Funktionen streben von $0$ nach $\infty$.} \lang{en}{The functions go from $0$ to $\infty$.}
  \\
  \end{table}
 \tab{$f(x) = a^{x}, \; 0 < a < 1$}
\begin{table}
  \head
  $f(x) = a^{x} = \left(\frac{1}{a}\right)^{-x}$ & $ 0 < a < 1, \;\; a \in \mathbb{R}$ 
  \body
  $D_f = \mathbb{R}$ & \lang{de}{Definitionsbereich ist die Menge aller reellen Zahlen.}\lang{en}{The domain is the set of all real numbers.} \\
  $W_f = \mathbb{R}^+$  &  \lang{de}{Wertemenge ist die Menge der positiven reellen Zahlen.}\lang{en}{The range is the set of all positive real numbers.} \\
  $f(x) \neq 0$& \lang{de}{Die Funktionen haben keine Nullstellen.}\lang{en}{The functions have no zeros.}\\
  $f(x_1) > f(x_2)$ \lang{de}{für}\lang{en}{for} $x_1 < x_2$&  \lang{de}{Die Funktionen sind streng monoton fallend.}\lang{en}{The functions are strictly monotonically decreasing.}\\
  \lang{de}{Die Steigung von $f$ nimmt ständig zu.}\lang{en}{The slope is increasing} & \lang{de}{Der Graph beschreibt eine Linkskurve.}\lang{en}{The graph is concave up}  \\
  $\lim_{x \to -\infty}f(x) = \infty, \ \lim_{x \to \infty}f(x) = 0$& \lang{de}{Die Funktionen streben von $\infty$ nach $0$.} \lang{en}{The functions go from $\infty$ to $0$.}\\
  \end{table}
 \end{tabs*}
\end{example}

\begin{quickcheck}
		\field{rational}
		\type{input.number}
		\begin{variables}
			\randint{a1}{1}{4}
			\randint{a2}{1}{4}		
			\function[calculate]{a}{a1/a2} % a=1 ausschließen???
			\randint[Z]{xp}{-1}{1} % sollte -4 .. 4 sein
			\function[calculate]{yp}{a^xp}
		\end{variables}
  		\lang{de}{
			\text{Für welches $a>0$ geht der Graph der Exponentialfunktion $f(x)=a^x$ durch den Punkt $P=(\var{xp};\var{yp})$?\\
			Für $a=$\ansref.}
     }
     \lang{en}{
      \text{For which $a>0$ does the graph of the exponential function
      $f(x)=a^x$ go through the point $P=(1;2)$?\\
      $a=$\ansref.}
     }
		
		\begin{answer}
			\solution{a}
		\end{answer}
    \lang{de}{
		\explanation{Wenn der Punkt $P=(\var{xp};\var{yp})$ auf dem Graphen liegt, gilt: $f(\var{xp})=\var{yp}$, 
		also $a^\var{xp}=\var{yp}$.}
    }
    \lang{en}{
    \explanation{If the point $P=(\var{xp};\var{yp})$ lies on the graph, then $f(\var{xp})=\var{yp}$,
    and therefore $a^\var{xp}=\var{yp}$.}
    }
	\end{quickcheck}


\lang{de}{Bei der Beschreibung von Prozessen in den Natur-, Ingenieur- und Wirtschaftswissenschaften durch
mathematische Modelle ("`mathematische Modellbildung"') treten sehr oft Exponentialfunktionen auf, 
insbesondere, um den zeitlichen Verlauf wiederzugeben. Die unabhängige Variable wird dann meist mit $t$
für die Zeit (lateinisch tempus) bezeichnet.}
\lang{en}{
Exponential functions often arise when describing processes in the natural sciences,
in engineering, or in economics through mathematical
models, especially when describing the evolution of a process over time.
In this context, the independent variable is usually
denoted by $t$, standing for time.
}

\lang{de}{Im nächsten Kapitel werden wir das Wachstum eines Kapitals mit festem
Zinssatz mit Hilfe von Exponentialfunktionen beschreiben und die Zinseszinsformel 
   \begin{equation*}K(t) = K_{0}  \cdot  (1 + \frac{p}{100} )^{t}
   \end{equation*}
   erhalten. }
\lang{en}{
In the next chapter, we will model the growth of
a sum of money at a fixed interest rate with
exponential functions and obtain the formula
    \begin{equation*}K(t) = K_{0}  \cdot  (1 + \frac{p}{100} )^{t}
   \end{equation*}
for compound interest.
}

\lang{de}{Um Exponentialgleichungen wie z.\,B. $3^x = 81$ lösen zu können, brauchen wir noch \emph{\ref[content_04_Funktionsbegriff][Umkehrfunktionen]{def:functionprops}} von
Exponentialfunktionen. Dies sind die \emph{Logarithmusfunktionen}. Diese Umkehrfunktionen existieren, weil die 
Exponentialfunktionen streng monoton wachsend oder fallend sind (und damit bijektiv auf den Wertebereich abbilden). 
}
\lang{en}{
To solve exponential equations such as $3^x = 81$, we
will also need the \emph{\ref[content_04_Funktionsbegriff][inverse functions]{def:functionprops}}
of the exponential functions. These are the
\emph{logarithms}. The inverse functions exist
because exponential functions are either strictly
monotonically increasing or decreasing (and therefore
bijective maps onto their images).
}

\begin{definition}[\lang{de}{Logarithmus} \lang{en}{Logarithm}]
  
  	\label{ln} \lang{de}{Zu vorgegebener Basis $a>0$ und $u>0$ ist der 
  	\notion{\emph{Logarithmus zur Basis $a$ von $u$}} die eindeutige Lösung der Gleichung $a^x = u$. Wir schreiben dafür $\log_a(u)$.}
    \lang{en}{
    For a given base $a>0$ and $u>0$, the
    \emph{base $a$ logarithm of $u$} is the unique solution $x$ of the equation $a^x = u$.
    We denote it by $\log_a(u)$.
    }
    \lang{de}{Wenn aus dem Kontext bekannt ist, welche Basis gemeint ist, oder wenn die Basis für die Aussage irrelevant ist, 
    schreibt man auch einfach $\log(u)$ statt 
    $\log_a(u)$. }
    \lang{en}{
    If the base is clear in context or if it is irrelevant
    then we can simply write $\log(u)$ instead of $\log_a(u)$.
    }
    \lang{de}{
    In der Praxis benutzen wir meistens die beiden folgenden Logarithmusfunktionen: \\  
    {Den \emph{Logarithmus zur Basis $10$}} nennen wir \notion{Zehnerlogarithmus} oder \notion{dekadischen Logarithmus}
     und schreiben dafür $\lg$.\\
     {Den \emph{Logarithmus zur Basis $e$}} nennen wir \notion{natürlichen Logarithmus} und schreiben dafür $\ln$.
    }
    \lang{en}{
    In practice, we mostly use one of the following
    two logarithms:\\
    {The \emph{base $10$ logarithm}} is called the \notion{common logarithm} or \notion{decimal logarithm}
    and is written $\lg$.\\
    {The \emph{base $e$ logarithm}} is called the \notion{natural logarithm}
    and is written $\ln$.
    }
\end{definition}
\lang{de}{
Es gilt also 
  	\begin{align*}	
  		\ln(u) = s \quad \Leftrightarrow \quad e^s = u
  	\end{align*}
und Exponentialfunktion und Logarithmus heben sich gegenseitig auf:
\begin{align*}
 e^{\ln(x)}  = x, \quad \ln(e^x) = x.
\end{align*}
Dies gilt auch für Exponential- und Logarithmusfunktionen zu anderen Basen. 
}
\lang{en}{
In other words,
  	\begin{align*}	
  		\ln(u) = s \quad \Leftrightarrow \quad e^s = u,
  	\end{align*}
and the exponential function and logarithm cancel each other out:
\begin{align*}
 e^{\ln(x)}  = x, \quad \ln(e^x) = x.
\end{align*}
This also applies to exponentials and logarithms with other bases.
}

\begin{center}
\image{T601_Logarithms_B}
\end{center}

\lang{de}{
In den folgenden Beispielen wird auf die Gleichungen jeweils eine Logarithmusfunktion angewandt.}
\lang{en}{
In the following examples, a logarithm is applied to each equation:
}
\begin{example}
	\begin{eqnarray*}
        e^{-1} = \frac{1}{e} &\Leftrightarrow & \ln\left(\frac{1}{e} \right) = -1 \\
		10^4 = 10\;\!000 &\Leftrightarrow & \lg(10\;\!000) = 4\\
		10^{-3} = \frac{1}{1\;\!000}  &\Leftrightarrow & \lg\left(\frac{1}{1\;\!000}\right) = -3\\
		e^{-\frac{1}{2}} = \frac{1}{\sqrt{e}} &\Leftrightarrow & \ln\left(\frac{1}{\sqrt{e}}\right)= -\frac{1}{2} 
	\end{eqnarray*}
\end{example}



\begin{quickcheckcontainer}
\randomquickcheckpool{1}{2}

\begin{quickcheck}
		\field{rational}
		\type{input.number}
		\begin{variables}
			\randint{a}{1}{4}
			\randint{k}{1}{2}
			\function[calculate]{n}{k*a+1}
			\function[calculate]{b}{10^a}
			\function[calculate]{c}{a/n}
		\end{variables}

      \lang{de}{
			\text{Bestimmen Sie $\lg(\sqrt[\var{n}]{\var{b}})$.\\ $\lg(\sqrt[\var{n}]{\var{b}})=$\ansref.}
  		}
      \lang{en}{
      \text{Calculate $\lg(\sqrt[\var{n}]{\var{b}})$.\\ $\lg(\sqrt[\var{n}]{\var{b}})=$\ansref.}
  		}
		\begin{answer}
			\solution{c}
		\end{answer}
    \lang{de}{
		\explanation{Es ist $\sqrt[\var{n}]{\var{b}}=10^{\var{c}}$, und daher 
		$\lg(\sqrt[\var{n}]{\var{b}})=\var{c}$.}
    }
    \lang{en}{
    \explanation{We have $\sqrt[\var{n}]{\var{b}}=10^{\var{c}}$,
    and therefore $\lg(\sqrt[\var{n}]{\var{b}})=\var{c}$.}
    }
	\end{quickcheck}
	
\begin{quickcheck}
		\field{rational}
		\type{input.number}
		\begin{variables}
			\randint{n}{2}{4}
			\randint{k}{1}{2}
			\function[calculate]{a}{k*n+1}
			\function[calculate]{b}{10^a}
			\function[calculate]{c}{a/n}
		\end{variables}
		
			\lang{de}{
			\text{Bestimmen Sie $\lg(\sqrt[\var{n}]{\var{b}})$.\\ $\lg(\sqrt[\var{n}]{\var{b}})=$\ansref.}
  		}
      \lang{en}{
      \text{Calculate $\lg(\sqrt[\var{n}]{\var{b}})$.\\ $\lg(\sqrt[\var{n}]{\var{b}})=$\ansref.}
  		}
    
		\begin{answer}
			\solution{c}
		\end{answer}
		\lang{de}{
		\explanation{Es ist $\sqrt[\var{n}]{\var{b}}=10^{\var{c}}$, und daher 
		$\lg(\sqrt[\var{n}]{\var{b}})=\var{c}$.}
    }
    \lang{en}{
    \explanation{We have $\sqrt[\var{n}]{\var{b}}=10^{\var{c}}$,
    and therefore $\lg(\sqrt[\var{n}]{\var{b}})=\var{c}$.}
    }

	\end{quickcheck}
\end{quickcheckcontainer}
	
 

\lang{de}{Zu den bereits bekannten Potenzrechengesetzen erhalten wir passende 
Logarithmusgesetze: }
\lang{en}{
We get logarithm rules from the already known
exponent rules:
}
 
\begin{rule}[\lang{de}{Logarithmusgesetze}
            \lang{en}{Logarithm rules}]
	\lang{de}{
    Für $u,v>0$, $t \in \R$ und $n \in \N$ gelten die folgenden Gleichheiten:}
  \lang{en}{
    The following equations are valid for $u,v>0$, $t \in \R$ and $n \in \N$:
  }
	\begin{align*}

		\text{a)}&\quad\log(u \cdot v)&=& \log(u) + \log(v),\\
		\text{b)}& \quad\log\left(\frac{u}{v}\right)& \ = \ & \log(u) - \log(v), \quad \text{insbesondere gilt}\\
			     & \quad\log\left(\frac{1}{v}\right)& \ = \ & - \log(v),\\
         \text{c)}& \quad\log(u^t) & \ = \ & t\, \log(u), \\
		\text{d)} & \quad\log(\sqrt[n]{u}) & \ = \ & \frac{1}{n} \cdot \log(u).
                 
	\end{align*}	
\end{rule}

\begin{proof*}[\lang{de}{Beweis} \lang{en}{Proof}]
\lang{de}{Die Beweise beruhen auf den bereits behandelten Potenzrechengesetzen. }
\lang{en}{
The proofs come from the rules for exponents discussed earlier.
}
\begin{showhide}
\begin{tabs*}[\initialtab{0}\class{example}]
		\tab{\lang{de}{Beweis a)}\lang{en}{Proof of a)}}
        \lang{de}{Es gelten:}
        \lang{en}{We have:}
		\begin{align*}
			&\log(u) &=& \ s \;\Leftrightarrow\; a^s = u& \\
			\text{\lang{de}{und}\lang{en}{and}}\quad &&&&\\
			&\log(v) &=& \ t \;\Leftrightarrow\; a^t = v, &\\
			\text{\lang{de}{also}\lang{en}{so}}\quad &&&& \\
			&\log(u \cdot v) &=& \log (a^s \cdot a^t) &\\
			&&=& \log (a^{s + t}) = s+t& \\
			& &=& \log(u) + \log(v).  &
		\end{align*}
		
		\tab{\lang{de}{Beweis b)}\lang{en}{Proof of b)}}
        \lang{de}{Es gelten:}
        \lang{en}{We have:}
		\begin{align*}
			&\log(u) &=& \ s \;\Leftrightarrow\; a^s = u& \\
			\text{\lang{de}{und}\lang{en}{and}}\quad &&&& \\
			&\log(v) &=& \ t \;\Leftrightarrow\; a^t = v, &\\
			\text{\lang{de}{also} \lang{en}{so}} \quad &&&& \\
			&\log\left(\frac{u}{v}\right) &=& \log \left(\frac{a^s}{a^t}\right)& \\
			&&=& \log (a^{s - t}) = s-t& \\
			& &=& \log(u) - \log(v).  &
		\end{align*}
        	\tab{\lang{de}{Beweis c)}\lang{en}{Proof of c)}}
            \lang{de}{Es gelten:}
            \lang{en}{We have:}
		\begin{align*}
			\log(u) &=& s \quad\Leftrightarrow \quad u=a^s,\\
			& u^t & = \left(a^s\right)^t = a^{s\cdot t},\\
			\log\left(u^t\right) &=& s\cdot t = t\cdot \log(u).
		\end{align*}
		
	\tab{\lang{de}{Beweis d)}\lang{en}{Proof of d)}}
    \lang{de}{Es gilt }
    \lang{en}{We have}
    \[
			\log (\sqrt[n]{u})  = \log (u^{\frac{1}{n}})
    \]
    \lang{de}{und aus c) folgt direkt }
    \lang{en}{and c) immediately yields}
    \[
    \log ( \sqrt[n]{u}) = \frac{1}{n} \log(u)\,.
    \]
	\end{tabs*}
    \end{showhide}
\end{proof*}


\lang{de}{Mit Hilfe der Logarithmusgesetze können Rechnungen häufig deutlich vereinfacht 
werden, wie die folgenden Beispiele zeigen. }
\lang{en}{
Calculations can often be simpified considerably using
the logarithm rules, as the following examples show.
}

\begin{example}[\lang{de}{Logarithmusgesetze} \lang{en}{Logarithm rules}]
\begin{tabs*}[\initialtab{1}\class{example}]
\tab{\lang{de}{Beispiel 1}\lang{en}{Example 1}}
\begin{eqnarray*}
\ln (2 e^5) 
& =& \ln(2) + \ln(e^5) \\ 
&= & \ln (2) + 5.
\end{eqnarray*}
\tab{\lang{de}{Beispiel 2}\lang{en}{Example 2}}
\lang{de}{Für $-5<x<5$ gilt}
\lang{en}{For $-5<x<5$,}
\begin{eqnarray*}
	\lg (5-x) + \lg(5+x) 
	& =& \lg\left((5-x)\cdot(5+x)\right)\\
	& =& \lg(25-x^2).
\end{eqnarray*}
\tab{\lang{de}{Beispiel 3}\lang{en}{Example 3}}
 \begin{eqnarray*} 
\lg \left( \frac{1}{1000^{3}}\right) &=& \lg (1000^{-3}) \\
&=& -3 \lg (1000) \\
&=& -3 \lg (10^3) \\
&=& -3 \cdot 3 \\
&=& -9.
\end{eqnarray*}
\tab{\lang{de}{Beispiel 4}\lang{en}{Example 4}}
 \begin{eqnarray*} 
 \ln \left( \sqrt{e}^{\;5} \right) &=& \ln \left( e^{\frac{5}{2}} \right) \\
&=& \frac{5}{2}\ln (e)\\
&=&   \frac{5}{2}.
\end{eqnarray*}
 \end{tabs*}
 \end{example}

 
 \begin{quickcheck}
		\field{rational}
		\type{input.number}
		\begin{variables}
			\randint{k}{2}{5}
			\randint{m}{2}{5}
			\randint{n}{2}{5}
			\function[calculate]{k2}{k^2}
			\function[calculate]{c}{2*n-m}
		\end{variables}
  		\lang{de}{
			\text{Vereinfachen Sie den Ausdruck $2\cdot \ln(\var{k}e^\var{n})-\ln(\var{k2}e^\var{m})$.\\ 
			$2\cdot \ln(\var{k}e^\var{n})-\ln(\var{k2}e^\var{m})=$\ansref.}
  		}
      \lang{en}{
      \text{Simplify the expression $2\cdot \ln(\var{k}e^\var{n})-\ln(\var{k2}e^\var{m})$.\\ 
      $2\cdot \ln(\var{k}e^\var{n})-\ln(\var{k2}e^\var{m})=$\ansref.}
      }
		\begin{answer}
			\solution{c}
		\end{answer}
    \lang{de}{
		\explanation{Mit obigen Rechenregeln ist \\
		\begin{align*}
		2\cdot \ln(\var{k}e^\var{n})-\ln(\var{k2}e^\var{m}) 
		&=2\cdot \big( \ln(\var{k})+\ln(e^\var{n})\big)-\ln (\var{k}^2)-\ln(e^\var{m}) \\
		&=2\cdot \ln(\var{k}) + 2\cdot \var{n} - 2\cdot \ln(\var{k}) - \var{m}\\
		&=2\cdot \var{n} -\var{m} = \var{c}.
		\end{align*}
		}
    }
    \lang{en}{
		\explanation{Using the above rules, \\
		\begin{align*}
		2\cdot \ln(\var{k}e^\var{n})-\ln(\var{k2}e^\var{m}) 
		&=2\cdot \big( \ln(\var{k})+\ln(e^\var{n})\big)-\ln (\var{k}^2)-\ln(e^\var{m}) \\
		&=2\cdot \ln(\var{k}) + 2\cdot \var{n} - 2\cdot \ln(\var{k}) - \var{m}\\
		&=2\cdot \var{n} -\var{m} = \var{c}.
		\end{align*}
		}
    }
	\end{quickcheck}
 
\lang{de}{Wie die folgende Regel zeigt, kann man Exponential- und Logarithmusfunktionen auch in andere 
Basen umrechnen. Grundsätzlich braucht man also nicht für jede beliebige Basis die entsprechende 
Logarithmusfunktion. Auf Taschenrechnern sind daher auch häufig nur der dekadische und der 
natürliche Logarithmus umgesetzt. }
\lang{en}{
The following rule shows that exponential and logarithm
functions can be converted to different bases.
Therefore, we do not strictly need a different
logarithm for every possible base.
This is why calculators often only provide the
common and natural logarithms.
}

\begin{rule}[\lang{de}{Umrechnung der Basis} \lang{en}{Change of base}]\label{change-base}
\lang{de}{Exponential- und Logarithmusfunktionen zu einer Basis $a > 0$ können in jede andere Basis umgerechnet werden. 
Für die Umrechnung in die Basis $e$ ergibt sich beispielhaft:}
\lang{en}{
The exponential and logarithm functions to a base $a > 0$ can be converted to
any other base. For the change of base to $e$, for example,
}
\begin{align*}
    a)  \ & \ \  e^x &= \ & a^{x\log_a(e)}, \\
	b) \ & \ \ a^x &= \ & e^{x\ln(a)},\\
	c) \ & \  \ln(x) &= \ & \frac{\log_a(x)}{\log_a(e)},\\
	d) \ & \ \log_a(x) &= \ & \frac{\ln(x)}{\ln(a)}.\\
\end{align*}
\lang{de}{In Fall $c)$ und $d)$ muss $x>0$ gelten, damit die Logarithmen definiert sind.
}
\lang{en}{
In cases $c)$ and $d)$, we need $x > 0$ so that the
logarithms are defined.
}
\end{rule}
 
 \begin{proof*}
 \begin{tabs*}[\initialtab{0}\class{example}]
\tab{\lang{de}{Beweis a) und b)}
\lang{en}{Proof of a) and b)}}
\lang{de}{Nach den Logarithmus- und Potenzrechengesetzen gilt}
\lang{en}{The rules for logarithms and exponents yield}
\begin{align*}
 e^x = a^{\log_a(e^x)} = a^{x\log_a(e)}
\end{align*}
\lang{de}{sowie umgekehrt}
\lang{en}{and conversely}
\begin{align*}
 a^x = e^{\ln(a^x)} = e^{x\ln(a)}.
\end{align*}

\tab{\lang{de}{Beweis c) und d)}
\lang{en}{Proof of c) and d)}}
\lang{de}{Nach den Logarithmus- und Potenzrechengesetzen gilt}
\lang{en}{The rules for logarithms and exponents yield}
\begin{align*}
\log_a(x) = \log_a(e^{\ln(x)}) = \ln(x) \cdot \log_a(e).
\end{align*}
\lang{de}{Wenn wir beide Seiten durch $\log_a(e)$ teilen, erhalten wir}
\lang{en}{After dividing both sides by $\log_a(e)$, we obtain}
\begin{align*}
\ln(x) = \frac{\log_a(x)}{\log_a(e)}.
\end{align*}
\lang{de}{Genau so folgt}
\lang{en}{Similarly,}
\begin{align*}
\ln(x) = \ln(a^{\log_a(x)}) = \log_a(x) \cdot \ln(a).
\end{align*}
\lang{de}{Wenn wir beide Seiten durch $\ln(a)$ teilen, erhalten wir}
\lang{en}{After dividing both sides by $\ln(a)$, we obtain}
\begin{align*}
\log_a(x) = \frac{\ln(x)}{\ln(a)}.
\end{align*}
\end{tabs*}
 \end{proof*}


\lang{de}{Alle Logarithmusfunktionen haben dieselbe Nullstelle $x=1$, denn $a^0=1$ für jede Basis $a>0$.
In der Praxis benutzt man lediglich die Funktionen $\lg$ und $\ln$. Die wichtigsten 
Eigenschaften sind in der folgenden Tabelle festgehalten.}
\lang{en}{
All logarithms have the same zero $x=1$, because $a^0=1$ for every base $a>0$.
In practice, one only uses the function $\lg$ and $\ln$.
Their most important properties are listed in the table below.
}


  \begin{table}
  \head
  $f(x) = \lg (x)$  \lang{de}{oder} \lang{en}{or} $f(x) = \ln (x)$  & $ $ 
  \body
  $D_f = \mathbb{R}^+$ & \lang{de}{Definitionsbereich ist die Menge der positiven Zahlen.}\lang{en}{The domain is the set of all positive numbers.} \\
  $W_f = \mathbb{R}$  &  \lang{de}{Wertemenge ist die Menge der reellen Zahlen.}\lang{en}{The range is the set of all real numbers.}  \\
  $f(1) = 0$& $x=1$ \lang{de}{ist die einzige Nullstelle.}\lang{en}{is the only zero.}\\
  $f(x_1) < f(x_2)\,$ \lang{de}{für}\lang{en}{for} $x_1 < x_2$&  \lang{de}{Die Funktion ist streng monoton wachsend.}\lang{en}{The functions are strictly monotonically increasing.}\\
  \lang{de}{Die Steigung von $f$ nimmt ständig ab.}\lang{en}{The slope is decreasing.} & \lang{de}{Der Graph beschreibt eine Rechtskurve.}\lang{en}{The graph is concave down.} \\
  \lang{de}{für}\lang{en}{for} $ x \rightarrow 0$ \lang{de}{gilt}\lang{en}{,} $\;f(x) \rightarrow -\infty$ & \lang{de}{y-Achse ist Asymptote.}\lang{en}{The y-axis is an asymptote.}   
  \end{table}

 \lang{de}{Die Rechenregeln für Logarithmen und für Potenzen
helfen auch, Exponential- und Logarithmusgleichungen umzuformen und zu lösen. Die Anwendung von 
Exponentialfunktion oder Logarithmus ist eine gültige Äquivalenzumformung. 
}
\lang{en}{
The rules for logarithms and exponents are also useful
for rearranging and solving exponential and logarithm equalities.
Taking exponentials or logarithms is an equivalence.
}


\begin{example}[
\lang{de}{Exponential- und Logarithmusgleichungen}
\lang{en}{Exponential and logarithm equations}
]
 \begin{tabs*}[\initialtab{1}\class{example}] 
 \tab{\lang{de}{Beispiel 1} \lang{en}{Example 1}}
 \begin{eqnarray*}
 e^{2x + 3} = 4 &\Leftrightarrow& \ln ( e^{2x + 3}) = \ln(4) \\
 &\Leftrightarrow& 2x + 3 = \ln(4) \\
  &\Leftrightarrow& x = \frac{1}{2} (\ln(4) -3 ).
 \end{eqnarray*}
  \tab{\lang{de}{Beispiel 2} \lang{en}{Example 2}}
 \begin{eqnarray*}
 \left( 2^{3 - x} \right)^{2-x} = 1  &\Leftrightarrow& 2^{(3-x)(2-x)} = 1 \\
 &\Leftrightarrow& \ln \left(2^{(3-x)(2-x)}\right) = \ln (1) \\
 &\Leftrightarrow& (3-x)(2-x)\ln(2)=0 \\
 &\Leftrightarrow& x = 3\;\;\; \text{\lang{de}{oder}\lang{en}{or}} \;\;\; x = 2.
 \end{eqnarray*}
  	\tab{\lang{de}{Beispiel 3}\lang{en}{Example 3}}
	 	\lang{de}{Für $x>-2$:}\lang{en}{For $x>-2$:} 
	 	\begin{eqnarray*}
	 	\ln \frac{1}{2+x} = 0 &\Leftrightarrow& \text{e}^{ \ln \frac{1}{2+x}} = \text{e}^0 \\
	 								    &\Leftrightarrow& \frac{1}{2+x} = 1 \\
	 									&\Leftrightarrow& 2 + x = 1 \\ 
	 									&\Leftrightarrow& x = -1.\\
	 	\end{eqnarray*}
	 	
 	\tab{\lang{de}{Beispiel 4}\lang{en}{Example 4}}
 		\lang{de}{Für $x>0$:}\lang{en}{For $x>0$:} 
		\begin{eqnarray*}
			\lg (x^3) + 2 \lg (x^2) = 21  
			&\Leftrightarrow& \lg (x^3) +  \lg \left((x^2)^2\right) = 21\\
			&\Leftrightarrow& \lg (x^3\cdot x^4) = 21\\
			&\Leftrightarrow& \lg (x^7) = 21\\
			&\Leftrightarrow& 10^{ \lg (x^7)} = 10^{21}\\ 
			&\Leftrightarrow& x^7 = 10^{21}\\
			&\Leftrightarrow& \sqrt[7]{x^7} = \sqrt[7]{10^{21}}\\
			&\Leftrightarrow& x = 10^{\frac{21}{7}}\\
			&\Leftrightarrow& x = 10^{3}.
		\end{eqnarray*}
 \end{tabs*}
 \end{example}
 


\begin{quickcheck}
		\field{real}
		\displayprecision{2}
 		\correctorprecision{2}
		\type{input.number}
		\begin{variables}
			\randint[Z]{c}{-2}{4}
			\randint[Z]{k}{-3}{3}
		    \function[normalize]{f}{x-k}
			\function[calculate]{ls}{k+10^c}
		\end{variables}

      \lang{de}{
			\text{Bestimmen Sie die Lösung der Gleichung $\lg(\var{f})=\var{c}$.\\ Die Lösung ist \ansref.}
      }
      \lang{en}{
      \text{Find the solution of the equation $\lg(\var{f})=\var{c}$.\\ The solution is \ansref.}
      }
		\begin{answer}
			\solution{ls}
		\end{answer}
    \lang{de}{
		\explanation{Umformen ergibt:
		\begin{align*}
			& \lg(\var{f}) &=\var{c} &\quad &\\
			\Leftrightarrow \ & \var{f} & =10^\var{c}  &\quad & (\text{und }\var{f}>0) \\
			\Leftrightarrow \  & x & =\var{k}+ 10^\var{c}= \var{ls}.&&
		\end{align*} }
    }
    \lang{en}{
    \explanation{After rearranging:
    \begin{align*}
			& \lg(\var{f}) &=\var{c} &\quad &\\
			\Leftrightarrow \ & \var{f} & =10^\var{c}  &\quad & (\text{and }\var{f}>0) \\
			\Leftrightarrow \  & x & =\var{k}+ 10^\var{c}= \var{ls}.&&
		\end{align*} }
    }
	\end{quickcheck}




\section{\lang{de}{Wurzelfunktion} \lang{en}{Radical functions}}
\lang{de}{Mit Hilfe von Wurzelfunktionen lassen sich die Potenzfunktionen $x^n$ rückgängig machen, denn es handelt sich 
um die Umkehrfunktionen von $f(x) = x^n$. Diese sind im Allgemeinen  nicht für negative Werte definiert. 
}
\lang{en}{
Radical functions are useful when inverting the function
$f(x) = x^n.$
In general, radicals are not defined for negative arguments. 
}

\begin{definition}
\lang{de}{Für $n \in \N$ ist die $n$-te Wurzelfunktion gegeben durch }
\lang{en}{For $n \in \N$, the $n$-th root function is defined by}
\[ f: [0,\infty)\to \R, \ f(x) = \sqrt[n]{x} = x^{\frac{1}{n}}. \]
%\floatright{\href{https://www.hm-kompakt.de/video?watch=150}{\image[75]{00_Videobutton_schwarz}}}\\\\
\end{definition}

\lang{de}{Der maximale Definitionsbereich der $n$-ten Wurzelfunktion ist also die Menge der nicht negativen 
reellen Zahlen $[0,\infty)$, die Wertemenge ist ebenso $[0,\infty)$.
}
\lang{en}{
The maximal domain of the $n$-th root function is the set
of nonnegative real numbers, $[0, \infty)$, and the
range is also $[0, \infty)$.
}

\lang{de}{Da die $n$-te Wurzelfunktion die reelle Umkehrfunktion ist, erhält man ihren Funktionsgraphen, indem man
den Graphen der Potenzfunktion $f:[0,\infty)\to \R, x\mapsto x^n$ an der Winkelhalbierenden $y=x$
spiegelt.}
\lang{en}{
Since the $n$-th root is an inverse function,
its graph is obtained by reflecting the graph of the
power function $f:[0,\infty)\to \R,\; x \mapsto x^n$
across the diagonal $y=x$.
}

	\begin{genericGWTVisualization}[550][1000]{mathlet1}
		\begin{variables}		
			\parametricFunction{p1}{real}{t, t, 0.01, 3, 1000}
			\parametricFunction{p2}{real}{t, t^2, 0.01, 3, 1000}
			\parametricFunction{p3}{real}{t, t^3, 0.01, 3, 1000}
			\parametricFunction{p4}{real}{t, t^4, 0.01, 3, 1000}
			\parametricFunction{f2}{real}{t^2, t, 0.01, 3, 1000}
			\parametricFunction{f3}{real}{t^3, t, 0.01, 3, 1000}
			\parametricFunction{f4}{real}{t^4, t, 0.01, 3, 1000}
		\end{variables}
		\color[0.5]{p1}{GRAY}
 		\color[0.1]{p2}{#0066CC}
 		\color[0.1]{p3}{#00CC00}
  		\color[0.1]{p4}{#CC6600}
		\color{f2}{#0066CC}
		\color{f3}{#00CC00}
		\color{f4}{#CC6600}


		\begin{canvas}
			\plotSize{450}
			\plotLeft{-1}
			\plotRight{5}
			\plot[coordinateSystem]{p1,p2,p3,p4,f2,f3,f4}
		\end{canvas}
    \lang{de}{
		\text{Hier sehen Sie die Graphen der
		Wurzelfunktionen $\textcolor{#0066CC}{f_2(x)=\sqrt{x}}$,
		$\textcolor{#00CC00}{f_3(x)=\sqrt[3]{x}}$ und $\textcolor{#CC6600}{f_4(x)=\sqrt[4]{x}}$, sowie die 
		zugehörenden Potenzfunktionen $\textcolor{#0066CC}{x^2}$, $\textcolor{#00CC00}{x^3}$ und 
		$\textcolor{#CC6600}{x^4}$ im Bereich $x\geq 0$.
		}
    }
    \lang{en}{
    These are the graphs of the radical functions
    $\textcolor{#0066CC}{f_2(x)=\sqrt{x}}$,
    $\textcolor{#00CC00}{f_3(x)=\sqrt[3]{x}}$ and
    $\textcolor{#CC6600}{f_4(x)=\sqrt[4]{x}}$,
    as well as the associated power functions
    $\textcolor{#0066CC}{x^2}$,
    $\textcolor{#00CC00}{x^3}$ and
    $\textcolor{#CC6600}{x^4}$ on the domain $x\geq 0$.
    }
	    	\end{genericGWTVisualization}

\begin{example}
\lang{de}{Wir berechnen den Definitionsbereich einer Wurzelfunktion: Die Funktion $f$ mit $f(x) = \sqrt{x^2 - 4}$ ist nur definiert, wenn $x^2 - 4 \geq 0$ gilt. Dies ist 
der Fall für $x\geq 4$ oder $x \leq -4$. Insgesamt folgt $D_f = (-\infty; -4] \cup [4; \infty)$. }
\lang{en}{
Let us find the domain of a radical function.
The function $f$ with $f(x)=\sqrt{x^2 - 4}$ is only defined
when $x^2 - 4 \ge 0$. This is the case if $x \ge 4$
or $x \le -4$. So $D_f = (-\infty, -4] \cup [4, \infty)$.
}
\end{example}

\begin{quickcheck}
		\field{rational}
		\type{input.number}
		\begin{variables}
			\randint[Z]{b}{1}{5}
		    \function[calculate]{a2}{b^2}
		    \function[calculate]{a}{-b}
		\end{variables}
      \lang{de}{
			\text{Der maximale Definitionsbereich der Funktion $f(x)=\sqrt{\var{a2}-x^2}$ ist\\
			$D_f=\{ x\in \R \,|\, $\ansref$\leq x\leq $\ansref $\}$.}
      }
      \lang{en}{
      \text{The maximal domain of the function $f(x)=\sqrt{\var{a2}-x^2}$ is\\
      $D_f=\{ x\in \R \,|\, $\ansref$\leq x\leq $\ansref $\}$.}
      }
		\begin{answer}
			\solution{a}
		\end{answer}
		\begin{answer}
			\solution{b}
		\end{answer}
    \lang{de}{
		\explanation{Der maximale Definitionsbereich besteht aus den Werten von $x$, für die der Ausdruck
		definiert ist, für die also $\var{a2}-x^2\geq 0$ gilt.}
    }
    \lang{en}{
    \explanation{The maximal domain consists of all values of $x$ for which
    the expression is defined, i.e. for which $\var{a2}-x^2\geq 0$.}
    }
	\end{quickcheck}
\begin{quickcheck}
		\field{rational}
		\type{input.number}
		\begin{variables}
			\randint[Z]{a}{1}{5}
			\randint[Z]{b}{1}{5}
		    \function[calculate]{na}{-a}
		\end{variables}
      \lang{de}{
			\text{Der maximale Definitionsbereich der Funktion $f(x)=\sqrt{x+\var{a}}-\sqrt{\var{b}-x}$ ist\\
			$D_f=\{ x\in \R \,|\, $\ansref$\leq x\leq $\ansref $\}$.}
      }
      \lang{en}{
      \text{The maximal domain of the function $f(x)=\sqrt{x+\var{a}}-\sqrt{\var{b}-x}$ is\\
      $D_f=\{ x\in \R \,|\, $\ansref$\leq x\leq $\ansref $\}$.}
      }
		\begin{answer}
			\solution{na}
		\end{answer}
		\begin{answer}
			\solution{b}
		\end{answer}
    \lang{de}{
		\explanation{Der maximale Definitionsbereich besteht aus den Werten von $x$, für die der gesamte Ausdruck
		definiert ist, für die also $x+\var{a}\geq 0$ ist und $\var{b}-x\geq 0$ ist.}
    }
    \lang{en}{
    \explanation{The maximal domain consists of all values of $x$
    at which the entire expression is defined, i.e.
    for which $x+\var{a}\geq 0$ and $\var{b}-x\geq 0$.}
    }
	\end{quickcheck}


\section{
\lang{de}{Trigonometrische Funktionen}
\lang{en}{Trigonometric functions}
}

\lang{de}{Periodische Vorgänge, die sich stets wiederholen, können häufig gut durch trigonometrische Funktionen wie Sinus und Kosinus 
modelliert werden. Wir wiederholen kurz die Definition. 
}
\lang{en}{
Periodic processes that continuously repeat can often
be modeled well by trigonometric functions such as the
sine and cosine. We will quickly review their definitions.
}

\begin{definition}[Sinus, Kosinus und Tangens] \label{def-trigFkt}
\lang{de}{
\textbf{Sinus}, \textbf{Kosinus} und \textbf{Tangens} ordnen einem Winkel im rechtwinkligen
Dreieck die L"angenverh"altnisse der Katheten und Hypotenuse zu.
F"ur die Definition betrachtet man zun"achst ein rechtwinkliges
Dreieck $ABC$ wie in der Abbildung.} 
\lang{en}{
The \textbf{sine}, \textbf{cosine} and \textbf{tangent}
of an angle in a right triangle describe the ratios
between the lengths of the legs and the hypotenuse.
To define them, consider a right triangle $ABC$
as in the figure below.
}

\label{sin-cos-im-dreieck}
\begin{figure}
  \image{T601_Triangle_A}
\caption{\lang{de}{Rechtwinkliges Dreieck $ABC$}\lang{en}{Right Triangle $ABC$}}
\end{figure}

\lang{de}{Man definiert dann \textbf{Sinus}, \textbf{Kosinus} und \textbf{Tangens} von
$\alpha$ durch
  \begin{eqnarray*}
    \sin(\alpha) &:=& \frac{\text{L"ange der Gegenkathete von }\alpha}{\text{L"ange der Hypotenuse}}
                 =  \frac{a}{c}, \\
    \cos(\alpha) &:=& \frac{\text{L"ange der Ankathete von }\alpha}{\text{L"ange der Hypotenuse}}
                 =  \frac{b}{c}, \\
    \tan(\alpha) &:=& \frac{\text{L"ange der Gegenkathete von }\alpha}{\text{L"ange der Ankathete von }\alpha}
                 =  \frac{a}{b} = \frac{\sin(\alpha)}{\cos(\alpha)}.
  \end{eqnarray*}
}

\lang{en}{The \textbf{sine}, \textbf{cosine} and \textbf{tangent} of
$\alpha$ are then defined as
  \begin{eqnarray*}
    \sin(\alpha) &:=& \frac{\text{length of opposite side}}{\text{length of hypotenuse}}
                 =  \frac{a}{c}, \\
    \cos(\alpha) &:=& \frac{\text{length of adjacent side}}{\text{length of hypotenuse}}
                 =  \frac{b}{c}, \\
    \tan(\alpha) &:=& \frac{\text{length of opposite side}}{\text{length of adjacent side}}
                 =  \frac{a}{b} = \frac{\sin(\alpha)}{\cos(\alpha)}.
  \end{eqnarray*}
}
\end{definition}


\lang{de}{Die Werte
$\sin(\alpha)$, $\cos(\alpha)$ und $\tan(\alpha)$ hängen nicht von der Gr"o"se des Dreiecks
ab, sondern nur von der Winkelgr"o"se von $\alpha$.}
\lang{en}{
The values of $\sin(\alpha)$, $\cos(\alpha)$ and $\tan(\alpha)$
depend only on the size of the angle $\alpha$,
not on the size of the triangle.
}

\lang{de}{Wenn $\alpha$ kein Winkel in Grad ist, sondern eine reelle Zahl (\glqq in Bogenmaß\grqq),
müssen wir zwischen Grad- und Bogenmaß umrechnen:}
\lang{en}{
If $\alpha$ is not given as an angle measured in degrees but instead
as a real number ("in radians")
then we have to convert between degrees and radians:
}

\lang{de}{
  \[ \text{Wert in Bogenma"s} = \frac{\text{Wert in Gradma"s}}{360^\circ}\cdot 2\pi \] 
und umkehrt
    \[ \text{Wert in Gradma"s} = \frac{\text{Wert in Bogenma"s}}{2\pi}\cdot
  360^\circ .
  \]
}
\lang{en}{
\[ \text{Size in radians} = \frac{\text{Size in degrees}}{360^\circ}\cdot 2\pi \] 
and conversely
\[ \text{Size in degrees} = \frac{\text{Size in radians}}{2\pi}\cdot
  360^\circ .
  \]
}

\lang{de}{Auf diese Weise ist $\sin(x)$ oder $\cos(x)$ für reelle Zahlen $x$ definiert. Dadurch können wir 
Sinus und Kosinus auch als reelle Funktionen auffassen.}
\lang{en}{
$\sin(x)$ and $\cos(x)$ are defined
in this way for arbitrary real numbers $x$. We can therefore view sine and
cosine as real functions.
}

\\
\begin{center}
\image{T601_SineCosine_B}
\end{center}
\\
\lang{de}{Sinus und Kosinus haben die folgenden Eigenschaften:}
\lang{en}{
The sine and cosine functions have the following properties:
}
\begin{rule}
\begin{itemize}
\item \lang{de}{Der Definitionsbereich von Sinus und Kosinus ist $D_{\sin}=D_{\cos}=\R$ und die 
Wertemenge ist das Intervall $W_{\sin}=W_{\cos}=[-1 ; 1]$.
}
\lang{en}{
The domain of sine and cosine is $D_{\sin}=D_{\cos}=\R$
and the range is the interval $W_{\sin}=W_{\cos}=[-1 ; 1]$.
}

\item \lang{de}{Sinus und Kosinus sind periodisch mit Periodenlänge $2\pi$, d.\,h.}
\lang{en}{Sine and cosine are periodic with a period of length $2\pi$, i.e.}
\begin{eqnarray*}
\sin(x + 2 \pi) &= &\sin(x)\, ,\\
\cos(x + 2 \pi) &= &\cos(x).
\end{eqnarray*}

\item \lang{de}{Sinus und Kosinus unterscheiden sich nur um eine Verschiebung um $\frac{\pi}{2}$, d.\,h. für alle $x \in \R$ gilt}
\lang{en}{
Sine and cosine differ only by an offset of $\frac{\pi}{2}$; that is, for all $x \in \R$,
}
\[
\sin(x+\frac{\pi}{2})=\cos(x).
\]


\item \lang{de}{Für $k \in \mathbb{Z}$ gilt}
\lang{en}{For $k \in \mathbb{Z}$,}
\[
\sin(k \pi) =  0, \quad \cos\left(\frac{\pi}{2} + k\pi \right) = 0
\]
\lang{de}{und alle Nullstellen von Sinus und Kosinus sind von dieser Form. }
\lang{en}{
and every zero of sine or cosine is of this form.
}


\item \lang{de}{Für alle $x \in \R$ gilt}
\lang{en}{For every $x \in \R$,}
\[
\sin(-x) =  -\sin(x), \quad \cos(-x) =  \cos(x).
\]

\end{itemize}
\end{rule}


\begin{quickcheckcontainer}
\randomquickcheckpool{1}{2}
\begin{quickcheck}
		\field{real}
		\type{input.number}
        
		\begin{variables}
			\randint[Z]{k1}{-2}{2}			
			\function[calculate]{k2}{k1+3}
			\function[calculate]{lb}{floor(k1*pi)}
			\function[calculate]{ub}{floor(k2*pi)+1}
			\function[normalize]{x1}{k1*pi}
			\function[normalize]{x2}{(k1+1)*pi}
			\function[normalize]{x3}{(k1+2)*pi}
			\function[normalize]{x4}{k2*pi}
		\end{variables}

      \lang{de}{
			\text{Welche Nullstellen hat die Funktion $\sin(x)$ im Intervall $[\var{lb};\var{ub}]$?
				Die Nullstellen sind (in aufsteigender Reihenfolge):\\ 
                $x_1=$\ansref,\\
                $x_2=$\ansref,\\
                $x_3=$\ansref,\\
                $x_4=$\ansref.\\
				(Geben Sie die Nullstellen als Vielfache von $\pi$ an durch Eingabe von z.\,B. $2*pi$).}
      }
      \lang{en}{
      \text{What are the zeros of the function $\sin(x)$ in the interval $[\var{lb};\var{ub}]$?
        The zeros (in increasing order) are: \\
                $x_1=$\ansref,\\
                $x_2=$\ansref,\\
                $x_3=$\ansref,\\
                $x_4=$\ansref.\\
        (Give the zeros as multiples of $\pi$ by entering $2*pi$ (for example))}
      }
        \begin{answer}
			\solution{x1}
		\end{answer}
		\begin{answer}
			\solution{x2}
		\end{answer}
		\begin{answer}
			\solution{x3}
		\end{answer}
		\begin{answer}
			\solution{x4}
		\end{answer}
	\end{quickcheck}
\begin{quickcheck}
		\field{real}
		\type{input.number}
		\begin{variables}
			\randint[Z]{k1}{-2}{2}			
			\function[calculate]{k2}{k1+4}
			\function[calculate]{lb}{floor(k1*pi)}
			\function[calculate]{ub}{floor(k2*pi)+1}
			\function[normalize]{x1}{(k1+1/2)*pi}
			\function[normalize]{x2}{(k1+3/2)*pi}
			\function[normalize]{x3}{(k1+5/2)*pi}
			\function[normalize]{x4}{(k1+7/2)*pi}
		\end{variables}

      \lang{de}{
			\text{Welche Nullstellen hat die Funktion $\cos(x)$ im Intervall $[\var{lb};\var{ub}]$?
				Die Nullstellen sind (in aufsteigender Reihenfolge):\\
                $x_1=$\ansref,\\
                $x_2=$\ansref,\\
                $x_3=$\ansref,\\
                $x_4=$\ansref.\\
				(Geben Sie die Nullstellen als Vielfache von $\pi$ an durch Eingabe von z.\,B. $2*pi$).}
      }
      \lang{en}{
      \text{What are the zeros of the function $\cos(x)$ in the interval $[\var{lb};\var{ub}]$?
        The zeros (in increasing order) are: \\
                $x_1=$\ansref,\\
                $x_2=$\ansref,\\
                $x_3=$\ansref,\\
                $x_4=$\ansref.\\
            (Enter the zeros as multiples of $\pi$ by entering $2*pi$ (for example))}
      }
		\begin{answer}
			\solution{x1}
		\end{answer}
		\begin{answer}
			\solution{x2}
		\end{answer}
		\begin{answer}
			\solution{x3}
		\end{answer}
		\begin{answer}
			\solution{x4}
		\end{answer}
	\end{quickcheck}
\end{quickcheckcontainer}


\end{visualizationwrapper}

\end{content}

