%$Id:  $
\documentclass{mumie.article}
%$Id$
\begin{metainfo}
  \name{
    \lang{de}{Trigonometrische Funktionen}
    \lang{en}{}
  }
  \begin{description} 
 This work is licensed under the Creative Commons License Attribution 4.0 International (CC-BY 4.0)   
 https://creativecommons.org/licenses/by/4.0/legalcode 

    \lang{de}{Beschreibung}
    \lang{en}{}
  \end{description}
  \begin{components}
    \component{generic_image}{content/rwth/HM1/images/g_tkz_T105_Tangent.meta.xml}{T105_Tangent}
    \component{generic_image}{content/rwth/HM1/images/g_tkz_T211_SineCosine.meta.xml}{T211_SineCosine}
    \component{generic_image}{content/rwth/HM1/images/g_img_00_Videobutton_schwarz.meta.xml}{00_Videobutton_schwarz}
    \component{js_lib}{system/media/mathlets/GWTGenericVisualization.meta.xml}{mathlet1}
    \component{generic_image}{content/rwth/HM1/images/g_img_hm-def-sin-cos.meta.xml}{image1}
    \component{generic_image}{content/rwth/HM1/images/g_img_hm-def-tancot.meta.xml}{tan_cot}
    %\component{js_lib}{content/rwth/HM1/mathlets/VisualizationSineCosineCircleMathlet.meta.xml}{SinCos}
  \end{components}
  \begin{links}
    \link{generic_article}{content/rwth/HM1/T211_Eigenschaften_stetiger_Funktionen/g_art_content_36_anwendungen.meta.xml}{content_36_anwendungen}
    \link{generic_article}{content/rwth/HM1/T211_Eigenschaften_stetiger_Funktionen/g_art_content_33_zwischenwertsatz.meta.xml}{zwischenwertsatz}
    \link{generic_article}{content/rwth/HM1/T210_Stetigkeit/g_art_content_30_elem_funktionen.meta.xml}{elem-funk}
    \link{generic_article}{content/rwth/HM1/T105_Trigonometrische_Funktionen/g_art_content_18_grad_und_bogenmass.meta.xml}{grad-bogenmass}
    \link{generic_article}{content/rwth/HM1/T105_Trigonometrische_Funktionen/g_art_content_19_allgemeiner_sinus_cosinus.meta.xml}{allg-sin-cos}
    \link{generic_article}{content/rwth/HM1/T209_Potenzreihen/g_art_content_28_exponentialreihe.meta.xml}{exp}
  \end{links}
  \creategeneric
\end{metainfo}
\begin{content}
\usepackage{mumie.ombplus}
\ombchapter{11}
\ombarticle{3}
\usepackage{mumie.genericvisualization}

\begin{visualizationwrapper}

\lang{de}{\title{Trigonometrische Funktionen}}
 

\begin{block}[annotation]
  Im Ticket-System: \href{http://team.mumie.net/issues/9795}{Ticket 9795}\\
\end{block}

\begin{block}[info-box]
\tableofcontents
\end{block}

\section{Sinus und Kosinus}

Im Abschnitt \link{allg-sin-cos}{Sinus und Kosinus am Einheitskreis} haben wir schon den Sinus und
 Kosinus als Funktionen von Winkelgrößen kennengelernt. Man fasst diese Funktionen als reelle
 Funktionen auf, indem man den Winkel im Bogenmaß verwendet, welcher gleich der Länge des
 entsprechenden Bogens auf dem Einheitskreis ist (vgl. Abschnitt \link{grad-bogenmass}{Gradmaß und Bogenmaß}).
 
 
\begin{definition}\label{sincos.definition.1}
    \lang{de}{Es sei $x \in \R$ und $(u;v)$ der Punkt auf dem Einheitskreis
    ($u^2 + v^2 = 1$), der mit der positiven $u$-Achse den Winkel $x$
    einschließt (entgegen dem Uhrzeigersinn gerechnet).
    
    Die \notion{Kosinusfunktion} ist die Funktion, die dem Winkel $x$ die Projektion des Punktes $(u;v)$ auf die 
    $u$-Achse zuordnet, also $\cos(x) = u$.

    Ordnet man dem Winkel $x$ die Projektion des Punktes $(u;v)$ auf die 
    $v$-Achse zu, erhält man die \notion{Sinusfunktion}, es gilt also $ \sin(x) = v$.}
    
\begin{center}
\image{T211_SineCosine}
\end{center}

\end{definition} 
\\
\lang{de}{Mit der nachfolgenden Visualisierung können Sie die Graphen von Sinus und Kosinus
durch die Bewegung des Punktes auf dem Einheitskreis nachvollziehen.}
\lang{en}{With the following applet you will be able to understand the graphs of sine and cosine via the movement of the point
on the unit circle.}
\\
% Beginn des jsxgraph-Teils
\begin{jsxgraph}

% werden die Tabellen mit den jxgboxen.
% im script wird der Inhalt geändert.
\jxgbox[300][0]{out-0}
\jxgbox[600][510]{out-1}

\button[ansichtButton]{Ansicht 2}{changeView(aktAnsicht)}
\button[speedButton]{1x}{changeSpeed(speed)}
\button[startPauseButton]{>}{animating(animationRunning)}
\button[resetButton]{Zurücksetzen}{resetView()}

% java-script-code für die Visualisierung
\begin{code}
// create new css-class:
const style = document.createElement('style');
style.type = 'text/css';
style.innerHTML = '.hideDiv { display: none; }';
document.getElementsByTagName('head')[0].appendChild(style);

// make a table of jxgboxes out of the jxgbox "out-0"
  const firstView =  document.getElementById("out-0");
  firstView.classList.remove("jxgbox");
  firstView.removeAttribute("style");
  firstView.innerHTML = '<table><tr><td>' +
  '<div id="jsx-board-master-0" class="MuiBox-root jss2 jss1 jxgbox" style="width:500px; height:140px"></div>'+
  '</td></tr><tr><td>'+
  '<div id="jsx-board-sine-0" class="MuiBox-root jss2 jss1 jxgbox" style="width:500px; height:140px"></div>'+
  '</td></tr><tr><td>'+
  '<div id="jsx-board-cosine-0" class="MuiBox-root jss2 jss1 jxgbox" style="width:500px; height:140px"></div>'+
  '</td></tr></table>';
// make a table of jxgboxes out of the jxgbox "out-1" 
  const secondView =  document.getElementById("out-1");
  secondView.classList.remove("jxgbox");
  secondView.removeAttribute("style");	
  secondView.innerHTML = '<table><tr><td>'+
  '<div id="jsx-board-master-1" class="MuiBox-root jss2 jss1 jxgbox" style="width:116px; height:116px"></div>'+
  '</td><td>'+
  '<div id="jsx-board-sine-1" class="MuiBox-root jss2 jss1 jxgbox" style="width:450px; height:116px"></div>'+
  '</td></tr><tr><td>'+
  '<div id="jsx-board-cosine-1" class="MuiBox-root jss2 jss1 jxgbox" style="width:116px; height:386px"></div>'+
  '</td></tr></table>'; 
  
// ----------------------------------
// global variables
// ----------------------------------
const maxSpeed = 8;
var aktAnsicht = 0; // 0 (wenn aktuell Ansicht 1), 1 (wenn aktuell Ansicht 2)
var speed = 1;
var animationRunning = false;
var tValue = Math.PI / 6;
var intervalID = null;
// ----------------------------------
// functions for button-events
// ----------------------------------
function changeView(n) {
  var jxgboxes = [
    [
      document.getElementById('jsx-board-master-0'),
      document.getElementById('jsx-board-sine-0'),
      document.getElementById('jsx-board-cosine-0'),
    ],
    [
      document.getElementById('jsx-board-master-1'),
      document.getElementById('jsx-board-sine-1'),
      document.getElementById('jsx-board-cosine-1'),
    ],
  ];
  for (var i = 0; i < 3; i++) {
    jxgboxes[n][i].classList.add('hideDiv');
    jxgboxes[1-n][i].classList.remove('hideDiv');
  }
  aktAnsicht = 1-n;
  document.getElementById('ansichtButton').value = 'Ansicht ' + (2 - aktAnsicht);
  document.getElementById('ansichtButton').innerHTML = 
        document.getElementById('ansichtButton').value;
}

function changeSpeed(n) {
  if (n < maxSpeed) {
    speed = 2 * speed;
  } else {
    speed = 1;
  }
  document.getElementById('speedButton').setAttribute('value', speed + 'x');
  document.getElementById('speedButton').innerHTML = 
        document.getElementById('speedButton').value;
  animating(!animationRunning);
}

function animating(isRunning) {
  animationRunning = !isRunning;
  if (intervalID != null) {
    // stop current animation.
    clearInterval(intervalID);
  }
  if (isRunning) {
    document.getElementById('startPauseButton').setAttribute('value', '>');
    intervalID = null;
  } else {
    document.getElementById('startPauseButton').setAttribute('value', '||');
    // start animation.
    intervalID = setInterval(() => {
      increaseTValue(speed);
    }, 160);
    console.log('interval started with ID:', intervalID, 'and speed', speed);
  }
  document.getElementById('startPauseButton').innerHTML = 
        document.getElementById('startPauseButton').value;
}

function increaseTValue(stepwidth) {
  tValue = tValue + stepwidth / 200;
  if (tValue > 2 * Math.PI) {
    tValue = tValue - 2 * Math.PI;
    console.log('tValue reduced to ', tValue);
  }
  masterBoard[0].elementsByName['p'].moveTo([Math.cos(tValue), Math.sin(tValue)]);
  masterBoard[1].elementsByName['p'].moveTo([Math.cos(tValue), Math.sin(tValue)]);
}

function resetView() {
  changeView(1);
  animating(true);
  changeSpeed(maxSpeed);
  tValue = Math.PI / 6;
  masterBoard[0].elementsByName['p'].moveTo([Math.cos(tValue), Math.sin(tValue)]);
  masterBoard[1].elementsByName['p'].moveTo([Math.cos(tValue), Math.sin(tValue)]);
}

// ----------------------------------
// function which is called when 'point' is dragged:
//  * tValue is adjusted,
//  * corresp. point in second board is synchronised.
// ----------------------------------
function setTValue(point) {
  tValue = Math.atan2(point.Y(), point.X());
  if (tValue < 0) {
    tValue += 2 * Math.PI;
  }
  var p2;
  if (point.board == masterBoard[0]) {
    p2 = masterBoard[1].elementsByName[point.name];
  }
  if (point.board == masterBoard[1]) {
    p2 = masterBoard[0].elementsByName[point.name];
  }
  p2.moveTo([point.X(), point.Y()]);
}

// ----------------------------------
// attributes for boards
// ----------------------------------
const attributesMaster = [
  {
    boundingBox: [-5, 1.4, 5, -1.4],
    axis: true,
    keepaspectratio: true,
    showZoom: false,
    showNavigation: false,
    showCopyright: false,
  },
  {
    boundingBox: [-1.4, 1.4, 1.4, -1.4],
    axis: true,
    keepaspectratio: true,
    showZoom: false,
    showNavigation: false,
    showCopyright: false,
  },
];

const attributesSine = [
  {
    boundingBox: [-1.5, 1.4, 8.5, -1.4],
    axis: true,
    keepaspectratio: true,
    showZoom: false,
    showNavigation: false,
    showCopyright: false,
    defaultAxes: {
      x: { ticks: {
          scale: Math.PI,
          scaleSymbol: '\u03c0',
          ticksDistance: 1,
          insertTicks: false
      }}}
  },
  {
    boundingBox: [-1.5, 1.4, 8.5, -1.4],
    axis: true,
    keepaspectratio: true,
    showZoom: false,
    showNavigation: false,
    showCopyright: false,
    defaultAxes: {
      x: { ticks: {
          scale: Math.PI,
          scaleSymbol: '\u03c0',
          ticksDistance: 1,
          insertTicks: false
      }}}
  },
];
const attributesCosine = [
  {
    boundingBox: [-1.5, 1.4, 8.5, -1.4],
    axis: true,
    keepaspectratio: true,
    showZoom: false,
    showNavigation: false,
    showCopyright: false,
    defaultAxes: {
      x: { ticks: {
          scale: Math.PI,
          scaleSymbol: '\u03c0',
          ticksDistance: 1,
          insertTicks: false
      }}}
  },
  {
    boundingBox: [-1.4, 1.5, 1.4, -6],
    axis: false,
    grid: false,
    keepaspectratio: true,
    showZoom: false,
    showNavigation: false,
    showCopyright: false,
  },
];

// ----------------------------------
// boards: master, sine and cosine for each view
// ----------------------------------
const masterBoard = [
  JXG.JSXGraph.initBoard('jsx-board-master-0', attributesMaster[0]),
  JXG.JSXGraph.initBoard('jsx-board-master-1', attributesMaster[1]),
];
const sineBoard = [
  JXG.JSXGraph.initBoard('jsx-board-sine-0', attributesSine[0]),
  JXG.JSXGraph.initBoard('jsx-board-sine-1', attributesSine[1]),
];
const cosineBoard = [
  JXG.JSXGraph.initBoard('jsx-board-cosine-0', attributesCosine[0]),
  JXG.JSXGraph.initBoard('jsx-board-cosine-1', attributesCosine[1]),
];

// enable MathJax in Text-Elements
JXG.Options.text.useMathJax = true;

// ----------------------------------
// initializations of the boards;
// has to be put in subroutines for correctly distinguished variables.
// ----------------------------------
for (i = 0; i < 2; i++) {
  initializeMasterBoard(masterBoard[i], i);
}
for (var i = 0; i < 2; i++) {
  initializeSineBoard(sineBoard[i], masterBoard[i], 'p');
}
initializeFirstCosineBoard(cosineBoard[0], masterBoard[0], 'p');
initializeSecondCosineBoard(cosineBoard[1], masterBoard[1], 'p');

// apply changeView(1) to hide the second set of boards.
changeView(1);

// ----------------------------------
// uncomment the following to see color deficiencies:
// * 'defic' is a string equal to either 'protanopia', 'deuteranopia' or 'tritanopia'.
// ----------------------------------
/* 
  const defic='tritanopia';
  for (var i=0;i<2;i++) { 
  masterBoard[i].emulateColorblindness(defic);
  sineBoard[i].emulateColorblindness(defic);
  cosineBoard[i].emulateColorblindness(defic);
} 
*/

// ----------------------------------
// Master configuration:
// ----------------------------------
function initializeMasterBoard(brd, n) {
  const origin = brd.create('point', [0, 0], { visible: false });
  const eins = brd.create('point', [1, 0], { visible: false });
  const kreis = brd.create('circle', [origin, 1], {
    strokeWidth: 2,
    strokeColor: 'black',
    fixed: true,
    highlight: false,
  });
  var p = brd.create('glider', [Math.cos(tValue), Math.sin(tValue), kreis], {
    name: 'p',
    color: 'black',
    size: 4,
    withLabel: false,
  });
  p.on('drag', function() {
    setTValue(p);
  });
  var bogen = brd.create('arc', [origin, eins, p], {
    strokeWidth: 4,
    strokeColor: '#0066CC',
    highlight: false,
  });
  var xText = brd.create(
    'text',
    [
      () => {
        const r = 1.21;
        return r * Math.cos(tValue / 2);
      },
      () => {
        const r = 1.21;
        return r * Math.sin(tValue / 2);
      },
      'x',
    ],
    {
      color: '#0066CC',
      fixed: true,
      highlight: false,
      name: 'xText',
    }
  );
  xText.addParents([p]); // p has to be set as a parent so that xText will be updated, when p is.
  var px = brd.create(
    'point',
    [
      () => {
        return p.X();
      },
      0,
    ],
    { visible: false }
  );
  var py = brd.create(
    'point',
    [
      0,
      () => {
        return p.Y();
      },
    ],
    { visible: false }
  );
  var l1 = brd.create('segment', [py, p], {
    color: '#00CC00',
    strokeWidth: 4,
    highlight: false,
  });
  var l2 = brd.create('segment', [p, px], {
    color: '#CC6600',
    strokeWidth: 4,
    highlight: false,
  });
  if (i == 0) {
    l1.setAttribute({ withLabel: true, labelColor: '#00CC00' });
    l1.setAttribute({ label: { position: 'bot' } });
    l1.setLabelText('cos(x)');
    l2.setAttribute({ withLabel: true, labelColor: '#CC6600' });
    l2.setLabelText('sin(x)');
    l2.setAttribute({ label: { position: 'top' } });
  }
}
// ----------------------------------
// sineBoards
// ----------------------------------
function initializeSineBoard(brd, parent, point) {
  parent.addChild(brd); // so that brd is updated, when parent is.
  var pcopy = parent.elementsByName[point];
  var origin = brd.create('point', [0, 0], { visible: false });
  var bb = brd.getBoundingBox();
  var f = brd.create('functiongraph', [Math.sin, bb[0], bb[2]], {
    strokeWidth: 2,
    strokeColor: 'black',
    fixed: true,
    highlight: false,
  });
  var qx = brd.create(
    'point',
    [
      function() {
        return tValue;
      },
      0,
    ],
    { visible: false }
  );
  var q = brd.create(
    'point',
    [
      function() {
        return tValue;
      },
      function() {
        return Math.sin(tValue);
      },
    ],
    { visible: false }
  );
  qx.addParents([pcopy]);
  q.addParents([pcopy]); // for correct update
  var l1 = brd.create('segment', [origin, qx], {
    color: '#0066CC',
    strokeWidth: 4,
    highlight: false,
    withLabel: false,
  });
  brd.create(
    'text',
    [
      function() {
        return qx.X() / 2;
      },
      -0.1,
      'x',
    ],
    { color: '#0066CC' }
  );
  var l2 = brd.create('segment', [q, qx], {
    color: '#CC6600',
    strokeWidth: 4,
    highlight: false,
    withLabel: false,
  });
  var tt = brd.create('text', [bb[0] + 0.1, bb[1] - 0.2, 'sin(x)'], {
    color: '#CC6600',
    fixed: true,
    highlight: false,
  });
}

// ----------------------------------
// first cosineBoard:
// ----------------------------------
function initializeFirstCosineBoard(brd, parent, point) {
  parent.addChild(brd);
  console.log('parent', parent);
  console.log('Elemente', parent.elementsByName);
  var pcopy = parent.elementsByName[point];
  console.log('pcopy', pcopy);
  var origin = brd.create('point', [0, 0], { visible: false });
  var bb = brd.getBoundingBox();
  console.log(bb);
  var f = brd.create('functiongraph', [Math.cos, bb[0], bb[2]], {
    strokeWidth: 2,
    strokeColor: 'black',
    fixed: true,
    highlight: false,
  });
  var qx = brd.create(
    'point',
    [
      function() {
        return tValue;
      },
      0,
    ],
    { visible: false }
  );
  var q = brd.create(
    'point',
    [
      function() {
        return tValue;
      },
      function() {
        return Math.cos(tValue);
      },
    ],
    { visible: false }
  );
  qx.addParents([pcopy]);
  q.addParents([pcopy]); // for correct update
  brd.create(
    'text',
    [
      () => {
        return qx.X() / 2;
      },
      0.2,
      'x',
    ],
    { color: '#0066CC' }
  );
  var l1 = brd.create('segment', [origin, qx], {
    color: '#0066CC',
    strokeWidth: 4,
    highlight: false,
    withLabel: false,
  });
  var l2 = brd.create('segment', [q, qx], {
    color: '#00CC00',
    strokeWidth: 4,
    highlight: false,
    withLabel: false,
  });
  var tt = brd.create('text', [bb[0] + 0.1, bb[1] - 0.2, 'cos(x)'], {
    color: '#00CC00',
    fixed: true,
    highlight: false,
  });
}

// ----------------------------------
// second cosineBoard
// ----------------------------------
function initializeSecondCosineBoard(brd, parent, point) {
  parent.addChild(brd);
  console.log('parent', parent);
  console.log('Elemente', parent.elementsByName);
  var pcopy = parent.elementsByName[point];
  console.log('pcopy', pcopy);
  var origin = brd.create('point', [0, 0], { visible: false });
  // im Vergleich zur anderen Visualisierung: Alles um -pi/2 gedreht.
  var xachse = brd.create(
    'axis',
    [
      [0, 0],
      [0, -1],
    ],
    { strokeOpacity: 1 }
  );
  var yachse = brd.create('axis', [
    [0, 0],
    [1, 0],
  ]);
  xachse.removeTicks(xachse.defaultTicks);
  var tic = brd.create('ticks', [xachse, Math.PI], {
    minorTicks: 1,
    drawZero: true,
  });
  tic.setAttribute({ strokeOpacity: 0.3, minorHeight: 5, majorHeight: -1 }); // negativ value for whole board
  brd.create('text', [
    -0.3,
    -Math.PI,
    () => {
      return '\u03c0';
    },
  ]);
  brd.create('text', [
    -0.4,
    -2 * Math.PI,
    () => {
      return '2'+'\u03c0';
    },
  ]);
  var bb = brd.getBoundingBox();
  console.log(bb);
  var f = brd.create(
    'curve',
    [
      t => {
        return Math.cos(t);
      },
      t => {
        return t;
      },
      bb[3],
      bb[1],
    ],
    {
      strokeWidth: 2,
      strokeColor: 'black',
      fixed: true,
      highlight: false,
    }
  );
  var qx = brd.create(
    'point',
    [
      0,
      function() {
        return -tValue;
      },
    ],
    { visible: false }
  );
  var q = brd.create(
    'point',
    [
      function() {
        return Math.cos(tValue);
      },
      function() {
        return -tValue;
      },
    ],
    { visible: false }
  );
  qx.addParents([pcopy]);
  q.addParents([pcopy]); // for correct update
  brd.create(
    'text',
    [
      0.2,
      () => {
        return qx.Y() / 2;
      },
      'x',
    ],
    { color: '#0066CC' }
  );
  var l1 = brd.create('segment', [origin, qx], {
    color: '#0066CC',
    strokeWidth: 4,
    highlight: false,
    withLabel: false,
  });
  var l2 = brd.create('segment', [q, qx], {
    color: '#00CC00',
    strokeWidth: 4,
    highlight: false,
    withLabel: false,
  });
  var tt = brd.create('text', [bb[0] + 0.1, bb[1] - 0.2, 'cos(x)'], {
    color: '#00CC00',
    fixed: true,
    highlight: false,
  });
}
\end{code}
\end{jsxgraph}


Viele Eigenschaften, wie die Periodizität $\cos(x+2\pi)=\cos(x)$ und $\sin(x+2\pi)=\sin(x)$ für alle $x\in \R$, haben wir im Abschnitt  \link{allg-sin-cos}{Sinus und Kosinus am Einheitskreis} auch schon gesehen, weshalb wir uns hier auf die Eigenschaften beschränken, die dort noch nicht aufgeführt sind.


\begin{theorem}[Additionstheoreme]
Die Funktionen $\sin$ und $\cos$ erfüllen die sogenannten \notion{Additionstheoreme}. Für alle $x,y\in \R$ gelten
\begin{eqnarray*}
\cos(x+y) &=& \cos(x)\cos(y)-\sin(x)\sin(y), \\
\cos(x-y) &=& \cos(x)\cos(y)+\sin(x)\sin(y), \\
\sin(x+y) &=& \sin(x)\cos(y)+\cos(x)\sin(y), \\
\sin(x-y) &=& \sin(x)\cos(y)-\cos(x)\sin(y).
\end{eqnarray*}
\floatright{\href{https://www.hm-kompakt.de/video?watch=136}{\image[75]{00_Videobutton_schwarz}}}\\\\
\end{theorem}


\begin{proof*}
\begin{incremental}
\step
Dies kann man mit geometrischen Argumenten beweisen, oder aber auch über den Zusammenhang mit der komplexen Exponentialfunktion (vgl. Abschnitt \link{exp}{Exponentialreihe, Sinus und Kosinus}), aus dem
man auch die Potenzreihen-Darstellungen für Sinus und Kosinus erhält:
\[ \cos(x)= \sum_{k=0}^\infty\;(-1)^k\; \frac{x^{2k}}{(2k)!} 
    = 1 -\frac{x^2}{2!} + \frac{x^4}{4!} - \ldots \]
    und
    \[ \sin(x) = \sum_{k=0}^\infty \; (-1)^k\;\frac{x^{2k+1}}{(2k+1)!} 
    =x - \frac{x^3}{3!} + \frac{x^5}{5!}- \ldots\, ,\]
welche für alle $x\in \R$ absolut konvergieren.
\end{incremental}
\end{proof*}

\begin{theorem}
\begin{enumerate}
\item Sinus und Kosinus sind stetige Funktionen.
\item Kosinus ist streng monoton fallend auf dem Intervall $[0;\pi]$ und $\cos([0;\pi])=[-1;1]$.
\item Sinus ist streng monoton wachsend auf dem Intervall $[-\frac{\pi}{2};\frac{\pi}{2}]$
 und $\sin( [-\frac{\pi}{2};\frac{\pi}{2}])=[-1;1]$.
\end{enumerate}
\end{theorem}



\begin{proof*}
\begin{incremental}
\step
Die strenge Monotonie von Sinus und Kosinus auf den Intervallen ist direkt aus der Definition ersichtlich.
Außerdem ist $\cos(0)=1=\sin(\frac{\pi}{2})$ und $\cos(\pi)=-1=\sin(-\frac{\pi}{2})$.
Wenn wir die Stetigkeit gezeigt haben, ist jeweils nach dem 
\ref[zwischenwertsatz][Zwischenwertsatz]{thm:zwischenwertsatz} und wegen der Monotonie der Wertebereich 
genau das Intervall $[-1;1]$.

\step

Für die Stetigkeit zeigen wir zunächst die Stetigkeit beider Funktionen an der Stelle $0$, d.\,h. wir zeigen
\[  \lim_{x\to 0} \sin(x)=0=\sin(0)\quad \text{und} \quad  \lim_{x\to 0} \cos(x)=1=\cos(0). \]

Aus der Definition sieht man direkt, dass für alle $x\in \R$ mit $-\frac{\pi}{2}\leq x\leq \frac{\pi}{2}$ 
die Ungleichung $|\sin(x)|\leq |x|$ gilt, denn $|x|$ ist die Länge des Kreisbogens von $(1;0)$ zum
Punkt $(u;v)=(\cos(x);\sin(x))$ und $|\sin(x)|$ ist der Abstand dieses Punktes zur $u$-Achse.
\step
Dann gilt aber $ \lim_{x\to 0} |\sin(x)|\leq  \lim_{x\to 0} |x|=0$ und daher
\[ \lim_{x\to 0} \sin(x)=0. \]
Aus dem trigonometrischen Pythagoras $\cos(x)^2+\sin(x)^2=1$ für alle $x\in \R$ und $\cos(x)\geq 0$ für 
$-\frac{\pi}{2}\leq x\leq \frac{\pi}{2}$ erhält man 
\[ \cos(x)=\sqrt{1-\sin(x)^2} \quad \text{für alle }x\in [-\frac{\pi}{2};\frac{\pi}{2}]. \]
Da mit $\sin(x)$ auch $1-\sin(x)^2$ stetig in $0$ ist und auch die Wurzelfunktion stetig ist, ist auch die
Komposition $\sqrt{1-\sin(x)^2}$ bei $0$ stetig 
(vgl. \ref[elem-funk][Kompositionen stetiger Funktionen]{sec:kompositionen}). Also ist auch $\cos(x)$ bei 
$0$ stetig und es gilt
\[   \lim_{x\to 0} \cos(x) =\lim_{x\to 0} \sqrt{1-\sin(x)^2} =1=\cos(0). \]
\step
Für die Stetigkeit in einem beliebigen Punkt $x^*\in \R$ verwenden wir die Additionstheoreme:
\begin{eqnarray*}
\lim_{x\to x^*} \sin(x) &=& \lim_{x\to x^*}  \sin(x^*+(x-x^*)) \\
&=& \lim_{x\to x^*}[ \sin(x^*)\cos(x-x^*)+  \cos(x^*)\sin(x-x^*)] \\
&=& \lim_{h\to 0}[\sin(x^*)\cos(h)+  \cos(x^*)\sin(h)] \\
&=& \sin(x^*)\cos(0)+  \cos(x^*)\sin(0)= \sin(x^*). 
\end{eqnarray*} 
Für $\cos$ rechnet man ganz entsprechend
\begin{eqnarray*}
\lim_{x\to x^*} \cos(x)  %&=& \lim_{x\to x^*}  \cos(x^*+(x-x^*)) \\
&=& \lim_{x\to x^*}[ \cos(x^*)\cos(x-x^*)-  \sin(x^*)\sin(x-x^*) ]\\
% &=& \lim_{h\to 0}\cos(x^*)\cos(h)-  \sin(x^*)\sin(h) \\
&=& \cos(x^*)\cos(0)-  \sin(x^*)\sin(0)= \cos(x^*). 
\end{eqnarray*} 


\step Anmerkung: Verwendet man die Potenzreihendarstellung, erhält man direkt aus Abschnitt 
\ref[elem-funk][Stetigkeit elementarer Funktionen]{sec:potenzreihen}, dass Sinus und Kosinus auf ihrem 
Konvergenzbereich (also auf ganz $\R$) stetig sind. 

\end{incremental}
\end{proof*}

\section{Tangens und Kotangens}\label{tan-cot.definition}

Auch Tangens und Kotangens haben wir schon als reelle Funktionen kennengelernt:

\begin{definition}
\lang{de}{Die \notion{Tangensfunktion} $\tan$ ist definiert durch
\begin{align*}
\tan(x) = \frac{\sin(x)}{\cos(x)} ,
\end{align*}
wobei $x \in \mathbb{R}$ nicht die Werte $\;\ldots, \frac{-5\pi}{2},
\frac{-3\pi}{2}, \frac{-\pi}{2}, \frac{\pi}{2}, \frac{3\pi}{2},
\frac{5\pi}{2}, \ldots\;$ annehmen darf, das heißt, ihr Definitionsbereich
ist gegeben durch $D_{\tan} = \mathbb{R} \setminus U$ mit $U = \big\{
 k\pi + \frac{\pi}{2} \; \big| \; k \in \mathbb{Z} \big\}$.

Die \notion{Kotangensfunktion} $\cot$ ist definiert durch
\begin{align*}
\cot(x) = \frac{\cos(x)}{\sin(x)} ,
\end{align*}
wobei $x \in \mathbb{R}$ nicht die Werte
$\;\ldots, -2\pi, -\pi, 0, \pi, 2\pi, \ldots\;$
annehmen darf, das heißt, ihr Definitionsbereich ist
gegeben durch $D_{\cot} = \mathbb{R} \setminus G$ mit
$G= \big\{ k\pi \; \big| \; k \in  \mathbb{Z} \big\}$. 
%Beide Funktionen haben die Menge der reellen Zahlen als Wertemenge: $W_{\tan}=W_{\cot}=\R$.
\begin{center}
\image{T105_Tangent}
\end{center}
}
\lang{en}{The \emph{tangent} of an angle $\tan(x)$ is defined by
\begin{align*}
\tan(x) = \frac{\sin(x)}{\cos(x)} ,
\end{align*}
where $x \in \mathbb{R}$ may not take on the values $\;\ldots, \frac{-5\pi}{2},
\frac{-3\pi}{2}, \frac{-\pi}{2}, \frac{\pi}{2}, \frac{3\pi}{2},
\frac{5\pi}{2}, \ldots\;$. In set notation the domain is given by
$D_{\tan} = \mathbb{R} \setminus U$ where $U = \big\{
 k\pi + \frac{\pi}{2} \; \big| \; k \in \mathbb{Z} \big\}$.
\\
The \emph{cotangent} of an angle $\cot(x)$ is defined by
\begin{align*}
\cot(x) = \frac{\cos(x)}{\sin(x)} ,
\end{align*}
where $x \in \mathbb{R}$ may not take on the values
$\;\ldots, -2\pi, -\pi, 0, \pi, 2\pi, \ldots\;$. In set notation the domain is given by $D_{\cot} = \mathbb{R} \setminus G$ where
$G= \big\{ k\pi \; \big| \; k \in  \mathbb{Z} \big\}$. \\
Both functions' ranges are the set of real numbers: $W_{\tan}=W_{\cot}=\R$.
\begin{center}
\image[350]{tan_cot} % Bild hm-def-tancot
\end{center}
}
\end{definition}

Wir kennen bereits die Periodizität $\tan(x+\pi)=\tan(x)$ und $\cot(x+\pi)=\cot(x)$. Weitere Eigenschaften  
der Funktionen fasst das folgende Theorem zusammen: 

\begin{theorem}
\begin{enumerate}
\item Die Funktionen $\tan$ und $\cot$ sind auf ihrem gesamten Definitionsbereich stetig.
\item Tangens ist auf dem Intervall $(-\frac{\pi}{2};\frac{\pi}{2})$ streng monoton wachsend und hat den Wertebereich $\tan\big((-\frac{\pi}{2};\frac{\pi}{2})\big)=\R$.
\item Kotangens ist auf dem Intervall $(0;\pi)$ streng monoton fallend und
hat den Wertebereich $\cot\big((0;\pi)\big)=\R$.
\end{enumerate}
\end{theorem}


\begin{proof*}
\begin{incremental}
\step
Als Quotienten stetiger Funktionen sind $\tan$ und $\cot$ stetig.

Für die Monotonie von $\tan$ auf dem Intervall $(-\frac{\pi}{2};\frac{\pi}{2})$ betrachten wir $x,y\in (-\frac{\pi}{2};\frac{\pi}{2})$ mit $x<y$. Dann gilt
\begin{eqnarray*}
\tan(y)-\tan(x) &=& \frac{\sin(y)}{\cos(y)}-  \frac{\sin(x)}{\cos(x)} \\
&=& \frac{\sin(y)\cos(x)-\cos(y)\sin(x)}{\cos(y)\cos(x)}= \frac{\sin(y-x)}{\cos(y)\cos(x)},
\end{eqnarray*}
wobei im letzten Schritt ein Additionstheorem verwendet wurde.\\
Nun ist $y-x\in (0;\pi)$ und daher $\sin(y-x)>0$. Es gelten auch
$\cos(y)>0$ und $\cos(x)>0$, da $x,y\in  (-\frac{\pi}{2};\frac{\pi}{2})$.
Also ist \[ \tan(y)-\tan(x) =\frac{\sin(y-x)}{\cos(y)\cos(x)}>0. \]
Damit ist $\tan$ auf dem Intervall $(-\frac{\pi}{2};\frac{\pi}{2})$
streng monoton wachsend.
\step
Weiter gilt für alle $x$ mit  $-\frac{\pi}{2}<x<0$
\[   \tan(x)=  \frac{\sin(x)}{\cos(x)}<0 \]
und
\[ \lim_{x\searrow -\frac{\pi}{2}} \frac{1}{\tan(x)}= 
\lim_{x\searrow -\frac{\pi}{2}} \frac{\cos(x)}{\sin(x)} = 
\frac{\cos(-\frac{\pi}{2})}{\sin(-\frac{\pi}{2})}=0.\]
Also 
\[ \lim_{x\searrow -\frac{\pi}{2}} \tan(x) =-\infty. \]
Entsprechend ist wegen $\tan(x)>0$ für $0<x<\frac{\pi}{2}$ und
wegen $\lim_{x\nearrow \frac{\pi}{2}} \frac{1}{\tan(x)}=0$ auch
\[ \lim_{x\nearrow \frac{\pi}{2}} \tan(x) =\infty. \]

Damit ist der Wertebereich von $\tan$ auf dem Intervall $(-\frac{\pi}{2};\frac{\pi}{2})$ 
gleich $(-\infty;\infty)=\R$.
\step

Um die Monotonie und den Wertebereich für $\cot$ zu zeigen, verfährt man ganz entsprechend, 
oder man verwendet, dass
\[  \cot(x)= \frac{\cos(x)}{\sin(x)}=\frac{\sin(\frac{\pi}{2}-x)}{\cos(\frac{\pi}{2}-x)}
=\tan\left(\frac{\pi}{2}-x\right). \]
\end{incremental}
\end{proof*}


\section{Arkusfunktionen}\label{arcus.funktionen}

Die trigonometrischen Funktionen sind alle periodisch und deshalb nicht injektiv. Schränkt
man sie jedoch auf ein geeignetes Teilintervall von $\R$ ein, dann sind sie dort streng monoton, also auch injektiv, wie wir oben gesehen haben.

Speziell haben wir gesehen:
\begin{itemize}
\item $\sin$ ist auf dem Intervall $[-\frac{\pi}{2};\frac{\pi}{2}]$ streng monoton wachsend mit Wertemenge $[-1;1]$,
\item $\cos$ ist auf dem Intervall $[0;\pi]$ streng monoton fallend mit Wertemenge $\cos([0;\pi])=[-1;1]$, 
\item $\tan$ ist auf dem Intervall $(-\frac{\pi}{2};\frac{\pi}{2})$ streng monoton wachsend und hat den Wertebereich $\tan\big((-\frac{\pi}{2};\frac{\pi}{2})\big)=\R$,
\item $\cot$ ist auf dem Intervall $(0;\pi)$ streng monoton fallend und
hat den Wertebereich $\cot\big((0;\pi)\big)=\R$.
\end{itemize}

Man definiert daher ihre partiellen Umkehrfunktionen.

\begin{definition}
\begin{enumerate}
\item Der \notion{Arkuskosinus} $\arccos: [-1;1]\to [0;\pi]$ ist die Umkehrfunktion des eingeschränkten Kosinus,
\item der \notion{Arkussinus} $\arcsin:[-1;1]\to [-\frac{\pi}{2};\frac{\pi}{2}]$ ist die Umkehrfunktion des eingeschränkten Sinus,
\item der \notion{Arkustangens} $\arctan:\R\to (-\frac{\pi}{2};\frac{\pi}{2})$ ist die Umkehrfunktion des eingeschränkten Tangens,
\item der \notion{Arkuskotangens} $\arccot:\R\to (0;\pi)$ ist die Umkehrfunktion des eingeschränkten Kotangens.
\end{enumerate}
\end{definition}

 	\begin{genericGWTVisualization}[550][1000]{mathlet1}
 		\begin{variables}
			\function{f1}{real}{sin(x)+sqrt((pi^2)/4-x^2)-((pi^2)/4-x^2)/sqrt((pi^2)/4-x^2)} 	% eingeschränkt auf $(-pi/2;pi/2)$
			\function{f11}{real}{sin(x)}  % zur Anzeige
			\function{f2}{real}{arcsin(x)} 			
			\function{g1}{real}{cos(x)+sqrt(x*(pi-x))-x*(pi-x)/sqrt(x*(pi-x))} 	% eingeschränkt auf $(0;pi)$
			\function{g11}{real}{cos(x)}  % zur Anzeige
			\function{g2}{real}{arccos(x)} 			
			\function{h1}{real}{tan(x)+sqrt((pi^2)/4-x^2)-((pi^2)/4-x^2)/sqrt((pi^2)/4-x^2)} 	% eingeschränkt auf $(-pi/2;pi/2)$
			\function{h11}{real}{tan(x)}  % zur Anzeige
			\function{h2}{real}{arctan(x)} 			
			\function{j1}{real}{cot(x)+sqrt(x*(pi-x))-x*(pi-x)/sqrt(x*(pi-x))} 	% eingeschränkt auf $(0;pi)$
			\function{j11}{real}{cot(x)}  % zur Anzeige
			\function{j2}{real}{arccot(x)} 			
			\function{w}{real}{x}
			\function{v}{real}{x}
			\number{a1}{real}{-(pi/2)}
			\number{a2}{real}{0}
			\number{a3}{real}{(pi/2)}
			\number{a4}{real}{pi}
			\point{p11}{real}{var(a1),var(a1)}
			\point{p12}{real}{var(a1),var(a2)}
			\point{p21}{real}{var(a2),var(a1)}
			\point{p22}{real}{var(a2),var(a2)}
			\point{p33}{real}{var(a3),var(a3)}
			\point{p34}{real}{var(a3),var(a4)}
			\point{p43}{real}{var(a4),var(a3)}
			\point{p44}{real}{var(a4),var(a4)}
			\line{s1}{real}{var(p11),var(p12)}
			\line{s2}{real}{var(p21),var(p22)}
			\line{s3}{real}{var(p33),var(p34)}
			\line{s4}{real}{var(p43),var(p44)}
			\line{w1}{real}{var(p11),var(p21)}
			\line{w2}{real}{var(p12),var(p22)}
			\line{w3}{real}{var(p33),var(p43)}
			\line{w4}{real}{var(p34),var(p44)}
			
 		\end{variables}
 		\color{f1}{#0066CC}
 		\color{f11}{#0066CC}
 		\color{f2}{#0066CC}
 		\color{g1}{#CC6600}
 		\color{g11}{#CC6600}
 		\color{g2}{#CC6600}
 		\color{h1}{#00CC00}
 		\color{h11}{#00CC00}
 		\color{h2}{#00CC00}
 		\color{j1}{#CC00CC}
 		\color{j11}{#CC00CC}
 		\color{j2}{#CC00CC}
 		\color{w}{LIGHT_GRAY}
 		\color{v}{LIGHT_GRAY}
 		\color{s1}{LIGHT_GRAY}
 		\color{s2}{LIGHT_GRAY}
 		\color{s3}{LIGHT_GRAY}
 		\color{s4}{LIGHT_GRAY}
 		\color{w1}{LIGHT_GRAY}
 		\color{w2}{LIGHT_GRAY}
 		\color{w3}{LIGHT_GRAY}
 		\color{w4}{LIGHT_GRAY}
 
 		\text{Hier zunächst die Graphen der eingeschränkten trigonometrischen Funktionen
 		\var{f11}, \var{g11}, \var{h11} und \var{j11}...
 		}
 		\begin{canvas}
 			\plotSize{400}
 			\plotLeft{-3.5}
 			\plotRight{3.5}
 			\plot[coordinateSystem]{s1,s2,s3,s4,f1,g1,h1,j1}
 		\end{canvas}
 		
  		\text{... und die Graphen der zugehörigen Arkusfunktionen
 		\var{f2}, \var{g2}, \var{h2} und \var{j2}.}

 		\begin{canvas}
 			\plotSize{400}
 			\plotLeft{-3.5}
 			\plotRight{3.5}
 			\plot[coordinateSystem]{w1,w2,w3,w4,f2,g2,h2,j2}
 		\end{canvas}

        \text{Um sich die Arkusfunktionen vorzustellen, ist es hilfreich, 
        ausgewählte Punkte zu betrachten und sie \glqq umzukehren". Wir erhalten:
     
        \textcolor{##0066CC}{
        \begin{table}
        $\sin(-\frac{\pi}{2})=-1$&$\sin(0)=0$&$\sin(\frac{\pi}{2})=1$\\
        $\arcsin(-1)=-\frac{\pi}{2}$&$\arcsin(0)=0$&$\arcsin(1)=\frac{\pi}{2}$
        \end{table}}
        
        \textcolor{#CC6600}{
        \begin{table}
        $\cos(0)=1$&$\cos(\frac{\pi}{2})=0$&$\cos(\pi)=-1$\\
        $\arccos(1)=0$&$\arccos(0)=\frac{\pi}{2}$&$\arccos(-1)=\pi$
        \end{table}}
        
        \textcolor{#00CC00}{
        \begin{table}
        $\tan(-\frac{\pi}{4})=-1$&$\tan(0)=0$&$\tan(\frac{\pi}{4})=1$\\
        $\arctan(-1)=-\frac{\pi}{4}$&$\arctan(0)=0$&$\arctan(1)=\frac{\pi}{4}$
        \end{table}}
        
        \textcolor{#CC00CC}{
        \begin{table}
        $\cot(\frac{3\pi}{4})=-1$&$\cot(\frac{\pi}{2})=0$&$\cot(\frac{\pi}{4})=1$\\
        $\arccot(-1)=\frac{3\pi}{4}$&$\arccot(0)=\frac{\pi}{2}$&$\arccot(1)=\frac{\pi}{4}$
        \end{table}}
        }


 	    	\end{genericGWTVisualization}


Aus dem \ref[elem-funk][Satz zur Stetigkeit der Umkehrfunktion]{thm:inverse-stetig} erhält man direkt:

\begin{theorem}
Die Arkusfunktionen sind alle auf ihrem Definitionsbereich stetig.
\end{theorem}

\begin{quickcheckcontainer}

\randomquickcheckpool{1}{2}
\randomquickcheckpool{3}{3}

\begin{quickcheck}
    \field{rational}
    \type{input.number}
    \begin{variables}
        \number{n}{0}
    \end{variables}
    \text{$\arcsin(0)=$ \ansref }
    \begin{answer}
        \solution{n}
    \end{answer}
\end{quickcheck}

\begin{quickcheck}
    \field{rational}
    \type{input.function}
    \begin{variables}
        \function{f}{pi/2}
    \end{variables}
    \text{$\arccos(0)=$ \ansref }
    \begin{answer}
        \solution{f}
    \end{answer}
\end{quickcheck}

\begin{quickcheck}
    \field{rational}
    \type{input.function}
    \begin{variables}
        \function{f}{pi/2}
    \end{variables}
    \text{$\lim_{u\to\infty}\arctan(u)=$ \ansref }
    \begin{answer}
        \solution{f}
    \end{answer}
\end{quickcheck}


\end{quickcheckcontainer}

\end{visualizationwrapper}
\end{content}