%$Id:  $
\documentclass{mumie.article}
%$Id$
\begin{metainfo}
  \name{
    \lang{de}{Überblick: Eigenschaften stetiger Funktionen}
    \lang{en}{overview: }
  }
  \begin{description} 
 This work is licensed under the Creative Commons License Attribution 4.0 International (CC-BY 4.0)   
 https://creativecommons.org/licenses/by/4.0/legalcode 

    \lang{de}{Beschreibung}
    \lang{en}{}
  \end{description}
  \begin{components}
  \end{components}
  \begin{links}
\link{generic_article}{content/rwth/HM1/T211_Eigenschaften_stetiger_Funktionen/g_art_content_36_anwendungen.meta.xml}{content_36_anwendungen}
\link{generic_article}{content/rwth/HM1/T211_Eigenschaften_stetiger_Funktionen/g_art_content_35_trigonom_funktionen.meta.xml}{content_35_trigonom_funktionen}
\link{generic_article}{content/rwth/HM1/T211_Eigenschaften_stetiger_Funktionen/g_art_content_34_exp_und_log.meta.xml}{content_34_exp_und_log}
\link{generic_article}{content/rwth/HM1/T211_Eigenschaften_stetiger_Funktionen/g_art_content_33_zwischenwertsatz.meta.xml}{content_33_zwischenwertsatz}
\end{links}
  \creategeneric
\end{metainfo}
\begin{content}
\begin{block}[annotation]
	Im Ticket-System: \href{https://team.mumie.net/issues/30126}{Ticket 30126}
\end{block}
\begin{block}[annotation]
Copy of : /home/mumie/checkin/content/rwth/HM1/T301_Differenzierbarkeit/art_T301_overview.src.tex
\end{block}





\begin{block}[annotation]
Im Entstehen: Überblicksseite für Kapitel  Eigenschaften stetiger Funktionen
\end{block}

\usepackage{mumie.ombplus}
\ombchapter{1}
\lang{de}{\title{Überblick: Eigenschaften stetiger Funktionen}}
\lang{en}{\title{}}



\begin{block}[info-box]
\lang{de}{\strong{Inhalt}}
\lang{en}{\strong{Contents}}


\lang{de}{
    \begin{enumerate}%[arabic chapter-overview]
   \item[11.1] \link{content_33_zwischenwertsatz}{Sätze zu stetigen Funktionen}
   \item[11.2] \link{content_34_exp_und_log}{Exponentialfunktion und Logarithmus}
   \item[11.3] \link{content_35_trigonom_funktionen}{Trigonometrische Funktionen}
   \item[11.4] \link{content_36_anwendungen}{Polardarstellung, Potenzen und Wurzeln komplexer Zahlen}
   \end{enumerate}
} %lang

\end{block}

\begin{zusammenfassung}

\lang{de}{}
Wir betrachten stetige Funktionen im Kleinen
und (oft eingeschränkt) auf abgeschlossenen Intervallen. 
Auf einem abgeschlossenen Intervall $[a;b]$ verhält sich eine stetige Funktion $f$ besonders nett: Sie nimmt dort ihr Minimum und ihr Maximum an, 
ebenso  jeden Wert zwischen $f(a)$ und $f(b)$.

Nun sind wir endlich in der Lage, die natürliche Exponentialfunktion $e^x$ für alle $x\in\R$ zu definieren, indem wir sie mit der Exponentialreihe identifizieren.
Das nehmen wir zum Anlass, Exponential- und Logarithmusfunktionen noch einmal mit ihren nun bewiesenen Eigenschaften darzustellen.

Auch die Eigenschaften der trigonometrischen Funktionen $\cos$, $\sin$, $\tan$, $\cot$ kennen wir jetzt wirklich, 
und wir betrachten ihre Umkehrfunktionen, die Arcus-Funktionen, auf sinnvollen Intervallen.

Mit unserem vertieften Wissen über die Exponentialfunktion statten wir den komplexen Zahlen wieder einen Besuch ab. 
Wir erklären die Polarkoordinatenschreibweise $z=r\cdot e^{i\phi}$ und befassen uns mit Potenzen und Wurzeln komplexer Zahlen.


\end{zusammenfassung}

\begin{block}[info]\lang{de}{\strong{Lernziele}}
\lang{en}{\strong{Learning Goals}} 
\begin{itemize}[square]
\item \lang{de}{Sie kennen den Zwischenwertsatz und den Satz vom Minimum und Maximum und wenden sie sicher an.}
\item \lang{de}{Sie kennen neben $\exp$ und $\ln$ noch viele weitere Exponential- und Logarithmusfunktionen und deren Eigenschaften.}
\item \lang{de}{Sie kennen die trigonometrischen Funktionen mit ihren Eigenschaften sowie die Additionstheoreme.}
\item \lang{de}{Sie wissen, in welchen Bereichen die trigonometrischen Funktionen umgekehrt werden können, kennen die Umkehrfunktionen und deren Eigenschaften.}
\item \lang{de}{Sie skizzieren bzw. identifizieren die Graphen der trigonometrischen und der Arcus-Funktionen und Verschiebungen von ihnen.}
\item \lang{de}{Sie kennen die Werte von $\cos$ und $\sin$ an den ganzzahligen Vielfachen von $\frac{\pi}{6}$.}
\item \lang{de}{Sie rechnen komplexe Zahlen aus Normalform in Polarkoordinaten um und umgekehrt.}
\item \lang{de}{Sie berechnen Potenzen komplexer Zahlen.}
\item \lang{de}{Sie wissen, dass und warum man beim Wurzelziehen aus komplexen Zahlen höllisch aufpassen muss.}
\end{itemize}
\end{block}




\end{content}
