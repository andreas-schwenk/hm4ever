\documentclass{mumie.problem.gwtmathlet}
%$Id$
\begin{metainfo}
  \name{
    \lang{de}{A04: Polarkoordinaten}
    \lang{en}{Problem 3}
  }
  \begin{description} 
 This work is licensed under the Creative Commons License Attribution 4.0 International (CC-BY 4.0)   
 https://creativecommons.org/licenses/by/4.0/legalcode 

    \lang{de}{Beschreibung}
    \lang{en}{description}
  \end{description}
  \corrector{system/problem/GenericCorrector.meta.xml}
  \begin{components}
    \component{js_lib}{system/problem/GenericMathlet.meta.xml}{gwtmathlet}
  \end{components}
  \begin{links}
  \end{links}
  \creategeneric
\end{metainfo}
\begin{content}
\usepackage{mumie.ombplus}
\usepackage{mumie.genericproblem}

\lang{de}{\title{A04: Polarkoordinaten}}
\lang{en}{\title{Problem 4}}

\begin{block}[annotation]
	Im Ticket-System: \href{http://team.mumie.net/issues/10058}{Ticket 10058}
\end{block}


\begin{problem}
   \begin{question}
    \lang{de}{\text{
      Schreiben Sie $z=\var{z}$ in der Form $r\cdot e^{i\phi}$, wobei $r>0$ und $0\le\phi < 2\pi$ gilt. Geben Sie das
      Ergebnis exakt oder mit einer Mindestgenauigkeit von zwei Nachkommastellen an.
    }}
    \type{input.function}
    \field{complex}
    %\precision{8}
    %\correctorprecision[atleast]{2}
    \begin{variables}
      \randint[Z]{a}{-9}{9}
      \randint[Z]{b}{-9}{9}
      \function[calculate]{z}{a+i*b}
      \function[calculate]{r}{sqrt(a^2+b^2)}
      \function[calculate]{d1}{theta(a)}
      \function[calculate]{d2}{theta(b)}
      \function[calculate]{d3}{theta(-1*a)}
      \function[calculate]{d4}{theta(-1*b)}
      \function[calculate]{phiTemp}{atan(b/a)}
      \function[calculate]{phi}{d1*d2*phiTemp + d3*(pi+phiTemp) + d1*d4*(2*pi+phiTemp)}
      % 1. Quadrant + 2./3. Quadrant + 4.Quandrant
    \end{variables}

    \begin{answer}
      \text{$r = $}
      \lang{de}{\explanation{
        Für eine komplexe Zahl $z=a+bi$ gilt: $r=\abs{z}$. 
      }}      
      \solution{r}
      \checkAsFunction[0.005]{x}{-10}{10}{10}
    \end{answer}

    \begin{answer}
      \text{$\phi = $}
      \lang{de}{\explanation{
        Wird der Winkel $\phi$ über $\arctan\left(\frac{\Im(z)}{\Re(z)}\right)$ berechnet, dann muss im 2. und 3. Quadranten
        $\pi$ addiert werden und im 4. Quadranten muss $2\pi$ addiert werden.}
        }        
      \solution{phi}
      \checkAsFunction[0.005]{x}{-10}{10}{10}
    \end{answer}

  \end{question}
\end{problem}

\embedmathlet{gwtmathlet}
\end{content}