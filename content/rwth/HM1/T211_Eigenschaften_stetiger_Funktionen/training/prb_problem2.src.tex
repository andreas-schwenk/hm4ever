\documentclass{mumie.problem.gwtmathlet}
%$Id$
\begin{metainfo}
  \name{
    \lang{de}{A02: Nullstellen}
    \lang{en}{Problem 2}
  }
  \begin{description} 
 This work is licensed under the Creative Commons License Attribution 4.0 International (CC-BY 4.0)   
 https://creativecommons.org/licenses/by/4.0/legalcode 

    \lang{de}{Beschreibung}
    \lang{en}{description}
  \end{description}
  \corrector{system/problem/GenericCorrector.meta.xml}
  \begin{components}
    \component{js_lib}{system/problem/GenericMathlet.meta.xml}{gwtmathlet}
  \end{components}
  \begin{links}
  \end{links}
  \creategeneric
\end{metainfo}
\begin{content}
\usepackage{mumie.ombplus}
\usepackage{mumie.genericproblem}

\lang{de}{\title{A02: Nullstellen}}
\lang{en}{\title{Problem 2}}

\begin{block}[annotation]
	Im Ticket-System: \href{http://team.mumie.net/issues/10056}{Ticket 10056}
\end{block}


\begin{problem}
    
  %  \begin{question} 
  %   
  %  	\lang{de}{ 
  %   	\text{Es sei $c\in\R$. Gegeben sei die Funktion $f:\R\to\R$ mit $f(x)=x^{6}+198 x^{4}-cx$ 
  %     für ein $c\in\R$. Bestimmen Sie $c\in\R$ so, dass $f$ mindestens zwei 
  %     reelle Nullstellen besitzt.}
  % 	}
  % 	\type{mc.unique}
  %    \permutechoices{1}{5}
    %		\begin{choice}
	%		\text{$c \in \mathbb{R}$}
	%		\solution{false}
	%	\end{choice}
	%	\begin{choice}
	%		\text{$c \neq 0$}
	%		\solution{true}
	%	\end{choice}
	%	\begin{choice}
	%		\text{$c < 0$}
	%		\solution{false}
	%	\end{choice}
	%	\begin{choice}
	%		\text{$c > 0$}
	%		\solution{false}
	%	\end{choice}
	%	\begin{choice}
	%		\text{$c \ge 1$}
	%		\solution{false}
	%	\end{choice}
	%	
	% \end{question}
    
     \begin{question}
        \text{Es sei $c\in\R$. Gegeben sei die Funktion $f:\R\to\R$ mit $f(x)=x^3+2x^2+cx$ 
        für ein $c\in\R$. Für welche Wahl von $c\in\R$ besitzt $f$ mindestens zwei 
        reelle Nullstellen?}
        \type{mc.unique}
	    \permutechoices{1}{5}
		\begin{choice}
			\text{$c \in \mathbb{R}$}
			\solution{false}
		\end{choice}
		\begin{choice}
			\text{$c \leq 1$}
			\solution{true}
		\end{choice}
		\begin{choice}
			\text{$c < 0$}
			\solution{false}
		\end{choice}
		\begin{choice}
			\text{$c > 0$}
			\solution{false}
		\end{choice}
		\begin{choice}
			\text{$c \ge 1$}
			\solution{false}
		\end{choice}
        \explanation{$f(x)=x(x^2+2x+c)$, d.h. $x=0$ ist auf alle Fälle eine Nullstelle. Damit noch 
        eine weitere Nullstelle existiert, muss das quadratische Polynom mindestens eine Nullstelle haben.
        Dies ist dann der Fall, wenn $1-c\geq 0$ ist (Wurzelterm in pq-Formel darf nicht negativ sein).
        }
     \end{question} 
     
 
    \begin{question}
    \text{Hat die Gleichung $\sin(x)=\frac{\cos(x)}{1+x^2}$ im Intervall $[0;\frac{\pi}{2}]$ eine Lösung?}
    \type{mc.unique}
        \begin{choice}
            \text{Es existiert eine Lösung.}
            \solution{true}
        \end{choice}
        \begin{choice}
            \text{Es existiert keine Lösung.}
            \solution{false}
        \end{choice}
        \explanation{Betrachten Sie die stetige Funktion $f(x)=\sin(x)-\frac{\cos(x)}{1+x^2}$.}
     \end{question}
     
     
\end{problem}

\embedmathlet{gwtmathlet}



\end{content}