\documentclass{mumie.problem.gwtmathlet}
%$Id$
\begin{metainfo}
  \name{
    \lang{en}{...}
    \lang{de}{A07: Potenz}
  }
  \begin{description} 
 This work is licensed under the Creative Commons License Attribution 4.0 International (CC-BY 4.0)   
 https://creativecommons.org/licenses/by/4.0/legalcode 

    \lang{en}{...}
    \lang{de}{...}
  \end{description}
  \corrector{system/problem/GenericCorrector.meta.xml}
  \begin{components}
    \component{js_lib}{system/problem/GenericMathlet.meta.xml}{gwtmathlet}
  \end{components}
  \begin{links}
  \end{links}
  \creategeneric
\end{metainfo}
\begin{content}
\begin{block}[annotation]
	Im Ticket-System: \href{https://team.mumie.net/issues/21538}{Ticket 21538}
\end{block}
\usepackage{mumie.genericproblem}

\lang{de}{\title{A07: Potenz}}

\begin{problem}
           \begin{variables}
                \randint{a}{-10}{-5}
                \randint{b}{1}{2}
                
                \function{r}{sqrt(a^2+b^2)}
                \function{phi}{pi+atan(b/a)}
                \function{r4}{(a^2+b^2)^2}
                \function[calculate,2]{phi4}{4*atan(b/a)}
                 % z^4 bildet (157,5°;180°)auf den 4. Quadranten ab.
           \end{variables}        
     
     
          \begin{question}
                \type{input.function}
                \field{real}
                    \text{Berechnen sie zu $z=\var{a}+\var{b}i$ die vierte 
                    Potenz $z^4=re^{i\phi}$. \\Geben Sie $\phi$ mit zwei 
                    Nachkommastellen ein und wählen Sie $\phi$ so, dass $-\pi<\phi\leq+\pi$.\\
                    $r=$\ansref und $\phi=$\ansref}
               \begin{answer}
                    \solution{r4}
                    \inputAsFunction{x}{antwortr}
                    \checkFuncForZero{r4-antwortr}{1}{2}{1}
               \end{answer}
               \begin{answer}
                    \solution{phi4}
                    \inputAsFunction{x}{antwortr}
                    \checkFuncForZero{r4-antwortr}{1}{2}{1}
               \end{answer} 
               \explanation{$r= |z|^4 = ((\var{a})^2+\var{b}^2)^2$ und 
               $\arg(z)=\pi+\arctan(b/a)$, so dass für $\arg(z^4)$ gilt: 
               $\phi=4\arg(z) -4\pi$. Die $4\pi$ berücksichtigen, dass $\arg(z)$
               durch Vervierfachung so groß geworden ist.}
               
          \end{question}     
     \end{problem}


\embedmathlet{gwtmathlet}

\end{content}
