\documentclass{mumie.problem.gwtmathlet}
%$Id$
\begin{metainfo}
  \name{
    \lang{en}{...}
    \lang{de}{A06: komplexe Potenzen}
  }
  \begin{description} 
 This work is licensed under the Creative Commons License Attribution 4.0 International (CC-BY 4.0)   
 https://creativecommons.org/licenses/by/4.0/legalcode 

    \lang{en}{...}
    \lang{de}{...}
  \end{description}
  \corrector{system/problem/GenericCorrector.meta.xml}
  \begin{components}
    \component{js_lib}{system/problem/GenericMathlet.meta.xml}{gwtmathlet}
  \end{components}
  \begin{links}
  \end{links}
  \creategeneric
\end{metainfo}
\begin{content}
\begin{block}[annotation]
	Im Ticket-System: \href{https://team.mumie.net/issues/21512}{Ticket 21512}
\end{block}
\usepackage{mumie.genericproblem}

\lang{de}{\title{A06: komplexe Potenzen}}

     \begin{problem}
     
        \begin{variables}
            \randint[Z]{a}{2}{5}
            \function[expand,normalize]{z}{a*exp(i*pi)}
            \function{z2}{a^2}
            %\function[calculate, expand,normalize]{z3}{a^3*e^(i*pi)}
            \function{z3}{a^3*e^(i*pi)}
            \function[calculate]{z3b}{a^3*e^(i*pi)}
            \function[expand,normalize]{r3}{a^3}
        \end{variables}
        
        \begin{question}
               \type{input.function}
               \field{complex}
               \text{Berechnen Sie zu $z=\var{a}e^{i\pi}$ das Quadrat sowie die dritte Potenz:\\
               $z^2=$\ansref \\
               $z^3=$\ansref
               }
               \begin{answer}
                    \solution{z2}
                    \inputAsFunction{x}{antwort2}
                    \checkFuncForZero{z2-antwort2}{1}{2}{1}
               \end{answer}
               \begin{answer}
                    \solution{z3b}
                    \inputAsFunction{x}{antwort3}
                    \checkStringsForRelation{equal(antwort3,z3) OR equal(antwort3,z3b)}
                    %\checkFuncForZero{z3-antwort3}{1}{2}{1}
               \end{answer}               
               \explanation{Für $z^2$ gilt $r=\var{a}^2$ und $\phi=2\pi$, also ist das Quadrat
               eine reelle Zahl. Für $z^3$ gilt $r=\var{a}^3$ und $\phi=3\pi$. 
               Dies ist wie $\phi=\pi$ und entspricht einer negativen, reellen Zahl, also $z^3=-\var{r3}$.}
          \end{question}
          
     \end{problem}



\embedmathlet{gwtmathlet}

\end{content}
