\documentclass{mumie.element.exercise}
%$Id$
\begin{metainfo}
  \name{
    \lang{de}{Ü03: Zwischenwertsatz}
    \lang{en}{Exercise 3}
  }
  \begin{description} 
 This work is licensed under the Creative Commons License Attribution 4.0 International (CC-BY 4.0)   
 https://creativecommons.org/licenses/by/4.0/legalcode 

    \lang{de}{Hier die Beschreibung}
    \lang{en}{}
  \end{description}
  \begin{components}
  \end{components}
  \begin{links}
\link{generic_article}{content/rwth/HM1/T211_Eigenschaften_stetiger_Funktionen/g_art_content_33_zwischenwertsatz.meta.xml}{content_33_zwischenwertsatz}
\end{links}
  \creategeneric
\end{metainfo}
\begin{content}
\title{
\lang{de}{Ü03: Zwischenwertsatz}
}
\begin{block}[annotation]
  Im Ticket-System: \href{http://team.mumie.net/issues/10051}{Ticket 10051}
\end{block}
 
Es sei $f:[0;1]\to[0;1]$ eine stetige Funktion. Beweisen Sie, dass $f$ einen Fixpunkt besitzt, das heißt, zeigen Sie, dass es ein $x\in[0;1]$ gibt mit $f(x)=x$.


\begin{tabs*}[\initialtab{0}\class{exercise}]

  \tab{
  \lang{de}{Lösung}}
\begin{incremental}[\initialsteps{1}]
	\step Wir betrachten die Funktion $g:[0;1]\to\R$, die gegeben ist durch $g(x)=f(x)-x$. Diese Funktion ist stetig, 
    da $f$ stetig ist. Die Behauptung ist äquivalent zu der Aussage, dass $g$ eine Nullstelle hat. 
	\step Es ist $g(0)=f(0)\geq 0$ und $g(1)=f(1)-1\leq 0$. Falls $f(0)=0$ oder
    $f(1)=1$ gilt, so ist nichts mehr zu zeigen. 
    Es sei daher ohne Einschränkung $g(0)>0$ und $g(1)<0$. Dann existiert nach dem \ref[content_33_zwischenwertsatz][Zwischenwertsatz]{thm:zwischenwertsatz}
     ein $x\in (0;1)$ mit der Eigenschaft $g(x)=0$, was zu zeigen war. 
\end{incremental}

\end{tabs*}

\end{content}