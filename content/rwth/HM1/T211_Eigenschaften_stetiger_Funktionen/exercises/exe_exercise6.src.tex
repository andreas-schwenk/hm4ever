\documentclass{mumie.element.exercise}
%$Id$
\begin{metainfo}
  \name{
    \lang{de}{Ü06: komplexe Gleichung}
    \lang{en}{Exercise 6}
  }
  \begin{description} 
 This work is licensed under the Creative Commons License Attribution 4.0 International (CC-BY 4.0)   
 https://creativecommons.org/licenses/by/4.0/legalcode 

    \lang{de}{Hier die Beschreibung}
    \lang{en}{}
  \end{description}
  \begin{components}
  \end{components}
  \begin{links}
  \end{links}
  \creategeneric
\end{metainfo}
\begin{content}
\title{
\lang{de}{Ü06: komplexe Gleichung}
}
\begin{block}[annotation]
  Im Ticket-System: \href{http://team.mumie.net/issues/10054}{Ticket 10054}
\end{block}
 
Bestimmen Sie alle $z\in\C$ mit der Eigenschaft $e^{z}=9$.

\begin{tabs*}[\initialtab{0}\class{exercise}]

  \tab{
  \lang{de}{Lösung}}
Wir schreiben $z=x+iy$ mit $x,y\in\R$. Dann ist die Gleichung äquivalent zu
\[9=e^{x+iy}=e^{x}\cdot e^{iy}=e^{x}(\cos(y)+i\sin(y))\,.\]
Da die linke Seite der Gleichung reell ist, muss $\sin(y)=0$ gelten. Daraus 
folgt $y\in\{\pi k\,|\,k\in\Z\}$. Der Term $e^{x}$ ist positiv für jedes $x\in\R$. 
Daher kann die Gleichung nur gelten, falls $\cos(y)>0$ ist. Wegen $\cos(k\pi)=(-1)^{k}$ 
folgern wir, dass $k$ gerade ist. Weiter gilt $e^{x}=9$ genau dann, wenn $x=\ln(9)=2\ln(3)$. 
Daher wird die Gleichung von allen komplexen Zahlen der Form $z=2\ln(3)+2\pi ik$ mit $k\in\Z$ gelöst.


\end{tabs*}

\end{content}