\documentclass{mumie.element.exercise}
%$Id$
\begin{metainfo}
  \name{
    \lang{de}{Ü02: Zwischenwertsatz}
    \lang{en}{Exercise 2}
  }
  \begin{description} 
 This work is licensed under the Creative Commons License Attribution 4.0 International (CC-BY 4.0)   
 https://creativecommons.org/licenses/by/4.0/legalcode 

    \lang{de}{Hier die Beschreibung}
    \lang{en}{}
  \end{description}
  \begin{components}
  \end{components}
  \begin{links}
\link{generic_article}{content/rwth/HM1/T211_Eigenschaften_stetiger_Funktionen/g_art_content_33_zwischenwertsatz.meta.xml}{content_33_zwischenwertsatz}
\end{links}
  \creategeneric
\end{metainfo}
\begin{content}
\title{
\lang{de}{Ü02: Zwischenwertsatz}
}
\begin{block}[annotation]
  Im Ticket-System: \href{http://team.mumie.net/issues/10050}{Ticket 10050}
\end{block}
 
Es sei die Funktion $f:\R\to\R$ gegeben durch $f(x)=\exp(x)-2$. Zeigen Sie, dass $f$ genau eine Nullstelle hat. 


\begin{tabs*}[\initialtab{0}\class{exercise}]

  \tab{
  \lang{de}{Lösung}}
\begin{incremental}[\initialsteps{1}]
	\step Die Funktion $f$ ist stetig als Komposition stetiger Funktionen. Es gilt $f(0)=\exp(0)-2=-1<0$ und 
    wegen $\exp(1)=e>2$ gilt $f(1)=e-2>0$. Nach dem\ref[content_33_zwischenwertsatz][ Zwischenwertsatz]{thm:zwischenwertsatz} existiert somit ein $\zeta\in (0;1)$ mit der Eigenschaft $f(\zeta)=0$.
	\step Weil die Exponentialfunktion streng monoton steigend ist, ist auch $f$ streng monoton steigend. Der Funktionsgraph von $f$ ist nämlich der um zwei Einheiten nach unten verschobene Funktionsgraph der Exponentialfunktion. Daher besitzt $f$ höchstens eine Nullstelle. 
	\step Insgesamt haben wir gezeigt, dass $f$ genau eine Nullstelle besitzt. 
  \end{incremental}

\end{tabs*}

\end{content}