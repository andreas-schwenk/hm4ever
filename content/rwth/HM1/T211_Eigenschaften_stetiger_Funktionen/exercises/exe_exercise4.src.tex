\documentclass{mumie.element.exercise}
%$Id$
\begin{metainfo}
  \name{
    \lang{de}{Ü04: Gleichungen}
    \lang{en}{Exercise 4}
  }
  \begin{description} 
 This work is licensed under the Creative Commons License Attribution 4.0 International (CC-BY 4.0)   
 https://creativecommons.org/licenses/by/4.0/legalcode 

    \lang{de}{Hier die Beschreibung}
    \lang{en}{}
  \end{description}
  \begin{components}
  \end{components}
  \begin{links}
  \end{links}
  \creategeneric
\end{metainfo}
\begin{content}
\title{
\lang{de}{Ü04: Gleichungen}
}
\begin{block}[annotation]
  Im Ticket-System: \href{http://team.mumie.net/issues/10052}{Ticket 10052}
\end{block}
 
\begin{itemize}
\item[a)] {Bestimmen Sie alle reellen Lösungen der Gleichung $e^{x}+e^{-x}=5$.}\\
\item[b)] {Bestimmen Sie alle reellen Lösungen der Gleichung $\tan(x)=\cos(x)$.}
\end{itemize}


\begin{tabs*}[\initialtab{0}\class{exercise}]

  \tab{
  \lang{de}{Lösung a)}}
  \begin{incremental}[\initialsteps{1}]
	\step 
Wir bemerken zunächst, dass die Funktion $f(x)=e^{x}+e^{-x}$ symmetrisch zur $y$-Achse ist: Es gilt $f(-x)=f(x)$ für alle $x\in\R$. Wir formulieren das Problem zunächst um:
\[e^{x}+e^{-x}=5\iff \frac{e^{2x}+1}{e^{x}}=5\iff e^{2x}-5e^{x}+1=0\,.\]
Wir setzen nun $u:=e^{x}$. Dann schreibt sich die letzte Gleichung in der Form
\[u^{2}-5u+1=0\,.\]
\step Dies ist eine quadratische Gleichung in $u$. Die Nullstellen können wir mit Hilfe der $p$-$q$-Formel bestimmen:
\[u^{2}-5u+1=0\iff u=\frac{5+\sqrt{21}}{2}\text{ oder } \ u=\frac{5-\sqrt{21}}{2}\,.\]
\step Die Lösungen der ursprünglichen Gleichung sind also $x_{1}=\ln\left(\frac{5+\sqrt{21}}{2}\right)$ und  $x_{2}=\ln\left(\frac{5-\sqrt{21}}{2}\right)$. Hier muss
natürlich darauf geachtet werden, dass die Argumente in der Logarithmusfunktion positiv sind, was hier aber der Fall ist. 

Beachte, dass wegen der Logarithmenregeln tatsächlich $x_{2}=-x_{1}$ gilt:
\[-x_{1}=-\ln\left(\frac{5+\sqrt{21}}{2}\right)=\ln\left(\frac{1}{\frac{5+\sqrt{21}}{2}}\right)=\ln\left(\frac{2}{5+\sqrt{21}}\right)\,.\]
Weiter ist
\[\frac{2}{5+\sqrt{21}}=\frac{2 \cdot (5-\sqrt{21})}{(5+\sqrt{21})(5-\sqrt{21})} = \frac{2 \cdot (5-\sqrt{21})}{4} = \frac{5-\sqrt{21}}{2}\,.\]
\end{incremental}

	\tab{
	\lang{de}{Lösung b)}}
	\begin{incremental}[\initialsteps{1}]
	\step Der Tangens ist definiert auf der Menge
\[M=\R\backslash\left\{\frac{\pi}{2}+k\pi\,|\,k\in\Z\right\}\,.\]
Die Lösungen sind also innerhalb dieser Menge zu suchen. 
\step Die Gleichung lässt sich folgendermaßen umstellen:
\begin{align*}
\tan(x)=\cos(x) &\iff \frac{\sin(x)}{\cos(x)}=\cos(x)\iff \sin(x)=\cos(x)^{2} \\ &\iff \sin(x)=1-\sin(x)^{2}\,,
\end{align*}
wobei die letzte Identität laut Vorlesung gilt. Substituiert man $u=\sin(x)$, so führt dies auf das Nullstellenproblem
\[u^{2}+u-1=0\,.\]
Jede Lösung des ursprünglichen Problems ist also auch eine Lösung des Nullstellenproblems. 
\step Wir können diese quadratische Gleichung wieder mit Hilfe der $p$-$q$-Formel lösen und erhalten als Lösungen
\[u_{1}=\frac{\sqrt{5}-1}{2}\quad,\quad u_{2}=\frac{-\sqrt{5}-1}{2}\,.\]
\step Die Umkehrfunktion des Sinus ist $\arcsin$. Diese ist definiert auf der Menge $[-1;1]$. Wegen $2<\sqrt{5}<3$ ist $u_{1}\in[-1;1]$, 
aber $u_{2}\notin [-1;1]$. Daher erhalten wir $x_{1}=\arcsin(u_{1})$ als eindeutige Lösung im Intervall $(-\frac{\pi}{2};\frac{\pi}{2})$. Offenbar 
ist $x_{1}\in M$. Aus der Identität $\sin(\pi-x)=\sin(x)$ erhalten wir $x_{2}=\pi-x_{1}$ als weitere Lösung im Intervall $(\frac{\pi}{2};\frac{3\pi}{2})$.

Wegen der $2\pi$-Periodizität des Sinus ist die Lösungsmenge der Gleichung gegeben durch
\[\{x_{1}+2\pi k\,|\,k\in\Z\}\cup\{x_{2}+2\pi k\,|\,k\in\Z\}\,.\]	
\end{incremental}
\end{tabs*}

\end{content}