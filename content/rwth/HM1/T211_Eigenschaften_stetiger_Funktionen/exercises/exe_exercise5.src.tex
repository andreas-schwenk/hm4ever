\documentclass{mumie.element.exercise}
%$Id$
\begin{metainfo}
  \name{
    \lang{de}{Ü05: Additionstheoreme}
    \lang{en}{Exercise 5}
  }
  \begin{description} 
 This work is licensed under the Creative Commons License Attribution 4.0 International (CC-BY 4.0)   
 https://creativecommons.org/licenses/by/4.0/legalcode 

    \lang{de}{Hier die Beschreibung}
    \lang{en}{}
  \end{description}
  \begin{components}
  \end{components}
  \begin{links}
  \end{links}
  \creategeneric
\end{metainfo}
\begin{content}
\title{
\lang{de}{Ü05: Additionstheoreme}
}
\begin{block}[annotation]
  Im Ticket-System: \href{http://team.mumie.net/issues/10053}{Ticket 10053}
\end{block}
 
 Es seien $x, y \in \R$. Zeigen Sie folgende Identitäten.
 
\begin{enumerate}
\item
 {$\ \sin(x-y) = \sin(x)\cos(y) - \sin(y)\cos(x)$}\\
\item
\begin{enumerate}
\item[a)] {$\ \sin(x)\cdot\cos(y)=\frac{1}{2}(\sin(x+y)+\sin(x-y))$}\\
\item[b)] {$\ \sin(x)\cdot\sin(y)=\frac{1}{2}(\cos(x-y)-\cos(x+y))$}\\
\item[c)] {$\ \sin(x)+\sin(y)=2(\sin(\frac{x+y}{2})\cdot\cos(\frac{x-y}{2}))$},\  Tipp: Teil a)\\
\item[d)] {$\ \sin(3x)=(3\cdot\cos^2(x)-\sin^2(x))\cdot\sin(x)$}\\
\end{enumerate}
\end{enumerate}


\begin{tabs*}[\initialtab{0}\class{exercise}]

  \tab{
  \lang{de}{Lösung 1a)}}
Es gilt die Identität $\sin(x+y)=\cos(x)\sin(y)+\cos(y)\sin(x)$. Folglich erhalten wir
 \[
  \sin(x+(-y))=\cos(x)\cdot \sin(-y)+ \cos(-y)\cdot\sin(x).
 \]
Außerdem gelten die Identitäten 
\[
\cos(x)=\cos(-x) \text{ und } -\sin(x)=\sin(-x). 
\]
Wir schließen, dass
\[
 \sin(x-y)=-\cos(x)\cdot \sin(y)+\cos(y) \cdot \sin(x)
\]
gilt. Nach Umsortieren ist dies die gesuchte Gleichung. 

\tab{
\lang{de}{Lösung 2a)}}
Es gilt mit der ersten Gleichung
\[
\sin(x-y) = \sin(x)\cos(y) - \sin(y)\cos(x)  
\]
und 
\[
 \sin(x+y) = \sin(x)\cos(y) +\sin(y)\cos(x).  
\]
Addiert man diese beiden Gleichungen und teilt danach durch $2$, erhält man die Gleichung
\[
 \frac{1}{2}\left(\sin(x-y)+\sin(x+y)\right)=\frac{1}{2}\cdot 2\cdot \sin(x)\cdot \cos(y).
\]

	\tab{
	\lang{de}{Lösung 2b)}}
Es gilt 
 \[
 \cos(x+y)=\cos(x)\cos(y)-\sin(x)\sin(y)
 \]
 und folglich auch
 \[
 \cos(x+(-y))=\cos(x)\cos(-y)-\sin(x)\sin(-y)=\cos(x)\cos(y)+\sin(x)\sin(y),
 \]
wobei hier wieder die Achsensymmetrie des Kosinus und die Punktsymmetrie des Sinus verwendet wurde.
Wir folgern, dass
\[
 \cos(x-y)-\cos(x+y)=\sin(x)\sin(y)+\sin(x)\sin(y)=2\sin(x)\sin(y)
\]
gilt. Teilen durch zwei liefert die Behauptung.



   \tab{\lang{de}{Lösungsvideo 2 a)-d)}}	
    \youtubevideo[500][300]{SoAMNFQ0mrw}\\


\end{tabs*}

\end{content}