\documentclass{mumie.element.exercise}
%$Id$
\begin{metainfo}
  \name{
    \lang{de}{Ü01: Minimum-Maximum-Satz}
    \lang{en}{Exercise 1}
  }
  \begin{description} 
 This work is licensed under the Creative Commons License Attribution 4.0 International (CC-BY 4.0)   
 https://creativecommons.org/licenses/by/4.0/legalcode 

    \lang{de}{}
    \lang{en}{}
  \end{description}
  \begin{components}
  \end{components}
  \begin{links}
\link{generic_article}{content/rwth/HM1/T211_Eigenschaften_stetiger_Funktionen/g_art_content_33_zwischenwertsatz.meta.xml}{content_33_zwischenwertsatz}
\end{links}
  \creategeneric
\end{metainfo}
\begin{content}
\title{
\lang{de}{Ü01: Minimum-Maximum-Satz}
}
\begin{block}[annotation]
  Im Ticket-System: \href{http://team.mumie.net/issues/10049}{Ticket 10049}
\end{block}
 
Wir betrachten die Funktion $f:[0;1]\to\R$ mit $f(x)=1-x^{2}$. Zeigen Sie, dass $f$ Minimum und Maximum auf $[0;1]$ annimmt und bestimmen Sie diese.


\begin{tabs*}[\initialtab{0}\class{exercise}]

  \tab{
  \lang{de}{Lösung}}
\begin{incremental}[\initialsteps{1}]
	\step  
Die Funktion $f$ ist stetig auf dem abgeschlossenen Intervall $[0;1]$ als Polynom vom Grad 2. Weiter ist $f$ nicht konstant. Nach dem \ref[content_33_zwischenwertsatz][Satz vom Minimum und Maximum]{thm:minmax} gibt es also Zahlen $m,M\in\R$ mit $m<M$, so dass
\[m=\min\{f(x)\,|\,x\in[0;1]\}, \ \quad M=\max\{f(x)\,|\,x\in[0;1]\}\,.\]

\step Wir zeigen, dass $f$ streng monoton fallend ist. Es seien dazu $x,y\in[0;1]$ mit $x<y$. Dann gilt (beachte $y>x \geq 0$)
\[x^{2}=x\cdot x\leq x\cdot y<y\cdot y=y^{2}\iff1-x^{2}>1-y^{2}\iff f(x)>f(y)\,.\]
Also ist $f$ streng monoton fallend und es gilt $f(0)=1=M$ sowie $f(1)=0=m$.
\end{incremental}
\end{tabs*}

\end{content}