%$Id:  $
\documentclass{mumie.article}
%$Id$
\begin{metainfo}
  \name{
    \lang{de}{Exponentialfunktion und Logarithmus}
    \lang{en}{}
  }
  \begin{description} 
 This work is licensed under the Creative Commons License Attribution 4.0 International (CC-BY 4.0)   
 https://creativecommons.org/licenses/by/4.0/legalcode 

    \lang{de}{Beschreibung}
    \lang{en}{}
  \end{description}
  \begin{components}
    \component{generic_image}{content/rwth/HM1/images/g_tkz_T211_ExpLog.meta.xml}{T211_ExpLog}
    \component{generic_image}{content/rwth/HM1/images/g_tkz_T104_Logarithms_A.meta.xml}{T104_Logarithms_A}
    \component{generic_image}{content/rwth/HM1/images/g_tkz_T104_Exponentials_A.meta.xml}{T104_Exponentials_A}
    \component{generic_image}{content/rwth/HM1/images/g_tkz_T209_Exponential.meta.xml}{T209_Exponential}
    \component{generic_image}{content/rwth/HM1/images/g_img_00_video_button_schwarz-blau.meta.xml}{00_video_button_schwarz-blau}
    \component{js_lib}{system/media/mathlets/GWTGenericVisualization.meta.xml}{mathlet1}
  \end{components}
  \begin{links}
    \link{generic_article}{content/rwth/HM1/T211_Eigenschaften_stetiger_Funktionen/g_art_content_34_exp_und_log.meta.xml}{content_34_exp_und_log}
    \link{generic_article}{content/rwth/HM1/T204_Abbildungen_und_Funktionen/g_art_content_12_reelle_funktionen_monotonie.meta.xml}{content_12_reelle_funktionen_monotonie}
    \link{generic_article}{content/rwth/HM1/T205_Konvergenz_von_Folgen/g_art_content_15_monotone_konvergenz.meta.xml}{monot-konv}
    \link{generic_article}{content/rwth/HM1/T210_Stetigkeit/g_art_content_30_elem_funktionen.meta.xml}{elem-funk}
    \link{generic_article}{content/rwth/HM1/T211_Eigenschaften_stetiger_Funktionen/g_art_content_33_zwischenwertsatz.meta.xml}{zwischenwertsatz}
    \link{generic_article}{content/rwth/HM1/T209_Potenzreihen/g_art_content_28_exponentialreihe.meta.xml}{exp}
  \end{links}
  \creategeneric
\end{metainfo}
\begin{content}
\usepackage{mumie.ombplus}
\ombchapter{11}
\ombarticle{2}
\usepackage{mumie.genericvisualization}

\begin{visualizationwrapper}

\lang{de}{\title{Exponentialfunktion und Logarithmus}}
 
\begin{block}[annotation]
  
  
\end{block}
\begin{block}[annotation]
  Im Ticket-System: \href{http://team.mumie.net/issues/9794}{Ticket 9794}\\
\end{block}

\begin{block}[info-box]
\tableofcontents
\end{block}

\section{Exponentialfunktion}

Im Abschnitt \link{exp}{Exponentialreihe} haben wir schon die Exponentialfunktion $\exp$
durch ihre Reihendarstellung kennengelernt:

\[ \exp(z)= \sum_{n=0}^\infty \frac{z^n}{n!} \]
für alle $z\in \C$. Im Folgenden betrachten wir jedoch nur die reelle Exponentialfunktion, d.\,h.
die auf die reellen Zahlen eingeschränkte Funktion
\[ \exp:\R\to \R, \ x\mapsto \sum_{n=0}^\infty \frac{x^n}{n!}. \]

Die Exponentialfunktion erfüllt drei ganz wichtige Eigenschaften, die diese auch eindeutig charakterisieren.
Das heißt, die Exponentialfunktion ist die einzige Funktion, die alle der drei folgenden 
Eigenschaften erfüllt:

\begin{enumerate}
\item Sie ist eine auf ganz $\R$ definierte stetige Funktion,
\item für alle $x,y\in \R$ gilt $\exp(x+y)=\exp(x)\exp(y)$,
\item es ist $\exp(1)=e$, wobei $e$ die \ref[monot-konv][Eulersche Zahl]{sec:eulersche-zahl} $e\approx 2,71828$ bezeichnet.
\end{enumerate}
(Vgl. die Abschnitte \link{exp}{Exponentialreihe} und \ref[elem-funk][Stetigkeit elementarer Funktionen]{sec:potenzreihen}.)

Aus diesen Eigenschaften haben wir noch viele weitere Eigenschaften hergeleitet:

\begin{rule}\label{rule:exp-auf-q}
\begin{enumerate}
\item Für alle rationalen Zahlen $x\in \Q$ ist
\[  \exp(x) = e^x \]
mit der \ref[monot-konv][Eulerschen Zahl]{sec:eulersche-zahl} $e\approx 2,71828$,
weshalb wir auch für reelle Zahlen $x$ oft $e^x$ statt $\exp(x)$ schreiben.
\item Es gilt $\exp(x)>0$ und $\exp(-x)=\frac{1}{\exp(x)}$ bzw. $e^{-x}=\frac{1}{e^x}$ für alle $x\in \R$.
\item Es gilt $\exp(rx)=\exp(x)^r$ bzw. $e^{rx}=(e^x)^r$ für alle $r\in \Q$ und $x\in \R$.
\end{enumerate}
\end{rule}

Die Exponentialfunktion erfüllt jedoch noch viele weitere Eigenschaften.

\begin{rule}
Es gelten weiter:
\begin{enumerate}
\item Die Exponentialfunktion $\exp$ ist streng monoton wachsend.
\item Es gilt $\lim_{x\to \infty} \exp(x)= \infty$.
\item Es gilt $\lim_{x\to -\infty} \exp(x)= 0$.
\end{enumerate}
\end{rule}


\begin{proof*}
\begin{incremental}
\step
Aus der Reihendarstellung sieht man direkt für $h>0$:
\[ \exp(h)= \sum_{n=0}^\infty \frac{h^n}{n!} >1. \]
Es seien nun $x,y\in \R$ mit $x<y$ und $h=y-x>0$. Dann gilt
\[ \exp(y)=\exp(x+h)=\exp(x)\cdot \exp(h)>\exp(x). \]
Also ist die Exponentialfunktion streng monoton wachsend.
\step
Aus der Reihendarstellung sieht man direkt $\exp(x)\geq x$ für alle $x>0$.
Da $\lim_{x\to \infty} x=\infty$, ist somit auch $\lim_{x\to \infty} \exp(x)= \infty$.
\step
Aus der vorigen Aussage folgt
\[ \lim_{x\to -\infty} \exp(x)=\lim_{x\to -\infty} \frac{1}{\exp(-x)}=\lim_{y\to \infty}  \frac{1}{\exp(y)}=0. \]
\end{incremental}
\end{proof*}

\begin{center}
\image{T209_Exponential}
\end{center}

Aus obigen Eigenschaften folgt:

\begin{theorem}\label{thm:eigenschaften_exp}
Die Exponentialfunktion $\exp:\R\to \R$ ist stetig, streng monoton wachsend und bildet $\R$ bijektiv
auf das Intervall $\R_{+}=(0;\infty)$ ab.
\end{theorem}


\begin{proof*}
\begin{incremental}
\step
Es muss nur noch gezeigt werden, dass die Wertemenge von $\exp$ das Intervall $(0;\infty)$ ist. Für jedes abgeschlossene endliche Intervall $[a;b]$ ist nach dem \ref[zwischenwertsatz][Zwischenwertsatz]{thm:zwischenwertsatz} die Bildmenge $\exp([a;b])$ ein Intervall $[m;M]$, wobei
$m$ das Minimum der Funktionswerte und $M$ das Maximum der Funktionswerte ist. Da die Funktion
$\exp$ streng monoton wachsend ist, ist das Minimum genau $f(a)$ und das Maximum genau $f(b)$,
also $\exp([a;b])=[\exp(a);\exp(b)]$.
Damit ist dann aber die Wertemenge von $\exp$ genau die Vereinigung aller dieser Mengen, d.\,h.
$W_{\exp}=(s;S)$ mit
\[ s=\inf W_{\exp}= \lim_{x\to -\infty} \exp(x)=0 \]
und
\[ S=\sup W_{\exp}= \lim_{x\to \infty} \exp(x)=\infty. \]
Also ist $W_{\exp}=(0;\infty)$.
\end{incremental}
\end{proof*}

Für das asymptotische Grenzverhalten gelten sogar stärkere Aussagen:

\begin{theorem}\label{thm:exp_asymptotik}
Für jedes $n\in \Nzero$ ist
\begin{enumerate}
\item $\lim_{x\to \infty} \frac{x^n}{\exp(x)}=0$ (die Exponentialfunktion wächst stärker als jede Potenz),
\item $\lim_{x\to -\infty} x^n\cdot \exp(x)=0$.
\end{enumerate}
\end{theorem}


\begin{proof*}
\begin{incremental}
\step

Der zweite Teil folgt aus dem ersten durch Setzen von $y=-x$ und $\exp(x)=\frac{1}{\exp(y)}$.
Um den ersten Teil einzusehen, bemerkt man aus der Reihendarstellung zunächst, dass für $x\geq 0$ die Abschätzung 
\[ \exp(x)\geq \frac{x^{n+1}}{(n+1)!} \]
gilt. Damit gilt auch
\[  \frac{x^n}{\exp(x)}\leq \frac{(n+1)!}{x} \quad \text{für alle }x\geq 0 \]
und daher
\[ 0\leq \lim_{x\to \infty} \frac{x^n}{\exp(x)} \leq \lim_{x\to \infty}\frac{(n+1)!}{x} =0.\]

\end{incremental}
\end{proof*}


 	\begin{genericGWTVisualization}[550][1000]{mathlet1}
 		\begin{variables}
			\function{f}{real}{exp(x)} 			
 			\function[editable]{g}{real}{x^4}
			\function{h}{real}{var(g)/var(f)}
%			\function{h}{real}{x^5/exp(x)}
 		\end{variables}
 		\color{g}{#0066CC}
 		\color{h}{#00CC00}
 
 		\begin{canvas}
 			\plotSize{400,300}
 			\plotLeft{-0.3}
 			\plotRight{13}
 			\plot[coordinateSystem]{f,g,h}
 		\end{canvas}
 		\text{Die schwarze Kurve ist der Graph der Funktion $f(x)=e^x$, die blaue Kurve der Graph
 		von $g(x)=$\var{g}, und die grüne Kurve der Graph von $h(x)=\frac{g(x)}{f(x)}=\var{h}$.\\
 		Sie können die Funktion $g$ verändern.}
 	    	\end{genericGWTVisualization}


\section{Der natürliche Logarithmus}

Da die Exponentialfunktion $\exp$ eine streng monoton wachsende (und daher injektive) Funktion ist, 
besitzt sie eine (partielle) Umkehrfunktion. Ihr Definitionsbereich ist dann die Wertemenge von $\exp$, 
also $(0;\infty)$, und ihre Wertemenge ist der Definitionsbereich von $\exp$, also $\R$. Außerdem ist auch 
die Umkehrfunktion \ref[content_12_reelle_funktionen_monotonie][wieder streng monoton wachsend]{thm:inversmonoton}.
Diese Umkehrfunktion wird der \emph{natürliche Logarithmus} genannt.

\begin{definition}
Die Umkehrfunktion zu $\exp : \R \to ( 0; \infty )$ heißt \notion{(natürlicher) Logarithmus}
und wird mit $\ln$ (oder manchmal auch $\log$) bezeichnet:

\[  \ln: ( 0; \infty )\to  \R, \ x\mapsto \ln(x). \]
\end{definition}

Das folgende Video geht ausführlich auf die Umkehrfunktion ein und bespricht kurz die Beispiele $f(x)=e^x$ sowie $f(x)=x^2$.
\floatright{\href{https://api.stream24.net/vod/getVideo.php?id=10962-2-10920&mode=iframe&speed=true}{\image[75]{00_video_button_schwarz-blau}}}\\



	\begin{genericGWTVisualization}[550][1000]{mathlet1}
 		\begin{variables}
			\function{f}{real}{exp(x)} 			
 			\function{g}{real}{ln(x)}
			\function{h}{real}{x}
%			\function{h}{real}{x^5/exp(x)}
 		\end{variables}
 		\color{g}{#0066CC}
 		\color{f}{#00CC00}
		\color{h}{LIGHT_GRAY}
 
 		\begin{canvas}
 			\plotSize{300}
 			\plotLeft{-3}
 			\plotRight{13}
 			\plot[coordinateSystem]{f,g,h}
 		\end{canvas}
 		\text{Die grüne Kurve ist der Graph der Funktion \textcolor{green}{$f(x)=e^x$} und die blaue 
 		Kurve der Graph
 		der Umkehrfunktion \textcolor{blue}{$g(x)=\ln(x)$.} Wie stets bei Umkehrfunktionen erhält man 
 		den Graphen von \textcolor{blue}{$\ln(x)$},
 		indem man den Graphen von \textcolor{green}{$e^x$} an der Winkelhalbierenden $y=x$ 
 		(grau eingezeichnet) spiegelt.
 		}
 	    	\end{genericGWTVisualization}

Aus der Definition als Umkehrfunktion folgt direkt (mit der Schreibweise $e^x$ statt $\exp(x)$ und dem 
\ref[content_34_exp_und_log][Satz von der stetigen Umkehrfunktion auf Intervallen]{thm:inverse-stetig}):

\begin{rule}
Es gelten 
\begin{enumerate}
\item $ \ \ln(e^x)=x$ für alle $x\in \R$,
\item $ \ e^{\ln(x)}=x$ für alle $x\in (0;\infty)$.
\item $ \ \ln$ ist eine stetige Funktion.
\end{enumerate}
\end{rule}

Damit erhält man aus den Regeln für die Exponentialfunktion auch viele Eigenschaften des Logarithmus:

\begin{rule}\label{rule:ln}
Der natürliche Logarithmus $ \ln: ( 0; \infty )\to  \R$ besitzt die folgenden Eigenschaften:
\begin{enumerate}
\item $ \ \ln(1)=0$ und $\ln(e)=1$,
\item $ \ \ln(xy)=\ln(x)+\ln(y)$ für alle $x,y>0$,
\item $ \ \ln(\frac{1}{x})=-\ln(x)$ für alle $x>0$,
\item $ \ \ln(x^r)=r\cdot \ln(x)$ für alle $x>0$ und $r\in \Q$,
\item $ \ \lim_{x\to \infty} \ln(x)=\infty$,
\item $ \ \lim_{x \searrow 0} \ln(x)=-\infty$,
\item $ \ \lim_{x\to \infty} \frac{\ln(x)}{\sqrt[n]{x}}=0$ für alle $n\in \N$ (Logarithmus wächst langsamer als jede Potenz/Wurzel von $x$).
\end{enumerate}
\end{rule}

\begin{proof*}
Da die Exponentialfunktion injektiv ist, sind die jeweiligen Seiten der Gleichungen 1.-4. 
genau dann gleich, wenn ihre Bilder unter der Exponentialfunktion gleich sind.
\begin{incremental}
\step
Zu 1. ist 
\[ e^{\ln(1)}=1 =e^0\quad \text{und}\quad e^{\ln(e)}=e=e^1, \]
und daher $\ln(1)=0$ und $\ln(e)=1$.
\step
Zu 2. ist für alle $x,y>0$
\[  e^{\ln(x)+\ln(y)}=e^{\ln(x)}\cdot e^{\ln(y)}=xy=e^{\ln(xy)}.\]
Also gilt $\ln(x)+\ln(y)=\ln(xy)$.
\step
Ebenso folgen $\ln(\frac{1}{x})=-\ln(x)$ und $\ln(x^r)=r\cdot \ln(x)$ aus
\[ e^{-\ln(x)}=\frac{1}{e^{\ln(x)}} =   \frac{1}{x} =e^{\ln(\frac{1}{x})} \]
bzw.
\[  e^{r\cdot \ln(x)}=\left( e^{\ln(x)}\right)^r=x^r=e^{\ln(x^r)}. \]
Da der natürliche Logarithmus stetig und streng monoton steigend mit Wertebereich ganz $\R$ ist (der Definitionsbereich von $\exp$),
folgt direkt $\lim_{x \to \infty} \ln(x)=\infty$ und $\lim_{x \searrow 0} \ln(x)=-\infty$.
\step
Für den letzten Grenzwert bemerken wir zunächst: Ist $x> 0$, so setzen wir $z=\sqrt[2n]{x}>0$ und 
erhalten aus obigen Abschätzungen für die Exponentialfunktion
\[  \sqrt[2n]{x}=z\leq e^z=e^{\sqrt[2n]{x}}. \]
\step
Da der Logaritmus streng monoton wachsend ist, gilt somit
\[ \frac{1}{2n} \ln(x) =  \ln(\sqrt[2n]{x})\leq \ln(e^{\sqrt[2n]{x}})=\sqrt[2n]{x} \quad \text{für alle }x>0. \]
Dies können wir umformen zu 
\[
\frac{\ln(x)}{\sqrt[2n]{x}} \leq 2n . 
\]
Damit gilt
\[ \lim_{x\to \infty} \frac{\ln(x)}{\sqrt[n]{x}}=\lim_{x\to \infty}  \frac{\ln(x)}{(\sqrt[2n]{x})^2}
=\lim_{x\to \infty}  \frac{\ln(x)}{\sqrt[2n]{x}}\cdot  \frac{1}{\sqrt[2n]{x}} =0,\]
da $\frac{\ln(x)}{\sqrt[2n]{x}}$ beschränkt und $\lim_{x\to \infty} \frac{1}{\sqrt[2n]{x}} =0$ ist.

\end{incremental}
\end{proof*}



%  	\begin{genericGWTVisualization}[550][1000]{mathlet1}
%  		\begin{variables}
% 			\function{f}{real}{ln(x)} 			
%  			\function[editable]{g}{rational}{x^(1/4)}
% 			\function{h}{rational}{var(f)/var(g)}
%  		\end{variables}
%  		\color{g}{BLUE}
%  		\color{h}{GREEN}
%  
%  		\begin{canvas}
%  			\plotSize{400,300}
%  			\plotLeft{-0.3}
%  			\plotRight{13}
%  			\plot[coordinateSystem]{f,g,h}
%  		\end{canvas}
%  		\text{Die schwarze Kurve ist der Graph der Funktion $f(x)=\ln(x)$, die blaue Kurve der Graph
%  		von $g(x)=$\var{g}, und die grüne Kurve der Graph von $h(x)=\frac{f(x)}{g(x)}=\var{h}$.\\
%  		Sie können die Funktion $g$ verändern.}
%  	    	\end{genericGWTVisualization}


\section{Exponentialfunktion und Logarithmus zu anderen Basen}

Mit Hilfe der Exponentialfunktion und des Logarithmus können wir auch allgemeine Potenzen $a^b$ mit $a>0$ und $b\in \R$ definieren.

\begin{definition}
Für reelle Zahlen $a,b$ mit $a>0$  definieren wir die Potenz $a^b$ als
\[   a^b=e^{b\ln(a)}=\exp(b\ln(a)). \]
\end{definition}

\begin{remark}
\begin{enumerate}
\item Für rationale Zahlen $b=\frac{p}{q}$ stimmt diese Definition aufgrund der 
\lref{rule:exp-auf-q}{obigen Regeln} mit der üblichen überein, d.\,h. es
ist  \[  a^b =\sqrt[q]{a^p} .\]
\item Durch Anwenden des natürlichen Logarithmus auf die Definition von $a^b$ erhält man für alle 
$a>0$ und $b\in \R$
\[   \ln(a^b)=b\cdot \ln(a). \]
\end{enumerate}
\end{remark}

\begin{definition}[\lang{de}{Exponentialfunktion}\lang{en}{The Exponential Function}]
   Für $a\in \R$ mit $a>0$ heißt die Funktion
     \begin{equation*}
    f:\R\to \R\quad \text{ mit } \ f(x) = a^x
      \end{equation*}
    \notion{Exponentialfunktion zur Basis $a$}.
\end{definition}

Alle Exponentialfunktionen haben die Eigenschaft $f(0)=a^0=1$. Der Verlauf der Exponentialfunktionen
ist von der Basis $a$ abhängig und kann in drei Fälle unterteilt werden, nämlich $0 < a < 1$, $a=1$ und $a > 1$.

Im Fall $a=1$ ist die Funktion nichts anderes als die konstante Funktion $f(x)=1$, ansonsten gilt:

\begin{theorem}\label{thm:andere-basen}
Für $a>1$ ist die Exponentialfunktion $f(x) = a^x$ streng monoton wachsend.\\
Für $0<a<1$ ist die Exponentialfunktion $f(x) = a^x$ streng monoton fallend.

In beiden Fällen ist die Funktion stetig mit Wertemenge $(0;\infty)$ und erfüllt die \emph{Funktionalgleichung}
\[  f(x+y)= f(x)\cdot f(y) \quad \text{für alle }x,y\in \R, \]
d.\,h. $ \ a^{x+y}=a^x\cdot a^y$.
\end{theorem}

\begin{example}
  \begin{table}[\cellaligns{cccccccc}]
%   \lang{de}{
  \head
  $x$ & $\quad-3 \quad$ &$\quad -2  \quad$& $\quad -1 \quad$ &$\quad 0 \quad$ & $\quad 1 \quad$& $\quad 2 \quad$ &$\quad 3 \quad$
  \body
  $2^x$ & $\frac{1}{8}$ & $\frac{1}{4}$ & $\frac{1}{2}$& $1$ & $2$& $4$ & $8$\\
  $\left( \frac{1}{2} \right)^x = 2^{-x}$& $8$ & $4$ & $2$ & $1$ & $\frac{1}{2}$ & $\frac{1}{4}$ & $\frac{1}{8}$ \\
  $3^x$ & $\frac{1}{27}$ & $\frac{1}{9}$ & $\frac{1}{3}$& $1$ & $3$& $9$ & $27$\\
  $\left( \frac{1}{3} \right)^x = 3^{-x}$& $27$ & $9$ & $3$ & $1$ & $\frac{1}{3}$ & $\frac{1}{9}$ & $\frac{1}{27}$\\
   $10^x$& $0,001$& $0,01$ &$0,1$& $1$ & $10$& $100$ & $1000$\\
  $\left( \frac{1}{10} \right)^x = 10^{-x}$ & $1000$ & $100$ & $10$ & $1$ &$0,1$ & $0,01$ & $0,001$
%   }
%   \lang{en}{\head
%   $x$ & $\quad-3 \quad$ &$\quad -2  \quad$& $\quad -1 \quad$ &$\quad 0 \quad$ & $\quad 1 \quad$& $\quad 2 \quad$ &$\quad 3 \quad$
%   \body
%   $2^x$ & $\frac{1}{8}$ & $\frac{1}{4}$ & $\frac{1}{2}$& $1$ & $2$& $4$ & $8$\\
%   $\left( \frac{1}{2} \right)^x = 2^{-x}$& $8$ & $4$ & $2$ & $1$ & $\frac{1}{2}$ & $\frac{1}{4}$ & $\frac{1}{8}$ \\
%   $3^x$ & $\frac{1}{27}$ & $\frac{1}{9}$ & $\frac{1}{3}$& $1$ & $3$& $9$ & $27$\\
%   $\left( \frac{1}{3} \right)^x = 3^{-x}$& $27$ & $9$ & $3$ & $1$ & $\frac{1}{3}$ & $\frac{1}{9}$ & $\frac{1}{27}$\\
%    $10^x$& $0.001$& $0.01$ &$0.1$& $1$ & $10$& $100$ & $1000$\\
%   $\left( \frac{1}{10} \right)^x = 10^{-x}$ & $1000$ & $100$ & $10$ & $1$ &$0.1$ & $0.01$ & $0.001$
%   }
  \end{table}
 \begin{center}
\image{T104_Exponentials_A}
  \end{center}
\end{example}

\begin{quickcheck}
		\field{rational}
		\type{input.number}
		\begin{variables}
			\randint{a1}{1}{4}
			\randint{a2}{5}{8}
			\function[calculate]{a}{a1/a2} 
			\randint[Z]{xp}{-4}{4} 
			\function[calculate]{yp}{a^xp}
		\end{variables}

			\text{Für welches $a>0$ geht der Graph der Exponentialfunktion $f(x)=a^x$ durch den Punkt $P=(\var{xp};\var{yp})$?\\
			Für $a=$\ansref.}

		\begin{answer}
			\solution{a}
		\end{answer}
		\explanation{Wenn der Punkt $P=(\var{xp};\var{yp})$ auf dem Graphen liegt, gilt $f(\var{xp})=\var{yp}$,
		also $a^\var{xp}=\var{yp}$.}
	\end{quickcheck}


Da die Exponentialfunktionen $f(x)=a^x$ für $a>0$ und $a\neq 1$ streng monoton sind, besitzen auch sie Umkehrfunktionen, die allgemeinen \emph{Logarithmusfunktionen}.

\begin{definition}\label{def:Log_arbitrary_base}
Für $a\in \R$ mit $a>0$ und $a\neq 1$ ist der \notion{Logarithmus zur Basis $a$} die Umkehrfunktion  

\[ \log_a: (0;\infty)\to \R, \ x\mapsto \log_a(x) \]
der Exponentialfunktion zur Basis $a$.

 \begin{center}
\image{T211_ExpLog}
  \end{center}


\end{definition}

\begin{remark}
Nach Definition der Umkehrfunktion gilt:
\[  y=\log_a(x)\Leftrightarrow  x=a^y. \]

Außerdem sind die Logarithmusfunktionen $\log_a$ für $a>1$ streng monoton wachsend und für $0<a<1$ streng monoton fallend. Genauer gilt sogar
\[   \log_{\frac{1}{a}}(x)=-\log_a(x) \]
für alle $x\in (0;\infty)$.
\end{remark}

\begin{quickcheck}
    \field{real}
    \type{input.function}
        \begin{variables}
            \function{f}{1/x}
            \function{g}{2}
            \function{h}{ln(3)+x}
        \end{variables}
            \text{Berechnen oder vereinfachen Sie:\\
            $e^{-\ln(x)}=$ \ansref}
        \begin{answer}
            \solution{f}
        \end{answer}
        \explanation{$e^{-\ln(x)}=e^{\ln(x^{-1})}=e^{\ln(\frac{1}{x})}=\frac{1}{x}$,\\
        $\log_{10}(4)+\log_{10}(25)=\log_{10}(4\cdot 25)=\lg(100)=2$,\\
        $\ln(3e^x)=\ln(3)+\ln(e^x)=\ln(3)+x$.}
            \text{$\log_{10}(4)+\log_{10}(25)=$ \ansref}
        \begin{answer}
            \solution{g}
        \end{answer}
            \text{$\ln(3e^x)=$ \ansref }
        \begin{answer}
            \solution{h}
        \end{answer}
\end{quickcheck}

\begin{example}
Neben dem natürlichen Logarithmus, welcher auch der Logarithmus zur Basis $e$ ist, werden häufig auch
der \emph{Zehnerlogarithmus} $\log_{10}$, auch \emph{dekadischer Logarithmus} genannt,  und der
\emph{Zweierlogarithmus} (\emph{binäre Logarithmus}) $\log_2$ verwendet.

Statt $\log_{10}$ wird oft auch $\lg$ geschrieben und statt $\log_2$ oft $\text{lb}$.


\begin{center}
\image{T104_Logarithms_A}
\end{center}
\end{example}

\begin{rule}
Alle Logarithmusfunktionen unterscheiden sich nur um ein skalares Vielfaches.
Genauer gilt für $a,b\in \R$ mit $a,b>0$ und $a,b\neq 1$:
\[   \log_b(x)= \frac{\log_a(x)}{\log_a(b)} \quad \text{für alle }x\in (0;\infty). \]
\end{rule}

\begin{proof*}
\begin{showhide}

Es ist $y= \log_b(x)$ genau dann, wenn $x=b^y$. Setzen wir dies in die rechte Seite ein, ergibt sich
\[ \frac{\log_a(b^y)}{\log_a(b)}= \frac{y\cdot \log_a(b)}{\log_a(b)}=y =\log_b(x). \]
\end{showhide}
\end{proof*}

Da sich alle Logarithmusfunktionen nur um ein skalares Vielfaches unterscheiden, wird in der Praxis immer nur mit einer
Logarithmusfunktion gerechnet, und zwar mit derjenigen, die für die gegebenen Zwecke am praktischsten ist.
Je nach Anwendungsgebiet ist dies der natürliche Logarithmus $\ln$, der dekadische Logarithmus 
$\log_{10}=\lg$
oder der binäre Logarithmus $\log_2=\text{lb}$.\\
Oft wird dann für den verwendeten Logarithmus auch $\log$ geschrieben, ohne die Basis explizit zu nennen.

\end{visualizationwrapper}

\end{content}