%$Id:  $
\documentclass{mumie.article}
%$Id$
\begin{metainfo}
  \name{
    \lang{de}{Integrierbarkeit}
    \lang{en}{Integrability}
  }
  \begin{description} 
 This work is licensed under the Creative Commons License Attribution 4.0 International (CC-BY 4.0)   
 https://creativecommons.org/licenses/by/4.0/legalcode 

    \lang{de}{Beschreibung}
    \lang{en}{Description}
  \end{description}
  \begin{components}
    \component{generic_image}{content/rwth/HM1/images/g_tkz_T304_Integral_E.meta.xml}{T303_Integral_E}
    \component{generic_image}{content/rwth/HM1/images/g_tkz_T304_Integral_D.meta.xml}{T303_Integral_D}
    \component{generic_image}{content/rwth/HM1/images/g_tkz_T304_Integral_C.meta.xml}{T303_Integral_C}
    \component{generic_image}{content/rwth/HM1/images/g_img_00_Videobutton_schwarz.meta.xml}{00_Videobutton_schwarz}
    \component{generic_image}{content/rwth/HM1/images/g_img_00_video_button_schwarz-blau.meta.xml}{00_video_button_schwarz-blau}
  \end{components}
  \begin{links}
    \link{generic_article}{content/rwth/HM1/T304_Integrierbarkeit/g_art_content_09_integrierbare_funktionen.meta.xml}{link2}
    \link{generic_article}{content/rwth/HM1/T107_Integralrechnung/g_art_content_24_integral_als_flaeche.meta.xml}{teil1}
    \link{generic_article}{content/rwth/HM1/T204_Abbildungen_und_Funktionen/g_art_content_12_reelle_funktionen_monotonie.meta.xml}{reelle-funk}
    \link{generic_article}{content/rwth/HM1/T304_Integrierbarkeit/g_art_content_07_ober_und_untersumme.meta.xml}{ober-und-unter}
    \link{generic_article}{content/rwth/HM1/T304_Integrierbarkeit/g_art_content_10_uneigentliches_integral.meta.xml}{uneigentlich-int}
  \end{links}
  \creategeneric
\end{metainfo}
\begin{content}
\usepackage{mumie.ombplus}
\ombchapter{3}
\ombarticle{2}


\title{\lang{de}{Integrierbarkeit}\lang{en}{Integrability}}
 
\begin{block}[annotation]
  
  
\end{block}
\begin{block}[annotation]
  Im Ticket-System: \href{http://team.mumie.net/issues/10037}{Ticket 10037}\\
\end{block}

\begin{block}[info-box]
\tableofcontents
\end{block}


\lang{de}{
In diesem Abschnitt verwenden wir die Ober-, Unter- und Zwischensummen aus dem 
\link{ober-und-unter}{vorigen Abschnitt}, um Integrierbarkeit und Integrale zu definieren und zu 
berechnen.
\\\\
Grundsätzlich existieren Integrale nur für Funktionen, die auf einem endlichen Intervall $[a;b]$ 
definiert sind und dort auch beschränkt sind.
\\\\
Wir werden später allerdings auch noch \textit{uneigentliche Integrale} kennenlernen, bei denen die 
Funktion z.\,B. am Rand nicht definiert sein muss. Diese werden wir aus den hier definierten 
Integralen durch Grenzwertbildung erhalten.
}
\lang{en}{
In this section we use the upper sum, lower sum and Riemann sum definitions from the 
\link{ober-und-unter}{previous section} to define and determine integrability and integrals.
\\\\
The integrals that are introduced here are for functions that are defined and bounded on a finite 
interval $[a;b]$.
\\\\
However, in later sections we also introduce definitions for \textit{improper integrals}, for which 
the interval may not be finite. To introduce those, we will require the definitions we provide in 
this section.
}


\section{\lang{de}{Integrierbarkeit}\lang{en}{Integrability}}\label{sec:integrierbarkeit}


\begin{definition}
\label{def:integral}
%\emph{Definition:} 
\lang{de}{
Es sei $f(x)$ eine auf dem Intervall $[a;b]$ beschränkte Funktion.
% $f$ heißt (Riemann-) integrierbar auf dem Intervall $[a;b]$, 
%falls es eine Folge von Zerlegungen $Z_n$ von $[a;b]$ gibt, deren Feinheit
%(maximale Breite der Teilintervalle) nach $0$ konvergiert für $n \rightarrow \infty$
%und deren Obersummen und Untersummen gegen einen gemeinsamen Grenzwert (das Integral) konvergieren:
%\[ \lim_{n\rightarrow \infty} U(Z_n) = \lim_{n\rightarrow \infty} O(Z_n) = \int_a^b f(x)\, dx \]
$f$ heißt \notion{integrierbar} auf dem Intervall $[a;b]$, falls sich Untersumme und Obersumme
durch Verfeinerung der Zerlegung beliebig genau annähern, d.\,h. falls es zu jeder
vorgegebenen Zahl $\epsilon>0$ (und sei sie noch so klein) eine Zerlegung $Z_\epsilon$ des
Intervalls $[a;b]$ gibt, sodass
}
\lang{en}{
Let $f(x)$ be a function that is defined and bounded on an interval $[a;b]$. 
We say that $f$ is \notion{integrable} on the interval $[a;b]$ if the lower sum and the upper sum 
tend towards the same limit as the partition is refined. That is to say, if for arbitrarily small 
$\epsilon>0$ there exists a partition $Z_\epsilon$ of the interval $[a;b]$ such that
}
\[ 0 \leq O(Z_\epsilon) - U(Z_\epsilon) \leq \epsilon \]
\lang{de}{
gilt. \\
Ist $f$ integrierbar auf $[a;b]$, so gibt es eine eindeutige reelle Zahl
$\big\int_a^b f(x)\, dx$, die als \notion{Integral von $f$ von $a$ nach $b$}
bezeichnet wird, sodass
}
\lang{en}{
holds. \\
If $f$ is integrable on $[a;b]$, then there exists a unique real number $\big\int_a^b f(x)\, dx$ 
called the \notion{integral of $f$ from $a$ to $b$} such that
}
\[ U(Z) \leq \int_a^b f(x)\, dx \leq O(Z) \]
\lang{de}{
für alle Zerlegungen $Z$ des Intervalls $[a;b]$ in endlich viele Teilintervalle gilt.
}
\lang{en}{
for all partitions $Z$ of the interval $[a;b]$ into finitely many subintervals.
}
\end{definition}
\lang{de}{
Die gesamte (bisherige) Einführung des allgemeinen Integralbegriffs kann auch noch einmal hier angesehen werden:
\floatright{\href{https://api.stream24.net/vod/getVideo.php?id=10962-2-10813&mode=iframe&speed=true}{\image[75]{00_video_button_schwarz-blau}}}\\
\\
}
\lang{en}{}
%Das Integral kann auch alternativ mit Hilfe der Zwischensummen eingeführt werden.
%Das Vorgehen ist sehr ähnlich und äquivalent. Dies wird in diesen Videos präsentiert:
%\center{\href{https://www.hm-kompakt.de/video?watch=601}{\image[75]{00_Videobutton_schwarz}}
%\href{https://www.hm-kompakt.de/video?watch=603}{\image[75]{00_Videobutton_schwarz}}
%\href{https://www.hm-kompakt.de/video?watch=605}{\image[75]{00_Videobutton_schwarz}}
%}\\\\


% \begin{block}[notation]

% \emph{\lang{de}{Schreibweise des bestimmten Integrals}\lang{en}{Notation for definite integrals}}\\\\
% \lang{de}{Für das Integral der Funktion $f(x)$ in den Grenzen von $a$ bis $b$ schreibt man }
% \lang{en}{The integral of the function $f(x)$ from $a$ to $b$ is written}
% \[ \int_a^b f(x)\, dx \,  . \]
% \lang{de}{$a$ und $b$ sind die \notion{Integrationsgrenzen}, 
% $x$ ist die \notion{Integrationsvariable}, 
% $f(x)$ der \notion{Integrand} und $dx$ das \notion{Differential}.}
% \lang{en}{The parameters $a$ and $b$ are called the \textit{limits} or \emph{bounds} of integration where $a$ is the lower bound and $b$ is the upper bound, 
% $x$ is the \textit{integration variable},
% $f(x)$ is called the \textit{integrand}, and $dx$ is called the \textit{differential}.}

% \end{block}



\lang{de}{
Das Integral kann also mit Hilfe von Obersummen und Untersummen definiert werden. 
Zur Berechnung des Integrals einer integrierbaren Funktion $f$ lässt man 
die Breite der Teilintervalle immer kleiner werden. Schließlich ergibt sich dann
der gesuchte Integralwert.
%Im Grenzwert konvergieren Untersummen und Obersummen dann gegen den gesuchten Integralwert.
Da es weitere Integralbegriffe gibt, spricht man hierbei 
vom \emph{Riemann-Integral}.}
\lang{en}{
The integral can thus be defined using lower and upper sums.
In order to calculate the integral of an integrable function $f$ we simply progressively smaller 
subintervals. 
As the subinterval size goes to zero, the upper and lower sums converge to the value of the 
integral. 
As there are other formal definitions of integrals, we call this particular construction the 
\emph{Riemann integral}.
}



\begin{example}
\lang{de}{
Im \ref[teil1][ersten Teil]{ex:integral-einfuehrung} haben wir das Integral als orientierten 
Flächeninhalt kennengelernt. Dort haben wir für die Funktion $f(x)=2x+1$ das Integral 
$\int_{-1}^2 f(x) dx$ in einem \ref[teil1][Beispiel]{ex:integral-dreieck} geometrisch bestimmt. 
Zur Erinnerung ist der Flächeninhalt eines Dreiecks 
\nowrap{$\frac{1}{2}$ $\cdot$ Breite $\cdot$ Höhe.}
Wir haben das Integral als die Summe von zwei Dreiecksflächen mit unterschiedlichem Vorzeichen 
berechnet: Ein kleines Dreieck zwischen $x=-1$ und $x=-0,5$ und ein größeres Dreieck zwischen 
$x=-0,5$ und $x=2$. Das Integral $\int_{-1}^2 (2x+1)\, dx $ haben wir als 
}
\lang{en}{
In the \ref[teil1][first part of the course]{ex:integral-einfuehrung} we introduced intervals as 
signed areas bounded by a function graph and the $x$-axis. In an 
\ref[teil1][example]{ex:integral-dreieck}, 
we geometrically determined the interval $\int_{-1}^2 f(x) dx$ of the function $f(x)=2x+1$. 
Recall that the area of a triangle is given by \nowrap{$\frac{1}{2}$ $\cdot$ width $\cdot$ height.} 
We calculated the integral as the sum of the areas of two triangles with different signs: a small 
triangle between $x=-1$ and $x=-0.5$, and a larger triangle between $x=-0.5$ and $x=2$. The integral 
$\int_{-1}^2 (2x+1)\, dx $ was determined to be
}
\[ \frac{1}{2} \big(0{,}5 \cdot (-1) \big) + \frac{1}{2}\big(2{,}5 \cdot 5 \big) = 6  \]
\lang{de}{bestimmt.}
\lang{en}{in this example.}

\begin{center}
\image{T303_Integral_C}
\end{center}

\lang{de}{
Stimmt jedoch der geometrische Integralbegriff aus dem ersten Teil des Kurses mit unserer Definition 
\ref{def:integral} überein? Dies wollen wir nun rechnerisch bestätigen. Bei einer äquidistanten 
Zerlegung $Z_n$ des Intervalls $[-1;2]$ in $n$ Teilintervalle sind die Randstellen der 
Teilintervalle 
}
\lang{en}{
Does our definition \ref{def:integral} correspond to the geometric idea behind the interval from the 
first part of the course? We will now confirm this computationally. For a partition $Z_n$ of the 
interval $[-1;2]$ into $n$ equal subintervals, the endpoints of the intervals are
}
\[  x_k=-1 +k\cdot \frac{3}{n}= \frac{3k-n}{n}\quad \text{ \lang{de}{für}\lang{en}{for} }k=0,\ldots, n \] 
\lang{de}{
und die Intervalllänge jeweils $\frac{3}{n}$.
\\\\
Da die Funktion monoton wachsend ist, werden die Suprema der Funktionswerte stets am rechten 
Intervallrand angenommen und die Infima stets am linken. Man erhält also als Unter- und Obersummen
}
\lang{en}{
and the length of each subinterval $\frac{3}{n}$.
\\\\
As the function is monotone increasing, the function takes the suprema of its image on each 
subinterval at the right end of the interval, and the infima at the left end. Hence the lower sums 
are
}
\begin{eqnarray*}
U(Z_n) &=& \sum_{k=1}^n f(x_{k-1})\cdot \frac{3}{n}=\left( \sum_{k=1}^n (2\cdot \frac{3(k-1)-n}{n}+1)\right)\cdot \frac{3}{n} \\
 &=& \frac{3}{n^2}\cdot \sum_{k=1}^n  (6k-6-2n+n) = \frac{3}{n^2}\cdot \left( 6\frac{n(n+1)}{2} - 6n-n^2 \right) \\
 &=& \frac{3}{n^2}\cdot (3n^2+3n-6n-n^2)=6-\frac{9}{n},
\end{eqnarray*}
\lang{de}{
wobei wir die Gaußsche Summenformel $\sum_{k=1}^n k=\frac{n(n+1)}{2}$ ausgenutzt haben. Für die 
Obersumme gilt entsprechend
}
\lang{en}{
where we use the Gaussian sum formula $\sum_{k=1}^n k=\frac{n(n+1)}{2}$. The upper sums are
}
\begin{eqnarray*}
O(Z_n) &=& \sum_{k=1}^n f(x_{k})\cdot \frac{3}{n}=\left( \sum_{k=1}^n (2\cdot \frac{3k-n}{n}+1)\right)\cdot \frac{3}{n} \\
 &=& \frac{3}{n^2}\cdot \sum_{k=1}^n  (6k-2n+n) = \frac{3}{n^2}\cdot \left( 6\frac{n(n+1)}{2} -n^2 \right) \\
 &=& \frac{3}{n^2}\cdot (3n^2+3n-n^2)=6+\frac{9}{n}.
\end{eqnarray*}

\lang{de}{
Die Differenz der beiden Summen beträgt $O(Z_n)-U(Z_n)=\frac{18}{n}$.
Ist also $\epsilon>0$ beliebig, so kann man eine natürliche Zahl $n>\frac{18}{\epsilon}$ wählen, und 
erhält für die Differenz der Ober- und Untersumme zur Zerlegung $Z_n$, dass
}
\lang{en}{
The difference of the two sums is $O(Z_n)-U(Z_n)=\frac{18}{n}$. 
Therefore given arbitrary $\epsilon>0$, we may choose any natural number $n>\frac{18}{\epsilon}$ 
such that
}
\[ 0 \leq O(Z_n) - U(Z_n) = \frac{18}{n} < \epsilon \]
\lang{de}{
gilt. Also ist die Funktion auf dem Intervall $[-1;2]$ integrierbar. 
\\\\
Wegen $\lim_{n\to \infty} O(Z_n)=\lim_{n\to \infty} 6+\frac{9}{n}=6$ und
$\lim_{n\to \infty} U(Z_n)=\lim_{n\to \infty} 6-\frac{9}{n}=6$ ist die einzige Zahl $I$ mit
$U(Z_n)\leq I \leq O(Z_n)$ für alle $n$ die Zahl $I=6$.
\\\\
Also gilt:
}
\lang{en}{
holds. Hence the function is integrable on the interval $[-1;2]$.
\\\\
As $\lim_{n\to \infty} O(Z_n)=\lim_{n\to \infty} 6+\frac{9}{n}=6$ and
$\lim_{n\to \infty} U(Z_n)=\lim_{n\to \infty} 6-\frac{9}{n}=6$, the only number $I$ with 
$U(Z_n)\leq I \leq O(Z_n)$ for all $n$ is $I=6$.
\\\\
Hence:
}
\[ \int_{-1}^2 (2x+1)\, dx = 6. \]
\end{example}


\begin{example}
\begin{tabs*}[\initialtab{0}]
\tab{$\int_0^2 (x^2+1)\, dx$}
%\emph{Beispiel: } 
\lang{de}{
Wir betrachten wieder die Funktion $f(x)=x^2+1$ auf dem Intervall $[0;2]$ 
(siehe Beispiel \ref[ober-und-unter][im vorigen Abschnitt]{zweizerl}). 
\\\\
Für die äquidistante Zerlegung $Z_n$ in $n$ Teilintervalle erhält man wie im vorigen Beispiel 
die Randstellen der Teilintervalle 
}
\lang{en}{
We once again consider the function $f(x)=x^2+1$ on the interval $[0;2]$ 
(see example \ref[ober-und-unter][in the previous section]{zweizerl}).
\\\\
For the partition $X_n$ into $n$ equal subintervals we obtain as the endpoints of the subintervals
}
\[  x_k=0 +k\cdot \frac{2}{n}= \frac{2k}{n}\quad \text{ \lang{de}{für}\lang{en}{for} }k=0,\ldots, n \] 
\lang{de}{
und die Intervalllängen $\frac{2}{n}$. Der minimale Funktionswert liegt jeweils am linken Rand jedes 
Teilintervalls und der maximale Funktionswert am rechten Rand. Um Ober- und Untersumme in diesem 
Beispiel auszurechnen, brauchen wir außerdem noch die Formel 
}
\lang{en}{
and each subinterval has length $\frac{2}{n}$. The minimum function value is always on the left end 
of each subinterval and the maximum function value on the right end. To calculate the upper and 
lower sums for this example, we once more employ the formula
}
\[ \sum_{k=1}^n \ k^2 = \frac{n(n+1)(2n+1)}{6},\]
\lang{de}{
welche mit vollständiger Induktion leicht bewiesen werden kann. (Dass wir eine derartige 
Summenformel brauchen, deutet bereits an, dass man Integrale mit diesem Ansatz nur schwer bestimmen 
kann.)
\\\\
Die Obersumme und Untersumme sind nun
}
\lang{en}{
which can be easily proven by induction. (The need for such a formula points towards this approach 
to computing integrals often being a difficult one.)
\\\\
The upper sums are
}
\begin{eqnarray*}
O(Z_n) &=& \sum_{k=1}^n f(x_{k})\cdot \frac{2}{n}=\frac{2}{n}\cdot \sum_{k=1}^n \left((\frac{2k}{n})^2+1\right) \\
&=&  \frac{2}{n}\cdot \sum_{k=1}^n \frac{4k^2}{n^2} + \frac{2}{n}\cdot n \\
&=& \frac{8}{n^3}\cdot \sum_{k=1}^n k^2 + 2 \\
&=& \frac{8}{n^3}\cdot \frac{n(n+1)(2n+1)}{6} + 2 =2+ \frac{8}{6}\cdot (1+\frac{1}{n})(2+\frac{1}{n})
\end{eqnarray*}
\lang{de}{bzw. entsprechend}
\lang{en}{and the lower sums are}
\begin{eqnarray*}
U(Z_n) &=& \sum_{k=1}^n f(x_{k-1})\cdot \frac{2}{n}\\
&=& \frac{8}{n^3}\cdot \frac{(n-1)n(2n-1)}{6} + 2 =2+ \frac{8}{6}\cdot (1-\frac{1}{n})(2-\frac{1}{n}) .
\end{eqnarray*}
\lang{de}{Für die Grenzwerte für $n\to \infty$ gilt dann:}
\lang{en}{The limits as $n\to \infty$ are then}
\begin{eqnarray*}
\lim_{n\to \infty} O(Z_n)&=&  \lim_{n\to \infty} \big( 2+ \frac{8}{6}\cdot (1+\frac{1}{n})(2+\frac{1}{n}) \big) =2+\frac{8}{6}\cdot 2=\frac{14}{3}, \\
\lim_{n\to \infty} U(Z_n)&=&  \lim_{n\to \infty} \big( 2+ \frac{8}{6}\cdot (1-\frac{1}{n})(2-\frac{1}{n}) \big) =2+\frac{8}{6}\cdot 2=\frac{14}{3}.
\end{eqnarray*}
\lang{de}{
Der Integralwert ist also $\frac{14}{3}$, d.\,h. 
$\big\int_0^2 (x^2+1)\, dx = \frac{14}{3} \approx 4{,}67$.
}
\lang{en}{
The value of the integral is therefore $\frac{14}{3}$, that is, 
$\big\int_0^2 (x^2+1)\, dx = \frac{14}{3} \approx 4{.}67$.
}
\end{tabs*}
\end{example}



\begin{block}[warning]
\lang{de}{
Über Definitionslücken darf nicht hinweg integriert werden und auch 
Integrale über unbeschränkte Intervalle wie z.\,B. $[0;\infty)$ 
oder Integrale unbeschränkter Funktionen sind 
zunächst nicht zugelassen.
In diesen Fällen kann der Flächeninhalt unendlich groß
oder unbestimmt sein. Manchmal existiert dann ein \emph{uneigentliches Integral}, das wir aber erst \link{uneigentlich-int}{später}
behandeln.
}
\lang{en}{
Gaps in the domain of a function cannot be integrated over; unbounded
intervals such as $[0,\infty)$ cannot be integrated over, and unbounded functions
also cannot be integrated. 
Attempting to integrate in such cases can end with areas being infinitely large or simply undefined. 
Sometimes the so-called \emph{improper integral} exists, but this is dealt with link{uneigentlich-int}{later}.
}
\end{block}

\lang{de}{
Die Warnung haben wir bereits im ersten Teil ausgesprochen und ein 
\ref[teil1][Beispiel]{ex:integral-def-luecke}
angegeben. Wir schauen uns das Beispiel noch einmal genauer an.
}
\lang{en}{
We already gave this warning in the first part of the course, and gave an 
\ref[teil1][example]{ex:integral-def-luecke}. 
We now consider this example more closely.
}

\begin{example}
%\emph {Beispiel:} 
\lang{de}{Wir betrachten die Funktion}
\lang{en}{We consider the function}
\[
f: \R \to \R, \ x \mapsto \begin{cases}\frac{1}{x^2}, & \text{\lang{de}{falls}\lang{en}{if} } x \neq 0, \\ 0, & \text{\lang{de}{falls}\lang{en}{if} } x = 0.\end{cases}
\]
\lang{de}{
Die Funktion $f$ hat zwar keine Definitionslücke, sie ist aber weder stetig noch beschränkt.
Auf abgeschlossenen Intervallen, die die Null nicht enthalten, ist $f$ beschränkt.
Die Integrale $\big\int_{-3}^{-1} f(x)\, dx$ und $\big\int_2^4 f(x) \, dx$ 
existieren also, da $f$
auf $[-3;-1]$ und auch auf $[2;4]$ beschränkt ist. Die Integrale
$\big\int_{-1}^1 f(x)\, dx$ oder $\big\int_0^1 f(x)\, dx$ existieren aber nicht, 
da in jeder Umgebung um $x=0$ die Funktion $f$ nicht beschränkt ist.
Genauso existiert auch $\big\int_0^1 \frac{1}{x^2}\, dx$ nicht. Hier können wir zusätzlich anführen,
dass $\frac{1}{x^2}$ in $x=0$ gar nicht definiert ist.
}
\lang{en}{
The function $f$ may not have any gaps in its definition, but it is neither continuous nor bounded. 
On a closed interval not containing zero, $f$ is bounded. Thus the integrals 
$\big\int_{-3}^{-1} f(x)\, dx$ and $\big\int_2^4 f(x) \, dx$ exist, as $f$ is bounded on both 
$[-3;-1]$ and $[2;4]$. The integrals $\big\int_{-1}^1 f(x)\, dx$ and $\big\int_0^1 f(x)\, dx$ do not exist however, as the function $f$ is not bounded on any neighbourhood of $x=0$. Likewise,  $\big\int_0^1 \frac{1}{x^2}\, dx$ does not exist. Here we can use that $\frac{1}{x^2}$ is 
not even defined at $x=0$.
}
\end{example}


\lang{de}{
Wir schließen den Paragraphen mit einem Beispiel für eine Funktion, die nicht integrierbar ist.
}
\lang{en}{
We give an example of a function that is not integrable at all.
}

\begin{example}\label{ex:dirichlet-nicht-intbar}
\lang{de}{
Wir betrachten die \ref[reelle-funk][Dirichletsche Sprungfunktion]{ex:unstetige-funktionen}
}
\lang{en}{
We consider the \ref[reelle-funk][Dirichlet function]{ex:unstetige-funktionen}
}
\[  D:\R\to {\{0;1\}}, \ x \mapsto \begin{cases} 1, & \text{\lang{de}{falls}\lang{en}{if} }x\in \Q, \\
0, & \text{\lang{de}{falls}\lang{en}{if} }x\notin \Q, \end{cases}\]
\lang{de}{
auf dem Intervall $[0;1]$.
Wählt man eine beliebige Zerlegung $Z$ des Intervalls mit Teilpunkten $a=x_0<x_1<\ldots < x_n=b$,
so ist das Infimum $m_k$ auf jedem Teilintervall $0$, da jedes Teilintervall  eine irrationale Zahl 
enthält (sogar unendlich viele).
Andererseits enthält jedes Teilintervall  auch unendlich viele rationale Zahlen, weshalb das 
Supremum $M_k$ auf jedem Teilintervall $1$ ist. Als Unter- und Obersummen hat man also
}
\lang{en}{
on the interval $[0;1]$.
If we choose an arbitrary partition $Z$ of subintervals with endpoints $a=x_0<x_1<\ldots < x_n=b$,
then the infimum $m_k$ on each subinterval is $0$, as each interval contains an irrational number 
(in fact, infinitely many). Each subinterval also contains infinitely many rational numbers, so the 
supremum $M_k$ on each subinterval is $1$. The lower and upper sums are therefore, respectively, 
}
\begin{eqnarray*}
 U(Z)&=& \sum_{k=1}^n 0\cdot \Delta x_k= 0 \quad \text{\lang{de}{und}\lang{en}{and}} \\
O(Z) &=& \sum_{k=1}^n 1\cdot \Delta x_k=\sum_{k=1}^n (x_k-x_{k-1})=x_n-x_0=1-0=1.
\end{eqnarray*} 
\lang{de}{
Unabhängig von der Zerlegung sind also die Untersummen gleich $0$ und die Obersummen gleich $1$.
Insbesondere ist die Differenz $O(Z)-U(Z)=1$ und kann nicht beliebig klein werden.
\\\\
Die Dirichletsche Sprungfunktion ist also nicht (Riemann-)integrierbar.
}
\lang{en}{
Independently of the partition, the lower sums are all equal to $0$, and the upper sums 
are all equal to $1$. In particular, the difference between them is $O(Z)-U(Z)=1$, and cannot be 
made arbitrarily small.
\\\\
The Dirichlet function is therefore not (Riemann) integrable.
}
\end{example}

\section{\lang{de}{Eigenschaften des Integrals}
         \lang{en}{Properties of integrals}}\label{eigenschaften}

\lang{de}{Das Integral ist additiv bezüglich des Integrationsintervalls:}
\lang{en}{Integrals are additive on their integration intervals:}

\begin{rule}\label{rule:intervall}
\lang{de}{
Für reelle Zahlen $a<b<c$ und eine auf dem Intervall $[a;c]$ integrierbare Funktion $f(x)$ gilt:
}
\lang{en}{
For all real numbers $a<b<c$ and all functions $f(x)$ integrable on $[a,c]$ we have:
}

\[ \int_a^b f(x)\, dx + \int_b^c f(x)\, dx = \int_a^c f(x)\, dx \,  . \]
\end{rule}

\begin{remark}
\lang{de}{
Für die Integrationsgrenzen $a$ und $b$ gilt üblicherweise $a<b$. 
Man betrachtet aber auch die Fälle $a=b$ und $a>b$:
\\\\
Im Fall $a=b$ setzt man $\big\int_a^a f(x)\, dx = 0$, und für $a>b$ setzt man
}
\lang{en}{
For the limits of integration $a$ and $b$ we usually assume $a<b$. Let us consider the
cases $a=b$ and $a>b$:
\\\\
In the case $a=b$ we set $\big\int_a^a f(x)\, dx = 0$, and if $a>b$ we set
}
\[ \int_a^b f(x)\, dx = - \int_b^a f(x)\, dx \, . \]
\lang{de}{
Ein Grund für diese Definition ist zum einen die Anschauung, dass bei Integration von rechts nach 
links die Orientierung und daher das Vorzeichen des Integrals umgekehrt ist, zum anderen aber gilt 
mit dieser Definition die Additivität des Integrals auch für Grenzen $a,b,c$, die nicht 
notwendigerweise angeordnet sind:
}
\lang{en}{
A reason for this definition is the idea that integrating from right to left reverses the 
orientation of the process, and hence changes the sign in front of the interval. Moreover, the 
property of additivity of integrals holds for limits $a,b,c$ that are not necessarily ordered:
}
\[ \int_a^b f(x)\, dx + \int_b^c f(x)\, dx = \int_a^c f(x)\, dx \,  . \]
\lang{de}{Insbesondere zum Beispiel, wenn $a=c$ ist, erhält man}
\lang{en}{In particular, if $a=c$, we obtain}

\[ \int_a^b f(x)\, dx  + \int_b^a f(x)\, dx= 0 = \int_a^a f(x)\, dx  .\]
\lang{de}{
\floatright{\href{https://www.hm-kompakt.de/video?watch=612}{\image[75]{00_Videobutton_schwarz}}}\\~
}
\lang{en}{}
\end{remark}



\lang{de}{
Neben der Additivität bezüglich der Intervallgrenzen verhält sich das Integral auch gut bezüglich 
der Addition und Skalierung von Funktionen:
}
\lang{en}{
Besides additivity in the sense of integral limits, integrals have nice properties in the sense 
of adding and scaling functions:
}

\begin{rule} \label{Linear_Integral}
\lang{de}{
Es seien $f$ und $g$ zwei auf dem Intervall $[a;b]$ integrierbare Funktionen und $r\in \R$ eine 
reelle Zahl. Dann sind auch die Summe $f+g$, die Differenz $f-g$ und das Vielfache $r\cdot f$ wieder 
auf $[a;b]$ integrierbar, und es gelten
}
\lang{en}{
Let $f$ and $g$ be two functions that are integrable on the interval $[a;b]$ and $r\in \R$ a real 
number. Then the sum $f+g$, the difference $f-g$ and the scaled function $r\cdot f$ are all also 
integrable on $[a;b]$, and we have
}
\begin{eqnarray*}
 \int_a^b (f+g)(x)\, dx &=& \int_a^b f(x)\, dx + \int_a^b g(x)\, dx, \\
 \int_a^b (f- g)(x)\, dx &=& \int_a^b f(x)\, dx  - \int_a^b g(x)\, dx, \\
  \int_a^b (rf)(x)\, dx &=& r\cdot  \int_a^b f(x)\, dx.
\end{eqnarray*}
\lang{de}{
\floatright{\href{https://www.hm-kompakt.de/video?watch=614}{\image[75]{00_Videobutton_schwarz}}}\\\\
}
\lang{en}{}
\end{rule}


\begin{example}
\lang{de}{
Betrachten wir die Funktion $h(x)=2x+\cos(x)$ auf dem Intervall $[0;\pi]$. 
Um das Integral $\int_0^{\pi} h(x)\, dx$ zu bestimmen, zerlegen wir $h$ als $h(x)=f(x)+g(x)$ mit
$f(x)=2x$ und $g(x)=\cos(x)$.
}
\lang{en}{
Consider the function $h(x)=2x+\cos(x)$ on the interval $[0;\pi]$. 
In order to determine the value of the integral $\int_0^{\pi} h(x)\, dx$, we write $h$ as 
$h(x)=f(x)+g(x)$ with $f(x)=2x$ and $g(x)=\cos(x)$.
}

\begin{center}
\image{T303_Integral_D}
\end{center}

\lang{de}{
Der Graph von $f$ ist eine Gerade durch den Ursprung und schließt mit der $x$-Achse im Bereich 
$[0;\pi]$ ein Dreieck ein. Die Grundseite des Dreiecks ist $\pi$ und die Höhe ist $f(\pi)=2\pi$. Der 
Flächeninhalt (und damit das Integral) beträgt daher
}
\lang{en}{
The graph of $f$ is a line through the origin, and with the $x$-axis, bounds a triangle in the 
interval $[0;\pi]$. The bottom side of the triangle has length $\pi$, and the height of the triangle 
is $f(\pi)=2\pi$. The area of the triangle (and hence the value of the integral) is hence
}
\[  \int_0^{\pi} f(x)\, dx =\frac{1}{2}\pi\cdot 2\pi=\pi^2.\]

\lang{de}{Anhand der Eigenschaften von $g(x)=\cos(x)$ (oder auch anhand der Skizze) erkennen wir}
\lang{en}{Making use of the properties of $g(x)=\cos(x)$ (or its graph) we see that}
\[  \int_0^{\pi} g(x)\, dx =0,\]
\lang{de}{
da die Flächen, die der Graph mit der $x$-Achse einschließt, im Bereich $[0;\frac{\pi}{2}]$ und 
$[\frac{\pi}{2};\pi]$ gleich groß sind und sich gegenseitig wegheben.
\\\\
Damit gilt:
}
\lang{en}{
as the areas bounded by the graph and the $x$-axis in the intervals $[0;\frac{\pi}{2}]$ and 
$[\frac{\pi}{2};\pi]$ are the same size, and thus cancel each other out.
\\\\
Hence:
}
\[  \int_0^{\pi} h(x)\, dx=\int_0^{\pi}  f(x)\, dx + \int_0^{\pi} g(x)\, dx = \pi^2+0=\pi^2. \]
\end{example}

\section{\lang{de}{Stetigkeit und Integrierbarkeit}
         \lang{en}{Continuity and integrability}}\label{sec:stet-u-int}

\lang{de}{
Es stellt sich noch die Frage, welche Funktionen integrierbar sind.
Die wichtigste Klasse bilden die stetigen Funktionen.
}
\lang{en}{
The question remains of which functions are integrable. 
The most important class of integrable functions are the continuous functions.
}

\begin{theorem}
\lang{de}{
Ist $f$ eine auf dem Intervall $[a;b]$ stetige Funktion, so ist $f$ auf $[a;b]$ integrierbar.
Das Integral $\int_a^b f(x)\, dx$ stimmt dann mit dem orientierten Flächeninhalt, den der Graph von 
$f$ mit der $x$-Achse im Bereich $a\leq x\leq b$ einschließt, überein.
\\\\
Außerdem sind sogar stückweise stetige Funktionen integrierbar, die abschnittsweise aus stetigen 
Funktionen zusammengefügt werden, sofern sie beschränkt sind.
\\\\
\floatright{\href{https://api.stream24.net/vod/getVideo.php?id=10962-2-10814&mode=iframe&speed=true}{\image[75]{00_video_button_schwarz-blau}}}\\
}
\lang{en}{
Let $f$ be a function continuous on the interval $[a;b]$. Then $f$ is integrable on $[a;b]$. The 
integral $\int_a^b f(x)\, dx$ then corresponds to the signed area bounded by the graph of $f$ and 
the $x$-axis in the interval $a\leq x\leq b$.
\\\\
Even piecewise continuous functions are integrable, provided that they are bounded.
}
\end{theorem}
\begin{proof*}
\lang{de}{
Die Integrierbarkeit stetiger Funktionen folgt im Wesentlichen aus der gleichmäßigen Stetigkeit der 
Funktionen auf Kompakta. (Der Begriff der gleichmäßigen Stetigkeit wird im Rahmen dieses Kurses 
nicht behandelt, der folgende Beweis sollte aber trotzdem zumindest von der Grundidee verständlich 
sein.)
}
\lang{en}{
The integrability of continuous functions follows in general from their uniform continuity on 
compact subsets of the real line. (The definition of uniform continuity is not handled in the 
context of this course, however the following proof should still be mostly understandable.)
}
\begin{incremental}
\step
\lang{de}{
Eine stetige Funktion $f$ ist gleichmäßig stetig auf dem kompakten Intervall $[a;b]$. 
Das heißt, dass es zu jedem $\varepsilon > 0$ ein $\delta>0$ gibt mit
}
\lang{en}{
A continuous function $f$ is uniformly continuous on the compact interval $[a;b]$. 
This means that for arbitrary $\varepsilon > 0$ there exists $\delta>0$ such that
}
\[
|f(x)-f(y)| < \varepsilon \ \text{ \lang{de}{für alle}\lang{en}{for all} } \ |x-y|<\delta.
\]
\step
\lang{de}{
Ersetze $\varepsilon$ von oben durch $\varepsilon/(b-a)$. Für eine Zerlegung $Z$ von $[a;b]$ mit 
Feinheit $\Delta Z < \delta$ und Zwischenstellen $x_0,\ldots, x_n$ ergibt sich
}
\lang{en}{
Replace the $\varepsilon$ from above with $\varepsilon/(b-a)$. For a partition $Z$ of $[a;b]$ with 
norm $\Delta Z < \delta$ and subinterval endpoints $x_0,\ldots, x_n$ we have
}
\begin{align*}
O(Z)-U(Z) & \leq \sum_{j=0}^{n} (\sup_{x\in [x_j;x_{j+1}]}f(x)-\inf_{x\in [x_j;x_{j+1}]}f(x)) (x_{j+1}-x_j) \\ &\leq \sum_{j=0}^n \frac{\varepsilon}{b-a} (x_{j+1}-x_j) 
\end{align*}
\lang{de}{aus der gleichmäßigen Stetigkeit.}
\lang{en}{by the uniform continuity.}
\step
\lang{de}{Dann gilt also}
\lang{en}{Hence we have}
\[
O(Z)-U(Z) \leq \varepsilon
\]
\lang{de}{und $f$ ist nach Definition integrierbar.}
\lang{en}{and $f$ is by definition integrable.}
\end{incremental}

\lang{de}{
Der Wert des Integrals stimmt mit dem orientierten Flächeninhalt überein, da dieser ebenfalls 
zwischen Ober- und Untersumme liegt und deren Differenz beliebig klein wird. 
\\\\
Der Funktionswert an einer einzelnen Stelle hat keinen Einfluss auf den Wert des Integrals.
Ist zum Beispiel
}
\lang{en}{
The value of the integral corresponds to the signed area, as it lies between the upper and lower 
sums and the difference between these two can be made arbitrarily small.
\\\\
The value of the function at a single point has no effect on the value of the integral. For example, 
if
}
\[
f(x) = \begin{cases}x, & x \neq 0, \\ 42, & x = 0,\end{cases}
\]
\lang{de}{
gegeben, dann ist $f$ auf $[-1;1]$ integrierbar und für das Integral gilt 
$\int_{-1}^1 f(x)\, dx = \int_{-1}^1 x\, dx = 0$. 
Der Einfluss des Wertes $f(0)=42$ wird bei Obersummen feinerer Zerlegungen immer geringer, bis er 
beim Wert des Integrals gar keine Rolle mehr spielt.
\\\\
Zusammen mit Regel \ref{rule:intervall} erklärt dies, warum stückweise stetige Funktionen 
integrierbar sind und wie stückweise stetige Funktionen integriert werden können. Ein Beispiel 
betrachten wir in \ref{stueck}.
}
\lang{en}{
then $f$ is integrable on $[-1;1]$ with $\int_{-1}^1 f(x)\, dx = \int_{-1}^1 x\, dx = 0$. 
The effect of the value $f(0)=42$ on the upper sum becomes smaller as the partition becomes finer, 
until it has no effect at all on the value of the integral.
\\\\
Together with rule \ref{rule:intervall}, this explains why piecewise continuous functions are 
integrable and how we can integrate them. We consider an example of this in \ref{stueck}.
}
\end{proof*}

\lang{de}{
Diese Aussage schließt die aus der Schule bekannten Funktionen wie Polynome, die Exponentialfunktion 
$e^x$, die trigonometrischen Funktionen $\sin(x)$, $\cos(x)$ und die Betragsfunktion über beliebigen 
Intervallen $[a;b]$ ein. Ebenso alle Funktionen, die aus diesen zusammengesetzt sind (Summen, 
Produkte, Quotienten, Verkettungen), sofern im Intervall $[a;b]$ keine Definitionslücke auftritt.
}
\lang{en}{
This theorem is applicable to many common functions such as polynomial functions, the exponential 
function $e^x$, the trigonometric functions $\sin(x)$, $\cos(x)$ and the absolute value function 
on any interval $[a;b]$. Likewise, it applies to functions which are sums, products, quotients and 
compositions of the above, provided that the function is defined everywhere in the interval 
$[a;b]$.
}
 
\begin{remark}
\lang{de}{
Beim Differenzieren gibt es für die aus der Schule bekannten Funktionen stets Formeln und Regeln, um 
deren Ableitungsfunktionen zu bestimmen. Für die Integration ist dies nicht der Fall. Zwar gibt es 
auch hier einige Formeln und Regeln, die im nächsten Themenblock behandelt werden, aber dennoch kann 
man nicht für alle Funktionen Formeln herleiten. Ein solches Beispiel ist die Funktion 
$f(x)=e^{-x^2}$.
\\\\
In solchen Fällen kann man aber durch Ober-, Unter- oder 
\ref[ober-und-unter][Zwischensummen]{sec:zwischensummen} das Integral annähern.
}
\lang{en}{
There exist formulas and rules for differentiating common functions such as the above. This is not 
the case for integration. The following chapter covers some formulas and rules for integration, 
but not every function can be integrated. An example of such a function is $f(x)=e^{-x^2}$.
\\\\
In these cases, we can still approximate an integral using 
\ref[ober-und-unter][Riemann sums]{sec:zwischensummen}.
}
\end{remark}


\lang{de}{
Wir wissen zwar inzwischen, dass stetige Funktionen integrierbar sind, jedoch ist die Berechnung des 
Integral mittels der Definition recht umständlich, da man laut Definition alle Zerlegungen des 
Intervalls betrachten muss. Im nächsten Abschnitt und im Themenblock "`Integrationstechniken"' 
werden wir schnelle Möglichkeiten kennenlernen, für viele Funktionen deren Integrale zu berechnen.\\
Bei solchen Funktionen jedoch, auf die diese Techniken nicht anwendbar sind, hilft die folgende 
Aussage.
}
\lang{en}{
We now know that continuous functions are integrable, but computing an integral using the definition 
is quite laborious, as one has to consider all partitions of the interval. In the next section and 
in the chapter 'Techniques for integration' we introduce some faster methods for calculating the 
integral of a variety of functions.\\
The following theorem is useful for functions to which these methods cannot be applied.
}

\begin{theorem}
\lang{de}{
Es sei $f$ eine auf dem Intervall $[a;b]$ integrierbare Funktion. Für jede natürliche Zahl 
$m\in \N$ sei eine Zerlegung $Z_m$ des Intervalls $[a;b]$ in $n_m$ Teilintervalle gegeben, deren 
Feinheiten gegen Null konvergieren, also mit $\lim_{m\to \infty} \Delta Z_m=0$. Weiter seien zu den 
Zerlegungen $Z_m$ Zwischenstellen $a_{m,k}$ für $k=1,\ldots, n_m$ gegeben, und 
}
\lang{en}{
Let $f$ be a function that is integrable on the interval $[a;b]$. For every natural number $m\in \N$ 
let $Z_m$ be a partition of the interval $[a;b]$ into $n_m$ subintervals, such that 
$\lim_{m\to \infty} \Delta Z_m=0$. Let $a_{m,k}$ be a point in the $k$th subinterval of each $Z_m$, 
for $k=1,\ldots, n_m$, and
}
\[  S_m:=S(Z_m; a_{m,1},\ldots, a_{m,n_m})=\sum_{k=1}^{n_m} f(a_{m,k})\cdot \Delta x_{m,k} \]
\lang{de}{
die Zwischensumme von $f$ zur Zerlegung $Z_m$ und den Zwischenstellen $a_{m,k}$ für 
$k=1,\ldots, n_m$. Dann gilt
}
\lang{en}{
the Riemann sum of $f$ over the partition $Z_m$. Then we have
}
\[  \lim_{m\to \infty} S_m = \int_{a}^b f(x) dx. \]

\lang{de}{
\floatright{\href{https://www.hm-kompakt.de/video?watch=607}{\image[75]{00_Videobutton_schwarz}}}\\
}
\lang{en}{}
\end{theorem}

\lang{de}{
Grob gesagt, bedeutet diese Aussage, dass man nicht alle Zwischensummen betrachten muss, sondern 
lediglich eine Folge von Zwischensummen, bei denen die Zerlegungen immer feiner werden. Oftmals 
nimmt man als $n$-tes Folgenglied die äquidistante Zerlegung in $n$ Teilintervalle und geeignete 
Zwischenstellen, z.\,B. jeweils die untere Grenze der Teilintervalle. 
}
\lang{en}{
Roughly speaking, this theorem states that not all Riemann sums need to be considered, only a 
sequence of Riemann sums whose partitions become finer and finer. Often we take the partition into 
$n$ equal subintervals as the $nth$ partition and a suitable point within each subinterval, for 
example the lower endpoint of each subinterval.
}

\begin{block}[warning]
\lang{de}{
Der Satz kann \textbf{nicht} dazu benutzt werden,  Integrierbarkeit zu zeigen! Er dient lediglich 
dazu, das Integral zu berechnen, wenn man schon weiß, dass das Integral existiert.
\\\\
Als einfaches Beispiel einer nicht-integrierbaren Funktion können wir die Dirichletsche 
Sprungfunktion aus \ref{ex:dirichlet-nicht-intbar} nehmen. Wählen wir ausschließlich rationale 
Zwischenstellen, die immer feiner werden, dann existiert der Grenzwert $ \lim_{m\to \infty} S_m=1$. 
Aber dennoch ist die Funktion nicht integrierbar.
}
\lang{en}{
The theorem \textbf{cannot} be used to prove integrability! It may only be used to calculate an 
integral once we know that it exist.
\\\\
As a simple example using a non-integrable function, consider the Dirichlet function from 
\ref{ex:dirichlet-nicht-intbar}. If we choose a rational number in each interval, then the limit 
$ \lim_{m\to \infty} S_m=1$ exists. However, the function is not integrable.
}
\end{block}

\lang{de}{
Zum Abschluss geben wir ein Beispiel, wie stückweise stetige Funktionen integriert werden.
}
\lang{en}{
Finally we give an example of how piecewise continuous functions can be integrated.
}
\begin{example}
\label{stueck}
%\emph{Beispiel: } 
\lang{de}{Wir betrachten eine abschnittsweise definierte Funktion $f(x)$: }
\lang{en}{Consider the piecewise function $f(x)$:}
\[ f(x) = \begin{cases} x, & x \in [0;1), \\
                        2x, & x \in [1;2].
          \end{cases} \]
\begin{center}
\image{T303_Integral_E}     
\end{center}
\lang{de}{
$f(x)$ hat also eine Sprungstelle bei $x=1$. An den übrigen Stellen ist sie aber stetig, und sie ist 
auf dem ganzen Intervall $[0;2]$ beschränkt. (Die Funktionswerte liegen alle im Intervall $[0;4]$.) 
Daher ist $f(x)$ auf $[0;2]$ integrierbar. 
Wir können für das Integral nun die Regel \ref{rule:intervall} anwenden:
}
\lang{en}{
$f(x)$ has a jump discontinuity at $x=1$. The function is continuous at all other points, and is 
bounded on the whole interval $[0;2]$. (The function values all lie in the interval $[0;4]$.) 
Hence $f(x)$ is integrable on $[0,2]$. 
We may now apply rule \ref{rule:intervall} to compute the integral:
}
\[
\int_0^2 f(x)\, dx = \int_0^1 f(x)\, dx + \int_1^2 f(x)\, dx.
\]
\lang{de}{
Die jeweiligen Integrale können beispielsweise geometrisch bestimmt werden. Das haben wir bereits im 
\ref[teil1][ersten Teil]{stueck} getan. Das erste Integral ist durch eine Dreiecksfläche und das 
zweite durch eine Trapezfläche gegeben:
}
\lang{en}{
Both of these integrals can be computed geometrically for example. In fact, we have already done so 
in the \ref[teil1][first section]{stueck}. The first integral is simply the area of a triangle, and 
the second is the area of a trapezium:
}
\[ \int_0^1 f(x)\, dx + \int_1^2 f(x)\, dx = \frac{1}{2}(1\cdot 1) + 1 \cdot 2 + 
\frac{1}{2}(1 \cdot 2) = \frac{7}{2}. \]                    
\end{example}


\end{content}