%$Id:  $
\documentclass{mumie.article}
%$Id$
\begin{metainfo}
  \name{
    \lang{de}{Uneigentliche Integrale}
    \lang{en}{Improper integrals}
  }
  \begin{description} 
 This work is licensed under the Creative Commons License Attribution 4.0 International (CC-BY 4.0)   
 https://creativecommons.org/licenses/by/4.0/legalcode 

    \lang{de}{Beschreibung}
    \lang{en}{Description}
  \end{description}
  \begin{components}
    \component{generic_image}{content/rwth/HM1/images/g_img_00_Videobutton_schwarz.meta.xml}{00_Videobutton_schwarz}
    \component{js_lib}{system/media/mathlets/GWTGenericVisualization.meta.xml}{mathlet1}
    \component{generic_image}{content/rwth/HM1/images/g_img_00_video_button_schwarz-blau.meta.xml}{00_video_button_schwarz-blau}
  \end{components}
  \begin{links}
    \link{generic_article}{content/rwth/HM1/T301_Differenzierbarkeit/g_art_content_02_ableitungsregeln.meta.xml}{content_02_ableitungsregeln}
    \link{generic_article}{content/rwth/HM1/T304_Integrierbarkeit/g_art_content_08_integral_eigenschaften.meta.xml}{int-barkeit}
    \link{generic_article}{content/rwth/HM1/T304_Integrierbarkeit/g_art_content_09_integrierbare_funktionen.meta.xml}{stammfunktion}
    \link{generic_article}{content/rwth/HM1/T305_Integrationstechniken/g_art_content_12_substitutionsregel.meta.xml}{substitution}
  \end{links}
  \creategeneric
\end{metainfo}
\begin{content}
\usepackage{mumie.ombplus}
\usepackage{mumie.genericvisualization}
\ombchapter{3}
\ombarticle{4}

\begin{visualizationwrapper}
\title{\lang{de}{Uneigentliche Integrale}\lang{en}{Improper integrals}}
 

\begin{block}[annotation]
  Im Ticket-System: \href{http://team.mumie.net/issues/10039}{Ticket 10039}\\
\end{block}

\begin{block}[info-box]
\tableofcontents
\end{block}

\lang{de}{
Wir haben in den vorigen Abschnitten \link{int-barkeit}{Integrale} $\int_a^b f(x)\, dx$ 
kennengelernt und gesehen, dass man sie oft mit Hilfe von \link{stammfunktion}{Stammfunktionen}
berechnen kann.
\\\\
Eine wichtige Voraussetzung war, dass das Intervall $[a;b]$ ein endliches Intervall ist und dass 
$f$ auf dem ganzen Intervall $[a;b]$ definiert und beschränkt ist.
In diesem Abschnitt wollen wir den Begriff etwas erweitern, und auch Integrale über unendlichen 
Intervallen definieren sowie Integrale, bei denen die Funktion $f$ unbeschränkt oder an einer 
Integralgrenze gar nicht definiert ist.
Da diese keine Integrale im eigentlichen Sinn sind, heißen sie \emph{uneigentliche Integrale}.
}
\lang{en}{
In the previous sections we introduced \link{int-barkeit}{integrals} $\int_a^b f(x)\, dx$ and 
saw that they can often be computed using \link{stammfunktion}{antiderivatives}.
\\\\
An important condition throughout the previous sections was that the interval $[a;b]$ should be 
finite and that $f$ should be defined and bounded on the whole interval $[a;b]$. 
In this section we will slightly extend the definition to include integrals over unbounded intervals 
and integrals of a function $f$ which is unbounded or not defined on one end of the interval. As 
these are not intervals in the 'proper' sense, they are fittingly called \emph{improper integrals}.
}

\section{\lang{de}{Uneigentliche Integrale über unbeschränkten Intervallen}
         \lang{en}{Improper integrals over unbounded intervals}}

\begin{definition}\label{def:uneig-int-interval}
\lang{de}{
Es sei $f:[a;\infty ) \to \R$  eine Funktion, die \"uber jedem Teilintervall 
$[a;d] \subseteq [a;\infty )$ integrierbar ist.
\\\\
Das \notion{uneigentliche Integral} der Funktion $f$ \"uber dem Intervall $[a;\infty)$ ist definiert als Grenzwert
}
\lang{en}{
Let $f:[a;\infty ) \to \R$ be a function that is integrable over every subinterval 
$[a;d] \subseteq [a;\infty )$.
\\\\
The \notion{improper integral} of the function $f$ over the interval $[a;\infty)$ is defined as the limit
}
  \\
\[ \int_a^\infty f\left(x\right) \; dx \coloneq
 \lim\limits_{d \to  \infty} \int_a^d f\left(x\right) \; dx,
 \]
\lang{de}{
sofern der Grenzwert existiert.
\\\\
Existiert dieser Grenzwert nicht, so sagen wir:\\
Das uneigentliche Integral $\int_a^\infty f\left(x\right) \; dx$ existiert nicht.\\
 \floatright{\href{https://api.stream24.net/vod/getVideo.php?id=10962-2-10816&mode=iframe&speed=true}{\image[75]{00_video_button_schwarz-blau}}
 \href{https://www.hm-kompakt.de/video?watch=609}{\image[75]{00_Videobutton_schwarz}}}\\\\
}
\lang{en}{
provided the limit exists.
\\\\
If the limit does not exist, we say that the improper integral $\int_a^\infty f\left(x\right) \; dx$ 
does not exist.
}
\end{definition}

\begin{example}\label{ex:motivation}
\lang{de}{
Wir betrachten die konstante Funktion $f(x)=1$ und das uneigentliche Integral $\int_0^\infty f(x)\, dx$.
}
\lang{en}{
We consider the constant function $f(x)=1$ and the improper integral $\int_0^\infty f(x)\, dx$.
}
\lang{de}{
\begin{genericGWTVisualization}[550][800]{mathlet1}
    \begin{variables}
        \function{f}{rational}{1}
        \parametricFunction{ax}{real}{t, 0, 1, 100, 100}
		\pointOnParametricCurve[editable]{P}{real}{var(ax)}{0.8}
        \set{s}{real}{x < var(P)[x] AND x > 0 AND y < 1 AND y > 0}
        \point{p1}{real}{0,0}
        \point{p2}{real}{0,1}
        \point{p4}{real}{var(P)[x],1}
        \segment{a}{real}{var(p1),var(p2)}
        \segment{b}{real}{var(P),var(p4)}
        \number{dummy}{rational}{var(P)[x]}
    \end{variables}
    \color{P}{#00CC00}
	\color{f}{BLACK}
    \color{s}{#0066CC}
    \color{a}{#0066CC}
    \color{b}{#0066CC}
    \begin{canvas}
			\plotLeft{0}
			\plotRight{20}
			\plotSize{400,300}
			\plot[coordinateSystem]{P,f,s,a,b}
	\end{canvas}
    \text{Für jedes $b > 0$ betrachten wir das Integral $\int_0^b f(x)\, dx$.
    Sie können die obere Grenze in der Grafik verschieben, in dem Sie den grünen Punkt verschieben.
    Für $b=\var{dummy}$ haben wir $\int_0^b f(x)\, dx = \var{dummy}$.
    Leicht zu sehen ist, dass die Fläche immer größer wird, je größer die obere Integrationsgrenze ist.
    Es ist nämlich $\int_0^b f(x)\, dx = b$. Das uneigentliche Integral
    $\int_0^\infty f(x)\, dx$ existiert damit nicht.}
\end{genericGWTVisualization}
}
\lang{en}{
\begin{genericGWTVisualization}[550][800]{mathlet1}
    \begin{variables}
        \function{f}{rational}{1}
        \parametricFunction{ax}{real}{t, 0, 1, 100, 100}
		\pointOnParametricCurve[editable]{P}{real}{var(ax)}{0.8}
        \set{s}{real}{x < var(P)[x] AND x > 0 AND y < 1 AND y > 0}
        \point{p1}{real}{0,0}
        \point{p2}{real}{0,1}
        \point{p4}{real}{var(P)[x],1}
        \segment{a}{real}{var(p1),var(p2)}
        \segment{b}{real}{var(P),var(p4)}
        \number{dummy}{rational}{var(P)[x]}
    \end{variables}
    \color{P}{#00CC00}
	\color{f}{BLACK}
    \color{s}{#0066CC}
    \color{a}{#0066CC}
    \color{b}{#0066CC}
    \begin{canvas}
			\plotLeft{0}
			\plotRight{20}
			\plotSize{400,300}
			\plot[coordinateSystem]{P,f,s,a,b}
	\end{canvas}
    \text{For every $b > 0$ we consider the integral $\int_0^b f(x)\, dx$.
    The upper limit can be moved in the visualisation by moving the green point.
    For $b=\var{dummy}$ we have $\int_0^b f(x)\, dx = \var{dummy}$.
    It is easy to see that the area boudned by the graph and the $x$-axis becomes larger as the 
    upper limit becomes 
    larger.
    After all, the area is $\int_0^b f(x)\, dx = b$. The improper integral
    $\int_0^\infty f(x)\, dx$ hence does not exist.}
\end{genericGWTVisualization}
}
\end{example}

\begin{remark}
\lang{de}{
Damit für eine stetige Funktion $f:[a;\infty)\to\R$ das uneigentliche Integral 
$\int_a^\infty f(x)\, dx$ existieren kann, muss notwendigerweise $\lim_{x\to\infty}f(x) = 0$ gelten. 
Ansonsten kann der Flächeninhalt unter dem Graphen wie im Beispiel \ref{ex:motivation} (betraglich) 
beliebig groß werden. In den nächsten Beispielen betrachten wir uneigentliche Integrale von 
Funktionen mit $\lim_{x\to\infty}f(x) = 0$. Wir werden sehen, dass die Bedingung für die Existenz 
nicht ausreicht. Dies ist vergleichbar mit der Reihenkonvergenz.
}
\lang{en}{
Given a continuous function $f:[a;\infty)\to\R$, for the improper integral $\int_a^\infty f(x)\, dx$ 
to exist we must have $\lim_{x\to\infty}f(x) = 0$. Otherwise the area bounded by the graph can 
be made arbitrarily large, as in the example \ref{ex:motivation}. In the upcoming examples we will 
consider improper integrals of functions where $\lim_{x\to\infty}f(x) = 0$. We will see that this 
condition, whilst necessary, is not sufficient. This is comparable to the convergence of series 
requiring the convergence of terms as a sequence.
}
\end{remark}

\begin{example}\label{ex1}
\begin{tabs*}
\tab{$\int_0^\infty \frac{1}{1+x^2} \, dx$}
\lang{de}{
Für die Funktion $f(x)=\frac{1}{1+x^2}$ existiert das uneigentliche Integral
$ \int_0^\infty \frac{1}{1+x^2}\, dx$ und es gilt
}
\lang{en}{
Given the function $f(x)=\frac{1}{1+x^2}$, the improper integral 
$ \int_0^\infty \frac{1}{1+x^2}\, dx$ exists, and
}
\[ 
 \int_0^\infty   \frac{1}{1+x^2}\, dx= \frac{\pi}{2}.
 \]

\lang{de}{
Da die Funktion $f(x)=\frac{1}{1+x^2}$ nämlich genau die 
\ref[content_02_ableitungsregeln][Ableitung des Arkustangens]{ex:abl-umkehrfunktion} ist, ist 
$\arctan(x)$ eine Stammfunktion für $f$. Damit gilt:
}
\lang{en}{
As the function $f(x)=\frac{1}{1+x^2}$ is precisely the 
\ref[content_02_ableitungsregeln][derivative of the arctangent]{ex:abl-umkehrfunktion}, $\arctan(x)$ 
is itself an antiderivative of $f$. Hence:
}
 \begin{eqnarray*}
 \int_0^\infty \;\frac{1}{1+x^2}\, dx &=& 
 \lim\limits_{b\rightarrow \infty}\;\int_0^b\; \frac{1}{1+x^2}\, dx\\
 &=&\lim\limits_{b\rightarrow \infty}\; \left[ \arctan(x)\right]_0^b\\
 &=&\lim\limits_{b\rightarrow \infty} \;(\arctan(b) - \arctan(0)\,)\\
 &=& \frac{\pi}{2}.
 \end{eqnarray*}
\tab{$\int_1^\infty \frac{1}{x}\,dx$}
\lang{de}{
Für die Funktion $f(x)=\frac{1}{x}$ existiert das uneigentliche Integral
$ \int_1^\infty \frac{1}{x}\, dx$ nicht, d.\,h. 
}
\lang{en}{
Given the function $f(x)=\frac{1}{x}$, the improper integral $ \int_1^\infty \frac{1}{x}\, dx$ 
does not exist, that is,
}
\[
  \lim\limits_{b\rightarrow \infty}\;\int_1^b \frac{1}{x}\, dx \quad \text{\lang{de}{existiert nicht}\lang{en}{does not exist}}.
 \]
 
\lang{de}{
Da die Funktion $f(x)=\frac{1}{x}$ für positive reelle Zahlen nämlich genau die Ableitung des Logarithmus ist, ist $\ln(x)$ eine Stammfunktion für $f$. Damit gilt:
}
\lang{en}{
As the function $f(x)=\frac{1}{x}$ is precisely the derivative of the natural logarithm function for 
positive real numbers, $\ln(x)$ is an antiderivative of $f$. Hence:
}
 \begin{eqnarray*}
  \lim\limits_{b\rightarrow \infty}\;\int_1^b \frac{1}{x}\, dx
  &=&\lim\limits_{b\rightarrow \infty} \;\left[ \ln(x)\right]_1^b\\
  &=&\lim\limits_{b\rightarrow \infty} \;\ln(b) =\infty \, .
  \end{eqnarray*}
\lang{de}{
Die bestimmten Integrale wachsen also unbeschränkt für $b \rightarrow \infty$, weshalb das 
uneigentliche Integral $ \int_1^\infty \frac{1}{x}\, dx$ nicht existiert.
}
\lang{en}{
These integrals grow without bounds as $b \rightarrow \infty$, so the improper integral 
$ \int_1^\infty \frac{1}{x}\, dx$ does not exist.
}
\tab{$\int_1^\infty \frac{1}{x^\alpha}\,dx$}
\lang{de}{
Für $\alpha>0,\; \alpha \neq 1$ betrachten wir die Funktion $f:(0;\infty)\to \R$ mit $f(x)=\frac{1}{x^\alpha}=x^{-\alpha}$. Eine Stammfunktion für $f$ ist die Funktion
}
\lang{en}{
Consider the function $f:(0;\infty)\to \R$ defined by $f(x)=\frac{1}{x^\alpha}=x^{-\alpha}$. For 
$\alpha>0,\; \alpha \neq 1$, the function
}
\[ F(x)=\frac{x^{1-\alpha}}{1-\alpha}, \]
\lang{de}{
wie man durch Ableiten von $F(x)$ sieht. Für $b>1$ ist damit
}
\lang{en}{
is an antiderivative of $f$. This can be seen by differentiating $F(X)$. For $b>1$ we therefore have
}
\begin{eqnarray*}
\int_1^b \;\frac{1}{x^\alpha}\, dx &=&  \left[ \frac{x^{1-\alpha}}{1-\alpha}\right]_1^b\\
&=& \frac{b^{1-\alpha}}{1-\alpha} -   \frac{1}{1-\alpha}. 
\end{eqnarray*} 

\lang{de}{Ist nun $\alpha  > 1$, so gilt}
\lang{en}{If $\alpha  > 1$, we have}
\begin{eqnarray*}
\int_1^\infty \;\frac{1}{x^\alpha}\, dx &=& 
\lim_{b\to \infty} \int_1^b \;\frac{1}{x^\alpha}\, dx \\
&=& \lim_{b\to \infty}  \frac{b^{1-\alpha}}{1-\alpha} -   \frac{1}{1-\alpha} \\
&=&\frac{1}{\alpha-1}.
\end{eqnarray*} 
\lang{de}{
Das uneigentliche Integral $\int_1^\infty \;\frac{1}{x^\alpha}\, dx$ existiert in diesem Fall also.
\\\\
Ist jedoch $0<\alpha<1$, so gilt
}
\lang{en}{
The improper integral $\int_1^\infty \;\frac{1}{x^\alpha}\, dx$ hence exists.
\\\\
If $0<\alpha<1$, then
}
\begin{eqnarray*}
\int_1^\infty \;\frac{1}{x^\alpha}\, dx &=& 
\lim_{b\to \infty} \int_1^b \;\frac{1}{x^\alpha}\, dx \\
&=& \lim_{b\to \infty}  \frac{b^{1-\alpha}}{1-\alpha} -   \frac{1}{1-\alpha} =\infty.
\end{eqnarray*}
\lang{de}{
In diesem Fall existiert das uneigentliche Integral $\int_1^\infty \;\frac{1}{x^\alpha}\, dx$
also nicht.
}
\lang{en}{
In this case the improper integral $\int_1^\infty \;\frac{1}{x^\alpha}\, dx$ does not exist.
}
\end{tabs*}
\end{example}
\begin{quickcheck}
    \begin{variables}
        \randint{a}{-1}{1}
        \randint[Z]{b}{-3}{3}
        \function[normalize]{f}{b*x^(a/2)}
    \end{variables}
    \text{\lang{de}{Existiert das uneigentliche Integral $\int_1^\infty \var{f} \, dx$?}
          \lang{en}{Does the improper integral $\int_1^\infty \var{f} \, dx$ exist?}}
    \explanation{\lang{de}{Betrachten Sie noch einmal die letzten Beispiele.}
                 \lang{en}{Review the previous examples.}}
    \begin{choices}{unique}
        \begin{choice}
            \text{Ja.}
            \solution{false}
        \end{choice}
        \begin{choice}
            \text{Nein.}
            \solution{true}
        \end{choice}
    \end{choices}
\end{quickcheck}


\lang{de}{In der nächsten Bemerkung behandeln wir andere Typen von uneigentlichen Integralen.}
\lang{en}{In the next remark we consider other types of improper integrals.}
\begin{remark}
\begin{enumerate}
\item \lang{de}{
  Ist $f:(-\infty; b]\to \R$ eine Funktion,  die \"uber jedem Teilintervall 
  $[d;b] \subseteq (-\infty;b]$ integrierbar ist, so definiert man entsprechend das uneigentliche Integral
  }
  \lang{en}{
  Let $f:(-\infty; b]\to \R$ be a function that is integrable over every subinterval 
  $[d;b] \subseteq (-\infty;b]$. Then we define the corresponding improper integral
  }
  \[ \int_{-\infty}^b f(x)\, dx :=\lim_{a\to -\infty}  \int_{a}^b f(x)\, dx, \]
  \lang{de}{
  sofern der Grenzwert existiert.
  Existiert der Grenzwert nicht, so sagt man, dass das uneigentliche Integral
  $\int_{-\infty}^b f(x)\, dx$ nicht existiert.
  }
  \lang{en}{
  provided that the limit exists. 
  If the limit does not exist, we say that the improper integral 
  $\int_{-\infty}^b f(x)\, dx$ does not exist.
  }
  \item \lang{de}{
  Ist $f:\R=(-\infty;\infty)\to \R$ eine Funktion, die über jedem endlichen Teilintervall $[a;b]$ 
  integrierbar ist, so definiert man auch das uneigentliche Integral
  }
  \lang{en}{
  Let $f:\R=(-\infty;\infty)\to \R$ be a function that is integrable on every finite subinterval 
  $[a;b]$. Then we define the corresponding improper integral
  }
\[  \int_{-\infty}^\infty f(x)\, dx \] 
\lang{de}{für beliebiges $c\in \R$ als}
\lang{en}{for any $c\in \R$ as}
\[  \int_{-\infty}^\infty f(x)\, dx= \int_{-\infty}^c f(x)\, dx +   \int_{c}^\infty f(x)\, dx, \]
\lang{de}{
sofern beide uneigentlichen Integrale auf der rechten Seite existieren. Die Wahl von $c$ hat keinen Einfluss 
auf den Wert des uneigentlichen Integrals. Alternativ kann das Integral als doppelter Grenzwert 
}
\lang{en}{
provided that both improper integrals on the right-hand side exist. The choice of $c$ has no effect 
on the value of the improper integral. Alternatively, the integral can be defined as the double 
limit
}
\[  \int_{-\infty}^\infty f(x)\, dx= \lim_{a\to -\infty}  \lim_{b\to \infty}  \int_{a}^b f(x)\, dx \]
\lang{de}{
definiert werden, sofern $ \lim_{b\to \infty}  \int_{a}^b f(x)\, dx= \int_a^\infty  f(x)\, dx$ für 
alle $a\in \R$ existiert und der gesamte Grenzwert existiert. Wichtig ist, dass man die 
Grenzübergänge $a\to -\infty$ und $b\to \infty$ unabhängig voneinander macht, wie nachfolgendes 
Beispiel zeigt.
}
\lang{en}{
provided that $ \lim_{b\to \infty}  \int_{a}^b f(x)\, dx= \int_a^\infty  f(x)\, dx$ exists for all 
$a\in \R$ and that the overall limit exists. It is important to consider the limits $a\to -\infty$ 
and $b\to \infty$ seperately, as the following example illustrates.
}
\end{enumerate}    
\end{remark}

\begin{example}
\begin{tabs*}
\tab{$\int_{-\infty}^\infty \sin(x)\, dx$}
\lang{de}{
Wir betrachten die Funktion $f(x)=\sin(x)$, welche auf ganz $\R$ definiert ist.
Wegen der Symmetrie von $\sin(x)$, nämlich $\sin(-x)=-\sin(x)$, gilt für alle $a>0$
}
\lang{en}{
Consider the function $f(x)=\sin(x)$, defined on $\R$. 
By the symmetry of $\sin(x)$, that is, $\sin(-x)=-\sin(x)$, for all $a>0$ we have
}
\[ \int_{-a}^a \sin(x)\, dx = \int_{-a}^0 \sin(x)\, dx + \int_{0}^a \sin(x)\, dx=0.\]
\lang{de}{Damit ist}
\lang{en}{Thus}
\[ \lim_{a\to \infty} \int_{-a}^a \sin(x)\, dx=\lim_{a\to \infty} 0 =0. \]
\lang{de}{
Das uneigentliche Integral $\int_{-\infty}^\infty \sin(x)\, dx$ existiert jedoch nicht.
Für $b>0$ ist nämlich
}
\lang{en}{
The improper integral $\int_{-\infty}^\infty \sin(x)\, dx$ does not exist, as for $b>0$,
}
\[ \int_{0}^b \sin(x)\, dx=  \left[ -\cos(x)\right]_0^b= -\cos(b) + 1\]
\lang{de}{
und $\lim_{b\to \infty}  (-\cos(b) + 1)$ existiert nicht. Also existiert das uneigentliche Integral
$\int_{0}^\infty \sin(x)\, dx$ nicht und erst recht nicht das uneigentliche Integral
$\int_{-\infty}^\infty \sin(x)\, dx$ .
}
\lang{en}{
and $\lim_{b\to \infty}  (-\cos(b) + 1)$ does not exist. Hence the improper integral 
$\int_{0}^\infty \sin(x)\, dx$ does not exist, and certainly the improper integral 
$\int_{-\infty}^\infty \sin(x)\, dx$ does not exist either.
}
\tab{$\int_{-\infty}^\infty \frac{1}{1+x^2}\, dx$}
\lang{de}{
Für die Funktion $f(x)=\frac{1}{1+x^2}$ existiert das uneigentliche Integral
$ \int_{-\infty}^\infty \frac{1}{1+x^2}\, dx$ und es gilt
}
\lang{en}{
Given the function $f(x)=\frac{1}{1+x^2}$, the improper integral 
$ \int_{-\infty}^\infty \frac{1}{1+x^2}\, dx$ exists and
}
\[ 
 \int_{-\infty}^\infty   \frac{1}{1+x^2}\, dx= \pi.
 \]
\lang{de}{
Wir hatten nämlich \lref{ex1}{oben} schon gesehen, dass das uneigentliche Integral
$ \int_0^\infty \frac{1}{1+x^2}\, dx$ existiert mit
}
\lang{en}{
We already saw \lref{ex1}{above} that the improper integral $ \int_0^\infty \frac{1}{1+x^2}\, dx$ 
exists with
}
\[  \int_0^\infty   \frac{1}{1+x^2}\, dx= \frac{\pi}{2}. \]
\lang{de}{Wegen der Symmetrie $f(-x)=\frac{1}{1+(-x)^2}=\frac{1}{1+x^2}=f(x)$ ist damit}
\lang{en}{By the symmetry $f(-x)=\frac{1}{1+(-x)^2}=\frac{1}{1+x^2}=f(x)$, we have}
\[  \int_{-\infty}^0   \frac{1}{1+x^2}\, dx= \int_0^\infty   \frac{1}{1+x^2}\, dx= \frac{\pi}{2}. \]
\lang{de}{
Insgesamt existiert also auch das uneigentliche Integral über $(-\infty; \infty)$ und es ist
}
\lang{en}{
Hence the improper integral also exists over $(-\infty; \infty)$, with
}
\[ 
 \int_{-\infty}^\infty   \frac{1}{1+x^2}\, dx= \int_{-\infty}^0   \frac{1}{1+x^2}\, dx
 + \int_0^\infty   \frac{1}{1+x^2}\, dx= \frac{\pi}{2}+ \frac{\pi}{2}= \pi.
 \] 
\end{tabs*}
\end{example}


\section{\lang{de}{Uneigentliche Integrale bei Definitionslücken}
         \lang{en}{Improper integrals with gaps in the domain}}

\lang{de}{
Nachdem im vorigen Paragraphen uneigentliche Integrale über unendlichen Intervallen
behandelt wurden, betrachten wir nun uneigentliche Integrale über endlichen Intervallen,
bei denen die zu integrierende Funktion an einem Rand nicht definiert ist.
}
\lang{en}{
We have introduced improper integrals over unbounded intervals. Now we introduce improper integrals 
over finite intervals, but where the function being integrated is not defined at one end of the 
interval.
}


\begin{definition}\label{def:uneig-int-def-luecke}
  \lang{de}{
  Es sei $f:\,\,( a; b] \to \R$  eine Funktion, die \"uber 
  jedem Teilintervall $[c;b] \subseteq ( a;b]$ integrierbar ist.
  Das \notion{uneigentliche Integral} der Funktion $f$ \"uber dem Intervall $( a;b]$ ist definiert als Grenzwert
  }
  \lang{en}{
  Let $f:\,\,( a; b] \to \R$ be a function that is integrable over every subinterval 
  $[c;b] \subseteq ( a;b]$. Then the improper integral of the function $f$ over the interval 
  $( a;b]$ is defined as the limit
  }
\[ \int_a^b f\left(x\right)\; dx =
 \lim_{c \searrow a} \int_c^b f\left(x\right) \; dx, \]
  \lang{de}{
  sofern der Grenzwert existiert.
  \\\\
  Existiert dieser Grenzwert nicht, so sagen wir:\\
  Das uneigentliche Integral $\int_a^b f\left(x\right) \; dx$ existiert nicht.
  }
  \lang{en}{
  provided that the limit exists.
  \\\\
  If the limit does not exist, then we say:\\
  The improper integral $\int_a^b f\left(x\right) \; dx$ does not exist.
  }
\end{definition}

\begin{example}\label{ex2}
\begin{tabs*}
\tab{$\int_0^1 \frac{1}{\sqrt{x}}\, dx$}
\lang{de}{
Für die Funktion $f(x)=\frac{1}{\sqrt{x}}$ existiert das uneigentliche Integral
$ \int_0^1 \frac{1}{\sqrt{x}}\, dx$ und es gilt
}
\lang{en}{
Given the function $f(x)=\frac{1}{\sqrt{x}}$, the improper integral 
$ \int_0^1 \frac{1}{\sqrt{x}}\, dx$ exists, with
}
\[  \int_0^1 \frac{1}{\sqrt{x}}\, dx =2 .\]

\lang{de}{
Eine Stammfunktion für $f(x)=\frac{1}{\sqrt{x}}=x^{-1/2}$ ist nämlich $F(x)=2x^{1/2}=2\sqrt{x}$. Damit folgt für $0<c<1$
}
\lang{en}{
The function $F(x)=2x^{1/2}=2\sqrt{x}$ is an antiderivative of $f(x)=\frac{1}{\sqrt{x}}=x^{-1/2}$. 
Thus for $0<c<1$,
}
\begin{eqnarray*}
 \int_c^1 \frac{1}{\sqrt{x}}\, dx &=& \left[ 2\sqrt{x}\right]_c^1 \\
 &=&  2\sqrt{1}-2\sqrt{c}=2-2\sqrt{c}. 
\end{eqnarray*}
Also ist
\[  \int_0^1 \frac{1}{\sqrt{x}}\, dx=\lim_{c\searrow 0} \int_c^1 \frac{1}{\sqrt{x}}\, dx
=\lim_{c\searrow 0} 2-2\sqrt{c} =2.\]
\tab{$\int_0^1 \frac{1}{x}\, dx$}
\lang{de}{
Für die Funktion $f(x)=\frac{1}{x}$ existiert das uneigentliche Integral
$ \int_0^1 \frac{1}{x}\, dx$ nicht, d.\,h. 
}
\lang{en}{
Given the function $f(x)=\frac{1}{x}$, the improper integral $ \int_0^1 \frac{1}{x}\, dx$ does 
not exist, that is,
}
\[
  \lim\limits_{c\searrow 0}\;\int_c^1 \frac{1}{x}\, dx \quad \text{\lang{de}{existiert nicht}\lang{en}{does not exist}}.
 \]
\lang{de}{
Da die Funktion $f(x)=\frac{1}{x}$ für positive reelle Zahlen nämlich genau die Ableitung des 
Logarithmus ist, ist $\ln(x)$ eine Stammfunktion für $f$. Damit gilt:
}
\lang{en}{
As the function $f(x)=\frac{1}{x}$ is precisely the derivative of the natural logarithm for positive 
real numbers, $\ln(x)$ is an antiderivative of $f$. Thus
}
 \begin{eqnarray*}
  \lim\limits_{c\searrow 0}\;\int_c^1 \frac{1}{x}\, dx
  &=&\lim\limits_{c\searrow 0} \;\left[ \ln(x)\right]_c^1\\
  &=&\lim\limits_{c\searrow 0} \; 0-\ln(c) =\infty \, .
  \end{eqnarray*}
  \lang{de}{
  Die bestimmten Integrale divergieren also gegen $+\infty$ im Grenzwert $c\searrow 0$, weshalb das 
  uneigentliche Integral $ \int_0^1 \frac{1}{x}\, dx$ nicht existiert.
  }
  \lang{en}{
  These integrals diverge to $+\infty$ in the limit $c\searrow 0$, so the improper integral 
  $ \int_0^1 \frac{1}{x}\, dx$ does not exist.
  }
 \tab{$\int_0^1 \frac{1}{x^\alpha}\, dx$}
 \lang{de}{
 Für $\alpha>0,\; \alpha \neq 1$ betrachten wir die Funktion $f:(0;\infty)\to \R$ mit 
 $f(x)=\frac{1}{x^\alpha}=x^{-\alpha}$. Für $\alpha=\frac{1}{2}$ ist es also genau die Funktion aus 
 dem ersten Beispiel. Eine Stammfunktion für $f$ ist die Funktion
 }
 \lang{en}{
 For $\alpha>0,\; \alpha \neq 1$, consider the function $f:(0;\infty)\to \R$ given by 
 $f(x)=\frac{1}{x^\alpha}=x^{-\alpha}$. For $\alpha=\frac{1}{2}$, this is precisely the function 
 from the first example. It is readily seen that the function
 }
 \[ F(x)=\frac{x^{1-\alpha}}{1-\alpha}, \]
 \lang{de}{wie man durch Ableiten von $F(x)$ sieht. Für $0<c<1$ ist damit}
 \lang{en}{is an antiderivative of $f$. For $0<c<1$ we have}
\begin{eqnarray*}
\int_c^1 \;\frac{1}{x^\alpha}\, dx &=&  \left[ \frac{x^{1-\alpha}}{1-\alpha}\right]_c^1\\
&=& \frac{1}{1-\alpha} -   \frac{c^{1-\alpha}}{1-\alpha}. 
\end{eqnarray*} 

\lang{de}{Ist nun $0<\alpha < 1$, so gilt }
\lang{en}{For $0<\alpha < 1$, we have}
\begin{eqnarray*}
\int_0^1 \;\frac{1}{x^\alpha}\, dx &=& 
\lim_{c\searrow 0} \int_c^1 \;\frac{1}{x^\alpha}\, dx \\
&=& \lim_{c\searrow 0} \frac{1}{1-\alpha} -   \frac{c^{1-\alpha}}{1-\alpha}\\
&=&\frac{1}{1-\alpha}.
\end{eqnarray*} 
\lang{de}{
Das uneigentliche Integral $\int_0^1 \;\frac{1}{x^\alpha}\, dx$ existiert in diesem Fall also.
\\\\
Ist jedoch $\alpha>1$, so gilt
}
\lang{en}{
Hence the improper integral $\int_0^1 \;\frac{1}{x^\alpha}\, dx$ exists in this case.
\\\\
For $\alpha>1$,
}
\begin{eqnarray*}
\int_0^1 \;\frac{1}{x^\alpha}\, dx &=& 
\lim_{c\searrow 0} \int_c^1 \;\frac{1}{x^\alpha}\, dx \\
&=& \lim_{c\searrow 0}  \frac{1}{1-\alpha} -   \frac{c^{1-\alpha}}{1-\alpha} =\infty.
\end{eqnarray*} 
\lang{de}{
In diesem Fall existiert das uneigentliche Integral $\int_0^1 \;\frac{1}{x^\alpha}\, dx$
also nicht.
}
\lang{en}{
In this case the improper integral $\int_0^1 \;\frac{1}{x^\alpha}\, dx$ does not exist.
}
\end{tabs*}
\end{example}

\begin{remark}
\lang{de}{
Vergleichen wir die \lref{ex2}{letzten Beispiele} mit den \lref{ex1}{Beispielen} aus dem vorigen 
Paragraphen, so sehen wir, dass für die Funktion $f(x)=\frac{1}{x}$ weder das uneigentliche Integral 
$\int_1^\infty f(x)\, dx$ noch das uneigentliche Integral $\int_0^1 f(x)\, dx$ existieren. Für die 
Funktionen $g(x)=\frac{1}{x^\alpha}$ mit $\alpha>0$, $\alpha \neq 1$, die ebenfalls bei $0$ einen 
Pol haben (d.\,h. $\lim_{x\searrow 0} g(x)=\infty$) existiert genau eines der beiden uneigentlichen 
Integrale. Für $\alpha>1$ existiert $\int_1^\infty g(x)\, dx$, aber $\int_0^1 g(x)\, dx$ existiert 
nicht. Für $0<\alpha<1$ existiert $\int_0^1 g(x)\, dx$, aber $\int_1^\infty g(x)\, dx$ existiert 
nicht.
}
\lang{en}{
Let us compare the \lref{ex2}{previous examples} with the \lref{ex1}{earlier examples}. We see that 
given the function $f(x)=\frac{1}{x}$, neither the improper integral $\int_1^\infty f(x)\, dx$ nor 
the improper integral $\int_0^1 f(x)\, dx$ exist. Given the function $g(x)=\frac{1}{x^\alpha}$ with 
$\alpha>0$, $\alpha \neq 1$, which have a pole at $0$ (that is, $\lim_{x\searrow 0} g(x)=\infty$), 
precisely one of the two improper integrals exists. For $\alpha>1$, the integral 
$\int_1^\infty g(x)\, dx$ exists, but $\int_0^1 g(x)\, dx$ does not. For $0<\alpha<1$, the integral 
$\int_0^1 g(x)\, dx$ exists, but $\int_1^\infty g(x)\, dx$ does not.
}
\end{remark}


\begin{rule}
\lang{de}{
Ist die Funktion $f:(a;b]\to \R$ beschränkt, so kann man einen beliebigen Wert $w\in \R$ wählen und 
eine Funktion $g:[a;b]\to \R$ definieren mittels
}
\lang{en}{
Let the function $f:(a;b]\to \R$ be bounded, so that we can choose an arbitrary value $w\in \R$ and 
define a function $g:[a;b]\to \R$ by
}
\[  g(x)=\left\{ \begin{mtable} w, \quad & x=a, \\ f(x), & x \neq a. \end{mtable} \right. \] 
\lang{de}{
Das uneigentliche Integral $\int_a^b f(x)\, dx$ existiert genau dann, wenn das (eigentliche) 
Integral $\int_a^b g(x)\, dx$ existiert, und im Falle der Existenz sind sie auch beide gleich.
\\\\
Das bedeutet zum einen, dass das uneigentliche Integral in diesem Fall ein eigentliches Integral 
ist, und zum anderen, dass das eigentliche Integral nicht von der Wahl von $g(a)$ abhängt.
}
\lang{en}{
The improper integral $\int_a^b f(x)\, dx$ exists precisely if the (proper) integral 
$\int_a^b g(x)\, dx$ exists, and in the case of their existence, they are equal.
\\\\
This means that the in this case, the improper integral is in fact a proper integral, which is of 
course not dependent on the choice of $g(a)$.
}
\end{rule}


\begin{remark}
\begin{enumerate}
\item \lang{de}{
  Ist $f:[a; b)\to \R$ eine Funktion, die \"uber jedem Teilintervall $[a;d] \subseteq [a;b)$ 
  integrierbar ist, so definiert man entsprechend das uneigentliche Integral
  }
  \lang{en}{
  Let $f:[a; b)\to \R$ be a function that is integrable over every subinterval 
  $[a;d] \subseteq [a;b)$. Then we define the corresponding improper integral
  }
  \[ \int_a^b f(x)\, dx :=\lim_{d\nearrow b}  \int_{a}^d f(x)\, dx, \]
  \lang{de}{
  sofern der Grenzwert existiert. Existiert der Grenzwert nicht, so sagt man, dass das uneigentliche 
  Integral $\int_a^b f(x)\, dx$ nicht existiert.
  }
  \lang{en}{
  provided that the limiit exists. If the limit does not exist, then we say that the improper 
  integral $\int_a^b f(x)\, dx$ does not exist.
  }
  \item \lang{de}{
  Ist $f:(a;b)\to \R$ eine Funktion, die über jedem Teilintervall $[c;d]$ integrierbar ist, so 
  definiert man auch das uneigentliche Integral
  }
  \lang{en}{
  Let $f:(a;b)\to \R$ be a function that is integrable over every subinterval $[c;d]$. 
  Then we define the corresponding improper integral
  }
  \[  \int_a^b f(x)\, dx \]
  \lang{de}{als Summe uneigentlicher Integrale}
  \lang{en}{as the sum of the improper integrals}
  \[  \int_{a}^b f(x)\, dx= \int_{a}^c f(x)\, dx +   \int_{c}^b f(x)\, dx, \]
  \lang{de}{mit einer beliebigen Wahl von $c\in (a;b)$.}
  \lang{en}{with an arbitrary choice of $c\in (a;b)$.}
\end{enumerate}    
\end{remark}


\section{\lang{de}{Weitere Fälle uneigentlicher Integrale}
         \lang{en}{Further types of improper integrals}}

\lang{de}{
Außer den Fällen uneigentlicher Integrale, die in den vorigen Paragraphen behandelt wurden,
gibt es noch viele weitere  Fälle. Diese werden jedoch stets durch Zerlegung des 
Integrationsintervalls auf die vorigen zurückgeführt.
Ist zum Beispiel eine Funktion $f:(0;\infty)\to \R$ gegeben, so definiert man auch das
uneigentliche Integral $\int_0^\infty f(x)\, dx$, und zwar als Summe der uneigentlichen Integrale 
$\int_0^c f(x)\, dx$ und $\int_c^\infty f(x)\, dx$:
}
\lang{en}{
There are many types of improper integrals besides the cases introduced above. We can define them 
using the above definitions by changing the intervals over which the integration is occuring. 
For example, given a function $f:(0;\infty)\to \R$, we define the improper integral 
$\int_0^\infty f(x)\, dx$ as the sum of two improper integrals $\int_0^c f(x)\, dx$ and 
$\int_c^\infty f(x)\, dx$:
}
\[ \int_0^\infty f(x)\, dx= \int_0^c f(x)\, dx + \int_c^\infty f(x)\, dx,\]
\lang{de}{
wobei ersteres im zweiten Paragraphen und letzteres im ersten Paragraphen behandelt wurden.
\\\\
Ebenso geht man vor, wenn die Funktion im Inneren des Intervalls eine (oder mehrere) 
Definitionslücke(n) hat oder die Funktion in der Umgebung eines Punktes unbeschränkt ist.
}
\lang{en}{
which were previously defined.
\\\\
We proceed similarly if the function has one (or several) gaps in its domain, or if the function 
is unbounded in a neighbourhood of a point.
}

\begin{example}
\lang{de}{Wir betrachten die Funktion $f:[0;2] \setminus \{1\}\to \R$, die gegeben ist durch}
\lang{en}{Consider the function $f:[0;2] \setminus \{1\}\to \R$ given by}
\[ f(x)=\frac{1}{\sqrt{|1-x|}}. \]
\lang{de}{
Dann ist das Integral $\int_{0}^2 f(x)\, dx$ kein eigentliches Integral, 
da die Funktion unbeschränkt ist und man über Definitionslücken nicht integrieren darf.
\\\\
Man kann es aber als uneigentliches Integral auffassen. Hierzu zerlegt man das Integrationsintervall 
in die Intervalle $[0;1]$ und $[1;2]$ und verwendet für die uneigentlichen Integrale
}
\lang{en}{
Then the integral $\int_{0}^2 f(x)\, dx$ is not a proper integral, as the function is unbounded and 
we are not allowed to integrate over gaps in the domain.
\\\\
We may approach it as an improper integral by seperately considering the intervals $[0;1]$ and 
$[1;2]$, and computing the improper integrals
}
\[ \int_{0}^1 f(x)\, dx=\int_{0}^1 \frac{1}{\sqrt{1-x}} \, dx \]
\lang{de}{und}
\lang{en}{and}
\[ \int_{1}^2 f(x)\, dx=\int_{1}^2 \frac{1}{\sqrt{x-1}} \, dx \]
\lang{de}{
die Definition aus dem vorigen Paragraphen.
\\\\
Mit Hilfe der \link{substitution}{Substitutionsregel} (wird im nächsten Kapitel behandelt) oder 
durch geschicktes Raten findet man als Stammfunktion für $\frac{1}{\sqrt{1-x}}$ die Funktion 
$F(x)=-2\sqrt{1-x}$ und als  Stammfunktion für $\frac{1}{\sqrt{x-1}}$ die Funktion 
$G(x)=2\sqrt{x-1}$. Damit ist
}
\lang{en}{
using the definitions from earlier.
\\\\
By \link{substitution}{substitution} (which we deal with in the next chapter) or by educated 
guesswork we find an antiderivative $F(x)=-2\sqrt{1-x}$ of the function $\frac{1}{\sqrt{1-x}}$, and 
so $G(x)=2\sqrt{x-1}$ is an antiderivative of $\frac{1}{\sqrt{x-1}}$. Hence
}
\begin{eqnarray*}
 \int_{0}^1 f(x)\, dx &=& \int_{0}^1 \frac{1}{\sqrt{1-x}} \, dx\\
 &=& \lim_{c\nearrow 1} \int_{0}^c \frac{1}{\sqrt{1-x}} \, dx\\
 &=&  \lim_{c\nearrow 1} \left[ -2\sqrt{1-x} \right]_0^c = \lim_{c\nearrow 1} (-2\sqrt{1-c})+2\sqrt{1-0}\\
 &=& 2
\end{eqnarray*}
\lang{de}{und}
\lang{en}{and}
\begin{eqnarray*}
 \int_{1}^2 f(x)\, dx &=& \int_{1}^2 \frac{1}{\sqrt{x-1}} \, dx\\
 &=& \lim_{c\searrow 1} \int_{c}^2 \frac{1}{\sqrt{x-1}} \, dx\\
 &=&  \lim_{c\searrow 1} \left[ 2\sqrt{x-1} \right]_c^2 = \lim_{c\searrow 1} 2\sqrt{2-1}-2\sqrt{1-c}\\
 &=& 2
\end{eqnarray*}
\lang{de}{Insgesamt also:}
\lang{en}{Finally,}
\[ \int_{0}^2 f(x)\, dx = \int_{0}^1 f(x)\, dx +  \int_{1}^2 f(x)\, dx=2+2=4.\]
\end{example}
\end{visualizationwrapper}
\end{content}