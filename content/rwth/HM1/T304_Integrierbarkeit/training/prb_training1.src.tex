\documentclass{mumie.problem.gwtmathlet}
%$Id$
\begin{metainfo}
  \name{
    \lang{de}{A01: Intervallzerlegung}
    \lang{en}{}
  }
  \begin{description} 
 This work is licensed under the Creative Commons License Attribution 4.0 International (CC-BY 4.0)   
 https://creativecommons.org/licenses/by/4.0/legalcode 

    \lang{de}{}
    \lang{en}{}
  \end{description}
  \corrector{system/problem/GenericCorrector.meta.xml}
  \begin{components}
    \component{js_lib}{system/problem/GenericMathlet.meta.xml}{mathlet}
  \end{components}
  \begin{links}
  \end{links}
  \creategeneric
\end{metainfo}
\begin{content}
\usepackage{mumie.genericproblem}

\lang{de}{\title{A01: Intervallzerlegung}}

\begin{block}[annotation]
	Im Ticket-System: \href{http://team.mumie.net/issues/10585}{Ticket 10585}
\end{block}

\begin{problem}
	\randomquestionpool{1}{2}
	
%######################################################QUESTION_START	
	\begin{question}	
	%+++++++++++++++++++VARIABLES++++++++++++++++++++++
		\begin{variables}
			\randint[Z]{a}{1}{4}
			\randint[Z]{d1}{1}{40}
			\randint[Z]{d2}{1}{40}
			\randint[Z]{d3}{1}{40}
			\randint[Z]{d4}{1}{40}
			\function[calculate,2]{dx1}{d1/20}
			\function[calculate,2]{dx2}{d2/20}
			\function[calculate,2]{dx3}{d3/20}
			\function[calculate,2]{dx4}{d4/20}
			\function[calculate, display, normalize]{x1}{a+dx1}
			\function[calculate,2]{x2}{x1+dx2}
			\function[calculate,2]{x3}{x2+dx3}
			\function[calculate,2]{x4}{x3+dx4}
			\function[calculate,2]{b}{x4}
			\function[calculate]{m1}{(dx1+dx2+|dx1-dx2|)/2} %=max(dx1,dx2)
			\function[calculate]{m2}{(dx3+dx4+|dx3-dx4|)/2}
			\function[calculate]{m}{(m1+m2+|m1-m2|)/2}  %=max(dx1,dx2,dx3,dx4)
			
		\end{variables}
	%+++++++++++++++++++TYPE+++++++++++++++++++++++++++	
		\type{input.number} %input.text %input.cases.function %input.finite-number-set %input.interval %...http://team.mumie.net/projects/support/wiki/DifferentAnswerType
		\field{real} 

		\precision{2}

	%+++++++++++++++++++TITLE++++++++++++++++++++++++++	
	    \lang{de}{\text{Für das Intervall $[\var{a};\var{b}]$ betrachten wir die
Zerlegung $\var{a}<\var{x1}<\var{x2}<\var{x3}<\var{b}$. Bestimmen Sie die Längen der
Teilintervalle und die Feinheit der Zerlegung. Die Längen der Teilintervalle sind dann.}}
	%+++++++++++++++++++ANSWERS++++++++++++++++++++++++    
	    \begin{answer}
	    	\text{$\Delta {x_1}=$}
			\solution{dx1}
            \explanation{Das erste Teilintervall ist $[\var{a};\var{x1}]$.}
	    \end{answer}
	    \begin{answer}
	    	\text{$\Delta {x_2}=$}
			\solution{dx2}
            \explanation{Das zweite Teilintervall ist $[\var{x1};\var{x2}]$.}
	    \end{answer}
	    \begin{answer}
	    	\text{$\Delta {x_3}=$}
			\solution{dx3}
            \explanation{Das dritte Teilintervall ist $[\var{x2};\var{x3}]$.}
	    \end{answer}
	    \begin{answer}
	    	\text{$\Delta {x_4}=$}
			\solution{dx4}
            \explanation{Das vierte Teilintervall ist $[\var{x3};\var{b}]$.}
	    \end{answer} 
	    \begin{answer}
	    	\text{Die Feinheit der Zerlegung beträgt:}
			\solution{m}
            \explanation{Die Feinheit ist gegeben durch $\max(\Delta x_1,\Delta x_2, \Delta x_3, \Delta x_4)$.}
	    \end{answer}
	\end{question}
%######################################################QUESTION_END

%######################################################QUESTION_START	
	\begin{question}	
	%+++++++++++++++++++VARIABLES++++++++++++++++++++++
		\begin{variables}
			\randint[Z]{a}{1}{4}
			\randint[Z]{d1}{1}{40}
			\randint[Z]{d2}{1}{40}
			\randint[Z]{d3}{1}{40}
			\randint[Z]{d4}{1}{40}
			\function[calculate,display]{dx1}{d1/20}
			\function[calculate,display]{dx2}{d2/20}
			\function[calculate,display]{dx3}{d3/20}
			\function[calculate,display]{dx4}{d4/20}
			\function[calculate]{x1}{a+dx1}
			\function[calculate]{x2}{x1+dx2}
			\function[calculate]{x3}{x2+dx3}
			\function[calculate]{x4}{x3+dx4}
			\function[calculate, display]{b}{x4}
			%\function{sol}{max(dx1,dx2,dx3,dx4)}
			
		\end{variables}
	%+++++++++++++++++++TYPE+++++++++++++++++++++++++++	
		\type{input.number} %input.text %input.cases.function %input.finite-number-set %input.interval %...http://team.mumie.net/projects/support/wiki/DifferentAnswerType
		\field{real} 
		\displayprecision{2} % <- dieser command gibt die Stellenzahl für numbers. Für functions muss bei
        % calculate noch "display" ergänzt werden, damit auf die Stellenzahl gerundet wird.
    	\correctorprecision{2}
	%+++++++++++++++++++TITLE++++++++++++++++++++++++++	
	    \lang{de}{\text{Für das Intervall $[\var{a};\var{b}]$ betrachten wir die
Zerlegung in $4$ Teilintervalle der Längen $\var{dx1}$, $\var{dx2}$, $\var{dx3}$ und $\var{dx4}$.
Bestimmen Sie die Teilungsstellen.
Diese sind\\
$ \var{a}< $\ansref $<$\ansref $<$\ansref $<\var{b}$.}}
	%+++++++++++++++++++ANSWERS++++++++++++++++++++++++    
	    \begin{answer}
%	    	\text{$a=$}
			\solution{x1}
            \explanation{Die Lösung ergibt sich durch Addition der ersten Länge zum linken Rand.}

	    \end{answer}
	    \begin{answer}
%	    	\text{$b=$}
			\solution{x2}
            \explanation{Die Lösung ergibt sich durch Addition der ersten beiden Längen zum linken Rand.}
            
	    \end{answer}
	    \begin{answer}
%	    	\text{$c=$}
			\solution{x3}
            \explanation{Die Lösung ergibt sich durch Addition der ersten drei Längen zum linken Rand.}
            
	    \end{answer}
	\end{question}
%######################################################QUESTION_END

\end{problem}

\embedmathlet{mathlet}

\end{content}