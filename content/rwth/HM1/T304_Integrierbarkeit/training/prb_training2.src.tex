\documentclass{mumie.problem.gwtmathlet}
%$Id$
\begin{metainfo}
  \name{
    \lang{de}{A02: Ober- und Untersummen}
    \lang{en}{}
  }
  \begin{description} 
 This work is licensed under the Creative Commons License Attribution 4.0 International (CC-BY 4.0)   
 https://creativecommons.org/licenses/by/4.0/legalcode 

    \lang{de}{}
    \lang{en}{}
  \end{description}
  \corrector{system/problem/GenericCorrector.meta.xml}
  \begin{components}
    \component{js_lib}{system/problem/GenericMathlet.meta.xml}{mathlet}
  \end{components}
  \begin{links}
  \end{links}
  \creategeneric
\end{metainfo}
\begin{content}
\usepackage{mumie.genericproblem}

\lang{de}{\title{A02: Ober- und Untersummen}}

\begin{block}[annotation]
	Im Ticket-System: \href{http://team.mumie.net/issues/10586}{Ticket 10586}
\end{block}

\begin{problem}
	\randomquestionpool{1}{3}
	
%######################################################QUESTION_START	
	\begin{question}	
	% Fall s\leq a
	%+++++++++++++++++++VARIABLES++++++++++++++++++++++
		\begin{variables}

		\randint[Z]{s}{-2}{2}
		\randint[Z]{c}{-2}{2}
		\randint{k}{0}{2}
		
		\randint[Z]{d1}{4}{12}
		\randint[Z]{d2}{4}{12}
		\randint[Z]{d3}{4}{12}
		\function[calculate,3]{dx1}{d1/10}
		\function[calculate,3]{dx2}{d2/10}
		\function[calculate,3]{dx3}{d3/10}
		
		\function[calculate,3]{a}{s+(2*k)/10}
		\function[calculate,3]{x1}{a+dx1}
		\function[calculate,3]{x2}{x1+dx2}
		\function[calculate,3]{b}{x2+dx3}
		
		\function[expand,normalize]{f}{(x-s)^2+c}
		
		\function[calculate,3]{fa}{(a-s)^2+c}
		\function[calculate,3]{fb}{(b-s)^2+c}
		\function[calculate,3]{fx1}{(x1-s)^2+c}
		\function[calculate,3]{fx2}{(x2-s)^2+c}
		
				
		\function[calculate,3]{s1}{fx1}
		\function[calculate,3]{s2}{fx2}
		\function[calculate,3]{s3}{fb}
		\function[calculate,3]{i1}{fa}
		\function[calculate,3]{i2}{fx1}
		\function[calculate,3]{i3}{fx2}
		
		
		\function[calculate,3]{U}{i1*dx1+i2*dx2+i3*dx3}
		\function[calculate,3]{O}{s1*dx1+s2*dx2+s3*dx3}

		\end{variables}
	%+++++++++++++++++++TYPE+++++++++++++++++++++++++++	
		\type{input.number} %input.text %input.cases.function %input.finite-number-set %input.interval %...http://team.mumie.net/projects/support/wiki/DifferentAnswerType
		\field{rational} 
		\precision{3}
	%+++++++++++++++++++TITLE++++++++++++++++++++++++++	
	    \lang{de}{
	    \text{Für die Funktion $f(x)=\var{f}$ (s. Grafik) auf dem Intervall $[\var{a};\var{b}]$
sollen die Ober- und Untersumme bezüglich der Zerlegung 
$\var{a}<\var{x1}<\var{x2}<\var{b}$ berechnet werden.}}
	%+++++++++++++++++++ANSWERS++++++++++++++++++++++++    
	    \begin{answer}
	    	\text{
$\sup \{f(x) | x\in [\var{a};\var{x1}] \}=$}   
			\solution{s1}
	    \end{answer}    
	    \begin{answer}
	    	\text{
$\inf \{f(x) | x\in [\var{a};\var{x1}] \}=$} 
			\solution{i1}
	    \end{answer}   
	    \begin{answer}
	    	\text{
$\sup \{f(x) | x\in [\var{x1};\var{x2}] \}=$}   
			\solution{s2}
	    \end{answer}    
	    \begin{answer}
	    	\text{
$\inf \{f(x) | x\in [\var{x1};\var{x2}] \}=$} 
			\solution{i2}
	    \end{answer}   
	    \begin{answer}
	    	\text{
$\sup \{f(x) | x\in [\var{x2};\var{b}] \}=$}   
			\solution{s3}
	    \end{answer}    
	    \begin{answer}
	    	\text{
$\inf \{f(x) | x\in [\var{x2};\var{b}] \}=$} 
			\solution{i3}
	    \end{answer}   
	    
	    \begin{answer}
	    	\text{Die Längen der Teilintervalle  sind\\
$\Delta x_1=$}
			\solution{dx1}
            \explanation{Gefragt ist nach der Länge von $[\var{a};\var{x1}]$.}
	    \end{answer} 
        \begin{answer}
	    	\text{$\Delta x_2=$}
			\solution{dx2}
            \explanation{Gefragt ist nach der Länge von $[\var{x1};\var{x2}]$.}
	    \end{answer}
        \begin{answer}
	    	\text{$\Delta x_3=$}
			\solution{dx3}
            \explanation{Gefragt ist nach der Länge von $[\var{x2};\var{b}]$.}
	    \end{answer}
	    \begin{answer}
		    %\type{input.number} %input.text %input.cases.function %input.finite-number-set %input.interval %...http://team.mumie.net/projects/support/wiki/DifferentAnswerType
	    	\text{Die Obersumme ist damit: $O=$}
			\solution{O}
	    \end{answer}  
	    \begin{answer}
	    	\text{Die Untersumme ist damit: $U=$}
			\solution{U}
	    \end{answer}  
	    \explanation{Die Funktion $f$ ist auf $[\var{a};\var{b}]$ monoton steigend.}
        
	    \plotF{1}{f}           %%domain
        \plotFrom{1}{-4}
        \plotTo{1}{6}
        \plotColor{4}{red}      %%domain color

        \plotLeft{-4}            %%defines the canvas bound left
        \plotRight{6}           %%and right
        \plotSize{400}          %%function plot canvas 
	    
	    
	\end{question}
%######################################################QUESTION_END
	
%######################################################QUESTION_START	
	\begin{question}	
	% Fall a<s< x1
    %+++++++++++++++++++VARIABLES++++++++++++++++++++++
		\begin{variables}
% $s,c\in \{-2,-1,1,2\}$ und $f=(x-s)^2+c$ (mit expand,normalize)\\
% $d_i\in \{4,\ldots, 12\}$ und $dx_i=d_i/10$ für $i=1,2,3$.\\
% $O=sup_1\cdot dx_1+sup_2\cdot dx_2+ sup_3\cdot dx_3$\\
% $U=inf_1\cdot dx_1+inf_2\cdot dx_2+ inf_3\cdot dx_3$\\
		\randint[Z]{s}{-2}{2}
		\randint[Z]{c}{-2}{2}
		\randint[Z]{k}{0}{2}
		
		\randint[Z]{d1}{2}{6}
		\randint[Z]{d2}{4}{12}
		\randint[Z]{d3}{4}{12}
		\function[calculate,3]{dx1}{d1/5}
		\function[calculate,3]{dx2}{d2/10}
		\function[calculate,3]{dx3}{d3/10}
		
		\function[calculate,3]{a}{s-dx1/2}
		\function[calculate,3]{x1}{a+dx1}
		\function[calculate,3]{x2}{x1+dx2}
		\function[calculate,3]{b}{x2+dx3}
		
		\function[expand,normalize]{f}{(x-s)^2+c}
		
		\function[calculate,3]{fa}{(a-s)^2+c}
		\function[calculate,3]{fb}{(b-s)^2+c}
		\function[calculate,3]{fx1}{(x1-s)^2+c}
		\function[calculate,3]{fx2}{(x2-s)^2+c}

		\function[calculate,3]{s1}{(fa+fx1+|fa-fx1|)/2} %=max(fa,fx1)
		\function[calculate,3]{s2}{fx2}
		\function[calculate,3]{s3}{fb}
		\function[calculate,3]{i1}{c}
		\function[calculate,3]{i2}{fx1}
		\function[calculate,3]{i3}{fx2}

		
		
		\function[calculate,3]{O}{s1*dx1+s2*dx2+s3*dx3}
		\function[calculate,3]{U}{i1*dx1+i2*dx2+i3*dx3}

		\end{variables}
	%+++++++++++++++++++TYPE+++++++++++++++++++++++++++	
		\type{input.number} %input.text %input.cases.function %input.finite-number-set %input.interval %...http://team.mumie.net/projects/support/wiki/DifferentAnswerType
		\field{rational} 
		\precision{3}
	%+++++++++++++++++++TITLE++++++++++++++++++++++++++	
	    \lang{de}{
	    \text{Für die Funktion $f(x)=\var{f}$ (s. Grafik) auf dem Intervall $[\var{a};\var{b}]$
sollen die Ober- und Untersumme bezüglich der Zerlegung 
$\var{a}<\var{x1}<\var{x2}<\var{b}$ berechnet werden.}}
	%+++++++++++++++++++ANSWERS++++++++++++++++++++++++    
		    \begin{answer}
	    	\text{
$\sup \{f(x) | x\in [\var{a};\var{x1}] \}=$}   
			\solution{s1}
	    \end{answer}    
	    \begin{answer}
	    	\text{
$\inf \{f(x) | x\in [\var{a};\var{x1}] \}=$} 
			\solution{i1}
	    \end{answer}   
	    \begin{answer}
	    	\text{
$\sup \{f(x) | x\in [\var{x1};\var{x2}] \}=$}   
			\solution{s2}
	    \end{answer}    
	    \begin{answer}
	    	\text{
$\inf \{f(x) | x\in [\var{x1};\var{x2}] \}=$} 
			\solution{i2}
	    \end{answer}   
	    \begin{answer}
	    	\text{
$\sup \{f(x) | x\in [\var{x2};\var{b}] \}=$}   
			\solution{s3}
	    \end{answer}    
	    \begin{answer}
	    	\text{
$\inf \{f(x) | x\in [\var{x2};\var{b}] \}=$} 
			\solution{i3}
	    \end{answer}   
	    
	   \begin{answer}
	    	\text{Die Längen der Teilintervalle  sind\\
$\Delta x_1=$}
			\solution{dx1}
            \explanation{Gefragt ist nach der Länge von $[\var{a};\var{x1}]$.}
	    \end{answer} 
        \begin{answer}
	    	\text{$\Delta x_2=$}
			\solution{dx2}
            \explanation{Gefragt ist nach der Länge von $[\var{x1};\var{x2}]$.}
	    \end{answer}
        \begin{answer}
	    	\text{$\Delta x_3=$}
			\solution{dx3}
            \explanation{Gefragt ist nach der Länge von $[\var{x2};\var{b}]$.}
	    \end{answer}
	    \begin{answer}
		    %\type{input.number} %input.text %input.cases.function %input.finite-number-set %input.interval %...http://team.mumie.net/projects/support/wiki/DifferentAnswerType
	    	\text{Die Obersumme ist damit: $O=$}
			\solution{O}
	    \end{answer}  
	    \begin{answer}
	    	\text{Die Untersumme ist damit: $U=$}
			\solution{U}
	    \end{answer}  
	    
        \explanation{Die Funktion $f$ ist auf $[\var{a};\var{s}]$ monoton fallend und auf $[\var{s};\var{b}]$ monoton steigend.}
        
	    \plotF{1}{f}           %%domain
        \plotFrom{1}{-4}
        \plotTo{1}{6}
        \plotColor{4}{red}      %%domain color

        \plotLeft{-4}            %%defines the canvas bound left
        \plotRight{6}           %%and right
        \plotSize{400}          %%function plot canvas 
	\end{question}
%######################################################QUESTION_END
	
%######################################################QUESTION_START	
	\begin{question}	
   	% Fall x1<s< x2
	%+++++++++++++++++++VARIABLES++++++++++++++++++++++
		\begin{variables}
% $s,c\in \{-2,-1,1,2\}$ und $f=(x-s)^2+c$ (mit expand,normalize)\\
% $d_i\in \{4,\ldots, 12\}$ und $dx_i=d_i/10$ für $i=1,2,3$.\\
% $O=sup_1\cdot dx_1+sup_2\cdot dx_2+ sup_3\cdot dx_3$\\
% $U=inf_1\cdot dx_1+inf_2\cdot dx_2+ inf_3\cdot dx_3$\\
		\randint[Z]{s}{-2}{2}
		\randint[Z]{c}{-2}{2}
		\randint[Z]{k}{0}{2}
		
		\randint[Z]{d1}{4}{12}
		\randint[Z]{d2}{2}{6}
		\randint[Z]{d3}{4}{12}
		\function[calculate,3]{dx1}{d1/10}
		\function[calculate,3]{dx2}{d2/5}
		\function[calculate,3]{dx3}{d3/10}
		
		\function[calculate,3]{a}{s-dx2/2-dx1}
		\function[calculate,3]{x1}{a+dx1}
		\function[calculate,3]{x2}{x1+dx2}
		\function[calculate,3]{b}{x2+dx3}
		
		\function[expand,normalize]{f}{(x-s)^2+c}
		
		\function[calculate,3]{fa}{(a-s)^2+c}
		\function[calculate,3]{fb}{(b-s)^2+c}
		\function[calculate,3]{fx1}{(x1-s)^2+c}
		\function[calculate,3]{fx2}{(x2-s)^2+c}
		

		\function[calculate,3]{s1}{fa}
		\function[calculate,3]{s2}{(fx1+fx2+|fx1-fx2|)/2} %=max(fx1,fx2)
		\function[calculate,3]{s3}{fb}
		\function[calculate,3]{i1}{fx1}
		\function[calculate,3]{i2}{c}
		\function[calculate,3]{i3}{fx2}
		
		
		\function[calculate,3]{O}{s1*dx1+s2*dx2+s3*dx3}
		\function[calculate,3]{U}{i1*dx1+i2*dx2+i3*dx3}

		\end{variables}
	%+++++++++++++++++++TYPE+++++++++++++++++++++++++++	
		\type{input.number} %input.text %input.cases.function %input.finite-number-set %input.interval %...http://team.mumie.net/projects/support/wiki/DifferentAnswerType
		\field{rational} 
		\precision{3}
	%+++++++++++++++++++TITLE++++++++++++++++++++++++++	
	    \lang{de}{
	    \text{Für die Funktion $f(x)=\var{f}$ (s. Grafik) auf dem Intervall $[\var{a};\var{b}]$
sollen die Ober- und Untersumme bezüglich der Zerlegung 
$\var{a}<\var{x1}<\var{x2}<\var{b}$ berechnet werden.}}
	%+++++++++++++++++++ANSWERS++++++++++++++++++++++++    
	    \begin{answer}
	    	\text{
$\sup \{f(x) | x\in [\var{a};\var{x1}] \}=$}   
			\solution{s1}
	    \end{answer}    
	    \begin{answer}
	    	\text{
$\inf \{f(x) | x\in [\var{a};\var{x1}] \}=$} 
			\solution{i1}
	    \end{answer}   
	    \begin{answer}
	    	\text{
$\sup \{f(x) | x\in [\var{x1};\var{x2}] \}=$}   
			\solution{s2}
	    \end{answer}    
	    \begin{answer}
	    	\text{
$\inf \{f(x) | x\in [\var{x1};\var{x2}] \}=$} 
			\solution{i2}
	    \end{answer}   
	    \begin{answer}
	    	\text{
$\sup \{f(x) | x\in [\var{x2};\var{b}] \}=$}   
			\solution{s3}
	    \end{answer}    
	    \begin{answer}
	    	\text{
$\inf \{f(x) | x\in [\var{x2};\var{b}] \}=$} 
			\solution{i3}
	    \end{answer}   
	    
	    \begin{answer}
	    	\text{Die Längen der Teilintervalle  sind\\
$\Delta x_1=$}
			\solution{dx1}
            \explanation{Gefragt ist nach der Länge von $[\var{a};\var{x1}]$.}
	    \end{answer} 
        \begin{answer}
	    	\text{$\Delta x_2=$}
			\solution{dx2}
            \explanation{Gefragt ist nach der Länge von $[\var{x1};\var{x2}]$.}
	    \end{answer}
        \begin{answer}
	    	\text{$\Delta x_3=$}
			\solution{dx3}
            \explanation{Gefragt ist nach der Länge von $[\var{x2};\var{b}]$.}
	    \end{answer}
	    \begin{answer}
		    %\type{input.number} %input.text %input.cases.function %input.finite-number-set %input.interval %...http://team.mumie.net/projects/support/wiki/DifferentAnswerType
	    	\text{Die Obersumme ist damit: $O=$}
			\solution{O}
	    \end{answer}  
	    \begin{answer}
	    	\text{Die Untersumme ist damit: $U=$}
			\solution{U}
	    \end{answer}
        
        \explanation{Die Funktion $f$ ist auf $[\var{a};\var{s}]$ monoton fallend und auf $[\var{s};\var{b}]$ monoton steigend.}
	    
	    \plotF{1}{f}           %%domain
        \plotFrom{1}{-4}
        \plotTo{1}{6}
        \plotColor{4}{red}      %%domain color

        \plotLeft{-4}            %%defines the canvas bound left
        \plotRight{6}           %%and right
        \plotSize{400}          %%function plot canvas 
        
	\end{question}
%######################################################QUESTION_END
\end{problem}

\embedmathlet{mathlet}

\end{content}