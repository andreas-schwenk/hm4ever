\documentclass{mumie.problem.gwtmathlet}
%$Id$
\begin{metainfo}
  \name{
    \lang{de}{A05: Integral bestimmen}
    \lang{en}{}
  }
  \begin{description} 
 This work is licensed under the Creative Commons License Attribution 4.0 International (CC-BY 4.0)   
 https://creativecommons.org/licenses/by/4.0/legalcode 

    \lang{de}{}
    \lang{en}{}
  \end{description}
  \corrector{system/problem/GenericCorrector.meta.xml}
  \begin{components}
    \component{js_lib}{system/problem/GenericMathlet.meta.xml}{mathlet}
  \end{components}
  \begin{links}
  \end{links}
  \creategeneric
\end{metainfo}
\begin{content}
\usepackage{mumie.ombplus}
\usepackage{mumie.genericproblem}


\lang{de}{
	\title{A05: Integral bestimmen}
}
\begin{block}[annotation]
  
      
\end{block}
\begin{block}[annotation]
  Im Ticket-System: \href{http://team.mumie.net/issues/10589}{Ticket 10589}
\end{block}


\begin{problem}

  \begin{variables}
	\randint{b}{-8}{-1}
	\randint{z}{2}{4}
    \randint{w}{1}{3}
	\randint{c}{1}{8}
	\randint{d}{1}{2}
    \randint{v}{-5}{5}
	\function[calculate,3]{m}{c/d}
	\function[normalize]{f}{m*x+b}
    \function[calculate,3]{a1}{0}
    \function[calculate,3]{a2}{z^2}
    \function[calculate,3]{a3}{(z+w)^2}
    \function[calculate,3]{z2}{z-1}
    \function[normalize]{g}{sqrt(x)}
	\function{L1}{m/2*a2^2+b*a2}
    \function{L2}{2/3*(a3*(z+w)-a2*z)}
    \function[calculate]{L}{L1+L2}
  \end{variables}

  \begin{question}
    \text{Wir betrachten die Funktion\\
    $f: [0;\var{a3}]\to \R, x \mapsto \begin{cases}\var{f}, & 0\leq x < \var{a2}, \\ \var{v}, & x = \var{a2}, \\ \var{g}, & x > \var{a2}.\end{cases}$\\
    Bestimmen Sie $\int_{0}^{\var{a3}} f(x) \, dx$.}
    \type{input.number}
    \field{rational}

    \explanation{Diese stückweise Funktion integriert man, indem man $\int_0^{\var{a2}} \var{f}\, dx + \int_{\var{a2}}^{\var{a3}} \var{g} \, dx$ rechnet.}
    
	\begin{answer}
      \lang{de}{ \text{Lösung: }}
       \lang{en}{\text{Solution: }}
	 \solution{L}
	\end{answer}

  \end{question}



\end{problem}


\embedmathlet{mathlet}



\end{content}