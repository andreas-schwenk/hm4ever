\documentclass{mumie.problem.gwtmathlet}
%$Id$
\begin{metainfo}
  \name{
    \lang{de}{A04: Integrierbarkeit}
    \lang{en}{Aufgabe 4}
  }
  \begin{description} 
 This work is licensed under the Creative Commons License Attribution 4.0 International (CC-BY 4.0)   
 https://creativecommons.org/licenses/by/4.0/legalcode 

    \lang{de}{Beschreibung}
    \lang{en}{description}
  \end{description}
  \corrector{system/problem/GenericCorrector.meta.xml}
  \begin{components}
    \component{js_lib}{system/problem/GenericMathlet.meta.xml}{gwtmathlet}
  \end{components}
  \begin{links}
  \end{links}
  \creategeneric
\end{metainfo}
\begin{content}
\usepackage{mumie.genericproblem}


\begin{block}[annotation]
	Im Ticket-System: \href{http://team.mumie.net/issues/10588}{Ticket 10588}
\end{block}

\lang{de}{
	\title{A04: Integrierbarkeit}
}

\begin{problem}
% 	

     \begin{question}
     
     \begin{variables}
%      $a1\in \{ 0,\ldots, 3\}$, $b1\in \{ 5,\ldots, 10\}$, $c1\in \{-3,\ldots, 3\}\setminus \{0\}$, 
% $c2\in \{0,\ldots, 8\}$\\
% $f1=\frac{c1}{x^2-c_2^2}$, $con1=(a1-c2)(b1-c2)$.
% 
% $i\in \{-3,\ldots, 3\}$, $j\in \{ 1,\ldots, 4\}$, $a2=\frac{i\cdot \pi}{2}$, $b2=\frac{(i+j)\cdot \pi}{2}$,\\
% $m=(a2+b2)/2$.
% 
% $b3 \in \{ 1,\ldots, 6\}$, $a3=-b3$, $c0\in \{-3,\ldots, 3\}\setminus \{0\}$, $c=b+c0$,\\
% $k\in \{0,1\}$, $con3=k\cdot c0$,\\
% $f3=k\cdot \ln(|x-c|)+(1-k)\cdot \exp(-1/|x-c|)$.
			\randint{a1}{0}{3}
			\randint[Z]{b1}{5}{10}
			\randint[Z]{c1}{-3}{3}

			\randint{c2}{0}{8}
			\function[expand,normalize]{f1}{c1/(x^2-c2^2)}
			\function[calculate]{con1}{(a1-c2)*(b1-c2)}

			
			\randint{i}{-3}{3}
			\randint[Z]{j}{1}{4}
			\function[normalize]{a2}{(i*pi)/2}
			\function[normalize]{b2}{((i+j)*pi)/2}
			\function[normalize]{m}{(a2+b2)/2}
			
			\randint[Z]{b3}{3}{6}
			\function[calculate]{a3}{-b3}
			\randint[Z]{c0}{-3}{3}

			\function[calculate]{c}{b3+c0}
			\randint{k}{0}{1}
			\function[calculate]{con3}{k*c0}

			\function[expand,normalize]{f3}{k*ln(abs(x-c))+(1-k)*(e^(-1/abs(x-c)))}
     
     
     
     \end{variables}
     \field{rational}
      
     \lang{de}{ 
      	\text{Existieren die folgenden Integrale?}
      }
    	\type{mc.yesno}
        \permutechoices{1}{3}
		\begin{choice}
  			\text{$\int_{\var{a1}}^{\var{b1}} \var{f1} \, dx$} % Antwort ist ja, wenn $\var{con1}>0$.
  			\solution{compute}
  			 \iscorrect{con1}{>}{0}
             \explanation[con1 > 0]{Die Definitionslücke der rationalen Funktion liegt außerhalb des Integrationsbereichs. Auf dem Integrationsbereich ist der Integrand beschränkt und stetig.}
             \explanation[con1 <= 0]{Die Definitionslücke der rationalen Funktion liegt innerhalb des Integrationsbereichs.}
		\end{choice}
		\begin{choice}
			\text{$\int_{\var{a2}}^{\var{b2}} f(x)\, dx$ mit $f(x)=\left\{ \begin{mtable}[\cellaligns{cc}] \cos(x) & \var{a2}\leq x\leq \var{m}, \\
\sin(x) \quad & \var{m}< x\leq \var{b2}. 
\end{mtable} \right.$}
  			\solution{true}
            \explanation{Die Funktion $f$ ist beschränkt und stückweise stetig.}
		\end{choice}
		\begin{choice}
			\text{$\int_{\var{a3}}^{\var{b3}} g(x)\, dx$ mit $g(x)=\left\{ \begin{mtable}[\cellaligns{cc}] \var{f3} & x\neq \var{c}, \\
0 \quad & x=\var{c}. 
\end{mtable} \right.$}
  			\solution{compute}
  			 \iscorrect{con3}{>=}{0}  %Antwort ist ja, wenn $\var{con3}\geq 0$.
             \explanation[con3 >= 0]{Die Funktion $g$ ist beschränkt und stetig auf dem Integrationsbereich.}
             \explanation[con3 < 0]{Die Funktion $g$ hat zwar keine Definitionslücke, sie ist aber unbeschränkt.}
		\end{choice}
    \end{question}

\end{problem}

\embedmathlet{gwtmathlet}

\end{content}