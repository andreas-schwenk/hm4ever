\documentclass{mumie.problem.gwtmathlet}
%$Id$
\begin{metainfo}
  \name{
    \lang{de}{A10: Uneigentliches Integral}
    \lang{en}{}
  }
  \begin{description} 
 This work is licensed under the Creative Commons License Attribution 4.0 International (CC-BY 4.0)   
 https://creativecommons.org/licenses/by/4.0/legalcode 

    \lang{de}{}
    \lang{en}{}
  \end{description}
  \corrector{system/problem/GenericCorrector.meta.xml}
  \begin{components}
    \component{js_lib}{system/problem/GenericMathlet.meta.xml}{mathlet}
  \end{components}
  \begin{links}
  \end{links}
  \creategeneric
\end{metainfo}
\begin{content}
\usepackage{mumie.genericproblem}

\lang{de}{\title{A10: Uneigentliches Integral}}

\begin{block}[annotation]
	Im Ticket-System: \href{http://team.mumie.net/issues/10644}{Ticket 10644}
\end{block}

\begin{problem}
	\randomquestionpool{1}{2}
	

	\begin{question}	
	%+++++++++++++++++++VARIABLES++++++++++++++++++++++
		\begin{variables}
			\randint[Z]{a}{-2}{2}
			\randint{k}{1}{3}
			\randint[Z]{c}{-6}{6}
			\randint{l}{1}{20}
			\function[normalize]{f}{x-a}
			\function[calculate]{b}{a+k}
			\function[calculate,1]{n}{l/5}
			\function[calculate,1]{n1}{n+1}
            \function[calculate]{cm}{-c}
			
			%\function{sol}{-c/(n*k^n)}
			\function{sol}{infinity}
		\end{variables}

		
	%+++++++++++++++++++TYPE+++++++++++++++++++++++++++	
		\type{input.function} %input.text %input.cases.function %input.finite-number-set %input.interval %...http://team.mumie.net/projects/support/wiki/DifferentAnswerType
		\field{real} 
		\precision{1}
	%+++++++++++++++++++TITLE++++++++++++++++++++++++++	
	    \lang{de}{\text{
			Bestimmen Sie das folgende uneigentliche Integral, sofern es existiert (als exakten Ausdruck oder auf 2 Stellen
            nach dem Komma gerundet). 
			Falls es nicht existiert, geben Sie \textit{infinity} ein.\\
$\int_{\var{a}}^{\var{b}} \frac{\var{c}}{(\var{f})^{\var{n1}}} dx=$ \ansref \\

Hinweis: Eine Stammfunktion des Integranden ist 
$\frac{\var{cm}}{\var{n}(\var{f})^{\var{n}}}$.
		}}
	%+++++++++++++++++++ANSWERS++++++++++++++++++++++++    
	    \begin{answer}
			\solution{sol}
			\checkAsFunction{x}{-10}{10}{10}
	    \end{answer}    
	     \explanation{Bezeichnet man die angegebene Stammfunktion mit $F$, so ist das uneigentliche Integral gleich 
	    $\lim_{a\searrow \var{a}} ( F(\var{b})-F(a) )=F(\var{b}) -\lim_{a\searrow \var{a}} F(a)$, falls der Grenzwert existiert,
	    und ansonsten existiert das Integral nicht.
	    }
	\end{question}
	
	\begin{question}	
	%+++++++++++++++++++VARIABLES++++++++++++++++++++++
		\begin{variables}
			\randint[Z]{a}{-2}{2}
			\randint{k}{1}{3}
			\randint{l}{-4}{-1}
			\function[normalize]{f}{x-a}
			\randint[Z]{c}{-6}{6}
			\function[calculate]{b}{a+k}
			\function[calculate,1]{n}{l/5}
			\function[calculate,1]{n1}{n+1}
			\function[calculate,1]{cm}{-c}
			\function[normalize]{sol}{-c/(n*k^n)}
			%\string{sol}{infinity}
		\end{variables}
    
		
	%+++++++++++++++++++TYPE+++++++++++++++++++++++++++	
		\type{input.function} %input.text %input.cases.function %input.finite-number-set %input.interval %...http://team.mumie.net/projects/support/wiki/DifferentAnswerType
		\field{real} 
		\precision{1}
	%+++++++++++++++++++TITLE++++++++++++++++++++++++++	
	    \lang{de}{\text{
			Bestimmen Sie das folgende uneigentliche Integral, sofern es existiert (als exakten Ausdruck oder auf 2 Stellen
            nach dem Komma gerundet). 
			Falls es nicht existiert, geben Sie \textit{infinity} ein.\\
$\int_{\var{a}}^{\var{b}} \frac{\var{c}}{(\var{f})^{\var{n1}}} dx=$ \ansref \\

Hinweis: Eine Stammfunktion des Integranden ist 
$\frac{\var{cm}}{\var{n}(\var{f})^{\var{n}}}$.
		}}
	%+++++++++++++++++++ANSWERS++++++++++++++++++++++++    
	    \begin{answer}
			\solution{sol}
			\checkAsFunction[1E-2]{x}{-10}{10}{10}
	    \end{answer}    
	    \explanation{Bezeichnet man die angegebene Stammfunktion mit $F$, so ist das uneigentliche Integral gleich 
	    $\lim_{a\searrow \var{a}} ( F(\var{b})-F(a) )=F(\var{b}) -\lim_{a\searrow \var{a}} F(a)$, falls der Grenzwert existiert,
	    und ansonsten existiert das Integral nicht.
	    }
	\end{question}

\end{problem}

\embedmathlet{mathlet}

\end{content}