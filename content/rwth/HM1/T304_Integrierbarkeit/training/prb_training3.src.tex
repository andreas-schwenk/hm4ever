\documentclass{mumie.problem.gwtmathlet}
%$Id$
\begin{metainfo}
  \name{
    \lang{de}{A03: Zwischensumme}
    \lang{en}{}
  }
  \begin{description} 
 This work is licensed under the Creative Commons License Attribution 4.0 International (CC-BY 4.0)   
 https://creativecommons.org/licenses/by/4.0/legalcode 

    \lang{de}{}
    \lang{en}{}
  \end{description}
  \corrector{system/problem/GenericCorrector.meta.xml}
  \begin{components}
    \component{js_lib}{system/problem/GenericMathlet.meta.xml}{mathlet}
  \end{components}
  \begin{links}
  \end{links}
  \creategeneric
\end{metainfo}
\begin{content}
\usepackage{mumie.genericproblem}

\lang{de}{\title{A03: Zwischensumme}}

\begin{block}[annotation]
	Im Ticket-System: \href{http://team.mumie.net/issues/10587}{Ticket 10587}
\end{block}

\begin{problem}
	%\randomquestionpool{}{}
	
%######################################################QUESTION_START	
	%+++++++++++++++++++VARIABLES++++++++++++++++++++++
		\begin{variables}
			\randint[Z]{n}{2}{4}
			\randint[Z]{a}{1}{4}
			\randint[Z]{s}{1}{4}
			\randint[Z]{c}{1}{4}
			\randint[Z]{b}{5}{8}
			\function[calculate]{dx}{(b-a)/n}
			\function[calculate]{k1}{1- (n-3)(n-4)/2} %k1=0, falls n=2    und =1, falls n=3, 4
			\function[calculate]{k2}{(n-2)(n-3)/2}   % k2=0, falls n=2, 3 und =1, falls n=4
			\function[calculate]{a1}{a}
			\function[calculate]{a2}{a+dx}
			\function[calculate]{a3}{a2+k1*dx}
			\function[calculate]{a4}{a3+k2*dx}
			\function[expand, normalize]{f}{c/(x+s)}
			\function[calculate]{f1}{c/(a1+s)}
			\function[calculate]{f2}{c/(a2+s)}
			\function[calculate]{f3}{k1*(c/(a3+s))}
			\function[calculate]{f4}{k2*(c/(a4+s))}
			\function[calculate,2]{S}{(f1+f2+f3+f4)*dx}
			%\randadjustIf{}{}
		\end{variables}
	%+++++++++++++++++++TYPE+++++++++++++++++++++++++++	
	\begin{question}	
		\type{input.generic} %input.text %input.cases.function %input.finite-number-set %input.interval %...http://team.mumie.net/projects/support/wiki/DifferentAnswerType
		\field{rational} 
%		\precision{2}
	%+++++++++++++++++++TITLE++++++++++++++++++++++++++	
	    \lang{de}{\text{Für die Funktion $f(x)=\var{f}$ auf dem Intervall $[\var{a};\var{b}]$
soll eine Zwischensumme zur äquidistanten Zerlegung in $\var{n}$ Teilintervalle
berechnet werden. Als Zwischenstellen für die Funktionswerte sollen die unteren Grenzen
der Teilintervalle gewählt werden.}}
	%+++++++++++++++++++ANSWERS++++++++++++++++++++++++    
	    \begin{answer}
		\type{input.number} %input.text %input.cases.function %input.finite-number-set %input.interval %...http://team.mumie.net/projects/support/wiki/DifferentAnswerType
	    	\text{Die Längen der Teilintervalle sind also sämtlich $\Delta x=$}
			\solution{dx}
	    \end{answer}    
	    \begin{answer}
	    	\type{input.finite-number-set}
	    	\text{und die Zerlegungsstellen sind:}
			\solution{a1,a2,a3,a4,b}
            \explanation{Die Anfangsstelle $\var{a}$ und die letzte Stelle $\var{b}$ sind ebenfalls Zerlegungsstellen. Insgesamt gibt es
            eine Zerlegungsstelle mehr als Teilintervalle.}
	    \end{answer}  
	\end{question}
        
	\begin{question}	
	    	\type{input.number}
            \field{real} 
    		\precision{2}
        \text{Damit ist die Zwischensumme $S$ (auf zwei Stellen nach dem Komma gerundet):\\ $S=$ \ansref}
	    \begin{answer}
        \explanation{Die Zwischensumme berechnet sich hier zu $\sum_{j=1}^{\var{n}} \Delta x f(x_j)$, wobei $x_j$ jeweils die untere Grenze des Teilintervalls ist.}
	    	
			\solution{S}
	    \end{answer}    
	\end{question}
%######################################################QUESTION_END

\end{problem}

\embedmathlet{mathlet}

\end{content}