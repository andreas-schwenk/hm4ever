\documentclass{mumie.problem.gwtmathlet}
%$Id$
\begin{metainfo}
  \name{
    \lang{de}{A08: Hauptsatz der Integralrechnung}
    \lang{en}{problem 8}
  }
  \begin{description} 
 This work is licensed under the Creative Commons License Attribution 4.0 International (CC-BY 4.0)   
 https://creativecommons.org/licenses/by/4.0/legalcode 

    \lang{de}{}
    \lang{en}{}
  \end{description}
  \corrector{system/problem/GenericCorrector.meta.xml}
  \begin{components}
    \component{js_lib}{system/problem/GenericMathlet.meta.xml}{mathlet}
  \end{components}
  \begin{links}
  \end{links}
  \creategeneric
\end{metainfo}
\begin{content}
\usepackage{mumie.ombplus}
\usepackage{mumie.genericproblem}


\lang{de}{
	\title{A08: Hauptsatz der Integralrechnung}
}


\begin{block}[annotation]
  Im Ticket-System: \href{http://team.mumie.net/issues/10592}{Ticket 10592}
\end{block}

\begin{problem}
	\begin{variables}
	 	\randint[Z]{a}{-5}{3}
	 	\randint[Z]{b}{-5}{4}
	 	\randint[Z]{c}{-3}{5}
	 	\randint[Z]{d}{-4}{5}
	 	\randint[Z]{r}{-20}{20}

		\function{f}{a*sin(x)+d*cos(x)}
		\derivative[normalize]{f_1}{f}{x}
		\functionNormalize{f_1}
	%Da Pi nicht geparst wird, wird das Ergebnis simuliert.
	%Sinus(n*pi)=0 forall n. Kosinus(n*pi) = +-1. Daher (-1)^n...
 	 	\randint{n}{2}{8}
		\function[calculate]{hilf}{(-1)^n}
		\function[calculate]{L}{d*hilf-d}

	\end{variables}


	\begin{question}
		\lang{de}{
			\text{



			Gegeben sei die Funktion $f(x) = \var{f_1}$. \\
		      Bestimmen Sie eine Stammfunktion $F(x)$ von $f(x)$:}
		}
		 \lang{en}{\text{
      Given is the function $f(x) = \var{f_1}$. \\
      Find an antiderivate $F(x)$ of $f(x)$:
     }}
		\type{input.function}

		\begin{answer}
			\text{ $F(x) = $}
			\solution{f}
			\inputAsFunction{x}{inp}
			\checkFuncForZero{D[inp]-f_1}{-10}{10}{100}
		\end{answer}
       \lang{de}{ \explanation{Finden Sie für beide Summanden eine Stammfunktion und nutzen Sie dann die Summen- und Faktorregel.}}
       \lang{en}{ \explanation{Find an antiderivative for each summand and use sum and factor rules.}}
	\end{question}


	\begin{question}
	    \lang{de}{
			\text{Berechnen Sie nun $\int_{0}^{\var{n}\pi} f(x) \, dx$ mit Hilfe des Hauptsatzes der \\
			Differential- und Integralrechnung:}
	    }
	     \lang{en}{\text{
        Now calculate $\int_{0}^{\var{n}\pi} f(x) \, dx$ with the help of the
        fundamental theorem of calculus:
       }}
		\type{input.number}
		\begin{answer}
			\text{ $\int_{0}^{\var{n}\pi} f(x) \, dx = $ }
			\solution{L}
            \lang{de}{\explanation{Benutzen Sie Teil a).}}
            \lang{en}{\explanation{Use the result of a).}}
		\end{answer}

	\end{question}

\end{problem}


\embedmathlet{mathlet}

\end{content}