%$Id:  $
\documentclass{mumie.article}
%$Id$
\begin{metainfo}
  \name{
    \lang{de}{Überblick: Integrierbarkeit}
    \lang{en}{Overview: Intgrability}
  }
  \begin{description} 
 This work is licensed under the Creative Commons License Attribution 4.0 International (CC-BY 4.0)   
 https://creativecommons.org/licenses/by/4.0/legalcode 

    \lang{de}{Beschreibung}
    \lang{en}{Description}
  \end{description}
  \begin{components}
  \end{components}
  \begin{links}
\link{generic_article}{content/rwth/HM1/T107_Integralrechnung/g_art_T107_overview.meta.xml}{T107_overview}
\link{generic_article}{content/rwth/HM1/T304_Integrierbarkeit/g_art_content_10_uneigentliches_integral.meta.xml}{content_10_uneigentliches_integral}
\link{generic_article}{content/rwth/HM1/T304_Integrierbarkeit/g_art_content_09_integrierbare_funktionen.meta.xml}{content_09_integrierbare_funktionen}
\link{generic_article}{content/rwth/HM1/T304_Integrierbarkeit/g_art_content_08_integral_eigenschaften.meta.xml}{content_08_integral_eigenschaften}
\link{generic_article}{content/rwth/HM1/T304_Integrierbarkeit/g_art_content_07_ober_und_untersumme.meta.xml}{content_07_ober_und_untersumme}
\end{links}
  \creategeneric
\end{metainfo}
\begin{content}
\begin{block}[annotation]
	Im Ticket-System: \href{https://team.mumie.net/issues/30122}{Ticket 30122}
\end{block}





\begin{block}[annotation]
Im Entstehen: Überblicksseite für Kapitel  Integrierbarkeit
\end{block}

\usepackage{mumie.ombplus}
\ombchapter{1}
\title{\lang{de}{Überblick:  Integrierbarkeit}\lang{en}{Overview: Integrability}}



\begin{block}[info-box]
\lang{de}{\strong{Inhalt}}
\lang{en}{\strong{Contents}}


\lang{de}{
    \begin{enumerate}%[arabic chapter-overview]
   \item[3.1] \link{content_07_ober_und_untersumme}{Ober- und Untersummen}
   \item[3.2] \link{content_08_integral_eigenschaften}{Integrierbarkeit}
   \item[3.3] \link{content_09_integrierbare_funktionen}{Stammfunktion}
   \item[3.4] \link{content_10_uneigentliches_integral}{Uneigentliche Integrale}
   \end{enumerate}
}
\lang{en}{
    \begin{enumerate}%[arabic chapter-overview]
   \item[3.1] \link{content_07_ober_und_untersumme}{Upper and lower sums}
   \item[3.2] \link{content_08_integral_eigenschaften}{Integrability}
   \item[3.3] \link{content_09_integrierbare_funktionen}{Antiderivatives}
   \item[3.4] \link{content_10_uneigentliches_integral}{Improper integrals}
   \end{enumerate}
}%lang

\end{block}

\begin{zusammenfassung}

\lang{de}{
Hier hinterfüttern wir den naiven Integralbegriff einer Fläche zwischen Graph und $x$-Achse aus 
\link{T107_overview}{Teil 1 Kapitel 7} mit einer handfesten Theorie.
Mit Hilfe von Ober- und Untersummen gelangen wir über einen Grenzprozess zum Begriff des 
Riemann-Integrals und seinen Eigenschaften und können dingfestmachen, 
was eine integrierbare Funktion auf einem abgeschlossenen Intervall überhaupt ist.
Dieses Vorgehen hat nicht nur theoretische Bedeutung. Es ist die Keimzelle aller praktischen, 
numerischen Integrationsmethoden.
\\\\
Eine wichtige Klasse integrierbarer Funktionen sind die  stetigen Funktionen.
Für sie gilt der Hauptsatz der Differential- und Integralrechnung, der Integration als die Umkehrung der Differentitation beschreibt.
Der Begriff der Stammfunktion eröffnet neue Möglichkeiten, Integrale zu berechnen.
\\\\
Weiter beschäftigen wir uns mit Grenzprozessen von Integralen an den Rändern des Definitionsbereichs 
integrierbarer Funktionen und gelangen so zum Begriff des uneigentlichen Integrals.
}
\lang{en}{
Here we supplement the naive definition of an integral as the signed area bounded by the graph of a 
function and the $x$-axis, first given in \link{T107_overview}{Part 1 Chapter 7}, with some more 
rigorous theory. Using upper and lower sums and a limiting process, we define the Riemann integral 
of a function, and investigate its properties. In this way we can define what it means for a 
function to be integrable on a bounded interval. The construction of the Riemann integral is not 
only useful in theory, it is used for most forms of numerical integration.
\\\\
Continuous functions are an important class of integrable functions. They satisfy the 
fundamental theorem of calculus, which describes how integration relates to differentiation. 
Introducing antiderivatives allows us new methods for computing integrals.
\\\\
Finally we consider how the limits for integrals of integrable functions behave at the ends of the 
intervals on which they are defined, leading us to the definitions of improper integrals.
}


\end{zusammenfassung}

\begin{block}[info]\lang{de}{\strong{Lernziele}}
\lang{en}{\strong{Learning Goals}} 
\begin{itemize}[square]
\item \lang{de}{
      Sie berechnen Ober- und Untersummen von Funktionen und führen den Grenzprozess des 
      Riemann-Integrals durch
      }
      \lang{en}{
      Being able to compute upper and lower sums of functions and take a limit to compute 
      Riemann integral.
      }
\item \lang{de}{
      Sie kennen den Hauptsatz der Differential- und Integralrechnung und Stammfunktionen 
      und wenden dieses Wissen in einfachen Situationen an.
      }
      \lang{en}{
      Knowing the fundamental theorem of calculus and the definition of antiderivatives, and 
      being able to apply these in simple cases.
      }
\item \lang{de}{Sie besitzen einen Fundus von elementaren Stammfunktionen.}
      \lang{en}{Knowing some elementary antiderivatives for common functions.}
\item \lang{de}{
      Sie kennen den Begriff des uneigentlichen Integrals und können in einfachen Situationen 
      entscheiden, funktion uneigentlich integrierbar ist.
      }
      \lang{en}{
      Knowing the definitions of improper integrals and being able to decide in some simple cases 
      whether the improper integral of a function exists.
      }
\end{itemize}
\end{block}




\end{content}
