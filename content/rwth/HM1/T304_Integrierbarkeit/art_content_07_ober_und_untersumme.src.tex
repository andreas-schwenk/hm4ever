%$Id:  $
\documentclass{mumie.article}
%$Id$
\begin{metainfo}
  \name{
    \lang{de}{Ober- und Untersummen}
    \lang{en}{Upper and lower sums}
  }
  \begin{description} 
 This work is licensed under the Creative Commons License Attribution 4.0 International (CC-BY 4.0)   
 https://creativecommons.org/licenses/by/4.0/legalcode 

    \lang{de}{Beschreibung}
    \lang{en}{Description}
  \end{description}
  \begin{components}
    \component{generic_image}{content/rwth/HM1/images/g_tkz_T304_Integral_B.meta.xml}{T303_Integral_B}
    \component{generic_image}{content/rwth/HM1/images/g_tkz_T304_Integral_A.meta.xml}{T303_Integral_A}
    \component{generic_image}{content/rwth/HM1/images/g_tkz_T107_Integral_A.meta.xml}{T107_Integral_A}
    \component{generic_image}{content/rwth/HM1/images/g_img_00_Videobutton_schwarz.meta.xml}{00_Videobutton_schwarz}
  \end{components}
  \begin{links}
    \link{generic_article}{content/rwth/HM1/T204_Abbildungen_und_Funktionen/g_art_content_12_reelle_funktionen_monotonie.meta.xml}{reelle-funk}
    \link{generic_article}{content/rwth/HM1/T107_Integralrechnung/g_art_content_24_integral_als_flaeche.meta.xml}{def-integral}
    \link{generic_article}{content/rwth/HM1/T202_Reelle_Zahlen_axiomatisch/g_art_content_07_vollstaendigkeit.meta.xml}{supinf}
  \end{links}
  \creategeneric
\end{metainfo}
\begin{content}
\usepackage{mumie.ombplus}
\ombchapter{3}
\ombarticle{1}

\title{\lang{de}{Ober- und Untersummen}\lang{en}{Upper and lower sums}}
 
\begin{block}[annotation]
  
  
\end{block}
\begin{block}[annotation]
  Im Ticket-System: \href{http://team.mumie.net/issues/10036}{Ticket 10036}\\
\end{block}

\begin{block}[info-box]
\tableofcontents
\end{block}

\lang{de}{
In \link{def-integral}{Teil 1} hatten wir das Integral als Flächeninhalt mit Vorzeichen 
kennengelernt. Das bestimmte Integral $\big\int_a^b f(x)\, dx$ einer Funktion $f(x)$ gab den 
\emph{Flächeninhalt mit Vorzeichen} (orientierter Flächeninhalt) der Fläche an, die der Graph der 
Funktion mit der $x$-Achse zwischen den Integrationsgrenzen $a$ und $b$ einschließt, wobei die 
Flächenteile oberhalb der $x$-Achse \emph{positiv} und die Flächenteile unterhalb der $x$-Achse 
\emph{negativ} in das Integral eingehen.
}
\lang{en}{
In the following we will always assume that $a$ and $b$ are real numbers and $a<b$. We recall our 
definitions from \link{def-integral}{Part 1} of the course. The definite integral 
$\big\int_a^b f(x)\, dx$ of a function $f(x)$ gives us the \emph{signed area} of the 
function bounded by its graph between the lower and upper bounds, $a$ and $b$ respectively. Areas 
above the $x$-axis count as \emph{positive} area, and areas below the $x$-axis count as \emph{negative} area.
}

\begin{center}
\image{T107_Integral_A}
\end{center}

\lang{de}{
Weist der Funktionsgraph jedoch Sprünge auf, müsste man sich noch senkrechte Verbindungslinien 
denken, um eine geschlossen umrahmte Fläche zu bekommen. Auch kann es bei sehr "`sprunghaften"' 
Funktionsgraphen (z.\,B. zur 
\ref[reelle-funk][Dirichletschen Sprungfunktion]{ex:unstetige-funktionen}) vorkommen, dass man keine 
richtige Fläche erhält. Wir benötigen daher eine mathematische Definition des Integrals, die in 
diesem und dem folgenden Abschnitt eingeführt werden. Bei "`schönen"' Funktionen stimmt dies dann 
wieder mit der naiven Anschauung als eingeschlossene Fläche überein.
\\\\
Die grobe Idee ist, dass das Integral durch Summation von Rechtecksflächen mit Vorzeichen entsteht, 
welche den gesuchten orientierten Flächeninhalt approximieren. Hierzu benötigt man zunächst 
\textit{Zerlegungen} des Intervalls.
}
\lang{en}{
If the graph of the function being integrated has 'jumps', we can use vertical lines to bound the 
area for integration. In the case of graphs with very many 'jumps' (for example the 
\ref[reelle-funk][Dirichlet function]{ex:unstetige-funktionen}), even this strategy may not work. 
Hence we require a rigorous definition of an integral, which we introduce in this section and the 
next. This should correspond to our previous naive definition in terms of bounded areas in the case 
of 'nice' functions.
\\\\
Roughly, the idea is to define the integral as the sum of signed areas of rectangles which 
approximate the area bounded by the function. Firstly we introduce the concept of 
\textit{partitioning} an interval.
}




\section{\lang{de}{Intervall-Zerlegung}\lang{en}{Partitioning intervals}}


\begin{definition}\label{def:intervall-zer}
\lang{de}{
Es sei $[a;b]$ ein Intervall. Eine \notion{Zerlegung} $Z$ des Intervalls $[a;b]$ in $n$ 
Teilintervalle besteht aus Stellen $x_0=a<x_1<x_2<\ldots <x_{n-1}<x_n=b$, und den zugehörigen 
Teilintervallen $[x_0;x_1]$, $[x_1;x_2],\ldots, [x_{n-1};x_n]$.\\
Der Wert $\Delta x_k=x_k-x_{k-1}$ ist die Länge des $k$-ten Teilintervalls (für $k=1,\ldots, n$) und
die \notion{Feinheit} der Zerlegung $Z$ ist das Maximum der Teilintervalllängen, d.\,h. der Wert
}
\lang{en}{
Let $[a;b]$ be an interval. A \notion{partition} $Z$ of the interval $[a;b]$ into $n$ subintervals 
is a collection of points $x_0=a<x_1<x_2<\ldots <x_{n-1}<x_n=b$ and the corresponding subintervals 
$[x_0;x_1]$, $[x_1;x_2],\ldots, [x_{n-1};x_n]$.\\
The value $\Delta x_k=x_k-x_{k-1}$ is the length of the $k$th subinterval (for $k=1,\ldots, n$) and 
the \notion{norm} of the partition $Z$ is the maximum of the lengths of the subintervals, that is, 
the value
}
\[ \Delta Z:= \max\{ \Delta x_1,\ldots, \Delta x_n\}. \]
\end{definition}

\begin{example}\label{ex:aequidistante-zerlegung}
\lang{de}{
Eine der wichtigsten Zerlegungen ist die sogenannte \notion{äquidistante Zerlegung}, bei der die 
Längen der Teilintervalle alle gleich sind. Bei einer äquidistanten Zerlegung des Intervalls $[1;6]$ 
in $8$ Teilintervalle hat also jedes Teilintervall die Länge $\Delta x_k=\frac{6-1}{8}=\frac{5}{8}$. 
Die Zwischenstellen sind also
}
\lang{en}{
One of the most important partitions is the so-called \notion{equal partition}, in which the 
intervals are all the same lengths. An equal partition of the interval $[1;6]$ into $8$ 
subintervals results in each subinterval having length $\Delta x_k=\frac{6-1}{8}=\frac{5}{8}$. The 
corresponding points are
}
\[   x_0=1,\, x_1=1+\frac{5}{8}=\frac{13}{8},\, x_2=\frac{13}{8}+\frac{5}{8}=\frac{18}{8}, \ldots \]
\lang{de}{bzw. allgemein}
\lang{en}{in this case, or more generally,}
\[   x_k= 1+k\cdot \frac{5}{8} \quad \text{\lang{de}{ für }\lang{en}{ for }}k=0,\ldots, 8. \]
\lang{de}{Die Teilintervalle sind also die Intervalle }
\lang{en}{The subintervals are therefore the intervals }
$[1;\frac{13}{8}]$, $[\frac{13}{8};\frac{18}{8}],\ldots, [\frac{43}{8};6]$.
  
\lang{de}{
Im Allgemeinen besteht also eine äquidistante Zerlegung eines Intervalls $[a;b]$ in $n$ 
Teilintervalle aus den Teilpunkten 
}
\lang{en}{
In general the equal partition of an interval $[a;b]$ into $n$ subintervals has the following 
endpoints:
}
\[   x_k= a+k\cdot \frac{b-a}{n} \quad \text{\lang{de}{ für }\lang{en}{ for }}k=0,\ldots, n. \]
\end{example}

\begin{quickcheckcontainer}
\randomquickcheckpool{1}{2}
\begin{quickcheck}
		\field{rational}
		\type{input.number}
		\begin{variables}
			\randint{a}{-3}{1}
            \randint{b}{2}{8}
            \function[calculate]{l}{(b-a)/4}
			\function[calculate]{sol1}{a+l}
            \function[calculate]{sol2}{a+2*l}
            \function[calculate]{sol3}{a+3*l}
		\end{variables}
		
		\text{\lang{de}{
        Das Intervall $[\var{a};\var{b}]$ soll in vier äquidistante Teilintervalle zerlegt werden.
        Die Zerlegungsstellen sind gegeben durch:
        }
        \lang{en}{
        If the interval $[\var{a};\var{b}]$ is partitioned into four equal subintervals, the 
        endpoints of the intervals are as follows:
        }\\
        $\var{a} <$ \ansref $<$ \ansref $<$ \ansref $< \var{b}$}
		
		\begin{answer}
			\solution{sol1}
		\end{answer}
        \begin{answer}
			\solution{sol2}
		\end{answer}
        \begin{answer}
			\solution{sol3}
		\end{answer}
		\explanation{\lang{de}{Die Länge jedes Teilintervalls ist $\var{l}$.}
                 \lang{en}{The length of each subinterval is $\var{l}$.}}
	\end{quickcheck}
	
	\begin{quickcheck}
		\field{rational}
		\type{input.number}
		\begin{variables}
			\randint{a}{-3}{1}
            \randint{b}{2}{8}
            \function[calculate]{l}{(b-a)/5}
			\function[calculate]{sol1}{a+l}
            \function[calculate]{sol2}{a+2*l}
            \function[calculate]{sol3}{a+3*l}
            \function[calculate]{sol4}{a+4*l}
		\end{variables}
		
		\text{\lang{de}{
        Das Intervall $[\var{a};\var{b}]$ soll in fünf äquidistante Teilintervalle zerlegt werden.
        Die Zerlegungsstellen sind gegeben durch:
        }
        \lang{en}{
        If the interval $[\var{a};\var{b}]$ is partitioned into five equal subintervals, the 
        endpoints of the intervals are as follows:
        }\\
        $\var{a} <$ \ansref $<$ \ansref $<$ \ansref $<$ \ansref $< \var{b}$}
		
		\begin{answer}
			\solution{sol1}
		\end{answer}
        \begin{answer}
			\solution{sol2}
		\end{answer}
        \begin{answer}
			\solution{sol3}
		\end{answer}
        \begin{answer}
			\solution{sol4}
		\end{answer}
		\explanation{\lang{de}{Die Länge jedes Teilintervalls ist $\var{l}$.}
                 \lang{en}{The length of each subinterval is $\var{l}$.}}
	\end{quickcheck}
\end{quickcheckcontainer}


\section{\lang{de}{Obersummen und Untersummen}
         \lang{en}{Upper sums and lower sums}}\label{sec:ober-untersumme}


\begin{definition}\label{def:Obersum-untersum}
\lang{de}{
Sei $f:[a;b]\to \R$ eine auf dem Intervall $[a;b]$ definierte beschränkte Funktion.
Für eine Zerlegung $Z$ des Intervalls $[a;b]$ in Teilintervalle $[x_{k-1};x_k]$ ($k=1,\ldots, n$)
seien für jedes $k=1,\ldots, n$
}
\lang{en}{
Let $f:[a;b]\to \R$ be a bounded function defined on the interval $[a;b]$. For a partition $Z$ of the 
interval $[a;b]$ into subintervals $[x_{k-1};x_k]$ ($k=1,\ldots, n$), for all $k=1,\ldots, n$ let 
}
\[ m_k=\inf\{ f(x) \,|\, x\in [x_{k-1};x_k] \}, \]
\lang{de}{
das \link{supinf}{Infimum} aller Funktionswerte auf dem Teilintervall $[x_{k-1};x_k]$, und
}
\lang{en}{
be the \link{supinf}{infimum} of the image of the function on the subinterval $[x_{k-1};x_k]$, and
}
\[ M_k=\sup\{ f(x) \,|\, x\in [x_{k-1};x_k] \}, \]
\lang{de}{
das \link{supinf}{Supremum} aller Funktionswerte auf dem Teilintervall $[x_{k-1};x_k]$. Dann 
definiert man die \notion{Untersumme} $U(Z)$ und die \notion{Obersumme} $O(Z)$ von $f$ zur Zerlegung 
$Z$ durch
}
\lang{en}{
be the \link{supinf}{supremum} of the image of the function on the subinterval $[x_{k-1};x_k]$. Then 
we define the \notion{lower sum} $U(Z)$ and the \notion{upper sum} $O(Z)$ of $f$ over the partition 
$Z$ as
}
  \[ U(Z)= \sum_{k=1}^n m_k \cdot \Delta x_k \, \, \text{\lang{de}{ und }\lang{en}{ and }} \, \, \,   O(Z)= \sum_{k=1}^n M_k \cdot \Delta x_k \, .\]

\lang{de}{Die Werte hängen von der gewählten Zerlegung $Z$ des Intervalls $[a;b]$ ab.}
\lang{en}{These values depend on the chosen partition $Z$ of the interval $[a;b]$.}

% Der gesuchte Integralwert (sofern er existiert) liegt \emph{zwischen der Untersumme und der Obersumme}. 
\end{definition}


\begin{example}\label{zweizerl}
%\emph {Beispiel:} 

\lang{de}{
Links wird das Integrationsintervall $[0;2]$ in zwei Teilintervalle und rechts in 
zehn Teilintervalle aufgeteilt. Die Untersumme ist die dunkel gefärbte Fläche,
die Obersumme ist die hell gefärbte plus die dunkel gefärbte Fläche.
In beiden Fällen liegt der Flächeninhalt unter dem Graphen zwischen der Untersumme und der Obersumme. 
Mit zehn Teilintervallen ist der Unterschied jedoch geringer.\\
}
\lang{en}{
On the left, the interval $[0,2]$ has been divided into two smaller subintervals. On the right, the 
same interval has been divided into ten equal intervals. The lower sum is the area of the darker 
rectangles and the upper sum is the area of the lighter rectangles. In both cases, the true value 
of the integral lies between the upper and lower sums, however the approximation 
is better with ten intervals than with two.\\
}


\begin{center}
\image{T303_Integral_A}\hspace{1cm} \image{T303_Integral_B}
\end{center}


%\end{example}
%\includegraphics[width=7cm]{summe2.png} %\hspace*{5mm}
%\includegraphics[width=7cm]{summe5.png}
%\caption{Untersummen und Obersummen für $f(x)=x^2+1$ im Intervall $[0,2]$ mit $2$ Teilintervallen (links) und $10$ Teilintervallen (rechts).}
%\label{summen2-5}
%\end{figure}

%\begin{example}
%\emph{Beispiel:}
\lang{de}{
Die Funktion in diesem Beispiel ist $f(x)=x^2+1$ und die Intervallgrenzen sind $a=0$ und $b=2$. 
Bei der Aufteilung in zwei Teilintervalle (linker Graph) wird das Intervall $[0;2]$ in zwei 
Teilintervalle $[0;1]$ und $[1;2]$ zerlegt und für diese Zerlegung die Untersumme $U(Z_2)$ und die 
Obersumme $O(Z_2)$ berechnet. Die Intervalllänge ist $1$, der kleinste Funktionswert liegt am linken 
Rand, der größte am rechten Rand. Man erhält:
}
\lang{en}{
The function in this example is $f(x)=x^2+1$, defined on an interval from $a=0$ to $b=2$. To 
partition this into two equal subintervals (left graph), the interval $[0;2]$ is divided into 
$[0;1]$ and $[1;2]$, and these are used to find the lower sum $U(Z_2)$ and the upper sum $O(Z_2)$. 
The length of the intervals is $1$, the smallest value of the function is on the left side of each 
subinterval, and the largest value is on the right side. We obtain:
}
\[ U(Z_2)= f(0)\cdot 1 + f(1)\cdot 1 = 1+2 = 3 \, \, \, \text{\lang{de}{ und }\lang{en}{ and }} \, \, \,
O(Z_2)=f(1) \cdot 1 + f(2)\cdot 1 = 2+5 = 7 . \]

\lang{de}{
Bei der Zerlegung in zehn gleich große Teilintervalle ist die Intervalllänge jeweils 
$\frac{2}{10}=0,2$. Auch hier wird das Supremum der Funktionswerte stets am rechten Rand und das 
Infimum am linken Rand der Teilintervalle angenommen. Für Unter- und Obersumme erhält man daher
}
\lang{en}{
To partition the interval into ten equal subintervals, the subintervals must have length 
$\frac{2}{10}=0.2$. Again, the smallest value of the function is on the left side of each 
subinterval, and the largest value is on the right side. The lower and upper sums are now:
}
\begin{eqnarray*}
U(Z_{10}) &=& f(0)\cdot 0\lang{de}{,}\lang{en}{.}2 + f(0,2)\cdot 0\lang{de}{,}\lang{en}{.}2 
            + \ldots + f(1\lang{de}{,}\lang{en}{.}8)\cdot 0\lang{de}{,}\lang{en}{.}2 
            =(1 + 1\lang{de}{,}\lang{en}{.}04 + 1\lang{de}{,}\lang{en}{.}16 + \ldots 
            + 4\lang{de}{,}\lang{en}{.}24)\cdot 0\lang{de}{,}\lang{en}{.}2 
            = 4\lang{de}{,}\lang{en}{.}28\, , \\
O(Z_{10}) &=& f(0\lang{de}{,}\lang{en}{.}2)\cdot 0\lang{de}{,}\lang{en}{.}2 
            + f(0\lang{de}{,}\lang{en}{.}4)\cdot 0\lang{de}{,}\lang{en}{.}2 
            +\ldots + f(2)\cdot 0\lang{de}{,}\lang{en}{.}2 
            = (1\lang{de}{,}\lang{en}{.}04 + 1\lang{de}{,}\lang{en}{.}16 + \ldots 
            + 4\lang{de}{,}\lang{en}{.}24+5)\cdot 0\lang{de}{,}\lang{en}{.}2 
            = 5\lang{de}{,}\lang{en}{.}08\, . \\
\end{eqnarray*} 

\lang{de}{
Bei der Zerlegung in zehn Teilintervalle ist die Differenz zwischen Ober- und Untersumme nicht so 
groß wie bei der Zerlegung in zwei Teilintervalle.
}
\lang{en}{
When we partition into ten subintervals, the difference between the upper sum and the lower sum is 
smaller than when we partition into two subintervals.
}
\end{example}

%
%\lang{de}{In der folgenden Definition wird der Begriff der beschränkten Funktion verwendet.
%Eine Funktion $f$ ist \emph{beschränkt}, falls es eine Zahl $R>0$ gibt, so dass
%$-R \leq f(x) \leq R$ für alle $x$ in ihrem Definitionsbereich gilt.}
%\lang{en}{In the following definition we'll deal with the idea of a bounded function. A function $f$ is called \emph{bounded} if there is a number 
%$R>0$ so that $-R \leq f(x) \leq R$ for all $x$ in the domain of $f$.}



\section{\lang{de}{Riemannsche Zwischensummen}
         \lang{en}{Riemann sums}}\label{sec:zwischensummen}

\lang{de}{
Wir hatten im obigen \lref{sec:ober-untersumme}{Abschnitt} Ober- und Untersummen definiert, indem wir 
das betreffende Intervall in Teilintervalle zerlegt hatten und dann rechteckige Flächen aufsummiert 
haben, deren Höhen genau die Suprema bzw. Infima der Funktion auf den Teilintervallen sind.
\\\\
Ganz entsprechend definiert man auch Zwischensummen:
}
\lang{en}{
In the above \lref{sec:ober-untersumme}{section} we defined upper sums and lower sums by partitioning 
an interval into equal subintervals and summing the rectangular areas whose heights are respectively 
the suprema or infima of the function values on the subintervals.
\\\\
We now define Riemann sums:
}


\begin{definition}\label{def:riemann-zwisch-sum}
\lang{de}{
Es sei $f:[a;b]\to \R$ eine auf dem Intervall $[a;b]$ definierte beschränkte Funktion
und $Z$ eine Zerlegung des Intervalls $[a;b]$ in Teilintervalle $[x_{k-1};x_k]$ ($k=1,\ldots, n$).
Weiter sei für jedes $k=1,\ldots, n$ eine Stelle $a_k\in [x_{k-1};x_k]$ gewählt.
\\\\
Dann definiert man die \notion{(Riemannsche) Zwischensumme} $S(Z;a_1,\ldots, a_n)$ von $f$ zur 
Zerlegung $Z$ und den Zwischenstellen $a_1,\ldots, a_n$ durch
}
\lang{en}{
Let $f:[a;b]\to \R$ be a bounded function defined on the interval $[a;b]$, and $Z$ a partition of the 
interval $[a;b]$ into subintervals $[x_{k-1};x_k]$ ($k=1,\ldots, n$). 
For each $k=1,\ldots, n$, let $a_k\in [x_{k-1};x_k]$ be a point.
\\\\
We define the \notion{Riemann sum} $S(Z;a_1,\ldots, a_n)$ of $f$ over the partition $Z$ with 
a point $a_1,\ldots, a_n$ in each subinterval as
}
  \[ S(Z;a_1,\ldots, a_n)= \sum_{k=1}^n f(a_k) \cdot \Delta x_k = \sum_{k=1}^n f(a_k) \cdot (x_k-x_{k-1}) \, .\]

\end{definition}

\begin{remark}
\lang{de}{
Sind die Zwischenstellen $a_1,\ldots,a_n$ so gewählt, dass 
$f(a_k)=\sup \{ f(x) | x\in [x_{k-1};x_k]\}$ 
für alle $k$ gilt (falls es überhaupt möglich ist), so ist die
Zwischensumme zur gewählten Zerlegung gleich der Obersumme zu dieser Zerlegung.\\
Entsprechend ist die Zwischensumme gleich der Untersumme, wenn 
$f(a_k)=\inf \{ f(x) | x\in [x_{k-1};x_k]\}$ für alle $k$.
\\\\
Im Allgemeinen liegt die Zwischensumme zu einer gewählten Zerlegung zwischen der Unter- und der Obersumme zu dieser Zerlegung.
}
\lang{en}{
If the points $a_1,\ldots,a_n$ are chosen such that $f(a_k)=\sup \{ f(x) | x\in [x_{k-1};x_k]\}$ for 
all $k$ (if this is even possible), then the Riemann sum for the given partition is equal to the 
upper sum for the partition.\\
Similarly, the Riemann sum is equal to the lower sum if $f(a_k)=\inf \{ f(x) | x\in [x_{k-1};x_k]\}$ 
for all $k$.
\\\\
In general, for any given partition, the Riemann sum lies between the upper sum and the lower sum.
}
\end{remark}

\lang{de}{
Da Zwischensummen zwischen der Unter- und der Obersumme liegen, dienen auch sie dazu, Integrale 
näherungsweise zu bestimmen. Sie sind im Allgemeinen sogar praktischer, weil man nicht die Suprema 
bzw. Infima auf Intervallen bestimmen muss, sondern lediglich Funktionswerte an Zwischenstellen.
}
\lang{en}{
As Riemann sums lie between upper sums and lower sums, they can be used to approximate integrals. In 
fact, they are more practical to this purpose than the upper and lower sums themselves, as there is 
no need to determine the suprema and infima on each subinterval. We simply need to choose some 
arbitrary value in the subinterval at which to evaluate the function.
}


\begin{example}
\lang{de}{
Im \lref{zweizerl}{obigen Beispiel} $f(x)=x^2+1$ auf dem Intervall $[a;b]=[0;2]$ kann man zur 
äquidistanten Zerlegung in zwei Teilintervalle auch die Zwischenstellen $a_1=\frac{1}{2}$ und 
$a_2=\frac{3}{2}$ wählen und erhält dementsprechend als Zwischensumme
}
\lang{en}{
In the \lref{zweizerl}{above example} of $f(x)=x^2+1$ on the interval $[a;b]=[0;2]$, we may partition 
$[0;2]$ into two equal subintervals, then choose the midpoints of each interval, $a_1=\frac{1}{2}$ 
and $a_2=\frac{3}{2}$. The Riemann sum is then
}
\[  S\left(Z_2; \frac{1}{2}, \frac{3}{2}\right)= f\left(\frac{1}{2}\right)\cdot 1+ f\left(\frac{3}{2}\right)\cdot 1=\frac{5}{4}+\frac{13}{4}=\frac{9}{2}=4\lang{de}{,}\lang{en}{.}5.\]

\lang{de}{
Wir sehen, dass hier der Wert der Zwischensumme zwischen den Werten der Ober- und Untersumme aus 
Beispiel \ref{zweizerl} liegt. Wenn wir das Integral mit den Methoden aus dem ersten Teil ermitteln, 
erhalten wir
}
\lang{en}{
Here we see that the value of the Riemann sum lies in between the values of the upper and lower sums. 
Evaluating the integrals using the methods from the first chapter yields
}
\[
\int_0^2 f(x)\, dx = \frac{14}{3} \approx 4\lang{de}{,}\lang{en}{.}67.
\]
\lang{de}{
Der Wert der Zwischensumme kann den Wert des Integrals also einigermaßen gut approximieren. Dies 
liegt aber auch an der gewählten Zerlegung und den Zwischenstellen.
}
\lang{en}{
The value of the Riemann sum can evidently approximate the value of the integral to some extent. It 
still depends, of course, on the chosen partition and at which points the function is evaluated in 
each subinterval.
}
\end{example}

\begin{quickcheck}
		\field{rational}
		\type{input.number}
		\begin{variables}
			\randint{a}{-3}{3}
            \randint{b}{-3}{3}
            \randint{c}{1}{4}
            \function[normalize]{f}{x^2+a*x+b}
            \function[calculate]{z1}{-1/c}
            \function[calculate]{z2}{2/c}
			\substitute[calculate]{v1}{f}{x}{z1}
            \substitute[calculate]{v2}{f}{x}{z2}
            \function[calculate]{sol}{v1+2*v2}
		\end{variables}
		
		\text{\lang{de}{
        Das Intervall $[-1;2]$ soll in zwei Teilintervalle zerlegt werden mit der Zwischenstelle $0$.
        Die Funktion $f$ sei gegeben durch $f(x)=\var{f}$. Berechnen Sie die Zwischensumme $S(Z;\var{z1},\var{z2}) =$\ansref
        }
        \lang{en}{
        Partition the interval $[-1;2]$ into two subintervals with endpoint $0$. 
        Let the function $f$ be given by $f(x)=\var{f}$. Calculate the Riemann sum $S(Z;\var{z1},\var{z2}) =$\ansref
        }}
		
		\begin{answer}
			\solution{sol}
		\end{answer}
		\explanation{$f(\var{z1})=\var{v1}$ \lang{de}{und}\lang{en}{and} $f(\var{z2})=\var{v2}$}
	\end{quickcheck}


\end{content}