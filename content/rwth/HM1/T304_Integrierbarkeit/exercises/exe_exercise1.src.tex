\documentclass{mumie.element.exercise}
%$Id$
\begin{metainfo}
  \name{
    \lang{de}{Ü01: Intervallzerlegung}
    \lang{en}{Exercise 1}
  }
  \begin{description} 
 This work is licensed under the Creative Commons License Attribution 4.0 International (CC-BY 4.0)   
 https://creativecommons.org/licenses/by/4.0/legalcode 

    \lang{de}{}
    \lang{en}{}
  \end{description}
  \begin{components}
  \end{components}
  \begin{links}
  \end{links}
  \creategeneric
\end{metainfo}
\begin{content}
\usepackage{mumie.ombplus}

\title{\lang{de}{Ü01: Intervallzerlegung}}

\begin{block}[annotation]
  Im Ticket-System: \href{http://team.mumie.net/issues/10576}{Ticket 10576}
\end{block}

%######################################################FRAGE_TEXT
\lang{de}{ Es sollen Zerlegungen des Intervalls $[2;5]$ betrachtet werden.

\begin{enumerate}[a)]
\item a) Bestimmen Sie für die Zerlegung des Intervalls $[2;5]$ in Teilintervalle der Längen
$\Delta x_1=1$, $\Delta x_2=\Delta x_3=\frac{1}{2}$, $\Delta x_4=\frac{1}{3}$ und $\Delta x_5=\frac{2}{3}$ die
zugehörigen Randstellen der Teilintervalle.
\item b) Bestimmen Sie die Längen $\Delta x_k$ der Teilintervalle zur Zerlegung $2<\frac{5}{2}<\frac{8}{3}<3,5<4<5$
sowie die Feinheit $\Delta Z$ der Zerlegung.
\item c) Bestimmen Sie die äquidistante Zerlegung des Intervalls $[2;5]$ in zehn Teilintervalle. Welche Feinheit hat diese
Zerlegung?
\end{enumerate}

 }

%##################################################ANTWORTEN_TEXT
\begin{tabs*}[\initialtab{0}\class{exercise}]

\tab{\lang{de}{    Antworten    }}
    \lang{de}{ \begin{enumerate}[a)]
\item a) Zerlegung mit Randstellen: $2<3<3,5<4<\frac{13}{3}<5$.
\item b) Die Feinheit $\Delta Z$ beträgt $1$. Die Intervalllängen sind
\begin{align*}
\Delta x_1 =\frac{1}{2}, \ \Delta x_2 =\frac{1}{6}, \ \Delta x_3 =\frac{5}{6}, \ 
\Delta x_4 =\frac{1}{2}, \ \Delta x_5 =1.
\end{align*}
\item c) Die Teilungsstellen sind
\begin{align*}
 x_0 &= 2, \quad & x_1 & =2,3\, ,\quad & x_2 & =2,6\, ,\\ 
 x_3 & =2,9\, ,\quad & x_4 & =3,2\, ,\quad & x_5 & =3,5\, , \\
 x_6 & =3,8\, ,\quad &x_7 & =4,1\, ,\quad &x_8 & =4,4\, , \\
 x_9 & =4,7\, ,\quad  &x_{10} &=5. & &
 \end{align*}
Die Feinheit $\Delta Z$ beträgt $0,3$. 
\end{enumerate} }

  %++++++++++++++++++++++++++++++++++++++++++START_TAB_X
  \tab{\lang{de}{    Lösung a)    }}
  \begin{incremental}[\initialsteps{1}]
  
  	 %----------------------------------START_STEP_X
    \step 
Da $5$ Intervalllängen angegeben sind, ist das Intervall $[2;5]$ also in $5$ Teilintervalle zu zerlegen.
Wir benötigen also Stellen $x_0,x_1,\ldots, x_5$, wobei $x_0=2$ (linker Rand des gesamten Intervalls) und 
$x_5=5$ (rechter Rand des gesamten Intervalls). Die Stellen $x_1,x_2,x_3,x_4$ findet man über die
Längen der Teilintervalle, da
\[  \Delta x_k=x_k-x_{k-1} \]
für $k=1,\ldots, 5$ gilt.
\step
Es sind also
\begin{eqnarray*}
x_1 &=& x_0+ \Delta x_1= 2+1=3, \\
x_2 &=& x_1+ \Delta x_2= 3+\frac{1}{2}=3,5 , \\
x_3 &=& x_2+ \Delta x_3= 3,5+\frac{1}{2}=4, \\
x_4 &=& x_3+ \Delta x_4= 4+\frac{1}{3}=\frac{13}{3}.
\end{eqnarray*}
\step
Zuletzt testen wir noch, ob dann auch die letzte Angabe stimmt, d.\,h. ob $\Delta x_5=\frac{2}{3}$ auch
wirklich mit der Länge des Intervalls $[x_4;x_5]$ übereinstimmt:
\[ \Delta x_5 = x_5-x_4= 5-\frac{13}{3}=\frac{15-13}{3}=\frac{2}{3}. \]
\step
Die Zerlegung ist also gegeben durch
\[ 2<3<3,5<4<\frac{13}{3}<5. \]
  	 %------------------------------------END_STEP_X
 
  \end{incremental}
  %++++++++++++++++++++++++++++++++++++++++++++END_TAB_X
  %++++++++++++++++++++++++++++++++++++++++++START_TAB_X
  \tab{\lang{de}{    Lösung b)    }}
  \begin{incremental}[\initialsteps{1}]
  
  	 %----------------------------------START_STEP_X
    \step 
Die Längen der Teilintervalle $[x_{k-1};x_k]$ sind gegeben durch
\[  \Delta x_k=x_k-x_{k-1} \]
für alle $k$.
\step
Für die Zerlegung $2<\frac{5}{2}<\frac{8}{3}<3,5<4<5$ sind dies also
\begin{eqnarray*}
\Delta x_1 &=& \frac{5}{2} - 2 =\frac{1}{2}, \\
\Delta x_2 &=& \frac{8}{3} - \frac{5}{2} =\frac{16-15}{6}=\frac{1}{6}, \\
\Delta x_3 &=& 3,5 - \frac{8}{3}=\frac{7}{2}-\frac{8}{3}=\frac{21-16}{6}=\frac{5}{6}, \\
\Delta x_4 &=& 4 - 3,5 =\frac{1}{2}, \\
\Delta x_5 &=& 5-4 =1.
\end{eqnarray*}
\step
Die Feinheit $\Delta Z$ der Zerlegung ist das Maximum der Teilintervalllängen, also
\[ \Delta Z=\max \{\frac{1}{2}; \frac{1}{6};\frac{5}{6};\frac{1}{2};1 \}=1. \]
  	 %------------------------------------END_STEP_X
 
  \end{incremental}
  %++++++++++++++++++++++++++++++++++++++++++++END_TAB_X
  %++++++++++++++++++++++++++++++++++++++++++START_TAB_X
  \tab{\lang{de}{    Lösung c)    }}
  \begin{incremental}[\initialsteps{1}]
  
  	 %----------------------------------START_STEP_X
    \step 
Bei einer äquidistanten Zerlegung sind die Längen der Teilintervalle alle gleich groß. Da das Intervall
$[2;5]$ in $10$ Teilintervalle zerlegt werden soll, beträgt die Länge eines jeden Teilintervalls also
ein Zehntel der Gesamtlänge, also genau $\frac{3}{10}=0,3$.

\step
Damit berechnen sich die Teilungsstellen durch
\begin{align*}
 x_0 &= 2, \quad & x_1 & =x_0+\frac{3}{10}=2,3\, ,\quad & x_2 & =x_1+0,3=2,6\, ,\\ 
 x_3 & =2,9\, ,\quad & x_4 & =3,2\, ,\quad & x_5 & =3,5\, , \\
 x_6 & =3,8\, ,\quad &x_7 & =4,1\, ,\quad &x_8 & =4,4\, , \\
 x_9 & =4,7\, \quad \text{ und } &x_{10} &=5. & &
 \end{align*}
\step
Die Feinheit der Zerlegung ist das Maximum der Längen der Teilintervalle. Da diese alle die Länge $0,3$ haben,
ist dies auch die Feinheit der Zerlegung.

  	 %------------------------------------END_STEP_X
 
  \end{incremental}
  %++++++++++++++++++++++++++++++++++++++++++++END_TAB_X
  
  


%#############################################################ENDE
\end{tabs*}
\end{content}