\documentclass{mumie.element.exercise}
%$Id$
\begin{metainfo}
  \name{
    \lang{de}{Ü05: Zwischensummen}
    \lang{en}{exercise 5}
  }
  \begin{description} 
 This work is licensed under the Creative Commons License Attribution 4.0 International (CC-BY 4.0)   
 https://creativecommons.org/licenses/by/4.0/legalcode 

    \lang{de}{Hier die Beschreibung}
    \lang{en}{}
  \end{description}
  \begin{components}
  \end{components}
  \begin{links}
  \link{generic_article}{content/rwth/HM1/T304_Integrierbarkeit/g_art_content_09_integrierbare_funktionen.meta.xml}{link1}
  \end{links}
  \creategeneric
\end{metainfo}
\begin{content}
\title{
  \lang{de}{Ü05: Zwischensummen}
}



\begin{block}[annotation]
  Im Ticket-System: \href{http://team.mumie.net/issues/10580}{Ticket 10580}
\end{block}

Es sei $f(x)=x^2-2x$ und $Z$ die Zerlegung von $[0;3]$ mit den Zwischenstellen $1$ und $\frac{5}{2}$.
\begin{enumerate}
\item[a)] Berechnen Sie die Ober- und Untersumme zur Zerlegung $Z$.
\item[b)] Berechnen Sie die Riemannsche Zwischensumme $S(Z;0, 1, 3)$.
\item[c)] Berechnen Sie mit dem Hauptsatz der Differential- und Integralrechnung $\int_0^3 f(x)\, dx$.
\item[d)] Vergleichen Sie die Ergebnisse aus a)-c).
\end{enumerate}

\begin{tabs*}[\initialtab{0}\class{exercise}]
  \tab{\lang{de}{Antworten}}
  a) $\ O(Z)=\frac{27}{8}$, $U(Z) = -\frac{15}{8}$,\\
  b) $\ S(Z;0, 1, 3)=0$, \\
  c) $\ \int_0^3 f(x)\, dx = 0$, \\
  d) $\, $ Wert von Zwischensumme und Integral stimmen überein.
  
  \tab{Lösung a)}
  \begin{incremental}[\initialsteps{1}]
  \step Zunächst gilt $f(x) = x^2-2x = (x-1)^2-1$. Damit ist $f$ eine verschobene Normalparabel.
  \step Auf dem Intervall $[0;1]$ ist $f$ monoton fallend und auf dem Intervall $[1;3]$ monoton steigend.
  \step Damit ist die Obersumme gegeben durch
  \[
  O(Z) = (1-0)\cdot f(0) + (\frac{5}{2}-1)\cdot f(\frac{5}{2})+ (3-\frac{5}{2})\cdot f(3) = 0 + \frac{3}{2} \cdot \frac{5}{4} + \frac{1}{2}\cdot 3 = \frac{27}{8}
  \]
  und die Untersumme durch
  \[
  U(Z) = (1-0) \cdot f(1) + (\frac{5}{2}-1)\cdot f(1) + (3-\frac{5}{2})\cdot f(\frac{5}{2})  = 1 \cdot (-1) + \frac{3}{2}\cdot (-1) + \frac{1}{2} \cdot \frac{5}{4} = - \frac{15}{8}.
  \]
  \end{incremental}
  
  \tab{Lösung b)}
  Die Riemannsche Zwischensumme berechnet sich zu
  \[
  S(Z;0, 1, 3) = (1-0) f(0) + (\frac{5}{2}-1)\cdot f(1) + (3-\frac{5}{2})\cdot f(3) = 1 \cdot 0 + \frac{3}{2}\cdot (-1) + \frac{1}{2} \cdot 3 = 0.
  \]
  
  \tab{Lösung c)}
  Mit dem \ref[link1][Hauptsatz zur Differential- und Integralrechnung]{thm:fundamental} erhalten wir
  \[
  \int_0^3 f(x)\, dx  = \left[\frac{1}{3}x^3-x^2\right]_0^3 = \frac{1}{3} \cdot 27 - 9 = 0.
  \]
  \tab{Lösung d)}
  Aus dem Vorlesungstext wissen wir
  \[
  U(Z) \leq \int_0^3 f(x)\, dx \leq O(Z)
  \]
  und
  \[
  U(Z)\leq S(Z;0, 1, 3) \leq O(Z).
  \]
  In diesem Beispiel sind die Ungleichungen gültig. Dass die Zwischensumme gleich dem Integral ist,
  liegt an der speziellen Wahl der Zerlegung und der Zwischenstellen.
  Die Zerlegung ist weder äquidistant noch sind die Zwischenstellen äquidistant verteilt.
  
  Das Beispiel zeigt, dass es möglich ist, Zerlegungen und Zwischenstellen zu wählen, sodass der Integralwert
  genau gleich ist oder diesen zumindest sehr gut approximieren kann.
  Das setzt aber voraus, dass man die Funktion schon gut kennt.
  \end{tabs*}


\end{content}