\documentclass{mumie.element.exercise}
%$Id$
\begin{metainfo}
  \name{
    \lang{de}{Ü03: Zwischensummen}
    \lang{en}{Exercise 3}
  }
  \begin{description} 
 This work is licensed under the Creative Commons License Attribution 4.0 International (CC-BY 4.0)   
 https://creativecommons.org/licenses/by/4.0/legalcode 

    \lang{de}{}
    \lang{en}{}
  \end{description}
  \begin{components}
  \end{components}
  \begin{links}
  \end{links}
  \creategeneric
\end{metainfo}
\begin{content}
\usepackage{mumie.ombplus}

\title{\lang{de}{Ü03: Zwischensummen}}

\begin{block}[annotation]
  Im Ticket-System: \href{http://team.mumie.net/issues/10578}{Ticket 10578}
\end{block}

\begin{enumerate}
\item
%######################################################FRAGE_TEXT
\lang{de}{ 
Bestimmen Sie zur Zerlegung $Z$: $0<\frac{1}{2}<\frac{3}{2}<2$ des Intervalls $[0;2]$
und den Zwischenstellen $a_1=0$, $a_2=1$ und $a_3=2$ die Zwischensummen zu den Funktionen

\begin{enumerate}[a)]
\item a) $\ f(x)=x^2-2x+2$, 
\item b) $\ g(x)=2x+1$,
\item c) $\ h(x)=\frac{1}{1+x^2}$.
\end{enumerate}

 }

%##################################################ANTWORTEN_TEXT
\begin{tabs*}[\initialtab{0}\class{exercise}]
\tab{Antworten}
a) $\ 3$, \\
b) $\ 6$, \\
c) $\ \frac{11}{10}$.

  %++++++++++++++++++++++++++++++++++++++++++START_TAB_X
  \tab{\lang{de}{    Lösung a)    }}
  \begin{incremental}[\initialsteps{1}]
  
  	 %----------------------------------START_STEP_X
    \step  
Die allgemeine Formel für die Zwischensumme ist
\[ S(Z; a_1, a_2, a_3)=\sum_{k=1}^3 f(a_k)\cdot \Delta x_k, \]
wobei $\Delta x_k$ die Länge des $k$-ten Teilintervalls ist.
\step
Hier also 
\[ \Delta x_1=\frac{1}{2}-0=\frac{1}{2},\quad \Delta x_2=\frac{3}{2}-\frac{1}{2}=1,\quad 
\Delta x_3=2-\frac{3}{2}=\frac{1}{2}. \]
\step
Damit ist also die Zwischensumme gegeben als:
\[ S(Z; a_1, a_2, a_3)= f(a_1)\cdot \frac{1}{2}+f(a_2)\cdot 1+f(a_3)\cdot \frac{1}{2}
= 2\cdot \frac{1}{2}+1\cdot 1+2\cdot \frac{1}{2}=3.\]
  	 %------------------------------------END_STEP_X
 
  \end{incremental}
  %++++++++++++++++++++++++++++++++++++++++++++END_TAB_X

  %++++++++++++++++++++++++++++++++++++++++++START_TAB_X
  \tab{\lang{de}{    Lösung b)    }}
  \begin{incremental}[\initialsteps{1}]
  
  	 %----------------------------------START_STEP_X
    \step   
Da das Intervall, seine Zerlegung und die Zwischenstellen dieselben sind wie in a), erhalten
wir die gleiche Formel für die Zwischensumme wie in a), d.\,h.
\[ S(Z; a_1, a_2, a_3)= g(0)\cdot \frac{1}{2}+g(1)\cdot 1+g(2)\cdot \frac{1}{2}.\]
\step
Einsetzen von $g(0)=1$, $g(1)=3$ und $g(2)=5$ ergibt dann:
\[ S(Z; a_1, a_2, a_3)=1\cdot \frac{1}{2}+3\cdot 1+5\cdot \frac{1}{2}=6.\]
  	 %------------------------------------END_STEP_X
 
  \end{incremental}
  %++++++++++++++++++++++++++++++++++++++++++++END_TAB_X

  %++++++++++++++++++++++++++++++++++++++++++START_TAB_X
  \tab{\lang{de}{    Lösung c)    }}
  \begin{incremental}[\initialsteps{1}]
  
  	 %----------------------------------START_STEP_X
    \step    
Da das Intervall, seine Zerlegung und die Zwischenstellen dieselben sind wie in a), erhalten
wir die gleiche Formel für die Zwischensumme wie in a), d.\,h.
\[ S(Z; a_1, a_2, a_3)= h(0)\cdot \frac{1}{2}+h(1)\cdot 1+h(2)\cdot \frac{1}{2}.\]
\step
Einsetzen von $h(0)=1$, $h(1)=\frac{1}{2}$ und $h(2)=\frac{1}{5}$ ergibt dann:
\[ S(Z; a_1, a_2, a_3)=1\cdot \frac{1}{2}+\frac{1}{2}\cdot 1+\frac{1}{5}\cdot \frac{1}{2}=\frac{11}{10}.\]

%------------------------------------END_STEP_X
 
  \end{incremental}
  %++++++++++++++++++++++++++++++++++++++++++++END_TAB_X


%#############################################################ENDE

 

\end{tabs*}

\item Sei $f(x) = -x^2+2x+3$. Berechnen Sie (mit einem Taschenrechner)
Näherungen zu $\int_0^3 f(x)dx$ durch eine Riemannsche Zwischensumme zu den folgenden Zerlegungen
\begin{enumerate}
\item $x_0=0; x_1=1; x_2=2; x_3=3$,
\item $x_0=0; x_1=0,5; x_2=1; x_3=1,5; x_4=2;x_5=2,5; x_6=3,$
\item $x_0=0; x_1=1; x_2=2,5; x_3=3$,
\end{enumerate}
und Zwischenstellen $\hat{x_k}$ am linken Intervallrand. 

Was ergibt sich bei der Zerlegung a) 
bei Zwischenstellen $\hat{x_k}$ in der Intervallmitte bzw. als Ober- und Untersumme?

Skizzieren Sie die Situationen.
\begin{tabs*}
\tab{Antworten}
\begin{enumerate}[alph]
\item $10$,
\item $\frac{77}{8}$,
\item $\frac{79}{8}$,
\end{enumerate}
Zu den verschiedenen Zwischenstellen bei der Zerlegung in a) ergibt sich:
\begin{itemize}
\item In der Intervallmitte: $9,25$,
\item Obersumme: $11$,
\item Untersumme: $6$.
\end{itemize}

   \tab{\lang{de}{Lösungsvideo}}
  \youtubevideo[500][300]{SYSS7_xvCas}\\
\end{tabs*}
\end{enumerate}
\end{content}