\documentclass{mumie.element.exercise}
%$Id$
\begin{metainfo}
  \name{
    \lang{de}{Ü08: Anwendungsaufgabe}
    \lang{en}{Exercise 8}
  }
  \begin{description} 
 This work is licensed under the Creative Commons License Attribution 4.0 International (CC-BY 4.0)   
 https://creativecommons.org/licenses/by/4.0/legalcode 

    \lang{de}{}
    \lang{en}{}
  \end{description}
  \begin{components}
  \end{components}
  \begin{links}
  \end{links}
  \creategeneric
\end{metainfo}
\begin{content}
\usepackage{mumie.ombplus}

\title{\lang{de}{Ü08: Anwendungsaufgabe}}

\begin{block}[annotation]
  Im Ticket-System: \href{http://team.mumie.net/issues/10583}{Ticket 10583}
\end{block}

Die Wassermenge in Liter in einem Wasserbehälter kann am Beobachtungstag zur Stunde $t$ annähernd durch die Funktion
$f(t) = \frac{t^3}{150} - \frac{t^2}{3} +4 t + 91$ beschrieben werden.

Bestimmen Sie die durchschnittliche Wassermenge im Wasserbehälter an diesem Tag und runden Sie Ihr Ergebnis auf zwei Nachkommastellen.

\begin{tabs*}[\initialtab{0}\class{exercise}]
  \tab{\lang{de}{Antwort}
  \lang{en}{Answer}
  }
$98,04$ Liter Wasser.

\tab{\lang{de}{L"osung}
\lang{en}{Solution}
}

Durchschnittlich sind in den ersten 24 Stunden
\[
\frac{\int_0^{24} f(t)\, dt}{24-0}
\]
Liter Wasser im Wasserbehälter.

Das Integral lösen wir mit dem Hauptsatz der Differential- und Integralrechnung.
Eine Stammfunktion ist durch
\[
F(t) = \frac{t^4}{600} - \frac{t^3}{9} + 2t^2 +91t
\]
gegeben.
Dann ist
\[
\frac{\int_0^{24} f(t)\, dt}{24-0} = \left[ \frac{t^4}{600\cdot 24} - \frac{t^3}{9 \cdot 24} + \frac{t^2}{12} + \frac{91t}{24} \right]_0^{24}
= 98,04.
\]
 
\end{tabs*}
\end{content}