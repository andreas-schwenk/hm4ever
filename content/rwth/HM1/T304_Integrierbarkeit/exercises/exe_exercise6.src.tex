\documentclass{mumie.element.exercise}
%$Id$
\begin{metainfo}
  \name{
    \lang{de}{Ü06: Integrierbarkeit}
    \lang{en}{Exercise 6}
  }
  \begin{description} 
 This work is licensed under the Creative Commons License Attribution 4.0 International (CC-BY 4.0)   
 https://creativecommons.org/licenses/by/4.0/legalcode 

    \lang{de}{}
    \lang{en}{}
  \end{description}
  \begin{components}
  \end{components}
  \begin{links}
  \link{generic_article}{content/rwth/HM1/T304_Integrierbarkeit/g_art_content_09_integrierbare_funktionen.meta.xml}{link1}
  \end{links}
  \creategeneric
\end{metainfo}
\begin{content}
\usepackage{mumie.ombplus}

\title{\lang{de}{Ü06: Integrierbarkeit}}

\begin{block}[annotation]
  Im Ticket-System: \href{http://team.mumie.net/issues/10581}{Ticket 10581}
\end{block}

%######################################################FRAGE_TEXT
\lang{de}{ 
Für welche der folgenden Funktionen $f$ existiert das Integral $\int_{0}^4 f(x)\, dx$, für welche nicht?
Falls das Integral existiert, bestimmen Sie den Wert.
\begin{enumerate}[a)]
\item a) $\ f(x)=x^2+x$
\item b) $\ f(x)=\tan(x)$ 

 \item c) $\ f(x)=\left\{ \begin{mtable}[\cellaligns{cc}] 1, & 0\leq x< 1, \\
 x^2-2x, \quad & 1\leq x\leq 2, \\
 \sin(\pi x), & 2<x\leq 4.
 \end{mtable} \right.$
 \item d) $\ f(x)=\left\{\begin{mtable}[\cellaligns{cc}]  1, & 0\leq x\leq 2,\\
 \frac{1}{x-2}, & 2<x\leq 4.
 \end{mtable} \right.$ 
\end{enumerate}
 }

%##################################################ANTWORTEN_TEXT
\begin{tabs*}[\initialtab{0}\class{exercise}]
%++++++++++++++++++++++++++++++++++++++++++START_TAB_X
  \tab{\lang{de}{    Antworten    }}
    \lang{de}{   a) existiert, der Wert ist $\frac{88}{3}$ ,\\ b) existiert nicht,\\
c) existiert, der Wert ist $\frac{1}{3}$,\\ d) existiert nicht.    }
  \tab{\lang{de}{    Lösung a)    }}
    \lang{de}{   
Die Funktion $f(x)=x^2+x$ ist auf dem Intervall $[0;4]$ stetig. Daher
existiert das Integral von $f$ auf diesem Intervall.

Nach dem \ref[link1][Hauptsatz der Differential- und Integralrechnung]{thm:fundamental} gilt
\[
\int_0^4 (x^2+x) \, dx = \left[\frac{1}{3}x^3+\frac{1}{2}x^2\right]_0^4 = \frac{1}{3} \cdot 4^3 +\frac{1}{2}\cdot 4^2 = \frac{64}{3}+ 8 = \frac{88}{3}.
\]
    }
  \tab{\lang{de}{    Lösung b)    }}
    \lang{de}{   
Die Funktion $f(x)=\tan(x)$ ist an der Stelle $\frac{\pi}{2}$ nicht definiert, welche im Intervall $[0;4]$ liegt. Daher 
ist das Integral $\int_{0}^4 \tan(x)\, dx$ nicht definiert.
    }
  \tab{\lang{de}{    Lösung c)    }} 
\begin{incremental}[\initialsteps{1}]
\step
Die Funktion $f$ ist zwar nicht stetig (z.\,B. macht sie bei $1$ einen Sprung), aber sie ist stückweise aus stetigen Funktionen zusammengesetzt.
Außerdem ist sie beschränkt, weshalb das Integral dennoch existiert.

\step
Das Integral kann aufgeteilt werden:
\[
\int_0^4 f(x)\, dx = \int_0^1 f(x)\, dx + \int_1^2 f(x)\, dx + \int_2^4 f(x)\, dx.
\]
\step Die einzelnen Integrale können wir wieder mit dem \ref[link1][Hauptsatz der Differential- und Integralrechnung]{thm:fundamental} berechnen.
Es ist
\[
\int_0^1 1 \, dx = [x]_0^1 = 1,
\]
\[
\int_1^2 (x^2-2x) \, dx = \left[\frac{1}{3}x^3-x^2\right]_1^2 = \frac{1}{3}\cdot 2^3 - 2^2 - \frac{1}{3}+1=-\frac{2}{3}
\]
und
\[
\int_2^4 \sin(\pi x)\, dx = \left[\frac{-1}{\pi}\cos(\pi x)\right]_2^4 = 0.
\]
\step
Damit ist $\int_0^4 f(x)\, dx = 1 -\frac{2}{3}+0 = \frac{1}{3}$.
\end{incremental}
  \tab{\lang{de}{    Lösung d)    }}
    \lang{de}{   
Die Funktion $f$ ist zwar stückweise aus stetigen Funktionen zusammengesetzt, jedoch ist sie nicht beschränkt, da 
\[ \lim_{x\searrow 2} f(x)=\lim_{x\searrow 2} \frac{1}{x-2}=\infty. \]

Also existiert das Integral $\int_{0}^4 f(x)\, dx$ nicht.
    }
\end{tabs*}
\end{content}