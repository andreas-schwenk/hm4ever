\documentclass{mumie.element.exercise}
%$Id$
\begin{metainfo}
  \name{
    \lang{de}{Ü10: Uneigentliches Integral}
    \lang{en}{Exercise 10}
  }
  \begin{description} 
 This work is licensed under the Creative Commons License Attribution 4.0 International (CC-BY 4.0)   
 https://creativecommons.org/licenses/by/4.0/legalcode 

    \lang{de}{}
    \lang{en}{}
  \end{description}
  \begin{components}
  \end{components}
  \begin{links}
    \link{generic_article}{content/rwth/HM1/T303_Approximationen/g_art_content_06_de_l_hospital.meta.xml}{link1}
  \end{links}
  \creategeneric
\end{metainfo}
\begin{content}
\usepackage{mumie.ombplus}

\title{\lang{de}{Ü10: Uneigentliches Integral}}

\begin{block}[annotation]
  Im Ticket-System: \href{http://team.mumie.net/issues/10642}{Ticket 10642}
\end{block}

%######################################################FRAGE_TEXT
\lang{de}{ Welche der folgenden uneigentlichen Integrale existieren? Berechnen Sie gegebenenfalls
das Integral.

\begin{enumerate}[alph]
\item $\int_0^\infty e^{-x} \; dx$
\item $ \int_1^\infty \frac{x^2+1}{x} \; dx$
\item $ \int_0^1 \frac{(x-1)(e^x-1)}{x^2}\; dx$ 
\item $ \int_1^2 \frac{x^2-4x+4}{x-2}\; dx$
\item $ \int_0^{\infty} \frac{(2x+1)e^{-x}}{\sqrt{x}^3}\, dx$
\end{enumerate}
\textit{Hinweis zu c)}: Eine Stammfunktion zu 
$f:(0;\infty)\to \R, \ x\mapsto \frac{(x-1)(e^x-1)}{x^2}$ ist 
$F(x)=\frac{e^x-1}{x}-\ln(x)$\\
\textit{Hinweis zu e):} Eine Stammfunktion zu 
$g:(0;\infty)\to \R, \ x\mapsto\frac{(2x+1)e^{-x}}{\sqrt{x}^3}$ ist
$G(x)=\frac{-2e^{-x}}{\sqrt{x}}$. }

%##################################################ANTWORTEN_TEXT
\begin{tabs*}[\initialtab{0}\class{exercise}]

	%++++++++++++++++++++++++++++++++++++++++++START_TAB_X
  \tab{\lang{de}{    Antworten    }}
    \lang{de}{ Die uneigentlichen Integrale b), c) und e) existieren nicht.
    
Der Wert des uneigentlichen Integrals in a) ist $1$\\
und der Wert in d) ist $-\frac{1}{2}$    }
  %++++++++++++++++++++++++++++++++++++++++++++END_TAB_X


     \tab{\lang{de}{Lösungsvideo a)}}
  \youtubevideo[500][300]{2VAB17XsKF0}\\

  %++++++++++++++++++++++++++++++++++++++++++START_TAB_X
  \tab{\lang{de}{    Lösung b)    }}
%   \begin{incremental}[\initialsteps{1}]
  
%   	 %----------------------------------START_STEP_X
%     \step 
% Um eine Stammfunktion von $f(x)=\frac{x^2+1}{x}$ zu bestimmen, zerlegen wir zunächst
% den Bruch und erhalten $f(x)=x+\frac{1}{x}$. Damit ist eine Stammfunktion
% im Bereich der positiven reellen Zahlen gegeben durch $F(x)=\frac{x^2}{2}+\ln(x)$.

% \step
% Das uneigentliche Integral ist dann gleich dem folgenden Grenzwert (sofern dieser existiert):
% \[ \lim_{b\to \infty} \int_1^b \frac{x^2+1}{x} \; dx
% =  \lim_{b\to \infty} \left[ \frac{x^2}{2}+\ln(x) \right]_1^b 
% =  \lim_{b\to \infty}  \frac{b^2}{2}+\ln(b)- ( \frac{1}{2}+\ln(1) ). 
% \]
% \step
% Jedoch ist $ \lim_{b\to \infty}  \frac{b^2}{2}+\ln(b)=\infty$, weshalb der Grenzwert nicht
% existiert und daher das uneigentliche Integral auch nicht existiert.

% \step
% Dass das uneigentliche Integral nicht existiert, hätte man hier auch grafisch sehen können, 
% da nämlich die Funktionswerte von $f(x)=x+\frac{1}{x}$ gegen unendlich divergieren. Dann werden bei Vergrößern der oberen Grenze der bestimmten 
% Integrale immer größere Flächen hinzugenommen, weshalb
% der Grenzwert der Integrale nicht existieren kann.    }
%   	 %------------------------------------END_STEP_X
 
%   \end{incremental}
  %++++++++++++++++++++++++++++++++++++++++++++END_TAB_X
Es ist $f(x)=\frac{x^2+1}{x}=x+\frac{1}{x}$ für alle $x\geq 1$.
Es gilt $\lim_{x\to \infty} f(x) = \infty$. Damit kann das uneigentliche Integral nicht existieren.

  %++++++++++++++++++++++++++++++++++++++++++START_TAB_X
  \tab{\lang{de}{    Lösung c)    }}
  \begin{incremental}[\initialsteps{1}]
  
  	 %----------------------------------START_STEP_X
    \step 
Die Funktion $f(x)=\frac{(x-1)(e^x-1)}{x^2}$ ist bei $0$ nicht definiert. Daher ist das
Integral $\int_0^1 \frac{(x-1)(e^x-1)}{x^2}\; dx$ ein uneigentliches Integral, sofern es überhaupt existiert, und gegeben durch
\[ \int_0^1 \frac{(x-1)(e^x-1)}{x^2}\; dx =\lim_{a\searrow 0} \int_a^1 \frac{(x-1)(e^x-1)}{x^2}\; dx. \]
\step
Da nach dem Hinweis $F(x)=\frac{e^x-1}{x}-\ln(x)$ eine Stammfunktion zu $f(x)$ ist, gilt für jedes $a>0$:
\[ \int_a^1 \frac{(x-1)(e^x-1)}{x^2}\; dx=\left[ \frac{e^x-1}{x}-\ln(x) \right]_a^1
= \left( \frac{e^1-1}{1}-\ln(1) \right)- \left( \frac{e^a-1}{a}-\ln(a) \right)
= (e-1) - \frac{e^a-1}{a}+\ln(a).\]

Zu bestimmen ist also das Grenzverhalten von $(e-1) - \frac{e^a-1}{a}+\ln(a)$ für $a\searrow 0$.

\step
Mit der \ref[link1][Regel von de l'Hospital]{thm:de-l-hospital}
erhält man
\[ \lim_{a\searrow 0} \frac{e^a-1}{a} =\lim_{a\searrow 0} \frac{e^a}{1}=e^0=1.\]
Andererseits ist $ \lim_{a\searrow 0} \ln(a)=-\infty$ und damit
divergiert der Ausdruck $\left( (e-1) - \frac{e^a-1}{a}+\ln(a)\right)$ gegen $-\infty$ für
$a\searrow 0$.\\
\step
Der Grenzwert existiert also nicht, weshalb das uneigentliche Integral nicht existiert.
  	 %------------------------------------END_STEP_X
 
  \end{incremental}
  %++++++++++++++++++++++++++++++++++++++++++++END_TAB_X


  %++++++++++++++++++++++++++++++++++++++++++START_TAB_X
  \tab{\lang{de}{    Lösung d)    }}
  \begin{incremental}[\initialsteps{1}]
  
  	 %----------------------------------START_STEP_X
    \step 
Beim uneigentlichen Integral $\int_1^2 \frac{x^2-4x+4}{x-2}\; dx$ ist der Integrand an der
oberen Grenze $2$ nicht definiert.\\
\step
Für $x$ gegen $2$ konvergiert aber sowohl Zähler als auch Nenner gegen $0$. Das bedeutet insbesondere, dass sowohl das Zählerpolynom als auch das Nennerpolynom bei
$x=2$ eine Nullstelle hat und wir durch Polynomdivision jeweils einen Linearfaktor $x-2$
abspalten können.

Hier könnte man auch direkt sehen, dass nach der 2. Binomischen Formel 
$x^2-4x+4=(x-2)^2$ gilt.

\step
Für alle $x\neq 2$ ist damit
\[ \frac{x^2-4x+4}{x-2}=\frac{(x-2)^2}{x-2}=x-2.\]
Insbesondere ist der Integrand auf dem Intervall $[1;2)$ beschränkt und das uneigentliche
Integral existiert.
Zur Berechnung des Integrals können wir als Fortsetzung von $h_1(x)=\frac{x^2-4x+4}{x-2}$ die Funktion $h_2:[1;2]\to \R$ mit $h_2(x)=x-2$ wählen und das bestimmte Integral
$\int_1^2  h_2(x)\;dx$ berechnen:

\[ \int_1^2 \frac{x^2-4x+4}{x-2}\; dx=\int_1^2  (x-2)\;dx
= \left[ \frac{x^2}{2}-2x \right]_1^2=\left(  \frac{2^2}{2}-2\cdot 2\right)
-\left(  \frac{1^2}{2}-2\right) =- \frac{1}{2}.\]
  	 %------------------------------------END_STEP_X
 
  \end{incremental}
  %++++++++++++++++++++++++++++++++++++++++++++END_TAB_X
  %++++++++++++++++++++++++++++++++++++++++++START_TAB_X
  \tab{\lang{de}{    Lösung e)    }}
  \begin{incremental}[\initialsteps{1}]
  
  	 %----------------------------------START_STEP_X
    \step 
Bei diesem uneigentlichen Integral ist sowohl das Intervall unendlich als auch der Integrand
an der unteren Grenze nicht definiert. Das Integrationsintervall ist also in zwei Teilintervalle 
aufzuteilen, z.\,B. in $(0;1]$ und $[1;\infty)$, und die uneigentlichen Integrale
$\int_0^1 \frac{(2x+1)e^{-x}}{\sqrt{x}^3}\; dx$ und $\int_1^{\infty} \frac{(2x+1)e^{-x}}{\sqrt{x}^3}\, dx$ zu untersuchen.

\step
Nach dem Hinweis ist $G(x)=\frac{-2e^{-x}}{\sqrt{x}}$ eine Stammfunktion für den
Integranden und daher
\[ \int_0^1 \frac{(2x+1)e^{-x}}{\sqrt{x}^3}\; dx=\lim_{a\searrow 0} \int_a^1 \frac{(2x+1)e^{-x}}{\sqrt{x}^3}\; dx
=\lim_{a\searrow 0}  \left[ \frac{-2e^{-x}}{\sqrt{x}}\right]_a^1
=  \frac{-2e^{-1}}{\sqrt{1}} - \lim_{a\searrow 0} \frac{-2e^{-a}}{\sqrt{a}}.\]
\step
Der Grenzwert $ \lim_{a\searrow 0} \frac{-2e^{-a}}{\sqrt{a}}$ existiert jedoch nicht,
da der Ausdruck für $a\searrow 0$ gegen $-\infty$ divergiert.

Also existiert das uneigentliche Integral $\int_0^1 \frac{(2x+1)e^{-x}}{\sqrt{x}^3}\; dx$
nicht.

Damit existiert aber auch das gesuchte Integral nicht, völlig unabhängig davon, ob
das uneigentliche Integral $\int_1^{\infty} \frac{(2x+1)e^{-x}}{\sqrt{x}^3}\, dx$
existiert und welchen Wert es hätte.
  	 %------------------------------------END_STEP_X
 
  \end{incremental}
  %++++++++++++++++++++++++++++++++++++++++++++END_TAB_X


%#############################################################ENDE



\end{tabs*}
\end{content}