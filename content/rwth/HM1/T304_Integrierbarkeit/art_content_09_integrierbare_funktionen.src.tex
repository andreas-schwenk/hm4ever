%$Id:  $
\documentclass{mumie.article}
%$Id$
\begin{metainfo}
  \name{
    \lang{de}{Stammfunktion}
    \lang{en}{Antiderivatives}
  }
  \begin{description} 
 This work is licensed under the Creative Commons License Attribution 4.0 International (CC-BY 4.0)   
 https://creativecommons.org/licenses/by/4.0/legalcode 

    \lang{de}{Beschreibung}
    \lang{en}{Description}
  \end{description}
  \begin{components}
    \component{generic_image}{content/rwth/HM1/images/g_tkz_T304_MeanValue_B.meta.xml}{T304_MeanValue_B}
    \component{generic_image}{content/rwth/HM1/images/g_tkz_T304_MeanValue_A.meta.xml}{T304_MeanValue_A}
    \component{generic_image}{content/rwth/HM1/images/g_img_00_Videobutton_schwarz.meta.xml}{00_Videobutton_schwarz}
    \component{generic_image}{content/rwth/HM1/images/g_img_00_video_button_schwarz-blau.meta.xml}{00_video_button_schwarz-blau}
\end{components}
  \begin{links}
    \link{generic_article}{content/rwth/HM1/T304_Integrierbarkeit/g_art_content_08_integral_eigenschaften.meta.xml}{link1}
    \link{generic_article}{content/rwth/HM1/T301_Differenzierbarkeit/g_art_content_02_ableitungsregeln.meta.xml}{link2}
    \link{generic_article}{content/rwth/HM1/T107_Integralrechnung/g_art_content_25_stammfunktion.meta.xml}{teil1}
    \link{generic_article}{content/rwth/HM1/T301_Differenzierbarkeit/g_art_content_01_differenzenquotient.meta.xml}{link3}
    \link{generic_article}{content/rwth/HM1/T211_Eigenschaften_stetiger_Funktionen/g_art_content_33_zwischenwertsatz.meta.xml}{link4}
    \link{generic_article}{content/rwth/HM1/T301_Differenzierbarkeit/g_art_content_03_hoehere_ableitungen.meta.xml}{link5}
    \link{generic_article}{content/rwth/HM1/T107_Integralrechnung/g_art_content_24_integral_als_flaeche.meta.xml}{link6}
  \end{links}
  \creategeneric
\end{metainfo}
\begin{content}
\usepackage{mumie.ombplus}
\ombchapter{3}
\ombarticle{3}

\title{\lang{de}{Stammfunktion und Hauptsatz der Differential- und Integralrechnung}
       \lang{en}{Antiderivatives and fundamental theorem of calculus}}
 
\begin{block}[annotation]
  
  
\end{block}
\begin{block}[annotation]
  Im Ticket-System: \href{http://team.mumie.net/issues/10038}{Ticket 10038}\\
\end{block}

\begin{block}[info-box]
\tableofcontents
\end{block}

\section{\lang{de}{Stammfunktionen}\lang{en}{Antiderivatives}}\label{stamm}

\lang{de}{
Im \link{link1}{vorigen Abschnitt} haben wir gelernt, wann eine Funktion auf einem Intervall 
integrierbar ist. Da die Berechnung des Integral mittels der Definition recht umständlich ist, sind 
Formeln zur Berechnung wünschenswert. Zwar lassen sich nicht einmal für alle elementaren Funktionen 
Formeln angeben, z.\,B. nicht für das Integral von $f(x)=e^{-x^2}$ auf einem Intervall $[a;b]$, aber 
bei manchen Funktionen helfen \emph{Stammfunktionen} und der 
\emph{Hauptsatz der Differential- und Integralrechnung}.
\\\\
Stammfunktionen haben wir auch schon im \ref[teil1][ersten Teil]{def:stammfkt} kennengelernt.
Wir setzen die Kenntnisse von jenem Abschnitt hier voraus und werden sie hier etwas vertiefen und 
erweitern. Zunächst erinnern wir an die Definition:
}
\lang{en}{
In the \link{link1}{previous section} we defined the integrability of a function on an interval. 
As the computation of the integral via the definitions is cumbersome, some formulas and rules to 
this end are helpful to know. We should also keep in mind that not every elementary function can 
be so simply integrated, for example $f(x)=e^{-x^2}$ on the interval $[a;b]$. For some functions 
though, their \emph{antiderivatives} and the \emph{fundamental theorem of calculus} can be very 
helpful.
\\\\
Antiderivatives were already introduced in the \ref[teil1][first section]{def:stammfkt}. We revisit 
these here and will further investigate them. Firstly we recall the definition:
}

\begin{definition}\label{def:stammf}
\lang{de}{
Eine differenzierbare Funktion $F(x)$ heißt Stammfunktion von $f(x)$ auf einem Intervall $I$, falls 
gilt:
}
\lang{en}{
A differentiable function $F(x)$ is called the antiderivative of $f(x)$ on an interval $I$ if:
}
\[ F'(x) = f(x) \, \, \text{ \lang{de}{für alle}\lang{en}{for all} } x \in I . \]
\end{definition}

% \begin{example}
% %\emph{Beispiel:} 
% \lang{de}{Sei $f(x)=x^2+1$. Dann ist $F(x)=\frac{1}{3}x^3+x$ eine Stammfunktion
% von $f(x)$ auf $\mathbb{R}$, denn es gilt $F'(x)=x^2+1=f(x)$.}
% \lang{en}{If $f(x)=x^2+1$ then $F(x)=\frac{1}{3}x^3+x$ is an antiderivative of $f(x)$ on $\mathbb{R}$, since
% $F'(x)=x^2+1=f(x)$.}
% \end{example} 
% \lang{de}{Wie bestimmt man eine Stammfunktion
% $F(x)$, wenn nur $f(x)$ gegeben ist? Durch Umkehrung der Ableitung, d.h. man
%  sucht eine Funktion, deren Ableitung $f(x)$ ist. \\
%  }
% \lang{en}{How do we calculate an antiderivative
% $F(x)$ if we're only given $f(x)$? By working backwards from derivatives, i.e.
%  we look for a function whose derivative is $f(x)$. \\
%  }

\lang{de}{
Aus \ref[link3][den Ableitungsregeln]{rule:diffregeln} erhalten wir für einige Funktionen die 
Stammfunktion.
}
\lang{en}{
Using the \ref[link3][rules for differentiation]{rule:diffregeln} we can already obtain the 
antiderivatives of certain functions.
}
\begin{rule}\label{rule:diffregeln}
\begin{align*}
\underline{\text{\lang{de}{Funktion}\lang{en}{Function}}\; f(x)}&\hspace{20pt}&  \underline{\text{\lang{de}{Stammfunktion}\lang{en}{Antiderivative}}\; F(x) }&&\underline{\text{\lang{de}{Bedingung}\lang{en}{Conditions}}\, }\\
c\; (c \in \R) && cx && \\
x^n\;( n\in\N_0)&& \frac{1}{n+1} x^{n+1}&&\\
x^n\;(n\in\Z, n<-1)&&\frac{1}{n+1}x^{n+1}&& x\neq 0\\
x^r\;( r\in\R, r \neq -1)&&\frac{1}{r+1}x^{r+1}&& x>0\\
\sqrt{x}=x^{1/2} &&\frac{2}{3}x^{\frac{3}{2}}=\frac{2}{3}x\sqrt{x} \ \ \ \ \ \ \ \ \ \ \ \ && x>0\\
e^x&&e^x&&\\
\frac{1}{x}&&\ln|x|&&x\neq 0\\
a^x \;(a>0)&&\frac{1}{\ln a}\cdot a^x&&\\
\sin(x)&&-\cos(x)&&\\
\cos(x)&&\sin(x)&&\\
\frac{1}{\cos(x)^2}&& \tan(x) && x\notin \{ \frac{\pi}{2}+k\pi \,|\, k\in\Z \}\\
\frac{1}{1+x^2} && \arctan(x) && \\
\tan(x) && -\ln|\cos(x)| && x\notin \{ \frac{\pi}{2}+k\pi \,|\, k\in\Z \}
\end{align*}

\lang{de}{
\floatright{\href{https://www.hm-kompakt.de/video?watch=621}{\image[75]{00_Videobutton_schwarz}}}\\
}
\lang{en}{}
\end{rule}

\begin{block}[warning]
\lang{de}{
Stammfunktionen sind nur auf Intervallen definiert und bei der Integration darf nicht über 
Definitionslücken hinweg integriert werden. Die obige Regel ist deshalb so zu lesen, dass die 
Bedingung an $x$ auch eine Bedingung an das Intervall $I$ ist!
}
\lang{en}{
Antiderivatives are only defined on certain intervals, and we must be careful not to integrate 
across gaps in the domain of definition. The conditions in the above rule should thus be 
interpreted as conditions on the interval $I$ on which the function is defined.
}
\end{block}

\lang{de}{
Wir erinnern uns außerdem daran, dass Stammfunktionen nicht eindeutig sind.
Die Menge aller Stammfunktionen von $f(x)$ lässt sich 
genau beschreiben: Wenn $F_1(x)$ und $F_2(x)$ zwei beliebige
Stammfunktionen von $f(x)$ auf dem Intervall $I$ sind, so gilt
}
\lang{en}{
We recall that antiderivatives are not necessarily unique. 
The set of all antiderivatives of $f(x)$ can be precisely described: If $F_1(x)$ and $F_2(x)$ 
are any two antiderivatives of $f(x)$ on the interval $I$, then we have
}
\[  F_1'(x) - F_2'(x)= f(x) - f(x) = 0 . \]

\lang{de}{
Die Ableitung von $F_1(x)-F_2(x)$ ist also konstant gleich $0$.
Die Funktion $F_1(x)-F_2(x)$ ändert sich auf dem Intervall $I$ nicht und ist daher gleich 
einer konstanten Funktion: $F_1(x)-F_2(x) = C$, oder auch:
$F_1(x)=F_2(x)+C$. Zwei Stammfunktionen unterscheiden sich
also auf dem Intervall $I$ immer nur um eine konstante Verschiebung des Funktionswertes. \\
}
\lang{en}{
The derivative of $F_1(x)-F_2(x)$ is equal to $0$ on the whole interval. 
Hence the function $F_1(x)-F_2(x)$ does not change on the interval $I$ and is therefore equal 
to a constant $F_1(x)-F_2(x) = C$, or alternnatively $F_1(x)=F_2(x)+C$. Therefore any two 
antiderivatives of the same function only differ on the interval $I$ by a constant.
}

\begin{quickcheckcontainer}
\randomquickcheckpool{1}{1}
\begin{quickcheck}
		\field{real}
		\type{input.function}
		\begin{variables}
			\randint{v}{0}{1}
			\randint[Z]{a}{-5}{5}
			\randint[Z]{b}{1}{4}
			\randint{c}{-4}{4}
		    \function[normalize]{f}{4*a*x^3+v*b*sin(x)+(1-v)*b*cos(x)+c*exp(x)}
			\function[normalize]{sol}{a*x^4-v*b*cos(x)+(1-v)*b*sin(x)+c*exp(x)+v*b-c}
		\end{variables}


         \text{\lang{de}{
         Bestimmen Sie diejenige Stammfunktion $F(x)$ von $f(x) =\var{f}$, 
         für die $F(0) = 0$ gilt. \\ Antwort: $F(x)=$\ansref.
         }
         \lang{en}{
         Determine the unique antiderivative $F(x)$ of $f(x) =\var{f}$ 
         such that $F(0) = 0$ holds. \\ Answer: $F(x)=$\ansref.
         }}


		\begin{answer}
			\solution{sol}
			\checkAsFunction{x}{-5}{5}{50}
		\end{answer}
		\explanation{\lang{de}{
        $F(x)$ ist nach Definition eine Stammfunktion von $f(x)$, wenn $F'(x)=f(x)$ gilt.
        Außerdem ist $F(0)=0$, also ist $F(x)=\var{sol}$.
        }
        \lang{en}{
        $F(x)$ is by definition an antiderivative of $f(x)$ if $F'(x)=f(x)$ holds.
        Adding the condition $F(0)=0$ gives us $F(x)=\var{sol}$.
        }} 
	\end{quickcheck}
\end{quickcheckcontainer}

\section{\lang{de}{Der Hauptsatz der Differential- und Integralrechnung}
         \lang{en}{The fundamental theorem of calculus}}\label{sec:hauptsatz}

\lang{de}{
In diesem Abschnitt stellen wir nun die Verbindung zwischen dem \emph{bestimmten
Integral} $\big\int_a^b f(x)\, dx$ und den Stammfunktionen her.
\\\\
Zunächst beschäftigen wir uns mit dem \emph{Mittelwertsatz der Integralrechnung}.
}
\lang{en}{
In this section we develop the connection between the \emph{definite integral} 
$\big\int_a^b f(x)\, dx$ and the antiderivative. This will also clarify why the set of 
antiderivatives of $f(x)$ is called the \emph{indefinite integral}.
\\\\
Firstly we consider the \emph{mean value theorem for integrals}.
}

\begin{theorem}[\lang{de}{Mittelwertsatz der Integralrechnung}
                \lang{en}{Mean value theorem for integrals}]\label{thm:mittelwertsatz}
\lang{de}{
Es seien $a < b$ reelle Zahlen und $f:[a;b]\to \R$ eine stetige Funktion. Dann existiert ein 
Zwischenwert $x_0 \in (a;b)$, sodass
}
\lang{en}{
Let $a < b$ be real numbers and $f:[a;b]\to \R$ a continuous function. Then there exists a point 
$x_0 \in (a;b)$ such that
}
\[
\int_a^b f(x)\, dx = (b-a) \cdot f(x_0).
\]
\end{theorem}

\begin{example}
\lang{de}{Die Aussage lässt sich anschaulich am nächsten Bild erklären.}
\lang{en}{The theorem is visualised in the following image.}
\begin{center}
\image{T304_MeanValue_A}
\end{center}
\lang{de}{
Das Bild zeigt den Graphen von $f(x) = x^3$.
Die orangene Fläche ist genauso groß wie die
blaue Fläche. Es ist $\int_0^1 f(x)\, dx = \frac{1}{4}$.
Also ist die Höhe des blauen Rechtecks $f(x_0)=\frac{1}{4}$.
Der Mittelwertsatz gibt keine Aussage über die genaue Lage von $x_0$. In diesem Beispiel
wäre $x_0 = \frac{1}{\sqrt[3]{4}}$.
}
\lang{en}{
The image shows the graph of $f(x) = x^3$. 
The size of the orange area is the same as that of the blue area. 
It is $\int_0^1 f(x)\, dx = \frac{1}{4}$, and so the height of the blue rectangle is 
$f(x_0)=\frac{1}{4}$. The mean value theorem does not tell us what $x_0$ is. In this 
example, it is $x_0 = \frac{1}{\sqrt[3]{4}}$.
}
\end{example}

\begin{proof*}[\lang{de}{Beweis des Mittelwertsatzes der Integralrechnung}
               \lang{en}{Proof of the mean value theorem for integrals}]
\lang{de}{
Der Mittelwertsatz ist im Wesentlichen eine Folgerung des 
\ref[link4][Zwischenwertsatzes]{thm:zwischenwertsatz}.
}
\lang{en}{
The mean value theorem is a consequence of the 
\ref[link4][intermediate value theorem]{thm:zwischenwertsatz}.
}

\begin{incremental}
\step
\lang{de}{
Wir betrachten das Intervall $[a;b]$. Wir betrachten die Ober- und Untersumme von $f$ auf dem Intervall ohne Zwischenstellen. (Wir betrachten also eine Zerlegung von $[a;b]$ ohne echte Teilintervalle.)
Dann ist
}
\lang{en}{
Consider the interval $[a;b]$. We will consider the lower and upper sums of $f$ on the partition 
into the whole interval (rather than on a partition into subintervals). Then
}
\[
U([a;b]) = \min_{x \in [a;b]} f(x) \cdot (b-a)
\]
\lang{de}{und}
\lang{en}{and}
\[
O([a;b]) = \max_{x \in [a;b]} f(x) \cdot (b-a).
\]
\lang{de}{Es gilt}
\lang{en}{We have}
\[
U([a;b]) \leq \int_a^b f(x) \, dx \leq O([a;b]).
\]
\lang{de}{Äquivalent umformuliert heißt das}
\lang{en}{Equivalently,}
\[
\min_{x \in [a;b]} f(x) \leq \frac{1}{b-a} \int_a^b f(x) \, dx \leq \max_{x \in [a;b]} f(x).
\]
\step
\lang{de}{
Nach dem \ref[link4][Zwischenwertsatz]{thm:zwischenwertsatz} (da $f$ stetig ist) existiert also eine
Zwischenstelle $x_0$ mit
}
\lang{en}{
By the \ref[link4][intermediate value theorem]{thm:zwischenwertsatz} (as $f$ is continuous), there 
exists a point $x_0$ between $a$ and $b$ such that
}
\[
f(x_0) = \frac{1}{b-a} \int_a^b f(x) \, dx.
\]
\end{incremental}
\end{proof*}

\lang{de}{Eine Folgerung aus dieser Beobachtung ist der Hauptsatz.}
\lang{en}{A consequence of this observation is the main theorem of this section.}

\begin{theorem}[\lang{de}{Hauptsatz der Differential- und Integralrechnung}
                \lang{en}{The fundamental theorem of calculus}]\label{thm:fundamental}

\lang{de}{
Ist $f(x)$ auf dem Intervall $[a;b]$ stetig, dann %(d.h. eine kontinuierliche Funktion ohne Sprungstellen oder Lücken), dann 
existiert eine Stammfunktion $F(x)$ von $f(x)$ und 
für eine beliebige Stammfunktion $F(x)$ gilt
}
\lang{en}{
Let $f(x)$ be continuous on an interval $[a,b]$. Then there exists an antiderivative $F(x)$ of 
$f(x)$ (so $F'(x)=f(x)$), and for any antiderivative $F(x)$ of $f(x)$ we have
}
\[ \int_a^b f(x)\, dx = F(b) - F(a) .\]
%\lang{de}{\floatright{\href{https://www.hm-kompakt.de/video?watch=615}{\image[75]{00_Videobutton_schwarz}}}\\~}
\end{theorem}

\lang{de}{
Das bestimmte Integral kann also durch Einsetzen der Grenzen in eine Stammfunktion ausgerechnet 
werden. Für die Differenz $F(b)-F(a)$ schreibt man auch $\big[F(x)\big]_a^b$. Der Hauptsatz
der Differential- und Integralrechnung besagt also
}
\lang{en}{
The definite integral can therefore be calculated by evaluating the antiderivative at both limits 
and taking the difference. To denote $F(b)-F(a)$ in this context we often write 
$\big[F(x)\big]_a^b$. The fundamental theorem of calculus essentially states that
}
\[ \int_a^b f(x)\, dx = \Big[\, F(x)\, \Big]_a^b  . \]

\begin{proof*}[\lang{de}{Beweis des Hauptsatzes}
               \lang{en}{Proof of the fundamental theorem of calculus}]
\lang{de}{Wir zeigen zunächst, dass eine Stammfunktion existiert.}
\lang{en}{We firstly prove that an antiderivative exists.}
\begin{incremental}
\step
\lang{de}{Dazu betrachten wir}
\lang{en}{We consider}
\[
F : [a;b] \to \R, \, \, x \mapsto \int_a^x f(t)\, dt.
\]
\lang{de}{
Wir zeigen, dass $F$ eine Stammfunktion von $f$ ist. Wir zeigen mit der Definition, dass $F'(x)=f(x)$ gilt.
}
\lang{en}{
We show that $F$ is an antiderivative of $f$. Using the definition, we verify that $F'(x)=f(x)$.
}
\[
\frac{F(x+h)-F(x)}{h} = \frac{\int_a^{x+h} f(t)\, dt - \int_a^x f(t)\, dt}{h}
= \frac{\int_x^{x+h} f(t)\, dt}{h}.
\]
\step
\lang{de}{
Der Mittelwertsatz \ref{thm:mittelwertsatz} liefert uns eine Zwischenstelle $x_0 \in (x;x+h)$ mit
}
\lang{en}{
The mean value theorem \ref{thm:mittelwertsatz} gives us a point $x_0 \in (x;x+h)$ such that
}
\[
\int_x^{x+h} f(t)\, dt = f(x_0)\cdot h.
\]
\lang{de}{Dann ist}
\lang{en}{Then}
\[
\lim_{h \to 0} \, \frac{F(x+h)-F(x)}{h} = \lim_{h \to 0} \, \frac{f(x_0)\cdot h}{h} = \lim_{h \to 0} \, f(x_0) = f(x),
\]
\lang{de}{
denn für $h \to 0$ läuft die Zwischenstelle $x_0 \to x$. (Hier geht wieder die Stetigkeit von $f$ ein.)
Damit ist $F$ eine Stammfunktion von $f$.
}
\lang{en}{
as taking $h \to 0$ results in $x_0 \to x$. (We use the continuity of $f$ for this.) 
Hence $F$ is an antiderivative of $f$.
}

\step
\lang{de}{
Der Wert hängt nicht von der Wahl einer Stammfunktion ab. In der Tat würde eine konstante
Verschiebung von $F(x)$ (d.\,h. $F(x)+C$ statt $F(x)$) bei der Bildung der Differenz wieder wegfallen!
}
\lang{en}{
The value of the difference of the antiderivative evaluated at the two limit points does not depend 
on the choice of antiderivative. This is because any constant added to the antiderivative $F(x)$ 
(i.e. using $F(x)+C$ instead of $F(x)$) is cancelled in the process of finding the difference.
}
\end{incremental}
\end{proof*}

\begin{remark}
\lang{de}{
Der Mittelwertsatz der Integralrechnung \ref{thm:mittelwertsatz} lässt sich zusammen mit dem Hauptsatz
der Integral- und Differentialrechnung \ref{thm:fundamental} so lesen:
\\\\
Für eine stetige Funktion $f:[a;b]\to \R$ mit Stammfunktion $F$ existiert eine Zwischenstelle $x_0 \in (a;b)$, sodass
}
\lang{en}{
The mean value theorem for integrals \ref{thm:mittelwertsatz} can be combined with the 
fundamental theorem of calculus \ref{thm:fundamental} to give:
\\\\
Given a continuous function $f:[a;b]\to \R$ with antiderivative $F$, there exists a point 
$x_0 \in (a;b)$ such that 
}
\[
f(x_0)= \frac{1}{b-a} \int_a^b f(x)\, dx = \frac{F(b)-F(a)}{b-a}.
\]
\lang{de}{
Das ist die gleiche Aussage wie vom 
\ref[link5][Mittelwertsatz der Differentialrechnung]{thm:mittelwertsatz} angewendet auf $F$.
}
\lang{en}{
This is simply the 
\ref[link5][mean value theorem for integrals]{thm:mittelwertsatz} applied to $F$.
}
\end{remark}

\begin{example}
\lang{de}{
Die Geschwindigkeit (in km/h) eines Wanderers in einem Beobachtungszeitraum von $2$ Stunden kann
näherungsweise durch die Funktion $f(t)= \frac{5}{t+1}+t +\cos(4\pi t)$
beschrieben werden.
\\\\
Wir wollen die Durchschnittsgeschwindigkeit ermitteln.
In der folgenden Abbildung ist die Funktion dargestellt:
}
\lang{en}{
Suppose that the speed (in km/h) of a hiker over an interval of $2$ hours can be approximated by 
the function $f(t)= \frac{5}{t+1}+t +\cos(4\pi t)$.
\\\\
We want to calculate the average speed of the hiker. 
The function is grapheed in the following diagram.
}
\begin{center}
\image{T304_MeanValue_B}
\end{center}

\lang{de}{
Die Durchschnittsgeschwindigkeit berechnen wir, indem
wir die zurückgelegte Strecke durch die verstrichene Zeit dividieren.
Die zurückgelegte Strecke in Kilometern seit Beobachtungsbeginn
ist durch die Fläche unter dem Graphen von $f$ gegeben. (Zur Begründung verweisen wir
auf \ref[link6][dieses Beispiel aus dem ersten Teil]{weg}.)
\\\\
Die blaue Fläche ist ungefähr $\approx 7,49$. Der Wanderer legt also circa $7,49$ km in den 2 Stunden zurück.
Die Durchschnittsgeschwindigkeit beträgt also $\frac{7,49 km}{2 h} \approx 3,75 \frac{km}{h}.$
\\\\
In der Grafik ist die Durchschnittsgeschwindigkeit in Orange eingezeichnet. Das orangene Rechteck in der Zeichnung ist genauso groß wie die blaue Fläche.
\\\\
Mit Hilfe des Hauptsatzes der Differential- und Integralrechnung können wir dies verifizieren:
Eine Stammfunktion $F$ von $f$ ist beispielsweise
}
\lang{en}{
The mean velocity is calculated by dividing the amount of distance covered by the hiker by the 
amount of time that has passed. The distance in kilometres covered by the hiker is given by the 
area under the graph of $f$. (The justification for this can be found in 
\ref[link6][this example from the first part]{weg}.)
\\\\
The blue area is approximately $\approx 7.49$. The hiker therefore covers about $7.49$ km in those 
two hours. The mean speed of the hiker is therefore $\frac{7.49 km}{2 h} \approx 3.75 \frac{km}{h}.$
\\\\
In the diagram the mean speed is denoted in orange. The orange rectangle has the exact same area as 
the blue shape under the graph.
\\\\
Using the fundamental theorem of calculus, we can verify this. The following function can easily be 
verified to be an antiderivative of $f$:
}
\[
F(t) = 5 \cdot \ln(t+1) + \frac{1}{2}t^2+ \frac{1}{4\pi} \sin(4\pi t).
\]
\lang{de}{Damit gilt für die Durchschnittsgeschwindigkeit $d$}
\lang{en}{Hence the mean speed is}
\[
d = \frac{F(2)-F(0)}{2-0} = \frac{5}{2}\cdot \ln(3)+ 1 \approx 3\lang{de}{,}\lang{en}{.}75.
\]

\lang{de}{
Die Aussage des Mittelwertsatzes können wir so interpretieren:
Es gibt einen Zeitpunkt, zu dem der Wanderer genau die Durchschnittsgeschwindigkeit $d$ schnell ist.
Dies ist offenbar mehrmals in den 2 Stunden der Fall.
}
\lang{en}{
The mean value theorem for integrals can be interpreted as follows: 
There exists a moment in time at which the hiker is travelling precisely at his mean speed. 
This visibly occurs multiple times across the two hours in this example.
}
\end{example}

\lang{de}{
Eine Zusammenfassung des Abschnitts findet sich in diesem Video:
\floatright{\href{https://api.stream24.net/vod/getVideo.php?id=10962-2-10815&mode=iframe&speed=true}{\image[75]{00_video_button_schwarz-blau}}}\\
}
\lang{en}{}

% \begin{example}
% %\emph{Beispiel:} 
% \lang{de}{$f(x)=\frac{2}{x^2} - 6 x^2$ ist auf dem Intervall $[1;2]$ stetig. Die Funktion
% $F(x)=- \frac{2}{x} - 2x^3 + 20$ ist eine Stammfunktion und
% \[ \int_1^2 \Big(\frac{2}{x^2} - 6 x^2 \Big)\, dx =
% \Big[- \frac{2}{x} - 2x^3 + 20 \Big]_1^2 = 3 - 16 = -13. \]

% Die Funktion $f(x)$, die Stammfunktion $F(x)$ und der orientierte Flächeninhalt 
% $\big\int_1^2 f(x)\, dx = F(2)-F(1)=-13$ sind
% in der folgenden Abbildung dargestellt:}
% \lang{en}{$f(x)=\frac{2}{x^2} - 6 x^2$ is continuous on the interval $[1,2]$. The function
% $F(x)=- \frac{2}{x} - 2x^3 + 20$ is an antiderivative and
% \[ \int_1^2 \Big(\frac{2}{x^2} - 6 x^2 \Big)\, dx =
% \Big[- \frac{2}{x} - 2x^3 + 20 \Big]_1^2 = 3 - 16 = -13. \]

% The function $f(x)$, the antiderivative $F(x)$ and the signed area 
% $\big\int_1^2 f(x)\, dx = F(2)-F(1)=-13$ are depicted
% in the following image:}
% \begin{center}
% \image{image2}
% \end{center}
% \end{example}

\end{content}