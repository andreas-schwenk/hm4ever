\documentclass{mumie.problem.gwtmathlet}
%$Id$
\begin{metainfo}
  \name{
    \lang{en}{Problem 10}
    \lang{de}{A10: Anwendungsaufgabe}
  }
  \begin{description} 
 This work is licensed under the Creative Commons License Attribution 4.0 International (CC-BY 4.0)   
 https://creativecommons.org/licenses/by/4.0/legalcode 

    \lang{en}{Problem 10}
    \lang{de}{A10: Anwendungsaufgabe}
  \end{description}
  \corrector{system/problem/GenericCorrector.meta.xml}
  \begin{components}
    \component{generic_image}{content/rwth/HM1/images/g_tkz_T107_Problem10.meta.xml}{T107_Problem10}
    \component{js_lib}{system/problem/GenericMathlet.meta.xml}{gwtmathlet}
  \end{components}
  \begin{links}
  \end{links}
  \creategeneric
\end{metainfo}



\begin{content}
\begin{block}[annotation]
	Im Ticket-System: \href{https://team.mumie.net/issues/18762}{Ticket 18762}
\end{block}
\usepackage{mumie.genericproblem}


\lang{de}{
	\title{A10: Anwendungsaufgabe}
}
\lang{en}{
	\title{Problem 10}
}


\begin{problem}

    \begin{variables}
    \randint{a1}{5}{10} %a/2 in image
    \randint{b1}{1}{4} %b/2 in image
    \randint{L}{5}{10}
    \function[calculate]{a}{2*a1}
    \function[calculate]{b}{2*b1}
    \function[calculate]{H}{a^2/4}
    \function[calculate]{h}{(a/2)^2 - (b/2)^2}
    \function[calculate]{sol1}{-2/3*(a/2)^3+(a^2/4)*a - (-2/3*(b/2)^3 + (a^2/4-h)*b)}
    \function[calculate]{sol}{sol1*L}
    \end{variables}

    \begin{question}
    \type{input.number}
    \field{rational}

    \lang{de}{
    \text{Ein wasserführender Stollen habe einen parabelförmigen Querschnitt, 
    der sich durch die Funktion 
    $ f(x)=-\left(x-\frac{a}{2}\right)\left(x+\frac{a}{2}\right)  $ 
    mit $a=\var{a}$, $b=\var{b}$, $H=\var{H}$,
    und $h=\var{h}$ beschreiben lasse 
    (Bezeichnungen wie in der Abbildung und Maße in Meter).
    \\
    Der Stollen ist bis zu einer Höhe von $h=\var{h}$ mit Wasser gefüllt.
\begin{figure}
\image{T107_Problem10}
\end{figure}
    \\
    Bestimmen Sie das Volumen des Wassers in $m^3$, das durch den Stollen fließen kann, wenn dieser \var{L} Meter lang ist!
    }
    }
    
    \lang{en}{
    \text{A waterbearing tunnel has a parabola shaped cross-section, 
    which can be described by the function 
    $     f(x)=-\left(x-\frac{a}{2}\right)\left(x+\frac{a}{2}\right)   $
    with $a=\var{a}$, $b=\var{b}$, $H=\var{H}$,
    and $h=\var{h}$ 
    (labels as in the illustration and units in meter).
    \\
    The tunnel is filled with water up to a height of $h=\var{h}$.
    \image{T107_Problem10_Stollen}
    \\
    Determine the volume of water in $m^3$ that can flow through the tunnel if the tunnel is $\var{L}$ meters long!
    }
    }
        \begin{answer}
        \lang{de}{
        \text{Das Volumen beträgt }
        }
        \lang{en}{
        \text{The volume is }
        }
        \solution{sol}
        \end{answer}
        
        \lang{de}{
        \explanation{Berechnen Sie zuerst den Flächeninhalt des Querschnitts des Stollens $\int_{-\frac{a}{2}}^{\frac{a}{2}} f(x) \, dx$ und subtrahieren Sie von diesem den Flächeninhalt des nicht wasserführenden Teils $\int_{-\frac{b}{2}}^{\frac{b}{2}} f(x) \, dx -h\cdot b$. Beachten Sie dann noch die gegebene Länge des Stollens, um das Volumen zu bestimmen.}
        }
        \lang{en}{
        \explanation{First calculate the area of the cross-section of the tunnel $\int_{-\frac{a}{2}}^{\frac{a}{2}} f(x) \, dx$ and then subtract the area of the part of the cross-section that is not waterbearing $\int_{-\frac{b}{2}}^{\frac{b}{2}} f(x) \, dx -h\cdot b$. After that, take the length of the tunnel into account, to calculate the volume.}
        }
    \end{question}
\end{problem}
    

\embedmathlet{gwtmathlet}

\end{content}








