\documentclass{mumie.problem.gwtmathlet}
%$Id$
 \begin{metainfo}
 \name{
    \lang{de}{A09: Unbestimmtes Integral}
    \lang{en}{Problem 9}
  }
 \begin{description} 
 This work is licensed under the Creative Commons License Attribution 4.0 International (CC-BY 4.0)   
 https://creativecommons.org/licenses/by/4.0/legalcode 

    \lang{de}{Aufgabe 9}
    \lang{en}{Problem 9}
 \end{description}
 \corrector{system/problem/GenericCorrector.meta.xml}
 \begin{components}
   \component{js_lib}{system/problem/GenericMathlet.meta.xml}{gwtmathlet}
 \end{components}
 \begin{links}
 \end{links}
 \creategeneric
 \end{metainfo}
 
 \begin{content}
\begin{block}[annotation]
	Im Ticket-System: \href{https://team.mumie.net/issues/18687}{Ticket 18687}
\end{block}
   \usepackage{mumie.genericproblem}
   \title {
       \lang{de}{A09: Unbestimmtes Integral}
       \lang{en}{Problem 9}
   }
 \begin{problem}
 
 \randomquestionpool{1}{3}
 
 \begin{question}
 \type{input.generic}
 
  \lang{de}{\text{Welche der folgenden Formeln für unbestimmte Integrale sind richtig, falls $x>0$ gilt und
$C \in \mathbb{R}$ eine beliebige Konstante ist?}}
\lang{en}{\text{Which of the following indefinite integral formulas are correct,
if $x>0$ and $C \in \mathbb{R}$ is an arbitrary constant?}}

      \begin{variables}
      	\randint{a}{2}{8}
      \end{variables}

     


\begin{answer}
\type{mc.multiple}
%\permutechoices{1}{5}

 \begin{choice}
     \text{$ \int \ln(x)\, \text{ d}x = \frac{1}{x} + C$}
     \solution{false}
 \end{choice}
 
 \begin{choice}
   \text{$\int \frac{1}{x^2}\, \text{ d}x = - \frac{1}{x}$}
   \solution{false}
 \end{choice}
 
 \begin{choice}
   \text{$\int \ln(x)\, \text{ d}x = x \ln(x) - x + C$}
   \solution{true} 
 \end{choice}
 
 \begin{choice}
   \text{$\int \frac{1}{x}\, \text{ d}x = \ln(\var{a}x) + C$}
   \solution{true} 
 \end{choice}
 
 \begin{choice}
   \text{$\int \frac{1}{x}\, \text{ d}x = C \ln(x)$}
   \solution{false} 
 \end{choice}
 
 
\lang{de}{
\explanation[equalChoice(ans,1????)]{$\frac{1}{x}$ ist die Ableitung von $\ln(x)$, aber nicht die Stammfunktion.}
\explanation[equalChoice(ans,?1???)]{Das unbestimmte Integral berechnet alle möglichen Stammfunktionen. Deshalb fehlt hier die Addition einer beliebigen Konstante $C$.}
\explanation[equalChoice(ans,??0??)]{Leiten Sie $x\ln(x)-x+C$ ab und prüfen Sie, ob Sie $\ln(x)$ erhalten.}
\explanation[equalChoice(ans,???0?)]{Hinweis: Es ist $\ln(\var{a}x)=\ln(x)+\ln(\var{a})$.}
\explanation[equalChoice(ans,????1)]{Die Konstante $C$ ist eine additive Konstante, sie darf nicht multipliziert werden, wie man auch durch Ableiten überprüfen kann.}
}

\lang{en}{
\explanation[equalChoice(ans,1????)]{$\frac{1}{x}$ is the derivative of $\ln(x)$, not its antiderivative.}
\explanation[equalChoice(ans,?1???)]{The indefinite integral is the set of all antiderivatives. Here the additive constant $C$ is missing.}
\explanation[equalChoice(ans,??0??)]{Derive $x\ln(x)-x+C$ and obtain $\ln(x)$.}
\explanation[equalChoice(ans,???0?)]{Hint: We have $\ln(\var{a}x)=\ln(x)+\ln(\var{a})$.}
\explanation[equalChoice(ans,????1)]{The constant $C$ is not multiplicative as one can see by computing the derivative.}
}

\end{answer}
\end{question}



 \begin{question}
 \type{input.generic}
 
  \lang{de}{\text{Welche der folgenden Formeln für unbestimmte Integrale sind richtig, falls $x>0$ gilt und
$C \in \mathbb{R}$ eine beliebige Konstante ist?}}
\lang{en}{\text{Which of the following indefinite integral formulas are correct,
if $x>0$ and $C \in \mathbb{R}$ is an arbitrary constant?}}

      \begin{variables}
      	\randint{a}{2}{8}
      \end{variables}

     


\begin{answer}
\type{mc.multiple}
%\permutechoices{1}{5}
 \begin{choice}
   \text{$\int \frac{1}{x}\, \text{ d}x = C \ln(x)$}
   \solution{false} 
 \end{choice}
 
 \begin{choice}
     \text{$ \int \ln(x)\, \text{ d}x = \frac{1}{x} + C$}
     \solution{false}
 \end{choice}
 
 \begin{choice}
   \text{$\int \frac{1}{x}\, \text{ d}x = \ln(\var{a}x) + C$}
   \solution{true} 
 \end{choice}
  
 \begin{choice}
   \text{$\int \frac{1}{x^2}\, \text{ d}x = - \frac{1}{x}$}
   \solution{false}
 \end{choice}
 \begin{choice}
   \text{$\int \ln(x)\, \text{ d}x = x \ln(x) - x + C$}
   \solution{true} 
 \end{choice}
 
\lang{de}{
\explanation[equalChoice(ans,1????)]{Die Konstante $C$ ist eine additive Konstante, sie darf nicht multipliziert werden, wie man auch durch Ableiten überprüfen kann.}
\explanation[equalChoice(ans,?1???)]{$\frac{1}{x}$ ist die Ableitung von $\ln(x)$, aber nicht die Stammfunktion.}
\explanation[equalChoice(ans,??0??)]{Hinweis: Es ist $\ln(\var{a}x)=\ln(x)+\ln(\var{a})$.}
\explanation[equalChoice(ans,???1?)]{Das unbestimmte Integral berechnet alle möglichen Stammfunktionen. Deshalb fehlt hier die Addition einer beliebigen Konstante $C$.}
\explanation[equalChoice(ans,????0)]{Leiten Sie $x\ln(x)-x+C$ ab und prüfen Sie, ob Sie $\ln(x)$ erhalten.}
}

\lang{en}{
\explanation[equalChoice(ans,1????)]{The constant $C$ is not multiplicative as one can see by computing the derivative.}
\explanation[equalChoice(ans,?1???)]{$\frac{1}{x}$ is the derivative of $\ln(x)$, not its antiderivative.}
\explanation[equalChoice(ans,??0??)]{Hint: We have $\ln(\var{a}x)=\ln(x)+\ln(\var{a})$.}
\explanation[equalChoice(ans,???1?)]{The indefinite integral is the set of all antiderivatives. Here the additive constant $C$ is missing.}
\explanation[equalChoice(ans,????0)]{Derive $x\ln(x)-x+C$ and obtain $\ln(x)$.}
}

\end{answer}
\end{question}
 
  \begin{question}
 \type{input.generic}
 
  \lang{de}{\text{Welche der folgenden Formeln für unbestimmte Integrale sind richtig, falls $x>0$ gilt und
$C \in \mathbb{R}$ eine beliebige Konstante ist?}}
\lang{en}{\text{Which of the following indefinite integral formulas are correct,
if $x>0$ and $C \in \mathbb{R}$ is an arbitrary constant?}}

      \begin{variables}
      	\randint{a}{2}{8}
      \end{variables}

     


\begin{answer}
\type{mc.multiple}
%\permutechoices{1}{5}

 \begin{choice}
   \text{$\int \frac{1}{x}\, \text{ d}x = \ln(\var{a}x) + C$}
   \solution{true} 
 \end{choice}
 
 \begin{choice}
   \text{$\int \frac{1}{x}\, \text{ d}x = C \ln(x)$}
   \solution{false} 
 \end{choice}
 
 \begin{choice}
     \text{$ \int \ln(x)\, \text{ d}x = \frac{1}{x} + C$}
     \solution{false}
 \end{choice}
 
 \begin{choice}
   \text{$\int \frac{1}{x^2}\, \text{ d}x = - \frac{1}{x}+C$}
   \solution{true}
 \end{choice}
 \begin{choice}
   \text{$\int \ln(x)\, \text{ d}x = x \ln(x) - x$}
   \solution{false} 
 \end{choice}
 
 \lang{de}{
 \explanation[equalChoice(ans,0????)]{Hinweis: Es ist $\ln(\var{a}x)=\ln(x)+\ln(\var{a})$.}
 \explanation[equalChoice(ans,?1???)]{Die Konstante $C$ ist eine additive Konstante, sie darf nicht multipliziert werden, wie man auch durch Ableiten überprüfen kann.}
 \explanation[equalChoice(ans,??1??)]{$\frac{1}{x}$ ist die Ableitung von $\ln(x)$, aber nicht die Stammfunktion.}
 \explanation[equalChoice(ans,???0?)]{Leiten Sie $-\frac{1}{x}$ ab und prüfen Sie, ob Sie $\frac{1}{x^2}$ erhalten.}
\explanation[equalChoice(ans,????1)]{Das unbestimmte Integral berechnet alle möglichen Stammfunktionen. Deshalb fehlt hier die Addition einer beliebigen Konstante $C$.}
}

 \lang{en}{
 \explanation[equalChoice(ans,0????)]{Hint: We have $\ln(\var{a}x)=\ln(x)+\ln(\var{a})$.}
 \explanation[equalChoice(ans,?1???)]{The constant $C$ is not multiplicative as one can see by computing the derivative.}
 \explanation[equalChoice(ans,??1??)]{$\frac{1}{x}$ is the derivative of $\ln(x)$, not its antiderivative.}
 \explanation[equalChoice(ans,???0?)]{Derive $- \frac{1}{x}+C$ and obtain $\frac{1}{x^2}$.}
\explanation[equalChoice(ans,????1)]{The indefinite integral is the set of all antiderivatives. Here the additive constant $C$ is missing.}
}
 
\end{answer}
\end{question}
 
 \end{problem}
\embedmathlet{gwtmathlet}
\end{content}

