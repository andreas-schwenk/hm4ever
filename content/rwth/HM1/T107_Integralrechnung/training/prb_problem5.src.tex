\documentclass{mumie.problem.gwtmathlet}
%$Id$
\begin{metainfo}
  \name{
    \lang{de}{A05: Hauptsatz der Integralrechnung}
    \lang{en}{problem 5}
  }
  \begin{description} 
 This work is licensed under the Creative Commons License Attribution 4.0 International (CC-BY 4.0)   
 https://creativecommons.org/licenses/by/4.0/legalcode 

    \lang{de}{}
    \lang{en}{}
  \end{description}
  \corrector{system/problem/GenericCorrector.meta.xml}
  \begin{components}
    \component{js_lib}{system/problem/GenericMathlet.meta.xml}{mathlet}
  \end{components}
  \begin{links}
  \end{links}
  \creategeneric
\end{metainfo}
\begin{content}
\usepackage{mumie.ombplus}
\usepackage{mumie.genericproblem}


\lang{de}{
	\title{A05: Hauptsatz der Integralrechnung}
}
\lang{en}{
	\title{Problem 5}
}

\begin{block}[annotation]
  Im Ticket-System: \href{http://team.mumie.net/issues/9613}{Ticket 9613}
\end{block}

\begin{problem}
	\begin{variables}
	 	\randint{a}{-2}{3}
	 	\randint{b}{0}{4}
	 	\randadjustIf{a,b}{b<=a}

	 	\randint[Z]{c}{-3}{5}
	 	\randint[Z]{d}{-4}{5}
	 	\randint[Z]{u}{-5}{5}

		\function{f}{d*x^4+u*x^2+c*x}
		\derivative[normalize]{f_1}{f}{x}
		\functionNormalize{f_1}

        %\substitute[normalize]{fx0}{f}{x}{x0} % if you want to have the value computed, use this option
        \substitute{Fb}{f}{x}{b}
        \substitute{Fa}{f}{x}{a}
        \function[calculate]{L}{Fb-Fa}


	\end{variables}


	\begin{question}
		\lang{de}{
			\text{In dieser Aufgabe sei die Funktion $f(x) = \var{f_1}$ gegeben.\\
			Das Ziel der Aufgabe ist, den orientierten Flächeninhalt $\int_{\var{a}}^{\var{b}} f(x) \, dx$ zu bestimmen. \\
			Dazu dienen die folgenden Schritte:\\ \\
			Bestimmen Sie zuerst eine Stammfunktion $F$ von $f$:}
		}
		 \lang{en}{\text{
      In this problem, we are given the function $f(x) = \var{f_1}$.\\
      The goal is to determine the size of the oriented area
      $\int_{\var{a}}^{\var{b}} f(x) \, dx$. \\
      In order to find this area, we will use the following steps: \\ \\
      First, determine an antiderivative $F$ of $f$:
     }}
		\type{input.function}

		\begin{answer}
			\text{ $F(x) = $}
			\solution{f}
			\inputAsFunction{x}{inp}
			\checkFuncForZero{D[inp]-f_1}{-10}{10}{100}
            
            \lang{de}{
            \explanation[edited(ans)]{Die eingegebene Stammfunktion ist falsch. Beachten Sie, dass $(x^n)\prime = nx^{n-1}$ für $n\geq 1$ gilt.}
            }
            \lang{en}{
            \explanation[edited(ans)]{The entered function is incorrect. Note that $(x^n)\prime = nx^{n-1}$ for $n\geq 1$.}
            }            
		\end{answer}
	\end{question}


	\begin{question}
		\lang{de}{
			\text{Berechnen Sie nun $\int_{\var{a}}^{\var{b}} f(x) \, dx$.}
		}
		 \lang{en}{\text{
      Now calculate $\int_{\var{a}}^{\var{b}} f(x) \, dx$.
     }}
		\type{input.number}
		\begin{answer}
			\text{ $\int_{\var{a}}^{\var{b}} f(x) \, dx = $ }
			\solution{L}
            
            \lang{de}{
            \explanation[edited(ans)]{Für eine beliebige Stammfunktion $F$ von $f$ gilt $\int_{\var{a}}^{\var{b}} f(x) \, dx = F(\var{b})-F(\var{a})$.}
            }
            \lang{en}{
            \explanation[edited(ans)]{For an arbitrary antiderivative $F$ of $f$ the equation $\int_{\var{a}}^{\var{b}} f(x) \, dx = F(\var{b})-F(\var{a})$ holds.}
            }            
		\end{answer}

	\end{question}

\end{problem}


\embedmathlet{mathlet}
\end{content}