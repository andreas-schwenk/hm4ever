\documentclass{mumie.problem.gwtmathlet}
%$Id$
\begin{metainfo}
  \name{
    \lang{de}{A03: Stammfunktion}
    \lang{en}{problem 3}
  }
  \begin{description} 
 This work is licensed under the Creative Commons License Attribution 4.0 International (CC-BY 4.0)   
 https://creativecommons.org/licenses/by/4.0/legalcode 

    \lang{de}{}
    \lang{en}{}
  \end{description}
  \corrector{system/problem/GenericCorrector.meta.xml}
  \begin{components}
    \component{js_lib}{system/problem/GenericMathlet.meta.xml}{mathlet}
  \end{components}
  \begin{links}
  \end{links}
  \creategeneric
\end{metainfo}
\begin{content}
\usepackage{mumie.ombplus}
\usepackage{mumie.genericproblem}


\lang{de}{
	\title{A03: Stammfunktion}
}
\lang{en}{
	\title{Problem 3}
}

\begin{block}[annotation]
  Im Ticket-System: \href{http://team.mumie.net/issues/9611}{Ticket 9611}
\end{block}

\begin{problem}

\randomquestionpool{1}{1}
\randomquestionpool{2}{2}
\randomquestionpool{3}{5}

	\begin{variables}
	 	\randint[Z]{a}{-5}{3}
	 	\randint[Z]{b}{-5}{4}
	 	\randint[Z]{c}{-3}{5}
	 	\randint[Z]{d}{-4}{5}
	 	\randint[Z]{r}{-20}{20}

		\function{f}{a*cos(x)+ b*x^3+c*sin(x)+d*ln(x)}
		\function{f_7}{a*cos(x)+ b*x^3+c*sin(x)+d*ln(x)+r}
		\derivative[normalize]{f_1}{f}{x}
		\functionNormalize{f_1}
        \functionNormalize{f_7}

	\end{variables}


	\begin{question}%1
		\lang{de}{
			\text{


			Gegeben sei die Funktion $f(x) = \var{f_1}$. \\
		      Bestimmen Sie für $x>0$ eine Stammfunktion $F(x)$ von $f(x)$:}
		}
		 \lang{en}{\text{
      Given is the function $f(x) = \var{f_1}$. \\
      For $x > 0$, determine an antiderivative $F(x)$ of $f(x)$:
     }}
		\type{input.function}

		\begin{answer}
			\text{ $F(x) = $}
			\solution{f}
			\inputAsFunction{x}{inp}
			\checkFuncForZero{D[inp]-f_1}{-10}{10}{100}
            
            \lang{en}{
            \explanation{Determine an antiderivative for each summand and note that an antiderivative
            of $f$ is then given by the sum of these functions. }
            }
            
            \lang{de}{
            \explanation{Bestimmen Sie für jeden Summanden eine Stammfunktion und beachten Sie, dass eine Stammfunktion
            von $f$ dann durch die Summe dieser Stammfunktionen gegeben ist. }            
            }
		\end{answer}
	\end{question}%1


	\begin{question}%2
		\lang{de}{
			\text{Gegeben sei nun die Funktion $g(x) = \var{f_7}$ für $x>0$. \\
		       Leiten Sie $g(x)$ ab!}
     \explanation{Beachten Sie, dass die Ableitung einer Summe von Funktionen durch die Summe der 
     Ableitungen der Summanden gegeben ist (Additivität der Ableitung).}               
               
		}
	 \lang{en}{\text{
      Given is the function $g(x) = \var{f_7}$, $x>0$. \\
      Find the derivative of $g(x)$!
     }
     \explanation{Note that the derivative of a sum of functions is the sum of the derivatives 
         of these functions (additivity of the derivative).}
     }
         
		\type{input.function}

		\begin{answer}
			\text{ $g'(x) = $}
			\solution{f_1}
			\checkAsFunction{x}{-10}{10}{100}
            

		\end{answer}

	\end{question}%2

	\begin{question}%3
    \type{input.generic}
    
	   \lang{de}{
	     \text{Vergleichen Sie nun $g'(x)$ mit $f(x)$.\\
	     Wie können Sie die Gleichheit erklären, obwohl $g(x)$ und $F(x)$ unterschiedlich sind? }
	   }
	    \lang{en}{\text{
        Now compare $g'(x)$ with $f(x)$. \\
        How can we explain the fact that they are the same, even though $g(x)$ and $F(x)$ are different?
      }}
      \begin{answer}
      
	   %\permutechoices{1}{4}
	   \type{mc.multiple}
	   \begin{choice}
      \lang{de}{\text{Es ist Zufall.}}
       \lang{en}{\text{It is coincidence.}}
	    \solution{false}
	   \end{choice}

	   \begin{choice}
      \lang{de}{\text{Es kann passieren, dass durch Rechenfehler die gleiche Stammfunktion entsteht.}}
       \lang{en}{\text{It can happen that by calculation error we get the same antiderivatives.}}
	    \solution{false}
	   \end{choice}

	   \begin{choice}
      \lang{de}{	    \text{Stammfunktionen sind nicht eindeutig, es können Konstanten addiert werden.}
}
      \lang{en}{\text{Antiderivatives are not unique, we can always add a constant.}}
      \solution{true}
	   \end{choice}

	   \begin{choice}
      \lang{de}{	    \text{Es gibt immer genau zwei verschiedene Stammfunktionen zu einer Funktion.}
}
 \lang{en}{\text{There are always exactly two different antiderivatives for a function.}}
      \solution{false}
	   \end{choice}
       
       \lang{de}{
       \explanation[equalChoice(1???)]{Zufall ist in diesem Fall nicht die gesuchte Erklärung.}
       \explanation[equalChoice(?1??)]{Rechenfehler erklären hier nicht die beobachtete Gleichheit.}
       \explanation[equalChoice(??0?)]{Die Ableitung einer Konstanten ist $0$, daher können sich Stammfunktionen um eine additive Konstante unterscheiden.}
       \explanation[equalChoice(???1)]{Wenn es eine Stammfunktion gibt, dann gibt es direkt unendlich viele.}
       }
       
       \lang{en}{
       \explanation[equalChoice(1???)]{In this case, the explanation you are looking for is not "coincidence".}
       \explanation[equalChoice(?1??)]{Calculation or rounding errors do not explain the observed equality here.}
       \explanation[equalChoice(??0?)]{Deriving a constant yields $0$, so two antiderivatives can differ by a constant.}
       \explanation[equalChoice(???1)]{If there is an antiderivative, there are infinitely many.}
       }       
\end{answer}
	\end{question}%3

\begin{question}%4
    \type{input.generic}
    
	   \lang{de}{
	     \text{Vergleichen Sie nun $g'(x)$ mit $f(x)$.\\
	     Wie können Sie die Gleichheit erklären, obwohl $g(x)$ und $F(x)$ unterschiedlich sind? }
	   }
	    \lang{en}{\text{
        Now compare $g'(x)$ with $f(x)$. \\
        How can we explain the fact that they are the same, even though $g(x)$ and $F(x)$ are different?
      }}
      \begin{answer}
      
	   %\permutechoices{1}{4}
	   \type{mc.multiple}
	   \begin{choice}
      \lang{de}{\text{Es ist Zufall.}}
       \lang{en}{\text{It is coincidence.}}
	    \solution{false}
	   \end{choice}
       
	   \begin{choice}
      \lang{de}{	    \text{Stammfunktionen sind nicht eindeutig, es können Konstanten addiert werden.}
}
      \lang{en}{\text{Antiderivatives are not unique, we can always add a constant.}}
      \solution{true}
	   \end{choice}

	   \begin{choice}
      \lang{de}{\text{Es kann passieren, dass durch Rechenfehler die gleiche Stammfunktion entsteht.}}
       \lang{en}{\text{It can happen that by calculation error we get the same antiderivatives.}}
	    \solution{false}
	   \end{choice}



	   \begin{choice}
      \lang{de}{	    \text{Es gibt immer genau zwei verschiedene Stammfunktionen zu einer Funktion.}
}
 \lang{en}{\text{There are always exactly two different antiderivatives for a function.}}
      \solution{false}
	   \end{choice}
       
       \lang{de}{
       \explanation[equalChoice(1???)]{Zufall ist in diesem Fall nicht die gesuchte Erklärung.}
       \explanation[equalChoice(?0??)]{Die Ableitung einer Konstanten ist $0$, daher können sich Stammfunktionen um eine additive Konstante unterscheiden.}
       \explanation[equalChoice(??1?)]{Rechenfehler erklären hier nicht die beobachtete Gleichheit.}
       \explanation[equalChoice(???1)]{Wenn es eine Stammfunktion gibt, dann gibt es direkt unendlich viele.}
       }
       
       \lang{en}{
       \explanation[equalChoice(1???)]{In this case, the explanation you are looking for is not "coincidence".}
       \explanation[equalChoice(?0??)]{Deriving a constant yields $0$, so two antiderivatives can differ by a constant.}
       \explanation[equalChoice(??1?)]{Calculation or rounding errors do not explain the observed equality here.}
       \explanation[equalChoice(???1)]{If there is an antiderivative, there are infinitely many.}
       }       
\end{answer}
	\end{question}%4


\begin{question}%5
    \type{input.generic}
    
	   \lang{de}{
	     \text{Vergleichen Sie nun $g'(x)$ mit $f(x)$.\\
	     Wie können Sie die Gleichheit erklären, obwohl $g(x)$ und $F(x)$ unterschiedlich sind? }
	   }
	    \lang{en}{\text{
        Now compare $g'(x)$ with $f(x)$. \\
        How can we explain the fact that they are the same, even though $g(x)$ and $F(x)$ are different?
      }}
      \begin{answer}
      
	   %\permutechoices{1}{4}
	   \type{mc.multiple}
       
\begin{choice}
      \lang{de}{	    \text{Es gibt immer genau zwei verschiedene Stammfunktionen zu einer Funktion.}
}
 \lang{en}{\text{There are always exactly two different antiderivatives for a function.}}
      \solution{false}
	   \end{choice}
       
	   \begin{choice}
      \lang{de}{\text{Es ist Zufall.}}
       \lang{en}{\text{It is coincidence.}}
	    \solution{false}
	   \end{choice}

	   \begin{choice}
      \lang{de}{\text{Es kann passieren, dass durch Rechenfehler die gleiche Stammfunktion entsteht.}}
       \lang{en}{\text{It can happen that by calculation error we get the same antiderivatives.}}
	    \solution{false}
	   \end{choice}

	   \begin{choice}
      \lang{de}{	    \text{Stammfunktionen sind nicht eindeutig, es können Konstanten addiert werden.}
}
      \lang{en}{\text{Antiderivatives are not unique, we can always add a constant.}}
      \solution{true}
	   \end{choice}


       
       \lang{de}{
       \explanation[equalChoice(1???)]{Wenn es eine Stammfunktion gibt, dann gibt es direkt unendlich viele.}
       \explanation[equalChoice(?1??)]{Zufall ist in diesem Fall nicht die gesuchte Erklärung.}
       \explanation[equalChoice(??1?)]{Rechenfehler erklären hier nicht die beobachtete Gleichheit.}
       \explanation[equalChoice(???0)]{Die Ableitung einer Konstanten ist $0$, daher können sich Stammfunktionen um eine additive Konstante unterscheiden.}
       }
       
       \lang{en}{
       \explanation[equalChoice(1???)]{If there is an antiderivative, there are infinitely many.}
       \explanation[equalChoice(?1??)]{In this case, the explanation you are looking for is not "coincidence".}
       \explanation[equalChoice(??1?)]{Calculation or rounding errors do not explain the observed equality here.}
       \explanation[equalChoice(???0)]{Deriving a constant yields $0$, so two antiderivatives can differ by a constant.}
       }

\end{answer}
	\end{question}%5

\end{problem}


\embedmathlet{mathlet}
\end{content}