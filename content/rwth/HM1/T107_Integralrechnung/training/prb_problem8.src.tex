\documentclass{mumie.problem.gwtmathlet}
%$Id$
\begin{metainfo}
  \name{
    \lang{de}{A08: Flächen zwischen zwei Graphen}
    \lang{en}{problem 8}
  }
  \begin{description} 
 This work is licensed under the Creative Commons License Attribution 4.0 International (CC-BY 4.0)   
 https://creativecommons.org/licenses/by/4.0/legalcode 

    \lang{de}{}
    \lang{en}{}
  \end{description}
  \corrector{system/problem/GenericCorrector.meta.xml}
  \begin{components}
    \component{js_lib}{system/problem/GenericMathlet.meta.xml}{mathlet}
  \end{components}
  \begin{links}
  \end{links}
  \creategeneric
\end{metainfo}
\begin{content}
\usepackage{mumie.ombplus}
\usepackage{mumie.genericproblem}


\lang{de}{
	\title{A08: Flächen zwischen zwei Graphen}
}
\lang{en}{
	\title{Problem 8}
}

\begin{block}[annotation]
  Im Ticket-System: \href{http://team.mumie.net/issues/9616}{Ticket 9616}
\end{block}

\begin{problem}

	\begin{variables}
	 	\randint[Z]{a}{-8}{-1}
	 	\function{b}{-a}
	 	\number{c}{0}

		\randint{m}{2}{7}
	 	\randint{n}{2}{7}

	 	\randint{j}{2}{6}
	 	\function[calculate]{k}{m-j*a^2}
	 	\function{l}{n}

	    \function{F}{(1/2)*m*x^2+n*x}
		\derivative[normalize]{f_1}{F}{x}
		\functionNormalize{f_1}

    	\function{G}{(1/4)*j*x^4+(1/2)*k*x^2+l*x}
		\derivative[normalize]{g_1}{G}{x}
		\functionNormalize{g_1}

		\function{H}{G-F}
		\derivative[normalize]{h_1}{H}{x}
		\functionNormalize{h_1}

		\substitute[normalize]{H_b}{H}{x}{b}
		\substitute[normalize]{H_a}{H}{x}{c}
		\function[calculate]{L}{|H_b-H_a|}
	\end{variables}

    \begin{question}

	   \lang{de}{
	     \text{In dieser Aufgabe sind die Funktionen $f(x) = \var{m}x+\var{n}$ und $g(x)=\var{g_1}$ gegeben. 
	     Diese beiden Funktionen schneiden sich an drei Stellen. Eine davon ist die Stelle $0$.\\
	     Bestimmen Sie die beiden anderen Schnittstellen - es sollte je eine negative und eine positive 
	     Schnittstelle sein. \\
         Geben Sie die positive Schnittstelle $b$ an:}}
	      \lang{en}{\text{
          In this problem we are given the functions $f(x) = \var{m}x+\var{n}$ and $g(x)=\var{g_1}$. 
          These functions intersect at three points. One of these is $x=0$.\\
          Determine the two other intersection points, one should be negative
          and the other positive.\\
          Give the positive intersection point $b$:
        }}
	    \type{input.number}
	   	\begin{answer}
			\text{ $b =$ }
			\solution{b}
            
            \lang{de}{
            \explanation{Da $0$ ein Schnittpunkt ist, müssen Sie lediglich noch die positive Nullstelle von $\frac{g(x)-f(x)}{x}$ bestimmen.}
            \explanation[ans<0]{Es ist der positive Schnittpunkt gesucht!}
            }
            \lang{en}{
            \explanation{To determine the intersection points, compute the zeros of $g-f$.}
            \explanation[ans<0]{The positive intersection point is to be determined.}
            }
		\end{answer}

	\end{question}

  \begin{question}

	   \lang{de}{
	     \text{Bestimmen Sie nun die Fläche, welche zwischen den Graphen der Funktionen $f(x) = \var{m}x+\var{n}$ 
	     und $g(x)=\var{g_1}$ über dem Intervall $[0;b]$ eingeschlossen wird.}}
	      \lang{en}{\text{
          Now determine the area enclosed by the graphs of $f(x) = \var{m}x+\var{n}$ 
          and $g(x)=\var{g_1}$ over the interval $[0, b]$.
        }}
	    \type{input.number}
	    \correctorprecision[rounded]{2}
	    \displayprecision{2}
	   	\begin{answer}
      \lang{de}{\text{ Fläche (auf zwei Nachkommastellen gerundet): }}
       \lang{en}{\text{ Area (rounded to two decimal places): }}
			\solution{L}
            
            \lang{de}{
            \explanation{Der Flächeninhalt kann über das Integral $\int_0^b \left\vert g(x)-f(x)\right\vert dx$ berechnet werden. Verwenden Sie die Schnittstelle aus a), um das Integral auszurechnen. Lösen Sie den Betrag geeignet auf.}
		    }
            \lang{en}{
            \explanation{The area can be computed via the integral $\int_0^b \left\vert g(x)-f(x)\right\vert dx$. Use the intersection point from a) to compute the integral and to resolve the absolute value. }
            }
        \end{answer}

	\end{question}

\end{problem}


\embedmathlet{mathlet}
\end{content}