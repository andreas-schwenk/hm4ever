\documentclass{mumie.problem.gwtmathlet}
%$Id$
\begin{metainfo}
  \name{
    \lang{de}{A02: Abschnittsweise definierte Funktion}
    \lang{en}{problem 2}
  }
  \begin{description} 
 This work is licensed under the Creative Commons License Attribution 4.0 International (CC-BY 4.0)   
 https://creativecommons.org/licenses/by/4.0/legalcode 

    \lang{de}{}
    \lang{en}{}
  \end{description}
  \corrector{system/problem/GenericCorrector.meta.xml}
  \begin{components}
    \component{js_lib}{system/problem/GenericMathlet.meta.xml}{mathlet}
  \end{components}
  \begin{links}
  \end{links}
  \creategeneric
\end{metainfo}
\begin{content}
\usepackage{mumie.ombplus}
\usepackage{mumie.genericproblem}


\lang{de}{
	\title{A02: Abschnittsweise definierte Funktion}
}
\lang{en}{
	\title{Problem 2}
}

\begin{block}[annotation]
  Im Ticket-System: \href{http://team.mumie.net/issues/9610}{Ticket 9610}
\end{block}

\begin{problem}

\randomquestionpool{1}{3}
\randomquestionpool{4}{4}

 \begin{variables}
  \randint{a}{3}{5}

  \randint{m}{2}{4}
  \randint{b}{1}{5}
  \function{f}{m*x+b}

  \randint{g}{1}{20}

  \function[calculate]{L}{(6-a)*g + (a*(m*a+2*b))/2}
 \end{variables}

 \begin{question}%1
 \type{input.generic}
    \lang{de}{
 %begin-cosh   
  \text{Gegeben sei eine Funktion $f$, wobei $f(x)= \begin{cases}\var{f}, & \text{ für }x \leq \var{a}\\
 \var{g}, &\text{ für }x > \var{a}\end{cases}$ gilt.\\
 Ist diese Funktion auf $[0;6]$ integrierbar?}}
   \lang{en}{\text{Let $f$ be a function where $f(x)= \begin{cases}\var{f}, & \text{ for }x \leq \var{a}\\
 \var{g}, &\text{ for }x > \var{a}\end{cases}$. Is this function integrable on $[0, 6]$?}}
    
  \begin{answer}
  %\permutechoices{1}{5}
  \type{mc.multiple}
  \field{integer}

  \begin{choice}
      \lang{de}{ \text{Nein, weil sie an der Stelle $x=\var{a}$ eine Sprungstelle hat.}}
       \lang{en}{\text{No, since there is a jump discontinuity at $x=\var{a}$.}}
   \solution{false}
  \end{choice}

  \begin{choice}
      \lang{de}{   \text{Ja, weil beschränkte Funktionen auf abgeschlossenen Intervallen immer integrierbar sind.}
}
 \lang{en}{\text{Yes, since bounded functions over closed intervals are always integrable.}}
      \solution{false}
  \end{choice}

  \begin{choice}
      \lang{de}{   \text{Nein, weil sie abschnittsweise definiert ist.}
}
 \lang{en}{\text{No, since it is defined piecewise.}}
      \solution{false}
  \end{choice}

   \begin{choice}
      \lang{de}{    \text{Ja, weil jede Funktion integrierbar ist.}
}
 \lang{en}{\text{Yes, since all functions are integrable.}}
      \solution{false}
   \end{choice}

   \begin{choice}
      \lang{de}{    \text{Ja, weil sie auf beiden Teilintervallen integrierbar ist.}
}
 \lang{en}{\text{Yes, since it is integrable on both subintervals.}}
      \solution{true}
   \end{choice}
   
\lang{de}{  
   \explanation[equalChoice(1????)]{Auch Funktionen mit Sprungstellen können integrierbar sein.}                                   
   \explanation[equalChoice(?1???)]{Beschränkheit alleine reicht nicht, um Integrierbarkeit einer Funktion zu garantieren.}                                    
   \explanation[equalChoice(??1??)]{Auch abschnittsweise definierte Funktionen können integrierbar sein.}                                    
   \explanation[equalChoice(???1?)]{Nicht jede Funktion ist integrierbar.}                                 
   \explanation[equalChoice(????0)]{Aus der Integrierbarkeit einer Funktion auf zwei Teilintervallen ergibt sich die Integrierbarkeit auf der Vereinigung der Intervalle.}
                                    
} 

\lang{en}{  
   \explanation[equalChoice(1????)]{Functions with jump discontinuities can also be integrable.}
   \explanation[equalChoice(?1???)]{Boundedness is not enough to guarantee that a function can be integrated.}
   \explanation[equalChoice(??1??)]{Piecewise defined functions can be integrable.}
   \explanation[equalChoice(???1?)]{There are also non-integrable functions.}
   \explanation[equalChoice(????0)]{The integrability on the union of intervals results from the integrability on the subintervals.}
} 

\end{answer}
 \end{question}%1
 
 \begin{question}%2
 \type{input.generic}
    \lang{de}{
 %begin-cosh   
  \text{Gegeben sei eine Funktion $f$, wobei $f(x)= \begin{cases}\var{f}, & \text{ für }x \leq \var{a}\\
 \var{g}, &\text{ für }x > \var{a}\end{cases}$ gilt.\\
 Ist diese Funktion auf $[0;6]$ integrierbar?}}
   \lang{en}{\text{Let $f$ be a function where where $f(x)= \var{f},$ when $x \leq \var{a}$\\
   and $f(x)=\var{g},$ when $x > \var{a}$. Is this function integrable on $[0, 6]$?}}
    
  \begin{answer}
  %\permutechoices{1}{5}
  \type{mc.multiple}
  \field{integer}
  
\begin{choice}
      \lang{de}{   \text{Ja, weil beschränkte Funktionen auf abgeschlossenen Intervallen immer integrierbar sind.}
}
 \lang{en}{\text{Yes, since bounded functions over closed intervals are always integrable.}}
      \solution{false}
\end{choice}

\begin{choice}
      \lang{de}{ \text{Nein, weil sie an der Stelle $x=\var{a}$ eine Sprungstelle hat.}}
       \lang{en}{\text{No, since there is a jump discontinuity at $x=\var{a}$.}}
   \solution{false}
\end{choice}

\begin{choice}
      \lang{de}{    \text{Ja, weil jede Funktion integrierbar ist.}
}
 \lang{en}{\text{Yes, since all functions are integrable.}}
      \solution{false}
\end{choice}

\begin{choice}
      \lang{de}{    \text{Ja, weil sie auf beiden Teilintervallen integrierbar ist.}
}
 \lang{en}{\text{Yes, since it is integrable on both subintervals.}}
      \solution{true}
\end{choice}

\begin{choice}
      \lang{de}{   \text{Nein, weil sie abschnittsweise definiert ist.}
}
 \lang{en}{\text{No, since it is defined piecewise.}}
      \solution{false}
\end{choice}

\lang{de}{   
   \explanation[equalChoice(1????)]{Beschränkheit alleine reicht nicht, um Integrierbarkeit einer Funktion zu garantieren.}                                    
   \explanation[equalChoice(?1???)]{Auch Funktionen mit Sprungstellen können integrierbar sein.}                                    
   \explanation[equalChoice(??1??)]{Nicht jede Funktion ist integrierbar.}                                    
   \explanation[equalChoice(???0?)]{Aus der Integrierbarkeit einer Funktion auf zwei Teilintervallen ergibt sich die Integrierbarkeit auf der Vereinigung der Intervalle.}                                    
   \explanation[equalChoice(????1)]{Auch abschnittsweise definierte Funktionen können integrierbar sein.}                                    
}   

\lang{en}{   
   \explanation[equalChoice(1????)]{Boundedness is not enough to guarantee that a function can be integrated.}
   \explanation[equalChoice(?1???)]{Functions with jump discontinuities can also be integrable.}
   \explanation[equalChoice(??1??)]{There are also non-integrable functions.}
   \explanation[equalChoice(???0?)]{The integrability on the union of intervals results from the integrability on the subintervals.}
   \explanation[equalChoice(????1)]{Piecewise defined functions can be integrable.}
}   
   
\end{answer}
 \end{question}%2
 
\begin{question}%3
 \type{input.generic}
    \lang{de}{
 %begin-cosh   
  \text{Gegeben sei eine Funktion $f$, wobei $f(x)= \begin{cases}\var{f}, & \text{ für }x \leq \var{a}\\
 \var{g}, &\text{ für }x > \var{a}\end{cases}$ gilt.\\
 Ist diese Funktion auf $[0;6]$ integrierbar?}}
   \lang{en}{\text{Let $f$ be a function where where $f(x)= \var{f},$ when $x \leq \var{a}$\\
   and $f(x)=\var{g},$ when $x > \var{a}$. Is this function integrable on $[0, 6]$?}}
    
  \begin{answer}
  %\permutechoices{1}{5}
  \type{mc.multiple}
  \field{integer}
  
\begin{choice}
      \lang{de}{    \text{Ja, weil sie auf beiden Teilintervallen integrierbar ist.}
}
 \lang{en}{\text{Yes, since it is integrable on both subintervals.}}
      \solution{true}
\end{choice}

\begin{choice}
      \lang{de}{ \text{Nein, weil sie an der Stelle $x=\var{a}$ eine Sprungstelle hat.}}
       \lang{en}{\text{No, since there is a jump discontinuity at $x=\var{a}$.}}
   \solution{false}
\end{choice}

\begin{choice}
      \lang{de}{   \text{Ja, weil beschränkte Funktionen auf abgeschlossenen Intervallen immer integrierbar sind.}
}
 \lang{en}{\text{Yes, since bounded functions over closed intervals are always integrable.}}
      \solution{false}
\end{choice}

\begin{choice}
      \lang{de}{    \text{Ja, weil jede Funktion integrierbar ist.}
}
 \lang{en}{\text{Yes, since all functions are integrable.}}
      \solution{false}
\end{choice}



\begin{choice}
      \lang{de}{   \text{Nein, weil sie abschnittsweise definiert ist.}
}
 \lang{en}{\text{No, since it is defined piecewise.}}
      \solution{false}
\end{choice}
   
\lang{de}{   
   \explanation[equalChoice(0????)]{Aus der Integrierbarkeit einer Funktion auf zwei Teilintervallen ergibt sich die Integrierbarkeit auf der Vereinigung der Intervalle.}                                   
   \explanation[equalChoice(?1???)]{Auch Funktionen mit Sprungstellen können integrierbar sein.}                                    
   \explanation[equalChoice(??1??)]{Beschränkheit alleine reicht nicht, um Integrierbarkeit einer Funktion zu garantieren.}                                    
   \explanation[equalChoice(???1?)]{Nicht jede Funktion ist integrierbar.}                                    
   \explanation[equalChoice(????1)]{Auch abschnittsweise definierte Funktionen können integrierbar sein.}
                                    
}

\lang{en}{   
   \explanation[equalChoice(0????)]{The integrability on the union of intervals results from the integrability on the subintervals.}
   \explanation[equalChoice(?1???)]{Functions with jump discontinuities can also be integrable.}
   \explanation[equalChoice(??1??)]{Boundedness is not enough to guarantee that a function can be integrated.}
   \explanation[equalChoice(???1?)]{There are also non-integrable functions.}
   \explanation[equalChoice(????1)]{Piecewise defined functions can be integrable.}
}

\end{answer}
 \end{question}%3
 
 
 

 \begin{question}%4
    \lang{de}{
%begin-cosh    
  \text{Bestimmen Sie den Wert des Integrals $\int_0^6 f(x)\, dx$ für die oben gegebene Funktion $f$.\\
%ende-cosh  
  \textit{Tipp: Denken Sie an die Additivität des Integrals!}}
}
 \lang{en}{\text{Determine the value of the integral $\int_0^6 f(x)\, dx$ for
 the function $f(x)$ given above. \\
 \textit{Hint: recall the additivity property of integrals!}}}

  \type{input.number}

  \begin{answer}
      \lang{de}{   \text{Lösung: }
}
 \lang{en}{\text{Solution: }}
      \solution{L}
 \lang{de}{ 
  \explanation{Verwenden Sie für $x$ zwischen $0$ und $\var{a}$ die obere Definition und für $x$ zwischen
  $\var{a}$ und $6$ die untere Definition in der abschnittsweisen Definition von $f$.}
  }
  
  \lang{en}{
  \explanation{For $x$ between $0$ and $\var{a}$ use the upper equation and for $x$ between
  $\var{a}$ and $6$ the lower equation in the piecewise definition of $f$.}
  }
  \end{answer}
  


 \end{question}%4


\end{problem}


\embedmathlet{mathlet}

\end{content}