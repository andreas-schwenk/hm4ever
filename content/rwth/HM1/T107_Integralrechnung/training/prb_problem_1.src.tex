\documentclass{mumie.problem.gwtmathlet}
%$Id$
\begin{metainfo}
  \name{
    \lang{de}{A01: Integral als Flächeninhalt}
    \lang{en}{problem_1}
  }
  \begin{description} 
 This work is licensed under the Creative Commons License Attribution 4.0 International (CC-BY 4.0)   
 https://creativecommons.org/licenses/by/4.0/legalcode 

    \lang{de}{}
    \lang{en}{}
  \end{description}
  \corrector{system/problem/GenericCorrector.meta.xml}
  \begin{components}
    \component{js_lib}{system/problem/GenericMathlet.meta.xml}{mathlet}
  \end{components}
  \begin{links}
  \end{links}
  \creategeneric
\end{metainfo}
\begin{content}
\usepackage{mumie.ombplus}
\usepackage{mumie.genericproblem}


\lang{de}{
	\title{A01: Integral als Flächeninhalt}
}
\lang{en}{
	\title{Problem 1}
}

\begin{block}[annotation]
  
      
\end{block}
\begin{block}[annotation]
  Im Ticket-System: \href{http://team.mumie.net/issues/9609}{Ticket 9609}
\end{block}


\begin{problem}

  \begin{variables}
	\randint{b}{-8}{-1}
	\randint{z}{2}{8}
	\randint{c}{1}{8}
	\randint{d}{1}{2}
	\function[expand, normalize]{m}{c/d}
	\function{f}{m*x+b}
	\function[expand, normalize]{a1}{-b/m}
	\function[expand, normalize]{a2}{-b/m +z}
	\function[expand, normalize]{a0}{a1+1}
	\function{L1}{(m*(z^2))/2}
	\function{L2}{(m*((z^2)-1))/2}
  \end{variables}


  \begin{question}

    \lang{de}{ \text{Betrachten Sie in der gesamten Aufgabe die Gerade $f(x)=\var{m}x\var{b}$.\\
     Bestimmen Sie (mit geometrischen Mitteln) den Wert des Integrals \\
       $\int_{a}^{b} f(x) \, dx,$ wobei $a=\var{a1}$ und $b=\var{a2}$ sind.  }
    }
     \lang{en}{\text{
      Throughout the exercise, refer back to the line $f(x)=\var{m}x\var{b}$.\\
      Determine geometrically the value of the integral \\
        $\int_{a}^{b} f(x) \, dx,$ where $a=\var{a1}$ and $b=\var{a2}$.
     }}
    \type{input.number}
    \field{rational}

	\begin{answer}
      \lang{de}{ \text{Lösung: }}
      \lang{en}{\text{Solution: }}
	 \solution{L1}
     
     \lang{de}{
     \explanation{ Das Integral $\int_{a}^{b} f(x) \, dx$ kann geometrisch als der Flächeninhalt, 
     der von dem Graphen von $f$ und der $x$-Achse zwischen $a$ und $b$ eingeschlossen wird, berechnet werden. 
     Eine Skizze der Situation könnte bei der Lösung hilfreich sein. }
     }
     
     \lang{en}{ 
     \explanation{The integral $\int_{a}^{b} f(x) \, dx$ can be calculated geometrically as the area which is enclosed 
      by the graph of $f$ and the $x$-axis between $a$ and $b$.
     A sketch of the situation could be helpful in solving it. }
     }
     
	\end{answer}

  \end{question}

  \begin{question}
    \lang{de}{ \text{Bestimmen Sie (mit geometrischen Mitteln) den Wert des Integrals
      $\int_{a}^{b} f(x) \, dx,$\\ wobei $a=\var{a0}$ und $b=\var{a2}$ sind.  }
    }
     \lang{en}{\text{Determine geometrically the value of the integral
     $\int_{a}^{b} f(x) \, dx,$\\ where $a=\var{a0}$ and $b=\var{a2}$ }}
    \type{input.number}
    \field{rational}

	\begin{answer}
      \lang{de}{\text{Lösung: }}
       \lang{en}{\text{Solution: }}
	 \solution{L2}
     
     \lang{de}{
     \explanation{Das Integral $\int_{a}^{b} f(x) \, dx$ kann geometrisch als der Flächeninhalt, 
     der von dem Graphen von $f$ und der $x$-Achse zwischen $a$ und $b$ eingeschlossen wird, berechnet werden. 
     Eine Skizze der Situation könnte bei der Lösung hilfreich sein. Der gesuchte Wert setzt sich zusammen aus 
     dem Flächeninhalt eines Rechtecks und eines Dreiecks.}
     }
     
     \lang{en}{ 
     \explanation{The integral $\int_{a}^{b} f(x) \, dx$ can be calculated geometrically as the area which is enclosed 
      by the graph of $f$ and the $x$-axis between $a$ and $b$.
     A sketch of the situation could be helpful in solving it. The area we are looking for is made up of
     the area of a rectangle and a triangle.}   
     }
	\end{answer}

  \end{question}



\end{problem}


\embedmathlet{mathlet}



\end{content}