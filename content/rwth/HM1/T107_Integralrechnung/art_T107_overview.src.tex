
%$Id:  $
\documentclass{mumie.article}
%$Id$
\begin{metainfo}
  \name{
    \lang{de}{Überblick: Integralrechnung}
    \lang{en}{Overview: Integration}
  }
  \begin{description} 
 This work is licensed under the Creative Commons License Attribution 4.0 International (CC-BY 4.0)   
 https://creativecommons.org/licenses/by/4.0/legalcode 

    \lang{de}{Beschreibung}
    \lang{en}{Description}
  \end{description}
  \begin{components}
  \end{components}
  \begin{links}
\link{generic_article}{content/rwth/HM1/T107_Integralrechnung/g_art_content_26_flaechen_zwischen_graphen.meta.xml}{content_26_flaechen_zwischen_graphen}
\link{generic_article}{content/rwth/HM1/T107_Integralrechnung/g_art_content_25_stammfunktion.meta.xml}{content_25_stammfunktion}
\link{generic_article}{content/rwth/HM1/T107_Integralrechnung/g_art_content_24_integral_als_flaeche.meta.xml}{content_24_integral_als_flaeche}
\end{links}
  \creategeneric
\end{metainfo}
\begin{content}
\begin{block}[annotation]
	Im Ticket-System: \href{https://team.mumie.net/issues/30142}{Ticket 30142}
\end{block}


\begin{block}[annotation]
Im Entstehen: Überblicksseite für Kapitel Integralrechnung
\end{block}

\usepackage{mumie.ombplus}
\ombchapter{1}
\title{\lang{de}{Überblick: Integralrechnung}\lang{en}{Overview: Integration}}



\begin{block}[info-box]
\lang{de}{\strong{Inhalt}}
\lang{en}{\strong{Contents}}


\lang{de}{
    \begin{enumerate}%[arabic chapter-overview]
   \item[7.1] \link{content_24_integral_als_flaeche}{Definition und Eigenschaften des Integrals}
   \item[7.2] \link{content_25_stammfunktion}{Stammfunktion}
   \item[7.3] \link{content_26_flaechen_zwischen_graphen}{Fläche zwischen Graphen}
     \end{enumerate}
}
\lang{en}{
    \begin{enumerate}%[arabic chapter-overview]
   \item[7.1] \link{content_24_integral_als_flaeche}{Definition and fundamental properties of 
                                                     integrals}
   \item[7.2] \link{content_25_stammfunktion}{Antiderivatives}
   \item[7.3] \link{content_26_flaechen_zwischen_graphen}{Computing integrals and areas}
     \end{enumerate}
} %lang

\end{block}

\begin{zusammenfassung}

\lang{de}{
Wir prägen mit der Fläche zwischen Funktionsgraph und $x$-Achse einen ersten Integralbegriff und 
bestimmen Eigenschaften des Integrals. Flächen oberhalb der $x$-Achse sind dabei positiv, Flächen 
unterhalb negativ. 
\\\\
Die Beobachtung, dass das Integral $\int_a^b f'(x)~dx$ einer Ableitungsfunktion gleich der Differenz 
$f(a)-f(b)$ von Werten der Funktion $f$ selbst ist, führt uns zum  Begriff der Stammfunktion. Dadurch 
können viele Intergale effektiv berechnet werden. Im Hauptsatz der Differential- und Integralrechnung 
wird diese wichtige Erkenntnis festgehalten.
\\\\
Als Anwendung berechnen wir Flächen zwischen Funktionsgraphen.
}
\lang{en}{
In this chapter we define the integral as the signed area bounded by the graph of a function and the 
$x$-axis, and give some properties of integrals. Areas above the $x$-axis count positively towards 
the integral, and areas below the $x$-axis count negatively towards the integral.
\\\\
We make the observation that the integral $\int_a^b f'(x)~dx$ of a derivative is equal to the 
difference $f(a)-f(b)$ of the function $f$ evaluated at the limits $a$ and $b$. This leads to the 
definition of the antiderivative and indefinite integral, which can be used to evaluate direct 
integrals of a function, by the fundamental theorem of calculus.
\\\\
As an application, we cover how to calculate the area bounded by two graphs.
}


\end{zusammenfassung}

\begin{block}[info]\lang{de}{\strong{Lernziele}}
\lang{en}{\strong{Learning Goals}} 
\begin{itemize}[square]
\item \lang{de}{
      Sie kennen den Begriff des bestimmten Integrals einer Funktion und ziehen aus 
      Symmetrieeigenschaften des Funktionsgraphen Rückschlüsse auf das Integral.
      }
      \lang{en}{
      Knowing the definition of the definite integral of a function, and being able to draw 
      conclusions about the interval where the symmetry of a graph makes this possible.
      }
\item \lang{de}{
      Sie kennen den Hauptsatz der Differentialrechnung und die Begriffe Stammfunktion und 
      unbestimmtes Integral.
      }
      \lang{en}{
      Knowing the statement of the fundamental theorem of algebra and the definitions of the 
      antiderivatives and indefinite integral of a function.
      }
\item \lang{de}{
      Sie kennen wichtige Eigenschaften des Integrals wie seine Additivität, Summen- und Faktorregel 
      und das Verhalten bei Vertauschung der Grenzen.
      }
      \lang{en}{
      Knowing some important properties of integrals such as their additivity in the limits, the sum 
      rule, the constant rule and their behaviour when their limits are switched. 
      }
\item \lang{de}{Sie berechnen einfache Integrale mit Hilfe von Stammfunktionen.}
      \lang{en}{Being able to evaluate basic definite integrals using indefinite integrals.}
\item \lang{de}{Sie berechnen die Fläche zwischen Funktionsgraphen.}
      \lang{en}{Being able to calculate the area bounded by the graphs of two functions.}
\end{itemize}
\end{block}




\end{content}
