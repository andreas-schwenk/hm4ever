%$Id:  $
\documentclass{mumie.article}
%$Id$
\begin{metainfo}
  \name{
    \lang{de}{Fläche zwischen Graphen}
    \lang{en}{Computing integrals and areas}
  }
  \begin{description} 
 This work is licensed under the Creative Commons License Attribution 4.0 International (CC-BY 4.0)   
 https://creativecommons.org/licenses/by/4.0/legalcode 

    \lang{de}{Beschreibung}
    \lang{en}{Description}
  \end{description}
  \begin{components}
    \component{generic_image}{content/rwth/HM1/images/g_tkz_T107_Area_D.meta.xml}{T107_Area_D}
    \component{generic_image}{content/rwth/HM1/images/g_tkz_T107_Area_C.meta.xml}{T107_Area_C}
    \component{generic_image}{content/rwth/HM1/images/g_tkz_T107_Area_B.meta.xml}{T107_Area_B}
    \component{generic_image}{content/rwth/HM1/images/g_tkz_T107_Area_A.meta.xml}{T107_Area_A}
  \end{components}
  \begin{links}
    \link{generic_article}{content/rwth/HM1/T107_Integralrechnung/g_art_content_24_integral_als_flaeche.meta.xml}{link2}
    \link{generic_article}{content/rwth/HM1/T106_Differentialrechnung/g_art_content_21_kettenregel.meta.xml}{link1}
    \link{generic_article}{content/rwth/HM1/T106_Differentialrechnung/g_art_content_20_ableitung_als_tangentensteigung.meta.xml}{link0}
  \end{links}
  \creategeneric
\end{metainfo}
\begin{content}
\usepackage{mumie.ombplus}
\ombchapter{7}
\ombarticle{3}

\lang{de}{\title{Berechnung von Integralen und Flächen}}
\lang{en}{\title{Computing integrals and areas}}
 

\begin{block}[annotation]
  übungsinhalt
  
\end{block}
\begin{block}[annotation]
  Im Ticket-System: \href{http://team.mumie.net/issues/9037}{Ticket 9037}\\
\end{block}

\begin{block}[info-box]
\tableofcontents
\end{block}

\section{\lang{de}{Berechnung von bestimmten Integralen}
         \lang{en}{Computing definite integrals}}\label{berechnung}

\lang{de}{
Im \link{link2}{vorigen Abschnitt} wurde behandelt, wie man bestimmte Integrale mit Hilfe des
\emph{Hauptsatzes der Differential- und Integralrechnung} berechnen kann:
}
\lang{en}{
In the \link{link2}{previous section} we introduced how to calculate definite integrals using 
the \emph{fundamental theorem of calculus}:
}
\[ \int_a^b f(x)\, dx = \Big[F(x)\Big]_a^b \, . \]
\lang{de}{
Die Berechnung eines Integrals $\big\int_a^b f(x)\, dx$ reduziert sich daher im 
Wesentlichen auf die Bestimmung einer Stammfunktion $F(x)$.
% beziehungsweise auf die
%Lösung des unbestimmten Integrals $\int f(x)\, dx$.
\\
Im vorigen Abschnitt wurden auch bereits Stammfunktionen zu mehreren 
wichtigen Funktionen wie $x^k$, $\sin(x)$, $\cos(x)$ und $e^x$ vorgestellt. 
Außerdem gilt die Summen- und Faktorregel für Integrale.
}
\lang{en}{
Calculating the integral $\big\int_a^b f(x)\, dx$ becomes the problem of finding an antiderivative of 
the integrand, which can then be evaluated at the limits of the integral.
\\
In the previous section we also covered the antiderivatives of some important functions such as 
$x^k$, $\sin(x)$, $\cos(x)$, and $e^x$, and we saw the sum and constant rules for integrals.
}

\begin{example}

%\emph{Beispiel:} 
\lang{de}{Sei $f(x)=e^x-x-1$. Dann gilt $\big\int f(x)\, dx = e^x - \frac{1}{2} x^2 - x +C$ und}
\lang{en}{Let $f(x)=e^x-x-1$. Then $\big\int f(x)\, dx = e^x - \frac{1}{2} x^2 - x +C$ and}
\begin{align*} \int_{-1}^1 (e^x-x-1)\, dx  & \ = \ & \Big[  e^x - \frac{1}{2} x^2 - x \Big]_{-1}^1 \\
 & \ = \ &
 (e - \frac{1}{2} -1)-(e^{-1} - \frac{1}{2} +1 ) \\ & \ = \ & e-\frac{1}{e} -2 \approx 0{,}35 .
\end{align*}

 \begin{center}
 \image{T107_Area_A}
 \end{center}
 \end{example}  
 
 \lang{de}{
 Die Bestimmung einer Stammfunktion  kann eine schwierige Aufgabe sein. Es existieren hierzu
 verschiedene Regeln, die in diesem Kurs nur teilweise behandelt werden. Im Unterschied zur 
 Ableitung führt die Anwendung der Regeln aber nicht automatisch auf eine Lösung, obwohl 
 jede stetige Funktion eine Stammfunktion besitzt.
 \\\\
 Wenn eine mögliche Stammfunktion gegeben ist, so kann man durch Ableiten überprüfen, ob es 
 sich wirklich um eine solche handelt.
 }
 \lang{en}{
 Calculating or determining the antiderivative of a function ('integrating the function') can be a 
 very challenging task. There are several different strategies for doing this that will not be dealt 
 with in this course. In comparison to differentiation, these strategies do not guarantee us a 
 solution, even though every continuous function has an antiderivative.
 \\\\
 If we are given a candidate antiderivative, we can easily check it by differentiating it to see 
 if its derivative is indeed our integrand.
 }
 \begin{example}
 %\emph{Beispiel: } 
 \lang{de}{Das Integral}
 \lang{en}{Suppose we want to check if}
 
 \[ \int \sin^2(x)\, dx = \frac{1}{2}x - \frac{1}{2} \sin(x) \cos(x) + C \]
 
 \lang{de}{
 soll durch Ableiten überprüft werden. Man verwendet 
 für den zweiten Summanden die Produktregel
 (siehe Abschnitt \link{link0}{Ableitung und Ableitungsformeln}) und erhält
 }
 \lang{en}{
 by differentiating the right-hand side. We use the product rule for the second summand (see the 
 section on \link{link0}{derivatives}) and get
 }
 
 \begin{align*} & \big(\frac{1}{2}x - \frac{1}{2} \sin(x) \cos(x)\big)'  =  \frac{1}{2} 
  - \frac{1}{2} \big( \cos(x) \cos(x) - \sin(x) \sin(x) \big)  \\
 & = \frac{1}{2} - \frac{1}{2} \big( 1 - \sin^2(x) -  \sin^2(x) \big)  =  \sin^2(x) .
 \end{align*}
 
 \lang{de}{
 Man beachte hierbei, dass $\sin^2(x) + \cos^2(x) = 1$ gilt. Der Strich $'$ bezeichnet wie üblich die 
 Ableitung nach dem Argument.
 \\\\
 Das folgende Integral lässt sich dann leicht berechnen:
 }
 \lang{en}{
 We also used the identity $\sin^2(x) + \cos^2(x) = 1$. The small stroke $'$ denotes the 
 derivative with respect to the argument, and $f'(x)$ is read: "f prime of x".
 \\\\
 The right-hand side was indeed the integral of $\sin^2(x)$ with respect to $x$, so the following 
 integral can then be calculated easily:
 }
 
 \[ \int_0^{2\pi} \sin^2(x)\, dx =  \Big[ \frac{1}{2}x - \frac{1}{2} \sin(x) \cos(x) \Big]_0^{2\pi}
  = \pi - 0  = \pi . \]

  \end{example}
 
 \begin{quickcheck}
		\field{rational}
		\type{input.number}
		\begin{variables}
			\randint{a}{1}{4}			
			\randint{b}{-4}{4}			
			\randint{c}{-4}{4}			
			\function[normalize]{g}{a*x^2+b*x+2*a*x+c+b}  %ge^x=((a*x^2+b*x+c)e^x)'
			\randint{k}{1}{4}	% Zufallsvariable zum Vertauschen:
			\function[calculate]{d1}{-(k-2)*(k-3)*(k-4)/6}  % "Dirac"-funktionen
			\function[calculate]{d2}{(k-1)*(k-3)*(k-4)/2}
			\function[calculate]{d3}{-(k-1)*(k-2)*(k-4)/2}
			\function[calculate]{d4}{(k-1)*(k-2)*(k-3)/6}
			\function{r}{a*x^2+b*x+c}	% richtiger Term
			\function{f1}{b*x+c}		% falsche Terme
			\function{f2}{a*x^2+(b-4)*x+c}	
			\function{f3}{a*x^3+b*x}	

			\function[expand,normalize]{G1}{d1*r+d2*f1+d3*f2+d4*f3}
			\function[expand,normalize]{G2}{d1*f1+d2*r+d3*f3+d4*f2}
			\function[expand,normalize]{G3}{d1*f3+d2*f2+d3*r+d4*f1}
			\function[expand,normalize]{G4}{d1*f2+d2*f3+d3*f1+d4*r}
			
		\end{variables}
		
        \text{\lang{de}{
              Welche der folgenden Funktionen $F(x)$ ist eine Stammfunktion von $f(x)=(\var{g})e^x$? 
              (Geben Sie die entsprechende Nummer an.)\\
              }
              \lang{en}{
              Which of the following functions $F(x)$ is an antiderivative of $f(x)=(\var{g})e^x$? 
              (Enter the corresponding number.)\\
              }
          		\begin{table}[\class{items}]
          			\nowrap{1) $F(x)=(\var{G1})e^x$} & \nowrap{2) $F(x)=(\var{G2})e^x$} \\
          			\nowrap{3) $F(x)=(\var{G3})e^x$} & \nowrap{4) $F(x)=(\var{G4})e^x$}
          		\end{table}
              \lang{de}{Richtig ist \ansref.}
              \lang{en}{The correct answer is \ansref.}
          		}

		
		\begin{answer}
			\solution{k}
		\end{answer}
		\explanation{\lang{de}{
    Man bestimme mit der Produktregel die Ableitungen der angegebenen Funktionen $F(x)$. 
		Die Funktion, deren Ableitung gleich $f(x)$ ist, ist die gesuchte Stammfunktion.
    }
    \lang{en}{
    Use the product rule to determine the derivatives. The function whose derivative equals $f(x)$ is 
    an antiderivative.
    }}

	\end{quickcheck}
	
 
 \section{\lang{de}{Die lineare Substitution}\lang{en}{Linear substitution}}\label{subst}
 
 \lang{de}{
 Bei der Ableitung verketteter (zusammengesetzter) Funktionen wird die \emph{Kettenregel} verwendet, 
 die im Abschnitt \link{link1}{Kettenregel} behandelt wurde. Für $h(x)=f\big(g(x)\big)$ gilt:
 }
 \lang{en}{
 When taking the derivative of a composite function we use the \link{link1}{chain rule}. For 
 $h(x)=f\big(g(x)\big)$ we have:
 }
 \[ h'(x)=\Big(f\big(g(x)\big)\Big)' = f'\big(g(x)\big) \cdot g'(x) . \]
 
 %\begin{tabs*} % [\initialtab{0}]
 % \tab{\lang{de}{Beispiele} }
  
 % \tab{\lang{de}{Beispiel 1}}
 % Sei $h(x)=f(g(x))=e^{2x+1}$. Die äußere Funktion ist $f(x)=e^x$ mit $f'(x)=e^x$ und die innere Funktion ist
 % $g(x)=2x+1$ mit $g'(x)=2$. Die Ableitung von $h(x)$ ist daher:
 % \[ h'(x) = e^{2x+1} \cdot 2 \]
  
  %\tab{\lang{de}{Beispiel 2}}
  %Sei $h(x)=\sin^3(x)$. Die äußere Funktion ist $f(x)=x^3$ mit $f'(x)=3x^2$ 
  %und die innere Funktion ist $g(x)=\sin(x)$ mit $g'(x)=\cos(x)$.
  %Die Ableitung von $h(x)$ ist daher:
  %\[ h'(x) = 3 \sin^2(x) \cos(x) \]
%\end{tabs*}
 
 \lang{de}{
 Ein Spezialfall sind zusammengesetzte Funktionen vom Typ $h(x)=f(mx+b)$.
 Die innere Funktion  $g(x)=mx+b$ beschreibt eine Gerade und ihre
 Ableitung ist eine Konstante. Die Ableitung von $h(x)$ ist daher
 }
 \lang{en}{
 One special case of composite functions are those of the form $h(x)=f(mx+b)$.
 The inner function $g(x)=mx+b$ is the equation of a line and its derivative is constant.
 The derivative of $h(x)$ is thus
 }
 \[ h'(x) = f'\big(g(x)\big) \cdot g'(x) = f'(mx+b) \cdot m .\]
 \lang{de}{
 Wie kann man nun $h(x)=f(mx+b)$ integrieren, wenn nur eine Stammfunktion $F(x)$
 von $f(x)$ bekannt ist?
 }
 \lang{en}{ 
 How can we integrate $h(x)=f(mx+b)$ if an antiderivative $F(x)$
 of $f(x)$ is known?
 }
 
 %Mit Hilfe der Kettenregel überprüft man leicht, dass 
 
 
 %\[ (\frac{1}{m} F(mx+b))' = \frac{1}{m} F'(mx+b) \cdot m = f(mx+b) = h(x)  \]
 
 %gilt. 
 %Man erkennt dann: $\frac{1}{m} F(mx+b)$ ist eine Stammfunktion von $f(mx+b)$.
 
 \begin{rule}[\lang{de}{Lineare Substitution}\lang{en}{Linear substitution}]\label{rule:lin-sub}
 %\emph{Satz:} 
 \lang{de}{
 Wenn $F(x)$ eine Stammfunktion von $f(x)$ ist und Konstanten $m \neq 0$ und $b \in \mathbb{R}$ 
 gegeben sind, so gilt
 }
 \lang{en}{
 If $F(x)$ is an antiderivative of $f(x)$, given constants $m \neq 0$ and $b \in \mathbb{R}$, 
 }
 \[ \int f(mx+b)\, dx = \frac{1}{m} F(mx+b) + C . \]
 \end{rule}
 
 \lang{de}{
 Die Regel kann man durch Ableiten leicht überprüfen.
 %, da sich der Ausdruck $mx+b$
 %durch eine neue Variable substitutieren (ersetzen) ließe.
  %und $dz=m\cdot dx$, d.h. $dx=\frac{1}{m} dz$, und integriert dann nach der neuen Variablen $z$.
 %\[ \int f(mx+b)\, dx = \int f(z) \frac{1}{m}\, dz = \frac{1}{m} F(z) + C = 
 %\frac{1}{m} F(mx+b) + C \]
 Die lineare Substitution ist ein Spezialfall
 einer allgemeineren Substitutionsregel, die hier nicht behandelt wird.\\
 }
 \lang{en}{
 This rule can easily be checked by taking the derivative. 
 Linear substitution is a special case of general substitution which will not be dealt with here.\\
 }
 
 %\begin{block}[example]
 %\emph{Beispiele:} 
 \begin{tabs*}  %[\initialtab{1}]
 \tab{\lang{de}{Beispiel 1}\lang{en}{Example 1}}
 \[ \int e^{2x+1}\, dx = \frac{1}{2} e^{2x+1} + C \]
 \tab{\lang{de}{Beispiel 2}\lang{en}{Example 2}}
 \[ \int_0^1 \sin(\pi x)\, dx = \Big[ - \frac{1}{\pi} \cos(\pi x) \Big]_0^1 =
 -\frac{1}{\pi}\cos(\pi) - \big(- \frac{1}{\pi} \cos(0)\big) = \frac{2}{\pi} \]

 \end{tabs*}
 
\begin{quickcheck}
	\field{rational}
		\type{input.function}
		\begin{variables}
			\randint[Z]{a}{-5}{5}
			\randint[Z]{b}{1}{4}
			\randint{c}{2}{4}
			\function{l}{a*x+b}
			\function[calculate]{t}{-b/a}
			%\function[calculate]{t}{0}

			\randint{k}{1}{4}	% Zufallsvariable zum Vertauschen:
			\function[calculate]{d1}{-(k-2)*(k-3)*(k-4)/6}  % "Dirac"-funktionen
			\function[calculate]{d2}{(k-1)*(k-3)*(k-4)/2}
			\function[calculate]{d3}{-(k-1)*(k-2)*(k-4)/2}
			\function[calculate]{d4}{(k-1)*(k-2)*(k-3)/6}
			
			
			\function[normalize]{f1}{sin(l)}
			\function[normalize]{f2}{cos(l)}
			\function[normalize]{f3}{exp(l)}
			\function[normalize]{f4}{l^c}

			\function[normalize]{g}{d1*f1+d2*f2+d3*f3+d4*f4}
			\function[normalize]{gg}{(d1*(-f2)+d2*f1+d3*f3+d4*(f4*l/(c+1)))/a}
			\function[calculate]{y0}{(d1*(-1)+d3)/a}			
		    \function[normalize]{ff}{gg-y0}
		\end{variables}
		
        \text{\lang{de}{
              Bestimmen Sie mittels linearer Substitution diejenige Stammfunktion $F(x)$ von 
              $f(x)=\var{g}$, für welche $F(\var{t})=0$ gilt.\\
          		$F(x)=$\ansref.}
              \lang{en}{
              Use linear substitution to determine the antiderivative $F(x)$ of $f(x)=\var{g}$ for 
              which	$F(\var{t})=0$ holds.\\
          		$F(x)=$\ansref.
              }}
		

		\begin{answer}
			\solution{ff}
			\checkAsFunction{x}{-5}{5}{50}
		\end{answer}

		\explanation{\lang{de}{
    Verwenden Sie lineare Substitution, sowie die Formeln für Stammfunktionen von Potenzen, Sinus, 
		Kosinus und Exponentialfunktion.
    }
    \lang{en}{
    Use linear substitution as well as the formulas for antiderivatives of powers, sine, cosine and 
    exponential function.
    }}   
	\end{quickcheck}

 
 \section{\lang{de}{Berechnung von Flächen zwischen zwei Graphen}
          \lang{en}{Calculating the area between two graphs}}\label{flaeche}
 
  \lang{de}{
  Schließlich betrachten wir noch die Berechnung von \emph{Flächen} (positiven Flächeninhalten) 
  zwischen dem Graph einer Funktion und der $x$-Achse oder allgemeiner die Berechnung der Fläche 
  zwischen zwei Graphen.
  \\\\
  Das Integral berechnet den \emph{orientierten Flächeninhalt}: die Flächen erhalten dabei ein 
  positives oder ein negatives Vorzeichen. Flächeninhalte mit entgegengesetztem Vorzeichen können 
  sich gegenseitig aufheben. Bei der Berechnung der Fläche zwischen zwei Graphen sollen nun aber alle 
  Flächeninhalte positiv zählen.
  \\\\
  Wir setzen voraus, dass $a$ und $b$ reelle Zahlen mit $a<b$ sind und dass $f(x)$ und $g(x)$ auf dem 
  Intervall $[a;b]$ integrierbar sind. Bei der Berechnung des positiven Flächeninhalts zwischen den 
  Graphen von $f(x)$ und $g(x)$  kommt es nun auf die Lage der beiden Funktionsgraphen an. Im 
  Spezialfall $f(x)=0$ oder $g(x)=0$ ist der entsprechende Graph die $x$-Achse. 
  \\\\
  Besonders einfach ist die Situation dann, wenn $f(x) \geq g(x)$ 
  im \emph{gesamten} Integrationsintervall $[a;b]$ gilt:
  }
  \lang{en}{
  Finally, we will consider how to calculate the area between the graph of a function and the 
  $x$-axis, or more generally, the area between two graphs.
  \\\\
  The integral gives us the \emph{signed area}: the areas we get from our calculations can be either 
  negative or positive. Areas with opposite signs can cancel each other out, so when calculating the 
  area between two graphs we need to make sure that all areas count positively.
  \\\\
  First, we need $f(x)$ and $g(x)$ to both be integrable on the interval $[a, b]$ for some $a$ and 
  $b$ with $a<b$. When calculating the positive area between the graphs of $f(x)$ and $g(x)$, we must 
  pay attention to the relative position of the graphs of both functions. In the special case that 
  $f(x)=0$ or $g(x)=0$, the corresponding graph of that function is the $x$-axis.   
  \\\\
  The case when $f(x) \geq g(x)$ for all $x \in [a, b]$ is especially simple:
  }
  \begin{rule} 
  %\emph{Satz:} 
  \lang{de}{
  Falls $f(x) \geq g(x)$ für alle $x \in [a;b]$  gilt, so beträgt die Fläche zwischen den Graphen von 
  $f(x)$ und $g(x)$ im Intervall von $a$ bis $b$
  }
  \lang{en}{
  If $f(x) \geq g(x)$ for all $x \in [a, b]$, then the area between the two graphs of $f(x)$ and 
  $g(x)$ on the interval from $a$ to $b$ is
  }
  \[ \int_a^b \big(f(x)-g(x)\big)\, dx \, . \]

  % Nach der Integration von $f(x)-g(x)$ (oder auch $g(x)-f(x)$) liefert also der Betrag die gesuchte Fläche.
  \end{rule}
  
  { }\\
  
  %\begin{block}[example]
  \begin{tabs*} % [\initialtab{0}]
 % \tab{\lang{de}{Beispiele} }
  { } \\
  
  \tab{\lang{de}{Beispiel 1}\lang{en}{Example 1}}
  \lang{de}{
  Die Fläche zwischen dem Graphen von $g(x)=x^2-1$ und der $x$-Achse soll bestimmt werden. Gesucht 
  ist die Fläche zwischen den Schnittpunkten. Dann gilt $f(x)=0$ und Gleichsetzen $f(x)=g(x)$ ergibt 
  die Integrationsgrenzen $a=-1$ und $b=1$. Im Intervall $[-1;1]$ gilt $0 \geq x^2-1$.
  \\\\
  Die gesuchte Fläche beträgt dann
  }
  \lang{en}{
  In this example we are looking for the area bounded by the graph of $g(x)=x^2-1$ and the $x$-axis, 
  hence the area between their intersection points. The first function is the $x$-axis, i.e. 
  $f(x)=0$, and setting $f(x)=g(x)$ gives us the limits of integration $a=-1$ and $b=1$. On the 
  interval $[-1, 1]$, we have $0 \geq x^2-1$.
  \\\\
  The area is therefore:
  }
  \[  \int_a^b \big(f(x)-g(x)\big)\, dx = \int_{-1}^1 -(x^2 -1)\, dx  =
   \big[ -\frac{1}{3} x^3 +x \big]_{-1}^1 = 
    \big(-\frac{1}{3} + 1\big)-\big(\frac{1}{3} -1 \big)  =  \frac{4}{3} . \]

  \begin{center}
   \image{T107_Area_B}
  \end{center}
    
  \tab{\lang{de}{Beispiel 2}\lang{en}{Example 2}}
   
  \lang{de}{
  Die Fläche zwischen den Graphen von $f(x)=\cos(x)$ und $g(x)=\sin(x)$ soll im Intervall 
  $[-\frac{3}{4} \pi ; \frac{1}{4} \pi]$ berechnet werden. In diesem Intervall gilt 
  $\cos(x) \geq \sin(x)$.
  %, also $\cos(x)-\sin(x) \geq 0$.
  %Der Betrag kann weggelassen werden und 
  \\\\
  Die Fläche beträgt dann
  }
  \lang{en}{
  In this example we are looking for the area between the graphs of $f(x)=\cos(x)$ and $g(x)=\sin(x)$ 
  on the interval $[-\frac{3}{4} \pi , \frac{1}{4} \pi]$. On this interval $\cos(x) \geq \sin(x)$. 
  %, also $\cos(x)-\sin(x) \geq 0$.
  %Der Betrag kann weggelassen werden und 
  \\\\
  The area is therefore
  }
   \[
    \int_{-\frac{3}{4}\pi}^{\frac{1}{4}\pi} \big(\cos(x)-\sin(x) \big)\, dx  =
    \Big[ \sin(x)+\cos(x) \Big]_{-\frac{3}{4}\pi}^{\frac{1}{4}\pi}  =
    \Big(\frac{\sqrt{2}}{2} + \frac{\sqrt{2}}{2}\Big) - 
   \Big(-\frac{\sqrt{2}}{2}-\frac{\sqrt{2}}{2}\Big)  = 2 \sqrt{2} . \]
   
   \begin{center}  
    \image{T107_Area_C}
    \end{center}  
  \end{tabs*} 
  %\end{block}
  
  
  
  \lang{de}{
  Etwas schwieriger ist die Situation, wenn das Vorzeichen von $f(x)-g(x)$ 
  im Integrationsintervall $[a;b]$ wechselt. Der Betrag 
  $|f(x)-g(x)|$ beschreibt dann den senkrechten Abstand zwischen den Graphen und das Integral
  darüber ergibt die gesuchte Fläche.
  }
  \lang{en}{
  The situation becomes a bit trickier when the sign of $f(x)-g(x)$ changes on the interval $[a, b]$. 
  The absolute value $|f(x)-g(x)|$ will give us the the positive distance between both graphs, and 
  the integral gives us the area desired.
  }
  %In dem Fall müssen die 
  %positiven und negativen Anteile getrennt integriert und die Beträge dann addiert
  %werden.   
  
  \begin{rule}[\lang{de}{Fläche zwischen zwei Graphen}\lang{en}{Area between two graphs}]
  %\emph{Satz:} 
  \lang{de}{
  Die Fläche zwischen den Graphen von zwei stetigen Funktionen $f(x)$ und $g(x)$ im Intervall 
  $[a;b]$ beträgt
  }
  \lang{en}{
  The area between the graphs of two continuous functions $f(x)$ and $g(x)$ on the interval $[a, b]$ 
  is
   }
   \[ \int_a^b \,{\big|\, f(x) - g(x) \, \big|} \, dx \]
  \lang{de}{und kann auf die folgende Weise berechnet werden: }
  \lang{en}{and can be calculated in the following way: }
  
  \begin{enumerate}
  \item \lang{de}{
        Bestimme die Nullstellen von $f(x)-g(x)$ im Intervall $[a;b]$. Dies sind die Schnittstellen.
        }
        \lang{en}{
        Determine the roots of the function $f(x)-g(x)$ on the interval $[a, b]$. These are the 
        intersection points of $f(x)$ and $g(x)$ (or the points where the two graphs simply touch).
        }
  \item \lang{de}{
        Integriere $f(x)-g(x)$ abschnittsweise zwischen dem linken Rand $a$, den Nullstellen von 
        $f(x)-g(x)$ im Intervall $(a;b)$ (sofern vorhanden) und dem rechten Rand $b$.
        }
       \lang{en}{
       Integrate $f(x)-g(x)$ piecewise between the left bound $a$, the roots of $f(x)-g(x)$ on the 
       interval $(a,b)$ (if any exist) and the right bound $b$.
       }
  \item \lang{de}{Addiere die Beträge der einzelnen Integrale.}
        \lang{en}{Add the absolute value of each integral.}
  \end{enumerate}
\end{rule}

\lang{de}{
Bei der beschriebenen abschnittsweisen  Integration darf der Betrag  \emph{nach} 
der Integration angewendet werden, so dass keine gesonderte Überlegung zum Vorzeichen
von $f(x)-g(x)$ erforderlich ist.
}
\lang{en}{
When doing the piece-by-piece integration we can use the absolute value \emph{after} the integration 
so that we do not have to think about the sign of $f(x)-g(x)$ while performing the integration.
}
 
\begin{example}
%\emph{ Beispiel:} 
\lang{de}{
Seien $f(x)=x^3-x^2$ und $g(x)=2x$. Gesucht ist die positive Fläche zwischen den Schnittpunkten der 
Graphen von $f(x)$ und $g(x)$. Man bestimmt zunächst die Nullstellen von
}
\lang{en}{
Let $f(x)=x^3-x^2$ and $g(x)=2x$. We are looking for the area bounded by the two graphs. This is 
between the intersection points of the graphs of $f(x)$ and $g(x)$. Firstly we need to find the roots 
of the function
}
\[ f(x)-g(x)=x^3-x^2-2x=x(x^2-x-2)=x(x+1)(x-2) . \]
\lang{de}{
Die Nullstellen (Schnittstellen von $f(x)$ und $g(x)$) sind daher $x=-1$, $x=0$ und $x=2$. 
Die Fläche zwischen den Graphen von  $f(x)$ und $g(x)$ im Intervall $[-1;2]$ ist dann die Summe der 
Beträge von zwei Integralen:
}
\lang{en}{
The roots (the intersection points of $f(x)$ and $g(x)$) are $x=-1$, $x=0$ and $x=2$. The area 
between the graphs of $f(x)$ and $g(x)$ on the interval $[-1, 2]$ is the sum of the absolute values 
of two integrals:
}
\[ \int_{-1}^2 {\big|\, x^3-x^2-2x\, \big|}\, dx = 
\big| \int_{-1}^0 (x^3-x^2-2x)\, dx \big| + \big| \int_0^2 (x^3-x^2-2x)\,
dx \big| \, . \]
\lang{de}{
Eine Stammfunktion von $f(x)-g(x)=x^3-x^2-2x$ ist $\frac{1}{4} x^4 - \frac{1}{3} x^3 - x^2$.
Die gesuchte Fläche beträgt dann
}
\lang{en}{
An antiderivative of $f(x)-g(x)=x^3-x^2-2x\;$ is $\; \frac{1}{4} x^4 - \frac{1}{3} x^3 - x^2$, and hence
the area can be calculated as
}

\[ \big| \Big[\frac{1}{4} x^4 - \frac{1}{3} x^3 - x^2 \Big]_{-1}^0 \big| + 
\big| \Big[\frac{1}{4} x^4 - \frac{1}{3} x^3 - x^2 \Big]_{0}^2 \big| = 
\big| \frac{5}{12} \big| + \big| - \frac{8}{3} \big| = \frac{37}{12} .\]
\begin{center}
\image{T107_Area_D}
\end{center}
\end{example}

    \begin{quickcheck}
	\field{rational}
		\type{input.number}
		\begin{variables}
			%\randint[Z]{c}{-2}{2}
			\number{c}{1}   % einfacher zu rechnen mit c=1
			\randint{ns1}{-2}{0}
			\randint{ns2}{1}{2}
			\function[expand,normalize]{d}{c*(x-ns1)*(x-ns2)}
			\randint[Z]{a}{-2}{2}
			\randint[Z]{b}{-2}{2}
			\function{g}{a*x+b}
			\function[normalize,sort]{f}{d+g}
			\function[normalize]{dd}{(c/3)*x^3-(c/2)*(ns1+ns2)*x^2+c*ns1*ns2*x}
			\function[calculate]{F1}{(c/3)*ns1^3-(c/2)*(ns1+ns2)*ns1^2+c*ns1*ns2*ns1}
			\function[calculate]{F2}{(c/3)*ns2^3-(c/2)*(ns1+ns2)*ns2^2+c*ns1*ns2*ns2}
			% Stammfunktion ist: c/3*x^3-c/2*(ns1+ns2)*x^2+c*ns1*ns2*x
			\function[calculate]{F}{|F2-F1|} % int_ns1^ns2 |d(x)| 

		\end{variables}
		
		\text{\lang{de}{
    Bestimmen Sie den Inhalt der Fläche, die die Graphen der Funktion $f(x)=\var{f}$ und der Funktion 
    $g(x)=\var{g}$ einschließen.\\
		Die Schnittstellen der Graphen liegen bei \ansref (kleinere Stelle) und \ansref (größere 
    Stelle).\\
		Der Flächeninhalt ist dann \ansref.
    }
    \lang{en}{
    Determine the size of the area enclosed by the graphs of $f(x)=\var{f}$ and $g(x)=\var{g}$.\\
		The intersection points of the graphs are at \ansref (smaller $x$-value) and \ansref 
    (larger $x$-value).\\
		The size of the area is \ansref.
    }}

		\begin{answer}
			\solution{ns1}
		\end{answer}
		\begin{answer}
			\solution{ns2}
		\end{answer}
		\begin{answer}
			\solution{F}
		\end{answer}

		\explanation{\lang{de}{
    Die Schnittstellen sind die Nullstellen der Differenzfunktion $(\var{f})-(\var{g})=\var{d}$.
		Diese sind $\var{ns1}$ und $\var{ns2}$ (Lösen der quadratischen Gleichung $\var{d}=0$).\\
		Der Flächeninhalt ergibt sich dann als 
		$\int_{\var{ns1}}^{\var{ns2}} |\var{d}| dx=| [ \var{dd} ]_{\var{ns1}}^{\var{ns2}}| =|\var{F2}-\var{F1}|=\var{F}$.
		}
    \lang{en}{
    The intersection points are the roots of the function $(\var{f})-(\var{g})=\var{d}$.
    These are $\var{ns1}$ and $\var{ns2}$ (solutions of the quadratic equation $\var{d}=0$).\\
    Hence the size of the area is 
		$\int_{\var{ns1}}^{\var{ns2}} |\var{d}| dx=| [ \var{dd} ]_{\var{ns1}}^{\var{ns2}}| =|\var{F2}-\var{F1}|=\var{F}$.
		}} 
	\end{quickcheck}
  
\end{content}