%$Id:  $
\documentclass{mumie.article}
%$Id$
\begin{metainfo}
  \name{
    \lang{de}{Definition und Eigenschaften des Integrals}
    \lang{en}{Definition and fundamental properties of integrals}
  }
  \begin{description} 
 This work is licensed under the Creative Commons License Attribution 4.0 International (CC-BY 4.0)   
 https://creativecommons.org/licenses/by/4.0/legalcode 

    \lang{de}{Beschreibung}
    \lang{en}{Description}
  \end{description}
  \begin{components}
    \component{generic_image}{content/rwth/HM1/images/g_tkz_T107_Integral_J.meta.xml}{T107_Integral_J}
    \component{generic_image}{content/rwth/HM1/images/g_tkz_T107_Integral_I.meta.xml}{T107_Integral_I}
    \component{generic_image}{content/rwth/HM1/images/g_tkz_T107_Integral_H.meta.xml}{T107_Integral_H}
    \component{generic_image}{content/rwth/HM1/images/g_tkz_T107_Integral_G.meta.xml}{T107_Integral_G}
    \component{generic_image}{content/rwth/HM1/images/g_tkz_T107_Integral_F.meta.xml}{T107_Integral_F}
    \component{generic_image}{content/rwth/HM1/images/g_tkz_T107_SineCosine.meta.xml}{T107_SineCosine}
    \component{generic_image}{content/rwth/HM1/images/g_tkz_T107_Integral_E.meta.xml}{T107_Integral_E}
    \component{generic_image}{content/rwth/HM1/images/g_tkz_T107_Integral_D.meta.xml}{T107_Integral_D}
    \component{generic_image}{content/rwth/HM1/images/g_tkz_T107_Integral_C.meta.xml}{T107_Integral_C}
    \component{generic_image}{content/rwth/HM1/images/g_tkz_T107_Integral_B.meta.xml}{T107_Integral_B}
    \component{generic_image}{content/rwth/HM1/images/g_tkz_T107_Integral_A.meta.xml}{T107_Integral_A}
    \component{generic_image}{content/rwth/HM1/images/g_img_00_Videobutton_schwarz.meta.xml}{00_Videobutton_schwarz}
    \component{generic_image}{content/rwth/HM1/images/g_img_00_video_button_schwarz-blau.meta.xml}{00_video_button_schwarz-blau}
  \end{components}
  \begin{links}
    \link{generic_article}{content/rwth/HM1/T304_Integrierbarkeit/g_art_content_07_ober_und_untersumme.meta.xml}{content_07_ober_und_untersumme}
    \link{generic_article}{content/rwth/HM1/T304_Integrierbarkeit/g_art_content_10_uneigentliches_integral.meta.xml}{content_10_uneigentliches_integral}
    \link{generic_article}{content/rwth/HM1/T107_Integralrechnung/g_art_content_25_stammfunktion.meta.xml}{link2}
  \end{links}
  \creategeneric
\end{metainfo}
\begin{content}
\usepackage{mumie.ombplus}
\ombchapter{7}
\ombarticle{1}

\lang{de}{\title{Definition und grundlegende Eigenschaften des Integrals}}
\lang{en}{\title{Definition and fundamental properties of integrals}}

\begin{block}[annotation]

Integral als orientierter Flächeninhalt,
Integrale von konstanten Funktionen und Geraden (orientierte Rechtecks- und Dreiecksflächen),
Integral der Änderungsrate (Ableitung einer Funktion),
Definition des Riemann-Integrals als Grenzwert von Obersummen und Untersummen,
Eigenschaften des Integrals: Additivität bezüglich der Grenzen, Vertauschung der Grenzen, Linearität.

\end{block}

\begin{block}[annotation]
  Im Ticket-System: \href{http://team.mumie.net/issues/9035}{Ticket 9035}\\
\end{block}

\begin{block}[info-box]
  \tableofcontents
\end{block}


\section{\lang{de}{Das bestimmte Integral als Flächeninhalt mit Vorzeichen}
         \lang{en}{The definite integral as an area}}\label{orient-flaeche}
\lang{de}{
Im Folgenden nehmen wir stets an, dass $a$ und $b$ reelle Zahlen sind und $a<b$ gelte.
Das bestimmte Integral $\big\int_a^b f(x)\, dx$ einer Funktion $f(x)$ gibt den \emph{Flächeninhalt 
mit Vorzeichen} (orientierter Flächeninhalt) an, die der Graph der Funktion mit der $x$-Achse 
zwischen den Integrationsgrenzen $a$ und $b$ einschließt, wobei die Flächenteile oberhalb der 
$x$-Achse \emph{positiv} und die Flächenteile unterhalb der $x$-Achse \emph{negativ} in das Integral 
eingehen.
}
\lang{en}{
In the following we will always assume that $a$ and $b$ are real numbers and $a<b$. The definite 
integral $\big\int_a^b f(x)\, dx$ of a function $f(x)$ gives us the \emph{signed area} bounded by the 
graph of the function, the $x$-axis and the lower and upper bounds, $a$ and $b$. Areas 
above the $x$-axis count as \emph{positive} area, and areas below the $x$-axis count as 
\emph{negative} area.
}

\begin{example}\label{ex:integral-einfuehrung}
%Beispiel: 
\lang{de}{
Das Integral $\big\int_a^b f(x)\, dx$ ist in diesem Beispiel die Summe von drei orientierten 
Flächeninhalten. Der erste und dritte Flächeninhalt oberhalb der $x$-Achse zählen positiv, der zweite 
Flächeninhalt unterhalb der $x$-Achse zählt negativ.
}
\lang{en}{
In this example the integral $\big\int_a^b f(x)\, dx$ is the sum of three signed areas. The first 
and third areas above the $x$-axis count positively towards the area, and the second area below the 
$x$-axis counts negatively.
}
\begin{center}
\image{T107_Integral_A}
\end{center}
\end{example}
{ }\\ 
  
\begin{block}[notation]\label{bestimmtes_Integral}

\lang{de}{
\emph{Schreibweise des bestimmten Integrals}\\\\
Für das Integral der Funktion $f(x)$ in den Grenzen von $a$ bis $b$ schreibt man
}
\lang{en}{
\emph{Notation for definite integrals}\\\\
The integral of the function $f(x)$ between $a$ and $b$ is written
}
\[ \int_a^b f(x)\, dx \,  . \]
\lang{de}{
$a$ und $b$ sind die \textit{Integrationsgrenzen}, 
$x$ ist die \textit{Integrationsvariable}, 
$f(x)$ der \textit{Integrand} und $dx$ das \textit{Differential}.
}
\lang{en}{
The parameters $a$ and $b$ are called the \textit{limits} or \emph{bounds} of integration where $a$ 
is the lower bound and $b$ is the upper bound. $x$ is called the \textit{integration variable}, 
$f(x)$ is called the \textit{integrand}, and $dx$ is called the \textit{differential}.
}

\end{block}

\lang{de}{
Das Integralzeichen $\int$ ist historisch aus dem Summenzeichen $\sum$ entstanden: 
Die Vorstellung des Integrals war, dass die Fläche ermittelt wird, indem schmale Rechtecksflächen mit Vorzeichen summiert werden, d.\,h. 
Produkte von Funktionswerten $f(x)$ als Höhe des Rechtecks und einer 
kleinen (infinitesimalen) Intervallbreite $dx$.
\\\\
Besonders einfach lassen sich Integrale von konstanten Funktionen bestimmen.
Das Integral einer konstanten Funktion $f(x)=c$ ist eine orientierte Rechtecksfläche der Breite $b-a$ und der positiven oder negativen Höhe $c$, also
}
\lang{en}{
The integral sign $\int$ comes from the sum symbol $\sum$. The formal definition of an integral 
divides the area under the curve into progressively thinner rectangles, whose signed areas are added 
together and a limit is taken to form an integral. The height of the rectangle centered at $x$ is the 
function value $f(x)$ and the base length $dx$ is taken to be infinitesimally small using the limit.
\\\\
Determining the integrals of constant functions is especially simple. The integral of a constant 
function $f(x)=c$ is the signed area of a rectangle of width $b-a$ and height (positive or negative) 
$c$, so 
}
\[ \int_a^b c\, dx = c \cdot (b-a) . \]

\lang{de}{
Wie diese Annäherung durch Rechtecksflächen zu verstehen ist, 
können Sie sich an drei Beispielen veranschaulichen.
}
\lang{en}{
How this approximation using rectangular areas is illustrated in the next three examples and is 
explained in more detail in a \link{content_07_ober_und_untersumme}{section} later in this course.
}

\begin{tabs*}[\initialtab{0}\class{exercise}]

 \tab{
 \lang{de}{Konstante Funktion}
 \lang{en}{Constant functions}
 }
\lang{de}{
Für konstante Funktionen liefert die Annäherung der Fläche mit Hilfe von Rechtecksflächen ein exaktes 
Ergebnis.
}
\lang{en}{
For constant functions, the approximation of the area with sums of areas of rectangles gives the 
exact result without taking a limit.
}
 
\begin{center}
\image{T107_Integral_B}
\end{center}

 \tab{   
 \lang{de}{Lineare Funktion}
 \lang{en}{Linear functions}
 }
 
\lang{de}{
Für lineare Funktionen ist das Ergebnis dieser Approximation in der Regel nicht exakt. Durch 
Approximation der Fläche durch $10$ Rechtecke erhält man für die abgebildete Funktion ein Ergebnis 
von $5,625$.
}
\lang{en}{
In general, for linear functions the result of this approximation is not exact. Through approximation 
with $10$ rectangles of the plotted function, we get an area of $5.625$.
}
 
\begin{center}
\image{T107_Integral_C}
\end{center}
 
\lang{de}{Verwendet man $20$ Rechtecke, so erhält man $5,9375$.}
\lang{en}{Using $20$ rectangles, we get $5.9375$.}
  
\begin{center}
\image{T107_Integral_D}
\end{center}

\lang{de}{
Das exakte Ergebnis ist durch die Fläche eines Dreiecks gegeben und beträgt $A=2,5\cdot 5/2=6,25$.
Je mehr Rechtecke man zur Approximation verwendet, umso näher kommt man diesem Ergebnis.
}
\lang{en}{
The exact result is given by the area of a triangle $A=2.5\cdot 5/2=6.25$. The more rectangles we 
use, the closer the approximated result is to the exact result of the limit.
}

 \tab{
 \lang{de}{Allgemein}
 \lang{en}{In general}
 }
 
\lang{de}{
Die gleiche Vorgehensweise kann für beliebige Funktionen angewendet werden, um den Flächeninhalt 
zwischen Funktionsgraph und $x$-Achse näherungsweise zu berechnen. Dabei muss beachtet werden, dass 
Flächen unterhalb der $x$-Achse mit einem negativen Vorzeichen versehen sind.
}
\lang{en}{
The same approach can be applied to arbitrary functions, in order to approximately calculate the area 
enclosed by the function and the $x$-axis. In doing so, one has to keep in mind that areas under the $x$-axis are given a negative sign.
}

\begin{center}
\image{T107_Integral_E}
\end{center}

\lang{de}{
Die Annäherung mit $20$ Rechtecken für die (vorzeichenbehaftete) Fläche, die obige Funktion von $1$ 
bis $2$ mit der $x$-Achse einschließt,  liefert ungefähr den Wert $-12,51$. Der exakte Wert wird in 
einem Beispiel im nächsten \link{link2}{Abschnitt} bestimmt und beträgt $-13$.
\\\\
Die Approximation des Integrals wird im folgenden Video erklärt und in einem 
\link{content_07_ober_und_untersumme}{Abschnitt} des Vertiefungsteils Analysis präzisiert und 
formalisiert.\\
\floatright{\href{https://www.hm-kompakt.de/video?watch=600}{\image[75]{00_Videobutton_schwarz}}}
\\\\
}
\lang{en}{
The approximation with $20$ rectangular areas between $1$ and $2$ for the function depicted above 
results in about $-12.51$. The precise value is calculated in an example in a \link{link2}{later 
section} to be $-13$.
}
\end{tabs*}


\begin{center}
\image{T107_SineCosine}
\end{center}

\begin{quickcheckcontainer}
\randomquickcheckpool{1}{2}
\begin{quickcheck}
		\field{rational}
		\type{input.number}
		\begin{variables}
			\randint{b}{1}{8}
			\function[calculate]{k1}{-b/8}
			\function[calculate]{k2}{b/8}
			\function{lb}{k1*pi}
			\function{ub}{k2*pi}
			\number{sol}{0}
		\end{variables}
		
		\text{\lang{de}{Bestimmen Sie mit Hilfe der obigen Grafik das folgende Integral: \\ } 
          \lang{en}{Use the graphic above to determine the following integral: \\ }
      		$\int_{\var{lb}}^{\var{ub}} \sin(x)dx =$\ansref.}
		
		\begin{answer}
			\solution{sol}
		\end{answer}
		\explanation{\lang{de}{
    Die positiv zählende Fläche und die negativ zählende Fläche sind gleich groß, weshalb das
		Integral gleich $0$ ist.
    }
    \lang{en}{
    The area bounded above the $x$-axis and below the $x$-axis is the same, which is why the 
    integral is $0$.
    }}
	\end{quickcheck}
	
	\begin{quickcheck}
		\field{rational}
		\type{input.number}
		\begin{variables}
			\randint{b}{1}{8}
			\function[calculate]{k1}{1/2-b/8}
			\function[calculate]{k2}{1/2+b/8}
			\function{lb}{k1*pi}
			\function{ub}{k2*pi}
			\number{sol}{0}
		\end{variables}
		
		\text{\lang{de}{
          Bestimmen Sie mit Hilfe der obigen Grafik das folgende Integral: \\ 
          }
          \lang{en}{
          Use the graphic above to determine the following integral: \\
          }
      		$\int_{\var{lb}}^{\var{ub}} \cos(x)dx =$\ansref.}
		
		\begin{answer}
			\solution{sol}
		\end{answer}
		\explanation{\lang{de}{
    Die positiv zählende Fläche und die negativ zählende Fläche sind gleich groß, weshalb das 
		Integral gleich $0$ ist.
    }
    \lang{en}{
    The area bounded above the $x$-axis and below the $x$-axis is the same, which is why the 
    integral is $0$.
    }}

	\end{quickcheck}
\end{quickcheckcontainer}


\lang{de}{
Die Ausnutzung von Symmetrien von Funktionen ist auch Thema dieses Videos:
\floatright{\href{https://www.hm-kompakt.de/video?watch=613}{\image[75]{00_Videobutton_schwarz}}}\\\\
}
\lang{en}{}


\begin{example}
%\emph{Beispiel:} 
\lang{de}{
Das Integral der konstanten Funktion $f(x)=-4$ in den Grenzen von $-1$ bis $2$ ist gleich $-12$:
}
\lang{en}{
The integral of the constant function $f(x)=-4$ from $-1$ to $2$ is equal to $-12$:
}
\[ \int_{-1}^2 -4\, dx = (-4)\cdot \big(2-(-1)\big) = (-4) \cdot 3 = -12. \]
\end{example}


\lang{de}{
Integrale von linearen Funktionen $f(x)=mx+b$ können geometrisch durch Summen von 
vorzeichenbehafteten Rechtecks- und Dreiecksflächen bestimmt werden. Im 
\link{link2}{nächsten Abschnitt} werden wir Integrale von linearen Funktionen zusätzlich auch mit 
Hilfe einer Stammfunktion berechnen.
}
\lang{en}{
Integrals of linear functions $f(x)=mx+b$ can be calculated geometrically by summing signed 
rectangular or triangular areas. In the \link{link2}{next section} we calculate the integral of linear functions with the help of antiderivatives.
}

\begin{example}\label{ex:integral-dreieck}
%\emph{Beispiel:} 
\lang{de}{
Das Integral der Funktion $f(x)=2x+1$ in den Grenzen von $-1$ bis $2$ ist eine Summe von zwei 
Dreiecksflächen mit unterschiedlichem Vorzeichen. Der Flächeninhalt des kleinen Dreiecks
zwischen $x=-1$ und $x=-0,5$ geht mit negativem, der Flächeninhalt des großen Dreiecks zwischen 
$x=-0,5$ und $x=2$ geht mit positivem Vorzeichen in das Integral ein. Der Flächeninhalt eines 
Dreiecks beträgt \nowrap{$\frac{1}{2}$ $\cdot$ Breite $\cdot$ Höhe}. Damit ergibt sich
\[ \int_{-1}^2 (2x+1)\, dx = \frac{1}{2} \big(0,5 \cdot (-1) \big) + \frac{1}{2}\big(2,5 \cdot 5 \big) = 6.\]
}
\lang{en}{
The integral of the function $f(x)=2x+1$ from $-1$ to $2$ is the sum of the areas of two triangles 
with different signs. The area of the small triangle between $x=-1$ and $x=-0.5$ counts negatively 
towards the integral, whereas the area between $x=-0.5$ and $x=2$ counts positively. Recall that the 
area of a triangle is \nowrap{$\frac{1}{2}$ base $\cdot$ height}. Hence, we get
\[ \int_{-1}^2 (2x+1)\, dx = \frac{1}{2} \big(0.5 \cdot (-1) \big) + \frac{1}{2}\big(2.5 \cdot 5 \big) = 6.\]
}

\begin{center}
\image{T107_Integral_F}
\end{center}
\end{example}

\begin{remark}
\lang{de}{
Der Name der Integrationsvariablen ($x$, $t$, ...) hat im bestimmten Integral keine Bedeutung:
}
\lang{en}{
The symbol chosen for the integration variable ($x$, $t$, ...) in definite integrals has no meaning 
outside of the integrand:
}
\[ \int_a^b f(x)\, dx = \int_a^b f(t)\, dt = \int_a^b f(u)\, du = \, ... \]
\lang{de}{Bei Integralen über ein Zeitintervall wird häufig die Variable $t$ verwendet.}
\lang{en}{When integrating over time intervals, the variable $t$ is chosen the most often.}
\end{remark}

\begin{block}[warning]
\lang{de}{
Über Definitionslücken darf nicht hinweg integriert werden und auch Integrale über unbeschränkte 
Intervalle wie z.\,B. $[0;\infty)$ oder Integrale unbeschränkter Funktionen sind zunächst nicht 
zugelassen. In diesen Fällen kann der Flächeninhalt unendlich groß oder unbestimmt sein. Manchmal 
existiert aber ein \emph{uneigentliches Integral}, das wir nicht in diesem Grundlagenteil behandeln. 
Für Interessierte wird das uneigentliche Integral in 
\link{content_10_uneigentliches_integral}{Teil 3a} behandelt.
}
\lang{en}{
Gaps in the domain of a function cannot be integrated over. Unbounded intervals like $[0,\infty)$ 
cannot be integrated over, and unbounded functions also cannot be integrated, as in such cases the 
area can end up being infinitely large or simply undefined. Sometimes the so-called \emph{improper 
integral} exists, but it will not be dealt with in this part of the course. For interested students, 
the improper integral is covered in \link{content_10_uneigentliches_integral}{part 3a}.
}
\end{block}

\begin{example}\label{ex:integral-def-luecke}
%\emph {Beispiel:} 
\lang{de}{
Die Funktion $f(x) = \frac{1}{x^2}$ hat eine Lücke (Polstelle) bei $x=0$. Die Integrale
$\big\int_{-3}^{-1} \frac{1}{x^2}\, dx$ und $\big\int_2^4 \frac{1}{x^2}\, dx$ existieren, nicht aber 
$\big\int_{-1}^1 \frac{1}{x^2}\, dx$ oder $\big\int_0^1 \frac{1}{x^2}\, dx$, 
da $x=0$ dann im Integrationsintervall (bzw. am Rand des Intervalls) liegt.
\\\\
Die Fläche, die zwischen $-3$ und $-1$ von der Funktion und der $x$-Achse eingeschlossen wird, kann 
durch folgendes Bild veranschaulicht werden:
}

\lang{en}{
The function $f(x) = \frac{1}{x^2}$ has a gap (also called a pole) at $x=0$. The integrals
$\big\int_{-3}^{-1} \frac{1}{x^2}\, dx$ and $\big\int_2^4 \frac{1}{x^2}\, dx$ exist, however  
$\big\int_{-1}^1 \frac{1}{x^2}\, dx$ and $\big\int_0^1 \frac{1}{x^2}\, dx$ do not, since $x=0$ is 
inside the limits of integration for these.
\\\\
The area enclosed by the function and the $x$-axis from $-3$ to $-1$ can be depicted as follows:
}

\begin{center}
\image{T107_Integral_G}
\end{center}

\lang{de}{
Die Fläche, die zwischen $-1$ und $1$ von der Funktion und der $x$-Achse eingeschlossen wird, ist in 
diesem Fall auch im uneigentlichen Sinne nicht endlich. Anhand einer Skizze kann dies nicht 
entschieden werden.
}
\lang{en}{
The area enclosed by the function between $-1$ and $1$ and the $x$-axis (in the improper sense 
mentioned in the warning above) is not finite. One cannot properly see this by only looking at an 
image.
}

\begin{center}
\image{T107_Integral_H}
\end{center}
\end{example}

\section{\lang{de}{Das Integral der Änderungsrate}
         \lang{en}{The integral of a derivative}}\label{aenderung}
\lang{de}{
Die Integration hat eine enge Beziehung zur Ableitung einer Funktion und kann als ihre Umkehrung 
angesehen werden. Wenn die momentane Änderungsrate einer Funktion (d.\,h. ihre Ableitung) gegeben 
ist, so liefert die Integration der Ableitung die ursprünglichen Funktionswerte, oder genauer gesagt: 
Das Integral von $f'(x)$ in den Grenzen von $a$ bis $b$ ergibt die gesamte Änderung $f(b)-f(a)$ der 
Funktionswerte. \\
}
\lang{en}{
Integration is related to the derivative of a function, and can be seen as the inverse of the 
differentiation. If we know the instantaneous rate of change of a function (its derivative), then 
integrating it gives us the original function values. More precisely, the integral of $f'(x)$ from 
$a$ to $b$ gives us the difference $f(b)-f(a)$ in the function's value from $a$ to $b$.\\
}

\begin{example}
\label{weg}
%\emph{Beispiel:} 
\lang{de}{
Die zeitabhängige Geschwindigkeit $v(t)$ eines Objekts sei gegeben. Die Geschwindigkeit $v(t)$ ist 
die Ableitung der Wegfunktion $s(t)$. Für ein kleines Zeitintervall $[t;t+\Delta t]$ beträgt die 
Wegänderung näherungsweise $\Delta s = v(t)\cdot \Delta t$. Dies entspricht einer kleinen 
Rechtecksfläche der Breite $\Delta t$ und der Höhe $v(t)$. Dann liefert das Integral 
$\big\int_a^b v(t)\, dt$ die Summation der Wegänderungen und somit die gesamte Weglänge 
$s=s(b)-s(a)$, die zwischen den Zeitpunkten $t=a$ und $t=b$ zurückgelegt wurde. Die Weglänge $s$ 
entspricht der markierten Fläche.
}
\lang{en}{
Let $v(t)$ be the time dependent velocity of an object. The velocity $v(t)$ is the derivative of the 
displacement of the object, $s(t)$. The change in displacement during a short time interval 
$[t,t+\Delta t]$ is approximately $\Delta s = v(t)\cdot \Delta t$. This corresponds with a small 
rectangular area of width $\Delta t$ and height $v(t)$. The integral $\big\int_a^b v(t)\, dt$ 
then gives us the sum of infinitesimally small changes in displacement, and hence the total distance 
traveled $s=s(b)-s(a)$ between times $t=a$ and $t=b$. The total distance traveled $s$ is the marked area.
}
\begin{center}
\image{T107_Integral_I}
\end{center}
\end{example}

\begin{quickcheck}
		\field{real}
		\type{input.number}
		\begin{variables}
			\randint{k}{2}{4}
			\function[calculate]{zk}{10*k}
			\function[calculate]{sol}{10*k^2/2}
		\end{variables}
		
		\text{\lang{de}{
          Ein Stein wird in einen Brunnen fallen gelassen. Die Geschwindigkeit des Steins zur Zeit 
          $t>0$ beträgt	dann ungefähr $v(t)=10t$ m/s (da die Beschleunigung konstant ist). Nach 
          $\var{k}$ Sekunden kommt der Stein unten an.\\
      		Wie tief ist der Brunnen? \\ Die Tiefe des Brunnens beträgt \ansref Meter.
          }
          \lang{en}{
          A stone is dropped into a well. The speed of the stone at the time $t > 0$ is then 
          approximately $v(t) = 10t$ m/s (since the acceleration due to gravity is constant). After 
          $ \var{k} $ seconds the stone arrives at the bottom. \\
          How deep is the well? \\ The well is \ansref meters deep.
          }}
		
		\begin{answer}
			\solution{sol}
		\end{answer}
		\explanation{\lang{de}{
    Die vom Stein zurückgelegte Strecke ergibt sich als $\int_{0}^\var{k} (10t)dt$ Meter. Sie ist 
    also gleich dem Flächeninhalt eines Dreiecks mit Breite $\var{k}$ und Höhe 
    $10\cdot \var{k}=\var{zk}$, also gleich	$\frac{1}{2}\cdot \var{k}\cdot \var{zk}=\var{sol}.$
    }
    \lang{en}{
    The distance covered by the stone is $\int_{0}^\var{k} (10t) dt$ meters. It is therefore equal to 
    the area of a triangle with width $\var{k}$ and height $ 10 \cdot \var{k} = \var{zk} $, thus 
    $\frac{1}{2} \cdot \var{k} \cdot \var{zk} = \var{sol}.$
    }}

	\end{quickcheck}


%begin-cosh
\lang{de}{
Funktionen, die eine Ableitung besitzen, heißen \emph{differenzierbar}. Der folgende Satz gilt für 
differenzierbare Funktionen mit stetiger Ableitung, d.\,h. für Funktionen, deren Ableitung keine 
Sprünge besitzt. Diese Voraussetzung ist in den meisten Fällen, die uns interessieren, erfüllt.
}
\lang{en}{
Functions that have derivatives are called \emph{differentiable}. The following theorem is valid for 
differentiable functions with continuous derivatives, i.e. for functions whose derivative has no 
jumps. This condition is satisfied in most cases we are interested in.
}

\begin{theorem}\label{thm:aenderung}
	\lang{de}{
  Wenn $f(x)$ differenzierbar ist und eine stetige Ableitung $f\prime(x)$ besitzt, so liefert die 
  Integration von $f'(x)$ im Intervall $[a;b]$ die Differenz $f(b)-f(a)$, d.\,h. die Änderung von 
  $f(x)$ zwischen den Stellen $x=a$ und $x=b$ :
	}
	%ende-cosh
	\lang{en}{
  If $f(x)$ is a differentiable function with continuous derivative $f\prime(x)$, then the integral 
  of $f'(x)$ on the interval $[a,b]$ is equal to the difference $f(b)-f(a)$. That is, the change of 
  $f(x)$ between the points $x=a$ and $x=b$ is
  }
	\[ \int_a^b f'(x)\, dx = f(b) - f(a) \, . \]
\end{theorem}

\begin{tabs*}[\initialtab{0}]
\tab{\lang{de}{Erläuterung}\lang{en}{Explanation}}
\lang{de}{
Der Satz folgt aus dem Hauptsatz der Differential- und Integralrechnung
(siehe \link{link2}{folgender Abschnitt}), der in diesem Grundlagenteil aber nicht
bewiesen wird.
\\\\
Warum ist die Aussage des Satzes plausibel? Setzt man $y=f(x)$ und betrachtet ein kleines 
Teilintervall $[x;x+\Delta x]$ von $[a;b]$, so beträgt die Differenz der Funktionswerte 
$\Delta y = f(x+\Delta x)-f(x)$. Ähnlich wie im Beispiel \ref{weg} ist $\Delta y$ näherungsweise 
gleich $f'(x) \cdot \Delta x$. Zerlegt man das Intervall $[a;b]$ in Teilintervalle der Breite 
$\Delta x$, so ergibt die Summation der $\Delta y$ (im Grenzwert $\Delta x \rightarrow 0$) daher das 
Integral $\big\int_a^b f'(x)\, dx$. Andererseits liefert die Summation der 
$\Delta y = f(x+\Delta x)-f(x) $ aber auch die gesamte Änderung $f(b)-f(a)$ der Funktionswerte 
zwischen $a$ und $b$.
}
\lang{en}{
The theorem follows from the fundamental theorem of calculus (see \link{link2}{the following 
section}) which will not be proven in this course.
\\\\
Why is the claim made by the theorem even plausible? Let $y=f(x)$ and consider a small subinterval 
$[x,x+\Delta x]$ of $[a,b]$. The difference in the function values at the start and the end of the 
interval is $\Delta y = f(x+\Delta x)-f(x)$. Much like in example \ref{weg}, $\Delta y$ is 
approximately equal to $f'(x) \cdot \Delta x$. If we break down the interval $[a,b]$ into smaller 
intervals of length $\Delta x$ and sum the $\Delta y$ (in the limit as $\Delta x \rightarrow 0$), we 
get the integral $\big\int_a^b f'(x)\, dx$. On the other hand, this sum gives us the total change of 
the function value between $a$ and $b$, $f(b)-f(a)$.
}
\end{tabs*}

\begin{example}
%\emph{Beispiel:}

\lang{de}{
Die Ableitung der Funktion $f(x)=\frac{1}{2}x^2+1$ ist $f'(x)=x$. Integriert man die Ableitung 
$f'(x)=x$ von $0$ bis $1$, so erhält man den positiven Flächeninhalt eines Dreiecks der Breite und 
Höhe $1$,sodass gilt
}
\lang{en}{
The derivative of the function $f(x)=\frac{1}{2}x^2+1$ is $f'(x)=x$. If we integrate the derivative
$f'(x)=x$ from $0$ to $1$ we get the positive area of a triangle with both base and height $1$ so that
}
%man die Änderung (den Zuwachs) der Funktion
%$f(x)$ zwischen den Stellen $x=0$ und $x=1$. 
\[ \int_0^1 x\, dx = \frac{1}{2} \cdot 1 \cdot 1 = \frac{1}{2} . \]
\lang{de}{
Andererseits ist die gesamte Änderung (der Zuwachs) der Funktionswerte von $f(x)$ zwischen den 
Stellen $x=0$ und $x=1$ nichts anderes als $f(1)-f(0)=\frac{3}{2} - 1 = \frac{1}{2}$. Man erhält in 
diesem Beispiel also
}
\lang{en}{
The total change in the function value of $f(x)$ between $x=0$ and $x=1$ is 
$f(1)-f(0)=\frac{3}{2} - 1 = \frac{1}{2}$. Hence in this example we have
}
\[ \int_0^1 f'(x)\, dx = f(1)-f(0) . \]

\end{example}

\lang{de}{
Wenn die momentane Änderungsrate $f'(x)$ einer Funktion $f(x)$ gegeben ist, so bietet die Integration
von $f'(x)$ also eine Möglichkeit der Bestimmung oder Rekonstruktion der Funktion. Sofern ein 
Anfangswert $f(a)$ gegeben ist, lässt sich jeder Wert $f(b)$ ausrechnen:
}
\lang{en}{
If the instantaneous rate of change (derivative) $f'(x)$ of a function $f(x)$ is given, then 
integrating $f'(x)$ allows us to reconstruct the original function. Given a starting value $f(a)$, 
we can calculate any $f(b)$:
}
\[ f(b) = f(a) +  \int_a^b f'(x)\, dx . \]


\begin{example}
\emph{\lang{de}{Anwendungen:}\lang{en}{Practical Applications}}\\

%Beispiele aus den Naturwissenschaften
\begin{tabs*}[\initialtab{0}] %\class{exercise}
 % \tab{Arbeit, Kraft}
 % Das Integral über die Kraft $F$ längs eines Weges ergibt die verrichtete Arbeit $W$
 % \[  W = \int_{s_0}^{s_1} F(s)\, ds \]
  
\tab{\lang{de}{Energie}\lang{en}{Energy}}
	\lang{de}{
  Das Integral über die Leistung $P(t)$ (Arbeit pro Zeiteinheit, d.\,h. die Ableitung der Arbeit nach 
  der Zeit)	ergibt die im Zeitintervall $[0;T]$ verrichtete Arbeit $W$ (aufgewendete Energie). Der 
  Anfangswert (d.\,h. die Energie bei $t=0$) ist hier gleich $0$:
  }
	\lang{en}{
  Power is the amount of work done per unit time, i.e. the derivative of work with respect to time. 
  The units of work are \emph{Joules} and the units	of power are \emph{Joules/sec}. If we take the 
  integral of the power $P(t)$ between $0$ and $T$, we get the amount of work done $W$ (in Joules) 
  during the time interval $[0,T]$. Assuming that the initial value (the energy at $t=0$) is $0$, we 
  get
  }
	\[ W = \int_0^T P(t)\, dt . \]

\tab{\lang{de}{Ladung}\lang{en}{Charge}}
	\lang{de}{
  Beim Aufladen eines Kondensators über einen Widerstand ergibt das Integral über die Stromstärke 
  $I(t)$ (Ladung pro Zeiteinheit, d.\,h. die Ableitung der Ladung nach der Zeit) die Ladungsmenge 
  $Q$, die der Kondensator zum Zeitpunkt $T$ speichert, sofern er bei $t=0$ völlig entladen war 
  (Anfangswert gleich $0$):
  }
	\lang{en}{
  Current is measured in \emph{charge per unit time}; when charging a capacitor through a resistor. 
  The integral of the current $I(t)$ (i.e. of the derivative of charge with respect to time) is the 
  total capacity $Q$ that the capacitor has stored from time $t=0$ to time $T$:
  } 
	\[ Q = \int_0^T I(t)\, dt . \]

\tab{\lang{de}{Volumen}\lang{en}{Volume}}
	\lang{de}{
  Die Zulaufrate eines Wasserbehälters sei $f(t)$ (Volumen pro Zeiteinheit, d.\,h. die Ableitung des 
  Volumens nach der Zeit). Das Integral über $f(t)$ ergibt das Volumen $V$ (die Wassermenge im 
  Behälter) zum Zeitpunkt $T$, sofern der Behälter bei $t=0$ leer war (Anfangswert gleich $0$):
  }
	\lang{en}{
  Suppose the filling rate of a container is $f(t)$ (\emph{volume per unit time}; the derivative of 
  volume with respect to time). The integral of $f(t)$ gives us the filled volume $V$ at the time 
  point $T$ assuming the container was empty at time $t=0$:
  }
	\[ V = \int_0^T f(t)\, dt . \]
%begin-cosh
\tab{\lang{de}{Kosten}\lang{en}{Marginal costs}}
	\lang{de}{
  Die Grenzkosten sind diejenigen Kosten,	die durch die Produktion einer zusätzlichen Mengeneinheit 
  eines Produktes entstehen. Die Grenzkostenfunktion für die Herstellung eines bestimmten Produktes 
  sei $f(x)$.	Dann können die Kosten $K$ für die Produktion von $N$ Einheiten durch Integration der 
  Grenzkosten und Additon der Fixkosten $K_0$ bestimmt werden:
  \[ K = K_0 + \int_0^N f(x)\, dx . \]
  }
  \lang{en}{
  In economics, the \textit{marginal cost} is the cost associated with the production of an 
  additional unit of a product. Let $f(x)$ be the marginal cost function for the production of a 
  specific product. The total cost $C$ for the production of $Q$ units can be written as
	the integral of the marginal cost plus the addition of any fixed costs, $C_0$:
	\[ C = C_0 + \int_0^Q f(x)\, dx . \]
  }
%ende-cosh

\end{tabs*}
\end{example}

% \section{\lang{de}{Obersummen und Untersummen}\lang{en}{Upper Sums and Lower Sums}}\label{summe}
% 
% \lang{de}{Das Integral entsteht durch Summation von Rechtecksflächen mit Vorzeichen, welche den 
% gesuchten orientierten Flächeninhalt approximieren. 
% Zur Bestimmung des Integrals zerlegt man das Integrationsintervall 
% $[a;b]$ in $n$ Teilintervalle der Breite $\Delta x_k$, mit $k \in \{1;2;...;n\}$.
%   Auf diesen Intervallen ermittelt man jeweils
%  den kleinsten Funktionswert $m_k$ und den größten Funktionswert $M_k$. % (eigentlich das Infimum und das 
%  % Supremum, aber diese Begriffe werden in diesem Kurs nicht behandelt). %, wobei auch das Vorzeichen berücksichtigt wird.
%   Man erhält je Teilintervall zwei orientierte Rechtecksflächen, $m_k \cdot \Delta x_k$ und $M_k \cdot \Delta x_k$. 
%  %Das Vorzeichen ist positiv oder negativ, je nachdem ob der Funktionswert 
%  %$\geq 0$ bzw. $<0$ ist. 
%  Die Summation 
%  %der orientierten Flächeninhalte, die aus dem 
%  %minimalen Funktionswert in jedem Teilintervall gebildet werden,
%   liefert die \emph{Untersumme} $U(Z)$ und die \emph{Obersumme} $O(Z)$. Die Werte hängen von 
%   der gewählten Zerlegung $Z$ des Intervalls $[a;b]$ ab.
%   \[ U(Z)= \sum_{k=1}^n m_k \cdot \Delta x_k \, \, \text{ und } \, \, \,   O(Z)= \sum_{k=1}^n M_k \cdot \Delta x_k \, .\]
%  %Die Summation der Flächeninhalte mit dem maximalen Funktionswert liefert die \emph{Obersumme}. 
%  Der gesuchte Integralwert liegt \emph{zwischen der Untersumme und der Obersumme}. 
%  Die Abschätzung des Integrals mit Hilfe der Untersumme und Obersumme ist umso 
%  genauer, je kleiner die Teilintervalle sind, d.h. je feiner die Zerlegung ist. } 
%  \lang{en}{The integral is made up of the sum of signed rectangular areas which approximate the 
%  area of a given region. In order to determine the value of the integral, we break down the limits of integration
% $[a,b]$ into $n$ equal parts of size $\Delta x_k$ where $k \in \{1,2,...,n\}$.
%   On each of these intervals we can identify the smallest function value $m_k$ as well as the largest $M_k$. 
%   On each interval we get two signed rectangles, $m_k \cdot \Delta x_k$ and $M_k \cdot \Delta x_k$. 
%  The sum of these rectangles gives us the \emph{lower sum} $L(Z)$ and the \emph{upper sum} $U(Z)$, respectively. These values 
% depend on the decomposition $Z$ of the interval $[a,b]$.
%   \[ L(Z)= \sum_{k=1}^n m_k \cdot \Delta x_k \, \, \text{ and } \, \, \,   U(Z)= \sum_{k=1}^n M_k \cdot \Delta x_k \, .\]
%  The true value of the integral lies \emph{between the upper sum and lower sum}. 
%  The estimate of the integral using the upper and lower sums improves the smaller the intervals get (i.e. the smaller the bases of the rectangles get).}
% 
% \begin{example}\label{zweizerl}
% %\emph {Beispiel:} 
% 
% \lang{de}{Links wird das Integrationsintervall $[0;2]$ in zwei 
% Teilintervalle und rechts in 
% zehn Teilintervalle aufgeteilt. Die Untersumme ist die dunkel gefärbte Fläche,
% die Obersumme ist die hell gefärbte plus die dunkel gefärbte Fläche.
% In beiden Fällen liegt der wahre Integralwert zwischen der Untersumme und der Obersumme. 
% Mit zehn Teilintervallen wird aber eine bessere Näherung als mit zwei Teilintervallen erzielt.\\
% }
% \lang{en}{On the left the interval $[0,2]$ has been divided into two smaller subintervals. On the right, the same
% interval has been divided into 10 equal intervals. The lower sum is calculated from the darker 
% rectangles and the upper sum is calculated from the lighter ones. In both cases, the true value
% of the integral lies between the upper and lower sums, however with ten intervals the approximation
% is better than the approximation with only two.\\
% }
% 
% \image{image4}
% \image{image5}
% \end{example}
% %\includegraphics[width=7cm]{summe2.png} %\hspace*{5mm}
% %\includegraphics[width=7cm]{summe5.png}
% %\caption{Untersummen und Obersummen für $f(x)=x^2+1$ im Intervall $[0,2]$ mit $2$ Teilintervallen (links) und $10$ Teilintervallen (rechts).}
% %\label{summen2-5}
% %\end{figure}
% 
% \begin{example}
% %\emph{Beispiel:} 
% \lang{de}{In obigem Beispiel \ref{zweizerl} ist $f(x)=x^2+1$ und die Integrationsgrenzen sind $a=0$ und $b=2$. 
% Bei der Aufteilung in zwei Teilintervalle (linker Graph) wird 
% das Intervall $[0;2]$ 
% in zwei Teilintervalle $[0;1]$ und $[1;2]$ zerlegt und für diese Zerlegung die Untersumme $U(Z_2)$ und die
% Obersumme $O(Z_2)$ berechnet. Die Intervallbreite ist $1$, der kleinste Funktionswert liegt am linken Rand, der größte am rechten Rand.
% Man erhält:
% \[ U(Z_2)= f(0)\cdot 1 + f(1)\cdot 1 = 1+2 = 3 \, \, \, \text{ und } \, \, \,
% O(Z_2)=f(1) \cdot 1 + f(2)\cdot 1 = 2+5 = 7 . \]
% }
% \lang{en}{In Example \ref{zweizerl} above, $f(x)=x^2+1$ with lower and upper bounds $a=0$ and $b=2$. 
% Dividing the interval $[0,2]$ into two (in the left image) gives us the intervals $[0,1]$ and $[1,2]$ 
% From this we can calculate the lower $L(Z_2)$ and upper $U(Z_2)$ sums to be:
% \[ L(Z_2)= f(0)\cdot 1 + f(1)\cdot 1 = 1+2 = 3 \, \, \, \text{ and } \, \, \,
% U(Z_2)=f(1) \cdot 1 + f(2)\cdot 1 = 2+5 = 7 . \]
% Note that for this function, the heights of the rectangles used in the lower sum are the function values $f(x)$ at the left boundary of each interval and
% the heights for the upper sums are the function values at the right boundary.
% }
% \end{example}
% 
% 
% \lang{de}{In der folgenden Definition wird der Begriff der beschränkten Funktion verwendet.
% Eine Funktion $f$ ist \emph{beschränkt}, falls es eine Zahl $R>0$ gibt, so dass
% $-R \leq f(x) \leq R$ für alle $x$ in ihrem Definitionsbereich gilt.}
% \lang{en}{In the following definition we'll deal with the idea of a bounded function. A function $f$ is called \emph{bounded} if there is a number 
% $R>0$ so that $-R \leq f(x) \leq R$ for all $x$ in the domain of $f$.}
% 
% \begin{definition}
% 
% %\emph{Definition:} 
% \lang{de}{Sei $f(x)$ eine auf dem Intervall $[a;b]$ beschränkte Funktion.
% % $f$ heißt (Riemann-) integrierbar auf dem Intervall $[a;b]$, 
% %falls es eine Folge von Zerlegungen $Z_n$ von $[a;b]$ gibt, deren Feinheit
% %(maximale Breite der Teilintervalle) nach $0$ konvergiert für $n \rightarrow \infty$
% %und deren Obersummen und Untersummen gegen einen gemeinsamen Grenzwert (das Integral) konvergieren:
% %\[ \lim_{n\rightarrow \infty} U(Z_n) = \lim_{n\rightarrow \infty} O(Z_n) = \int_a^b f(x)\, dx \]
% $f$ heißt integrierbar auf dem Intervall $[a;b]$, falls sich Untersumme und Obersumme
% durch Verfeinerung der Zerlegung beliebig genau annähern lassen, d.h. falls es zu jeder
% vorgegebenen Zahl $\epsilon>0$ (und sei sie noch so klein) eine Zerlegung $Z_n$ des
% Intervalls $[a;b]$ in $n$ Teilintveralle gibt, sodass
% \[ 0 \leq O(Z_n) - U(Z_n) \leq \epsilon \]
% gilt. \\
% Ist $f$ integrierbar auf $[a;b]$, so gibt es eine eindeutige reelle Zahl
% $\big\int_a^b f(x)\, dx$, die als \emph{Integral von $f$ von $a$ nach $b$}
% bezeichnet wird, sodass
% \[ U(Z) \leq \int_a^b f(x)\, dx \leq O(Z) \]
% für alle Zerlegungen $Z$ des Intervalls $[a;b]$ in endlich viele Teilintervalle gilt.}
% \lang{en}{Let $f(x)$ be a bounded function on the interval $[a,b]$.
% A function $f$ is called integrable on the interval $[a,b]$ if the upper and lower sums 
% converge to the same limit as the interval size decreases. Integrability is also equivalent to the following:
% a function is integrable if for every $\epsilon>0$ there exists a decomposition of the interval $[a,b]$ into $n$ parts such that
% \[ 0 \leq U(Z_n) - L(Z_n) \leq \epsilon .\]  \\
% If $f$ is integrable on $[a,b]$ then there is a unique real number 
% $\big\int_a^b f(x)\, dx$ called the \emph{integral of $f$ from $a$ to $b$}
% and
% \[ L(Z) \leq \int_a^b f(x)\, dx \leq U(Z) \]
% for any decomposition $Z$ of the interval $[a,b]$ into finitely many subintervals.}
% \end{definition}
% 
% \lang{de}{Das Integral kann also mit Hilfe von Obersummen und Untersummen definiert werden. 
% Zur Berechnung des Integrals einer integrierbaren Funktion $f$ lässt man 
% die Breite der Teilintervalle immer kleiner werden. Schließlich ergibt sich dann
% der gesuchte Integralwert.
% %Im Grenzwert konvergieren Untersummen und Obersummen dann gegen den gesuchten Integralwert.
% Da es weitere Integralbegriffe gibt, spricht man hierbei 
% vom \emph{Riemann-Integral}.}
% \lang{en}{In this way, the integral can be defined with the help of lower and upper sums.
% In order to calculate the integral of an integrable function $f$ we simply take smaller and smaller subintervals. 
% In the limit as the interval size goes to zero, the upper and lower sums converge to the value of the integral. 
% There are other definitions of the integral, and this construction gives the so-called \emph{Riemann Integral}.}
% 
% \begin{example}
% %\emph{Beispiel: } 
% \lang{de}{Wir betrachten weiterhin $f(x)=x^2+1$ und unterteilen $[0;2]$ nun in $10$ 
% Teilintervalle der Breite $\frac{1}{5}$ (siehe Beispiel \ref{zweizerl}). % Diese Zerlegung bezeichnen wir
% %mit $Z_{10}$.
% Der minimale Funktionswert liegt jeweils am linken Rand jedes Teilintervalls. Die Untersumme ist daher:
% \[ U(Z_{10})=\Big(  f(0)+f(\frac{1}{5})+f(\frac{2}{5})+f(\frac{3}{5})+f(\frac{4}{5})
% +f(1)+f(\frac{6}{5})+f(\frac{7}{5})+f(\frac{8}{5})+
% f(\frac{9}{5}) \Big) \cdot \frac{1}{5} . \]
% Mit Hilfe einer Wertetabelle lässt sich dies leicht ausrechnen: $U(Z_{10})=\frac{107}{25}=4{,}28$.\\
% 
% Der maximale Funktionswert liegt jeweils am rechten Rand der Teilintervalle. Die Obersumme ist daher:
% \[ O(Z_{10})= \Big(  f(\frac{1}{5})+f(\frac{2}{5})+f(\frac{3}{5})+f(\frac{4}{5})
% +f(1)+f(\frac{6}{5})+f(\frac{7}{5})+f(\frac{8}{5})+
% f(\frac{9}{5}) + f(2) \Big) \cdot \frac{1}{5} . \]
% Man erhält $O(Z_{10})=\frac{127}{25}=5{,}08$. Der wahre Integralwert, den wir mit 
% feineren Zerlegungen (oder mit
% Hilfe einer Stammfunktion im \link{link2}{nächsten Abschnitt}) ausrechnen können, 
% beträgt $\big\int_0^2 (x^2+1)\, dx = \frac{14}{3} \approx 4{,}67$.}
% \lang{en}{Let's consider the function $f(x)=x^2+1$ on the interval $[0,2]$ divided into $10$ equal subintervals
% of size $\frac{1}{5}$ (see Example \ref{zweizerl}).
% The minimal function value lies on the left boundary of each subinterval, and hence the lower sum is:
% \[ L(Z_{10})=\Big(  f(0)+f(\frac{1}{5})+f(\frac{2}{5})+f(\frac{3}{5})+f(\frac{4}{5})
% +f(1)+f(\frac{6}{5})+f(\frac{7}{5})+f(\frac{8}{5})+
% f(\frac{9}{5}) \Big) \cdot \frac{1}{5}  \]
% \[ L(Z_{10})=\frac{107}{25}=4{.}28 \]\\
% 
% The maximal function value is on the right boundary of each subinterval, hence the upper sum is:
% \[ U(Z_{10})= \Big(  f(\frac{1}{5})+f(\frac{2}{5})+f(\frac{3}{5})+f(\frac{4}{5})
% +f(1)+f(\frac{6}{5})+f(\frac{7}{5})+f(\frac{8}{5})+
% f(\frac{9}{5}) + f(2) \Big) \cdot \frac{1}{5}  \]
% \[ U(Z_{10})=\frac{127}{25}=5.08 \] The true value of the integral can be calculated
% with finer decompositions or simply by using the antiderivative from the \link{link2}{next section} and is
% $\big\int_0^2 (x^2+1)\, dx = \frac{14}{3} \approx 4.67$.}
% 
% \end{example}
% 
% %\begin{block}[info-box]
% \lang{de}{Welche Funktionen sind integrierbar? Alle stetigen Funktionen wie 
% Polynome, die Exponentialfunktion $e^x$, die trigonometrischen Funktionen 
% $\sin(x)$, $\cos(x)$ und die Betragsfunktion sind integrierbar über beliebigen Intervallen $[a;b]$. 
% Eine Funktion $y=f(x)$ ist \emph{stetig}, wenn kleine (infinitesimale) Änderungen des $x$-Wertes nur zu kleinen
% (infinitesimalen) Änderungen des $y$-Wertes führen. Solche Funktionen besitzen keine Sprünge der Funktionswerte.\\
% 
% Außerdem sind sogar stückweise stetige Funktionen integrierbar, die abschnittsweise aus stetigen Funktionen
% zusammengefügt werden (siehe Beispiel \ref{stueck}).}
% \lang{en}{Which functions are integrable? All continuous functions, such as polynomials, the exponential function $e^x$, the trig functions $\sin(x)$, $\cos(x)$, and the
% absolute value function are integrable on any interval $[a,b]$. 
% A function $y=f(x)$ is called \emph{continuous} if small (infinitesimal) changes in the $x$-value only lead to small (infinitesimal) changes in the $y$-value. Such functions 
% don't have any jumps in their function values.\\
% 
% Piecewise continuous functions are also integrable; see Example \ref{stueck}.}
% %\end{block}



\section{\lang{de}{Eigenschaften des Integrals}
         \lang{en}{Properties of integrals}}\label{eigenschaften}

\lang{de}{Das Integral ist additiv bezüglich des Integrationsintervalls:}
\lang{en}{Integrals are additive on their integration intervals:}

\begin{rule}\label{thm:additiv}
\lang{de}{
Für reelle Zahlen $a<b<c$ und eine auf dem Intervall $[a;c]$ integrierbare Funktion $f(x)$ gilt:
\[ \int_a^b f(x)\, dx + \int_b^c f(x)\, dx = \int_a^c f(x)\, dx \,  . \]
\floatright{\href{https://api.stream24.net/vod/getVideo.php?id=10962-2-10772&mode=iframe&speed=true}{\image[75]{00_video_button_schwarz-blau}}}\\
}
\lang{en}{
Let $f(x)$ be an integrable function on the interval $[a,c]$. For all real numbers $a<b<c$, we have
\[ \int_a^b f(x)\, dx + \int_b^c f(x)\, dx = \int_a^c f(x)\, dx \,  . \]
}
\end{rule}

\begin{example}
\label{stueck}
%\emph{Beispiel: } 
\lang{de}{Wir betrachten eine abschnittsweise definierte Funktion $f(x)$: }
\lang{en}{Consider the following piecewise function $f(x)$:}
\[ f(x) = \begin{cases} x, & x \in [0\lang{de}{;}\lang{en}{,}1) \\
                        2x, & x \in [1\lang{de}{;}\lang{en}{,}2]
          \end{cases} \]

\begin{center}
    \image{T107_Integral_J}     
\end{center}

\lang{de}{
$f(x)$ hat also eine Sprungstelle bei $x=1$. An den übrigen Stellen ist sie aber stetig.
Daher ist $f(x)$ auf $[0;2]$ integrierbar und
das Integral ergibt sich als Summe der Integrale von $0$ bis $1$ und von $1$ bis $2$.
Das erste Integral ist durch eine Dreiecksfläche und das zweite durch eine Trapezfläche 
gegeben, die man beispielsweise als Summe einer Rechtecks- und einer Dreiecksfläche berechnen
kann. Die Flächeninhalte der markierten Kästchen lassen sich auch direkt aus der 
Abbildung ablesen. Alle 
Integrale sind positiv, da $f(x) \geq 0$ für $x \in [0;2]$ gilt. Insgesamt ist also 
}   
\lang{en}{
$f(x)$ has a jump discontinuity at $x=1$. At all other points the function is continuous, hence 
$f(x)$ is integrable on $[0,2]$ and the integral can be given via the sum of the integrals from $0$ 
to $1$ and from $1$ to $2$. The first integral is given via the triangular area, the second via the 
trapezoidal area. Both can be calculated via the known formulas for their areas. The area of the 
marked regions can be read directly from the graph. The integrals are positive, since 
$f(x) \geq 0$ for $x \in [0, 2]$. So
}
\[ \int_0^2 f(x)\, dx = \int_0^1 f(x)\, dx + \int_1^2 f(x)\, dx = \frac{1}{2}(1\cdot 1) + 1 \cdot 2 + 
\frac{1}{2}(1 \cdot 2) = \frac{7}{2} . \]                    
\end{example}

\lang{de}{
Für die Integrationsgrenzen $a$ und $b$ gilt üblicherweise $a<b$. 
Man betrachtet aber auch die Fälle $a=b$ und $a>b$:
}
\lang{en}{
For the limits $a$ and $b$, by convention $a<b$. Let us however consider the case $a \geq b$:
}

\begin{definition}\label{def:Integr_grenz_tausch} 
%\emph{Definition:} 
\lang{de}{
Wenn $f(x)$ auf dem Intervall $[a;b]$ mit $a<b$ integrierbar ist, dann definiert man 
auf folgende Weise das Integral mit vertauschten Grenzen:
}
\lang{en}{
If $f(x)$ is integrable on the interval $[a,b]$ where $a<b$, then we define the integral with
swapped upper and lower bounds in the following way:
}
\[ \int_b^a f(x)\, dx = - \int_a^b f(x)\, dx \, . \]
\lang{de}{Außerdem setzt man $\big\int_a^a f(x)\, dx = 0$.}
\lang{en}{In addition, we define $\big\int_a^a f(x)\, dx = 0$.}
\end{definition}

\lang{de}{
Warum ändert man das Vorzeichen? Bei der Integration von $b$ nach $a$ (d.\,h. in negativer 
$x$-Richtung) ändert sich die Orientierung und daher das Vorzeichen des Integrals. Eine weitere 
Begründung liefert die Additivität des Integrals bezüglich der Integrationsgrenzen:
}
\lang{en}{
Why does the sign change when swapping the limits? Intuitively, if we integrate from $b$ to $a$, we 
are integrating in the negative $x$-direction and hence we need to add a negative sign. More 
convincing perhaps is the need to keep the additive property of limits of integrals:
}
\[ \int_a^b f(x)\, dx  + \int_b^a f(x)\, dx = \int_a^a f(x)\, dx = 0 .\]


\end{content}