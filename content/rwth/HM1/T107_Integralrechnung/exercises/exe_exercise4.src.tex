\documentclass{mumie.element.exercise}
%$Id$
\begin{metainfo}
  \name{
    \lang{de}{Ü04: Unbestimmtes Integral}
    \lang{en}{exercise 4}
  }
  \begin{description} 
 This work is licensed under the Creative Commons License Attribution 4.0 International (CC-BY 4.0)   
 https://creativecommons.org/licenses/by/4.0/legalcode 

    \lang{de}{Hier die Beschreibung}
    \lang{en}{}
  \end{description}
  \begin{components}
    \component{generic_image}{content/TU9/omb_plus/calculus_of_integration/media/images/g_img_piecewise.meta.xml}{piecewise}
  \end{components}
  \begin{links}
  \end{links}
  \creategeneric
\end{metainfo}
\begin{content}
\title{
  \lang{de}{Ü04: Unbestimmtes Integral}
  \lang{en}{Exercise 4}
}



\begin{block}[annotation]
  Im Ticket-System: \href{http://team.mumie.net/issues/9604}{Ticket 9604}
\end{block}

%\begin{enumerate}

%\item[a)]
%  \lang{de}{Welche der folgenden Formeln für unbestimmte Integrale sind richtig, falls $x>0$ gilt und
%$C \in \mathbb{R}$ eine beliebige Konstante ist?}
%\lang{en}{Which of the following indefinite integral formulas are correct,
%if $x>0$ and $C \in \mathbb{R}$ is an arbitrary constant?}

%\begin{enumerate}
%\item[1)] $\big \int \ln(x)\, dx = \frac{1}{x} + C$
%\item[2)] $\big \int \frac{1}{x^2}\, dx = - \frac{1}{x}$
%\item[3)] $\big \int \ln(x)\, dx = x \ln(x) - x + C$
%\item[4)] $\big \int \frac{1}{x}\, dx = \ln(2x) + C$
%\item[5)] $\big \int \frac{1}{x}\, dx = C \ln(x)$
%\end{enumerate}

%\item[b)]
  \lang{de}{Lösen Sie die folgenden unbestimmten Integrale, wobei $a$ und $b$ konstante reelle
Zahlen sind. }
\lang{en}{Solve the following indefinite integrals, where $a$ and
$b$ are real-valued constants.}
\begin{enumerate}
\item[1)] $\big \int \big(a \sin(x) + b \cos(x)\big)\, dx$
\item[2)] $\big \int \frac{ax+b}{x^3}\, dx$
\item[3)] $\big \int \left(\frac{1}{(ax)^6} - b^2 e^x \right)\, dx $
\end{enumerate}

%\end{enumerate}

\begin{tabs*}[\initialtab{0}\class{exercise}]
  %\tab{\lang{de}{Antworten}
  \tab{\lang{de}{Antwort}
  %\lang{en}{Answers}
  \lang{en}{Answer}
  }

  %\lang{de}{a) Die Integrale 3) und 4) sind richtig, die übrigen sind fehlerhaft.}
  %\lang{en}{a) Integrals 3) and 4) are correct, the rest have errors.}
  \\

%b) 
\begin{enumerate}
\item[1)] $\big\int \big(a \sin(x) + b \cos(x)\big)\, dx = -a \cos(x) + b\sin(x) + C$
\item[2)] $\big\int \frac{ax+b}{x^3}\, dx = - a \frac{1}{x} - \frac{b}{2} \frac{1}{x^2} + C$
\item[3)] $\big\int \left(\frac{1}{(ax)^6} - b^2e^x \right)\, dx = -\frac{1}{5a^6} \frac{1}{x^5} - b^2e^x + C$
\end{enumerate}

%\tab{\lang{de}{L"osung zu a}
%\lang{en}{Solution for a}
%}

 %\begin{incremental}[\initialsteps{1}]
 %   \step 1)
 %  \lang{de}{Die Ableitung von $\frac{1}{x}$ ist $- \frac{1}{x^2}$.
 %   Die Formel ist also falsch.}
 %   \lang{en}{The derivative of $\frac{1}{x}$ is $- \frac{1}{x^2}$.
 %   Thus the equation is wrong.}

 %   \step 2)
 %   \lang{de}{Es gilt $\big \int \frac{1}{x^2}\, dx = - \frac{1}{x} + C$. Die additive Konstante $C$ fehlt
 %   in der Formel; das unbestimmte Integral ist eine \emph{Menge} von Stammfunktionen.}
 %   \lang{en}{$\big \int \frac{1}{x^2}\, dx = - \frac{1}{x} + C$ holds.
 %   The constant of integration $C$ is missing in the equation;
 %   the indefinite integral is a \emph{set} of antiderivatives.}

 %   \step 3)
 %   \lang{de}{Die Formel ist richtig, denn die Ableitung von $x \ln(x) - x$ ist
 %   $x \cdot \frac{1}{x} + 1 \cdot \ln(x) - 1 = \ln(x)$.}
 %   \lang{en}{The equation is correct, since the derivative of $x \ln(x) - x$ is
 %   $x \cdot \frac{1}{x} + 1 \cdot \ln(x) - 1 = \ln(x)$.}

 %   \step 4)
 %   \lang{de}{Die Formel ist richtig, obwohl $\ln(2x)$ statt $\ln(x)$ angegeben ist.
 %   Es gilt $\ln(2x)=\ln(2) + \ln(x)$ und dies ist ebenso eine Stammfunktion von $\frac{1}{x}$.}
 %   \lang{en}{The equation is correct, even though $\ln(2x)$, instead of $\ln(x)$, is given.
 %    Remember that $\ln(2x)=\ln(2) + \ln(x)$ which is also an antiderivative of $\frac{1}{x}$.}


  %  \step 5)
  %  \lang{de}{$C \cdot \ln(x)$ ist keine Stammfunktion
  %  von $\frac{1}{x}$, denn $(C \ln(x))' = C \frac{1}{x}$. Die Integrationskonstante ist \emph {additiv}.}
  %  \lang{en}{$C \cdot \ln(x)$ is not an antiderivative of $\frac{1}{x}$,
  %  since $(C \ln(x))' = C \frac{1}{x}$. The constant of integration is \emph {additive}.}

 %\end{incremental}
\tab{\lang{de}{L"osung}
\lang{en}{Solution}
%\tab{\lang{de}{L"osung zu b}
%\lang{en}{Solution for b}
}
\begin{incremental}[\initialsteps{1}]
     \step
    \lang{de}{Wir wenden die Summen- und Faktorregel an. }
    \lang{en}{We use the sum and constant rules.}
     \step 1)
    \lang{de}{Wegen $\big \int \sin(x)\, dx = - \cos(x) + C$ und $\big\int \cos(x)\, dx = \sin(x) + C$ folgt aus
     der Summen- und Faktorregel}
     \lang{en}{Given $\big \int \sin(x)\, dx = - \cos(x) + C$ and $\big\int \cos(x)\, dx = \sin(x) + C$,
     using the sum and constant rules we get}
     \[ \int \big(a \sin(x) + b \cos(x)\big)\, dx = -a \cos(x) + b\sin(x) + C .\]

     \step 2)
    \lang{de}{Es gilt }
    \lang{en}{It holds that}
     \[ \frac{ax+b}{x^3} = \frac{ax}{x^3} + \frac{b}{x^3} =
     a \cdot \frac{1}{x^2} + b \cdot \frac{1}{x^3} = a \cdot x^{-2} + b \cdot x^{-3}.\]
    \lang{de}{Wegen $\big \int x^{-2} \, dx = - x^{-1} + C  =  - \frac{1}{x} + C$ und
     $\big \int x^{-3} \, dx = - \frac{1}{2} \frac{1}{x^2} + C$ folgt aus der Summen- und
     Faktorregel}
     \lang{en}{Given $\big \int x^{-2} \, dx = - x^{-1} + C = - \frac{1}{x} + C$ and
      $\big \int x^{-3} \, dx = - \frac{1}{2} \frac{1}{x^2} + C$,
      using the sum and constant rules we get}
     \[ \int \frac{ax+b}{x^3}\, dx = - a \frac{1}{x} - \frac{b}{2} \frac{1}{x^2} + C . \]

     \step 3)
    \lang{de}{ Wegen $\big \int \frac{1}{x^6}\, dx = -\frac{1}{5} x^{-5} + C$ und
     $\big \int e^x\, dx = e^x + C$ folgt aus der Summen- und Faktorregel}
     \lang{en}{Given $\big \int \frac{1}{x^6}\, dx = -\frac{1}{5} x^{-5} + C$ and
      $\big \int e^x\, dx = e^x + C$, using the sum and constant rules we get}

     \[ \int \left(\frac{1}{(ax)^6} - b^2e^x \right)\, dx =
     \int \left(\frac{1}{a^6 x^6} - b^2e^x \right)\, dx = -\frac{1}{5a^6} \frac{1}{x^5} - b^2e^x + C .\]

     \end{incremental}
\end{tabs*}
\end{content}