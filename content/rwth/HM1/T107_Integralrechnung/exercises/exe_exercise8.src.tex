\documentclass{mumie.element.exercise}
%$Id$
\begin{metainfo}
  \name{
    \lang{de}{Ü08: Fläche zwischen zwei Graphen}
    \lang{en}{exercise 8}
  }
  \begin{description} 
 This work is licensed under the Creative Commons License Attribution 4.0 International (CC-BY 4.0)   
 https://creativecommons.org/licenses/by/4.0/legalcode 

    \lang{de}{Hier die Beschreibung}
    \lang{en}{}
  \end{description}
  \begin{components}
    \component{generic_image}{content/rwth/HM1/images/g_tkz_T107_Exercise08.meta.xml}{T107_Exercise08}
    \component{generic_image}{content/TU9/omb_plus/calculus_of_integration/media/images/g_img_flaeche-fg.meta.xml}{image1}
  \end{components}
  \begin{links}
  \end{links}
  \creategeneric
\end{metainfo}
\begin{content}
\title{
  \lang{de}{Ü08: Fläche zwischen zwei Graphen}
  \lang{en}{Exercise 8}
}



\begin{block}[annotation]
  Im Ticket-System: \href{http://team.mumie.net/issues/9608}{Ticket 9608}
\end{block}


\lang{de}{Bestimmen Sie die Fläche zwischen den Schnittpunkten der Graphen der
Funktionen $f(x)=-x^2+x$ und $g(x)=x^2-x-12$.}
\lang{en}{Determine the area between the graphs of the functions $f(x)=-x^2+x$
and $g(x)=x^2-x-12$, between their two points of intersection.}

\begin{enumerate}

\item[a)]
  \lang{de}{Skizzieren Sie (in einer Grafik) die Graphen der Funktionen $f(x)$ und $g(x)$
und schraffieren Sie die gesuchte Fläche. }
\lang{en}{Sketch (in one diagram) the graphs of $f(x)$ and $g(x)$ and shade the
desired area.}

\item[b)]
  \lang{de}{Bestimmen Sie die Schnittpunkte der Graphen von $f(x)$ und $g(x)$.}
  \lang{en}{Determine the intersection points of the graphs of $f(x)$ and $g(x)$.}

\item[c)]
  \lang{de}{Berechnen Sie dann die Fläche zwischen den Schnittpunkten der Graphen von $f(x)$ und $g(x)$.}
  \lang{en}{Calculate the area between the graphs of $f(x)$ and $g(x)$.}

\end{enumerate}

\begin{tabs*}[\initialtab{0}\class{exercise}]

 \tab{\lang{de}{Graphen}
 \lang{en}{Graphs}
 }
  \begin{center}
  \image{T107_Exercise08}
  \end{center}

\tab{\lang{de}{Schnittpunkte}
\lang{en}{Intersection points}
}

 \begin{incremental}[\initialsteps{1}]
  \step
    \lang{de}{Man setzt $f(x)=g(x)$ und erhält }
    \lang{en}{We set $f(x)=g(x)$ and get}
  \[ -x^2+x=x^2-x-12 . \]
  \step
    \lang{de}{Durch Äquivalenzumformung ergibt sich hieraus }
    \lang{en}{We transform this equation and obtain}
  \[ 2x^2-2x-12=0 \Leftrightarrow x^2-x-6=0 \Leftrightarrow (x+2)(x-3) = 0. \]
    \lang{de}{Die Lösungen (Schnittstellen) sind also $x_1=-2$ und $x_2=3$. }
    \lang{en}{The solutions (intersection points) are $x_1=-2$ and $x_2=3$.}
  \step
    \lang{de}{Durch Einsetzen von $x_1$ und $x_2$ in $f(x)$ oder $g(x)$ erhält man dann die Schnittpunkte
  $P_1=(-2;-6)$ und $P_2=(3;-6)$.}
  \lang{en}{We substitute $x_1$ and $x_2$ into $f(x)$ or $g(x)$ to get the
  intersection points $P_1=(-2,-6)$ and $P_2=(3,-6)$.}

 \end{incremental}

   \tab{\lang{de}{Fläche}
   \lang{en}{Area}
   }
   \begin{incremental}[\initialsteps{1}]
  \step
    \lang{de}{Zwischen den Schnittstellen $x_1=-2$ und $x_2=3$ gilt $f(x) \geq g(x)$. }
    \lang{en}{Between the intersection points $x_1=-2$ and $x_2=3$ we can see that
    $f(x) \geq g(x)$.}
  \step
    \lang{de}{Der Flächeninhalt zwischen den Schnittpunkten der Graphen beträgt dann}
    \lang{en}{We can get the area between the intersection points of the graphs by}
  \[ \int_{-2}^3 \big(f(x)-g(x)\big)\, dx = \int_{-2}^3 (-2x^2+2x+12)\, dx . \]
  \step
    \lang{de}{Eine Stammfunktion des Integranden ist $F(x)=-\frac{2}{3}x^3 + x^2 + 12x$.}
    \lang{en}{An antiderivative of the integrand is $F(x)=-\frac{2}{3}x^3 + x^2 + 12x$.}
  \step
    \lang{de}{Die Fläche beträgt also:}
    \lang{en}{So we get an area of:}
  \[ \left[  -\frac{2}{3}x^3 + x^2 + 12x \right]_{-2}^3
  = -\frac{2}{3} \cdot 27 + 9 + 36 - \left(  \frac{2}{3} \cdot 8 + 4 - 24 \right) =
  \frac{125}{3}
  \]

  \end{incremental}
\end{tabs*}

\end{content}