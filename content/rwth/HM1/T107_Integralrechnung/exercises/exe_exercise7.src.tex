\documentclass{mumie.element.exercise}
%$Id$
\begin{metainfo}
  \name{
    \lang{de}{Ü07: Lineare Substitution}
    \lang{en}{exercise 7}
  }
  \begin{description} 
 This work is licensed under the Creative Commons License Attribution 4.0 International (CC-BY 4.0)   
 https://creativecommons.org/licenses/by/4.0/legalcode 

    \lang{de}{Hier die Beschreibung}
    \lang{en}{}
  \end{description}
  \begin{components}
    \component{generic_image}{content/TU9/omb_plus/calculus_of_integration/media/images/g_img_piecewise.meta.xml}{piecewise}
  \end{components}
  \begin{links}
    \link{generic_article}{content/rwth/HM1/T107_Integralrechnung/g_art_content_26_flaechen_zwischen_graphen.meta.xml}{content_26_flaechen_zwischen_graphen}
  \end{links}
  \creategeneric
\end{metainfo}
\begin{content}
\title{
  \lang{de}{Ü07: Lineare Substitution}
  \lang{en}{Exercise 7}
}



\begin{block}[annotation]
  Im Ticket-System: \href{http://team.mumie.net/issues/9607}{Ticket 9607}
\end{block}


\lang{de}{Berechnen Sie die folgenden unbestimmten Integrale mit Hilfe einer\ref[content_26_flaechen_zwischen_graphen][ linearen Substitution.]{rule:lin-sub}}
\lang{en}{Calculate the following indefinite integrals with the help of a \ref[content_26_flaechen_zwischen_graphen][ linear substitution.]{rule:lin-sub}}

\begin{enumerate}

\item[a)] $ \ \big\int 2 e^{-3x+1}\, dx$

\item[b)] $ \ \big\int (2x+1)^5\, dx$

\item[c)] $ \ \big\int \frac{3}{4x+1}\, dx$

\item[d)] $ \ \big\int \cos\left(2\pi t + \frac{\pi}{4}\right)\, dt$

\end{enumerate}

\begin{tabs*}[\initialtab{0}\class{exercise}]

 \tab{\lang{de}{Antworten}
 \lang{en}{Answers}}

 \begin{enumerate}
\item[a)] $ \ \big\int 2 e^{-3x+1}\, dx = - \frac{2}{3} e^{-3x+1} + C$.
\item[b)] $ \ \big\int (2x+1)^5\, dx = \frac{1}{12} (2x+1)^6 + C$.
\item[c)] $ \ \big\int \frac{3}{4x+1}\, dx = \frac{3}{4} \ln |4x+1| + C$.

\item[d)] $ \ \big\int \cos\left( 2\pi t + \frac{\pi}{4}\right)\, dt = \frac{1}{2\pi} \sin \left(2\pi t + \frac{\pi}{4}\right) + C$.

\end{enumerate}
\tab{\lang{de}{L"osung zu a)}
\lang{en}{Solution for a)}
}

 \begin{incremental}[\initialsteps{1}]
  \step
    \lang{de}{Die innere Funktion ist die Gerade $g(x)=-3x+1$, die Ableitung ist $g'(x)=-3$.}
    \lang{en}{The inner function is the line $g(x)=-3x+1$, its derivative is $g'(x)=-3$.}
  \step
    \lang{de}{Die äußere Funktion ist $f(x)=e^x$ und $F(x)=e^x$ ist eine Stammfunktion.}
    \lang{en}{The outer function is $f(x)=e^x$ and $F(x)=e^x$ is an antiderivative.}
  \step
    \lang{de}{Dann folgt für das unbestimmte Integral:}
    \lang{en}{Therefore for the indefinite integral we get:}
  \[ \int 2 e^{-3x+1}\, dx = 2 \cdot \frac{1}{-3} e^{-3x+1} + C = - \frac{2}{3} e^{-3x+1} + C \]
  \end{incremental}

\tab{\lang{de}{L"osung zu b)}
\lang{en}{Solution for b)}
}
   \begin{incremental}[\initialsteps{1}]
  \step
    \lang{de}{Die innere Funktion ist die Gerade $g(x)=2x+1$, die Ableitung ist $g'(x)=2$. }
    \lang{en}{The inner function is the line $g(x)=2x+1$, its derivative is $g'(x)=2$.}
  \step
    \lang{de}{Die äußere Funktion ist $f(x)=x^5$ und $F(x)=\frac{1}{6} x^6$ ist eine Stammfunktion.}
    \lang{en}{The outer function is $f(x)=x^5$ and $F(x)=\frac{1}{6} x^6$ is an antiderivative.}
  \step
    \lang{de}{Dann folgt für das unbestimmte Integral:}
    \lang{en}{Therefore for the indefinite integral we get:}
  \[ \int (2x+1)^6 \, dx = \frac{1}{2} \cdot \frac{1}{6} (2x+1)^6 + C =  \frac{1}{12} (2x+1)^6 + C \]
  \end{incremental}

  \tab{\lang{de}{L"osung zu c)}
  \lang{en}{Solution for c)}
  }

 \begin{incremental}[\initialsteps{1}]
  \step
    \lang{de}{Die innere Funktion ist die Gerade $g(x)=4x+1$, die Ableitung ist $g'(x)=4$. }
    \lang{en}{The inner function is the line $g(x)=4x+1$, its derivative is $g'(x)=4$.}
  \step
    \lang{de}{Die äußere Funktion ist $f(x)=\frac{1}{x}$ und $F(x)=\ln |x|$ ist eine Stammfunktion.}
    \lang{en}{The outer function is $f(x)=\frac{1}{x}$ and $F(x)=\ln |x|$ is an antiderivative.}

  \step
    \lang{de}{Dann folgt für das unbestimmte Integral:}
    \lang{en}{Therefore for the indefinite integral we get:}
  \[ \int\,  \frac{3}{4x+1}\, dx = 3 \cdot \frac{1}{4} {\ln |4x+1|} + C
  = \frac{3}{4} {\ln |4x+1|} + C \]
  \end{incremental}

   \tab{\lang{de}{L"osung zu d)}
   \lang{en}{Solution for d)}
   }

 \begin{incremental}[\initialsteps{1}]
  \step
    \lang{de}{Die innere Funktion ist die Gerade $g(t)=2\pi t + \frac{\pi}{4}$, die Ableitung ist
  $g'(t)=2\pi$. }
  \lang{en}{The inner function is the line $g(t)=2\pi t + \frac{\pi}{4}$,
  its derivative is $g'(t)=2\pi$.}
  \step
    \lang{de}{Die äußere Funktion ist $f(t)=\cos(t)$ und $F(t)=\sin(t)$ ist eine Stammfunktion.}
    \lang{en}{The outer function is $f(t)=\cos(t)$ and $F(t)=\sin(t)$ is an antiderivative.}
  \step
    \lang{de}{Dann folgt für das unbestimmte Integral:}
    \lang{en}{Therefore for the indefinite integral we get:}
  \[ \int \cos\left( 2\pi t + \frac{\pi}{4}\right)\, dt = \frac{1}{2\pi} \sin \big(2\pi t + \frac{\pi}{4} \big) + C \]
  \end{incremental}

\end{tabs*}


\end{content}