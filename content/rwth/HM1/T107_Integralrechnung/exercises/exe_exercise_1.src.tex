\documentclass{mumie.element.exercise}
%$Id$
\begin{metainfo}
  \name{
    \lang{de}{Ü01: Integral als Flächeninhalt}
    \lang{en}{exercise 1}
  }
  \begin{description} 
 This work is licensed under the Creative Commons License Attribution 4.0 International (CC-BY 4.0)   
 https://creativecommons.org/licenses/by/4.0/legalcode 

    \lang{de}{Hier die Beschreibung}
    \lang{en}{}
  \end{description}
  \begin{components}
    \component{generic_image}{content/rwth/HM1/images/g_tkz_T107_Exercise01.meta.xml}{T107_Exercise01}
  \end{components}
  \begin{links}
  \end{links}
  \creategeneric
\end{metainfo}
\begin{content}
\title{
  \lang{de}{Ü01: Integral als Flächeninhalt}
	\lang{en}{Exercise 1}
}



\begin{block}[annotation]
  Im Ticket-System: \href{http://team.mumie.net/issues/9601}{Ticket 9601}
\end{block}

\lang{de}{Es seien die Funktion $f(x) = -\frac{1}{2}x+2$ und ihr Graph gegeben (siehe unten). Gesucht ist der Wert des Integrals
$\int_a^b f(x)\, dx$ für verschiedene Grenzen $a$ und $b$. Lösen Sie die Aufgabe
geometrisch!}
\lang{en}{Given the function $f(x) = -\frac{1}{2}x+2$ and its graph (see below),
find the value of the integral $\int_a^b f(x)\, dx$ for different values of
$a$ and $b$. Solve this exercise geometrically!}
\begin{center}
\image{T107_Exercise01}
\end{center}
 \begin{table}[\class{items}]
\nowrap{a) $ \ \int_0^4 f(x)\, dx$}\\
\nowrap{b) $ \ \int_1^3 f(x)\, dx$}\\
\nowrap{c) $ \ \int_2^6 f(x)\, dx$
  \lang{de}{Was passiert hier?}
  \lang{en}{What happens here?}
}
\end{table}

\begin{tabs*}[\initialtab{0}\class{exercise}]
  \tab{
  \lang{de}{Antworten}
  \lang{en}{Answers}
  }


\begin{table}[\class{items}]
a) $ \ \int_0^4 f(x) \, dx = 4$\\
b) $ \ \int_1^3 f(x) \, dx = 2$ \\
c) $ \ \int_2^6 f(x) \, dx = 0$
\end{table}

\tab{\lang{de}{L"osung zu a)}
\lang{en}{Solution for a)}
}

 \begin{incremental}[\initialsteps{1}]

    \step
    \lang{de}{Die Gerade $f(x) = -\frac{1}{2}x+2$ schneidet die $x$-Achse an der Stelle $x=4$. Da sie die Steigung $-\frac{1}{2}$ hat, fällt sie
    und verläuft somit im Intervall $[0;4]$ oberhalb der $x$-Achse.}
    \lang{en}{The line $f(x) = -\frac{1}{2}x+2$ intersects the $x$-axis at the point
    $x=4$. Since the slope is $-\frac{1}{2}$, the function is decreasing and thus
    is above the $x$-axis over the interval $[0, 4]$.}
    \step
    \lang{de}{Daher liegt die Fläche, welche von dem Graphen von $f$ und der $x$-Achse zwischen $x=0$ und $x=4$
    eingeschlossen wird, komplett oberhalb der $x$-Achse.
     Somit ist das gesuchte Integral positiv und kann als Flächeninhalt eines Dreiecks berechnet werden.}
     \lang{en}{Therefore the area enclosed by the graph of $f$ and the $x$-axis
     between $x=0$ and $x=4$ lies completely above the $x$-axis. Thus the integral
     is positive and calculated as the area of a triangle.}
    \step
    \lang{de}{Die Länge der Dreieckseite auf der $x$-Achse beträgt $4-0=4$. Die Höhe des Dreiecks auf der $y$-Achse beträgt $2-0=2$.
    Als Flächeninhalt ergibt sich daher $\frac{4 \cdot 2}{2}=4$ und somit $\int_0^4 f(x) \, dx = 4$.}
    \lang{en}{The length of the base of the triangle, which lies along the $x$-axis, is $4-0=4$.
    The height of the triangle along the $y$-axis is $2-0=2$. Thus the area of the
    triangle is $\frac{4 \cdot 2}{2}=4$, and so $\int_0^4 f(x) \, dx = 4$.}
   \step
    \lang{de}{Der orientierte Flächeninhalt lässt sich auch anhand der Grafik ablesen. Jedes Kästchen hat
    die Fläche $1$. }
    \lang{en}{The signed area can also be read off the graphic. Every box has an
    area of $1$.}
 \end{incremental}

\tab{\lang{de}{L"osung zu b)}
\lang{en}{Solution for b)}
}

\begin{incremental}[\initialsteps{1}]
    \step
    \lang{de}{Die Gerade $f(x) = -\frac{1}{2}x+2$ schneidet die $x$-Achse an der Stelle $x=4$. Da sie die Steigung $-\frac{1}{2}$ hat, fällt sie
    und verläuft somit im Intervall $[1;3]$ oberhalb der $x$-Achse.}
    \lang{en}{The line $f(x) = -\frac{1}{2}x+2$ intersects the $x$-axis at the point
    $x=4$. Since its slope is $-\frac{1}{2}$, the function is decreasing and thus
    is above the $x$-axis over the interval $[1, 3]$.}
    \step
    \lang{de}{Daher liegt die gesuchte Fläche komplett oberhalb der $x$-Achse.
     Das Integral ist positiv und kann als Flächeninhalt eines \emph{Trapezes}
     mit den Eckpunkten $(1;0),(3;0),(3;\frac{1}{2}),(1;\frac{3}{2})$
     oder als
     Summe einer Rechtecks- und einer Dreiecksfläche berechnet werden.}
     \lang{en}{Therefore the required area lies completely above the $x$-axis.
     The integral is positive and can be calculated either as the area of a
     \emph{trapezium} with vertices $(1,0),(3,0),(3,\frac{1}{2}),(1,\frac{3}{2})$,
     or as the sum of the areas of a rectangle and a triangle.}
    \step
    \lang{de}{Der Flächeninhalt des Rechtecks ist $(3-1) \cdot \frac{1}{2} = 1$, der Flächeninhalt
    des Dreiecks beträgt  $((3-1)\cdot 1)\cdot \frac{1}{2} = 1$,
    was $\int_1^3 f(x) \, dx = 2$ liefert.}
    \lang{en}{The area of the rectangle is $(3-1) \cdot \frac{1}{2} = 1$, the area
    of the triangle is $((3-1)\cdot 1)\cdot \frac{1}{2} = 1$, thus $\int_1^3 f(x) \, dx = 2$.}
    \step
    \lang{de}{Der orientierte Flächeninhalt lässt sich auch anhand der Grafik ablesen. Jedes Kästchen hat
    die Fläche $1$. }
    \lang{en}{The signed area can also be read off the graphic. Every box has an
    area of $1$.}
\end{incremental}

\tab{\lang{de}{L"osung zu c)}
\lang{en}{Solution for c)}
}

\begin{incremental}[\initialsteps{1}]
    \step
    \lang{de}{Die Gerade $f(x) = -\frac{1}{2}x+2$ schneidet die $x$-Achse an der Stelle $x=4$.
    Damit liegt sie bezüglich des Integrationsintervalls $[2;6]$
    teilweise oberhalb und teilweise unterhalb der $x$-Achse.}
    \lang{en}{The line $f(x) = -\frac{1}{2}x+2$ intersects the $x$-axis at the point
    $x=4$. Thus it lies both above and below the $x$-axis over the integration
    interval $[2, 6]$.}
    \step
    \lang{de}{Bei der Berechnung des Integrals muss nun aufgepasst werden! Das Integral besteht aus zwei orientierten Dreiecksflächen.}
    \lang{en}{So we have to pay attention when calculating the integral! The
    integral consists of two signed triangular areas.}
    \step
    \lang{de}{Das erste Dreieck (oberhalb der $x$-Achse) hat eine Grundseite von $x=2$ bis $x=4$ und eine Höhe von $|f(2)|=|1|=1$. Der Flächeninhalt
    beträgt folglich $\frac{(4-2) \cdot 1}{2}=1$. Diese Fläche trägt positiv zum Integral bei.}
    \lang{en}{The first triangle (above the $x$-axis) has its base from $x=2$ to $x=4$
    and a height of $|f(2)|=|1|=1$. Therefore its area is $\frac{(4-2) \cdot 1}{2}=1$.
    This area contributes positively to the integral.}
    \step
    \lang{de}{Das zweite Dreieck (unterhalb der $x$-Achse) hat eine Grundseite von $x=4$ bis $x=6$ und eine Höhe von $|f(6)|=|-1|=1$. Der Flächeninhalt
    beträgt folglich $\frac{(6-4) \cdot 1}{2}=1$. Diese Fläche trägt negativ zum Integral bei.}
    \lang{en}{The second triangle (below the $x$-axis) has its base from $x=4$ to $x=6$
    and a height of $|f(6)|=|-1|=1$. Therefore its area is $\frac{(6-4) \cdot 1}{2}=1$.
    This area contributes negatively to the integral.}
    \step
    \lang{de}{Das Integral berechnet sich nun aus der Summe der positiven und der negativen Dreiecksfläche: $1-1=0$.
    Die Flächeninhalte heben sich im Integral wegen ihrer unterschiedlichen Orientierung auf.
     Daher ist $\int_2^6 f(x) \, dx = 0$.}
     \lang{en}{The integral is calculated as the sum of the positive and negative
     triangular areas: $1-1=0$. The areas cancel in the integral because of their
     differing orientations. Thus $\int_2^6 f(x) \, dx = 0$.}
     \step
    \lang{de}{Anhand der Grafik erkennt man, dass die Flächen oberhalb und unterhalb der $x$-Achse
     gleich groß sind.}
     \lang{en}{From the graphic we can see that the areas above and below the
     $x$-axis are the same size.}

     \end{incremental}

 \end{tabs*}


\end{content}