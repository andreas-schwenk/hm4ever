\documentclass{mumie.element.exercise}
%$Id$
\begin{metainfo}
  \name{
    \lang{de}{Ü09: Fläche zwischen zwei Graphen}
    \lang{en}{Exercise 9}
  }
  \begin{description} 
 This work is licensed under the Creative Commons License Attribution 4.0 International (CC-BY 4.0)   
 https://creativecommons.org/licenses/by/4.0/legalcode 

    \lang{de}{Ü09: Fläche zwischen zwei Graphen}
    \lang{en}{Exercise 9}
  \end{description}
  \begin{components}
\component{generic_image}{content/rwth/HM1/images/g_tkz_T107_Exercise09_B.meta.xml}{T107_Exercise09_B}
\component{generic_image}{content/rwth/HM1/images/g_tkz_T107_Exercise09_A.meta.xml}{T107_Exercise09_A}
\end{components}
  \begin{links}
  \end{links}
  \creategeneric
\end{metainfo}

\begin{content}

\begin{block}[annotation]
	Im Ticket-System: \href{https://team.mumie.net/issues/18756}{Ticket 18756}
\end{block}
\title{
  \lang{de}{Ü09: Fläche zwischen zwei Graphen}
  \lang{en}{Exercise 9}
}


\lang{de}{Gegeben sei die Funktion $f$ durch $f(x)=2x-0,5x^2$.}
\lang{en}{Let the funtion $f$ be given by $f(x)=2x-0.5x^2$.}

\begin{enumerate}

\item[a)]
  \lang{de}{Skizzieren Sie den Graphen von $f$ im Bereich $0\leq x \leq 6$.\\
  
  Die $x$-Achse, der Graph von $f$ und die Gerade $x=6$ begrenzen eine Fläche.
  \\ Berechnen Sie den Inhalt dieser Fläche. }
\lang{en}{Sketch the graph of $f$ for $0\leq x\leq 6$. The $x$-axis, the graph of $f$ and the line 
    $x=6$ enclose an area.
    \\ Calculate this area.}

\item[b)]
  \lang{de}{Bestimmen Sie die Gleichung der Tangente $t$ an den Graphen von $f$ im Punkt $(4;0)$.
  Der Graph von $f$, die Tangente $t$ und die $y$-Achse begrenzen eine Fläche.
  \\ Berechnen Sie den Inhalt dieser Fläche.}
  \lang{en}{Determine the equation of the tangent $t$ of $f$ in $(4,0)$.
  The graph of $f$, this tangent and the $y$-axis enclose an area.
  \\ Calculate this area.}

\end{enumerate}

\begin{tabs*}[\initialtab{0}\class{exercise}]

 \tab{  \lang{de}{Antworten}
        \lang{en}{Answers}
 }
 \begin{enumerate}
 
  \item[a)] \lang{de}{Der Inhalt der Fläche ist  $\frac{16}{3}$. Skizze siehe "Lösung zu a)".}
            \lang{en}{The area is  $\frac{16}{3}$. For sketch see "Solution for a)".}
  \item[b)] \lang{de}{Die Gleichung der Tangente ist $t(x)=-2x+8$. Der Inhalt der Fläche ist $\frac{32}{3}$.}
            \lang{en}{The tangent is given by $t(x)=-2x+8$. The area is  $\frac{32}{3}$.}
 \end{enumerate}

\tab{   
    \lang{de}{L"osung zu a)}
    \lang{en}{Solution for a)}
    }

 \begin{incremental}[\initialsteps{1}]
  \step
\begin{center}
\image{T107_Exercise09_A}
\end{center}
    \lang{de}{Der Flächeninhalt ist gegeben durch  }
    \lang{en}{The area is given by}
    \[\left\vert \ \int_4^6 f(x)\ \text{d}x \ \right\vert.\]
  \step
    \lang{de}{Wir berechnen daher zuerst }
    \lang{en}{First, we calculate}
  \lang{de}{\[ \int_4^6 f(x)\ \text{d}x = \int_4^6 2x-0,5x^2\ \text{d}x. \]}
  \lang{en}{\[ \int_4^6 f(x)\ \text{d}x = \int_4^6 2x-0.5x^2\ \text{d}x. \]}
    \lang{de}{Eine Stammfunktion des Integranden ist}
    \lang{en}{An antiderivative of the integrand is}
    \[F(x)=x^2-\frac{1}{6}x^3.\]
  \step
    \lang{de}{Damit ist der Wert des Integrals gegeben durch}
    \lang{en}{Therefore the value of the integral is}
    \[\left[x^2-\frac{1}{6}x^3\right]_4^6=6^2-\frac{6^3}{6}-\left(4^2-\frac{4^3}{6}\right) = -\left(16-\frac{4^3}{6}\right)=-\frac{16}{3}.\]
  \step
    \lang{de}{Da Flächeninhalte positiv sind, besitzt die gesuchte Fläche einen Flächeninhalt von}
    \lang{en}{Since areas are positive, the value asked for is}
    \[\left\vert-\frac{16}{3}\right\vert=\frac{16}{3}.\]
 \end{incremental}

   \tab{
   \lang{de}{L"osung zu b)}
   \lang{en}{Solution for b)}
   }
   
   \begin{incremental}[\initialsteps{1}]
  \step
\begin{center}
\image{T107_Exercise09_B}
\end{center}
    \lang{de}{Die Gleichung der Tangente von $f$ im Punkt $(4;0)$ ist}
    \lang{en}{The equation of the tangent of $f$ in the point $(4,0)$ is}
    \[t(x)=f(4) + f'(4)(x-4) = 0 + (2-4)(x-4)=-2x+8.\]
  \step
    \lang{de}{Der gesuchte Flächeninhalt ist gegeben durch}
    \lang{en}{The area asked for is given by}
  \lang{de}{\[\int_0^4(t(x)-f(x))\ \text{d}x =  \int_0^4 (0,5x^2-4x+8)\ \text{d}x. \]}
  \lang{en}{\[\int_0^4(t(x)-f(x))\ \text{d}x =  \int_0^4 (0.5x^2-4x+8)\ \text{d}x. \]}
  \step
    \lang{de}{Eine Stammfunktion des Integranden ist}
    \lang{en}{An antiderivative of the integrand is}
    \[F(x)= \frac{x^3}{6}-2x^2+8x.\]
  \step
    \lang{de}{Damit ist der Flächeninhalt}
    \lang{en}{Therefore, the area is}
  \[  \left[\frac{x^3}{6}-2x^2+8x\right]_0^4= \frac{4^3}{6}-2\cdot 4^2+8\cdot 4 - 0 = \frac{32}{3} .\]

  \end{incremental}
\end{tabs*}





\end{content}

