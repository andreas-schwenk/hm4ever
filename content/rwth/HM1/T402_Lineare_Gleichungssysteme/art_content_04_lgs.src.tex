%$Id:  $
\documentclass{mumie.article}
%$Id$
\begin{metainfo}
  \name{
    \lang{de}{Lineare Gleichungssysteme}
    \lang{en}{System of linear equations}
  }
  \begin{description} 
 This work is licensed under the Creative Commons License Attribution 4.0 International (CC-BY 4.0)   
 https://creativecommons.org/licenses/by/4.0/legalcode 

    \lang{de}{Beschreibung}
    \lang{en}{}
  \end{description}
  \begin{components}
  \component{generic_image}{content/rwth/HM1/images/g_img_00_video_button_schwarz-blau.meta.xml}{00_video_button_schwarz-blau}
  \end{components}
  \begin{links}
    \link{generic_article}{content/rwth/HM1/T112neu_Lineare_Gleichungssysteme/g_art_content_40_lineare_gleichungssysteme.meta.xml}{{def:lgs}}
    \link{generic_article}{content/rwth/HM1/T401_Matrizenrechnung/g_art_content_01_matrizen.meta.xml}{matrizen}
    \link{generic_article}{content/rwth/HM1/T108_Vektorrechnung/g_art_content_29_linearkombination.meta.xml}{lin-komb}
    \link{generic_article}{content/rwth/HM1/T402_Lineare_Gleichungssysteme/g_art_content_05_gaussverfahren.meta.xml}{gauss-verfahren}
    \link{generic_article}{content/rwth/HM1/T403a_Vektorraum/g_art_content_10a_vektorraum.meta.xml}{def:Unter_VR}
  \end{links}
  \creategeneric
\end{metainfo}
\begin{content}
\usepackage{mumie.ombplus}
\ombchapter{2}
\ombarticle{1}
\usepackage{mumie.genericvisualization}

\begin{visualizationwrapper}

\title{\lang{de}{Lineare Gleichungssysteme über allgemeinen Körpern} \lang{en}{Systems of linear equations over general fields}}

\begin{block}[annotation]
 
  
\end{block}
\begin{block}[annotation]
  Im Ticket-System: \href{http://team.mumie.net/issues/11195}{Ticket 11195}\\
\end{block}

\begin{block}[info-box]
\tableofcontents
\end{block}

\section{\lang{de}{Lineare Gleichungssysteme}\lang{en}{Systems of Linear Equations}}\label{lgs}

\lang{de}{
In \link{def:lgs}{Teil 1} haben wir bereits lineare Gleichungssysteme (LGS) über dem Körper der reellen Zahlen
kennengelernt. 
Analog dazu lassen sich LGS mit Koeffizienten eines beliebigen 
Körpers $\mathbb{K}$ (beispielsweise den komplexen Zahlen) 
definieren und entsprechende Lösungen
in dem jeweiligen Körper finden. 
Auch hier lässt sich die Lösungsmenge systematisch mit dem \link{gauss-verfahren}{Gauß-Verfahren} bestimmen.\\

Die Verallgemeinerung des Körpers der reellen Zahlen zu einem beliebigen Körper wird hier vor allem in den Beispielen deutlich. 
Außerdem werden lineare Gleichungssysteme in diesem Teil stets in Matrixschreibweise betrachtet.}

\lang{en}{
In \link{def:lgs}{Part 1} we already got to know to systems of linear equations (or linear systems) over the field of real
numbers. By analogy, we can define linear systems with coefficients in any field $\K$ (e.g. the complex numbers) and find 
solutions for the system.
The solution set can here be determined with the \link{gauss-verfahren}{Gaussian elimination} too.
The generalisation from real numbers to any field will be performed by discussing various examples.
Furthermore, systems of linear equations are considered in matrix notation in this part.}


\begin{definition}[\lang{de}{Koeffizientenmatrix} \lang{en}{Coefficient matrix}]\label{def:koeffizientenmatrix}
\lang{de}{
Ein \notion{lineares Gleichungssystem} (kurz \notion{LGS}) über einem Körper $\mathbb{K}$ mit $m$ Gleichungen und $n$ Unbestimmten $x_1,\ldots,x_n$ ist ein Gleichungssystem, welches durch Umformungen
der einzelnen Gleichungen stets in die Form
\begin{equation*} 
\begin{mtable}[\cellaligns{ccccccccc}]
 a_{11} x_1 & + & a_{12} x_2 & + & \cdots & + & a_{1n} x_n & = & b_1 \\
 a_{21} x_1 & + & a_{22} x_2 & + & \cdots & + & a_{2n} x_n & = & b_2 \\
 \vdots     &   & \vdots     &   &        &   & \vdots     &   & \vdots \\
 a_{m1} x_1 & + & a_{m2} x_2 & + & \cdots & + & a_{mn} x_n & = & b_m
\end{mtable} 
\end{equation*}
mit Elementen $a_{ij}\in \mathbb{K}$ für $1 \leq i \leq m$ und $1 \leq j \leq n$, sowie $b_1, \ldots, b_m\in \mathbb{K}$ gebracht werden kann.\\
Für ein solches lineares Gleichungssystem werden die zum linearen Gleichungssystem geh"origen Matrizen
\begin{equation*}
A=\underbrace{\left( \begin{smallmatrix}
a_{11} & a_{12} & \cdots & a_{1n} \\
a_{21} & a_{22} & \cdots & a_{2n} \\
\vdots & \vdots & \ddots & \vdots \\
a_{m1} & a_{m2} & \cdots & a_{mn}
\end{smallmatrix} \right)}_{ \in M(m,n;\mathbb{K})}
 \quad \text{ und } \quad 
\end{equation*}
\begin{equation*}
(A \ \mid \ b) = \underbrace{\left(  \begin{smallmatrix}
                                      a_{11} & a_{12} & \cdots & a_{1n} & | & b_1\\
                                      a_{21} & a_{22} & \cdots & a_{2n} & | &b_2\\
                                      \vdots & \vdots & \ddots & \vdots & \mid &\vdots\\
                                      a_{m1} & a_{m2} & \cdots & a_{mn} &| & b_m
                                      \end{smallmatrix} \right) }_{
\in M(m,n+1;\mathbb{K})}
\end{equation*}
\notion{Koeffizientenmatrix} und \notion{erweiterte Koeffizientenmatrix} genannt.\\
Die Trennstriche bei der erweiterten Koeffizientenmatrix
dienen nur der besseren Übersicht und können auch weggelassen werden.}\\

\lang{en}{
Any \notion{system of linear equations} over a field $\K$  with $m$ equations and in $n$ unknowns $x_1,\ldots,x_n$ can
be written in the following form, by manipulating each individual equation:
\begin{equation*} 
\begin{mtable}[\cellaligns{ccccccccc}]
 a_{11} x_1 & + & a_{12} x_2 & + & \cdots & + & a_{1n} x_n & = & b_1 \\
 a_{21} x_1 & + & a_{22} x_2 & + & \cdots & + & a_{2n} x_n & = & b_2 \\
 \vdots     &   & \vdots     &   &        &   & \vdots     &   & \vdots \\
 a_{m1} x_1 & + & a_{m2} x_2 & + & \cdots & + & a_{mn} x_n & = & b_m
\end{mtable} 
\end{equation*}
The entries $a_{ij}$ and $b_i$ for $1 \leq i \leq m$ and $1 \leq j \leq n$ have their values in $\K$.\\
For such a linear system, there are the matrices
\begin{equation*}
A=\underbrace{\left( \begin{smallmatrix}
a_{11} & a_{12} & \cdots & a_{1n} \\
a_{21} & a_{22} & \cdots & a_{2n} \\
\vdots & \vdots & \ddots & \vdots \\
a_{m1} & a_{m2} & \cdots & a_{mn}
\end{smallmatrix} \right)}_{ \in M(m,n;\mathbb{K})}
 \quad \text{ and } \quad 
\end{equation*}
\begin{equation*}
(A \ \mid \ b) = \underbrace{\left(  \begin{smallmatrix}
                                      a_{11} & a_{12} & \cdots & a_{1n} & | & b_1\\
                                      a_{21} & a_{22} & \cdots & a_{2n} & | &b_2\\
                                      \vdots & \vdots & \ddots & \vdots & \mid &\vdots\\
                                      a_{m1} & a_{m2} & \cdots & a_{mn} &| & b_m
                                      \end{smallmatrix} \right). }_{
\in M(m,n+1;\mathbb{K})}
\end{equation*}
They are called \notion{coefficient matrix} and \notion{augmented matrix}.\\
The vertical dividing line in the augmented matrix is useful notation, but may also be omitted.}\\


\end{definition}

\lang{de}{
Mit der \ref[matrizen][Matrix-Vektor-Multiplikation]{sec:produkt} kann das obige lineare Gleichungssystem 
dann kompakt geschrieben werden als
\[ A\cdot x = b \qquad \text{mit}\quad b=\left(  \begin{smallmatrix}  b_1 \\    \vdots\\ b_m  \end{smallmatrix} \right) \quad \text{und}\quad x=\left(  \begin{smallmatrix}  x_1 \\    \vdots\\ x_n  \end{smallmatrix} \right). \]
Diese Schreibweise werden wir im Folgenden stets benutzen.}

\lang{en}{
Using \ref[matrizen][matrix-vector multiplication]{sec:produkt} the linear systems above can be written as
\[ A\cdot x = b \qquad \text{with}\quad b=\left(  \begin{smallmatrix}  b_1 \\    \vdots\\ b_m  \end{smallmatrix} \right) \quad \text{and}\quad x=\left(  \begin{smallmatrix}  x_1 \\    \vdots\\ x_n  \end{smallmatrix} \right). \]
By now, we will always use this notation.}

\begin{example}
\begin{tabs*}
\tab{\lang{de}{1. Beispiel} \lang{en}{1. Example}} 
\lang{de}{
Das lineare Gleichungssystem 
\[ \begin{mtable}[\cellaligns{crcrcr}]
(I)&\qquad 2 \cdot  x & - & 4 \cdot  y & = & 10 \\
(II)&-3 \cdot  x & + & 6\cdot y & =  & -15
\end{mtable} \]
mit Koeffizienten in $\Q$ (oder in $\R$) hat als Koeffizientenmatrix 
$A=\begin{pmatrix} 2 & -4 \\ -3 & 6\end{pmatrix}$ und als rechte Seite $b=\begin{pmatrix} 10\\ -15\end{pmatrix}$ 
sowie die erweiterte Koeffizientenmatrix
\[ (A \mid b )= \begin{pmatrix} 2 & -4& | & 10 \\ -3 & 6& | & -15\end{pmatrix}  .\]
In Matrixschreibweise lautet das LGS daher
\[   \begin{pmatrix} 2 & -4 \\ -3 & 6\end{pmatrix}\cdot \begin{pmatrix} x\\ y\end{pmatrix}= \begin{pmatrix} 10\\ -15\end{pmatrix}.\]}
\lang{en}{
For the linear system
\[ \begin{mtable}[\cellaligns{crcrcr}]
(I)&\qquad 2 \cdot  x & - & 4 \cdot  y & = & 10 \\
(II)&-3 \cdot  x & + & 6\cdot y & =  & -15
\end{mtable} \]
with its coefficients in $\Q$ (or in $\R$), we have the coefficient matrix 
$A=\begin{pmatrix} 2 & -4 \\ -3 & 6\end{pmatrix}$ and the right side $b=\begin{pmatrix} 10\\ -15\end{pmatrix}$. 
So, the augmented matrix is given by
\[ (A \mid b )= \begin{pmatrix} 2 & -4& | & 10 \\ -3 & 6& | & -15\end{pmatrix}  .\]
In matrix notation the system is 
\[   \begin{pmatrix} 2 & -4 \\ -3 & 6\end{pmatrix}\cdot \begin{pmatrix} x\\ y\end{pmatrix}= \begin{pmatrix} 10\\ -15\end{pmatrix}.\]}

\tab{\lang{de}{2. Beispiel} \lang{en}{2. Example}} 
\lang{de}{
Das lineare Gleichungssystem 
\[ \begin{mtable}[\cellaligns{crcrcr}]
(I)&\qquad i \cdot  x & - & 4 \cdot  y & = & 4+2i \\
(II)&-2 \cdot  x & + & (8-8i)\cdot y & =  & -4
\end{mtable} \]
mit Koeffizienten in $\C$ hat als Koeffizientenmatrix 
$A=\begin{pmatrix} i & -4 \\ -2 & 8-8i\end{pmatrix}$ und als rechte Seite $b=\begin{pmatrix} 4+2i\\ -4\end{pmatrix}$ 
sowie die erweiterte Koeffizientenmatrix
\[ (A \mid b )= \begin{pmatrix} i & -4& | & 4+2i \\ -2 & 8-8i& | & -4\end{pmatrix}  .\]
In Matrixschreibweise lautet das LGS daher
\[   \begin{pmatrix} i & -4 \\ -2& 8-8i\end{pmatrix}\cdot \begin{pmatrix} x\\ y\end{pmatrix}= \begin{pmatrix} 4+2i\\ -4\end{pmatrix}.\]}
\lang{en}{
For the linear system 
\[ \begin{mtable}[\cellaligns{crcrcr}]
(I)&\qquad i \cdot  x & - & 4 \cdot  y & = & 4+2i \\
(II)&-2 \cdot  x & + & (8-8i)\cdot y & =  & -4
\end{mtable} \]
with its coefficients in $\C$, we have 
$A=\begin{pmatrix} i & -4 \\ -2 & 8-8i\end{pmatrix}$ for the coefficient matrix and $b=\begin{pmatrix} 4+2i\\ -4\end{pmatrix}$ for the right side.
So, the augmented matrix is given by
\[ (A \mid b )= \begin{pmatrix} i & -4& | & 4+2i \\ -2 & 8-8i& | & -4\end{pmatrix}  .\]
In matrix notation the system is 
\[   \begin{pmatrix} i & -4 \\ -2& 8-8i\end{pmatrix}\cdot \begin{pmatrix} x\\ y\end{pmatrix}= \begin{pmatrix} 4+2i\\ -4\end{pmatrix}.\]}

\tab{\lang{de}{3. Beispiel} \lang{en}{3. Example}}
\lang{de}{
Das lineare Gleichungssystem 
\[ \begin{mtable}[\cellaligns{ccrcrcrcr}]
(I)&&x_{1}&+&2x_{2}&+&x_{3}&=&4\\
(II)&&x_{1}&-&x_{2}&+&\frac{3}{2}x_{3}&=&-7\\
(III)&\qquad-&4x_{1}&+&2x_{2}&&&=&-2
\end{mtable} \]
hat als Koeffizientenmatrix
$ A=\begin{pmatrix} 1 & 2 & 1 \\ 1 & -1 & \frac{3}{2} \\ -4 & 2 & 0\end{pmatrix} $
und als rechte Seite $b=\begin{pmatrix} 4 \\ -7 \\ -2 \end{pmatrix} $ sowie die erweiterte Koeffizientenmatrix
\[ (A \mid b )= \begin{pmatrix} 1 & 2 & 1 & | & 4 \\ 1 & -1 & \frac{3}{2}& | & -7\\ -4 & 2 & 0& | &-2\end{pmatrix}.\]
In Matrixschreibweise lautet das LGS daher
\[ \begin{pmatrix} 1 & 2 & 1 \\ 1 & -1 & \frac{3}{2} \\ -4 & 2 & 0\end{pmatrix}\cdot \begin{pmatrix}x_1 \\x_2\\ x_3\end{pmatrix}
= \begin{pmatrix} 4 \\ -7 \\ -2 \end{pmatrix}. \]}
\lang{en}{
For the linear system 
\[ \begin{mtable}[\cellaligns{ccrcrcrcr}]
(I)&&x_{1}&+&2x_{2}&+&x_{3}&=&4\\
(II)&&x_{1}&-&x_{2}&+&\frac{3}{2}x_{3}&=&-7\\
(III)&\qquad-&4x_{1}&+&2x_{2}&&&=&-2
\end{mtable} \]
the coefficient matrix is given by
$ A=\begin{pmatrix} 1 & 2 & 1 \\ 1 & -1 & \frac{3}{2} \\ -4 & 2 & 0\end{pmatrix} $. With the right side given by
$b=\begin{pmatrix} 4 \\ -7 \\ -2 \end{pmatrix} $, the augmented matrix is then given by
\[ (A \mid b )= \begin{pmatrix} 1 & 2 & 1 & | & 4 \\ 1 & -1 & \frac{3}{2}& | & -7\\ -4 & 2 & 0& | &-2\end{pmatrix}.\]
In matrix notation the system is
\[ \begin{pmatrix} 1 & 2 & 1 \\ 1 & -1 & \frac{3}{2} \\ -4 & 2 & 0\end{pmatrix}\cdot \begin{pmatrix}x_1 \\x_2\\ x_3\end{pmatrix}
= \begin{pmatrix} 4 \\ -7 \\ -2 \end{pmatrix}. \]}
\end{tabs*}
\end{example}

\lang{de}{
Allgemein gibt man die Lösungsmenge eines LGS wie folgt an, wobei hier zusätzlich zu beachten ist, über welchem Körper $\K$ das LGS betrachtet wird 
und in welchem Vektorraum $\mathbb{K}^n$ damit die entsprechenden Lösungsvektoren liegen.}
\lang{en}{
In general, the solution set of a linear system is given as follows. Here, it must also be taken into account over which field $\K$ the system is considered 
and to which vector space $\K^n$ the solution vectors belong.}

\begin{definition}[\lang{de}{Lösungsmenge} \lang{en}{Solution set}]\label{def:loesungsmenge}
\lang{de}{Eine \notion{L\"osung} eines solchen linearen Gleichungssystems über $\mathbb{K}$ 
mit $n$ Unbekannten ist ein $n$-Tupel $(x_1; x_2; \ldots; x_n)$ mit $n$ Elementen $x_1, x_2, \ldots, x_n\in \mathbb{K}$, das alle Gleichungen des Systems erf\"ullt.}
\lang{en}{A \notion{solution} to a linear system of equations over $\mathbb{K}$ in $n$ knowns is called an $n$-tuple $(x_1, x_2, \ldots, x_n)$ comprised of $n$ elements $x_1$, $x_2$, $\ldots$, $x_n\in \mathbb{K}$ which fulfill all the equations in the system.}
\lang{de}{
Oft schreiben wir diese Lösungen als  \emph{Spaltenvektoren}
\[ \left(\begin{smallmatrix} x_1\\ \vdots \\ x_n\end{smallmatrix}\right).\]

Die \notion{Lösungsmenge} eines linearen Gleichungssystems $Ax=b$ ist dann die Menge aller Lösungen, also
\[ \mathbb{L}=\Big\lbrace \left(\begin{smallmatrix} x_1\\ \vdots  \\ x_n\end{smallmatrix}\right) \in \K^{n}\ \Big\vert \ 
\left(\begin{smallmatrix} x_1\\ \vdots \\ x_n\end{smallmatrix}\right) \text{ erf\"ullt } Ax=b \Big\rbrace .
\]}

\lang{en}{
The solutions are often written as  \emph{column vectors}
\[ \left(\begin{smallmatrix} x_1\\ \vdots \\ x_n\end{smallmatrix}\right).\]

The \notion{solution set} of a linear system $Ax=b$ is the set of all solution, so
\[ \mathbb{L}=\Big\lbrace \left(\begin{smallmatrix} x_1\\ \vdots  \\ x_n\end{smallmatrix}\right) \in \K^{n}\ \Big\vert \ 
\left(\begin{smallmatrix} x_1\\ \vdots \\ x_n\end{smallmatrix}\right) \text{ fulfills } Ax=b \Big\rbrace .
\]}
\end{definition}

%\begin{remark}
%Die Lösungsmenge des linearen Gleichungssystems hängt also auch davon ab, über welchem Körper $\mathbb{K}$
%man das LGS betrachtet, da man nach Lösungen im Vektorraum $\mathbb{K}^n$ sucht.\\
%Beim \link{gauss-verfahren}{Gauß-Verfahren} werden wir aber sehen, dass sich alle Rechnungen im kleinsten Körper abspielen,
%in dem alle Koeffizienten liegen. Bei der Beschreibung der Lösungsmenge in einer parametrisierten Form, 
%wie man sie aus
%dem Gauß-Verfahren erhält, unterscheidet sich diese dann nur darin, welchen Körper man als Werte für die Parameter zulässt 
%(vgl. auch die \lref{ex:lsgsmenge-homogen}{Beispiele}).
%\end{remark}

%Um im nächsten Abschnitt explizite Lösungen zu berechnen zu können, 
%unterscheiden wir zunächst zwischen homogenen und inhomogenen 
%linearen Gleichungssystemen.
\lang{de}{
Um eine Struktur der Lösungsmenge anzugeben, 
unterscheiden wir zwischen homogenen und inhomogenen linearen Gleichungssystemen:\\ 

\begin{definition}[homogenes/inhomogenes LGS] \label{def:homogen-inhomogen}
Ein lineares Gleichungssystem $Ax=b$ heißt \notion{homogenes LGS}, wenn $b$ gleich dem Nullvektor ist, also
$b_1= \cdots = b_m =0$ gilt. Anderenfalls spricht man von einem
\notion{inhomogenen LGS}.

Ist $Ax=b$ ein inhomogenes LGS, so nennt man das lineare Gleichungssystem $Ax=0$ das \notion{zugehörige
homogene LGS}.
\end{definition}}

\lang{en}{
To structure the solution set, we distinguish the systems in homogeneous and nonhomogeneous linear systems:

\begin{definition}[(Non-)Homogeneous system of linear equations]\label{def:homogen-inhomogen}
A linear system $Ax=b$ is called \notion{homogeneous system}, when $b$ is the zero vector, so 
$b_1= \cdots = b_m =0$ holds. Otherwise, it is called \notion{nonhomogeneous linear system.}

If $Ax=b$ is a nonhomogeneous linear system, we call the system $Ax=0$ the \notion{corresponding homogeneous linear system}.
\end{definition}}

\begin{example}
\begin{tabs*}
\tab{\lang{de}{1. Beispiel} \lang{en}{1. Example}}
\lang{de}{
Das obige LGS
\[ \begin{pmatrix} 2 & -4 \\ -3 & 6\end{pmatrix}\cdot \begin{pmatrix} x\\ y\end{pmatrix}= \begin{pmatrix} 10\\ -15\end{pmatrix}\]
ist inhomogen, da die rechte Seite $b=\begin{pmatrix} 10\\ -15\end{pmatrix}$ nicht der Nullvektor ist. Das zugehörige homogene LGS
lautet
\[ \begin{pmatrix} 2 & -4 \\ -3 & 6\end{pmatrix}\cdot \begin{pmatrix} x\\ y\end{pmatrix}= \begin{pmatrix} 0\\ 0\end{pmatrix}.\]}
\lang{en}{
The above linear system 
\[ \begin{pmatrix} 2 & -4 \\ -3 & 6\end{pmatrix}\cdot \begin{pmatrix} x\\ y\end{pmatrix}= \begin{pmatrix} 10\\ -15\end{pmatrix}\]
is nonhomogeneous, because the right side $b=\begin{pmatrix} 10\\ -15\end{pmatrix}$ is not the zero vector. The corresponding homogeneous system is
\[ \begin{pmatrix} 2 & -4 \\ -3 & 6\end{pmatrix}\cdot \begin{pmatrix} x\\ y\end{pmatrix}= \begin{pmatrix} 0\\ 0\end{pmatrix}.\]}

\tab{\lang{de}{2. Beispiel} \lang{en}{2. Example}}
\lang{de}{
Das obige LGS
\[   \begin{pmatrix} i & -4 \\ -2& 8-8i\end{pmatrix}\cdot \begin{pmatrix} x\\ y\end{pmatrix}= \begin{pmatrix} 4+2i\\ -4\end{pmatrix}\]
ist inhomogen, da die rechte Seite $b=\begin{pmatrix} 4+2i\\ -4\end{pmatrix}$ nicht der Nullvektor ist. Das zugehörige homogene LGS
lautet
\[   \begin{pmatrix} i & -4 \\ -2& 8-8i\end{pmatrix}\cdot \begin{pmatrix} x\\ y\end{pmatrix}= \begin{pmatrix} 0\\ 0\end{pmatrix}.\]}
\lang{en}{
The above linear system
\[   \begin{pmatrix} i & -4 \\ -2& 8-8i\end{pmatrix}\cdot \begin{pmatrix} x\\ y\end{pmatrix}= \begin{pmatrix} 4+2i\\ -4\end{pmatrix}\]
is nonhomogeneous, because the right side $b=\begin{pmatrix} 4+2i\\ -4\end{pmatrix}$ is not the zero vector. The corresponding homogeneous system
is
\[   \begin{pmatrix} i & -4 \\ -2& 8-8i\end{pmatrix}\cdot \begin{pmatrix} x\\ y\end{pmatrix}= \begin{pmatrix} 0\\ 0\end{pmatrix}.\]}

\tab{\lang{de}{3. Beispiel} \lang{en}{3. Example}}
\lang{de}{
Das obige LGS 
\[ \begin{pmatrix} 1 & 2 & 1 \\ 1 & -1 & \frac{3}{2} \\ -4 & 2 & 0\end{pmatrix}\cdot \begin{pmatrix}x_1 \\x_2\\ x_3\end{pmatrix}
= \begin{pmatrix} 4 \\ -7 \\ -2 \end{pmatrix} \]
ist inhomogen, da die rechte Seite $b=\begin{pmatrix} 4 \\ -7 \\ -2 \end{pmatrix}$ nicht der Nullvektor ist. Das zugehörige homogene LGS
lautet
\[ \begin{pmatrix} 1 & 2 & 1 \\ 1 & -1 & \frac{3}{2} \\ -4 & 2 & 0\end{pmatrix}\cdot \begin{pmatrix}x_1 \\x_2\\ x_3\end{pmatrix}
= \begin{pmatrix} 0 \\ 0 \\ 0 \end{pmatrix}. \]}
\lang{en}{
The above linear system 
\[ \begin{pmatrix} 1 & 2 & 1 \\ 1 & -1 & \frac{3}{2} \\ -4 & 2 & 0\end{pmatrix}\cdot \begin{pmatrix}x_1 \\x_2\\ x_3\end{pmatrix}
= \begin{pmatrix} 4 \\ -7 \\ -2 \end{pmatrix} \]
is nonhomogeneous, because the right side $b=\begin{pmatrix} 4 \\ -7 \\ -2 \end{pmatrix}$ is not the zero vector. The corresponding homogeneous linear system
is
\[ \begin{pmatrix} 1 & 2 & 1 \\ 1 & -1 & \frac{3}{2} \\ -4 & 2 & 0\end{pmatrix}\cdot \begin{pmatrix}x_1 \\x_2\\ x_3\end{pmatrix}
= \begin{pmatrix} 0 \\ 0 \\ 0 \end{pmatrix}. \]}
\end{tabs*}
\end{example}

\lang{de}{
\begin{quickcheck}
		\type{input.interval}
        \field{rational}
        \precision{3}
      \field{real}
      \begin{variables}
           \randint[Z]{a}{2}{6}
           \randint[Z]{b}{2}{8}
           \randint[Z]{c}{2}{8}
           \randint[Z]{d}{2}{8}
           \end{variables}
      \text{Das folgende lineare Gleichungssystem
\[ \begin{mtable}[\cellaligns{crcrcr}]
(I)&\qquad \var{a} \cdot  x & - & \var{b}  & = & -\var{b} \\
(II)&   &  & \var{c}\cdot y & =  &-\var{d} \cdot  x 
\end{mtable} \]
ist
        } 
    \begin{choices}{unique}

        \begin{choice}
            \text{homogen.}
			\solution{true}
		\end{choice}
                    
        \begin{choice}
            \text{inhomogen.}
			\solution{false}
		\end{choice}
    \end{choices}{unique}
 \explanation{Das lineare Gleichungssystem ist homogen, da es sich in der folgenden Form darstellen lässt:
 \[ \begin{mtable}[\cellaligns{crcrcr}]
(I)&\qquad \var{a} \cdot  x &  &   & = & 0 \\
(II)&  \var{d} \cdot  x & + & \var{c}\cdot y & =&0
\end{mtable} \]
 }
	\end{quickcheck}}

 
\lang{en}{
\begin{quickcheck}
		\type{input.interval}
        \field{rational}
        \precision{3}
      \field{real}
      \begin{variables}
           \randint[Z]{a}{2}{6}
           \randint[Z]{b}{2}{8}
           \randint[Z]{c}{2}{8}
           \randint[Z]{d}{2}{8}
           \end{variables}
      \text{The following system of linear equations
\[ \begin{mtable}[\cellaligns{crcrcr}]
(I)&\qquad \var{a} \cdot  x & - & \var{b}  & = & -\var{b} \\
(II)&   &  & \var{c}\cdot y & =  &-\var{d} \cdot  x 
\end{mtable} \]
is
        } 
    \begin{choices}{unique}

        \begin{choice}
            \text{homogeneous.}
			\solution{true}
		\end{choice}
                    
        \begin{choice}
            \text{nonhomogeneous.}
			\solution{false}
		\end{choice}
    \end{choices}{unique}
 \explanation{The linear system is homogeneous, because it can be written as:
 \[ \begin{mtable}[\cellaligns{crcrcr}]
(I)&\qquad \var{a} \cdot  x &  &   & = & 0 \\
(II)&  \var{d} \cdot  x & + & \var{c}\cdot y & =&0
\end{mtable} \]
 }
	\end{quickcheck}}

\lang{de}{
Im folgenden Video wird ein lineares Gleichungssystem in Matrixschreibweise dargestellt und anschließend mit den bereits bekannten Methoden gelöst. \\
\floatright{\href{https://api.stream24.net/vod/getVideo.php?id=10962-2-10840&mode=iframe&speed=true}{\image[75]{00_video_button_schwarz-blau}}}}\\ \\


\section{\lang{de}{Die L"osungsmenge homogener linearer Gleichungssysteme} \lang{en}{The solution set of homogeneous linear systems}}\label{sec:homogen}

\lang{de}{
Auch in Fällen, in denen lineare Gleichungssysteme nicht eindeutig lösbar sind, 
kann die Lösungsmenge strukturiert angegeben werden.}
\lang{en}{
The solution set can be notated in a structured way, even if the linear system does not have a single unique solution.}

\begin{rule}\label{rule-lgs_raum}
\lang{de}{
Für  ein homogenes LGS $Ax=0$ mit $m$ Gleichungen und $n$ Unbekannten gilt:
\begin{enumerate}
\item Es besitzt stets die Lösung $x_1=\ldots=x_n=0$. Diese wird die \notion{triviale Lösung} genannt.
\item Die Lösungsmenge bildet einen \ref[def:Unter_VR][Untervektorraum]{supp:allg-vektorraum}, d.\,h. entweder ist $\mathbb{L}=\{ 0\}$ oder
es gibt %eine natürliche Zahl $k\leq n$ und
$k \leq n$ linear unabhängige Lösungen
\[ v_1=\begin{pmatrix}v_{11}\\ \vdots \\ v_{1n} \end{pmatrix},\quad v_2=\begin{pmatrix}v_{21}\\ \vdots \\ v_{2n} \end{pmatrix},\quad \ldots
\quad v_k=\begin{pmatrix}v_{k1}\\ \vdots \\ v_{kn}\end{pmatrix}, \]
so dass
\[ \mathbb{L}= \left\{ r_1v_1+r_2v_2+\cdots + r_kv_k \mid r_1,r_2,\ldots, r_k\in \mathbb{K} \right\}.\]
\item Jedes homogene LGS mit mehr Variablen als Gleichungen hat mindestens eine nichttriviale L"osung. 
\end{enumerate}
\floatright{\href{https://api.stream24.net/vod/getVideo.php?id=10962-2-11012&mode=iframe&speed=true}{\image[75]{00_video_button_schwarz-blau}}}}\\

\lang{en}{
For a homogeneous linear system with $m$ equations and $n$ unknowns holds:
\begin{enumerate}
\item $x_1=\ldots=x_n=0$ is always a solution. It is called the \notion{trivial solution}.
\item The solution set is a \ref[def:Unter_VR][linear subspace.]{supp:allg-vektorraum}. So, it is either $\mathbb{L}=\{ 0\}$ or
there exist %eine natürliche Zahl $k\leq n$ und
$k \leq n$ linear independent solutions
\[ v_1=\begin{pmatrix}v_{11}\\ \vdots \\ v_{1n} \end{pmatrix},\quad v_2=\begin{pmatrix}v_{21}\\ \vdots \\ v_{2n} \end{pmatrix},\quad \ldots
\quad v_k=\begin{pmatrix}v_{k1}\\ \vdots \\ v_{kn}\end{pmatrix}, \]
such that
\[ \mathbb{L}= \left\{ r_1v_1+r_2v_2+\cdots + r_kv_k \mid r_1,r_2,\ldots, r_k\in \mathbb{K} \right\}.\]
\item Every homogeneous linear system with more variables than equations has at least one non-trivial solution. 
\end{enumerate}}
\end{rule}

\begin{proof*}[\lang{de}{Beweis Regel} \lang{en}{Proof for the rule}]
\begin{showhide}
\lang{de}{
\begin{enumerate}
\item Bei einem homogenen LGS entspricht $b$ dem Nullvektor. Wählt man nun für $x$ ebenfalls den Nullvektor, so ist die Gleichung
\[Ax=A\cdot \begin{pmatrix} 0 \\ 0 \\ \vdots \\ 0 \end{pmatrix}= \begin{pmatrix} 0 \\ 0 \\ \vdots \\ 0 \end{pmatrix}=b\]
stets erfüllt.
\item Sind $v_1,\ldots, v_{\ell}$ Lösungen des LGS $Ax=0$, dann sieht man aufgrund der 
\ref[matrizen][Rechenregeln für die Matrix-Vektor-Multiplikation]{sec:rechenregeln}, dass für beliebige Elemente
$r_1,\ldots, r_{\ell} \in\mathbb{K}$ die Gleichung
\[ A(r_1v_1+\ldots +r_{\ell} v_{\ell})=r_1\cdot Av_1+\ldots + r_{\ell}\cdot Av_{\ell}=0+\ldots +0=0 \]
erfüllt ist. Damit ist jede Linearkombination der Lösungen $v_1,\ldots, v_{\ell}$ auch eine Lösung.

Um alle Lösungen darzustellen, müssen hinreichend viele linear unabhängige Vektoren gefunden werden, so dass alle Lösungen als Linearkombination dieser Vektoren
geschrieben werden können. Praktisch wird dies mit dem \link{gauss-verfahren}{Gauß-Verfahren} durchgeführt.
\item Hat ein homogenes LGS $m$ Gleichungen und $n$ Variablen mit $m<n$, so lässt sich das LGS höchstens nach $m$ Variablen auflösen.
Die übrigen Variablen lassen sich frei wählen. Damit besitzt ein solches LGS mindestens eine nichttriviale Lösung. \\
Im nächsten Kapitel \link{gauss-verfahren}{Gauß-Verfahren} werden wir diese Variablen genauer kennenlernen.
\end{enumerate}}

\lang{en}{
\begin{enumerate}
\item In a homogeneous linear system $b$ corresponds to the zero vector. If we choose the zero vector for $x$, the equation
\[Ax=A\cdot \begin{pmatrix} 0 \\ 0 \\ \vdots \\ 0 \end{pmatrix}= \begin{pmatrix} 0 \\ 0 \\ \vdots \\ 0 \end{pmatrix}=b\]
is always solved.
\item If $v_1,\ldots, v_{\ell}$ are solutions for the linear system $Ax=0$, then we see 
(using the \ref[matrizen][calculation rules for matrix-vector multiplication]{sec:rechenregeln}) that for any $r_1,\ldots, r_{\ell} \in\mathbb{K}$ 
the equation
\[ A(r_1v_1+\ldots +r_{\ell} v_{\ell})=r_1\cdot Av_1+\ldots + r_{\ell}\cdot Av_{\ell}=0+\ldots +0=0 \]
is solved. This shows, that any linear combination of the solutions $v_1,\ldots, v_{\ell}$ is a solution too.

To display all solutions, sufficiently many linearly independent vectors need to be found, such that all solutions can be written
as linear combinations of these vectors.
In practise, this is done using the \link{gauss-verfahren}{Gaussian elimination}.
\item If a homogeneous linear system has $m$ equations and $n$ variables with $m<n$, the linear system can at most be solved for $m$ variables.
The remaining variables can be chosen freely. Thus, such an LGS has at least one non-trivial solution. \\
In the next chapter, about the \link{gauss-verfahren}{Gaussian elimination}, we will get to know these variables more detailed.
\end{enumerate}}

\end{showhide}
\end{proof*}

\lang{de}{
\begin{example}\label{ex:lsgsmenge-homogen}
\begin{tabs*}
\tab{1. Beispiel}
Das homogene LGS
\[ \begin{pmatrix} 2 & -4 \\ -3 & 6\end{pmatrix}\cdot \begin{pmatrix} x\\ y\end{pmatrix}= \begin{pmatrix} 0\\ 0\end{pmatrix}\]
über $\Q$ hat als Lösungsmenge
\[ \mathbb{L}=\left\{ r\cdot \begin{pmatrix} 2 \\ 1 \end{pmatrix} \, \big| \, r\in \mathbb{Q} \right\}. \]
%wie man mit den bereits bekannten Verfahren berechnen kann. 

Hier ist also $k=1$ und $v_1=\begin{pmatrix} 2 \\ 1 \end{pmatrix}$ eine Lösung, so dass die Lösungsmenge
des LGS geschrieben werden kann als $\mathbb{L}= \left\{ r_1v_1 \, | \, r_1\in \mathbb{Q} \right\}$,
mit $r_1 = r$.

Betrachtet man das lineare Gleichungssystem als LGS über den reellen Zahlen, wäre die Lösungsmenge 
$\mathbb{L}= \left\{ r_1v_1 | r_1\in \R \right\}$. Sie unterscheidet sich also darin, welche Werte für den
Parameter $r_1$ zugelassen werden.

Das bedeutet beispielsweise, dass für $r=\sqrt{2}$ der Vektor $\begin{pmatrix} 2\sqrt{2} \\ \sqrt{2} \end{pmatrix}$ in der Lösungsmenge liegt, wenn das LGS über den reellen Zahlen betrachtet wird. 
Wird das LGS über den rationalen Zahlen betrachtet, stellt er keinen Lösungsvektor dar.
\tab{2. Beispiel}Die erweiterte Koeffizientenmatrix des homogenen LGS
\[   \begin{pmatrix} i & -4 \\ -2& 8-8i\end{pmatrix}\cdot \begin{pmatrix} x\\ y\end{pmatrix}= \begin{pmatrix} 0\\ 0\end{pmatrix} \]

lässt sich durch Multiplikation der ersten Zeile mit $-2i$ und anschließender Addition zur zweiten Gleichung in Stufenform bringen:
\[ \begin{pmatrix} i & -4 & | & 0 \\ 0& 8 & | & 0 \end{pmatrix}\]

Aus der zweiten Zeile folgt $y=0$ und durch Einsetzen in die erste Gleichung muss damit auch $x=0$ sein.
Das homogene LGS besitzt folglich nur die triviale Lösung:
\[ \mathbb{L}=\left\{ r\cdot \begin{pmatrix} 0 \\ 0 \end{pmatrix} \, \big| \, r\in \mathbb{C} \right\} =\left\{\begin{pmatrix} 0 \\ 0 \end{pmatrix}\right\}. \]


\tab{3. Beispiel}
Für das homogene LGS 
\[ \begin{pmatrix} 1 & 2 & 1 \\ 1 & -1 & \frac{3}{2} \\ -4 & 2 & 0\end{pmatrix}\cdot \begin{pmatrix}x_1 \\x_2\\ x_3\end{pmatrix}
= \begin{pmatrix} 0 \\ 0 \\ 0 \end{pmatrix} \]
über $\R$ (oder über $\Q$) zeigt eine Rechnung (siehe Abschnitt \link{gauss-verfahren}{Gauß-Verfahren}), dass es nur die 
triviale Lösung als Lösung besitzt.
\end{tabs*}
\end{example}}


\lang{en}{
\begin{example}\label{ex:lsgsmenge-homogen}
\begin{tabs*}
\tab{1. Example}
The solution set of the homogeneous linear system
\[ \begin{pmatrix} 2 & -4 \\ -3 & 6\end{pmatrix}\cdot \begin{pmatrix} x\\ y\end{pmatrix}= \begin{pmatrix} 0\\ 0\end{pmatrix}\]
over $\Q$ is
\[ \mathbb{L}=\left\{ r\cdot \begin{pmatrix} 2 \\ 1 \end{pmatrix} \, \big| \, r\in \mathbb{Q} \right\}. \]
%wie man mit den bereits bekannten Verfahren berechnen kann. 

Here we have $k=1$ and $v_1=\begin{pmatrix} 2 \\ 1 \end{pmatrix}$ is a solution. The solution set can then be written
as $\mathbb{L}= \left\{ r_1v_1 \, | \, r_1\in \mathbb{Q} \right\}$,
with $r_1 = r$.


If we consider the linear system as a system over the real numbers, the solution set would be 
$\mathbb{L}= \left\{ r_1v_1 | r_1\in \R \right\}$. The difference is therefore which values are allowed for the parameter $r_1$.

For example, for $r=\sqrt{2}$ the vector $\begin{pmatrix} 2\sqrt{2} \\ \sqrt{2} \end{pmatrix}$ is in the solution set, when the linear 
system is considered over the real numbers. But is the system considered as a system over $\Q$, the vector is no solution and therefore no
element of the solution set, because $\sqrt{2}\notin\Q$.

\tab{2. Example}
The first equation of the homogeneous linear system
\[   \begin{pmatrix} i & -4 \\ -2& 8-8i\end{pmatrix}\cdot \begin{pmatrix} x\\ y\end{pmatrix}= \begin{pmatrix} 0\\ 0\end{pmatrix} \]

can be multiplied with $-2i$ and added to the second equation. The augmented matrix is then in row echelon form:
\[ \begin{pmatrix} i & -4 & | & 0 \\ 0& 8 & | & 0 \end{pmatrix}\]

From the second line follows $y=0$. By inserting $y$ in the first equation, we see, that $x=0$ holds.
Hence, the homogeneous linear system has only the trivial solution:
\[ \mathbb{L}=\left\{ r\cdot \begin{pmatrix} 0 \\ 0 \end{pmatrix} \, \big| \, r\in \mathbb{C} \right\} =\left\{\begin{pmatrix} 0 \\ 0 \end{pmatrix}\right\}. \]


\tab{3. Example}
The homogeneous linear system
\[ \begin{pmatrix} 1 & 2 & 1 \\ 1 & -1 & \frac{3}{2} \\ -4 & 2 & 0\end{pmatrix}\cdot \begin{pmatrix}x_1 \\x_2\\ x_3\end{pmatrix}
= \begin{pmatrix} 0 \\ 0 \\ 0 \end{pmatrix} \]
over $\R$ (or over $\Q$) has only the trivial solution. This can be seen by applying the \link{gauss-verfahren}{Gaussian elimination}).
\end{tabs*}
\end{example}}


\section{\lang{de}{Die L"osungsmenge inhomogener linearer Gleichungssysteme} \lang{en}{The solution set of nonhomogeneous linear systems}}\label{sec:inhomogen}

\lang{de}{
Die Lösungsmengen inhomogener LGS und der zugehörigen homogenen LGS sind eng miteinander verbunden.}
\lang{en}{
The solution sets of inhomogeneous LGS and the corresponding homogeneous LGS are closely related.}

\begin{rule}
\lang{de}{
Es sei $Ax=b$ ein inhomogenes LGS über $\mathbb{K}$ mit $m$ Gleichungen und $n$ Variablen und $\mathbb{L}_0$ die
Lösungsmenge des zugehörigen homogenen LGS $Ax=0$. Dann gilt für die Lösungsmenge $\mathbb{L}$ des inhomogenen LGS:
\begin{itemize}
    \item Entweder ist die Lösungsmenge leer, d.\,h.~$\mathbb{L}=\emptyset$, oder es gilt
    \item $\mathbb{L}= v +\mathbb{L}_0=\left\{ v+ w \mid w\in \mathbb{L}_0 \right\}$, 
    wobei $v$ eine beliebige Lösung des inhomogenen LGS $Ax=b$ ist. 
\end{itemize}
\floatright{\href{https://api.stream24.net/vod/getVideo.php?id=10962-2-11013&mode=iframe&speed=true}{\image[75]{00_video_button_schwarz-blau}}}}\\

\lang{en}{
Let $Ax=b$ be a nonhomogeneous linear system over $\mathbb{K}$ with $m$ equations and $n$ unknowns and $\mathbb{L}_0$ the
solution set of the corresponding homogeneous linear system $Ax=0$. Then for the solution set $\mathbb{L}$ of the inhomogeneous LGS applies:
\begin{itemize}
    \item The solution set is empty, i.\,e.~$\mathbb{L}=\emptyset$, or
    \item $\mathbb{L}= v +\mathbb{L}_0=\left\{ v+ w \mid w\in \mathbb{L}_0 \right\}$, 
    with $v$ being any solution of the inhomogeneous linear system $Ax=b$. 
\end{itemize}}
\end{rule}


\begin{proof*}[\lang{de}{Beweis Regel} \lang{en}{Proof of the rule}]
\begin{showhide}
\lang{de}{
Diese Aussage erhält man wieder aus den \ref[matrizen][Rechenregeln für die Matrix-Vektor-Multiplikation]{sec:rechenregeln}.
Wenn nämlich die Lösungsmenge $\mathbb{L}$ nicht leer ist und $v\in \mathbb{L}$ eine beliebige Lösung ist, dann gilt $Av=b$.\\
Für jede Lösung $w\in \mathbb{L}_0$ des zugehörigen homogenen LGS $Ax=0$ gilt dann
\[  A(v+w)=Av+Aw=b+0=b. \]
Also ist $v+w\in \mathbb{L}$.

Ist andererseits $v' \in \mathbb{L}$ eine weitere Lösung des LGS $Ax=b$, so gilt
\[  A(v'-v)=Av'-Av=b-b=0. \]
Die Differenz $w=v'-v$ ist also eine Lösung des zugehörigen homogenen LGS $Ax=0$. Damit gilt $v'=v+(v'-v)=v+w$ 
und folglich lässt sich jede Lösung $v'$ des inhomogenen Systems als Summe einer festen inhomogenen Lösung $v$ 
und einer homogenen Lösung $w$ schreiben.}

\lang{en}{
This statement is again obtained from the \ref[matrizen][calculating rules for matrix-vector multiplication]{sec:rechenregeln}.
It is $Av=b$ for any $v\in\mathbb{L}$, when the solution set $\mathbb{L}$ is not empty.\\
For every solution $w\in \mathbb{L}_0$ of the corresponding homogeneous linear system $Ax=0$ we have
\[  A(v+w)=Av+Aw=b+0=b. \]
So, it is $v+w\in \mathbb{L}$.

Is on the other hand $v' \in \mathbb{L}$ another solution of the linear system $Ax=b$, we have
\[  A(v'-v)=Av'-Av=b-b=0. \]
So, the difference $w=v'-v$ is a solution of the corresponding homogeneous linear system $Ax=0$. Therefore it holds $v'=v+(v'-v)=v+w$
and any solution $v'$ of the nonhomogeneous system can be written as the sum of a fixed inhomogeneous solution $v$ and
and a homogeneous solution $w$.}
\end{showhide}
\end{proof*}


\lang{de}{
\begin{example}
\begin{tabs*}
\tab{1. Beispiel}
Das inhomogene LGS
\[ \begin{mtable}[\cellaligns{crcrcr}]
(I)&\qquad 2 \cdot  x & - & 4 \cdot  y & = & 10 \\
(II)&-3 \cdot  x & + & 6\cdot y & =  & -15
\end{mtable} \]
über $\R$ bzw. in Matrixform
\[   \begin{pmatrix} 2 & -4 \\ -3 & 6\end{pmatrix}\cdot \begin{pmatrix} x\\ y\end{pmatrix}= \begin{pmatrix} 10\\ -15\end{pmatrix}\]
hat die Lösungsmenge
\[ \mathbb{L}=\left\{ \begin{pmatrix}5\\ 0 \end{pmatrix}+r\cdot \begin{pmatrix} 2 \\ 1 \end{pmatrix} \, \big| \, r\in \mathbb{R} \right\}. \]
Hierbei ist $v=\begin{pmatrix}5\\ 0 \end{pmatrix}$ eine spezielle Lösung und die Lösungsmenge des zugehörigen homogenen LGS
ist 
\[ \mathbb{L}_0=\left\{ r\cdot \begin{pmatrix} 2 \\ 1 \end{pmatrix} \, \big| \, r\in \mathbb{R} \right\} \]
(siehe \lref{ex:lsgsmenge-homogen}{oben}).
\tab{2. Beispiel}
Die erweiterte Koeffizientenmatrix des inhomogenen LGS
\[   \begin{pmatrix} i & -4 \\ -2& 8-8i\end{pmatrix}\cdot \begin{pmatrix} x\\ y\end{pmatrix}= \begin{pmatrix} 4+2i\\ -4\end{pmatrix}\]

lässt sich durch Multiplikation der ersten Zeile mit $-2i$ und anschließender Addition zur zweiten Gleichung in Stufenform bringen:
\[ \begin{pmatrix} i & -4 & | & 4+2i \\ 0& 8 & | & -8i \end{pmatrix}\]

Aus der zweiten Zeile folgt $y=-i$ und durch Einsetzen in die erste Zeile ergibt sich
\[i\cdot x+4i=4+2i\Leftrightarrow x=\frac{4-2i}{i}=-4i-2.\]
(Im letzten Schritt wurde der Bruch mit $-i$ erweitert.)

Das inhomogene LGS besitzt damit als spezielle Lösung $v=\begin{pmatrix} -4i-2 \\ -i\end{pmatrix}$. 
Das dazugehörige homogene LGS (siehe \lref{ex:lsgsmenge-homogen}{oben}) besitzt nur die triviale Lösung.
Die Lösungsmenge enthält damit nur den Vektor $v$:
\[ \mathbb{L}=\{ \begin{pmatrix} -4i-2 \\ -i \end{pmatrix} \}. \]
\tab{3. Beispiel}
Das LGS 
\[ \begin{pmatrix} 1 & 2 & 1 \\ 1 & -1 & \frac{3}{2} \\ -4 & 2 & 0\end{pmatrix}\cdot \begin{pmatrix}x_1 \\x_2\\ x_3\end{pmatrix}
= \begin{pmatrix} 4 \\ -7 \\ -2 \end{pmatrix} \]
über $\R$ besitzt die Lösung $\begin{pmatrix} 2 \\ 3 \\ -4 \end{pmatrix}$, wie man leicht nachrechnet.
Das zugehörige homogene LGS 
\[ \begin{pmatrix} 1 & 2 & 1 \\ 1 & -1 & \frac{3}{2} \\ -4 & 2 & 0\end{pmatrix}\cdot \begin{pmatrix}x_1 \\x_2\\ x_3\end{pmatrix}
= \begin{pmatrix} 0 \\ 0 \\ 0 \end{pmatrix} \]
besitzt nur die triviale Lösung als Lösung (siehe \lref{ex:lsgsmenge-homogen}{oben}). Also ist die Lösungsmenge des inhomogenen LGS
\[ \mathbb{L}=\{ \begin{pmatrix} 2 \\ 3 \\ -4 \end{pmatrix} \}. \]
\end{tabs*}
\end{example}}

\lang{en}{
\begin{example}
\begin{tabs*}
\tab{1. Example}
The inhomogeneous linear system
\[ \begin{mtable}[\cellaligns{crcrcr}]
(I)&\qquad 2 \cdot  x & - & 4 \cdot  y & = & 10 \\
(II)&-3 \cdot  x & + & 6\cdot y & =  & -15
\end{mtable} \]
over $\R$ or in matrix representation
\[   \begin{pmatrix} 2 & -4 \\ -3 & 6\end{pmatrix}\cdot \begin{pmatrix} x\\ y\end{pmatrix}= \begin{pmatrix} 10\\ -15\end{pmatrix}\]
has the solution set
\[ \mathbb{L}=\left\{ \begin{pmatrix}5\\ 0 \end{pmatrix}+r\cdot \begin{pmatrix} 2 \\ 1 \end{pmatrix} \, \big| \, r\in \mathbb{R} \right\}. \]
 $v=\begin{pmatrix}5\\ 0 \end{pmatrix}$ is a particular solution and the solution set of the corresponding homogeneous linear system 
is
\[ \mathbb{L}_0=\left\{ r\cdot \begin{pmatrix} 2 \\ 1 \end{pmatrix} \, \big| \, r\in \mathbb{R} \right\} \]
(see \lref{ex:lsgsmenge-homogen}{above}).
\tab{2. Example}
In augmented matrix of the inhomogeneous system
\[   \begin{pmatrix} i & -4 \\ -2& 8-8i\end{pmatrix}\cdot \begin{pmatrix} x\\ y\end{pmatrix}= \begin{pmatrix} 4+2i\\ -4\end{pmatrix}\]

the first row can be multiplied with $-2i$. After adding the result to the second equation, we get the row echelon form:
\[ \begin{pmatrix} i & -4 & | & 4+2i \\ 0& 8 & | & -8i \end{pmatrix}\]

From the second row follows $y=-i$. Inserting this in the first row, we have
\[i\cdot x+4i=4+2i\Leftrightarrow x=\frac{4-2i}{i}=-4i-2.\]
(In the last step we multiplied  the numerator and denominator with $-i$.)

The nonhomogeneous system has the particular solution
$v=\begin{pmatrix} -4i-2 \\ -i\end{pmatrix}$. 
The corresponding homogeneous linear system (see \lref{ex:lsgsmenge-homogen}{above}) only has the trivial solution.
The solutions set only contains the vector $v$:
\[ \mathbb{L}=\{ \begin{pmatrix} -4i-2 \\ -i \end{pmatrix} \}. \]
\tab{3. Example}
The system of linear equations
\[ \begin{pmatrix} 1 & 2 & 1 \\ 1 & -1 & \frac{3}{2} \\ -4 & 2 & 0\end{pmatrix}\cdot \begin{pmatrix}x_1 \\x_2\\ x_3\end{pmatrix}
= \begin{pmatrix} 4 \\ -7 \\ -2 \end{pmatrix} \]
over $\R$ has the solution $\begin{pmatrix} 2 \\ 3 \\ -4 \end{pmatrix}$. This can be calculated easily.
The corresponding homogeneous linear system 
\[ \begin{pmatrix} 1 & 2 & 1 \\ 1 & -1 & \frac{3}{2} \\ -4 & 2 & 0\end{pmatrix}\cdot \begin{pmatrix}x_1 \\x_2\\ x_3\end{pmatrix}
= \begin{pmatrix} 0 \\ 0 \\ 0 \end{pmatrix} \]
only has the trivial solution (see \lref{ex:lsgsmenge-homogen}{above}). Hence, the solution set of the nonhomogeneous system is
\[ \mathbb{L}=\{ \begin{pmatrix} 2 \\ 3 \\ -4 \end{pmatrix} \}. \]
\end{tabs*}
\end{example}}

\lang{de}{
\begin{quickcheck}

    \begin{variables}
    
      \randint[Z]{a}{2}{2}
      \randint[Z]{b}{6}{9}
      \randint[Z]{c}{2}{5}
      
    
      \function[calculate]{a3}{a*3}
      \function[calculate]{b3}{b*3}
      \function[calculate]{c2}{c*2}
       \function[calculate]{c4}{c*4}
      \function[calculate]{x}{(c2-b*2)/a}
     
    \end{variables}
       \text{Betrachte die erweiterte Koeffizientenmatrix

\[ \begin{pmatrix} \var{a} & \var{b}  &| & \var{c2}  \\ \var{a3} & \var{b3}  & |& \var{c4} \\\end{pmatrix}\cdot \]
Was gilt für die dazugehörige Lösungsmenge?
 
        } 
    \begin{choices}{unique}

      \begin{choice}
        \text{$\mathbb{L}=\left\{ r\cdot \begin{pmatrix} \var{x} \\ 2 \end{pmatrix} \mid\, r\in \R \right\}$}
        \solution{false}
      \end{choice}

      \begin{choice}
        \text{\text{$\mathbb{L}=\left\{\begin{pmatrix} \var{x} \\ 2 \end{pmatrix}\right\}$}}
        \solution{false}
        \end{choice}
      \begin{choice}
        \text{\text{$\mathbb{L}=\emptyset$}}
        \solution{true}
      \end{choice}
\end{choices}{unique}
\explanation{Die Lösungsmenge entspricht der leeren Menge, da durch $(II)-3\cdot(I)$ die Zeile $0=-\var{c2}$ entsteht. Somit besitzt das LGS keine Lösung.}
\end{quickcheck}}


\lang{en}{
\begin{quickcheck}

    \begin{variables}
    
      \randint[Z]{a}{2}{2}
      \randint[Z]{b}{6}{9}
      \randint[Z]{c}{2}{5}
      
    
      \function[calculate]{a3}{a*3}
      \function[calculate]{b3}{b*3}
      \function[calculate]{c2}{c*2}
       \function[calculate]{c4}{c*4}
      \function[calculate]{x}{(c2-b*2)/a}
     
    \end{variables}
       \text{Consider the augmented matrix

\[ \begin{pmatrix} \var{a} & \var{b}  &| & \var{c2}  \\ \var{a3} & \var{b3}  & |& \var{c4} \\\end{pmatrix}\cdot \]
What is true for the corresponding solution set?
 
        } 
    \begin{choices}{unique}

      \begin{choice}
        \text{$\mathbb{L}=\left\{ r\cdot \begin{pmatrix} \var{x} \\ 2 \end{pmatrix} \mid\, r\in \R \right\}$}
        \solution{false}
      \end{choice}

      \begin{choice}
        \text{\text{$\mathbb{L}=\left\{\begin{pmatrix} \var{x} \\ 2 \end{pmatrix}\right\}$}}
        \solution{false}
        \end{choice}
      \begin{choice}
        \text{\text{$\mathbb{L}=\emptyset$}}
        \solution{true}
      \end{choice}
\end{choices}{unique}
\explanation{The solution set is the empty set, because the operation $(II)-3\cdot(I)$ creates the row $0=-\var{c2}$ is created. Therefore, the linear system has no solution.}
\end{quickcheck}}

\end{visualizationwrapper}

\end{content}