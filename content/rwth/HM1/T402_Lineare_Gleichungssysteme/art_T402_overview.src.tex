%$Id:  $
\documentclass{mumie.article}
%$Id$
\begin{metainfo}
  \name{
    \lang{de}{Überblick: Lineare Gleichungssyteme}
    \lang{en}{Overview: Linear system }
  }
  \begin{description} 
 This work is licensed under the Creative Commons License Attribution 4.0 International (CC-BY 4.0)   
 https://creativecommons.org/licenses/by/4.0/legalcode 

    \lang{de}{Lineare Gleichungssysteme}
    \lang{en}{Linear system}
  \end{description}
  \begin{components}
  \end{components}
  \begin{links}
\link{generic_article}{content/rwth/HM1/T402_Lineare_Gleichungssysteme/g_art_content_06_umformungen_rang.meta.xml}{content_06_umformungen_rang}
\link{generic_article}{content/rwth/HM1/T402_Lineare_Gleichungssysteme/g_art_content_05_gaussverfahren.meta.xml}{content_05_gaussverfahren}
\link{generic_article}{content/rwth/HM1/T402_Lineare_Gleichungssysteme/g_art_content_04_lgs.meta.xml}{content_04_lgs}
\end{links}
  \creategeneric
\end{metainfo}
\begin{content}
\begin{block}[annotation]
	Im Ticket-System: \href{https://team.mumie.net/issues/30117}{Ticket 30117}
\end{block}




\begin{block}[annotation]
Im Entstehen: Überblicksseite für Kapitel Lineare Gleichungssyteme
\end{block}

\usepackage{mumie.ombplus}
\ombchapter{1}
\lang{de}{\title{Überblick: Lineare Gleichungssyteme}}
\lang{en}{\title{Overview: Linear system}}



\begin{block}[info-box]
\lang{de}{\strong{Inhalt}}
\lang{en}{\strong{Contents}}


\lang{en}{
    \begin{enumerate}%[arabic chapter-overview]
   \item[2.1] \link{content_04_lgs}{Linear system}
   \item[2.2] \link{content_05_gaussverfahren}{Gaussian elimination}
   \item[2.3] \link{content_06_umformungen_rang}{Row transformations by matrix multiplication and rank}
  \end{enumerate}
} %lang

\lang{de}{
    \begin{enumerate}%[arabic chapter-overview]
   \item[2.1] \link{content_04_lgs}{Lineare Gleichungssyteme}
   \item[2.2] \link{content_05_gaussverfahren}{Gauß-Verfahren}
   \item[2.3] \link{content_06_umformungen_rang}{Zeilenumformungen durch Matrizenmultiplikation und Rang}
  \end{enumerate}
}

\end{block}

\begin{zusammenfassung}
\lang{de}{
Lineare Gleichungssysteme mit reellwertigen Koeffizienten haben wir bereits in Kursteil 1 behandelt. In diesem Kapitel liegt der Fokus auf allgemeinen Körpern $\K$. Wir formalisieren die Notation etwas weiter und sind damit in der Lage, das Gaußverfahren algorithmisch anzugeben. 
Nachdem wir das Lösen mehrerer LGS mit unterschiedlichen rechten Seiten bei gleicher Matrix $A$ behandeln,
stellen wir im letzten Teilkapitel Zeilenumformungen als Matrizenmultiplikation dar.
}
\lang{en}{
Linear systems with real coefficients were already discussed in Part 1. In this chapter, we will focus on general fields $\K$.
We formalise the notation a little further and are thus able to state the Gaussian elimination algorithmically.
We will solve several linear systems with different right sides for the same matrix $A$. Afterwards we will express row transformations as matrix multiplications.
}
\end{zusammenfassung}

\begin{block}[info]\lang{de}{\strong{Lernziele}}
\lang{en}{\strong{Learning Goals}} 
\begin{itemize}[square]
\item \lang{de}{Sie lösen homogene und inhomogene lineare Gleichungssysteme, deren Koeffizienten reellwertig oder komplexwertig sind, algorithmisch durch das Gaußverfahren.}
      \lang{en}{Being able to solve homogeneous and nonhomogeneous systems with real or complex coefficients by using the Gaussian elimination.}
\item \lang{de}{Sie erkennen wie die Lösungsmenge beschaffen ist und führen bei Bedarf freie Variablen ein.}
      \lang{en}{Being able to recognise the nature of the solution set and introduce free variables where necessary.}
\item \lang{de}{Sie stellen lineare Gleichungssysteme anhand von textuellen Beschreibungen auf (Modellierung).}
      \lang{en}{Being able to set up linear systems by using verbal descriptions (modelling).}
\item \lang{de}{Sie konstruieren für Zeilenumformungen des Gaußverfahrens zugehörige Elementarmatrizen.}
      \lang{en}{Being able to construct elementary matrices for the row transformations in the Gaussian elimination.}
\end{itemize}
\end{block}


\end{content}
