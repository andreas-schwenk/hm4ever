\documentclass{mumie.element.exercise}
%$Id$
\begin{metainfo}
  \name{
    \lang{de}{Ü03: Lösungsmenge}
    \lang{en}{Ex03: Solution set}
  }
  \begin{description} 
 This work is licensed under the Creative Commons License Attribution 4.0 International (CC-BY 4.0)   
 https://creativecommons.org/licenses/by/4.0/legalcode 

    \lang{de}{Hier die Beschreibung}
    \lang{en}{}
  \end{description}
  \begin{components}
  \end{components}
  \begin{links}
  \end{links}
  \creategeneric
\end{metainfo}
\begin{content}
\title{
  \lang{de}{Ü03: Lösungsmenge}
  \lang{en}{Ex03: Solution set}
}
\begin{block}[annotation]
	Matritzen, Lineare Gleichungen: Übungen
\end{block}
\begin{block}[annotation]
  Im Ticket-System: \href{http://team.mumie.net/issues/11290}{Ticket 11290}
\end{block}

\lang{de}{Welche Lösungsmengen sind für das folgende lineare Gleichungssystem korrekt?}
\lang{en}{Which of the following solution sets are correct for the given linear system?}
\begin{displaymath}
\begin{mtable}[\cellaligns{ccrcrcrcr}]
\text{(I)}&\qquad&2x_1&+&3x_2&+&x_3&=&3\\
\text{(II)}&&4x_1&-&4x_2&+&2x_3&=&1\\
\text{(III)}&&2x_1&-&5x_2&+&x_3&=&-1
\end{mtable}
\end{displaymath}

\begin{table}[\class{items}]
  \nowrap{(a) $\mathbb{L}=\{(2-\frac{1}{2}s\lang{de}{;}\lang{en}{,} \frac{1}{2}\lang{de}{;}\lang{en}{,} -\frac{5}{2}+s) ~|~ s\in\R\}$}\\
  \nowrap{(b) $\mathbb{L}=\{(-3-t\lang{de}{;}\lang{en}{,} \frac{1}{2}\lang{de}{;}\lang{en}{,} 1+2t) ~|~ t\in\R\}$}\\
  \nowrap{(c) $\mathbb{L}=\emptyset$}\\
  \nowrap{(d) $\mathbb{L}=\{(\frac{3}{4}-\frac{1}{2}t\lang{de}{;}\lang{en}{,} \frac{1}{2}\lang{de}{;}\lang{en}{,} t) ~|~ t\in\R\}$}\\
  \nowrap{(e) $\mathbb{L}=\{(r\lang{de}{;}\lang{en}{,} \frac{1}{2}\lang{de}{;}\lang{en}{,} \frac{3}{2}-2r) ~|~ r\in\R\}$}\\
  \nowrap{(f) $\mathbb{L}=\{(\frac{1}{2}\lang{de}{;}\lang{en}{,} \frac{1}{2}\lang{de}{;}\lang{en}{,} \frac{1}{2})\}$}\\
  \nowrap{(g) $\mathbb{L}=\{(r\lang{de}{;}\lang{en}{,} s\lang{de}{;}\lang{en}{,} 3-2r-3s) ~|~ r, s\in\R\}$}
\end{table}

\begin{tabs*}[\initialtab{0}\class{exercise}]
  \tab{
  \lang{de}{Antwort}
  \lang{en}{Answer}
  }
\begin{table}[\class{items}]
    \lang{de}{Die Lösungsmengen in (a), (d) und (e) sind korrekt. Es handelt sich hierbei um verschiedene Darstellungen derselben Lösungsmenge.}
    \lang{en}{The solution sets in (a), (d), and (e) are correct. Each of the sets are different representations of the same solution set.}
\end{table}

\tab{
  \lang{de}{Lösung}
  \lang{en}{Solution}
  }
  
  \begin{incremental}[\initialsteps{1}]

    \step \lang{de}{Zunächst stellen wir die erweiterte Koeffizientenmatrix des reellen LGS auf:
\[
(A \mid b) ~=~
\begin{pmatrix}
    2 &  3 & 1 &|&  3 \\
    4 & -4 & 2 &|&  1 \\
    2 & -5 & 1 &|& -1
\end{pmatrix}
\]
    }

  \lang{en}{First we set up the augmented matrix of the real-valued linear system:
\[
(A \mid b) ~=~
\begin{pmatrix}
    2 &  3 & 1 &|&  3 \\
    4 & -4 & 2 &|&  1 \\
    2 & -5 & 1 &|& -1
\end{pmatrix}
\]
    }

    \lang{de}{Um zu entscheiden, welche Lösungsmengen für das lineare Gleichungssystem korrekt sind, prüfen wir zuerst die Lösbarkeit des linearen Gleichungssystems.
    Dazu bringen wir es mit dem Gauß-Verfahren auf Stufenform.}
    \lang{en}{To decide which solution sets are correct for the linear system,
    we first check the solvability of the linear system. Therefore we transform it into row echelon form by using Gaussian elimination.}


   \step \lang{de}{Der aktuelle \textit{Stufeneintrag} ist $a_{11}$ und wird im Folgenden orange markiert.
   Um die Einträge $a_{21}$ und $a_{31}$ (blau) zu eliminieren, werden die folgenden Zeilenoperationen
   durchgeführt:}
   \lang{de}{Right now the \textit{leading coefficient} is $a_{11}$ and is colored orange in the following.
   To eliminate the entries $a_{21}$ and $a_{31}$ (blue), we perform the following row operations:}
   \[
\begin{pmatrix}
    \textcolor{#CC6600}{2} & 3 & 1 &|& 3 \\
    \textcolor{#0066CC}{4} & -4 & 2 &|& 1 \\
    \textcolor{#0066CC}{2} & -5 & 1 &|& -1
\end{pmatrix}
~
\begin{matrix}
    \\
   |(-2) \cdot (I) + (II)\\
    | -(I) + (III)
\end{matrix}
\]

    \lang{de}{Im nächsten Schritt können die zweite und die dritte Zeile vereinfacht werden:}
    \lang{en}{in the next step we simplify the second and the third equation:}
\[
\rightsquigarrow~~
\begin{pmatrix}
    2 & 3 & 1 &|& 3 \\
    0 & \textcolor{#0066CC}{-10} & \textcolor{#0066CC}{0} &|& \textcolor{#0066CC}{-5} \\
    0 & \textcolor{#0066CC}{-8} & \textcolor{#0066CC}{0} &|& \textcolor{#0066CC}{-4} \\
\end{pmatrix}
~
\begin{matrix}
    \\
    | (II):(-5)\\
    | (III):(-4)
\end{matrix}
\]
    \lang{de}{Nun stellt man fest, dass die Zeilen (II) und (III) identisch sind.
    Die dritte Zeile kann eliminiert werden:}
    \lang{en}{We see, that the rows (II) and (III) are identical. The third row can be eliminated: }
    \[
    \rightsquigarrow~~
    \begin{pmatrix}
        2 & 3 & 1 &|& 3 \\
        0 & \textcolor{#0066CC}{2} & \textcolor{#0066CC}{0} &|& \textcolor{#0066CC}{1} \\
        0 & \textcolor{#0066CC}{2} & \textcolor{#0066CC}{0} &|& \textcolor{#0066CC}{1} \\
    \end{pmatrix}
    ~
    \begin{matrix}
        \\
        \\
       | -(II)+(III)
    \end{matrix}
    \]
    \lang{de}{Man erhält:}
    \lang{en}{We get:}
    \[
    \rightsquigarrow~~
    \begin{pmatrix}
        2 & 3 & 1 &|& 3 \\
        0 & 2 & 0 &|& 1 \\
        0 & \textcolor{#0066CC}{0} & \textcolor{#0066CC}{0} &|& \textcolor{#0066CC}{0} \\
    \end{pmatrix}
    \]



%\lang{en}{In order to decide which solution sets are correct for the given linear system, we need to first check the solvability of the linear system. To do this, we use Gaussian Elimination to bring the system to row echelon form.} 
    %\step \lang{de}{TODO: TODO: TODO: Multiplizieren Sie beispielsweise die erste Gleichung auf beiden Seiten mit $-2$ bzw. $-1$ und addieren Sie sie dann zur zweiten  bzw. dritten Gleichung, so dass in diesen $x$ nicht mehr vorkommt.}
    %\lang{en}{Multiply both sides of the first equation by $-2$ and add it to the second; multiply both sides of the first equation by $-1$ and add it to the third to remove the variable $x$ from both equations.}
    %\step \lang{de}{Das lineare Gleichungssystem sieht dann wie folgt aus:}
    %\lang{en}{The linear system then looks as follows:}
%    \begin{displaymath}
%\begin{mtable}[\cellaligns{ccrcrcrcr}]
%\text{(I)}&\qquad&2x&+&3y&+&z&=&3\phantom{.}\\
%\text{(II)}&&&-&10y&&&=&-5\phantom{.}\\
%\text{(III)}&&&-&8y&&&=&-4\lang{de}{.}\lang{en}{\phantom{.}}
%\end{mtable}
%\end{displaymath}
%
%    \step \lang{de}{Die zweite bzw. dritte Gleichung lässt sich hier noch vereinfachen, indem man auf beiden Seiten durch $-5$ bzw. $-4$ teilt.\\
%    Das lineare Gleichungssystem sieht dann wie folgt aus:}
%    \lang{en}{The second and third equations can be simplified by dividing the second by $-5$ and the third by $-4$.\\
%    The linear system then looks as follows:}
%    \begin{displaymath}
%\begin{mtable}[\cellaligns{ccrcrcrcr}]
%\text{(I)}&\qquad&2x&+&3y&+&z&=&3\phantom{.}\\
%\text{(II)}&&&&2y&&&=&1\phantom{.}\\
%\text{(III)}&&&&2y&&&=&1\lang{de}{.}\lang{en}{\phantom{.}}
%\end{mtable}
%\end{displaymath}

 %   \step\lang{de}{Multipliziert man nun die zweite Gleichung mit $-1$ und addiert sie dann zur dritten Gleichung dazu, so enthält die dritte Gleichung schließlich $y$ nicht mehr.\\
  %  Das lineare Gleichungssystem sieht dann wie folgt aus:}
  %  \lang{en}{Now we multiply the second equation by $-1$ and add it to the third equation, so that the third equation doesn't contain the variable $y$ anymore.\\
  %  The linear system then looks as follows:}
  %  \begin{displaymath}
%\begin{mtable}[\cellaligns{ccrcrcrcr}]
%\text{(I)}&\qquad&2x&+&3y&+&z&=&3\phantom{.}\\
%\text{(II)}&&&&2y&&&=&1\phantom{.}\\
%\text{(III)}&&&&&&0&=&0\lang{de}{.}\lang{en}{\phantom{.}}
%\end{mtable}
%\end{displaymath}
    
    \step \lang{de}{Das lineare Gleichungssystem ist jetzt in Stufenform und man erkennt, dass genau eine Variable frei wählbar ist, allerdings nicht $x_2$, da aus der zweiten Gleichung $x_2=\frac{1}{2}$ folgt.}
    \lang{en}{The linear system is now in row echelon form and we can see that one of the variables can be chosen freely. The free variable can't be $x_2$, since the second equation gives us $y=\frac{1}{2}$.}
    \step\lang{de}{Aufgrund dieser Feststellung scheiden die Lösungsmengen in (c), (f) und (g) als korrekte Lösungsmengen aus.}
    \lang{en}{Because of this restriction, the solution sets c), f), and g) can be thrown out as possibilities for solution sets.}
    
    \step\lang{de}{Bei den restlichen Lösungsmengen muss durch Einsetzen in das lineare Gleichungssystem geprüft werden, ob die angegebenen Lösungsvorschläge das lineare Gleichungssystem lösen.
    
    Das Tripel aus (a) beispielsweise erfüllt die Gleichungen (I) und (II) aus der Stufenform, denn $2\cdot (2-\frac{1}{2}s)+3\cdot\frac{1}{2}+(-\frac{5}{2}+s)=4-s+\frac{3}{2}-\frac{5}{2}+s=3$ und $2\cdot \frac{1}{2}=1$. Somit ist diese Lösungsmenge korrekt.\\
    Analog erfüllen die Tripel aus (d) und (e) die Gleichungen (I) und (II) aus der Stufenform, nicht aber das Tripel aus (b).
    
    Insgesamt sind also die Lösungsmengen in (a), (d) und (e) korrekt.}  
    
    \lang{en}{In order to check the remaining solution sets, we need to substitute the solutions for $x$, $y$, and $z$ into the equations to check whether each of the given possibilities do in fact solve the linear system.
    
    The triple from (a) fulfills equations (I) and (II) from row echelon form, since $2\cdot (2-\frac{1}{2}s)+3\cdot\frac{1}{2}+(-\frac{5}{2}+s)=4-s+\frac{3}{2}-\frac{5}{2}+s=3$ and $2\cdot \frac{1}{2}=1$, hence this solution set is correct.\\
    Analogously, the triples from (d) and (e) fulfill equations (I) and (II) from row echelon form, but the triple from b) doesn't.
    
    In total, the solution sets from (a), (d), and (e) are correct.}  
  \end{incremental}

\tab{
  \lang{de}{Hinweis}
  \lang{en}{Note}
  }
  
  \lang{de}{Beachten Sie, dass es auch andere Möglichkeiten gibt, das lineare Gleichungssystem mit dem Gauß-Verfahren auf Stufenform zu bringen. Der aufgeführte Weg ist nur eine Variante. Außerdem ist die Stufenform nicht eindeutig, die Antwort beeinflusst dies aber nicht.}
  \lang{en}{Note that there are other possibilities for reducing the linear system into row echelon form using Gaussian Elimination. The steps performed here are only one variant and the row echelon form of a linear system is not unique, but the answer isn't influenced by this at all.}

  \tab{
  \lang{de}{Warnung}
  \lang{en}{Warning}
  }
	
	\lang{de}{Es genügt hier nicht, nur die jeweils angegebenen Lösungsvorschläge für $x_1$, $x_2$ und $x_3$ in die Gleichungen des linearen Gleichungssystems einzusetzen und zu schauen, ob alle drei Gleichungen erfüllt sind.
	Damit lässt sich nur feststellen, ob die angegebene Lösungsmenge falsch ist. Sind alle Gleichungen erfüllt, heißt das nur, dass die gegebene Lösung zur Lösungsmenge gehört, nicht aber, dass die Lösungsmenge damit auch vollständig ist.
	
	In der vorliegenden Aufgabe ist z.\,B. das Tripel $(\frac{1}{2}; \frac{1}{2}; \frac{1}{2})$ eine Lösung des linearen Gleichungssystems, $\mathbb{L}=\{(\frac{1}{2}; \frac{1}{2}; \frac{1}{2})\}$ ist aber falsch, da die Lösungsmenge noch nicht vollständig ist.}
	\lang{en}{It is not enough here to only substitute the proposed solutions for $x$, $y$, and $z$ into the linear system to see whether all three equations are fulfilled.
	This can only help determine whether the given solution set is false. If all of the equations are fulfilled, that only means that the given solution is part of the solution set, but not that the solution set is complete.
	
	For example, the triple $(\frac{1}{2}, \frac{1}{2}, \frac{1}{2})$ is a solution of the linear system, however $\mathbb{L}=\{(\frac{1}{2}, \frac{1}{2}, \frac{1}{2})\}$ is wrong since it doesn't represent the complete solution set.}
  

 \end{tabs*}
\end{content}