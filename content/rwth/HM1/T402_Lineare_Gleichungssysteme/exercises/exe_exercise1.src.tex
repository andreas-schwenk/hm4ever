\documentclass{mumie.element.exercise}
%$Id$
\begin{metainfo}
  \name{
    \lang{de}{Ü01: Komplexwertiges LGS}
    \lang{en}{Ex01: Complex-valued linear system}
  }
  \begin{description} 
 This work is licensed under the Creative Commons License Attribution 4.0 International (CC-BY 4.0)   
 https://creativecommons.org/licenses/by/4.0/legalcode 

    \lang{de}{Hier die Beschreibung}
    \lang{en}{}
  \end{description}
  \begin{components}
  \end{components}
  \begin{links}
  \end{links}
  \creategeneric
\end{metainfo}
\begin{content}
\usepackage{mumie.ombplus}

\title{\lang{de}{Ü01: Komplexwertiges LGS} \lang{en}{Ex01: Complex-valued linear system}}

\begin{block}[annotation]
  Im Ticket-System: \href{http://team.mumie.net/issues/11288}{Ticket 11288}
\end{block}

%######################################################FRAGE_TEXT
\lang{de}{ Wir betrachten das folgende inhomogene LGS mit komplexwertigen Koeffizienten:}
\lang{en}{
We consider the nonhomogeneous complex-valued linear system:}
\begin{equation}\tag{1}
\begin{mtable}[\cellaligns{rcrcrcr}]
 x_{1}&+&2x_{2}&-&x_{3}&=&1+i\\
3x_{1}&&&+& x_{3}&=&1-i\\
\end{mtable}
\end{equation}
\lang{de}{Wie üblich ist $i\in \C$ die imaginäre Einheit.} \lang{en}{As usual, $i\in \C$ is the imaginary unit.}

\begin{enumerate}[(a)]
\item (a) \lang{de}{Beschreiben Sie das LGS durch die Koeffizientenmatrix $A$ sowie den Vektor $b$ für die rechte Seite.
Stellen Sie dann die erweiterte Koeffizientenmatrix für das angegebene LGS auf.}
\lang{en}{Describe the linear system by the coefficient matrix $A$ and the vector $b$ as the right side.
Then set up the augmented matrix for the given linear system.}
\item (b) \lang{de}{Bestimmen Sie die Lösungsmenge des zugehörigen homogenen LGS.} \lang{en}{Determine the solution set of the corresponding homogeneous linear system.}
\item (c) \lang{de}{Rechnen Sie nach, dass $(x_1;x_2;x_3)=(0;1;1-i)$ 
%\[  \begin{pmatrix} x_1\\ x_2 \\ x_3 \end{pmatrix}
%    =
%    \begin{pmatrix} 0\\ 1 \\ 1-i \end{pmatrix}   \]
eine Lösung des inhomogenen LGS ist.}
\lang{en}{Calculate, that $(x_1;x_2;x_3)=(0;1;1-i)$ is a solution of the nonhomogeneous linear system.}
\item (d) \lang{de}{Geben Sie mit Hilfe von (b) und (c) die Lösungsmenge des inhomogenen LGS an.}
\lang{en}{Use (b) and (c) to give the solution set of the nonhomogeneous linear system.}
\end{enumerate} 

%##################################################ANTWORTEN_TEXT
\begin{tabs*}[\initialtab{0}\class{exercise}]

  %++++++++++++++++++++++++++++++++++++++++++START_TAB_X
  \tab{\lang{de}{   Antwort  } \lang{en}{Answer}}
  \begin{incremental}[\initialsteps{1}]
  
  	 %----------------------------------START_STEP_X
    \step 
    \lang{de}{   

(a) Koeffizientenmatrix $A$ und die rechte Seite $b$ sehen wie folgt aus:
\[
A=
\left(\begin{smallmatrix}
    1 & 2 & -1 \\
    3 & 0 & 1 
\end{smallmatrix}\right), \quad
b=
\left(\begin{smallmatrix}
    1+i \\
    1-i 
\end{smallmatrix}\right)
\]

Die erweiterte Koeffizientenmatrix lautet:

\[
(A \mid b) =
\left(\begin{smallmatrix}
    1 & 2 & -1 & | & 1+i \\
    3 & 0 & 1  & | & 1-i 
\end{smallmatrix}\right)
\]

(b) Die erweiterte Koeffizientenmatrix für das zugehörige homogene LGS lautet:
\[
(A \mid 0) =
\left(\begin{smallmatrix}
    1 & 2 & -1 & | & 0 \\
    3 & 0 & 1  & | & 0 
\end{smallmatrix}\right)
\]

%\[
%\begin{mtable}[\cellaligns{rcrcrcr}]
% x_{1}&+&2x_{2}&-&x_{3}&=&0\\
%3x_{1}&&&+& x_{3}&=&0\\
%\end{mtable} \]
Die Lösungsmenge des homogenen LGS ist:
\[  \mathbb{L}_0= \left\{ t\cdot \left(\begin{smallmatrix}
-1 \\ 2 \\ 3 \end{smallmatrix} \right) ~|~ t\in \mathbb{C} \right\} \]

(c) Man prüft durch Einsetzen, ob $A \cdot x = b$ gilt.

(d) Die Lösungsmenge des inhomogenen LGS ist:
\[  \mathbb{L}= \left\{  \left(\begin{smallmatrix}
-t \\ 1+2t \\ 1-i+3t \end{smallmatrix} \right) ~|~ t\in \mathbb{C} \right\}.\]


    }


\lang{en}{   

(a) The coefficient matrix $A$ and the right side $b$ are as follows:
\[
A=
\left(\begin{smallmatrix}
    1 & 2 & -1 \\
    3 & 0 & 1 
\end{smallmatrix}\right), \quad
b=
\left(\begin{smallmatrix}
    1+i \\
    1-i 
\end{smallmatrix}\right)
\]

The augmented matrix is:

\[
(A \mid b) =
\left(\begin{smallmatrix}
    1 & 2 & -1 & | & 1+i \\
    3 & 0 & 1  & | & 1-i 
\end{smallmatrix}\right)
\]

(b) The augmented matrix for the corresponding homogeneous linear system is:
\[
(A \mid 0) =
\left(\begin{smallmatrix}
    1 & 2 & -1 & | & 0 \\
    3 & 0 & 1  & | & 0 
\end{smallmatrix}\right)
\]

%\[
%\begin{mtable}[\cellaligns{rcrcrcr}]
% x_{1}&+&2x_{2}&-&x_{3}&=&0\\
%3x_{1}&&&+& x_{3}&=&0\\
%\end{mtable} \]
The solution set of the homogeneous linear system is:
\[  \mathbb{L}_0= \left\{ t\cdot \left(\begin{smallmatrix}
-1 \\ 2 \\ 3 \end{smallmatrix} \right) ~|~ t\in \mathbb{C} \right\} \]

(c) We use substitution to check, whether $A \cdot x = b$ holds.

(d) The solution set of the nonhomogeneous linear system is:
\[  \mathbb{L}= \left\{  \left(\begin{smallmatrix}
-t \\ 1+2t \\ 1-i+3t \end{smallmatrix} \right) ~|~ t\in \mathbb{C} \right\}.\]


    }
  	 %------------------------------------END_STEP_X
 
  \end{incremental}
  %++++++++++++++++++++++++++++++++++++++++++++END_TAB_X




  %++++++++++++++++++++++++++++++++++++++++++START_TAB_X
  \tab{\lang{de}{    Lösung (a)    } \lang{en}{Solution (a)}}
  \begin{incremental}[\initialsteps{1}]
  
  	 %----------------------------------START_STEP_X
    \step 
    \lang{de}{  
Das gegebene inhomogene LGS liegt in der folgenden Form vor:
\[
\begin{mtable}[\cellaligns{cccccccc}]
 a_{11} x_1 & + & a_{12} x_2 & + & a_{13} x_3 & = & b_1 \\
 a_{21} x_1 & + & a_{22} x_2 & + & a_{23} x_3 & = & b_2
\end{mtable}
\]
Die Einträge $a_{ij}\in \mathbb{C}$ für $1 \leq i \leq 2$ und $1 \leq j \leq 3$
schreiben wir als Matrix $A$:
\[
A=
\left(\begin{smallmatrix}
    1 & 2 & -1 \\
    3 & 0 & 1 
\end{smallmatrix}\right)
\]
Die Einträge $b_1, b_2 \in \mathbb{C}$ stellen wir als rechte Seite $b$ dar:
\[
b=
\left(\begin{smallmatrix}
    1+i \\
    1-i 
\end{smallmatrix}\right)
\]
Nun kann die erweiterte Koeffizientenmatrix $(A \mid b)$ angegeben werden:
\[
\left(\begin{smallmatrix}
    1 & 2 & -1 & | & 1+i \\
    3 & 0 & 1  & | & 1-i 
\end{smallmatrix}\right)
\]

    }


\lang{en}{  
The given nonhomogeneous linear system is:
\[
\begin{mtable}[\cellaligns{cccccccc}]
 a_{11} x_1 & + & a_{12} x_2 & + & a_{13} x_3 & = & b_1 \\
 a_{21} x_1 & + & a_{22} x_2 & + & a_{23} x_3 & = & b_2
\end{mtable}
\]
The entries $a_{ij}\in \mathbb{C}$ for $1 \leq i \leq 2$ and $1 \leq j \leq 3$
can be written in a matrix $A$:
\[
A=
\left(\begin{smallmatrix}
    1 & 2 & -1 \\
    3 & 0 & 1 
\end{smallmatrix}\right)
\]
The entries $b_1, b_2 \in \mathbb{C}$ will be displayed in the right side $b$:
\[
b=
\left(\begin{smallmatrix}
    1+i \\
    1-i 
\end{smallmatrix}\right)
\]
We can now build the augmented matrix $(A \mid b)$:
\[
\left(\begin{smallmatrix}
    1 & 2 & -1 & | & 1+i \\
    3 & 0 & 1  & | & 1-i 
\end{smallmatrix}\right)
\]

    }
  	 %------------------------------------END_STEP_X

  \end{incremental}
  %++++++++++++++++++++++++++++++++++++++++++++END_TAB_X


  %++++++++++++++++++++++++++++++++++++++++++START_TAB_X
  \tab{\lang{de}{    Lösung (b)    } \lang{en}{Solution (b)}}
  \begin{incremental}[\initialsteps{1}]
  
  	 %----------------------------------START_STEP_X
    \step 
    \lang{en}{   We receive the corresponding homogeneous linear system by substituting the values on the right side $b$ by $0$:    
\[
\begin{mtable}[\cellaligns{rcrcrcl}]
 x_{1}&+&2x_{2}&-&x_{3}&=&0\\
3x_{1}&&&+& x_{3}&=&0.\\
\end{mtable} \]

Therefore the augmented matrix for the corresponding homogeneous linear system is:
\[
(A \mid 0) =
\left(\begin{smallmatrix}
    1 & 2 & -1 & | & 0 \\
    3 & 0 & 1  & | & 0 
\end{smallmatrix}\right)
\]

By adding $(-3)$-times the first equation to the second equation, we get:
\[
\left(\begin{smallmatrix}
    1 & 2 & -1 & | & 0 \\
    0 & -6 & 4 & | & 0 
\end{smallmatrix}\right)
\]

In the following example exercises of this chapters the transformations will be done systematically with the Gaussian elimination.
The steps will be noted next to the augmented matrix.

We now determine the solution from the remaining linear system:
\[
\begin{mtable}[\cellaligns{rcrcrcl}]
 x_{1}&+&2x_{2}&-&x_3&=&0\\
&-&6x_2&+& 4x_{3}&=&0\\
\end{mtable} \]

The variable $x_3$  never appears in the leading term and is because of that a free variable.
Set $t=x_3$ $(t \in \C)$, which transforms the second equation to:
\[
    -6x_2 + 4t = 0 ~\Leftrightarrow~ x_2 = \frac{2}{3} t
\]
We substitute the solution for $x_2$ and $x_3=t$ in the first equation. We have:
\[
    x_1 + 2 \left( \frac{2}{3} t \right) - t
    = 0 ~\Leftrightarrow~ x_1 = -\frac{1}{3} t
\]
The solution set is:
\[
    \mathbb{L}_0 = 
    \left\{ 
        t\cdot 
        \left(\begin{smallmatrix}
            -1/3 \\ 2/3 \\ 1 
          \end{smallmatrix}\right) 
        ~|~ 
        t \in \mathbb{C} \right\}
\]
Because $t$ is a free variable, the solution set can be indicated without any fractions:
\[
    \mathbb{L}_0 = 
    \left\{ 
        t\cdot 
        \left(\begin{smallmatrix}
            -1 \\ 2 \\ 3 
          \end{smallmatrix}\right) 
        ~|~ 
        t \in \mathbb{C} \right\}
\]

    }


\lang{de}{   Das zugehörige homogene LGS erhält man, indem man die Werte auf der rechten Seite $b$ alle durch $0$ ersetzt:    
\[
\begin{mtable}[\cellaligns{rcrcrcl}]
 x_{1}&+&2x_{2}&-&x_{3}&=&0\\
3x_{1}&&&+& x_{3}&=&0.\\
\end{mtable} \]

Damit ist die erweiterte Koeffizientenmatrix für das zugehörige homogene LGS:
\[
(A \mid 0) =
\left(\begin{smallmatrix}
    1 & 2 & -1 & | & 0 \\
    3 & 0 & 1  & | & 0 
\end{smallmatrix}\right)
\]

Addiert man das $(-3)$-fache der ersten Gleichung zur zweiten Gleichung, erhält man:
\[
\left(\begin{smallmatrix}
    1 & 2 & -1 & | & 0 \\
    0 & -6 & 4 & | & 0 
\end{smallmatrix}\right)
\]

In den weiteren Beispielaufgaben dieses Kapitels werden wir Umformungen
dieser Art systematisch durch das Gauß-Verfahren vornehmen 
und die Schritte neben die erweiterte Koeffizientenmatrix notieren.

Wir ermitteln nun die Lösung aus dem verbleibenden Gleichungssystem:
\[
\begin{mtable}[\cellaligns{rcrcrcl}]
 x_{1}&+&2x_{2}&-&x_3&=&0\\
&-&6x_2&+& 4x_{3}&=&0\\
\end{mtable} \]

Variable $x_3$ hat keinen Stufeneintrag und ist daher eine freie Variable.
Setzen wir $t=x_3$ $(t \in \C)$, so ergibt sich für die zweite Gleichung:
\[
    -6x_2 + 4t = 0 ~\Leftrightarrow~ x_2 = \frac{2}{3} t
\]
Setzt man nun die Lösung für $x_2$ (sowie $x_3=t$) in die erste Gleichung ein, erhält man:
\[
    x_1 + 2 \left( \frac{2}{3} t \right) - t
    = 0 ~\Leftrightarrow~ x_1 = -\frac{1}{3} t
\]
Die Lösungsmenge lautet:
\[
    \mathbb{L}_0 = 
    \left\{ 
        t\cdot 
        \left(\begin{smallmatrix}
            -1/3 \\ 2/3 \\ 1 
          \end{smallmatrix}\right) 
        ~|~ 
        t \in \mathbb{C} \right\}
\]
Da $t$ eine freie Variable ist, 
kann die Lösungsmenge alternativ auch ohne Brüche angegeben werden:
\[
    \mathbb{L}_0 = 
    \left\{ 
        t\cdot 
        \left(\begin{smallmatrix}
            -1 \\ 2 \\ 3 
          \end{smallmatrix}\right) 
        ~|~ 
        t \in \mathbb{C} \right\}
\]


%Man bringt $x_1$ auf die rechte Seite und teilt die erste Gleichung durch $2$:
%\[
%\begin{mtable}[\cellaligns{rcl}]
% x_{2}&=&-2x_1\\
%x_{3}&=&-3x_1.\\
%\end{mtable} \]
%Unter Betrachtung von $x_1 \in \C$ als freie Variable lautet die Lösungsmenge:
%\[  \mathbb{L}_0=\left\{  \left(\begin{smallmatrix}
%x_1 \\ -2x_1 \\ -3x_1 \end{smallmatrix} \right) ~|~ x_1\in \mathbb{C} \right\} \]
%Wir schreiben:
%\[
%\mathbb{L}_0 = \left\{ t\cdot \left(\begin{smallmatrix}
%1 \\ -2 \\ -3 \end{smallmatrix} \right) ~|~ t\in \mathbb{C} \right\}.\]

    }
  	 %------------------------------------END_STEP_X
 
  \end{incremental}
  %++++++++++++++++++++++++++++++++++++++++++++END_TAB_X

\tab{\lang{de}{    Lösung (c)    } \lang{en}{Solution (c)}}
  \begin{incremental}[\initialsteps{1}]
  
  	 %----------------------------------START_STEP_X
    \step 
    \lang{de}{   
Um zu sehen, dass das angegebene Tripel $(0;1;1-i)$ eine Lösung ist, muss man es lediglich in beide Gleichungen einsetzen:
\[ 
\begin{mtable}[\cellaligns{rcrcrcr}]
 0&+&2\cdot 1&-&(1-i)&=&1+i & \checkmark\\
3\cdot 0&&&+& (1-i)&=&1-i& \checkmark\\
\end{mtable} \]
Beide Gleichungen sind erfüllt, weshalb das Tripel $(0;1;1-i)$ eine Lösung ist.


Eleganter ist es jedoch zu prüfen, ob $A \cdot x = b$ gilt:
\[ 
A \cdot x =
\left(\begin{smallmatrix}
    1 & 2 & -1 \\
    3 & 0 & 1 
\end{smallmatrix}\right)
\cdot
\left(\begin{smallmatrix}
    0 \\
    1 \\
    1-i
\end{smallmatrix}\right)
=
\left(\begin{smallmatrix}
    1 \cdot 0 + 2 \cdot 1 + (-1) \cdot (1-i) \\
    3 \cdot 0 + 0 \cdot 1 + 1 \cdot (1-i)
\end{smallmatrix}\right)
=
\left(\begin{smallmatrix}
    1+i \\
    1-i
\end{smallmatrix}\right)
\]
Das Produkt entspricht genau der rechten Seite $b$:
\[
b=
\left(\begin{smallmatrix}
    1+i \\
    1-i 
\end{smallmatrix}\right)
\]
    }

\lang{en}{   
To check, whether the given triple $(0;1;1-i)$ is a solution or not, we simply need to insert it in both sides of the equation:
\[ 
\begin{mtable}[\cellaligns{rcrcrcr}]
 0&+&2\cdot 1&-&(1-i)&=&1+i & \checkmark\\
3\cdot 0&&&+& (1-i)&=&1-i& \checkmark\\
\end{mtable} \]
Both equations are fulfilled, which is why the triple $(0;1;1-i)$ is a solution.


However, it is more elegant to check whether $A \cdot x = b$ holds:
\[ 
A \cdot x =
\left(\begin{smallmatrix}
    1 & 2 & -1 \\
    3 & 0 & 1 
\end{smallmatrix}\right)
\cdot
\left(\begin{smallmatrix}
    0 \\
    1 \\
    1-i
\end{smallmatrix}\right)
=
\left(\begin{smallmatrix}
    1 \cdot 0 + 2 \cdot 1 + (-1) \cdot (1-i) \\
    3 \cdot 0 + 0 \cdot 1 + 1 \cdot (1-i)
\end{smallmatrix}\right)
=
\left(\begin{smallmatrix}
    1+i \\
    1-i
\end{smallmatrix}\right)
\]
The product equals exactly the right side $b$:
\[
b=
\left(\begin{smallmatrix}
    1+i \\
    1-i 
\end{smallmatrix}\right)
\]
    }
  	 %------------------------------------END_STEP_X
 
  \end{incremental}

  %++++++++++++++++++++++++++++++++++++++++++START_TAB_X
  \tab{\lang{de}{    Lösung (d)    } \lang{en}{Solution (d)}}
  \begin{incremental}[\initialsteps{1}]
  
  	 %----------------------------------START_STEP_X
    \step 
    \lang{de}{   
        Die Lösungsmenge des inhomogenen LGS erhält man, 
        indem man zur speziellen Lösung
        $(0;1;1-i)$ aus Aufgabenteil (c) 
        alle Lösungen des zugehörigen homogenen LGS aus (b) addiert:
\begin{eqnarray*}
 \mathbb{L} &=& \left\{  \left(\begin{smallmatrix}
0 \\ 1 \\ 1-i \end{smallmatrix} \right) + \left(\begin{smallmatrix}
y_1 \\ y_2 \\ y_3 \end{smallmatrix} \right) ~|~ \left(\begin{smallmatrix}
y_1 \\ y_2 \\ y_3 \end{smallmatrix} \right)\in \mathbb{L}_0 \right\} \\
&=& \left\{  \left(\begin{smallmatrix}
0 \\ 1 \\ 1-i \end{smallmatrix} \right) + t\cdot \left(\begin{smallmatrix}
-1 \\ 2 \\ 3 \end{smallmatrix} \right) ~|~  t\in \mathbb{C} \right\}  \\
&=& \left\{  \left(\begin{smallmatrix}
-t \\ 1+2t \\ 1-i+3t \end{smallmatrix} \right) ~|~ t\in \mathbb{C} \right\}.
\end{eqnarray*}
    }

\lang{en}{ 
We receive the solution set of the nonhomogeneous linear system, by adding $(0;1;1-i)$  from (b) to all the solutions
of the corresponding homogeneous linear system from (b):
\begin{eqnarray*}
 \mathbb{L} &=& \left\{  \left(\begin{smallmatrix}
0 \\ 1 \\ 1-i \end{smallmatrix} \right) + \left(\begin{smallmatrix}
y_1 \\ y_2 \\ y_3 \end{smallmatrix} \right) ~|~ \left(\begin{smallmatrix}
y_1 \\ y_2 \\ y_3 \end{smallmatrix} \right)\in \mathbb{L}_0 \right\} \\
&=& \left\{  \left(\begin{smallmatrix}
0 \\ 1 \\ 1-i \end{smallmatrix} \right) + t\cdot \left(\begin{smallmatrix}
-1 \\ 2 \\ 3 \end{smallmatrix} \right) ~|~  t\in \mathbb{C} \right\}  \\
&=& \left\{  \left(\begin{smallmatrix}
-t \\ 1+2t \\ 1-i+3t \end{smallmatrix} \right) ~|~ t\in \mathbb{C} \right\}.
\end{eqnarray*}
    }
  	 %------------------------------------END_STEP_X
 
  \end{incremental}
  %++++++++++++++++++++++++++++++++++++++++++++END_TAB_X


%#############################################################ENDE
\end{tabs*}
\end{content}