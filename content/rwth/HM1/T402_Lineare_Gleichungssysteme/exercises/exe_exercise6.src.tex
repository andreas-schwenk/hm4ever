\documentclass{mumie.element.exercise}
%$Id$
\begin{metainfo}
  \name{
    \lang{de}{Ü06: Elementarmatrizen}
    \lang{en}{Ex 06: Elementary matrices}
  }
  \begin{description} 
 This work is licensed under the Creative Commons License Attribution 4.0 International (CC-BY 4.0)   
 https://creativecommons.org/licenses/by/4.0/legalcode 

    \lang{de}{}
    \lang{en}{}
  \end{description}
  \begin{components}
  \end{components}
  \begin{links}
  \end{links}
  \creategeneric
\end{metainfo}
\begin{content}
\usepackage{mumie.ombplus}

\title{\lang{de}{Ü06: Elementarmatrizen} \lang{en}{Ex06: Elementary matrices}}

\begin{block}[annotation]
  Im Ticket-System: \href{http://team.mumie.net/issues/11516}{Ticket 11516}
\end{block}

%######################################################FRAGE_TEXT
\lang{de}{In dieser Aufgabe sollen die Zeilenumformungen des Gauß-Verfahrens
als Elementarmatrizen angegeben werden.}
\lang{en}{In this exercise the row operations of Gaussian elimination should be expressed as elementary matrices.}

\lang{de}{
Gegeben sei die Matrix 
%$A \in M(2,3;\R)$ mit 
$A= \left(\begin{smallmatrix}
2 & 2 & -1 \\ 3 & 1 & 4 \end{smallmatrix} \right) $ über $\R$.
Die folgenden Zeilenumformungen bringen die Matrix auf die
reduzierte Zeilenstufenform:}
\lang{en}{
Given a matrix 
%$A \in M(2,3;\R)$ mit 
$A= \left(\begin{smallmatrix}
2 & 2 & -1 \\ 3 & 1 & 4 \end{smallmatrix} \right) $ over $\R$.
The following row operations transform the matrix into reduced row echelon form:}

\lang{de}{Matrix $A$:}
\lang{en}{Matrix $A$:}
\[
\begin{pmatrix}
    2 & 2 & -1 \\
    3 & 1 & 4\\
\end{pmatrix}
~
\begin{matrix}
     \\
    | (-\frac{3}{2}) \cdot (I) + (II) \\
\end{matrix}
\]

\lang{de}{Schritt 1:}
\lang{en}{Step 1:}
\[
\rightsquigarrow~~
\begin{pmatrix}
    2 & 2 & -1 \\
    0 & -2 & \frac{11}{2}\\
\end{pmatrix}
~
\begin{matrix}
    | 1 \cdot (II) + (I) \\
    \\
\end{matrix}
\]

\lang{de}{Schritt 2:}
\lang{en}{Step 2:}
\[
\rightsquigarrow~~
\begin{pmatrix}
    2 & 0 & \frac{9}{2} \\
    0 & -2 & \frac{11}{2}\\
\end{pmatrix}
~
\begin{matrix}
    | \frac{1}{2} \cdot (I) \\
    \\
\end{matrix}
\]

\lang{de}{Schritt 3:}
\lang{en}{Step 3:}
\[
\rightsquigarrow~~
\begin{pmatrix}
    1 & 0 & \frac{9}{4} \\
    0 & -2 & \frac{11}{2}\\
\end{pmatrix}
~
\begin{matrix}
    \\
    | (-\frac{1}{2}) \cdot (II) \\
\end{matrix}
\]

\lang{de}{Schritt 4:}
\lang{en}{Step 4:}
\[
\rightsquigarrow~~
\begin{pmatrix}
    1 & 0 & \frac{9}{4} \\
    0 & 1 & -\frac{11}{4}\\
\end{pmatrix}
\]

%\begin{eqnarray*}
%&& \begin{pmatrix}  2 & 2 & -1 \\ 3 & 1 & 4\end{pmatrix} \begin{matrix}  \phantom{1}\\ /  -\frac{3}{2}\cdot \text{(I)}  \end{matrix} \\
%& \rightsquigarrow &
%\begin{pmatrix}  2 & 2 & -1 \\ 0 & -2 & \frac{11}{2} \end{pmatrix} \begin{matrix}  / +\text{(II)}   \\ \phantom{1}\end{matrix} \\ 
%& \rightsquigarrow &
%\begin{pmatrix}  2 & 0 & \frac{9}{2} \\ 0 & -2 & \frac{11}{2} \end{pmatrix} \begin{matrix}  / \cdot \frac{1}{2}   \\ \phantom{1} \end{matrix}\\
%& \rightsquigarrow &
%\begin{pmatrix}  1 & 0 & \frac{9}{4} \\ 0 & -2 & \frac{11}{2} \end{pmatrix} \begin{matrix} \phantom{1}\\ / \cdot (-\frac{1}{2})  \end{matrix} \\
%& \rightsquigarrow &
%\begin{pmatrix}  1 & 0 & \frac{9}{4} \\ 0 & 1 & -\frac{11}{4} \end{pmatrix}
%\end{eqnarray*}

\lang{de}{
\begin{enumerate}[(a)]
\item (a) Geben Sie Elementarmatrizen für alle vier Umformungsschritte an.

\item (b) Bestimmen Sie aus der Lösung von (a) eine Matrix $B$, sodass Folgendes gilt:
\[
    \begin{pmatrix}  1 & 0 & \frac{9}{4} \\ 0 & 1 & -\frac{11}{4} \end{pmatrix} = B \cdot A = B \cdot \begin{pmatrix}  2 & 2 & -1 \\ 3 & 1 & 4\end{pmatrix}
\]

\item (c) Bestimmen Sie weiterhin eine Matrix $C$, für die gilt:
\[
    A = 
    \begin{pmatrix}  2 & 2 & -1 \\ 3 & 1 & 4\end{pmatrix}=C\cdot \begin{pmatrix}  1 & 0 & \frac{9}{4} \\ 0 & 1 & -\frac{11}{4} \end{pmatrix}
\]
\end{enumerate}}

\lang{en}{
\begin{enumerate}[(a)]
\item (a) Give the elementary matrices for all four transformation steps.

\item (b) Determin a matrix $B$, using (a), such that holds:
\[
    \begin{pmatrix}  1 & 0 & \frac{9}{4} \\ 0 & 1 & -\frac{11}{4} \end{pmatrix} = B \cdot A = B \cdot \begin{pmatrix}  2 & 2 & -1 \\ 3 & 1 & 4\end{pmatrix}
\]

\item (c) Determine a matric $C$, that holds:
\[
    A = 
    \begin{pmatrix}  2 & 2 & -1 \\ 3 & 1 & 4\end{pmatrix}=C\cdot \begin{pmatrix}  1 & 0 & \frac{9}{4} \\ 0 & 1 & -\frac{11}{4} \end{pmatrix}
\]
\end{enumerate}}

\lang{de}{
\textit{Bemerkung:} In dieser Aufgabe sind die in (b) und (c) gesuchten Matrizen $B$ und $C$
eindeutig durch die angegebene Bedingung festgelegt.
Im Allgemeinen (d.\,h. bei anderen Ausgangsmatrizen) könnte es jedoch mehrere solcher Matrizen geben.}
\lang{en}{
\textit{Note:} The matrices $B$ and $C$ are unambiguously determined by the given conditions. In general (i.e. for other matrices) 
there may be several matrices.}

%##################################################ANTWORTEN_TEXT
\begin{tabs*}[\initialtab{0}\class{exercise}]

  \tab{\lang{de}{    Antwort    } \lang{en}{Answer}}
  \begin{incremental}[\initialsteps{1}]
  
  	 %----------------------------------START_STEP_X
    \step 
    \lang{de}{   
\begin{enumerate}
\item[(a)] Die Elementarmatrizen sind (in dieser Reihenfolge):
\[  \begin{pmatrix} 1 & 0 \\ -\frac{3}{2} & 1 \end{pmatrix}, \quad \begin{pmatrix} 1 & 1 \\ 0 & 1 \end{pmatrix},\quad \begin{pmatrix} \frac{1}{2} & 0 \\ 0 & 1 \end{pmatrix},\quad \begin{pmatrix} 1 & 0 \\ 0 & -\frac{1}{2} \end{pmatrix}.\]
\item[(b)] Multiplikation der Elementarmatrizen in umgekehrter Reihenfolge zu (a) ergibt:\\
\[ B= \begin{pmatrix} -\frac{1}{4} & \frac{1}{2} \\ \frac{3}{4} & -\frac{1}{2} \end{pmatrix} \]
%\item[c)] Multiplikation der Elementarmatrizen in der Reihenfolge aus a) ergibt:\\
\item[(c)] Der Lösungsweg ist wie in (a) und (b), 
jedoch mit den gegenteiligen Umformungsschritten.
Man erhält:
\[ C= \begin{pmatrix}2 & 2 \\ 3 & 1\end{pmatrix} \]
\end{enumerate}    }

\lang{en}{   
\begin{enumerate}
\item[(a)] The elementary matrices are (in this order):
\[  \begin{pmatrix} 1 & 0 \\ -\frac{3}{2} & 1 \end{pmatrix}, \quad \begin{pmatrix} 1 & 1 \\ 0 & 1 \end{pmatrix},\quad \begin{pmatrix} \frac{1}{2} & 0 \\ 0 & 1 \end{pmatrix},\quad \begin{pmatrix} 1 & 0 \\ 0 & -\frac{1}{2} \end{pmatrix}.\]
\item[(b)] Multiplication of the elementary matrices in reversed order (compared to (a)) yields:\\
\[ B= \begin{pmatrix} -\frac{1}{4} & \frac{1}{2} \\ \frac{3}{4} & -\frac{1}{2} \end{pmatrix} \]
%\item[c)] Multiplikation der Elementarmatrizen in der Reihenfolge aus a) ergibt:\\
\item[(c)] We receive $C$ by going on as done in (a) and (b), but with the inversed transformationsteps.
We get:
\[ C= \begin{pmatrix}2 & 2 \\ 3 & 1\end{pmatrix} \]
\end{enumerate}    }
  	 %------------------------------------END_STEP_X
 
  \end{incremental}
  %++++++++++++++++++++++++++++++++++++++++++START_TAB_X
  \tab{\lang{de}{    Lösung (a)    } \lang{en}{Solution (a)}}
  \begin{incremental}[\initialsteps{1}]
  
  	 %----------------------------------START_STEP_X
    \step\lang{de}{
    Im ersten Umformungsschritt wird das $(-\frac{3}{2})$-fache der
    ersten Zeile zur zweiten Zeile hinzuaddiert.
    
    Wir definieren nun eine Matrix $A_{ij}(c)$, um das $c$-fache 
der $j$-ten Zeile zur $i$-ten Zeile hinzuzuaddieren.
Die Matrix $A_{ij}(c)$ ist wie folgt aufgebaut:
Die Einträge auf der Hauptdiagonalen sind 1.
An der Stelle $(i,j)$ ist der Eintrag $c$.
Alle anderen Einträge sind 0.
Hier sind $c=-\frac{3}{2}$, $j=1$ und $i=2$.
Die zur ersten Umformung gehörende Matrix ist daher:}

\lang{en}{
In the first step we add $(-\frac{3}{2})$-times the first row to the second.
Now we define a matrix $A_{ij}(c)$ to add $c$-times the $j$th row to the $i$th row. The matrix $A_{ij}(c)$ is composed as follows:
The entries on the main diagonal are all 1.
The entry $a_{i,j}$ is $c$.
The other entries are 0.
Here we have $c=-\frac{3}{2}$, $j=1$ and $i=2$.
The matrix for the first transformation is then:}
\[  A_{21}(-\frac{3}{2})=\begin{pmatrix} 1 & 0 \\ -\frac{3}{2} & 1 \end{pmatrix}. \]

    \step\lang{de}{
    Im zweiten Umformungsschritt wird das $1$-fache der zweiten Zeile zur ersten Zeile hinzuaddiert.    
Entsprechend definieren wir:}
\lang{en}{In the second transformatio stept we add ($1$-times) the second row to the first row.
Analogously we define:}
\[ A_{12}(1)= \begin{pmatrix} 1 & 1 \\ 0 & 1 \end{pmatrix}.\]

    \step\lang{de}{
    Im dritten Schritt wird die erste Zeile mit dem Faktor $\frac{1}{2}$ multipliziert.
    Die Matrix, mit der man die $i$-te Zeile mit $c$ multipliziert, bezeichnen wir mit
$M_i(c)$.
    Der $i$-te Diagonaleintrag dieser Matrix ist gleich $c$.
    Alle anderen Diagonaleintäge sind 1.
    Die Einträge, die nicht auf der Hauptdiagonalen liegen, sind 0.
    Hier ist $c=\frac{1}{2}$ und wir erhalten:}

    \lang{en}{
    In the third step we multiply the first row with $\frac{1}{2}$.
    The matrix, which multiplies the $i$th row with $c$, is called $M_i(c)$.
    The $i$th diagonal entry equals $c$. The other diagonal entries are $1$.
    All the other entries are $0$.
    Here we have $c=\frac{1}{2}$, which gives us:}
\[ M_1(\frac{1}{2})= \begin{pmatrix} \frac{1}{2} & 0 \\ 0 & 1 \end{pmatrix}. \]

    \step\lang{de}{
    Im letzten Schritt wird die zweite Zeile mit einem Faktor multipliziert.
    Wir verfahren wie im letzen Schritt und erhalten die Matrix:}
    \lang{en}{
    In the last step the second row is multiplied with a factor.
    We proceed, like we have done in the last step and we get the matrix:}
\[ M_2(-\frac{1}{2})= \begin{pmatrix} 1 & 0 \\ 0 & -\frac{1}{2} \end{pmatrix}. \]

  	 %------------------------------------END_STEP_X
 
  \end{incremental}
  %++++++++++++++++++++++++++++++++++++++++++++END_TAB_X


  %++++++++++++++++++++++++++++++++++++++++++START_TAB_X
  \tab{\lang{de}{    Lösung (b)    } \lang{en}{Solution (b)}}
  \begin{incremental}[\initialsteps{1}]
  
  	 %----------------------------------START_STEP_X
    \step 
    \lang{de}{   
Durch die Umformungen und die in (a) bestimmten zugehörigen Matrizen erhält man:}
\lang{en}{Through the transformations and the matrices determined in (a) we get:}


\begin{eqnarray*}

A' = A_{21}(-\frac{3}{2}) \cdot A =
\begin{pmatrix} 1 & 0 \\ -\frac{3}{2} & 1 \end{pmatrix}\cdot \begin{pmatrix}  2 & 2 & -1 \\ 3 & 1 & 4\end{pmatrix}
&=& 
\begin{pmatrix}  2 & 2 & -1 \\ 0 & -2 & \frac{11}{2} \end{pmatrix} \\

A'' = A_{12}(1) \cdot A' = 
\begin{pmatrix} 1 & 1 \\ 0 & 1 \end{pmatrix}\cdot \begin{pmatrix}  2 & 2 & -1 \\ 0 & -2 & \frac{11}{2} \end{pmatrix}
&=& 
\begin{pmatrix}  2 & 0 & \frac{9}{2} \\ 0 & -2 & \frac{11}{2} \end{pmatrix} \\

A''' = M_1(\frac{1}{2}) \cdot A'' =
\begin{pmatrix} \frac{1}{2} & 0 \\ 0 & 1 \end{pmatrix}\cdot \begin{pmatrix}  2 & 0 & \frac{9}{2} \\ 0 & -2 & \frac{11}{2} \end{pmatrix}
&=& 
\begin{pmatrix}  1 & 0 & \frac{9}{4} \\ 0 & -2 & \frac{11}{2} \end{pmatrix} \\

M_2(-\frac{1}{2}) \cdot A''' =
\begin{pmatrix} 1 & 0 \\ 0 & -\frac{1}{2} \end{pmatrix}\cdot  \begin{pmatrix}  1 & 0 & \frac{9}{4} \\ 0 & -2 & \frac{11}{2} \end{pmatrix}
&=&
\begin{pmatrix}  1 & 0 & \frac{9}{4} \\ 0 & 1 & -\frac{11}{4} \end{pmatrix} \\

\end{eqnarray*}

    \lang{de}{   
Dies führt sukzessive von unten nach oben eingesetzt zu}
\lang{en}{This leads successively from the bottom to the top to}

\[
    B = M_2(-\frac{1}{2}) \cdot M_1(\frac{1}{2}) \cdot A_{12}(1) \cdot A_{21}(-\frac{3}{2}).
\]


%\begin{eqnarray*}
%\begin{pmatrix}  1 & 0 & \frac{9}{4} \\ 0 & 1 & -\frac{11}{4} \end{pmatrix} &=&
% \begin{pmatrix} 1 & 0 \\ 0 & -\frac{1}{2} \end{pmatrix}\cdot \begin{pmatrix}  1 & 0 & \frac{9}{4} \\ 0 & -2 & \frac{11}{2} \end{pmatrix} =  \begin{pmatrix} 1 & 0 \\ 0 & -\frac{1}{2} \end{pmatrix}\cdot 
% \begin{pmatrix} \frac{1}{2} & 0 \\ 0 & 1 \end{pmatrix}\cdot \begin{pmatrix}  2 & 0 & \frac{9}{2} \\ 0 & -2 & \frac{11}{2} \end{pmatrix}, \\
% &=&  \begin{pmatrix} 1 & 0 \\ 0 & -\frac{1}{2} \end{pmatrix}\cdot 
% \begin{pmatrix} \frac{1}{2} & 0 \\ 0 & 1 \end{pmatrix}\cdot \begin{pmatrix} 1 & 1 \\ 0 & 1 \end{pmatrix}\cdot
%\begin{pmatrix}  2 & 2 & -1 \\ 0 & -2 & \frac{11}{2} \end{pmatrix} , \\
% &=&  \begin{pmatrix} 1 & 0 \\ 0 & -\frac{1}{2} \end{pmatrix}\cdot 
% \begin{pmatrix} \frac{1}{2} & 0 \\ 0 & 1 \end{pmatrix}\cdot \begin{pmatrix} 1 & 1 \\ 0 & 1 \end{pmatrix}\cdot
% \begin{pmatrix} 1 & 0 \\ -\frac{3}{2} & 1 \end{pmatrix}\cdot  \begin{pmatrix}  2 & 2 & -1 \\ 3 & 1 & 4\end{pmatrix}.
%\end{eqnarray*}

    \lang{de}{Als Matrix $B$ kann also das Produkt der Elementarmatrizen gewählt werden:}
    \lang{en}{Matrix $B$ can be chosen as the product of the elementary matrices:}

\begin{eqnarray*}
 B&=& \begin{pmatrix} 1 & 0 \\ 0 & -\frac{1}{2} \end{pmatrix}\cdot 
 \begin{pmatrix} \frac{1}{2} & 0 \\ 0 & 1 \end{pmatrix}\cdot \begin{pmatrix} 1 & 1 \\ 0 & 1 \end{pmatrix}\cdot
 \begin{pmatrix} 1 & 0 \\ -\frac{3}{2} & 1 \end{pmatrix} \\
  &=& \qquad \begin{pmatrix} \frac{1}{2} & 0 \\ 0 & -\frac{1}{2} \end{pmatrix}\quad \cdot \quad \begin{pmatrix} -\frac{1}{2} & 1 \\ -\frac{3}{2} & 1 \end{pmatrix}\\
  &=&  \begin{pmatrix} -\frac{1}{4} & \frac{1}{2} \\ \frac{3}{4} & -\frac{1}{2} \end{pmatrix}.
\end{eqnarray*}

    \lang{de}{Wir prüfen:}
    \lang{en}{We check:}
\[
B \cdot A = 
\begin{pmatrix} -\frac{1}{4} & \frac{1}{2} \\ \frac{3}{4} & -\frac{1}{2} \end{pmatrix}
\cdot
\begin{pmatrix}  2 & 2 & -1 \\ 3 & 1 & 4\end{pmatrix}
= 
\begin{pmatrix}  1 & 0 & \frac{9}{4} \\ 0 & 1 & -\frac{11}{4} \end{pmatrix}
\]


    
  	 %------------------------------------END_STEP_X
 
  \end{incremental}
  %++++++++++++++++++++++++++++++++++++++++++++END_TAB_X


  %++++++++++++++++++++++++++++++++++++++++++START_TAB_X
  \tab{\lang{de}{    Lösung (c)    } \lang{en}{Solution (c)}}
  \begin{incremental}[\initialsteps{1}]
  
  	 %----------------------------------START_STEP_X
    \step 
    \lang{de}{   
Die Bestimmung von $C$ kann man entsprechend zur Bestimmung von $B$ durchführen.
Man führt die Umformungsschritte in umgekehrter Reihenfolge durch.}
\lang{en}{
The Determination of $C$ can be done analogously to the determination of $B$.
Therefor the transformation steps need to be done in reversed order.}

    \lang{de}{Die Elementarmatrizen zu den umgekehrten Umformungen sind von unten nach oben:}
    \lang{en}{The elementary matrices to the reversed transformations are vom the bottom to the top:}

\[  M_2(-2)= \begin{pmatrix} 1 & 0\\ 0 & -2 \end{pmatrix},\quad M_1(2)= \begin{pmatrix} 2 & 0\\ 0 & 1 \end{pmatrix}, \quad A_{12}(-1)= \begin{pmatrix} 1 & -1\\  0 & 1 \end{pmatrix},  \quad A_{21}(\frac{3}{2})= \begin{pmatrix} 1 & 0\\ \frac{3}{2} & 1 \end{pmatrix}.\]

    \lang{de}{Entsprechend der Überlegung in (b) ist dann:}
    \lang{en}{Corresponding to (b) we have:}
\begin{eqnarray*}
  \begin{pmatrix}  2 & 2 & -1 \\ 3 & 1 & 4\end{pmatrix}
&=& A_{21}(\frac{3}{2})\cdot A_{12}(-1)\cdot M_1(2)\cdot  M_2(-2)\cdot \begin{pmatrix}  1 & 0 & \frac{9}{4} \\ 0 & 1 & -\frac{11}{4} \end{pmatrix}\\
&=& \begin{pmatrix} 1 & 0\\ \frac{3}{2} & 1 \end{pmatrix}\cdot \begin{pmatrix} 1 & -1\\  0 & 1 \end{pmatrix}\cdot \begin{pmatrix} 2 & 0\\ 0 & 1 \end{pmatrix}\cdot  \begin{pmatrix} 1 & 0\\ 0 & -2 \end{pmatrix}
\cdot \begin{pmatrix}  1 & 0 & \frac{9}{4} \\ 0 & 1 & -\frac{11}{4} \end{pmatrix}
\end{eqnarray*}

    \lang{de}{Man erhält letztendlich die gewünschte Matrix $C$ wie folgt:}
    \lang{en}{We receive $C$ as follows:}
\begin{eqnarray*}
C&=&  \begin{pmatrix} 1 & 0\\ \frac{3}{2} & 1 \end{pmatrix}\cdot \begin{pmatrix} 1 & -1\\  0 & 1 \end{pmatrix}\cdot \begin{pmatrix} 2 & 0\\ 0 & 1 \end{pmatrix}\cdot  \begin{pmatrix} 1 & 0\\ 0 & -2 \end{pmatrix} \\
  &=& \qquad \begin{pmatrix} 1 & -1 \\ \frac{3}{2} & -\frac{1}{2} \end{pmatrix}\quad \cdot \quad \begin{pmatrix} 2 & 0 \\ 0 & -2 \end{pmatrix}\\
  &=&  \begin{pmatrix} 2 & 2 \\ 3 & 1 \end{pmatrix}.
\end{eqnarray*}    

    \lang{de}{Vorausblick: Im nächsten Kapitel (quadratische Matrizen) werden wir feststellen,
    dass wir für die Lösung dieses Aufgabenteils alternativ auch die \textit{inverse Matrix}
    zur Matrix des Aufgabenteils (b) bestimmen können.}
    \lang{en}{Preview: In the next chapter (square matrix), we well see, that we may determine the \textit{inverse}
    of the matrix in (b) to solve this exercise part.}


  	 %------------------------------------END_STEP_X
 
  \end{incremental}
  %++++++++++++++++++++++++++++++++++++++++++++END_TAB_X


%#############################################################ENDE
\end{tabs*}
\end{content}