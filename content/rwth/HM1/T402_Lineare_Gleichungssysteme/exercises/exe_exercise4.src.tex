\documentclass{mumie.element.exercise}
%$Id$
\begin{metainfo}
  \name{
    \lang{de}{Ü04: Parabelgleichung}
    \lang{en}{Exercise 4}
  }
  \begin{description} 
 This work is licensed under the Creative Commons License Attribution 4.0 International (CC-BY 4.0)   
 https://creativecommons.org/licenses/by/4.0/legalcode 

    \lang{de}{Hier die Beschreibung}
    \lang{en}{}
  \end{description}
  \begin{components}
  \end{components}
  \begin{links}
  \end{links}
  \creategeneric
\end{metainfo}
\begin{content}
\title{
  \lang{de}{Ü04: Parabelgleichung}
  \lang{en}{Exercise 4}
}

\begin{block}[annotation]
	Matritzen, Lineare Gleichungen: Übungen
\end{block}
\begin{block}[annotation]
  Im Ticket-System: \href{http://team.mumie.net/issues/11291}{Ticket 11291}
\end{block}

\lang{de}{
    \begin{itemize}
\item[(a)]
Durch die Punkte $P=(-1; -4)$, $Q=(1; 6)$ und $R=(3; 0)$ verläuft genau eine Parabel. Stellen Sie ein lineares Gleichungssystem
        für die Koeffizienten $a$, $b$ und $c$ der Parabelgleichung $y=ax^2+bx+c$ auf und bestimmen Sie dessen Lösungsmenge und damit die Gleichung der
        Parabel, die durch die Punkte $P$, $Q$ und $R$ verläuft.
\item[(b)]
Für welches Polynom $p$ dritten Grades gilt
$p(-1)=-3$, $p(0)=3$, $p(1)=1$ und $p(2)=3$?
\end{itemize}
}

\lang{en}{
    \begin{itemize}
\item[(a)]
There exists exactly one parabola, that goes trough the points $P=(-1; -4)$, $Q=(1; 6)$ and $R=(3; 0)$. Create a linear system for the coefficients
$a$, $b$ und $c$ of the parabola $y=ax^2+bx+c$ . Determine the solution set of the linear system and hence the equation of the parabola,
that goes through the points.
\item[(b)]
For which third grade polynomial $p$ holds
$p(-1)=-3$, $p(0)=3$, $p(1)=1$ und $p(2)=3$?
\end{itemize}
}
%\lang{en}{Exactly one parabola goes through the points $P=(-1, -4)$, $Q=(1, 6)$, and $R=(3, 0)$. Create a linear system for the coefficients $a$, $b$, and $c$ of the parabola
        %$y=ax^2+bx+c$. Determine the solution set of the system and hence the equation of the parabola that goes through the points $P$, $Q$, and $R$.
%}

\begin{tabs*}[\initialtab{0}\class{exercise}]
  \tab{
  \lang{de}{Antwort}
  \lang{en}{Answer}
  }
\lang{de}{
\begin{table}[\class{items}]
(a)
Das lineare Gleichungssystem für die Koeffizienten lautet:
\begin{displaymath}
\begin{mtable}[\cellaligns{ccrcrcrcr}]
(I)&\qquad&a&-&b&+&c&=&-4\\
(II)&&a&+&b&+&c&=&6\\
(III)&&9a&+&3b&+&c&=&0
\end{mtable}
\end{displaymath}
Die Lösungsmenge des linearen Gleichungssystems ist:
\begin{displaymath}
\mathbb{L}=\{ \begin{pmatrix} -2\\ 5\\ 3 \end{pmatrix} \}
\end{displaymath}
Das lineare Gleichungssystem hat also die eindeutige Lösung
$a=-2$, $b=5$ und $c=3$.\\
Die Gleichung der Parabel, die durch die Punkte $P$, $Q$ und $R$ verläuft,
ist somit $y=ax^2+bx+c=-2x^2+5x+3$.\\
\\
(b) 
\[p(x)=2x^3-4x^2+3\]
\end{table}}

\lang{en}{
\begin{table}[\class{items}]
(a) The linear system for the coefficients is
\begin{displaymath}
\begin{mtable}[\cellaligns{ccrcrcrcr}]
(I)&\qquad&a&-&b&+&c&=&-4\\
(II)&&a&+&b&+&c&=&6\\
(III)&&9a&+&3b&+&c&=&0
\end{mtable}
\end{displaymath}
and its solution set is
\begin{displaymath}
\mathbb{L}=\{(-2, 5, 3)\}.
\end{displaymath}
The unique solution of the linear system for $a$, $b$, and $c$ is $a=-2$, $b=5$ and $c=3$.\\
(b)
$p(x)=-2x^2+5x+3$
\end{table}}


\tab{
  \lang{de}{Lösung (a)}
  \lang{en}{Solution (a)}
  }
  
  \begin{incremental}[\initialsteps{1}]
    \step 
    \lang{de}{Da die drei Punkte $P$, $Q$ und $R$ auf einer Parabel mit der Gleichung $y=ax^2+bx+c$ liegen sollen, müssen ihre Koordinaten diese Gleichung
    jeweils erfüllen.}
    \lang{en}{Since the three points $P$, $Q$ and $R$ lie on a parabola with the equation $y=ax^2+bx+c$, their coordinates have to fulfill the equation.}
    
     
    \step \lang{de}{Setzt man $P=(-1;-4)$ in die Parabelgleichung ein, also $x=-1$ und $y=-4$, so folgt $-4=a \cdot (-1)^2 + b \cdot (-1) + c$.}
    \lang{en}{If we subsitute $P=(-1;-4)$ (i.e. $x=-1$ and $y=-4$) into the parabola, we get $-4=a \cdot (-1)^2 + b \cdot (-1) + c$. }
    
    \lang{de}{Analog folgt durch Einsetzen von $Q=(1;6)$ die Gleichung $6=a \cdot 1^2 + b \cdot 1 +c$.}
    \lang{de}{Schließlich erhält man durch Einsetzen von $R=(3;0)$ die dritte Gleichung $0=a \cdot 3^2 + b \cdot 3 + c $.}
    \lang{en}{Analogously we can substitute in $Q=(1;6)$ to $6=a \cdot 1^2 + b \cdot 1 +c$ and $R=(3;0)$ to get $0=a \cdot 3^2 + b \cdot 3 + c $.}
    
    \lang{de}{Dies führt auf das folgende lineare Gleichungssystem für die Koeffizienten $a$, $b$ und $c$:}
    \lang{en}{This leads to the the following linear system for the coefficients $a$, $b$, and $c$:}
    \begin{displaymath}
\begin{mtable}[\cellaligns{ccrcrcrcr}]
\text{(I)}&\qquad&a&-&b&+&c&=&-4\phantom{.}\\
\text{(II)}&&a&+&b&+&c&=&6\phantom{.}\\
\text{(III)}&&9a&+&3b&+&c&=&0\lang{de}{\phantom{.}}\lang{en}{\phantom{.}}
\end{mtable}
\end{displaymath}
    
    \step \lang{de}{Die Lösungsmenge dieses linearen Gleichungssystems erhält man nun durch Anwendung des Gauß-Verfahrens.}
    \lang{en}{The solution set of the linear system is obtained via Gaussian Elimination.}


    \lang{de}{
    Im Vorlesungsteil haben wir die Unbekannten als $x_i$ mit $1 \leq i \leq n$ bezeichnet.
    Beispielsweise können wir in dieser Aufgabe $x_1=a$, $x_2=b$ und $x_3=c$ setzen.
    Dann lautet die erweiterte Koeffizientenmatrix wie folgt:
    }
    \lang{en}{In the lecture part, we named the unknowns as $x_i$ with $1 \leq i \leq n$.
    In this exercise, we set $x_1=a$, $x_2=b$ and $x_3=c$.
    Then, the augmented matrix is as follows:
    }
\[
(A \mid b) ~=~
\begin{pmatrix}
    1 & -1 & 1 &|& -4 \\
    1 &  1 & 1 &|&  6 \\
    9 &  3 & 1 &|&  0
\end{pmatrix}
\]
    
    \step \lang{de}{Für eine möglichst einfache Rechnung ist es in dieser Aufgabe jedoch geschickter,
    $x_1=c$, $x_2=b$ und $x_3=a$ festzulegen.
    Nun sind alle Einträge in der ersten Spalte von $A$ gleich eins.
    Wir führen die nachfolgenden Schritte durch:}
    \lang{en}{For the simplest possible calculation it makes more sense to name the unknowns $x_1=c$, $x_2=b$ und $x_3=a$ festzulegen.
    Now all entries in the first column of $A$ equal 1.
    We perform the following steps:}

\[
(A \mid b) ~=~
\begin{pmatrix}
    1 & -1 & 1 &|& -4 \\
    1 &  1 & 1 &|&  6 \\
    1 &  3 & 9 &|&  0
\end{pmatrix}
~
\begin{matrix}
    \\
   |-(I) + (II)\\
    |-(I) + (III)
\end{matrix}
\]

\[
\rightsquigarrow~~
\begin{pmatrix}
    1 & -1 & 1 &|& -4 \\
    0 &  2 & 0 &|& 10 \\
    0 &  4 & 8 &|&  4
\end{pmatrix}
~
\begin{matrix}
    \\
    \\
    | -2 \cdot (II) + (III)
\end{matrix}
\]

\[
\rightsquigarrow~~
\begin{pmatrix}
    1 & -1 & 1 &|&  -4 \\
    0 &  2 & 0 &|&  10 \\
    0 &  0 & 8 &|& -16
\end{pmatrix}
\]

    \step \lang{de}{
        Durch Rückwärtseinsetzen erhält man $8 x_3 = -16$.
        Damit ist $x_3 = -2$.
        Weiterhin erhält man $2 x_2 = 10$, also $x_2 = 5$.
        Durch Einsetzen in die erste Gleichung ergibt sich
        $x_1 - x_2 + x_3 = -4$, also ist $x_1 = 3$.
        
        Nun können wir die Koeffizienten $a$, $b$ und $c$ bestimmen:
        $a = x_3 = -2$, $b = x_2 = 5$, $c = x_1 = 3$.
    }
    \lang{en}{
    By using back substitution we get $8 x_3 = -16$.
    Therefore, it is $x_3=-2$.
    Furthermore we get $2 x_2 = 10$, so $x_2 = 5$.
    By inserting this in the first equation, we get
    $x_1 - x_2 + x_3 = -4$, so $x_1 = 3$.}

    \step \lang{de}{
        Die Lösungsmenge des linearen Gleichungssystems ist somit 
        $\mathbb{L}=\{ \begin{pmatrix}-2\\ 5\\ 3 \end{pmatrix} \}$
        und die Gleichung der Parabel, die durch die Punkte $P$, $Q$ und $R$ verläuft, 
        lautet $y=-2x^2+5x+3$.
    }
    \lang{en}{
        The solution set of the linear system is 
        $\mathbb{L}=\{ \begin{pmatrix}-2\\ 5\\ 3 \end{pmatrix} \}$
        and the equation of the parabola, that goes through $P$, $Q$ and $R$, 
        is $y=-2x^2+5x+3$.
    }

%    \step \lang{de}{Multiplizieren Sie beispielsweise die erste Gleichung auf beiden Seiten mit $-1$ und addieren Sie sie dann zur zweiten bzw. dritten Gleichung, so dass in diesen $c$ nicht mehr vorkommt.}
%    \lang{en}{Multiply both sides of the first equation by $-1$ and add the result to the second and third equation so that $c$ does not appear in either the second or third equation anymore.}
%    \step \lang{de}{Das lineare Gleichungssystem sieht dann wie folgt aus:}
%    \lang{en}{The linear system then looks as follows:}
%    \begin{displaymath}
%\begin{mtable}[\cellaligns{ccrcrcrcr}]
%\text{(I)}&\qquad&a&-&b&+&c&=&-4\phantom{.}\\
%\text{(II)}&&&&2b&&&=&10\phantom{.}\\
%\text{(III)}&&8a&+&4b&&&=&4\lang{de}{.}\lang{en}{\phantom{.}}
%\end{mtable}
%\end{displaymath}

%    \step \lang{de}{Auch wenn das lineare Gleichungssystem noch nicht in Stufenform ist, lässt sich seine Lösungsmenge bereits jetzt ermitteln.\\
%    Aus der zweiten Gleichung folgt nämlich sofort $b=5$. Einsetzen von $b$ in die dritte Gleichung ergibt dann $8a+20=4$ und somit $a=-2$. Zuletzt lässt sich $c$ durch 
%    Einsetzen von $a$ und $b$ in die erste Gleichung bestimmen. Es ergibt sich $-2-5+c=-4$, woraus $c=3$ folgt.
    
%    Anmerkung:\\
%    Um das lineare Gleichungssystem in Stufenform zu bringen, hätte man entweder noch in der dritten Gleichung $b$ eliminieren müssen oder aber auch einfach
%    die zweite und dritte Gleichung vertauschen können.}
%    \lang{en}{Even though the linear system is not in row echelon form, the solution set can already be determined.\\
%    From the second equation we get immediately that $b=5$. Substituting $b$ into the third equation results in $8a+20=4$ and hence $a=-2$. Lastly, $c$ can be determined by
%    substituting $a$ and $b$ into the first equation. This results in $-2-5+c=-4$ from which we get that $c=3$.
    
%    Note:\\
%    In order to put the linear system into row echelon form, we could have either eliminated $b$ from the third equation or swapped the second and third equation.}
%    \step\lang{de}{Die Lösungsmenge des linearen Gleichungssystems ist somit $\mathbb{L}=\{(-2; 5; 3)\}$ und die Gleichung der Parabel, die durch die Punkte $P$, $Q$ und $R$ geht 
%    lautet $y=-2x^2+5x+3$.} 
%    \lang{en}{The solution set of the linear system is $\mathbb{L}=\{(-2, 5, 3)\}$ and the equation of the parabola that goes through the points $P$, $Q$, and $R$ is $y=-2x^2+5x+3$.} 
  \end{incremental}
         \tab{\lang{de}{Lösungsvideo (b)} \lang{en}{Solution (b)}}	
   \lang{de}{ \youtubevideo[500][300]{9WDeeUdiiFc}}
   \lang{en}{The equation for a general third grad polonymial is $p(x)=ax^3+bx^2+cx+d$.
   Because the polynomial should go through the give points, their coordinates need to fulfill the polynomial equation:
   \begin{align*}
   p(-1)=&-a&+b&-c&+d&=-3\\
   p(0)=& & & & d&=3  \\
   p(1)=& a&+b&+c&+d&=1\\
   p(2)=&8a&+4b&+2c&+d&=3
   \end{align*}
   For setting up the augmented matrix, we identify the unknowns $x_i$ with $1\leq i\leq n$ with the coefficients of the polynomial.
   The augmented matrix is:
   \[
(A \mid b) ~=~
\begin{pmatrix}
    -1 & 1 & -1 & 1 &|& -3 \\
    0 &  0 & 0 & 1 &|&  3 \\
    1 &  1 & 1 & 1 &|&  1 \\
    8 & 4 & 2 & 1 &|& 3
\end{pmatrix}
\]
After performing Gaussian elimination the matrix has row echelon form:
 \[
(A \mid b) ~=~
\begin{pmatrix}
    -1 & 1 & -1 & 1 &|& -3 \\
    0 &  1 & 0 & 1 &|&  -1 \\
    0 &  0 & -6 & -3 &|&  -9 \\
    0 & 0 & 0 & 1 &|& -1
\end{pmatrix}
\]
Back substitution gives the coefficients of the polynomial: $a=2$, $b=-4$, $c=0$ and $d=3$.
The polnynomial, that holds $p(-1)=-3$, $p(0)=3$, $p(1)=1$ und $p(2)=3$, is
\begin{align*}
p(x)=2x^3-4x^2+3.
\end{align*}}\\
\tab{
  \lang{de}{Hinweis}
  \lang{en}{Note}
  }

   \lang{de}{Beachten Sie, dass es auch andere Möglichkeiten gibt, das lineare Gleichungssystem mit dem Gauß-Verfahren umzuformen. Der aufgeführte Weg ist nur eine Variante und entspricht exakt den im Vorlesungsteil beschriebenen Einzelschritten. 
   %Außerdem ist die Stufenform nicht eindeutig, die Antworten beeinflusst dies aber nicht.
   } 
   \lang{en}{Note, that there are other possibilities for reducing the linear system into row echelon form using Gaussian Elimination. The steps performed here are only one variant and the row echelon form of a linear system is not unique, but the answers aren't influenced by this at all.}
   
   
 \end{tabs*}
\end{content}