\documentclass{mumie.element.exercise}
%$Id$
\begin{metainfo}
  \name{
    \lang{de}{Ü05: Stufenform}
    \lang{en}{Ex05: Row echelon form}
  }
  \begin{description} 
 This work is licensed under the Creative Commons License Attribution 4.0 International (CC-BY 4.0)   
 https://creativecommons.org/licenses/by/4.0/legalcode 

    \lang{de}{}
    \lang{en}{}
  \end{description}
  \begin{components}
  \end{components}
  \begin{links}
  \end{links}
  \creategeneric
\end{metainfo}
\begin{content}
\usepackage{mumie.ombplus}

\title{\lang{de}{Ü05: Stufenform} \lang{en}{Ex05: Row echelon form}}

\begin{block}[annotation]
  Im Ticket-System: \href{http://team.mumie.net/issues/11517}{Ticket 11517}
\end{block}

%######################################################FRAGE_TEXT
\lang{de}{ 
Bestimmen Sie die erweiterte Koeffizientenmatrix für das folgende lineare Gleichungssystem.
Die Koeffizienten sind über den komplexen Zahlen $\C$ definiert.
Wie üblich ist $i\in \C$ die imaginäre Einheit.
\[
 \begin{mtable}[\cellaligns{rcrcrcl}]
 ix_1&-& x_{2}&+&2x_3&=&2\\
-2ix_1&+& 3x_2&-&4x_{3}&=&0\\
-x_1&-&ix_2&+&(1+2i)x_3 &=& i
\end{mtable} \]
Bringen Sie die erweiterte Koeffizientenmatrix anschließend in die reduzierte Stufenform
und berechnen Sie daraus die Lösungsmenge des LGS. }

\lang{en}{ 
Determine the augmented matrix for the following linear system. The coefficients are defined over the complex numbers and $i\in\C$ is,
as usual, the imaginary unit.
\[
 \begin{mtable}[\cellaligns{rcrcrcl}]
 ix_1&-& x_{2}&+&2x_3&=&2\\
-2ix_1&+& 3x_2&-&4x_{3}&=&0\\
-x_1&-&ix_2&+&(1+2i)x_3 &=& i
\end{mtable} \]
Transform the augmented matrix in reduced row echelon form and determine the solution set of the linear system.}

%##################################################ANTWORTEN_TEXT
\begin{tabs*}[\initialtab{0}\class{exercise}]

  %++++++++++++++++++++++++++++++++++++++++++START_TAB_X
  \tab{\lang{de}{   Antwort   } \lang{en}{Answer}}
  \begin{incremental}[\initialsteps{1}]
  
  	 %----------------------------------START_STEP_X
    \step 
    \lang{de}{Die erweiterte Koeffizientenmatrix des LGS ist
\[ 
(A \mid b) ~=~
\begin{pmatrix}
i & -1 & 2 & | & 2 \\
-2i & 3 & -4 & | & 0\\
-1 & -i& 1+2i &|&i
\end{pmatrix}. \]
Deren reduzierte Stufenform lautet
\[ \begin{pmatrix}
1 & 0 & 0 & | & 2-6i \\
0 & 1 & 0 & | & 4\\
0 & 0& 1 &|& -i
\end{pmatrix}. \]
Die Lösungsmenge besteht aus genau einer Lösung:
\[ \mathbb{L}= \left\{  \left(\begin{smallmatrix}
2-6i \\ 4 \\ -i \end{smallmatrix} \right) \right\}.\]
    }

\lang{en}{The augmented matrix is
\[ 
(A \mid b) ~=~
\begin{pmatrix}
i & -1 & 2 & | & 2 \\
-2i & 3 & -4 & | & 0\\
-1 & -i& 1+2i &|&i
\end{pmatrix}. \]
The reduced row echelon form is
\[ \begin{pmatrix}
1 & 0 & 0 & | & 2-6i \\
0 & 1 & 0 & | & 4\\
0 & 0& 1 &|& -i
\end{pmatrix}. \]
The solution set consists of exactly one solution:
\[ \mathbb{L}= \left\{  \left(\begin{smallmatrix}
2-6i \\ 4 \\ -i \end{smallmatrix} \right) \right\}.\]
    }
  	 %------------------------------------END_STEP_X
 
  \end{incremental}
  %++++++++++++++++++++++++++++++++++++++++++++END_TAB_X


  %++++++++++++++++++++++++++++++++++++++++++START_TAB_X
  \tab{\lang{de}{    Lösung    } \lang{en}{Solution}}
  \begin{incremental}[\initialsteps{1}]
  
  	 %----------------------------------START_STEP_X
    \step 
    \lang{de}{Die erweiterte Koeffizientenmatrix zum LGS besteht aus den Koeffizienten des 
Gleichungssystems:}
\lang{en}{The augmented matrix of the linear system consists of the coefficients of the linear system:}

\[
(A \mid b) ~=~
\begin{pmatrix}
i & -1 & 2 & | & 2 \\
-2i & 3 & -4 & | & 0\\
-1 & -i& 1+2i &|&i
\end{pmatrix}
\]


    \step \lang{de}{Um die reduzierte Stufenform zu erhalten,
        werden die folgenden Zeilenumformungen vorgenommen:}
        \lang{en}{To recevie the reduced row echelon form, we perfom the following row operations:}

\[
\begin{pmatrix}
i & -1 & 2 & | & 2 \\
-2i & 3 & -4 & | & 0\\
-1 & -i& 1+2i &|&i
\end{pmatrix}
~
\begin{matrix}
     \\
    | 2 \cdot (I) + (II) \\
   | -i \cdot (I) + (III) \\
\end{matrix}
\]

\[
\rightsquigarrow~~
\begin{pmatrix}
i & -1 & 2 & | & 2 \\
0 & 1 & 0 & | & 4\\
0 & 0& 1 &|&-i
\end{pmatrix}
~
\begin{matrix}
    | - 2 \cdot (III)+(I) \\
    \\
    \\
\end{matrix}
\]

\[
\rightsquigarrow~~
\begin{pmatrix}
i & -1 & 0 & | & 2+2i \\
0 & 1 & 0 & | & 4\\
0 & 0& 1 &|&-i
\end{pmatrix}
~
\begin{matrix}
    | (II) + (I) \\
    \\
    \\
\end{matrix}
\]

\[
\rightsquigarrow~~
\begin{pmatrix}
i & 0 & 0 & | & 6+2i \\
0 & 1 & 0 & | & 4\\
0 & 0& 1 &|&-i
\end{pmatrix}
~
\begin{matrix}
    | (-i) \cdot (I) \\
    \\
    \\
\end{matrix}
\]

\[
\rightsquigarrow~~
\begin{pmatrix}
1 & 0 & 0 & | & 2-6i \\
0 & 1 & 0 & | & 4\\
0 & 0& 1 &|&-i
\end{pmatrix}
\]

%\begin{eqnarray*}
%&& \begin{pmatrix}
%i & -1 & 2 & | & 2 \\
%-2i & 3 & -4 & | & 0\\
%-1 & -i& 1+2i &|&i
%\end{pmatrix} \begin{matrix}  \phantom{1}\\  /  +2\cdot \text{(I)}\\  /  -i\cdot \text{(I)}
%\end{matrix} \quad  \rightsquigarrow 
% \begin{pmatrix} 
%i & -1 & 2 & | & 2 \\
%0 & 1 & 0 & | & 4\\
%0 & 0& 1 &|&-i
%\end{pmatrix} \begin{matrix} /  -2\cdot \text{(III)}\\    \phantom{1}\\    \phantom{1}  \end{matrix} \\
%& \rightsquigarrow &
% \begin{pmatrix} 
%i & -1 & 0 & | & 2+2i \\
%0 & 1 & 0 & | & 4\\
%0 & 0& 1 &|&-i
%\end{pmatrix} \begin{matrix} /  +1\cdot \text{(II)}\\    \phantom{1}\\    \phantom{1}  \end{matrix} 
%  \quad\rightsquigarrow 
% \begin{pmatrix} 
%i & 0 & 0 & | & 6+2i \\
%0 & 1 & 0 & | & 4\\
%0 & 0& 1 &|&-i
%\end{pmatrix} \begin{matrix} /  \cdot (-i)\\    \phantom{1}\\    \phantom{1}  \end{matrix} \\
%& \rightsquigarrow &
% \begin{pmatrix} 
%1 & 0 & 0 & | & 2-6i \\
%0 & 1 & 0 & | & 4\\
%0 & 0& 1 &|&-i
%\end{pmatrix}
%\end{eqnarray*}

    \step \lang{de}{Die Lösungsmenge des ursprünglichen LGS ist dann die Lösungsmenge des LGS, 
    welches zur letzten Matrix gehört, also zu}
    \lang{en}{The solution set of the primary linear system is the solution of the linear system, that corresponds to the last matrix, which is}

\[
 \begin{mtable}[\cellaligns{rcrcrcc}]
 x_1&& &&&=&2-6i\\
&& x_2&&&=&4\\
&&&&x_3 &=& -i.
\end{mtable}
\]

\lang{de}{Die Lösungsmenge ist also}
\lang{en}{Therefore the solution set is}

\[ \mathbb{L}= \left\{  \left(\begin{smallmatrix}
2-6i \\ 4 \\ -i \end{smallmatrix} \right) \right\}.\]

  	 %------------------------------------END_STEP_X
 
  \end{incremental}
  %++++++++++++++++++++++++++++++++++++++++++++END_TAB_X



%#############################################################ENDE
\end{tabs*}
\end{content}