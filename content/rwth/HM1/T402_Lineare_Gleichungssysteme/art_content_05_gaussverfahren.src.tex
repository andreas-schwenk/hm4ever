%$Id:  $
\documentclass{mumie.article}
%$Id$
\begin{metainfo}
  \name{
    \lang{de}{Gauß-Verfahren}
    \lang{en}{Gaussian elimination}
  }
  \begin{description} 
 This work is licensed under the Creative Commons License Attribution 4.0 International (CC-BY 4.0)   
 https://creativecommons.org/licenses/by/4.0/legalcode 

    \lang{de}{Beschreibung}
    \lang{en}{}
  \end{description}
  \begin{components}
  \component{generic_image}{content/rwth/HM1/images/g_img_00_video_button_schwarz-blau.meta.xml}{00_video_button_schwarz-blau}
  \end{components}
  \begin{links}
    \link{generic_article}{content/rwth/HM1/T112neu_Lineare_Gleichungssysteme/g_art_content_41_gauss_verfahren.meta.xml}{gauss}
  \end{links}
  \creategeneric
\end{metainfo}
\begin{content}
\usepackage{mumie.ombplus}
\ombchapter{2}
\ombarticle{2}
\usepackage{mumie.genericvisualization}


\title{\lang{de}{Gauß-Verfahren über allgemeinen Körpern} \lang{en}{Gaussian elinimation over any field}}

\begin{block}[annotation]
 
  
\end{block}
\begin{block}[annotation]
  Im Ticket-System: \href{http://team.mumie.net/issues/11196}{Ticket 11196}\\
\end{block}

\begin{block}[info-box]
\tableofcontents
\end{block}

\lang{de}{
Das \notion{Gauß-Verfahren} zum L\"osen linearer Gleichungssysteme (LGS) über den reellen Zahlen wurde bereits in
\link{gauss}{Teil 1} eingeführt.
Die Idee des Gauß-Verfahrens ist es, die Gleichungen des LGS (bzw. die Zeilen der zugehörigen erweiterten Koeffizientenmatrix) 
so umzuformen, dass man am Ende
die Lösungsmenge des Gleichungssystems leicht angeben kann. 
In diesem Abschnitt wird das Gauß-Verfahren für die erweiterte Koeffizientenmatrix wiederholt. 
Allerdings werden die Koeffizienten nun über beliebigen Körpern $\mathbb{K}$ betrachtet.\\
Außerdem werden wir zeigen, wie man eine Menge von linearen Gleichungssystemen $Ax=b$ lösen kann,
bei denen sich jeweils nur der Vektor $b$ unterscheidet.\\
Im Folgenden betrachten wir stets ein lineares Gleichungssystem der Form
\begin{equation*} 
\begin{mtable}[\cellaligns{ccccccccc}]
 a_{11} x_1 & + & a_{12} x_2 & + & \cdots & + & a_{1n} x_n & = & b_1 \\
 a_{21} x_1 & + & a_{22} x_2 & + & \cdots & + & a_{2n} x_n & = & b_2 \\
 \vdots     &   & \vdots     &   &        &   & \vdots     &   & \vdots \\
 a_{m1} x_1 & + & a_{m2} x_2 & + & \cdots & + & a_{mn} x_n & = & b_m
\end{mtable} \label{eq:lgs1}
\end{equation*}
mit Elementen $a_{ij}\in \mathbb{K}$ für $1 \leq i \leq m$ und $1 \leq j \leq n$ und $b_1, \ldots, b_m\in \mathbb{K}$, sowie die zugehörige erweiterte Koeffizientenmatrix
\begin{equation*}
(A \ \mid \ b) = \underbrace{\left(  \begin{smallmatrix}
                                      a_{11} & a_{12} & \cdots & a_{1n} & | & b_1\\
                                      a_{21} & a_{22} & \cdots & a_{2n} & | &b_2\\
                                      \vdots & \vdots & \ddots & \vdots & \mid &\vdots\\
                                      a_{m1} & a_{m2} & \cdots & a_{mn} &| & b_m
                                      \end{smallmatrix} \right) }_{
\in M(m,n+1;\mathbb{K})}.
\end{equation*}
Ziel des Gauß-Verfahrens ist es, eine Matrix schrittweise in Stufenform zu bringen und anschließend daraus die Lösungsmenge anzugeben.}


\lang{en}{
The \notion{Gaussian elimination} for solving linear system over the real numbers was already introduced in \link{gauss}{Part 1}.
The idea of the Gaussian elimination is to manipulate the equations of the system (or rather the rows of the corresponding augmented matrix) 
into a form, which makes it easy to find the solution set.
This section reviews the Gaussian elimination for the augmented matrix, but the coefficients will be elements of any field $\mathbb{K}$.\\ 
Further on, we will see, how we can solve a set of linear equations $Ax=b$, that differs only in the vector $b$.
In the following linear systems will have the form
\begin{equation*} 
\begin{mtable}[\cellaligns{ccccccccc}]
 a_{11} x_1 & + & a_{12} x_2 & + & \cdots & + & a_{1n} x_n & = & b_1 \\
 a_{21} x_1 & + & a_{22} x_2 & + & \cdots & + & a_{2n} x_n & = & b_2 \\
 \vdots     &   & \vdots     &   &        &   & \vdots     &   & \vdots \\
 a_{m1} x_1 & + & a_{m2} x_2 & + & \cdots & + & a_{mn} x_n & = & b_m
\end{mtable} \label{eq:lgs1}
\end{equation*}
with $a_{ij}\in \mathbb{K}$ for $1 \leq i \leq m$ and $1 \leq j \leq n$ and $b_1, \ldots, b_m\in \mathbb{K}$, plus the corresponding augmented coefficient matrix
\begin{equation*}
(A \ \mid \ b) = \underbrace{\left(  \begin{smallmatrix}
                                      a_{11} & a_{12} & \cdots & a_{1n} & | & b_1\\
                                      a_{21} & a_{22} & \cdots & a_{2n} & | &b_2\\
                                      \vdots & \vdots & \ddots & \vdots & \mid &\vdots\\
                                      a_{m1} & a_{m2} & \cdots & a_{mn} &| & b_m
                                      \end{smallmatrix} \right) }_{
\in M(m,n+1;\mathbb{K})}.
\end{equation*}
The aim of the Gaussian elimination is to manipulate the matrix step-by-step into the reduced echelon row form, from which the solution can be read.}




\section{\lang{de}{Die Stufenform}\lang{en}{Row echelon form}} 
\begin{definition}[\lang{de}{Stufenform} \lang{en}{row echelon form}]\label{def:stufenformen}\label{stufenform}
\lang{de}{
Ein LGS liegt in \notion{Stufenform} vor, wenn für jede Gleichung die 
%(lexikographisch) 
erste auftretende Variable
in allen folgenden Gleichungen nicht mehr auftritt (oder die Gleichung gar keine Variablen enthält).\\
Betrachtet man die Stufenform der dazugehörigen erweiterten Koeffizientenmatrix, so hat diese die Form}
\lang{en}{
A linear system is in \notion{row echelon form}, if for each equation the leading term is not appearing in the following equations (i.e. the coefficient is zero).\\
The augmented matrix corresponding to a linear system in row echelon form has the following form}
\[ \begin{pmatrix}
    a_{1} & \star & \cdots & \star & \star&| & b_1\\ 
     0 & a_{2} & \cdots & \star & \star&| & b_2\\ 
    \vdots & \, & \ddots & \, & \vdots&| & \vdots\\ 
 0 & 0 & \cdots & a_{k} & \star&| & b_k\\ 
   0 & 0 & \cdots & 0 & 0&| & \vdots \\ 
  0 & 0 & \cdots & 0 & 0&| & b_{m}\\ 
    \end{pmatrix},\]
\lang{de}{wobei der Eintrag $a_{i}$ jeweils dem ersten Koeffizienten ungleich Null der $i$-ten Gleichung entspricht. Diesen Koeffizienten nennen wir \notion{Stufenelement} (oder Stufeneintrag). 
Der "$\star$" steht jeweils für einen beliebigen Eintrag. 
Das LGS liegt in \notion{reduzierter Stufenform} vor, wenn die ersten auftretenden Variablen sogar in keiner anderen Gleichung
auftreten, also auch nicht in den darüber liegenden, und ihre Koeffizienten zusätzlich den Wert 1 besitzen.\\
Die entsprechende reduzierte Stufenform der erweiterten Koeffizientenmatrix hat damit die Form}

\lang{en}{where $a_{i}$ is the first non-zero coefficient of the $i$th row. This coefficient is called the \notion{leading coefficient} of the row/equation. 
The "$\star$" here may be any element. 
The linear system is in  \notion{reduced row echelon form}, if the variable in the leading term do not appear in any other equation/row (not only in the following equations/rows)
and the leading coefficient of each row is 1.\\
The augmented matrix corresponding to a linear system in reduced row echelon form has the following form:}
 
 \[ \begin{pmatrix}
    1 & 0 & \cdots & 0 & \star&| & b_1\\ 
     0 & 1 & \cdots & 0 & \star&| & b_2\\ 
    \vdots & \, & \ddots & \, & \vdots &| & \vdots\\ 
 0 & 0 & \cdots & 1 & \star&| & b_k\\ 
   0 & 0 & \cdots & 0 & 0&| & \vdots \\ 
  0 & 0 & \cdots & 0 & 0&| & b_{m}\\ 
  \end{pmatrix}.\]
\lang{de}{Dabei ist zu beachten, dass nur die Einträge über und unter einem Stufenelement Null sein müssen.}
\lang{en}{It is important to note, that the elements below and above each leading coefficient must be zero.}

\end{definition}
\lang{de}{
In der Literatur spricht man oft präziser von der Zeilenstufenform.}

\begin{remark}
\lang{de}{
Falls eine Koeffizientenmatrix in Stufenform eine Nullzeile besitzt, 
so ist die entsprechende Gleichung des LGS $0=b_i$. 
In dieser Gleichung treten also keine Variablen mehr auf.}
\lang{en}{
If the augmented matrix in row echelon form contains a row of zero, the corresponding equation of the linear system is $0=b_i$.
This equation contains no variables.}\\
\end{remark}

\lang{de}{
\begin{example}
\begin{tabs*}
\tab{1. Beispiel}
Die erweiterte Koeffizientenmatrix
\[ \begin{pmatrix} -1 & 2 & 3 & | & 5 \\\textcolor{#0066CC}{0} & 1 & 4 & | & 11 \\\textcolor{#0066CC}{0} & \textcolor{#0066CC}{0} & -2 & | & -6 \end{pmatrix} \]
liegt in Stufenform vor, da die Einträge unter den Stufenelementen Null sind. \\
Es liegt aber keine reduzierte Stufenform vor, da zum einen die Stufenelemente nicht alle Wert 1 haben und die Einträge über ihnen nicht Null sind.
\[ \begin{pmatrix} \textcolor{#CC6600}{-1} & \textcolor{#CC6600}{2} & \textcolor{#CC6600}{3} & | & 5 \\0 & \textcolor{#0066CC}{1} & \textcolor{#CC6600}{4} & | & 11 \\0 & 0 & \textcolor{#CC6600}{-2} & | & -6 \end{pmatrix} \]
\tab{2. Beispiel}
Die erweiterte Koeffizientenmatrix 
\[ \begin{pmatrix} 2 & -4-2i & | & 10 \\\textcolor{#CC6600}{6i} &6-12i & | & 30i \end{pmatrix} \]
über den komplexen Zahlen liegt nicht in Stufenform vor, da der Koeffizient $6i$ Null entsprechen müsste.\\
Anders verhält es sich bei der erweiterten Koeffizientenmatrix
\[ \begin{pmatrix} \textcolor{#0066CC}{1}& -2-i & | & 10 \\0 &0 & | & 0 \end{pmatrix}, \]
bei der der Koeffizient $a_{21}=0$ ist. Außerdem hat das Stufenelement hier den Wert 1 und es existieren keine darüber liegenden Einträge. Damit ist sie sogar in reduzierter Stufenform.
\tab{3. Beispiel}
Die erweiterte Koeffizientenmatrix
\[ \begin{pmatrix}1 & 2 &1 & -1 &| & 7 \\\textcolor{#0066CC}{0} & 1 &-1 & 0 &| & 1\\\textcolor{#0066CC}{0} &\textcolor{#0066CC}{0} &0 & 1 &| & -1 \end{pmatrix}\]
liegt in Stufenform vor, da die Einträge unter den Stufenelementen alle Null entsprechen.\\
Die Stufenform ist jedoch keine reduzierte Stufenform, obwohl die Stufenelemente den Wert 1 besitzen. Denn nicht alle Koeffizienten über den Stufenelementen sind Null.
\[ \begin{pmatrix}\textcolor{#0066CC}{1} & \textcolor{#CC6600}{2} &1 & \textcolor{#CC6600}{-1} &| & 7 \\0 & \textcolor{#0066CC}{1} &-1 & \textcolor{#0066CC}{0} &| & 1\\0 & 0 &0 & \textcolor{#0066CC}{1} &| & -1 \end{pmatrix}\]
\end{tabs*}
\end{example}}



\lang{en}{
\begin{example}
\begin{tabs*}
\tab{1. Example}
The augmented matrix
\[ \begin{pmatrix} -1 & 2 & 3 & | & 5 \\\textcolor{#0066CC}{0} & 1 & 4 & | & 11 \\\textcolor{#0066CC}{0} & \textcolor{#0066CC}{0} & -2 & | & -6 \end{pmatrix} \]
is in row echelon form, because the coefficients below the leading coefficients are zero. \\
The row echelon form is not reduced, because the coefficients of the leading terms have values $\neq 1$ and the entries above are non-zero.
\[ \begin{pmatrix} \textcolor{#CC6600}{-1} & \textcolor{#CC6600}{2} & \textcolor{#CC6600}{3} & | & 5 \\0 & \textcolor{#0066CC}{1} & \textcolor{#CC6600}{4} & | & 11 \\0 & 0 & \textcolor{#CC6600}{-2} & | & -6 \end{pmatrix} \]
\tab{2. Example}
The augmented matrix 
\[ \begin{pmatrix} 2 & -4-2i & | & 10 \\\textcolor{#CC6600}{6i} &6-12i & | & 30i \end{pmatrix} \]
over the complex numbers is not in row echelon form, because the coefficient $6i$ should be zero.\\
The situations is different for the augmented matrix
\[ \begin{pmatrix} \textcolor{#0066CC}{1}& -2-i & | & 10 \\0 &0 & | & 0 \end{pmatrix}, \]
because we have $a_{21}=0$. The leading coefficient is 1 and there are no entries above it. For this reason, the matrix is even in reduced row echelon form.
\tab{3. Example}
The augmented matrix
\[ \begin{pmatrix}1 & 2 &1 & -1 &| & 7 \\\textcolor{#0066CC}{0} & 1 &-1 & 0 &| & 1\\\textcolor{#0066CC}{0} &\textcolor{#0066CC}{0} &0 & 1 &| & -1 \end{pmatrix}\]
is in row echelon form, because the entries below the leading coefficients are all zero.\\
The row echelon form is not reduced, although all the leading coefficients have the value 1, because there are non-zero entries above the pivots.
\[ \begin{pmatrix}\textcolor{#0066CC}{1} & \textcolor{#CC6600}{2} &1 & \textcolor{#CC6600}{-1} &| & 7 \\0 & \textcolor{#0066CC}{1} &-1 & \textcolor{#0066CC}{0} &| & 1\\0 & 0 &0 & \textcolor{#0066CC}{1} &| & -1 \end{pmatrix}\]
\end{tabs*}
\end{example}}

\lang{de}{
\begin{quickcheck}
		\type{input.interval}
        \field{rational}
        \precision{3}
      \field{real}
      \begin{variables}
           \randint[Z]{a}{2}{6}
           \randint[Z]{b}{2}{8}
           \randint[Z]{c}{2}{8}
           \randint[Z]{d}{2}{8}
           \end{variables}
      \text{Für das lineare Gleichungssystem 
\[ \begin{mtable}[\cellaligns{crcrcrcr}]
(I)&\var{a} \cdot  x & - & \var{c} \cdot y       &   &                &= & 0 \\
(II)&                &   & \var{b}\cdot y & + & \var{d} \cdot z&=  & \var{a} 
\end{mtable} \]
       gilt: } 
    \begin{choices}{unique}
        \begin{choice}
            \text{Es liegt in Stufenform vor, aber nicht in reduzierter Stufenform.}
			\solution{true}
		\end{choice}             
        \begin{choice}
            \text{Es liegt in reduzierter Stufenform vor.}
			\solution{false}
		\end{choice}
                \begin{choice}
            \text{Es liegt nicht in Stufenform vor.}
			\solution{false}
		\end{choice}      
    \end{choices}{unique}
            	\explanation{Das LGS liegt in Stufenform vor, da die Variable $x$ nur in der ersten Gleichung auftritt.\\ 
                Es liegt aber nicht in reduzierter Stufenform vor, da die Variable $y$ (als erste Variable der zweiten Gleichung) ebenfalls in der ersten Gleichung auftaucht. 
                Zusätzlich besitzen die Stufenelemente nicht den Wert 1.}		
	\end{quickcheck}}

 \lang{en}{
\begin{quickcheck}
		\type{input.interval}
        \field{rational}
        \precision{3}
      \field{real}
      \begin{variables}
           \randint[Z]{a}{2}{6}
           \randint[Z]{b}{2}{8}
           \randint[Z]{c}{2}{8}
           \randint[Z]{d}{2}{8}
           \end{variables}
      \text{For the linear system 
\[ \begin{mtable}[\cellaligns{crcrcrcr}]
(I)&\var{a} \cdot  x & - & \var{c} \cdot y       &   &                &= & 0 \\
(II)&                &   & \var{b}\cdot y & + & \var{d} \cdot z&=  & \var{a} 
\end{mtable} \]
       holds: } 
    \begin{choices}{unique}
        \begin{choice}
            \text{The linear system is in row echelon form, but is not reduced.}
			\solution{true}
		\end{choice}                    
        \begin{choice}
            \text{The linear system is in reduced row echelon form.}
			\solution{false}
		\end{choice}
                \begin{choice}
            \text{The linear system is not in row echelon form.}
			\solution{false}
		\end{choice}      
    \end{choices}{unique}
            	\explanation{The linear system is in row echelon form, because the variable x is found only in the firs equations.
             But the row echelon form is not reduced, since the variable $y$ (as the first variable in the second euqation) is also found in the first equation.
             Further on, the pivot elements do not have the value 1.}		
	\end{quickcheck}}
    
    
%Im Allgemeinen sieht ein LGS in Stufenform dann folgendermaßen aus mit gewissen Zahlen $k$ ($1\leq k\leq m$) und
%$1\leq i_1<i_2<\ldots <i_k\leq n$, sowie $c_i\in \R\setminus \{0\}$ und $\tilde{a}_{ij}, \tilde{b}_i\in \R$:
%\begin{equation}\label{eq:lgs2}
% \begin{mtable}[\cellaligns{rrrrrrrrrrrrrrl}]
% c_1 x_{i_1} & + & \cdots & + & \tilde{a}_{1i_2}x_{i_2} & + & \cdots & &\cdots &&\cdots && \cdots & = & \tilde{b}_1 \\
%			 &   	&  &  &   c_2 x_{i_2} & + & \cdots &+ & \tilde{a}_{2i_3}x_{i_3} & + &\cdots&&  \cdots & = & \tilde{b}_2 \\
% 			 &   	&  			 &  &                          &    &            &   &   c_3 x_{i_3} & + &\cdots&&  \cdots & = & \tilde{b}_3 \\
%    &    &          &   &                   &      &      &    &   &   & \vdots   &&   &     & \vdots \\
% 			 &   &               &   &        &    & &           &        &   &  c_k x_{i_k} & + & \cdots & = & \tilde{b}_k \\
% 			 &&&&&&&&&&&& 0 & = & \tilde{b}_{k+1} \\
% 			 &&&&&&&&&&&& \vdots & & \vdots \\
% 			 &&&&&&&&&&&& 0 & = & \tilde{b}_{m} 
%\end{mtable} 
%\end{equation}
%In der reduzierten Stufenform sind zusätzlich die Koeffizienten über den $c$'s gleich Null.
\lang{de}{
Anhand der Stufenform lässt sich erkennen, ob ein LGS lösbar ist.}
\lang{en}{
From the row echelon form of a linear system we can tell whether it has solutions.}
%und sie lässt sich damit explizit durch Rückwärtseinsetzen berechnen. 
\lang{de}{
\begin{theorem}[Lösbarkeit]\label{rule:loesung-parametrisiert}
Ein LGS ist genau dann lösbar,
wenn für die Stufenform  (Definition \ref{stufenform}) 
seiner erweiterten Koeffizientenmatrix
entweder
%Ein LGS und ihre dazugehörige erweiterte Koeffizientenmatrix ist genau 
%dann l"osbar, wenn für ihre Stufenform (\ref{stufenform}) entweder 
\begin{itemize}
\item $k=m \ $ oder
\item${b}_{k+1}=\ldots = {b}_{m} =0$
\end{itemize}
gilt.\\
Lösbarkeit bedeutet also, dass nach Umformung in Stufenform keine Zeile der Form
$(0 ~$\cdots$~ 0 | b_i)$ mit $b_i \neq 0$
auftritt.
\end{theorem}}

\lang{en}{
\begin{theorem}[\lang{de}{Lösbarkeit} \lang{en}{Solvability}]\label{rule:loesung-parametrisiert}
A linear system is solvable if and only if we have for the row echelon form (Definition \ref{stufenform}) of the augmented matrix
either
%Ein LGS und ihre dazugehörige erweiterte Koeffizientenmatrix ist genau 
%dann l"osbar, wenn für ihre Stufenform (\ref{stufenform}) entweder 
\begin{itemize}
\item $k=m \ $ or
\item${b}_{k+1}=\ldots = {b}_{m} =0$.
\end{itemize}
Hence, solvability of a linear system corresponds to the non-existence any rows of the form $(0\ldots 0|bi)$ with $bi\neq 0$ in the row echelon form of the system.
\end{theorem}}

\lang{de}{
Ist ein LGS nicht lösbar, so gilt für seine Lösungsmenge $\mathbb{L}=\emptyset$.\\
Für die Bestimmung der Lösungsmenge eines lösbaren LGS gehen wir wie folgt vor.}

\lang{en}{
If a linear system is not solvable, then its solution set is $\mathbb{L}=\emptyset$.
We determine the solution set of a solvable linear system as follows:}

\lang{de}{
\begin{rule}[Bestimmung der Lösungsmenge]\label{rule:bestimmungDerLoesungsmenge}
Um die Lösungsmenge zu berechnen, stellt man die erweiterte Koeffizientenmatrix wieder als lineares Gleichungssystem dar.\\
Unbekannte, die als Koeffizienten einen Stufeneintrag besitzen, werden \notion{abhängige Variablen} genannt und folglich sind alle Unbekannten 
ohne einen Stufeneintrag als Koeffizienten \notion{unabhängige} oder \notion{freie Variablen}.\\
Für letztere werden Parameter gewählt und 
anschließend wird das Gleichungssystem nach den abhängigen Variablen durch \notion{Rückwärtseinsetzen} aufgelöst.\\
Ist das LGS in reduzierter Stufenform, so enthält jede Gleichung höchstens eine abhängige Variable und diese kann direkt in Abhängigkeit der Parameter abgelesen werden (falls vorhanden).
Besitzt ein lösbares LGS keine freien Variablen, so besteht die Lösungsmenge aus genau einem Lösungsvektor.
\end{rule}}

\lang{en}{
\begin{rule}[Determining the solution set]\label{rule:bestimmungDerLoesungsmenge}
For determining the solution of a linear system, it needs to be considered as a system of linear equations, instead of a matrix.\\
The leading variable of a row is called \notion{dependent variable} and all variables in non-leading terms are naturally called \notion{independent} or \notion{free variables}.\\
For those, we choose parameters to represent them and afterwards we solve the linear system in a process called \notion{back substitution}.\\
If the linear system is in reduced row echelon form, each non-zero equation contains at most one dependent variable and can be rearranged to give the solution of these dependent variables in terms of parameters (if existing).
If the linear system has no independent variables, then the solution set consists of exactly one solution vector.
\end{rule}}

\lang{de}{
Rückwärtseinsetzen bedeutet, dass man bei der letzten Gleichung beginnend alle Gleichungen von unten nach oben nach der
ersten auftretenden Variable l\"ost,
 indem man alle bereits in Parametern ausgedrückten oder durch Werte berechneten Variablen in die darüber liegenden Gleichungen einsetzt.\\
 Dadurch erhält man für alle Variablen Werte oder Ausdrücke in den Parametern. 
 \begin{remark}
Werden die freien bzw. abhängigen Variablen anders gewählt, so ändert sich die Lösungsmenge nicht. Sie erhält dadurch aber eine andere Darstellungsform. 
\end{remark}}

\lang{en}{
Back substitution is a process, that dissolves all equations for its leading variables. We rearrange the last equation by expressing its leading variable in terms of the
parameters and substitute the calculated variable in the above equation, and so on. 
Thereby all variables will be expressed in terms of the parameters. 
 \begin{remark}
Changing the (in-)dependent variables does not change the solution set, but the representation may change.
\end{remark}}


\begin{example} \label{gauss1}
\begin{tabs*}
\tab{\lang{de}{1. Beispiel} \lang{en}{1. Example}}
\lang{de}{
Im LGS
\begin{displaymath}
\begin{mtable}[\cellaligns{ccrcrcrcr}]
(I)&\qquad-&x&+&2y&+&3z&=&5\\
(II)&&&&y&+&4z&=&11\\
(III)&&&&&-&2z&=&-6
\end{mtable}
\end{displaymath}
ist jede Variable die erste auftretende Variable einer Gleichung. Es gibt daher keine freien Variablen, und für alle Variablen können
 Werte berechnet werden. 
Zun\"achst wird die dritte Gleichung nach der Variable $z$ aufgel\"ost.
\begin{displaymath}
(III) \qquad z=3
\end{displaymath}
Diesen Wert setzt man in die zweite Gleichung ein, um die Variable $y$ zu bestimmen.
\begin{displaymath}
(II) \qquad y+4\cdot3=11 \quad \Leftrightarrow \quad y=-1
\end{displaymath}
Anschlie{\ss}end werden beide Werte in die erste Gleichung eingesetzt, um daraus $x$ zu berechnen.
\begin{displaymath}
(I) \qquad -x+2\cdot(-1)+3\cdot3=5 \quad \Leftrightarrow \quad x=2
\end{displaymath}
Das LGS ist eindeutig l\"osbar und besitzt die L\"osungsmenge
\begin{displaymath}
\mathbb{L}=\left\{\begin{pmatrix}2\\ -1\\ 3\end{pmatrix} \right\}.
\end{displaymath}}
\lang{en}{
In the linear system
\begin{displaymath}
\begin{mtable}[\cellaligns{ccrcrcrcr}]
(I)&\qquad-&x&+&2y&+&3z&=&5\\
(II)&&&&y&+&4z&=&11\\
(III)&&&&&-&2z&=&-6
\end{mtable}
\end{displaymath}
each variable is the leading variable of an equation, which is why there are no independent variables. Hence, the value of each
variable can be determined. 
First, the third equation is solved for the variable $z$:
\begin{displaymath}
(III) \qquad z=3
\end{displaymath}
This value is then substituted into the second equation in order to calculate the variable $y$:
\begin{displaymath}
(II) \qquad y+4\cdot3=11 \quad \Leftrightarrow \quad y=-1
\end{displaymath}
Next, both values are substituted into the first equation for solving it for $x$:
\begin{displaymath}
(I) \qquad -x+2\cdot(-1)+3\cdot3=5 \quad \Leftrightarrow \quad x=2
\end{displaymath}
The linear system is uniquely solvable and has the solution set:
\begin{displaymath}
\mathbb{L}=\left\{\begin{pmatrix}2\\ -1\\ 3\end{pmatrix} \right\}.
\end{displaymath}}

\tab{\lang{de}{2. Beispiel} \lang{en}{2. Example}}
\lang{de}{
Das LGS über den komplexen Zahlen
\[ \begin{mtable}[\cellaligns{crcrcr}]
(I)&\qquad x & - & (2+i) \cdot  y & = & 5 \\
(II)& &  & 0 & =  & 0
\end{mtable} \]
ist lösbar, da es zwar eine Gleichung ohne Variablen gibt, diese aber $0=0$ entspricht, d.\,h. stets erfüllt ist. Diese Gleichung kann nun
weggelassen werden. In der verbliebenen Gleichung $(I)$ ist $x$ die erste Variable, also eine abhängige Variable und $y$ ist eine
freie Variable. Wir führen daher einen komplexen Parameter $r$ für $y$ ein, setzen also $y=r$. 
Nun setzt man $y=r$ in die Gleichung $(I)$ ein, löst sie nach $x$ auf und erhält 
\[ x=5+(2+i)r.\]
Die Lösungsmenge ist daher
\[ \mathbb{L}= \left\{ \begin{pmatrix}5+(2+i)r\\ r \end{pmatrix} \, \big| \, r\in \C \right\}
= \left\{ \begin{pmatrix}5\\ 0 \end{pmatrix}+r\cdot \begin{pmatrix} 2+i \\ 1 \end{pmatrix} \, \big| \, r\in \C \right\}. \]}

\lang{en}{
The linear system
\[ \begin{mtable}[\cellaligns{crcrcr}]
(I)&\qquad x & - & (2+i) \cdot  y & = & 5 \\
(II)& &  & 0 & =  & 0
\end{mtable} \]
over the complex numbers is solvable. Though there is an equation without any variables, this equation is $0=$, i.e. always fulfilled.
This equation can be omitted.
In theremaining equationIn der verbliebenen Gleichung $(I)$ the variable $x$ is the first one and therefore a dependent variable.
$y$ is a free variable. We introduce the complex parameter $r$ for $y$ and set $y=r$. 
We now subsitute $y=r$ in equation $(I)$, dissolve it for $x$ and we get 
\[ x=5+(2+i)r.\]
Therefore the solution set is
\[ \mathbb{L}= \left\{ \begin{pmatrix}5+(2+i)r\\ r \end{pmatrix} \, \big| \, r\in \C \right\}
= \left\{ \begin{pmatrix}5\\ 0 \end{pmatrix}+r\cdot \begin{pmatrix} 2+i \\ 1 \end{pmatrix} \, \big| \, r\in \C \right\}. \] }

\tab{\lang{de}{3. Beispiel} \lang{en}{3. Example}}
\lang{de}{
Im LGS über $\mathbb{R}$
\[ \begin{mtable}[\cellaligns{crcrcrcrcr}]
(I)&\qquad x_1 & + & 2 \cdot  x_2 &+ & x_3 & - & x_4 &= & 7 \\
(II)& 		  &  &           x_2 &- & x_3 &   &     &=  & 1 \\
(III)& 		  &  &               &  &     &   & x_4 &= & -1
\end{mtable} \]   
sind die Variablen $x_1$, $x_2$ und $x_4$ die ersten auftretenden Variablen einer der Gleichungen. Dies sind also die abhängigen Variablen
und die Variable $x_3$ ist eine freie Variable. Wir setzen also $x_3=t$ mit einem reellen Parameter $t$, und rechnen die anderen Variablen
von unten nach oben aus. Die dritte Gleichung ist
\[   x_4=-1.  \]
Für die zweite Gleichung gilt:
\[  x_2-t=1 \Leftrightarrow x_2=1+t. \]
Nun setzen wir die Ausdrücke für $x_2$, $x_3$ und $x_4$ in die erste Gleichung ein und lösen nach $x_1$ auf:
\[ x_1+2\cdot (1+t)+t- (-1)=7  \Leftrightarrow x_1=7-2\cdot (1+t)-t-1=4-3t. \]
Die Lösungsmenge ist daher:
\[ \mathbb{L}
= \left\{ \begin{pmatrix} 4-3t \\ 1+t \\t \\ -1\end{pmatrix} \, \big| \, t\in \R \right\}
= \left\{ \begin{pmatrix} 4 \\ 1 \\ 0 \\ -1\end{pmatrix}+ t\cdot \begin{pmatrix} -3\\ 1\\ 1\\ 0 \end{pmatrix} \, \big| \, t\in \R \right\}. \]}

\lang{en}{
In the linear system over $\mathbb{R}$,
\[ \begin{mtable}[\cellaligns{crcrcrcrcr}]
(I)&\qquad x_1 & + & 2 \cdot  x_2 &+ & x_3 & - & x_4 &= & 7 \\
(II)& 		  &  &           x_2 &- & x_3 &   &     &=  & 1 \\
(III)& 		  &  &               &  &     &   & x_4 &= & -1
\end{mtable} \]   
the variables $x_1$, $x_2$ and $x_4$ are the first one in one of the equatiosn. 
They are then the dependent variables. $x_3$ is the free variable.
We set $x_3=t$ with $t\in\R$ and determine the other variables by back subituting.
The thrid equation is
\[   x_4=-1.  \]
For the second equation we have:
\[  x_2-t=1 \Leftrightarrow x_2=1+t. \]
Now we subsitute the terms for $x_2$, $x_3$ and $x_4$ in the first equation and dissolve it for $x_1$:
\[ x_1+2\cdot (1+t)+t- (-1)=7  \Leftrightarrow x_1=7-2\cdot (1+t)-t-1=4-3t. \]
So, the solution set is:
\[ \mathbb{L}
= \left\{ \begin{pmatrix} 4-3t \\ 1+t \\t \\ -1\end{pmatrix} \, \big| \, t\in \R \right\}
= \left\{ \begin{pmatrix} 4 \\ 1 \\ 0 \\ -1\end{pmatrix}+ t\cdot \begin{pmatrix} -3\\ 1\\ 1\\ 0 \end{pmatrix} \, \big| \, t\in \R \right\}. \]}
\end{tabs*}
\end{example}




\section{\lang{de}{Gauß-Verfahren mit erweiterter Koeffizientenmatrix} \lang{en}{Gaussian elimination for the augmented matrix}}\label{sec:gauss-mit-matrizen}
\lang{de}{
Mit dem Gauß-Verfahren lässt sich jedes lineare Gleichungssystem systematisch in die beschriebene (reduzierte) Stufenform bringen. 
Es stellt eine Verallgemeinerung des Additions-Verfahrens dar.}
\lang{en}{
The Gaussian elimination is an algorithm to systematically transform any linear system into (reduced) row echelon form. It is a generalisation of the addition method.}


\begin{definition} \label{def:Gauß-Verfahren}
\lang{de}{
Für das Gauß-Verfahren können folgende sogenannte \notion{elementare Umformungen} auf die Zeilen einer Matrix angewendet werden:}
\lang{en}{
The following are called \notion{elementary row operations} and can be utilized for the Gaussian elimination:}
\begin{itemize}
\item \lang{de}{Vertauschen zweier Zeilen.} \lang{en}{Exchanging/swapping two rows}
\item \lang{de}{Multiplikation einer Zeile mit einer Konstanten $c\neq 0$ mit $c \in \mathbb{K}$.} \lang{en}{Multiplication of a row with a number $c\neq0$}
\item \lang{de}{Addition (Subtraktion) eines $c$-fachen einer Zeile zu (von) einer anderen.} \lang{en}{Addition (or subtraction) of one multiple of a row to (from) another}
\end{itemize}
\end{definition}


\lang{de}{Bevor wir das Gauß-Verfahren betrachten, zunächst eine \textit{Vorbemerkungen}:}
\lang{en}{There are some \textit{preliminary premarks} to look at before considering the Gaussian elimination:}
\lang{de}{
\begin{itemize}
    \item Zu Beginn betrachten wir die gesamte 
        erweiterte Koeffizientenmatrix.
    \item Die Schritte des Verfahrens werden wiederholt auf 
        einer Teilmatrix durchgeführt. 
        Wenn im Folgenden z.\,B. von der ersten Zeile die Rede ist, 
        dann ist hiermit immer die erste Zeile 
        der aktuell betrachteten Teilmatrix gemeint.
    \item Wir schreiben $(K)$ für alle Einträge der $k$-ten Zeile,
    also $a_{k1}, a_{k2}, \ldots, a_{kn}$.
    
    Handschriftlich adressiert man die Zeilen meist mit
    römischen Zahlen: $(I)$, $(II)$, $\ldots$
\end{itemize}}

\lang{en}{
\begin{itemize}
    \item First of all we consider the entire augmented matrix.
    \item The steps of the elimination are repeatedly applied to a submatrix.
    Hereinafter, when talking about e.g. the first row, we always talk about the first row of the currently used submatrix.
    \item We write $(K)$ for all entries in the $k$th row,
    so $a_{k1}, a_{k2}, \ldots, a_{kn}$.
    
    When solving a linear system by hand, the rows a usually labelled with roman numerals:
    römischen Zahlen: $(I)$, $(II)$, $\ldots$
\end{itemize}}

\begin{rule}[\lang{de}{Gauß-Verfahren} \lang{en}{The Gaussian elimination}] 
\lang{de}{
\begin{enumerate}
\item[1.] Suche die erste Spalte, die auch Einträge ungleich $0$ enthält. Den Index dieser Spalte bezeichnen wir nun mit $j$. 
\item[2.] Vertausche die Zeilen der Matrix ggf. so, dass in dieser Spalte der erste Koeffizient $a_{kj}$ nicht gleich $0$ ist.\\
Dieser Koeffizient entspricht dem Stufenelement dieser Zeile.
%\item Teile die erste Zeile durch den Koeffizienten der ersten Variablen.
\item[3.] Addiere jeweils geeignete Vielfache der ersten Zeile $(K)$ zu den darunter liegenden Zeilen $(L)$, 
so dass alle anderen Koeffizienten dieser Spalte gleich $0$ werden:\\
Für alle $l$ mit  $k<l$ forme die Zeile $(L)$ zu $(L)-\frac{a_{lj}}{a_{kj}}(K)$ um.
\item[4.] Die erste Zeile der aktuellen Teilmatrix wird nun 
    nicht mehr verändert. 
    Wir führen das Gauß-Verfahren ab Schritt~1 für eine neue
    Teilmatrix fort: Diese besteht aus der zweiten bis zur letzten Zeile 
    der aktuellen Teilmatrix.
    
Das Verfahren wird beendet, wenn nur noch eine Zeile übrig ist
oder die linken Seiten der restlichen Zeilen alle gleich 0 sind.
\end{enumerate}}

\lang{en}{
\begin{enumerate}
\item[1.] Find the first non-zero column. The index of this column, we now call $j$. 
\item[2.] If necessary, swap the rows so, that we have $a_{kj}\neq0$ for the first coefficient in the $j$th column. 
This coefficient is now the leading element of this row. 
%\item Teile die erste Zeile durch den Koeffizienten der ersten Variablen.
\item[3.] Add a suitable multiple of the first row $(K)$ to the rows $(L)$ below so, that all the other coefficients
in this row equal zero:\\
For all $l$ with $k<l$ transform the row $(L)$ to $(L)-\frac{a_{lj}}{a_{kj}}(K)$.
\item[4.] The first row of the current submatrix will not be changed any more. Continue (starting at step~1) with
the Gaussian elimination for a new submatrix, that consists of the second to the last row of the current submatrix.

The algorithm stops, if there is only one row left or  the left sides of the remaining rows are all 0.
\end{enumerate}}

\lang{de}{
Die erweiterte Koeffizientenmatrix liegt nun in Stufenform vor.

Um eine reduzierte Stufenform zu erhalten, verfährt man nach dieser Umformung weiter:
\begin{enumerate}
\item[5.] Bringe jedes Stufenelement auf den Wert 1:\\
Teile dazu jede Zeile, bei der die linke Seite nicht komplett $0$ ist, durch ihren Stufeneintrag:\\
Für alle entsprechenden $l$ mit  $l\geq 1$ forme die Zeile $(L)$ zu $\frac{1}{a_{lj}}(L)$ um.
\item[6.] Bringe alle Koeffizienten über einem Stufenelement auf den Wert 0.\\
Addiere dazu geeignete Vielfache der in 5. genannten Zeilen $(L)$ zu den darüber liegenden Zeilen $(K)$.\\
Für alle $k$ mit $1\leq k < l$ forme die Zeile $(K)$ zu $(K)-a_{kj}(L)$ um.
\end{enumerate}
\floatright{\href{https://api.stream24.net/vod/getVideo.php?id=10962-2-10841
&mode=iframe}{\image[75]{00_video_button_schwarz-blau}}}\\
}


\lang{en}{
We now have the augmented matrix in row echelon form.

To receive a reduced row echelon form, we must proceed with the following transformations:
\begin{enumerate}
\item[5.] Each leading element needs to have the value $1$:\\
Divide each row with a non-zero left side, by its leading element:\\
For all $l$ with  $l\geq 1$ transform the row $(L)$ to $\frac{1}{a_{lj}}(L)$.
\item[6.] All coefficients above a leading element need to have value $0$:
Add suitable multiples of the row $(L)$, mentioned in 5., to the rows $(K)$ above.\\
For all $k$ with $1\leq k < l$ transform the row $(K)$ to $(K)-a_{kj}(L)$.
\end{enumerate}}
\end{rule}

\lang{de}{
\begin{remark}
Beim handschriftlichen Lösen eines LGS wird der Algorithmus 
oft nicht streng umgesetzt. 
Um Brüche zu vermeiden, werden im ersten Schritt die Zeilen 
zum Beispiel so getauscht, dass der 
%Koeffizient $a_{ik}=1$ 
Stufeneintrag gleich 1
ist. 
Außerdem kann man in Schritt 3 beispielsweise anstelle 
der Umformung $(L)-\frac{a_{lj}}{a_{kj}}(K)$ 
auch $-a_{kj}(L) + a_{lj}(K)$ verwenden. 
Sowohl die Lösungsmenge als auch die reduzierte Stufenform 
bleiben hierbei unverändert.
\end{remark}}

\lang{en}{
\begin{remark}
When solving a linear system by hand we slightly change the algorithm for convenience.
In the first step, the rows are swapped so, that the leading element is $1$ to avoid fractions.
In step 3, it is possible to use the transformation $-a_{kj}(L) + a_{lj}(K)$ 
instead of $(L)-\frac{a_{lj}}{a_{kj}}(K)$.
Both the solution set and the reduced row echelon form stay unmodified.
\end{remark}}

\begin{example}
\begin{tabs*}
\tab{\lang{de}{1. Beispiel} \lang{en}{2. Example}}
\lang{de}{
Für das LGS 
\[  \begin{mtable}[\cellaligns{crcrcrcrcrl}]
(I)&\qquad x_1 &+& 2x_2 &+ &\phantom{3}x_3 &-  &x_4 & = & 7 & \phantom{\qquad|\,\, -2\cdot \text{(II)}}\\
(II)&  x_1  &+ &3x_2 &  &    & -  & x_4 & = & 8& \\
(III)& -x_1  &  &      &  -   &3x_3 &+   &5x_4  & = & -9& 
\end{mtable} \]
über $\mathbb{R}$ erhalten wir die folgende erweiterte Koeffizientenmatrix:
\[
\begin{pmatrix}
    1 & 2 & 1 & -1 & | & 7\\ 
    \textcolor{#CC6600}{1} & 3 & 0 & -1  &| & 8\\
    \textcolor{#CC6600}{-1} & 0 & -3 & 5 &| &  -9
 %    \rowops \add[-1]01 \add[+1]02
   \end{pmatrix}
   ~
\begin{matrix}
    \phantom{}\\
    |-(I) + (II)\\
    |(I) + (III)
\end{matrix}\]
Die erste Spalte enthält Koeffizienten ungleich $0$, sogar in  der ersten Zeile, weshalb wir keine Zeilen tauschen müssen.
Subtrahieren der ersten Zeile von der zweiten und Addieren der ersten Zeile auf die dritte liefert
\[    \rightsquigarrow~~
\begin{pmatrix}
    1 & 2 & 1 & -1 &| & 7\\ 
    0 & 1 & -1& 0  &| & 1\\
    0 & 2 & -2 & 4 & | &  -2
 %    \rowops \add[-2]12
\end{pmatrix}.\]
Damit ist der erste Durchlauf fertig und man fährt genauso mit der Matrix aus der zweiten und dritten Zeile fort, schreibt aber
weiterhin die erste Zeile mit:

\[    \rightsquigarrow~~
\begin{pmatrix}
    \textcolor{gray}{1} & \textcolor{gray}{2} & \textcolor{gray}{1} & \textcolor{gray}{-1} &\textcolor{gray}{|} & \textcolor{gray}{7}\\ 
    0 & 1 & -1& 0  &| & 1\\
    0 & \textcolor{#CC6600}{2} & -2 & 4 & | &  -2
 %    \rowops \add[-2]12
   \end{pmatrix}
   ~
\begin{matrix}
    \phantom{}\\
    \phantom{}\\
    |(-2)(II) + (III)
\end{matrix}\]

Betrachtet man also nur die kleinere Matrix, enthält die erste Spalte nur $0$, aber die zweite nicht. Auch hier sind keine 
Zeilen zu tauschen.
Dann wird das Doppelte der zweiten Zeile von der dritten subtrahiert 
\[     \rightsquigarrow~~
\begin{pmatrix}
    \textcolor{gray}{1} & \textcolor{gray}{2} & \textcolor{gray}{1} & \textcolor{gray}{-1} &\textcolor{gray}{|} & \textcolor{gray}{7}\\ 
    0 & 1 & -1  & 0&| & 1\\
    0 & 0 & 0 &4 &| &   -4
% \rowops \mult[\frac{1}{4}]2   
   \end{pmatrix}, \]
wonach auch der zweite Durchlauf fertig ist und man die Stufenform erreicht hat.

Die Stufeneinträge sind im Folgenden blau markiert:
\[    \rightsquigarrow~~
\begin{pmatrix}
    \textcolor{#0066CC}{1} & {2} & {1} &{-1} &{|} & {7}\\ 
    0 & \textcolor{#0066CC}{1} & -1  & 0&| & 1\\
    0 & 0 & 0 &\textcolor{#0066CC}{4} &| &   -4
% \rowops \mult[\frac{1}{4}]2   
   \end{pmatrix}
   ~
\begin{matrix}
    \phantom{}\\
    \phantom{}\\
    |\cdot \frac{1}{4} (III)
\end{matrix}\]


Für die reduzierte Stufenform teilen wir also die dritte Zeile durch $4$   
\[   
   \rightsquigarrow~~\begin{pmatrix}
    \textcolor{#0066CC}{1} & 2 & 1 & \textcolor{red}{-1} &| & 7\\ 
    0 & \textcolor{#0066CC}{1} & -1  & 0&| & 1\\
    0 & 0 & 0 &\textcolor{#0066CC}{1} &| &   -1
   \end{pmatrix}
   ~
\begin{matrix}
    |(III)+(I)\\
    \phantom{}\\
    \phantom{}
\end{matrix}\]
und addieren die dritte Zeile auf die erste Zeile, um die rote $1$ über dem
letzten Stufeneintrag zu eliminieren,
\[     \rightsquigarrow~~
\begin{pmatrix}
    \textcolor{#0066CC}{1} & \textcolor{#CC6600}{2} & 1 & 0 &| & 6\\ 
    0 & \textcolor{#0066CC}{1} & -1  & 0&| & 1\\
    0 & 0 & 0 &\textcolor{#0066CC}{1} &| &   -1
%     \rowops \add[-2]10
   \end{pmatrix}
   ~
\begin{matrix}
    |(-2)(II)+(I)\\
    \phantom{}\\
    \phantom{}
\end{matrix}\]
und subtrahieren schließlich das Doppelte der zweiten Zeile von der ersten,
um auch die rote $2$ zu eliminieren:
\[     \rightsquigarrow~~
\begin{pmatrix}
    \textcolor{#0066CC}{1} & 0 & 3 & 0 &| & 4\\ 
    0 & \textcolor{#0066CC}{1} & -1  & 0&| & 1\\
    0 & 0 & 0 &\textcolor{#0066CC}{1} &| &   -1
   \end{pmatrix}
\]
Als Gleichungssystem ist dies dann 
\[  \begin{mtable}[\cellaligns{crcrcrcrcrl}]
(I)&\qquad \phantom{+}x_1 &\phantom{+}& \phantom{2x_2} &+ &3x_3 &\phantom{-}  & & = & 4 &  \\
(II)&   & &x_2 & - &  x_3  &   & & = & 1 & \\
(III)&  &  &    &     & &   &\phantom{4}x_4  & = & -1& 
\end{mtable} \]
und daraus lässt sich die Lösungsmenge direkt ablesen, indem man $x_3=t$ mit $t \in \R$ setzt:
\[ \mathbb{L}
= \left\{ \begin{pmatrix} 4-3t \\ 1+t \\t \\ -1\end{pmatrix} \, \big| \, t\in \R \right\}
= \left\{ \begin{pmatrix} 4 \\ 1 \\ 0 \\ -1\end{pmatrix}+ t\cdot \begin{pmatrix} -3\\ 1\\ 1\\ 0 \end{pmatrix} \, \big| \, t\in \R \right\} \]
.}

\lang{en}{
For the linear system
\[  \begin{mtable}[\cellaligns{crcrcrcrcrl}]
(I)&\qquad x_1 &+& 2x_2 &+ &\phantom{3}x_3 &-  &x_4 & = & 7 & \phantom{\qquad|\,\, -2\cdot \text{(II)}}\\
(II)&  x_1  &+ &3x_2 &  &    & -  & x_4 & = & 8& \\
(III)& -x_1  &  &      &  -   &3x_3 &+   &5x_4  & = & -9& 
\end{mtable} \]
over $\mathbb{R}$ we have the following augmented matrix:
\[
\begin{pmatrix}
    1 & 2 & 1 & -1 & | & 7\\ 
    \textcolor{#CC6600}{1} & 3 & 0 & -1  &| & 8\\
    \textcolor{#CC6600}{-1} & 0 & -3 & 5 &| &  -9
 %    \rowops \add[-1]01 \add[+1]02
   \end{pmatrix}
   ~
\begin{matrix}
    \phantom{}\\
    |-(I) + (II)\\
    |(I) + (III)
\end{matrix}\]
The first column contains non-zero coefficients in every row, even in the first one, which is why we do not need to swap rows.
Subtracting the first row from the second and adding the first row to the thirds yields
\[    \rightsquigarrow~~
\begin{pmatrix}
    1 & 2 & 1 & -1 &| & 7\\ 
    0 & 1 & -1& 0  &| & 1\\
    0 & 2 & -2 & 4 & | &  -2
 %    \rowops \add[-2]12
\end{pmatrix}.\]
The first cycle is done and we go on the same way with the resulting submatrix, but
continue to note the first row:

\[    \rightsquigarrow~~
\begin{pmatrix}
    \textcolor{gray}{1} & \textcolor{gray}{2} & \textcolor{gray}{1} & \textcolor{gray}{-1} &\textcolor{gray}{|} & \textcolor{gray}{7}\\ 
    0 & 1 & -1& 0  &| & 1\\
    0 & \textcolor{#CC6600}{2} & -2 & 4 & | &  -2
 %    \rowops \add[-2]12
   \end{pmatrix}
   ~
\begin{matrix}
    \phantom{}\\
    \phantom{}\\
    |(-2)(II) + (III)
\end{matrix}\]

Condering only the small matrix, the first column is a zero-column, but the second is not, so they do not need to be swapped.
The double of the second row is subtracted from the third
\[     \rightsquigarrow~~
\begin{pmatrix}
    \textcolor{gray}{1} & \textcolor{gray}{2} & \textcolor{gray}{1} & \textcolor{gray}{-1} &\textcolor{gray}{|} & \textcolor{gray}{7}\\ 
    0 & 1 & -1  & 0&| & 1\\
    0 & 0 & 0 &4 &| &   -4
% \rowops \mult[\frac{1}{4}]2   
   \end{pmatrix}.\]
The second cycle is done and the matrix is in row echelon form.

The leading elements are colored blue:
\[    \rightsquigarrow~~
\begin{pmatrix}
    \textcolor{#0066CC}{1} & {2} & {1} &{-1} &{|} & {7}\\ 
    0 & \textcolor{#0066CC}{1} & -1  & 0&| & 1\\
    0 & 0 & 0 &\textcolor{#0066CC}{4} &| &   -4
% \rowops \mult[\frac{1}{4}]2   
   \end{pmatrix}
   ~
\begin{matrix}
    \phantom{}\\
    \phantom{}\\
    |\cdot \frac{1}{4} (III)
\end{matrix}\]


To receive the reduced row echelon form, we divide the third row by $4$   
\[   
   \rightsquigarrow~~\begin{pmatrix}
    \textcolor{#0066CC}{1} & 2 & 1 & \textcolor{red}{-1} &| & 7\\ 
    0 & \textcolor{#0066CC}{1} & -1  & 0&| & 1\\
    0 & 0 & 0 &\textcolor{#0066CC}{1} &| &   -1
   \end{pmatrix}
   ~
\begin{matrix}
    |(III)+(I)\\
    \phantom{}\\
    \phantom{}
\end{matrix}\]
and add the result to the first row. So the red $1$ above the last leading element is eliminated
\[     \rightsquigarrow~~
\begin{pmatrix}
    \textcolor{#0066CC}{1} & \textcolor{#CC6600}{2} & 1 & 0 &| & 6\\ 
    0 & \textcolor{#0066CC}{1} & -1  & 0&| & 1\\
    0 & 0 & 0 &\textcolor{#0066CC}{1} &| &   -1
%     \rowops \add[-2]10
   \end{pmatrix}
   ~
\begin{matrix}
    |(-2)(II)+(I)\\
    \phantom{}\\
    \phantom{}
\end{matrix}.\]
In the last step we substract the double of the first row from the first to eliminate the red $2$:
\[     \rightsquigarrow~~
\begin{pmatrix}
    \textcolor{#0066CC}{1} & 0 & 3 & 0 &| & 4\\ 
    0 & \textcolor{#0066CC}{1} & -1  & 0&| & 1\\
    0 & 0 & 0 &\textcolor{#0066CC}{1} &| &   -1
   \end{pmatrix}
\]
This yields the system of linear equations as follows
\[  \begin{mtable}[\cellaligns{crcrcrcrcrl}]
(I)&\qquad \phantom{+}x_1 &\phantom{+}& \phantom{2x_2} &+ &3x_3 &\phantom{-}  & & = & 4 &  \\
(II)&   & &x_2 & - &  x_3  &   & & = & 1 & \\
(III)&  &  &    &     & &   &\phantom{4}x_4  & = & -1& 
\end{mtable}.\]
Now the solution set can be read right away by subsituting $x_3=t$ with $t \in \R$:
\[ \mathbb{L}
= \left\{ \begin{pmatrix} 4-3t \\ 1+t \\t \\ -1\end{pmatrix} \, \big| \, t\in \R \right\}
= \left\{ \begin{pmatrix} 4 \\ 1 \\ 0 \\ -1\end{pmatrix}+ t\cdot \begin{pmatrix} -3\\ 1\\ 1\\ 0 \end{pmatrix} \, \big| \, t\in \R \right\} \]
.}
\tab{\lang{de}{2. Beispiel} \lang{en}{3. Example}}
\lang{de}{
Für das LGS über den komplexen Zahlen
\[  \begin{mtable}[\cellaligns{crcrcrcrl}]
(I)&\qquad ix_1 &+& (2+i)x_2 &+ &5x_3& = & 0 & \phantom{\qquad|\,\, -2\cdot \text{(II)}}\\
(II)&  -x_1  &+ &(-1+2i)x_2 & +  & 4ix_3 & = & 2i& \\
(III)& -3ix_1  & - &  6-3ix_2    &  -   &12x_3 &= & -6& 
\end{mtable} \]
 erhalten wir die folgende erweiterte Koeffizientenmatrix:
\[ \begin{pmatrix}
    i & 2+i & 5 & | & 0\\ 
    -1 & 2i-1 & 4i &| & 2i\\
    -3i & -6-3i & -12 &| & -6
   \end{pmatrix}  
   ~
\begin{matrix}
    \phantom{}\\
    |(-i)(I)+(II)\\
    |3(I)+(III)
\end{matrix}\]
Die erste Spalte enthält Koeffizienten ungleich $0$, sogar in  der ersten Zeile, weshalb wir keine Zeilen tauschen müssen.
Die erste Zeile wird mit $-i$ multipliziert und anschließend zu der zweiten Zeile addiert. 
Ebenso wird das Dreifache der ersten zu der dritten Zeile addiert.
\[ 
 \rightsquigarrow~~\begin{pmatrix}
    i & 2+i & 5 &| & 0\\ 
    0 & 0 & -i &| & 2i\\
    0 & 0 & 3 &| & -6
   \end{pmatrix}\]
Damit ist der erste Durchlauf fertig und man fährt genauso mit der Matrix aus der zweiten und dritten Zeile fort, schreibt aber
weiterhin die erste Zeile mit:

\[  \rightsquigarrow~~
\begin{pmatrix}
    \textcolor{gray}{i} & \textcolor{gray}{2+i} & \textcolor{gray}{5} & {|} & \textcolor{gray}{0}\\ 
    0 & 0 & -i &| & 2i\\
    0 & 0 & 3 &| & -6
   \end{pmatrix}  
   ~
\begin{matrix}
    \phantom{}\\
    \phantom{}\\
    |(-3i)(II)+(III)
\end{matrix}\]

Betrachtet man also nur die kleinere Matrix, enthält die erste und die zweite Spalte nur Nullen. 
In der dritten Spalte sind alle Koeffizienten nicht Null und müssen somit nicht getauscht werden.
Die zweite Zeile wird mit $-3i$ multipliziert und anschließend zur dritten Zeile addiert
\[    \rightsquigarrow~~
\begin{pmatrix}
        \textcolor{gray}{i} & \textcolor{gray}{2+i} & \textcolor{gray}{5} & {|} & \textcolor{gray}{0}\\
    0 & 0 & -i &| & 2i\\
    0 & 0 & 0 &| & 0
   \end{pmatrix}, \]
wonach auch der zweite Durchlauf fertig ist und man die Stufenform erreicht hat.

Die Stufeneinträge sind im Folgenden blau markiert:
\[   \rightsquigarrow~~
\begin{pmatrix}
    \textcolor{#0066CC}{i} & {2+i} & {5} &{|} & {0}\\ 
    0 & 0 & \textcolor{#0066CC}{-i}&| & 2i\\
    0 & 0 & 0 &| & 0  
   \end{pmatrix}  
   ~
\begin{matrix}
    |(-i)(I)\\
    |i(II)\\
     \phantom{}
\end{matrix}\]


Um die Stufenelemente auf $1$ zu bringen, multiplizieren wir die erste Zeile mit $-i$ und die zweite Zeile mit $i$:
\[  \rightsquigarrow~~
\begin{pmatrix}
    \textcolor{#0066CC}{1} & 1-2i &\textcolor{#CC6600}{-5i} &| & 0\\ 
    0 & 0&\textcolor{#0066CC}{1} &| & -2\\
    0 & 0 & 0 &| &0
   \end{pmatrix}  
   ~
\begin{matrix}
    |5i(II)+(I)\\
\phantom{}\\
\phantom{}
\end{matrix}\]
Anschließend addieren wir das $5i$-fache der zweiten Zeile zu der ersten Zeile:
\[   \rightsquigarrow~~
\begin{pmatrix}
    \textcolor{#0066CC}{1} & 1-2i&0 &| & -10i\\ 
    0 & 0&\textcolor{#0066CC}{1} &| & -2\\
    0 & 0 & 0 &| & 0
   \end{pmatrix} \]
Damit ist die reduzierte Stufenform erreicht.%\\

Als Gleichungssystem ist dies  
\[ 
\begin{mtable}[\cellaligns{crcrcrl}]
(I)&\qquad x_1 &+&(1-2i)x_2 &\phantom{-}  & & = & -10i &  \\
(II)&   & & &  &  x_3  & = & -2 & \\ 
\end{mtable} \]
und man erhält die Lösungsmenge
\[ \mathbb{L}
= \left\{ \begin{pmatrix} -10i-(1-2i)t \\ t \\ -2 \end{pmatrix} \, \big| \, t\in \C \right\}
= \left\{ \begin{pmatrix} -10i \\ 0 \\ -2 \end{pmatrix}+ t\cdot \begin{pmatrix} -1+2i\\ 1\\ 0 \end{pmatrix} \, \big| \, t\in \C \right\}. \]}

\lang{en}{
For the complex-valued linear system
\[  \begin{mtable}[\cellaligns{crcrcrcrl}]
(I)&\qquad ix_1 &+& (2+i)x_2 &+ &5x_3& = & 0 & \phantom{\qquad|\,\, -2\cdot \text{(II)}}\\
(II)&  -x_1  &+ &(-1+2i)x_2 & +  & 4ix_3 & = & 2i& \\
(III)& -3ix_1  & - &  6-3ix_2    &  -   &12x_3 &= & -6& 
\end{mtable} \]
the augmented matrix is given by:
\[ \begin{pmatrix}
    i & 2+i & 5 & | & 0\\ 
    -1 & 2i-1 & 4i &| & 2i\\
    -3i & -6-3i & -12 &| & -6
   \end{pmatrix}  
   ~
\begin{matrix}
    \phantom{}\\
    |(-i)(I)+(II)\\
    |3(I)+(III)
\end{matrix}\]
The first column contains non-zero coefficients, even in the first row, which is why no rows need to be exchanged.
The first row ist multiplied with $-i$ and then added to the second row.
Likewise, the triple of the first row is added to the third.

\[ 
 \rightsquigarrow~~\begin{pmatrix}
    i & 2+i & 5 &| & 0\\ 
    0 & 0 & -i &| & 2i\\
    0 & 0 & 3 &| & -6
   \end{pmatrix}\]
The first cycle is done and we continue with the submatrix consisting of the second and the third row, but keep noting down the first row:

\[  \rightsquigarrow~~
\begin{pmatrix}
    \textcolor{gray}{i} & \textcolor{gray}{2+i} & \textcolor{gray}{5} & {|} & \textcolor{gray}{0}\\ 
    0 & 0 & -i &| & 2i\\
    0 & 0 & 3 &| & -6
   \end{pmatrix}  
   ~
\begin{matrix}
    \phantom{}\\
    \phantom{}\\
    |(-3i)(II)+(III)
\end{matrix}\]

Considering only the submatrix, the first and the second column contain only zeros, but there are non-zero entries in the third column.
Therefore, nothing needs to be exchanged.
The second row is multiplied with $-3i$ and is then added to the third row
\[    \rightsquigarrow~~
\begin{pmatrix}
        \textcolor{gray}{i} & \textcolor{gray}{2+i} & \textcolor{gray}{5} & {|} & \textcolor{gray}{0}\\
    0 & 0 & -i &| & 2i\\
    0 & 0 & 0 &| & 0
   \end{pmatrix}, \]
which is the row echelon form.

The leading elements are colored blue in the following:
\[   \rightsquigarrow~~
\begin{pmatrix}
    \textcolor{#0066CC}{i} & {2+i} & {5} &{|} & {0}\\ 
    0 & 0 & \textcolor{#0066CC}{-i}&| & 2i\\
    0 & 0 & 0 &| & 0  
   \end{pmatrix}  
   ~
\begin{matrix}
    |(-i)(I)\\
    |i(II)\\
     \phantom{}
\end{matrix}\]


To get only ones as leading elements, the first row is multiplied with $-i$ and the second row with $i$:
\[  \rightsquigarrow~~
\begin{pmatrix}
    \textcolor{#0066CC}{1} & 1-2i &\textcolor{#CC6600}{-5i} &| & 0\\ 
    0 & 0&\textcolor{#0066CC}{1} &| & -2\\
    0 & 0 & 0 &| &0
   \end{pmatrix}  
   ~
\begin{matrix}
    |5i(II)+(I)\\
\phantom{}\\
\phantom{}
\end{matrix}\]
Afterwards we add $5i$-times the second row to the first:
\[   \rightsquigarrow~~
\begin{pmatrix}
    \textcolor{#0066CC}{1} & 1-2i&0 &| & -10i\\ 
    0 & 0&\textcolor{#0066CC}{1} &| & -2\\
    0 & 0 & 0 &| & 0
   \end{pmatrix} \]
Now the linear system is in reduced row echelon form.%\\

Written as a system of linear equations we have  
\[ 
\begin{mtable}[\cellaligns{crcrcrl}]
(I)&\qquad x_1 &+&(1-2i)x_2 &\phantom{-}  & & = & -10i &  \\
(II)&   & & &  &  x_3  & = & -2 & \\ 
\end{mtable} \]
and the solution set is given by
\[ \mathbb{L}
= \left\{ \begin{pmatrix} -10i-(1-2i)t \\ t \\ -2 \end{pmatrix} \, \big| \, t\in \C \right\}
= \left\{ \begin{pmatrix} -10i \\ 0 \\ -2 \end{pmatrix}+ t\cdot \begin{pmatrix} -1+2i\\ 1\\ 0 \end{pmatrix} \, \big| \, t\in \C \right\}. \]}
\end{tabs*}
\end{example}


\begin{remark}
\lang{de}{
Ist allgemein eine Matrix $A$ gegeben (die nicht von einem LGS induziert sein muss),
reden wir dennoch von Zeilenumformungen und reduzierter Stufenform der Matrix $A$.
Diese werden entsprechend dem obigen Gauß-Verfahren gebildet, wobei keine rechte Seite vorhanden ist.}
\lang{en}{
The terms "row operations" and "reduced row echelon form" may also be used for a general matrix $A$, that does not necesseraly
need to be given by a system of linear equations.
For that, the above Gaussian elimination can be applied, even if no right side exists. }
\end{remark}

\lang{de}{
Eine Matrix $A$ (bzw. ein LGS) kann durch unterschiedliche Umformungen eine andere Stufenform annehmen. 
Im Gegensatz dazu existiert genau eine dazugehörige reduzierte Stufenform.}
\lang{en}{
A matrix $A$ (or a linear system) may have different row echelon forms depending on the transformations applied.
In contrast to that, there exists exactly on corresponding reduced row echelon form.}
\label{rule:red-stufenform-eind}

\begin{theorem}
\begin{enumerate}
\item \lang{de}{Die reduzierte Stufenform einer Matrix $A$ hängt nicht von den getätigten
Zwischenschritten ab. Das heißt, wenn man im Gauß-Verfahren andere Schritte
durchführt - zum Beispiel am Anfang noch Zeilen vertauscht, um einen anderen von Null
verschiedenen Eintrag an erster Position zu erhalten -, ist die reduzierte Stufenform mit Einsen als Stufenelementen
am Ende wieder dieselbe.}
\lang{en}{The reduced row echelon form of a matrix $A$ does not depend on the intermediate steps taken.
This means, that if we perform different steps in the Gaussian elimination - e.g. exchanging rows to start with a different non-zero entry -,
the reduced row echelon form with only ones as leading elements is the same in the end.}
\item \lang{de}{Sind $A$ und $B$ zwei Matrizen, die durch Zeilenumformungen auseinander
hervorgegangen sind, so sind ihre reduzierten Stufenformen dieselben.}
\lang{en}{If $A$ and $B$ are two matrices that have been created from each other by row transformations,
then their reduced row echelon forms are the same.}
\end{enumerate}
\end{theorem}

\begin{block}[explanation]
			\begin{showhide}
            \begin{enumerate}
\item \lang{de}{Durch die Zeilenumformungen ändert sich die Lösungsmenge des entsprechenden
homogenen linearen Gleichungssystems $Ax=0$ nicht.\\
Aus der reduzierten Stufenform lässt sich die Lösungsmenge in Parameterform direkt ablesen (s. \lref{rule:loesung-parametrisiert}{unten}). 
Diese Darstellung hängt von den
als Parametern gewählten Variablen ab (siehe \ref{rule:loesung-parametrisiert}), welche aber bei dem angewendeten Verfahren
stets die möglichst letzten sind.\\
Die reduzierte Stufenform ist also immer dieselbe, egal welche elementaren Umformungen dafür gemacht wurden.}

\lang{en}{
The row operations do not change the solution set of the corresponding homogeneous linear system $Ax=0$.\\
From the reduced row echelon form, the solution set can be read directly in parameter form (see \lref{rule:loesung-parametrisiert}{below}).
The representation depends on the chosen parameters for the variables (see \ref{rule:loesung-parametrisiert}), which, however,
are always the last possible in the procedure used.\\
So, the reduced row echelon form is always the same, no matter which elementary operations were used.}

\item \lang{de}{Eine Matrix $A$ lässt sich immer in reduzierter Stufenform angeben und somit lässt sich auch jede Matrix $B$, die durch elementare Umformungen von $A$ erzeugt wurde, in dieser darstellen.
Da die reduzierte Stufenform von $A$ eindeutig ist, entspricht sie auch der von $B$.}
\lang{en}{
A matrix $A$ can always be presented in reduced row echelon form. Therefore, every matrix $B$, created from $A$ by using elementary operations,
can be displayed in reduced row echelon form too. Because the reduced row echelon form of $A$ is unique, it also corresponds to that of $B$.}\\
\end{enumerate}
\end{showhide}
\end{block}

\section{\lang{de}{Lineare Gleichungssysteme mit mehreren rechten Seiten} \lang{en}{Linear systems with several right sides}}\label{sec:mehrere-rechte-seiten}

\lang{de}{In der Praxis möchte man oft mehrere lineare Gleichungssysteme lösen, wobei die
linke Seite ($Ax$) durch das konkrete Problem stets dieselbe ist und sich lediglich die
rechte Seite ($b$) ändert. 

Mit dem Gauß-Verfahren kann man diese LGS in einer Rechnung
lösen. Man muss in der erweiterten Koeffizientenmatrix lediglich alle rechten Seiten nebeneinander aufführen.}

\lang{en}{In practise we often want to solve several linear system, in which the left side ($Ax$) is always the same because of 
the given problem and only the righ side ($b$) changes.

With the Gaussian elimination these linear systems can be solved with only one calculation. For this, all the right sides need to be listed
in the augmented matrix.}

\begin{example}
\begin{tabs*}
\tab{\lang{de}{1. Beispiel} \lang{en}{2. Example}}\ref{ex:gauss}
\lang{de}{
Zu lösen sind die linearen Gleichungssysteme}
\begin{displaymath}
\begin{mtable}[\cellaligns{ccrcrcrcr}]
(I)&\quad-&x&+&2y&+&3z&=&5\\
(II)&\quad&x&-&y&+&z&=&6\\
(III)&\quad&2x&-&3y&-&4z&=&-5
\end{mtable}\qquad \text{und} \qquad 
\begin{mtable}[\cellaligns{ccrcrcrcr}]
(I)&\quad-&x&+&2y&+&3z&=&2\\
(II)&\quad&x&-&y&+&z&=&-1\\
(III)&\quad&2x&-&3y&-&4z&=&3
\end{mtable}
\end{displaymath}
und bei beiden ist die Koeffizientenmatrix
\[ A= \begin{pmatrix}
-1 & 2 & 3 \\ 1 & -1 & 1 \\ 2 & -3 & -4 \end{pmatrix}. \]
Die Spaltenvektoren zur rechten Seite sind
\[  b=\begin{pmatrix} 5 \\ 6 \\ -5\end{pmatrix} \quad \text{bzw.} \quad 
c= \begin{pmatrix} 2 \\ -1 \\ 3\end{pmatrix}. \]
Um beide LGS gleichzeitig zu lösen (und damit Rechenaufwand zu sparen),
schreibt man eine erweiterte Koeffizientenmatrix, bei der beide Spaltenvektoren auf der rechten Seite stehen:
\[ (A \ \mid \ b\ c)=  \begin{pmatrix}
-1 & 2 & 3 & | & 5 & 2 \\ 1 & -1 & 1& | & 6 & -1  \\ 2 & -3 & -4& | & -5 & 3  \end{pmatrix}. \]
Anschließend wendet man auf diese Matrix das oben beschriebene Gauß-Verfahren an
und kann danach beide LGS in reduzierter Stufenform ablesen:
\begin{eqnarray*}
&& \begin{pmatrix} -1 & 2 & 3 & | & 5 & 2 \\ 1 & -1 & 1& | & 6 & -1  \\ 2 & -3 & -4& | & -5 & 3  \end{pmatrix} 
\begin{matrix} \text{|} \cdot (-1) \\ \phantom{1}\\ \phantom{1} \end{matrix} \rightsquigarrow 
\begin{pmatrix} 1 & -2 & -3 & | & -5 & -2 \\ 1 & -1 & 1& | & 6 & -1  \\ 2 & -3 & -4& | & -5 & 3  \end{pmatrix} 
\begin{matrix} \phantom{1} \\ \text{|} - 1\cdot (I)\\ \text{|}  -2\cdot (I) \end{matrix}  \\ &&\\
& \rightsquigarrow & \begin{pmatrix} 1 & -2 & -3 & | & -5 & -2 \\ 0 & 1 & 4& | & 11 & 1  \\ 0 & 1 & 2& | & 5 & 7  \end{pmatrix} 
\begin{matrix} \phantom{1} \\  \phantom{1}\\ \text{|} -1\cdot (II) \end{matrix} \rightsquigarrow 
\begin{pmatrix} 1 & -2 & -3 & | & -5 & -2 \\ 0 & 1 & 4& | & 11 & 1  \\ 0 & 0 & -2& | & -6 & 6  \end{pmatrix} 
\begin{matrix} \phantom{1} \\  \phantom{1}\\ \text{|}  \cdot (-1/2) \end{matrix}   \\ &&\\
& \rightsquigarrow & \begin{pmatrix} 1 & -2 & -3 & | & -5 & -2 \\ 0 & 1 & 4& | & 11 & 1  \\ 0 & 0 & 1& | & 3 & -3  \end{pmatrix} 
\begin{matrix}\text{|} +3\cdot (III) \\  \text{|}  -4\cdot (III) \\\phantom{1} \end{matrix} \rightsquigarrow 
\begin{pmatrix} 1 & -2 & 0 & | & 4 & -11 \\ 0 & 1 & 0& | & -1 & 13  \\ 0 & 0 & 1& | & 3 & -3  \end{pmatrix} 
\begin{matrix}\text{|} +2\cdot (II) \\  \phantom{1} \\ \phantom{1} \end{matrix}  \\ &&\\
& \rightsquigarrow & \begin{pmatrix} 1 & 0 & 0 & | & 2 & 15 \\ 0 & 1 & 0& | & -1 & 13  \\ 0 & 0 & 1& | & 3 & -3  \end{pmatrix}
\end{eqnarray*}
Die Lösungsmenge für das erste LGS lautet damit
\[  \mathbb{L}=\{ \left(\begin{smallmatrix}
2 \\ -1\\ 3 \end{smallmatrix}\right) \},\]
und die Lösungsmenge für das zweite LGS ist
\[  \mathbb{L}=\{ \left(\begin{smallmatrix}
15 \\ 13\\ -3 \end{smallmatrix}\right) \}.\]
%}

\lang{en}{
The following linear systems need to be solved
\begin{displaymath}
\begin{mtable}[\cellaligns{ccrcrcrcr}]
(I)&\quad-&x&+&2y&+&3z&=&5\\
(II)&\quad&x&-&y&+&z&=&6\\
(III)&\quad&2x&-&3y&-&4z&=&-5
\end{mtable}\qquad \text{und} \qquad 
\begin{mtable}[\cellaligns{ccrcrcrcr}]
(I)&\quad-&x&+&2y&+&3z&=&2\\
(II)&\quad&x&-&y&+&z&=&-1\\
(III)&\quad&2x&-&3y&-&4z&=&3
\end{mtable}.
\end{displaymath}
Both have the coefficient matrix
\[ A= \begin{pmatrix}
-1 & 2 & 3 \\ 1 & -1 & 1 \\ 2 & -3 & -4 \end{pmatrix}. \]
The column vectors of the right side are
\[  b=\begin{pmatrix} 5 \\ 6 \\ -5\end{pmatrix} \quad \text{bzw.} \quad 
c= \begin{pmatrix} 2 \\ -1 \\ 3\end{pmatrix}. \]
To solve both linear systems at a time (and save calculation effort),
we write a augmented matrix with both column vectors on the right side:
\[ (A \ \mid \ b\ c)=  \begin{pmatrix}
-1 & 2 & 3 & | & 5 & 2 \\ 1 & -1 & 1& | & 6 & -1  \\ 2 & -3 & -4& | & -5 & 3  \end{pmatrix}. \]
Afterwards we apply the Gaussian elimination from above. In the end it is possible to read both linear systems in 
reduced row echelon form:
\begin{eqnarray*}
&& \begin{pmatrix} -1 & 2 & 3 & | & 5 & 2 \\ 1 & -1 & 1& | & 6 & -1  \\ 2 & -3 & -4& | & -5 & 3  \end{pmatrix} 
\begin{matrix} \text{|} \cdot (-1) \\ \phantom{1}\\ \phantom{1} \end{matrix} \rightsquigarrow 
\begin{pmatrix} 1 & -2 & -3 & | & -5 & -2 \\ 1 & -1 & 1& | & 6 & -1  \\ 2 & -3 & -4& | & -5 & 3  \end{pmatrix} 
\begin{matrix} \phantom{1} \\ \text{|} - 1\cdot (I)\\ \text{|}  -2\cdot (I) \end{matrix}  \\ &&\\
& \rightsquigarrow & \begin{pmatrix} 1 & -2 & -3 & | & -5 & -2 \\ 0 & 1 & 4& | & 11 & 1  \\ 0 & 1 & 2& | & 5 & 7  \end{pmatrix} 
\begin{matrix} \phantom{1} \\  \phantom{1}\\ \text{|} -1\cdot (II) \end{matrix} \rightsquigarrow 
\begin{pmatrix} 1 & -2 & -3 & | & -5 & -2 \\ 0 & 1 & 4& | & 11 & 1  \\ 0 & 0 & -2& | & -6 & 6  \end{pmatrix} 
\begin{matrix} \phantom{1} \\  \phantom{1}\\ \text{|}  \cdot (-1/2) \end{matrix}   \\ &&\\
& \rightsquigarrow & \begin{pmatrix} 1 & -2 & -3 & | & -5 & -2 \\ 0 & 1 & 4& | & 11 & 1  \\ 0 & 0 & 1& | & 3 & -3  \end{pmatrix} 
\begin{matrix}\text{|} +3\cdot (III) \\  \text{|}  -4\cdot (III) \\\phantom{1} \end{matrix} \rightsquigarrow 
\begin{pmatrix} 1 & -2 & 0 & | & 4 & -11 \\ 0 & 1 & 0& | & -1 & 13  \\ 0 & 0 & 1& | & 3 & -3  \end{pmatrix} 
\begin{matrix}\text{|} +2\cdot (II) \\  \phantom{1} \\ \phantom{1} \end{matrix}  \\ &&\\
& \rightsquigarrow & \begin{pmatrix} 1 & 0 & 0 & | & 2 & 15 \\ 0 & 1 & 0& | & -1 & 13  \\ 0 & 0 & 1& | & 3 & -3  \end{pmatrix}
\end{eqnarray*}
The solution set of the first linear system ist
\[  \mathbb{L}=\{ \left(\begin{smallmatrix}
2 \\ -1\\ 3 \end{smallmatrix}\right) \},\]
and the solution set of the second linear system is
\[  \mathbb{L}=\{ \left(\begin{smallmatrix}
15 \\ 13\\ -3 \end{smallmatrix}\right) \}.\]}

\tab{\lang{de}{2. Beispiel} \lang{en}{2. Example}}
\lang{de}{
Zu lösen sind die linearen Gleichungssysteme
\begin{displaymath}
\begin{mtable}[\cellaligns{ccrcrcrcr}]
(I)&\qquad ix_1 &+& (2+i)x_2 &+ &5x_3& = & 0 \\
(II)&  -x_1  &+ &(-1+2i)x_2 & +  & 4ix_3 & = & 2i \\
(III)& -3ix_1  & + &  (-6-3i)x_2    &  -   &12x_3 &= & -6
\end{mtable}\qquad \text{und} \qquad 
\begin{mtable}[\cellaligns{ccrcrcrcr}]
(I)&\qquad ix_1 &+& (2+i)x_2 &+ &5x_3& = & 1\\
(II)&  -x_1  &+ &(-1+2i)x_2 & +  & 4ix_3 & = & i \\
(III)& -3ix_1  & + &  (-6-3i)x_2    &  -   &12x_3 &= & -3
\end{mtable}
\end{displaymath}
und bei beiden ist die Koeffizientenmatrix
\[ \begin{pmatrix}
    i & 2+i & 5  \\ 
    -1 & 2i-1 & 4i \\
    -3i & -6-3i & -12 
   \end{pmatrix}. \]
Die Spaltenvektoren zur rechten Seite sind
\[  b=\begin{pmatrix} 0\\ 2i \\ -6\end{pmatrix} \quad \text{bzw.} \quad 
c= \begin{pmatrix} 1 \\ i \\ -3\end{pmatrix}. \]
Um beide LGS gleichzeitig zu lösen (und damit Rechenaufwand zu sparen),
schreibt man eine erweiterte Koeffizientenmatrix, bei der beide Spaltenvektoren auf der rechten Seite stehen:
\[ (A \ \mid \ b\ c)=\begin{pmatrix}
    i & 2+i & 5 & | & 0 &1\\ 
    -1 & 2i-1 & 4i &| & 2i&i\\
    -3i & -6-3i & -12 &| & -6&-3
   \end{pmatrix}. \]
Anschließend wendet man auf diese Matrix das oben beschriebene Gauß-Verfahren an
und kann danach beide LGS in reduzierter Stufenform ablesen:
\begin{eqnarray*}
&& \begin{pmatrix}
    i & 2+i & 5 & | & 0 &1\\ 
    -1 & 2i-1 & 4i &| & 2i&i\\
    -3i & -6-3i & -12 &| & -6&-3
   \end{pmatrix}
   \begin{matrix} \phantom{1} \\ \text{|} -i \cdot (I) \\ \text{|} +3 \cdot (I)   \end{matrix} \rightsquigarrow 
\begin{pmatrix}
    i & 2+i & 5 &| & 0 & 1\\ 
    0 & 0 & -i &| & 2i & 0\\
    0 & 0 & 3 &| & -6 & 0
   \end{pmatrix}
\begin{matrix} \phantom{1} \\\phantom{1} \\ \text{|} -3i \cdot (II)  \end{matrix}   \\ &&\\
& \rightsquigarrow & 
\begin{pmatrix}
    i & 2+i & 5 &| & 0 & 1\\ 
    0 & 0 & -i &| & 2i & 0\\
    0 & 0 & 0 &| & 0 & 0
   \end{pmatrix}
\begin{matrix} \text{|} \cdot (-i) \\ \text{|} \cdot i\\ \phantom{1} \end{matrix} \rightsquigarrow 
\begin{pmatrix}
    1 & 1-2i &{-5i} &| & 0 & -i\\ 
    0 & 0&{1} &| & -2 & 0\\
    0 & 0 & 0 &| & 0 & 0
   \end{pmatrix}  
\begin{matrix} \text{|} +5i \cdot (II) \\ \phantom{1} \\  \phantom{1} \end{matrix}  \\ &&\\& \rightsquigarrow & 
\begin{pmatrix}
    1 & 1-2i&0 &| & -10i& -i\\ 
    0 & 0   &1 &| & -2 & 0\\
    0 & 0   &0 &| & 0 & 0
   \end{pmatrix}
\end{eqnarray*}
Die Lösungsmenge des ersten LGS lautet damit
\[  \mathbb{L}=\left\{ \begin{pmatrix} -10i-(1-2i)t \\ t \\ -2 \end{pmatrix} \, \big| \, t\in \C \right\}\]
und die Lösungsmenge des zweiten LGS ist
\[ \mathbb{L}=\left\{ \begin{pmatrix} -i-(1-2i)t \\ t \\ 0 \end{pmatrix} \, \big| \, t\in \C \right\}.\]}

\lang{en}{
The following linear systems need to be solved
\begin{displaymath}
\begin{mtable}[\cellaligns{ccrcrcrcr}]
(I)&\qquad ix_1 &+& (2+i)x_2 &+ &5x_3& = & 0 \\
(II)&  -x_1  &+ &(-1+2i)x_2 & +  & 4ix_3 & = & 2i \\
(III)& -3ix_1  & + &  (-6-3i)x_2    &  -   &12x_3 &= & -6
\end{mtable}\qquad \text{und} \qquad 
\begin{mtable}[\cellaligns{ccrcrcrcr}]
(I)&\qquad ix_1 &+& (2+i)x_2 &+ &5x_3& = & 1\\
(II)&  -x_1  &+ &(-1+2i)x_2 & +  & 4ix_3 & = & i \\
(III)& -3ix_1  & + &  (-6-3i)x_2    &  -   &12x_3 &= & -3
\end{mtable}.
\end{displaymath}
For both the coefficient matrix is
\[ \begin{pmatrix}
    i & 2+i & 5  \\ 
    -1 & 2i-1 & 4i \\
    -3i & -6-3i & -12 
   \end{pmatrix}. \]
The column vectors on the right side are
\[  b=\begin{pmatrix} 0\\ 2i \\ -6\end{pmatrix} \quad \text{bzw.} \quad 
c= \begin{pmatrix} 1 \\ i \\ -3\end{pmatrix}. \]

To solve both linear systems at a time (and save calculation effort),
we write a augmented matrix with both column vectors on the right side:
\[ (A \ \mid \ b\ c)=\begin{pmatrix}
    i & 2+i & 5 & | & 0 &1\\ 
    -1 & 2i-1 & 4i &| & 2i&i\\
    -3i & -6-3i & -12 &| & -6&-3
   \end{pmatrix}. \]
Afterwards we apply the Gaussian elimination on the matrix. Then it is possible to
read both linear systems in reduced row echelon form:
\begin{eqnarray*}
&& \begin{pmatrix}
    i & 2+i & 5 & | & 0 &1\\ 
    -1 & 2i-1 & 4i &| & 2i&i\\
    -3i & -6-3i & -12 &| & -6&-3
   \end{pmatrix}
   \begin{matrix} \phantom{1} \\ \text{|} -i \cdot (I) \\ \text{|} +3 \cdot (I)   \end{matrix} \rightsquigarrow 
\begin{pmatrix}
    i & 2+i & 5 &| & 0 & 1\\ 
    0 & 0 & -i &| & 2i & 0\\
    0 & 0 & 3 &| & -6 & 0
   \end{pmatrix}
\begin{matrix} \phantom{1} \\\phantom{1} \\ \text{|} -3i \cdot (II)  \end{matrix}   \\ &&\\
& \rightsquigarrow & 
\begin{pmatrix}
    i & 2+i & 5 &| & 0 & 1\\ 
    0 & 0 & -i &| & 2i & 0\\
    0 & 0 & 0 &| & 0 & 0
   \end{pmatrix}
\begin{matrix} \text{|} \cdot (-i) \\ \text{|} \cdot i\\ \phantom{1} \end{matrix} \rightsquigarrow 
\begin{pmatrix}
    1 & 1-2i &{-5i} &| & 0 & -i\\ 
    0 & 0&{1} &| & -2 & 0\\
    0 & 0 & 0 &| & 0 & 0
   \end{pmatrix}  
\begin{matrix} \text{|} +5i \cdot (II) \\ \phantom{1} \\  \phantom{1} \end{matrix}  \\ &&\\& \rightsquigarrow & 
\begin{pmatrix}
    1 & 1-2i&0 &| & -10i& -i\\ 
    0 & 0   &1 &| & -2 & 0\\
    0 & 0   &0 &| & 0 & 0
   \end{pmatrix}
\end{eqnarray*}
The solution set of the first linear system is
\[  \mathbb{L}=\left\{ \begin{pmatrix} -10i-(1-2i)t \\ t \\ -2 \end{pmatrix} \, \big| \, t\in \C \right\}\]
The solution set of the second linear system is
\[ \mathbb{L}=\left\{ \begin{pmatrix} -i-(1-2i)t \\ t \\ 0 \end{pmatrix} \, \big| \, t\in \C \right\}.\]}
\end{tabs*}
\end{example}
\end{content}