%$Id:  $
\documentclass{mumie.article}
%$Id$
\begin{metainfo}
  \name{
    \lang{de}{Zeilenumformungen durch Matrizenmultiplikation und Rang}
    \lang{en}{Row operations by matrix multiplication and rank}
  }
  \begin{description} 
 This work is licensed under the Creative Commons License Attribution 4.0 International (CC-BY 4.0)   
 https://creativecommons.org/licenses/by/4.0/legalcode 

    \lang{de}{Beschreibung}
    \lang{en}{}
  \end{description}
  \begin{components}
  \component{generic_image}{content/rwth/HM1/images/g_img_00_video_button_schwarz-blau.meta.xml}{00_video_button_schwarz-blau}
\component{generic_image}{content/rwth/HM1/images/g_img_00_Videobutton_schwarz.meta.xml}{00_Videobutton_schwarz}
\end{components}
  \begin{links}
    \link{generic_article}{content/rwth/HM1/T402_Lineare_Gleichungssysteme/g_art_content_05_gaussverfahren.meta.xml}{gauss}
    \link{generic_article}{content/rwth/HM1/T402_Lineare_Gleichungssysteme/g_art_content_05_gaussverfahren.meta.xml}{gauss1}
    \link{generic_article}{content/rwth/HM1/T401_Matrizenrechnung/g_art_content_03_transponierte.meta.xml}{transponierte}
    \link{generic_article}{content/rwth/HM1/T403_Quadratische_Matrizen,_Determinanten/g_art_content_07_quadratische_matrizen.meta.xml}{content_07_quadratische_matrizen}
    \link{generic_article}{content/rwth/HM1/T108_Vektorrechnung/g_art_content_30_basen_eigenschaften.meta.xml}{def_lin_unabh}
 \end{links}
  \creategeneric
\end{metainfo}
\begin{content}
\usepackage{mumie.ombplus}
\ombchapter{2}
\ombarticle{3}
\usepackage{mumie.genericvisualization}



\title{\lang{de}{Zeilenumformungen durch Matrizenmultiplikation und Rang} \lang{en}{Row operations by matrix multiplication and rank}}

\begin{block}[annotation]
 
  
\end{block}
\begin{block}[annotation]
  Im Ticket-System: \href{http://team.mumie.net/issues/11197}{Ticket 11197}\\
\end{block}

\begin{block}[info-box]
\tableofcontents
\end{block}

\section{\lang{de}{Zeilenumformung mit Matrizenmultiplikation}\lang{en}{Row operations as matrix multiplications}}\label{sec:zeilenumformungMitMatrixmul}

\lang{de}{
Beim Gauß-Verfahren im \link{gauss}{letzten Abschnitt} haben wir gelernt, dass
bei dem Verfahren in einer systematischen Reihenfolge drei Arten von
 elementaren Zeilenumformungen vorgenommen werden:
\begin{itemize}
\item Tausch zweier Zeilen
\item Multiplikation einer Zeile mit $c\in \mathbb{K}$, $c\neq 0$
\item Addition (Subtraktion) des $c$-fachen einer Zeile zu (von) einer anderen Zeile
\end{itemize}

In diesem Kapitel werden wir die elementaren Umformungsschritte
als Matrizenmultiplikation durchführen.}
%soll dies mit der Matrizenmultiplikation in 
%Zusammenhang gebracht werden.


\lang{en}{
While studying the Gaussian elimination in the \link{gauss}{last section} we learned,
that there exist three types of elementary row operations in a particular order:
 \begin{itemize}
\item Exchanging/swapping two rows
\item Multiplication of a row with $c\in \mathbb{K}$, $c\neq 0$
\item Addition (or subtraction) of one multiple of a row to (from) another
\end{itemize}

In this chapter we will undertake the elementary transformation steps as matrix multiplications.}

\lang{de}{
\begin{definition}[Elementarmatrizen]\label{rule:elementarmatrizen}
Für eine $(m\times n)$-Matrix $A$ über einem Körper $\mathbb{K}$ lassen sich die jeweiligen resultierenden Matrizen, die durch elementare Zeilenumformungen von $A$ entstehen, 
durch eine Matrixmultiplikation geeigneter $(m\times m)$-Matrizen beschreiben. Die Matrizen, die dabei jeweils von links an $A$ multipliziert werden, nennt man \notion{Elementarmatrizen}.
\begin{enumerate}
\item Für $1\leq i,j\leq m$ ist die Matrix, die man aus $A$ durch Vertauschen der $i$-ten und der $j$-ten Zeile erhält, gegeben durch
\[   V_{ij}\cdot A, \]
wobei
\[
\begin{mtable}[\cellaligns{cccc}]
 { \text{i-te Spalte}} &  &   {\text{j-te Spalte }} &   \\
  \downarrow           &  & \downarrow &
\end{mtable}
\]
\[ V_{ij} \ \ \ \ = \ \ \ \ 
\left(
\begin{matrix}
1       &        &        &        &        &            &        &   &   &        &   &  \\
        & \ddots &        &        &        &            &        &   &   &        &   &  \\ 
        &        &  1     &        &        &            &        &   &   &        &   &  \\
        &        &        & 0      &        &            &        & 1 &   &        &   &  \\
        &        &        &        & 1      &            &        &   &   &        &   &  \\
        &        &        &        &        & \ddots     &        &   &   &        &   &  \\
        &        &        &        &        &            & 1      &   &   &        &   &  \\
        &        &        & 1      &        &            &        & 0 &   &        &   &  \\
        &        &        &        &        &            &        &   & 1 &        &   &  \\
        &        &        &        &        &            &        &   &   & \ddots &   &  \\
        &        &        &        &        &            &        &   &   &        & 1 & 
\end{matrix}
\right)
\begin{mtable}[\cellaligns{c}]
\\
\\
\leftarrow {\text{i-te Zeile}} \\
\\
\\
\\
\leftarrow {\text{j-te Zeile}} \\
\\
\\
\end{mtable}
 \]
 \item Für $1\leq i\leq m$ und $c \neq 0$ ist die Matrix, die man aus $A$ durch
 Multiplikation der $i$-ten Zeile mit $c$ erhält, gegeben durch
 \[ M_{i}(c)\cdot A, \]
 wobei
 \[
\begin{mtable}[\cellaligns{cc}]
 { \text{i-te Spalte }} &\\
  \downarrow        &
\end{mtable}
\]
\[
M_{i}(c) \ = 
\left(
\begin{matrix}
1       &        &        &        &        &            &        &   &            \\
        & \ddots &        &        &        &            &        &   &            \\ 
        &        &        & 1      &        &            &        &   &  \\
        &        &        &        & c      &            &        &   &  \\
        &        &        &        &        & 1          &        &   &  \\
        &        &        &        &        &            & \ddots &   &  \\
        &        &        &        &        &            &        & 1 & 
\end{matrix}
\right)
\begin{mtable}[\cellaligns{c}]
\\
\\
\\
\leftarrow {\text{i-te Zeile}} \\
\\
\\
\\
\end{mtable}
 \]
 \item Für $1\leq i,j\leq m$ und $c \neq 0$ ist die Matrix, die man aus $A$ durch
 Addition des $c$-fachen der $j$-ten Zeile zur $i$-ten Zeile erhält, gegeben durch
 \[  A_{ij}(c)\cdot A, \]
 wobei
 \[
\begin{mtable}[\cellaligns{cc}]
     &   & { \text{j-te Spalte }}& \\
     &   & \downarrow        &
\end{mtable}
\]
\[
A_{ij}(c)=
\left(
\begin{matrix}
1       &                &        &        &            &        &   &            \\
        & \ddots         &        &        &            &        &   &            \\ 
        &                & 1      &        & c          &        &   &  \\
        &                &        & \ddots &            &        &   &  \\
        &                &        &        & 1          &        &   &  \\
        &                &        &        &            & \ddots &   &  \\
        &                &        &        &            &        & 1 & 
\end{matrix}
\right)
\begin{mtable}[\cellaligns{c}]
\\
\leftarrow {\text{i-te Zeile}} \\
\\
\\
\\
\end{mtable}
\]
\end{enumerate}
\end{definition}}


\lang{en}{
\begin{definition}[Elementary matrices]\label{rule:elementarmatrizen}
For a $(m\times n)$-matrix $A$ over a field $\mathbb{K}$ the matrices resulting from elementary row operations of $A$
can be expressed as matrix multiplications with suitabler $(m\times m)$-matrices.
The matrices, $A$ is multiplied with, are called \notion{elementary matrices}.
\begin{enumerate}
\item For $1\leq i,j\leq m$ the matrix, that results from exchanging the $i$th and the $j$th row of $A$ is given by
\[   V_{ij}\cdot A, \]
with
\[
\begin{mtable}[\cellaligns{cccc}]
 { i\text{th column}} &  &   {j\text{th column}} &   \\
  \downarrow           &  & \downarrow &
\end{mtable}
\]
\[ V_{ij} \ \ \ \ = \ \ \ \ 
\left(
\begin{matrix}
1       &        &        &        &        &            &        &   &   &        &   &  \\
        & \ddots &        &        &        &            &        &   &   &        &   &  \\ 
        &        &  1     &        &        &            &        &   &   &        &   &  \\
        &        &        & 0      &        &            &        & 1 &   &        &   &  \\
        &        &        &        & 1      &            &        &   &   &        &   &  \\
        &        &        &        &        & \ddots     &        &   &   &        &   &  \\
        &        &        &        &        &            & 1      &   &   &        &   &  \\
        &        &        & 1      &        &            &        & 0 &   &        &   &  \\
        &        &        &        &        &            &        &   & 1 &        &   &  \\
        &        &        &        &        &            &        &   &   & \ddots &   &  \\
        &        &        &        &        &            &        &   &   &        & 1 & 
\end{matrix}
\right)
\begin{mtable}[\cellaligns{c}]
\\
\\
\leftarrow {i\text{th row}} \\
\\
\\
\\
\leftarrow {j\text{th row}} \\
\\
\\
\end{mtable}
 \]
 \item For $1\leq i\leq m$ and $c \neq 0$ the matrix, that results from multiplying the $i$th row of $A$ with
  $c$ is given by
 \[ M_{i}(c)\cdot A, \]
 with
 \[
\begin{mtable}[\cellaligns{cc}]
 { \text{i-te Spalte }} &\\
  \downarrow        &
\end{mtable}
\]
\[
M_{i}(c) \ = 
\left(
\begin{matrix}
1       &        &        &        &        &            &        &   &            \\
        & \ddots &        &        &        &            &        &   &            \\ 
        &        &        & 1      &        &            &        &   &  \\
        &        &        &        & c      &            &        &   &  \\
        &        &        &        &        & 1          &        &   &  \\
        &        &        &        &        &            & \ddots &   &  \\
        &        &        &        &        &            &        & 1 & 
\end{matrix}
\right)
\begin{mtable}[\cellaligns{c}]
\\
\\
\\
\leftarrow {i\text{th row}} \\
\\
\\
\\
\end{mtable}
 \]
 \item For $1\leq i,j\leq m$ and $c \neq 0$ the matrix that results from adding the row $c\cdot j$ to the $i$th row is given by
 \[  A_{ij}(c)\cdot A, \]
 with
 \[
\begin{mtable}[\cellaligns{cc}]
     &   & { j\text{th column}}& \\
     &   & \downarrow        &
\end{mtable}
\]
\[
A_{ij}(c)=
\left(
\begin{matrix}
1       &                &        &        &            &        &   &            \\
        & \ddots         &        &        &            &        &   &            \\ 
        &                & 1      &        & c          &        &   &  \\
        &                &        & \ddots &            &        &   &  \\
        &                &        &        & 1          &        &   &  \\
        &                &        &        &            & \ddots &   &  \\
        &                &        &        &            &        & 1 & 
\end{matrix}
\right)
\begin{mtable}[\cellaligns{c}]
\\
\leftarrow {i\text{th row}} \\
\\
\\
\\
\end{mtable}
\]
\end{enumerate}
\end{definition}}

\lang{de}{
\begin{remark}
Für eine $(m\times n)$-Matrix $A$ über einem Körper $\mathbb{K}$ 
können wir durch Linksmultiplikation
mit gewissen Elementarmatrizen $S_1, \cdots, S_{\ell}$ 
die Matrix $A$ in eine Matrix $D$ in Zeilenstufenform überführen. 
Somit gilt dann
\[
D = S_{\ell} \cdot S_{{\ell}-1} \cdot \cdots \cdot S_1 \cdot A.
\]
\end{remark}}

\lang{en}{
\begin{remark}
A $(m\times n)$-matrix $A$ over a field $\mathbb{K}$ can be transformed into a matrix $D$ in row echelon form
by multiplying several elementary matrices $S_1, \cdots, S_{\ell}$ from the left with $A$.
Therefore, we have
\[
D = S_{\ell} \cdot S_{{\ell}-1} \cdot \cdots \cdot S_1 \cdot A.
\]
\end{remark}}

\lang{de}{
\begin{example}\label{ex:Elementarmatrizen3kreuz3}
Betrachten wir ein lineares Gleichungssystem mit $3$ Gleichungen und $n$ Unbekannten, das hei"st
\begin{equation*}
\begin{mtable}[\cellaligns{ccccccccc}]
 a_{11} x_1 & + & a_{12} x_2 & + & \cdots & + & a_{1n} x_n & = & b_1 \\
 a_{21} x_1 & + & a_{22} x_2 & + & \cdots & + & a_{2n} x_n & = & b_2 \\
 a_{31} x_1 & + & a_{32} x_2 & + & \cdots & + & a_{3n} x_n & = & b_3  
\end{mtable}
\end{equation*}
in allgemeiner Form, dann hat die zugehörige (erweiterte) Koeffizientenmatrix $3$ Zeilen und die Elementarmatrizen sind $(3\times 3)$-Matrizen. 
Sie sehen folgendermaßen aus:
\begin{tabs*}
 \tab{$A_{ij}(c)$} Die Matrix
       \begin{equation*}
       A_{12}(c)=\begin{pmatrix}
       1 & c & 0 \\ 0 & 1 & 0 \\ 0 & 0 & 1 
       \end{pmatrix}
       \text{ addiert das } c \text{-fache der } 2. \text{Zeile zur } 1. \text{Zeile.}
       \end{equation*}
       Die Matrix
       \begin{equation*}
       A_{13}(c)=\begin{pmatrix}
       1 & 0 & c \\ 0 & 1 & 0 \\ 0 & 0 & 1 
       \end{pmatrix}
       \text{ addiert das } c \text{-fache der } 3. \text{Zeile zur } 1. \text{Zeile.}
       \end{equation*}
       Die Matrix
       \begin{equation*}
       A_{23}(c)=\begin{pmatrix}
       1 & 0 & 0 \\ 0 & 1 & c \\ 0 & 0 & 1 
       \end{pmatrix}
       \text{ addiert das } c \text{-fache der } 3. \text{Zeile zur } 2. \text{Zeile.}
       \end{equation*}
       Die Matrix
       \begin{equation*}
       A_{21}(c)=\begin{pmatrix}
       1 & 0 & 0 \\ c & 1 & 0 \\ 0 & 0 & 1 
       \end{pmatrix}
       \text{ addiert das } c \text{-fache der } 1. \text{Zeile zur } 2. \text{Zeile.}
       \end{equation*}
       Die Matrix
       \begin{equation*}
       A_{31}(c)=\begin{pmatrix}
       1 & 0 & 0 \\ 0 & 1 & 0 \\ c & 0 & 1 
       \end{pmatrix}
       \text{ addiert das } c \text{-fache der } 1. \text{Zeile zur } 3. \text{Zeile.}
       \end{equation*}
        Die Matrix
       \begin{equation*}
       A_{32}(c)=\begin{pmatrix}
       1 & 0 & 0 \\ 0 & 1 & 0 \\ 0 & c & 1 
       \end{pmatrix}
       \text{ addiert das } c \text{-fache der } 2. \text{Zeile zur } 3. \text{Zeile.}
       \end{equation*}
 \tab{$M_i(c)$} Die Matrix
       \begin{equation*}
       M_1(c)=\begin{pmatrix}
       c & 0 & 0 \\ 0 & 1 & 0 \\ 0 & 0 & 1 
       \end{pmatrix}
       \text{ multipliziert die } 1. \text{Zeile mit } {c \in \mathbb{K} \backslash \lbrace 0 \rbrace.}
       \end{equation*}
       Die Matrix
       \begin{equation*}
       M_2(c)=\begin{pmatrix}
       1 & 0 & 0 \\ 0 & c & 0 \\ 0 & 0 & 1 
       \end{pmatrix}
       \text{ multipliziert die } 2. \text{Zeile mit } {c \in \mathbb{K} \backslash \lbrace 0 \rbrace.}
       \end{equation*}
       Die Matrix
       \begin{equation*}
       M_3(c)=\begin{pmatrix}
       1 & 0 & 0 \\ 0 & 1 & 0 \\ 0 & 0 & c 
       \end{pmatrix}
       \text{ multipliziert die } 3. \text{Zeile mit } {c \in \mathbb{K} \backslash \lbrace 0 \rbrace.}
       \end{equation*}
 \tab{$V_{ij}$} Die Matrix
       \begin{equation*}
       V_{13}=V_{31}=\begin{pmatrix}
       0 & 0 & 1 \\ 0 & 1 & 0 \\ 1 & 0 & 0 
       \end{pmatrix}
       \text{ vertauscht die } 1. \text{ und } 3. \text{Zeile.}
       \end{equation*}
       Die Matrix
       \begin{equation*}
       V_{12}=V_{21}=\begin{pmatrix}
       0 & 1 & 0 \\ 1 & 0 & 0 \\ 0 & 0 & 1 
       \end{pmatrix}
       \text{ vertauscht die } 1. \text{ und } 2. \text{Zeile.}
       \end{equation*}
       Die Matrix
       \begin{equation*}
       V_{23}=V_{32}=\begin{pmatrix}
       1 & 0 & 0 \\ 0 & 0 & 1 \\ 0 & 1 & 0 
       \end{pmatrix}
       \text{ vertauscht die } 2. \text{ und } 3. \text{Zeile.}
       \end{equation*}
\end{tabs*}
\end{example}}

\lang{en}{
\begin{example}\label{ex:Elementarmatrizen3kreuz3}
We consider a system of linear equations with $3$ equations and $n$ unknowns,
\begin{equation*}
\begin{mtable}[\cellaligns{ccccccccc}]
 a_{11} x_1 & + & a_{12} x_2 & + & \cdots & + & a_{1n} x_n & = & b_1 \\
 a_{21} x_1 & + & a_{22} x_2 & + & \cdots & + & a_{2n} x_n & = & b_2 \\
 a_{31} x_1 & + & a_{32} x_2 & + & \cdots & + & a_{3n} x_n & = & b_3  
\end{mtable}.
\end{equation*}
The corresponding (augmented) coefficient matrix has $3$ rows and the elementary matrices are $(3\times 3)$-matrices, as follows:
\begin{tabs*}
 \tab{$A_{ij}(c)$} The matrix
       \begin{equation*}
       A_{12}(c)=\begin{pmatrix}
       1 & c & 0 \\ 0 & 1 & 0 \\ 0 & 0 & 1 
       \end{pmatrix}
       \text{ adds } c \text{ times the } 2. \text{row to the } 1. \text{row.}
       \end{equation*}
       The matrix
       \begin{equation*}
       A_{13}(c)=\begin{pmatrix}
       1 & 0 & c \\ 0 & 1 & 0 \\ 0 & 0 & 1 
       \end{pmatrix}
       \text{ add  } c \text{ times the  } 3. \text{row to the } 1. \text{row.}
       \end{equation*}
       The matrix
       \begin{equation*}
       A_{23}(c)=\begin{pmatrix}
       1 & 0 & 0 \\ 0 & 1 & c \\ 0 & 0 & 1 
       \end{pmatrix}
       \text{ adds } c \text{ times the } 3. \text{row to the } 2. \text{row.}
       \end{equation*}
       The matrix
       \begin{equation*}
       A_{21}(c)=\begin{pmatrix}
       1 & 0 & 0 \\ c & 1 & 0 \\ 0 & 0 & 1 
       \end{pmatrix}
       \text{ adds } c \text{ times the } 1. \text{row to the } 2. \text{row.}
       \end{equation*}
       The matrix
       \begin{equation*}
       A_{31}(c)=\begin{pmatrix}
       1 & 0 & 0 \\ 0 & 1 & 0 \\ c & 0 & 1 
       \end{pmatrix}
       \text{ adds } c \text{ times the } 1. \text{row to the } 3. \text{row.}
       \end{equation*}
        The matrix
       \begin{equation*}
       A_{32}(c)=\begin{pmatrix}
       1 & 0 & 0 \\ 0 & 1 & 0 \\ 0 & c & 1 
       \end{pmatrix}
       \text{ adds } c \text{ times the } 2. \text{row to the } 3. \text{row.}
       \end{equation*}
 \tab{$M_i(c)$} The matrix
       \begin{equation*}
       M_1(c)=\begin{pmatrix}
       c & 0 & 0 \\ 0 & 1 & 0 \\ 0 & 0 & 1 
       \end{pmatrix}
       \text{ multiplies the } 1. \text{row with } {c \in \mathbb{K} \backslash \lbrace 0 \rbrace.}
       \end{equation*}
       The matrix
       \begin{equation*}
       M_2(c)=\begin{pmatrix}
       1 & 0 & 0 \\ 0 & c & 0 \\ 0 & 0 & 1 
       \end{pmatrix}
       \text{ multiplies the } 2. \text{row with } {c \in \mathbb{K} \backslash \lbrace 0 \rbrace.}
       \end{equation*}
       The matrix
       \begin{equation*}
       M_3(c)=\begin{pmatrix}
       1 & 0 & 0 \\ 0 & 1 & 0 \\ 0 & 0 & c 
       \end{pmatrix}
       \text{ multiplies the } 3. \text{row with } {c \in \mathbb{K} \backslash \lbrace 0 \rbrace.}
       \end{equation*}
 \tab{$V_{ij}$} The matrix
       \begin{equation*}
       V_{13}=V_{31}=\begin{pmatrix}
       0 & 0 & 1 \\ 0 & 1 & 0 \\ 1 & 0 & 0 
       \end{pmatrix}
       \text{ exchanges the } 1. \text{ and the } 3. \text{row.}
       \end{equation*}
       The Matrix
       \begin{equation*}
       V_{12}=V_{21}=\begin{pmatrix}
       0 & 1 & 0 \\ 1 & 0 & 0 \\ 0 & 0 & 1 
       \end{pmatrix}
       \text{ exchanges the } 1. \text{ and the } 2. \text{row.}
       \end{equation*}
       The matrix
       \begin{equation*}
       V_{23}=V_{32}=\begin{pmatrix}
       1 & 0 & 0 \\ 0 & 0 & 1 \\ 0 & 1 & 0 
       \end{pmatrix}
       \text{ exchanges the } 2. \text{ and the } 3. \text{Zeile.}
       \end{equation*}
\end{tabs*}
\end{example}}

\lang{de}{
\begin{quickcheck}

    \begin{variables}
    
      \randint[Z]{t}{2}{6}
        
    \end{variables}
       \text{Betrachte die Matrix $A=\begin{pmatrix} 1 & 0&0\\0&7&0\\0 &\var{t}&1\end{pmatrix}$
und die Einheitsmatrix $E$.\\
Für welche Matrix $S$, die aus einem Produkt von $(3 \times 3)$-Elementarmatrizen besteht, gilt $E=S\cdot A$?
 
 
        } 
    \begin{choices}{unique}

      \begin{choice}
        \text{$S=A_{23}(\var{t})\cdot M_2(7)\cdot V_{23}$}
        \solution{false}
      \end{choice}

      \begin{choice}
         \text{$S=A_{32}(\var{t})\cdot M_2(7)$}
        \solution{false}
        \end{choice}
      \begin{choice}
        \text{$S=A_{32}(-\var{t})\cdot M_2(\frac{1}{7})$}
        \solution{true}
      \end{choice}
            \begin{choice}
        \text{$S=V_{23}\cdot M_2(-\var{t})\cdot A_{32}(\frac{1}{7})$}
        \solution{false}
      \end{choice}
\end{choices}{unique}

 \explanation{
 \[E=\begin{pmatrix} 1 & 0&0\\0&1&0\\0 &0&1\end{pmatrix}\]
 Um aus der Matrix $A$ die Einheitsmatrix $E$ zu erhalten, multiplizieren wir zuerst die zweite Zeile mit $\frac{1}{7}$: $M_2(\frac{1}{7})$\\
 Anschließend subtrahiert man das $\var{t}$-fache der zweiten Zeile von der dritten Zeile: $A_{32}(-\var{t})$\\
 Durch Linksmultiplikation mit der Matrix $A$ erhält man damit die Einheitsmatrix $E$:
 \[E=A_{32}(-\var{t})\cdot(M_2(\frac{1}{7})\cdot A)=A_{32}(-\var{t})\cdot M_2(\frac{1}{7})\cdot A\]}
  
\end{quickcheck}}


\lang{en}{
\begin{quickcheck}

    \begin{variables}
    
      \randint[Z]{t}{2}{6}
        
    \end{variables}
       \text{Consider the matrix $A=\begin{pmatrix} 1 & 0&0\\0&7&0\\0 &\var{t}&1\end{pmatrix}$
and the elementary matrix $E$.\\
For which matrix $S$, that consists of a product of $(3 \times 3)$ elementary matrices, yields $E=S\cdot A$?
 
 
        } 
    \begin{choices}{unique}

      \begin{choice}
        \text{$S=A_{23}(\var{t})\cdot M_2(7)\cdot V_{23}$}
        \solution{false}
      \end{choice}

      \begin{choice}
         \text{$S=A_{32}(\var{t})\cdot M_2(7)$}
        \solution{false}
        \end{choice}
      \begin{choice}
        \text{$S=A_{32}(-\var{t})\cdot M_2(\frac{1}{7})$}
        \solution{true}
      \end{choice}
            \begin{choice}
        \text{$S=V_{23}\cdot M_2(-\var{t})\cdot A_{32}(\frac{1}{7})$}
        \solution{false}
      \end{choice}
\end{choices}{unique}

 \explanation{
 \[E=\begin{pmatrix} 1 & 0&0\\0&1&0\\0 &0&1\end{pmatrix}\]
 To receive the elementary matrix $E$ having $A$, we multiply the first row with $\frac{1}{7}$: $M_2(\frac{1}{7})$\\
 Afterwars we substract $\var{t}$ times the second row from the third row: $A_{32}(-\var{t})$\\
 By multiplying from the left with $A$, we get the elementary matrix $E$:
 \[E=A_{32}(-\var{t})\cdot(M_2(\frac{1}{7})\cdot A)=A_{32}(-\var{t})\cdot M_2(\frac{1}{7})\cdot A\]}
  
\end{quickcheck}}


\section{\lang{de}{Zeilenrang und Spaltenrang} \lang{en}{Row rank and column rank}}

\lang{de}{
\begin{definition}[Zeilenrang/Spaltenrang]\label{def:zeilenSpaltenRang}
Für eine $(m\times n)$-Matrix $A$ über $\mathbb{K}$ ist der \notion{Zeilenrang von $A$} die maximale Anzahl an \link{def_lin_unabh}{linear unabhängigen} Zeilenvektoren.\\
Analog ist der \notion{Spaltenrang von $A$} definiert als die maximale Anzahl an linear unabhängigen Spaltenvektoren.
\end{definition}}

\lang{en}{
\begin{definition}[row rank/column rank]\label{def:zeilenSpaltenRang}
For a  $(m\times n)$-matrix $A$ over $\mathbb{K}$ the \notion{row rank of $A$} is the  maximal number of \link{def_lin_unabh}{linear independent} row vectors.\\
Analogous the \notion{column rank of $A$} is defined as the maximal number of linear independent column vectors.
\end{definition}}

\lang{en}{
The \ref[gauss][Gaussian elimination]{sec:gauss-mit-matrizen} makes it possible to transform any matrix into
reduced row echelon form.}

\lang{de}{
Mit dem \ref[gauss][Gauß-Verfahren]{sec:gauss-mit-matrizen} kann man beliebige Matrizen durch elementare Umformungen der Zeilen auf
die reduzierte Stufenform bringen.}


\begin{remark}
\lang{de}{
Der Zeilenrang von $A$ entspricht der Anzahl der Zeilen, die ungleich der Nullzeile sind, nachdem $A$ durch das
Gauß-Verfahren auf Stufenform gebracht worden ist.\\
Der Spaltenrang von $A$ ist folglich der Zeilenrang der transponierten Matrix $A^T$.}
\lang{en}{
The row rank of $A$ is the number of non-zero rows, after the Gaussian elimination was applied.\\
The column rank of $A$ is the row rank of the transpose $A^T$.}
\end{remark}

\lang{de}{
\begin{example}
Wir betrachten die $(4\times 5)$-Matrix \[A=
\begin{pmatrix}
    1 & 2 & 1 & -1 & 7\\ 
    1 & 3 & 0 & -1 & 8\\
    -1 & 0 & -3 & 5 & 9 \\
    1 & 1 & 2 & -1 & 6
   \end{pmatrix}. \]
Durch elementare Zeilenumformungen erhalten wir   
\[ 
\rightarrow  \begin{pmatrix}
    1 & 2 & 1 & -1 & 7\\ 
    0 & 1 & -1& 0  & 1\\
    0 & 2 & -2 & 4 &16 \\
	0 & -1& 1 &  0 & -1    
   \end{pmatrix}  \rightarrow \begin{pmatrix}
     1 & 2 & 1 & -1 & 7\\ 
     0 & 1 & -1& 0  & 1\\
   	 0 & 0 & 0 &4 &  14 \\
	 0 & 0 & 0 & 0 & 0
   \end{pmatrix}.   
\]
Die Stufenform besitzt also drei Zeilen ungleich $0$, weshalb der Zeilenrang der Matrix $A$ gleich $3$ ist.
 
Für den Spaltenrang von $A$ bringen wir \[A^T= \begin{pmatrix} 1&1&-1&1 \\ 2&3&0&1\\ 1&0&-3&2\\ -1&-1&5&-1\\ 7&8&9&6   \end{pmatrix}\]
auf Stufenform:
\[ \rightarrow \begin{pmatrix} 1&1&-1&1 \\ 0&1&2&-1\\ 0&-1&-2&1\\ 0&0&4&0\\ 0&1&16&-1 \end{pmatrix}
\rightarrow \begin{pmatrix}1&1&-1&1 \\ 0&1&2&-1\\ 0&0&0&0\\ 0&0&4&0\\ 0&0&14&0\end{pmatrix}
\rightarrow \begin{pmatrix}1&1&-1&1 \\ 0&1&2&-1\\ 0&0&14&0\\ 0&0&4&0\\0&0&0&0\end{pmatrix}
\rightarrow \begin{pmatrix}1&1&-1&1 \\ 0&1&2&-1\\ 0&0&14&0\\ 0&0&0&0\\0&0&0&0\end{pmatrix}.\]
Auch hier bleiben drei Zeilen ungleich der Nullzeile. Der Spaltenrang der Matrix $A$ ist also auch gleich $3$.
\end{example}}

\lang{en}{
\begin{example}
Consider the $(4\times 5)$-matrix \[A=
\begin{pmatrix}
    1 & 2 & 1 & -1 & 7\\ 
    1 & 3 & 0 & -1 & 8\\
    -1 & 0 & -3 & 5 & 9 \\
    1 & 1 & 2 & -1 & 6
   \end{pmatrix}. \]
The elementary row operations transform it to  
\[ 
\rightarrow  \begin{pmatrix}
    1 & 2 & 1 & -1 & 7\\ 
    0 & 1 & -1& 0  & 1\\
    0 & 2 & -2 & 4 &16 \\
	0 & -1& 1 &  0 & -1    
   \end{pmatrix}  \rightarrow \begin{pmatrix}
     1 & 2 & 1 & -1 & 7\\ 
     0 & 1 & -1& 0  & 1\\
   	 0 & 0 & 0 &4 &  14 \\
	 0 & 0 & 0 & 0 & 0
   \end{pmatrix}.   
\]
The row echelon form has three non-zero rows, which is why the row rank of $A$ is $3$.

For the row rank of $A$ the transpose \[A^T= \begin{pmatrix} 1&1&-1&1 \\ 2&3&0&1\\ 1&0&-3&2\\ -1&-1&5&-1\\ 7&8&9&6   \end{pmatrix}\]
 is transformed into row echelon form:
\[ \rightarrow \begin{pmatrix} 1&1&-1&1 \\ 0&1&2&-1\\ 0&-1&-2&1\\ 0&0&4&0\\ 0&1&16&-1 \end{pmatrix}
\rightarrow \begin{pmatrix}1&1&-1&1 \\ 0&1&2&-1\\ 0&0&0&0\\ 0&0&4&0\\ 0&0&14&0\end{pmatrix}
\rightarrow \begin{pmatrix}1&1&-1&1 \\ 0&1&2&-1\\ 0&0&14&0\\ 0&0&4&0\\0&0&0&0\end{pmatrix}
\rightarrow \begin{pmatrix}1&1&-1&1 \\ 0&1&2&-1\\ 0&0&14&0\\ 0&0&0&0\\0&0&0&0\end{pmatrix}.\]
There are again three non-zero rows. So the column rank of $A$ is also $3$.
Auch hier bleiben drei Zeilen ungleich der Nullzeile. Der Spaltenrang der Matrix $A$ ist also auch gleich $3$.
\end{example}}

\lang{de}{
\begin{remark}
Um den Spaltenrang einer Matrix $A$ zu bestimmen, kann man auch die Matrix $A$
mittels Spaltenumformungen, welche entsprechend zu den Zeilenumformungen gemacht werden, auf eine Spalten-Stufenform bringen, und dann den Spaltenrang anhand der
von Null verschiedenen Spalten ablesen.\\

Bei entsprechenden Umformungsschritten ist die Spalten-Stufenform am Ende nämlich
genau die transponierte Matrix zu der Stufenform, die man aus $A^T$ durch Zeilenumformungen erhält.
\end{remark}}



\lang{en}{
\begin{remark}
To determine the column rank of a matrix $A$, we may transform $A$ into column echelon form by applying column operations, which are analog
to the row operations. The rank can the be found by countin the non-zero columns.\\

The column echelon form of $A$ is similar to the row echelon form of $A^T$.
\end{remark}}

\lang{de}{
Allgemeiner gilt für den Zeilen- und Spaltenrang der folgende Satz.
\begin{theorem}
Für jede $(m\times n)$-Matrix $A$ über $\mathbb{K}$ ist der Zeilenrang gleich dem Spaltenrang.
\end{theorem}}

\lang{en}{
In general we have the following theorem for the row and column rank:
\begin{theorem}
For each $(m\times n)$-matrix $A$ over $\mathbb{K}$ the row rank equals the column rank.
\end{theorem}}

\begin{proof*}[\lang{de}{Beweis}\lang{en}{Proof}]
\begin{showhide}
\lang{de}{
Es sei $A$ eine $(m\times n)$-Matrix und $A_z$ ihre reduzierte Zeilenstufenform, 
bzw. $A_s$ ihre reduzierte Spaltenstufenform.
Wir wissen, dass 
$\text{Zeilenrang}(A)=\text{Zeilenrang}(A_z)$ und $\text{Spaltenrang}(A)=\text{Spaltenrang}(A_s)$
gilt.\\

Um den Satz zu beweisen, gehen wir nun wie folgt vor. Zeige:}\\

\lang{en}{
Let $A$ be a $(m\times n)$-matrix and $A_r$ its reduced row echelon form and
$A_c$ its reduced column echelon form.
We know, that
$\text{row rank}(A)=\text{row rank}(A_r)$ and $\text{column rank}(A)=\text{column rank}(A_c)$
holds.\\

To proof the theorem, we go on as follows. Proof:}\\

\lang{de}{
\begin{itemize}
\item[1.] $\text{Zeilenrang}(A_z)=\text{Spaltenrang}(A_z)$
\item[2.] $\text{Zeilenrang}(A_s)=\text{Spaltenrang}(A_s)$
\item[3.] Für Spaltenumformungen einer Matrix bleibt der Zeilenrang unverändert: $\text{Zeilenrang}(A)=\text{Zeilenrang}(A_s)$
\end{itemize}

Das heißt insgesamt: 
$\text{Zeilenrang}(A)=\text{Zeilenrang}(A_s)=\text{Spaltenrang}(A_s)=\text{Spaltenrang}(A)$}


\lang{en}{
\begin{itemize}
\item[1.] $\text{row rank}(A_r)=\text{column rank}(A_c)$
\item[2.] $\text{row rank}(A_c)=\text{column rank}(A_r)$
\item[3.] The row rank does not change whil column operations: $\text{row rank}(A)=\text{row rank}(A_c)$
\end{itemize}

Altogether we have:
$\text{Zeilenrang}(A)=\text{Zeilenrang}(A_s)=\text{Spaltenrang}(A_s)=\text{Spaltenrang}(A)$}

\begin{incremental}

\step \begin{itemize}
\item[1.]
\lang{de}{
Betrachtet man zunächst eine Matrix $A_z$ in reduzierter (Zeilen-)Stufenform 
\[
A_z= \left(
\begin{matrix}
0 & \cdots & 0 & 1 & \ast & \cdots & \ast & 0    & \ast & \cdots & \ast & 0    & \ast & \cdots & \ast & 0    & \ast & \cdots & \ast \\ 
0 & \cdots & 0 & 0 & 0    & \cdots & 0    & 1    & \ast & \cdots & \ast & 0    & \ast & \cdots & \ast & 0    & \ast & \cdots & \ast \\
0 & \cdots & 0 & 0 & 0    & \cdots & 0    & 0    & 0    & \cdots & 0    & 1    & \ast & \cdots & \ast & 0    & \ast & \cdots & \ast \\
  & \vdots &   &   &      & \vdots &      &      &      & \vdots &      &      &      & \vdots &      &      &      & \vdots &      \\
0 & \cdots & 0 & 0 & 0    & \cdots & 0    & 0    & 0    & \cdots & 0    & 0    & 0    & \cdots & 0    & 1    & \ast & \cdots &      \\
0 & \cdots & 0 & 0 & 0    & \cdots & 0    & 0    & 0    & \cdots & 0    & 0    & 0    & \cdots & 0    & 0    & 0    & \cdots & 0    \\
  & \vdots &   &   &      & \vdots &      &      &      & \vdots &      &      &      & \vdots &      &      &      & \vdots &      \\
0 & \cdots & 0 & 0 & 0    & \cdots & 0    & 0    & 0    & \cdots & 0    & 0    & 0    & \cdots & 0    & 0    & 0    & \cdots & 0    
\end{matrix}
\right)\]
und bringt diese durch Spaltenumformungen auf Spalten-Stufenform, so sieht man direkt, dass die resultierende Matrix von der Form
\[ \left( \begin{matrix}
1       &        &        &        &        &            &        &   &            \\
        & \ddots &        &        &        &            &        &   &            \\ 
        &        &        & \ddots      &        &            &        &   &  \\
        &        &        &        & 1      &            &        &   &  \\
        &        &        &        &        & 0          &        &   &  \\
        &        &        &        &        &            & \ddots &   &  \\
        &        &        &        &        &            &        & 0 & 
\end{matrix} \right) \]
ist, wobei die Anzahl der Einsen gleich der Anzahl der Stufenelemente von $A$ ist.\\
Für eine solche Matrix $A$ ist also der Zeilenrang gleich dem Spaltenrang:
\[\text{Zeilenrang}(A_z)=\text{Spaltenrang}(A_z)\]}

\lang{en}{
Consider a matrix $A_r$ in reduced row echelon form 
\[
A_z= \left(
\begin{matrix}
0 & \cdots & 0 & 1 & \ast & \cdots & \ast & 0    & \ast & \cdots & \ast & 0    & \ast & \cdots & \ast & 0    & \ast & \cdots & \ast \\ 
0 & \cdots & 0 & 0 & 0    & \cdots & 0    & 1    & \ast & \cdots & \ast & 0    & \ast & \cdots & \ast & 0    & \ast & \cdots & \ast \\
0 & \cdots & 0 & 0 & 0    & \cdots & 0    & 0    & 0    & \cdots & 0    & 1    & \ast & \cdots & \ast & 0    & \ast & \cdots & \ast \\
  & \vdots &   &   &      & \vdots &      &      &      & \vdots &      &      &      & \vdots &      &      &      & \vdots &      \\
0 & \cdots & 0 & 0 & 0    & \cdots & 0    & 0    & 0    & \cdots & 0    & 0    & 0    & \cdots & 0    & 1    & \ast & \cdots &      \\
0 & \cdots & 0 & 0 & 0    & \cdots & 0    & 0    & 0    & \cdots & 0    & 0    & 0    & \cdots & 0    & 0    & 0    & \cdots & 0    \\
  & \vdots &   &   &      & \vdots &      &      &      & \vdots &      &      &      & \vdots &      &      &      & \vdots &      \\
0 & \cdots & 0 & 0 & 0    & \cdots & 0    & 0    & 0    & \cdots & 0    & 0    & 0    & \cdots & 0    & 0    & 0    & \cdots & 0    
\end{matrix}
\right)\]
and transform it into column echelon form using column transformations. The result matrix is as follows
\[ \left( \begin{matrix}
1       &        &        &        &        &            &        &   &            \\
        & \ddots &        &        &        &            &        &   &            \\ 
        &        &        & \ddots      &        &            &        &   &  \\
        &        &        &        & 1      &            &        &   &  \\
        &        &        &        &        & 0          &        &   &  \\
        &        &        &        &        &            & \ddots &   &  \\
        &        &        &        &        &            &        & 0 & 
\end{matrix} \right) \]
The number of ones equals the number of the leading elements of $A$.\\
For such a matrix $A$ the row rank equals the column rank:
\[\text{ro rank}(A_r)=\text{column rank}(A_c)\]}
\end{itemize}

\step 
\begin{itemize}
\item[2.]
\lang{de}{
Entsprechend (durch Übergang zur Transponierten) sieht man, dass auch für eine
Matrix in Spalten-Stufenform der Zeilenrang gleich dem Spaltenrang ist:
\[\text{Zeilenrang}(A_s)=\text{Spaltenrang}(A_s)\]}
\lang{en}{
Looking at the transpose we see, that the matrix is in column echelon form and that the row rank is the same
as the column rank:
\[\text{row rank}(A_c)=\text{column rank}(A_c)\]}
\end{itemize}
\step
\begin{itemize}
\item[3.]
\lang{de}{
Zuletzt zeigen wir noch, dass bei Spaltenumformungen nicht nur der Spaltenrang gleich
bleibt, sondern auch der Zeilenrang.\\
Betrachten wir also eine allgemeine $(m\times n)$-Matrix $A$ über $\mathbb{K}$ und 
eine Matrix $B$, die aus $A$ durch Spaltenumformung entstanden ist.\\
Da man Spaltenumformungen auch durch Multiplikation mit $(n\times n)$-Matrizen von rechts darstellen kann, gibt es also eine solche 
Matrix $D$ mit $B=A\cdot D$.\\
Da man Zeilenumformungen auch durch Multiplikation mit $(m\times m)$-Matrizen von links darstellen kann, gibt es des Weiteren eine 
Matrix $C$, so dass $A_z=C\cdot A$ die
reduzierte Zeilenstufenform von $A$ ist.\\
Die Matrix $C\cdot B$ erhält man also aus $B$ durch Zeilenumformungen, weshalb ihre
Zeilenränge gleich sind. Andererseits ist
\[ C\cdot B= C\cdot (A\cdot D)=(C\cdot A)\cdot D=A_z \cdot D \]
eine Matrix, die aus der reduzierten Zeilenstufenform $A_z=C\cdot A$ durch Spaltenumformungen entsteht. Da die Anzahl der Nullzeilen dabei 
nicht weniger werden
kann, ist also
\begin{eqnarray*}
 \text{Zeilenrang}(B) &=& \text{Zeilenrang}(C\cdot B)=\text{Zeilenrang}((C\cdot A)\cdot D)\\ 
 & \leq & \text{Zeilenrang}(C\cdot A)=\text{Zeilenrang}(A_z)=\text{Zeilenrang}(A).
\end{eqnarray*}
Durch Vertauschen der Rollen von $A$ und $B$ erhält man auch
\[ \text{Zeilenrang}(A) \leq  \text{Zeilenrang}(B),  \]
also insgesamt die Gleichheit.}

\lang{en}{
Lastly, we need to proof, that not only the column rank but also the row rank remain the same during column operations.\\
So we consider a general $(m\times n)$-matrix $A$ over $\mathbb{K}$ and 
a matrix $B$, which resulting from $A$ through column operations.\\
Since multiplication with $(n\times n)$-matrices from the right represent column operations, there is 
a matrix $D$ with $B=A\cdot D$.\\
Row operations are represetend by multiplication with $(m\times m)$-matrices from the left, there is a matrix
 $C$ such, that $A_z=C\cdot A$ is the reduced row echelon form of $A$.\\
 We receive $C\cdot B$ by perfoming row operations on $B$, which is why the have the same row rank. 
On the other hand we have
\[ C\cdot B= C\cdot (A\cdot D)=(C\cdot A)\cdot D=A_r \cdot D \].
This is a matrix, that is the outcome of performing column operations on $A_r=C\cdot A$. Because the number of
non-zero row can not get less, we have
\begin{eqnarray*}
 \text{row rank}(B) &=& \text{row rank}(C\cdot B)=\text{row rank}((C\cdot A)\cdot D)\\ 
 & \leq & \text{row rank}(C\cdot A)=\text{row rank}(A_r)=\text{row rank}(A).
\end{eqnarray*}
By swapping the positions of $A$ and $B$ we receive
\[ \text{row rank}(A) \leq  \text{row rank}(B),  \]
which is overall the equality.}
\end{itemize}

\step
\lang{de}{
Damit erhält man dann für eine allgemeine $(m \times n)$-Matrix $A$ über $\mathbb{K}$ mit reduzierter Spalten-Stufenform $A_s$
\[\text{Zeilenrang}(A)=\text{Zeilenrang}(A_s)=\text{Spaltenrang}(A_s)=\text{Spaltenrang}(A).\]}
\lang{en}{
For a general $(m \times n)$-matrix $A$ over $\mathbb{K}$ with its reduced column echelon form $A_c$ we have
\[\text{row rank}(A)=\text{row rank}(A_c)=\text{column rank}(A_c)=\text{column rank}(A).\]}
\end{incremental}
\end{showhide}
\end{proof*}

\lang{de}{
Wir müssen also nicht mehr zwischen dem Zeilen- und dem Spaltenrang unterscheiden.}
\lang{en}{
There is no need to distinguish the row rank from the column rank.}

\begin{definition}[\lang{de}{(voller) Rang} \lang{de}{(Full) rank}]\label{def:vollerRang}
\lang{de}{
Da der Zeilenrang und der Spaltenrang einer Matrix $A$ immer gleich sind, wird diese Zahl auch kurz \notion{Rang} von $A$ 
genannt und mit Rang($A$) bezeichnet.

Man sagt, dass eine $(m\times n)$-Matrix $A$ \notion{vollen Rang} hat, wenn der Rang von $A$ gleich der kleineren der 
beiden Zahlen $m$ und $n$ ist, d.\,h. wenn Rang($A$)$=\min\{m,n\}$ gilt.\\
\floatright{\href{https://www.hm-kompakt.de/video?watch=816}{\image[75]{00_Videobutton_schwarz}}
\href{https://www.hm-kompakt.de/video?watch=817}{\image[75]{00_Videobutton_schwarz}}}}\\\\

\lang{en}{
Since the row rank and the column rank of a matrix $A$ are always the same, this number is also called \notion{rank} of $A$
and is denoted by rank($A$).

A $(m\times n)$-matrix $A$ is said to have \notion{full rank} if the rank of $A$ is equal to the lesser of the 
two numbers $m$ and $n$, i.e. if rank($A$)$=\min\{m,n\}$ holds.}
\end{definition}

\begin{remark}
\lang{de}{
Eine quadratische $(n\times n)$-Matrix $A$ hat vollen Rang, 
wenn der Rang von $A$ gleich $n$ ist.\\
Diese Matrizen werden \notion{regulär} genannt. Wir nennen quadratische Matrizen \notion{singulär}, wenn sie keinen vollen Rang besitzen.}


\lang{en}{
A square $(n\times n)$-matrix has full rank, when the rank of $A$ equals $n$.\\
Such a matrix is called \notion{regular}. We call a square matrix \notion{singular}, if it does not have full rank.}\\
\end{remark}
\lang{de}{
\link{content_07_quadratische_matrizen}{Quadratische Matrizen} werden ausführlich im nächsten Abschnitt behandelt.}

\lang{en}{
\link{content_07_quadratische_matrizen}{Square matrices} will be discussed in detail in the next section.}
 
\begin{example}
\lang{de}{
Die $(4\times 5)$-Matrix \[A=
\begin{pmatrix}
    1 & 2 & 1 & -1 & 7\\ 
    1 & 3 & 0 & -1 & 8\\
    -1 & 0 & -3 & 5 & 9 \\
    1 & 1 & 2 & -1 & 6
   \end{pmatrix} \]
   aus obigem Beispiel hat Rang $3$. Der Rang ist also sowohl kleiner als die Zeilenanzahl, als auch kleiner als die Spaltenanzahl. 
   Daher hat $A$ keinen vollen Rang.}

\lang{en}{
The $(4\times 5)$-matrix \[A=
\begin{pmatrix}
    1 & 2 & 1 & -1 & 7\\ 
    1 & 3 & 0 & -1 & 8\\
    -1 & 0 & -3 & 5 & 9 \\
    1 & 1 & 2 & -1 & 6
   \end{pmatrix} \]
   from the above example has rank $3$. The rank is less than the number of rows and the number ob columns.
   Therefore, $A$ does not have full rank.}
   
\end{example}

\begin{quickcheck}
    \type{input.number}
      \precision{3}
      \field{real}
      \begin{variables}
           \randint[Z]{m}{2}{4}
           \randint[Z]{n}{5}{6}
           \randint[Z]{k}{5}{6}
           \randint[Z]{l}{2}{4}
\function[calculate]{k4}{4*k}
\function[calculate]{l4}{4*l}
       \end{variables}
      \text{
      \lang{de}{
      Die Matrix 
  $A=\begin{pmatrix} \var{m} & \var{n} & 1\\ 0 &\var{k}&\var{l} \\ 0 & \var{k4} & c \end{pmatrix}$
      hat für $c=$\ansref keinen vollen Rang.}
      \lang{en}{
      The matrix 
  $A=\begin{pmatrix} \var{m} & \var{n} & 1\\ 0 &\var{k}&\var{l} \\ 0 & \var{k4} & c \end{pmatrix}$
      does not have full rank for $c=$\ansref.}\\
      }
      \explanation{
      \lang{de}{$A$ ist eine $(3 \times 3)$-Matrix. Für $c=\var{l4}$ ist die dritte Zeile ein Vielfaches der zweiten Zeile.
      Damit existiert nach Zeilenumformung eine Nullzeile. Der Rang ist dadurch $\text{Rang}(A)=2$ und nicht voll.}
      \lang{en}{$A$ is a $(3 \times 3)$-matrix. For $c=\var{l4}$ the third row is a multiple of the second row.
      Because of that, there is a zero row after the row operations. Thereby the rank is $\text{rank}(A)=2$ and not full.}
      }
  
    \begin{answer}
            \solution{l4}
      \end{answer}
\end{quickcheck}

\section{\lang{de}{Rang und Lösbarkeit eines LGS} \lang{en}{Rank and solvability of a linear system}}
\lang{de}{
Mit Hilfe des Rangs einer Koeffizientenmatrix lassen sich Aussagen über die Lösbarkeit des dazugehörigen linearen Gleichungssystems treffen.}
\lang{en}{
Given the rank of a coefficient matrix, we may make statements about the solvability of the associated linear system. }

\begin{theorem}
\lang{de}{
Ein lineares Gleichungssystem über einen Körper $\mathbb{K}$ mit $m$ Gleichungen und $n$ Variablen besitzt...}
\lang{en}{A system of linear equations of a field $\mathbb{K}$ with $m$ equations and $n$ variables has...}
\begin{itemize}
\item[1.] \lang{de}{\notion{keine Lösung}, wenn der Rang der Koeffizientenmatrix nicht dem Rang der erweiterten Koeffizientenmatrix entspricht, also wenn
\[\text{Rang}(A)\neq \text{Rang}(A|b).\]}
\lang{en}{\notion{no solution}, when the rank of its coefficient matrix is equal to the rank of the augmented matrix, i.e.
\[\text{rank}(A)\neq \text{rank}(A|b).\]}

\item[2.] \lang{de}{\notion{genau eine Lösung}, wenn der Rang von Koeffizientenmatrix und von erweiterter Koeffizientenmatrix $n$ ist, also wenn
\[\text{Rang}(A)=\text{Rang}(A|b)\,\,\, \text{und}\,\,\, \text{Rang}(A)=n. \]}
\lang{en}{\notion{exactly one solution}, if the rank of its coefficient matrix and the augmented matrix both equal $n$, i.e.
\[\text{Rang}(A)=\text{rank}(A|b)\,\,\, \text{and}\,\,\, \text{rank}(A)=n. \]}

\item[3.] \lang{de}{\notion{unendlich viele Lösungen}, wenn die Ränge von Koeffizientenmatrix und erweiterter Koeffizientenmatrix zwar gleich, aber kleiner als $n$ sind, d.\,h.
\[\text{Rang}(A)=\text{Rang}(A|b)\,\,\, \text{und}\,\,\, \text{Rang}(A)<n.\]
(Dieser Punkt gilt nur, wenn der Körper $\mathbb{K}$ selbst unendlich viele Elemente enthält, zum Beispiel $\mathbb{Q},\mathbb{R}$ oder $\mathbb{C}$. 
Über einem endlichen Körper würde die Lösungsmenge in diesem Fall genau so viele Elemente enthalten wie der Körper.)} 
\lang{en}{\notion{infinitly many solutions}, if the rank of the coefficient matrix is equal to the rank of the augmented
matrix, but less than $n$, i.e.
\[\text{rank}(A)=\text{rank}(A|b)\,\,\, \text{and}\,\,\, \text{rank}(A)<n.\]
(This only holds, wenn the field $\mathbb{K}$ consists of infinitly many elements, e.g. $\Q$, $\R$ or $\C$.
Over a finite body, the solution set would contain exactly as many elements as the field.}
\end{itemize}
\end{theorem}

\begin{proof*}[\lang{de}{Beweis Theorem} \lang{en}{Proof of the theorem}]
\begin{showhide}
\begin{itemize}
\item[1.] \lang{de}{Wenn der Rang der Koeffizientenmatrix nicht dem Rang der erweiterten Koeffizientenmatrix entspricht, so hat diese die reduzierte Stufenform
 \[ \begin{pmatrix}
    1 & 0 & \cdots & 0 & \star&| & b_1\\ 
     0 & 1 & \cdots & 0 & \star&| & b_2\\ 
    \vdots & \, & \ddots & \, & \vdots&| & \vdots\\ 
 0 & 0 & \cdots & 1 & \star&| & b_k\\ 
   0 & 0 & \cdots & 0 & 0&| & \vdots \\ 
  0 & 0 & \cdots & 0 & 0&| & b_{m}\\ 
  \end{pmatrix}\]
mit mindestens einem $b_i \neq 0$ für $k<i\leq m$, was im Widerspruch zu der Nullzeile der linken Seite steht. 
Damit besitzt das dazugehörige LGS keine Lösung.\\
Für die Lösbarkeit eines LGS muss der Rang der Koeffizientenmatrix folglich immer gleich dem Rang der erweiterten Koeffizientenmatrix sein.} 

\lang{en}{If the rank of the coefficient matrix does not equal the rank of the augmented matrix, this has the reduced row echelon form
 \[ \begin{pmatrix}
    1 & 0 & \cdots & 0 & \star&| & b_1\\ 
     0 & 1 & \cdots & 0 & \star&| & b_2\\ 
    \vdots & \, & \ddots & \, & \vdots&| & \vdots\\ 
 0 & 0 & \cdots & 1 & \star&| & b_k\\ 
   0 & 0 & \cdots & 0 & 0&| & \vdots \\ 
  0 & 0 & \cdots & 0 & 0&| & b_{m}\\ 
  \end{pmatrix}\]
with at least one $b_i \neq 0$ for $k<i\leq m$, which contradicts the zero row on the left side. 
Therefore the linear system has no solution.\\
For solvability of a linear system the rank of the coefficient matrix must equal the rank of the augmented coefficient matrix.}

\item[2.] \lang{de}{Wenn der Rang der Koeffizientenmatrix $n$ ist, so ist in ihrer reduzierten Stufenform dieselbe Anzahl an Spalten und Zeilen ungleich Null.
Die reduzierte Stufenform der erweiterten Koeffizientenmatrix ist
 \[ \begin{pmatrix}
    1 & 0 & \cdots & 0 & 0&| & b_1\\ 
     0 & 1 & \cdots & 0 & 0&| & b_2\\ 
    \vdots & \, & \ddots & \, & \vdots&| & \vdots\\ 
 0 & 0 & \cdots & 1 & 0&| & b_{n-1}\\ 
  0 & 0 & \cdots & 0 & 1&| & b_n\\
    0 & 0 & \cdots & 0 & 0&| & 0\\ 
     \vdots & \, & \cdots & & \vdots&| & \vdots\\  
  0 & 0 & \cdots & 0 & 0&| & 0\\ 
  \end{pmatrix}.\]
Das heißt, es existieren $n$ Stufenelemente (Einsen) zu $n$ Variablen und damit ist keine von ihnen eine freie Variable.\\
Eine eindeutige Lösung lässt sich hier direkt ablesen.}

\lang{en}{If the rank of the coefficient matrix is $n$, in reduced row echelon form the number of non-zero rows equals the number of non-zero columns.
The reduced row echelon form of the augmented matrix is
 \[ \begin{pmatrix}
    1 & 0 & \cdots & 0 & 0&| & b_1\\ 
     0 & 1 & \cdots & 0 & 0&| & b_2\\ 
    \vdots & \, & \ddots & \, & \vdots&| & \vdots\\ 
 0 & 0 & \cdots & 1 & 0&| & b_{n-1}\\ 
  0 & 0 & \cdots & 0 & 1&| & b_n\\
    0 & 0 & \cdots & 0 & 0&| & 0\\ 
     \vdots & \, & \cdots & & \vdots&| & \vdots\\  
  0 & 0 & \cdots & 0 & 0&| & 0\\ 
  \end{pmatrix}.\]
There exist $n$ leading elements (ones) for $n$ variables and therefore non of them is a free variable.
A unique solution can be read directly here.}

\item[3.] \lang{de}{Wenn der Rang der (erweiterten) Koeffizientenmatrix $k$ kleiner als $n$ ist, so existieren im Gegensatz zu 2. freie Variablen. 
 \[ \begin{pmatrix}
    1 & 0 & \cdots & 0 & \star&| & b_1\\ 
     0 & 1 & \cdots & 0 & \star&| & b_2\\ 
    \vdots & \, & \ddots & & \vdots&| & \vdots\\ 
 0 & 0 & \cdots & 1 & \star&| & b_k\\ 
    0 & 0 & \cdots & 0 & 0&| & 0\\ 
     \vdots & \, & \cdots & \, & \vdots&| & \vdots\\  
  0 & 0 & \cdots & 0 & 0&| & 0\\ 
  \end{pmatrix}\]
Da freie Variablen als Parameter in $\mathbb{K}$ gesetzt werden können, existieren unendlich viele Lösungen.}

\lang{en}{If the rank of the augmented matrix $k$ is less than $n$, there exist free variables (in contrast to 2.). 
 \[ \begin{pmatrix}
    1 & 0 & \cdots & 0 & \star&| & b_1\\ 
     0 & 1 & \cdots & 0 & \star&| & b_2\\ 
    \vdots & \, & \ddots & & \vdots&| & \vdots\\ 
 0 & 0 & \cdots & 1 & \star&| & b_k\\ 
    0 & 0 & \cdots & 0 & 0&| & 0\\ 
     \vdots & \, & \cdots & \, & \vdots&| & \vdots\\  
  0 & 0 & \cdots & 0 & 0&| & 0\\ 
  \end{pmatrix}\]
  Because fre
Since free variables can be set as parameters in $\mathbb{K}$, there exist infinitely many solutions.}
\end{itemize}
\end{showhide}
\end{proof*}

\begin{example}
\begin{tabs*}
\tab{\lang{de}{1. Beispiel} \lang{en}{1. Example}}
\lang{de}{
Für das lineare Gleichungssystem
\begin{displaymath}
\begin{mtable}[\cellaligns{ccrcrcrcr}]
(I)&\qquad-&x&+&2y&+&3z&=&5\\
(II)&&&&y&+&4z&=&11\\
(III)&&&&&-&2z&=&-6
\end{mtable}
\end{displaymath}
mit erweiterter Koeffizientenmatrix 
\[ \begin{pmatrix} -1 & 2 & 3 & | & 5 \\0 & 1 & 4 & | & 11 \\0 & 0 & -2 & | & -6 \end{pmatrix} \]
gilt $\text{Rang}(A)=\text{Rang}(A|b)=3$, was auch der Anzahl der dazugehörigen Variablen entspricht. Das heißt, das LGS besitzt genau eine Lösung:

\[\mathbb{L}=\left\{\begin{pmatrix}2\\ -1\\ 3\end{pmatrix} \right\}\]
(siehe \link{gauss1}{letztes Kapitel}).}

\lang{en}{
For the linear system
\begin{displaymath}
\begin{mtable}[\cellaligns{ccrcrcrcr}]
(I)&\qquad-&x&+&2y&+&3z&=&5\\
(II)&&&&y&+&4z&=&11\\
(III)&&&&&-&2z&=&-6
\end{mtable}
\end{displaymath}
with its augmented matrix 
\[ \begin{pmatrix} -1 & 2 & 3 & | & 5 \\0 & 1 & 4 & | & 11 \\0 & 0 & -2 & | & -6 \end{pmatrix} \]
we have $\text{rank}(A)=\text{rank}(A|b)=3$, which is also the number of the corresponding variables. 
This means that the LGS has exactly one solution:

\[\mathbb{L}=\left\{\begin{pmatrix}2\\ -1\\ 3\end{pmatrix} \right\}\]
(see \link{gauss1}{last chapter}).}
\\

\tab{\lang{de}{2. Beispiel} \lang{en}{2. Example}}
\lang{de}{
Für das komplexwertige lineare Gleichungssystem
\[ \begin{mtable}[\cellaligns{crcrcr}]
(I)&\qquad x & - & (2+i) \cdot  y & = & 5 \\
(II)& &  & 0 & =  & 0
\end{mtable} \]
mit erweiterter Koeffizientenmatrix 
\[ \begin{pmatrix} 1 & -2-i & | & 5 \\0 &0 & | & 0 \end{pmatrix} \]
gilt $\text{Rang}(A)=\text{Rang}(A|b)=1$. Das LGS besitzt aber 2 Variablen und hat damit unendlich viele Lösungen:

\[ \mathbb{L}= \left\{ \begin{pmatrix}5\\ 0 \end{pmatrix}+r\cdot \begin{pmatrix} 2+i \\ 1 \end{pmatrix} \mid\, r\in \C \right\} \] 
(siehe \link{gauss1}{letztes Kapitel}).}

\lang{en}{
For the complex-valued linear system
\[ \begin{mtable}[\cellaligns{crcrcr}]
(I)&\qquad x & - & (2+i) \cdot  y & = & 5 \\
(II)& &  & 0 & =  & 0
\end{mtable} \]
with its augmented matrix 
\[ \begin{pmatrix} 1 & -2-i & | & 5 \\0 &0 & | & 0 \end{pmatrix} \]
we have $\text{rank}(A)=\text{rank}(A|b)=1$. But the linear system hast 2 variables and therefore iniftly many solutions:

\[ \mathbb{L}= \left\{ \begin{pmatrix}5\\ 0 \end{pmatrix}+r\cdot \begin{pmatrix} 2+i \\ 1 \end{pmatrix} \mid\, r\in \C \right\} \] 
(see \link{gauss1}{last chapter}).}

\tab{\lang{de}{3. Beispiel} \lang{en}{3. Example}}
\lang{de}{
Für das lineare Gleichungssystem
\begin{displaymath}
\begin{mtable}[\cellaligns{ccrcrcrcr}]
(I)&\qquad-&x&+&2y&+&3z&=&5\\
(II)&&&&y&+&4z&=&11\\
(III)&&&&&&0&=&-6
\end{mtable}
\end{displaymath}
mit erweiterter Koeffizientenmatrix 
\[ \begin{pmatrix} -1 & 2 & 3 & | & 5 \\0 & 1 & 4 & | & 11 \\0 & 0 & 0 & | & -6 \end{pmatrix} \]
gilt $\text{Rang}(A)=2$ und $\text{Rang}(A|b)=3$.\\
Das heißt, das LGS ist nicht lösbar und 
\[\mathbb{L}=\emptyset.\]}

\lang{en}{
For the linear system
\begin{displaymath}
\begin{mtable}[\cellaligns{ccrcrcrcr}]
(I)&\qquad-&x&+&2y&+&3z&=&5\\
(II)&&&&y&+&4z&=&11\\
(III)&&&&&&0&=&-6
\end{mtable}
\end{displaymath}
with its augmented matrix 
\[ \begin{pmatrix} -1 & 2 & 3 & | & 5 \\0 & 1 & 4 & | & 11 \\0 & 0 & 0 & | & -6 \end{pmatrix} \]
we have $\text{rank}(A)=2$ and $\text{rank}(A|b)=3$.\\
Therefore the linear system is not solvable and 
\[\mathbb{L}=\emptyset.\]}

\end{tabs*}
\end{example}




\end{content}