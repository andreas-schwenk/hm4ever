\documentclass{mumie.problem.gwtmathlet}
%$Id$
\begin{metainfo}
  \name{
    \lang{de}{A02: Lösungsmenge}
    \lang{en}{P02: Solution set}
  }
  \begin{description} 
 This work is licensed under the Creative Commons License Attribution 4.0 International (CC-BY 4.0)   
 https://creativecommons.org/licenses/by/4.0/legalcode 

    \lang{de}{Beschreibung}
    \lang{en}{}
  \end{description}
  \corrector{system/problem/GenericCorrector.meta.xml}
  \begin{components}
    \component{js_lib}{system/problem/GenericMathlet.meta.xml}{mathlet}
  \end{components}
  \begin{links}
  \end{links}
  \creategeneric
\end{metainfo}
\begin{content}
\usepackage{mumie.genericproblem}

\lang{de}{
	\title{A02: Lösungsmenge}
}
\lang{en}{
	\title{P02: Solution set}
}


\begin{block}[annotation]
  Im Ticket-System: \href{http://team.mumie.net/issues/11295}{Ticket 11295}
\end{block}

\begin{problem}

	\randomquestionpool{1}{3}

	\begin{question} %1. Frage 
        \lang{de}{\text{Welche Lösungsmengen sind für das folgende lineare Gleichungssystem über $\R$ korrekt?\\
        %\begin{align}
        %\var{a}x&+\var{b}y&\var{c}z&=&\var{j}\\
        %\var{d}x&\var{d1}y&+\var{f}z&=&\var{k}\\
        %\var{g}x&\var{h}y&+\var{i1}z&=&\var{l}
        %\end{align}        
        $~~~~~
            \begin{pmatrix}
               \var{a} & \var{b}&  \var{c} &|& \var{j}\\
               \var{d} & \var{d1}& \var{f} &|& \var{k}\\
               \var{g} & \var{h}&  \var{i1} &|& \var{l}
            \end{pmatrix}
        $
        }}

        \lang{en}{\text{Which solution sets are correct for the following linear system over $\R$?\\
        %\begin{align}
        %\var{a}x&+\var{b}y&\var{c}z&=&\var{j}\\
        %\var{d}x&\var{d1}y&+\var{f}z&=&\var{k}\\
        %\var{g}x&\var{h}y&+\var{i1}z&=&\var{l}
        %\end{align}        
        $~~~~~
            \begin{pmatrix}
               \var{a} & \var{b}&  \var{c} &|& \var{j}\\
               \var{d} & \var{d1}& \var{f} &|& \var{k}\\
               \var{g} & \var{h}&  \var{i1} &|& \var{l}
            \end{pmatrix}
        $
        }}

        %\lang{en}{\text{Which solution sets are correct for the following linear system?\\
        %\begin{align}
        %\var{a}x&+\var{b}y&\var{c}z&=&\var{j}\\
        %\var{d}x&\var{d1}y&+\var{f}z&=&\var{k}\\
        %\var{g}x&\var{h}y&+\var{i1}z&=&\var{l}
        %\end{align}
        %}}
        \lang{de}{\explanation{
            Bringen Sie das lineare Gleichungssystem zunächst mit dem Gauß-Verfahren
            auf Stufenform.\\
            Nun erkennt man, ob (und wenn ja, wieviele) freie Variablen
            gewählt werden müssen.\\
            Nun können die noch in Frage kommenden Auswahlmöglichkeiten 
            durch Einsetzen überprüft werden.}}
        \lang{en}{\explanation{Reduce the linear system to row echelon form using Gaussian elimination.\\
        Now we see whether we need to define free variables (and if so, how many).\\
        The options can be checked by inserting them into the linear system.}}
        %\lang{en}{\explanation{Reduce the linear system to row echelon form with Gaussian Elimination in order to see how many (if any) and which variables are free choices.\\
        %If you know that, you can identify wrong options and check the remaining ones via substitution.}}
        \permutechoices{1}{7}
        \type{mc.multiple}
        \field{rational}
        
        \begin{variables}
            % zuvor: 5 statt 3:
            \randint[Z]{a}{-3}{3}
            \randint[Z]{b}{0}{3}
            \randint[Z]{c}{-3}{0}
            \randint[Z]{d}{-3}{3}
            \randint[Z]{d1}{-3}{0}
            \randint[Z]{f}{0}{3}
            
            \randadjustIf{a,b,d,d1,c,f}{a*d1-b*d=0 OR a*f-d*c=0 OR b*f-d1*c=0}
                               
            \randint[Z]{m}{-3}{-1}
            \randint[Z]{n}{1}{3}
                                    
            \function[calculate]{g}{m*a+n*d}
            \function[calculate]{h}{m*b+n*d1}
            \function[calculate]{i1}{m*c+n*f}
                     
            % zuvor 5 statt 3
            \randint{r}{-3}{3}
            \randint{s}{-3}{3}
            \randint{q}{-3}{3}
            
            \function[calculate]{j}{a*r+b*s+c*q}
            \function[calculate]{k}{d*r+d1*s+f*q} 
            \function[calculate]{l}{g*r+h*s+i1*q}
            
            \function[calculate]{p1}{(j*d1-k*b)/(a*d1-b*d)-2*j/a+(1-abs(sign(j)))/a}
            \function[calculate]{p2}{(f*b-c*d1)/(a*d1-b*d)}
            \function[calculate]{p3}{(a*k-d*j)/(a*d1-b*d)}
            \function[calculate]{p4}{(d*c-a*f)/(a*d1-b*d)}
                      
            \function{lsg1}{p1+p2*t}
            \function{lsg2}{p3+p4*t}
            \function{lsg3}{t}     
            
            \function[calculate]{p5}{(j*f-k*c)/(a*f-c*d)}
            \function[calculate]{p6}{(d1*c-b*f)/(a*f-c*d)}
            \function[calculate]{p7}{(a*k-d*j)/(a*f-c*d)}
            \function[calculate]{p8}{(d*b-a*d1)/(a*f-c*d)}
                      
            \function{lsg4}{p5+p6*u}
            \function{lsg5}{u}
            \function{lsg6}{p7+p8*u}  
            
            \function[calculate]{p9}{((j+2*c)*d1-(k+2*f)*b)/(a*d1-b*d)}
            \function[calculate]{p10}{(b*f-c*d1)/(a*d1-b*d)}
            \function[calculate]{p11}{(a*(k+2*f)-d*(j+2*c))/(a*d1-b*d)}
            \function[calculate]{p12}{(c*d-a*f)/(a*d1-b*d)}
                      
            \function{lsg7}{p9+p10*v}
            \function{lsg8}{p11+p12*v}
            \function{lsg9}{-2+v}
            
            \function[calculate]{p13}{j/a}
            \function[calculate]{p14}{-b/a}
            \function[calculate]{p15}{-c/a}
                                  
            \function{lsg10}{p13+p14*u+p15*v}
            \function{lsg11}{u}
            \function{lsg12}{v}            
		            
			\function[calculate]{p20}{((j-a)*f-(k-d)*c)/(b*f-c*d1)}
            \function[calculate]{p21}{(d*c-a*f)/(b*f-c*d1)}
            \function[calculate]{p22}{(b*(k-d)-d1*(j-a))/(b*f-c*d1)}
            \function[calculate]{p23}{(d1*a-b*d)/(b*f-c*d1)}
                      
            \function{lsg19}{1+v}
            \function{lsg20}{p20+p21*v}
            \function{lsg21}{p22+p23*v}   
        \end{variables}

		\begin{choice}
              \lang{de}{\text{$\mathbb{L}=\{(\var{r}; \var{s}; \var{q})\}$}}
              \lang{en}{\text{$\mathbb{L}=\{(\var{r}, \var{s}, \var{q})\}$}}
              \solution{false}
        \end{choice}
        
        \begin{choice}
              \text{$\mathbb{L}=\emptyset$}
              \solution{false}
        \end{choice}
        
        \begin{choice}
              \lang{de}{\text{$\mathbb{L}=\{(\var{lsg1}; \var{lsg2}; \var{lsg3}) ~|~ t\in\R\}$}}
              \lang{en}{\text{$\mathbb{L}=\{(\var{lsg1}, \var{lsg2}, \var{lsg3}) ~|~ t\in\R\}$}}
              \solution{false}
        \end{choice}
        
        \begin{choice}
              \lang{de}{\text{$\mathbb{L}=\{(\var{lsg4}; \var{lsg5}; \var{lsg6}) ~|~ u\in\R\}$}}
              \lang{en}{\text{$\mathbb{L}=\{(\var{lsg4}, \var{lsg5}, \var{lsg6}) ~|~ u\in\R\}$}}
              \solution{true}
        \end{choice}
        
        \begin{choice}
              \lang{de}{\text{$\mathbb{L}=\{(\var{lsg7}; \var{lsg8}; \var{lsg9}) ~|~ v\in\R\}$}}
              \lang{en}{\text{$\mathbb{L}=\{(\var{lsg7}, \var{lsg8}, \var{lsg9}) ~|~ v\in\R\}$}}
              \solution{true}
        \end{choice}
        
        \begin{choice}
              \lang{de}{\text{$\mathbb{L}=\{(\var{lsg10}; \var{lsg11}; \var{lsg12}) ~|~ u,v\in\R\}$}}
              \lang{en}{\text{$\mathbb{L}=\{(\var{lsg10}, \var{lsg11}, \var{lsg12}) ~|~ u,v\in\R\}$}}
              \solution{false}
        \end{choice}
        
        \begin{choice}
              \lang{de}{\text{$\mathbb{L}=\{(\var{lsg19}; \var{lsg20}; \var{lsg21}) ~|~ v\in\R\}$}}
              \lang{en}{\text{$\mathbb{L}=\{(\var{lsg19}, \var{lsg20}, \var{lsg21}) ~|~ v\in\R\}$}}
              \solution{true}
        \end{choice}
        
	\end{question} %CHECKED

	\begin{question} %2. Frage
        \lang{de}{\text{Welche Lösungsmengen sind für das folgende lineare Gleichungssystem über $\R$ korrekt?\\
        %\begin{align}
        %\var{a}x&+\var{b}y&\var{c}z&=&\var{j}\\
        %\var{d}x&+\var{d1}y&\var{f}z&=&\var{k}\\
        %\var{g}x&\var{h}y&+\var{i1}z&=&\var{l}
        %\end{align}
        $~~~~~
            \begin{pmatrix}
                \var{a}&\var{b}&\var{c}&|&\var{j}\\
                \var{d}&\var{d1}&\var{f}&|&\var{k}\\
                \var{g}&\var{h}&\var{i1}&|&\var{l}
            \end{pmatrix}
        $        
        }}
        \lang{en}{\text{Which solution sets are correct for the following linear system?\\
        \begin{align}
        \var{a}x&+\var{b}y&\var{c}z&=&\var{j}\\
        \var{d}x&+\var{d1}y&\var{f}z&=&\var{k}\\
        \var{g}x&\var{h}y&+\var{i1}z&=&\var{l}
        \end{align}
        }}
        \lang{de}{\explanation{
            Bringen Sie das lineare Gleichungssystem zunächst mit dem Gauß-Verfahren
            auf Stufenform.\\
            Nun erkennt man, ob (und wenn ja, wieviele) freie Variablen
            gewählt werden müssen.\\
            Nun können die noch in Frage kommenden Auswahlmöglichkeiten 
            durch Einsetzen überprüft werden.}}
        \lang{en}{\explanation{Reduce the linear system to row echelon form using Gaussian elimination.\\
        Now we see whether we need to define free variables (and if so, how many).\\
        The options can be checked by inserting them into the linear system.}}
        
        \permutechoices{1}{7}
        \type{mc.multiple}
        \field{rational}
        
        \begin{variables}
            % zuvor 6 statt 3
            \randint[Z]{x1}{0}{3}
            \randint[Z]{x2}{0}{3}
            \randint[Z]{x3}{-3}{0}
            
            \randadjustIf{x1,x2,x3}{(x1-x2)*(x1-x3)*(x2-x3)=0}
            
            \function[calculate]{a}{x1*x1}
            \function[calculate]{b}{x1*x2}
            \function[calculate]{c}{x1*x3}
            \function[calculate]{d}{x2*x1}
            \function[calculate]{d1}{x2*x2}
            \function[calculate]{f}{x2*x3}         
            \function[calculate]{g}{x3*x1}
            \function[calculate]{h}{x3*x2}
            \function[calculate]{i1}{x3*x3}
                     
            % zuvor 5 statt 3
            \randint{r}{-3}{3}
            \randint{s}{-3}{3}
            \randint{q}{-3}{3}
            
            \function[calculate]{j}{a*r+b*s+c*q}
            \function[calculate]{k}{d*r+d1*s+f*q} 
            \function[calculate]{l}{g*r+h*s+i1*q}        
            
            \function[calculate]{p1}{(j-b*s)/a}%%%
            \function[calculate]{p2}{-c/a}%%%
                      
            \function{lsg1}{p1+p2*u}%%%
            \function{lsg2}{s}%%%
            \function{lsg3}{u}%%%     
            
            \function[calculate]{p3}{-j/a+(1-abs(sign(j)))/a}%%%
            \function[calculate]{p4}{-b/a}%%%
            \function[calculate]{p5}{-c/a}%%%
                      
            \function{lsg4}{p3+p4*v+p5*w}%%%
            \function{lsg5}{v}%%%
            \function{lsg6}{w}%%%  
            
            \function[calculate]{p6}{j/c}%%%
            \function[calculate]{p7}{-a/c}%%%
            \function[calculate]{p8}{-b/c}%%%
                      
            \function{lsg7}{t}%%%
            \function{lsg8}{u}%%%
            \function{lsg9}{p6+p7*t+p8*u}%%%
            
            \function[calculate]{p9}{(j-a)/b}%%%	
            \function[calculate]{p10}{-a/b}%%%
            \function[calculate]{p11}{-c/b}%%%
                                  
            \function{lsg10}{1+u}%%%
            \function{lsg11}{p9+p10*u+p11*v}%%%
            \function{lsg12}{v}%%%
            
            \function[calculate]{p16}{j/b}%%%
            \function[calculate]{p17}{-(c+a)/b}%%%
                                  
            \function{lsg13}{t}%%%
            \function{lsg14}{p16+p17*t}%%%
            \function{lsg15}{t}%%%
        \end{variables}

		\begin{choice} %ok
              \lang{de}{\text{$\mathbb{L}=\{(\var{r}; \var{s}; \var{q})\}$}}
              \lang{en}{\text{$\mathbb{L}=\{(\var{r}, \var{s}, \var{q})\}$}}
              \solution{false}
        \end{choice}
        
        \begin{choice} %ok
              \text{$\mathbb{L}=\emptyset$}
              \solution{false}
        \end{choice}
        
        \begin{choice} %ok
              \lang{de}{\text{$\mathbb{L}=\{(\var{lsg1}; \var{lsg2}; \var{lsg3}) ~|~ u\in\R\}$}}
              \lang{en}{\text{$\mathbb{L}=\{(\var{lsg1}, \var{lsg2}, \var{lsg3}) ~|~ u\in\R\}$}}
              \solution{false}
        \end{choice}
        
        \begin{choice} %ok
              \lang{de}{\text{$\mathbb{L}=\{(\var{lsg4}; \var{lsg5}; \var{lsg6}) ~|~ v,w\in\R\}$}}
              \lang{en}{\text{$\mathbb{L}=\{(\var{lsg4}, \var{lsg5}, \var{lsg6}) ~|~ v,w\in\R\}$}}
              \solution{false}
        \end{choice}
        
        \begin{choice} %ok
              \lang{de}{\text{$\mathbb{L}=\{(\var{lsg13}; \var{lsg14}; \var{lsg15}) ~|~ t\in\R\}$}}
              \lang{en}{\text{$\mathbb{L}=\{(\var{lsg13}, \var{lsg14}, \var{lsg15}) ~|~ t\in\R\}$}}
              \solution{false}
        \end{choice}
        
        \begin{choice} %ok
              \lang{de}{\text{$\mathbb{L}=\{(\var{lsg7}; \var{lsg8}; \var{lsg9}) ~|~ t,u\in\R\}$}}
              \lang{en}{\text{$\mathbb{L}=\{(\var{lsg7}, \var{lsg8}, \var{lsg9}) ~|~ t,u\in\R\}$}}
              \solution{true}
        \end{choice}
        
        \begin{choice} %ok
              \lang{de}{\text{$\mathbb{L}=\{(\var{lsg10}; \var{lsg11}; \var{lsg12}) ~|~ u,v\in\R\}$}}
              \lang{en}{\text{$\mathbb{L}=\{(\var{lsg10}, \var{lsg11}, \var{lsg12}) ~|~ u,v\in\R\}$}}
              \solution{true}
        \end{choice}
        
	\end{question} %CHECKED

	\begin{question} %3. Frage
        \lang{de}{\text{Welche Lösungsmengen sind für das folgende lineare Gleichungssystem über $\R$ korrekt?\\
        %\begin{align}
        %\var{a}x&+\var{b}y&\var{c}z&=&\var{j}\\
        %\var{d}x&\var{d1}y&+\var{f}z&=&\var{k}\\
        %\var{g}x&\var{h}y&+\var{i1}z&=&\var{l}
        %\end{align}
        $~~~~~
            \begin{pmatrix}
                \var{a}&\var{b}&\var{c}&|&\var{j}\\
                \var{d}&\var{d1}&\var{f}&|&\var{k}\\
                \var{g}&\var{h}&\var{i1}&|&\var{l}
            \end{pmatrix}
        $     
        }}
        \lang{en}{\text{Which solution sets are correct for the following linear system?\\
        \begin{align}
        \var{a}x&+\var{b}y&\var{c}z&=&\var{j}\\
        \var{d}x&\var{d1}y&+\var{f}z&=&\var{k}\\
        \var{g}x&\var{h}y&+\var{i1}z&=&\var{l}
        \end{align}
        }}
        \lang{de}{\explanation{
            Bringen Sie das lineare Gleichungssystem zunächst mit dem Gauß-Verfahren
            auf Stufenform.\\
            Nun erkennt man, ob (und wenn ja, wieviele) freie Variablen
            gewählt werden müssen.\\
            Nun können die noch in Frage kommenden Auswahlmöglichkeiten 
            durch Einsetzen überprüft werden.}}
        \lang{en}{\explanation{Reduce the linear system to row echelon form with Gaussian Elimination in order to see how many (if any) and which variables are free choices.\\
        If you know that, you can check the remaining ones via substitution.}}
        
        \permutechoices{1}{7}
        \type{mc.multiple}
        \field{rational}
        
        \begin{variables}
            % zuvor 5 statt 3
            \randint[Z]{b}{0}{3}
            \randint[Z]{c}{-3}{0}
            \randint[Z]{d1}{-3}{0}
            \randint[Z]{f}{0}{3}
            
            \randadjustIf{b,d1,c,f}{b*f-d1*c=0}
                               
            \randint[Z]{m}{-3}{-1}
            \randint[Z]{n}{1}{3}
            \randint[Z]{o}{-3}{3}      
            
            \function[calculate]{a}{o*b}
            \function[calculate]{d}{o*d1}                        
            \function[calculate]{g}{m*a+n*d}
            \function[calculate]{h}{m*b+n*d1}
            \function[calculate]{i1}{m*c+n*f}
                            
            % zuvor 5 statt 3
            \randint{r}{-3}{3}
            \randint{s}{-3}{3}
            \randint{q}{-3}{3}
            
            \function[calculate]{j}{a*r+b*s+c*q}
            \function[calculate]{k}{d*r+d1*s+f*q} 
            \function[calculate]{l}{g*r+h*s+i1*q}
            
            \function[calculate]{p1}{(j-b)/a}%%%
            \function[calculate]{p2}{(2*b-c)/a}%%%
  
            \function{lsg1}{p1+p2*t}%%%
            \function{lsg2}{1-2*t}%%%
            \function{lsg3}{t}%%%     
            
            \function[calculate]{p7}{(b*k-d1*j)/(b*f-c*d1)}%%%
            \function[calculate]{p5}{(j-c*p7)/a-2*j/a+(1-abs(sign(j)))/a}%%%
            \function[calculate]{p6}{-b/a}%%%
                      
            \function{lsg4}{p5+p6*u}%%%
            \function{lsg5}{u}%%%
            \function{lsg6}{p7}%%%  
            
            \function[calculate]{p9}{(j-p7*c)/b}%%%
            \function[calculate]{p10}{-a/b}%%%
                      
            \function{lsg7}{v}%%%
            \function{lsg8}{p9+p10*v}%%%
            \function{lsg9}{p7}%%%
            
            \function[calculate]{p13}{j/c}%%%
            \function[calculate]{p14}{-a/c}%%%
            \function[calculate]{p15}{-b/c}%%%
                                  
            \function{lsg10}{u}%%%
            \function{lsg11}{v}%%%
            \function{lsg12}{p13+p14*u+p15*v}%%%
            
			\function[calculate]{p18}{(j-b-c*p7)/a}%%%
            \function[calculate]{p19}{-b/a}%%%
                                  
            \function{lsg16}{p18+p19*t}%%%
            \function{lsg17}{1+t}%%%
            \function{lsg18}{p7}%%%
        \end{variables}

		\begin{choice} %ok
              \lang{de}{\text{$\mathbb{L}=\{(\var{r}; \var{s}; \var{q})\}$}}
              \lang{en}{\text{$\mathbb{L}=\{(\var{r}, \var{s}, \var{q})\}$}}
              \solution{false}
        \end{choice}
        
        \begin{choice} %ok
              \text{$\mathbb{L}=\emptyset$}
              \solution{false}
        \end{choice}
        
        \begin{choice} %ok
              \lang{de}{\text{$\mathbb{L}=\{(\var{lsg1}; \var{lsg2}; \var{lsg3}) ~|~ t\in\R\}$}}
              \lang{en}{\text{$\mathbb{L}=\{(\var{lsg1}, \var{lsg2}, \var{lsg3}) ~|~ t\in\R\}$}}
              \solution{false}
        \end{choice}
        
        \begin{choice} %ok
              \lang{de}{\text{$\mathbb{L}=\{(\var{lsg4}; \var{lsg5}; \var{lsg6}) ~|~ u\in\R\}$}}
              \lang{en}{\text{$\mathbb{L}=\{(\var{lsg4}, \var{lsg5}, \var{lsg6}) ~|~ u\in\R\}$}}
              \solution{false}
        \end{choice}
        
        \begin{choice} %ok
              \lang{de}{\text{$\mathbb{L}=\{(\var{lsg7}; \var{lsg8}; \var{lsg9}) ~|~ v\in\R\}$}}
              \lang{en}{\text{$\mathbb{L}=\{(\var{lsg7}, \var{lsg8}, \var{lsg9}) ~|~ v\in\R\}$}}
              \solution{true}
        \end{choice}
        
        \begin{choice} %ok
              \lang{de}{\text{$\mathbb{L}=\{(\var{lsg10}; \var{lsg11}; \var{lsg12}) ~|~ u,v\in\R\}$}}
              \lang{en}{\text{$\mathbb{L}=\{(\var{lsg10}, \var{lsg11}, \var{lsg12}) ~|~ u,v\in\R\}$}}
              \solution{false}
        \end{choice}
        
        \begin{choice} %ok
              \lang{de}{\text{$\mathbb{L}=\{(\var{lsg16}; \var{lsg17}; \var{lsg18}) ~|~ t\in\R\}$}}
              \lang{en}{\text{$\mathbb{L}=\{(\var{lsg16}, \var{lsg17}, \var{lsg18}) ~|~ t\in\R\}$}}
              \solution{true}
        \end{choice}
        
	\end{question} %CHECKED

\end{problem}

\embedmathlet{mathlet}

\end{content}