\documentclass{mumie.problem.gwtmathlet}
%$Id$
\begin{metainfo}
  \name{
    \lang{de}{A04: Elementarmatrizen}
    \lang{en}{P04: Elementary matrices}
  }
  \begin{description} 
 This work is licensed under the Creative Commons License Attribution 4.0 International (CC-BY 4.0)   
 https://creativecommons.org/licenses/by/4.0/legalcode 

    \lang{de}{}
    \lang{en}{}
  \end{description}
  \corrector{system/problem/GenericCorrector.meta.xml}
  \begin{components}
    \component{js_lib}{system/problem/GenericMathlet.meta.xml}{mathlet}
  \end{components}
  \begin{links}
  \end{links}
  \creategeneric
\end{metainfo}
\begin{content}
\usepackage{mumie.genericproblem}

\lang{de}{\title{A04: Elementarmatrizen}}
\lang{en}{\title{P04: Elementary matrices}}

\begin{block}[annotation]
	Im Ticket-System: \href{http://team.mumie.net/issues/11519}{Ticket 11519}
\end{block}

\begin{problem}
	 \randomquestionpool{1}{3}
     \randomquestionpool{4}{6}
     \randomquestionpool{7}{9}

%%----- Addition einer Zeile zu einer anderen ------
\lang{de}{\begin{question}
		
		\begin{variables}

            \number{m}{2}  %$m=2$-Fall

            \randint{n}{2}{5} 
            \randint[Z]{r1}{-4}{5} 
            \randint[Z]{r2}{-4}{5} 
            \function[calculate]{r}{r1/r2}

            % $k2\in \{1,2\}$ \% Zufallsvariable für $m=2$-Fall \\
            \randint{k2}{1}{2}

            \function[calculate]{i}{k2}
		    \function[calculate]{j}{3-k2}
		    \matrix[calculate]{mm}{1 & (2-i)*r \\ (i-1)*r & 1}

		\end{variables}
		
        \type{input.matrix}
        \field{rational}
		
	    \text{Gegeben sei eine $(\var{m}\times \var{n})$-Matrix $A$ mit reellen Einträgen.
        In einer Zeilenoperation soll das
        $(\var{r})$-fache der $\var{j}$-ten Zeile zur $\var{i}$-ten Zeile addiert werden.\\

        Geben Sie die Elementarmatrix $B$ an, so dass die durch die Zeilenoperation
        entstehende Matrix gleich $B\cdot A$ ist.\\
        
        Bestimmen Sie zunächst das Format (also die Anzahl an Zeilen und Spalten) von $B$ und bestimmen
        Sie dann die Einträge wertmäßig.}
	    
	    \begin{answer}
	    	\text{$B=$}
			\solution{mm}
            \explanation{Die gesuchte Matrix $B$ hat das Format $(m \times m)$,
            wobei $m$ der Anzahl der Zeilen von $A$ entspricht.
            Der Eintrag $(i,j)$ ist mit $\var{r}$ zu wählen.
            Hierbei ist $i$ die zu ändernde Zeile, zu welcher
            das $\var{r}$-fache der Zeile $j$ hinzuaddiert wird.
            Die Einträge auf der Hauptdiagonalen sind $1$.
            Die restlichen Einträge der Matrix sind $0$.            
            }
	    \end{answer}    
	    	    
\end{question}

\begin{question}
		
	\begin{variables}
        \number{m}{3}

        \randint{n}{2}{5} 
        \randint[Z]{r1}{-4}{5} 
        \randint[Z]{r2}{-4}{5} 
        \function[calculate]{r}{r1/r2}

        % $k3,l3\in \{1,2,3\}$ \% Zufallsvariablen für $m=3$-Fall \\
        \randint{k3}{1}{3}
        \randint{l3}{1}{3}
        \randadjustIf{k3,l3}{k3=l3}

		\function[calculate]{i}{k3}
		\function[calculate]{j}{l3}
		\function[calculate]{d12}{(1-|sign(i-1)|)*(1-|sign(j-2)|)}
		\function[calculate]{d13}{(1-|sign(i-1)|)*(1-|sign(j-3)|)}
		\function[calculate]{d21}{(1-|sign(i-2)|)*(1-|sign(j-1)|)}
		\function[calculate]{d23}{(1-|sign(i-2)|)*(1-|sign(j-3)|)}
		\function[calculate]{d31}{(1-|sign(i-3)|)*(1-|sign(j-1)|)}
		\function[calculate]{d32}{(1-|sign(i-3)|)*(1-|sign(j-2)|)}
		\matrix[calculate]{mm}{1 & d12*r & d13*r \\ d21*r & 1 & d23*r\\ d31*r & d32*r & 1}

    \end{variables}
		
        \type{input.matrix}
        \field{rational}
		
	    %\text{Bei der $(\var{m}\times \var{n})$-Matrix $A$ mit reellen Einträgen soll
%das $(\var{r})$-fache der $\var{j}$-ten Zeile zur $\var{i}$-ten Zeile addiert werden.\\
%Geben Sie die Elementarmatrix $B$ an, so dass die entstandene Matrix gleich $B\cdot A$ ist.}

	    \text{Gegeben sei eine $(\var{m}\times \var{n})$-Matrix $A$ mit reellen Einträgen.
        In einer Zeilenoperation soll das
        $(\var{r})$-fache der $\var{j}$-ten Zeile zur $\var{i}$-ten Zeile addiert werden.\\

        Geben Sie die Elementarmatrix $B$ an, so dass die durch die Zeilenoperation
        entstehende Matrix gleich $B\cdot A$ ist.\\
        
        Bestimmen Sie zunächst das Format (also die Anzahl an Zeilen und Spalten) von $B$ und bestimmen
        Sie dann die Einträge wertmäßig.}

	    
	    \begin{answer}
	    	\text{$B=$}
			\solution{mm}
            \explanation{Die gesuchte Matrix $B$ hat das Format $(m \times m)$,
            wobei $m$ der Anzahl der Zeilen von $A$ entspricht.
            Der Eintrag $(i,j)$ ist mit $\var{r}$ zu wählen.
            Hierbei ist $i$ die zu ändernde Zeile, zu welcher
            das $\var{r}$-fache der Zeile $j$ hinzuaddiert wird.
            Die Einträge auf der Hauptdiagonalen sind $1$.
            Die restlichen Einträge der Matrix sind $0$.            
            }
	    \end{answer}    
	    
\end{question}
	
\begin{question}
		
	\begin{variables}        
        \number{m}{4}


        \randint{n}{2}{5} 
        \randint[Z]{r1}{-4}{5} 
        \randint[Z]{r2}{-4}{5} 
        \function[calculate]{r}{r1/r2}

        % $k4,l4\in \{1,2,3,4\}$ \% Zufallsvariablen für $m=4$-Fall \\
        \randint{k4}{1}{4}
        \randint{l4}{1}{4}
        \randadjustIf{k4,l4}{k4=l4}

        \function[calculate]{i}{k4}
		\function[calculate]{j}{l4}
		\function[calculate]{d12}{(1-|sign(i-1)|)*(1-|sign(j-2)|)}
		\function[calculate]{d13}{(1-|sign(i-1)|)*(1-|sign(j-3)|)}
		\function[calculate]{d14}{(1-|sign(i-1)|)*(1-|sign(j-4)|)}
		\function[calculate]{d21}{(1-|sign(i-2)|)*(1-|sign(j-1)|)}
		\function[calculate]{d23}{(1-|sign(i-2)|)*(1-|sign(j-3)|)}
		\function[calculate]{d24}{(1-|sign(i-2)|)*(1-|sign(j-4)|)}
		\function[calculate]{d31}{(1-|sign(i-3)|)*(1-|sign(j-1)|)}
		\function[calculate]{d32}{(1-|sign(i-3)|)*(1-|sign(j-2)|)}
		\function[calculate]{d34}{(1-|sign(i-3)|)*(1-|sign(j-4)|)}
		\function[calculate]{d41}{(1-|sign(i-4)|)*(1-|sign(j-1)|)}
		\function[calculate]{d42}{(1-|sign(i-4)|)*(1-|sign(j-2)|)}
		\function[calculate]{d43}{(1-|sign(i-4)|)*(1-|sign(j-3)|)}
		\matrix[calculate]{mm}{1 & d12*r & d13*r & d14*r \\ d21*r & 1 & d23*r & d24*r \\ d31*r & d32*r & 1 & d34*r \\ d41*r & d42*r & d43*r & 1}


	\end{variables}
		
        \type{input.matrix}
        \field{rational}
		
	    %\text{Bei der $(\var{m}\times \var{n})$-Matrix $A$ mit reellen Einträgen soll
%das $(\var{r})$-fache der $\var{j}$-ten Zeile zur $\var{i}$-ten Zeile addiert werden.\\
%Geben Sie die Elementarmatrix $B$ an, so dass die entstandene Matrix gleich $B\cdot A$ ist.}
	    \text{Gegeben sei eine $(\var{m}\times \var{n})$-Matrix $A$ mit reellen Einträgen.
        In einer Zeilenoperation soll das
        $(\var{r})$-fache der $\var{j}$-ten Zeile zur $\var{i}$-ten Zeile addiert werden.\\

        Geben Sie die Elementarmatrix $B$ an, so dass die durch die Zeilenoperation
        entstehende Matrix gleich $B\cdot A$ ist.\\
        
        Bestimmen Sie zunächst das Format (also die Anzahl an Zeilen und Spalten) von $B$ und bestimmen
        Sie dann die Einträge wertmäßig.}

	    
	    \begin{answer}
	    	\text{$B=$}
			\solution{mm}
            \explanation{Die gesuchte Matrix $B$ hat das Format $(m \times m)$,
            wobei $m$ der Anzahl der Zeilen von $A$ entspricht.
            Der Eintrag $(i,j)$ ist mit $\var{r}$ zu wählen.
            Hierbei ist $i$ die zu ändernde Zeile, zu welcher
            das $\var{r}$-fache der Zeile $j$ hinzuaddiert wird.
            Die Einträge auf der Hauptdiagonalen sind $1$.
            Die restlichen Einträge der Matrix sind $0$.            
            }
	    \end{answer}    
	    
\end{question}
	
    
%%--- Vertauschung zweier Zeilen ---
%%----------------------------------
\begin{question}
		
	\begin{variables}
        \number{m}{2}
 
        \randint{n}{2}{5} 
        % $k2\in \{1,2\}$ \% Zufallsvariable für $m=2$-Fall \\
        \randint{k2}{1}{2}

		\function[calculate]{i}{k2}
		\function[calculate]{j}{3-k2}
		\matrix[calculate]{mm}{ 0 & 1 \\ 1 & 0 }

		\end{variables}
		
            \type{input.matrix}
            \field{rational}

	    \text{Gegeben sei eine $(\var{m}\times \var{n})$-Matrix $A$ mit reellen Einträgen.
        In einer Zeilenoperation sollen die
        $\var{i}$-te und die $\var{j}$-te Zeile vertauscht werden.\\

        Geben Sie die Elementarmatrix $B$ an, so dass die durch die Zeilenoperation
        entstehende Matrix gleich $B\cdot A$ ist.\\
        
        Bestimmen Sie zunächst das Format (also die Anzahl an Zeilen und Spalten) von $B$ und bestimmen
        Sie dann die Einträge wertmäßig.}
	    
	    \begin{answer}
	    	\text{$B=$}
			\solution{mm}
            \explanation{Die gesuchte Matrix $B$ hat das Format $(m \times m)$,
            wobei $m$ der Anzahl der Zeilen von $A$ entspricht.
            %Jede Zeile von Matrix $B$ enthält einen Eintrag mit dem Wert $1$.
            %Alle anderen Einträge sind jeweils $0$.
            Alle Einträge der Matrix sind $0$, außer die folgenden Ausnahmen:
            Falls die $k$-te Zeile $(1 \leq k \leq m)$ nicht zu den zu tauschenden
            Zeilen gehört, dann ist der Eintrag $(k,k)$ mit 1 zu wählen.
            Weiterhin sind die Einträge $(i,j)$ und $(j,i)$ mit 1 zu belegen,
            wobei $i$ und $j$ die Indizes für die zu tauschenden Zeilen sind.            
            }
	    \end{answer}    
 \end{question}

\begin{question}
		
	\begin{variables}
        \number{m}{3}

        \randint{n}{2}{5} 
            
        % $k3,l3\in \{1,2,3\}$ \% Zufallsvariablen für $m=3$-Fall \\
        \randint{k3}{1}{3}
        \randint{l3}{1}{3}
        \randadjustIf{k3,l3}{k3=l3}

        \function[calculate]{i}{k3}
		\function[calculate]{j}{l3}
		\function[calculate]{i1}{(1-|sign(i-1)|)}
		\function[calculate]{i2}{(1-|sign(i-2)|)}
		\function[calculate]{i3}{(1-|sign(i-3)|)}
		\function[calculate]{j1}{(1-|sign(j-1)|)}
		\function[calculate]{j2}{(1-|sign(j-2)|)}
		\function[calculate]{j3}{(1-|sign(j-3)|)}
		\matrix[calculate]{mm}{1-i1-j1 & i1*j2+i2*j1 &i1*j3+i3*j1 \\ i2*j1+i1*j2 & 1-i2-j2 & i2*j3+i3*j2\\ i3*j1+i1*j3 & i3*j2+i2*j3 & 1-i3-j3}

	\end{variables}
		
            \type{input.matrix}
            \field{rational}
		    
%            \text{Bei der $(\var{m}\times \var{n})$-Matrix $A$ mit reellen Einträgen sollen
%die $\var{i}$-te und die $\var{j}$-te Zeile vertauscht werden.\\
%Geben Sie die Elementarmatrix $B$ an, so dass die entstandene Matrix gleich $B\cdot A$ ist.}
\text{Gegeben sei eine $(\var{m}\times \var{n})$-Matrix $A$ mit reellen Einträgen.
        In einer Zeilenoperation sollen die
        $\var{i}$-te und die $\var{j}$-te Zeile vertauscht werden.\\

        Geben Sie die Elementarmatrix $B$ an, so dass die durch die Zeilenoperation
        entstehende Matrix gleich $B\cdot A$ ist.\\
        
        Bestimmen Sie zunächst das Format (also die Anzahl an Zeilen und Spalten) von $B$ und bestimmen
        Sie dann die Einträge wertmäßig.}
        
        
	    \begin{answer}
	    	\text{$B=$}
			\solution{mm}
            \explanation{Die gesuchte Matrix $B$ hat das Format $(m \times m)$,
            wobei $m$ der Anzahl der Zeilen von $A$ entspricht.
            %Jede Zeile von Matrix $B$ enthält einen Eintrag mit dem Wert $1$.
            %Alle anderen Einträge sind jeweils $0$.
            Alle Einträge der Matrix sind $0$, außer die folgenden Ausnahmen:
            Falls die $k$-te Zeile $(1 \leq k \leq m)$ nicht zu den zu tauschenden
            Zeilen gehört, dann ist der Eintrag $(k,k)$ mit 1 zu wählen.
            Weiterhin sind die Einträge $(i,j)$ und $(j,i)$ mit 1 zu belegen,
            wobei $i$ und $j$ die Indizes für die zu tauschenden Zeilen sind.            
            }
	    \end{answer}    
	    
\end{question}

\begin{question}
		
	\begin{variables}
        \number{m}{4}
        
        \randint{n}{2}{5}
        % $k4,l4\in \{1,2,3,4\}$ \% Zufallsvariablen für $m=4$-Fall \\
        \randint{k4}{1}{4}
        \randint{l4}{1}{4}
        \randadjustIf{k4,l4}{k4=l4}

		\function[calculate]{i}{k4}
		\function[calculate]{j}{l4}
		\function[calculate]{i1}{(1-|sign(i-1)|)}
		\function[calculate]{i2}{(1-|sign(i-2)|)}
		\function[calculate]{i3}{(1-|sign(i-3)|)}
		\function[calculate]{i4}{(1-|sign(i-4)|)}
		\function[calculate]{j1}{(1-|sign(j-1)|)}
		\function[calculate]{j2}{(1-|sign(j-2)|)}
		\function[calculate]{j3}{(1-|sign(j-3)|)}
		\function[calculate]{j4}{(1-|sign(j-4)|)}
		\matrix[calculate]{mm}{1-i1-j1 & i1*j2+i2*j1 &i1*j3+i3*j1 &i1*j4+i4*j1 \\ 
        i2*j1+i1*j2 & 1-i2-j2 & i2*j3+i3*j2& i2*j4+i4*j2\\ i3*j1+i1*j3 & i3*j2+i2*j3 & 
        1-i3-j3 & i3*j4+i4*j3\\ i4*j1+i1*j4 & i4*j2+i2*j4 & i4*j3+i3*j4 & 1-i4-j4}

	\end{variables}
		
            \type{input.matrix}
            \field{rational}
		    
%            \text{Bei der $(\var{m}\times \var{n})$-Matrix $A$ mit reellen Einträgen sollen
%die $\var{i}$-te und die $\var{j}$-te Zeile vertauscht werden.\\
%Geben Sie die Elementarmatrix $B$ an, so dass die entstandene Matrix gleich $B\cdot A$ ist.}
\text{Gegeben sei eine $(\var{m}\times \var{n})$-Matrix $A$ mit reellen Einträgen.
        In einer Zeilenoperation sollen die
        $\var{i}$-te und die $\var{j}$-te Zeile vertauscht werden.\\

        Geben Sie die Elementarmatrix $B$ an, so dass die durch die Zeilenoperation
        entstehende Matrix gleich $B\cdot A$ ist.\\
        
        Bestimmen Sie zunächst das Format (also die Anzahl an Zeilen und Spalten) von $B$ und bestimmen
        Sie dann die Einträge wertmäßig.}

	    \begin{answer}
	    	\text{$B=$}
			\solution{mm}
            \explanation{Die gesuchte Matrix $B$ hat das Format $(m \times m)$,
            wobei $m$ der Anzahl der Zeilen von $A$ entspricht.
            %Jede Zeile von Matrix $B$ enthält einen Eintrag mit dem Wert $1$.
            %Alle anderen Einträge sind jeweils $0$.
            Alle Einträge der Matrix sind $0$, außer die folgenden Ausnahmen:
            Falls die $k$-te Zeile $(1 \leq k \leq m)$ nicht zu den zu tauschenden
            Zeilen gehört, dann ist der Eintrag $(k,k)$ mit 1 zu wählen.
            Weiterhin sind die Einträge $(i,j)$ und $(j,i)$ mit 1 zu belegen,
            wobei $i$ und $j$ die Indizes für die zu tauschenden Zeilen sind.            
            }
	    \end{answer}    
	    
\end{question}

%%-- Multiplikation einer Zeile mit Faktor ---
%%--------------------------------------------
\begin{question}
		
	\begin{variables}
        \number{m}{2}
        
        \randint{n}{2}{5} 
        \randint[Z]{r1}{-4}{5} 
        \randint[Z]{r2}{-4}{5} 
        %\randadjustIf{r1,r2}{r1=r2}
        \function[calculate]{r}{r1 + r2*i}
        
        % $k2\in \{1,2\}$ \% Zufallsvariable für $m=2$-Fall \\
        \randint{k2}{1}{2}

        \function[calculate]{ii}{k2}
		\function[calculate]{d1}{2-ii}
		\function[calculate]{d2}{ii-1}
		\matrix[calculate]{mm}{1+d1*(r-1) & 0 \\ 0 & 1+d2*(r-1) }

	\end{variables}
		
        \type{input.matrix}
        \field{complex} % rational
        
		%\text{Bei der $(\var{m}\times \var{n})$-Matrix $A$ mit reellen Einträgen soll
%die $\var{i}$-te Zeile mit $\var{r}$ multipliziert werden.\\
%Geben Sie die Elementarmatrix $B$ an, so dass die entstandene Matrix gleich $B\cdot A$ ist.}

\text{Gegeben sei eine $(\var{m}\times \var{n})$-Matrix $A$ mit komplexen Einträgen.
        Wie üblich ist $i \in \C$ die imaginäre Einheit.\\
        
        In einer Zeilenoperation soll
        die $\var{ii}$-te Zeile mit dem Faktor $(\var{r})$ multipliziert werden.\\

        Geben Sie die Elementarmatrix $B$ an, so dass die durch die Zeilenoperation
        entstehende Matrix gleich $B\cdot A$ ist.\\
        
        Bestimmen Sie zunächst das Format (also die Anzahl an Zeilen und Spalten) von $B$ und bestimmen
        Sie dann die Einträge wertmäßig.}
        

	    \begin{answer}
	    	\text{$B=$}
			\solution{mm}
            \explanation{Die gesuchte Matrix $B$ hat das Format $(m \times m)$,
            wobei $m$ der Anzahl der Zeilen von $A$ entspricht.
            Der Eintrag $(k,k)$ ist mit $\var{r}$ zu wählen.
            Hierbei ist $k$ die zu ändernde Zeile. 
            Alle anderen Einträge auf der Hauptdiagonalen sind $1$.
            Die restlichen Einträge der Matrix sind $0$.            
            }
	    \end{answer}    
	    	    
\end{question}



\begin{question}
		
	\begin{variables}
        \number{m}{3}
        
        \randint{n}{2}{5} 
        \randint[Z]{r1}{-4}{5} 
        \randint[Z]{r2}{-4}{5} 
        %\randadjustIf{r1,r2}{r1=r2}
        \function[calculate]{r}{r1 + r2*i}
        
        % $k3\in \{1,2,3\}$ \% Zufallsvariable für $m=3$-Fall \\
        \randint{k3}{1}{3}

        \function[calculate]{ii}{k3}
		\function[calculate]{d1}{(1-|sign(ii-1)|)}
		\function[calculate]{d2}{(1-|sign(ii-2)|)}
		\function[calculate]{d3}{(1-|sign(ii-3)|)}
		\matrix[calculate]{mm}{ 1+d1*(r-1) & 0 &0 \\ 0 & 1+d2*(r-1) &0\\ 0&0& 1+d3*(r-1) }

		\end{variables}
		
        \type{input.matrix}
        \field{complex} % rational

%\text{Bei der $(\var{m}\times \var{n})$-Matrix $A$ mit reellen Einträgen soll
%die $\var{i}$-te Zeile mit $\var{r}$ multipliziert werden.\\
%Geben Sie die Elementarmatrix $B$ an, so dass die entstandene Matrix gleich $B\cdot A$ ist.}
\text{Gegeben sei eine $(\var{m}\times \var{n})$-Matrix $A$ mit komplexen Einträgen.
        Wie üblich ist $i \in \C$ die imaginäre Einheit.\\
        
        In einer Zeilenoperation soll
        die $\var{ii}$-te Zeile mit dem Faktor $(\var{r})$ multipliziert werden.\\

        Geben Sie die Elementarmatrix $B$ an, so dass die durch die Zeilenoperation
        entstehende Matrix gleich $B\cdot A$ ist.\\
        
        Bestimmen Sie zunächst das Format (also die Anzahl an Zeilen und Spalten) von $B$ und bestimmen
        Sie dann die Einträge wertmäßig.}
        
        
	    \begin{answer}
	    	\text{$B=$}
			\solution{mm}
            \explanation{Die gesuchte Matrix $B$ hat das Format $(m \times m)$,
            wobei $m$ der Anzahl der Zeilen von $A$ entspricht.
            Der Eintrag $(k,k)$ ist mit $\var{r}$ zu wählen.
            Hierbei ist $k$ die zu ändernde Zeile.
            Alle anderen Einträge auf der Hauptdiagonalen sind $1$.
            Die restlichen Einträge der Matrix sind $0$.            
            }
	    \end{answer}    
	    
\end{question}




\begin{question}
		
	\begin{variables}
        \number{m}{4}
        
        \randint{n}{2}{5} 
        \randint[Z]{r1}{-4}{5} 
        \randint[Z]{r2}{-4}{5} 
        %\randadjustIf{r1,r2}{r1=r2}        
        \function[calculate]{r}{r1 + r2*i}
        
        % $k4\in \{1,2,3,4\}$ \% Zufallsvariable für $m=4$-Fall \\
        \randint{k4}{1}{4}
        
		\function[calculate]{ii}{k4}
		\function[calculate]{d1}{(1-|sign(ii-1)|)}
		\function[calculate]{d2}{(1-|sign(ii-2)|)}
		\function[calculate]{d3}{(1-|sign(ii-3)|)}
		\function[calculate]{d4}{(1-|sign(ii-4)|)}
		\matrix[calculate]{mm}{1+d1*(r-1) & 0 &0  &0\\ 0 & 1+d2*(r-1) &0 &0\\ 0&0& 1+d3*(r-1)  &0\\ 0&0&0& 1+d4*(r-1)}


	\end{variables}
		
        \type{input.matrix}
        \field{complex} % rational

%\text{Bei der $(\var{m}\times \var{n})$-Matrix $A$ mit reellen Einträgen soll
%die $\var{i}$-te Zeile mit $\var{r}$ multipliziert werden.\\
%Geben Sie die Elementarmatrix $B$ an, so dass die entstandene Matrix gleich $B\cdot A$ ist.}

\text{Gegeben sei eine $(\var{m}\times \var{n})$-Matrix $A$ mit komplexen Einträgen.
        Wie üblich ist $i \in \C$ die imaginäre Einheit.\\
        
        In einer Zeilenoperation soll
        die $\var{ii}$-te Zeile mit dem Faktor $(\var{r})$ multipliziert werden.\\

        Geben Sie die Elementarmatrix $B$ an, so dass die durch die Zeilenoperation
        entstehende Matrix gleich $B\cdot A$ ist.\\
        
        Bestimmen Sie zunächst das Format (also die Anzahl an Zeilen und Spalten) von $B$ und bestimmen
        Sie dann die Einträge wertmäßig.}
        
        
	    \begin{answer}
	    	\text{$B=$}
			\solution{mm}
            \explanation{Die gesuchte Matrix $B$ hat das Format $(m \times m)$,
            wobei $m$ der Anzahl der Zeilen von $A$ entspricht.
            Der Eintrag $(k,k)$ ist mit $\var{r}$ zu wählen.
            Hierbei ist $k$ die zu ändernde Zeile.
            Alle anderen Einträge auf der Hauptdiagonalen sind $1$.
            Die restlichen Einträge der Matrix sind $0$.            
            }
	    \end{answer}    
	    
\end{question}}


%%-----ENGLISCHE VERSION -----
%%----- Addition einer Zeile zu einer anderen ------
\lang{en}{\begin{question}
		
		\begin{variables}

            \number{m}{2}  %$m=2$-Fall

            \randint{n}{2}{5} 
            \randint[Z]{r1}{-4}{5} 
            \randint[Z]{r2}{-4}{5} 
            \function[calculate]{r}{r1/r2}

            % $k2\in \{1,2\}$ \% Zufallsvariable für $m=2$-Fall \\
            \randint{k2}{1}{2}

            \function[calculate]{i}{k2}
		    \function[calculate]{j}{3-k2}
		    \matrix[calculate]{mm}{1 & (2-i)*r \\ (i-1)*r & 1}

		\end{variables}
		
        \type{input.matrix}
        \field{rational}
		
	    \text{Given is a $(\var{m}\times \var{n})$-matrix with real-valued entries.
     A row operations should add $(\var{r})$-times the $\var{j}$th row to the $\var{i}$th row.\\

     Give the elementary matrix $B$ such that the resulting matrix equals $B\cdot A$.\\

     First of all determine the form (the number of rows and columns) of $B$ and then determine the value of the entries.}
    
	    
	    \begin{answer}
	    	\text{$B=$}
			\solution{mm}
            \explanation{
            The matrix $B$ has the form $(m\times m$), at which $m$ equals the numbers of rows of $A$.
            We choose $\var{r}$ for the entry $(i,j)$.
            Here is $i$ the index of the row, which we multiply by $\var{r}$ and add it to the row $j$.
            The entries on the main diagonal equal $1$ and the entries not on the main diagonal equal $0$.           
            }
	    \end{answer}    
	    	    
\end{question}

\begin{question}
		
	\begin{variables}
        \number{m}{3}

        \randint{n}{2}{5} 
        \randint[Z]{r1}{-4}{5} 
        \randint[Z]{r2}{-4}{5} 
        \function[calculate]{r}{r1/r2}

        % $k3,l3\in \{1,2,3\}$ \% Zufallsvariablen für $m=3$-Fall \\
        \randint{k3}{1}{3}
        \randint{l3}{1}{3}
        \randadjustIf{k3,l3}{k3=l3}

		\function[calculate]{i}{k3}
		\function[calculate]{j}{l3}
		\function[calculate]{d12}{(1-|sign(i-1)|)*(1-|sign(j-2)|)}
		\function[calculate]{d13}{(1-|sign(i-1)|)*(1-|sign(j-3)|)}
		\function[calculate]{d21}{(1-|sign(i-2)|)*(1-|sign(j-1)|)}
		\function[calculate]{d23}{(1-|sign(i-2)|)*(1-|sign(j-3)|)}
		\function[calculate]{d31}{(1-|sign(i-3)|)*(1-|sign(j-1)|)}
		\function[calculate]{d32}{(1-|sign(i-3)|)*(1-|sign(j-2)|)}
		\matrix[calculate]{mm}{1 & d12*r & d13*r \\ d21*r & 1 & d23*r\\ d31*r & d32*r & 1}

    \end{variables}
		
        \type{input.matrix}
        \field{rational}
		
	    %\text{Bei der $(\var{m}\times \var{n})$-Matrix $A$ mit reellen Einträgen soll
%das $(\var{r})$-fache der $\var{j}$-ten Zeile zur $\var{i}$-ten Zeile addiert werden.\\
%Geben Sie die Elementarmatrix $B$ an, so dass die entstandene Matrix gleich $B\cdot A$ ist.}

	    \text{Given is a $(\var{m}\times \var{n})$-matrix with real-valued entries.
     A row operations should add $(\var{r})$-times the $\var{j}$th row to the $\var{i}$th row.\\

     Give the elementary matrix $B$ such that the resulting matrix equals $B\cdot A$.\\

     First of all determine the form (the number of rows and columns) of $B$ and then determine the value of the entries.}

	    
	    \begin{answer}
	    	\text{$B=$}
			\solution{mm}
            \explanation{The matrix $B$ has the form $(m\times m$), at which $m$ equals the numbers of rows of $A$.
            We choose $\var{r}$ for the entry $(i,j)$.
            Here is $i$ the index of the row, which we multiply by $\var{r}$ and add it to the row $j$.
            The entries on the main diagonal equal $1$ and the ones not on the main diagonal equal $0$.             
            }
	    \end{answer}    
	    
\end{question}
	
\begin{question}
		
	\begin{variables}        
        \number{m}{4}


        \randint{n}{2}{5} 
        \randint[Z]{r1}{-4}{5} 
        \randint[Z]{r2}{-4}{5} 
        \function[calculate]{r}{r1/r2}

        % $k4,l4\in \{1,2,3,4\}$ \% Zufallsvariablen für $m=4$-Fall \\
        \randint{k4}{1}{4}
        \randint{l4}{1}{4}
        \randadjustIf{k4,l4}{k4=l4}

        \function[calculate]{i}{k4}
		\function[calculate]{j}{l4}
		\function[calculate]{d12}{(1-|sign(i-1)|)*(1-|sign(j-2)|)}
		\function[calculate]{d13}{(1-|sign(i-1)|)*(1-|sign(j-3)|)}
		\function[calculate]{d14}{(1-|sign(i-1)|)*(1-|sign(j-4)|)}
		\function[calculate]{d21}{(1-|sign(i-2)|)*(1-|sign(j-1)|)}
		\function[calculate]{d23}{(1-|sign(i-2)|)*(1-|sign(j-3)|)}
		\function[calculate]{d24}{(1-|sign(i-2)|)*(1-|sign(j-4)|)}
		\function[calculate]{d31}{(1-|sign(i-3)|)*(1-|sign(j-1)|)}
		\function[calculate]{d32}{(1-|sign(i-3)|)*(1-|sign(j-2)|)}
		\function[calculate]{d34}{(1-|sign(i-3)|)*(1-|sign(j-4)|)}
		\function[calculate]{d41}{(1-|sign(i-4)|)*(1-|sign(j-1)|)}
		\function[calculate]{d42}{(1-|sign(i-4)|)*(1-|sign(j-2)|)}
		\function[calculate]{d43}{(1-|sign(i-4)|)*(1-|sign(j-3)|)}
		\matrix[calculate]{mm}{1 & d12*r & d13*r & d14*r \\ d21*r & 1 & d23*r & d24*r \\ d31*r & d32*r & 1 & d34*r \\ d41*r & d42*r & d43*r & 1}


	\end{variables}
		
        \type{input.matrix}
        \field{rational}
		
	    %\text{Bei der $(\var{m}\times \var{n})$-Matrix $A$ mit reellen Einträgen soll
%das $(\var{r})$-fache der $\var{j}$-ten Zeile zur $\var{i}$-ten Zeile addiert werden.\\
%Geben Sie die Elementarmatrix $B$ an, so dass die entstandene Matrix gleich $B\cdot A$ ist.}
	    \text{Given is a $(\var{m}\times \var{n})$-matrix with real-valued entries.
     A row operations should add $(\var{r})$-times the $\var{j}$th row to the $\var{i}$th row.\\

     Give the elementary matrix $B$ such that the resulting matrix equals $B\cdot A$.\\

     First of all determine the form (the number of rows and columns) of $B$ and then determine the value of the entries.}

	    
	    \begin{answer}
	    	\text{$B=$}
			\solution{mm}
            \explanation{The matrix $B$ has the form $(m\times m$), at which $m$ equals the numbers of rows of $A$.
            We choose $\var{r}$ for the entry $(i,j)$.
            Here is $i$ the index of the row, which we multiply by $\var{r}$ and add it to the row $j$.
            The entries on the main diagonal equal $1$ and the entries not on the main diagonal equal $0$.          
            }
	    \end{answer}    
	    
\end{question}
	
    
%%--- Vertauschung zweier Zeilen ---
%%----------------------------------
\begin{question}
		
	\begin{variables}
        \number{m}{2}
 
        \randint{n}{2}{5} 
        % $k2\in \{1,2\}$ \% Zufallsvariable für $m=2$-Fall \\
        \randint{k2}{1}{2}

		\function[calculate]{i}{k2}
		\function[calculate]{j}{3-k2}
		\matrix[calculate]{mm}{ 0 & 1 \\ 1 & 0 }

		\end{variables}
		
            \type{input.matrix}
            \field{rational}

	    \text{Given is a $(\var{m}\times \var{n})$-matrix $A$ with real-valued entries.
     A row operation should swap the $\var{i}$th and the $\var{j}$th row.\\

    Give the elementary matrix $B$ such, that the resulting matrix equals $B\cdot A$.\\
        
    First of all determine the form (the numbers of rows and columns) and then determine the value of the entries.}
	    
	    \begin{answer}
	    	\text{$B=$}
			\solution{mm}
            \explanation{
            The matrix $B$ has the form $(m \times m)$, where $m$ is the number of rows of the matrix $A$.
            All entries equal $0$, besides the following exceptions:
            If the $k$th row $(1\leq k\leq m)$ is not one of those to be swapped, then the entry $(j,i)$ equals $1$.
            Furthermore we set the entries $(i,j)$ and $(j,i)$ equal to $1$, with $i$ and $j$ being
            the indices of the swapped rows.          
            }
	    \end{answer}    
 \end{question}

\begin{question}
		
	\begin{variables}
        \number{m}{3}

        \randint{n}{2}{5} 
            
        % $k3,l3\in \{1,2,3\}$ \% Zufallsvariablen für $m=3$-Fall \\
        \randint{k3}{1}{3}
        \randint{l3}{1}{3}
        \randadjustIf{k3,l3}{k3=l3}

        \function[calculate]{i}{k3}
		\function[calculate]{j}{l3}
		\function[calculate]{i1}{(1-|sign(i-1)|)}
		\function[calculate]{i2}{(1-|sign(i-2)|)}
		\function[calculate]{i3}{(1-|sign(i-3)|)}
		\function[calculate]{j1}{(1-|sign(j-1)|)}
		\function[calculate]{j2}{(1-|sign(j-2)|)}
		\function[calculate]{j3}{(1-|sign(j-3)|)}
		\matrix[calculate]{mm}{1-i1-j1 & i1*j2+i2*j1 &i1*j3+i3*j1 \\ i2*j1+i1*j2 & 1-i2-j2 & i2*j3+i3*j2\\ i3*j1+i1*j3 & i3*j2+i2*j3 & 1-i3-j3}

	\end{variables}
		
            \type{input.matrix}
            \field{rational}
		    
%            \text{Bei der $(\var{m}\times \var{n})$-Matrix $A$ mit reellen Einträgen sollen
%die $\var{i}$-te und die $\var{j}$-te Zeile vertauscht werden.\\
%Geben Sie die Elementarmatrix $B$ an, so dass die entstandene Matrix gleich $B\cdot A$ ist.}
\text{Given is a $(\var{m}\times \var{n})$-matrix $A$ with real-valued entries.
     A row operation should swap the $\var{i}$th and the $\var{j}$th row.\\

    Give the elementary matrix $B$ such, that the resulting matrix equals $B\cdot A$.\\
        
    First of all determine the form (the numbers of rows and columns) and then determine the value of the entries.}
        
        
	    \begin{answer}
	    	\text{$B=$}
			\solution{mm}
            \explanation{
            The matrix $B$ has the form $(m \times m)$, where $m$ is the number of rows of the matrix $A$.
            All entries equal $0$, besides the following exceptions:
            If the $k$th row $(1\leq k\leq m)$ is not one of those to be swapped, then the entry $(j,i)$ equals $1$.
            Furthermore we set the entries $(i,j)$ and $(j,i)$ equal to $1$, with $i$ and $j$ being
            the indices of the swapped rows. 
            }
	    \end{answer}    
	    
\end{question}

\begin{question}
		
	\begin{variables}
        \number{m}{4}
        
        \randint{n}{2}{5}
        % $k4,l4\in \{1,2,3,4\}$ \% Zufallsvariablen für $m=4$-Fall \\
        \randint{k4}{1}{4}
        \randint{l4}{1}{4}
        \randadjustIf{k4,l4}{k4=l4}

		\function[calculate]{i}{k4}
		\function[calculate]{j}{l4}
		\function[calculate]{i1}{(1-|sign(i-1)|)}
		\function[calculate]{i2}{(1-|sign(i-2)|)}
		\function[calculate]{i3}{(1-|sign(i-3)|)}
		\function[calculate]{i4}{(1-|sign(i-4)|)}
		\function[calculate]{j1}{(1-|sign(j-1)|)}
		\function[calculate]{j2}{(1-|sign(j-2)|)}
		\function[calculate]{j3}{(1-|sign(j-3)|)}
		\function[calculate]{j4}{(1-|sign(j-4)|)}
		\matrix[calculate]{mm}{1-i1-j1 & i1*j2+i2*j1 &i1*j3+i3*j1 &i1*j4+i4*j1 \\ 
        i2*j1+i1*j2 & 1-i2-j2 & i2*j3+i3*j2& i2*j4+i4*j2\\ i3*j1+i1*j3 & i3*j2+i2*j3 & 
        1-i3-j3 & i3*j4+i4*j3\\ i4*j1+i1*j4 & i4*j2+i2*j4 & i4*j3+i3*j4 & 1-i4-j4}

	\end{variables}
		
            \type{input.matrix}
            \field{rational}
		    
%            \text{Bei der $(\var{m}\times \var{n})$-Matrix $A$ mit reellen Einträgen sollen
%die $\var{i}$-te und die $\var{j}$-te Zeile vertauscht werden.\\
%Geben Sie die Elementarmatrix $B$ an, so dass die entstandene Matrix gleich $B\cdot A$ ist.}
\text{Given is a $(\var{m}\times \var{n})$-matrix $A$ with real-valued entries.
     A row operation should swap the $\var{i}$th and the $\var{j}$th row.\\

    Give the elementary matrix $B$ such, that the resulting matrix equals $B\cdot A$.\\
        
    First of all determine the form (the numbers of rows and columns) and then determine the value of the entries.}

	    \begin{answer}
	    	\text{$B=$}
			\solution{mm}
            \explanation{
            The matrix $B$ has the form $(m \times m)$, where $m$ is the number of rows of the matrix $A$.
            All entries equal $0$, besides the following exceptions:
            If the $k$th row $(1\leq k\leq m)$ is not one of those to be swapped, then the entry $(j,i)$ equals $1$.
            Furthermore we set the entries $(i,j)$ and $(j,i)$ equal to $1$, with $i$ and $j$ being
            the indices of the swapped rows. 
            }
	    \end{answer}    
	    
\end{question}

%%-- Multiplikation einer Zeile mit Faktor ---
%%--------------------------------------------
\begin{question}
		
	\begin{variables}
        \number{m}{2}
        
        \randint{n}{2}{5} 
        \randint[Z]{r1}{-4}{5} 
        \randint[Z]{r2}{-4}{5} 
        %\randadjustIf{r1,r2}{r1=r2}
        \function[calculate]{r}{r1 + r2*i}
        
        % $k2\in \{1,2\}$ \% Zufallsvariable für $m=2$-Fall \\
        \randint{k2}{1}{2}

        \function[calculate]{ii}{k2}
		\function[calculate]{d1}{2-ii}
		\function[calculate]{d2}{ii-1}
		\matrix[calculate]{mm}{1+d1*(r-1) & 0 \\ 0 & 1+d2*(r-1) }

	\end{variables}
		
        \type{input.matrix}
        \field{complex} % rational
        
		%\text{Bei der $(\var{m}\times \var{n})$-Matrix $A$ mit reellen Einträgen soll
%die $\var{i}$-te Zeile mit $\var{r}$ multipliziert werden.\\
%Geben Sie die Elementarmatrix $B$ an, so dass die entstandene Matrix gleich $B\cdot A$ ist.}

\text{Given is a $(\var{m}\times \var{n})$-matrix $A$ with complex-valued entries. $i\in\C$ is the imaginary unit.\\
A row operations should multiply the $\var{ii}$th row with the factor $(\var{r})$.\\

Give the elementary matrix $B$ such that the resulting matrix equals $B\cdot A$.\\
First of all determine the form (the number of rows and columns) of $B$ und then determine the values of its entries.}
        

	    \begin{answer}
	    	\text{$B=$}
			\solution{mm}
            \explanation{
            The matrix $B$ has the form $(m\times m)$, where $m$ equals the number of rows of the matrix $A$.
            $k$ is the index of the row we want to multiply with the factor. We then set the entry $(k,k)$ equal to $\var{r}$.
            The other entries on the main diagonal are $1$ and the entries not on the main diagonal equal $0$.         
            }
	    \end{answer}    
	    	    
\end{question}



\begin{question}
		
	\begin{variables}
        \number{m}{3}
        
        \randint{n}{2}{5} 
        \randint[Z]{r1}{-4}{5} 
        \randint[Z]{r2}{-4}{5} 
        %\randadjustIf{r1,r2}{r1=r2}
        \function[calculate]{r}{r1 + r2*i}
        
        % $k3\in \{1,2,3\}$ \% Zufallsvariable für $m=3$-Fall \\
        \randint{k3}{1}{3}

        \function[calculate]{ii}{k3}
		\function[calculate]{d1}{(1-|sign(ii-1)|)}
		\function[calculate]{d2}{(1-|sign(ii-2)|)}
		\function[calculate]{d3}{(1-|sign(ii-3)|)}
		\matrix[calculate]{mm}{ 1+d1*(r-1) & 0 &0 \\ 0 & 1+d2*(r-1) &0\\ 0&0& 1+d3*(r-1) }

		\end{variables}
		
        \type{input.matrix}
        \field{complex} % rational

%\text{Bei der $(\var{m}\times \var{n})$-Matrix $A$ mit reellen Einträgen soll
%die $\var{i}$-te Zeile mit $\var{r}$ multipliziert werden.\\
%Geben Sie die Elementarmatrix $B$ an, so dass die entstandene Matrix gleich $B\cdot A$ ist.}
\text{Given is a $(\var{m}\times \var{n})$-matrix $A$ with complex-valued entries. $i\in\C$ is the imaginary unit.\\
A row operations should multiply the $\var{ii}$th row with the factor $(\var{r})$.\\

Give the elementary matrix $B$ such that the resulting matrix equals $B\cdot A$.\\
First of all determine the form (the number of rows and columns) of $B$ und then determine the values of its entries.}
        
        
	    \begin{answer}
	    	\text{$B=$}
			\solution{mm}
            \explanation{The matrix $B$ has the form $(m\times m)$, where $m$ equals the number of rows of the matrix $A$.
            $k$ is the index of the row we want to multiply with the factor. We then set the entry $(k,k)$ equal to$\var{r}$.
            The other entries on the main diagonal are $1$ and the entries not on the main diagonal equal $0$.            
            }
	    \end{answer}    
	    
\end{question}




\begin{question}
		
	\begin{variables}
        \number{m}{4}
        
        \randint{n}{2}{5} 
        \randint[Z]{r1}{-4}{5} 
        \randint[Z]{r2}{-4}{5} 
        %\randadjustIf{r1,r2}{r1=r2}        
        \function[calculate]{r}{r1 + r2*i}
        
        % $k4\in \{1,2,3,4\}$ \% Zufallsvariable für $m=4$-Fall \\
        \randint{k4}{1}{4}
        
		\function[calculate]{ii}{k4}
		\function[calculate]{d1}{(1-|sign(ii-1)|)}
		\function[calculate]{d2}{(1-|sign(ii-2)|)}
		\function[calculate]{d3}{(1-|sign(ii-3)|)}
		\function[calculate]{d4}{(1-|sign(ii-4)|)}
		\matrix[calculate]{mm}{1+d1*(r-1) & 0 &0  &0\\ 0 & 1+d2*(r-1) &0 &0\\ 0&0& 1+d3*(r-1)  &0\\ 0&0&0& 1+d4*(r-1)}


	\end{variables}
		
        \type{input.matrix}
        \field{complex} % rational

%\text{Bei der $(\var{m}\times \var{n})$-Matrix $A$ mit reellen Einträgen soll
%die $\var{i}$-te Zeile mit $\var{r}$ multipliziert werden.\\
%Geben Sie die Elementarmatrix $B$ an, so dass die entstandene Matrix gleich $B\cdot A$ ist.}

\text{Given is a $(\var{m}\times \var{n})$-matrix $A$ with complex-valued entries. $i\in\C$ is the imaginary unit.\\
A row operations should multiply the $\var{ii}$th row with the factor $(\var{r})$.\\

Give the elementary matrix $B$ such that the resulting matrix equals $B\cdot A$.\\
First of all determine the form (the number of rows and columns) of $B$ und then determine the values of its entries.}
        
        
	    \begin{answer}
	    	\text{$B=$}
			\solution{mm}
            \explanation{The matrix $B$ has the form $(m\times m)$, where $m$ equals the number of rows of the matrix $A$.
            $k$ is the index of the row we want to multiply with the factor. We then set the entry $(k,k)$ equal to$\var{r}$.
            The other entries on the main diagonal are $1$ and the entries not on the main diagonal equal $0$.            
            }
	    \end{answer}    
	    
\end{question}}




\end{problem}

\embedmathlet{mathlet}

\end{content}