\documentclass{mumie.problem.gwtmathlet}
%$Id$
\begin{metainfo}
  \name{
    \lang{de}{A03: Stufenform}
    \lang{en}{P03: Row echelon form}
  }
  \begin{description} 
 This work is licensed under the Creative Commons License Attribution 4.0 International (CC-BY 4.0)   
 https://creativecommons.org/licenses/by/4.0/legalcode 

    \lang{de}{}
    \lang{en}{}
  \end{description}
  \corrector{system/problem/GenericCorrector.meta.xml}
  \begin{components}
    \component{js_lib}{system/problem/GenericMathlet.meta.xml}{mathlet}
  \end{components}
  \begin{links}
  \end{links}
  \creategeneric
\end{metainfo}
\begin{content}
\usepackage{mumie.genericproblem}

\lang{de}{\title{A03: Stufenform} \lang{en}{P03: Row echelon form}}

\begin{block}[annotation]
	Im Ticket-System: \href{http://team.mumie.net/issues/11518}{Ticket 11518}
\end{block}

\begin{problem}
	\randomquestionpool{1}{3}
    \randomquestionpool{4}{4}

    \begin{question}
		 
		\begin{variables}
		
			\randint{l1}{-3}{5}
			\randint{l2}{-3}{5}
			\randint{l3}{-3}{5}
			\randint{b}{-3}{5}
			\randint{f}{-3}{5}
			\randint[Z]{a}{-3}{5}
		
        	\matrix[calculate]{aa}{2 & 2 & 4\\ -1 & a-1 & b-2 \\ f & -2*a + f  & -2*b + 2*f + 1}
        	\matrix[calculate]{bb}{2*l1 + 2*l2 + 4*l3 \\ 
a*l2 + b*l3 - l1 - l2 - 2*l3\\ 
-2*a*l2 - 2*b*l3 + f*l1 + f*l2 + 2*f*l3 + l3}
        	
        	\function{s11}{1}
        	\function{s12}{0}
        	\function{s13}{0}
        	\function{s14}{l1}
        	\function{s21}{0}
        	\function{s22}{1}
        	\function{s23}{0}
        	\function{s24}{l2}
        	\function{s31}{0}
        	\function{s32}{0}
        	\function{s33}{1}
        	\function{s34}{l3}
		\end{variables}
		
		\type{input.number}
        \field{real}
		\correctorprecision{3}
		\displayprecision{3}
		\lang{de}{
	    \text{Wir betrachten das folgende lineare Gleichungssystem:\\ 
        $\var{aa}\cdot \left(\begin{smallmatrix}
x_1\\ x_2\\ x_3\end{smallmatrix}  \right) = \var{bb}$\\
Bestimmen Sie zur zugehörigen erweiterten Koeffizientenmatrix die reduzierte Stufenform.
In Ihrer Lösung sollen die Koeffizienten vor den verbleibenden Variablen
immer genau 1 sein.\\ 

%Die reduzierte Stufenform ist: \\
\ansref  \ansref  \ansref |  \ansref \\
\ansref  \ansref  \ansref |  \ansref \\
\ansref  \ansref  \ansref | \ansref 
    }
    }

\lang{en}{
	    \text{Consider the following linear system:\\ 
        $\var{aa}\cdot \left(\begin{smallmatrix}
x_1\\ x_2\\ x_3\end{smallmatrix}  \right) = \var{bb}$\\
Determine the reduced row echelon form for the corresponding augmented matrix.
In your solution the coefficients of the remaining variables should all be exactly $1$.\\

%Die reduzierte Stufenform ist: \\
\ansref  \ansref  \ansref |  \ansref \\
\ansref  \ansref  \ansref |  \ansref \\
\ansref  \ansref  \ansref | \ansref 
    }
    }


	    \begin{answer}
			\solution{s11}
	    \end{answer}    
	    \begin{answer}
			\solution{s12}
	    \end{answer}    
	    
	    \begin{answer}
			\solution{s13}
	    \end{answer}    
	    
	    \begin{answer}
			\solution{s14}
	    \end{answer}    
	    \begin{answer}
			\solution{s21}
	    \end{answer}    
	    \begin{answer}
			\solution{s22}
	    \end{answer}    
	    
	    \begin{answer}
			\solution{s23}
	    \end{answer}    
	    
	    \begin{answer}
			\solution{s24}
	    \end{answer}    
	    
	    \begin{answer}
			\solution{s31}
	    \end{answer}    
	    \begin{answer}
			\solution{s32}
	    \end{answer}    
	    
	    \begin{answer}
			\solution{s33}
	    \end{answer}    
	    
	    \begin{answer}
			\solution{s34}
	    \end{answer}    
	    
        \lang{de}{\explanation{
            Zunächst ermittelt man die allgemeine Stufenform mittels der Regeln
            des Gauß-Verfahrens (Zeilentausch, Addition eines Vielfachen
            einer anderen Zeile zu einer Zeile und Multiplikation einer Zeile
            mit einem Faktor).
            Nun sind alle Einträge unterhalb der Hauptdiagonalen 0.
            Jetzt werden die Einträge oberhalb der Hauptdiagonalen
            mittels Zeilenumformungen eliminiert, also auf 0 gesetzt.
            Man erhält die reduzierte Stufenform.
            Diese ist jedoch noch nicht eindeutig und muss noch normiert werden:
            Für alle Zeilen der linken Seite, die einen Eintrag haben, der ungleich 0 und
            ungleich 1 ist, dividiert man alle Einträge der entsprechenden Zeile durch
            diesen Koeffizienten.
        }}

          \lang{en}{\explanation{
          For determining the row echelon form we use the "tools" of Gaussian elimination (swapping rows, adding rows, multiplying rows
          with a factor).
          All the entries below the main diagonal equal now 0.
          The next step is to eliminate the entries above the main diagonal by transforming the rows.
          We end up with the reduced row echelon form.
          Since it is not unique, we need to norm ist:
          For all non-zero rows with entries unequal to 1, we divide all entries of this row by this entry.
        }}
	  
	\end{question}


    \begin{question}
		
		\begin{variables}
        	
			\randint{l1}{-3}{5}
			\randint{l2}{-3}{5}
			\randint{l3}{-3}{5}
			\randint{b}{-3}{5}
			\randint{f}{-3}{5}
			\randint{s}{-3}{5}
			\randint{t}{-3}{5}
			\randint[Z]{a}{-3}{5}
		
        	\matrix[calculate]{aa}{2&2& 2*s+2*t \\ -1 & a-1&a*t-s-t \\ f & -2*a+f & -2*a*t + f*s + f*t}
        	\matrix[calculate]{bb}{2*l1 + 2*l2 + 4*l3 \\
 a*l2 + b*l3 - l1 - l2 - 2*l3\\
 -2*a*l2 - 2*b*l3 + f*l1 + f*l2 + 2*f*l3 + l3}
        	
        	\function{s11}{1}
        	\function{s12}{0}
        	\function{s13}{s}
        	\function{s14}{l1}
        	\function{s21}{0}
        	\function{s22}{1}
        	\function{s23}{t}
        	\function{s24}{l2}
        	\function{s31}{0}
        	\function{s32}{0}
        	\function{s33}{0}
        	\function{s34}{l3}
        	
        	\function{h}{|sign(l3)|}
		\end{variables}
		
		\type{input.generic}
        \field{real}
		\correctorprecision{3}
		\displayprecision{3}
		\lang{de}{
	    \text{Wir betrachten das folgende lineare Gleichungssystem:\\ $\var{aa}\cdot \left(\begin{smallmatrix}
x_1\\ x_2\\ x_3\end{smallmatrix}  \right) = \var{bb}$\\
Bestimmen Sie zur zugehörigen erweiterten Koeffizientenmatrix die reduzierte Stufenform:
In Ihrer Lösung sollen die Koeffizienten vor den verbleibenden Variablen
immer genau 1 sein.\\ 

%Die reduzierte Stufenform ist: \\
\ansref  \ansref  \ansref |  \ansref \\
\ansref  \ansref  \ansref |  \ansref \\
\ansref  \ansref  \ansref | \ansref 
	    }}

\lang{en}{
	    \text{Consider the following linear system:\\ $\var{aa}\cdot \left(\begin{smallmatrix}
x_1\\ x_2\\ x_3\end{smallmatrix}  \right) = \var{bb}$\\
Determine the reduced row echelon form of the corresponding augmented matrix. The coefficients of the remaining variables in your solutions
should all be exactly $1$.\\ 

%Die reduzierte Stufenform ist: \\
\ansref  \ansref  \ansref |  \ansref \\
\ansref  \ansref  \ansref |  \ansref \\
\ansref  \ansref  \ansref | \ansref 
	    }}
     
	    \begin{answer}
	    	\type{input.number}
			\solution{s11}
	    \end{answer}    
	    \begin{answer}
	    	\type{input.number}
			\solution{s12}
	    \end{answer}    
	    
	    \begin{answer}
	    	\type{input.number}
			\solution{s13}
	    \end{answer}    
	    
	    \begin{answer}
	    	\type{input.function}
			\solution{s14}
			\inputAsFunction{x}{k1}
			\checkFuncForZero{(1-h)*(k1-l1)}{-1}{1}{10}
	    \end{answer}    
	    \begin{answer}
	    	\type{input.number}
			\solution{s21}
	    \end{answer}    
	    \begin{answer}
	    	\type{input.number}
			\solution{s22}
	    \end{answer}    
	    
	    \begin{answer}
	    	\type{input.number}
			\solution{s23}
	    \end{answer}    
	    
	    \begin{answer}
	    	\type{input.function}
			\solution{s24}
			\inputAsFunction{x}{k2}
			\checkFuncForZero{(1-h)*(k2-l2)}{-1}{1}{10}
	    \end{answer}    
	    
	    \begin{answer}
	    	\type{input.number}
			\solution{s31}
	    \end{answer}    
	    \begin{answer}
	    	\type{input.number}
			\solution{s32}
	    \end{answer}    
	    
	    \begin{answer}
	    	\type{input.number}
			\solution{s33}
	    \end{answer}    
	    
	    \begin{answer}
	    	\type{input.function}
			\solution{s34}
			\inputAsFunction{x}{k3}
			\checkFuncForZero{|sign(k3)|-h}{-1}{1}{10}
	    \end{answer}    

        \lang{de}{\explanation{
            Zunächst ermittelt man die allgemeine Stufenform mittels der Regeln
            des Gauß-Verfahrens (Zeilentausch, Addition eines Vielfachen
            einer anderen Zeile zu einer Zeile und Multiplikation einer Zeile
            mit einem Faktor).
            Nun sind alle Einträge unterhalb der Hauptdiagonalen 0.
            Jetzt werden die Einträge oberhalb der Hauptdiagonalen
            mittels Zeilenumformungen eliminiert, also auf 0 gesetzt.
            Man erhält die reduzierte Stufenform.
            Diese ist jedoch noch nicht eindeutig und muss noch normiert werden:
            Für alle Zeilen der linken Seite, die einen Eintrag haben, der ungleich 0 und
            ungleich 1 ist, dividiert man alle Einträge der entsprechenden Zeile durch
            diesen Koeffizienten.
        }}

        \lang{en}{\explanation{
          For determining the row echelon form we use the "tools" of Gaussian elimination (swapping rows, adding rows, multiplying rows
          with a factor).
          All the entries below the main diagonal equal now 0.
          The next step is to eliminate the entries above the main diagonal by transforming the rows.
          We end up with the reduced row echelon form.
          Since it is not unique, we need to norm ist:
          For all non-zero rows with entries unequal to 1, we divide all entries of this row by this entry.
        }}
	  
	\end{question}

    \begin{question}
		
		\begin{variables}
        	
			\randint{l1}{-3}{5}
			\randint{l2}{-3}{5}
			\randint{l3}{-3}{5}
			\randint{b}{-3}{5}
			\randint{f}{-3}{5}
			\randint{s}{-3}{5}
			\randint{t}{-3}{5}
			\randint[Z]{a}{-3}{5}
		
        	\matrix[calculate]{aa}{2& 2*s & 2\\ -1& -s & a-1 \\ f & f*s & -2*a+f}
        	\matrix[calculate]{bb}{2*l1 + 2*l2 + 4*l3\\ 
a*l2 + b*l3 - l1 - l2 - 2*l3\\ 
 -2*a*l2 - 2*b*l3 + f*l1 + f*l2 + 2*f*l3 + l3}
        	
        	\function{s11}{1}
        	\function{s12}{s}
        	\function{s13}{0}
        	\function{s14}{l1}
        	\function{s21}{0}
        	\function{s22}{0}
        	\function{s23}{1}
        	\function{s24}{l2}
        	\function{s31}{0}
        	\function{s32}{0}
        	\function{s33}{0}
        	\function{s34}{l3}
        	
        	\function{h}{|sign(l3)|}
		\end{variables}
		
		\type{input.generic}
        \field{real}
		\correctorprecision{3}
		\displayprecision{3}
		\lang{de}{
	    \text{Wir betrachten das folgende lineare Gleichungssystem\\ $\var{aa}\cdot \left(\begin{smallmatrix}
x_1\\ x_2\\ x_3\end{smallmatrix}  \right) = \var{bb}$:\\
Bestimmen Sie zur zugehörigen erweiterten Koeffizientenmatrix die reduzierte Stufenform.
In Ihrer Lösung sollen die Koeffizienten vor den verbleibenden Variablen
immer genau 1 sein.\\ 

%Die reduzierte Stufenform ist: \\
\ansref  \ansref  \ansref |  \ansref \\
\ansref  \ansref  \ansref |  \ansref \\
\ansref  \ansref  \ansref | \ansref 
	    }}

\lang{en}{
	    \text{Consider the following linear system:\\ $\var{aa}\cdot \left(\begin{smallmatrix}
x_1\\ x_2\\ x_3\end{smallmatrix}  \right) = \var{bb}$:\\
Determine the reduced row echelon form of corresponding augmented matrix.
The coefficients of the remaining variables in your solutions
should all be exactly $1$.\\

%Die reduzierte Stufenform ist: \\
\ansref  \ansref  \ansref |  \ansref \\
\ansref  \ansref  \ansref |  \ansref \\
\ansref  \ansref  \ansref | \ansref 
	    }}
     
	    \begin{answer}
		\type{input.number}
			\solution{s11}
	    \end{answer}    
	    \begin{answer}
		\type{input.number}
			\solution{s12}
	    \end{answer}    
	    
	    \begin{answer}
		\type{input.number}
			\solution{s13}
	    \end{answer}    
	    
	    \begin{answer}
		\type{input.function}
			\solution{s14}
			\inputAsFunction{x}{k1}
			\checkFuncForZero{(1-h)*(k1-l1)}{-1}{1}{10} 
	    \end{answer}    
	    \begin{answer}
		\type{input.number}
			\solution{s21}
	    \end{answer}    
	    \begin{answer}
		\type{input.number}
			\solution{s22}
	    \end{answer}    
	    
	    \begin{answer}
		\type{input.number}
			\solution{s23}
	    \end{answer}    
	    
	    \begin{answer}
		\type{input.function}
			\solution{s24}
			\inputAsFunction{x}{k2}
			\checkFuncForZero{(1-h)*(k2-l2)}{-1}{1}{10}
	    \end{answer}    
	    
	    \begin{answer}
		\type{input.number}
			\solution{s31}
	    \end{answer}    
	    \begin{answer}
		\type{input.number}
			\solution{s32}
	    \end{answer}    
	    
	    \begin{answer}
		\type{input.number}
			\solution{s33}
	    \end{answer}    
	    
	    \begin{answer}
		\type{input.function}
			\solution{s34}
			\inputAsFunction{x}{k3}
			\checkFuncForZero{|sign(k3)|-h}{-1}{1}{10}
	    \end{answer}    
        
        \lang{de}{\explanation{
            Zunächst ermittelt man die allgemeine Stufenform mittels der Regeln
            des Gauß-Verfahrens (Zeilentausch, Addition eines Vielfachen
            einer anderen Zeile zu einer Zeile und Multiplikation einer Zeile
            mit einem Faktor).
            Nun sind alle Einträge unterhalb der Hauptdiagonalen 0.
            Jetzt werden die Einträge oberhalb der Hauptdiagonalen
            mittels Zeilenumformungen eliminiert, also auf 0 gesetzt.
            Man erhält die reduzierte Stufenform.
            Diese ist jedoch noch nicht eindeutig und muss noch normiert werden:
            Für alle Zeilen der linken Seite, die einen Eintrag haben, der ungleich 0 und
            ungleich 1 ist, dividiert man alle Einträge der entsprechenden Zeile durch
            diesen Koeffizienten.
        }}

        \lang{en}{\explanation{
          For determining the row echelon form we use the "tools" of Gaussian elimination (swapping rows, adding rows, multiplying rows
          with a factor).
          All the entries below the main diagonal equal now 0.
          The next step is to eliminate the entries above the main diagonal by transforming the rows.
          We end up with the reduced row echelon form.
          Since it is not unique, we need to norm ist:
          For all non-zero rows with entries unequal to 1, we divide all entries of this row by this entry.
        }}
	    
	\end{question}



    \begin{question}
		 
		\begin{variables}
		
            % draw some random numbers in range [-2,2] \ { 0 }, ti in ZZ
			\randint[Z]{t1}{-2}{2}
            \randint[Z]{t2}{-2}{2}
            \randint[Z]{t3}{-2}{2}
            \randint[Z]{t4}{-2}{2}
            \randint[Z]{t5}{-2}{2}
            \randint[Z]{t6}{-2}{2}
            \randint[Z]{t7}{-2}{2}
            
            % generate elements for the following system:
            % [ a b | c
            %   d e | f ]
            \function[calculate]{a}{t1+t2*i}
            \function[calculate]{b}{t3}
            \function[calculate]{c}{0}
            \function[calculate]{d}{t4+t5*i}
            \function[calculate]{e_}{t6}
            \function[calculate]{f}{t7*i}

            % setup matrix A (-> alias AA)
            \matrix[calculate]{AA}{a & b\\ d & e_}
            % setup right-hand side b (-> alias bb)
            \matrix[calculate]{bb}{c\\ f}

            % calculate the solution in form:
            % [ u v | w 
            %   x y | z ]
            \function[calculate]{u}{1}
            \function[calculate]{v}{0}
            \function[calculate]{x}{0}
            \function[calculate]{y}{1}
            \function[calculate]{w}{((-b*(-d/a*c+f))/(-d/a*b+e_)+c)/a}
            \function[calculate]{z}{(-d/a*c+f)/(-d/a*b+e_)}
            
            % setup solution matrix X (-> alias XX)
            \matrix[calculate]{XX}{u & v\\ x & y}
            % setup solution vector u (-> alias uu)
            \matrix[calculate]{uu}{w\\ z}
                    
		\end{variables}
		
		\type{input.matrix}
        \field{complex-rational}
		\correctorprecision[rounded]{2}
		\displayprecision{5}
		\lang{de}{
	    \text{Wir betrachten das folgende lineare Gleichungssystem über den komplexen Zahlen.
        Wie üblich ist $i \in \C$ die imaginäre Einheit.\\ 
        $\var{AA}\cdot \left( \begin{smallmatrix} x_1\\ x_2 \end{smallmatrix}  \right) = \var{bb}$\\
Bestimmen Sie die reduzierte Stufenform.
Ihre Lösung hat zunächst die Form $(A|b)$. 
In den Eingabefeldern geben Sie die Koeffizientenmatrix $A$,
sowie die rechte Seite $b$ separat an.
Wählen Sie zunächst das richtige Format, 
und geben Sie dann die Elemente wertmäßig an.
In Ihrer Lösung sollen die Einträge vor den verbleibenden Variablen
immer genau 1 sein.%$\var{XX}$, $\var{uu}$
}}

\lang{en}{
	    \text{We consider the following linear system over the field of the complex numbers.
         $i \in \C$ is the imaginary unit.\\ 
        $\var{AA}\cdot \left( \begin{smallmatrix} x_1\\ x_2 \end{smallmatrix}  \right) = \var{bb}$\\
Determine the redcued row echelon form. First of all your solution has the form $(A|b)$.
Enter the coefficient matrix $A$ and the right side $b$ in the input boxes seperatly.
First off all choose the right form, then enter the elements by value.
The entries of the remaining variables in your solution should all have the value $1$.
}}

	    \begin{answer}
            \text{$A=$}
            \type{input.matrix}
			\solution{XX}
            \lang{de}{\explanation{
                Zunächst ermittelt man die allgemeine Stufenform mittels der Regeln
                des Gaußverfahrens (Zeilentausch, Addition eines Vielfachen
                einer anderen Zeile zu einer Zeile und Multiplikation einer Zeile
                mit einem Faktor).
                Nun sind alle Einträge unterhalb der Hauptdiagonalen 0.
                Jetzt werden die Einträge oberhalb der Hauptdiagonalen
                mittels Zeilenumformungen eliminiert, also auf 0 gesetzt.
                Man erhält die reduzierte Stufenform.
                Diese ist jedoch noch nicht eindeutig und muss noch normiert werden:
                Für alle Zeilen der linken Seite, die einen Eintrag haben, der ungleich 0 und
                ungleich 1 ist, dividiert man alle Einträge der entsprechenden Zeile durch
                diesen Koeffizienten.
            }}

            \lang{en}{\explanation{
          For determining the row echelon form we use the "tools" of Gaussian elimination (swapping rows, adding rows, multiplying rows
          with a factor).
          All the entries below the main diagonal equal now 0.
          The next step is to eliminate the entries above the main diagonal by transforming the rows.
          We end up with the reduced row echelon form.
          Since it is not unique, we need to norm ist:
          For all non-zero rows with entries unequal to 1, we divide all entries of this row by this entry.
        }}
            
	    \end{answer}  

	    \begin{answer}
            \text{$b=$}
            \type{input.matrix}
			\solution{uu}
	    \end{answer}  

	\end{question}



\end{problem}

\embedmathlet{mathlet}

\end{content}