\documentclass{mumie.problem.gwtmathlet}
%$Id$
\begin{metainfo}
  \name{
    \lang{de}{A05: Rang}
    \lang{en}{P05: Rank}
  }
  \begin{description} 
 This work is licensed under the Creative Commons License Attribution 4.0 International (CC-BY 4.0)   
 https://creativecommons.org/licenses/by/4.0/legalcode 

    \lang{de}{}
    \lang{en}{}
  \end{description}
  \corrector{system/problem/GenericCorrector.meta.xml}
  \begin{components}
    \component{js_lib}{system/problem/GenericMathlet.meta.xml}{mathlet}
  \end{components}
  \begin{links}
  \end{links}
  \creategeneric
\end{metainfo}

\begin{content}
\begin{block}[annotation]
	Im Ticket-System: \href{https://team.mumie.net/issues/21544}{Ticket 21544}
\end{block}
\usepackage{mumie.genericproblem}


\lang{de}{\title{A05: Rang}}
\lang{en}{\title{P05: Rank}}
% Übernahme aus Ticket 9673; Inhalt angepasst

\lang{de}{
\begin{problem}

     \begin{question} 
     
     
     \begin{variables}
		\randint[Z]{a}{-9}{9}
		\randint[Z]{b}{-9}{9}
		\randint[Z]{c}{-9}{9}
		\randint[Z]{d}{-9}{9}
		\randint[Z]{ee}{-9}{9}
		\randint[Z]{f}{-9}{9}
		\randint[Z]{k}{-9}{9}
        \randint[Z]{l}{-9}{9}
		
			%\matrix[calculate]{m1}{
  			%a & a & k*a \\
     %0 & a & k*a \\
     %0 & a & k*a 
     % 	}
      		\matrix[calculate]{m2}{
  			 a & -l*a & -(l+1)*a \\
     a & -l*a & -(l+1)*a \\
     0 & (l+1)*a & (l+1)*a 
      	}
     % 		\matrix[calculate]{m3}{
  		%	a & b & c \\
     %0 & 0 & c \\
     %0 & b & 0 
     % 	}
      	\matrix[calculate]{m4}{
  			a & b & c \\
     d & ee & f \\
     0 & 0 & 0
      	}
      	\matrix[calculate]{m5}{
  			a & 0 & 0 & 0\\
     0 & b & b & 0 \\
     0 & 2*b & 3*b & 4*b \\
     0 & c & c & 5*c
      	}
        \matrix[calculate]{m6}{
  			a & b & c \\
     a*2 & b*4 & c*5
      	}
	\end{variables}
     
     
     \lang{de}{ 
      	\text{
            Geben Sie für alle der folgenden Matrizen über $\R$ an, ob diese
            vollen Rang haben.
        }
      }
    	\permutechoices{1}{3}
    	\type{mc.yesno}
		%\begin{choice}
  		%	\text{$\var{m1}$}
  		%	\solution{false}
		%\end{choice}
		\begin{choice}
  			\text{$\var{m2}$}
  			\solution{false}
		\end{choice}
		\begin{choice}
  			\text{$\var{m4}$}
  			\solution{false}
		\end{choice}
		%\begin{choice}
  		%	\text{$\var{m3}$}
  		%	\solution{true}
		%\end{choice}
		\begin{choice}
  			\text{$\var{m5}$}
  			\solution{true}
		\end{choice}
		\begin{choice}
  			\text{$\var{m6}$}
  			\solution{true}
		\end{choice}
                
        \explanation{
         Bringen Sie die jeweilige Matrix zunächst mit Hilfe des Gauß-Verfahrens in Stufenform.
         Der Rang enspricht dann der Anzahl an Zeilen, die ungleich der Nullzeile sind.
         Eine $(m \times n)$-Matrix $A$ hat dann vollen Rang, wenn der Rang von $A$ gleich $\min\{m,n\}$ ist.
        }
        
    \end{question}
    
    
    
    
\begin{question} 
   \begin{variables}
      \randint[Z]{ar}{-9}{9}
      \randint[Z]{ai}{-9}{9}
      \function[calculate]{a}{ar+ai*i}
      \randint[Z]{b}{-9}{9}
      \randint[Z]{cr}{-9}{9}
      \randint[Z]{ci}{-9}{9}
      \function[calculate]{c}{cr+ci*i}
      \randint[Z]{d}{-9}{9}
      \randint[Z]{ee}{-9}{9}
      \randint[Z]{f}{-9}{9}
      \randint[Z]{k}{-9}{9}
      \matrix[calculate]{m6}{
          a & b & cr \\
          a*3 & b*3 & c*3 
      }
    \end{variables}
    \field{complex}
    \lang{de}{ 
        \text{Entscheiden Sie, ob die folgende Matrix über $\C$ vollen Rang besitzt.
        Wie üblich ist $i \in \C$ die imaginäre Einheit.}
    }
    %\permutechoices{1}{3}
    \type{mc.yesno}
    \begin{choice}
        \text{$\var{m6}$}
        \solution{true}
    \end{choice}      
    \explanation{
     Bringen Sie die Matrix zunächst mit Hilfe des Gauß-Verfahrens in Stufenform.
     Der Rang enspricht dann der Anzahl an Zeilen, die ungleich der Nullzeile sind.
     Eine $(m \times n)$-Matrix $A$ hat dann vollen Rang, wenn der Rang von $A$ gleich $\min\{m,n\}$ ist.
     Der Rang ist hier 2 und entspricht $\min\{2, 3\}=2$.
    }        
\end{question}    
    
    
    
    
\begin{question}
  \begin{variables}
    \randint[Z]{a}{-9}{9}
    \randint[Z]{b}{-9}{9}
    \randint[Z]{c}{-9}{9}
    \randint[Z]{d}{-9}{9}
    \randint[Z]{ee}{-9}{9}
    \randint[Z]{f}{-9}{9}
    \randint[Z]{k}{-9}{9}
    \matrix[calculate]{r1}{
      a & a & k*a \\
      0 & a & k*a \\
      0 & a & k*a 
    }
    \number{r1s}{2}
  \end{variables}
  \type{input.number}
  \displayprecision{3}
  \correctorprecision{2}
  \field{integer}
  \lang{de}{
      \text{Geben Sie den Rang für die folgende Matrix über $\R$ an:\\
      $A=\var{r1}$}
  }
  \begin{answer}
      \text{Rang(A)=}\solution{r1s}
      \explanation{
         Bringen Sie die Matrix zunächst mit Hilfe des Gauß-Verfahrens in Stufenform.
         Sie stellen fest, dass die dritte Zeile in dieser Aufgabe dann eine Nullzeile ist.
         Der Rang enspricht dann der Anzahl an Zeilen, die ungleich der Nullzeile sind,
         also hier 2.
      }
  \end{answer}
\end{question} 


\begin{question}
  \begin{variables}
    \randint[Z]{a}{-9}{9}
    \randint[Z]{b}{-9}{9}
    \randint[Z]{c}{-9}{9}
    \randint[Z]{d}{-9}{9}
    \randint[Z]{ee}{-9}{9}
    \randint[Z]{f}{-9}{9}
    \randint[Z]{k}{-9}{9}
    \matrix[calculate]{r2}{
      a & b & c \\
      0 & 0 & c \\
      0 & b & 0 
    }    
    \number{r2s}{3}
  \end{variables}
  \type{input.number}
  \displayprecision{3}
  \correctorprecision{2}
  \field{integer}
  \lang{de}{
      \text{Geben Sie den Rang für die folgende Matrix über $\R$ an:\\
      $B=\var{r2}$}
  }
  \begin{answer}
      \text{Rang(B)=}\solution{r2s}
      \explanation{
        Bringen Sie die Matrix zunächst mit Hilfe des Gauß-Verfahrens in Stufenform.
        Am einfachsten tauscht man hier die zweite und die dritte Zeile.
        Der Rang enspricht dann der Anzahl an Zeilen, die ungleich der Nullzeile sind,
        also hier 3.
      }
  \end{answer}
\end{question} 

\end{problem}}

%% ------ ENGLISCHE VERSION -----------------------------------
\lang{en}{
\begin{problem}

     \begin{question} 
     
     
     \begin{variables}
		\randint[Z]{a}{-9}{9}
		\randint[Z]{b}{-9}{9}
		\randint[Z]{c}{-9}{9}
		\randint[Z]{d}{-9}{9}
		\randint[Z]{ee}{-9}{9}
		\randint[Z]{f}{-9}{9}
		\randint[Z]{k}{-9}{9}
        \randint[Z]{l}{-9}{9}
		
			%\matrix[calculate]{m1}{
  			%a & a & k*a \\
     %0 & a & k*a \\
     %0 & a & k*a 
     % 	}
      		\matrix[calculate]{m2}{
  			 a & -l*a & -(l+1)*a \\
     a & -l*a & -(l+1)*a \\
     0 & (l+1)*a & (l+1)*a 
      	}
     % 		\matrix[calculate]{m3}{
  		%	a & b & c \\
     %0 & 0 & c \\
     %0 & b & 0 
     % 	}
      	\matrix[calculate]{m4}{
  			a & b & c \\
     d & ee & f \\
     0 & 0 & 0
      	}
      	\matrix[calculate]{m5}{
  			a & 0 & 0 & 0\\
     0 & b & b & 0 \\
     0 & 2*b & 3*b & 4*b \\
     0 & c & c & 5*c
      	}
        \matrix[calculate]{m6}{
  			a & b & c \\
     a*2 & b*4 & c*5
      	}
	\end{variables}
      
      	\text{
       Decide for each of the following matrices over $\R$, if they have full rank.
        }
  
    	\permutechoices{1}{3}
    	\type{mc.yesno}
		%\begin{choice}
  		%	\text{$\var{m1}$}
  		%	\solution{false}
		%\end{choice}
		\begin{choice}
  			\text{$\var{m2}$}
  			\solution{false}
		\end{choice}
		\begin{choice}
  			\text{$\var{m4}$}
  			\solution{false}
		\end{choice}
		%\begin{choice}
  		%	\text{$\var{m3}$}
  		%	\solution{true}
		%\end{choice}
		\begin{choice}
  			\text{$\var{m5}$}
  			\solution{true}
		\end{choice}
		\begin{choice}
  			\text{$\var{m6}$}
  			\solution{true}
		\end{choice}
                
        \explanation{
        Transform each matrix via Gaussian elimination into row echelon form.
        The rank is equal to the number of non-zero rows.
        A $(m\times n)$-matrix $A$ has full rank, when the rank of $A$ is $\min\{m,n\}$.}
        
    \end{question}
    
    
    
    
\begin{question} 
   \begin{variables}
      \randint[Z]{ar}{-9}{9}
      \randint[Z]{ai}{-9}{9}
      \function[calculate]{a}{ar+ai*i}
      \randint[Z]{b}{-9}{9}
      \randint[Z]{cr}{-9}{9}
      \randint[Z]{ci}{-9}{9}
      \function[calculate]{c}{cr+ci*i}
      \randint[Z]{d}{-9}{9}
      \randint[Z]{ee}{-9}{9}
      \randint[Z]{f}{-9}{9}
      \randint[Z]{k}{-9}{9}
      \matrix[calculate]{m6}{
          a & b & cr \\
          a*3 & b*3 & c*3 
      }
    \end{variables}
    \field{complex}
    \lang{de}{ 
        \text{Decide for each of the following matrices over $\C$, if they have full rank. $i\in\C$ is the imaginary unit.}
    }
    %\permutechoices{1}{3}
    \type{mc.yesno}
    \begin{choice}
        \text{$\var{m6}$}
        \solution{true}
    \end{choice}      
    \explanation{
     Transform the matrix via Gaussian elimination into row echelon form.
        The rank is equal to the number of non-zero rows.
        A $(m\times n)$-matrix $A$ has full rank, when the rank of $A$ is $\min\{m,n\}$.
     Here the rank is $2$ and we have $\min \{2,3\}=2$.
    }        
\end{question}    
    
    
    
    
\begin{question}
  \begin{variables}
    \randint[Z]{a}{-9}{9}
    \randint[Z]{b}{-9}{9}
    \randint[Z]{c}{-9}{9}
    \randint[Z]{d}{-9}{9}
    \randint[Z]{ee}{-9}{9}
    \randint[Z]{f}{-9}{9}
    \randint[Z]{k}{-9}{9}
    \matrix[calculate]{r1}{
      a & a & k*a \\
      0 & a & k*a \\
      0 & a & k*a 
    }
    \number{r1s}{2}
  \end{variables}
  \type{input.number}
  \displayprecision{3}
  \correctorprecision{2}
  \field{integer}
  \lang{de}{
      \text{Determine the rank of the following matrix over $\R$:\\
      $A=\var{r1}$}
  }
  \begin{answer}
      \text{Rang(A)=}\solution{r1s}
      \explanation{
      Transform the matrix via Gaussian elimination into row echelon form.
      We see, that the third row is a zero row.
      The rank is the number of non-zero rows, which is here $2$.
      }
  \end{answer}
\end{question} 


\begin{question}
  \begin{variables}
    \randint[Z]{a}{-9}{9}
    \randint[Z]{b}{-9}{9}
    \randint[Z]{c}{-9}{9}
    \randint[Z]{d}{-9}{9}
    \randint[Z]{ee}{-9}{9}
    \randint[Z]{f}{-9}{9}
    \randint[Z]{k}{-9}{9}
    \matrix[calculate]{r2}{
      a & b & c \\
      0 & 0 & c \\
      0 & b & 0 
    }    
    \number{r2s}{3}
  \end{variables}
  \type{input.number}
  \displayprecision{3}
  \correctorprecision{2}
  \field{integer}
  \lang{de}{
      \text{Determine the rank of the following matrix over $\R$:\\
      $B=\var{r2}$}
  }
  \begin{answer}
      \text{Rang(B)=}\solution{r2s}
      \explanation{
      Transform the matrix via Gaussian elimination into row echelon form. 
      Therefore it is the easiest to swap the second and the third row.
      The rank is the number of non-zero rows, which is here $3$.
      }
  \end{answer}
\end{question} 

\end{problem}}

\embedmathlet{mathlet}

\end{content}
