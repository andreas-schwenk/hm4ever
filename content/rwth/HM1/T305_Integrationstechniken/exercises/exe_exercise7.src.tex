\documentclass{mumie.element.exercise}
%$Id$
\begin{metainfo}
  \name{
    \lang{de}{Ü07: Partialbruchzerlegung}
    \lang{en}{Exercise 7}
  }
  \begin{description} 
 This work is licensed under the Creative Commons License Attribution 4.0 International (CC-BY 4.0)   
 https://creativecommons.org/licenses/by/4.0/legalcode 

    \lang{de}{}
    \lang{en}{}
  \end{description}
  \begin{components}
  \end{components}
  \begin{links}
  \end{links}
  \creategeneric
\end{metainfo}
\begin{content}
\usepackage{mumie.ombplus}

\title{\lang{de}{Ü07: Partialbruchzerlegung}}

\begin{block}[annotation]
  Im Ticket-System: \href{http://team.mumie.net/issues/10925}{Ticket 10925}
\end{block}

%######################################################FRAGE_TEXT
\lang{de}{ 
Bestimmen Sie jeweils die Partialbruchzerlegung der rationalen Funktionen
\[  {a)}\quad R_1(x)=\frac{1}{x^3-4x^2+4x}\, , \qquad\qquad {b)}\quad R_2(x)=\frac{x^3}{(x^2+4)^3}\, .
\]
Bestimmen Sie des Weiteren Stammfunktionen zu diesen Funktionen.
 }

%##################################################ANTWORTEN_TEXT
\begin{tabs*}[\initialtab{0}\class{exercise}]

  %++++++++++++++++++++++++++++++++++++++++++START_TAB_X
  \tab{\lang{de}{    Antwort    }}
    \lang{de}{   Die Partialbruchzerlegungen der gegebenen Funktionen sind:
\begin{enumerate}[a)]
\item a) $\ R_1(x)=\frac{1}{x^3-4x^2+4x}= \frac{1}{4}\cdot \frac{1}{x}-\frac{1}{4}\cdot  \frac{1}{x-2}+\frac{1}{2}\cdot\frac{1}{(x-2)^2}$,
\item b) $\ R_2(x)=\frac{x^3}{(x^2+4)^3}=  \frac{x}{(x^2+4)^2} + \frac{-4x}{(x^2+4)^3}$.
\end{enumerate}   


Eine Stammfunktion zu $R_1(x)$ ist 
\[  \frac{1}{4}\cdot \ln({|x|})-\frac{1}{4}\cdot  \ln({|x-2|})+ \frac{1}{2}\cdot\frac{-1}{(x-2)} .\]
Eine Stammfunktion zu $R_2(x)$ ist
\[  - \frac{1}{2}\cdot \frac{1}{(x^2+4)} +  \frac{1}{(x^2+4)^2} .\]
 }
  	 %------------------------------------END_STEP_X
 
  %++++++++++++++++++++++++++++++++++++++++++++END_TAB_X
  %++++++++++++++++++++++++++++++++++++++++++START_TAB_X
  \tab{\lang{de}{    Lösung a)    }}
  \begin{incremental}[\initialsteps{1}]
  
  	 %----------------------------------START_STEP_X
    \step    
Für die Partialbruchzerlegung ist zunächst der Nenner in Faktoren zu zerlegen.
Lineare Faktoren findet man im Allgemeinen durch Bestimmung der Nullstellen.
Hier kann man die Faktorisierung aber durch Ausklammern von $x$ und mit Hilfe der zweiten binomischen Formel direkt finden:
\[ x^3-4x^2+4x=x(x^2-4x+4)=x(x-2)^2. \]
\step Da der Nenner nun einen Faktor in höherer Potenz hat, sind die Nenner in der Partialbruchzerlegung die Polynome $x$, $x-2$ und $(x-2)^2$. Wir machen den Ansatz
\[ R_1(x)=\frac{1}{x(x-2)^2}=\frac{A_1}{x}+ \frac{A_{2,1}}{x-2}+\frac{A_{2,2}}{(x-2)^2} \]
und bringen die rechte Seite auf den Hauptnenner $x(x-2)^2$ durch die Rechnung
\begin{eqnarray*}
 \frac{A_1}{x}+ \frac{A_{2,1}}{x-2}+\frac{A_{2,2}}{(x-2)^2}
&=&\frac{A_1(x-2)^2+A_{2,1}x(x-2)+A_{2,2}x}{x(x-2)^2}\\
&=& \frac{(A_1+A_{2,1})x^2+(-4A_1-2A_{2,1}+A_{2,2})x+4A_1}{x(x-2)^2}.
\end{eqnarray*}
\step Anschließend vergleichen wir die Koeffizienten der Zähler:
\[ \begin{matrix}
A_1 &+ A_{2,1}& &=& 0,& \qquad \text{(Koeff. von $x^2$)} \\
-4A_1 &-2A_{2,1} & + A_{2,2}&=& 0,& \qquad \text{(Koeff. von $x$)} \\
4A_1 & &  &= & 1. &\qquad \text{(Koeff. von $x^0$)}
\end{matrix} \]
\step Daraus erhalten wir sukzessive
\[  A_1=\frac{1}{4},\qquad A_{2,1}=-A_1=-\frac{1}{4}\quad \text{und}\quad
A_{2,2}=4A_1+2A_{2,1}=\frac{1}{2}.\]
Wir erhalten damit die Partialbruchzerlegung
\[ R_1(x)=\frac{1}{x(x-2)^2}=\frac{1}{4}\cdot \frac{1}{x}-\frac{1}{4}\cdot  \frac{1}{x-2}+\frac{1}{2}\cdot\frac{1}{(x-2)^2}. \]

\step Eine Stammfunktion für $R_1$ ergibt sich dann aus der Summe der Stammfunktionen für die
einzelnen Summanden, also zum Beispiel
\[  \frac{1}{4}\cdot \ln({|x|})-\frac{1}{4}\cdot  \ln({|x-2|})+ \frac{1}{2}\cdot\frac{-1}{(x-2)}. \]

  	 %------------------------------------END_STEP_X
 
  \end{incremental}
  %++++++++++++++++++++++++++++++++++++++++++++END_TAB_X


  %++++++++++++++++++++++++++++++++++++++++++START_TAB_X
  \tab{\lang{de}{    Lösung b)    }}
  \begin{incremental}[\initialsteps{1}]
  
  	 %----------------------------------START_STEP_X
    \step    
Da $x^2+4$ keine reellen Nullstellen hat, ist in diesem Fall der Nenner $(x^2+4)^3$ schon vollständig in Faktoren
zerlegt. 
\step Für die Partialbruchzerlegung müssen wir die rationale Funktion
$R_2(x)=\frac{x^3}{(x^2+4)^3}$ jedoch in eine Summe
\[  \frac{B_1x+C_1}{x^2+4}+  \frac{B_2x+C_2}{(x^2+4)^2}+ \frac{B_3x+C_3}{(x^2+4)^3} \]
zerlegen.
Um die Zahlen $B_1,C_1,B_2,C_2,B_3,C_3\in \R$ zu bestimmen, bringen wir die rechte Seite wieder auf den Hauptnenner $(x^2+4)^3$, also
\begin{eqnarray*}
 \frac{B_1x+C_1}{x^2+4}+  \frac{B_2x+C_2}{(x^2+4)^2}+ \frac{B_3x+C_3}{(x^2+4)^3} &=& \frac{(B_1x+C_1)(x^2+4)^2+(B_2x+C_2)(x^2+4)+(B_3x+C_3)}{(x^2+4)^3} \\
 &=& \frac{B_1x^5+C_1x^4+(8B_1+B_2)x^3+(8C_1+C_2)x^2+(16B_1+4B_2+B_3)x+(16C_1+4C_2+C_3)}{(x^2+4)^3},
\end{eqnarray*}
und vergleichen anschließend die Koeffizienten der Zähler:
\step \[ \begin{matrix}
B_1 & = & 0,& \qquad \text{(Koeff. von $x^5$)} \\
C_1 &=& 0,& \qquad \text{(Koeff. von $x^4$)} \\
8B_1+B_2 &=& 1,& \qquad \text{(Koeff. von $x^3$)} \\
8C_1+C_2 &=& 0,& \qquad \text{(Koeff. von $x^2$)} \\
16B_1+4B_2+B_3 &=& 0,& \qquad \text{(Koeff. von $x$)} \\
16C_1+4C_2+C_3 &= & 0. &\qquad \text{(Koeff. von $x^0$)}
\end{matrix} \]
\step Daraus erhalten wir von oben nach unten:
\[  B_1=0, \quad C_1=0,\quad B_2=1-8B_1=1,\quad C_2=-8C_1=0, \]
\[ B_3=-16B_1-4B_2=-4,\quad C_3=-16C_1-4C_2=0.\]
Wir erhalten damit die Partialbruchzerlegung
\[ R_2(x)=\frac{x^3}{(x^2+4)^3} = \frac{x}{(x^2+4)^2} + \frac{-4x}{(x^2+4)^3}. \]

\step Um eine Stammfunktion zu finden, erkennen wir, dass in beiden Brüchen der Zähler bis auf einen reellen Faktor genau die Ableitung von $x^2+4$ ist.
 Mit der Substitutionsregel findet man für die Funktion
\[ \frac{x}{(x^2+4)^2} =\frac{1}{2}\cdot \frac{2x}{(x^2+4)^2} \]
 die Stammfunktion
\[ \frac{1}{2}\cdot \frac{-1}{(x^2+4)}. \]
Ebenso erhält man für 
\[  \frac{-4x}{(x^2+4)^3}=-2\cdot  \frac{2x}{(x^2+4)^3} \]
die Stammfunktion
\[ -2\cdot \frac{1}{-2}\cdot \frac{1}{(x^2+4)^2}= \frac{1}{(x^2+4)^2}.\]

Insgesamt ist also eine Stammfunktion von $R_2(x)$ gegeben durch
\[  - \frac{1}{2}\cdot \frac{1}{(x^2+4)} +  \frac{1}{(x^2+4)^2}. \]


  	 %------------------------------------END_STEP_X
 
  \end{incremental}
  %++++++++++++++++++++++++++++++++++++++++++++END_TAB_X

%#############################################################ENDE

%    \tab{\lang{de}{Video: ähnliche Übungsaufgabe}}
%  \youtubevideo[500][300]{XdAyAs5-bRU}\\

\end{tabs*}
\end{content}