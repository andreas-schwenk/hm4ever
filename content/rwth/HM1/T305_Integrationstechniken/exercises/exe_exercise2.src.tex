\documentclass{mumie.element.exercise}
%$Id$
\begin{metainfo}
  \name{
    \lang{de}{Ü02: Partielle Integration}
    \lang{en}{Exercise 2}
  }
  \begin{description} 
 This work is licensed under the Creative Commons License Attribution 4.0 International (CC-BY 4.0)   
 https://creativecommons.org/licenses/by/4.0/legalcode 

    \lang{de}{}
    \lang{en}{}
  \end{description}
  \begin{components}
  \end{components}
  \begin{links}
  \end{links}
  \creategeneric
\end{metainfo}
\begin{content}
\usepackage{mumie.ombplus}

\title{\lang{de}{Ü02: Partielle Integration}}

\begin{block}[annotation]
  Im Ticket-System: \href{http://team.mumie.net/issues/10920}{Ticket 10920}
\end{block}

%######################################################FRAGE_TEXT

\begin{enumerate}
\item Bestimmen Sie mit Hilfe der partiellen Integration eine Stammfunktion von
$f(x)=\frac{\ln(x)}{x^2}$ und berechnen Sie das Integral
\[ \int_1^e \frac{\ln(x)}{x^2}\; dx. \]

\item
\begin{enumerate}
\item Berechnen Sie mittels partieller Integration eine Stammfunktion zu $f(x)=\sin(x)\cos(x)$ 
und $g(x)=\sin(x)\cdot \sin(2x)$.
\item Fallen Ihnen auch andere Wege zur Bestimmung von Stammfunktionen zu den Funktionen aus a) ein?\\
\textit{(Tipp: Additionstheoreme)}
\item Bestimmen Sie die Werte von
\[
\int_0^{2 \pi} \sin(x) \cdot \cos(x)\, dx \ \text{ und } \int_0^{2 \pi}\sin(x)\cdot \sin(2x)\, dx.
\]
\end{enumerate}
\end{enumerate}
%##################################################ANTWORTEN_TEXT
\begin{tabs*}[\initialtab{0}\class{exercise}]

\tab{Antworten}
\begin{enumerate}
\item Eine Stammfunktion für $f$ ist gegeben durch
\[ F(x)= - \frac{\ln(x)}{x} -\frac{1}{x} .\]
Weiter ist
\[
\int_1^e \frac{\ln(x)}{x^2}\; dx = -\frac{2}{e} +1.
\]
\item Eine Stammfunktion ist z.\,B. $F(x)= \frac{1}{2}\sin(x)^2$ und\\
$G(x) = \frac{1}{3}\cos(x)\cdot \sin(2x) - \frac{2}{3} \sin(x)\cos(2x).$\\
Weiter sind
\[
\int_0^{2 \pi} \sin(x) \cdot \cos(x)= 0=\int_0^{2 \pi}\sin(x)\cdot \sin(2x)\, dx.
\]
\end{enumerate}

  %++++++++++++++++++++++++++++++++++++++++++START_TAB_X
  \tab{\lang{de}{    Lösung zu 1.   }}
  \begin{incremental}[\initialsteps{1}]
  
  	 %----------------------------------START_STEP_X
    \step   
Bei Funktionen, die den Logarithmus als Faktor enthalten, ist der erste Ansatz, $u(x)=\ln(x)$
zu setzen und den anderen Faktor als $v'(x)$, in diesem Fall $v'(x)=\frac{1}{x^2}=x^{-2}$.

\step Dann ist nämlich $u'(x)=\frac{1}{x}$ und für $v$ kann $v(x)=-x^{-1}=- \frac{1}{x}$
gewählt werden.
Damit gilt für jedes Intervall $[a;b]$, auf dem $f(x)=\frac{\ln(x)}{x^2}$ definiert ist, d.\,h. für alle $0<a<b$:
\begin{eqnarray*}
\int_a^b \frac{\ln(x)}{x^2}\; dx &=&
 \left[ \ln(x)\cdot (- \frac{1}{x}) \right]_a^b -\int_a^b\frac{1}{x}\cdot (- \frac{1}{x})  \; dx \\
&=&  \left[ - \frac{\ln(x)}{x} \right]_a^b + \int_a^b\frac{1}{x^2 } \; dx\\
&=&  \left[ - \frac{\ln(x)}{x} \right]_a^b + \left[ - \frac{1}{x} \right]_a^b  \\
&=&  \left[ - \frac{\ln(x)}{x} -\frac{1}{x} \right]_a^b 
\end{eqnarray*}
\step Eine Stammfunktion für $f$ ist damit gegeben durch
\[ F(x)= - \frac{\ln(x)}{x} -\frac{1}{x} .\]


\step Um das Integral zu berechnen, muss man nun lediglich die Grenzen in $F(x)$ einsetzen:
\begin{eqnarray*}
\int_1^e \frac{\ln(x)}{x^2}\; dx &=&  \left[  F(x) \right]_1^e =  \left[ - \frac{\ln(x)}{x} -\frac{1}{x} \right]_1^e  \\
&=& \left( -\frac{\ln(e)}{e}-\frac{1}{e}\right) - \left( -\frac{\ln(1)}{1}-\frac{1}{1}\right) \\
&=& -\frac{2}{e} +1 
\end{eqnarray*}

  	 %------------------------------------END_STEP_X
 
  \end{incremental}
  
  \tab{Lösungsvideo zu 2.}
  \youtubevideo[500][300]{QfPYB7vKYdg}\\

  
  %++++++++++++++++++++++++++++++++++++++++++++END_TAB_X

%#############################################################ENDE
\end{tabs*}
\end{content}