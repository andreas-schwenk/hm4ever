\documentclass{mumie.element.exercise}
%$Id$
\begin{metainfo}
  \name{
    \lang{de}{Ü06: Partialbruchzerlegung}
    \lang{en}{Exercise 6}
  }
  \begin{description} 
 This work is licensed under the Creative Commons License Attribution 4.0 International (CC-BY 4.0)   
 https://creativecommons.org/licenses/by/4.0/legalcode 

    \lang{de}{}
    \lang{en}{}
  \end{description}
  \begin{components}
  \end{components}
  \begin{links}
  \end{links}
  \creategeneric
\end{metainfo}
\begin{content}
\usepackage{mumie.ombplus}

\title{\lang{de}{Ü06: Partialbruchzerlegung}}

\begin{block}[annotation]
  Im Ticket-System: \href{http://team.mumie.net/issues/10924}{Ticket 10924}
\end{block}


%######################################################FRAGE_TEXT
\lang{de}{ 
Bestimmen Sie jeweils die Partialbruchzerlegung der rationalen Funktionen
\begin{enumerate}[alph]
\item $\frac{4x-3}{x^2+x-6}$,
\item $\frac{2x}{x^2-1}$,
\item $\frac{2x+3}{x^2+2x+1}$,
\item $\frac{-2x^2-3}{x^3+x}$,
\item $\frac{1}{2x^2-2x-4}$,
\item $f_1(x)=\frac{x}{x^2-2},$
\item $f_2(x)=\frac{2x+1}{x^3+4x^2+3x}$,
\item $f_3(x)= \frac{1}{x^3+x^2+x+1} $.
\end{enumerate}
 }

%##################################################ANTWORTEN_TEXT
\begin{tabs*}[\initialtab{0}\class{exercise}]

  %++++++++++++++++++++++++++++++++++++++++++START_TAB_X
  \tab{\lang{de}{    Antworten    }}
    \lang{de}{   
Die Partialbruchzerlegungen der gegebenen Funktionen sind:
\begin{enumerate}[alph]
\item $ \frac{1}{x-2}+\frac{3}{x+3}$,
\item $\frac{1}{x-1}+\frac{1}{x+1}$,
\item $\frac{2}{x+1}+\frac{1}{(x+1)^2}$,
\item $\frac{-3}{x}+\frac{x}{x^2+1}$,
\item $\frac{1}{6}\cdot\frac{1}{x-2}+\frac{-1}{6}\cdot\frac{1}{x+1}$,
\item $ f_1(x)=\frac{x}{x^2-2}=\frac{1}{2} \cdot \frac{1}{x-\sqrt{2}}+ \frac{1}{2}\cdot   \frac{1}{x+\sqrt{2}} $,
\item $ f_2(x)=\frac{2x+1}{x^3+4x^2+3x}= \frac{1}{3}\cdot \frac{1}{x}- \frac{5}{6}\cdot \frac{1}{x+3}+ \frac{1}{2}\cdot \frac{1}{x+1}$,
\item $ f_3(x)= \frac{1}{x^3+x^2+x+1}=\frac{1}{2}\cdot \frac{1}{x+1} +\frac{1}{2}\cdot  \frac{-x+1}{x^2+1} $.
\end{enumerate}    }
  	 %------------------------------------END_STEP_X
  %++++++++++++++++++++++++++++++++++++++++++++END_TAB_X
 \tab{\lang{de}{    Lösungsvideos a) - e) }}
 
 \youtubevideo[500][300]{A49VvVmjxBg}\\
 
  %++++++++++++++++++++++++++++++++++++++++++START_TAB_X
  \tab{\lang{de}{    Lösung f)    }}
  \begin{incremental}[\initialsteps{1}]
  
  	 %----------------------------------START_STEP_X
    \step   
Für die Partialbruchzerlegung ist zunächst der Nenner in Faktoren zu zerlegen.
Mit der dritten binomischen Formel erhalten wir
\[ x^2-2=(x-\sqrt{2})(x+\sqrt{2}). \]
\step Nun macht man den Ansatz
\[ f_1(x)=\frac{x}{x^2-2}=\frac{A_1}{x-\sqrt{2}}+\frac{A_2}{x+\sqrt{2}},\]
bringt die rechte Seite auf den Hauptnenner $(x-\sqrt{2})(x+\sqrt{2})$, was
\[ \frac{A_1}{x-\sqrt{2}}+\frac{A_2}{x+\sqrt{2}}=\frac{A_1(x+\sqrt{2})+A_2(x-\sqrt{2})}{(x-\sqrt{2})(x+\sqrt{2})}=\frac{(A_1+A_2)x+(A_1\sqrt{2}-A_2\sqrt{2})}{x^2-2} \]
ergibt, und vergleicht die Koeffizienten im Zähler:
\step
\[ \begin{matrix}
A_1+A_2 &=& 1,& \qquad \text{(Koeff. von $x$)} \\
A_1\sqrt{2}-A_2\sqrt{2} &=& 0.& \qquad \text{(Koeff. von $x^0$)} 
\end{matrix} \]
\step Aus der zweiten Gleichung erhalten wir $A_1=A_2$ und damit aus der ersten $2A_2=1$ bzw. $A_2=\frac{1}{2}$.
Also $A_1=A_2=\frac{1}{2}$.

Wir erhalten damit die Partialbruchzerlegung
\[ \frac{x}{x^2-2}= \frac{1}{2} \cdot \frac{1}{x-\sqrt{2}}+ \frac{1}{2}\cdot   \frac{1}{x+\sqrt{2}}. \]
  	 %------------------------------------END_STEP_X
 
  \end{incremental}
  %++++++++++++++++++++++++++++++++++++++++++++END_TAB_X


  %++++++++++++++++++++++++++++++++++++++++++START_TAB_X
  \tab{\lang{de}{    Lösung g)    }}
  \begin{incremental}[\initialsteps{1}]
  
  	 %----------------------------------START_STEP_X
    \step 
Für die Partialbruchzerlegung ist zunächst der Nenner in Faktoren zu zerlegen.
Lineare Faktoren findet man durch Bestimmung der Nullstellen:
\begin{eqnarray*}
&&  x^3+4x^2+3x =0\\ 
&\Leftrightarrow& x(x^2+4x+3)=0 \\
&\Leftrightarrow& x=0 \text{ oder }x^2+4x+3=0 \\
&\Leftrightarrow&  x=0 \text{ oder } x=\frac{-4}{2}\pm \sqrt{\frac{16}{4}-3}=-2\pm 1\\
&\Leftrightarrow&  x=0 \text{ oder } x=-3 \text{ oder } x=-1.
\end{eqnarray*}
\step
Damit gilt:
\[   x^3+4x^2+3x= (x-0)(x-(-3))(x-(-1))=x(x+3)(x+1). \]
\step Um nun die Partialbruchzerlegung zu bestimmen, macht man den Ansatz
\[ f_2(x)=\frac{2x+1}{x^3+4x^2+3x}= \frac{A_1}{x}+\frac{A_2}{x+3}+\frac{A_3}{x+1}. \]
\step Wir bringen die rechte Seite auf den Hauptnenner, also 
\begin{eqnarray*}  
\frac{A_1}{x}+\frac{A_2}{x+3}+\frac{A_3}{x+1}&=& 
\frac{A_1(x+3)(x+1)+A_2x(x+1)+A_3x(x+3)}{x(x+3)(x+1)}\\
&=&\frac{(A_1+A_2+A_3)x^2+(4A_1+A_2+3A_3)x+3A_1}{x(x+3)(x+1)},
 \end{eqnarray*}
und vergleichen die Koeffizienten der Zähler:
\[ \begin{matrix}
A_1+A_2+A_3 &=& 0,& \qquad \text{(Koeff. von $x^2$)} \\
4A_1+A_2+3A_3 &=& 2,& \qquad \text{(Koeff. von $x$)} \\
3A_1 &=& 1. &\qquad \text{(Koeff. von $x^0$)}
\end{matrix} \]
\step Den Wert $A_1=\frac{1}{3}$ aus der dritten Gleichung setzen wir direkt ein und berechnen $A_2$ und $A_3$ durch
Umformung des Gleichungssystems:\\
\[ \begin{bmatrix}
A_2 &+\, A_3 &= -\frac{1}{3} \\
A_2 &+3A_3&=\frac{2}{3}
\end{bmatrix} \rightsquigarrow \begin{bmatrix}
A_2 &+ A_3 &= -\frac{1}{3} \\
      &  \,2A_3&=1
\end{bmatrix} \rightsquigarrow \begin{bmatrix}
A_2 &+ A_3 &= -\frac{1}{3} \\
      & \,\, A_3&=\frac{1}{2}
\end{bmatrix}  \rightsquigarrow \begin{bmatrix}
A_2 &   &= -\frac{5}{6} \\
      &  A_3&=\frac{1}{2}
\end{bmatrix}  
\]
Wir erhalten damit die Zerlegung:
\[ \frac{2x+1}{x^3+4x^2+3x}= \frac{1}{3}\cdot \frac{1}{x}- \frac{5}{6}\cdot \frac{1}{x+3}+ \frac{1}{2}\cdot \frac{1}{x+1}.\]

  	 %------------------------------------END_STEP_X
 
  \end{incremental}
  %++++++++++++++++++++++++++++++++++++++++++++END_TAB_X


  %++++++++++++++++++++++++++++++++++++++++++START_TAB_X
  \tab{\lang{de}{    Lösung h)    }}
  \begin{incremental}[\initialsteps{1}]
  
  	 %----------------------------------START_STEP_X
    \step 
Für die Partialbruchzerlegung ist zunächst der Nenner in Faktoren zu zerlegen.
Lineare Faktoren findet man durch Bestimmung der Nullstellen, welche bei einem Polynom
dritten Grades "`geraten"' werden müssen. Im Fall des Polynoms $x^3+x^2+x+1$ erkennt man leicht $x=-1$ als Nullstelle, weshalb das Polynom den Linearfaktor $x+1$ hat.
Mittels Polynomdivision (oder hier direkt durch $x^3+x^2+x+1=x^2(x+1)+(x+1)$) sieht man
\[ x^3+x^2+x+1 =(x+1)(x^2+1). \]
Da das Polynom $x^2+1$ keine reellen Nullstellen hat, können wir es nicht weiter zerlegen.

\step Um nun die Partialbruchzerlegung zu bestimmen, macht man daher den Ansatz
\[ f_3(x)= \frac{1}{x^3+x^2+x+1} = \frac{A}{x+1} + \frac{Bx+C}{x^2+1} ,\]
bringt die rechte Seite auf den Hauptnenner $(x+1)(x^2+1)$, also
\[ \frac{A}{x+1} + \frac{Bx+C}{x^2+1}=\frac{A(x^2+1)+(Bx+C)(x+1)}{(x+1)(x^2+1)}
=\frac{(A+B)x^2+(B+C)x+(A+C)}{(x+1)(x^2+1)}, \]
und vergleicht die Koeffizienten der Zähler:
\[ \begin{matrix}
A &+ B& &=& 0,& \qquad \text{(Koeff. von $x^2$)} \\
 & \,\,B & +C &=& 0,& \qquad \text{(Koeff. von $x$)} \\
A & & +C &= & 1. &\qquad \text{(Koeff. von $x^0$)}
\end{matrix} \]
\step Aus den ersten beiden Gleichungen erhält man $B=-A$ und $C=-B=A$, und  daher
aus der dritten $2A=1$, d.\,h. $A=\frac{1}{2}$.\\
Schließlich $C=A=\frac{1}{2}$ und $B=-A=-\frac{1}{2}$. 

Wir erhalten damit die Partialbruchzerlegung
\[  \frac{1}{x^3+x^2+x+1} =  \frac{1}{2}\cdot \frac{1}{x+1} +\frac{1}{2}\cdot  \frac{-x+1}{x^2+1} .\]
  	 %------------------------------------END_STEP_X
 
  \end{incremental}
  %++++++++++++++++++++++++++++++++++++++++++++END_TAB_X
%#############################################################ENDE
\end{tabs*}
\end{content}