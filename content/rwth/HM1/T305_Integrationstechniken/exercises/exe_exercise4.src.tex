\documentclass{mumie.element.exercise}
%$Id$
\begin{metainfo}
  \name{
    \lang{de}{Ü04: Partielle Integration}
    \lang{en}{}
  }
  \begin{description} 
 This work is licensed under the Creative Commons License Attribution 4.0 International (CC-BY 4.0)   
 https://creativecommons.org/licenses/by/4.0/legalcode 

    \lang{de}{Hier die Beschreibung}
    \lang{en}{}
  \end{description}
  \begin{components}
  \end{components}
  \begin{links}
    \link{generic_article}{content/rwth/HM1/T209_Potenzreihen/g_art_content_28_exponentialreihe.meta.xml}{link1}
  \end{links}
  \creategeneric
\end{metainfo}
\begin{content}
\title{
  \lang{de}{Ü04: Partielle Integration}
}



\begin{block}[annotation]
  Im Ticket-System: \href{http://team.mumie.net/issues/10922}{Ticket 10922}
\end{block}

Bestimmen Sie mit Hilfe von partieller Integration eine Stammfunktion von \\
a) $\ f(x)=\sin(x)\cos(x)$ und \\
b) $\ g(x)=\exp(x)\sin(x)$.

\begin{tabs*}[\initialtab{0}\class{exercise}]

 \tab{\lang{de}{Antwort}
 \lang{en}{Answer}}

a) Mögliche Stammfunktionen von $f$ sind $F(x)=\frac{1}{2}\sin(x)^2+C$ oder $F(x)=-\frac{1}{2}\cos(x)^2+C$ mit $C \in \R$.\\
b) Eine mögliche Stammfunktion von $g$ ist $G(x)=\frac{1}{2}\exp(x)\cdot(\sin(x)-\cos(x))$.

\tab{\lang{de}{L"osung zu a)}
\lang{en}{Solution}
}

 \begin{incremental}[\initialsteps{1}]
  \step Wir führen eine partielle Integration aus. Wir wählen $u(x)=\sin(x)$ und $v'(x)=\cos(x)$. Wir werden gleich
  noch die Rollen vertauschen und sehen, dass eine Vertauschung der beiden Rollen von $u$ und $v'$ auch zum Ziel führt.
  
  \step Also sei $u(x)=\sin(x)$ und $v'(x)=\cos(x)$. Dann sind $u'(x)=\cos(x)$ und $v(x)=\sin(x)$.
  Damit gilt für $a<b$ nach partieller Integration:
  \[
  \int_a^b \textcolor{#0066CC}{\sin(x)\cos(x)} \, dx = \int_a^b u(x)v'(x)\, dx = [\sin(x)\sin(x)]_a^b - \int_a^b \cos(x)\sin(x)\, dx
  = [\sin(x)^2]_a^b - \int_a^b \textcolor{#0066CC}{\sin(x)\cos(x)}\, dx.
\]
 
\step Wir haben nun eine Gleichung, die wir auflösen können, um eine Stammfunktion zu erhalten:
\[
2 \int_a^b \textcolor{#0066CC}{\sin(x)\cos(x)}\, dx = [\sin(x)^2]_a^b.
\]
\step Eine Stammfunktion ist also durch $F_1(x)=\frac{1}{2}\sin(x)^2$ gegeben.

\step Nun vergleichen wir dieses Ergebnis mit dem, das wir erhalten, wenn wir $u(x)=\cos(x)$ und $v'(x) = \sin(x)$ gewählt hätten.
Dann ist $u'(x) = -\sin(x)$ und $v(x)= -\cos(x)$.

\step Damit gilt für $a<b$
\[
\int_a^b \textcolor{#0066CC}{\sin(x)\cos(x)} \, dx = \int_a^b u(x)v'(x)\, dx = [\cos(x)(-\cos(x))]_a^b - \int_a^b (-\sin(x))(-\cos(x))\, dx
  = [-\cos(x)^2]_a^b - \int_a^b \textcolor{#0066CC}{\sin(x)\cos(x)}\, dx.
\]
\step Wenn wir hier die Gleichung auflösen, dann erhalten wir eine mögliche Stammfunktion
\[
F_2(x)=-\frac{1}{2}\cos(x)^2.
\]
\step Wir haben nun zwei verschiedene Stammfunktionen von $f$ bestimmt. Wie passt das zusammen? Haben wir einen Fehler gemacht?
Wir wissen bereits, dass Stammfunktionen eindeutig bis auf Addition einer Konstanten sind.
Das heißt, es müsste ein $C \in \R$ geben, sodass
\[
C= F_1(x)-F_2(x)=\frac{1}{2} \sin(x)^2 - \left(-\frac{1}{2}\cos(x)^2\right)= \frac{1}{2}(\sin(x)^2+\cos(x)^2).
\]
Wir haben aber auch schon \ref[link1][kennengelernt]{rem:sin_cos_exp}, dass $\sin(x)^2+\cos(x)^2=1$ für alle $x\in \R$ gilt. Damit ist $C=\frac{1}{2}$ und beide Ergebnisse sind kein Widerspruch.
\end{incremental}

\tab{Lösung zu b)}
\begin{incremental}[\initialsteps{1}]
\step Wir führen eine partielle Integration durch mit $u(x)=\sin(x)$ und $v'(x)=\exp(x)$. (Sie können die beiden Wahlen aber auch vertauschen. Dies führt zum gleichen Ergebnis.)
Dann ist $u'(x)=\cos(x)$ und $v(x)=\exp(x)$.
\step Damit gilt für $a<b$
\[
\int_a^b \textcolor{#0066CC}{\exp(x)\sin(x)}\, dx = [\exp(x)\sin(x)]_a^b - \int_a^b \textcolor{#CC6600}{\cos(x)\exp(x)}\, dx.
\]
\step Das ursprüngliche Integral erkennen wir hier noch nicht wieder. Wir führen eine erneute partielle Integration durch.
Nun setzen wir $u(x)=\cos(x)$ und $v'(x)=\exp(x)$. Wichtig ist hier, dass wir wieder $v'(x)=\exp(x)$ setzen, ansonsten würden wir wieder zum Anfangspunkt zurückkehren und so gesehen unsere partielle Integration rückgängig machen.
Wir haben dann $u'(x)=-\sin(x)$ und $v(x)=\exp(x)$. Und für $a<b$ haben wir dann
\[
\int_a^b \textcolor{#CC6600}{\cos(x)\exp(x)}\, dx = [\cos(x)\exp(x)]_a^b-\int_a^b (-\sin(x))\exp(x)\, dx = [\cos(x)\exp(x)]_a^b + \int_a^b \textcolor{#0066CC}{\exp(x)\sin(x)}\, dx.
\]

\step Setzen wir dies zusammen, gilt
\[
\int_a^b \textcolor{#0066CC}{\exp(x)\sin(x)}\, dx = [\exp(x)\sin(x) - \exp(x)\cos(x)]_a^b - \int_a^b \textcolor{#0066CC}{\exp(x)\sin(x)}\, dx.
\]
Die Gleichung ergibt
\[
\int_a^b \textcolor{#0066CC}{\exp(x)\sin(x)}\, dx = \frac{1}{2}[\exp(x)\sin(x)-\exp(x)\cos(x)]_a^b.
\]
Damit ist $\frac{1}{2}\exp(x)\cdot(\sin(x)-\cos(x))$ eine gesuchte Stammfunktion.
\end{incremental}
\end{tabs*}


\end{content}