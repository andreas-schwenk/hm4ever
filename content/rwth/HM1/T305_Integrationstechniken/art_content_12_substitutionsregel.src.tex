%$Id:  $
\documentclass{mumie.article}
%$Id$
\begin{metainfo}
  \name{
    \lang{de}{Substitutionsregel}
    \lang{en}{Substitution rules}
  }
  \begin{description} 
 This work is licensed under the Creative Commons License Attribution 4.0 International (CC-BY 4.0)   
 https://creativecommons.org/licenses/by/4.0/legalcode 

    \lang{de}{Beschreibung}
    \lang{en}{Description}
  \end{description}
  \begin{components}
\component{generic_image}{content/rwth/HM1/images/g_img_00_Videobutton_schwarz.meta.xml}{00_Videobutton_schwarz}
\end{components}
  \begin{links}
    \link{generic_article}{content/rwth/HM1/T211_Eigenschaften_stetiger_Funktionen/g_art_content_34_exp_und_log.meta.xml}{content_34_exp_und_log}
    \link{generic_article}{content/rwth/HM1/T301_Differenzierbarkeit/g_art_content_02_ableitungsregeln.meta.xml}{abl-regeln}
    \link{generic_article}{content/rwth/HM1/T107_Integralrechnung/g_art_content_26_flaechen_zwischen_graphen.meta.xml}{lin-sub}
  \end{links}
  \creategeneric
\end{metainfo}
\begin{content}
\usepackage{mumie.ombplus}
\ombchapter{4}
\ombarticle{2}

\title{\lang{de}{Substitutionsregel}\lang{en}{Substitution rules}}
 
\begin{block}[annotation]
  
  
\end{block}
\begin{block}[annotation]
  Im Ticket-System: \href{http://team.mumie.net/issues/10042}{Ticket 10042}\\
\end{block}

\begin{block}[info-box]
\tableofcontents
\end{block}

\lang{de}{
In diesem Abschnitt werden wir eine Integrationstechnik behandeln, die sich aus der
\ref[abl-regeln][Kettenregel der Differentiation]{sec:kettenregel} ergibt.
\\\\
Einen Spezialfall dieser Integrationstechnik haben wir mit der \ref[lin-sub][linearen Substitution]{rule:lin-sub}
schon gesehen.
}
\lang{en}{
In this section we will introduce an integration technique which derives from the 
\ref[abl-regeln][chain rule for differentiation]{sec:kettenregel}.
\\\\
We have already seen a special case of this technique, \ref[lin-sub][linear substitution]{rule:lin-sub}.
}

\section{\lang{de}{Allgemeine Substitutionsregel}
         \lang{en}{General substitution rules}}\label{sec:allg-subst-reg}

\begin{theorem}[\lang{de}{Substitutionsregel}
                \lang{en}{Substitution rules}]\label{thm:substitutionsregel}
\lang{de}{
Es sei $I\subseteq \R$ ein Intervall und $f\colon I\to\R$ habe eine Stammfunktion $F.$\\
Weiter seien $\alpha < \beta$ und $g\colon [\alpha;\beta] \to I, \;x\mapsto g(x),$ stetig 
differenzierbar mit Ableitung $g'(x).$
Dann gilt: 
}
\lang{en}{
Let $I\subseteq \R$ be an interval and suppose $f\colon I\to\R$ has an antiderivative $F.$\\
Let $\alpha < \beta$ and let $g\colon [\alpha;\beta] \to I, \;x\mapsto g(x),$ be continuously 
differentiable with derivative $g'(x)$. 
Then we have:
}
    \[
  \int_\alpha^\beta \;f(\,g(x)\,)\cdot g'(x)\;dx = \int_{g(\alpha)}^{g(\beta)}\; f(y)\;dy
  = \left[ F(y) \right]_{g(\alpha)}^{g(\beta)}= \left[ F(g(x)) \right]_\alpha^\beta.
  \]
  %\floatright{\href{https://www.hm-kompakt.de/video?watch=627}{\image[75]{00_Videobutton_schwarz}}}\\\\
\end{theorem}

\begin{proof*}
\begin{incremental}[\initialsteps{1}]
\step
\lang{de}{Nach der Kettenregel ist die Ableitung der Funktion $F(g(x))$ gegeben durch}
\lang{en}{By the chain rule, the derivative of the function $F(g(x))$ is given by}
\[ (F\circ g)'(x)= F'(g(x))\cdot g'(x). \]
\lang{de}{
Da $F$ eine Stammfunktion von $f$ ist, gilt $F'=f$, d.\,h. $ (F\circ g)'(x)=f(g(x))\cdot g'(x)$.
\\\\
Damit ist aber nach dem Hauptsatz der Differential- und Integralrechnung
}
\lang{en}{
As $F$ is an antiderivative of $f$, we have $F'=f$, that is, $ (F\circ g)'(x)=f(g(x))\cdot g'(x)$.
\\\\
By the fundamental theorem of calculus,
}
  \[
  \int_\alpha^\beta \;f(\,g(x)\,)\cdot g'(x)\;dx = \left[ F(g(x)) \right]_{\alpha}^{\beta}.\]
\step
\lang{de}{
Die Gleichheit zu den anderen angegebenen Termen ergibt sich nun von hinten nach vorne durch
}
\lang{en}{
The other equalities now follow from
}
\[ \left[ F(g(x)) \right]_\alpha^\beta= F(g(\beta))-F(g(\alpha))= \left[ F(y) \right]_{g(\alpha)}^{g(\beta)}, \]
\lang{de}{
sowie mit dem Hauptsatz der Differential- und Integralrechnung angewandt auf die Funktion $f$:
}
\lang{en}{
and by applying the fundamental theorem of calculus to the function $f$:
}
\[  \int_{g(\alpha)}^{g(\beta)}\; f(y)\;dy  = \left[ F(y) \right]_{g(\alpha)}^{g(\beta)}. \]
\end{incremental}
\end{proof*}

\lang{de}{Wir diskutieren Anwendungen der allgemeinen Substitutionsregel.}
\lang{en}{Let us see some applications of the general substitution rules.}
\begin{example}
\begin{tabs*}[\initialtab{0}]
\tab{$\int_1^3 \frac{10}{(3-5x)^2}\;dx$} Wir betrachten das Integral $\int_1^3 \frac{10}{(3-5x)^2}\;dx$.\\
\lang{de}{
Setzen wir $g:[1;3]\to (-\infty;0), \ x \mapsto 3-5x$ und $f:(-\infty;0) \to \R, \ x \mapsto \frac{-2}{x^2}$, dann ist
}
\lang{en}{
Let $g:[1;3]\to (-\infty;0), \ x \mapsto 3-5x$ und $f:(-\infty;0) \to \R, \ x \mapsto \frac{-2}{x^2}$. Then
}
\[ f(\,g(x)\,)\cdot g'(x)= \frac{-2}{g(x)^2}\cdot g'(x)=\frac{-2}{(3-5x)^2}\cdot (-5)
= \frac{10}{(3-5x)^2} \]
\lang{de}{
genau der angegebene Integrand. Eine Stammfunktion von $f$ ist gegeben durch
$F(x)=\frac{2}{x}$ und daher ist
}
\lang{en}{
is precisely the given integrand. $F(x)=\frac{2}{x}$ is an antiderivative of $f$ and so
}
\begin{align*}
\int_1^3 \frac{10}{(3-5x)^2}\;dx &= \int_1^3\;f(g(x))\cdot g'(x)\;dx
= \int_{g(1)}^{g(3)} f(y)\, dy \\
&= \left[ F(y) \right]_{g(1)}^{g(3)} = \left[ \frac{2}{y} \right]_{-2}^{-12} \\ 
&= -\frac{2}{12}+\frac{2}{2}=\frac{5}{6}\,.
\end{align*}

\tab{$\int_0^1\; r\;\exp (-r^2/2)\;dr$}
\lang{de}{
Wir betrachten das Integral $\int_0^1\; r\;\exp (-r^2/2)\;dr$.\\
Hier bietet sich an, als innere Funktion $g(r)=-r^2/2$ zu setzen, wodurch $g'(r)=-r$ ist.
Der Integrand ist dann
}
\lang{en}{
Consider the integral $\int_0^1\; r\;\exp (-r^2/2)\;dr$.\\
It is convenient to choose $g(r)=-r^2/2$ as the inner function, so that $g'(r)=-r$. 
Then the integrand is
}
\[ r\;\exp (-r^2/2)=-g'(r)\cdot e^{g(r)}. \]
\lang{de}{Also ist}
\lang{en}{Thus}
\[ \int_0^1\; r\;\exp (-r^2/2)\;dr =-\int_0^1 e^{g(r)}\cdot g'(r)\, dr
= -\int_{g(0)}^{g(1)} e^{x}\;dx =- \left[ e^x\right]_{0}^{-1/2}
= -\frac{1}{\sqrt{e}} +1.
\]
\lang{de}{
Analog kann man so Funktionen der Art $x\cdot f(x^2)$ oder allgemeiner $x^{n-1}\cdot f(x^n)$ 
integrieren, wenn f\"ur $f$ eine Stammfunktion bekannt ist.
}
\lang{en}{
We can analogously integrate functions of the form $x\cdot f(x^2)$ or more generally 
$x^{n-1}\cdot f(x^n)$ if we know an antiderivative of $f$.
}
\tab{$\int_0^{\pi/4} \tan(t)\,dt$}
\lang{de}{
Wir betrachten das Integral $\int_0^{\pi/4} \tan(t)\,dt$.\\
Zunächst ist ja $\tan(t)= \frac{\sin(t)}{\cos(t)}=\sin(t)\cdot (\cos(t))^{-1}$.
Weil die Ableitung von $\cos(t)$ die Funktion $-\sin(t)$ ist, gilt also
$\tan(t)=-g'(t)\cdot g(t)^{-1}$ für $g(t)=\cos(t)$. Wir müssen aber auch aufpassen, dass wir nicht über eine 
Definitionslücke der Funktion $x \mapsto x^{-1}$ integrieren und auch eine richtige Stammfunktion finden. 
Daher betrachten wir den Wertebereich von $g(x)$: Für jeden Wert $x \in [0 ; \frac{\pi}{4}]$ ist $g(x)$ positiv. Damit ist
}
\lang{en}{
Consider the integral $\int_0^{\pi/4} \tan(t)\,dt$.\\
Firstly note that we have $\tan(t)= \frac{\sin(t)}{\cos(t)}=\sin(t)\cdot (\cos(t))^{-1}$.
As the derivative of $\cos(t)$ is the function $-\sin(t)$, we have 
$\tan(t)=-g'(t)\cdot g(t)^{-1}$ where $g(t)=\cos(t)$. 
We must be careful not to integrate over a gap in the domain of the function $x \mapsto x^{-1}$ and 
to find a correct antiderivative. Thankfully, $g(x)$ is positive for all 
$x \in [0 ; \frac{\pi}{4}]$, so
}
\begin{align*} 
\int_0^{\pi/4} \tan(t)\;dt \ &= \int_0^{\pi/4}-g'(t)\cdot g(t)^{-1} \;dt = 
-\int_{g(0)}^{g(\pi/4)} x^{-1}\; dx\\
 &= - \big[ \ln({|x|})\big]_{g(0)}^{g(\pi/4)} = -\big[ \ln({|\cos(t)|})\big]_0^{\pi/4} \\
 &= - \ln({|\cos(\frac{\pi}{4})|}) + \ln({|\cos(0)|})=-\ln(\frac{\sqrt{2}}{2}) + \ln(1)\\
 & =-\ln(2^{-1/2})=\frac{1}{2}\cdot \ln(2).
\end{align*}
\lang{de}{
Hier wurden im letzten Schritt die \ref[content_34_exp_und_log][Eigenschaften des Logarithmus]{rule:ln} genutzt.
}
\lang{en}{
In the final step we used some \ref[content_34_exp_und_log][properties of the logarithm]{rule:ln}.
}
\end{tabs*}
\end{example}


\section{\lang{de}{Anmerkungen zur Substitution}\lang{en}{Remarks about substitution}}
\lang{de}{
Die \ref[lin-sub][lineare Substitution]{rule:lin-sub} ist der Spezialfall der allgemeineren 
Substitutionsregel, wenn die innere Funktion $g(x)=mx+b$ ist. 
Wir wiederholen die Aussage hier:
}
\lang{en}{
\ref[lin-sub][Linear substitution]{rule:lin-sub} is a special case of the general substitution rule 
introduced above in which the inner function is of the form $g(x)=mx+b$. We repeat the former rule 
here:
}


% Ein spezieller Fall der Substitution ist die sogenannte \textit{lineare Substitution}. Die
% innere Funktion $g$ ist hierbei eine lineare Funktion. Die Ableitung $g'$ ist in diesem Fall 
% lediglich eine konstante reelle Zahl.
 
 
%  %\begin{tabs*} % [\initialtab{0}]
%  % \tab{\lang{de}{Beispiele} }
  
%  % \tab{\lang{de}{Beispiel 1}}
%  % Sei $h(x)=f(g(x))=e^{2x+1}$. Die äußere Funktion ist $f(x)=e^x$ mit $f'(x)=e^x$ und die innere Funktion ist
%  % $g(x)=2x+1$ mit $g'(x)=2$. Die Ableitung von $h(x)$ ist daher:
%  % \[ h'(x) = e^{2x+1} \cdot 2 \]
  
%   %\tab{\lang{de}{Beispiel 2}}
%   %Sei $h(x)=\sin^3(x)$. Die äußere Funktion ist $f(x)=x^3$ mit $f'(x)=3x^2$ 
%   %und die innere Funktion ist $g(x)=\sin(x)$ mit $g'(x)=\cos(x)$.
%   %Die Ableitung von $h(x)$ ist daher:
%   %\[ h'(x) = 3 \sin^2(x) \cos(x) \]
% %\end{tabs*}
 

 \begin{rule}[\lang{de}{Lineare Substitution}\lang{en}{Linear substitution}]
 \label{rule:lin-subst}
 \lang{de}{
 Wenn $F(x)$ eine Stammfunktion von $f(x)$ ist und Konstanten $m \neq 0$ und 
 $b \in \mathbb{R}$ gegeben sind, so ist
 }
 \lang{en}{
 If $F(x)$ is an antiderivative of $f(x)$ and constants $m \neq 0$ and $b \in \mathbb{R}$ are given, 
 then
 }
 \[ 
 \frac{1}{m} F(mx+b)
 \]
 \lang{de}{eine Stammfunktion von $f(mx+b)$.}
 \lang{en}{is an antiderivative of $f(mx+b)$.}
 \[ \int f(mx+b)\, dx = \frac{1}{m} F(mx+b) + C . \]
 \end{rule}
\lang{de}{
Es empfiehlt sich, bei der Bestimmung einer Stammfunktion mittels Substitution immer zu einem 
bestimmten Integral mit variablen Grenzen überzugehen. Dies ist übersichtlicher und so passieren 
seltener Fehler bei der Rücksubstitution.
}
\lang{en}{
When determining an antiderivative via substitution, it is advisable to choose a proper integral 
with variable limits. These are easier to keep track of, leading to fewer calculation mistakes.
}
\begin{remark}\label{rem:kalkuel}
\lang{de}{
Weit verbreitet in der Literatur ist auch folgendes Kalkül:
Bestimmen möchten wir $\int_a^b f(g(x)) g'(x) \, dx$. Wir wählen als Substitution $t = \textcolor{#0066CC}{g(x)}$.
Die Ableitung können wir auch als $\frac{dt}{dx} = g'(x)$ schreiben.
Umgeformt ergibt dies $\textcolor{#CC6600}{dx} = \frac{dt}{g'(x)}$.
Durch Ersetzen der Ausdrücke mit $x$ und entsprechendem Kürzen bekommen wir die bekannte Formel
}
\lang{en}{
The following reasoning is informal but is commonly seen in the literature on this topic. 
Suppose we wish to determine $\int_a^b f(g(x)) g'(x) \, dx$. As a substitution we choose 
$t = \textcolor{#0066CC}{g(x)}$. The derivative can also be written as $\frac{dt}{dx} = g'(x)$. 
Rearranging this would get us $\textcolor{#CC6600}{dx} = \frac{dt}{g'(x)}$. 
Now if we substitute $x$ and simplify, we get the well-known formula
}
\[
\int_{x=a}^{x=b} f(\textcolor{#0066CC}{g(x)}) \cdot g'(x)\, \textcolor{#CC6600}{dx} = \int_{t=g(a)}^{t=g(b)} f(t)\, dt.
\]
\lang{de}{
Daher stammt auch der Name der Substitutionsregel.
\\\\
Hier sollte aber darauf geachtet werden, dass die Substitution umkehrbar ist, d.\,h. $g:[a;b]\to \R$ injektiv.
}
\lang{en}{
This is from where the term 'substitution' comes from in reference to the above theorem.
\\\\
We must be careful here that the substitution is invertible, that is, that $g:[a;b]\to \R$ is 
injective.
}
\end{remark}
\lang{de}{
In den folgenden beiden Videos wird das Kalkül aus Bemerkung \ref{rem:kalkuel} zur Substitution genutzt:
\floatright{\href{https://www.hm-kompakt.de/video?watch=628}{\image[75]{00_Videobutton_schwarz}}
  \href{https://www.hm-kompakt.de/video?watch=629}{\image[75]{00_Videobutton_schwarz}}
 %\href{https://www.hm-kompakt.de/video?watch=630}{\image[75]{00_Videobutton_schwarz}}
  }\\\\
Wir diskutieren noch ein Beispiel dazu.
}
\lang{en}{Let us see an example of the above reasoning being used in a substitution.}
\begin{example}
\lang{de}{
Wir möchten das Integral $\int_0^{\pi/4} \frac{2\sin(x)}{\cos(x)^2}\, dx$ bestimmen.
Die offensichtliche Wahl wäre die Substitution $g(x)=\cos(x)$. Dann hätten wir
}
\lang{en}{
Suppose we want to compute the integral $\int_0^{\pi/4} \frac{2\sin(x)}{\cos(x)^2}\, dx$. The 
obvious choice would be the substitution $g(x)=\cos(x)$. Then we would have
}
\[
\int_0^{\pi/4} \frac{-2g'(x)}{g(x)^2} \, dx = -2\cdot \int_{g(0)}^{g(\pi/4)} \frac{1}{u^2}\, du = \left[\frac{2}{u}\right]_{1}^{1/\sqrt{2}}= 2\sqrt{2} -2.
\]
\lang{de}{
Nun kann man aber auch auf die Idee kommen, $t = h(x) = \textcolor{#0066CC}{\cos(x)^2}$ zu 
substituieren. Dann ist
}
\lang{en}{
Alternatively, we may choose $t = h(x) = \textcolor{#0066CC}{\cos(x)^2}$. Then
}
\[
\frac{dt}{dx} = h'(x) = -2 \sin(x) \cos(x) \Leftrightarrow \textcolor{#CC6600}{dx} = \frac{dt}{-2\sin(x) \cos(x)}.
\]
\lang{de}{Wenn wir dies im Integral einsetzen, erhalten wir}
\lang{en}{If we substitute this into the integral, we obtain}
\[
\frac{2 \sin(x) \, \textcolor{#CC6600}{dx}}{\textcolor{#0066CC}{\cos(x)^2}} = \frac{2\sin(x) dt}{-2\sin(x) \cos(x) t} = \frac{dt}{-\cos(x) \cdot t}.
\]
\lang{de}{
Wichtig ist nun, dass wir den Integranden nur in Termen mit $t$ schreiben. Dies ist möglich, da
$h:[0;\pi/4]\to \R$ injektiv ist und damit $\cos(x) = \sqrt{t}$ für $x \in [0;\pi/4]$.
Wir erhalten das alternative Integral
}
\lang{en}{
It is important to express the integrand entirely in terms of $t$. This is possible because 
$h:[0;\pi/4]\to \R$ is injective and so $\cos(x) = \sqrt{t}$ for $x \in [0;\pi/4]$. 
We obtain the alternative interval
}
\[
\int_0^{\pi/4} \frac{2 \sin(x)}{\cos(x)^2} \, dx = \int_{h(0)}^{h(\pi/4)} \frac{1}{-t^{3/2}}\, dt = \left[2\cdot \frac{1}{\sqrt{t}}\right]_{h(0)}^{h(\pi/4)} = \left[\frac{2}{\sqrt{t}}\right]_{1}^{1/2}=2\sqrt{2}-2.
\]
\lang{de}{Dies ist das gleiche Ergebnis wie oben.}
\lang{en}{This is the same result as for the first choice of function.}
\end{example}

\begin{quickcheckcontainer}
\randomquickcheckpool{1}{1}
\begin{quickcheck}
		\field{rational}
		\type{input.number}
		\begin{variables}
			\randint[Z]{a}{3}{5}
            \randint[Z]{n}{-1}{4}
		    \function[normalize]{f}{n*2*a*x*cos(a*x^2)}
            \function[normalize]{g}{a*x^2}
			\function[normalize]{ff}{n*sin(a*x^2)}
			\function[calculate]{ffupper}{n*sin(a*(pi/2))}
			\function[calculate]{fflower}{0}
			\function[calculate]{sol}{ffupper-fflower}
		\end{variables}
        
            \text{\lang{de}{Bestimmen Sie den Wert des folgenden Integrals:}
                  \lang{en}{Determine the value of the following integral:}\\
			$\int_{0}^{\sqrt{\frac{\pi}{2}}} (\var{f})dx=$\ansref.}	

		\begin{answer}
			\solution{sol}
		\end{answer}
		\explanation{\lang{de}{
    Eine Stammfunktion von $f(x)=\var{f}$ ist $F(x)=\var{ff}$ (substituiere $g(x)=\var{g}$).\\
		Damit ist $\int_{0}^{\sqrt{\frac{\pi}{2}}} (\var{f})dx=F(\sqrt{\frac{\pi}{2}})-F(0)=\var{ffupper}-\var{fflower}=\var{sol}$.
    }
    \lang{en}{
    The function $F(x)=\var{ff}$ is an antifderivative of $f(x)=\var{f}$ (substitute 
    $g(x)=\var{g}$).\\
    Hence $\int_{0}^{\sqrt{\frac{\pi}{2}}} (\var{f})dx=F(\sqrt{\frac{\pi}{2}})-F(0)=\var{ffupper}-\var{fflower}=\var{sol}$.
    }}

	\end{quickcheck}
\end{quickcheckcontainer}


\end{content}