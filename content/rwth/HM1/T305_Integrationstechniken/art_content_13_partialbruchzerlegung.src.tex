%$Id:  $
\documentclass{mumie.article}
\begin{metainfo}
  \name{
    \lang{de}{Partialbruchzerlegung}
    \lang{en}{Partial fraction decomposition}
  }
  \begin{description} 
 This work is licensed under the Creative Commons License Attribution 4.0 International (CC-BY 4.0)   
 https://creativecommons.org/licenses/by/4.0/legalcode 

    \lang{de}{Beschreibung}
    \lang{en}{Description}
  \end{description}
  \begin{components}
    \component{generic_image}{content/rwth/HM1/images/g_tkz_T305_Chart.meta.xml}{T305_Chart}
    \component{generic_image}{content/rwth/HM1/images/g_img_T305_summary.meta.xml}{image}
    \component{generic_image}{content/rwth/HM1/images/g_img_00_Videobutton_schwarz.meta.xml}{00_Videobutton_schwarz}
  \end{components}
  \begin{links}
    \link{generic_article}{content/rwth/HM1/T305_Integrationstechniken/g_art_content_13_partialbruchzerlegung.meta.xml}{content_13_partialbruchzerlegung}
    \link{generic_article}{content/rwth/HM1/T304_Integrierbarkeit/g_art_content_09_integrierbare_funktionen.meta.xml}{stammfunktion}
    \link{generic_article}{content/rwth/HM1/T103_Polynomfunktionen/g_art_content_10_polynomdivision.meta.xml}{polydiv}
    \link{generic_article}{content/rwth/HM1/T305_Integrationstechniken/g_art_content_12_substitutionsregel.meta.xml}{substitution}
  \end{links}
  \creategeneric
\end{metainfo}
\begin{content}
\usepackage{mumie.ombplus}
\ombchapter{4}
\ombarticle{3}

\title{\lang{de}{Partialbruchzerlegung}\lang{en}{Partial fraction decomposition}}
 
\begin{block}[annotation]
  
  
\end{block}
\begin{block}[annotation]
  Im Ticket-System: \href{http://team.mumie.net/issues/10040}{Ticket 10040}\\
\end{block}

\begin{block}[info-box]
\tableofcontents
\end{block}

\lang{de}{
In diesem Abschnitt geht es darum, Stammfunktionen zu rationalen Funktionen zu bestimmen, um
Integrale über solchen berechnen zu können. 
Es stellt sich heraus, dass es im Allgemeinen schwierig ist, von rationalen Funktionen
Stammfunktionen zu bestimmen.
Ein wichtiges Hilfsmittel ist dabei die
\notion{Partialbruchzerlegung}, bei der eine rationale Funktion in kleinere Bausteine bestimmter Form
zerlegt wird. Wir geben hier das allgemeine Verfahren an. Die Resultate decken alle theoretischen Fälle ab
und sind teilweise nicht einfach zu lesen. Die Beispiele decken aber einen Großteil der in der Anwendung auftretenden Fälle ab.
Wir beginnen mit den verschiedenen Bausteinen. Am Ende des Kapitels, in Abschnitt \ref{sec:summary},
geben wir eine Zusammenfassung.
}
\lang{en}{
In this section we study how to determine antiderivatives of rational functions in order to be able 
to integrate them. It turns out that in general, it is difficult to find antiderivatives of rational 
functions. \notion{Partially decomposing} a rational function into smaller expressions of 
a particular form can be a useful tool for this, and here we give the general method for doing so. 
The results cover every theoretical case and can be difficult to read. The examples cover most cases 
that appear in application. We begin with the individual components. At the end of the chapter, in 
section \ref{sec:summary}, we give a summary.
}

\section{\lang{de}{Stammfunktionen negativer Potenzen linearer Funktionen}
         \lang{en}{Antiderivatives of negative powers of linear functions}
         }\label{sec:partialbruch_int}

\lang{de}{
Für die Funktionen $f(x)=\frac{1}{x^n}$ mit fester natürlicher Zahl $n$ haben wir
\link{stammfunktion}{Stammfunktionen} kennengelernt. Diese waren gegeben durch
}
\lang{en}{
We already know antiderivatives for functions $f(x)=\frac{1}{x^n}$ where $n$ is some constant:
}
\begin{align}
  F(x) &=\ln(|x|),\quad \text{\lang{de}{falls }\lang{en}{if }} &f(x)=\frac{1}{x},\\
  F(x) &=\frac{1}{(1-n)x^{n-1}},\quad \text{\lang{de}{falls }\lang{en}{if }} &f(x)=\frac{1}{x^n}\,\text{\lang{de}{ mit }\lang{en}{ where }}n>1.
\end{align}

\lang{de}{
Mit Hilfe der \ref[substitution][Substitutionsregel für lineare Funktionen]{rule:lin-subst} lassen 
sich auch Stammfunktionen für negative Potenzen beliebiger linearer Funktionen $g(x)=mx+b$ angeben.
}
\lang{en}{
Using the \ref[substitution][substitution rule for linear functions]{rule:lin-subst} we can also 
find antiderivatives for negative powers of a linear function $g(x)=mx+b$.
}

\begin{rule}\label{rule:linearrational}
 \lang{de}{
 Es seien $n\in \N$ eine natürliche Zahl und $m,b\in \R$ reelle Zahlen mit $m\neq 0$. Weiter sei
 }
 \lang{en}{
 Let $n\in \N$ be a natural number and $m,b\in \R$ real numbers with $m\neq 0$. Furthermore, let
 }
 \[ h:\R\setminus \{-\frac{b}{m}\}\to \R, \ \ x\mapsto \frac{1}{(mx+b)^{n}}. \]
 \lang{de}{Dann ist eine Stammfunktion von $h$ gegeben durch}
 \lang{en}{Then the following function is an antiderivative of $h$:}
 \[ H(x)=\left\{ \begin{matrix} \frac{\ln{(|mx+b|)}}{m}, & \text{\lang{de}{falls }\lang{en}{if }}n=1,\\
 \frac{(mx+b)^{-n+1}}{(1-n)m},& \text{\lang{de}{falls }\lang{en}{if }}n>1.
 \end{matrix} \right. \]
 \end{rule}

 \begin{proof*}
 \lang{de}{
 Schreibt man $h(x)=f(g(x))$ mit äußerer Funktion $f(x)=x^{-n}$ und innerer Funktion $g(x)=mx+b$, so
 ist nach der \ref[substitution][linearen Substitutionsregel]{rule:lin-subst} die
 Stammfunktion genau die angegebene Funktion.
 }
 \lang{en}{
 If we write $h(x)=f(g(x))$ with outer function $f(x)=x^{-n}$ and inner function $g(x)=mx+b$, then 
 by the \ref[substitution][linear substitution rule]{rule:lin-subst}, the given function is an 
 antiderivative.
 }
 \end{proof*}

% \begin{remark}
% Zum Rechnen ist es oft übersichtlicher, mit Integralen mit variablen Grenzen $\alpha$ und $\beta$ zu rechnen. 
% Die Umformungen für obige Regel ist dann im Fall $n>1$ mit $g(x)=mx+b$ und $g'(x)=m$:
% \begin{eqnarray*}
%  \int_{\alpha}^{\beta} (mx+b)^{-n}\; dx &=& \int_{\alpha}^{\beta} g(x)^{-n}\; dx=\int_{\alpha}^{\beta} \frac{g'(x)}{m}g(x)^{-n}\; dx\\
%  &=&   \int_{g(\alpha)}^{g(\beta)} \frac{y^{-n}}{m}\; dy = \left[ \frac{y^{-n+1}}{(-n+1)m}\right]_{g(\alpha)}^{g(\beta)} \\
%  &=& \left[ \frac{(mx+b)^{-n+1}}{(-n+1)m}\right]_{\alpha}^{\beta}.
%  \end{eqnarray*}
% \end{remark}

 \section{\lang{de}{Stammfunktionen rationaler Funktionen mit quadratischen Nennern}
          \lang{en}{Antiderivatives of rational functions with quadratic denominators}
          }\label{sec:stammf-rat-fkt-quad}

\lang{de}{
Wir betrachten eine rationale Funktion $f(x)=\frac{p(x)}{q(x)}$, wobei $q(x)=ax^2+bx+c$ ein 
quadratisches Polynom ist. Ist der Grad von $p$ größer oder gleich dem Grad von $q$, so kann man 
zunächst eine \link{polydiv}{Polynomdivision} durchführen und erhält
}
\lang{en}{
Consider a rational function $f(x)=\frac{p(x)}{q(x)}$, where $q(x)=ax^2+bx+c$ is a quadratic 
polynomial. If the degree of $p$ is greater than or equal to the degree of $q$, then we can begin 
by performing \link{polydiv}{polynomial division} to obtain
}
 \[ f(x)=g(x)+\frac{r(x)}{q(x)} \]
\lang{de}{
mit Polynomfunktionen $g(x)$ und $r(x)$, wobei der Grad von $r$ echt kleiner als der von $q$ ist 
(oder sogar $r=0$ gilt). Eine Stammfunktion von $f$ erhält man dann als Summe von Stammfunktionen 
von $g$ und von $\frac{r}{q}$.
\\\\
Da wir Stammfunktionen von Polynomen schon behandelt haben, beschränken wir uns in diesem 
Paragraphen auf die Bestimmung einer Stammfunktion für eine rationale Funktion $\frac{r}{q}$ mit 
$\deg(r)<\deg(q)$. Da $q$ als quadratisch angenommen wurde, ist $r$ dann also konstant oder linear.
}
\lang{en}{
where $g(x)$ and $r(x)$ are polynomials, and the degree of $r$ is strictly smaller than the degree 
of $q$ (or even $r=0$). An antiderivative of $f$ can now be found as the sum of antiderivatives 
of $g$ and $\frac{r}{q}$.
\\\\
We already have a method of finding antiderivatives of polynomials, so we restrict ourselves in this 
section to determining antiderivatives of rational functions $\frac{r}{q}$ where $\deg(r)<\deg(q)$. 
As $q$ is supposed to be quadratic, $r$ must be constant or linear.
}


\begin{theorem}\label{thm:stammfkt_rationale}
\lang{de}{
Es sei $q(x)=ax^2+bx+c$ eine Polynomfunktion vom Grad $2$ mit $a,b,c\in \R$ und $a\neq 0$.\\ 
Weiter setzen wir 
}
\lang{en}{
Let $q(x)=ax^2+bx+c$ be a polynomial of degree $2$ with $a,b,c\in \R$ und $a\neq 0$.\\ 
Let 
}
$d:=\left(\frac{b}{2a}\right)^2-\frac{c}{a}$.
% $d:=-\frac{c}{a}$.
\lang{de}{Dann gelten:}
\lang{en}{Then:}
\begin{enumerate}
\item \lang{de}{Eine Stammfunktion von $\frac{2ax+b}{ax^2+bx+c}$ ist gegeben durch}
      \lang{en}{The function $\frac{2ax+b}{ax^2+bx+c}$ has an antiderivative}
      \[\ln(|ax^2+bx+c|). \]
\item \lang{de}{Für eine Stammfunktion von $\frac{1}{ax^2+bx+c}$ gibt es drei Fälle:}
      \lang{en}{There are three cases concerning the function $\frac{1}{ax^2+bx+c}$:}
\begin{itemize}
\item \lang{de}{
      Besitzt das Polynom $ax^2+bx+c$ keine reelle Nullstelle (genau in diesem Fall ist $d<0$), so 
      ist eine Stammfunktion gegeben durch
      }
      \lang{en}{
      If the polynomial $ax^2+bx+c$ does not have any real roots (which is precisely when $d<0$), 
      then the following function is an antiderivative:
      }
      \[ \frac{\arctan\left(\frac{x+\frac{b}{2a}}{\sqrt{-d}}\right)}{a\cdot \sqrt{-d}}.\]
\item \lang{de}{
      Besitzt das Polynom $ax^2+bx+c$ genau eine reelle Nullstelle (in diesem Fall ist $d=0$ und die 
      Nullstelle ist gegeben durch $x_0=-b/(2a)$), so ist $ax^2+bx+c=a(x-x_0)^2$ und eine 
      Stammfunktion von $\frac{1}{ax^2+bx+c}=\frac{1}{a(x-x_0)^2}$ ist gegeben durch
      }
      \lang{en}{
      If the polynomial $ax^2+bx+c$ has exactly one real root (which is precisely when $d=0$ and 
      in this case the root is $x_0=-b/(2a)$), then $ax^2+bx+c=a(x-x_0)^2$ and so the following 
      function is an antiderivative of $\frac{1}{ax^2+bx+c}=\frac{1}{a(x-x_0)^2}$:
      }
      \[ \frac{1}{-a(x-x_0)}.\]
\item \lang{de}{
      Besitzt das Polynom $ax^2+bx+c$ zwei verschiedene reelle Nullstellen $x_1$ und $x_2$ (genau in 
      diesem Fall ist $d>0$ und die Nullstellen sind gegeben durch $x_{1/2}=-b/(2a)\pm \sqrt{d}$), 
      so ist eine Stammfunktion von $\frac{1}{ax^2+bx+c}=\frac{1}{a(x-x_1)(x-x_2)}$ gegeben durch
      }
      \lang{en}{
      If the polynomial $ax^2+bx+c$ has two distinct roots $x_1$ and $x_2$ (which is precisely 
      when $d>0$ and in this case the roots are $x_{1/2}=-b/(2a)\pm \sqrt{d}$), then the following 
      function is an antiderivative of $\frac{1}{ax^2+bx+c}=\frac{1}{a(x-x_1)(x-x_2)}$:
      }
\[ \frac{1}{a(x_1-x_2)} \ln\left( |\frac{x-x_1}{x-x_2}|\right). \]
\end{itemize}
\end{enumerate}
\end{theorem}

\lang{de}{
Wichtiger als sich die Formeln zu merken, ist eigentlich die Herangehensweise zur Bestimmung der 
Stammfunktionen, die im Folgenden erklärt wird.
}
\lang{en}{
It is more important to learn the strategy for determining antiderivatives than it is to memorise 
formulas.
}

\begin{proof*}
\begin{enumerate}
\item \lang{de}{
      Bei der Funktion $f(x)=\frac{2ax+b}{ax^2+bx+c}$ steht im Zähler genau die Ableitung des 
      Nenners, d.\,h. $f(x)=\frac{q'(x)}{q(x)}=q'(x)\cdot q(x)^{-1}=q'(x)\cdot h(q(x))$ mit 
      $h(x)=x^{-1}$. Da $H(x)=\ln(|x|)$ eine Stammfunktion von $h(x)$ ist, ist nach der 
      \ref[substitution][Substitutionsregel]{thm:substitutionsregel} eine Stammfunktion von $f$ 
      gegeben durch $F(x)=H(q(x))=\ln(|ax^2+bx+c|)$.
      }
      \lang{en}{
      Consider the function $f(x)=\frac{2ax+b}{ax^2+bx+c}$. The numerator is precisely the 
      derivative of the denominator, that is, 
      $f(x)=\frac{q'(x)}{q(x)}=q'(x)\cdot q(x)^{-1}=q'(x)\cdot h(q(x))$ where 
      $h(x)=x^{-1}$. As $H(x)=\ln(|x|)$ is an antiderivative of $h(x)$, by the 
      \ref[substitution][substitution rule]{thm:substitutionsregel} the function 
      $F(x)=H(q(x))=\ln(|ax^2+bx+c|)$ is an antiderivative of $f$.
      }

\item
\begin{itemize}
\item \lang{de}{
      Um die Stammfunktionen für $\frac{1}{ax^2+bx+c}$ im ersten Fall zu bestimmen, erinnern wir 
      zunächst daran, dass $\arctan(x)$ eine Stammfunktion von $\frac{1}{x^2+1}$ ist.
      }
      \lang{en}{
      In order to determine the antiderivatives of $\frac{1}{ax^2+bx+c}$ in the first case, we 
      firstly recall that $\arctan(x)$ is an antiderivative of $\frac{1}{x^2+1}$.
      }
\begin{incremental}
\step \lang{de}{
      Die Stammfunktion von $f(x)=\frac{1}{ax^2+bx+c}$ ergibt sich daraus wieder durch die 
      Substitutionsregel und die Faktorregel:\\
      Mittels quadratischer Ergänzung und geschicktem Ausklammern erhält man nämlich
      }
      \lang{en}{
      An antiderivative of $f(x)=\frac{1}{ax^2+bx+c}$ can be found using the substitution 
      rule and the constant factor rule:\\
      By completing the square and carefully factoring, we obtain
      }
      \[ ax^2+bx+c =a\cdot \left( \left(x+\frac{b}{2a}\right)^2 +\frac{c}{a}- \left(\frac{b}{2a}\right)^2 \right)
= a\sqrt{-d}^2\cdot  \left(  \left( \frac{x+\frac{b}{2a}}{\sqrt{-d}}\right)^2 +1 \right), \]
\lang{de}{
      wobei $d=\left(\frac{b}{2a}\right)^2-\frac{c}{a}<0$ nach Annahme.\\
      Setzt man nun $g(x)=\frac{x+b/(2a)}{\sqrt{-d}}$ und daher $g'(x)=\frac{1}{\sqrt{-d}}$,
      so ist
      }
      \lang{en}{
      where $d=\left(\frac{b}{2a}\right)^2-\frac{c}{a}<0$ by the assumption.\\
      If we now set $g(x)=\frac{x+b/(2a)}{\sqrt{-d}}$ and thus $g'(x)=\frac{1}{\sqrt{-d}}$, we have
      }
      \[  f(x)=\frac{1}{ax^2+bx+c}=\frac{1}{a\sqrt{-d}^2}\cdot  \left(  \left( \frac{x+\frac{b}{2a}}{\sqrt{-d}}\right)^2 +1 \right)^{-1} 
= \frac{g'(x)}{a\sqrt{-d}}\cdot \frac{1}{g(x)^2+1}.\]
      \lang{de}{Mit der Substitutionsregel ist dann also}
      \lang{en}{By the substitution rule,}
      \[ F(x)=\frac{1}{a\sqrt{-d}} \cdot \arctan (g(x)) \ = \ \frac{\arctan\left(\frac{x+\frac{b}{2a}}{\sqrt{-d}}\right)}{a\cdot \sqrt{-d}}\]
      \lang{de}{eine Stammfunktion von $f(x)$.}
      \lang{en}{is an antiderivative of $f(x)$.}
\end{incremental}
\item \lang{de}{
      Der zweite Fall, in welchem $f(x)=\frac{1}{ax^2+bx+c}=\frac{1}{a(x-x_0)^2}$ gilt, ist 
      ein Spezialfall von \ref[content_13_partialbruchzerlegung][obiger Regel]{rule:linearrational}.
      }
      \lang{en}{
      The second case $f(x)=\frac{1}{ax^2+bx+c}=\frac{1}{a(x-x_0)^2}$ is a special case of the 
      \ref[content_13_partialbruchzerlegung][above rule]{rule:linearrational}.
      }
\item \lang{de}{
      Der dritte Fall ergibt sich aus der Partialbruchzerlegung, die im nächsten Paragraphen 
      behandelt wird; siehe unten stehendes \lref{ex:deg2}{Beispiel}.
      }
      \lang{en}{
      The third case derives from the partial fraction decomposition discussed below; see the 
      \lref{ex:deg2}{example} below.
      }
\end{itemize}
\end{enumerate}
\end{proof*}


\begin{remark}
\begin{enumerate}
\item \lang{de}{
      Um eine Stammfunktion von $\frac{rx+s}{ax^2+bx+c}$ zu bestimmen (mit $r \neq 0$),
      zerlegt man die rationale Funktion in eine Summe zweier rationaler Funktionen mit Nenner
      $ax^2+bx+c$, so dass der Zähler des einen Summanden ein Vielfaches von $2ax+b$ und der Zähler 
      des anderen Summanden konstant ist. Für diese zwei Typen lässt sich nämlich der obige Satz 
      anwenden.\\
      Das heißt, es lassen sich stets $\alpha, \beta \in \R$ berechnen, sodass
      }
      \lang{en}{
      In order to determine an antiderivative of $\frac{rx+s}{ax^2+bx+c}$ (where $r \neq 0$), 
      we decompose the rational function into the sum of two rational functions with denominator 
      $ax^2+bx+c$, such that the numerator of one of these is a multiple of $2ax+b$ and the 
      numerator of the other is constant. The above theorem can then be applied to each summand.\\
      We can always find $\alpha, \beta \in \R$ such that
      }
\[
\frac{rx+s}{ax^2+bx+c}= \alpha \cdot  \frac{2ax+b}{ax^2+bx+c} + \beta \cdot \frac{1}{ax^2+bx+c}.
\]

\begin{incremental}
\step \lang{de}{Eine solche Zerlegung ist stets möglich, da}
      \lang{en}{Such a decomposition is always possible, as}
\begin{align*}
\frac{rx+s}{ax^2+bx+c} &= \frac{r}{2a}\frac{2ax+\frac{2as}{r}+b-b}{ax^2+bx+c}\\
&=\frac{r}{2a}\left(\frac{2ax+b}{ax^2+bx+c}\right) + \frac{r}{2a}\left(\frac{\frac{2as}{r}-b}{ax^2+bx+c}\right).
\end{align*}
\end{incremental}

\item \lang{de}{
      Als Bausteine in der Partialbruchzerlegung treten die Funktionen $\frac{rx+s}{ax^2+bx+c}$ nur 
      in dem Fall auf, dass $ax^2+bx+c$ keine reelle Nullstelle hat. In den anderen Fällen wäre ein 
      solcher Bruch schon weiter zerlegt.
      }
      \lang{en}{
      When decomposing partial fractions, functions of the form $\frac{rx+s}{ax^2+bx+c}$ only appear 
      in the case where $ax^2+bx+c$ does not have real roots. In every other case, such a fraction 
      could be further decomposed.
      }
\end{enumerate}
\end{remark}

\begin{example}
\lang{de}{
Um eine Stammfunktion für die rationale Funktion $R(x)=\frac{x+2}{x^2-x+3}$ zu bestimmen, zerlegen 
wir den Zähler als Summe $x+2=\alpha (2x-1)+\beta$ mit reellen Zahlen $\alpha$ und $\beta$, da 
$2x-1$ genau die Ableitung von $x^2-x+3$ ist.\\
Die Zahlen $\alpha$ und $\beta$ bestimmen wir mit Koeffizientenvergleich:
}
\lang{en}{
In order to determine an antiderivative of the rational function $R(x)=\frac{x+2}{x^2-x+3}$, we 
decompose the numerator into the sum $x+2=\alpha (2x-1)+\beta$ for some real numbers $\alpha$ and 
$\beta$, as $2x-1$ is the derivative of $x^2-x+3$.\\
}
\begin{align*}
& \textcolor{#0066CC}{1} \cdot x+\textcolor{#CC6600}{2} &=\alpha (2x-1)+\beta= \textcolor{#0066CC}{2\alpha} x+ (\textcolor{#CC6600}{\beta-\alpha}) \\
\Leftrightarrow  & \ \textcolor{#0066CC}{1=2\alpha} &\text{\lang{de}{ und }\lang{en}{ and }} \ \textcolor{#CC6600}{2=\beta-\alpha} \\
\Leftrightarrow  & \ \alpha=\frac{1}{2} &\text{\lang{de}{ und }\lang{en}{ and }} \ \beta= \frac{5}{2}\, .
\end{align*}
\lang{de}{Also ist}
\lang{en}{Hence}
\[ \frac{x+2}{x^2-x+3} = \frac{1}{2}\cdot \frac{2x-1}{x^2-x+3}+  \frac{5}{2}\cdot \frac{1}{x^2-x+3}. \]
\lang{de}{
Eine Stammfunktion für $\frac{2x-1}{x^2-x+3}$ ist $\ln{|x^2-x+3|}$.
Weiter erhält man für das $d$ aus obiger Formel $d=(1/2)^2-3=-11/4<0$, weshalb
$x^2-x+3$ keine reelle Nullstelle hat. Eine Stammfunktion
für $\frac{1}{x^2-x+3}$ ist daher gegeben durch
}
\lang{en}{
The function $\ln{|x^2-x+3|}$ is an antiderivative of $\frac{2x-1}{x^2-x+3}$. 
For $d$ we obtain the formula $d=(1/2)^2-3=-11/4<0$, by which we know that 
$x^2-x+3$ does not have any real roots. The following function is therefore an antiderivative of 
$\frac{1}{x^2-x+3}$:
}
\[ \sqrt{\frac{4}{11}}\cdot\arctan\left( \frac{x-\frac{1}{2}}{\sqrt{\frac{11}{4}}}\right)=\frac{2}{\sqrt{11}}\cdot\arctan\left( \frac{2x-1}{\sqrt{11}}\right). \]
\lang{de}{Insgesamt erhält man also als Stammfunktion von $R(x)$:}
\lang{en}{To conclude, $R(x)$ has the following antiderivative:}
\[   \frac{1}{2}\cdot \ln{|x^2-x+3|} + \frac{5}{\sqrt{11}} \arctan\left( \frac{2x-1}{\sqrt{11}}\right). \]
\end{example}


\begin{remark}
\lang{de}{Ein weiterer Baustein für die Partialbruchzerlegung sind rationale Funktionen der Form}
\lang{en}{Another component of partial fraction decomposition are the rational functions of the form}
\[ r(x)=\frac{p(x)}{(x^2+bx+c)^n}, \]
\lang{de}{
wobei $p(x)$ ein konstantes oder lineares Polynom ist, der Nenner keine reellen Nullstellen hat und 
$n>1$ ist. Wie in der vorherigen Bemerkung kann man $p(x)$ dann noch zerlegen in 
$\alpha (2x+b)+\beta$ mit reellen Zahlen $\alpha$ und $\beta$, wodurch man sich auf die zwei Fälle
}
\lang{en}{
where $p(x)$ is a constant or linear polynomial, the denominator has no real roots and $n>1$. As 
in the earlier remark, we can decompose $p(x)$ into $\alpha (2x+b)+\beta$ for some real numbers 
$\alpha$ and $\beta$. We arrive at one of two cases:
}
\[ \frac{2x+b}{(x^2+bx+c)^n}\quad\text{\lang{de}{und}\lang{en}{and}}\quad \frac{1}{(x^2+bx+c)^n} \]
\lang{de}{
zurückziehen kann. 
Für das erstere erhält man mit der Substitutionsregel als Stammfunktion direkt
}
\lang{en}{
In the first case, the substitution rule immediately yields an antiderivative,
}
\[ \frac{1}{(1-n)(x^2+bx+c)^{n-1}}. \]
\lang{de}{
Die Bestimmung der Stammfunktion von $ \frac{1}{(x^2+bx+c)^n}$ ist deutlich schwieriger und 
behandeln wir in diesem Kurs nicht.
}
\lang{en}{
Determining an antiderivative of $ \frac{1}{(x^2+bx+c)^n}$ is significantly more difficult, and is 
not considered in this course.
}
% . Das Prinzip beruht darauf, 
% zunächst die Substitution
% \[ t=\arctan \left(\frac{x+b/(2)}{\sqrt{-d}}\right) \]
% durchzuführen und das resultierende unbestimmte Integral (welches dann $\cos(t)$ und/oder $\sin(t)$ enthält) 
% geschickt partiell zu integrieren.
\end{remark}


\section{\lang{de}{Partialbruchzerlegung}
         \lang{en}{Partial fraction decomposition}}\label{sec:partbruchzerl}

\lang{de}{
Jedes reelle Polynom $q(x)$ hat eine Zerlegung in ein Produkt linearer und quadratischer Polynome,
wobei die quadratischen Polynome keine reellen Nullstellen haben, d.\,h.
}
\lang{en}{
Every real polynomial $q(x)$ has a decomposition into a product of linear and quadratic polynomials, 
where the quadratic polynomials do not have any real roots, that is,
}
\[ q(x) = (x-a_1)^{k_1}\cdots (x-a_s)^{k_s}\cdot(x^2+p_1\,x+q_1)^{l_1}\cdots(x^2+p_t\,x+q_t)^{l_t}\lang{de}{,}
\]
\lang{de}{wobei }
\lang{en}{where }
$k_1+\ldots+k_s+2l_1+\ldots+2l_t = n = \text{\lang{en}{degree}\lang{de}{Grad}}(q(x))$, 
$a_j,\;p_j,\;q_j\in\R,\; p_j^2<4q_j\,$
\lang{de}{
und die Nullstellen $a_j$ von $q$ paarweise verschieden sind (dieses Resultat nennt man den Fundamentalsatz der Algebra).
\\\\
Eine solche Produkt-Zerlegung wird benutzt, um rationale Funktionen $ r(x) = \frac{p(x)}{q(x)}$ in 
eine Summe der oben genannten Bausteine zu zerlegen und damit Stammfunktionen bzw. Integrale zu 
bestimmen.
}
\lang{en}{
and the roots $a_j$ of $q$ are pairwise different (this result is called the fundamental theorem of 
algebra).
\\\\
Such a decomposition into a product is used to decompose a rational function of the form 
$ r(x) = \frac{p(x)}{q(x)}$ into a sum of components of the forms mentioned above, with the aim of 
finding antiderivatives and computing intervals.
}

\begin{theorem}[\lang{de}{Partialbruchzerlegung}
                \lang{en}{Partial fraction decomposition}]\label{thm:pzb}
\lang{de}{Für die rationale Funktion}
\lang{en}{Given the rational function}
\[ r(x) = \frac{p(x)}{q(x)}\qquad
\text{\lang{de}{mit Grad$(p(x))<$ Grad$(q(x))=n$}\lang{en}{with degree$(p(x))<$ degree$(q(x))=n$,}}
\]
\lang{de}{
sei die Produkt-Zerlegung des Polynoms $q(x)$ im Nenner gegeben durch
}
\lang{en}{
let the decomposition of the polynomial $q(x)$ into a product of linear factors and irreducible 
quadratic factors be given by
}
\[ q(x) = (x-a_1)^{k_1}\cdots (x-a_s)^{k_s}\cdot(x^2+p_1\,x+q_1)^{l_1}\cdots(x^2+p_t\,x+q_t)^{l_t}\lang{de}{,}
\]
\lang{de}{wobei }
\lang{en}{where }
$k_1+\ldots+k_s+2l_1+\ldots+2l_t = n = \text{\lang{en}{degree}\lang{de}{Grad}}(q(x))$.
\lang{de}{Dann gibt es eindeutige reelle Zahlen}
\lang{en}{Then there exist unique real numbers}
\[
A_{1,1},\ldots,A_{1,k_1},\ldots,A_{s,1},\ldots,A_{s,k_s},
\,B_{1,1},C_{1,1},\ldots ,B_{t,l_t},C_{t,l_t}\lang{de}{,}
\]
\lang{de}{so dass}
\lang{en}{such that}
\begin{eqnarray*}
r(x)&=& \frac{A_{1,1}}{(x-a_1)} +\ldots + \frac{A_{1,k_1}}{(x-a_1)^{k_1}} +\ldots+
	\frac{A_{s,1}}{(x-a_s)} +\ldots + \frac{A_{s,k_s}}{(x-a_s)^{k_s}} \\
	&&+\frac{B_{1,1}x+C_{1,1}}{(x^2+p_1\,x+q_1)} +\ldots + \frac{B_{1,l_1}x+C_{1,l_1}}{(x^2+p_1\,x+q_1)^{l_1}}
	+\ldots + \frac{B_{t,l_t}x+C_{t,l_t}}{(x^2+p_t\,x+q_t)^{l_t}} .
\end{eqnarray*}
\lang{de}{
Um die Zahlen $A_{1,1},\ldots,A_{s,k_s},\,B_{1,1},C_{1,1},\ldots ,B_{t,l_t},C_{t,l_t}$ zu berechnen, 
bringt man dies auf den Hauptnenner $q(x)$ und vergleicht die Koeffizienten der Zählerpolynome links 
und rechts. Dies führt zu einem linearen Gleichungssystem mit $n=\deg(q)$ Gleichungen für die $n$ 
Variablen $A_{1,1},\ldots,A_{s,k_s},\,B_{1,1},C_{1,1},\ldots ,B_{t,l_t},C_{t,l_t}$, welches man 
lösen muss (siehe Beispiele).
}
\lang{en}{
To calculate the numbers $A_{1,1},\ldots,A_{s,k_s},\,B_{1,1},C_{1,1},\ldots ,B_{t,l_t},C_{t,l_t}$, 
we add the terms on the RHS (for which we use the common denominator $q(x)$), and compare the 
coefficients on both sides of the equation. This leads to a uniquely solvable system of $n=\deg(q)$ 
linear equations in the variables 
$A_{1,1},\ldots,A_{s,k_s},\,B_{1,1},C_{1,1},\ldots ,B_{t,l_t},C_{t,l_t}$ (see the examples).}
\end{theorem}

\lang{ded}{
Das Resultat deckt alle theoretischen Fälle ab.
Die Faktorisierung des Nennerpolynoms ist häufig schon ein sehr schweres Problem, sodass
in der Praxis nur Polynome kleinen Grades berechnet werden können.
Die auftretenden Summanden bleiben überschaubar.
Wichtig ist, dass Sie die bisher besprochenen Bausteine wiedererkennen.
Wir berechnen nun explizite Beispiele einer Partialbruchzerlegung.
}
\lang{en}{
This result covers every case in theory. 
Factorising the polynomial in the denominators is often a difficult problem on its own, so in 
practice this is only done for polynomials of small degree. 
Furthermore, this way, the summands remain managable. 
It is important to recognise the different forms the summands come in. 
We now give some explicit examples of partial fraction decomposition.
}

\begin{example}\label{ex:deg2}
\begin{tabs*}
\tab{$\frac{1}{x^2-2x-3}$}
\lang{de}{
Wir bestimmen die Partialbruchzerlegung von $R(x)=\frac{1}{x^2-2x-3}$.\\
Die Nullstellen des Nenners sind $x_1=-1$ und $x_2=3$, weshalb
}
\lang{en}{
We will decompose $R(x)=\frac{1}{x^2-2x-3}$ into partial fractions.\\
The denominator has roots $x_1=-1$ and $x_2=3$, so
}
\[ R(x)=\frac{1}{(x+1)(x-3)}. \]
\lang{de}{Für die Partialbruchzerlegung müssen wir also reelle Zahlen $A_1$ und $A_2$ finden mit}
\lang{en}{We need to find real numbers $A_1$ and $A_2$ such that}
\[   \frac{1}{(x+1)(x-3)} =\frac{A_{1}}{x+1} + \frac{A_{2}}{x-3}=
\frac{A_{1}\cdot (x-3)+A_{2}\cdot (x+1) }{(x+1)(x-3)}. \]
\lang{de}{Wir vergleichen die Zählerpolynome}
\lang{en}{We compare the numerators}
\[
\textcolor{#0066CC}{0} \cdot x + \textcolor{#CC6600}{1} = (\textcolor{#0066CC}{A_1+A_2})\cdot x 
+(\textcolor{#CC6600}{-3A_1 + A_2}).
\]
\lang{de}{Vergleich der Koeffizienten ergibt}
\lang{en}{Equating coefficients yields}
\[  \left[  \begin{mtable}[\cellaligns{llr}] \textcolor{#0066CC}{0} &=& \textcolor{#0066CC}{A_{1}+A_{2}} \\
\textcolor{#CC6600}{1} &=& \textcolor{#CC6600}{-3A_{1} +A_{2}}
\end{mtable} \right]. \]
\lang{de}{
Löst man das Gleichungssystem auf, erhält man $A_{1}=-\frac{1}{4}$ und $A_{2}=\frac{1}{4}$.
\\\\
Die Partialbruchzerlegung von $R(x)=\frac{1}{x^2-2x-3}$ ist also
}
\lang{en}{
Solving the system of equations gives us $A_{1}=-\frac{1}{4}$ and $A_{2}=\frac{1}{4}$.
\\\\
Hence $R(x)=\frac{1}{x^2-2x-3}$ decomposes into 
}
\[  \frac{1}{(x+1)(x-3)} = -\frac{1}{4}\cdot \frac{1}{x+1} + \frac{1}{4}\cdot \frac{1}{x-3}. \]

\tab{$\frac{1}{(x-x_1)(x-x_2)}$} 
\lang{de}{
Betrachten wir allgemeiner eine rationale Funktion $R(x)=\frac{1}{(x-x_1)(x-x_2)}$ mit zwei 
verschiedenen reellen Zahlen $x_1$ und $x_2$, so berechnet man die Partialbruchzerlegung ganz 
entsprechend:
}
\lang{en}{
Consider more generally a rational function $R(x)=\frac{1}{(x-x_1)(x-x_2)}$ where $x_1$ and $x_2$ 
are distinct real numbers. The partial fraction decomposition is computed in the same way:
}
\[  \frac{1}{(x-x_1)(x-x_2)} =\frac{A_{1}}{x-x_1} + \frac{A_{2}}{x-x_2}=
\frac{A_{1}\cdot (x-x_2)+A_{2}\cdot (x-x_1) }{(x-x_1)(x-x_2)}. \]
\lang{de}{Vergleich der Koeffizienten im Zähler ergibt}
\lang{en}{Equating coefficients in the numerator yields}
\[  \left[  \begin{mtable}[\cellaligns{llr}] 0 &=& A_{1}+A_{2} \\
1 &=& -x_2A_{1} -x_1A_{2}
\end{mtable} \right]. \]
\lang{de}{
Die erste Zeile erhalten wir durch Vergleich der Koeffizienten von $x$ und die zweite Zeile durch 
Vergleich der konstanten Terme.
\\\\
Löst man das Gleichungssystem auf, erhält man $A_{1}=\frac{1}{x_1-x_2}$ und 
$A_{2}=-\frac{1}{x_1-x_2}$.
\\\\
Die Partialbruchzerlegung von $R(x)=\frac{1}{(x-x_1)(x-x_2)}$ ist also
}
\lang{en}{
The first row is obtained by comparing the coefficients of $x$, and the second by comparing the 
constant terms.
\\\\
Solving the system of equations yields $A_{1}=\frac{1}{x_1-x_2}$ and $A_{2}=-\frac{1}{x_1-x_2}$.
\\\\
Hence $R(x)=\frac{1}{(x-x_1)(x-x_2)}$ decomposes into
}
\[  \frac{1}{(x-x_1)(x-x_2)} = \frac{1}{x_1-x_2}\cdot \frac{1}{x-x_1} - \frac{1}{x_1-x_2}\cdot \frac{1}{x-x_2}. \]
\end{tabs*}
\end{example}

\lang{de}{
Beispiele zu den verschiedenen, auftretenden Fällen bei einer Partialbruchzerlegung sind auch in den folgenden Videos zu finden:
\floatright{
\href{https://www.hm-kompakt.de/video?watch=122}{\image[75]{00_Videobutton_schwarz}}
\href{https://www.hm-kompakt.de/video?watch=123}{\image[75]{00_Videobutton_schwarz}}
\href{https://www.hm-kompakt.de/video?watch=124}{\image[75]{00_Videobutton_schwarz}}}\\
}
\lang{en}{}

\begin{quickcheck}
    \begin{variables}
        \randint[Z]{a}{-2}{2}
        \randint[Z]{b}{-3}{3}
        \randint[Z]{c}{-5}{5}
        \function[normalize]{f}{x^2+b*x+c}
        \function[normalize]{n}{(x-a)^2}
        \function[normalize]{m}{x-a}
    \end{variables}
    \text{\lang{de}{
    Wir wollen eine Partialbruchzerlegung von $\frac{\var{f}}{\var{n} (x^2+x+1)}$
    durchführen. Welcher Ansatz für den zu berechnenden Koeffizientenvergleich ist zu wählen?
    }
    \lang{en}{
    Suppose we wish to decompose $\frac{\var{f}}{\var{n} (x^2+x+1)}$ into partial fractions using 
    the above method. What form of expression do we choose with the aim of comparing coefficients?
    }}
    %\permuteChoices{1}{4}
    \begin{choices}{unique}
        \begin{choice}
            \text{$\frac{a}{\var{n}}+\frac{b}{\var{m}}+\frac{c}{(x^2+x+1)}$}
            \solution{false}
        \end{choice}
        \begin{choice}
            \text{$\frac{a}{\var{n}}+\frac{b}{\var{m}}+\frac{cx+d}{(x^2+x+1)}$}
            \solution{true}
        \end{choice}
        \begin{choice}
            \text{$\frac{ax+b}{\var{n}}+\frac{c}{\var{m}}+\frac{d}{(x^2+x+1)}$}
            \solution{false}
        \end{choice}
        \begin{choice}
            \text{$\frac{a}{\var{n}}+\frac{bx+c}{(x^2+x+1)}+ \frac{d}{2x+1}$}
            \solution{false}
        \end{choice}
    \end{choices}
\end{quickcheck}

\begin{remark}
\begin{itemize}
\item \lang{de}{
      Wir haben oben angenommen, dass $q(x)$ von der Form $q(x) = x^n + $ Terme niedrigerer Ordnung 
      ist. Wenn der Nenner von der Form $a_n\,x^n + \ldots$ mit $a_n\neq 0$ ist, dann teile zuerst 
      Zähler und Nenner durch $a_n$.
      }
      \lang{en}{
      We have assumed above that $q(x)$ is of the form $q(x) = x^n + $ lower order terms. If instead 
      the denominator is of the form $a_n\,x^n + \ldots$ with $a_n\neq 0$, then we can simply divide 
      the numerator and denominator by $a_n$ to satisfy this assumption.
      }
\item \lang{de}{
      Wenn bei der rationalen Funktion $R(x)= P(x)/q(x)$ der Grad des Polynoms im Zähler nicht 
      kleiner als im Nenner ist, also Grad$(P(x))\geq $ Grad$(q(x))$, dann wende zunächst die 
      \link{polydiv}{Polynomdivision} an, um ein Polynom $P_1(x)$ so abzuspalten, dass
      }
      \lang{en}{
      If in the rational function $R(x)= P(x)/q(x)$ the degree of the polynomial in the numerator is 
      not less than the degree of the polynomial in the denominator, that is, if 
      $\text{degree}(P(x)) \geq \text{degree}(q(x))$, then we begin by applying 
      \link{polydiv}{polynomial division} to split off a polynomial $P_1(x)$ such that
      }
      \[ R(x)= P_1(x) + \frac{p(x)}{q(x)}, \quad \text{\lang{de}{Grad}\lang{en}{degree}}(p(x)) <
      \text{\lang{de}{Grad}\lang{en}{degree}}(q(x)).
      \]
\end{itemize}
\end{remark}

\section{\lang{de}{Zusammenfassung: Stammfunktionen von rationalen Funktionen}
         \lang{en}{Overview: antiderivatives of rational functions}}\label{sec:summary}

\lang{de}{Wir fassen die Resultate und das Vorgehen in einem Schaubild zusammen:}
\lang{en}{We summarise the results and methods in the following chart:}
\begin{center}
\image{T305_Chart}
\end{center}

\lang{de}{
Die wichtigsten Fälle, die bei der Bestimmung einer Stammfunktion oder eines Integrals einer 
gebrochen-rationalen Funktion auftreten, werden auch in den beiden folgenden Videos behandelt und 
erklärt:
  \floatright{\href{https://www.hm-kompakt.de/video?watch=632}{\image[75]{00_Videobutton_schwarz}}
  \href{https://www.hm-kompakt.de/video?watch=633}{\image[75]{00_Videobutton_schwarz}}}\\
}
\lang{en}{}  
  


\begin{example}
\lang{de}{
Um eine Stammfunktion der Funktion $R(x)=\frac{1}{(x-x_1)(x-x_2)}$ mit verschiedenen reellen Zahlen 
$x_1$ und $x_2$ zu berechnen, bestimmen wir zunächst die Partialbruchzerlegung (siehe 
\lref{ex:deg2}{obiges Beispiel}):
}
\lang{en}{
In order to determine an antiderivative of a function $R(x)=\frac{1}{(x-x_1)(x-x_2)}$ with distinct 
real numbers $x_1$ and $x_2$, we firstly decompose it into partial fractions (see 
\lref{ex:deg2}{above example}):
}
\[  \frac{1}{(x-x_1)(x-x_2)} = \frac{1}{x_1-x_2}\cdot \frac{1}{x-x_1} - \frac{1}{x_1-x_2}\cdot \frac{1}{x-x_2}. \]
\lang{de}{
Stammfunktionen von $\frac{1}{x-x_1}$ und $ \frac{1}{x-x_2}$ sind gegeben durch
$\ln(|x-x_1|)$ bzw. $\ln(|x-x_2|)$ und daher ist eine gesuchte Stammfunktion
}
\lang{en}{
The functions $\ln(|x-x_1|)$ and $\ln(|x-x_2|)$ are antiderivatives of $\frac{1}{x-x_1}$ and 
$ \frac{1}{x-x_2}$ respectively, so an antiderivative of $R(x)$ is
}
\begin{eqnarray*} 
\frac{1}{x_1-x_2}\cdot  \ln({|x-x_1|}) - \frac{1}{x_1-x_2}\cdot \ln({|x-x_2|})
&=& \frac{1}{x_1-x_2}\cdot \left(  \ln({|x-x_1|}) - \ln({|x-x_2|})\right) \\
&=& \frac{1}{x_1-x_2}\cdot \ln  \left( \frac{|x-x_1|}{|x-x_2|}\right). \end{eqnarray*}
\end{example}



\begin{example}
\begin{tabs*}[\initialtab{0}]
\tab{$\frac{x^3+x^2-3x+7}{x^2+x-2}$}
\begin{enumerate}
\item \lang{de}{Wir bestimmen eine Stammfunktion von $R(x)=\frac{x^3+x^2-3x+7}{x^2+x-2}$.}
      \lang{en}{Suppose we wish to find an antiderivative of $R(x)=\frac{x^3+x^2-3x+7}{x^2+x-2}$.}
\begin{itemize}
\item \lang{de}{
      Bei der rationalen Funktion $R(x)$ ist der Grad des Polynoms im Zähler nicht kleiner als im 
      Nenner. Deshalb müssen wir zunächst die \link{polydiv}{Polynomdivision} anwenden, um den 
      polynomialen Anteil $\,P_1(x)$ von $R$ abzuspalten:
      }
      \lang{en}{
      The degree of the polynomial in the numerator of $R(x)$ is not smaller than that in the 
      denominator. Hence we begin via \link{polydiv}{polynomial division} to split a polynomial 
      $\,P_1(x)$ from $R$:
      }
      \[ R(x) = \frac{x^3+x^2-3x+7}{x^2+x-2} = x + \frac{-x+7}{x^2+x-2}\eqcolon P_1(x)+r(x).\]
      \lang{de}{
      Der polynomiale Anteil ist $\,P_1(x)=x\,$ und der Rest $\,r(x)\,$ erfüllt 
      \nowrap{Grad$(-x+7)=1 < $ Grad$(x^2+x-2)=2$.}
      }
      \lang{en}{
      The polynomial part is $P_1(x)=x\,$ and the remainder $\,r(x)\,$ satisfies 
      \nowrap{degree$(-x+7)=1 < $ degree$(x^2+x-2)=2$.}
      }
\item \lang{de}{
      Als zweiten Schritt berechnen wir die Nullstellen des Nenners, welche $x_1=1$ und $x_2=-2$ 
      sind. Also ist
      }
      \lang{en}{
      The second step is to calculate the roots of the denominator, which turn out to be $x_1=1$ and 
      $x_2=-2$. Hence
      }
      \[ x^2+x-2 = (x-1)\cdot (x+2).\]
\item \lang{de}{Danach berechnen wir die Partialbruchzerlegung von $\,r(x)$.}
      \lang{en}{Next we compute the decomposition of $\,r(x)$ into partial fractions.}\\
\begin{align*} &&r(x) &=& \frac{-x+7}{(x-1)\cdot (x+2)} = \frac{A_1}{x-1}+\frac{A_2}{x+2}\\
	&&&=& \frac{A_1(x+2)+A_2(x-1)}{(x-1)\cdot (x+2)} = \frac{(A_1+A_2)\,x+(2A_1-A_2)}{(x-1)\cdot (x+2)}\\
	&\Longleftrightarrow&&& A_1+A_2 = -1,\qquad 2A_1-A_2 = 7\\
	&\Longleftrightarrow&&& A_1=2,\qquad A_2=-3.\\
\end{align*}
\lang{de}{Somit $\qquad r(x) =\frac{-x+7}{(x-1)\cdot (x+2)} = \frac{2}{x-1}-\frac{3}{x+2}$.}
\lang{en}{Hence $\qquad r(x) =\frac{-x+7}{(x-1)\cdot (x+2)} = \frac{2}{x-1}-\frac{3}{x+2}$.}
\item \lang{de}{
      Da $R(x)=x + \frac{2}{x-1} - \frac{3}{x+2}$ gilt, ist eine Stammfunktion gegeben durch
      }
      \lang{en}{
      As $R(x)=x + \frac{2}{x-1} - \frac{3}{x+2}$, we have an antiderivative
      }
      \[  \frac{x^2}{2}+2\,\ln{|x-1|}-3\,\ln{|x+2|}. \]
\end{itemize}
\item \lang{de}{Wir wollen nun das bestimmte Integral}
      \lang{en}{Now suppose we wish to compute the definite integral}
      \[ \int_{-1}^0\,R(x)\,dx=\int_{-1}^0\,\frac{x^3+x^2-3x+7}{x^2+x-2}\; dx\lang{de}{}\lang{en}{.}\]
      \lang{de}{
      berechnen. Zu beachten ist, dass das Integral definiert ist, da keine Nullstelle des Nenners 
      im Integrationsintervall $\,[-1;\,0]\,$ liegt.
      }
      \lang{en}{
      We remark that the integral is defined, the denominator has no roots in the 
      interval $\,[-1;\,0]\,$.
      }
      \begin{align*} \int_{-1}^0\,R(x)\,dx&=&\left[\frac{x^2}{2}+2\,\ln{|x-1|}-3\,\ln{|x+2|}\right]_{-1}^0\\
    	&=& -\frac{1}{2}+2\left(\ln(1)-\ln(2)\,\right)-3(\ln(2)-\ln(1)\,)\\
    	&=& -\frac{1}{2} -5\ln(2)=-\frac{1}{2} -\ln(32).
      \end{align*}
\end{enumerate}
\tab{$\frac{2x^3-2x^2+12x-3}{(x^2-x+3)^2}$}
\begin{enumerate}
\item \lang{de}{
      Wir bestimmen eine Stammfunktion von $R(x)=\frac{2x^3-2x^2+12x-3}{(x^2-x+3)^2}$.
      }
      \lang{en}{
      Suppose we wish to find an antiderivative of $R(x)=\frac{2x^3-2x^2+12x-3}{(x^2-x+3)^2}$.
      }
\begin{itemize}
  \item \lang{de}{
        Zunächst ist der Grad des Polynoms im Zähler kleiner als im Nenner: 
        \nowrap{Grad$(2x^3-2x^2+12x-3)=3 < $ Grad$((x^2-x+3)^2)=4$.} 
        Deshalb brauchen wir keine \link{polydiv}{Polynomdivision} anzuwenden, um einen polynomialen 
        Anteil von $R$ abzuspalten.
        }
        \lang{en}{
        Firstly, we remark that the degree of the polynomial in the numerator is smaller than that 
        in the denominator: 
        \nowrap{Grad$(2x^3-2x^2+12x-3)=3 < $ Grad$((x^2-x+3)^2)=4$.} 
        Thus no \link{polydiv}{polynomial division} is needed.
        }
\item \lang{de}{
      Anschließend versuchen wir Nullstellen des Nenners zu finden: 
      Mit der pq-Formel erhält man jedoch
      }
      \lang{en}{
      Finally we look for roots of the denominator. The pq-formula yields
      }
      \[ x_{1/2}=\frac{1}{2}\pm \sqrt{\frac{1}{4}-3}=\frac{1}{2}\pm \sqrt{-\frac{11}{4}}. \]
      \lang{de}{
      Da es aber keine reellen Wurzeln einer negativen Zahl gibt, besitzt der Nenner keine reellen 
      Nullstellen, deshalb gibt es keine weitere reelle Faktorisierung.
      }
      \lang{en}{
      As there are no real square roots of a negative number, the denominator has no real roots. 
      Hence there is no further real factorisation.
      }
\item \lang{de}{Nun berechnen wir die Partialbruchzerlegung von $\,R(x)$.}
      \lang{en}{Now we decompose $\,R(x)\,$ into partial fractions.}
      \begin{align*} &R(x) &=& \frac{2x^3-2x^2+12x-3}{(x^2-x+3)^2}\\
      	&&=&\frac{B_1x+C_1}{(x^2-x+3)} + \frac{B_2x+C_2}{(x^2-x+3)^2}\\
      	&&=& \frac{(B_1x+C_1)\cdot(x^2-x+3) + B_2x+C_2}{(x^2-x+3)^2}\\
      	&&=&\frac{B_1\,x^3 + (-B_1+C_1)\,x^2 + (3B_1-C_1+B_2)\,x +(3C_1+C_2)}{(x^2-x+3)^2},\\
      	\Longleftrightarrow&&& B_1=2,\quad -B_1+C_1=-2,\quad 3B_1-C_1+B_2=12,\quad 3C_1+C_2=-3,\\
      	\Longleftrightarrow&&& B_1=2,\quad C_1=0,\quad B_2=6,\quad C_2=-3.\\
      \end{align*}
      \lang{de}{Somit $\qquad R(x)= \frac{2x}{(x^2-x+3)}+\frac{6x-3}{(x^2-x+3)^2}$.}
      \lang{en}{Hence $\qquad R(x)= \frac{2x}{(x^2-x+3)}+\frac{6x-3}{(x^2-x+3)^2}$.}
\item \lang{de}{
      Schließlich zerlegen wir die Terme so, dass möglichst der Zähler die Ableitung des 
      quadratischen Terms im Nenner ist.
      }
      \lang{en}{
      Finally we break up the summands.
      }
      \begin{align*} R(x)&=& \frac{2x}{(x^2-x+3)}+\frac{6x-3}{(x^2-x+3)^2}\\
       &=& \frac{2x-1}{(x^2-x+3)}+\frac{1}{(x^2-x+3)}+3\frac{2x-1}{(x^2-x+3)^2}.
      \end{align*}
\item \lang{de}{
      Dann können wir schließlich eine Stammfunktion als Summe der Stammfunktionen der Bausteine 
      berechnen: Stammfunktion von
      }
      \lang{en}{
      Finally, we can compute the sum of antiderivatives of the components in order to determine 
      the antiderivative of the original function. The derivative of
      }
      \[ R(x)= \frac{2x-1}{(x^2-x+3)}+\frac{1}{(x^2-x+3)}+3\frac{2x-1}{(x^2-x+3)^2}\]
      \lang{de}{ist}
      \lang{en}{is}
      \[ \ln{\left|x^2-x+3\right|} + \frac{2}{\sqrt{11}}\arctan\left(\frac{2x-1}{\sqrt{11}}\right)-\frac{3}{(x^2-x+3)}. \]
\end{itemize}
\item \lang{de}{Wir wollen nun das bestimmte Integral}
      \lang{en}{Now suppose we wish to compute the definite integral}
      \[ \int_{0}^2\,R(x)\,dx= \int_{0}^2\,\frac{2x^3-2x^2+12x-3}{(x^2-x+3)^2}\; dx\]
      \lang{de}{berechnen.}
      \lang{en}{using the antiderivative.}
\begin{align*} \int_{0}^2\,R(x)\,dx&=& 
\left[ \ln{\left|x^2-x+3\right|} + \frac{2}{\sqrt{11}}\arctan\left(\frac{2x-1}{\sqrt{11}}\right)-\frac{3}{(x^2-x+3)}\right]_0^2\\
	&=& (\ln(5)-\ln(3)) + \frac{2}{\sqrt{11}}\left(\arctan\left(\frac{3}{\sqrt{11}}\right)-\arctan\left(\frac{-1}{\sqrt{11}}\right)\,\right)
	-(\frac{3}{5}-1)\\
	&=& \ln\left(\frac{5}{3}\right) +
	\frac{2}{\sqrt{11}}\left(\arctan\left(\frac{3}{\sqrt{11}}\right)-\arctan\left(\frac{-1}{\sqrt{11}}\right)\,\right) +\frac{2}{5}.
\end{align*}
\end{enumerate}
\end{tabs*}
\end{example}

\end{content}