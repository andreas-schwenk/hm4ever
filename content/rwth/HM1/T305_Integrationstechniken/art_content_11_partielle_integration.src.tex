%$Id:  $
\documentclass{mumie.article}
%$Id$
\begin{metainfo}
  \name{
    \lang{de}{Partielle Integration}
    \lang{en}{Integration by parts}
  }
  \begin{description} 
 This work is licensed under the Creative Commons License Attribution 4.0 International (CC-BY 4.0)   
 https://creativecommons.org/licenses/by/4.0/legalcode 

    \lang{de}{Beschreibung}
    \lang{en}{Description}
  \end{description}
  \begin{components}
\component{generic_image}{content/rwth/HM1/images/g_img_00_Videobutton_schwarz.meta.xml}{00_Videobutton_schwarz}
\end{components}
  \begin{links}
    \link{generic_article}{content/rwth/HM1/T304_Integrierbarkeit/g_art_content_09_integrierbare_funktionen.meta.xml}{stammfunktion}
    \link{generic_article}{content/rwth/HM1/T301_Differenzierbarkeit/g_art_content_02_ableitungsregeln.meta.xml}{abl-regeln}
  \end{links}
  \creategeneric
\end{metainfo}
\begin{content}
\usepackage{mumie.ombplus}
\ombchapter{4}
\ombarticle{1}

\title{\lang{de}{Partielle Integration}\lang{en}{Integration by parts}}
 
\begin{block}[annotation]
  
  
\end{block}
\begin{block}[annotation]
  Im Ticket-System: \href{http://team.mumie.net/issues/10041}{Ticket 10041}\\
\end{block}

\begin{block}[info-box]
\tableofcontents
\end{block}

\lang{de}{
Wir haben im Abschnitt \link{stammfunktion}{Stammfunktion} gesehen, dass
die Bestimmung des Integrals einer Funktion $f$ über einem Intervall $[a;b]$
am einfachsten ist, wenn man zu $f$ eine Stammfunktion $F$ kennt oder sie berechnen kann,
weil dann nach dem \ref[stammfunktion][Hauptsatz der Differential- und Integralrechnung]{sec:hauptsatz} gilt:
}
\lang{en}{
In the section on \link{stammfunktion}{antiderivatives}, we saw that computing the integral of a 
function $f$ over an interval $[a;b]$ is made much easier if we know an antiderivative $F$ of the 
function $f$. In this case, by the \ref[stammfunktion][fundamental theorem of calculus]{sec:hauptsatz} we have: 
}
\[  \int_a^b f(x)\, dx =[F(x)]_a^b=F(b)-F(a). \]
\lang{de}{
Für einige elementare Funktionen wie die Potenzfunktionen, die Exponentialfunktion, Sinus und 
Kosinus haben wir dort auch die Stammfunktionen gesehen.
\\\\
Wie für die Differentiation gilt bei der Bildung der Stammfunktion auch die Summen- und Faktorregel:
}
\lang{en}{
We gave antiderivatives of certain simple functions such as power functions, the exponential 
function and the sine and cosine functions.
\\\\
As for differentiation, we may use the sum and constant factor rules for determining antiderivatives:
}

\begin{rule}\label{rule:Linear_Stammfkt}
\lang{de}{
Sind $F(x)$ und $G(x)$ Stammfunktionen der Funktionen $f(x)$ bzw. $g(x)$, und sind $\alpha$ und $\beta$ reelle Zahlen, so ist eine Stammfunktion der Funktion $\alpha f(x)+\beta g(x)$ gegeben durch
}
\lang{en}{
Let $F(x)$ und $G(x)$ be antiderivatives of the functions $f(x)$ and $g(x)$ respectively, and let 
$\alpha$ and $\beta$ be real numbers. Then the following function is an antiderivative of the function $\alpha f(x)+\beta g(x)$:
}
\[  \alpha F(x)+\beta G(x). \]
\end{rule}
\lang{de}{
Der Beweis erfolgt durch Differentiation der Funktion $\alpha F(x)+\beta G(x)$ mit Hilfe der 
\ref[abl-regeln][Summen- und Faktorregel]{rule:summenregel} für die Ableitung.
\\\\
Die anderen Ableitungsregeln wie Produktregel, Quotientenregel und Kettenregel lassen sich nicht 
direkt auf die Integration übertragen, da zum Beispiel die Ableitung eines Produkts 
$u(x)\cdot v(x)$ gegeben ist durch
}
\lang{en}{
The proof follows by differentiating the function $\alpha F(x)+\beta G(x)$ using the 
\ref[abl-regeln][sum and constant factor rules]{rule:summenregel}.
\\\\
The other rules for differentiation such as the product rule, quotient rule and chain rule cannot 
be directly translated into rules for integration. For example, the derivative of a product 
$u(x)\cdot v(x)$ is given by
}
\[  (u(x)v(x))'= u'(x)v(x)+u(x)v'(x), \]
\lang{de}{
was jedoch kein Produkt, sondern die Summe zweier Produkte ist.
\\\\
Die Produktregel und die Kettenregel führen jedoch auf Techniken, um Integrale bzw. Stammfunktionen dennoch zu bestimmen.
}
\lang{en}{
which is not necessarily a product itself, rather the sum of two products.
\\\\
However, we may still derive some techniques for integrating, or rather determining antiderivatives, 
using the product rule and the chain rule.
}


\section{\lang{de}{Partielle Integration}\lang{en}{Integration by parts}}\label{sec:part-int}

\lang{de}{
Wir behandeln hier zunächst die \textit{partielle Integration}, die aus der Produktregel entstanden ist.
}
\lang{en}{
Firstly we introduce \textit{integration by parts}, which is derived from the product rule.
}

\begin{theorem}[\lang{de}{Partielle Integration}\lang{en}{Integration by parts}]\label{thm:partint}
\lang{de}{
Es seien $u,v\colon [c,d]\to \R$ stetig differenzierbar (d.\,h. differenzierbar mit stetiger 
Ableitung). Dann gilt für alle $a,b\in [c,d]$ 
}
\lang{en}{
Let $u,v\colon [c,d]\to \R$ be continuously differentiable (that is, differentiable and with a 
continuous derivative) functions. Then for all $a,b\in [c,d]$ we have
}
  \[ \int_a^b u(x)\,v'(x)\;dx = \left[ u(x)\,v(x)\right]_a^b - \int_a^b u'(x)\,v(x)\;dx. \]
\lang{de}{
\floatright{\href{https://www.hm-kompakt.de/video?watch=624}{\image[75]{00_Videobutton_schwarz}}}\\\\
}
\lang{en}{}
\end{theorem}

\begin{proof*}
\lang{de}{Nach dem Hauptsatz der Differential- und Integralrechnung gilt}
\lang{en}{By the fundamental theorem of calculus,}
\[ \int_a^b (u(x)v(x))'\;dx =  \left[ u(x)\,v(x)\right]_a^b. \]
\lang{de}{Nach der Produktregel ist aber $(u(x)v(x))'= u'(x)v(x)+u(x)v'(x)$ und daher}
\lang{en}{By the product rule $(u(x)v(x))'= u'(x)v(x)+u(x)v'(x)$ and therefore}
\[ \int_a^b (u(x)v(x))'\;dx =\int_a^b\big(u'(x)v(x)+u(x)v'(x)\big)\;dx 
=\int_a^b u'(x)v(x)\;dx + \int_a^b u(x)v'(x)\;dx .\]
\lang{de}{
Setzt man den letzten Ausdruck in die erste Gleichung ein und zieht auf beiden Seiten
$\int_a^b u'(x)v(x)\;dx$ ab, erhält man die Gleichung aus dem Satz.
}
\lang{en}{
Substituting the final expression into the first equation and subtracting $\int_a^b u'(x)v(x)\;dx$ 
from both sides yields the equation in the statement of the theorem.
}
\end{proof*}

\lang{de}{
Die \textit{partielle Integration} \"uberf\"uhrt ein Integral in ein anderes, das (hoffentlich) 
einfacher zu l\"osen ist. Manchmal muss man die partielle Integration mehrmals anwenden, um zum Ziel 
zu kommen. Aussichtsreiche Strategien lernt man durch typische Beispiele wie die folgenden.
}
\lang{en}{
\textit{Integration by parts} transforms one integral into another, which is (hopefully) easier to 
compute. Sometimes it is necessary to apply the technique multiple times to reach a function which 
is easy to integrate. The following examples provide some idea of which integrals are promising 
candidates for the technique.
}

\begin{example}\label{ex:1}
\begin{tabs*}[\initialtab{0}]
\tab{$\int_0^{2\pi} x\cdot \cos(x)\, dx$} 
\lang{de}{
Wir betrachten das Integral $\int_0^{2\pi} x\cdot \cos(x)\, dx$. \\
W\"ahlen wir $u(x)=x$ und $v'(x)=\cos x$ mit $u'(x)=1$ und $v(x)=\sin x,$ so erhalten wir
}
\lang{en}{
Consider the integral $\int_0^{2\pi} x\cdot \cos(x)\, dx$. \\
If we choose $u(x)=x$ and $v'(x)=\cos x$ we have $u'(x)=1$ and $v(x)=\sin x$, so
}
\[ \int_0^{2\pi} x \cos x\; dx = [x \sin x]_0^{2\pi} - \int_0^{2\pi} 1\cdot\sin x\; dx.\]
\lang{de}{Für die Funktion $\sin(x)$ kennen wir aber eine Stammfunktion, nämlich $-\cos(x)$, d.\,h.}
\lang{en}{We know an antiderivative of the function $\sin(x)$, namely $-\cos(x)$. Hence}
\[ \int_0^{2\pi} x \cos x\; dx = [x \sin x]_0^{2\pi} - [-\cos x]_0^{2\pi} = [ x \sin x + \cos x]_0^{2\pi}.
\]
\lang{de}{
Zum einen zeigt dies, dass $x \sin(x) + \cos( x)$ eine Stammfunktion von $x\cos(x)$ ist, und zum 
anderen können wir schließlich das Integral berechnen als
}
\lang{en}{
This proves that $x \sin(x) + \cos( x)$ is an antiderivative of $x\cos(x)$, and lets us compute 
the integral:
}
\[ \int_0^{2\pi} x \cos x\; dx= 2\pi\cdot \sin(2\pi)+\cos(2\pi) -\big( 0\cdot \sin(0)+\cos(0)\big)
= 0+1-0-1=0.\]

\tab{$\int_1^2 x^2\;e^x\;dx$}
\lang{de}{
Wir betrachten das Integral $\int_1^2 x^2\;e^x\;dx$.\\
Mit dem Ansatz $u(x)=x^2$ und $v'(x)=e^x$ ist $u'(x)=2x$ sowie $v(x)=e^x$ und damit
}
\lang{en}{
Consider the integral $\int_1^2 x^2\;e^x\;dx$.\\
With the substitutions $u(x)=x^2$ and $v'(x)=e^x$ we have $u'(x)=2x$ and $v(x)=e^x$, so
}
\[ \int_1^2 x^2\;e^x\;dx = [x^2e^x]_1^2 - \int_1^2 2x e^x\;dx= [x^2e^x]_1^2 -2\int_1^2 x e^x\;dx  .\]
\lang{de}{
Da wir auch keine Stammfunktion von $xe^x$ kennen, wenden wir für den letzten Term erneut partielle 
Integration an, diesmal mit
}
\lang{en}{
We do not know any antiderivatives of $xe^x$, but we may simply apply integration by parts again to 
the last integral, this time setting
}
\[ u(x)=x,\, v'(x)=e^x, \quad u'(x)=1,\, v(x)=e^x, \]
\lang{de}{und erhalten}
\lang{en}{and obtaining}
\[\int_1^2 x e^x\;dx =[x\cdot e^x]_1^2 - \int_1^2 1\cdot e^x\;dx =[x\cdot e^x]_1^2 
- [e^x]_1^2 =[xe^x-e^x]_1^2. \]
\lang{de}{Insgesamt also}
\lang{en}{We can conclude that}
\begin{eqnarray*}
  \int_1^2 x^2\;e^x\;dx &=& [x^2e^x]_1^2 - 2\cdot [xe^x-e^x]_1^2=[x^2e^x-2xe^x+2e^x]_1^2\\
&=& [(x^2-2x+2)e^x]_1^2=(2^2-4+2)e^2-(1^2-2+2)e^1=2e^2-e.
\end{eqnarray*}

\tab{$\int_1^e x\ln(x)\,dx$}
\lang{de}{
Um das Integral $\int_1^e x\ln(x)\,dx$ zu berechnen, führt der Ansatz mit $u(x)=x$ und
$v'(x)=\ln(x)$ nicht zum Erfolg, weil wir keine Stammfunktion von $\ln(x)$ kennen.
Setzen wir jedoch $u(x)=\ln(x)$ und $v'(x)=x$, so erhalten wir $u'(x)=\frac{1}{x}$ und $v(x)=\frac{x^2}{2}$ und damit mit partieller Integration
}
\lang{en}{
In order to compute the integral $\int_1^e x\ln(x)\,dx$, attempting to apply the theorem with 
$u(x)=x$ and $v'(x)=\ln(x)$ is unsuccessful, as we do not know an antiderivative of $\ln(x)$. 
However, if we choose $u(x)=\ln(x)$ and $v'(x)=x$, we obtain $u'(x)=\frac{1}{x}$ and 
$v(x)=\frac{x^2}{2}$, and can integrate by parts:
}
\begin{eqnarray*}
 \int_1^e x\ln(x)\,dx &=& [\frac{x^2}{2}\cdot \ln(x)]_1^e -\int_1^e \frac{x^2}{2}\cdot \frac{1}{x} \, dx \\
 &=& \left[\frac{x^2}{2}\cdot \ln(x)\right]_1^e -\frac{1}{2}\int_1^e x\, dx\\
 &=&  \left[\frac{x^2}{2}\cdot \ln(x)\right]_1^e -\frac{1}{2}\cdot \left[\frac{x^2}{2}\right]_1^e =  \left[\frac{x^2}{2}\cdot \ln(x) -\frac{x^2}{4}\right]_1^e \\
&=& \left( \frac{e^2}{2}\cdot 1- \frac{e^2}{4}\right) -\left( \frac{1}{2}\cdot 0- \frac{1}{4}\right) =  \frac{e^2}{4} +\frac{1}{4}.
\end{eqnarray*}
\lang{de}{
Da wir die Grenzen erst am Schluss eingesetzt haben, sehen wir auch, dass 
$F(x)=\frac{x^2}{2}\cdot \ln(x) -\frac{x^2}{4}$ eine Stammfunktion für die Funktion
$f(x)=x\ln(x)$ ist.
}
\lang{en}{
As we waited until the end to substitute in the limits of integration, we even see that 
$F(x)=\frac{x^2}{2}\cdot \ln(x) -\frac{x^2}{4}$ is an antiderivative of the function
$f(x)=x\ln(x)$.
}
\end{tabs*}
\end{example}

\begin{remark}
\lang{de}{
Die Vorgehensweise zur Anwendung der partiellen Integration ist also die, den Integranden
\textit{geschickt} in ein Produkt von Funktionen zu zerlegen, so dass man von einem der Faktoren 
eine Stammfunktion kennt. Diesen Faktor nennt man $v'(x)$ und dessen Stammfunktion ist $v(x)$, 
während der zweite Faktor $u(x)$ genannt wird.
\\\\
Geschickt ist die Zerlegung dann, wenn das Integral, das neu entsteht, also das Integral über die 
Funktion $u'(x)v(x)$, einfacher zu handhaben ist.
\\\\
In den ersten beiden Beispielen hatte man als $u(x)$ ein Polynom gewählt und als $v'(x)$ die 
trigonometrische Funktion bzw. die Exponentialfunktion. Das neu entstandene Integral hatte dann als 
Integrand das Produkt aus der Ableitung $u'(x)$ des Polynoms und einer trigonometrischen bzw.  
Exponentialfunktion. Im ersten Beispiel war $u'(x)$ sogar konstant, weshalb man dann die 
trigonometrische Funktion integrieren konnte. Im zweiten Beispiel musste man die partielle 
Integration ein zweites Mal anwenden, um aus dem polynomialen Faktor einen konstanten Faktor zu 
bekommen.
\\\\
Hat man allgemeiner das Produkt eines Polynoms $n$-ten Grades mit $\sin(x)$, $\cos(x)$ oder $e^x$, 
so führt dasselbe Verfahren nach $n$-facher Anwendung der partiellen Integration zum Erfolg.
}
\lang{en}{
The strategy used in integration by parts is to write the integrands as a product of two functions 
such that we know an antiderivative of one of the two factors. We call that factor $v'(x)$ and its 
antiderivative $v(x)$. The other factor is denoted $u(x)$.
\\\\
The factors are well chosen if the new integral produced by the procedure, that is, the integral of 
the function $u'(x)v(x)$, is easier to compute than the original integral.
\\\\
In the first two examples, we chose a polynomial as $u(x)$ and a trigonometric or exponential 
function as $v'(x)$. The new integrand was then the product of a lower degree polynomial $u'(x)$ and 
the trigonometric or exponential function. In the first example $u'(x)$ was constant, so we could 
simply integrate the trigonometric function. In the second example we applied the procedure twice 
in order to get a constant polynomial, and be able to simply integrate the exponential function.
\\\\
In general, given an integral whose integrand is the product of a polynomial of degree $n$ and one 
of $\sin(x)$, $\cos(x)$ or $e^x$, we may integrate by parts $n$ times to compute the integral.
}
\end{remark}

\begin{quickcheck}
\text{
\lang{de}{
Wie wählen Sie $u$ und $v$, um $\int_a^b \sin(x)\cdot(x+1)\, dx$ mit partieller Integration in einem 
Schritt zu bestimmen?
}
\lang{en}{
What choice of $u$ and $v$ lets us compute $\int_a^b \sin(x)\cdot(x+1)\, dx$ by parts in a single 
step?
}}
%\type{mc.unique}
\explanation{
\lang{de}{
Wie in obiger Bemerkung erklärt, ist es geschickt, $u$ als Polynom zu wählen, damit $u'$ eine Konstante wird.
}
\lang{en}{
As explained in the above remark, it makes sense to choose $u$ as the polynomial, as then $u'$ is 
a constant.
}}
\begin{choices}{unique}
\begin{choice}
\text{$u(x) = \sin(x)$, $v'(x)=x+1$}
\solution{false}
\end{choice}
\begin{choice}
\text{$u(x) = x+1$, $v'(x)=\sin(x)$}
\solution{true}
\end{choice}
\end{choices}
\end{quickcheck}

\begin{example}\label{ex:int-ln}
\lang{de}{
Wir wollen eine Stammfunktion von $\ln(x)$ finden. Da es im obigen Beispiel von Erfolg gekrönt war, 
beim Integral von $x\ln(x)$ den Faktor $\ln(x)$ als $u(x)$ zu setzen, schreiben wir 
$\ln(x)=1\cdot \ln(x)$ und setzen $v'(x)=1$ und $u(x)=\ln(x)$. 
Damit sind $v(x)=x$ und $u'(x)=\frac{1}{x}$ und wir erhalten
}
\lang{en}{
We wish to find an antiderivative of $\ln(x)$. In the above example we successfully set $u(x)$ to 
be the factor $\ln(x)$ of the integrand, so let us try writing $\ln(x)=1\cdot \ln(x)$ and setting 
$v'(x)=1$ and $u(x)=\ln(x)$. Then we have $v(x)=x$ and $u'(x)=\frac{1}{x}$, and obtain
}
\begin{eqnarray*}
 \int_a^b \ln(x)\,dx &=& \left[ x\ln(x)\right]_a^b - \int_a^b x\cdot \frac{1}{x} \,dx
= \left[ x\ln(x)\right]_a^b -\int_a^b 1\, dx \\
&=&  \left[ x\ln(x)\right]_a^b - [x]_a^b
= \left[ x\ln(x) -x \right]_a^b. 
\end{eqnarray*}
\lang{de}{
Also ist $x\ln(x) -x $ eine Stammfunktion von $\ln(x)$.
  \floatright{\href{https://www.hm-kompakt.de/video?watch=625}{\image[75]{00_Videobutton_schwarz}}}\\\\
}
\lang{en}{Hence $x\ln(x) -x $ is an antiderivative of $\ln(x)$.}
\end{example}

\section{\lang{de}{Partielle Integration mit anschließender Gleichungsauflösung}
         \lang{en}{Integration by parts to obtain an equation}}\label{sec:part-int-gleich-aufl}


\lang{de}{
Manchmal erhält man bei der partiellen Integration auch ein reelles Vielfaches des ursprünglichen 
Integrals wieder. Durch Umformen der Gleichung kommt man dann auch zum Ziel (sofern der reelle 
Faktor nicht gleich $1$ ist).
}
\lang{en}{
Sometimes integration by parts yields a constant multiple of the original integral. In these cases, 
we can rearrange the resulting equation for the integral (provided the constant factor is not $1$) 
and solve it.
}

\begin{example}
\lang{de}{
Wir betrachten wieder das Integral $\int_1^e x\ln(x)\,dx$ aus dem \lref{ex:1}{obigen Beispiel }. Da 
wir in Beispiel \ref{ex:int-ln} eine Stammfunktion $F(x)=x\ln(x) -x$ von $\ln(x)$ berechnet haben, 
können wir zur partiellen Integration auch den Ansatz $u(x)=x$, $v'(x)=\ln(x)$ machen mit $u'(x)=1$ 
und $v(x)=x\ln(x) -x$. Dann ist
}
\lang{en}{
Consider once again the integral $\int_1^e x\ln(x)\,dx$ from the \lref{ex:1}{above example}. As we 
found an antiderivative $F(x)=x\ln(x) -x$ of $\ln(x)$ in example \ref{ex:int-ln}, we may proceed via 
integration by parts with $u(x)=x$ and $v'(x)=\ln(x)$, so that $u'(x)=1$ and $v(x)=x\ln(x) -x$. Then
}
\begin{eqnarray*}
  \int_1^e \textcolor{#0066CC}{x\ln(x)}\,dx &=& \left[ x(x\ln(x) -x)\right]_1^e - \int_1^e 1\cdot (x\ln(x) -x)\, dx\\
&=&  \left[ x(x\ln(x) -x)\right]_1^e - \int_1^e x\ln(x) \, dx +  \int_1^e x\, dx \\
&=&  \left[ x(x\ln(x) -x)\right]_1^e- \int_1^e \textcolor{#0066CC}{x\ln(x)} \, dx + \left[ \frac{x^2}{2}\right]_1^e .
\end{eqnarray*}
\lang{de}{Damit erhält man}
\lang{en}{We obtain}
\[ 2\cdot  \int_1^e x\ln(x)\,dx  =\left[ x(x\ln(x) -x) +  \frac{x^2}{2}\right]_1^e, \]
\lang{de}{d.\,h.}
\lang{en}{that is,}
\[   \int_1^e x\ln(x)\,dx  =\left[ \frac{x}{2}(x\ln(x) -x) +  \frac{x^2}{4}\right]_1^e. \]
\lang{de}{Als Stammfunktion von $x\ln(x)$ erhalten wir also}
\lang{en}{Hence we get an antiderivative of $x\ln(x)$, namely}
\[ \frac{x}{2}(x\ln(x) -x) +  \frac{x^2}{4}= \frac{x^2}{2}\cdot\ln(x) -\frac{x^2}{4},\]
\lang{de}{was mit dem Ergebnis \lref{ex:1}{oben} übereinstimmt.}
\lang{en}{which confirms the result \lref{ex:1}{above}.}
\end{example}


\end{content}