\documentclass{mumie.problem.gwtmathlet}
%$Id$
\begin{metainfo}
  \name{
    \lang{de}{A07: Partialbruchzerlegung}
    \lang{en}{}
  }
  \begin{description} 
 This work is licensed under the Creative Commons License Attribution 4.0 International (CC-BY 4.0)   
 https://creativecommons.org/licenses/by/4.0/legalcode 

    \lang{de}{}
    \lang{en}{}
  \end{description}
  \corrector{system/problem/GenericCorrector.meta.xml}
  \begin{components}
    \component{js_lib}{system/problem/GenericMathlet.meta.xml}{mathlet}
  \end{components}
  \begin{links}
  \end{links}
  \creategeneric
\end{metainfo}
\begin{content}
\usepackage{mumie.genericproblem}

\lang{de}{\title{A07: Partialbruchzerlegung}}

\begin{block}[annotation]
	Im Ticket-System: \href{http://team.mumie.net/issues/11224}{Ticket 11224}
\end{block}

\begin{problem}
	\begin{question}
		
		\begin{variables}
			\randint[Z]{c1}{-4}{-1} %Z heißt "ohne 0"
			\randint[Z]{a}{-2}{2} 
			\randint[Z]{k}{-1}{1} 
			\function[calculate]{c2}{-c1+1}
			\function[normalize]{g}{x*(x-c1)*(x-c2)}
			\function[calculate]{b}{k*a+1}
			\randadjustIf{a}{b=0}%wenn a, b und (a+b)\neq 0, da ansonsten weniger Bruchstriche benötigt werden
			
			\function[normalize]{xc1}{x-c1}
			 
			\function[normalize]{xc2}{x-c2}
			
			\function[expand,normalize]{f}{a*xc2*x+b*xc1*x-(a+b)*xc2*xc1}
			\function{s1}{a/(xc1)}
			\function{s2}{b/(xc2)}
			\function[normalize]{s3}{(-a-b)/x}
			\function[normalize]{s}{s1+s2+s3}
			\function{fdg}{f/g}
            
            \function{F}{a*ln(abs(xc1))+b*ln(abs(xc2))+(-a-b)*ln(abs(x))}
			
		\end{variables}
		
		\lang{de}{
	    \text{Bestimmen Sie die Partialbruchzerlegung der rationalen Funktion 
        $f(x)=\var{fdg}$.\\        
        $\var{fdg}=$\ansref.\\
        Bestimmen Sie nun noch eine Stammfunktion $F(x)$ von der rationalen Funktion $f(x)$.\\
        $F(x)=$\ansref.}
		}
     
	\type{input.function}
	\field{rational}
	\begin{answer}
		\solution{s}
        \inputAsFunction{x}{kk}
        \checkAsFunction{x}{-10}{10}{100}
        \checkStringsForRelation{count(/,kk)>2 AND count((,kk)<3 AND count(x,kk)<4 AND equal(s,kk)}
        \explanation{Es ist der Ansatz $\frac{A}{\var{xc1}}+\frac{B}{\var{xc2}}+\frac{C}{x}$ mit reellen Zahlen $A,B,C$ zu wählen.}
        \explanation[count((,kk)>2)]{Nutzen Sie nur so viele Klammern wie nötig.}
	\end{answer}
    \begin{answer}
        \solution{F}
        \inputAsFunction{x}{Fx}
        \checkFuncForZero{D[Fx]-kk}{-10}{10}{100}
        \explanation{Nutzen Sie die Partialbruchzerlegung, um eine Stammfunktion zu bestimmen.}
    \end{answer}
	\end{question}
\end{problem}

\embedmathlet{mathlet}

\end{content}