\documentclass{mumie.problem.gwtmathlet}
%$Id$
\begin{metainfo}
  \name{
    \lang{de}{A08: Partialbruchzerlegung}
    \lang{en}{}
  }
  \begin{description} 
 This work is licensed under the Creative Commons License Attribution 4.0 International (CC-BY 4.0)   
 https://creativecommons.org/licenses/by/4.0/legalcode 

    \lang{de}{}
    \lang{en}{}
  \end{description}
  \corrector{system/problem/GenericCorrector.meta.xml}
  \begin{components}
    \component{js_lib}{system/problem/GenericMathlet.meta.xml}{mathlet}
  \end{components}
  \begin{links}
  \end{links}
  \creategeneric
\end{metainfo}
\begin{content}
\begin{block}[annotation]
	Im Ticket-System: \href{https://team.mumie.net/issues/19138}{Ticket 19138}
\end{block}
\usepackage{mumie.genericproblem}

\lang{de}{\title{A08: Partialbruchzerlegung}}

\begin{problem}
	\begin{question}
		
		\begin{variables}
			\randint[Z]{a}{-5}{5} %Z heißt "ohne 0"
			\randint[Z]{b}{-5}{5} 
			\randint[Z]{c}{-5}{5}
            \function[calculate]{A}{c+b*a}
            \randadjustIf{b}{A=0}%Damit Aufgabe immer richtig korrigiert wird, siehe Tr7
            \function[normalize]{xc1}{x-a}  %xc1 sollte besser xa heißen?
            \function[normalize]{f}{(b*x+c)/(xc1)^2}
            \function{s1}{b/xc1}
            \function{s2}{A/xc1^2}
            \function{s}{s1+s2}
            
            \function{F}{b*ln(abs(xc1))-(c+a)/xc1}
			
            
		\end{variables}
		
		\lang{de}{
	    \text{Bestimmen Sie die Partialbruchzerlegung der rationalen Funktion 
        $f(x)=\var{f}$.\\        
        $\var{f}=$\ansref.\\
        Bestimmen Sie nun noch eine Stammfunktion $F(x)$ der rationalen Funktion $f(x)$.\\
        $F(x)=$\ansref.}
		}
     
	\type{input.function}
	\field{rational}
	\begin{answer}
		\solution{s}
        \inputAsFunction{x}{k}
        \checkAsFunction{x}{-10}{10}{100}
        \checkStringsForRelation{count(/,k)>1 AND count((,k)<3 AND count(x,k)<3 AND equal(s,k)}
        \explanation{Es ist der Ansatz $\frac{A}{\var{xc1}}+\frac{B}{(\var{xc1})^2}$ mit reellen Zahlen $A$ und $B$ zu wählen.}
        \explanation[count((,k)>2)]{Nutzen Sie nur so viele Klammern wie nötig.}
	\end{answer}
    \begin{answer}
        \solution{F}
        \inputAsFunction{x}{Fx}
        \checkFuncForZero{D[Fx]-s}{-10}{10}{100}
        \explanation{Nutzen Sie die Partialbruchzerlegung, um eine Stammfunktion zu bestimmen.}
    \end{answer}
	\end{question}
\end{problem}

\embedmathlet{mathlet}
\end{content}

