\documentclass{mumie.problem.gwtmathlet}
%$Id$
\begin{metainfo}
  \name{
    \lang{de}{A05: Stammfunktion rationaler Funktionen}
    \lang{en}{}
  }
  \begin{description} 
 This work is licensed under the Creative Commons License Attribution 4.0 International (CC-BY 4.0)   
 https://creativecommons.org/licenses/by/4.0/legalcode 

    \lang{de}{}
    \lang{en}{}
  \end{description}
  \corrector{system/problem/GenericCorrector.meta.xml}
  \begin{components}
    \component{js_lib}{system/problem/GenericMathlet.meta.xml}{mathlet}
  \end{components}
  \begin{links}
  \end{links}
  \creategeneric
\end{metainfo}
\begin{content}
\usepackage{mumie.genericproblem}

\lang{de}{\title{A05: Stammfunktion rationaler Funktionen}}

\begin{block}[annotation]
	Im Ticket-System: \href{http://team.mumie.net/issues/11222}{Ticket 11222}
\end{block}

\begin{problem}
	\randomquestionpool{1}{3}
	
	\begin{question}	
		\begin{variables}
		%g = ax 3 + bx 2 + d mit b, d ∈ {− 4; 4 } \ { 0 } und a ∈ { 0; 1 }

			\randint[Z]{a}{-3}{4}
			\randint[Z]{b}{-3}{4}
			\randint[Z]{c}{-3}{4}
			\randint[Z]{n}{2}{4}

			\function[normalize]{f}{2*a*x+b}
			\function[normalize]{g0}{a*x^2+b*x+c}
			\function{g}{g0^n}
			
			\function[normalize]{S}{1/((1-n)*g0^(n-1))}
    

		\end{variables}
	%+++++++++++++++++++TYPE+++++++++++++++++++++++++++	
		  \type{input.function}
          \explanation{Nutzen Sie die Substitutionsregel. Im Zähler steht die Ableitung der zu substituierenden Funktion.}
      \field{rational}
		\correctorprecision{3}
		\displayprecision{3}
	    \lang{de}{\text{Bestimmen Sie eine Stammfunktion von $\frac{\var{f}}{\var{g}}$.}} 
	    \begin{answer}
	    	\text{Eine Stammfunktion ist }
			\solution{S}
          \inputAsFunction{x}{s}%D[s]=f/g
          \checkFuncForZero{D[s]-f/g}{-10}{10}{100}
	    \end{answer}  
        
	    
	\end{question}
	
	\begin{question}	
		\begin{variables}

			\randint[Z]{a}{-3}{4}
			\randint[Z]{c}{1}{3}
            \function{d}{c^2}

			\function{f}{a}
			\function{g}{x^2+d}
			
			\function{S}{a*arctan(x/c)/c}


		\end{variables}
	%+++++++++++++++++++TYPE+++++++++++++++++++++++++++	
		  \type{input.function}
     \field{real}
		\correctorprecision{3}
		\displayprecision{3}
	    \lang{de}{\text{Bestimmen Sie eine Stammfunktion von $\frac{\var{f}}{\var{g}}$.}} 
	    \begin{answer}
	    	\text{Eine Stammfunktion ist }
			\solution{S}
         \inputAsFunction{x}{s}%D[s]=f/g
         \checkFuncForZero{D[s]-f/g}{-10}{10}{100}
	    \end{answer}  
	    \explanation{Der Nenner hat keine reelle Nullstelle. Nutzen Sie aus, dass $\arctan$ eine Stammfunktion von $\frac{1}{x^2+1}$ ist.}
	\end{question}
	
	\begin{question}	
		\begin{variables}
		%g = ax 3 + bx 2 + d mit b, d ∈ {− 4; 4 } \ { 0 } und a ∈ { 0; 1 }

			\randint[Z]{a}{-3}{4}
			\randint[Z]{b}{-3}{4}
			\randint[Z]{c}{-3}{4}

			\function[normalize]{f}{2*a*x+b}
			\function[normalize]{g}{a*x^2+b*x+c}
			
			\function[normalize]{S}{ln(g)}


		\end{variables}
	%+++++++++++++++++++TYPE+++++++++++++++++++++++++++	
		  \type{input.function}
      \field{real}
		\correctorprecision{3}
		\displayprecision{3}
	    \lang{de}{\text{Bestimmen Sie eine Stammfunktion von $\frac{\var{f}}{\var{g}}$.}} 
	    \begin{answer}
	    	\text{Eine Stammfunktion ist }
			\solution{S}
          \inputAsFunction{x}{s}%D[s]=f/g
          \checkFuncForZero{D[s]-f/g}{-10}{10}{100}
	    \end{answer}  
	    \explanation{Im Zähler steht die Ableitung des Nenners. Nutzen Sie also die Substitutionsregel.}
	\end{question}
	
\end{problem}

\embedmathlet{mathlet}

\end{content}