\documentclass{mumie.problem.gwtmathlet}
%$Id$
\begin{metainfo}
  \name{
    \lang{de}{A06: Partialbruchzerlegung}
    \lang{en}{}
  }
  \begin{description} 
 This work is licensed under the Creative Commons License Attribution 4.0 International (CC-BY 4.0)   
 https://creativecommons.org/licenses/by/4.0/legalcode 

    \lang{de}{}
    \lang{en}{}
  \end{description}
  \corrector{system/problem/GenericCorrector.meta.xml}
  \begin{components}
    \component{js_lib}{system/problem/GenericMathlet.meta.xml}{mathlet}
  \end{components}
  \begin{links}
  \end{links}
  \creategeneric
\end{metainfo}
\begin{content}
\usepackage{mumie.genericproblem}

\lang{de}{\title{A06: Partialbruchzerlegung}}

\begin{block}[annotation]
	Im Ticket-System: \href{http://team.mumie.net/issues/11223}{Ticket 11223}
\end{block}

\begin{problem}
	%\randomquestionpool{}{}
	\begin{question}
		
		\begin{variables}
			\randint[Z]{c1}{-4}{-1} %Z heißt "ohne 0"
			\randint[Z]{a}{-2}{2} 
            \randint[Z]{k}{-1}{1} 
			%\randadjustIf{}{}
			%\derivative{g1}{g}{x}
			%\substitute[expand,normalize]{fg}{f}{x}{gcal}
			\function[calculate]{c2}{-c1+1}
			\function[normalize]{g}{(x-c1)*(x-c2)}
			\function[calculate]{b}{k*a+1}
			\randadjustIf{a}{b=0}
			\function[expand,normalize]{f}{a*(x-c2)+b*(x-c1)}
		
			\function[normalize]{xc1}{x-c1}
			\function[normalize]{xc2}{x-c2}
			\function{s1}{a/(xc1)}
			\function{s2}{b/(xc2)}
            \function{s}{s1+s2}
			
			\function{fdg}{f/g}
		\end{variables}
		
%		\type{input.generic}
%        \field{real}
%		\correctorprecision{3}
%		\displayprecision{3}
		
		\lang{de}{
	    \text{Bestimmen Sie die Partialbruchzerlegung der rationalen Funktion $\var{fdg}$.\\
        $\var{fdg}=$\ansref.
        }
	    }
%	    \permuteAnswers{1, 2} 
%	    %http://team.mumie.net/projects/support/wiki/DifferentAnswerType
%	    \begin{answer}
%            \type{input.function}
%			\solution{s1}
%			\checkAsFunction{x}{-10}{10}{100} 
%            %\inputAsFunction{x}{s3}
%			%\checkFuncForZero{D[s3]-h}{-10}{10}{100}
%	    \end{answer}    
%	    \begin{answer}
%            \type{input.function}
%			\solution{s2}
%			\checkAsFunction{x}{-10}{10}{100} 
%            %\inputAsFunction{x}{s3}
%			%\checkFuncForZero{D[s3]-h}{-10}{10}{100}
%	    \end{answer}   
%	    %generic viz: http://team.mumie.net/projects/support/wiki/Example
%	    
%	\end{question}

	\type{input.function}
	\field{rational}
	\begin{answer}
		\solution{s}
        \inputAsFunction{x}{kk}
        \checkAsFunction{x}{-10}{10}{100}
        \checkStringsForRelation{count(/,kk)>1 AND count((,kk)<3 AND count(x,kk)<3 AND equal(s,kk)}
        \explanation{Es ist der Ansatz $\frac{A}{\var{xc1}}+ \frac{B}{\var{xc2}}$ zu wählen. $A$ und $B$ sind reelle Zahlen, die mit Koeffizientenvergleich berechnet werden müssen.}
        \explanation[count((,kk)>2)]{Nutzen Sie nur so viele Klammern wie nötig.}
	\end{answer}

  \end{question}


\end{problem}

\embedmathlet{mathlet}

\end{content}