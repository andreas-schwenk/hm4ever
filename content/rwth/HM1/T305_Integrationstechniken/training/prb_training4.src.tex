\documentclass{mumie.problem.gwtmathlet}
%$Id$
\begin{metainfo}
  \name{
    \lang{de}{A04: Substitution}
    \lang{en}{}
  }
  \begin{description} 
 This work is licensed under the Creative Commons License Attribution 4.0 International (CC-BY 4.0)   
 https://creativecommons.org/licenses/by/4.0/legalcode 

    \lang{de}{Beschreibung}
    \lang{en}{}
  \end{description}
  \corrector{system/problem/GenericCorrector.meta.xml}
  \begin{components}
    \component{js_lib}{system/problem/GenericMathlet.meta.xml}{mathlet}
  \end{components}
  \begin{links}
  \end{links}
  \creategeneric
\end{metainfo}
\begin{content}
\usepackage{mumie.genericproblem}

\lang{de}{\title{A04: Substitution}}
\lang{en}{\title{Problem 4}}


\begin{block}[annotation]
  Im Ticket-System: \href{http://team.mumie.net/issues/11221}{Ticket 11221}
\end{block}

\begin{problem}

		\begin{variables}
		    \randint{p}{1}{5}
		    \randint{q}{1}{5}
		    \function[calculate]{PP}{p*4}
		    \function[calculate]{QQ}{q*3}
		    \function{f}{p*y^4+q*y^3}
		    \derivative[normalize]{f_1}{f}{y}
		    \functionNormalize{f_1}

		    \randint[Z]{m}{-3}{5}
		    \randint{n}{1}{6}
		    \function{g}{m*x+n}

		    \substitute{h}{f_1}{y}{g}
		    \substitute{j}{f}{y}{g}

		\end{variables}

\begin{question}
	\type{input.function}
	\field{real}
	\lang{de}{
			\text{Geben Sie zwei Funktionen $f(y)$ und $g(x)$ an, sodass für ihre Verkettung \\
			$f(g(x)) =\var{PP} \cdot (\var{g})^3+\var{QQ} \cdot (\var{g})^2$ gilt.\\
			%begin-cosh		
			Dabei soll der Graph von $g$ eine Gerade beschreiben. }
			%ende-cosh		
	}
	\lang{en}{
		\text{Find two functions $f(y)$ and $g(x)$ such that their composition is 
	    $f(g(x)) =\var{PP} \cdot (\var{g})^3+\var{QQ} \cdot (\var{g})^2$ and such that the graph of $g(x)$ is a line. }
	}
	\begin{answer}
		\text{ $f(y) = $}
		\solution{f_1}
		\inputAsFunction{y}{R}
	\end{answer}
	\begin{answer}
		\text{ $g(x) = $}
		\solution{g}
		\inputAsFunction{x}{S}
		\checkFuncForZero{R[S]-h}{-10}{10}{100}
        \explanation{Sie haben $f$ und $g$ nicht richtig gewählt.}
		\score{2.0}
	\end{answer}
\end{question}


	\begin{question}

	  \begin{variables}
	 	\randint{u}{-3}{4}
	 	\randint{v}{1}{5}
		\randadjustIf{u,v}{u>=v}

		\substitute[normalize]{JU}{j}{x}{u}
		\substitute[normalize]{JV}{j}{x}{v}

	 	\function[calculate]{L}{(JV-JU)/m}
	  \end{variables}

		\lang{de}{
			\text{Berechnen Sie nun das Integral $\int_a^b \var{PP} \cdot (\var{g})^3+\var{QQ} \cdot (\var{g})^2\, dx$\\
			mit den Grenzen $a=\var{u}$ und $b=\var{v}$.\\
			%\textit{Hinweis: Lineare Substitution mit Hilfe der in a) bestimmten Funktionen nutzen!}\\
			\textit{Für große Zahlen nutzen Sie am besten einen Taschenrechner.}}
		}
		 \lang{en}{\text{
      Now calculate the integral $\int_a^b \var{PP} \cdot (\var{g})^3+\var{QQ} \cdot (\var{g})^2\, dx$\\
      with the limits $a=\var{u}$ and $b=\var{v}$.\\
      \textit{Hint: use a linear substitution with the function determined in a)!}\\
      \textit{For large numbers it is best to use a calculator.}
     }}
		\type{input.number}
		\begin{answer}
      \lang{de}{	\text{ Lösung: }
}
 \lang{en}{\text{ Solution: }}
      \solution{L}
		\end{answer}
        \explanation{Lineare Substitution mit Hilfe der in a) bestimmten Funktionen nutzen!}

	\end{question}

\end{problem}


\embedmathlet{mathlet}

\end{content}