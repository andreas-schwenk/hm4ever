%$Id:  $
\documentclass{mumie.article}
%$Id$
\begin{metainfo}
  \name{
    \lang{de}{Überblick: Integrationstechniken}
    \lang{en}{Overview: Integration techniques}
  }
  \begin{description} 
 This work is licensed under the Creative Commons License Attribution 4.0 International (CC-BY 4.0)   
 https://creativecommons.org/licenses/by/4.0/legalcode 

    \lang{de}{Beschreibung}
    \lang{en}{Description}
  \end{description}
  \begin{components}
  \end{components}
  \begin{links}
\link{generic_article}{content/rwth/HM1/T305_Integrationstechniken/g_art_content_13_partialbruchzerlegung.meta.xml}{content_13_partialbruchzerlegung}
\link{generic_article}{content/rwth/HM1/T305_Integrationstechniken/g_art_content_12_substitutionsregel.meta.xml}{content_12_substitutionsregel}
\link{generic_article}{content/rwth/HM1/T305_Integrationstechniken/g_art_content_11_partielle_integration.meta.xml}{content_11_partielle_integration}
\end{links}
  \creategeneric
\end{metainfo}
\begin{content}
\begin{block}[annotation]
	Im Ticket-System: \href{https://team.mumie.net/issues/30121}{Ticket 30121}
\end{block}





\begin{block}[annotation]
Im Entstehen: Überblicksseite für Kapitel  Integrationstechniken
\end{block}

\usepackage{mumie.ombplus}
\ombchapter{1}
\title{\lang{de}{Überblick:  Integrationstechniken}\lang{en}{Overview: Integration techniques}}



\begin{block}[info-box]
\lang{de}{\strong{Inhalt}}
\lang{en}{\strong{Contents}}


\lang{de}{
    \begin{enumerate}%[arabic chapter-overview]
   \item[4.1] \link{content_11_partielle_integration}{Partielle Integration}
   \item[4.2] \link{content_12_substitutionsregel}{Substitutionsregel}
   \item[4.3] \link{content_13_partialbruchzerlegung}{Partialbruchzerlegung}
   \end{enumerate}
}
\lang{en}{
    \begin{enumerate}%[arabic chapter-overview]
   \item[4.1] \link{content_11_partielle_integration}{Integration by parts}
   \item[4.2] \link{content_12_substitutionsregel}{Substitution rules}
   \item[4.3] \link{content_13_partialbruchzerlegung}{Partial fraction decomposition}
   \end{enumerate}
} %lang

\end{block}

\begin{zusammenfassung}

\lang{de}{
Integration ist verglichen mit der Differentiation eine Kunst.
Durch den Hauptsatz der Differentiation und Integration wissen wir zwar, dass jede stetige Funktion 
eine Stammfunktion hat, aber es ist oft schwierig, manchmal sogar unmöglich diese Stammfunktion in 
einer geschlossenen Form anzugeben. Dieses Kapitel behandelt mit der partiellen Integration und der 
Substitutionsregel wichtige und mächtige Methoden, Stammfunktionen und Integrale zu berechnen.
\\\\
Beide Methoden sind Anwendungen des Hauptsatzes der Differential- und Integralrechnung.
Bei der partiellen Integration schieben wir die Produktregel der Differentiation unter das Integral.
Bei der Substitutionsregel machen wir dasselbe mit der Kettenregel.
\\\\
Die Integration von rationalen Funktionen benötigt neben der Substitionsregel häufig noch eine 
Rechenmethode, die Partialbruchzerlegung. Diese behandeln wir im letzten Teil des Kapitels. Die 
Partialbruchzerlegung hat Anwendungen über die Integration hinaus.
}
\lang{en}{
Compared to differentiation, integration is somewhat of an art. 
The fundamental theorem of calculus tells us that every continuous function has an antiderivative, 
however it is often difficult and sometimes impossible to give a closed-form expression for an 
antiderivative. This chapter introduces integration by parts and the substitution rule, which are 
both important and powerful methods for computing antiderivatives and integrals.
\\\\
Both methods involve an application of the fundamental theorem of calculus. Integration by parts 
derives from the product rule for differentiation, turning it into a technique for integration. 
The substitution rule does the same with the chain rule.
\\\\
Integration of rational functions requires, besides the substitution rule, a technique we call 
decomposition of rational functions into partial fractions. This is introduced in the last section 
of the chapter. 
}


\end{zusammenfassung}

\begin{block}[info]\lang{de}{\strong{Lernziele}}
\lang{en}{\strong{Learning Goals}} 
\begin{itemize}[square]
\item \lang{de}{Sie kennen die Techniken der partiellen Integration und der Substitutionsregel.}
      \lang{en}{Knowing the technique of integration by parts and the substitution rule.}
\item \lang{de}{
      Sie entscheiden ob und wie diese Methoden in konkreten Situtationen angewendet werden können.
      }
      \lang{en}{
      Being able to decide which (if any) of these methods can be applied in a given problem.
      }
\item \lang{de}{
      Sie berechnen mit diesen Techniken bestimmte Integrale und Stammfunktionen.
      }
      \lang{en}{
      Being able to compute definite integrals and find antiderivatives using these methods.
      }
\item \lang{de}{
      Sie erkennen an einer rationalen Funktion, welche Nenner für die Partialbruchzerlegung in 
      Frage kommen.
      }
      \lang{en}{
      Being able to recoginse which denominators are used in a decomposition of a rational function 
      into partial fractions.
      }
\item \lang{de}{Sie zerlegen rationale Funktionen in Summen von Partialbrüchen.}
      \lang{en}{Being able to decompose rational functions into sums of partial fractions.}
\end{itemize}
\end{block}




\end{content}
