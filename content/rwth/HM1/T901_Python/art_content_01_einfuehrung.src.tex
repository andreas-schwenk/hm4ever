%$Id:  $
\documentclass{mumie.article}
%$Id$
\begin{metainfo}
  \name{
    \lang{de}{Python Einführung}
    \lang{en}{...}
   }
  \begin{description} 
 This work is licensed under the Creative Commons License Attribution 4.0 International (CC-BY 4.0)   
 https://creativecommons.org/licenses/by/4.0/legalcode 

    \lang{de}{Python Einführung}
    \lang{en}{...}
  \end{description}
  \begin{components}
  \end{components}
  \begin{links}
%\\link{generic_article}{content/rwth/HM1/T901_Python/g_art_content_01_einfuehrung.meta.xml}{content_01_einfuehrung}
\link{generic_article}{content/rwth/HM1/T901_Python/g_art_content_03_programmablauf.meta.xml}{content_03_programmablauf}
%\link{generic_article}{content/rwth/HM1/T901_Python/g_art_content_05_kursverlinkungen.meta.xml}{content_05_kursverlinkungen}
\end{links}
  \creategeneric
\end{metainfo}

\begin{content}
\begin{block}[annotation]
	Im Ticket-System: \href{https://team.mumie.net/issues/18344}{Ticket 18344}
\end{block}
\usepackage{mumie.ombplus}
%\usepackage{listings}
\ombchapter{12}
\ombarticle{1}
\usepackage{mumie.genericvisualization}

\begin{visualizationwrapper}


\begin{block}[annotation]
DIESER KURSTEIL BEFINDET SICH NOCH IM AUFBAU!
\end{block}

\lang{de}{\title{Einführung in Python}}

%\begin{block}[annotation]
%  Im Ticket-System: \href{http://team.mumie.net/issues/18325}{Ticket 18325}\\
%\end{block}

\begin{remark}
Die Inhalte dieses Python-Kurses sind \textit{nicht} prüfungsrelevant.
\end{remark}

\begin{block}[info-box]
\tableofcontents
\end{block}


\section{Motivation}

In der Praxis ist es meist sinnvoll,
Berechnungen automatisiert durchzuführen. 
Während die Beispiele und Trainingsaufgaben
in diesem Kurs meist so gewählt sind,
dass die Rechnung möglichst einfach ist,
treten im experimentellen Kontext nur sehr selten 
ganzzahlige Koeffizienten und Konstanten auf.
Darüber hinaus sind die
die vorkommenden Terme meist eher lang.
Es gibt weiterhin sogar Aufgabenstellungen, 
für die gar kein geschlossener
Lösungsweg bekannt ist. 
In diesen Fällen kann die Lösung nur
numerisch angenähert werden.

\begin{example}
\begin{tabs*}[\initialtab{1}]
\tab{Integral} 
    Das folgende Integral lässt sich 
    mit den im Kurs aufgezeigten Ansätzen nicht bestimmen:
    \[ \int_0^1 e^{-x^2} dx \]
\tab{Nullstellen}
    Die Nullstellen von Polynomen höherer Ordnung
    können (abgesehen von Spezialfällen) nur numerisch 
    bestimmt werden.
\end{tabs*}
\end{example}

Ensprechend setzt man bei praktischen Problemstellen
fast immer auf \textit{rechnergestützte} Lösungen.

\subsection{Die Programmiersprache Python}

In diesem Kapitel werden die Grundzüge der 
\textit{Programmierung mit Python} aufgezeigt.

\begin{definition}[Python]
\textbf{Python} ist eine höhere \textit{Programmiersprache} in imperativer Form (Befehlsform).
\end{definition}

Die Syntax ist im Vergleich zu anderen Programmiersprachen eher einfach
zu lernen.
Durch die Vielzahl an verfügbaren Software-Bibliotheken, 
liegt für sehr viele Teilprobleme bereits eine fertige Lösung vor,
die in wenigen Schritten verwendet werden kann.
Die in diesem Kurs behandelten Themengebiete lassen sich also meist 
durch sehr kurze Programme ausdrücken.

Nachteilig ist die geringere Ausführungsgeschwindigkeit im Vergleich zu anderen
Programmiersprachen. Insbesondere bei selbst umgesetzten numerischen Verfahren ist dies
zu beachten.


%Vor allem im Bereich \textbf{Data Science} wird Python großflächig eingesetzt.


%\begin{remark}
%Diese Kurzeinführung gibt erste Einblicke in die grundlegenden Konzepte 
%der Programmierung mittels Python.
%Der der Fokus liegt auf den Inhalten dieses Mathematikonlinekurses.
%Auf weiterführende Aspekte, \textit{wie zum Beispiel objektorientierte Programmierung,}
%wird hier nicht weiter eingegangen.
%\end{remark}
%Sie finden am Ende vieler Kurskapitel \textbf{Programmierbeipiele}.
%Verlinkungen zu allen im Kurs integrierten Beispielen findet man HIER: TODO: kommt später :-)% \link{content_05_kursverlinkungen}{hier}.

\begin{example}
\begin{tabs*}[\initialtab{1}]

\tab{"Hallo Welt"} 
    Traditionell wird eine neue Programmiersprache über 
    ein "Hallo-Welt"-Programm eingeführt.
    Das folgende kurze Python-Programm gibt nach Ausführung den Text "Hello, World!" aus.\\

    (Im Abschnitt \textit{Werkzeuge} findet man Informationen
    darüber, wie man das Programm ausführen kann.)
\begin{block}[code]
print("Hello, World!")
\end{block}

\tab{Determinante}
    Bestimme die Determinante einer Matrix $A \in M(2;\C)$:
    \[
        A=
        \begin{pmatrix}
            1+2i & 3 \\
            4 & 5
        \end{pmatrix}
    \]
    \[
        det(A)
    \]
\begin{block}[code]
import numpy
A = [ [complex(1,2), 3], [4,5] ]
print( numpy.linalg.det( A ) )
\end{block}

\end{tabs*}
\end{example}


\subsection{Aufbau dieses Kursteils}

Der weitere Aufbau dieses Kursteils ist wie folgt:
\begin{itemize}
    \item \textbf{Kapitel 1} (dieses Kapitel) definiert den Begriff \textbf{Algorithmus}.
    \item \textbf{Kapitel 2} führt arithmetische \textbf{Ausdrücke}, sowie das Konzept der \textbf{Variablen} ein.
    \item \textbf{Kapitel 3} beschäftigt sich mit dem \textbf{Ablauf von Programmen}.
    \item \textbf{Kapitel 4} zeigt die grundlegenden mathematischen Software-\textbf{Bibliotheken} die hier relevant sind.
\end{itemize}
Themenspezifische Beispiele und Aufgaben sind über den gesamten Kurs verteilt.
Diese findet man meist am Ende eines Vorlesungskapitels.


\section{Algorithmen}

Innerhalb des Kurses haben wir bereits einige \textit{Rechenverfahren} kennengelernt.
Zum Beispiel können lineare Gleichungssysteme mit dem \textit{Gauß-Verfahren}
gelöst werden.

Nun sind wir daran interessiert, die Verfahren so exakt zu formulieren, 
dass diese auch von Rechnern verarbeitet werden können:

\begin{definition}[Algorithmus]
    Ein {Algorithmus} ist eine 
    \textbf{eindeutige Handlungsvorschrift}
    für eine gegebene Problemstellung.

    Er beschreibt \textbf{exakt} und in \textbf{Einzelschritten}, 
    wie eine gegebene \textit{Eingabe} in eine \textit{Ausgabe} überführt wird.
\end{definition}

Die folgenden Beispiele zeigen sehr einfache Algorithmen.

\begin{example}
\begin{tabs*}

\tab{Maximum}
Bestimmen des Maximum einer Menge bestehend aus ganzen Zahlen:
\begin{itemize}
    \item \textbf{Eingabe:} 
        \begin{itemize}
            %\item $x = \{x_1, x_2, \cdots, x_n\}, \quad x_{1 \leq i \leq n} \in \mathbb{Z}$
            \item $x = \{x_i ~|~ x_i \in \Z; 1 \leq i \leq n \}$
        \end{itemize}
    \item \textbf{Ausgabe:}
        \begin{itemize}
            \item $y = \max\{x\}$
        \end{itemize}
    \item \textbf{Wertspeicher (Variablen):}
        \begin{itemize}
            \item Anzahl der Elemente der Menge $n \in \N$.
            \item Elemente der Menge $x_1, x_2, \cdots, x_n$.
            \item (Zwischen-)Lösung $y \in \Z$.
            \item Hilfsvariable: 
                Index $i \in \N$ zur Adressierung von $x_i$ mit $1 \leq i \leq n$.
        \end{itemize}
    \item \textbf{Einzelschritte:}
        \begin{enumerate}
            \item Lese $n$ und $\{ x_1, x_2, \cdots, x_n \}$ ein.
            \item Setze die Zwischenlösung $y$ auf $-\infty$.
            \item Setze den Index $i$ auf $1$.
            \item Wenn $i$ größer als $n$ ist, dann gehe zu Schritt 8.
            \item Wenn $x_i$ größer als $y$ ist, dann setze $y := x_i$.
            \item Erhöhe $i$ um 1.
            \item Gehe zurück zu Schritt 4.
            \item Gebe $y$ aus.
            \item ENDE.
        \end{enumerate}
    \item \textbf{Ausblick:}\\
        In Python kann man beispielsweise das folgende Programm schreiben:
\begin{block}[code]
x = [ 3, 4, 43, 1, 4, 54 ]
y = max(x)
print(y)
\end{block}
        Ohne die Nutzung von \texttt{max} könnte die Lösung beispielsweise wie folgt aussehen:
\begin{block}[code]
x = [ 3, 4, 43, 1, 4, 54 ]
y = float("-inf")
i = 0
while i < len(x):
	if x[i] > y:
		y = x[i]
	i = i + 1
print(y)
\end{block}
\end{itemize}


%\tab{Schriftliche Addition}
%Schriftliche Addition zweier natürlicher Zahlen $a, b \in \mathbb{N}$:
%\begin{itemize}
%    \item \textbf{Eingabedaten:} $a, b \in \mathbb{N}$
%    \item \textbf{Ausgabedaten:} $c = a + b$
%    %\item \textbf{Algorithmus:} Funktion $f : \mathbb{N}^2 \to \mathbb{N}, \quad n \mapsto f(n), \quad f(a,b) = a + b$.
%\end{itemize}
%Der \textbf{Algorithmus} besteht aus den folgenden Einzelschritten.
%Wir nehmen im folgenden an, dass $100 \leq a, b < 1000$:
%\begin{enumerate}
%    \item Setze eine Hilfsvariable $k$ zur Beschreibung der akutell betrachteten Stelle auf den Wert 3. 
%    \item Addiere die beiden Ziffern der Stelle $k$.
%    \item Schreibe das Ergebnis XXX TODO
%\end{enumerate}

\tab{Fakultät}
Berechnung der Fakultät einer Zahl $n \in \mathbb{N}$:
\[ n! = \prod_{k=1}^{n} k \]
\begin{itemize}
    \item \textbf{Eingabe:}
        \begin{itemize}
            \item $n \in \mathbb{N}$
        \end{itemize}
    \item \textbf{Ausgabe:}
        \begin{itemize}
            \item $y = n!$
        \end{itemize}
    %\item \textbf{Algorithmus:}
    %    \begin{itemize}
    %        \item Funktion $f : \mathbb{N} \to \mathbb{N}, \quad n \mapsto f(n), \quad f(n) = 1 \cdot 2 \cdot \cdots \cdot n$
    %    \end{itemize}
    \item \textbf{Wertspeicher (Variablen)}:
    \begin{itemize}
        \item Eingabe $n \in \N$.
        \item Hilfsvariable $k \in \N$ für den Laufindex des Produkts.
    \end{itemize}
    \item \textbf{Einzelschrite:}
        \begin{enumerate}
            \item Lese $n$ ein.
            \item Setze die Zwischenlösung $y$ auf 1.
            \item Setze die Hilfsvariable $k$ auf 1.
            \item Wenn $k$ größer als $n$ ist, dann gehe zu Schritt 8.
            \item Setze $y_{neu}$ auf den Wert $y_{alt} \cdot k$.\\
                Hierzu schreiben wir kurz $y := y \cdot k$
            \item Erhöhe $k$ um 1.
            \item Gehe zu Schritt 4.
            \item Gebe das Ergebnis $y$ aus.
            \item ENDE.
        \end{enumerate}

    \item \textbf{Ausblick:}\\
        In Python kann man beispielsweise das folgende Programm schreiben:
\begin{block}[code]
import math
n = 4
y = math.factorial(n)
print(y)
\end{block}
        Ohne die Nutzung von \texttt{math.factorial} könnte die Lösung beispielsweise wie folgt aussehen:
\begin{block}[code]
n = 4
y = 1
for k in range(1, n+1):
    y = y * k
print(y)
\end{block}

\end{itemize}

\end{tabs*}
\end{example}



\begin{quickcheck}
  \type{input.number}
  \displayprecision{3}
  \correctorprecision{3}
  \begin{variables}
    \randint{num}{1}{6}
    \function[calculate]{ergebnis}{13}
  \end{variables}
  \text{
    Gegeben sei der folgende Algorithmus:
    \begin{enumerate}
        \item $x$ := 0, ~~ $y$ := 1, ~~ $z$ := 0.
        \item Wenn $z$ > 10, dann gehe zu Schritt 7.
        \item $z$ := $x$ + $y$.
        \item $x$ := $y$.
        \item $y$ := $z$.
        \item Gehe zu Schritt 3.
        \item Gebe $z$ aus.
    \end{enumerate}
    Welcher Wert wird ausgegeben? \ansref  
  }
  \explanation{
    Der Algorithmus berechnet die ersten Glieder der 
    \textit{Fibonacci-Folge}:
    Das jeweils nächste Glied berechnet sich aus der
    Summe der beiden vorangegangenen Glieder.
    Hier erhält der Wertspeicher $z$ nacheinander die Werte 
    0, 1, 1, 2, 3, 5, 8, 13.
    Sobald $z$ größer als 10 ist, 
    terminiert (= beendet sich) der Algorithmus.
    Es wird 13 ausgegeben.
  }
  \begin{answer}
    \solution{ergebnis}
  \end{answer}
\end{quickcheck}


\section{Programme}

Bisher haben wir Algorithmen als eine Folge von Einzelschritten angegeben.
Die Darstellungen waren noch nicht auf eine bestimmte Programmiersprache bezogen.
\begin{definition}[Programm]
    Ein \textbf{Programm} ist eine konkrete Umsetzung eines Algorithmus in einer 
    \textbf{Programmiersprache} in Form einer Folge von Anweisungen.
\end{definition}

Beispiele für Programmiersprachen sind C, C++, Java, JavaScript, Python, Rust usw.


\begin{example}
Das folgende \textbf{Programm} besteht aus 3 \textbf{Zeilen} korrektem Python-Code.
\begin{block}[code]
x = 4
y = 5
z = x*x + y
\end{block}
Bei Ausführung des Programmes werden die Zeilen im Allgemeinen von oben nach unten ausgewertet.
Auf der linken Seite steht jeweils ein Bezeicher für einen Wertspeicher (Variable).
Zunächst wird jeweils der Term auf der rechten Seite ausgewertet 
und der Ergebniswet in den Wertspeicher der linken Seite zugewiesen.

In diesem Beispiel sind die Schritte im Einzelnen:
\begin{enumerate}
    \item Weise der Variablen $x$ den Wert 4 zu.
    \item Weise der Variablen $y$ den Wert 5 zu.
    \item Weise der Variablen $z$ den Wert von $x^2 + y$, also $4^2 + 5 = 21$ zu.
\end{enumerate}
\end{example}

%\begin{remark}
%Im Unterkaptitel über \link{content_03_programmablauf}{Programmabläufe} werden wir alternative Folgen von Operationen kennenlernen.
%\end{remark}

\begin{block}[warning]
Hier wird \textbf{nicht} im mathematischen Sinne ein Gleichungssystem definiert und gelöst.
Auf dem Papier (nicht in Python selbst) schreibt man daher auch häufig $x := t$, 
um zu verdeutlichen, dass der aus Term $t$ berechnete Ergebniswert
den Wertspeicher $x$ geschrieben wird.
\end{block}

\begin{quickcheck}
    \type{input.number}
      \precision{3}
      \field{real}
      \begin{variables}
            \number{a}{7}
            \number{b}{18}
       \end{variables}
      \text{
Gegeben sei das folgende Python-Programm:
\begin{block}[code]
a = 4 + 2*3
b = (a-1) * 2 
a = a - 3
\end{block}
      Nach Ausführung enthält Variable $a$ den Wert \ansref und
      Variable $b$ den Wert \ansref.
        }
      \explanation{Die Zeilen von oben nach unten abarbeiten. Zunächst jeweils die
      Seite rechts des Gleichheitszeichens auswerten und dann dem Wertspeicher 
      der linken Seite zuweisen.}
      \begin{answer}
            \solution{a}
      \end{answer}
       \begin{answer}
            \solution{b}
      \end{answer}
\end{quickcheck}


%\begin{remark}
%Man bezeichnet eine Programmiersprache als \textbf{imperativ}, 
%wenn ein Programm als Folge von Anweisungen definiert ist,
%die nacheinander ausgeführt werden.
%Python ist ein Beispiel für eine imperative Sprache.
%\end{remark}



\section{Werkzeuge}

Es gibt vielfältige Möglichkeiten Python zu verwenden.

Während sich die hier im Kurs gestellten Programmieraufgaben 
direkt in den bereitgestellten Eingabefeldern lösen lassen, 
hat man im Allgemeinen die folgenden Optionen:
\begin{itemize}
    \item Nutzung von sogenannten \textbf{Online-Interpretern}.
    Diese können direkt im Browser, also ohne Installation auf dem Rechner,
    ausprobiert werden:
    \begin{itemize}
        \item \href{https://www.python.org/shell/}{https://www.python.org/shell/}
        \item \href{https://jupyter.org/try}{https://jupyter.org/try}
    \end{itemize}
    \item Download der Software von der \textbf{offiziellen Python-Seite} 
    mit anschließender Installation auf dem Rechner:
    \begin{itemize}
        \item \href{https://www.python.org/downloads/}{https://www.python.org/downloads/}
    \end{itemize}
    \item Download von \textbf{Distributionen} mit umfangreichen Softwarepaketen:
    \begin{itemize}
        \item Z.B. \href{https://www.anaconda.com}{Anaconda}.
        \item Eine vollständige Liste findet man \href{https://wiki.python.org/moin/PythonDistributions}{hier}.
    \end{itemize}
\end{itemize}

\begin{block}[warning]
In diesem Kurs wird Python in der \textbf{Versionsnummer 3} verwendet.
Programme die für die frühere Versionen 2 geschrieben wurden, 
sind teilsweise nicht kompatibel.

Python 2 wird seit längerem nicht mehr weiterentwickelt und 
sollte in der Praxis nicht weiter verwendet werden.
\end{block}



\section{Literatur}

Diese Einführung beschränkt sich auf wesentliche Grundlagen.
Sie können beispielsweise die folgenden Quellen nutzen, 
um sich tiefer in die Thematik einzuarbeiten.
Weiterhin findet man im Netz eine Vielzahl an kostenlosen Kursen.

\begin{itemize}
    \item Edmund Weitz, "Konkrete Mathematik (nicht nur) für Informatiker", Springer Spektrum, ISBN 978-3-658-21564-4, \href{https://doi.org/10.1007/978-3-658-21565-1}{DOI}.
    \item Python Software Foundation, "Python Documentation", \href{https://docs.python.org/3/}{Link}.
\end{itemize}


\end{visualizationwrapper}

\end{content}
