%$Id:  $
\documentclass{mumie.article}
%$Id$
\begin{metainfo}
  \name{
    \lang{de}{Bibliotheken}
    \lang{en}{...}
   }
  \begin{description} 
 This work is licensed under the Creative Commons License Attribution 4.0 International (CC-BY 4.0)   
 https://creativecommons.org/licenses/by/4.0/legalcode 

    \lang{de}{Bibliotheken}
    \lang{en}{...}
  \end{description}
  \begin{components}
  \end{components}
  \begin{links}
%\link{generic_article}{content/rwth/HM1/T901_Python/g_art_content_05_kursverlinkungen.meta.xml}{content_05_kursverlinkungen}
\end{links}
  \creategeneric
\end{metainfo}

\begin{content}
\begin{block}[annotation]
	Im Ticket-System: \href{https://team.mumie.net/issues/18347}{Ticket 18347}
\end{block}
\usepackage{mumie.ombplus}
%\usepackage{listings}
\ombchapter{12}
\ombarticle{1}
\usepackage{mumie.genericvisualization}

\begin{visualizationwrapper}

\begin{block}[annotation]
DIESER KURSTEIL BEFINDET SICH NOCH IM AUFBAU!
\end{block}

\lang{de}{\title{Bibliotheken}}

\begin{block}[info-box]
\tableofcontents
\end{block}

Für die Programmiersprache Python gibt es eine sehr große Zahl an
Software-Bibliotheken. Diese bieten z.B. die Möglichkeit Integrale
numerisch zu berechen oder den Rang einer Matrix zu bestimmen.

Bevor man also selbst einen Algorithmus umsetzt (was zu Lern-
zwecken empfehlenswert ist!), sollte man danach Ausschau halten,
ob bereits jemand anderes eine Lösung gefunden und bereit gestellt hat.

In diesem Kapitel werden die im mathematischen Kontext sehr häufig 
eingesetzten Bibliotheken \textbf{Numpy} und \textbf{Scipy} kurz 
vorgestellt.
%Für Details sei auf die zugehörigen Bibliothekedokumentation,
%sowie die weitern On- und Offline verfügbaren Quellen hingewiesen.
Man beachte auch über den Kurs verteilten Programmierbeispiele. 
Siehe dazu auch das Unterkapitel TODO: Link kommmt später :-)
%\link{content_05_kursverlinkungen}{Kursverlinkungen}.


\section{Einleitung}

Für die Nutzung von Bibliotheken müssen diese auf dem Rechner
bereits installiert sein. Die in der Einführung genannten Python-
Distributionen (z.B. Anaconda) liefern die erforderlichen Pakete meist mit.

Vor der Nutzung einer Bibliothek, muss diese zunächst 
durch das Schlüsselwort \texttt{import} in das aktuelle Programm eingebunden werden.

\begin{example}
Das folgende Programm bindet die Bibliothek "numpy" ein,
und berechnet $\sin(\frac{\pi}{2})$.
\begin{block}[code]
import numpy
print( numpy.sin( numpy.pi / 2) )
\end{block}
Ausgegeben wird \texttt{1.0}
\end{example}



\section{Numpy}

Die Bibliothek \textbf{Numpy} stellt insbesondere Datenstrukturen
und Funktionen für Vektoren und Matrizen bereit.

TODO

\section{Scipy}

TODO


\end{visualizationwrapper}

\end{content}



