%$Id:  $
\documentclass{mumie.article}
%$Id$
\begin{metainfo}
  \name{
    \lang{en}{...}
    \lang{de}{...}
   }
  \begin{description} 
 This work is licensed under the Creative Commons License Attribution 4.0 International (CC-BY 4.0)   
 https://creativecommons.org/licenses/by/4.0/legalcode 

    \lang{en}{...}
    \lang{de}{...}
  \end{description}
  \begin{components}
  \end{components}
  \begin{links}
    \end{links}
  \creategeneric
\end{metainfo}
\begin{content}
\begin{block}[annotation]
	Im Ticket-System: \href{https://team.mumie.net/issues/19291}{Ticket 19291}
\end{block}


\begin{block}[annotation]
    T401_01: Matrizenrechnung - Matrizen
\end{block}
\begin{block}[annotation]
    Die hiesigen Programmierbeispiele und -aufgaben werden später 
    in den Hauptkurs in die entsprechenden Kapitel am Ende eingebunden.
    Die Sichtbarkeit ist zurnächst "verborgen".
\end{block}


\textbf{Ausblick: Programmierung mit Python}
\begin{showhide}

\begin{tabs*}[\initialtab{1}\class{example}]

\tab{Einführung}
\begin{itemize}
  \item In \textbf{Python} kann eine Matrix durch die Datenstruktur 
    \textbf{array} aus der Bibliothek \href{https://numpy.org}{\textbf{NumPy}} ausgedrückt werden. 
  \item \textbf{Beispiel} zur Erzeugung zweier Matrizen 
    $A \in {M}(2,2;\mathbb{R})$, $B \in {M}(2,3;\mathbb{C})$:


\begin{block}[code]
import numpy
A = numpy.array( [[1, 2], [3, 4]] )
B = numpy.array( [[5+2.j, 6, 7], [8, 9-7.j, 10]] )
\end{block} 

\begin{block}[code-output]
\begin{equation*}
        A :=
        \begin{pmatrix}
            1 & 2\\
            3 & 4
        \end{pmatrix},\quad
        B :=
        \begin{pmatrix}
            5+2i & 6 & 7\\
            8 & 9-7i & 10
        \end{pmatrix}
    \end{equation*}
\end{block}


  \item Die Anzahl an Zeilen $m$ und Spalten $n$ von $A$ erhält man durch:

\begin{block}[code]
m, n = A.shape
\end{block} 

\end{itemize}


\tab{Skalare Multiplikation}

    Beispiel zur \textbf{skalaren Multiplikation} eines Skalars $a \in \R$
    und einer Matrix $B \in {M}(2,2;\mathbb{R})$:
    \begin{block}[code]
import numpy
a = 2
B = numpy.array( [[5, 6], [7, 8]] )
C = a * B
    \end{block}

    \begin{block}[code-output]
     \[
        C :=
        2
        \cdot
        \begin{pmatrix}
            5 & 6\\
            7 & 8
        \end{pmatrix}
        =
        \begin{pmatrix}
            10 & 12\\
            14 & 16
        \end{pmatrix}
    \]
    \end{block}
    

\tab{Addition und Multiplikation}

    Beispiel zur \textbf{Addition} und \textbf{Multiplikation} zweier Matrizen $A \in {M}(2,2;\mathbb{R})$, $B \in {M}(2,2;\mathbb{R})$:
    \begin{block}[code]
import numpy
A = numpy.array( [[1, 2], [3, 4]] )
B = numpy.array( [[5, 6], [7, 8]] )
C = A + B
D = A * B
    \end{block}
    \begin{block}[code-output]
     \[
        C :=
        \begin{pmatrix}
            1 & 2\\
            3 & 4
        \end{pmatrix}
        +
        \begin{pmatrix}
            5 & 6\\
            7 & 8
        \end{pmatrix}
        =
        \begin{pmatrix}
            6 & 8\\
            10 & 12
        \end{pmatrix}
    \]
     \[
        D :=
        \begin{pmatrix}
            1 & 2\\
            3 & 4
        \end{pmatrix}
        \cdot
        \begin{pmatrix}
            5 & 6\\
            7 & 8
        \end{pmatrix}
        =
        \begin{pmatrix}
            5 & 12 \\
            21 & 32
        \end{pmatrix}
    \]
    \end{block}
    
    \begin{block}[warning]
    Man erhält keine Fehlermeldung, wenn die Matrizen bezüglich der Operation
    nicht kompatibel zueinander sind.
    Für die Multipliation zweier Matrizen $A$ und $B$ kann man alternativ
    auch die Funktion \texttt{numpy.matmul(A,B)} nutzen. 
    Hier erhält man eine Meldung, wenn die Dimensionen nicht passen.
    \end{block}


\tab{Transponierte Matrix}

    Eine zu einer Matrix $A$ transponierte Matrix $A^T$
    erhält man wie folgt:
    \begin{block}[code]
import numpy
A = numpy.array( [[1, 2], [3, 4]] )
B = numpy.transpose( A )
    \end{block}
    \begin{block}[code-output]
     \[
        B :=
        \begin{pmatrix}
            1 & 2\\
            3 & 4
        \end{pmatrix}
        ^T
        =
        \begin{pmatrix}
            1 & 3\\
            2 & 4
        \end{pmatrix}
    \]
    \end{block}
    


% TODO: komplexe Matrix


\end{tabs*}

\end{showhide}



\end{content}

