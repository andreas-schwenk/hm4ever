\documentclass{mumie.problem.python}
%$Id$
\begin{metainfo}
  \name{
    \lang{en}{...}
    \lang{de}{...}
  }
  \begin{description} 
 This work is licensed under the Creative Commons License Attribution 4.0 International (CC-BY 4.0)   
 https://creativecommons.org/licenses/by/4.0/legalcode 

    \lang{en}{...}
    \lang{de}{...}
  \end{description}
  \corrector{system/problem/PythonCorrector.meta.xml}
  \begin{components}
  \end{components}
  \begin{links}
  \end{links}
  \creategeneric
\end{metainfo}
\begin{content}
\begin{block}[annotation]
	Im Ticket-System: \href{https://team.mumie.net/issues/20100}{Ticket 20100}
\end{block}

\title{Approximation: Newtonverfahren}

\begin{question}

Implementieren Sie eine \textbf{Funktion} \texttt{def newton(f, s)} in Python.
\begin{itemize}
    \item Sei der Parameter $f$ eine per \texttt{sympy} definierte symbolische Funktion. TODO: Link zur Sympy-Einführung im Kurs (noch nicht vorhanden) 
    \item Seit $s \in \R$ der Startwert (im Vorlesungsteil $x_0$).
    \item Zurückgegeben wird eine approxmierte Nullstelle $x_n \in \R$. 
\end{itemize}
Innerhalb der zu implementierenden Funktion wird $f$ zunächst symbolisch 
per \texttt{f.diff(x)} differenziert.
Ausgeführt werden sollen die ersten 100 Schritte.
Ein dynamisches Abbruchkriterium ist nicht gefordert.

\pythonanswer{int_sinc}

\begin{pythonsolution}{int_sinc}   
from sympy import *

def newton(f, s):
        xn = start
        fd = f.diff()
        for i in range(0, 100):
                n = f.subs(x, xn).evalf()
                d = fd.subs(x, xn).evalf()
                q = n / d
                xn = xn - q
        return xn
\end{pythonsolution}

\begin{pythonevaluator}
x = Symbol('x')
f = x - exp(-x)
if abs(newton(f, 0.0) - 0.56714) > 1e-3:
    exit(11)
exit(0)                                                   	
\end{pythonevaluator}

\pythongrading{11}{0.5}{Dein Code liefert das falsche Ergebnis}

\end{question}


\end{content}
