\documentclass{mumie.problem.python}
%$Id$
\begin{metainfo}
  \name{
    \lang{en}{...}
    \lang{de}{...}
  }
  \begin{description} 
 This work is licensed under the Creative Commons License Attribution 4.0 International (CC-BY 4.0)   
 https://creativecommons.org/licenses/by/4.0/legalcode 

    \lang{en}{...}
    \lang{de}{...}
  \end{description}
  \corrector{system/problem/PythonCorrector.meta.xml}
  \begin{components}
  \end{components}
  \begin{links}
  \end{links}
  \creategeneric
\end{metainfo}
\begin{content}
\begin{block}[annotation]
	Im Ticket-System: \href{https://team.mumie.net/issues/19325}{Ticket 19325}
\end{block}

\title{Integration: Mittelpunktsregel}

\begin{question}

Einige Funktionen lassen sich nur sehr aufwändig oder gar nicht explizit integrieren.
In dieser Programmieraufgabe berechnen Sie Integrale \textbf{numerisch}.

Ein Integral 
\[I = \int_a^b f(x) \,dx\]
kann durch die \textbf{Mittelpunktsregel} wie folgt approximativ gelöst werden:
\[ I \approx M = h \sum_{i=0}^{n-1} f\left(a+\left(i+0.5\right) \cdot h\right) \]
Dabei beschreibt $n \in \N$ die Anzahl der Schritte und $h := \frac{b-a}{n}$ die Schrittweite.

Implementieren Sie eine \textbf{Funktion} \texttt{def int_sinc(a, b, n)} in Python, 
um das folgende Integral nährungsweise zu bestimmen:
\[ \int_a^b \frac{sin(x)}{x} dx \]
.

\pythonanswer{int_sinc}

\begin{pythonsolution}{int_sinc}   
import numpy

def int_sinc(a, b, n):
    h = (b-a)/n
    M = 0
    for i in range(0, n):
        x = a + (i+0.5)*h
        fx = numpy.sin(x) / x
        M = M + fx
    M = h * M
    return M
\end{pythonsolution}

\begin{pythonevaluator}
if abs(int_sinc(0,1,10000) - 0.9460830) > 1e-3:
    exit(11)
exit(0)                                                   	
\end{pythonevaluator}

\pythongrading{11}{0.5}{Dein Code liefert das falsche Ergebnis}

\end{question}


\end{content}
