\documentclass{mumie.problem.python}
%$Id$
\begin{metainfo}
  \name{
    \lang{en}{...}
    \lang{de}{...}
  }
  \begin{description} 
 This work is licensed under the Creative Commons License Attribution 4.0 International (CC-BY 4.0)   
 https://creativecommons.org/licenses/by/4.0/legalcode 

    \lang{en}{...}
    \lang{de}{...}
  \end{description}
  \corrector{system/problem/PythonCorrector.meta.xml}
  \begin{components}
  \end{components}
  \begin{links}
  \end{links}
  \creategeneric
\end{metainfo}
\begin{content}

\begin{block}[annotation]
DIESE DATEI BITTE LÖSCHEN. MIR (A.Schwenk) FEHLT DAZU LEIDER DIE BERECHTIGUNG.
\end{block}

\begin{block}[annotation]
	Im Ticket-System: \href{https://team.mumie.net/issues/19326}{Ticket 19326}
\end{block}

\title{Signum Funktion}

\begin{question}

Die \textbf{Vorzeichenfunktion} für reelle Zahlen ist wie folgt definiert:
\[
    sgn(x) =
    \begin{cases}
        1 & \text{ falls } x > 0 \\
        0 & \text{ falls } x = 0 \\
        -1 & \text{ sonst }
    \end{cases}
\]

Implementieren Sie eine entsprechende \textbf{Funktion} in Python.
Wählen Sie \textbf{sgn} als Namen für Ihre Funktion.

\pythonanswer{sgn}

\begin{pythonsolution}{sgn}       
def sgn(x):
    if x > 0:
        return 1
    elif x == 0:
        return 0
    else:
        return -1
\end{pythonsolution}

\begin{pythonevaluator}
if sgn(1) != 1 or sgn(100) != 1 or sgn(0) != 0 or sgn(-1) != -1 or sgn(-25) != -1:
    exit(11)
exit(0)                                                   	
\end{pythonevaluator}

\pythongrading{11}{0.5}{Dein Code liefert das falsche Ergebnis}

\end{question}


\end{content}

