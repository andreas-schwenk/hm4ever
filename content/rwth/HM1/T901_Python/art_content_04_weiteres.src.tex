%$Id:  $
\documentclass{mumie.article}
%$Id$
\begin{metainfo}
  \name{
    \lang{de}{Weiterführende Themen}
    \lang{en}{...}
   }
  \begin{description} 
 This work is licensed under the Creative Commons License Attribution 4.0 International (CC-BY 4.0)   
 https://creativecommons.org/licenses/by/4.0/legalcode 

    \lang{de}{Weiterführende Themen}
    \lang{en}{...}
  \end{description}
  \begin{components}
  \end{components}
  \begin{links}
    \end{links}
  \creategeneric
\end{metainfo}

\begin{content}
\begin{block}[annotation]
	Im Ticket-System: \href{https://team.mumie.net/issues/18827}{Ticket 18827}
\end{block}

\usepackage{mumie.ombplus}
%\usepackage{listings}
\ombchapter{12}
\ombarticle{1}
\usepackage{mumie.genericvisualization}

\begin{visualizationwrapper}

\lang{de}{\title{Weiterführende Themen}}

\begin{block}[info-box]
\tableofcontents
\end{block}

Dieses Kapitel gibt einen kurzen Ausblick über wichtige weiterführende Themen.

\section{Listen}

\begin{definition}[Liste]
    Eine \textbf{Liste} ist eine dynamische Datenstrukturen zur Speicherung von
    Datenelementen.
\end{definition}

TODO

\begin{remark}
    Listen können auch wie folgt in for-Schleifen verwendet werden:
\begin{block}[code]
v = [21, 22, 23]
for x in v:
    print( x*2 )
\end{block}
Das Programm gibt $42, 44, 46$ aus.
\end{remark}


TODO: Zwei Listen konkatenieren, ...:
% [1,2,3] + [4,5,6]
% ->  [1, 2, 3, 4, 5, 6]



\section{Numerische Instablität}

Im Rahmen der Programmierung ist stets darauf zu achten,
dass Rechner die reellen Zahlen $\R$ nicht exakt darstellen können,
sondern nur als \textbf{Fließkommazahlen} mit \textbf{endlicher} Genauigkeit
gespeichern.

\begin{block}[warning]
    Für den numerischen Vergleich zweier Zahlen $a, b \in \mathbb{R}$ sollte statt
    $a == b$ immer \[ |a-b| < \varepsilon \] geprüft werden. 
    In Python schreibt man dann:
\begin{block}[code]
if abs(a-b) < 1e-15:
    print(" a und b sind numerisch gleich")
\end{block}
\end{block}

\begin{remark}
Die Wahl von $\varepsilon = 10^{-k}$ hängt von der jeweiligen Maschine, 
sowie vom gewählen Datentyp ab. In der Regel werden Gleitkommazahlen 
heute als 64-bit Werte gespeichert.
In diesem Fall ist die Wahl von $k = 15$ üblich. % TODO: 14??
\end{remark}


TODO: numerische Verfahren / Instabilität (Beispiel Skript Stoffel / PNV)




%%% SONSTIGES (IN DIESES ODER ANDERE KAPITEL EINSORTIEREN:)

%\begin{remark}
%    Schleifen können vorzeitig mit dem Schlüsselwort \texttt{break} abgebrochen werden.
%    TODO: Beispiel
%\end{remark}



\end{visualizationwrapper}

\end{content}
