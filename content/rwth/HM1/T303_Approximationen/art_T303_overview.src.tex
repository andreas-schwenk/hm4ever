%$Id:  $
\documentclass{mumie.article}
%$Id$
\begin{metainfo}
  \name{
    \lang{de}{Überblick: Approximationsverfahren}
    \lang{en}{Overview: Approximation techniques}
  }
  \begin{description} 
 This work is licensed under the Creative Commons License Attribution 4.0 International (CC-BY 4.0)   
 https://creativecommons.org/licenses/by/4.0/legalcode 

    \lang{de}{Beschreibung}
    \lang{en}{Description}
  \end{description}
  \begin{components}
  \end{components}
  \begin{links}
\link{generic_article}{content/rwth/HM1/T303_Approximationen/g_art_content_06_de_l_hospital.meta.xml}{content_06_de_l_hospital}
\link{generic_article}{content/rwth/HM1/T303_Approximationen/g_art_content_05_newtonverfahren.meta.xml}{content_05_newtonverfahren}
\link{generic_article}{content/rwth/HM1/T303_Approximationen/g_art_content_04_taylor_polynom.meta.xml}{content_04_taylor_polynom}
\end{links}
  \creategeneric
\end{metainfo}
\begin{content}
\begin{block}[annotation]
	Im Ticket-System: \href{https://team.mumie.net/issues/30123}{Ticket 30123}
\end{block}
\begin{block}[annotation]
Copy of : /home/mumie/checkin/content/rwth/HM1/T304_Integrierbarkeit/art_T304_overview.src.tex
\end{block}






\begin{block}[annotation]
Im Entstehen: Überblicksseite für Kapitel  Approximationsverfahren
\end{block}

\usepackage{mumie.ombplus}
\ombchapter{1}
\title{\lang{de}{Überblick: Approximationsverfahren}\lang{en}{Overview: Approximation techniques}}




\begin{block}[info-box]
\lang{de}{\strong{Inhalt}}
\lang{en}{\strong{Contents}}


\lang{de}{
    \begin{enumerate}%[arabic chapter-overview]
   \item[2.1] \link{content_04_taylor_polynom}{Taylor-Polynom und Restglied}
   \item[2.2] \link{content_05_newtonverfahren}{Newton-Verfahren}
   \item[2.3] \link{content_06_de_l_hospital}{Regel von de l'Hospital}
   \end{enumerate}
}
\lang{en}{
    \begin{enumerate}%[arabic chapter-overview]
   \item[2.1] \link{content_04_taylor_polynom}{Taylor expansions}
   \item[2.2] \link{content_05_newtonverfahren}{Newton's method}
   \item[2.3] \link{content_06_de_l_hospital}{L'Hopital's rules}
   \end{enumerate}
} %lang

\end{block}

\begin{zusammenfassung}

\lang{de}{
Approximationen treten in der Mathematik sehr häufig auf. Eigentlich besteht die ganze Analysis 
aus Näherungsprozessen. Approximation ist oft die einzige praktische Lösung, wenn die ideale Lösung 
nicht erreichbar oder schlicht zu teuer ist. Die Mathematik braucht man dann auch, um abschätzen zu 
können, welchen Fehler man dabei in Kauf nehmen muss. Wir behandeln hier drei Approximationsthemen, 
die für die Anwendung besondere Bedeutung haben und die mit den bisher erlernten mathematischen 
Werkzeugen verfügbar sind.
\\\\
Die Taylor-Approximation nähert Funktionen durch Polynome an. Dabei wird der Gedanke, dass die 
Tangente an eine Funktion diese nahe dem Aufpunkt annähert, für höhere Ableitungen weiterentwickelt. 
Sie lernen Taylor-Polynome und Taylor-Reihen kennen sowie Restglied-, also Fehlerabschätzungen.\\
Die Taylor-Approximation wird in der Praxis sehr oft verwendet. Übrigens: Unsere klassische Mechanik 
ist nichts anderes als die  Taylor-Approximation zweiter Ordnung der speziellen Relativitätstheorie.
\\\\
Das Newton-Verfahren bietet die Möglichkeit, Nullstellen einer Funktion effektiv schnell bis auf sehr 
viele Nachkommastellen genau zu berechnen. Hier werden iterativ immer wieder Tangenten an die 
Funktion angelegt, also auch von deren Differenzierbakeit ausgegangen.
\\\\
Zuletzt bietet sich mit der Regel von de l'Hospital die Möglichkeit, Funktionsgrenzwerte oft 
wesentlich zu vereinfachen. Eigentlich ist diese Regel nur eine Anwendung der Taylor-Approximation.
}
\lang{en}{
Approximations frequently appear in the world of mathematics. In fact, the entirety of Analysis is 
based on limiting processes. Approximation is often the only practical alternative when the exact 
solution to a problem is unattainable or computationally expensive to compute. Rigorous mathematics 
must in these cases be used to determine what error is to be expected from such approximations. In 
this chapter we handle three particularly useful methods in which approximation is used, and which 
can be understood using the mathematical machinery that has been covered so far.
\\\\
Taylor approximation is a way of approximating functions using polynomials. The idea behind this is 
to extend to higher derivatives the notion that the tangent of a function approximates the function 
in a small neighbourhood of a point. Taylor series, Taylor polynomials and their remainders are all 
introduced.\\
Taylor approximation is used in many applications. In fact: classical mechanics is, in a sense, 
the second order Taylor approximation of special relativity.
\\\\
Newton's method gives us the ability to compute roots of a function very efficiently and accurately 
to many decimal places. It is an iterative process involving finding tangents of the function, and 
therefore also has a differentiability condition.
\\\\
Finally we consider l'Hopital's rules, which allow us to compute limits of functions that would 
otherwise be out of reach. We see that these rules are nothing more than an application of Taylor 
approximations.
}


\end{zusammenfassung}

\begin{block}[info]\lang{de}{\strong{Lernziele}}
                   \lang{en}{\strong{Learning Goals}} 
\begin{itemize}[square]
\item \lang{de}{
      Sie kennen die Begriffe Entwicklungsstelle, Taylor-Polynom $n$-ter Ordnung und Restglied.
      }
      \lang{en}{
      Knowing the definitions of $n$th order Taylor polynomials expanded at a given point and their 
      remainders.
      }
\item \lang{de}{Sie berechnen Taylor-Polynome und Restglieder.}
      \lang{en}{Being able to calculate Taylor polynomials and their remainders.}
\item \lang{de}{
      Sie schätzen das Taylor-Restglied nach Lagrange in konkreten Situationen explizit ab.
      }
      \lang{en}{
      Being able to approximate the remainder of a Taylor polynomial by the Lagrange remainder.
      }
\item \lang{de}{
      Sie kennen den Begriff der Taylor-Reihe mit Beispielen und stellen einfache Taylor-Reihen 
      selbst auf.
      }
      \lang{en}{
      Knowing the definition of a Taylor series with examples and being able to determine some simple 
      Taylor series.
      }
\item \lang{de}{
      Sie wenden die Newton-Rekursionsformel zur Nullstellenberechnung sicher an. Sie kennen Gründe, 
      auf die ein Versagen des Verfahrens ggf. zurückgeführt werden kann.
      }
      \lang{en}{
      Being able to apply Newton's method to find roots of a function. Knowing reasons for which the 
      method may fail.
      }
\item \lang{de}{
      Sie kennen die verschiedenen Ausführungen der de l'Hospitalschen Regel. Sie wissen, unter 
      welchen Voraussetzungen sie angewendet werden darf, und Sie wenden sie sicher an.
      }
      \lang{en}{
      Knowing the different versions of l'Hopital's rule. Knowing under which conditions they may be 
      used, and being able to apply the rules.
      }
\end{itemize}
\end{block}




\end{content}
