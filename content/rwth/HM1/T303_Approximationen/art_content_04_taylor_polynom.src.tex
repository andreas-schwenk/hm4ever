%$Id:  $
\documentclass{mumie.article}
%$Id$
\begin{metainfo}
  \name{
    \lang{de}{Taylor-Entwicklung}
    \lang{en}{Taylor expansions}
  }
  \begin{description} 
 This work is licensed under the Creative Commons License Attribution 4.0 International (CC-BY 4.0)   
 https://creativecommons.org/licenses/by/4.0/legalcode 

    \lang{de}{Beschreibung}
    \lang{en}{Description}
  \end{description}
  \begin{components}
\component{generic_image}{content/rwth/HM1/images/g_tkz_T303_TaylorCosine.meta.xml}{T303_TaylorCosine}
\component{generic_image}{content/rwth/HM1/images/g_tkz_T303_TaylorParabola.meta.xml}{T303_TaylorParabola}
\component{generic_image}{content/rwth/HM1/images/g_tkz_T303_TaylorSine.meta.xml}{T303_TaylorSine}
\component{generic_image}{content/rwth/HM1/images/g_tkz_T303_TrigApprox_B.meta.xml}{T303_TrigApprox_B}
\component{generic_image}{content/rwth/HM1/images/g_tkz_T303_TrigApprox_A.meta.xml}{T303_TrigApprox_A}
\component{generic_image}{content/rwth/HM1/images/g_img_00_Videobutton_schwarz.meta.xml}{00_Videobutton_schwarz}
\component{generic_image}{content/rwth/HM1/images/g_img_00_video_button_schwarz-blau.meta.xml}{00_video_button_schwarz-blau}
\end{components}
  \begin{links}
    \link{generic_article}{content/rwth/HM1/T207_Intervall_Schachtelung/g_art_content_21_intervalle.meta.xml}{content_21_intervalle}
    \link{generic_article}{content/rwth/HM1/T301_Differenzierbarkeit/g_art_content_02_ableitungsregeln.meta.xml}{content_02_ableitungsregeln}
    \link{generic_article}{content/rwth/HM1/T303_Approximationen/g_art_content_04_taylor_polynom.meta.xml}{content_04_taylor_polynom}
    \link{generic_article}{content/rwth/HM1/T209_Potenzreihen/g_art_content_27_konvergenzradius.meta.xml}{content_27_konvergenzradius}
    \link{generic_article}{content/rwth/HM1/T201neu_Vollstaendige_Induktion/g_art_content_02_vollstaendige_induktion.meta.xml}{content_02_vollstaendige_induktion}
    %\link{generic_article}{content/rwth/HM1/T503_Differentialgleichungen/g_art_content_57_gewoehnliche_DGL_erster_Ordnung.meta.xml}{content_57_gewoehnliche_DGL_erster_Ordnung}
    \link{generic_article}{content/rwth/HM1/T103_Polynomfunktionen/g_art_content_10_polynomdivision.meta.xml}{content_10_polynomdivision}
    \link{generic_article}{content/rwth/HM1/T209_Potenzreihen/g_art_content_28_exponentialreihe.meta.xml}{content_28_exponentialreihe}
    \link{generic_article}{content/rwth/HM1/T301_Differenzierbarkeit/g_art_content_01_differenzenquotient.meta.xml}{diffbarkeit}
    \link{generic_article}{content/rwth/HM1/T211_Eigenschaften_stetiger_Funktionen/g_art_content_34_exp_und_log.meta.xml}{exp-und-log}
    \link{generic_article}{content/rwth/HM1/T303_Approximationen/g_art_content_06_de_l_hospital.meta.xml}{l-hospital}
  \end{links}
  \creategeneric
\end{metainfo}
\begin{content}
\usepackage{mumie.ombplus}
\ombchapter{2}
\ombarticle{1}

\title{\lang{de}{Taylor-Entwicklung}\lang{en}{Taylor expansions}}
 
\begin{block}[annotation]
  
  
\end{block}
\begin{block}[annotation]
  Im Ticket-System: \href{http://team.mumie.net/issues/10035}{Ticket 10035}\\
\end{block}
\begin{block}[annotation]
  ToDo wen JSX-Graphs auf den Servern läuft: Kleinwinkelnäherung neu einpflegen.\\
\end{block}

\begin{block}[info-box]
\tableofcontents
\end{block}

\lang{de}{
Häufig ist eine durch die Analysis bereitgestellte Lösungsfunktion in der Praxis nur schwer zu 
handhaben, weil die beteiligten Gleichungen zu kompliziert sind oder die explizite Berechnung des 
Funktionsausdrucks zu teuer oder unmöglich ist. Dann ist eine näherungsweise Darstellung der 
Funktion durch möglichst einfache Ausdrücke wünschenswert.
\\\\
Wir betrachten in diesem Abschnitt ein Approximationsverfahren, die \notion{Taylor-Entwicklung}, das 
eine solche Näherungsdarstellung mit Hilfe von Polynomen erzeugt. Polynome haben den Vorteil, dass 
sie sich einfach ableiten und integrieren lassen und dass es effektive Auswertungswerkzeuge für sie 
gibt, zum Beispiel das \ref[content_10_polynomdivision][Horner-Schema]{sec:poly-div-mit-horner}. 
% Referenz kontrollieren, wenn Polynom-Kapitel fertig entwickelt ist.
Man verwendet die sogenannten \notion{Taylor-Polynome}, um Funktionen in der Umgebung einer Stelle 
$x_0$ zu approximieren, und spricht dann auch von \notion{Taylor-Näherung}.
\\\\
Ein Beispiel für eine Taylor-Näherung ist die Kleinwinkelnäherung, die sich zunächst durch eine 
einfache Betrachtung am Einheitskreis finden lässt. Für große Winkel weichen $\sin x$ und $\tan x$ 
deutlich von $x$ ab. Für kleine Winkel $x$ sind sie Näherungsweise gleich diesem.
}
\lang{en}{
Often a function defined by its own properties is difficult to work with in practice, simply 
because its definition may be complicated, or it may be computationally difficult to even compute 
the values of the function. In these cases, an approximation to the function using some simpler 
functions is useful.
\\\\
In this section we consider a method of approximating functions, the so-called 
\notion{Taylor series}, which give such an approximate representation of a function in terms of 
polynomials. Polynomials have the advantage of being easy to differentiate and integrate, and 
computationally efficient to evaluate, for example using 
\ref[content_10_polynomdivision][Horner's method]{sec:poly-div-mit-horner}. 
So-called \notion{Taylor polynomials} are used to approximate functions in a neighbourhood of a 
point $x_0$, in a process sometimes called finding the \notion{Taylor expansion} of the function at 
the point.
\\\\
An example of a Taylor approximation is the approximation for small angles that can be justified 
using a simple visual argument on the unit circle. For larger angles, $\sin x$ and $\tan x$ diverge 
considerably from $x,$ but for small angles $x$ they are approximated well by $x$ itself.
}
\begin{table}[\class{item}]
 \lang{de}{\image{T303_TrigApprox_A}} & \lang{de}{\image{T303_TrigApprox_B}}
 \lang{en}{\image{T303_TrigApprox_A}} & \lang{en}{\image{T303_TrigApprox_B}}
\end{table}
%Abbildungen werden durch einen jsx-graph ersetzt, wenn diese auf den normalen Servern laufen.
\lang{de}{
Dabei wird angenommen, dass der Winkel $x$ so klein ist, dass man dessen Sinus oder Tangens durch 
den Winkel selbst ersetzen kann
}
\lang{en}{
Clearly the assumption being made for this approximation is that the angle $x$ is so small that its 
sine or cosine may be approximated by the angle itself:
}
\[\sin x\approx x\quad \text{\lang{de}{ bzw. }\lang{en}{ or }}\quad\tan x\approx x.\]
\lang{de}{
Wir ersetzen also die Funktionen $\sin x$ bzw. $\tan x$ durch das Polynom $x$ vom Grad eins.
Wir werden sehen, dass die Kleinwinkelnäherung nichts anderes ist als die nach dem ersten Glied 
abgebrochene Taylor-Entwicklung des Sinus bzw. Tangens an der Stelle $x_0=0$. Diese ist wiederum 
identisch mit der ersten Partialsumme der 
\ref[content_28_exponentialreihe][Potenzreihe des Sinus]{sec:sinus-kosinus} 
bzw. Tangens. Der Zusammenhang zwischen Potenzreihenentwicklung und Taylor-Reihenentwicklung ergibt 
sich im weiteren Verlauf dieses Abschnitts.
\\\\
Anschaulich handelt es sich bei dieser ersten Näherung um die Tangente an den Funktionsgraphen, die
bereits eine gute Näherung einer differenzierbaren Funktion $f$ in einer ausreichend kleinen 
Umgebung der Stelle $x_0=0$ darstellt.
}
\lang{en}{
We replace the function $\sin x$ or the function $\tan x$ by the polynomial $x$ of degree one. We 
will see that this small angle approximation is nothing more than the Taylor series of the sine or 
cosine function about the point $x_0$ respectively truncated after their first terms. This is 
in fact identical to the first partial sum of the 
\ref[content_28_exponentialreihe][power series of the sine function]{sec:sinus-kosinus} 
or tangent function respectively. The relationship between power series and Taylor series is 
revealed later in this section.
\\\\
Visually, this first approximation of the sine graph corresponds to the tangent of the sine graph 
at the point $x_0=0$, which approximates the graph well in a neighbourhood of $0$.
}
\begin{center}
\image{T303_TaylorSine}
\end{center}

\lang{de}{
Anwendung findet diese Näherung kleiner Winkel zum Beispiel beim (mathematischen) Fadenpendel. 
% Referenz setzen, wenn Teil Mehrdimensionale Analysis eingebunden wird: \ref[content_57_gewoehnliche_DGL_erster_Ordnung][(mathematischen) Fadenpendel]{ex:fadenpendel}
Dort wird eine elementar nicht lösbare nichtlineare Differentialgleichung in Näherung
durch eine explizit lösbare lineare Differentialgleichung ersetzt, 
deren Lösung die Bewegung des Fadenpendels für kleine Auslenkungen sehr gut approximiert.
\\\\
Die Näherung einer mehrfach differenzierbaren Funktion $f$ lässt sich dadurch verbessern, dass wir 
den Grad des Taylor-Polynoms erhöhen. Letzteres werden wir nun erst einmal definieren.
}
\lang{en}{
This approximation has an application when modelling a (mathematical) pendulum. 
A non-solvable non-linear differential equation can be replaced by a solvable approximation, whose 
solution accurately describes the movement of the pendulum when a small pertubation is applied to it.
\\\\
The approximation of a multiple times differentiable function $f$ may be improved by increasing the 
degree of the Taylor polynomial used in the approximation. We now give a definition of the latter.
}




\section{\lang{de}{Taylor-Polynom}\lang{en}{Taylor polynomial}}\label{sec:polynom}

\begin{definition}[\lang{de}{Taylor-Polynom $\text{n}$-ter Ordnung}
                   \lang{en}{$n$th order Taylor polynomial}]\label{def:taylorpoly}

\lang{de}{
Sei $f:[a,b]\rightarrow\R$ eine im Innern des Intervalls $[a,b]\subset\R\;$ $n$-mal differenzierbare Funktion, wobei $n\in\N_0$ ist. 
Dann heißt das Polynom
}
\lang{en}{
Let $f:[a,b]\rightarrow\R$ be a function that is $n$ times differentiable on the interior of the 
interval $[a,b]\subset\R\;$ for some $n\in\N_0$. 
Then the polynomial
}
%\lang{en}{Let $n \in \N_0$  and let $f$ be a 
%real valued function defined on the interval $(b,c)$.  Assuming that $f$ is $n$ times differentiable at 
%$\,a \in (b,c)\,$ we denote by$\, f^{(k)}(a)\,$ the \nowrap{$k$-th} derivative of $\, f\,$ at $\, a, \; 0 \leq k \leq n$.}
%\begin{eqnarray*}
  \[ T_n(x)\coloneq T_{f,x_0,n}(x) \coloneq\sum_{k=0}^{n} \frac{f^{(k)}(x_0)}{k!}(x-x_0)^k \] %\\
 %       &=& f(x_0)+\frac{f'(x_0)}{1!}(x-x_0)+\frac{f''(x_0)}{2!}(x-x_0)^2+\ldots +\frac{f^{(n)}(x_0)}{n!}(x-x_0)^n
%\end{eqnarray*}
%  \lang{en}{is called \notion{Taylor Polynomial} of $f$ of order $n$ at $a$.\\
%   The coefficients of the Taylor Polynomial are called \notion{Taylor coefficients}.}
\lang{de}{
\notion{Taylor-Polynom $n$-ter Ordnung} von $f$ zur \notion{Entwicklungsstelle} 
$x_0\in (a,b)$.\\\newline
Dabei bezeichnet $f^{(k)}(x_0)$ die $k$-te Ableitung von $f$ an der Entwicklungsstelle $x_0$ für 
$0 \leq k \leq n$.\\
Die Koeffizienten $\frac{f^{(k)}(x_0)}{k!}$ des Taylor-Polynoms heißen 
\notion{Taylor-Koeffizienten}.\\
\floatright{\href{https://api.stream24.net/vod/getVideo.php?id=10962-2-10738&mode=iframe&speed=true}{\image[75]{00_video_button_schwarz-blau}}\href{https://www.hm-kompakt.de/video?watch=536}{\image[75]{00_Videobutton_schwarz}}}
}
\lang{en}{
is called the \notion{$n$th order Taylor polynomial} of $f$ at a point $x_0\in (a,b)$.\\\newline
Here $\frac{f^{(k)}(x_0)}{k!}$ denotes the $k$th derivative of $f$ at the point $x_0$ for 
$0 \leq k \leq n$.\\
The coefficients $\frac{f^{(k)}(x_0)}{k!}$ of the Taylor polynomial are called \notion{Taylor 
coefficients}.
}
\end{definition}

%Alternative Erklärungen des Taylor-Polynoms aus Definition \ref{def:taylorpoly} können den folgenden Videos entnommen werden:\\

\begin{remark}
\begin{itemize}
\item[(i)] \lang{de}{
Das Taylor-Polynom $T_n$ der Ordnung $n$ einer Funktion ist ein Polynom vom Grad höchstens $n$. 
Ist die $n$-te Ableitung von $f$ an der Stelle $x_0$ gleich Null, $f^{(n)}(x_0)=0$,
so hat es echt einen kleineren Grad als Ordnung. Das Taylor-Polynom $n$-ter Ordnung zieht die 
Ableitungen der Funktion kleiner gleich $n$-ter Ordnung heran.
}
\lang{en}{
The $n$th order Taylor polynomial $T_n$ of a function $f$ is a polynomial of degree at most $n$. If 
the $n$th derivative of $f$ at the point $x_0$ is equal to zero, $f^{(n)}(x_0)=0$, then $T_n$ has a 
degree smaller than the order. The expression for the $n$th order Taylor polynomial is built using 
the higher derivatives of the function $f$ up to the $n$th of these.
}
\item[(ii)] \lang{de}{
Wir unterdrücken in der Notation $T_n=T_{f,x_0,n}$ im folgenden der Lesbarkeit halber meist sowohl 
$f$ als auch $x_0$, wenn klar ist, um welche Funktion und welche Entwicklungsstelle $x_0$ es sich 
handelt. Dass aber gerade diese von Bedeutung für die Gestalt des Taylor-Polynoms ist, zeigt 
untenstehendes einfaches Beispiel für ein Polynom.
}
\lang{en}{
We will often surpress $f$ and $x_0$ in the notation for $T_n=T_{f,x_0,n}$ when it is clear which 
function and point $x_0$ we are referring to. The below example shows the importance of the latter 
to the Taylor polynomial obtained.
}
\end{itemize}
\end{remark}

\begin{example}\label{ex:taylorpoly-exp}\label{ex:taylorpoly-geom-reihe}\label{ex:taylorpoly-ln}
\begin{tabs*}[\initialtab{0}]
\tab{\lang{de}{Polynom}\lang{en}{Polynomial}}
\lang{de}{
Wir bestimmen die Taylor-Polynome erster Ordnung der Funktion $f:\R\to\R$, $f(x)=x^2$, zu den zwei 
Entwicklungsstellen $x_0=0$ und $x_0=1$. Für $x_0=0$ finden wir
}
\lang{en}{
We determine the first order Taylor polynomials of the function $f:\R\to\R$, $f(x)=x^2$, at the two 
points $x_0=0$ und $x_0=1$. For $x_0=0$ we find
}
\[T_{f,0,1}(x)=f(0)+f'(0)\cdot(x-0)=0+0\cdot x=0,\]
\lang{de}{
also die Nullfunktion. Insbesondere hat es nicht den Grad Eins, sondern den Grad Null.\\
Für $x_0=1$ erhalten wir
}
\lang{en}{
i.e. the zero function. In particular, this is a polynomial of degree zero rather than degree one.\\
For $x_0=1$ we obtain
}
\[T_{f,1,1}(x)=f(1)+f'(1)\cdot(x-1)=1+2\cdot (x-1)\neq T_{f,0,1}(x).\]
\tab{\lang{de}{Natürliche Exponentialfunktion}
     \lang{en}{Natural exponent function}}%\label{ex:taylorpoly-exp}
\lang{de}{
Wir betrachten die Exponentialfunktion $f:\R\to\R$, $f(x)=e^x$, an der Entwicklungsstelle $x_0=0$.
Da sich $f$ beim Ableiten repliziert, $f^{(k)}(x)=e^x$ für alle $k\in\N_0$, und $f(0)=e^0=1$ ist, 
erhalten wir das Taylor-Polynom $n$-ter Ordnung an der Stelle $x_0=0$ zu
}
\lang{en}{
We consider the exponential function $f:\R\to\R$, $f(x)=e^x$ and the point $x_0=0$. As the 
derivative of $f$ is $f$ itself, $f^{(k)}(x)=e^x$ for all $k\in\N_0$, and as $f(0)=e^0=1$, we obtain 
for the $n$th order Taylor polynomial:
}
\[ T_n(x)=T_{f,0,n}(x)=\sum_{k=0}^n \frac{1}{k!}x^k .\]
%\end{example}
%\begin{example}
\tab{\lang{de}{Gebrochenrationale Funktion}
     \lang{en}{Fractional rational function}}%\label{ex:taylorpoly-geom-reihe}
\lang{de}{
Wir betrachten die Funktion $f:\R\setminus\{1\}\to\R$, $f(x)=\frac{1}{1-x}$ an der Stelle $x_0=0$.
Schreibt man $f(x)$ als $f(x)=(1-x)^{-1}$, erhält man mit der Kettenregel für die ersten Ableitungen
}
\lang{en}{
We consider the function $f:\R\setminus\{1\}\to\R$, $f(x)=\frac{1}{1-x}$ at the point $x_0=0$. 
If we write $f(x)$ as $f(x)=(1-x)^{-1}$, we may use the chain rule to obtain
}
\begin{eqnarray*}
 f'(x) &=& (-1)\cdot (-1)\cdot (1-x)^{-2}=(1-x)^{-2}, \\
f''(x) &=&  (-1)\cdot (-2)\cdot (1-x)^{-3}=2 (1-x)^{-3} ,\\
f'''(x) &=& 2\cdot (-1)\cdot (-3)\cdot (1-x)^{-4}=3!\cdot  (1-x)^{-4} ,
\end{eqnarray*}
\lang{de}{und induktiv dann allgemein}
\lang{en}{and by induction we have the general expression}
\[  f^{(k)}(x)= k! (1-x)^{-k-1}=\frac{k!}{(1-x)^{k+1}} .\]
\lang{de}{
Somit ist $f^{(k)}(0)=\frac{k!}{(1-0)^{k+1}}=k!$, und daher das Taylor-Polynom $n$-ter Ordnung zum 
Entwicklungspunkt $x_0=0$ gegeben durch
}
\lang{en}{
Therefore $f^{(k)}(0)=\frac{k!}{(1-0)^{k+1}}=k!$ and hence the $n$th order Taylor polynomial at the 
point $x_0=0$ is given by
}
\[ T_n(x)=\sum_{k=0}^n \frac{k!}{k!}x^k=\sum_{k=0}^n x^k .\]
\tab{\lang{de}{Natürlicher Logarithmus}\lang{en}{Natural logarithm}}%\label{ex:taylorpoly-ln}
\lang{de}{
Wir betrachten die natürliche Logarithmusfunktion $f:\R_{>0}\to\R$, $x\mapsto \ln (x)$, an der 
Entwicklungsstelle $x_0=1$. Die ersten Ableitungen ergeben sich zu
}
\lang{en}{
We consider the natural logarithm function $f:\R_{>0}\to\R$, $x\mapsto \ln (x)$, at the point 
$x_0=1$. The first few derivatives are given by
}
\begin{eqnarray*}
 f'(x) &=& \frac{1}{x}=x^{-1} ,\\
f''(x) &=&  -x^{-2},\\
f'''(x) &=& 2\cdot x^{-3},\\
f^{(4)}(x)&=&-6\cdot x^{-4}.
\end{eqnarray*}
\lang{de}{Daraus findet man induktiv die $k$-te Ableitung, $k\in\N$}
\lang{en}{By induction we find an expression for the $k$th derivative for $k\in\N$}
\[ f^{(k)}(x)=(-1)^{k-1}\cdot (k-1)!\cdot x^{-k} .\]
\lang{de}{
Die $k$-te Ableitung ausgewertet an der Stelle $x_0=1$ ist also 
$f^{(k)}(1)=(-1)^{k-1}\cdot (k-1)!$.\\
Das Taylor-Polynom $n$-ter Ordnung um die Stelle $x_0=1$ lautet somit
}
\lang{en}{
The $k$th derivative evaluated at the point $x_0=1$ is therefore 
$f^{(k)}(1)=(-1)^{k-1}\cdot (k-1)!$.\\
This gives us the $n$th order Taylor polynomial at the point $x_0=1$,
}
\[ T_n(x)=\ln(1)+\sum_{k=1}^n (-1)^{k-1}\cdot\frac{(k-1)!}{k!}\cdot (x-1)^k= \sum_{k=1}^n (-1)^{k-1}\cdot\frac{1}{k}\cdot (x-1)^k .\]
\tab{\lang{de}{Anwendungsbeispiel}\lang{en}{Application}} \%\label{ex:taylorpoly-Anwendung}
\lang{de}{
Ein Anwendungsbeispiel zur Bestimmung der relativistischen kinetischen Energie ist nachfolgend gegeben:\\
\floatright{\href{https://api.stream24.net/vod/getVideo.php?id=10962-2-11278&mode=iframe&speed=true}{\image[75]{00_video_button_schwarz-blau}}}
}
\lang{en}{
A Taylor expansion can be used to find a formula for the kinetic energy of a particle. In 
particular, the relativistic kinetic energy may be approximated using a Taylor series, and higher 
order Taylor expansions give higher degree terms that only become relevant at relativistic speeds.
}


\end{tabs*}
\end{example}

\begin{quickcheck}
		\field{rational}
		\type{input.function}
		\begin{variables}
			\function{b}{6}
		    \function{f}{1/(x-5)}
            \function{df1}{-1/(x-5)^2}
            \function[normalize]{df2}{2/(x-5)^3}
            \function{fb}{1}
            \function{df1b}{-1}
            \function{df2b}{2}
%             \substitute[calculate]{dfb}{df}{x}{b}
%             \substitute[calculate]{ddfb}{ddf}{x}{b}
%             \substitute[calculate]{dfb3}{df3}{x}{b}
            \function{p0}{1}
            \function{p1}{1-(x-6)}
            \function{p2}{1-(x-6)+(x-6)^2}
		\end{variables}
		
		\text{\lang{de}{
        Bestimmen Sie die Taylor-Polynome nullter, erster und zweiter Ordnung der Funktion 
        $f(x)=\var{f}$ um die Stelle $x_0=\var{b}$.\\ 
        Die Ableitungen der Funktion $f(x)=\var{f}$ sind\\ $f'(x)= $\ansref und $\quad f''(x)= $\ansref. \\
        Daraus ergeben sich die Werte $f(\var{b})=$\ansref, $f'(\var{b})=$\ansref und $\quad f''(\var{b})=$\ansref.\\
        Die Taylor-Polynome sind daher\\
        }
        \lang{en}{
        Determine the zeroth, first and second order Taylor polynomial of the function 
        $f(x)=\var{f}$ at the point $x_0=\var{b}$.\\
        The derivatives of the function $f(x)=\var{f}$ are\\ $f'(x)= $\ansref and $\quad f''(x)= $\ansref. \\
        From this we get the values $f(\var{b})=$\ansref, $f'(\var{b})=$\ansref und $\quad f''(\var{b})=$\ansref.\\
        The Taylor polynomials are thus
        }
        $T_0(x)=$ \ansref,\\                
        $T_1(x)=$ \ansref,\\                
        $T_2(x)=$ \ansref.                  
        }
		
		\begin{answer}
			\solution{df1}
			\checkAsFunction{x}{-1}{1}{10}
		\end{answer}
		\begin{answer}
			\solution{df2}
			\checkAsFunction{x}{-1}{1}{10}
		\end{answer}
		\begin{answer}
			\solution{fb}
			\checkAsFunction{x}{-1}{1}{10}
		\end{answer}
		\begin{answer}
			\solution{df1b}
			\checkAsFunction{x}{-1}{1}{10}
		\end{answer}
        \begin{answer}
			\solution{df2b}
			\checkAsFunction{x}{-1}{1}{10}
		\end{answer}
		\begin{answer}
			\solution{p0}
			\checkAsFunction{x}{-1}{1}{10}
		\end{answer}
		\begin{answer}
			\solution{p1}
			\checkAsFunction{x}{-1}{1}{10}
		\end{answer}
		\begin{answer}
			\solution{p2}
			\checkAsFunction{x}{-1}{1}{10}
		\end{answer}
		\explanation{\lang{de}{Benutzen Sie die Formel für die Taylorpolynome.}
                 \lang{en}{Use the formula for the Taylor polynomials.}}
\end{quickcheck}

%      \begin{quickcheck}
% 		\field{rational}
% 		\type{input.function}
% 		\begin{variables}
% 			\randint[Z]{a}{-3}{5}
% 			\randint[Z]{a0}{-3}{1}
%            \function[calculate]{b}{a+a0}
% 		    \function[normalize]{f}{(x-a)/(x+a)}
%             \function[normalize]{df}{(2*a)/(x+a)^2}
%             \derivative[normalize]{ddf}{df}{x}
%             \derivative[normalize]{df3}{ddf}{x}
%             \substitute[calculate]{fb}{f}{x}{b}
%             \substitute[calculate]{dfb}{df}{x}{b}
%             \substitute[calculate]{ddfb}{ddf}{x}{b}
%             \substitute[calculate]{dfb3}{df3}{x}{b}
%             \function[normalize]{p1}{fb+dfb*(x-b)^1}
%             \function[normalize]{p2}{p1+ddfb/2*(x-b)^2}
%             \function[normalize]{p3}{p2+dfb3/6*(x-b)^3}
% 		\end{variables}
		
% 		\text{Zu bestimmen sind die Taylorpolynome 1., 2. und 3. Grades der Funktion $f(x)=\var{f}$ um die Stelle $x_0=\var{b}$.\\ 
%         Die höheren Ableitungen der Funktion $f(x)=\var{f}$ sind\\ $f'(x)= $\ansref, 
%         $\quad f''(x)= $\ansref und $\quad f'''(x)= $\ansref.\\
%         Die Taylorpolynome sind daher\\
%         $T_1(x)=$ \ansref,\\                %{f,1,\var{b}}
%         $T_2(x)=$ \ansref,\\                %{f,2,\var{b}}
%         $T_3(x)=$ \ansref.                  %{f,3,\var{b}}
%         }
		
% 		\begin{answer}
% 			\solution{df}
% 			\checkAsFunction{x}{-1}{1}{10}
% 		\end{answer}
% 		\begin{answer}
% 			\solution{ddf}
% 			\checkAsFunction{x}{-1}{1}{10}
% 		\end{answer}
% 		\begin{answer}
% 			\solution{df3}
% 			\checkAsFunction{x}{-1}{1}{10}
% 		\end{answer}
% 		\begin{answer}
% 			\solution{p1}
% 			\checkAsFunction{x}{-1}{1}{10}
% 		\end{answer}
% 		\begin{answer}
% 			\solution{p2}
% 			\checkAsFunction{x}{-1}{1}{10}
% 		\end{answer}
% 		\begin{answer}
% 			\solution{p3}
% 			\checkAsFunction{x}{-1}{1}{10}
% 		\end{answer}
% 		\explanation{Es ist die Formel für die Taylorpolynome anzuwenden und einzusetzen.
%        }
% 	\end{quickcheck}
\begin{remark}
\lang{de}{
Damit eine $n$-mal differenzierbare Funktion $f$ in der Umgebung einer Stelle $x_0$ durch das 
Taylor-Polynom $T_n$ angenähert wird, muss dort das Verhalten des Taylor-Polynoms dem der Funktion 
ähnlich sein. Ähnliches Verhalten heißt, dass das Taylor-Polynom zum einen im Entwicklungspunkt 
$x_0$ denselben Wert hat wie die Funktion selbst $f(x_0)=T_n(x_0)$. Weiter soll sich $T_n$ in 
gleicher Weise ändern wie die Funktion. Das heißt, dass die Ableitungen in $x_0$ übereinstimmen 
müssen
}
\lang{en}{
In order to approximate an $n$ times differentiable function $f$ in a neighbourhood of a point 
$x_0$ using a Taylor polynomial $T_n$, the behaviour of the Taylor polynomial in the neighbourhood 
must be similar to that of the function. This means that the Taylor polynomial should have the same 
value as the function itself at the point $x_0$. Furthermore, it should change at the same rate as 
the function, so the derivative at $x_0$ should have the same value too:
}
\[T_n'(x_0)= f'(x_0).\]
\lang{de}{
Weiter soll die Änderung der Ableitungen gleich sein, und wiederum die Änderung dieser, etc.
Damit $f$ und das $n$-te Taylor-Polynom näherungsweise gleich sind, muss also
}
\lang{en}{
The rate of change of the derivatives should also have the same values, as should the rate of change 
of that, and etcetera. For $f$ to be well approximated by the $n$th order Taylor polynomial, we 
hence require
}
\[  T_n^{(j)}(x_0)= f^{(j)}(x_0)\]
\lang{de}{
für jede Ableitungsordnung $j=0,1,\ldots ,n$ erfüllt sein.
An Ableitungen höherer Ordnung als $n$ stellen wir bei der Taylor-Approximation $n$-ter Ordnung 
keine Forderungen.
\\\\
Das Taylor-Polynom $n$-ter Ordnung hat also an der Stelle $x_0$ die gleichen Ableitungen 
bis zur Ordnung $n$ wie die Funktion $f$ selbst. Das rechnen wir jetzt explizit nach durch Induktion.
}
\lang{en}{
for every order of derivative $j=0,1,\ldots ,n$. We need not place restrictions on the derivatives 
higher than the $n$th derivative of an $n$th order Taylor polynomial.
\\\\
Indeed, the $n$th order Taylor polynomial has the same $j$th derivative at the point $x_0$ as the 
function $f$. We now use induction to explicitly confirm this.
}
%Letztere Erkenntnis müsste man als zu erfüllende Forderung an ein Polynom $n$-ten Grades der Form 
%$T_n(x)=\sum_{k=0}^{n} a_k\cdot (x-x_0)^k$ mit $a_k\in\R$ stellen, wenn dieses eine Funktion $f$ in der 
%Umgebung einer Stelle $x_0$ approximieren soll. Damit gelangt man gerade wieder zu \textsc{Taylor}.
\begin{incremental}[\initialsteps{0}]
\step 
\begin{itemize}
\item[$\underline{j=0}$:] \lang{de}{
An der Stelle $x=x_0$ stimmt der Funktionswert des Taylor-Polynoms $n$-ter Ordnung mit dem der zu 
nähernden Funktion $f$ überein, $T_n(x_0)=f(x_0)$. Das ergibt sich unmittelbar aus der 
Definition~\ref{def:taylorpoly} des Taylor-Polynoms: 
Alle Potenzen $(x-x_0)^k$ mit $k>0$ verschwinden, während der Summand 
$\frac{f(x_0)}{0!}(x-x_0)^0=f(x_0)$ zu $k=0$ verbleibt. (Hier haben wir einmal mehr die Definition 
$0^0=1$ benutzt.)
\\\\
Insbesondere finden wir das $0$-te Taylor-Polynom $T_0(x)=f(x_0)$ als das konstante Polynom mit Wert 
$f(x_0)$. Anschaulich interpretiert beschreibt $T_0(x)=f(x_0)$ die horizontale Gerade, die die 
Funktion $f$ im Punkt $(x_0,f(x_0))$ schneidet bzw. berührt. Das ist im Allgemeinen noch keine gute 
Näherung der Funktion.
}
\lang{en}{
At the point $x=x_0$ the value of the $n$th order Taylor polynomial corresponds to the value of the 
function $f$, $T_n(x_0)=f(x_0)$. This is clear from the definition~\ref{def:taylorpoly} of the 
Taylor polynomial: 
every power $(x-x_0)^k$ with $k>0$ vanishes, and using the definition $0^0=1$, the $k=0$ term 
$\frac{f(x_0)}{0!}(x-x_0)^0=f(x_0)$ equals the function evaluated at the point.
\\\\
In particular, the $0$th Taylor polynomial $T_0(x)=f(x_0)$ is the constant polynomial with value 
$f(x_0)$. Visually we may interpret $T_0(x)=f(x_0)$ as the horizontal line that intersects the 
function $f$ at the point $(x_0,f(x_0))$. In general, this is still a poor approximation of the 
function.
}
% Damit ist offensichtlich, dass die Taylor-Näherung \nowrap{$0$. Ordnung} für eine \glqq gute\grqq Näherung der Funktion $f$ an der Stelle $x_0$
% nicht hinreichend ist, das Kriterium $T_n(x_0)=f(x_0)$ aber \notion{eine} Notwendigkeit darstellt, um an der Stelle $x_0$ mit $T_n(x)$ gleiches Verhalten wie
% die Funktion $f$ zu erreichen.
%\link{def:taylorpoly}
\item[$\underline{j=1}$:]
\lang{de}{Leitet man das Taylor-Polynom $T_n$ nach $x$ ab, so erhält man}
\lang{en}{If we differentiate the Taylor polynomial $T_n$ with respect to $x$, we obtain}
\[T_n'(x)=\sum_{k=0}^n k\cdot \frac{f^{(k)}(x_0)}{k!}(x-x_0)^{k-1}=\sum_{k=1}^n \frac{f^{(k)}(x_0)}{(k-1)!}(x-x_0)^{k-1}.\]
\lang{de}{
Ausgewertet in $x_0$ ist das wie gefordert $T_n'(x_0)=f'(x_0)$, denn hier ist der Summand zu $k=0$ 
identisch Null, während für $k>1$ die Potenz $(x_0-x_0)^k$ verschwindet.
\\\\
Aus dem Abschnitt \link{diffbarkeit}{Differenzierbarkeit} ist die Tangente an den Graphen von $f$ an 
der Stelle $x_0$ als lineare Funktion
}
\lang{en}{
Evaluating this at $x_0$ we have $T_n'(x_0)=f'(x_0)$ as required, as here the term where $k=0$ is 
equal to zero, and for $k>1$ the expression $(x_0-x_0)^k$ is also equal to zero.
\\\\
As covered in the section on \link{diffbarkeit}{differentiability}, the tangent of the graph of $f$ 
at the point $x_0$ can be expressed as a linear function
}
\[  T(x)=f(x_0)+f'(x_0)\cdot (x-x_0) \]
\lang{de}{bekannt.}
\lang{en}{of $x$.}
%Nett, aber hier unnötiges Beiwerk: \textsc{Leibniz} definierte mit ihrer Steigung die Ableitung der Funktion $f$ an der Stelle $x_0$.
\lang{de}{
Diese Tangente ist identisch mit dem Taylor-Polynom $T(x)=T_1(x)$ erster Ordnung ($n=1$) von $f$ an 
der Stelle $x_0$. Bei der Taylor-Näherung erster Ordnung stimmen also bereits der Funktionswert und 
das Steigungsverhalten von $f$ und $T_1$ an der Stelle $x_0$ überein.
}
\lang{en}{
This tangent is identical to the first order Taylor polynomial $T(x)=T_1(x)$ of $f$ at the point 
$x_0$. As required, we note that the function value and slope of $f$ and $T_1$ are equal at 
the point $x_0$.
}
\item[$\underline{j=2}$:]
\lang{de}{Die zweite Ableitung des Taylor-Polynoms ist}
\lang{en}{The second derivative of the Taylor polynomial is}
\[T_n''(x)=(T_n')'(x)=\sum_{k=1}^n (k-1)\frac{f^{(k)}(x_0)}{(k-1)!}(x-x_0)^{k-2}=
\sum_{k=2}^n \frac{f^{(k)}(x_0)}{(k-2)!}(x-x_0)^{k-1}.\]
\lang{de}{
Daraus liest man wiederum ab $T_n''(x_0)=f''(x_0)$.
\\\\
Insbesondere erhält man mit der Taylor-Entwicklung bis zur zweiten Ordnung eine verbesserte 
Näherung von $f$ in der Umgebung der Stelle $x_0$,
}
\lang{en}{
From this we immediately read $T_n''(x_0)=f''(x_0)$.
\\\\
In particular, the second order Taylor expansion serves as a better approximation of $f$ near the 
point $x_0$,
}
\[ T_2(x)=f(x_0)+f'(x_0)\cdot (x-x_0)+\frac{f''(x_0)}{2}\cdot (x-x_0)^2. \]
\lang{de}{
Anschaulich schmiegt sich die durch $T_2(x)$ gegebene Näherungsparabel im Gegensatz zur Tangente an 
den Graphen von $f$ an.\\%Einfügen einer entsprechenden Abb.
Die Näherungsparabel berücksichtigt im Vergleich zur Tangente zusätzlich zur Steigung auch das 
Krümmungsverhalten der Funktion $f$ an der Stelle $x_0$. Im Bild wird eine Funktion $f$ genähert 
durch ihre Taylorpolynome nullter, erster und zweiter Ordnung im Punkt $x_0$.
}
\lang{en}{
Visually, the parabola given by $T_2(x)$ is clearly a closer approximation to the graph of $f$ than 
the tangent. It meets the graph of $f$ at the same point as the tangent does, and has the same slope 
at that point, but in addition it takes into consideration the curvature of $f$ at $x_0$. We see 
this clearly in the graph below, which shows the graph of the function $f$, and its approximation by 
a zeroth, first and second order Taylor polynomial at the point $x_0$.
}
\begin{center}
\image{T303_TaylorParabola}
\end{center}

% Ist mit $n>2$ der Grad des Taylorpolynoms größer als die Ableitungsordnung $j$, wie z.B.
% im Fall von\nowrap{$\;T_3''(x)=f''(x_0)+f'''(x_0)\cdot (x-x_0)$}, dann stimmt das Krümmungsverhalten
% der zu nähernden Funktion $f$ mit dem des Näherungspolynoms $n$-ten Grades nur an der Stelle $x_0$
% exakt überein:$\;T_n''(x_0)=f''(x_0)$. An dieser Stelle lässt sich die Einschränkung auf die Stelle $x=x_0$ für die $j-te$ Ableitung
% des Taylor-Polynoms $T_n^{(j)}(x_0)= f^{(j)}(x_0)$ nachvollziehen, denn $T_3''(x)$ ist offenkundig eine lineare Approximation von $f''(x)$ um die Stelle $x_0$.
%
\item[ $\underline{j\geq 3}$:] 
\lang{de}{
Wir zeigen die Gültigkeit von \[ T_n^{(j)}(x_0)= f^{(j)}(x_0) \] schließlich für beliebiges $j=0,\ldots,n$ induktiv.
Wir zeigen dazu mehr, nämlich für alle $j\in \N_0$
}
\lang{en}{
We finally show that \[ T_n^{(j)}(x_0)= f^{(j)}(x_0) \] for all $j=0,\ldots,n$ inductively. 
Moreover, we show that for all $j\in \N_0$
}
\begin{equation}\label{eq:abl_taylor-pol}
T_n^{(j)}(x)=\sum_{k=j}^n \frac{f^{(k)}(x_0)}{(k-j)!}(x-x_0)^{k-j}.
\end{equation}
\lang{de}{
Wir haben dies bereits sicher verankert für die Fälle $j=0,1,2$. Ist die Behauptung richtig für ein 
festes $j\geq 0$, dann folgt durch nochmaliges Ableiten der $j$-ten Ableitung, weglassen von 
Nulltermen und Kürzen
}
\lang{en}{
We have already covered the base cases $j=0,1,2$. If we assume that the proposition is true for a 
fixed $j\geq 0$, then simply differentiating the $j$th derivative once more yields, after 
simplification,
}
\[ T_n^{(j+1)}(x)=(T_n^{(j)})'(x)=\sum_{k=j}^n (k-j)\frac{f^{(k)}(x_0)}{(k-j)!}(x-x_0)^{k-j-1}=
\sum_{k=j+1}^n \frac{f^{(k)}(x_0)}{(k-(j+1))!}(x-x_0)^{k-(j+1)}.\]
\lang{de}{
Das ist die Behauptung für $j+1$.
In der nun bewiesenen Formel (\ref{eq:abl_taylor-pol}) setzen wir $x=x_0$ und erhalten für $j\leq n$
}
\lang{en}{
This is precisely the proposition for $j+1$. 
We set $x=x_0$ in the formula (\ref{eq:abl_taylor-pol}), which we have just justified, and obtain 
for $j\leq n$
}
\[ T_n^{(j)}(x_0)=\sum_{k=j}^n \frac{f^{(k)}(x_0)}{(k-j)!}(x_0-x_0)^{k-j}=f^{(j)}(x_0),\]
\lang{de}{
denn in der Summe ist die Potenz $0^{k-j}=(x_0-x_0)^{k-j}$ nur dann von Null verschieden, wenn 
$k=j$. In dem Fall ist sie gleich $0^0=1$.
\\\\
Für alle $j>n$ sehen wir durch (\ref{eq:abl_taylor-pol}) auch $T_n^{(j)}=0$ für alle $j>n$, 
wie wir das von einem Polynom höchstens $n$-ten Grades erwarten.
}
\lang{en}{
as in the above sum, the power $0^{k-j}=(x_0-x_0)^{k-j}$ is only non-zero when $k=j$, in which case 
$0^0=1$.
\\\\
We also see by (\ref{eq:abl_taylor-pol}) that $T_n^{(j)}=0$ for all $j>n$, as is to be expected for 
a polynomial of degree at most $n$.
}
\end{itemize}
\end{incremental}
\end{remark}  
\begin{block}[tip]
\lang{de}{
Ein Taylor-Polynom $T_{f,x_0,n}$ ist durch seine Definition bereits in optimaler Form gegeben, 
nämlich dargestellt in Potenzen des linearen Terms $(x-x_0)$ um seinen Entwicklungspunkt $x_0$.
Belassen Sie es in dieser Darstellung!\\ 
Sie gibt Ihnen am besten Auskunft über das Verhalten des Taylor-Polynoms in der Nähe von $x_0$, 
durch das Sie die zugehörigen Funktion $f$ beschreiben wollen.
Vermeiden Sie zum Beispiel auch die Umformung zu einer vermeindlich übersichtlichen Darstellung in 
Potenzen von $x$, wenn Sie keinen ausdrücklichen Grund dazu haben.
}
\lang{en}{
The Taylor polynomial $T_{f,x_0,n}$ is already given in a convenient form by its definition, that 
is, in terms of powers of the linear expression $(x-x_0)$. There is no need to change this!\\
The behaviour of the Taylor polynomial near $x_0$ is what interests us, as this is where it best 
approximates the function $f$. Hence despite seeming simpler or clearer at first glance, writing 
a Taylor polynomial in powers of $x$ is not often useful.
}
\end{block}
\begin{example}
\begin{tabs*}[\initialtab{0}]
\tab{$\cos x$}
\lang{de}{
Wir berechnen das Taylor-Polynom vierter Ordnung der Funktion $f(x)=\cos x$ an der 
Entwicklungsstelle $x_0=0$. Die Ableitungen sind
}
\lang{en}{
We calculate the $4$th order Taylor polynomial of the function $f(x)=\cos x$ at the point $x_0=0$. 
The derivatives of $f$ are
}
\[ f'(x)=-\sin x,\quad f''(x)=-\cos x,\quad f^{(3)}(x)=\sin x,\quad f^{(4)}(x)=\cos x.\]
\lang{de}{Damit ergibt sich}
\lang{en}{From these we get}
\begin{eqnarray*}  
T_{4}(x)&=&\cos(0)-\sin(0)(x-0)-\frac{\cos(0)}{2!}(x-0)^2+\frac{\sin(0)}{3!}(x-0)^3
+ \frac{\cos(0)}{4!}(x-0)^4 \\
&=& 1-\frac{1}{2}x^2+\frac{1}{24}x^4
\end{eqnarray*}
\begin{center}
\image{T303_TaylorCosine}\\\\
\lang{de}{
Weitere Informationen zu den Taylor-Polynomen von Kosinus und Sinus sind im folgenden Video gegeben:\\
\floatright{\href{https://api.stream24.net/vod/getVideo.php?id=10962-2-11277&mode=iframe&speed=true}{\image[75]{00_video_button_schwarz-blau}}}\\ 
}
\lang{en}{}
\end{center}
\tab{\lang{de}{Polynom}\lang{en}{Polynomial}}
\lang{de}{
Für die Polynomfunktion $p(x)=x^3+2x^2-x+3$ berechnen wird das Taylor-Polynom fünfter Ordnung an der
Entwicklungsstelle $x_0=0$. Die Ableitungen von $f$ sind
}
\lang{en}{
We calculate the $5$th order Taylor polynomial of the function $p(x)=x^3+2x^2-x+3$ at the point 
$x_0=0$. The derivatives of $f$ are
}
\[ p'(x)=3x^2+4x-1,\quad p''(x)=6x+4,\quad p^{(3)}(x)=6,\quad  p^{(4)}(x)=p^{(5)}(x)=p^{(m)}(x)=0\]
\lang{de}{für $m\geq 4$, und}
\lang{en}{for $m\geq 4$, and}
\[ p(0)=3,\quad p'(0)=-1,\quad p''(0)=4,\quad p^{(3)}(0)=6,\quad  p^{(m)}(0)=0\]
\lang{de}{$m\geq 4$. Damit ist}
\lang{en}{$m\geq 4$. Thus}
\begin{eqnarray*} 
T_3(x)&=&T_4(x)=T_5(x)=\ldots=T_m(x)\\
&=&3-1\cdot x+\frac{4}{2}x^2+\frac{6}{3!}x^3+\frac{0}{4!}x^4+\frac{0}{5!}x^5\\
&=& x^3+2x^2-x+3=f(x).
\end{eqnarray*}
\lang{de}{
Alle Taylor-Polynome $T_m$ mit $m\geq 3$ stimmen mit der Polynomfunktion $p$ dritten Grades überein! 
Dies gilt viel allgemeiner, wie der folgende Satz zeigt.
}
\lang{en}{
Every Taylor polynomial $T_m$ where $m\geq 3$ is precisely the polynomial $p$, which is of degree 
$3$. This holds more generally, as we see in the following theorem.
}
\end{tabs*}
\end{example}

\begin{theorem}\label{thm:thm:polynome-taylor-entwickelt}
\lang{de}{
Ist $p$ eine Polynomfunktion vom Grad $n$, so stimmt das zugehörige Taylor-Polynom $T_{m}$ der 
Ordnung $m$ (zu einer beliebigen Entwicklungsstelle $x_0$) mit $p$ überein, wann immer $m\geq n$. 
Insbesondere ist der \emph{Grad} des Taylor-Polynoms $T_m$ dann $n$ und 
 \[T_n=f=T_m \quad\text{ für alle } m\geq n.\]
 \floatright{\href{https://api.stream24.net/vod/getVideo.php?id=10962-2-11276&mode=iframe&speed=true}{\image[75]{{00_video_button_schwarz-blau}}} \href{https://www.hm-kompakt.de/video?watch=539}{\image[75]{00_Videobutton_schwarz}}}\\\\
}
\lang{en}{
Let $p$ be a polynomial of degree $n$. Then the corresponding $m$th order Taylor polynomial $T_{m}$ 
(calculated at any point $x_0$) is equal to $p$ for $m\geq n$. 
In particular, the \emph{degree} of the Taylor polynomial $T_m$ is then $n$ and 
 \[T_n=f=T_m \quad\text{ for all } m\geq n.\]
}
\end{theorem}

\lang{de}{
Dieser Satz besagt, dass die beste Art, ein Polynom durch ein ein Taylor-Polynom hohen Grades zu 
approximieren, durch das Polynom selbst gegeben ist. Das $m$-te Taylor-Polynom $T_m$ einer 
Polynomfunktion vom Grad $n$ zu bestimmen macht also nur dann Arbeit, wenn $m<n$. Man kann aber die 
Taylor-Entwicklung in einem Punkt $x_0$ sehr gut dazu nutzen, ein Polynom umzusummieren in Potenzen 
$(x-x_0)^k$ mit $0 \leq k \leq n$.\\
%Weitere Erklärungen zu Satz \ref{thm:thm:polynome-taylor-entwickelt} sind den folgenden Videos zu entnehmen:\\
}
\lang{en}{
This theorem tells us that the approximation of a polynomial using a Taylor polynomial of a 
greater degree is given simply by the polynomial itself. That is, we only need to calculate the 
$m$th order Taylor polynomial $T_m$ of a polynomial function of degree $n$ for $m<n$: beyond this, 
we simply take the polynomial. We can, however, use a Taylor approximation at a point $x_0$ to 
express a polynomial in terms of powers of $(x-x_0)^k$ where $0 \leq k \leq n$.\\
}



\begin{proof*}[\lang{de}{Beweis}\lang{en}{Proof}] % von Satz \ref{thm:polynome-taylor-entwickelt}]
\begin{incremental}[\initialsteps{1}]
\step
\lang{de}{Wir benutzen unser Wissen über die Ableitungen von Polynomen.}
\lang{en}{We use our knowledge of the derivatives of polynomials.}
\step
\lang{de}{
Die $k$-te Ableitung eines Monoms $m(x)=x^l$ (für $l\in\N_0$) ist gegeben durch
}
\lang{en}{
The $k$-th derivative of a monomial $m(x)=x^l$ (for $l\in\N_0$) is given by
}
\[m^{(k)}(x)=l\cdot(l-1)\cdots(l-k)x^{l-k}\: \text{ falls }k\leq l, \quad \text{ bzw. }\quad m^{(k)}(x)=0,\:\text { falls } k>l.\]
\lang{de}{
Beschreibt dann $p(x)=a_nx^n+a_{n-1}x^{n-1}+\ldots+a_1x+a_0$ eine Polynomfunktion vom Grad $n$, 
(d.h. $a_j\in\R$ für $j=0,\ldots,n$ und $a_n\neq 0$), dann gilt $p^{(k)}=0$ für alle $k>n$. Damit 
sind im Taylor-Polynom $T_m$ von $p$ (zum Entwicklungspunkt $x_0$) alle Summanden zum Index $k>n$ 
gleich Null. Es gilt also für alle $m\geq n$
}
\lang{en}{
Now let $p(x)=a_nx^n+a_{n-1}x^{n-1}+\ldots+a_1x+a_0$ be a polynomial of degree $n$, (with $a_j\in\R$ 
for $j=0,\ldots,n$ and $a_n\neq 0$), then we have $p^{(k)}=0$ for all $k>n$. Hence the Taylor 
polynomial $T_m$ of $p$ (at the point $x_0$) has all terms of index $k>n$ equal to zero, and for all 
$m\geq n$ we have
}
\[T_m(x)=\sum_{k=0}^m\frac{p^{(k)}(x_0)}{k!}(x-x_0)^k=\sum_{k=0}^n\frac{p^{(k)}(x_0)}{k!}(x-x_0)^k=T_n(x).\]
\lang{de}{
Weil $p^{(n)}(x)=a_n\cdot n!\neq 0$, ist der $n$-te Summand hier nicht identisch Null, also hat das 
Polynom den Grad $n$.
}
\lang{en}{
As $p^{(n)}(x)=a_n\cdot n!\neq 0$, the $n$th term here is not equal to zero, and so the polynomial 
has degree $n$.
}
\end{incremental}
\end{proof*}

%\begin{remarks}
%  \begin{enumerate}
%\item 
%    \lang{de}{Das Taylor-Polynom einer \strong{geraden Funktion an der Stelle $a=0$}  hat nur 
%    \strong{gerade Potenzen}.\\
%    \lang{de}{Das Taylor-Polynom einer \strong{ungeraden Funktion an der Stelle $a=0$} hat nur \strong{ungerade Potenzen}.\\ 
%\item
%  \lang{de}{Aus der Definition folgt unmittelbar die Linearit\"at:}
%  \lang{en}{From the definition follows the linearity:}\\
%
%    \[
%    P\left[\alpha f + \beta g ,n,a \right](x) = \alpha P\left[f ,n,a \right](x)
%                                              + \beta  P\left[g ,n,a \right](x)
%    \]
%    \lang{de}{$\alpha, \beta \in \R$ Konstanten; $f,g$ differenzierbare Funktionen.} 
%    \lang{en}{$\alpha, \beta \in \R$ constants; $f,g$ differentiable functions.} 
%    \end{enumerate}
%\end{remarks}


\lang{de}{
Es gibt Funktionen, deren Verhalten in der Umgebung bestimmter Stellen des Definitionsbereichs durch 
das zugehörige Taylor-Polynom nicht abgebildet wird. Das heißt, eine solche Funktion kann durch ihr 
Taylor-Polynom in der Umgebung dieser Stelle nicht ausreichend \glqq gut\grqq genähert werden. 
Das zeigt folgendes Beispiel. Es ist also wichtig, sich auch um die Güte einer Approximation durch 
Taylor-Polynome Gedanken zu machen. Das tun wir im nächsten Abschnitt.
}
\lang{en}{
There exist functions whose behaviour near certain points of their domain is not well approximated 
by the corresponding Taylor polynomial. The following example demonstrates this. It is important 
to consider how well a Taylor polynomial serves as an approximation.
}

\begin{example}\label{ex:gegenbeispiel}
\lang{de}{Gegeben ist die Funktion $f:\R\to\R$,}
\lang{en}{Consider the function $f:\R\to\R$,}
\begin{displaymath}
  f(x)=
  \begin{cases}
    e^{-\frac{1}{x}}, & x>0,\\
    0, & x\leq 0.
  \end{cases}
\end{displaymath}
\lang{de}{
Wir zeigen unten, dass alle Ableitungen in $x_0=0$ existieren und gleich Null sind, dass also für 
alle $k\in\N$ gilt
}
\lang{en}{
Below we show that every derivative exists at $x_0=0$ and is equal to zero, i.e. that for all 
$k\in\N$ we have
}
\[f^{(k)}(0)=0.\]
\lang{de}{
Damit ist jedes Taylor-Polynom $T_n(x)$ um den Entwicklungspunkt $x_0=0$ identisch mit der 
Nullfunktion. Die Nullfunktion beschreibt aber nicht, wie sich der Graph der Funktion $f$ im Bereich 
$x>0$ verhält, zum Beispiel wie ihr Graph dort aussieht.
}
\lang{en}{
Hence every Taylor polynomial $T_n(x)$ at the point $x_0=0$ is equal to the zero function. The zero 
function does not, however, do a good job of describing how the graph of the function $f$ behaves 
for $x>0$.
}

\begin{incremental}[\initialsteps{0}]
\step
\lang{de}{
Wir betrachten zunächst die Ableitungsfunktionen im Bereich $x>0$.
Dort haben alle Ableitungsfunktionen der Ordnung $k\in\N$ die Form
}
\lang{en}{
Next we consider the derivatives in the domain $x>0$. 
There, every derivative function of order $k\in\N$ has the form
}
\[ f^{(k)}(x) =\frac{p_{k-1}(x)}{x^{2k}}e^{-\frac{1}{x}}=p_{k-1}(x) x^{-2k}e^{-\frac{1}{x}}, \]
\lang{de}{
wobei $p_{k-1}(x)$ ein Polynom vom Grad $k-1$, für das $p_{k-1}(0)=1$ gilt.
Sogar für $k=0$ können wir in dieser Art schreiben $f^{(0)}(x)=f(x)=p_{k-1}(x) x^{-2k}e^{-\frac{1}{x}}$, 
wenn wir $p_{-1}(x)=1$ setzen.
\\\\
Für $k>0$ zeigen wir das durch vollständige Induktion.
}
\lang{en}{
where $p_{k-1}(x)$ is a polynomial of degree $k-1$, for which $p_{k-1}(0)=1$ holds. 
Even for $k=0$ we may write $f^{(0)}(x)=f(x)=p_{k-1}(x) x^{-2k}e^{-\frac{1}{x}}$ in this form, if we 
define $p_{-1}(x)=1$.
\\\\
We use induction to show this for $k>0$.
}
\step
\lang{de}{Für $k=1$ gilt für $x>0$}
\lang{en}{For $k=1$ and $x>0$,}
\[f'(x)=\frac{1}{x^2}e^{-\frac{1}{x}},\]
\lang{de}{
Also gilt die Behauptung mit dem Polynom $p_{0}(x)=1$ vom Grad Null. Dies ist unser 
Induktionsanfang.
}
\lang{en}{
so the proposition holds for the polynomial $p_{0}(x)=1$ of degree zero. This is our base case.
}
\step
\lang{de}{
Nehmen wir nun für ein beliebiges, festes $k\in \N$ die obige Form der Ableitung an. 
Dann lässt sich mit Hilfe der Produkt- und Kettenregel induktiv zeigen, dass auch $f^{(k+1)}(x)$ von solcher Form  ist.
%Undzwar deshalb, weil das in $f^{(k+1)}(x)$ enthaltene Polynom vom Grad $k$ ist und als $p_k(x)$ geschrieben werden darf:
}
\lang{en}{
We fix $k\in \N$ and take the derivative of the above expression for the $k$th derivative. 
Using the product rule and the chain rule, we show that $f^{(k+1)}(x)$ is also of this form:
}
\begin{eqnarray*}
 f^{(k+1)}(x) &=& (f^{(k)})'(x)
 =p'_{k-1}(x)\cdot  x^{-2k}\cdot e^{-\frac{1}{x}}
 + p_{k-1}(x)\cdot (-2k)\cdot x^{-2k-1}\cdot e^{-\frac{1}{x}}
 + p_{k-1}(x)\cdot x^{-2k}\cdot x^{-2}\cdot e^{-\frac{1}{x}} \\
 &=& \left( x^2 p'_{k-1}(x)-2 k x p_{k-1}(x) + p_{k-1}(x)\right)x^{-2k-2}e^{-\frac{1}{x}}\\
 &=& \left( x^2 p'_{k-1}(x)-2 k x p_{k-1}(x) + p_{k-1}(x)\right)x^{-2(k+1)}e^{-\frac{1}{x}}
\end{eqnarray*}
\lang{de}{Das Polynom in Klammern}
\lang{en}{The polynomial in the parentheses}
\[p_{k}(x):=x^2 p'_{k-1}(x)-2 k x p_{k-1}(x) + p_{k-1}(x)\]
\lang{de}{
hat den Grad $k$, weil nach Induktionsannahme der Grad von $p_{k-1}$ gleich $k-1$ ist. Ebenso 
verwenden wir die Induktionsannahme $p_{k-1}(0)=1$, um den Wert $p_k(0)$ zu bestimmen
}
\lang{en}{
has degree $k$, as by the induction hypothesis, $p_{k-1}$ has degree $k-1$. We also apply the 
hypothesis $p_{k-1}(0)=1$ to determine the value of $p_k(0)$,
}
\[p_k(0)=0^2\cdot p'_{k-1}(0)-2 k \cdot 0\cdot p_{k-1}(0) + p_{k-1}(0)=0+0+1=1. \]
\lang{de}{Damit ist die behauptete Gestalt der Ableitungsfunktion $f^{(k)}$ bewiesen für alle $k$.}
\lang{en}{By induction, we have shown that $f^{(k)}$ has this form for all $k$.}
\step
\lang{de}{
Im Bereich $x<0$ sind offensichtlich alle Ableitungsfunktionen $f^{(k)}(x)$ $k$-ter Ordnung mit 
$k>0$ gleich $0$.
}
\lang{en}{
For $x<0$ the $k$th derivative function $f^{(k)}(x)$ where $k>0$ is obviously simply the zero 
function.
}
\step
\lang{de}{
Schließlich zeigen wir ebenfalls induktiv, dass $f$ in $x_0=0$ beliebig oft differenzierbar ist,
und alle Ableitungen dort gleich $0$ sind.
}
\lang{en}{
Finally we show inductively that $f$ is differentiable arbitrarily many times at $x_0=0$, and that 
all derivatives are equal to $0$ at that point.
}
\step
\lang{de}{
Unmittelbar aus der Definition der Funktion $f$, ist klar $f^{(0)}(0) =f(0)=0$.
Wir setzen nun voraus, dass $f^{(k)}(0)=0$ für ein beliebiges $k\in \N$ gilt. 
Um dieselbe Aussage für $k+1$ zu zeigen, müssen wir zeigen, dass der Grenzwert des Differenzenquotienten von $f^{(k)}$
}
\lang{en}{
We immediately get $f^{(0)}(0) =f(0)=0$ from the definition of the function $f$. We now assume 
that $f^{(k)}(0)=0$ for any $k\in \N$. To prove the proposition for $k+1$, we must show that the 
limit
}
\[  \lim_{h\to 0} \frac{f^{(k)}(h)-f^{(k)}(0)}{h}= \lim_{h\to 0}\frac{f^{(k)}(h)}{h}  \]
\lang{de}{
existiert und gleich Null ist. In dem Fall ist $f^{(k+1)}(0)=0$ bestimmt als Ableitung von 
$f^{(k)}$ in $x_0=0$. Da für $h<0$ $f^{(k)}(h)=0$ ist, folgt für den linksseitigen Grenzwert
}
\lang{en}{
exists and is equal to zero. If we show this, we will have $f^{(k+1)}(0)=0$. As $f^{(k)}(h)=0$ for 
$h<0$, the limit from the left is given by
}
\[  \lim_{h\nearrow 0}\frac{f^{(k)}(h)}{h}=0. \]
\lang{de}{Für den rechtsseitigen Grenzwert gilt}
\lang{en}{The limit from the right is}
\begin{eqnarray*}
 \lim_{h\searrow 0}\frac{f^{(k)}(h)}{h} &=& 
 \lim_{h\searrow 0}\left( p_{k-1}(h) h^{-2k}\, e^{-\frac{1}{h}}\right)\cdot h^{-1}=\lim_{h\searrow 0}p_{k-1}(h) h^{-2k-1}\, e^{-\frac{1}{h}} \\
&=&  \lim_{h\searrow 0} p_{k-1}(h) \cdot  \lim_{h\searrow 0} (\frac{1}{h})^{2k+1}\, e^{-\frac{1}{h}} \\
&=& 1\cdot  \lim_{r\to \infty} \frac{r^{2k+1}}{e^r}=0.
\end{eqnarray*}
\lang{de}{
Der letzte Grenzwert ist hier gleich Null, weil die Exponentialfunktion schneller als jede 
Potenzfunktion wächst  
(siehe \ref[exp-und-log][Abschnitt "'Exponentialfunktion und Logarithmus"']{thm:exp-staerker-als-potenz}).
Damit stimmen links- und rechtsseitiger Grenzwert an der Stelle $x_0=0$ überein, und die $(k+1)$-te 
Ableitung von $f$ existiert dort mit Wert $f^{(k+1)}(0)=0$.
}
\lang{en}{
The last limit above is equal to zero, as the exponential function grows faster than any power 
function 
(see the \ref[exp-und-log][section on the exponential function and logarithm]{thm:exp-staerker-als-potenz}).
Hence the left and right sided limits agree at the point $x_0=0$, and the $(k+1)$th derivative of 
$f$ exists at $x_0$ and has value $f^{(k+1)}(0)=0$.
}
\end{incremental}
\end{example}


%\begin{example}\label{ex:gegenbeispiel}
% An dem folgenden Beispiel ist der Unterschied zwischen der Funktion und 
%dem approximierenden Polynom gut zu sehen:
%\begin{displaymath}
%  f(x)=
%  \begin{cases}
%    e^{-\frac{1}{x}}, & x>0,\\
%    0, & x\leq 0.
%  \end{cases}
%
%\end{displaymath}
%Induktiv lässt sich berechnen, dass im Bereich $x>0$ alle höheren Ableitungsfunktionen
%$f^{(n)}(x)$ von der Form
%\[ f^{(n)}(x) =\frac{p_n(x)}{x^{2n}}e^{-\frac{1}{x}}=p_n(x) x^{-2n}e^{-\frac{1}{x}} \]
%ist, wobei $p_n(x)$ ein Polynom mit $p_n(0)=1$ ist.
%\begin{incremental}[\initialsteps{0}]
%\step Für $n=0$ ist nämlich 
%\[ f^{(0)}(x) =f(x)=\frac{1}{x^{2\cdot 0}}e^{-\frac{1}{x}} \]
%von dieser Form mit $p_0(x)=1$.
%\step Ist für ein bestimmtes $n\in \N$ die Ableitung $f^{(n)}(x)$ von dieser Form,
%so gilt mit der Produkt- und Kettenregel
%\begin{eqnarray*}
% f^{(n+1)}(x) &=& (f^{(n)})'(x)
% =p'_n(x)\cdot  x^{-2n}\cdot e^{-\frac{1}{x}}
% + p_n(x)\cdot (-2n)x^{-2n-1}\cdot e^{-\frac{1}{x}}
% + p_n(x)\cdot x^{-2n}\cdot \frac{1}{x^2}e^{-\frac{1}{x}} \\
% &=& \left( x^2 p'_n(x)-2nx p_n(x)+p_n(x)\right)x^{-2n-2}e^{-\frac{1}{x}} 
%\end{eqnarray*}
%Setzt man $p_{n+1}(x)=x^2 p'_n(x)-2nx p_n(x)+p_n(x)$, so ist also
%$f^{(n+1)}(x)=p_{n+1}(x) x^{-2(n+1)}e^{-\frac{1}{x}}$ und $p_{n+1}(0)=p_n(0)=1$.
%\end{incremental}
%Im Bereich $x<0$ sind offensichtlich alle höheren Ableitungsfunktionen $f^{(n)}(x)$ gleich $0$.
%
%Es lässt sich schließlich induktiv zeigen, dass auch an der Stelle $x=0$ die Funktion beliebig oft differenzierbar ist und alle höheren Ableitungen gleich $0$ sind.
%\begin{incremental}[\initialsteps{0}]
%\step Die $0$-te Ableitung $f^{(0)}(0) =f(0)$ ist nach Definition gleich $0$.
%\step Induktiv nehmen wir nun an, dass $f^{(n)}(0)=0$ für ein $n\in \N$.\\
%Um $f^{(n+1)}(0)$ zu bestimmen, muss man dann den Grenzwert
%\[  \lim_{h\to 0} \frac{f^{(n)}(h)-f^{(n)}(0)}{h}= \lim_{h\to 0}\frac{f^{(n)}(h)}{h} \]
%berechnen.
%
%Da $f^{(n)}(h)=0$ für $h<0$ ist, ist der linksseitige Grenzwert
%\[  \lim_{h\nearrow 0}\frac{f^{(n)}(h)}{h}=0. \]
%
%Für den rechtsseitigen Grenzwert gilt:
%\begin{eqnarray*}
% \lim_{h\searrow 0}\frac{f^{(n)}(h)}{h} &=& 
% \lim_{h\searrow 0}p_n(h) h^{-2n-1}e^{-\frac{1}{h}} \\
%&=&  \lim_{h\searrow 0} p_n(h) \cdot  \lim_{h\searrow 0} (\frac{1}{h})^{2n+1}e^{-\frac{1}{h}} \\
%&=& 1\cdot  \lim_{r\to \infty} \frac{r^{2n+1}}{e^r} \\
%\end{eqnarray*}
%Im \ref[exp-und-log][Abschnitt "'Exponentialfunktion und Logarithmus"']{thm:exp-staerker-als-potenz} wurde gezeigt, dass der letzte Grenzwert gleich $0$ ist.\\
%Also ist auch der rechtsseitige Grenzwert gleich $0$ und damit existiert die $(n+1)$-te Ableitung von $f$ an der Stelle $0$ und ist gleich $0$.
%\end{incremental}
%
%Da $f^{(k)}(0)=0$ f\"ur alle $k\in\N$, sind alle Taylor-Polynome im Entwicklungspunkt $x=0$ gleich
%  null. Diese Taylor-Polynome bilden daher nicht ab, wie der Funktionsgraph im Bereich $x>0$ aussieht.
%\end{example}


\section{\lang{de}{Taylor-Restglied}\lang{en}{Taylor polynomial remainder}}\label{sec:restglied}

\lang{de}{
Approximiert man eine Funktion $f$ in der Umgebung einer Stelle $x_0$ durch ein Taylor-Polynom 
$n$-ter Ordnung, so muss man also auch die Güte dieser Approximation abschätzen können. Das heißt,
wie stark das approximierende Taylor-Polynom $T_n$ von der gegebenen Funktion $f$ abweicht. 
Diesen Fehler messen wir dadurch, dass wir die Differenz $f-T_n$ abschätzen. Er hängt unter anderem 
davon ab, wie weit man sich von der gegebenen Entwicklungsstelle $x_0$ entfernt. \\
%Hier wird eine entsprechende ABBILDUNG eingebaut. Es bietet sich z.B. die Sinusfunktion zur Approxomation an.
}
\lang{en}{
If we intend to approximate a function $f$ in a neighbourhood of a point $x_0$ using an $n$th order 
Taylor polynomial, we ought to know how good this approximation is. That is, we should know how fast 
the Taylor polynomial $T_n$ deviates from the function $f$. This error is measured by approximating 
the difference $f-T_n$. The error is of course dependent on how far from the point $x_0$ we are. \\ 
}
\begin{definition}\label{def:restglied}
\lang{de}{
Die Funktion $f:D\to \R$ sei (mindestens) $n$-mal differenzierbar und es sei $x_0\in D$ eine 
Entwicklungsstelle. Die Differenz zwischen der Funktion $f(x)$ und ihrem $n$-ten Taylor-Polynom 
$T_n$ zur Entwicklungsstelle $x_0$ heißt das \notion{Restglied} $R_n:D\to\R$,
}
\lang{en}{
Let the function $f:D\to \R$ be (at least) $n$-times differentiable and let $x_0\in D$. The 
difference between the function $f(x)$ and its $n$th order Taylor polynomial $T_n$ at the point 
$x_0$ is called the \notion{remainder} $R_n:D\to\R$,
}
\[   R_{n}(x)=f(x)-T_{n}(x). \]
% Hat man eine bestimmte Form des Restgliedes gefunden, so bezeichnet man die Gleichung 
% \[   f(x)=T_{n}(x)+R_{n}(x) \]
% auch als \notion{Taylorsche Formel}.
\end{definition}

     \begin{quickcheck}
		\field{rational}
		\type{input.function}
		\begin{variables}
			\function{a}{1}
            \function{b}{0}
            \function{c}{1}
            \function{d}{1/2}
            \function{d2}{1/4}
		    \function[normalize]{f}{1/(x+1)}
            \function[normalize]{df}{-1/(x+1)^2}
            \function[normalize]{ddf}{2/(x+1)^3}
            \substitute[calculate]{fb}{f}{x}{b}
            \substitute[calculate]{dfb}{df}{x}{b}
            \substitute[calculate]{ddfb}{ddf}{x}{b}
%            \function[calculate]{fb}{(b-x0)/(b+x0)}
%            \function[calculate]{dfb}{(2*x0)/(b+x0)^2}
%            \function[calculate]{dfb}{(-6*x0)/(b+x0)^3}
            \function[normalize]{p1}{fb+dfb*(x-b)^1}
            \function[normalize]{p2}{p1+ddfb/2*(x-b)^2}
            \function[normalize]{r2}{f-p2}
            \substitute[calculate]{r2c}{r2}{x}{c}
            \substitute[calculate]{r2d}{r2}{x}{d}
            \substitute[calculate]{r2d2}{r2}{x}{d2}
  		\end{variables}
		
		\text{\lang{de}{
        Bestimmen Sie für Funktion $f(x)=\var{f}$ das Taylorpolynom $T_2$ zweiter Ordnung an der 
        Stelle $x_0=\var{b}$.\\
        Die erste und zweite Ableitung der Funktion $f(x)=\var{f}$ sind $\quad f'(x)= $\ansref und 
        $\quad f''(x)= $\ansref.\\
        Damit ergibt sich das Taylorpolynom zweiter Ordnung $T_{2}(x)=$ \ansref.\\
        Bestimmen Sie auch das zugehörige Restglied $R_{2}(x)=$ \ansref.\\
        Bestimmen Sie weiter die drei speziellen Werte des Restglieds 
        $R_2(1)=$\ansref, $\:R_2(\frac{1}{2})=$\ansref und $\:R_2(\frac{1}{4})=$\ansref.
        }
        \lang{en}{
        Let $f(x)=\var{f}$. Determine the second order Taylor polynomial $T_2$ at the point 
        $x_0=\var{b}$.\\
        The first and second derivatives of the function $f(x)=\var{f}$ are $\quad f'(x)= $\ansref 
        and $\quad f''(x)= $\ansref.\\
        Hence the second order Taylor polynomial is $T_{2}(x)=$ \ansref.\\
        Determine also the corresponding remainder $R_{2}(x)=$ \ansref.\\
        Determine the following three values of the remainder: 
        $R_2(1)=$\ansref, $\:R_2(\frac{1}{2})=$\ansref and $\:R_2(\frac{1}{4})=$\ansref.
        }}
		
		\begin{answer}
			\solution{df}
			\checkAsFunction{x}{-1}{1}{10}
		\end{answer}
		\begin{answer}
			\solution{ddf}
			\checkAsFunction{x}{-1}{1}{10}
		\end{answer}
		\begin{answer}
			\solution{p2}
			\checkAsFunction{x}{-1}{1}{10}
		\end{answer}
		\begin{answer}
			\solution{r2}
			\checkAsFunction{x}{-1}{1}{10}
		\end{answer}
        \begin{answer}
        \solution{r2c}
			\checkAsFunction{x}{-1}{1}{10}
		\end{answer}
        \begin{answer}
        \solution{r2d}
			\checkAsFunction{x}{-1}{1}{10}
		\end{answer}
        \begin{answer}
        \solution{r2d2}
			\checkAsFunction{x}{-1}{1}{10}
		\end{answer}
\end{quickcheck}
\lang{de}{
An diesem Kurztest sieht man: 
Bei kleinerem Abstand $|\frac{1}{4}-x_0|=|\frac{1}{4}-0|=\frac{1}{4}$ liegt die Abweichung 
$|R_{2}(\frac{1}{4})|<\!<\frac{1}{16}=\big(\frac{1}{4}\big)^2$ im niedrigsten Prozentbereich,
für doppelten Abstand $|\frac{1}{2}-x_0|=\frac{1}{2}$ gilt noch 
$| R_2(\frac{1}{2})|<\!<\frac{1}{4}=\big(\frac{1}{2}\big)^2$, 
während bei größerem Abstand $|1-x_0|=1$ die Abweichung zumindest noch $| R_2(1)|<\!<1=1^2$ erfüllt.
\\\\
Je näher also $x$ an der Entwicklungsstelle $x_0$ liegt, desto besser approximiert das 
Taylor-Polynom $T_n(x)$ die Funktion $f(x)$. Die Abweichung $R_n(x)$ verschwindet, wenn $x=x_0$ ist. 
% Umgekehrt formuliert heißt das: Liegt die Stelle $x$ sehr weit von $x_0$ entfernt, so kann bei festem $n$ der Approximationsfehler $R_n(x)$ schnell
% sehr groß werden.
Außerdem geht die Abweichung $R_2(x)$ in diesem Beispiel stärker gegen Null  als der Abstand zum 
Quadrat $|x-x_0|^{2}$. Das ist kein Zufall, wie uns der nächste Satz zeigt.
}
\lang{en}{
From this exercise we see:
For the $x$ closest to $x_0$, i.e. the smallest distance 
$|\frac{1}{4}-x_0|=|\frac{1}{4}-0|=\frac{1}{4}$, the remainder 
$|R_{2}(\frac{1}{4})|<\!<\frac{1}{16}=\big(\frac{1}{4}\big)^2$ is also the smallest.
For $x$ twice as far, i.e. a distance of $|\frac{1}{2}-x_0|=\frac{1}{2}$ we still have
$| R_2(\frac{1}{2})|<\!<\frac{1}{4}=\big(\frac{1}{2}\big)^2$, 
and taking $x$ further still, i.e. a distance of $|1-x_0|=1$, the remainder is larger but at least 
still satisfies $| R_2(1)|<\!<1=1^2$.
}
\begin{theorem}\label{thm:taylor-approximation}\label{thm:qual-taylor-formel}
\lang{de}{Es sei $f:D\to\R$ eine in $x_0\in D$ $n$-mal stetig differenzierbare Funktion.}
\lang{en}{Let $f:D\to\R$ be an $n$-times differentiable function at $x_0\in D$.}
\begin{enumerate}
\item[(a)]
\notion{\lang{de}{Approximationseigenschaft des Taylor-Polynoms}
        \lang{en}{Properties of a Taylor polynomial approximation}}\\
%Sei $f:[a,b]\rightarrow\R$ eine im Innern des Intervalls $[a,b]\subset\R\;$ $n$-mal differenzierbare Funktion und $T_n(x)$ das $n$-te
%Taylor-Polynom von $f$ mit Entwicklungsstelle $x_0\in (a,b)$.
\lang{de}{Für das Restglied $R_n(x)$ zum Entwicklungspunkt $x_0$ gilt}
\lang{en}{Let $R_n(x)$ be the remainder of the $n$th order Taylor polynomial at the point $x_0$. Then}
\[  \lim_{x\to x_0} \frac{R_{n}(x)}{(x-x_0)^n} =\lim_{x\to x_0} \frac{f(x)-T_{n}(x)}{(x-x_0)^n}=0. \] 
\lang{de}{
Das Restglied geht also selbst dann noch gegen Null, wenn es durch den immer kleiner werdenden 
Abstand $\vert x-x_0\vert$ in $n$-ter Potenz geteilt wird.
}
\lang{en}{
The remainder tends towards zero even when it is divided by the progressively smaller distance 
$\vert x-x_0\vert$ to the power $n$.
}
\item[(b)]
\notion{\lang{de}{Qualitative Taylor-Formel}
        \lang{en}{Qualitative Taylor formula}}\\
\lang{de}{
Es gibt eine in $x_0$ stetige Funktion $r=r_{f,x_0,n}:D\to\R$ mit $r(x_0)=0$ so, dass
}
\lang{en}{
There exists a function $r=r_{f,x_0,n}:D\to\R$, continuous at $x_0$ and with $r(x_0)=0$, such that
}
\[f(x)=T_n(x)+(x-x_0)^n\cdot r(x).\]
\end{enumerate}
\end{theorem}

%\begin{incremental}[\initialsteps{0}]
%\step

%\begin{block}[explanation]
%Da $R_{f,n,a}(x)$ die Differenz zwischen $f$ und dem Taylor-Polynom $P_{f,n,a}(x)$ ist und 
%\[  f^{(k)}(a)=P_{f,n,a}^{(k)}(a) \]
%für alle $0\leq k\leq n$ gilt, ist somit
%\[  R_{f,n,a}^{(k)}(a)=0 \]
%für alle $0\leq k\leq n$. Durch mehrfache Anwendung der \link{l-hospital}{Regel von de l'Hospital}
%erhält man dann
%\begin{eqnarray*}
% \lim_{x\to a} \frac{R_{f,n,a}(x)}{(x-a)^n}&=& \lim_{x\to a} \frac{R'_{f,n,a}(x)}{n(x-a)^{n-1}} \\
% &=& \ldots =  \lim_{x\to a} \frac{R_{f,n,a}^{(n-1)}(x)}{n!\cdot (x-a)^1} \\
% &=&  \lim_{x\to a} \frac{R_{f,n,a}^{(n)}(x)}{n!} \\
%&=& \frac{1}{n!}  R_{f,n,a}^{(n)}(a)=0.
%\end{eqnarray*}
%\end{block}
%\end{incremental}



\begin{proof}\label{proof:Beweis der Taylor-Approximationseigenschaft}
\begin{incremental}[\initialsteps{0}]
\step
\lang{de}{
Zu (a):
Für alle Ableitungsordnungen $j=0,1,\ldots ,n $ gilt 
}
\lang{en}{
For (a):
For every order of derivative $j=0,1,\ldots ,n $ we have
}
\[ f^{(j)}(x_0) = T_n^{(j)}(x_0)\quad\Leftrightarrow\quad f^{(j)}(x_0) - T_n^{(j)}(x_0) = R_n^{(j)}(x_0) = 0. \]
\lang{de}{Damit erhält man durch mehrfache Anwendung der \link{l-hospital}{Regel von de l'Hospital}}
\lang{en}{By applying \link{l-hospital}{l'Hopital's rule} multiple times, we obtain}
\begin{eqnarray*}
 \lim_{x\to x_0} \frac{R_{n}(x)}{(x-x_0)^n}&=& \lim_{x\to x_0} \frac{R'_{n}(x)}{n\cdot (x-x_0)^{n-1}} \\
 &=& \lim_{x\to x_0} \frac{R''_{n}(x)}{n\cdot (n-1)\cdot (x-x_0)^{n-2}} =  \ldots \\
 &=& \lim_{x\to x_0} \frac{R_{n}^{(n-1)}(x)}{n\cdot (n-1)\cdot\ldots\cdot 2\cdot (x-x_0)^1} \\
 &=&  \lim_{x\to x_0} \frac{R_{n}^{(n)}(x)}{n!} \\
&=& \frac{1}{n!} \lim_{x\to x_0} R_{n}^{(n)}(x) = \frac{1}{n!}  R_{n}^{(n)}(x_0)=0.
\end{eqnarray*}
\lang{de}{
Bei der  Grenzwertbildung wurde die Stetigkeit der $j$-ten Ableitung der Restgliedfunktion $R_n(x)$ 
in $x_0$ ausgenutzt für $j=0,\ldots,n$, die sich aus der Stetigkeit von $f^{(j)}$ und $T_n^{(j)}$ in 
$x_0$ ergibt.
}
\lang{en}{
To be able to do this, we use the continuity of the $j$th derivative of the remainder function 
$R_n(x)$ at $x_0$ for $j=0,\ldots,n$, which comes from the continuity of $f^{(j)}$ and $T_n^{(j)}$ 
at $x_0$.
}
\step
\lang{de}{
Zu (b): Die Funktion $r$ ist gegeben durch $r(x)=\frac{R_{n}(x)}{(x-x_0)^n}$. Die Eigenschaft stetig 
in $x_0$ zu sein mit $r(x_0)=0$ ist genau die Aussage von Teil (a).
}
\lang{en}{
For (b): The function $r$ is given by $r(x)=\frac{R_{n}(x)}{(x-x_0)^n}$. Its continuity at $x_0$, follows directly from (a) and $r(x_0)=0$.
}
\end{incremental}
\end{proof}

\begin{example}\label{ex:taylorrestglied-geom-reihe}
\lang{de}{
Für die Funktion $f(x)=\frac{1}{1-x}$ hatten wir in Beispiel \ref{ex:taylorpoly-geom-reihe} die 
Ableitungen $ f^{(k)}(x)=\frac{k!}{(1-x)^{k+1}}$ sowie das Taylor-Polynom $n$-ter Ordnung
}
\lang{en}{
Let $f(x)=\frac{1}{1-x}$. In example \ref{ex:taylorpoly-geom-reihe} we calculated its derivatives 
$ f^{(k)}(x)=\frac{k!}{(1-x)^{k+1}}$ and $n$th order Taylor polynomial
}
\[ T_n(x)=\sum_{k=0}^n \frac{k!}{k!}x^k=\sum_{k=0}^n x^k \]
\lang{de}{an der Stelle $x_0=0$ berechnet. Damit ergibt sich das $n$-te Restglied}
\lang{en}{at the point $x_0=0$. Hence the $n$th remainder is}
\begin{eqnarray*}
R_{n}(x)&=&  f(x)- T_{n}(x)= \frac{1}{1-x}- \sum_{k=0}^n x^k \\
&=&  \frac{1}{1-x} - \frac{1-x^{n+1}}{1-x} \\
&=& \frac{1-(1-x^{n+1})}{1-x}=\frac{x^{n+1}}{1-x}.
\end{eqnarray*} 
\lang{de}{
Beim Schritt in die zweite Zeile wurde die 
\ref[content_02_vollstaendige_induktion][geometrische Summenformel]{rule:geom-summe} 
benutzt. Berechnet man nun an der Stelle $x_0=0$ den entsprechenden Grenzwert des obiger Satzes, 
ergibt sich das erwartete Ergebnis
}
\lang{en}{
For the step in the second row, we use the 
\ref[content_02_vollstaendige_induktion][geometric sum formula]{rule:geom-summe}. If we now 
explicitly calculate the limit from the above theorem at $x_0=0$, we get as expected
}
\[     \lim_{x\to 0} \frac{R_{n}(x)}{x^n}=\lim_{x\to 0} \frac{x^{n+1}}{(1-x)x^n} =\lim_{x\to 0} \frac{x}{1-x}=0. \]
\end{example}


\lang{de}{
Die Abweichung zwischen der Funktion $f(x)$ und ihrem Taylor-Polynom $T_n(x)$ an einer beliebigen 
Stelle $x$ können wir also mittels Restglied exakt berechnen. 
Allerdings muss man dafür die Funktion selbst \emph{und} das Taylor-Polynoms $T_n$ kennen.
Das ist recht rechenintensiv und hilft uns in der Praxis nicht weiter, 
denn wir möchten die Auswertung der Funktion ja eigentlich durch das 
einfacher zu handhabende Polynom $T_n$ abkürzen.
Deshalb sind wir interessiert daran, das Restglied $R_{n}(x)$ für alle $x$ in einer 
\ref[content_21_intervalle][Umgebung]{sec:epsilon-umg} $U(x_0)$ abzuschätzen.
Der erste Schritt dazu ist es, bessere Formeln für das Restglied zu entwickeln.
Eine davon ist das Lagrange-Restglied. Der Beweis dieser Formel, auf den wir hier verzichten, beruht auf dem Mittelwertsatz.
}
\lang{en}{
The remainder can be used to find the exact deviation of a Taylor polynomial $T_n(x)$ from the 
function $f(x)$ at any point $x$. Of course, to find the remainder we need both the function itself 
\emph{and} the Taylor polynomial $T_n$. This is computationally intensive and in practice not very 
useful, as the Taylor polynomial was initially intended to facilitate the evaluation of the 
function. For this reason, we are interested in estimating the remainder $R_{n}(x)$ for all $x$ in 
a \ref[content_21_intervalle][neighbourhood]{sec:epsilon-umg} $U(x_0)$. As a first step towards 
this, let us find a better formula for the remainder. One such formula is the Lagrange remainder. 
The proof of the validity of this formula, which we do not cover here, uses the mean value theorem. 
}
\begin{theorem}[\lang{de}{Lagrange-Restgliedformel}
                \lang{en}{Lagrange remainder formula}]\label{thm:rg-Lagrange}
\lang{de}{
Die Funktion $f:(a;b)\to \R$ sei $(n+1)$-mal differenzierbar. Wir betrachten das $n$-te 
Taylor-Restglied $R_n$ zu einer Entwicklungsstelle $x_0\in (a;b)$. 
Für jedes $x\in (a;b)$ gibt es eine Stelle $\xi$ zwischen $x_0$ und $x$, so dass gilt
}
\lang{en}{
Let the function $f:(a;b)\to \R$ be $(n+1)$-times differentiable. Consider the remainder $R_n$ of 
the $n$th order Taylor polynomial at a point $x_0\in (a;b)$. 
For all $x\in (a;b)$ there exists a point $\xi$ between $x_0$ and $x$ such that
}
\[  R_{n}(x) = \frac{f^{(n+1)}(\xi)}{(n+1)!}\, (x-x_0)^{n+1} .\]
\end{theorem}

\lang{de}{
Das Lagrange-Restglied sieht fast wie der $(n+1)$-te Summand des Taylor-Polynoms aus, 
nur muss die Ableitung nicht bei $x_0$, sondern an einer geeigneten Stelle $\xi$ zwischen zwischen 
$x_0$ und $x$ bestimmt werden. Da die Zwischenstelle $\xi$ zu jeweils gegebenem $x$ unbekannt ist, 
ist es nicht möglich, den Fehler an der Stelle $x$ mit der Lagrange-Restglieddarstellung tatsächlich 
zu berechnen. Die obige Regel sagt lediglich aus, dass eine Stelle $\xi$ existiert, an der diese 
Restgliedformel gilt.\\
Daraus kann man aber die meistgebrauchte Abschätzung herleiten, mit der man in vielen Fällen einen 
Maximalwert für den Fehler $|R_{n}(x)|$ angeben kann.
}
\lang{en}{
The Lagrange remainder does not look dissimilar to the $(n+1)$th term of the Taylor polynomial, 
except that the derivative is evaluated at a point $\xi$ between $x_0$ and $x$ instead of at $x_0$. 
As the point $\xi$ is not known for any particular $x$, it is not possible to calculate the exact 
error at the point $x$ using only the Lagrange remainder. The above rule says only that a point 
$\xi$ exists at which this remainder formula holds.\\
However, we can often use this to deduce a maximal value for the error $|R_{n}(x)|$.
}
\begin{remark}[\lang{de}{Restglied-Abschätzung}
               \lang{en}{Remainder approximation}]\label{rg-abschaetzung}
\lang{de}{
Es sei $f:(a;b)\to\R$ eine $(n+1)$-mal differenzierbare Funktion. 
Wenn es eine obere Schranke $M>0$ für die $(n+1)$-te Ableitung von $f$ gibt, 
wenn also für alle $x\in (a;b)$ gilt
}
\lang{en}{
Let $f:(a;b)\to\R$ be a $(n+1)$-times differentiable functions. 
If there exists an upper bound $M>0$ for the $(n+1)$th derivative of $f$, that is, 
if for all $x\in (a;b)$ we have
}
\[|f^{(n+1)}(x)|\leq M ,\]
\lang{de}{dann gilt  für das Taylor-Restglied die Abschätzung}
\lang{en}{then we have a bound on the remainder of the $n$th order Taylor polynomial,}
\[ {|R_{n}(x) |} 
\leq  \frac{M}{(n+1)!}\, {|x-x_0|}^{n+1} .\]
\begin{incremental}[\initialsteps{0}]
\step
\lang{de}{
Denn durch die Schranke $M$ haben wir die Abschätzung $|f^{(n+1)}(\xi)|\leq M$ auch für das 
spezielle $\xi$ aus dem Lagrange-Restglied. Es gilt also
}
\lang{en}{
This is because the bound $M$ in particular holds for the $\xi$ given by the Lagrange remainder, so $|f^{(n+1)}(\xi)|\leq M$. Hence
}
\[ {|R_{n}(x) |}=  \frac{|f^{(n+1)}(\xi)|}{(n+1)!}\, {|x-x_0|}^{n+1} 
\leq  \frac{M}{(n+1)!}\, {|x-x_0|}^{n+1} .\]
\end{incremental}
\end{remark}
\lang{de}{
Eine Zusammenfassung zur Thematik des Taylor-Restglieds inklusive Rechenbeispielen ist dem folgenden Video zu entnehmen:\\
\floatright{\href{https://api.stream24.net/vod/getVideo.php?id=10962-2-10739&mode=iframe&speed=true}{\image[75]{00_video_button_schwarz-blau}}}\\\\
Eine alternative Erklärung des Taylor-Restglieds und ein zusätzliches konkretes Rechenbeispiel können den folgenden zwei Videos entnommen werden:\\
\floatright{
    \href{https://www.hm-kompakt.de/video?watch=540}{\image[75]{00_Videobutton_schwarz}}
    \href{https://www.hm-kompakt.de/video?watch=541}{\image[75]{00_Videobutton_schwarz}}
}\\\\
}
\lang{en}{}


\begin{example}\label{ex:taylorrestglied-exp}
\lang{de}{Wir entwickeln die Exponentialfunktion $f(x)=e^x$ an der Stelle $x_0=0$.}
\lang{en}{We expand the exponential function $f(x)=e^x$ at the point $x_0=0$.}
\begin{incremental}[\initialsteps{0}]
\step
\lang{de}{
Das Taylor-Polynom $n$-ten Grades $T_{n}(x)=\sum_{k=0}^n \frac{1}{k!}x^k$ zum Entwicklungspunkt 
$x_0=0$ hatten wir bereits in Beispiel~\ref{ex:taylorpoly-exp} berechnet. Gemäß der 
Lagrange-Restgliedformel gibt es also zu jedem $x\in (a,b)\subset\R$ eine Zahl $\xi$ zwischen $0$ 
und $x$ mit
}
\lang{en}{
We have already calculated the $n$th order Taylor polynomial $T_{n}(x)=\sum_{k=0}^n \frac{1}{k!}x^k$ 
at the point $x_0=0$ in example~\ref{ex:taylorpoly-exp}. By the Lagrange remainder formula there 
hence exists for each  $x\in (a,b)\subset\R$ a number $\xi$ between $0$ and $x$ such that
}
\[ R_{n}(x)=e^x-  T_{n}(x) =\frac{e^{\xi}}{(n+1)!}\, x^{n+1} .\]
\lang{de}{
Betrachten wir das Restglied $R_{n}(x)$ zu $T_{n}(x)$ von $e^x$ lediglich in der Umgebung 
$U_{\frac{1}{2}}(0)=(-\frac{1}{2};\frac{1}{2})$ des Entwicklungspunkts $x_0=0$, 
dann können wir eine obere Schranke $M$ für die $(n+1)$-te Ableitung angeben. 
Wegen der strengen Monotonie der Exponentialfunktion gilt nämlich für alle $x\in U_{\frac{1}{2}}(0)$ 
die Abschätzung $\left|e^x\right|=e^x < e^{\frac{1}{2}}=\sqrt{e}$. Wir können also $M:=\sqrt{e}$ 
wählen. Damit ergibt sich die Restglied-Abschätzung
}
\lang{en}{
Consider the remainder $R_{n}(x)$ of $T_{n}(x)$ for the function $e^x$ only in the neighbourhood 
$U_{\frac{1}{2}}(0)=(-\frac{1}{2};\frac{1}{2})$ of the point $x_0=0$. We can give an upper bound 
$M$ for the $(n+1)$th derivative. As the exponential function is strictly monotone increasing, we 
have $\left|e^x\right|=e^x < e^{\frac{1}{2}}=\sqrt{e}$ for all $x\in U_{\frac{1}{2}}(0)$. We can 
therefore choose $M:=\sqrt{e}$ and obtain
}
\[ \left| R_{n}(x)\right| = \left| e^x-  T_{n}(x) \right| < \frac{\sqrt{e}}{(n+1)!}\,\left| x^{n+1}\right| <  \frac{\sqrt{e}}{(n+1)!\cdot 2^{n+1}} .\]
\lang{de}{
Schon für $n=5$ wird der Fehler 
$\left| R_{5}(x)\right| <\frac{\sqrt{e}}{6!\cdot  2^{6}}\approx 4\cdot 10^{-5}$ sehr klein.
}
\lang{en}{
Even for $n=5$, the error 
$\left| R_{5}(x)\right| <\frac{\sqrt{e}}{6!\cdot  2^{6}}\approx 4\cdot 10^{-5}$ is very small.
}
\end{incremental}
\end{example}


	\begin{quickcheck}
		\field{rational}
%		\type{mc.unique}
		\begin{variables}
			\randint[Z]{n}{2}{6}
			\randint[Z]{a}{-3}{3}
            \function[calculate]{am1}{a-1}
            \function[calculate]{ap1}{a+1}
            \function[calculate]{np1}{n+1}
			\randint{v}{0}{1}
            \function[normalize]{f}{v*sin(x)+(1-v)*cos(x)}
		\end{variables}
		
		\text{\lang{de}{
    Welche der Abschätzung des Restgliedes $R_{\var{n}}(x)$
		der Funktion $f(x)=\var{f}$ in der Umgebung der Stelle $x_0=\var{a}$ sind korrekt?
    }
    \lang{en}{
    Which of the following bounds on the Taylor remainder $R_{\var{n}}(x)$ of the function 
    $f(x)=\var{f}$ in the neighbourhood of the point $x_0=\var{a}$ are correct?
    }}
%        \permutechoices{1}{3}
\begin{choices}{multiple}
\begin{choice}
			\text{$|R_{\var{n}}(x)|\leq \frac{1}{\var{np1}!}$ \lang{de}{für alle}\lang{en}{for all} $x\in [\var{am1};\var{ap1}]$}
            \solution{true}
       	\end{choice}
		\begin{choice}
			\text{$|R_{\var{n}}(x)|\leq \frac{1}{\var{np1}!}$ \lang{de}{für alle}\lang{en}{for all} $x\in \R$}
            \solution{false}
       	\end{choice}
		\begin{choice}
			\text{$|R_{\var{n}}(x)|\leq \frac{1}{\var{np1}!}\big(\frac{1}{2})^{\var{np1}}$ 
			\lang{de}{für alle}\lang{en}{for all} $x\in [-\frac{1}{2};\frac{1}{2}]$}
            \solution{false}
       	\end{choice}
\end{choices}
\explanation{\lang{de}{
Die höheren Ableitungen von $\var{f}$ sind $\pm \sin(x)$ und $\pm\cos(x)$, weshalb für alle
$\xi\in \R$ die Abschätzung $|f^{(\var{np1})}(\xi)|\leq 1$ gilt, und daher
$|R_{\var{n}}(x)|\leq \frac{1}{\var{np1}!}\cdot {|x-(\var{a})|}^{\var{np1}}$ für alle $x\in \R$.
}
\lang{en}{
The higher derivatives of $\var{f}$ are $\pm \sin(x)$ und $\pm\cos(x)$, so the bound 
$|f^{(\var{np1})}(\xi)|\leq 1$ holds for all $\xi\in \R$, and therefore 
$|R_{\var{n}}(x)|\leq \frac{1}{\var{np1}!}\cdot {|x-(\var{a})|}^{\var{np1}}$ for all $x\in \R$.
}}
	\end{quickcheck}



\section{\lang{de}{Taylor-Reihe}\lang{en}{Taylor series}}\label{sec:reihe}
\lang{de}{
Bisher haben wir uns mit der Approximation von Funktionen durch Taylor-Polynome $T_n$ beschäftigt. 
Wenn eine Funktion allerdings beliebig oft differenzierbar ist, dann können können wir auch 
Taylor-Polynome beliebig hoher Ordnung betrachten und überlegen, ob der Grenzwertprozess 
$\lim_{n\to\infty}T_n$ sinnvoll ist. Eigentlich erwarten wir, dass mit wachsendem $n$ die Funktion 
um die Entwicklungsstelle $x_0$ herum immer besser durch ihre Taylor-Polynome $T_n$ ausgedrückt 
wird. Das ist oft, aber nicht immer, der Fall, wie wir weiter unten sehen werden.
}
\lang{en}{
Up to this point, we have considered approximations of functions using Taylor polynomials $T_n$. 
However, if a function is differentiable arbitrarily many times, then a Taylor polynomial of any 
order can be found, and consider whether it makes sense to consider the limit 
$\lim_{n\to\infty}T_n$. An obvious expectation here is that as $n$ increases, the Taylor polynomial 
$T_n$ approximates the function progressively better in a neighbourhood of $x_0$. As we see below, 
this is often, but not always, the case.
}

\begin{definition}\label{def:taylorreihe}
\lang{de}{
Ist $f:(a;b)\to \R$ eine in $x_0\in(a;b)$ unendlich oft differenzierbare Funktion,
dann heißt die \ref[content_27_konvergenzradius][Potenzreihe]{def:Potenzreihe}
}
\lang{en}{
Let $f:(a;b)\to \R$ be a function that is infinitely differentiable on $x_0\in(a;b)$. Then 
the \ref[content_27_konvergenzradius][power series]{def:Potenzreihe}
}
\[    T(x)\coloneq T_{\infty}(x)= \sum_{k=0}^\infty \frac{f^{(k)}(x_0)}{k!}(x-x_0)^k \]
\lang{de}{
die \notion{Taylor-Reihe} von $f$ zur Entwicklungsstelle $x_0$. Die Koeffizienten $a_k$ der 
Potenzreihe sind genau die \ref[content_04_taylor_polynom][Taylor-Koeffizienten]{def:taylorpoly}
}
\lang{en}{
is called the \notion{Taylor series} of $f$ at the point $x_0$. The coefficients $a_k$ of the power 
series are precisely the \ref[content_04_taylor_polynom][Taylor coefficients]{def:taylorpoly}
}
\[a_k=\frac{f^{(k)}(x_0)}{k!}.\]
\end{definition}

\begin{remark}
\begin{enumerate}
\item \lang{de}{
Der \ref[content_27_konvergenzradius][Konvergenzbereich]{thm:konvergenzbereich} der Taylor-Reihe ist 
durch ihre Eigenschaft, eine Potenzreihe zu sein, gegeben. Er wird bestimmt durch den 
\ref[content_27_konvergenzradius][Konvergenzradius]{def:konvergenzradius} $R$, für den wir 
\ref[content_27_konvergenzradius][Berechnungsmethoden]{sec:konvergenzradius} kennen.\\
Ist $R>0$, so konvergiert die Taylorreihe zum Entwicklungspunkt $x_0$ absolut auf dem Intervall 
$(x_0-R;x_0+R)$. Ist $R=0$, so konvergiert die Taylor-Reihe nur im Entwicklungspunkt $x_0$.
Dieses Konvergenzgebiet muss nicht mit dem ursprünglichen Definitionsbereich von $f$ übereinstimmen.
}
\lang{en}{
The \ref[content_27_konvergenzradius][domain of convergence]{thm:konvergenzbereich} of the Taylor 
series is found by considering it as a power series. It is determined by the 
\ref[content_27_konvergenzradius][radius of convergence]{def:konvergenzradius} $R$, which we 
\ref[content_27_konvergenzradius][have already seen how to calculate]{sec:konvergenzradius}.\\
If $R>0$, then the Taylor series at the point $x_0$ converges absolutely on the interval 
$(x_0-R;x_0+R)$. If $R=0$, then the Taylor series only converges at the point $x_0$. The domain in 
which the series converges need not correspond to the original domain of $f$.
}
\item \lang{de}{
Unter allen Potenzreihen ist die Taylor-Reihe zur Entwicklungsstelle $x_0$ die einzige Kandidatin, 
die die Chance hat, die Funktion $f$ in einer Umgebung von $x_0$ darzustellen. Denn wenn $f$ in der 
Form $f(x)= \sum_{k=0}^\infty a_k (x-x_0)^k$ darstellbar ist, auf einer geeigneten Umgebung $U(x_0)$,
dann sind die Koeffizienten $a_k$ durch die Bedingung $a_k=\frac{f^{(k)}(x_0)}{k!}$ festgelegt.
}
\lang{en}{
Out of all the power series, the Taylor series at the point $x_0$ is the only candidate with a 
possibility of perfectly describing the function $f$ in a neighbourhood of $x_0$. After all, if $f$ 
can be written in the form $f(x)= \sum_{k=0}^\infty a_k (x-x_0)^k$ on some neighbourhood $U(x_0)$, 
then the coefficients $a_k$ are determined by the condition $a_k=\frac{f^{(k)}(x_0)}{k!}$.
}
\begin{incremental}[\initialsteps{0}]
\step \lang{de}{
Um das einzusehen bemerken wir zunächst, dass $f$ in diesem Fall unendlich oft differenzierbar ist, 
weil das alle Potenzreihen auf ihrem Konvergenzbereich sind. Die Ableitungen $f^{(j)}$ von
$f$ müssen also auf $U(x_0)$ mit denen der Potenzreihe übereinstimmen. Weil 
\ref[content_02_ableitungsregeln][Potenzreihen gliedweise abgeleitet]{thm:potenzreihen-ableitung}
werden, erhalten wir zunächst die erste Ableitung
}
\lang{en}{
In order to show this, we note that $f$ is infinitely differentiable on $U(x_0)$ in this case, as 
power series are infinitely differentiable on their domain of convergence. The 
derivatives $f^{(j)}$ of $f$ must therefore correspond to those of the power series on $U(x_0)$. As 
\ref[content_02_ableitungsregeln][power series are differentiated term-by-term]{thm:potenzreihen-ableitung}, 
we obtain the first derivative
}
\[ f'(x)=\sum_{k=1}^\infty k\,a_k\,(x-x_0)^{k-1},\]
\lang{de}{und induktiv dann die $j$-te Ableitung}
\lang{en}{and inductively the $j$th derivative}
\[ f^{(j)}(x)=\sum_{k=j}^\infty k\,(k-1)\cdot\ldots\cdot (k-(j-1))\,a_k\,(x-x_0)^{k-j}. \]
\lang{de}{Die Auswertung in $x_0$ liefert daraus}
\lang{en}{Evaluating at $x_0$ yields}
\[f^{(j)}(x_0)=j\cdot (j-1)\cdot\ldots\cdot (j-j+1)\cdot a_j\cdot(x_0-x_0)^0=j!a_j.\]
\lang{de}{Also sind die Koeffizienten $a_j$ genau die Taylor-Koeffizienten.}
\lang{en}{Hence the coefficients $a_j$ are precisely the Taylor coefficients.}
% Damit können wir die in~\ref{def:taylorreihe} definierte Taylor-Reihe $T(x)$ eindeutig auf 
% die vorausgesetzte Potenzreihendarstellung von $f(x)$ zurückführen:
% \[  T(x)=\sum_{k=0}^\infty \frac{f^{(k)}(x_0)}{k!}(x-x_0)^k = \sum_{k=0}^\infty \frac{k!a_k}{k!}(x-x_0)^k =\sum_{k=0}^\infty a_k (x-x_0)^k=f(x). \]
\end{incremental}
\item \lang{de}{
Ob eine Funktion $f:(a;b)\to\R$ tatsächlich in einer Umgebung vom Entwicklungspunkt $x_0$ durch ihre 
Taylor-Reihe $T$ in $x_0$ darsgestellt wird, hängt von folgenden Bedingungen ab.
}
\lang{en}{
Whether a function $f:(a;b)\to\R$ is equal to its Taylor series $T$ at $x_0$ in a neighbourhood of 
$x_0$ is conditional on the following:
}
\begin{enumerate}
\item[(i)] \lang{de}{Die Taylor-Reihe benötigt einen positiven Konvergenzradius $R>0$.}
           \lang{en}{The Taylor series must have a positive radius of convergence $R>0$.}
\item[(ii)] \lang{de}{
           Falls $R>0$, ist der Durchschnitt der beiden Definitionsbereiche $(a;b)\cap(x_0-R;x_0+R)$ 
           eine Umgebung $U(x_0)$, auf der $f$ potentiell durch die Taylor-Reihe dargestellt werden 
           könnte.
           }
           \lang{en}{
           If $R>0$, the intersection of the two domains $(a;b)\cap(x_0-R;x_0+R)$ is a neighbourhood 
           $U(x_0)$ on which $f$ may possibly be equal to its Taylor series.
           }
\item[(iii)] \lang{de}{
           Für jedes Element $x$ aus dieser Umgebung $U(x_0)$ gilt es dann zu prüfen, ob 
           }
           \lang{en}{
           We then need to check if, for all $x$ in this neighbourhood $U(x_0)$, we have
           }
           \[\lim_{n\to\infty}R_n(x)=\lim_{n\to\infty}(f(x)-T_n(x))=0.\]
\end{enumerate}  
\lang{de}{
Sind diese Bedingungen allesamt erfüllt, dann wird $f$ auf $U(x_0)$ durch seine Taylor-Reihe 
$T_{x_0}$ im Entwicklungspunkt $x_0$ dargestellt.
}
\lang{en}{
If the above conditions are all fulfilled, then $f$ is given on $U(x_0)$ precisely by its Taylor 
series $T_{x_0}$ at the point $x_0$.
}
\end{enumerate}
\end{remark}

\lang{de}{
Funktionen, die die Bedingungen (i)-(iii) in obiger Bemerkung sogar in jedem Punkt $x_0$ des 
Definitionsbereichs erfüllen, erhalten einen eigenen Namen.
}
\lang{en}{
Functions that satisfy the above conditions (i)-(iii) at every point $x_0$ of its domain are given a 
name.
}

\begin{definition}[\lang{de}{Analytische Funktionen}
                   \lang{en}{Analytic functions}]\label{thm:reell-analyt-Fktn}\label{def:reell-analyt-Fktn}
\lang{de}{
Die Funktion $f:D\to \R$ sei unendlich oft differenzierbar. Gibt es zu jedem $x_0\in D$ eine 
Umgebung $U(x_0)\subset D$, auf der $f$ durch seine Taylor-Reihe $T_{x_0}$ dargestellt wird, 
so heißt $f$ \notion{(reell-)analytisch}.
}
\lang{en}{
Let the function $f:D\to \R$ be infinitely differentiable. Suppose for all $x_0\in D$ there exists 
a neighbourhood $U(x_0)\subset D$ on which $f$ is given precisely by its Taylor series $T_{x_0}$. 
Them $f$ is called \notion{(real) analytic}.
}
\end{definition}
\lang{de}{
Die Eigenschaft, analytisch zu sein, ist zwar eine ganz besondere, sie wird aber von den meisten uns
bekannten unendlich oft differenzierbaren Funktionen erfüllt.
}
\lang{en}{
The property of being analytic is quite special, but many well-known infinitely differentiable 
functions are in fact analytic.
}
\begin{example}\label{ex:reell-analytische-Funktionen}
\begin{tabs*}[\initialtab{0}]
\tab{\lang{de}{Polynome}\lang{en}{Polynomials}}
\lang{de}{
Die einfachsten analytischen Funktionen sind die Polynomfunktionen $p:\R\to\R$. Nach Satz 
\ref{thm:thm:polynome-taylor-entwickelt} gilt für alle $n\geq \text{grad}(p)$, dass $p=T_n$ für 
jedes Taylor-Polynom $T_n$ zu jedem beliebigen Entwicklungspunkt $x_0\in\R$. Also ist für alle 
$x\in\R$
}
\lang{en}{
The simplest analytic functions are the polynomials $p:\R\to\R$. By theorem 
\ref{thm:thm:polynome-taylor-entwickelt} we have $p=T_n$ for all $n\geq \text{grad}(p)$ where 
$T_n$ is a Taylor polynomial of $p$ at any point $x_0\in\R$. Therefore for all $x\in\R$ we have
}
\[\lim_{n\to\infty} R_n(x)=\lim_{n\to\infty}(f(x)- T_n(x))=\lim_{n\to\infty} 0=0\]
\lang{de}{
erfüllt. Ebenso hat die Taylorreihe $T=\lim_{n\to\infty}T_n=p$ als endliche Summe den 
Konvergenzradius unendlich, ist also überall absolut konvergent. Eine Polynomfunktion stimmt in 
jedem Punkt $x\in\R$ mit seiner Taylor-Reihe überein.
}
\lang{en}{
and the Taylor series $T=\lim_{n\to\infty}T_n=p$ is a finite sum, and hence has an infinite radius 
of convergence. That is, $T$ is absolutely convergent. A polynomial is equal to its Taylor series 
at every point $x\in\R$.
}
\tab{\lang{de}{Exponentialfunktion}\lang{en}{Exponential function}}
\lang{de}{
Für die Exponentialfunktion $f:x\mapsto e^x$ wissen wir bereits, dass diese mit der (reellen) 
Exponentialreihe $\exp(x)=\sum_{n=0}^\infty \frac{x^n}{n!}$ übereinstimmt. Das sieht man zunächst 
für \ref[content_28_exponentialreihe][rationale $x\in\Q$]{thm:exp_auf_QQ}, und dann aus 
Stetigkeitsgründen für beliebige $x\in\R$. Die Exponentialreihe hat einen 
\ref[content_27_konvergenzradius][Konvergenzradius gleich unendlich]{ex:geom-und-exp-reihe}. 
Die Taylor-Polynome $T_n$ von $f$ zum Entwicklungspunkt $x_0=0$ (siehe Beispiel 
\ref{ex:taylorpoly-exp}) sind genau die Partialsummen der Exponentialreihe. Somit ist für alle 
$x\in\R$
}
\lang{en}{
We know that the exponential function $f:x\mapsto e^x$ is also given by the (real) power series 
$\exp(x)=\sum_{n=0}^\infty \frac{x^n}{n!}$. This can be shown first for 
\ref[content_28_exponentialreihe][rational $x\in\Q$]{thm:exp_auf_QQ}, and then extended by 
continuity for any $x\in\R$. The series has an 
\ref[content_27_konvergenzradius][infinite convergence ratio]{ex:geom-und-exp-reihe}. 
The Taylor polynomials $T_n$ of $f$ at the point $x_0=0$ (see example \ref{ex:taylorpoly-exp}) are 
precisely the partial sums of this power series. Hence for all $x\in\R$,
}
\[\lim_{n\to\infty}(e^x-T_n(x))=e^x-\exp(x)=0,\]
\lang{de}{
also: Die Taylor-Reihe in $x_0=0$ stellt überall die Exponentialfunktion dar.\\
Auch die Taylor-Entwicklung zu einem beliebigen anderen Entwicklungspunkt $x_0\in\R$ können wir mit 
Hilfe der 
\ref[content_28_exponentialreihe][Funktionalgleichung der Exponentialfunktion]{thm:eigenschaften_von_exp} leicht angeben. 
Weil für alle $x\in\R$ gilt
}
\lang{en}{
so: the Taylor series expanded at $x_0=0$ is equal to the exponential function.\\
The Taylor expansion at even an arbitrary point $x_0\in\R$ can easily be found using the 
\ref[content_28_exponentialreihe][properties of the exponential function]{thm:eigenschaften_von_exp}. As we have
}
\[e^x=e^{x_0}\cdot e^{x-x_0},\]
\lang{de}{folgt}
\lang{en}{for all $x\in\R$, it follows that}
\[e^x=e^{x_0}\cdot \sum_{n=0}^\infty \frac{(x-x_0)^n}{n!}=\sum_{n=0}^\infty e^{x_0}\cdot \frac{(x-x_0)^n}{n!}.\]
\lang{de}{
Dies ist eine Potenzreihe mit Entwicklungspunkt $x_0$, die in jedem Punkt mit der 
Exponentialfunktion übereinstimmt. Es muss sich also um die Taylor-Reihe von $f(x)=e^x$ im 
Entwicklungspunkt $x_0$ handeln. In der Tat ist $f^{(n)}(x_0)=e^{x_0}$.
\\\\
Wir haben also zu jedem $x_0\in\R$ eine auf ganz $\R$ konvergente Taylor-Reihe gefunden. 
Die Exponentialfunktion ist analytisch.
}
\lang{en}{
This is a power series expanded at a point $x_0$, whose value at every point corresponds to that of 
the exponential function. Hence it must be the Taylor series of $f(x)=e^x$ at the point $x_0$. 
Indeed, $f^{(n)}(x_0)=e^{x_0}$.
\\\\
We have found for each $x_0\in\R$ a Taylor series that converges on the whole of $\R$. The 
exponential function is analytic.
}


% Um die Exponentialfunktion $f(x)=e^x$ in der Umgebung $U(x_0)$ mit $x_0=0$ näherungsweise darzustellen, hatten wir in Beispiel \ref{ex:taylorpoly-exp}
% das Taylor-Polynom bis zum Grad $n$ berechnet:
% \[ T_{n}(x)=\sum_{k=0}^n \frac{1}{k!}x^k .\]
% In Beispiel~\ref{ex:taylorrestglied-exp} hatten wir dann das zugehörige Restglied gemäß \textsc{Lagrange} bestimmt:
% \[ R_{n}(x)=\frac{e^{\xi}}{(n+1)!}\, x^{n+1} .\]
% Dabei gibt es zu jedem $x\in (a,b)\subset\R$ eine unbekannte Zahl $\xi$ zwischen $0$ und $x$.

% Mit der \ref[content_04_taylor_polynom][Abschätzung des Restgliedes]{rg-abschaetzung} folgt wegen der Monotonie der Exponentialfunktion
% für den gesamten Definitionsbereich, d.h. für jedes $x\in \R$:
% \[ \left|R_{n}(x)\right|\leq e^{\left| x\right|}\frac{\left| x\right|^{n+1}}{(n+1)!}\, . \]
% Für jedes $x\in\R$ ist aber $\lim_{n\to \infty} \frac{|x|^{n+1}}{(n+1)!}=0$, womit auch $\lim_{n\to\infty} \left|R_n(x)\right|=0$ gezeigt ist.
% Deshalb konvergiert die Taylor-Reihe
% \[ \sum_{k=0}^\infty \frac{1}{k!}x^k  = T(x) = f(x) = e^x = \exp(x) \]
% gegen die Funktion $f(x)=e^x$ auf deren Definitionsbereich, d.h. für jedes $x\in\R$, und stellt diese exakt und vollständig dar.  
% Die Taylor-Reihendarstellung ist also identisch mit der bereits aus Abschnitt~\ref{sec:exp-reihe} bekannten Potenzreihendarstellung
% für die reell analytische Exponentialfunktion $f(x)=e^x$.
\tab{$\frac{1}{1-x}$}
\lang{de}{
Für die Funktion $f:\R\setminus\{1\}\to\R$, $f(x)=\frac{1}{1-x}$, hatten wir in 
Beispiel~\ref{ex:taylorpoly-geom-reihe} das Taylor-Polynom $n$-ten Grades
}
\lang{en}{
Consider the function $f:\R\setminus\{1\}\to\R$, $f(x)=\frac{1}{1-x}$. 
In example~\ref{ex:taylorpoly-geom-reihe}, we found the $n$th order Taylor polynomial
}
\[ T_{n}(x)=\sum_{k=0}^n \frac{k!}{k!}x^k=\sum_{k=0}^n x^k \]
\lang{de}{
zur Entwicklungsstelle $x_0=0$ und in Beispiel~\ref{ex:taylorrestglied-geom-reihe} das zugehörige 
Restglied
}
\lang{en}{
at the point $x_0=0$, and in example~\ref{ex:taylorrestglied-geom-reihe} we found the corresponding 
remainder
}
\[ R_{n}(x)=\frac{x^{n+1}}{1-x}\]
\lang{de}{
berechnet. Damit ergibt sich die Taylor-Reihe als \ref[content_27_konvergenzradius][geometrische Reihe]{ex:geom-und-exp-reihe}
}
\lang{en}{
of the Taylor polynomial. Hence the Taylor series is a 
\ref[content_27_konvergenzradius][geometric series]{ex:geom-und-exp-reihe}
}
\[ T(x)=\sum_{k=0}^\infty  x^k, \]
\lang{de}{
die den Konvergenzradius $R=1$ hat. Wählen wir also $U(0) = (-1;1)$, dann gilt für jedes $x\in U(0)$
}
\lang{en}{
with radius of convergence $R=1$. If we choose $U(0) = (-1;1)$, then for all $x\in U(0)$ we have
}
\[\lim_{n\to \infty} R_{n}(x)=\lim_{n\to \infty} \frac{x^{n+1}}{1-x}=0.\] 
\lang{de}{
Die Funktion $f(x)=\frac{1}{1-x}$ wird also nur im Intervall $(-1,1)$ durch die geometrische Reihe 
exakt dargestellt. Der Konvergenzbereich der Taylor-Reihe ist somit kleiner als der 
Definitionsbereich der Funktion.
\\\\
Auch an jede andere Entwicklungsstelle $x_0\neq 1$ können wir $f$ Taylor-entwickeln. Dazu nutzen wir 
die Ableitungen von $f$, die wir schon in Beispiel~\ref{ex:taylorpoly-geom-reihe} bestimmt hatten zu 
$f^{(n)}(x)=\frac{N!}{(1-x)^{n+1}}$. Die Taylor-Reihe zur Entwicklungsstelle $x_0$ ist also
}
\lang{en}{
The function $f(x)=\frac{1}{1-x}$ is therefore only equal to the geometric series in the interval 
$(-1,1)$. The domain on which the Taylor series converges is thus smaller than the domain that we 
originally defined the function on.
\\\\
We may also consider the Taylor expansion of $f$ at any other point $x_0\neq 1$. We use the 
derivatives of $f$ that we determined in 
example~\ref{ex:taylorpoly-geom-reihe} to be $f^{(n)}(x)=\frac{N!}{(1-x)^{n+1}}$. The 
Taylor series at the point $x_0$ is therefore
}
\[T_{x_0}(x)=\sum_{n=0}^\infty \frac{(x-x_0)^n}{(1-x_0)^{n+1}}.\]
\lang{de}{
Auch das ist eine geometrische Reihe, nämlich $\frac{1}{1-x_0}\cdot\sum_{n=0}^\infty q^n$ für 
$q=\frac{x-x_0}{1-x_0}$. Sie konvergiert absolut genau für $| q|<1$, also für $|x-x_0|<|1-x_0|$. 
Damit ist der Konvergenzradius von $T_{x_0}$ gleich $R_{x_0}=|1-x_0|$. Berechnen wir für 
$x\in U(x_0):=(x_0-R_{x_0};x_0+R_{x_0})$ den Grenzwert
}
\lang{en}{
This is also a geometric series, namely $\frac{1}{1-x_0}\cdot\sum_{n=0}^\infty q^n$ for 
$q=\frac{x-x_0}{1-x_0}$. It converges absolutely exactly when $| q|<1$, that is, $|x-x_0|<|1-x_0|$. 
The convergence radius of $T_{x_0}$ is therefore $R_{x_0}=|1-x_0|$. We calculate, for 
$x\in U(x_0):=(x_0-R_{x_0};x_0+R_{x_0})$, the limits
}
\begin{align*}
\lim_{n\to\infty}(f(x)-T_{x_0,n})=&\frac{1}{1-x}-\frac{1}{1-x_0}\cdot\frac{1}{1-q}\\
=&
\frac{1}{1-x}-\frac{1}{(1-x_0)(1-\frac{x-x_0}{1-x_0})}=
\frac{1}{1-x}-\frac{1}{1-x}=0,
\end{align*}
\lang{de}{
so folgt, dass $f$ auf $U(x_0)$ durch die Taylor-Reihe $T_{x_0}$ dargestellt wird.
Auch $f$ ist somit analytisch.
}
\lang{en}{
from which we deduce that $f$ is given exactly by the Taylor series $T_{x_0}$ in the neighbourhood 
$U(x_0)$. $f$ is therefore analytic.
}

\end{tabs*}
\end{example}
\lang{de}{
Aber auch eine unendlich oft differenzierbare Funktion, die nicht analytisch ist, ist kennen wir 
bereits:
}
\lang{en}{
We even know of an infinitely differentiable function which is not analytic:
}
\begin{example}
\lang{de}{Die Funktion $f:\R\to\R$,}
\lang{en}{The function $f:\R\to\R$,}
\begin{displaymath}
  f(x)=
  \begin{cases}
    e^{-\frac{1}{x}}, & x>0,\\
    0, & x\leq 0,
  \end{cases}
 \end{displaymath}
  \lang{de}{
  ist nicht reell-analytisch, denn die Taylor-Reihe zum Entwicklungspunkt $x_0=0$ stellt 
  in keiner Umgebung von $x_0$ die Funktion dar.
  }
  \lang{en}{
  is not real-analytic, as the Taylor series at the point $x_0=0$ does not equal the function in 
  any neighbourhood of $x_0$.
  }
  \begin{incremental}[\initialsteps{0}]
  \step
\lang{de}{
Im Beispiel~\ref{ex:gegenbeispiel} hatten wir gesehen, $f$ in $x_0=0$ unendlich oft differenzierbar 
ist mit $f^{(k)}(0)=0$ für alle $k\in\N_0$. Deshalb ist an der Entwicklungsstelle $x_0=0$ die 
Taylor-Reihe 
}
\lang{en}{
In example~\ref{ex:gegenbeispiel} we saw that $f$ is infinitely differentiable at $x_0=0$, with 
$f^{(k)}(0)=0$ for all $k\in\N_0$. Hence the Taylor series at the point $x_0=0$ is
}
\[ T(x)= \sum_{k=0}^\infty \frac{0}{k!}x^k =0 \]
\lang{de}{
die Nullfunktion. Natürlich stellt diese Taylor-Reihe die Funktion $f$ im Bereich $(-\infty;0]$ dar.
Für jedes $x>0$ ist aber $f(x)-T(x)=e^{-\frac{1}{x}}>0$. Es gibt also keine Umgebung von $x_0$, in 
der $T$ und $f$ vollständig übereinstimmen. Die Funktion $f$ ist also nicht analytisch.
}
\lang{en}{
the zero function. Of course this Taylor series represents the function $f$ on $(-\infty;0]$. 
However, for all $x>0$ we have $f(x)-T(x)=e^{-\frac{1}{x}}>0$. There hence does not exist a 
neighbourhood of $x_0$ in which $T$ equals $f$ for all values. The function $f$ is thus not analytic.
}
%Offenkundig konvergiert die Taylor-Reihe von $f(x)$ zur Entwicklungsstelle $x_0=0$ überall (gegen die Nullfunktion), für $x>0$ aber nicht gegen $f(x)$.
\end{incremental}
%Ein weiteres Beispiel einer überall unendlich oft differenzierbaren, aber nicht reell-analytischen Funktion ist $h:\R\to\R$, $h(x)=e^{1\frac{1}{x^2}}$ für $x\neq 0$ und $h(0)=0$.
\end{example}

\begin{quickcheck}
		\field{rational}
%		\type{mc.unique}

		\begin{variables}
			\randint[Z]{t}{-1}{1}
			\randint[Z]{a}{-5}{-2}
			\function[calculate]{a2}{t*a} 
			\function[calculate]{a1}{-a2} 
			\function{f}{1/(1-x)}
            \function[normalize]{m1}{x-a1}
            \function[normalize]{m2}{x-a2}
            \function[calculate]{c}{1/(1-a)} 
			
		\end{variables}
		
		\text{\lang{de}{
    Welche der folgenden Potenzreihen ist die Taylorreihe der Funktion $f(x)=\var{f}$
		zur Entwicklungsstelle $x_0=\var{a}$?
    }
    \lang{en}{
    Which of the following power series is the Taylor series of the function $f(x)=\var{f}$ at the 
    point $x_0=\var{a}$?
    }}
%        \permutechoices{1}{3}
\begin{choices}{multiple}
		\begin{choice}
			\text{$\sum_{n=0}^\infty x^n$}
            \solution{false}
       	\end{choice}
        \begin{choice}
			\text{$\sum_{n=0}^\infty (\var{c})^{n+1}(\var{m1})^n$}
            \solution[compute]{t = -1}
       	\end{choice}
		\begin{choice}
			\text{$\sum_{n=0}^\infty (\var{c})^{n+1}(\var{m2})^n$}
            \solution[compute]{t = 1}
       	\end{choice}
\end{choices}
\explanation{\lang{de}{
Die Taylorreihe von $f$ mit Entwicklungsstelle $x_0$ ist eine Potenzreihe in
Potenzen des Linearfaktors $(x-(\var{a}))$.
}
\lang{en}{
The Taylor series of $f$ at the point $x_0$ is a power series in powers of the linear factor 
$(x-(\var{a}))$.
}}
	\end{quickcheck}
    
\lang{de}{
Eine Zusammenfassung der in Abschnitt \ref{sec:reihe} angesprochenen Themen kann dem folgenden Video 
entnommen werden:\\
\floatright{\href{https://api.stream24.net/vod/getVideo.php?id=10962-2-10740&mode=iframe&speed=true}{\image[75]{00_video_button_schwarz-blau}}}
}
\lang{en}{}

\end{content}
