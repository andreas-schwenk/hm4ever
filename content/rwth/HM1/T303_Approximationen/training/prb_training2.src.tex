\documentclass{mumie.problem.gwtmathlet}
%$Id$
\begin{metainfo}
  \name{
    \lang{de}{A03: Taylor}
    \lang{en}{}
  }
  \begin{description} 
 This work is licensed under the Creative Commons License Attribution 4.0 International (CC-BY 4.0)   
 https://creativecommons.org/licenses/by/4.0/legalcode 

    \lang{de}{}
    \lang{en}{}
  \end{description}
  \corrector{system/problem/GenericCorrector.meta.xml}
  \begin{components}
    \component{js_lib}{system/problem/GenericMathlet.meta.xml}{mathlet}
  \end{components}
  \begin{links}
  \end{links}
  \creategeneric
\end{metainfo}
\begin{content}
\usepackage{mumie.genericproblem}

\lang{de}{\title{A03: Taylor}}

\begin{block}[annotation]
	Im Ticket-System: \href{http://team.mumie.net/issues/10407}{Ticket 10407}
\end{block}

\begin{problem}
	%\randomquestionpool{}{}
	
%######################################################QUESTION_START	
	\begin{question}	
	%+++++++++++++++++++VARIABLES++++++++++++++++++++++
		\begin{variables}
			\randint[Z]{m}{1}{5}
			\randint[Z]{b}{1}{4}
			\randint[Z]{n}{1}{3}
            \function[calculate]{N}{n+1}
%			\randint[Z]{n}{2}{3}
			\randint[Z]{k}{-2}{2}
            %Entwicklungspunkt und Intervallgrenzen
			\function[calculate]{a}{m-k}
            \function[calculate]{c1}{a-1/2}
            \function[calculate]{c2}{a+1/2}
			
			\function{f}{b/(m-x)}
			
			%ableitungen
			\derivative{f1}{f}{x}
			\derivative{f2}{f1}{x}
			\derivative{f3}{f2}{x}
            \derivative{f4}{f3}{x}
			
			%substitutionen
			\substitute[calculate]{fa}{f}{x}{a}
			\substitute[calculate]{fa1}{f1}{x}{a}
			\substitute[calculate]{fa2}{f2}{x}{a}
			\substitute[calculate]{fa3}{f3}{x}{a}
			
            %
            %"Kronecker-deltas" für die Summanden des Taylorpolynoms und Fälle für die Abschätzung des Restglieds
            \begin{switch}
                  \begin{case}{n=1}
                       \function[calculate]{d2}{0}
			           \function[calculate]{d3}{0}
                       \substitute[calculate]{M1}{f2}{x}{c1}
                       \substitute[calculate]{M2}{f2}{x}{c2}
                       \function[calculate]{M}{(M1+M2+|M1-M2|)/4}
                       \function[normalize]{rest}{M*|x-a|^2}
                  \end{case}

                  \begin{case}{n=2}
                       \function[calculate]{d2}{1}
			           \function[calculate]{d3}{0}
                       \substitute[calculate]{M1}{f3}{x}{c1}
                       \substitute[calculate]{M2}{f3}{x}{c2}
                       \function[calculate]{M}{(M1+M2+|M1-M2|)/12}
                       \function[normalize]{rest}{M*(|x-a|^3)}
                      
                  \end{case}

                  \begin{default}
                       \function[calculate]{d2}{1}
			           \function[calculate]{d3}{1}
                       \substitute[calculate]{M1}{f4}{x}{c1}
                       \substitute[calculate]{M2}{f4}{x}{c2}
                       \function[calculate]{M}{(M1+M2+|M1-M2|)/48}
                       \function[normalize]{rest}{M*(|x-a|^4)}

                  \end{default}
            \end{switch}
			%d-s

			\function[normalize]{p}{fa + (fa1)*(x-a) + (fa2)*(d2/2)*(x-a)^2 + (fa3)*(d3/6)*(x-a)^3}
			\function{sol}{f-p}
			
		\end{variables}
	%+++++++++++++++++++TYPE+++++++++++++++++++++++++++	
		\type{input.function} %input.text %input.cases.function %input.finite-number-set %input.interval %...http://team.mumie.net/projects/support/wiki/DifferentAnswerType
		\field{rational} 
		%\precision{3}
	%+++++++++++++++++++TITLE++++++++++++++++++++++++++	
	    \lang{de}{\text{Bestimmen Sie das Restglied zum Taylor-Polynom $\var{n}$-ter Ordnung von $f(x)=\var{f}$ an der Stelle $x_0=\var{a}$.\\
        Benutzen Sie die Lagrange-Restgliedformel, um eine Abschätzung des Fehlers auf dem  Intervall $(\var{c1};\var{c2})$ mit Hilfe des Abstands zum Entwicklungspunkt anzugeben.\\
% 	    \\
%         $\var{f}$, $\var{fa}$\\
%         $\var{f1}$, $\var{fa1}$\\
%         $\var{f2}$, $\var{fa2}$\\
%         $\var{f3}$, $\var{fa3}$\\
%         $\var{f4}$.
}}
	%+++++++++++++++++++ANSWERS++++++++++++++++++++++++    
	    \begin{answer}
	    	\text{Das Restglied ist: $R_{\var{n}}(x)= $}
			\solution{sol}
			\checkAsFunction{x}{-10}{10}{100} 
	        
	    \explanation{Das Taylor-Polynom ist in diesem Fall $T_{\var{n}}(x)=\var{p}$ und das Restglied ist die Differenz der Funktion
	    und des Taylorpolynoms.}
        \end{answer}
        \begin{answer}
        \text{Der Fehler auf $(\var{c1};\var{c2})$ ist höchstens: $| R_{\var{n}}(x)|\leq$ }
        \solution{rest}
        \checkAsFunction{x}{-10}{10}{100}
        \explanation{Hier muss die Konstante in der Lagrange-Restgliedformel abgeschätzt werden.
        Bei der Abschätzung der darin auftretenden $\var{N}$-ten Ableitung an einer Zwischenstelle $\xi$ hilft folgende Überlegung:
        Die Funktion $f$ und alle ihre Ableitungen sind auf $[\var{c1};\var{c2}]$ stetig und monoton.
Das können Sie benutzen, um Supremum und Infimum der $\var{N}$-ten Ableitung auf $(\var{c1};\var{c2})$ zu erhalten.\\
Die Fehlerabschätzung hat dann die Form $const.\cdot | x-x_0|^{\var{N}}$.}
        \end{answer}
	\end{question}
%######################################################QUESTION_END

\end{problem}

\embedmathlet{mathlet}

\end{content}