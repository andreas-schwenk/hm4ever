\documentclass{mumie.problem.gwtmathlet}
%$Id$
\begin{metainfo}
  \name{
    \lang{de}{A05: Newton}
    \lang{en}{}
  }
  \begin{description} 
 This work is licensed under the Creative Commons License Attribution 4.0 International (CC-BY 4.0)   
 https://creativecommons.org/licenses/by/4.0/legalcode 

    \lang{de}{}
    \lang{en}{}
  \end{description}
  \corrector{system/problem/GenericCorrector.meta.xml}
  \begin{components}
    \component{js_lib}{system/problem/GenericMathlet.meta.xml}{mathlet}
  \end{components}
  \begin{links}
  \end{links}
  \creategeneric
\end{metainfo}
\begin{content}
\usepackage{mumie.genericproblem}

\lang{de}{\title{A05: Newton}}

\begin{block}[annotation]
	Im Ticket-System: \href{http://team.mumie.net/issues/10408}{Ticket 10408}
\end{block}

\begin{problem}

	\begin{variables}
		
			\randint[Z]{az}{-6}{-2}
			\randint[Z]{an}{2}{5}
			\function{a}{az/an}
		
			\randint[Z]{b0}{1}{4}
			\function{b}{b0/2}
		
			\randint[Z]{c0}{3}{6}
			\function{c}{c0/2}
		
			\randint[Z]{d}{1}{4}
			
			\function[expand,normalize]{f}{(x-a)*(x-b)*(x-c)*(x-d)}
			\derivative[normalize]{f1}{f}{x}
			
			\function{g}{x-(f/f1)}
			
			\function[calculate]{x1}{floor(a)-1}
			\substitute{x2}{g}{x}{x1}
			\substitute{x3}{g}{x}{x2}
			
	\end{variables}

	\begin{question}	
		
	%+++++++++++++++++++TYPE+++++++++++++++++++++++++++
		
		\type{input.function} 
		\field{rational}
	    \lang{de}{\text{Wir betrachten das Newton-Verfahren zur Bestimmung einer Nullstelle der Funktion $f=\var{f}$.
Geben Sie die Rekursionsformel für das Newton-Verfahren an (schreiben sie lediglich $x$ statt $x_n$). }}
    	\explanation{Wenden sie die Rekursionsformel des Newton-Verfahrens auf die gegebene Funktion an.}

	    \begin{answer}
	    	\text{$x_{n+1}= $}
			\solution{g}
			\checkAsFunction{x}{-10}{10}{100} 
	    \end{answer}  
    
	\end{question}
	 
	\begin{question}	
		
		\type{input.number} %input.text %input.cases.function %input.finite-number-set %input.interval %...http://team.mumie.net/projects/support/wiki/DifferentAnswerType
		\field{real}
		\precision{3}
		
	    \lang{de}{\text{ Berechnen Sie die Folgeglieder $x_2$ und $x_3$ (auf drei Stellen nach dem Komma), wenn $x_1=\var{x1}$ ist.}}
    	\explanation{Wenden sie die Rekursionsformel des Newton-Verfahrens auf die gegebene Funktion an.}

	    \begin{answer}
	    
		\text{$x_{2}= $}
			\solution{x2}
		\end{answer}

		\begin{answer}
	    
		\text{$x_{3}= $}
			\solution{x3}
		\end{answer}  
    
	\end{question}   

\end{problem}

\embedmathlet{mathlet}

\end{content}