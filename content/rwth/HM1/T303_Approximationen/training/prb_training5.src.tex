\documentclass{mumie.problem.gwtmathlet}
%$Id$
\begin{metainfo}
  \name{
    \lang{de}{A07: l'Hospital}
    \lang{en}{}
  }
  \begin{description} 
 This work is licensed under the Creative Commons License Attribution 4.0 International (CC-BY 4.0)   
 https://creativecommons.org/licenses/by/4.0/legalcode 

    \lang{de}{}
    \lang{en}{}
  \end{description}
  \corrector{system/problem/GenericCorrector.meta.xml}
  \begin{components}
    \component{js_lib}{system/problem/GenericMathlet.meta.xml}{mathlet}
  \end{components}
  \begin{links}
  \end{links}
  \creategeneric
\end{metainfo}
\begin{content}
\usepackage{mumie.genericproblem}

\lang{de}{\title{A07: l'Hospital}}

\begin{block}[annotation]
	Im Ticket-System: \href{http://team.mumie.net/issues/10410}{Ticket 10410}
\end{block}

\begin{block}[annotation]
Aufgabenpool mit 3 leicht randomisierten Aufgaben.
\end{block}

\begin{problem}
	%\randomquestionpool{}{}
	\randomquestionpool{1}{3}
%######################################################QUESTION_START	
	\begin{question}	
	%+++++++++++++++++++VARIABLES++++++++++++++++++++++
		\begin{variables}
			\randint[Z]{x0}{0}{3}
			\randint[Z]{n}{2}{5}
			\function{sol}{1/n}
			\function[normalize]{h}{(x-(x0-1))}
			\function{f}{e^(h)-e} 
			\function{g}{e^(h^n)-e}
		\end{variables}
	%+++++++++++++++++++TYPE+++++++++++++++++++++++++++	
		\type{input.function} %input.text %input.cases.function %input.finite-number-set %input.interval %...http://team.mumie.net/projects/support/wiki/DifferentAnswerType
		\field{rational}
		\precision{3}
	%+++++++++++++++++++TITLE++++++++++++++++++++++++++	
	    \lang{de}{
	    	\text{Bestimmen Sie den folgenden Grenzwert\\ $ \lim_{x\to \var{x0}} \frac{\var{f}}{\var{g}} = $\ansref.
            \\
            Überlegen Sie sich insbesondere auch, was Ihr Vorgehen rechtfertigt.}
	    }
	%+++++++++++++++++++ANSWERS++++++++++++++++++++++++    
	    \begin{answer}
			\solution{sol}
	    	\checkAsFunction{x}{-1}{1}{10} 
	    \end{answer}  
        \explanation{Wenden Sie die Regel von de l'Hospital an.}
	\end{question}
%######################################################QUESTION_END

%######################################################QUESTION_START	
	\begin{question}	
	%+++++++++++++++++++VARIABLES++++++++++++++++++++++
		\begin{variables}
			\randint{b}{-3}{3}
			\function[calculate]{x0}{b}
			\randint[Z]{n}{1}{2}
			\function[normalize]{sol}{((-1)^(n+b))/pi}
			\function[normalize]{f}{(-1)^n*(x-b)}
			\function{g}{sin(pi*x)}
		\end{variables}
	%+++++++++++++++++++TYPE+++++++++++++++++++++++++++	
		\type{input.function} %input.text %input.cases.function %input.finite-number-set %input.interval %...http://team.mumie.net/projects/support/wiki/DifferentAnswerType
		\field{rational}
		\precision{3}
	%+++++++++++++++++++TITLE++++++++++++++++++++++++++	
	    \lang{de}{
	    	\text{Bestimmen Sie den folgenden Grenzwert\\ $ \lim_{x\to \var{x0}} \frac{\var{f}}{\var{g}} = $\ansref.
            \\
            Überlegen Sie sich insbesondere auch, was Ihr Vorgehen rechtfertigt.}
	    }
	%+++++++++++++++++++ANSWERS++++++++++++++++++++++++    
	    \begin{answer}
			\solution{sol}
			\checkAsFunction{x}{-1}{1}{10} 
	    \end{answer} 
        \explanation{Wenden Sie die Regel von de l'Hospital an. }
	\end{question}
%######################################################QUESTION_END

%######################################################QUESTION_START	
	\begin{question}	
	%+++++++++++++++++++VARIABLES++++++++++++++++++++++
		\begin{variables}
			\randint{b}{-3}{3}
			\number{n}{2}%n=1 funktioniert nicht, FS
			\function[normalize]{x0}{(-1)^n*infinity}
			\function{sol}{0}
			\function[normalize]{f}{((-1)^n)*(x-b)}
			\function[normalize]{g}{sqrt(((-1)^n)*(x-b))}
		\end{variables}
	%+++++++++++++++++++TYPE+++++++++++++++++++++++++++	
		\type{input.number} %input.text %input.cases.function %input.finite-number-set %input.interval %...http://team.mumie.net/projects/support/wiki/DifferentAnswerType
		\field{rational}
		\precision{3}
	%+++++++++++++++++++TITLE++++++++++++++++++++++++++	
	    \lang{de}{
	    	\text{Bestimmen Sie den folgenden Grenzwert\\ $ \lim_{x\to \var{x0}} \frac{\ln(\var{f})}{\var{g}} = $\ansref.
            \\
            Zusatz: Überlegen Sie sich auch, was Ihr Vorgehen rechtfertigt.}
	    }    
    %+++++++++++++++++++ANSWERS++++++++++++++++++++++++    
	    \begin{answer}
			\solution{sol}
	    \end{answer} 
        \explanation{Wenden Sie die Regel von de l'Hospital an. }
	\end{question}
%######################################################QUESTION_END

\end{problem}

\embedmathlet{mathlet}

\end{content}