\documentclass{mumie.problem.gwtmathlet}
%$Id$
\begin{metainfo}
  \name{
    \lang{en}{...}
    \lang{de}{A04: Taylor}
    \lang{zh}{...}
    \lang{fr}{...}
  }
  \begin{description} 
 This work is licensed under the Creative Commons License Attribution 4.0 International (CC-BY 4.0)   
 https://creativecommons.org/licenses/by/4.0/legalcode 

    \lang{en}{...}
    \lang{de}{...}
    \lang{zh}{...}
    \lang{fr}{...}
  \end{description}
  \corrector{system/problem/GenericCorrector.meta.xml}
  \begin{components}
    \component{js_lib}{system/problem/GenericMathlet.meta.xml}{gwtmathlet}
  \end{components}
  \begin{links}
  \end{links}
  \creategeneric
\end{metainfo}
\begin{content}
\begin{block}[annotation]
	Im Ticket-System: \href{https://team.mumie.net/issues/22383}{Ticket 22383}
\end{block}


\usepackage{mumie.genericproblem}
\lang{de}{\title{A04: Taylor}}
\begin{problem}
	%\randomquestionpool{}{}
	
%######################################################QUESTION_START	
	\begin{question}	
	%+++++++++++++++++++VARIABLES++++++++++++++++++++++
		\begin{variables}
			\randint[Z]{a}{1}{7}
            \randint[Z]{k}{-2}{2}
			\function[normalize]{b}{a+k}
%            \number{n}{5}
            \function[normalize]{f}{(a+x)/(a-x)}
            \function[normalize]{fn}{2*a*fac(n)/((a-x)^(n+1))}
            \substitute[calculate]{f1}{f}{x}{b}
%            \substitute[calculate]{fn1}{fn}{x}{b}
            \function{summand}{(2*a)/((a-b)^(n+1))*(x-b)^n}
		\end{variables}
	%+++++++++++++++++++TYPE+++++++++++++++++++++++++++	
		\type{input.function} %input.text %input.cases.function %input.finite-number-set %input.interval %...http://team.mumie.net/projects/support/wiki/DifferentAnswerType
		\field{rational} 
		%\precision{3}
	%+++++++++++++++++++TITLE++++++++++++++++++++++++++	
	    \lang{de}{\text{
        Es sei $f(x)=\var{f}$. \\ Bestimmen Sie allgemein die $n$-te Ableitung von $f$:
        \\
        $f^{(n)}(x)=$\ansref .\\
        Bestimmen Sie weiter die Taylor-Entwicklung von $f$ an der Entwicklungsstelle $x_0=\var{b}$
        \\
        $T_{f,\var{b}}(x)=$\ansref $+\sum_{n=1}^\infty$\ansref.
         
	    }}
	%+++++++++++++++++++ANSWERS++++++++++++++++++++++++
        \begin{answer}
          \solution{fn}
          \inputAsFunction{x,n}{stud1}
          \checkFuncForZero{|stud1[2,n]-fn[2,n]|+|stud1[3,n]-fn[3,n]|+|stud1[4,n]-fn[4,n]|}{8}{10}{100}
          \explanation{Nach Bildung einiger Ableitungen lässt sich ein allgemeiner Ausdruck für $f^{(n)}(x)$ erkennen, 
          den Sie durch vollständige Induktion beweisen können.}
        \end{answer}
        \begin{answer}
          \solution{f1}
          \checkAsFunction{x}{8}{10}{100}
          \explanation{Das nullte Glied der Taylor-Reihe ist $f(x_0)$.}
        \end{answer}
	    \begin{answer}
            \inputAsFunction{x,n}{stud}
	    	\solution{summand}
            \checkFuncForZero{|stud[2,n]-summand[2,n]|+|stud[3,n]-summand[3,n]|+|stud[4,n]-summand[4,n]|}{8}{10}{100}
			%\checkAsFunction{x,n}{8}{10}{100} 
            \explanation{Überlegen Sie, wie sich die Koeffizienten der Taylor-Reihe aus den Ableitungen der Funktion ergeben.}
	    \end{answer}    
	\end{question}
%######################################################QUESTION_END

\end{problem}



\embedmathlet{gwtmathlet}

\end{content}
