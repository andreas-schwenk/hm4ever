\documentclass{mumie.problem.gwtmathlet}
%$Id$
\begin{metainfo}
  \name{
    \lang{de}{A01: Taylor}
    \lang{en}{}
  }
  \begin{description} 
 This work is licensed under the Creative Commons License Attribution 4.0 International (CC-BY 4.0)   
 https://creativecommons.org/licenses/by/4.0/legalcode 

    \lang{de}{}
    \lang{en}{}
  \end{description}
  \corrector{system/problem/GenericCorrector.meta.xml}
  \begin{components}
    \component{js_lib}{system/problem/GenericMathlet.meta.xml}{mathlet}
  \end{components}
  \begin{links}
  \end{links}
  \creategeneric
\end{metainfo}
\begin{content}
\usepackage{mumie.genericproblem}

\lang{de}{\title{A01: Taylor}}

\begin{block}[annotation]
	Im Ticket-System: \href{http://team.mumie.net/issues/10406}{Ticket 10406}
\end{block}

\begin{problem}
	%\randomquestionpool{}{}
	
%######################################################QUESTION_START	
	\begin{question}	
	%+++++++++++++++++++VARIABLES++++++++++++++++++++++
		\begin{variables}
			\randint[Z]{m}{1}{4}
			\randint[Z]{b}{1}{4}
			\randint[Z]{n}{1}{3}
			\randint[Z]{k}{-2}{2}
			\randadjustIf{k}{k=0}
			\function[normalize]{a}{pi/m}
			
			\function[normalize]{f}{b*sin((m/k)*x)}
			
			%ableitungen
			\derivative{f1}{f}{x}
			\derivative{f2}{f1}{x}
			\derivative{f3}{f2}{x}
			
			%substitutionen
			\substitute[calculate]{fa}{f}{x}{a}
			\substitute[calculate]{fa1}{f1}{x}{a}
			\substitute[calculate]{fa2}{f2}{x}{a}
			\substitute[calculate]{fa3}{f3}{x}{a}
			
			%d-s
			\function[calculate]{d3}{((n-1)*(n-2))/2}
			\function[calculate]{d2}{1-(((n-3)*(n-2))/2)}

			\function[normalize]{p}{fa + (fa1)*(x-a) + (fa2)*(d2/2)*(x-a)^2 + (fa3)*(d3/6)*(x-a)^3}
			
		\end{variables}
	%+++++++++++++++++++TYPE+++++++++++++++++++++++++++	
		\type{input.function} %input.text %input.cases.function %input.finite-number-set %input.interval %...http://team.mumie.net/projects/support/wiki/DifferentAnswerType
		\field{rational} 
		%\precision{3}
	%+++++++++++++++++++TITLE++++++++++++++++++++++++++	
	    \lang{de}{\text{Bestimmen Sie das Taylor-Polynom von $f(x)=\var{f}$ der 
	    Ordnung $\var{n}$ an der Stelle $x_0=\var{a}$. \\
	 
	    }}
	%+++++++++++++++++++ANSWERS++++++++++++++++++++++++    
	    \begin{answer}
	    	\text{Das Taylor-Polynom ist $T_{\var{n}}(x)= $}
			\solution{p}
			\checkAsFunction{x}{-10}{10}{100} 
            \explanation{Rufen Sie sich die Definition des Taylor-Polynoms in Erinnerung.}
	    \end{answer}    
	\end{question}
%######################################################QUESTION_END

\end{problem}

\embedmathlet{mathlet}

\end{content}