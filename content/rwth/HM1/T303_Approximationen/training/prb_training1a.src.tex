\documentclass{mumie.problem.gwtmathlet}
%$Id$
\begin{metainfo}
  \name{
    \lang{en}{...}
    \lang{de}{A02: Taylor}
    \lang{zh}{...}
    \lang{fr}{...}
  }
  \begin{description} 
 This work is licensed under the Creative Commons License Attribution 4.0 International (CC-BY 4.0)   
 https://creativecommons.org/licenses/by/4.0/legalcode 

    \lang{en}{...}
    \lang{de}{...}
    \lang{zh}{...}
    \lang{fr}{...}
  \end{description}
  \corrector{system/problem/GenericCorrector.meta.xml}
  \begin{components}
    \component{js_lib}{system/problem/GenericMathlet.meta.xml}{gwtmathlet}
  \end{components}
  \begin{links}
  \end{links}
  \creategeneric
\end{metainfo}
\begin{content}
\begin{block}[annotation]
	Im Ticket-System: \href{https://team.mumie.net/issues/22380}{Ticket 22380}
\end{block}
\begin{block}[annotation]
Polynom in Punkt $x_0$ entwickeln.\\
Generische Aufgabe mit differenziertem Feedback.
\end{block}
\begin{block}[annotation]
	Entlehnt von \href{https://team.mumie.net/issues/17203}{Ticket 17203}.
\end{block}
\usepackage{mumie.genericproblem}
\lang{de}{\title{A02: Taylor}}
\lang{en}{\title{Problem 2}}

\begin{problem}
  \begin{question}
    \type{input.number}
    \field{rational}
    \begin{variables}
      \begin{pool}
        \begin{variables}
          \randint[Z]{sol3}{-4}{4}
          \randint{sol2}{-1}{1}
          \randint{sol1}{-2}{2}
          \randint[Z]{sol0}{-2}{2}
        \end{variables}
        \begin{variables}
          \randint[Z]{sol3}{-2}{2}
          \randint{sol2}{-9}{9}
          \randint{sol1}{-1}{1}
          \randint[Z]{sol0}{-1}{1}
        \end{variables}
        \begin{variables}
          \randint[Z]{sol3}{-2}{2}
          \randint{sol2}{-1}{1}
          \randint{sol1}{-9}{9}
          \randint[Z]{sol0}{-2}{2}
        \end{variables}
        \begin{variables}
          \randint[Z]{sol3}{-2}{2}
          \randint{sol2}{-2}{2}
          \randint{sol1}{-2}{2}
          \randint[Z]{sol0}{-9}{9}
        \end{variables}
      \end{pool}
      \randint[Z]{x0}{-3}{3}
      \function[calculate]{x00}{abs(x0)}
      \begin{switch}
        \begin{case}{x0<0}
          \string{op}{+}
        \end{case}
        \begin{default}
          \string{op}{-}
        \end{default}
      \end{switch}
      \function[expand, sort, normalize]{p}{sol3*(x-x0)^3 + sol2*(x-x0)^2 + sol1*(x-x0) + sol0}
      % typische Fehler
      \function[calculate]{err3}{sol3*6}
      \function[calculate]{err2}{sol2*2}
    \end{variables}
    \lang{de}{
      \text{
        Entwickeln Sie 
         die Polynomfunktion $p(x) = \var{p}$
        nach Potenzen $(x\var{op} \var{x00})^j$ um den Entwicklungspunkt $x_0 = \var{x0}$.\\
        \\
        $
            p(x) = $ \ansref $ \cdot\; (x \var{op} \var{x00})^3\;
            + $ \ansref $ \cdot\; (x \var{op} \var{x00})^2\;
            + $ \ansref $ \cdot\; (x \var{op} \var{x00})\;
            + $ \ansref 
      }
      \explanation{
        Dies entspricht der Taylor-Entwicklung im Entwicklungspunkt $x_0 = \var{x0}$.
      }
      \explanation[equal(err3,ans_1)]{\\
        Der Koeffizient des kubischen Terms $(x-x_0)^3$ ist 
        $\frac{p\prime\prime\prime(x_0)}{3!}$. \\
        Beachten Sie, dass Sie den Wert der dritten Ableitung durch $3!$ dividieren müssen.
      }
      \explanation[[equal(err2, ans_2)] AND [NOT [equal(sol2, 0)]]]{\\
        Der Koeffizient des quadratischen Terms $(x-x_0)^2$ ist 
        $\frac{p\prime\prime(x_0)}{2!}$.\\
        Beachten Sie, dass Sie den Wert der zweiten Ableitung durch $2!$ dividieren müssen.
      }
    }
    \begin{answer}
      \solution{sol3}
    \end{answer}
    \begin{answer}
      \solution{sol2}
    \end{answer}
    \begin{answer}
      \solution{sol1}
    \end{answer}
    \begin{answer}
      \solution{sol0}
    \end{answer}
  \end{question}
\end{problem}




\embedmathlet{gwtmathlet}

\end{content}
