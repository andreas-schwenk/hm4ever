\documentclass{mumie.problem.gwtmathlet}
%$Id$
\begin{metainfo}
  \name{
    \lang{de}{A08: l'Hospital}
    \lang{en}{}
  }
  \begin{description} 
 This work is licensed under the Creative Commons License Attribution 4.0 International (CC-BY 4.0)   
 https://creativecommons.org/licenses/by/4.0/legalcode 

    \lang{de}{}
    \lang{en}{}
  \end{description}
  \corrector{system/problem/GenericCorrector.meta.xml}
  \begin{components}
    \component{js_lib}{system/problem/GenericMathlet.meta.xml}{mathlet}
  \end{components}
  \begin{links}
  \end{links}
  \creategeneric
\end{metainfo}
\begin{content}
\usepackage{mumie.genericproblem}

\lang{de}{\title{A08: l'Hospital}}

\begin{block}[annotation]
	Im Ticket-System: \href{http://team.mumie.net/issues/10411}{Ticket 10411}
\end{block}
\begin{block}[annotation]
Pool mit zwei randomisierten Aufgaben zur Regel von de l'Hospital.
\end{block}

\begin{problem}
	%\randomquestionpool{}{}
	\randomquestionpool{1}{2}
%######################################################QUESTION_START	
	\begin{question}	
	%+++++++++++++++++++VARIABLES++++++++++++++++++++++
		\begin{variables}
			\randint{x0}{1}{3}
			\randint[Z]{n}{1}{3}
			\function{sol}{0}
			\function[normalize]{h}{(x-(x0-1))}
			\function{f}{1/(x-x0)} 
			\function{g}{x-x0}
		\end{variables}
	%+++++++++++++++++++TYPE+++++++++++++++++++++++++++	
		\type{input.function} %input.text %input.cases.function %input.finite-number-set %input.interval %...http://team.mumie.net/projects/support/wiki/DifferentAnswerType
		\field{rational}
		
	%+++++++++++++++++++TITLE++++++++++++++++++++++++++	
	    \lang{de}{\text{Bestimmen Sie den Grenzwert\\
        $ \lim_{x\searrow \var{x0}} \ln(\var{f})\cdot (\var{g}) = $\ansref \\
        Schreiben Sie ~\texttt{infty}~ bzw. ~\texttt{-infty}~, falls die Funktion
bestimmt gegen unendlich (bzw. minus unendlich) divergieren sollte.}}
	%+++++++++++++++++++ANSWERS++++++++++++++++++++++++    
	    \begin{answer}
			\solution{sol}
            \checkAsFunction{x}{-10}{10}{10}
	    \end{answer}  
        \explanation{Formen Sie geeignet um, um die Regel von de l'Hospital anwenden zu können.}
	\end{question}
%######################################################QUESTION_END

%######################################################QUESTION_START	
	\begin{question}	
	%+++++++++++++++++++VARIABLES++++++++++++++++++++++
		\begin{variables}
			\function{x0}{infinity}
			\randint[Z]{n}{2}{4}
			\function{f}{e^((x^n))}
			\function{g}{(e^x)^n}
			\function{sol}{infinity}
		\end{variables}
	%+++++++++++++++++++TYPE+++++++++++++++++++++++++++	
		\type{input.function} %input.text %input.cases.function %input.finite-number-set %input.interval %...http://team.mumie.net/projects/support/wiki/DifferentAnswerType
		\field{rational}
		\precision{3}
	%+++++++++++++++++++TITLE++++++++++++++++++++++++++	
	    \lang{de}{\text{Bestimmen Sie den Grenzwert \\$ \lim_{x\to \var{x0}} (\var{f}-\var{g}) = $\ansref \\
        Schreiben Sie ~\texttt{infty}~ bzw.~\texttt{-infty}~, falls die Funktion
bestimmt gegen unendlich (bzw. minus unendlich) divergieren sollte.}}
	%+++++++++++++++++++ANSWERS++++++++++++++++++++++++    
	    \begin{answer}
			\solution{sol}
			\checkAsFunction{x}{-10}{10}{10}
	    \end{answer} 
        \explanation{Formen Sie geeignet um, um die Regel von de l'Hospital anwenden zu können.}
	\end{question}
%######################################################QUESTION_END

\end{problem}

\embedmathlet{mathlet}

\end{content}