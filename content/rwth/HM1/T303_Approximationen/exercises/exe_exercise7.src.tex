\documentclass{mumie.element.exercise}
%$Id$
\begin{metainfo}
  \name{
    \lang{de}{Ü08: l'Hospital}
    \lang{en}{Exercise 8}
  }
  \begin{description} 
 This work is licensed under the Creative Commons License Attribution 4.0 International (CC-BY 4.0)   
 https://creativecommons.org/licenses/by/4.0/legalcode 

    \lang{de}{}
    \lang{en}{}
  \end{description}
  \begin{components}
  \end{components}
  \begin{links}
  \end{links}
  \creategeneric
\end{metainfo}
\begin{content}
\usepackage{mumie.ombplus}

\title{\lang{de}{Ü08: l'Hospital}}

\begin{block}[annotation]
  Im Ticket-System: \href{http://team.mumie.net/issues/10498}{Ticket 10498}
\end{block}

%######################################################FRAGE_TEXT
\lang{de}{ 
Bestimmen Sie mit Hilfe der Regel von de l'Hospital die folgenden Grenzwerte:
\begin{enumerate}[a)]
\item a) $\lim_{x\to \infty} (\ln(x)-\sqrt{x})$,
\item b) $\lim_{x\nearrow 1} \ln(x)\cdot \tan(\frac{\pi}{2}x)$.
\end{enumerate}
}

%##################################################ANTWORTEN_TEXT
\begin{tabs*}[\initialtab{0}\class{exercise}]

  \tab{\lang{de}{    Antworten    }}
    \lang{de}{  
Die gesuchten "`Grenzwerte"' sind:
\begin{enumerate}[a)]
\item a) $\lim_{x\to \infty} (\ln(x)-\sqrt{x})=-\infty$,
\item b) $\lim_{x\nearrow 1} \ln(x)\cdot \tan(\frac{\pi}{2}x)=-\frac{2}{\pi}$.
\end{enumerate} } 

  %++++++++++++++++++++++++++++++++++++++++++START_TAB_X
  \tab{\lang{de}{    Lösung a)    }}
    \lang{de}{   
Berechnet man den Grenzwert summandenweise, so erhält man den Ausdruck "`$\infty-\infty$"'.
Wir sollten also den Term so umformen, dass die Regel von de l'Hospital angewendet werden kann, z.B.
durch Ausklammern von $\sqrt{x}$:
\[ \ln(x)-\sqrt{x}=\sqrt{x}\cdot (\frac{\ln(x)}{\sqrt{x}}-1). \]
Für den Bruch $\frac{\ln(x)}{\sqrt{x}}$ prüfen wir die Voraussetzungen der Regel von de l'Hospital:
Zähler und Nenner sind beliebig oft differenzierbar und $\lim_{x\to\infty} \ln(x)=\infty=\lim_{x\to\infty}\sqrt{x}$. Also gilt
\[ \lim_{x\to \infty} \frac{\ln(x)}{\sqrt{x}}= \lim_{x\to \infty} \frac{1/x}{1/(2\sqrt{x})}
=\lim_{x\to \infty} \frac{2}{\sqrt{x}}=0.\]
Mit den Grenzwertregeln auch für uneigentliche Grenzwerte ist damit
\begin{eqnarray*}
  \lim_{x\to \infty} (\ln(x)-\sqrt{x})
&=& \lim_{x\to \infty} \sqrt{x}\cdot (\frac{\ln(x)}{\sqrt{x}}-1)\\
&=& \lim_{x\to \infty}\sqrt{x}\cdot \left(\lim_{x\to \infty}\frac{\ln(x)}{\sqrt{x}}-1\right)\\
&=& \infty\cdot (0-1)=-\infty.
\end{eqnarray*}    }
  %++++++++++++++++++++++++++++++++++++++++++++END_TAB_X

  %++++++++++++++++++++++++++++++++++++++++++START_TAB_X
  \tab{\lang{de}{    Lösung b)    }}
     
Berechnet man den Grenzwert faktorenweise, erhält man den Ausdruck "`$0\cdot \infty$"'.
Wir müssen den Term daher umformen, um die Regel von de l'Hospital anwenden zu können.
Schreiben wir
\[\tan(\frac{\pi}{2}x)=\frac{\sin(\frac{\pi}{2}x)}{\cos(\frac{\pi}{2}x)}, \]
so erhält man
\[ \ln(x)\cdot \tan(\frac{\pi}{2}x)=\frac{\ln(x)}{\cos(\frac{\pi}{2}x)}\cdot \sin(\frac{\pi}{2}x).\]
Darin ist $\lim_{x\to 1}\sin(\frac{\pi}{2}x)=1$ und 
\[ \lim_{x\nearrow 1} \frac{\ln(x)}{\cos(\frac{\pi}{2}x)}\quad \big(= \text{\glqq{}} \frac{0}{0}\text{\grqq{}}\big).\]
Dafür wenden wir die Regel von de l'Hospital an. Zähler und Nenner sind beliebig oft differenzierbar. 
Wir berechnen  $(\ln(x))'=1/x$ und $(\cos(\frac{\pi}{2}x))'=-\sin(\frac{\pi}{2}x)\cdot \frac{\pi}{2}$.

Damit ist
\[ \lim_{x\nearrow 1} \frac{\ln(x)}{\cos(\frac{\pi}{2}x)}= \lim_{x\nearrow 1} \frac{1/x}{-\sin(\frac{\pi}{2}x)\cdot \frac{\pi}{2}}=\frac{1}{-1\cdot \pi/2}=-\frac{2}{\pi}. \]
Mit der Grenzwertregel für Produkte ist also schließlich der gesuchte Grenzwert
\[ \lim_{x\nearrow 1}\frac{\ln(x)}{\cos(\frac{\pi}{2}x)}\cdot \sin(\frac{\pi}{2}x)
=\lim_{x\nearrow 1}\frac{\ln(x)}{\cos(\frac{\pi}{2}x)}\cdot \lim_{x\nearrow 1}\sin(\frac{\pi}{2}x)=
-\frac{2}{\pi}\cdot 1= -\frac{2}{\pi}. \]
  %++++++++++++++++++++++++++++++++++++++++++++END_TAB_X


%#############################################################ENDE
\end{tabs*}
\end{content}