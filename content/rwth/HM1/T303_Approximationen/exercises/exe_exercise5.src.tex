\documentclass{mumie.element.exercise}
%$Id$
\begin{metainfo}
  \name{
    \lang{de}{Ü06: Newton}
    \lang{en}{Exercise 6}
  }
  \begin{description} 
 This work is licensed under the Creative Commons License Attribution 4.0 International (CC-BY 4.0)   
 https://creativecommons.org/licenses/by/4.0/legalcode 

    \lang{de}{}
    \lang{en}{}
  \end{description}
  \begin{components}
  \end{components}
  \begin{links}
  \end{links}
  \creategeneric
\end{metainfo}
\begin{content}
\usepackage{mumie.ombplus}

\title{\lang{de}{Ü06: Newton}}

\begin{block}[annotation]
  Im Ticket-System: \href{http://team.mumie.net/issues/10496}{Ticket 10496}
\end{block}

%######################################################FRAGE_TEXT
\lang{de}{ 
1) Bestimmen Sie mit Hilfe des Newton-Verfahrens eine lokale Extremstelle der Funktion
$f(x)=x^4-3x^3+x^2-x$.\\
2) Bestimmen Sie die lokalen Extremstellen der Funktion $f(x)=x^3-3x+1$, skizzieren Sie mit diesen Informationen den Funktionsgraph, und bestimmen Sie (mit Hilfe eines Taschenrechners) Näherungen für sämtliche Nullstellen mittels des Newton-Verfahrens.
 }

%##################################################ANTWORTEN_TEXT
\begin{tabs*}[\initialtab{0}\class{exercise}]
  \tab{\lang{de}{Antworten}}
    Aufgabe 1: 
    \begin{itemize}
        \item $x_{min}\approx 2,066593$
    \end{itemize}
    Aufgabe 2: 
    \begin{itemize}
        \item Extremstellen bei $x_{min}=-1$ und $x_{max}=1$
        \item Nullstellen bei $x_{1}\approx -1,8793852$, $x_{2}\approx 0,34729636$ und $x_{3}\approx 1,53208889$
    \end{itemize}

  %++++++++++++++++++++++++++++++++++++++++++START_TAB_X
  \tab{\lang{de}{Lösung 1)}}
  \begin{incremental}[\initialsteps{1}]
  
  	 %----------------------------------START_STEP_X
    \step 
    \lang{de}{   
Da die Funktion $f$ auf ganz $\R$ definiert ist, ist eine Extremstelle der Funktion $f$ 
eine Nullstelle ihrer Ableitungsfunktion. Man wendet daher das Newton-Verfahren auf die Ableitungsfunktion $f'(x)=4x^3-9x^2+2x-1$ an.

Die zweite Ableitung ist $f''(x)=12x^2-18x+2$ und daher ist die Rekursionsformel
\[ x_{n+1}=x_n-\frac{f'(x_n)}{f''(x_n)}=x_n-\frac{4x_n^3-9x_n^2+2x_n-1}{12x_n^2-18x_n+2}. \]

Nun ist ein Startwert zu wählen. Es gelten:
\[ f'(0)=-1,\quad f'(1)=-4,\quad f'(2)=-1\quad\text{und }\:f'(3)=32. \]
Also besitzt die Ableitung zwischen $2$ und $3$ eine Nullstelle.

Wählt man nun $x_0=2$, dann erhält man:
\begin{eqnarray*}
x_1 &=& x_0-\frac{4x_0^3-9x_0^2+2x_0-1}{12x_0^2-18x_0+2} = \frac{29}{14}\approx 2,071429, \\
x_2 &=& x_1-\frac{4x_1^3-9x_1^2+2x_1-1}{12x_1^2-18x_1+2} \approx 2,066615, \\
x_3 &=& x_2-\frac{4x_2^3-9x_2^2+2x_2-1}{12x_2^2-18x_2+2} \approx 2,066593, \\
x_4 &=& x_3-\frac{4x_3^3-9x_3^2+2x_3-1}{12x_3^2-18x_3+2} \approx 2,066593. \\
\end{eqnarray*}
Die Extremstelle von $f$ liegt daher bei $2,066593$.    }
  	 %------------------------------------END_STEP_X
 
  \end{incremental}
  %++++++++++++++++++++++++++++++++++++++++++++END_TAB_X


%#############################################################ENDE

  \tab{\lang{de}{Lösungsvideo 2)}}
    Der Lösungsweg zu Aufgabe 2 kann dem folgenden Video entnommen werden:\youtubevideo[500][300]{yB_CMXXLGKQ}\\
  
\end{tabs*}
\end{content}