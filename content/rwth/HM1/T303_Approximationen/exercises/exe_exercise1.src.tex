\documentclass{mumie.element.exercise}
%$Id$
\begin{metainfo}
  \name{
    \lang{de}{Ü01: Taylor}
    \lang{en}{Exercise 1}
  }
  \begin{description} 
 This work is licensed under the Creative Commons License Attribution 4.0 International (CC-BY 4.0)   
 https://creativecommons.org/licenses/by/4.0/legalcode 

    \lang{de}{}
    \lang{en}{}
  \end{description}
  \begin{components}
    \component{js_lib}{system/media/mathlets/GWTGenericVisualization.meta.xml}{mathlet1}
  \end{components}
  \begin{links}
  \end{links}
  \creategeneric
\end{metainfo}
\begin{content}
\usepackage{mumie.ombplus}
\usepackage{mumie.genericvisualization}

\begin{visualizationwrapper}

\title{\lang{de}{Ü01: Taylor}}

\begin{block}[annotation]
  Im Ticket-System: \href{http://team.mumie.net/issues/10405}{Ticket 10405}
\end{block}

%######################################################FRAGE_TEXT
\lang{de}{1) Bestimmen Sie das Taylorpolynom  $4$-ter Ordnung zu der Funktion 
$f:\R\setminus\{0\}\to\R$, $f(x)=\frac{1}{x}$, an den Stellen
\begin{enumerate}
\item[a)] $x_0=1$,
\item[b)] $x_0=2$,
\item[c)] $x_0=-1$.
\end{enumerate}\\ }

\lang{de}{2) 
\begin{enumerate}
\item[a)] Bestimmen Sie das 3-te Taylorpolynom in $0$ von $f(x)=e^x\,\sin(x)$.
\item[b)] Bestimmen Sie das 3-te Taylorpolynom in $1$ von $f(x)=x^3-2x$.
\item[c)] Bestimmen Sie das 3-te Taylorpolynom in $0$ von $f(x)=\text{arcsin}(x)$.\\
          Hinweis: $[\text{arcsin}(x)]'=\frac{1}{\sqrt{1-x^2}}$
\end{enumerate}\\ }

%##################################################ANTWORTEN_TEXT
\begin{tabs*}[\initialtab{0}\class{exercise}]

  \tab{\lang{de}{Antworten}}
  Aufgabe 1
  \begin{itemize}
    \item[a)] $T_{4,1}(x)=1-(x-1)+(x-1)^2-(x-1)^4+(x-1)^5$
    \item[b)] $T_{4,2}(x)=\frac{1}{2}-\frac{1}{4}(x-2)+\frac{1}{8}(x-2)^2-\frac{1}{16}(x-2)^3+\frac{1}{32}(x-2)^4$
    \item[c)] $T_{4,-1}(x)=-1-(x+1)-(x+1)^2+(x+1)^3-(x+1)^4$
  \end{itemize}
  Aufgabe 2
  \begin{itemize}
    \item[a)] $T_{3,0}(x)=x+x^2+\frac{1}{3}x^3$
    \item[b)] $T_{3,1}(x)=x^3-2x$
    \item[c)] $T_{3,0}(x)=x+\frac{1}{6}x^3$
  \end{itemize}
  %++++++++++++++++++++++++++++++++++++++++++START_TAB_X
  \tab{\lang{de}{Allgemeiner Ansatz zu 1)}}
  \begin{incremental}[\initialsteps{1}]
  
  	 %----------------------------------START_STEP_X
    \step 
    \lang{de}{   Für das Taylorpolynom $4$-ter Ordnung  m"ussen zun"achst die ersten vier Ableitungsfunktionen
bestimmt werden. Mit der Regel für Potenzen gilt wegen $f(x)=\frac{1}{x}=x^{-1}$:
\begin{eqnarray*}
  f'(x)&=& (-1)\cdot x^{-2}\\
f''(x) &=& 2\cdot x^{-3}\\
f^{(3)}(x) &=& -6\cdot x^{-4}\\
f^{(4)}(x) &=& 24\cdot x^{-5}\\
\end{eqnarray*}
Das Taylorpolynom an einer Stelle $x_0$ ist damit gegeben durch
\begin{eqnarray*}
T_{4,x_0}(x)&=& f(x_0)+f'(x_0)\cdot (x-x_0)+\frac{f''(x_0)}{2}\cdot (x-x_0)^2+\frac{f^{(3)}(x_0)}{3!}\cdot (x-x_0)^3+\frac{f^{(4)}(x_0)}{4!}\cdot (x-x_0)^4 \\
&=& \frac{1}{x_0}-\frac{x-x_0}{x_0^2}+\frac{(x-x_0)^2}{x_0^3}-\frac{(x-x_0)^3}{x_0^4}+\frac{(x-x_0)^4}{x_0^5}
%&=& \frac{1}{a}\cdot \left( 1+ \frac{a-x}{a}+(\frac{a-x}{a})^2+(\frac{a-x}{a})^3+(\frac{a-x}{a})^4
%\right) 
\end{eqnarray*}     }
  	 %------------------------------------END_STEP_X
 
  \end{incremental}
  %++++++++++++++++++++++++++++++++++++++++++++END_TAB_X
  
   %++++++++++++++++++++++++++++++++++++++++++START_TAB_X
  \tab{\lang{de}{   Lösung 1a)    }}
  \begin{incremental}[\initialsteps{1}]
  
  	 %----------------------------------START_STEP_X
    \step 
    \lang{de}{   Hier müssen wir lediglich in das allgemeiner berechnete Taylorpolynom $x_0=1$ einsetzen. Dann
erhalten wir mit der allgemeinen Binomischen Formel:
\begin{eqnarray*}
T_{4,1}(x)&=& \frac{1}{1}-\frac{x-1}{1^2}+\frac{(x-1)^2}{1^3}-\frac{(x-1)^3}{1^4}+\frac{(x-1)^4}{1^5} \\
&=&1-(x-1)+(x-1)^2-(x-1)^4+(x-1)^5\:.
% &=& 1-(x-1)+(x^2-2x+1)-(x^3-3x^2+3x-1)+(x^4-4x^3+6x^2-4x+1) \\
% &=& x^4-5x^3+10x^2-10x+5
\end{eqnarray*}   }
  	 %------------------------------------END_STEP_X
 
  \end{incremental}
  %++++++++++++++++++++++++++++++++++++++++++++END_TAB_X
  
   %++++++++++++++++++++++++++++++++++++++++++START_TAB_X
  \tab{\lang{de}{   Lösung 1b)    }}
  \begin{incremental}[\initialsteps{1}]
  
  	 %----------------------------------START_STEP_X
    \step 
    \lang{de}{   Hier müssen wir in das allgemeiner berechnete Taylor-Polynom $x_0=2$ einsetzen. Dann
erhalten wir mit der allgemeinen binomischen Formel:
\begin{eqnarray*}
T_{4,2}(x)&=& \frac{1}{2}-\frac{x-2}{2^2}+\frac{(x-2)^2}{2^3}-\frac{(x-2)^3}{2^4}+\frac{(x-2)^4}{2^5} \\
&=& \frac{1}{2}-\frac{1}{4}(x-2)+\frac{1}{8}(x-2)^2-\frac{1}{16}(x-2)^3+\frac{1}{32}(x-2)^4\:.
% &=& \frac{1}{2}-\frac{1}{4}(x-2)+\frac{1}{8}(x^2-4x+4)-\frac{1}{16}(x^3-6x^2+12x-8)+\frac{1}{32}(x^4-8x^3+24x^2-32x+16) \\
% &=& \frac{1}{32}x^4-(\frac{1}{16}+\frac{8}{32}) x^3+( \frac{1}{8}+\frac{6}{16}+\frac{24}{32}) x^2-(\frac{1}{4}+\frac{4}{8}+\frac{12}{16}+\frac{32}{32}) x+(\frac{1}{2}+\frac{2}{4}+\frac{4}{8}+\frac{8}{16}+\frac{16}{32}) \\
% &=& \frac{1}{32}x^4-\frac{5}{16}x^3+\frac{5}{4}x^2-\frac{5}{2}x+\frac{5}{2}
\end{eqnarray*}  }
  	 %------------------------------------END_STEP_X
 
  \end{incremental}
  %++++++++++++++++++++++++++++++++++++++++++++END_TAB_X
  
   %++++++++++++++++++++++++++++++++++++++++++START_TAB_X
  \tab{\lang{de}{   Lösung 1c)    }}
  \begin{incremental}[\initialsteps{1}]
  
  	 %----------------------------------START_STEP_X
    \step 
    \lang{de}{  Hier müssen wir in das allgemeiner berechnete Taylor-Polynom $x_0=-1$ einsetzen. Dann
erhalten wir mit der allgemeinen binomischen Formel:
\begin{eqnarray*}
T_{4,-1}(x)&=& \frac{1}{-1}-\frac{x+1}{(-1)^2}-\frac{(x+1)^2}{(-1)^3}-\frac{(x+1)^3}{(-1)^4}+\frac{(x+1)^4}{(-1)^5} \\
&=& -1-(x+1)-(x+1)^2+(x+1)^3-(x+1)^4\:.
% &=& -1-(x+1)-(x^2+2x+1)-(x^3+3x^2+3x+1)-(x^4+4x^3+6x^2+4x+1) \\
% &=& -x^4-5x^3-10x^2-10x-5
\end{eqnarray*}

Bemerkung: Es ist kein Zufall, dass das Taylor-Polynom an der Stelle $x_0=-1$ und das Taylor-Polynom
an der Stelle $x_0=1$ bis auf Vorzeichen die gleichen Koeffizienten haben. Die Funktion $f$ 
erfüllt nämlich die Gleichung $f(-x)=-f(x)$ (d.h. sie ist punktsymmetrisch zum Ursprung). Damit
erhält man aber den Graphen des Taylor-Polynoms bei $x_0=-1$ aus dem bei $x_0=1$ durch Spiegeln
am Ursprung. Dies bedeutet aber, dass auch
\[ T_{4,-1}(-x)=-T_{4,1}(x). \] }
  	 %------------------------------------END_STEP_X
 
  \end{incremental}
  %++++++++++++++++++++++++++++++++++++++++++++END_TAB_X

%#############################################################ENDE

    \tab{\lang{de}{Lösungsvideo 2)}}
        Die Lösungswege zu Aufgabe 2 können dem folgenden Video entnommen werden:\\
        \youtubevideo[500][300]{ImC__oYmtVo}\\
\end{tabs*}

\begin{genericGWTVisualization}[700][1000]{mathlet1}
\lang{de}{\title{Das Taylor-Polynom $P_{f,4,x_0}(x)$ im Vergleich zur Funktion $f(x)$ für wählbares $x_0$}}
\begin{variables}
	
	\function{f}{real}{1/x}
	
	\number{n}{real}{3}
	
	\slider[0.1]{n_slider}{n}{-10, editable}{10, editable}
	
	\function{p}{real}{1/(var(n))-(x-var(n))/((var(n))^2)+((x-var(n))^2)/((var(n))^3)-((x-var(n))^3/(var(n))^4)+((x-var(n))^4/(var(n))^5)}
		
 \color{f}{#0066CC}
 \color{p}{#CC6600}
 \label{n}{$x_0=$}
 
	\point{x0}{rational}{var(n),0}
	\point{x01}{rational}{var(n),1/var(n)}
	
\end{variables}
\begin{canvas}
	\plotSize{600}
	\plotLeft{-10}
	\plotRight{10}
	\plot[coordinateSystem]{f,p,x0,x01}
    \slider{n_slider}
\end{canvas}
\lang{de}{\text{\textcolor{#CC6600}{Roter} Graph: $P(x_0)=\var{p}$}}
\lang{de}{\text{\textcolor{#0066CC}{Blauer} Graph: $f(x)=\var{f}$}}


\end{genericGWTVisualization}

\end{visualizationwrapper}



\end{content}