\documentclass{mumie.element.exercise}
%$Id$
\begin{metainfo}
  \name{
    \lang{de}{Ü03: Taylor}
    \lang{en}{Exercise 3}
  }
  \begin{description} 
 This work is licensed under the Creative Commons License Attribution 4.0 International (CC-BY 4.0)   
 https://creativecommons.org/licenses/by/4.0/legalcode 

    \lang{de}{}
    \lang{en}{}
  \end{description}
  \begin{components}
  \end{components}
  \begin{links}
  \end{links}
  \creategeneric
\end{metainfo}
\begin{content}
\usepackage{mumie.ombplus}

\title{\lang{de}{Ü03: Taylor}}

\begin{block}[annotation]
  Im Ticket-System: \href{http://team.mumie.net/issues/10494}{Ticket 10494}
\end{block}

%######################################################FRAGE_TEXT
\lang{de}{ 
Bestimmen Sie die Taylor-\notion{Reihen} zu den Funktionen\\
\textbf{a)} $f(x)=\frac{x}{1-x} \quad$ \textbf{b)} $g(x)=xe^x$\\
jeweils an der Stelle $x_0=0$.
 }

%##################################################ANTWORTEN_TEXT
\begin{tabs*}[\initialtab{0}\class{exercise}]

  %++++++++++++++++++++++++++++++++++++++++++START_TAB_X
  \tab{\lang{de}{    Lösung a)    }}
  \begin{incremental}[\initialsteps{1}]
  
  	 %----------------------------------START_STEP_X
    \step 
    \lang{de}{   Man berechnet zunächst die ersten Ableitungen, um eine Idee für die allgemeine Ableitung zu
bekommen.  Mit der Quotientenregel gilt:
\begin{eqnarray*}
f'(x)&=& \frac{1\cdot (1-x)-x\cdot (-1)}{(1-x)^2}=\frac{1}{(1-x)^2}=(1-x)^{-2}, \\
\end{eqnarray*}
Weiter erhält man mit der Kettenregel:
\begin{eqnarray*}
f''(x) &=& (-2)(1-x)^{-3}\cdot (-1)=2(1-x)^{-3}, \\
f'''(x) &=& 2\cdot (-3)(1-x)^{-4}\cdot (-1)=3!\cdot (1-x)^{-4},\\
f^{(4)}(x) &=& 3!\cdot (-4)(1-x)^{-5}\cdot (-1)=4!\cdot (1-x)^{-5} 
\end{eqnarray*}

Dies legt die allgemeine Formel $f^{(n)}(x)=n!\cdot (1-x)^{-n-1}$ nahe. Dass die Formel stimmt, weist man induktiv nach: Für $n=1$ ist die Formel richtig, und wenn sie für ein festes $n$
gilt, so folgt mit der Kettenregel
\[  f^{(n+1)}(x)=(f^{(n)})'(x)=\left( n!\cdot (1-x)^{-n-1} \right)'=n!\cdot (-n-1)(1-x)^{-n-2}\cdot (-1)=(n+1)!\cdot (1-x)^{-n-2}.\]
D.h. die Formel gilt auch für $n+1$.

Die Taylor-Reihe erhalten wir dann durch Einsetzen in die allgemeine Formel:
\[ P_{f,0}(x)=\sum_{n=0}^\infty \frac{f^{(n)}(0)}{n!}(x-0)^n
= 0+ \sum_{n=1}^\infty \frac{n!\cdot (1-0)^{-n-1}}{n!}x^n=\sum_{n=1}^\infty x^n.\]
   }
  	 %------------------------------------END_STEP_X
 
  \end{incremental}
  %++++++++++++++++++++++++++++++++++++++++++++END_TAB_X
\tab{\lang{de}{    Lösung b)    }}
  \begin{incremental}[\initialsteps{1}]
  
  	 %----------------------------------START_STEP_X
    \step 
    \lang{de}{   Man berechne zunächst die ersten Ableitungen, um eine Idee für die allgemeine Ableitung zu
bekommen. Mit der Produktregel gelten:
\begin{eqnarray*}
  g'(x)&=& 1\cdot e^x+x\cdot e^x=(1+x)e^x,\\
  g''(x) &=& 1\cdot e^x+(1+x)\cdot e^x=(2+x)e^x.
\end{eqnarray*}
Dies legt die allgemeine Formel
\[  g^{(n)}(x)=(n+x)e^x \]
nahe, welche man induktiv nachweist: Für $n=1$ stimmt die Formel, und wenn sie für ein festes $n$
gilt, so folgt
\[ g^{(n+1)}(x)=(g^{(n)})'(x)=\left( (n+x)e^x\right)'= 1\cdot e^x+(n+x)e^x=(n+1+x)e^x. \]
D.h. die Formel gilt auch für $n+1$.

Die Taylor-Reihe erhalten wir dann durch Einsetzen in die allgemeine Formel:
\[ T_{g,0}(x)=\sum_{n=0}^\infty \frac{g^{(n)}(0)}{n!}(x-0)^n
= \sum_{n=0}^\infty \frac{(n+0)e^0}{n!}x^n = \sum_{n=1}^\infty \frac{x^n}{(n-1)!}
\]
    }
  	 %------------------------------------END_STEP_X
 
  \end{incremental}
 

%#############################################################ENDE
\end{tabs*}
\end{content}