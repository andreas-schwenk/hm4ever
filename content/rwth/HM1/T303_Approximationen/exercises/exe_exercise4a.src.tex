\documentclass{mumie.element.exercise}
%$Id$
\begin{metainfo}
  \name{
    \lang{de}{Ü04: Newton}
    \lang{en}{Exercise 4}
  }
  \begin{description} 
 This work is licensed under the Creative Commons License Attribution 4.0 International (CC-BY 4.0)   
 https://creativecommons.org/licenses/by/4.0/legalcode 

    \lang{en}{...}
    \lang{de}{...}
  \end{description}
  \begin{components}
\component{generic_image}{content/rwth/HM1/images/g_tkz_T303_Exercise04.meta.xml}{T303_Exercise04}
\end{components}
  \begin{links}
  \end{links}
  \creategeneric
\end{metainfo}
\begin{content}
\begin{block}[annotation]
	Im Ticket-System: \href{https://team.mumie.net/issues/22494}{Ticket 22494}
\end{block}
\title{Ü04: Newton}
1) Bestimmen Sie mit Hilfe des Newton-Verfahrens die Nullstellen der Funktion
$f(x)=x^3-2x+2$. Setzen Sie den Startwert auf
\begin{itemize}
    \item[a)] $x_0=-2$
    \item[b)] $x_0=-0,85$
    \item[c)] $x_0=0$.
\end{itemize}
Als Abbruchkriterium wird eine Genauigkeit von sechs Dezimalstellen vorausgesetzt. 
\begin{itemize}
    \item[d)] Vergleichen Sie die Ergebnisse. Was fällt auf?
\end{itemize}

2) Was ergibt die Anwendung des Newton-Verfahrens auf die Funktion $f(x)=\frac{x}{x^2+3}$ mit den Startwerten
\begin{itemize}
    \item[a)] $x_0=0,5$
    \item[b)] $x_0=1$
    \item[c)] $x_0=1,5$?
\end{itemize}
Führen Sie jeweils drei Iterationen durch. Nutzen Sie einen Taschrechner. Skizzieren Sie die Situationen.

%##################################################ANTWORTEN_TEXT
  \begin{tabs*}[\initialtab{0}\class{exercise}]
    \tab{Antworten} 
        Aufgabe 1
        \begin{itemize}
            \item[a)]  $x_5 \approx -1,76929235$
            \item[b)]  $x_{12} \approx -1,76929235$
            \item[c)]  $x_3 = 1,\, x_4 = 0,\,x_5 = 1 ,\,x_6 = 0,\,...$
        \end{itemize}
        Aufgabe 2
        \begin{itemize}
            \item[a)] $x_3\approx -8\cdot10^{-11}$
            \item[b)] $x_3=-1 $           
            \item[c)] $x_3\approx -37,7$
        \end{itemize}
    
    \tab{Lösung 1a)}      
      Wird das Newton-Verfahren auf die gegebene Funktion angewendet, ergibt sich die Berechnungsvorschrift
      \begin{align*}
        x_{n+1}=x_n-\frac{f(x_n)}{f'(x_n)}=x_n-\frac{x_n^3-2x_n+2}{3x_n^2-2}.
      \end{align*}
      Wählt man nun $x_0=-2$, dann erhält man:
      \begin{eqnarray*}
      x_1 &=& x_0-\frac{x_0^3-2x_0+2}{3x_0^2-2} = -1,8, \\
      x_2 &=& x_1-\frac{x_1^3-2x_1+2}{3x_1^2-2} \approx -1,76994819, \\
      x_3 &=& x_2-\frac{x_2^3-2x_2+2}{3x_2^2-2} \approx -1,76929266, \\
      x_4 &=& x_3-\frac{x_3^3-2x_3+2}{3x_3^2-2} \approx -1,76929235, \\
      x_5 &=& x_4-\frac{x_4^3-2x_4+2}{3x_4^2-2} \approx -1,76929235.
      \end{eqnarray*}
    
    \tab{Lösung 1b)}
      Wird das Newton-Verfahren auf die gegebene Funktion angewendet, ergibt sich die Berechnungsvorschrift
      \begin{align*}
        x_{n+1}=x_n-\frac{f(x_n)}{f'(x_n)}=x_n-\frac{x_n^3-2x_n+2}{3x_n^2-2}.
      \end{align*}
      Wählt man nun $x_0=-0.85$, dann erhält man:
      \begin{eqnarray*}
      x_1 &=& x_0-\frac{x_0^3-2x_0+2}{3x_0^2-2} \approx -19,2731343, \\
      x_2 &=& x_1-\frac{x_1^3-2x_1+2}{3x_1^2-2} \approx -12,8736560, \\
      {} &{}& \quad\vdots \\
      x_{10} &=& x_{9}-\frac{x_{9}^3-2x_{9}+2}{3x_{9}^2-2} \approx -1,769300, \\
      x_{11} &=& x_{10}-\frac{x_{10}^3-2x_{10}+2}{3x_{10}^2-2} \approx -1,76929235, \\
      x_{12} &=& x_{11}-\frac{x_{11}^3-2x_{11}+2}{3x_{11}^2-2} \approx -1,76929235.
      \end{eqnarray*}
   
   \tab{Lösung 1c)}
      Wird das Newton-Verfahren auf die gegebene Funktion angewendet, ergibt sich die Berechnungsvorschrift
      \begin{align*}
        x_{n+1}=x_n-\frac{f(x_n)}{f'(x_n)}=x_n-\frac{x_n^3-2x_n+2}{3x_n^2-2}.
      \end{align*}
      Wählt man nun $x_0=0$, dann erhält man:
      \begin{eqnarray*}
      x_1 &=& x_0-\frac{x_0^3-2x_0+2}{3x_0^2-2} = 1, \\
      x_2 &=& x_1-\frac{x_1^3-2x_1+2}{3x_1^2-2} = 0, \\
      x_3 &=& x_2-\frac{x_2^3-2x_2+2}{3x_2^2-2} = 1, \\
      x_4 &=& x_3-\frac{x_3^3-2x_3+2}{3x_3^2-2} = 0, \\
      {} &{}& \vdots
      \end{eqnarray*}
    
    \tab{Lösung 1d)}
        Als Referenz wird die einzige reelle Lösung der gegebenen Funktion (bspw. mit Hilfe der Cardanischen Formel) bestimmt
        \begin{align*}
            x^*=-\sqrt[3]{\frac{9-\sqrt{57}}{9}}-\sqrt[3]{\frac{8/3}{9-\sqrt{57}}} \approx -1,769292354
        \end{align*}
        Zur Einordnung der numerischen Ergebnisse wird zudem der Funktionsgraph mit den drei gegebenen 
        Startwerten und den zugehörigen Tangenten $T_{x_0}$ dargestellt. 
        \begin{figure}
           \image{T303_Exercise04}
        \end{figure}
        Wie zu erkennen ist, befindet sich der Startwert 
        von Aufgabenteil a (rot) in unmittelbarer Nähe der analytischen Lösung. Die zugehörige Tangente weist 
        eine Steigung auf, die dafür sorgt, dass die geforderte Genauigkeit nach fünf Iterationen
        erreicht wird.\\
        Der Startwert von Aufgabenteil b (gelb) befindet sich in der Nähe des lokalen 
        Maximums $x_{Max}=-\sqrt{2/3}\approx -0,82$. Die zugehörige Tangente verläuft entsprechend flach.
        Ergo weicht die erste Approximation relativ weit von der
        analytischen Lösung ab. Die geforderte Genauigkeit wird nach 12 Iterationen erreicht.\\
        Der Startwert von Aufgabenteil c (violett) ist so plaziert, dass das Newton-Verfahren niemals die
        analytische Lösung erreicht. Ein oszillierendes Verhalten stellt sich ein, und die Lösung
        springt ständig zwischen zwei Werten hin und her. Die geforderte Genauigkeit wird niemals erreicht.
        
  \tab{\lang{de}{Lösungsvideo 2)}}
    Die Lösungswege zu Aufgabe 2 können dem folgenden Video entnommen werden:\\
    \youtubevideo[500][300]{YrVs3uaTuPE}\\
        
  \end{tabs*}
\end{content}

