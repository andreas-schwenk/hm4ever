\documentclass{mumie.element.exercise}
%$Id$
\begin{metainfo}
  \name{
    \lang{de}{Ü05: Newton}
    \lang{en}{Exercise 5}
  }
  \begin{description} 
 This work is licensed under the Creative Commons License Attribution 4.0 International (CC-BY 4.0)   
 https://creativecommons.org/licenses/by/4.0/legalcode 

    \lang{de}{}
    \lang{en}{}
  \end{description}
  \begin{components}
    \component{js_lib}{system/media/mathlets/GWTGenericVisualization.meta.xml}{mathlet1}
  \end{components}
  \begin{links}
  \end{links}
  \creategeneric
\end{metainfo}
\begin{content}
\usepackage{mumie.ombplus}
\usepackage{mumie.genericvisualization}

\begin{visualizationwrapper}

\title{\lang{de}{Ü05: Newton}}

\begin{block}[annotation]
  Im Ticket-System: \href{http://team.mumie.net/issues/10495}{Ticket 10495}
\end{block}

%######################################################FRAGE_TEXT
\lang{de}{ Gegeben sei die Funktion $f(x)=x+\cos(x)$ (siehe Grafik).
\begin{enumerate}[a)]
\item a) Anhand der Grafik sieht man, dass $f$ genau eine Nullstelle besitzt. Mit welchen
Kriterien lässt sich dies zeigen?
\item b) Untersuchen Sie anhand der Grafik, zwischen welchen ganzen Zahlen die Nullstelle von 
$f(x)=x+\cos(x)$ liegt, und bestätigen Sie dies rechnerisch.
\item c) Bestimmen Sie (mit Hilfe des Taschenrechners) die ersten Folgeglieder des Newton-Verfahrens, bis sie sich in den ersten 6 Nachkommastellen nicht mehr unterscheiden, und zwar startend mit
\begin{enumerate}[1.]
\item 1) der kleineren der in b) bestimmten Zahlen bzw.
\item 2) der größeren der in b) bestimmten Zahlen.
\end{enumerate}
Runden Sie die Zwischenergebnisse auf $8$ Stellen nach dem Komma.
\end{enumerate}
}

%##################################################ANTWORTEN_TEXT
\begin{tabs*}[\initialtab{0}\class{exercise}]

   %++++++++++++++++++++++++++++++++++++++++++START_TAB_X
  \tab{\lang{de}{   Lösung a)    }}
  \begin{incremental}[\initialsteps{1}]
  
  	 %----------------------------------START_STEP_X
    \step 
    \lang{de}{ In diesem (und vielen anderen Fällen) lässt sich die Existenz genau einer Nullstelle wie folgt zeigen:
    Wenn die  Funktion $f$     streng monoton  ist, dann hat sie höchstens eine Nullstelle.
    Wenn  sie stetig ist und außerdem $\lim_{x\to \infty} f(x)=\infty$ und
$\lim_{x\to -\infty} f(x)=-\infty$ gilt, dann muss sie nach dem Zwischenwertsatz auch mindestens eine Nullstelle haben.


Hier ist $f$ sogar differenzierbar, also sehen wir $f'(x)=1-\sin(x)$ und wegen $|\sin(x)|\leq 1$ ist also $f'(x)\geq 0$ für alle $x\in \R$. 
Also ist $f$ monoton wachsend. Wäre $f$ nicht streng monoton wachsend, so wäre $f$ auf einem (kleinen) Intervall konstant, auf welchem dann $f'(x)=0$ gälte. Jedoch ist $f'(x)=0$ nur für
$x\in \{ 2k\pi+\frac{\pi}{2} | k\in \Z\}$. Daher ist $f$ sogar streng monoton wachsend.

Weiter gilt für alle $x\in \R$:
\[  x-1\leq x+\cos(x)\leq x+1 \]
und damit
\[ \lim_{x\to \infty} f(x)\geq \lim_{x\to \infty} (x-1)=\infty \]
sowie
\[ \lim_{x\to -\infty} f(x)\leq \lim_{x\to -\infty} x+1=-\infty. \] }
  	 %------------------------------------END_STEP_X
 
  \end{incremental}
  %++++++++++++++++++++++++++++++++++++++++++++END_TAB_X
  
   %++++++++++++++++++++++++++++++++++++++++++START_TAB_X
  \tab{\lang{de}{   Lösung b)    }}
  \begin{incremental}[\initialsteps{1}]
  
  	 %----------------------------------START_STEP_X
    \step 
    \lang{de}{ 
Anhand der Grafik sieht man, dass die Nullstelle im Intervall $[-1, 0]$ liegt.
Rechnerisch bestätigt man dies, indem man die Funktionswerte bei $x=-1$ und bei $x=0$ betrachtet:
\begin{eqnarray*}
f(-1) &=& (-1)+\cos(-1)< -1+1=0,\quad \text{da }\cos(-1)\neq 1, \\
f(0) &=& 0+\cos(0)=1>0. 
\end{eqnarray*}
Da $f(-1)<0$ und $f(0)>0$ ist, und die Funktion stetig ist, gibt es nach dem Zwischenwertsatz eine
Stelle $x_0$ im Intervall $[-1, 0]$ mit $f(x_0)=0$, also eine Nullstelle.\\
Das gegebene Intervall ist ein Beispiel. Andere Intervalle wären möglich.
 }
  	 %------------------------------------END_STEP_X
 
  \end{incremental}
  %++++++++++++++++++++++++++++++++++++++++++++END_TAB_X
  
   %++++++++++++++++++++++++++++++++++++++++++START_TAB_X
  \tab{\lang{de}{   Lösung c)    }}
  \begin{incremental}[\initialsteps{1}]
  
  	 %----------------------------------START_STEP_X
    \step 
    \lang{de}{ 
Die Rekursionsformel für das Newton-Verfahren lautet:
\[ x_{n+1}=x_n-\frac{f(x_n)}{f'(x_n)}=x_n-\frac{x_n+\cos(x_n)}{1-\sin(x_n)}=\frac{x_n-x_n\sin(x_n)-x_n-\cos(x_n)}{1-\sin(x_n)}=-\frac{x_n\sin(x_n)+\cos(x_n)}{1-\sin(x_n)}. \]

Für $x_0=-1$ erhält man daher:
\begin{eqnarray*}
x_1&=& -\frac{-1\cdot \sin(-1)+\cos(-1)}{1-\sin(-1)}\approx -0,75036387 \\
x_2&=& -\frac{x_1\sin(x_1)+\cos(x_1)}{1-\sin(x_1)}\approx -0,73911289 \\
x_3&=& -\frac{x_2\sin(x_2)+\cos(x_2)}{1-\sin(x_2)}\approx -0,73908513 \\
x_4&=& -\frac{x_3\sin(x_3)+\cos(x_3)}{1-\sin(x_3)}\approx -0,73908513
\end{eqnarray*}
Auf 6 Stellen nach dem Komma liegt die Nullstelle also bei $-0,739085$.

Für $x_0=0$ erhält man entsprechend:
\[  x_1= -\frac{0\cdot \sin(0)+\cos(0)}{1-\sin(0)}=-1. \]
Die weiteren Folgeglieder sind daher die obigen, jedoch im Index um $1$ verschoben.
}
  	 %------------------------------------END_STEP_X
 
  \end{incremental}
  %++++++++++++++++++++++++++++++++++++++++++++END_TAB_X

%#############################################################ENDE
\end{tabs*}

\begin{genericGWTVisualization}[700][1000]{mathlet1}
\lang{de}{\title{Die Funktion cos(x)+x}}
\begin{variables}
	
	\function{f}{real}{x+cos(x)}

		
 \color{f}{#0066CC}
 
	
\end{variables}
\begin{canvas}
	\plotSize{600}
	\plotLeft{-10}
	\plotRight{10}
	\plot[coordinateSystem]{f}
\end{canvas}


\end{genericGWTVisualization}

\end{visualizationwrapper}
\end{content}