%$Id:  $
\documentclass{mumie.article}
%$Id$
\begin{metainfo}
  \name{
    \lang{de}{Regel von de l'Hospital}
    \lang{en}{L'Hopital's rules}
  }
  \begin{description} 
 This work is licensed under the Creative Commons License Attribution 4.0 International (CC-BY 4.0)   
 https://creativecommons.org/licenses/by/4.0/legalcode 

    \lang{de}{Beschreibung}
    \lang{en}{Description}
  \end{description}
  \begin{components}
    \component{generic_image}{content/rwth/HM1/images/g_img_00_Videobutton_schwarz.meta.xml}{00_Videobutton_schwarz}
    \component{generic_image}{content/rwth/HM1/images/g_img_00_video_button_schwarz-blau.meta.xml}{00_video_button_schwarz-blau}
  \end{components}
  \begin{links}
    \link{generic_article}{content/rwth/HM1/T303_Approximationen/g_art_content_04_taylor_polynom.meta.xml}{content_04_taylor_polynom}
    \link{generic_article}{content/rwth/HM1/T303_Approximationen/g_art_content_06_de_l_hospital.meta.xml}{content_06_de_l_hospital}
    \link{generic_article}{content/rwth/HM1/T210_Stetigkeit/g_art_content_31_grenzwerte_von_funktionen.meta.xml}{grenzw-funk}
    \link{generic_article}{content/rwth/HM1/T301_Differenzierbarkeit/g_art_content_03_hoehere_ableitungen.meta.xml}{hoehere-abl}
  \end{links}
  \creategeneric
\end{metainfo}
\begin{content}
\usepackage{mumie.ombplus}
\ombchapter{2}
\ombarticle{3}

\title{\lang{de}{Regel von de l'Hospital}\lang{en}{L'Hopital's rules}}
 
\begin{block}[annotation]
  
  
\end{block}
\begin{block}[annotation]
  Im Ticket-System: \href{http://team.mumie.net/issues/10033}{Ticket 10033}\\
\end{block}

\begin{block}[info-box]
\tableofcontents
\end{block}

\section{\lang{de}{Regel von de l'Hospital}\lang{en}{L'Hopital's rule}}\label{sec:lHospital}

\lang{de}{
In Anwendungen hat man oft mit Funktionen zu tun, die zum Beispiel Summen, Produkte oder Quotienten 
einfacher Funktionen sind. Die Grenzwerte solcher komplizierterer Funktionen kann man oft aus den 
Grenzwerten ihrer einfachen Bausteine erhalten, indem man die 
\ref[grenzw-funk][Grenzwertregeln]{sec:grenzwertregeln} anwendet 
- sofern die Ausdr\"ucke auch f\"ur die Grenzwerte definiert sind.
\\\\
Ist zum Beispiel 
$\lim_{x\to x_0}\;f(x) =f_0$ und $\lim_{x\to x_0}\;g(x) =g_0\neq 0,$ dann gilt 
$\nowrap{\lim_{x\to x_0}\;\frac{f(x)}{g(x)} = \frac{f_0}{g_0}}$.
\\\\
Wenn aber $f$ und $g$ beide gegen  Null oder Unendlich streben, dann gibt es keinen offensichtlichen Grenzwert 
 $\lim_{x\to x_0}\;\frac{f(x)}{g(x)}$. Ausdrücke wie $\;$ \glqq $\frac{0}{0}$\grqq{} $\,$ oder $\,$ \glqq $\frac{\infty}{\infty}$\grqq{}
 $\,$ sind nicht definiert.
\\\\
In solchen F\"allen kommt es nämlich darauf an, welche der Funktionen  $f$ und $g$ \emph{stärker} oder \emph{schneller} gegen ihren Grenzwert streben.
Es kommt also auch auf die Art der Änderung der Funktionen an. Und  die haben wir durch deren Ableitung beschrieben.
\\\\
Die folgenden Resultate über Grenzwerte, die als Regeln von de l'Hospital bezeichnet werden, machen sich das zu Nutze.
Mit Hilfe der Differentialrechnung liefern sie uns die Möglichkeit, Grenzwerte so umzuformen, 
dass wir in Situation gelangen können, in denen wir die Grenzwertsätze anwenden dürfen.
\\\\
%Wir erwarten einen Grenzwert Null,
% wenn der Z\"ahler $f(x)$ wesentlich schneller gegen Null strebt als der Nenner $g(x)$. Und wenn es anders herum ist,
% sollte der Grenzwert Unendlich sein. Einen endlichen Grenzwert erwarten wir, wenn $f$ und $g$ vergleichbar schnell 
% konvergieren. Beispielsweise konvergieren $\sin x$ und $x$ f\"ur $x\to 0$ vergleichbar schnell gegen Null, es gilt
%   $\lim\; (\sin x)/x =1$ f\"ur $x\to 0.$ Andererseits strebt $1 - \cos x$ quadratisch gegen Null, daher gilt
% $\lim \;(1-\cos x)/x =0$ und $\lim \;(1-\cos x)/x^2 =1/2$ f\"ur $x\to 0.$ 
 
%  Die Differentialrechnung liefert ein zus\"atzliches Hilfsmittel f\"ur solche Untersuchungen. 
%  Die Ableitung beschreibt gerade quantitativ die \"Anderungsrate einer Funktion. Mit ihr reduzieren wir das 
%  Problem auf einen Fall, in dem die Grenzwerts\"atze anwendbar sind.
Wir beginnen mit der einfachsten Variante dieser Idee.
}
\lang{en}{
In applications we often deal with functions that are sums, products or quotients of other 
functions. Limits of such complicated functions can sometimes be determined using the limits of 
their more simple components by using the \ref[grenzw-funk][limit rules]{sec:grenzwertregeln}, 
provided every expression is well-defined.
\\\\
For example, if 
$\lim_{x\to x_0}\;f(x) =f_0$ and $\lim_{x\to x_0}\;g(x) =g_0\neq 0,$ then 
$\nowrap{\lim_{x\to x_0}\;\frac{f(x)}{g(x)} = \frac{f_0}{g_0}}$.
\\\\
However, if both $f$ and $g$ tend towards zero or diverge to infinity, then there is no obvious limit 
$\lim_{x\to x_0}\;\frac{f(x)}{g(x)}$. Expressions such as $\;$ ' $\frac{0}{0}$' $\,$ and 
$\,$ ' $\frac{\infty}{\infty}$' $\,$ are not well defined.
\\\\
These cases effectively come down to which of the two functions $f$ and $g$ tends towards their 
limit \emph{faster}, or which is \emph{stronger}. We therefore need to consider the rate of change 
of the functions, that is, their derivatives.
\\\\
The following results, which are referred to as l'Hopital's rules, make use of derivatives to 
determine the values of such limits. We begin with the simplest variant of this idea.
}
\begin{theorem}\label{thm:de-l-Hospital_1}
  \lang{de}{
  Sei $I\subseteq\R$ ein offenes Intervall, $f,g\colon I\to \R$ differenzierbar mit $g'(x)\neq 0$ f\"ur $x\in I$, sowie $x_0\in I$
  oder $x_0$ ein Randpunkt von $I$.
  Weiter sei
  }
  \lang{en}{
  Let $I\subseteq\R$ be an interval, $f,g\colon I\to \R$ differentiable with $g'(x)\neq 0$ for 
  $x\in I$, and $x_0\in I$, or $x_0$ on the boundary (edge) of $I$. 
  Assume that
  }
  \[ \lim_{x\to x_0} f(x) = \lim_{x\to x_0} g(x) =0.
  \]
%  \lang{de}{oder}
%  \lang{en}{or that}
%  \\
%  \[ \lim_{x\to x_0} f(x) = \infty,\quad \lim_{x\to x_0} g(x) =\infty.
%  \]

  \lang{de}{
  Wenn der Grenzwert $\lim_{x\to x_0}\;\frac{f'(x)}{g'(x)} = A$ existiert, 
  dann konvergiert auch der Quotient der Funktionen und es gilt
  }
  \lang{en}{
  If the limit $\lim_{x\to x_0}\;\frac{f'(x)}{g'(x)} = A$ exists, then the quotient of the initial 
  functions converges, with
  }
  \[\lim_{x\to x_0}\;\frac{f(x)}{g(x)}= \lim_{x\to x_0}\;\frac{f'(x)}{g'(x)} = A.
  \]
\end{theorem}

\begin{proof*}
\begin{incremental}[\initialsteps{0}]
\step \lang{de}{
Zunächst interpretieren wir die Voraussetzung $\lim_{x\to x_0} f(x) = \lim_{x\to x_0} g(x) =0$.
Ist $x_0\in I$, dann folgt aus der Stetigkeit von $f$ und $g$, dass $f(x_0)=0=g(x_0)$.
Falls $x_0$ ein Randpunkt von $I$ ist, so können wir $f$ und $g$ mittels $f(x_0)=0=g(x_0)$ stetig in $x_0$ fortsetzen.
}
\lang{en}{
Firstly we consider the condition $\lim_{x\to x_0} f(x) = \lim_{x\to x_0} g(x) =0$. 
If $x_0\in I$, then by the continuity of $f$ and $g$, $f(x_0)=0=g(x_0)$. 
If $x_0$ lies on the boundary of $I$, then we may extend the functions $f$ and $g$ by setting 
$f(x_0)=0=g(x_0)$.
}
\step \lang{de}{
Nun wenden wir den \lref{sec:verallg-mws}{verallgemeinerten Mittelwertsatz} an, dessen Beweis wir auf 
das Ende dieses Kapitels verschieben. Nach diesem gibt es zu jedem $x$ eine Zahl $\xi$ (abhängig von 
$x$) zwischen $x$ und $x_0$ mit
}
\lang{en}{
Now we apply the \lref{sec:verallg-mws}{general mean value theorem}, whose proof we give at the end 
of the chapter. By this theorem, for each $x$ there exists a number $\xi$ (dependent on $x$) 
between $x$ and $x_0$ with
}
\[ \frac{f'(\xi)}{g'(\xi)}=\frac{f(x)-f(x_0)}{g(x)-g(x_0)}. \]
\lang{de}{
Betrachtet man nun $x\to x_0$, dann strebt auch dieses kleine dazwischen liegende $\xi$ gegen $x_0$.
Somit gilt, weil  $f(x_0)=0=g(x_0)$,
}
\lang{en}{
If we take $x\to x_0$, then $\xi$ tends towards $x_0$ too. As $f(x_0)=0=g(x_0)$, we have
}
\begin{eqnarray*}
 \lim_{x\to x_0} \;\frac{f(x)}{g(x)} &=& \lim_{x\to x_0} \frac{f(x)-f(x_0)}{g(x)-g(x_0)}=\lim_{x\to x_0}\frac{f'(\xi)}{g'(\xi)}\\
&=&  \lim_{\xi\to x_0}\frac{f'(\xi)}{g'(\xi)} = \lim_{x\to x_0} \frac{f'(x)}{g'(x)}.
\end{eqnarray*}
\end{incremental}
\end{proof*}

\lang{de}{
Etwas komplizierter nachzuweisen sind die folgenden Varianten der Regel von de l'Hospital, die 
ebenfalls unter diesem Namen firmieren.
\\\\
Die Unterschiede zur obigen Regel sind lediglich, dass bei 1. die Funktionen $f$ und $g$ gegen 
unendlich streben für $x$ gegen $x_0$, und dass bei 2. die Variable $x$ gegen $\infty$ strebt statt 
gegen einen Wert $x_0$.
}
\lang{en}{
Somewhat more complicated to verify are the following variants of l'Hopital's rule, which are given 
the same name.
\\\\
The differences between these are as follows: in 1. the functions $f$ and $g$ diverge 
towards infinity as $x$ tends towards $x_0$, and in 2. the variable $x$ itself tends towards $\infty$ 
rather than some value $x_0$.
}

\begin{theorem}\label{thm:de-l-hospital}
\begin{enumerate}
\item \lang{de}{
Sei $I\subseteq\R$ ein offenes Intervall, $f,g\colon I\to \R$ differenzierbar mit $g'(x)\neq 0$ f\"ur 
$x\in I$, sowie $x_0\in I$ oder $x_0$ ein Randpunkt von $I$. Gilt
}
\lang{en}{
Let $I\subseteq\R$ be an open interval, $f,g\colon I\to \R$ differentiable with $g'(x)\neq 0$ for 
$x\in I$, and either $x_0\in I$ or $x_0$ on the boundary of $I$. If
}
   \[ \lim_{x\to x_0} f(x) = \infty,\quad\text{\lang{de}{und}\lang{en}{and}}\quad \lim_{x\to x_0} g(x) =\infty, \]
\lang{de}{
und existiert der Grenzwert 
$ \lim_{x\to x_0}\;\frac{f'(x)}{g'(x)} = A, $
so gilt
}
\lang{en}{
holds, and the limit 
$ \lim_{x\to x_0}\;\frac{f'(x)}{g'(x)} = A$
exists, then
}
\[   \lim_{x\to x_0}\;\frac{f(x)}{g(x)} = \lim_{x\to x_0}\;\frac{f'(x)}{g'(x)} = A. \]
\item \lang{de}{
Ist $I=(a,\infty)\subseteq \R$ und $f,g\colon I\to \R$ differenzierbar mit $g'(x)\neq 0$ f\"ur $x\in I$. 
Gilt nun
}
\lang{en}{
Let $I=(a,\infty)\subseteq \R$ and let $f,g\colon I\to \R$ be differentiable functions with 
$g'(x)\neq 0$ for $x\in I$. If
}
  \[ \lim_{x\to \infty} f(x) = \lim_{x\to \infty} g(x) =0
  \quad
  \text{\lang{de}{oder}\lang{en}{or}}
  \quad \lim_{x\to \infty} f(x) = \infty= \lim_{x\to \infty} g(x),
  \]
\lang{de}{
und existiert der Grenzwert 
$ \lim_{x\to \infty}\;\frac{f'(x)}{g'(x)} = A,$
so gilt
}
\lang{en}{
holds and the limit 
$ \lim_{x\to \infty}\;\frac{f'(x)}{g'(x)} = A$ 
exists, then
}
\[   \lim_{x\to \infty}\;\frac{f(x)}{g(x)} = \lim_{x\to \infty}\;\frac{f'(x)}{g'(x)} = A. \]
\end{enumerate}
\end{theorem}

\begin{remark}
\begin{enumerate}
\item \lang{de}{Die Regel ist auch richtig, wenn $x$ gegen $-\infty$ strebt.}
\lang{en}{The rule also applies when $x$ tends towards $-\infty$. }
%\item Die Regel ist auch richtig, wenn $\lim_{x\to x_0}\;\frac{f'(x)}{g'(x)}=\infty$ oder $=-\infty$ ist.
%
\item \lang{de}{
Ergibt $\lim_{x\to x_0}\;\frac{f'(x)}{g'(x)}$ (mit $x_0\in \R$ oder $x_0=\infty$) wieder einen 
Ausdruck wie $\;$ "$0/0$" $\,$ oder $\,$ "$\infty/\infty$" $\,$, so kann man oft die Regel von de 
l'Hospital erneut (oder sogar mehrfach) anwenden und erhält eventuell einen Grenzwert
$\lim_{x\to x_0}\;\frac{f''(x)}{g''(x)}$ bzw. $\lim_{x\to x_0}\;\frac{f^{(n)}(x)}{g^{(n)}(x)}$ für $n>2$.
\\\\
Jeder Schritt erhält dann erst nachträglich seine Legitimation, wenn im letzten Schritt die Existenz 
des Limes gezeigt wurde.
\\\\
Dabei ist es wichtig, in jedem Schritt die Voraussetzungen an die Differenzierbarkeit von Zähler und 
Nenner sowie die Nullstellenfreiheit der Ableitung des Nenners in einer Umgebung des Grenzpunkts zu 
prüfen.
}
\lang{en}{
If $\lim_{x\to x_0}\;\frac{f'(x)}{g'(x)}$ (where $x_0\in \R$ or $x_0=\infty$) yields yet again an 
expression such as $\;$ "$0/0$" $\,$ or $\,$ "$\infty/\infty$" $\,$, then often l'Hopital's rule can 
be applied again (and even multiple times) until finally a limit 
$\lim_{x\to x_0}\;\frac{f''(x)}{g''(x)}$, or indeed $\lim_{x\to x_0}\;\frac{f^{(n)}(x)}{g^{(n)}(x)}$ 
for $n>2$, is obtained.
\\\\
Every step requires the existence of the limit, which is given only when (or rather, if) this final 
limit is reached, and which then recursively justifies the previous uses of the theorem.
\\\\
Thus it is important to verify at every step the differentiability of the numerator and the 
denominator, as well as checking that the denominator is non-zero within the interval in question.
}
\end{enumerate}
\end{remark}


\begin{example}
\begin{tabs*}[\initialtab{0}]
\tab{\glqq $\frac{0}{0}$ \grqq{}}
\lang{de}{
Um den Grenzwert $\lim_{x\to 0} \;\frac{\tan(x)}{x} $ zu berechnen, bemerken wir zum einen 
$\lim_{x\to 0} \tan(x)=0=\lim_{x\to 0} x$ und zum anderen, dass $f(x)=\tan x$ und $g(x)=x$ stetige 
Funktionen repräsentieren mit $f'(x)=1+\tan(x)^2$ und $g'(x)=1\neq 0$. Also können wir die Regel von 
de l'Hospital anwenden und erhalten
}
\lang{en}{
To determine the value of the limit $\lim_{x\to 0} \;\frac{\tan(x)}{x}$, we firstly note both that
$\lim_{x\to 0} \tan(x)=0=\lim_{x\to 0} x$, and that $f(x)=\tan x$ and $g(x)=x$ are continuous 
functions with $f'(x)=1+\tan(x)^2$ and $g'(x)=1\neq 0$. Hence we can apply l'Hopital's rule here and 
obtain
}
\[ \lim_{x\to 0} \;\frac{\tan(x)}{x} =\lim_{x\to 0} \frac{1+\tan(x)^2}{1} =1. \]
\tab{\glqq $\frac{\infty}{\infty}$\grqq{}}
\lang{de}{
Berechnung von $\lim_{x\to \infty} \;\frac{\ln(x)}{x}$:
\\\\
In diesem Fall gilt $\lim_{x\to \infty} \ln(x)=\infty$ und $\lim_{x\to \infty} x=\infty$, 
sowie $\ln'(x)=\frac{1}{x}$. Der Nenner $x$ hat eine konstante Ableitung. Also können wir die 
Regel von de l'Hospital anwenden und erhalten
}
\lang{en}{
We determine $\lim_{x\to \infty} \;\frac{\ln(x)}{x}$:
\\\\
In this case we have $\lim_{x\to \infty} \ln(x)=\infty$, $\lim_{x\to \infty} x=\infty$ and 
$\ln'(x)=\frac{1}{x}$. The denominator $x$ has a constant derivative. Hence we may apply l'Hopital's 
rules and obtain
}
  \[ \lim_{x\to \infty} \;\frac{\ln(x)}{x} =\lim_{x\to \infty} \;\frac{1/x}{1} =0.\]
\tab{\glqq $\frac{0}{0}$\grqq{} (\lang{de}{iterierte Anwendung}
                                 \lang{en}{iterating l'Hopital's rule})}
\lang{de}{
Um den Grenzwert $\lim_{x\to\infty}\,\frac{x^n}{e^x}$ zu berechnen, muss man die Regel von de 
l'Hospital $n$-mal anwenden.
\\\\
Dazu überlegt man sich, dass $f(x)=x^n$ beliebig oft stetig differenzierbar ist mit 
$\lim_{x\to\infty}f^{(k)}(x)=\infty$ für $k<n$, und dass $g(x)=e^x$ ebenfalls beliebig oft stetig 
differenzierbar ist mit $g^{(k)}(x)=g(x)\neq 0$ und $\lim_{x\to\infty}g^{(k)}(x)=\infty$.
\\\\
Also gilt
}
\lang{en}{
To determine the value of the limit $\lim_{x\to\infty}\,\frac{x^n}{e^x}$, we can apply l'Hopital's 
rules $n$ times.
\\\\
To do this, we use the fact that both $f(x)=x^n$ and $g(x)=e^x$ are arbitrarily many times 
continuously differentiable, with $\lim_{x\to\infty}f^{(k)}(x)=\infty$ for $k<n$, and 
$g^{(k)}(x)=g(x)\neq 0$ so $\lim_{x\to\infty}g^{(k)}(x)=\infty$.
\\\\
Hence we have
}
\[ \lim_{x\to\infty}\,\frac{x^n}{e^x} =\lim_{x\to\infty}\,\frac{nx^{n-1}}{e^x}=\ldots =\lim_{x\to\infty}\,\frac{n! x^0}{e^x}=0.\]
\lang{de}{Dies zeigt, dass die Exponentialfunktion schneller wächst als jede Potenz von $x$.}
\lang{en}{This shows that the exponential function grows faster than any power of $x$ does.}
\end{tabs*}
\end{example}
\begin{quickcheck}
\text{\lang{de}{
Welche der folgenden Grenzwertprobleme bieten das Potential, mit der Regel von de l'Hospital berechnet zu werden?
}
\lang{en}{
Which of the following limits could be calculated using l'Hopital's rule?
}}
\begin{choices}{multiple}

        \begin{choice}
            \text{$\lim_{x\to 0}\frac{x^4}{\cos x}$}
			\solution{false}
         \end{choice}
                    
        \begin{choice}
            \text{$\lim_{x\to \infty}\frac{e^x}{e^x-x}$}
			\solution{true}
		\end{choice} 
        \begin{choice}
            \text{$\lim_{x\to \infty}\frac{x^3}{\cos x}$}
			\solution{false}
        \end{choice} 
        \begin{choice}
            \text{$\lim_{x\to 0}\frac{x^3}{\sin x}$}
			\solution{true}
		\end{choice}
        \begin{choice}
            \text{$\lim_{x\to 0}\frac{\ln \frac{1}{x}}{\frac{1}{x}}$}
			\solution{true}
		\end{choice}
\end{choices}
\explanation{\lang{de}{
Den ersten Grenzwert berechnet man direkt $\lim_{x\to 0}\frac{x^4}{\cos x}=\frac{0}{1}=1$.\\
Bei $\lim_{x\to \infty}\frac{e^x}{e^x-x}$ liegt der Fall \glqq$\frac{\infty}{\infty}$\grqq{} vor.\\
Weil der Grenzwert $\lim_{x\to\infty}\cos x$ nicht exisitiert, kann man für 
$\lim_{x\to \infty}\frac{x^3}{\cos x}$ keine de l'Hospitalsche Regel anwenden.\\
Bei $\lim_{x\to 0}\frac{x^3}{\sin x}$ liegt der Fall \glqq$\frac{0}{0}$\grqq{} vor, und bei
$\lim_{x\to \infty}\frac{\ln \frac{1}{x}}{\frac{1}{x}}$ der Fall \glqq$(-1)\cdot\frac{\infty}{\infty}$\grqq{}.
}
\lang{en}{
The first limit can be directly computed as $\lim_{x\to 0}\frac{x^4}{\cos x}=\frac{0}{1}=1$.\\
For $\lim_{x\to \infty}\frac{e^x}{e^x-x}$ we may use l'hopital's rule for '$\frac{\infty}{\infty}$'.
As the limit $\lim_{x\to\infty}\cos x$ does not exist, we may not apply l'Hopital's rule to 
$\lim_{x\to \infty}\frac{x^3}{\cos x}$.\\
The limit $\lim_{x\to 0}\frac{x^3}{\sin x}$ allows us to use the $\frac{0}{0}$ version of l'Hopital's 
rule, and the limit $\lim_{x\to \infty}\frac{\ln \frac{1}{x}}{\frac{1}{x}}$ the version for 
$(-1)\cdot\frac{\infty}{\infty}$.
}}
\end{quickcheck}


\lang{de}{
Neben den F\"allen $\;$ \glqq $\frac{0}{0}$ \grqq{} $\;$ und 
$\,$ \glqq $\frac{\infty}{\infty}$\grqq{}, $\,$ die oben erw\"ahnt wurden, kann man auch F\"alle wie 
$\;$ \glqq $0\cdot\infty$\grqq{} oder \glqq $\infty - \infty$\grqq{} handhaben. Die Idee dabei ist, 
die Terme so umzuformen, dass diese Terme, oder Teile davon, die Ausdrücke 
\glqq $\frac{0}{0}$\grqq{} oder \glqq $\frac{\infty}{\infty}$\grqq{} als Grenzwerte hätten.
\\\\
Direkte Regeln für solche Ausdrücke gibt es nicht! Zum Beispiel ist es falsch, den Grenzwert 
$\lim_{x\to x_0} f(x)g(x)$, wobei $f$ gegen $0$ und $g$ gegen $\infty$ streben, mit dem Produkt der 
Ableitungen von $f$ und $g$ in $x_0$ in Beziehung zu setzen.
\\\\
Auch wenn Potenzen vorkommen, helfen die Grenzwertsätze oft, um die Grenzwerte zu berechnen. 
So ist zum Beispiel
}
\lang{en}{
In addition to the cases $\;$ '$\frac{0}{0}$' $\;$ and $\,$ '$\frac{\infty}{\infty}$' $\,$ that were 
handled above, it also makes sense to consider cases $\;$ '$0\cdot\infty$' and '$\infty - \infty$'. 
The idea behind computing limits in these cases is to manipulate the expressions in question in order 
to get expressions with '$\frac{0}{0}$' or '$\frac{\infty}{\infty}$' as limits.
\\\\
There are no direct rules for such expressions! For example, it would be incorrect to assume that the 
limit $\lim_{x\to x_0} f(x)g(x)$, where $f$ tends towards $0$ and $g$ tends towards $\infty$, 
corresponds to the product of the derivatives of $f$ and $g$ at $x_0$.
\\\\
Even when powers appear in a limit, we can often use the limits of the powers to compute the value of 
the limit itself. For example, if both limits exist \emph{and} the corresponding exponential function 
$(\lim_{x\to x_0}f(x))^y$ is continuous in $y$, then we have
}
\[\lim_{x\to x_0}f(x)^{g(x)}=(\lim_{x\to x_0}f(x))^{\lim_{x_0\to x}g(x)},\]
\lang{de}{
\emph{falls} beide Grenzwerte existieren \emph{und} die entstehende Exponentialfunktion 
$(\lim_{x\to x_0}f(x))^y$ stetig in $y$ ist. Es gibt im Wesentlichen drei Fälle, wo diese Bedingungen 
verletzt sind, nämlich Grenzwertausdrücke der Form $\,$ "$1^\infty$" $\,$, "$0^0$" oder "$\infty^0$".
\\\\
Auch dazu zeigen die folgenden Beispiele Vorgehensweisen, sie zu berechnen.
Ein festes Schema gibt es dabei nicht, es braucht viel mehr ein scharfes Auge bzw. Erfahrung, 
um eine zielführende Umformung zu erkennen.
}
\lang{en}{
There are three common cases in which these conditions are not met: $\,$ '$1^\infty$' $\,$, '$0^0$' 
and '$\infty^0$'.
\\\\
The following exercises show possible approaches even to calculate limits which fall into these 
cases. There is no exact algorithm for this, and it requires intuition, built through practice, 
to know which manipulations are to be done.
}
 
% \begin{enumerate}
% \item Sind Funktionen $f$, $g$ und eine "`Stelle"' $x_0\in \R\cup \{\infty\}$ mit
% \[  \lim_{x\to x_0} f(x)=0\quad\text{und}\quad \lim_{x\to x_0} g(x)=\infty \]
% gegeben und möchte man den Grenzwert $\lim_{x\to x_0} f(x)\cdot g(x)$ berechnen, schreibt man das Produkt
% um in die Form $f(x)\cdot g(x)=\frac{f(x)}{1/g(x)}$ und wendet die Regel von de l'Hospital auf diesen Quotienten an.
%Wenn $f(x)>0$ für alle $x$ und daher $\lim_{x\to x_0} 1/f(x)=\infty$, kann man auch alternativ die Regel von de l'Hospital 
%auf den Quotienten $\frac{g(x)}{1/f(x)}$ anwenden.
%\item Will man den Grenzwert $\lim_{x\to x_0} f(x)-g(x)$ berechnen, wobei $\lim_{x\to x_0} f(x)=\infty$ und
%$\lim_{x\to x_0} g(x)=\infty$, so klammert man einen geeigneten Faktor, z.B. $f(x)$ aus und erhält
%$f(x)-g(x)=f(x)\cdot (1-g(x)/f(x))$ 
% 
% \end{enumerate}


\begin{example}
\begin{tabs*}[\initialtab{0}]
\tab{\glqq $0\cdot\infty$\grqq{} (a)}
\lang{de}{
Versuchte man den Ausdruck $\lim_{x\nearrow \frac{\pi}{2}} \;(x- \frac{\pi}{2})\cdot \tan (x)$ 
faktorweise zu berechnen, erhielte man einen Ausdruck der Form "`$0\cdot\infty$"'. Dies formen wir zu 
einem Ausdruck "$0/0$" um
}
\lang{en}{
If we try to compute the expression 
$\lim_{x\nearrow \frac{\pi}{2}} \;(x- \frac{\pi}{2})\cdot \tan (x)$ factor by factor, we obtain an 
expression of the form $0\cdot\infty$. We can manipulate the expressions to obtain '$0/0$' instead.
}
\[   (x- \frac{\pi}{2})\cdot \tan (x) =  \frac{x- \frac{\pi}{2}}{\cot (x)} ,\]
\lang{de}{und wenden dann die Regel von de l'Hospital an, wobei wir $\cot'(x)=-1-\cot(x)^2$ verwenden}
\lang{en}{and then apply l'Hopital's rule, using $\cot'(x)=-1-\cot(x)^2$:}
 \[ \lim_{x\nearrow \frac{\pi}{2}} \;(x- \frac{\pi}{2})\cdot \tan (x) = \lim_{x\nearrow \frac{\pi}{2}} \frac{x- \frac{\pi}{2}}{\cot (x)}
  =  \lim_{x\nearrow \frac{\pi}{2}} \frac{1}{-1-\cot (x)^2} = -1.\]
\lang{de}{
(Weil $\cot'(\frac{\pi}{2})-1$ und $\cot'(x)$ stetig ist, gilt $\cot(x)\neq 0$ für $x$ nahe 
$\frac{\pi}{2}$.)
}
\lang{en}{
(As $\cot'(\frac{\pi}{2})-1$ and as $\cot'(x)$ is continuous, we have $\cot(x)\neq 0$ for $x$ near 
$\frac{\pi}{2}$.)
}
\tab{ \glqq $0\cdot\infty$\grqq{} (b)}
\lang{de}{
In anderen F\"allen ist es g\"unstiger, den Term \glqq $0\cdot\infty$\grqq{} zu "$\infty/\infty$" 
umzuformen, wie im folgenden Beispiel.\\
Versucht man den Ausdruck $\lim_{x\to \infty} x\cdot e^{-x}$ faktorweise zu berechnen, erhielte man 
einen Ausdruck der Form "`$\infty\cdot 0$"'. Hier schreibt man den Term besser um zu
}
\lang{en}{
In other cases it makes more sense to manipulate an expression '$0\cdot\infty$' into an expression 
'$\infty/\infty$', like in the following example.\\
If we try to compute the expression $\lim_{x\to \infty} x\cdot e^{-x}$ factor by factor, we obtain 
an expression of the form '$\infty\cdot 0$'. We can rewrite the expression as
}
\[   x\cdot e^{-x}=\frac{x}{e^x} \]
\lang{de}{und berechnet}
\lang{en}{and calculate}
\[ \lim_{x\to \infty} x\cdot e^{-x}=\lim_{x\to \infty} \frac{x}{e^x} =\lim_{x\to \infty} \frac{1}{e^x} = 0.\]
\lang{de}{
Würde man jedoch in der Absicht, de l'Hospital anzuwenden, den Term als $\frac{e^{-x}}{x^{-1}}$ 
schreiben, erhielte man den Quotienten der Ableitung als  $\frac{-e^{-x}}{-x^{-2}}$. Dessen Grenzwert 
kann wieder nicht direkt berechnet werden.
}
\lang{en}{
If we insisted on applying l'Hopital's rule to the expression rewritten instead as
$\frac{e^{-x}}{x^{-1}}$, the quotient of the derivatives would be $\frac{-e^{-x}}{-x^{-2}}$. 
However, the limit of this cannot be calculated directly.
}
\tab{"`$\infty - \infty$"'}
\lang{de}{
Würde man den Grenzwert $\lim_{x\to 0} \;\left(\frac{1}{e^x-1} - \frac{1}{x}\right) $ summandenweise 
berechnen wollen, erhielte man einen Ausdruck "`$\infty - \infty$"'. Auch in diesem Fall formt man 
den Term um, um als Grenzwert einen Ausdruck der Form "$0/0$" oder "$\infty/\infty$" zu bekommen, auf 
welchen man wieder die Regel von de l'Hospital anwenden kann (hier am Ende sogar zweimal)
}
\lang{en}{
If we tried to calculate the limit $\lim_{x\to 0} \;\left(\frac{1}{e^x-1} - \frac{1}{x}\right) $ term 
by term, we would obtain an expression '$\infty - \infty$'. Instead, we can manipulate the expression 
in order to get something of the forms '$0/0$' or '$\infty/\infty$', and apply l'Hopital's rule to 
this (twice, in this case).
}
  \[ \lim_{x\to 0} \;\left(\frac{1}{e^x-1} - \frac{1}{x}\right) =  \lim_{x\to 0} \frac{x-(e^x-1)}{x\,e^x - x}
  =  \lim_{x\to 0} \frac{1-e^x}{e^x+x\,e^x -1} = \lim_{x\to 0} \;\frac{-e^x}{2e^x+x\,e^x} =\frac{-1}{2}.
  \]
  \tab{\glqq $1^\infty$\grqq{}}
\lang{de}{
Bei dem Grenzwert $ \lim_{x\to \infty} \;\left(1+\frac{1}{x}\right)^x$ würde man einen Ausdruck 
"$1^\infty$" bekommen, wenn man die Grenzwerte von Basis und Exponent separat berechnet. Auch dies 
macht keinen Sinn.
\\\\    
In solchen Fällen berechnet man zunächst den Logarithmus des Terms, also
}
\lang{en}{
The limit $ \lim_{x\to \infty} \;\left(1+\frac{1}{x}\right)^x$ yields an expression '$1^\infty$' if 
we try to compute the limits of the base and the exponent seperately. This makes no sense.
\\\\
In such cases we first calculate the logarithm of the expression,
}
  \[ \ln\left( \big(1+\frac{1}{x}\big)^x\right) =x\cdot \ln \big(1+\frac{1}{x}\big) = \frac{\ln \big(1+\frac{1}{x}\big)}{1/x} \]
\lang{de}{und dessen Grenzwert mit der Regel von de l'Hospital (Voraussetzungen prüfen!)}
\lang{en}{and find its limit using l'Hopital's rule (and remembering to check the conditions!)}
   \[  \lim_{x\to \infty}\frac{\ln \big(1+\frac{1}{x}\big)}{1/x} = \lim_{x\to \infty}\frac{-\frac{1}{x^2}\cdot \frac{1}{1+1/x}}{-\frac{1}{x^2}} = \lim_{x\to \infty} \frac{1}{1+1/x} =1.\]

\lang{de}{Da die Exponentialfunktion stetig ist, gilt dann}
\lang{en}{As the exponential function is continuous, we then have}
	\[    \lim_{x\to \infty} \;\left(1+\frac{1}{x}\right)^x= \lim_{x\to \infty} \exp\left(\ln\left( \big(1+\frac{1}{x}\big)^x\right) \right)
	=  \exp\left( \lim_{x\to \infty}  \ln\left( \big(1+\frac{1}{x}\big)^x\right) \right)=\exp(1)=e.\]	
\tab{\glqq $0^0$\grqq{}}
\lang{de}{
Würde als Grenzwert ein Ausdruck "$0^0$" entstehen, verfährt man wie bei "$1^\infty$", d.h. man 
berechnet zunächst den Logarithmus des Terms und dessen Grenzwert mittels der Regel von de l'Hospital 
und wendet anschließend wieder die Exponentialfunktion an.
\\\\
Im Beispiel $\lim_{x\searrow 0} x^x$ berechnen wir zunächst
}
\lang{en}{
If a limit yields an expression '$0^0$', we proceed as with '$1^\infty$'. That is, we first calculate 
the logarithm of the expression and then its limit using l'Hopital's rule, then apply the exponential 
function once again.
\\\\
In the example $\lim_{x\searrow 0} x^x$, we start by finding
}
\[ \ln(x^x)=x\cdot \ln(x)= \frac{\ln(x)}{1/x} \]
\lang{de}{und}
\lang{en}{and}
\[ \lim_{x\searrow 0} \frac{\ln(x)}{1/x} = \lim_{x\searrow 0} \frac{1/x}{-1/x^2}
= \lim_{x\searrow 0} (-x) =0.\]
\lang{de}{Daher ist}
\lang{en}{Thus}
\[ \lim_{x\searrow 0} x^x =\lim_{x\searrow 0} \exp( \frac{\ln(x)}{1/x})=\exp\left( \lim_{x\searrow 0} \frac{\ln(x)}{1/x}\right)=\exp(0)=1.\]
\tab{\glqq $\infty^0$\grqq{}}
\lang{de}{
Würde als Grenzwert ein Ausdruck "$\infty^0$" entstehen, verfährt man wie bei "$1^\infty$", d.h.
man berechnet zunächst den Logarithmus des Terms und dessen Grenzwert mittels der Regel von de 
l'Hospital und wendet anschließend wieder die Exponentialfunktion an.
\\\\
Im Beispiel $\lim_{x\to \infty} x^{1/x}$ berechnen wir zunächst
}
\lang{en}{
If a limit yields an expression '$\infty^0$', we proceed as with '$1^\infty$'. That is, we first 
calculate the logarithm of the expression and then its limit using l'Hopital's rule, then apply the 
exponential function once again.
\\\\
In the example $\lim_{x\to \infty} x^{1/x}$, we start by finding
}
\[ \ln(x^{1/x})=\frac{1}{x}\cdot \ln(x)=\frac{\ln(x)}{x} \]
\lang{de}{und}
\lang{en}{and}
\[ \lim_{x\to \infty} \frac{\ln(x)}{x}= \lim_{x\to \infty} \frac{1/x}{1}=0.\]
\lang{de}{Daher ist, und hier benutzen wir wieder die Stetigkeit der Exponentialfunktion,}
\lang{en}{Hence, using once more the continuity of the exponential function,}
\[ \lim_{x\to \infty} x^{1/x}=\lim_{x\to \infty}\exp(\frac{\ln(x)}{x} )=\exp\left( \lim_{x\to \infty} \frac{\ln(x)}{x}\right)=\exp(0)=1.\]
\end{tabs*}
\end{example}
\lang{de}{
Ergänzende Zusammenfassungen der bisher angesprochenen Themen aus Abschnitt \ref{sec:lHospital}, samt Rechenbeispielen, können den folgenden Videos entnommen werden:\\
\floatright{
    \href{https://api.stream24.net/vod/getVideo.php?id=10962-2-10765&mode=iframe&speed=true}{\image[75]{00_video_button_schwarz-blau}}
    \href{https://www.hm-kompakt.de/video?watch=531}{\image[75]{00_Videobutton_schwarz}}
}\\\\
}
\lang{en}{}




\section{\lang{de}{Zusammenhang zur Taylor-Approximation}
         \lang{en}{Relationship with Taylor approximations}}\label{sec:lhopital-taylor}
\lang{de}{
Weshalb finden Sie diesen Abschnitt über die Regeln von de l'Hospital im Kapitel Approximation?
Wenn wir den Quotienten $\frac{f(x)}{g(x)}$ im Grenzwertprozess ersetzen durch den Quotienten 
$\frac{f'(x)}{g'(x)}$, dann approximieren wir die Funktion selbst durch ihre Ableitung. Das können 
wir besonders gut dann beobachten, wenn die Funktionen $f$ und $g$ im Punkt $x_0$ stetig 
differenzierbar sind, denn dann gibt uns die 
\ref[content_04_taylor_polynom][Taylor-Formel]{thm:taylor-approximation}
}
\lang{en}{
Why is this section on l'Hopital's rules in the chapter on approximations? In replacing the quotient 
$\frac{f(x)}{g(x)}$ with the quotient $\frac{f'(x)}{g'(x)}$ in the limit, we are approximating the 
functions themselves by their derivatives. This can be seen in particular if the functions $f$ and 
$g$ are continuously differentiable at the point $x_0$, as then there exists a 
\ref[content_04_taylor_polynom][Taylor polynomial]{thm:taylor-approximation}
}
\begin{align*}
f(x)&=& f(x_0)+f'(x_0)(x-x_0)+(x-x_0)\, r_f(x),\\
g(x)&=& g(x_0)+g'(x_0)(x-x_0)+(x-x_0)\,r_g(x),
\end{align*}
\lang{de}{
wobei die Restfunktionen erster Ordnung $r_f$ und $r_g$ stetig sind in $x_0$ mit 
$\lim_{x\to x_0}r_f(x)=0=\lim_{x\to x_0}r_g(x)$. Jetzt nehmen wir zusätzlich zur stetigen 
Differenzierbarkeit von $f$ und $g$ in $x_0$ noch die Voraussetzung von Satz 
\ref{thm:de-l-Hospital_1} an. Das heißt, wir nehmen an $f(x_0)=0=g(x_0)$ und $g'(x_0)\neq 0$. 
Dann gilt
}
\lang{en}{
where the first order remainder functions $r_f$ and $r_g$ are continuous at $x_0$ with 
$\lim_{x\to x_0}r_f(x)=0=\lim_{x\to x_0}r_g(x)$. Now if we additionally suppose the conditions of 
Theorem \ref{thm:de-l-Hospital_1} hold, that is, $f(x_0)=0=g(x_0)$ and $g'(x_0)\neq 0$, we have
}
\begin{align*}
\lim_{x\to x_0}\frac{f(x)}{g(x)}&=&\lim_{x\to x_0}\frac{f'(x_0)(x-x_0)+( x-x_0)\cdot r_f(x)}{g'(x_0)(x-x_0)+(x-x_0)\cdot r_g(x)}\\
&=&\lim_{x\to x_0}\frac{f'(x_0)}{g'(x_0)+r_g(x)}
+\lim_{x\to x_0}\frac{ r_f(x)}{g'(x_0)+ r_g(x)}\\
&=&\frac{f'(x_0)}{g'(x_0)}=\lim_{x\to x_0}\frac{f'(x)}{g'(x)}.
\end{align*}
\lang{de}{
Damit haben wir die erste de l'Hospitalsche Regel (Satz \ref{thm:de-l-Hospital_1}) gezeigt mit Hilfe 
der Taylor-Approximation unter der Voraussetzung, dass $f$ und $g$ stetig differenzierbare Funktionen 
sind. Die erste de l'Hospitalsche Regel fordert statt der Annahme der stetigen Differenzierbarkeit 
ein klein wenig schwächere Voraussetzungen, unter der die Regel trotzdem noch anwendbar ist.
\\\\
Wieso haben wir die de l'Hospitalsche Regel nicht gleich  durch die Taylor-Approximation bewiesen? 
Das liegt daran, dass wir bereits für den \ref[content_04_taylor_polynom][Beweis der Taylor-Approximationseigenschaft]{proof:Beweis der Taylor-Approximationseigenschaft} 
die de l'Hospitalsche Regel verwendet haben. Ohne einen anderen Beweis der de l'Hospitalsche Regel 
hätten wir einen mathematischen Zirkelschluss getan: Die eine Aussage ist genau dann wahr, wenn die 
andere wahr ist. Aber ob eine der Aussage wirklich wahr ist, hätten wir nicht sagen können.
\\\\
Der de l'Hospitalschen Regel liegt also dieselbe \emph{Idee} zugrunde wie der Taylor-Approximation. 
Um beide zu beweisen, müssen wir an einer Stelle härter arbeiten, nämlich den 
\ref[content_06_de_l_hospital][verallgemeinerten Mittelwertsatz]{thm:verallg_mws} beweisen.
Wir haben uns dafür entschieden, das bei der de l'Hospitalschen Regel zu tun.
\\\\
Auch das $n$-fache Anwenden der Regel von de l'Hospital können wir durch Taylor-Approximation ganz 
leicht verstehen. Dazu nehmen wir an, dass $f$ und $g$ sogar $n$-mal stetig differenzierbar sind in 
$x_0$ und dass alle Ableitungen bis zur $(n-1)$-ten verschwinden, nicht aber die $n$-te von $g$, 
also $f^{(k)}(x_0)=0=g^{(k)}(x_0)$ für $k=0,\ldots,n-1$ und $g^{(n)}(x_0)\neq 0$. Wir machen somit 
genau die Annahme, die wir beim $n$-fachen Anwenden von de l'Hospital benötigen. Dann gilt mit Hilfe 
der $n$-ten Restglieder $r_f$ und $r_g$ von $f$ bzw. $g$, die ja stetig sind in $x_0$ und dort 
verschwinden,
}
\lang{en}{
Thus we have shown the validity of the first of l'Hopital's rules (Theorem \ref{thm:de-l-Hospital_1}) 
using a Taylor approximation, under the condition that $f$ and $g$ are continuously differentiable 
functions. The first of l'Hopital's rule has weaker conditions than this, under which it is still 
valid.
\\\\
Why did we not initially use Taylor approximations to prove the rule? In fact, we used l'Hopital's 
rule in the \ref[content_04_taylor_polynom][proof of uniqueness of Taylor approximations]{proof:Beweis der Taylor-Approximationseigenschaft}. If we did not provide a different proof of 
l'Hopital's rule, we would have a circular proof: the statement is true if and only if the other 
statement is true. However, this does not tell us whether either theorem is true by itself.
\\\\
L'Hopital's rules therefore rely on the same \emph{idea} as Taylor approximations. In order to prove 
both, we must do some more work on one end. We choose to prove the 
\ref[content_06_de_l_hospital][general mean value theorem]{thm:verallg_mws}, 
which is then used in the proof of l'Hopital's rule earlier.
\\\\
Even the repeated application of l'Hopital's rule can be understood using Taylor approximations. 
Suppose that $f$ and $g$ are $n$-times continuously differentiable at $x_0$, and that all derivatives 
up to the $(n-1)$th vanish there, but not the $n$th derivative of $g$. That is, 
$f^{(k)}(x_0)=0=g^{(k)}(x_0)$ for $k=0,\ldots,n-1$ and $g^{(n)}(x_0)\neq 0$. We have required exactly 
the conditions that are required for $n$ consecutive applications of l'Hopital's rule. Then, using 
the $n$th order remainders $r_f$ and $r_g$ of $f$ and $g$, which are continuous and vanish at $x_0$,
}
\begin{align*}
f(x)&=& f^{(n)}(x_0)(x-x_0)^n+(x-x_0)^n\cdot r_f(x),\\
g(x)&=& g^{(n)}(x_0)(x-x_0)^n+(x-x_0)^n\cdot r_g(x).
\end{align*}
\lang{de}{Genau wie im Fall $n=1$ erhalten wir daraus}
\lang{en}{Like in the case $n=1$, we obtain}
\[\lim_{x\to x_0}\frac{f(x)}{g(x)}=\frac{f^{(n)}(x_0)}{g^{(n)}(x_0)}=\lim_{x\to x_0}\frac{f^{(n)}(x)}{g^{(n)}(x)}.\]

\section{\lang{de}{Verallgemeinerter Mittelwertsatz}
         \lang{en}{General mean value theorem}}\label{sec:verallg-mws}

\lang{de}{
Im Beweis der Regel von de l'Hospital haben wir den verallgemeinerten Mittelwertsatz verwendet, den 
wir jetzt nachholen.
}
\lang{en}{
The proof of l'Hopital's rule used the general mean value theorem, which we now provide a proof of.
}

\begin{theorem}[\lang{de}{Verallgemeinerter Mittelwertsatz}
                \lang{en}{General mean value theorem}]\label{thm:verallg_mws}
  \lang{de}{
  Sei $I\subseteq \R$ ein Intervall und seien $f,g\colon I\to\R$ stetige Funktionen, die im Innern 
  von $I$ differenzierbar sind. Die Ableitung erf\"ulle 
  \nowrap{$g'(x)\neq 0$} f\"ur alle $x$ aus dem Inneren von $ I$.
  \\\\
  F\"ur alle $a,b\in I,\; a<b,$ gibt es ein $\xi\in I$ zwischen $a$ und $b,$ so dass
  }
  \lang{en}{
  Let $I\subseteq \R$ be an interval and $f,g\colon I\to\R$ differentiable. Let the derivative 
  satisfy \nowrap{$g'(x)\neq 0$} for all $x\in I.$
  \\\\
  For all $a,b\in I,\; a< b,$ there is a $\xi\in I$ between $a$ and $b$ such that
  }
  \[ \frac{f(b)-f(a)}{g(b) - g(a)} = \frac{f'(\xi)}{g'(\xi)}\,.\]
\end{theorem}

\begin{remark}
  \begin{enumerate}
  \item
    \lang{de}{"`$\xi\in I$ zwischen  $a$ und $b$"' $\,$ bedeutet $a<\xi<b$.}
    \lang{en}{'$\xi\in I$ between $a$ and $b$' $\,$  means $a<\xi<b.$}
  \item
    \lang{de}{Aus $g'(x)\neq 0$ f\"ur alle $x\in I$ folgt $g(b) - g(a)\neq 0$ f\"ur alle $a\neq b.$}
    \lang{en}{'$g'(x)\neq 0$ for all $x\in I$' implies $g(b) - g(a)\neq 0$ for all $a\neq b.$}
  \item
    \lang{de}{Mit $g(x)=x$ erh\"alt man den  \ref[hoehere-abl][gew\"ohnlichen Mittelwertsatz]{thm:mittelwertsatz}.}
    \lang{en}{Setting $g(x)=x$ yields the usual \ref[hoehere-abl][mean value theorem]{thm:mittelwertsatz}. }
\end{enumerate}
\end{remark}
\begin{proof*}[\lang{de}{Beweis von Satz \ref{thm:verallg_mws}}
               \lang{en}{Proof of Theorem \ref{thm:verallg_mws}}]
\lang{de}{
Mit einer geeigneten Hilfsfunktion führen wir den verallgemeinerten Mittelwertsatz auf den 
\ref[hoehere-abl][gew\"ohnlichen Mittelwertsatz]{thm:mittelwertsatz} zurück.
}
\lang{en}{
With a suitable helping function, we can use the usual 
\ref[hoehere-abl][mean value theorem]{thm:mittelwertsatz} to prove the general mean value 
theorem.
}
\begin{incremental}[\initialsteps{0}]
\step
\lang{de}{Die Funktion $h:I\to\R$,}
\lang{en}{The function $h:I\to\R$,}
\[x\mapsto h(x) = [f(b)-f(a)\,]\,g(x) -[g(b)-g(a)\,]\,f(x)\]
\lang{de}{ist differenzierbar und erfüllt}
\lang{en}{is differentiable and satisfies}
\[h(a)=f(b)g(a)-g(b)f(a)=h(b).\]
\lang{de}{Nach dem Mittelwertsatz existiert eine Stelle $\xi$, $a<\xi<b$, für die gilt}
\lang{en}{By the mean value theorem, there exists a point $\xi$ with $a<\xi<b$ for which}
\[h'(\xi)=\frac{h(b)-h(a)}{b-a}=0.\]
\lang{de}{Das heißt aber}
\lang{en}{However, then}
\[ 0=h'(\xi) = [f(b)-f(a)\,]\,g'(\xi) -[g(b)-g(a)\,]\,f'(\xi).     \]
\lang{de}{
Umstellen dieser Gleichung liefert die Behauptung. Dabei bemerken wir, dass wir nicht nur durch 
$g'(\xi)$ sondern auch durch $(g(b)-g(a))$ teilen dürfen, weil nach Voraussetzung $g'$ keine 
Nullstelle hat. Somit ist die Funktion $g$ streng monoton, also $g(b)\neq g(a)$.
}
\lang{en}{
Manipulating this equation yields the assumption. We note that not only can we divide by $g'(\xi)$, 
but we can also divide by $(g(b)-g(a))$, because by our condition, $g'$ is non-zero at every point. 
Hence the function $g$ is strictly monotone, and $g(b)\neq g(a)$.
}
\end{incremental}

\end{proof*}


\end{content}