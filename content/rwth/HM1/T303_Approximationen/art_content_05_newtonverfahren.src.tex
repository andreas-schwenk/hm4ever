%$Id:  $
\documentclass{mumie.article}
%$Id$
\begin{metainfo}
  \name{
    \lang{de}{Newton-Verfahren}
    \lang{en}{Newton's method}
  }
  \begin{description} 
 This work is licensed under the Creative Commons License Attribution 4.0 International (CC-BY 4.0)   
 https://creativecommons.org/licenses/by/4.0/legalcode 

    \lang{de}{Beschreibung}
    \lang{en}{Description}
  \end{description}
  \begin{components}
    \component{generic_image}{content/rwth/HM1/images/g_tkz_T303_Newton.meta.xml}{T303_Newton}
    \component{generic_image}{content/rwth/HM1/images/g_img_00_Videobutton_schwarz.meta.xml}{00_Videobutton_schwarz}
    \component{generic_image}{content/rwth/HM1/images/g_img_00_video_button_schwarz-blau.meta.xml}{00_video_button_schwarz-blau}
    \component{js_lib}{system/media/mathlets/GWTGenericVisualization.meta.xml}{mathlet1}
    \component{generic_image}{content/rwth/HM1/images/g_img_hm-newton-verfahren.meta.xml}{newton-verfahren}
  \end{components}
  \begin{links}
    \link{generic_article}{content/rwth/HM1/T301_Differenzierbarkeit/g_art_content_01_differenzenquotient.meta.xml}{diffbarkeit}
    \link{generic_article}{content/rwth/HM1/T301_Differenzierbarkeit/g_art_content_02_ableitungsregeln.meta.xml}{abl-reg}
    \link{generic_article}{content/rwth/HM1/T303_Approximationen/g_art_content_04_taylor_polynom.meta.xml}{taylor-poly}
  \end{links}
  \creategeneric
\end{metainfo}
\begin{content}
\usepackage{mumie.ombplus}
\ombchapter{2}
\ombarticle{2}
\usepackage{mumie.genericvisualization}

\begin{visualizationwrapper}

\title{\lang{de}{Newton-Verfahren}\lang{en}{Newton's Method}}
 
\begin{block}[annotation]
  
\end{block}
\begin{block}[annotation]
  Im Ticket-System: \href{http://team.mumie.net/issues/10034}{Ticket 10034}\\
\end{block}

\begin{block}[info-box]
\tableofcontents
\end{block}


\section{\lang{de}{Newton-Verfahren}\lang{en}{Newton's method}}\label{newton}
\lang{de}{
\textbf{Motivation:} Viele praktische Probleme in den Ingenieur- und Naturwissenschaften lassen sich nicht 
analytisch exakt lösen. Sind Näherungswerte (Approximationen) für eine gesuchte Lösung ausreichend,
können sogenannte numerische Verfahren verwendet werden. Numerische Verfahren sind als gezielte 
Rate-Algorithmen zu verstehen, die zumeist einen wiederholenden (iterativen) Charakter besitzen. 
Mit zunehmender Anzahl an Iterationen sollte die numerische Lösung gegen die analytische Lösung 
laufen (konvergieren).\\
Im Rahmen dieses Abschnitts wollen wir uns mit der numerischen Nullstellenberechnung $f(x)=0$ beschäftigen.
Im Speziellen beschränken wir uns auf differenzierbare Funktionen und wollen deren Nullstellen mit dem
sogenannten Newton-Verfahren lösen.
}
\lang{en}{
\textbf{Motivation:} Many practical problems in engineering and science cannot be solved analytically 
and do not need exact solutions. If approximations are sufficient for such an application, we may 
apply so-called numerical methods. Numerical methods are algorithms, often iterative, that give 
approximate solutions to a problem which converge on the analytical solutions by iteration. \\
In this section we consider the equation $f(x)=0$. In particular, we consider the case where $f$ is 
differentiable, and use the so-called Newton's method to find the roots of $f$.
}
\begin{center}
    \image{T303_Newton}
\end{center}
\lang{de}{
\textbf{Idee:} Die grundlegende Idee des Newton-Verfahrens lässt sich mit Hilfe der oben dargestellten Abbildung wie folgt zusammenfassen:\\
}
\lang{en}{
\textbf{Idea:} The underlying idea of Newton's method can be understood using the above diagram: \\
}
\begin{enumerate}
  \item \lang{de}{
        Wähle einen Startwert $x_0$ in der Nähe der gesuchten Nullstelle $x^\ast$ (Schätzung anhand 
        einer Skizze der Funktion $f$).
        }
        \lang{en}{
        Choose a starting point $x_0$ somewhere near the root $x^\ast$ that we wish to find (we may 
        choose this by sketching $f$).
        }
  \item \lang{de}{
        Lege die Tangente $T_{x_0}(x)$ im Punkt $P_0=(x_0,f(x_0))$ an $f$.
        }
        \lang{en}{
        Consider the tangent $T_{x_0}(x)$ of $f$ at the point $P_0=(x_0,f(x_0))$.
        }
  \item \lang{de}{
        Bestimme die Nullstelle $x_1$ der Tangente $T_{x_0}(x)$.
        }
        \lang{en}{
        Determine the root $x_1$ of the tangent $T_{x_0}(x)$ (its intersection with the x-axis).
        }
  \item \lang{de}{
        Wiederhole die Schritte 2 und 3, ausgehend von der Nullstelle $x_1$ als neuer Berührstelle 
        der Tangente $T_{x_1}(x)$.
        }
        \lang{en}{
        Repeat steps 2 and 3, starting with the approximation $x_1$ and considering the tangent 
        $T_{x_1}(x)$.
        }
  \item \lang{de}{
        Ausgehend von der jeweils zuletzt ermittelten Nullstelle $x_n$ wiederhole man die Schritte 2 
        und 3 solange, bis ein Abbruchkriterium erfüllt ist. Mögliche Abbruchkriterien sind:
        }
        \lang{en}{
        Repeat steps 2 and 3 from the last determined approximation $x_n$ until a chosen stopping 
        criterion is fulfilled. Possible stopping criteria are:
        }
  \begin{enumerate}
     \item \lang{de}{
           Der Funktionswert $f(x_n)$ weicht betragsmäßig nicht mehr als ein vorgegebenes 
           $\epsilon_1$ von $f(x_n)=0$ ab, so dass $|f(x_n)|<\epsilon_1$ gilt.
           }
           \lang{en}{
           The difference between $f(x_n)=0$ and the value $f(x_n)$ smaller than a pre-chosen 
           $\epsilon_1$, that is, $|f(x_n)|<\epsilon_1$ gilt.
           }
     \item \lang{de}{
           Der Abstand zweier aufeinanderfolgender Näherungswerte $x_n$ und $x_{n+1}$ wird 
           unterschritten: $|x_{n+1}-x_n|<\epsilon_2$.
           }
           \lang{en}{
           The difference between two consecutive approximations $x_n$ und $x_{n+1}$ is sufficiently 
           small: $|x_{n+1}-x_n|<\epsilon_2$.
           }
  \end{enumerate}
\end{enumerate}
\lang{de}{
Aufbauend auf dieser Idee lässt sich eine Berechnungsvorschrift für das 
Newton-Verfahren entwicklen. Für einen gewählten Startpunkt $x_0$ berechnet sich die Tangente zu
}
\lang{en}{
Building on this idea, we can be more precise in our rules. For a chosen starting point $x_0$, the 
tangent can be found by
}
\[ 
    T_{x_0}(x)=f(x_0)+f'(x_0)(x-x_0). 
\]
\lang{de}{
Die Nullstelle dieser Tangente wird als erste Näherungslösung des ursprünglichen Problems aufgefasst
}
\lang{en}{
The root of this tangent, and our first approximation to the root of the function $f$, is hence
}
\begin{align*}
   & T_{x0}(x)&=0\\
   \Leftrightarrow\quad & f(x_0)+f'(x_0)(x-x_0)&=0\\
  \Leftrightarrow\quad &  f'(x_0)(x-x_0) &=-f(x_0) \\
  \Leftrightarrow\quad & x-x_0  &=-\frac{f(x_0)}{f'(x_0)} \\
  \Leftrightarrow\quad & x  &= x_0-\frac{f(x_0)}{f'(x_0)}:=x_1.
\end{align*}
\lang{de}{
Mit Hilfe der Nullstellen-Approximation $x_1$ kann die nächste Iteration durchgeführt werden.
Die allgemeine Berechnungsvorschrift für eine beliebige Iteration des Newton-Verfahrens lautet somit
}
\lang{en}{
Using the root approximation $x_1$, we can reiterate these steps. A general iteration of Newton's 
method can be written
}
\begin{align*}
   x_{n+1}  &= x_n-\frac{f(x_n)}{f'(x_n)}.
\end{align*}

\begin{theorem}[\lang{de}{Newton-Verfahren}\lang{en}{Newton's method}]\label{theo:newton-verfahren}
\lang{de}{
Sei $f:I\rightarrow\R$ eine auf einem Intervall $I$ (nicht konstante) differenzierbare Funktion, die 
eine Nullstelle $x^*\in I$ hat.\\
Für einen Startwert $x_0$, der \textit{nahe genug} bei $x^*$ liegt, konvergiert die Folge 
$(x_n)_{n\in \N}$ mit
}
\lang{en}{
Let $f:I\rightarrow\R$ be a (non-constant) differentiable function on an interval $I$, that has a 
root $x^*\in I$.\\
For a starting point $x_0$, which lies \textit{lies sufficiently close to} to $x^*$, the sequence 
$(x_n)_{n\in \N}$ where
}
\[ x_{n+1}=x_n-\frac{f(x_n)}{f'(x_n)} \]
\lang{de}{gegen die Nullstelle $x^*$.}
\lang{en}{converges towards the root $x^*$.}\\
\end{theorem}

%Eine alternative Erklärung des \lref{newton}{Newton-Verfahrens} ist folgendem Video zu entnehmen:\\


\begin{remark}
\begin{enumerate}
\item \lang{de}{
    Es gibt viele (komplizierte) hinreichende Bedingungen, die die
    Konvergenz des Newton-Verfahrens sicherstellen. Eine M\"oglichkeit, die
    Bedingung "`$x_0$ liegt nahe bei $x^*$"' zu pr\"azisieren, ist im \lref{sec:konv-geschw}{folgenden
    Paragraphen} gegeben.
    \\\\
    In der Praxis probiert man oft den Algorithmus aus und sieht sich das Resultat an.
    }
    \lang{en}{
    There exist many (complicated) sufficient conditions which guarantee the convergence of Newton's 
    method. A possible way to make the condition '$x_0$ lies sufficiently close to $x^*$' can be made 
    precise \lref{sec:konv-geschw}{as described in the following paragraphs}.
    \\\\
    In practice, we may simply try the algorithm several times until a good starting point is found.
    }
\item \lang{de}{
    Wenn $f$ mehr als eine Nullstelle hat, dann hängt es meist von $x_0$ ab, gegen welche Nullstelle 
    die Iteration strebt.
    }
    \lang{en}{
    If $f$ has more than one root, then it is mostly the choice of $x_0$ that determines which root 
    the algorithm will converge to.
    }
\end{enumerate}
\end{remark}


\begin{example}
\begin{tabs*}[\initialtab{0}]
\tab{$f(x)=x-e^{-x}$}
 	\begin{genericGWTVisualization}[550][1000]{mathlet1}
 		% \title{$f(x)=x-e^{-x}$}
 		\begin{variables}
			\pointOnCurve{st}{real}{0}{-1}  % verschiebbarer Punkt auf x-Achse, Anfangswert (-1,0)
 			\number{x0}{real}{var(st)[x]}
			\function{f}{real}{x-exp(-x)}
			\number{y0}{real}{var(x0)-exp(-var(x0))}
			\number{m}{real}{1+exp(-var(x0))}
			\number{q0}{real}{var(y0)-var(m)*var(x0)}
			\point{P}{real}{var(x0),var(y0)}
			\point{R}{real}{var(x0)+1,var(y0)+var(m)}
			\line{T}{real}{var(P),var(R)}
			\point{P2}{real}{var(x0)-var(y0)/var(m) ,0}
			\segment{v}{real}{var(st),var(P)}
		\end{variables}
 		\color{P}{#0066CC}
		\color{T}{#0066CC}
		\color{st}{#CC6600}
		\color{P2}{#00CC00}
		\color{v}{#CC6600}
		\begin{canvas}
			\plotSize{400}
			\plotLeft{-2}
			\plotRight{6}
			\plot[coordinateSystem]{f,T,v, P, st, P2}
		\end{canvas}
		
		\lang{de}{\text{
    Hier können Sie sich für das folgende Beispiel $f=\var{f}$ den Funktionsgraphen und zu einem 
		\textcolor{#CC6600}{Näherungswert} (aktuell $\var{st}[x]$) der Nullstelle jeweils den nächsten 
    \textcolor{#00CC00}{Näherungswert} (aktuell $\var{P2}[x]$) anzeigen lassen.
    }}
    \lang{en}{\text{
    Here is an opportunity to visualise Newton's method with the example $f=\var{f}$, using 
    \textcolor{#CC6600}{approximation} (currently $\var{st}[x]$) as a starting point, used to find 
    the next \textcolor{#00CC00}{approximation} (currently $\var{P2}[x]$).
    }}
 		
\end{genericGWTVisualization}

\lang{de}{
Wir betrachten die Funktion $f(x)=x-e^{-x}$.
Wegen $f'(x)=1+e^{-x}>0$ für alle $x\in \R$ ist sie streng monoton wachsend. Da außerdem
}
\lang{en}{
We consider the function $f(x)=x-e^{-x}$.
As $f'(x)=1+e^{-x}>0$ for all $x\in \R$, it is strictly monotone increasing. As 
}
\[ \lim_{x\to -\infty} f(x)=  \lim_{x\to -\infty} (x-e^{-x}) =-\infty \]
\lang{de}{und}
\lang{en}{and}
\[ \lim_{x\to \infty} f(x)=  \lim_{x\to \infty} (x-e^{-x}) =\infty, \]
\lang{de}{
besitzt $f$ genau eine Nullstelle. Diese Nullstelle wollen wir mit dem Newton-Verfahren 
näherungsweise berechnen. Die Rekursionsformel lautet hier
}
\lang{en}{
the function $f$ has exactly one root. We want to approximate this root using Newton's method. The 
recursion formula in this case is
}
\begin{eqnarray*}
x_{n+1} &=& x_n-\frac{f(x_n)}{f'(x_n)} = x_n-\frac{x_n-e^{-x_n}}{1+e^{-x_n}} \\
&=& \frac{x_n(1+e^{-x_n})-x_n+e^{-x_n}}{1+e^{-x_n}} =\frac{(x_n+1)e^{-x_n}}{1+e^{-x_n}} \\
&=& \frac{x_n+1}{e^{x_n}+1}.
\end{eqnarray*} 

\lang{de}{Starten wir mit $x_0=0$, so folgen}
\lang{en}{We begin with $x_0=0$. Then}
\begin{align*}
x_1 &= \frac{x_0+1}{e^{x_0}+1}&=\frac{0+1}{1+1}&=\frac{1}{2}=0,5 \, , \\
x_2 &= \frac{x_1+1}{e^{x_1}+1}=\frac{\frac{1}{2}+1}{e^{\frac{1}{2}}+1}&=\frac{3}{2\sqrt{e}+2}& \approx 0,5663110 \, , \\
x_3 &= \frac{x_2+1}{e^{x_2}+1} &=  \frac{\frac{3}{2\sqrt{e}+2}+1}{e^{\frac{3}{2\sqrt{e}+2}}+1}& \approx 0,5671431 \, , \\
x_4 &=  \frac{x_3+1}{e^{x_3}+1}& \qquad &\approx 0,5671432
\end{align*}
\lang{de}{
Das vierte Folgeglied unterscheidet sich also vom dritten nur noch in der siebten Nachkommastelle.
}
\lang{en}{
The fourth approximation only differs from the third on the seventh digit after the decimal point.
}

\tab{$f(x)=x^3-3x^2+1$}
 	\begin{genericGWTVisualization}[550][1000]{mathlet1}
 		%\title{$f(x)=x^3-3x^2+1$}
 			\begin{variables}
			\pointOnCurve{st}{real}{0}{-1}
 			\number{x0}{real}{var(st)[x]}
			\function{f}{real}{x^3-3*x^2+1}
			\number{y0}{real}{var(x0)^3-3*var(x0)^2+1}
			\number{m}{real}{3*var(x0)^2-6*var(x0)}
			\number{q0}{real}{var(y0)-var(m)*var(x0)}
			\point{P}{real}{var(x0),var(y0)}
			\point{R}{real}{var(x0)+1,var(y0)+var(m)}
			\line{T}{real}{var(P),var(R)}
			\point{P2}{real}{var(x0)-var(y0)/var(m) ,0}
			\segment{v}{real}{var(st),var(P)}
		\end{variables}
 		\color{P}{#0066CC}
		\color{T}{#0066CC}
		\color{st}{#CC6600}
		\color{P2}{#00CC00}
		\color{v}{#CC6600}
		\begin{canvas}
			\plotSize{400}
			\plotLeft{-2}
			\plotRight{6}
			\plot[coordinateSystem]{f,T,v, P, st, P2}
		\end{canvas}
		\lang{de}{\text{
    Hier können Sie für die Funktion $f(x)=\var{f}$ zu einer Stelle die nächste Stelle im 
    Newton-Verfahren berechnen lassen. Den orangenen Punkt auf der $x$-Achse können Sie verschieben.\\
		Für $x_0=\var{st}[x]$ ist die nächste Stelle $x_1=\var{P2}[x]$ (siehe grüner Punkt).
    }}
    \lang{en}{\text{
    For the function $f(x)=\var{f}$ we can calculate an approximation from the previous one using 
    Newton's method. The orange point on the $x$-axis can be moved.\\
    If $x_0=\var{st}[x]$, the next approximation is $x_1=\var{P2}[x]$ (the green point).
    }}
\end{genericGWTVisualization}

\lang{de}{
Die Polynomfunktion $f(x)=x^3-3x^2+1$ hat $3$ verschiedene Nullstellen, wie man anhand der Grafik sieht.
\\\\
Anschaulich erkennt man auch Folgendes:\\
Wird das Newton-Verfahren mit einem Anfangswert $x_0$ gestartet, welcher kleiner als die kleinste 
Nullstelle $x^*_{min}$ ist, so ist sichergestellt, dass die numerische Lösung gegen die exakte 
Nullstelle konvergiert. Eine mögliche Erklärung liefert der Funktionsverlauf. Wie zu erkennen ist, 
weist die Funktion im betrachteten Bereich für alle Anfangswerte $x_0<x^*_{min}$ eine Rechtskurve 
auf. Diese Rechtskurve stellt sicher, dass erstens alle Anfangswerte gegen die exakte Lösung gedrückt 
werden und zweitens, dass alle Folgewerte kleiner-gleich der exakten Lösung sind. In Summe ist also 
sichergestellt, dass das Newton-Verfahren konvergiert.
\\\\
Bei einem Anfangswert $x_0$ zwischen der kleinsten Nullstelle und dem lokalen Maximum liegt der 
zweite Wert $x_1$ wieder links von der kleinsten Nullstelle. Auch in diesem Fall nähert man sich dann 
im Weiteren der kleinsten Nullstelle an.
\\\\
Entsprechend konvergiert das Newton-Verfahren gegen die größte Nullstelle, wenn man einen Anfangswert 
rechts des lokalen Minimums wählt.
\\\\
Um eine Iteration zu erhalten, die gegen die mittlere Nullstelle konvergiert, muss man daher schon 
einen Wert wählen, der zwischen den beiden Extrema liegt. Jedoch nicht zu nah an den Extrema, da der 
Wert sonst nach dem ersten Schritt wieder zu klein bzw. zu groß wird.
}
\lang{en}{
The polynomials $f(x)=x^3-3x^2+1$ has $3$ distinct roots, which can be seen on the graph.
\\\\
Visually we also recognise the following:\\
If we perform Newton's method with a starting point of $x_0$ that is smaller than the smallest root 
$x^*_{min}$, then the numerical approximations converge on the root. A possible explanation is given 
by the shape of the graph. We can see that the function is 'concave', or has a decreasing derivative, 
for all starting points $x_0<x^*_{min}$. This ensures that Newton's method will converge to the same 
root for every starting point, and that each approximation is smaller than or equal to the root 
itself.
\\\\
If we choose a starting point $x_0$ between the smallest root and the local maximum, the second 
approximation $x_1$ will lie to the left of the smallest root. In this case, as above, the 
approximations converge to the smallest root.
\\\\
Correspondingly, Newton's method converges to the largest root if we choose a starting point to the 
right of the local minimum.
\\\\
For the approximations to converge to the middle root of the function, we must choose a starting 
point that lies between the two extrema. Not only this, but we cannot choose a point too close to 
either extrema, lest the second approximation be once more too large or too small.
}
\tab{$f(x)=\arctan(x)$}
\lang{de}{
Bei diesem Beispiel kennen wir natürlich die einzige Nullstelle, nämlich $x^*=0$ schon. Dieses 
Beispiel soll aber illustrieren, dass es durchaus auch Situationen gibt, in denen ein schlechter 
Anfangswert eine nicht konvergente Folge liefert.
}
\lang{en}{
We already know that the unique root for this example is $x^*=0$. This example is intended to 
illustrate the situations in which a bad choice of starting point leads to a non-convergent 
sequence of approximations.
}

	\begin{genericGWTVisualization}[550][1000]{mathlet1}
 		% \title{$f(x)=\arctan(x)$}
 		\begin{variables}
			\pointOnCurve{st}{real}{0}{2}  % verschiebbarer Punkt auf x-Achse, Anfangswert (-1,0)
 			\number{x0}{real}{var(st)[x]}
			\function{f}{real}{arctan(x)}
			\number{y0}{real}{arctan(var(x0))}
			\number{m}{real}{1/(1+var(x0)^2)}
			\number{q0}{real}{var(y0)-var(m)*var(x0)}
			\point{P}{real}{var(x0),var(y0)}
			\point{R}{real}{var(x0)+1,var(y0)+var(m)}
			\line{T}{real}{var(P),var(R)}
			\point{P2}{real}{var(x0)-var(y0)/var(m) ,0}
			\segment{v}{real}{var(st),var(P)}
		\end{variables}
 		\color{P}{#0066CC}
		\color{T}{#0066CC}
		\color{st}{#CC6600}
		\color{P2}{#00CC00}
		\color{v}{#CC6600}
		\begin{canvas}
			\plotSize{400}
			\plotLeft{-4}
			\plotRight{4}
			\plot[coordinateSystem]{f,T,v, P, st, P2}
		\end{canvas}
		
		\lang{de}{\text{
    Hier können Sie sich für das folgende Beispiel $f=\var{f}$ den Funktionsgraphen und zu einem 
		\textcolor{#CC6600}{Näherungswert} (aktuell $\var{st}[x]$) der Nullstelle jeweils den nächsten 
    \textcolor{#00CC00}{Näherungswert} (aktuell $\var{P2}[x]$) anzeigen lassen.
    }}
    \lang{en}{\text{
    Here is an opportunity to visualise Newton's method with the example $f=\var{f}$, using 
    \textcolor{#CC6600}{approximation} (currently $\var{st}[x]$) as a starting point, used to find 
    the next \textcolor{#00CC00}{approximation} (currently $\var{P2}[x]$).
    }}
 		
\end{genericGWTVisualization}
\lang{de}{
Die \ref[abl-reg][Ableitung von $f(x)=\arctan(x)$]{ex:abl-umkehrfunktion} ist 
$f'(x)=\frac{1}{1+x^2}$, weshalb die Rekursionsformel im Newton-Verfahren hier
}
\lang{en}{
The \ref[abl-reg][derivative of $f(x)=\arctan(x)$]{ex:abl-umkehrfunktion} is $f'(x)=\frac{1}{1+x^2}$, 
so in this case
}
\[ x_{n+1} = x_n-\frac{f(x_n)}{f'(x_n)} = x_n-\frac{\arctan(x_n)}{(1+x_n^2)^{-1}}=x_n-\arctan(x_n)\cdot (1+x_n^2) \]
\lang{de}{lautet. Ist nun $x_n\geq \sqrt{3}$ und damit $\arctan(x_n)\geq 1$, so gilt}
\lang{en}{is the recursion formula. If $x_n\geq \sqrt{3}$ and hence $\arctan(x_n)\geq 1$, we have}
\[  x_{n+1} = x_n-\arctan(x_n)\cdot (1+x_n^2)\leq x_n- (1+x_n^2)=-x_n - (1-2x_n+x_n^2)=-x_n-(1-x_n)^2<-x_n. \]
\lang{de}{Entsprechend gilt für $x_n\leq -\sqrt{3}$ und damit $\arctan(x_n)\leq -1$ die Ungleichung}
\lang{en}{Similarly, if $x_n\leq -\sqrt{3}$ and hence $\arctan(x_n)\leq -1$, we have}
\[ x_{n+1} = x_n-\arctan(x_n)\cdot (1+x_n^2)\geq x_n+(1+x_n^2)=-x_n+(1+2x_n+x_n^2)=-x_n+(1+x_n)^2>-x_n. \]
\lang{de}{
Insgesamt erhält man also für $|x_n|\geq \sqrt{3}$, dass $|x_{n+1}|>|x_n|$ ist.
\\\\
Startet man also das Newton-Verfahren mit einem Wert $x_0$ mit $|x_0|\geq \sqrt{3}$, so springt die 
Folge $(x_n)_{n\in \N}$ immer von positiv nach negativ (und umgekehrt) und entfernt sich dabei auch 
immer weiter von der Nullstelle $x^*=0$.
}
\lang{en}{
In general, for $|x_n|\geq \sqrt{3}$ we get $|x_{n+1}|>|x_n|$.
\\\\
If we begin Newton's method with a value $x_0$ where $|x_0|\geq \sqrt{3}$, then the sequence 
$(x_n)_{n\in \N}$ switches between positive and negative values, whilst diverging from the root 
$x^*=0$.
}
\end{tabs*}
\end{example}



	\begin{quickcheck}	
	\begin{variables}
		
			\randint[Z]{a}{1}{4}
			
			\function[expand,normalize]{f}{x^2-a*cos(x)}
			\derivative[normalize]{f1}{f}{x}   %2*x+a*sin(x)
			
			\function{g}{x-(f/f1)}
			
 			\function[calculate]{x1}{1}
 			\substitute[calculate]{x2}{g}{x}{x1}
 			\substitute[calculate]{x3}{g}{x}{x2}
% 			\function[calculate]{x2}{x1-(x1^2-a*cos(x1))/(2*x1+a*sin(x1))}
% 			\function[calculate]{x3}{x2-(x2^2-a*cos(x2))/(2*x2+a*sin(x2))}
									
	\end{variables}

		
	%+++++++++++++++++++TYPE+++++++++++++++++++++++++++
		
		\field{real}
		\precision{2}
		\type{input.function} 
		
	    \lang{de}{\text{
      Eine Nullstelle der Funktion $f(x)=\var{f}$ soll mit dem Newton-Verfahren
	    angenähert werden.\\
	    Geben Sie die Rekursionsformel für das Newton-Verfahren an.\\
	    $x_{n+1}= $\ansref   (schreiben sie lediglich $x$ statt $x_n$) \\ \\
	    Berechnen Sie anschließend mit Hilfe des Taschenrechners die ersten Folgenglieder des Verfahrens,
	    wenn als Startwert $x_0=\var{x1}$ gewählt wird (auf drei Stellen hinter dem Komma genau).\\
	    $x_{1}= $\ansref, \\
	    $x_{2}= $\ansref .
	    }}
      \lang{en}{\text{
      We approximate a root of the function $f(x)=\var{f}$ using Newton's method.\\
      Give the recursion formula of Newton's formula.\\
      $x_{n+1}= $\ansref   (simply write $x$ instead of $x_n$). \\ \\
      Determine, using a calculator, the first approximations given by the method, if we use the 
      starting point $x_0=\var{x1}$ (to three decimal places).\\
      $x_{1}= $\ansref, \\
	    $x_{2}= $\ansref .
      }}

	    \begin{answer}
			\solution{g}
			\checkAsFunction[1E-3]{x}{-10}{10}{10} 
	    \end{answer}  
    		
	    \begin{answer}
	    	\solution{x2}
			\checkAsFunction[1E-3]{x}{-10}{10}{10} 
		\end{answer}

		\begin{answer}
			\solution{x3}
			\checkAsFunction[1E-2]{x}{-10}{10}{10} 
		\end{answer}  
    
	\end{quickcheck}   



\begin{remark}
\lang{de}{
Das Newton-Verfahren lässt sich - etwas abgeändert - auch verwenden, um lokale Extremstellen von 
zweimal differenzierbaren Funktionen zu berechnen. Extremstellen sind ja Stellen, an denen die 
Ableitungsfunktion Nullstellen hat, weshalb man das Newton-Verfahren auf die Ableitungsfunktion 
anwenden kann, also die Rekursion
}
\lang{en}{
Newton's method can also be used - with slight modification - to find local extrema of 
twice-differentiable functions. Extrema are, after all, points at which the derivative of a function 
has a root. Thus we can use Newton's method on the derivative, i.e. iterate
}
    \begin{equation*}
      x_{n+1}=x_n-\frac{f'\left(x_n\right)}{f''\left(x_n\right)}
    \end{equation*}
\lang{de}{
benutzt, um Kandidaten f\"ur Extremstellen näherungsweise zu berechnen.
}
\lang{en}{
to converge on some candidates for extrema.
}
\end{remark}

	\begin{quickcheck}	
	\begin{variables}
		
			\randint[Z]{a}{1}{4}
			
			\function[expand,normalize]{f}{x^2-a*sin(x)}
			\derivative[normalize]{f1}{f}{x}   
			\derivative[normalize]{f2}{f1}{x}   
			
			\function{g}{x-(f1/f2)}
			
									
	\end{variables}

		
	%+++++++++++++++++++TYPE+++++++++++++++++++++++++++
		
		\field{rational}
		\type{input.function} 
		
	    \lang{de}{\text{
      Wie lautet die Rekursionsformel des Newton-Verfahrens, wenn eine mögliche Extremstelle 
      der Funktion $f(x)=\var{f}$ angenähert werden soll?\\
	    $x_{n+1}= $\ansref   (Schreiben sie lediglich $x$ statt $x_n$.)
      }}
      \lang{en}{\text{
      What is the recursion formula for Newton's method if we wish to find an extremum of the 
      function $f(x)=\var{f}$?\\
      $x_{n+1}= $\ansref   (simply write $x$ instead of $x_n$).
      }}

	    \begin{answer}
			\solution{g}
			\checkAsFunction{x}{-10}{10}{10} 
	    \end{answer}  
    	\explanation{\lang{de}{
      Die Ableitung von $f(x)=\var{f}$ ist $f'(x)=\var{f1}$ und die zweite Ableitung ist 
    	$f''(x)=\var{f2}$. Die Rekursionsformel für die Extremstelle ist allgemein 
        $x_{n+1}=x_n-\frac{f'(x_n)}{f''(x_n)}$,
   		wofür nur noch die Funktionen eingesetzt werden müssen.
    	}
      \lang{en}{
      The derivative of $f(x)=\var{f}$ is $f'(x)=\var{f1}$, and the second derivative of 
      $f''(x)=\var{f2}$. The recursion formula for the extrema is 
        $x_{n+1}=x_n-\frac{f'(x_n)}{f''(x_n)}$,
      which we then substitute the functions into.
      }}	    
	\end{quickcheck}   
\lang{de}{
Zusammenfassungen der angesprochenen Themen aus Abschnitt \ref{newton} können den folgenden Videos entnommen werden:\\
\floatright{
    \href{https://api.stream24.net/vod/getVideo.php?id=10962-2-10741&mode=iframe&speed=true}{\image[75]{00_video_button_schwarz-blau}}
    \href{https://www.hm-kompakt.de/video?watch=532}{\image[75]{00_Videobutton_schwarz}} 
}\\
}
\lang{en}{}

\section{\lang{de}{Konvergenzgeschwindigkeit des Newton-Verfahren}
         \lang{en}{Rate of convergence of Newton's method}}\label{sec:konv-geschw}

\lang{de}{
Bei der Vorstellung des Newton-Verfahrens wurde gesagt, dass das Verfahren konvergiert, wenn der 
Anfangswert $x_0$ "`nahe genug"' bei der Nullstelle $x^*$ liegt. Hier soll ein Kriterium gegeben werden,
was "`nahe genug"' heißt, und auch erklärt werden, dass das Newton-Verfahren in diesem Fall "`schnell"'
konvergiert.
}
\lang{en}{
In our introduction to Newton's method, we asserted that the sequence of approximations it provides 
converges if the starting value $x_0$ is 'sufficiently close' to a root $x^*$. Now we give a 
criterion for a point being 'sufficiently close', and explain that if Newton's method converges, 
it does so 'quickly'.
}

\begin{theorem}\label{theo:konv-geschw}
\lang{de}{
Sei $f:D\rightarrow\R$ eine zweimal differenzierbare Funktion, die eine Nullstelle $x^*\in D$ hat.\\
Wir nehmen an, dass es eine Umgebung $I=U_\epsilon(x^*)\subseteq D$ von $x^*$ gibt, so dass 
}
\lang{en}{
Let $f:D\rightarrow\R$ be a twice-differentiable function with a root at $x^*\in D$.\\
We assume that there exists a neighbourhood $I=U_\epsilon(x^*)\subseteq D$ of $x^*$ such that
}
\[ |f'(x)|\geq m_1>0 \text{ f\"ur alle }x\in I \]
\lang{de}{
für eine geeignete reelle Zahl $m_1>0$.
Weiter sei $M_2>0$ eine obere Schranke f\"ur den Betrag der zweiten Ableitung:
}
\lang{en}{
for some real number $m_1>0$.
Furthermore, let $M_2>0$ be an upper bound for the absolute value of the second derivative:
}
 \[ |f''(x)|\leq M_2 \text{\lang{de}{ f\"ur alle }\lang{en}{ for all }} x\in I.\]
\lang{de}{
Für einen Startwert $x_0\in I$ mit 
}
\lang{en}{
For a starting point $x_0\in I$ with
}
\[ {|x_0-x^*|}<\frac{2m_1}{M_2} \]
\lang{de}{
konvergiert das Newton-Verfahren \textit{quadratisch}, d.h. es gilt die Abschätzung
}
\lang{en}{
Newton's method converges \textit{quadratically}, that is, we have
}
\[  {|x_{n+1}-x^*|} \leq {|x_n-x^*|}^2 \cdot \frac{M_2}{2m_1} \]
\lang{de}{für alle $n\in \N$.}
\lang{en}{for all $n\in \N$.}
\end{theorem}

\begin{proof}[\lang{de}{Beweis}\lang{en}{Proof}]
  \begin{showhide}
    \lang{de}{
    Nach Definition der \lref{theo:newton-verfahren}{Berechnungsvorschrift des Newton-Verfahrens} gilt
    }
    \lang{en}{
    By the definition of \lref{theo:newton-verfahren}{Newton's method},
    }
    \begin{eqnarray*}
    |x_{n+1}-x^*| &=& | x_n-\frac{f(x_n)}{f'(x_n)}-x^*| =| \frac{f(x_n)-(x_n-x^*)f'(x_n)}{f'(x_n)} | \\
    &=& \frac{1}{|f'(x_n)|}\cdot \left| f(x_n)-(x_n-x^*)f'(x_n)-f(x^*)\right| ,
    \end{eqnarray*}
    \lang{de}{
    da $f(x^*)=0$ ist. Nun ist aber $f(x_n)-(x_n-x^*)f'(x_n)-f(x^*)=R_{f,1,x^*}(x_n)$ das 
    \ref[taylor-poly][Restglied zum Taylor-Polynom]{sec:restglied} der Ordnung $1$, welches wir auch mit der zweiten Ableitung
    darstellen können:
    }
    \lang{en}{
    as $f(x^*)=0$. Now we have $f(x_n)-(x_n-x^*)f'(x_n)-f(x^*)=R_{f,1,x^*}(x_n)$ the 
    \ref[taylor-poly][remainder of the Taylor polynomial]{sec:restglied} of order $1$, which can also 
    be expressed using the second derivative:
    }
    \[  R_{f,1,x^*}(x_n) =\frac{f''(\xi_n)}{2}(x_n-x^*)^2 \]
    \lang{de}{
    mit einem geeigneten Wert $\xi_n$ zwischen $x^*$ und $x_n$, insbesondere also im Intervall $I$.
    Damit folgt unter Berücksichtigung der \lref{theo:konv-geschw}{oben definierten Schranken} $m_1$ und $M_2$
    }
    \lang{en}{
    for some $\xi_n$ between $x^*$ and $x_n$, in particular, in the interval $I$. Hence by 
    considering the \lref{theo:konv-geschw}{bounds} $m_1$ and $M_2$,
    }
    \begin{eqnarray*}
    {|x_{n+1}-x^*|} &=& \frac{|R_{f,1,x^*}(x_n)|}{|f'(x_n)|} = \frac{|f''(\xi_n)|}{2|f'(x_n)|} \cdot {|x_n-x^*|}^2 \\
    &\leq & {|x_n-x^*|}^2 \cdot \frac{M_2}{2m_1}.
    \end{eqnarray*}
    \lang{de}{
    Wir haben also die oben angegebene rekursive Abschätzung. Aus dieser folgt auch die Konvergenz:\\
    Setzt man nämlich $c:=\frac{|x_0-x^*| M_2}{2m_1}<1$, so erhält man induktiv aus der rekursiven Abschätzung
    }
    \lang{en}{
    We therefore have the recursive bounds given above. From these we get convergence:\\
    if we set $c:=\frac{|x_0-x^*| M_2}{2m_1}<1$, via recursion we obtain
    }
    \[  {|x_n-x^*|} \leq c^{2^{n-1}-1}\cdot {|x_0-x^*|}. \]
  \end{showhide}
\end{proof}



\begin{remark}
\begin{enumerate}
\item \lang{de}{
Ist $f$ auf einer Umgebung von $x^*$ definiert, und $f'(x^*)\neq 0$, so lässt sich eine Umgebung $I$ 
wie im \lref{sec:konv-geschw}{Satz} immer finden. Wie oben schon erwähnt, wird in der Praxis aber 
lediglich $x_0$ nahezu beliebig gewählt und das Newton-Verfahren angewandt. 
Wenn sich keine Konvergenz absehen lässt, wird mit einem neuen Startwert begonnen.
}
\lang{en}{
If $f$ is defined on a neighbourhood of $x^*$, and $f'(x^*)\neq 0$, then a neighbourhood $I$ as in 
the \lref{sec:konv-geschw}{theorem} can always be found. As has already been mentioned, in practice 
we often choose some practically arbitrary $x_0$ and run through Newton's method. If this does not 
converge, we can simply choose another starting value and try again.
}
\item \lang{de}{
Dass die Umgebung im \lref{sec:konv-geschw}{Satz} symmetrisch um $x^*$ gewählt ist, hat den Grund, 
dass damit - und mit den folgenden Bedingungen - garantiert  ist, dass mit $x_0$ auch alle anderen 
Folgeglieder in $I$ liegen. Um wirklich eine symmetrische Umgebung zu finden, müsste man aber die 
Nullstelle schon genau kennen. Der Satz gilt jedoch auch mit Intervallen $I$, die $x^*$ enthalten, 
aber nicht symmetrisch sind, sofern für $x\in I$ auch $x-\frac{f(x)}{f'(x)}$ in $I$ liegt.
}
\lang{en}{
The neighbourhood from the \lref{sec:konv-geschw}{theorem} can be chosen to be symmetric about 
$x^*$ because then it is guaranteed that not only $x_0$, but all other approximations, lie in $I$. To 
find such a symmetric neighbourhood, we would need to know where the root is. However, the theorem 
still holds for intervals $I$ that contain $x^*$ but are not symmetric about it, provided that for 
all $x\in I$, the point $x-\frac{f(x)}{f'(x)}$ also lies in $I$.
}
\item \lang{de}{
Ist die Nullstelle auch kritischer Punkt, d.h. $f'(x^*)=0$, so konvergiert das Newtonverfahren 
nicht so gut. In diesem Fall kann man aber das Newton-Verfahren auf $f'$ anwenden.
}
\lang{en}{
If the root is also a critical point, i.e. $f'(x^*)=0$, then Newton's method does not converge too 
well. In this case, we can apply Newton's method to $f'$.
}
\end{enumerate}
\end{remark}

\begin{example}
%\begin{tabs*}[\initialtab{0}]
%\tab{$f(x)=x-e^{-x}$}
\lang{de}{
Im obigen Beispiel $f(x)=x-e^{-x}$ sind $f'(x)=1+e^{-x}$ und $f''(x)=-e^{-x}$. Da 
$f(0)=0-e^{-0}=-1<0$ und $f(1)=1-e^{-1}>0$ ist, liegt die Nullstelle von $f$ im Intervall $I=[0,1]$. 
Damit haben wir die Abschätzungen
}
\lang{en}{
In the above example $f(x)=x-e^{-x}$ we have $f'(x)=1+e^{-x}$ and $f''(x)=-e^{-x}$. As 
$f(0)=0-e^{-0}=-1<0$ and $f(1)=1-e^{-1}>0$, the root of $f$ lies in the interval $I=[0,1]$. Thus we 
have the bounds
}
\[  f'(x)=1+e^{-x}\geq 1+e^{-1}=:m_1 \text{\lang{de}{ f\"ur alle }\lang{en}{ for all }}x\in I \]
\lang{de}{und}
\lang{en}{and}
\[ |f''(x)|=|-e^{-x}|\leq 1=:M_2 \text{\lang{de}{ f\"ur alle }\lang{en}{ for all }}x\in I. \]
\lang{de}{
Ist nun $x^*\in I$ die Nullstelle und $x_0\in I$ ein Anfangswert, so gilt $|x_0-x^*|<1$, da beide in 
$I$ liegen. Andererseits ist:
}
\lang{en}{
If $x^*\in I$ is the root and $x_0\in I$ a starting point, then $|x_0-x^*|<1$ as both lie in $I$. 
Moreover:
}
\[ \frac{2m_1}{M_2}=2+\frac{2}{e}>1. \]
\lang{de}{
Für jeden Startwert im Intervall $I$ konvergiert damit das Newton-Verfahren nach obigem Satz.
\\\\
Man sieht an diesem Beispiel auch gut, dass das Kriterium nur eine hinreichende Bedingung für die 
Konvergenz gibt und kein notwendiges Kriterium: Das Newton-Verfahren konvergiert nämlich für jeden 
reellen Startwert $x_0$, obwohl die im Satz angegebende Bedingung für betragsmäßig große $x_0$ nicht 
erfüllt ist:
\\\\
Falls $x_0>-1$ ist, liegt nämlich nach \lref{ex:newton-exp}{obiger Rechnung} 
}
\lang{en}{
By the above theorem, Newton's method converges for every starting point in the interval $I$.
\\\\
This example illustrates that the given criterium is a sufficient condition but not a necessary 
condition for convergence: Newton's method actually converges for all real starting values $x_0$, not 
just those fulfilling the condition on the absolute value:
\\\\
If $x_0>-1$, then \lref{ex:newton-exp}{by the above calculation}
}
\[  x_1=\frac{x_0+1}{e^{x_0}+1} \]
\lang{de}{
im Intervall $[0,1]$, weshalb das Newton-Verfahren dann konvergiert.
\\\\
Falls $x_0\leq -1$ ist, gilt
}
\lang{en}{
in the interval $[0,1]$, so Newton's method converges in this case.
\\\\
If $x_0\leq -1$, then
}
\[ x_1=\frac{x_0+1}{e^{x_0}+1}\geq x_0+1, \]
\lang{de}{
weshalb für diese Startwerte die Folgenglieder zunächst sukzessiv um mindestens $1$ größer werden, 
bis irgendwann $x_n>-1$ gilt, woraufhin nach dem ersten Fall das Newton-Verfahren konvergiert.
}
\lang{en}{
and so for these starting values, each approximation is at least $1$ greater than the last, until 
eventually $x_0\leq -1$, the first case. Hence Newton's method converges for all real starting points.
}
%\end{tabs*}
\end{example}


\section{\lang{de}{Approximation von Wurzeln}
         \lang{en}{Approximation of roots}}\label{sec:n-te-wurzeln}

\lang{de}{
Als Spezialfall des Newton-Verfahrens lassen sich $n$-te Wurzeln 
%ganzer Zahlen durch rationale Zahlen 
annähern.
Die $n$-te Wurzel einer positiven Zahl $b$ lässt sich auf die Nullstelle des Polynoms $f(x)=x^n-b$ zurückführen
}
\lang{en}{
As a special case of Newton's method, we can approximate $n$th roots. The $n$th root of a positive 
number $b$ can be viewed as the root of the polynomial $f(x)=x^n-b$:
}
\begin{align*}
    \sqrt[n]{b}=x \quad\Rightarrow\quad b=x^n \quad\Rightarrow\quad 0=x^n-b.
\end{align*}

\begin{theorem}
\lang{de}{Sei $b>0$ und $n\in \N$ mit $n\geq 2$. Die rekursiv definierte Folge}
\lang{en}{Let $b>0$ and $n\in \N$ with $n\geq 2$. The recursively defined sequence}
\begin{eqnarray}
x_0 &\coloneq & 1\\
x_{m+1}&\coloneq & \left( 1- \frac{1}{n}\right) \cdot x_{m} + \frac{b}{n\cdot x_{m}^{n-1}}, \quad  m \geq 1 \\
\end{eqnarray}
\lang{de}{
konvergiert gegen $\sqrt[n]{b}$.\\
Ist $b$ eine rationale Zahl, so sind alle Folgeglieder rationale Zahlen.
}
\lang{en}{
converges towards $\sqrt[n]{b}$.\\
If $b$ is a rational number then every approximation in the sequence is a rational number.
}
\end{theorem}

\begin{proof*}[\lang{de}{Begründung}\lang{en}{Reasoning}]
\lang{de}{
Wendet man das Newton-Verfahren zur Nullstellenbestimmung auf die Funktion $f(x)=x^n-b$ an,
so erhält man wegen $f'(x)=nx^{n-1}$ die Rekursionsformel
}
\lang{en}{
Applying Newton's method to find the roots of the function $f(x)=x^n-b$, we get (thanks to 
$f'(x)=nx^{n-1}$) the recursion formula
}
\[ x_{m+1} =x_m -\frac{f(x_m)}{f'(x_m)}=x_m -\frac{x_m^n-b}{nx_m^{n-1}}
= \left( 1- \frac{1}{n}\right) \cdot x_{m} + \frac{b}{n\cdot x_{m}^{n-1}}. \]

\lang{de}{
Auf den Beweis, dass das Newton-Verfahren hier auch tatsächlich konvergiert, wollen wir verzichten.
}
\lang{en}{
We do not provide the proof of the convergence of Newton's method in this case.
}
\end{proof*}

\begin{remark}\label{rem:heron-verfahren}
\lang{de}{
Im Spezialfall $n=2$ ist diese Folge auch als \emph{Heron-Verfahren} zur Berechnung von 
Quadratwurzeln bekannt.
}
\lang{en}{
In the special case $n=2$, the so-called \emph{Heron's formula} is known for finding square roots.
}
\end{remark}

\begin{example}
\lang{de}{
Wir wollen die dritte Wurzel von $2$ berechnen. In diesem Fall ist also $n=3$ und $b=2$ und wir 
erhalten die Folge $(x_m)_{m\in \N}$ mit $x_0=1$ und
}
\lang{en}{
Suppose we want to find the third root of $2$. In this case, $n=3$, $b=2$, and we obtain the 
sequence $(x_m)_{m\in \N}$ with $x_0=1$ and
}
\[ x_{m+1}= \frac{2x_m}{3}+\frac{2}{3x_m^2} \]
\lang{de}{für alle $m\geq 1$. Die ersten Folgeglieder lauten also}
\lang{en}{for all $m\geq 1$. The first terms of the sequence are therefore}
\begin{align*} 
x_0 &=  1 &&& \\
x_1 &= \frac{2x_0}{3}+\frac{2}{3x_0^2} &=\frac{2}{3}+\frac{2}{3} &=\frac{4}{3} &=1,\bar{3}\, , \\
x_2 &= \frac{2x_1}{3}+\frac{2}{3x_1^2} &=\frac{8}{9}+\frac{3}{8}&=\frac{91}{72}&=1,263\bar{8}\, , \\
x_3 &= \frac{2x_2}{3}+\frac{2}{3x_2^2}&                                  &=\frac{1126819}{894348} &\approx 1,259933\, , \\
x_4 &= \frac{2x_3}{3}+\frac{2}{3x_3^2}&		&		 &\approx 1,259921\, .
\end{align*}
\lang{de}{
Das letzte berechnete Folgeglied $x_4$ stimmt mit $\sqrt[3]{2}$ sogar in den angegebenen Stellen 
überein, wie man durch das Berechnen von $x_5$ feststellen könnte.
}
\lang{en}{
The last term $x_4$ that we calculated even corresponds to $\sqrt[3]{2}$ in the given decimal places.
}
\end{example}
\lang{de}{
Eine Zusammenfassung der in den Abschnitten \ref{sec:konv-geschw} und \ref{sec:n-te-wurzeln} 
angesprochenen Themen kann dem folgenden Video entnommen werden:\\
\floatright{\href{https://api.stream24.net/vod/getVideo.php?id=10962-2-10742&mode=iframe&speed=true}{\image[75]{00_video_button_schwarz-blau}}}\\
}
\lang{en}{}


\end{visualizationwrapper}
\end{content}
