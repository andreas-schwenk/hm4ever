%$Id:  $
\documentclass{mumie.article}
%$Id$
\begin{metainfo}
  \name{
    \lang{de}{Elementares Rechnen und Termumformungen}
    \lang{en}{Writing and rearranging expressions}
  }
  \begin{description} 
 This work is licensed under the Creative Commons License Attribution 4.0 International (CC-BY 4.0)   
 https://creativecommons.org/licenses/by/4.0/legalcode 

    \lang{de}{Beschreibung}
    \lang{en}{Description}
  \end{description}
  \begin{components}
    \component{generic_image}{content/rwth/HM1/images/g_img_00_video_button_schwarz-blau.meta.xml}{00_video_button_schwarz-blau}    
  \end{components}
  \begin{links}
    \link{generic_article}{content/rwth/HM1/T104_weitere_elementare_Funktionen/g_art_content_13_wurzelfunktionen.meta.xml}{wurzelfunktionen}
    \link{generic_article}{content/rwth/HM1/T101neu_Elementare_Rechengrundlagen/g_art_content_03_bruchrechnung.meta.xml}{bruchrechnung}
    \link{generic_article}{content/rwth/HM1/T104_weitere_elementare_Funktionen/g_art_content_14_potenzregeln.meta.xml}{potenzregeln}
  \end{links}
  \creategeneric
\end{metainfo}
\begin{content}

\usepackage{mumie.ombplus}
\ombchapter{1}
\ombarticle{2}

\usepackage{mumie.genericvisualization}

\begin{visualizationwrapper}

\title{\lang{de}{Elementares Rechnen und Termumformungen}\lang{en}{Writing and rearranging expressions}}

 
\begin{block}[annotation]
  Dieses Kapitel wurde erweitert. Der Abschnitt 1.1 Grundrechenarten wurde neu vorangestellt, und beinhaltet eine Wiederholung 
  der Grundrechenarten, eine Kurz-Definition von Potenzen als Basis für den Umgang mit Termen. In Zusammenhang mit den Regeln
  zur Termumformung werden auch weitere Grund-Rechenregeln wie Kommutativ-, Assoziativ- und Distributivgesetz sowie die
  binomischen Formeln wiederholt. Der letzte Abschnitt, die Definition des Summen- und Produktzeichens, bleibt wie gehabt. 
  
\end{block}

%
% ursprüngliche Version: T101_Rechengrundlagen/content_01_terme_umformungen
%
% \begin{block}[annotation]
%  Im Ticket-System: \href{http://team.mumie.net/issues/8964}{Ticket 8964}\\
% \end{block}
%
%Ticket neu:
%
\begin{block}[annotation]
	Im Ticket-System: \href{https://team.mumie.net/issues/19191}{Ticket 19191}
\end{block}

\begin{block}[info-box]
\tableofcontents
\end{block}

%%%

\section {\lang{de}{Grundrechenarten}\lang{en}{Operators}} \label{sec:grundrechenarten}

\lang{de}{Der Erfolg im Umgang mit mathematischen Ausdr\"ucken h\"angt letztlich entscheidend von der 
         sicheren Beherrschung der vier \emph{Grundrechenarten} und den damit zusammenhängenden 
         \emph{Rechenregeln} ab. Deshalb werden im folgenden Abschnitt die wichtigsten \emph{elementaren 
         Rechengrundlagen} noch einmal zusammengefasst.}
\lang{en}{Understanding the four elementary operators and rules associated with them is crucial when working 
          with mathematical expressions. For this reason we review these in the following section.}

%%%%%%%%%%%%%%%%%%%%%%%%%%%%%%%%%%%%%%%%%%%%%%%%%%%%%%%
% noch zu klären ?
%%%%%%%%%%%%%%%%%%%%%%%%%%%%%%%%%%%%%%%%%%%%%%%%%%%%%%%
%
%   Der spezielle Umgang mit Dezimalzahlen wird in diesem Kapitel nicht behandelt. 
%   Dies kann bei Bedarf im Modul 
%   \href{https://www.ombplus.de/ombplus/link/ElemenRechneMengenZahlen/OrdnunIntervDezima}{\glqq{}IA Elemntares Rechnen: Mengen und Zahlen\grqq{}}
%    des \emph{OMB+} nachgeschlagen werden.}
%        
%%%%%%%%%%%%%%%%%%%%%%%%%%%%%%%%%%%%%%%%%%%%%%%%%%%%%%


\begin{definition}[\lang{de}{Grundrechenarten}\lang{en}{Basic operators}] \label{def:grundrechenarten}
\\
 \begin{table}[\class{layout} \cellaligns{lccl} \cellvaligns{tttt}]
    \notion{\lang{de}{Addition}\lang{en}{Addition}} &$\;(\mathbf{+})\quad$ &$ 5+8  =13 \quad$  
%    & Die Zahlen $5$ und $8$ nennt man \emph{Summanden} und das Ergebnis, im Beispiel die $13$, bezeichnet man als \emph{Summe.}\\
     &\lang{de}{"`$5+8 $"' sowie das Ergebnis $13$ bezeichnet man als \emph{Summe,} 
     die Zahlen $5$ und $8$ nennt man \emph{Summanden.}}
     \lang{en}{'$5+8 $' and the result $13$ are both called the \emph{sum}, and the numbers $5$ and $8$ 
     are sometimes called the \emph{summands.}} \\
    &&&\\
    \notion{\lang{de}{Subtraktion}\lang{en}{Subtraction}} &$\;(\mathbf{-})\quad$ &$ 12-4  =8 \quad$  
%    & Die Zahl $12$ wird \emph{Minuend} genannt, die $4$ als \emph{Subtrahend} bezeichnet und
%      das Ergebnis, hier die $8$, ist die \emph{Differenz}.\\
    &\lang{de}{"`$12 - 4 $"' sowie das Ergebnis $8$ bezeichnet man als \emph{Differenz,} 
    die Zahl $12$ nennt man den \emph{Minuend} und die $4$ ist der \emph{Subtrahend.}}
    \lang{en}{'$12 - 4 $' and the result $8$ are both called the \emph{difference}.}\\
    
    \notion{\lang{de}{Multiplikation}\lang{en}{Multiplication}} &$\; (\mathbf{\cdot})\quad$ &$ 3\cdot 6 =18 \quad$  
%    & Die Zahlen $3$ und $6$ nennt man \emph{Faktoren} und das Ergebnis, hier die $18$, bezeichnet man als \emph{Produkt.}\\
    &\lang{de}{"`$3\cdot 6 $"' sowie das Ergebnis $18$ bezeichnet man als \emph{Produkt,}
    die Zahlen $3$ und $6$ nennt man \emph{Faktoren}.}
    \lang{en}{'$3\cdot 6 $' and the result $18$ are both called the \emph{product,} 
    and the numbers $3$ and $6$ are called factors of $18$.}\\
    
    \notion{\lang{de}{Division}\lang{en}{Division}} &$\; (\mathbf{:})\quad$ &$ 12:2 = \frac{12}{2}=6  \quad $  
%    & Die $12$ wird als \emph{Dividend} bezeichnet, die $2$ als \emph{Divisor} oder auch \emph{Teiler}.  % \\  &&& 
%      Mit Dividend im Zähler und Divisor im Nenner lässt sich die Division auch als Bruch % $\frac{12}{2}$ 
%      darstellen. Das Ergebnis, $\frac{12}{2} \,$ bzw. $6$, bezeichnet man als \emph{Quotienten}. 
    &\lang{de}{Die $12$ nennt man \emph{Dividend}, die $2$ \emph{Divisor} oder auch \emph{Teiler}. 
    Mit Dividend im Zähler und Divisor im Nenner lässt sich die Division auch als Bruch $\frac{12}{2}$ darstellen.   
    Diesen bezeichnet man dann, ebenso wie "`$12:2 $"' und das Ergebnis $6$, als \emph{Quotient.}}
    \lang{en}{The $12$ is called the \emph{dividend}, the $2$ is called the \emph{divisor}.  
    Putting the dividend in the numerator and the divisor in the denominator, division can also be represented as a fraction $\frac{12}{2}$.   
    The result $6$ is called the \emph{quotient} of the division.}
    
 \end{table}
\end{definition}


\begin{remark} \label{rem:grundrechenarten}

  \begin{itemize}  
    \item \lang{de}{Die Grundrechenarten werden in vorstehender Definition an Beispielen mit natürlichen Zahlen veranschaulicht. 
         Sie sind aber gleichermaßen auch für ganze, rationale und reelle Zahlen anwendbar.}
         \lang{en}{The above examples of basic operations were performed on the natural numbers, but can also be used with integers, rational numbers 
         and real numbers.}

    \item \lang{de}{Beim Rechnen mit negativen Zahlen ist zu beachten, dass grunds\"atzlich nie zwei Rechensymbole 
        direkt hintereinander stehen d\"urfen. Um das Minuszeichen als Vorzeichen vom Rechensymbol \glqq $-$\grqq abzugrenzen, 
        verwendet man Klammern.}
        \lang{en}{When calculating with negative numbers it is important to remember that two operators cannot be directly next to each other 
        in an expression. To distinguish the minus sign from the subtraction operator we use parentheses.}\\

    \begin{itemize}  
        \item \lang{de}{Statt $5+-3$ schreibt man daher $5+(-3)$, was dasselbe wie $5-3$ ist.\\
        Statt $5--3$ schreibt man $5-(-3)$, was dasselbe wie $5+3$ ist.}
        \lang{en}{Instead of $5+-3$ we write $5+(-3)$, which is the same as $5-3$. Furthermore
        instead of $5--3$ we write $5-(-3)$, which is the same as $5+3$.} \\
      
        \item \lang{de}{Ebenso muss \glqq{}Vier multipliziert mit minus Zwei\grqq mit Klammern als $4\cdot (-2)$ geschrieben werden, und
            \glqq{}Vier geteilt durch minus Zwei\grqq als $4:(-2)$.}
            \lang{en}{Similarly, 'four multiplied by minus two' should be written as $4\cdot (-2)$, and 'four divided by minus two' as $4:(-2)$.}
      \end{itemize}
      

    \item \lang{de}{Für die Addition, Subtraktion, Multiplikation und Division von rationalen Zahlen sind besondere 
        Regeln zu beachten. Diese werden im Abschnitt \ref[bruchrechnung][\glqq{}Bruchrechnung\grqq{}]{add} erläutert.}
        \lang{en}{For addition, subtraction, multiplication and division of rational numbers we must follow special rules, explained in section 
        \ref[bruchrechnung]['Fraction arithmetic']{add}.}
    
%    \item \lang{de}{Das Rechnen mit Dezimalzahlen wird in diesem Kapitel nicht behandelt. Dies kann bei Bedarf im Modul 
%                   \href{https://www.ombplus.de/ombplus/link/ElemenRechneMengenZahlen/OrdnunIntervDezima}{\glqq{}IA Elemntares Rechnen: Mengen und Zahlen\grqq{}}
%                   des \emph{OMB+} nachgeschlagen werden.}
    
  \end{itemize}
\end{remark}

\lang{de}{Aus den Grundrechenarten lassen sich weitere mathematische Ausdrücke ableiten, wie 
        zum Beispiel die \emph{Potenz}, die als Kurzschreibweise für das mehrfache Produkt
        einer reellen Zahl mit sich selbst steht:}
\lang{en}{We can use the above operations to define other operations. For example, raising a 
number to a \emph{power}, which refers to repeated multiplication of a number by itself:}
 
\begin{definition}[\lang{de}{Potenzen / Potenzieren}\lang{en}{Powers / Exponents}] \label{def:potenz}
 \lang{de}{Für beliebige reelle Zahlen $a$ und natürliche Zahlen $n$ gilt:
  \\
  Die \notion{\emph{n-te Potenz von} $a\,$} bezeichnet das $n$-fache Produkt einer reellen Zahl $a$ mit sich 
  selbst. Man schreibt \[a^n := \underbrace{a\cdot a\cdot \ldots \cdot a}_{\text{$n$-mal}}.\]            
  Dabei nennt man $a$  die \notion{\emph{Basis}}, $\,n$ den \notion{\emph{Exponenten}} von $a^n$
  und die mathematische Operation \notion{\emph{Potenzieren}.}}
 \lang{en}{For any real $a$ and natural $n$, the \notion{\emph{n-th power of} $a\,$} is \[a^n := \underbrace{a\cdot a\cdot \ldots \cdot a}_{\text{\lang{de}{$n$-mal}\lang{en}{$n$-times}}}.\]
  Here $a$ is called the \notion{\emph{base}}, $\,n$ is called the \notion{\emph{exponent}} of $a^n$, and the operation itself 
  is called exponentiation (or 'raising $a$ to the $n$-th power').}
\end{definition}

\begin{example} \label{ex:potenz}
  \begin{itemize}
    \item $2^5=2 \cdot 2 \cdot 2 \cdot 2 \cdot 2=32$
    \item $a^1=a$
  \end{itemize}
\end{example}

\lang{de}{
Wir werden Potenzen später noch allgemeiner definieren und weiter 
vertiefen (s. Kapitel \ref[potenzregeln][\glqq{}Allgemeine Potenzen\grqq{}]{sec:potenzen}).
Hier beschränken wir uns zunächst auf Potenzen mit natürlichen Zahlen
als Exponenten. Für diese lassen sich aus der Definition die folgenden Rechenregeln
ableiten:
}
\lang{en}{
We will define exponents more generally later, and deepen our understanding of them (see the chapter 
\ref[potenzregeln]['General exponents']{sec:potenzen}). Here we limit ourselves to 
having natural numbers as exponents, for which the following rules can be derived:
}

\begin{rule}[\lang{de}{Potenzgesetze}\lang{en}{Power laws}]\label{rule:potenzgesetze}
\lang{de}{F\"ur $a,b\in\mathbb{R}$ und $n,m\in\mathbb{N}$ gilt:}
\lang{en}{For $a,b\in\mathbb{R}$ and $n,m\in\mathbb{N}$ we have:}
\\
\\	\begin{table}[\class{layout} \cellvaligns{tt}]
		& $a^n\cdot a^m=a^{n+m} \quad$
		& \textit{\lang{de}{Potenzen mit gleichen Basen werden multipliziert, indem man die Exponenten addiert.}
              \lang{en}{Powers with the same bases can be multiplied together by adding their exponents together.}}\\ 
		& $a^n\cdot b^n=(a\cdot b)^{n}\quad$
		& \textit{\lang{de}{Potenzen mit gleichen Exponenten werden multipliziert, indem man die Basen multipliziert und den Exponenten beibehält.}
              \lang{en}{Powers with the same exponents can be multiplied together by multiplying the their bases together and keeping the exponent the same.}}\\ 
 %		& $\displaystyle \frac{a^n}{b^n}=\big(\frac{a}{b}\big)^{n}\quad$
 %		& \textit{Potenzen mit gleichen Exponenten werden dividiert, indem man die Basen dividiert und den Exponenten beibehält.}\\ 
		& $(a^n)^m=a^{n\cdot m}\quad$
		& \textit{\lang{de}{Eine Potenz wird potenziert, indem man die Exponenten multipliziert.}
              \lang{en}{A power can itself be exponentiated by multiplying together the two exponents.}}\\ 
	
	\end{table}
\end{rule}

\begin{example} 
  \begin{itemize}
    \item $2^4 \cdot 2^2=\underbrace{2 \cdot 2 \cdot 2 \cdot 2}_{\text{\lang{de}{$4$-mal}\lang{en}{$4$-times}}} \cdot \underbrace{2 \cdot 2}_{\text{\lang{de}{$2$-mal}\lang{en}{$2$-times}}}
       = \underbrace{2 \cdot 2 \cdot 2 \cdot 2 \cdot 2 \cdot 2}_{\text{\lang{de}{$6$-mal}\lang{en}{$6$-times}}}= 2^6$
    \item $5^2 \cdot 2^2=\underbrace{5 \cdot 5 }_{\text{\lang{de}{$2$-mal}\lang{en}{$2$-times}}} \cdot \underbrace{2 \cdot 2}_{\text{\lang{de}{$2$-mal}\lang{en}{$2$-times}}}
       = \underbrace{(5 \cdot 2) \cdot (5 \cdot 2) }_{\text{\lang{de}{$2$-mal}\lang{en}{$2$-times}}}= (5 \cdot 2)^2=10^2$
    \item $(5^2)^4 =\underbrace{5^2 \cdot 5^2 \cdot 5^2 \cdot 5^2}_{\text{\lang{de}{$4$-mal}\lang{en}{$4$-times}}} 
                =\underbrace{(5 \cdot 5) \cdot (5 \cdot 5) \cdot (5 \cdot 5) \cdot (5 \cdot 5)}_{\text{$4 \cdot 2$\lang{de}{-mal}\lang{en}{-times}}}= 5^{2 \cdot 4}=5^8$
                
  \end{itemize}
\end{example}

%%%%%%%%%%%
\lang{de}{Als eine Umkehrung zum \emph{Potenzieren} führen wir an dieser Stelle auch noch das 
\emph{Wurzelziehen} ein.}
\lang{en}{To reverse exponentiation we introduce the concept of roots.}

\begin{definition}[\lang{de}{Wurzeln / Wurzelziehen}\lang{en}{Roots}] \label{def:wurzel}
  \lang{de}{
  Für reelle Zahlen $a\geq 0$ und natürliche Zahlen $n$ gilt:    
  \\
  Die \notion{\emph{n-te Wurzel aus $a$}} ist definiert als die eindeutige nichtnegative 
  Zahl, die $\,n\,$-mal mit sich selbst multipliziert $\,a\,$ ergibt, d.\,h. deren 
  \emph{n-te Potenz} gleich $a$ ist.
  \\
  Man schreibt $\quad \sqrt[n]{a} \quad$
  und definitionsgemäß gilt $\quad (\sqrt[n]{a})^n = a.$
  \\
%  Dabei nennt man  $\,a\,$ den \emph{\glqq Radikand\grqq } (oder \glqq \emph{Wurzelbasis}\grqq) und 
%  $n$ \emph{\glqq Wurzelexponent\grqq}.
  Im Spezialfall $n=2$ sprechen wir von der \notion{\emph{Quadratwurzel}} oder einfach \notion{\emph{Wurzel}}
  aus $a$ und schreiben auch einfach $\, \sqrt{a}\, $ statt $\, \sqrt[2]{a}.\;$
  Für $n=1$ gilt  $\, \sqrt[1]{a}=a.\;$
  }
  \lang{en}{
  For real $a$ and natural $n$, we define the \notion{\emph{$n$-th root of $a$}} to be the unique positive number, 
  which when multiplied to itself $n$ times equals $a$. That is, it is the positive number whose $n$-th power is $a$.
  \\
  In the special case of $n=2$ we refer to the \notion{\emph{square root}} or just \notion{\emph{root}} of $a$ and 
  simply write $\, \sqrt{a}\, $ instead of $\, \sqrt[2]{a}.\;$ 
  For $n=1$ we have $\, \sqrt[1]{a}=a.\;$
  }

\end{definition}

\begin{example}
  \lang{de}{Nach Beispiel \ref{ex:potenz} gilt:}\lang{en}{By example \ref{ex:potenz} we have:}
  \begin{itemize}
    \item  $\sqrt[5]{32}=\sqrt[5]{(2^5)}=2$
    \item  $\sqrt[1]{a}=\sqrt[1]{(a^1)}=a$
  \end{itemize}
\end{example}

\lang{de}{
Weitere Details sowie Regeln zum Rechnen mit Wurzeln werden später im Kapitel 
\ref[wurzelfunktionen][\glqq{}Wurzeln und Wurzelfunktionen\grqq{}]{sec:n-te_wurzel} behandelt.
}
\lang{en}{
Further details regarding rules for calculating with roots will be given later, in the chapter 
\ref[wurzelfunktionen]['Roots']{sec:n-te_wurzel}.
}

%%%%%%%%%%%

\section {\lang{de}{Terme}\lang{en}{Expressions}} \label{sec:terme}

\lang{de}{
H\"aufig wird man Ausdr"ucken begegnen, die zwei oder mehr Rechenoperationen enthalten und 
zudem neben Zahlen eventuell auch noch Buchstaben (sogenannte \emph{Variablen}), die als Platzhalter 
für eine beliebige oder noch unbenannte Zahl verwendet werden. 
Einen solchen Ausdruck bezeichnet man auch als \emph{Term.}
}
\lang{en}{
Often we will encounter \notion{\emph{expressions}}: collections of numbers and letters (so-called \emph{variables}), linked together by two or 
more operations. Variables are placeholders for unknown numbers.
}

\begin{definition}[\lang{de}{Terme}\lang{en}{Expressions}]
\lang{de}{
\notion{\emph{Terme}} sind sinnvolle Ausdrücke, die Zahlen, Variablen, Klammern und verschiedene Rechenoperationen 
enthalten können.}
\lang{en}{
An \notion{\emph{expression}} is a coherent ('well formed') collection of numbers, variables, parentheses and operators.
}
\end{definition}

\begin{example} \label{ex:terme}   
\lang{de}{Die folgenden Ausdr"ucke sind Beispiele für Terme:}
\lang{en}{The following are examples of expressions. }

\begin{itemize}    

  \item $2\cdot 4+5\cdot (3+2)  $\\  

  \item $15 \cdot b-\frac{1}{4}b^4   $\\

  \item $(2x-1)\cdot y^6 - 2x \cdot (y^2)^3 $\\

  \item $\displaystyle{\frac{a^2-b^2}{a-b}} $ 

\end{itemize}
 
\lang{de}{
Die Rechenoperation $\mathbf{\cdot}$ ("mal") wird, wie in den Beispielen, oftmals nicht extra geschrieben. 
So steht beispielsweise $2x\;$ für $2\cdot x \;$ und $\frac{1}{4}b^4 \;$ steht f"ur $\frac{1}{4} \cdot b^4$.
}
\lang{en}{
The operation $\mathbf{\cdot}$ ('multiplication') is often omitted when it is unambiguous, such as when a number and a 
variable are being multiplied, or two variables are being multiplied. For example, $\;2x\;$ stands for $2\cdot x \;$ 
and $\frac{1}{4}b^4 \;$ stands for $\frac{1}{4} \cdot b^4$.
}
\end{example}

\lang{de}{
Um mit Termen richtig umgehen zu k"onnen, sind eine Reihe von Regeln zu beachten. Dieses Thema wird 
im Teilgebiet \emph{Algebra} der Mathematik behandelt und vertieft. Die folgende Regel gibt die 
Reihenfolge zur Durchführung verschiedener Rechenoperationen und den grundlegenden Umgang mit 
Klammerausdrücken in einem Term vor.
}
\lang{en}{
We will establish a set of rules which allow us to consistently interpret and manipulate expressions. The field of \emph{algebra} 
generalises and elaborates on these rules. We first consider the order of operations, i.e. the order in which the 
operations are to be performed, which is modified by parentheses in the expression.
}

\begin{rule}[\lang{de}{Klammer- vor Potenz- vor Punkt- vor Strich-Regel}\lang{en}{Order of operations}]\label{rule:punkt-vor-strich}

\lang{de}{Bei der Berechnung von Termen ist die folgende Reihenfolge zu beachten:}
\lang{en}{Operations in an expression are to be interpreted in the following order:}
\begin{enumerate}
	\item \lang{de}{Ausdrücke in Klammern berechnen \textit{(die innersten Klammern zuerst)}}
        \lang{en}{Calculate parts of the expression in parentheses \textit{(starting with the innermost ones)}}
 	\item \lang{de}{Potenzen \textit{(mit Klammer von innen nach au"sen}, ansonsten \textit{von oben nach unten)}}
        \lang{en}{Calculate powers \textit{(considering parentheses first, and interpreting the innermost exponent first)}}
	\item \lang{de}{Multiplikationen und Divisionen \textit{(von links nach rechts)}}
        \lang{en}{Multiplication and division \textit{(both with the same priority, from left to right)}}
	\item \lang{de}{Additionen und Subtraktionen \textit{(von links nach rechts)}}
        \lang{en}{Addition and subtraction \textit{(both with the same priority, from left to right)}}
\end{enumerate} 


\\ \lang{de}{Die letzten beiden Schritte sind auch bekannt als die \notion{Punkt-vor-Strich-Regel:}\\
        \emph{\glqq Punktrechnung\grqq} (Multiplikation und Division) geht vor
        \emph{\glqq Strichrechnung\grqq} (Addition und Subtraktion).\\
    \\Insgesamt gilt also: \notion{Klammer- vor Potenz- vor Punkt- vor Strichrechnung.}    
        }

\end{rule}

\begin{example}
 \begin{tabs*}[\initialtab{0}]
 \tab{\lang{de}{Anwendung der \emph{Klammer- vor Potenz- vor Punkt- vor Strich-Regel}}\lang{en}{Using order of operations}}                  
  \begin{itemize} 
    \item  \lang{de}{Wir berechnen den Wert des ersten Terms aus Beispiel \ref{ex:terme}  
      unter Berücksichtigung der \emph{Klammer- vor Potenz- vor Punkt- vor Strich-Regel}:}
      \lang{en}{We calculate the value of the first expression from example \ref{ex:terme}, 
      keeping in mind the above order of operations:}
           
    \begin{description} 
 This work is licensed under the Creative Commons License Attribution 4.0 International (CC-BY 4.0)   
 https://creativecommons.org/licenses/by/4.0/legalcode 
  
        \item[\lang{de}{Zun"achst wird der Ausdruck in der Klammer berechnet:}
              \lang{en}{Firstly we calculate the expression in the parentheses:}]
             \[2\cdot 4+5\cdot (3+2)=2\cdot 4+5\cdot 5 \]
        \item[\lang{de}{Danach werden die Multiplikationen (von links nach rechts) ausgef"uhrt:}
              \lang{en}{Then we perform multiplication (from left to right):}] % $2\cdot 4=8$ und $5\cdot 5=25$. 
             \[2\cdot 4+5\cdot 5 = 8 + 5\cdot 5= 8 + 25  \]
        \item[\lang{de}{Und zuletzt erfolgt die Addition:}
              \lang{en}{And finally we perform the addition:}] \[8+25=33 \]
    \end{description}
               
%  Wenn in einem Term keine Variablen auftreten, werden die Regeln dazu verwendet, den Term zu berechnen.
 
    \item \lang{de}{
        Der zweite Term aus \ref{ex:terme} ist ein Beispiel dafür, wie wichtig die 
        Klammersetzung in Termen sein kann: 
        \[15 \cdot b-\frac{1}{4}b^4 \]
        Der Faktor $15$ wird hier nämlich nur mit $b$ multipliziert, nicht mit dem Term
        $b-\frac{1}{4}b^4 \;$. F"ur Letzteres m"ussten aufgrund der \emph{Klammer- vor Potenz- vor Punkt- vor Strich-Regel }
        entsprechende Klammern gesetzt werden:
        }
        \lang{en}{
        The second expression from \ref{ex:terme} is an example of how important parentheses can be:
        \[15 \cdot b-\frac{1}{4}b^4 \]
        The factor $15$ here is multiplied by $b$, rather than by the expression $b-\frac{1}{4}b^4 \;$. For it to be interpreted as 
        the latter, we would require a set of parentheses:
        }
        \[15\cdot \left( b-\frac{1}{4}b^4 \right) \neq 15 \cdot b-\frac{1}{4}b^4\]

        
 
    \item \lang{de}{
            Auch bei Potenz-Termen spielen Klammern eine wichtige Rolle.
            Gem\"a\"s der \emph{Klammer- vor Potenz-Regel } in \ref{rule:punkt-vor-strich} 
            werden im folgenden Term die Klammern von innen nach außen gerechnet:
            \[\left(2^4\right)^3=16^3=4.096\] 
            Wendet man die Regel \ref{rule:potenzgesetze} über das Potenzieren von 
            Potenzen an, führt dies zum gleichen Ergebnis:
            \[\left(2^4\right)^3=2^{(4 \cdot3)}=2^12=4.096\]
            \textbf{Ohne} Klammern jedoch werden Potenzen \textit{von oben nach unten} 
            berechnet. Das bedeutet:
            \[2^{4^3}=2^{(4^3)}=2^64\neq \left(2^4\right)^3 \]
            }
            \lang{en}{
            Parentheses also play a large role in how powers are interpreted. 
            Following the order of operations (rule \ref{rule:punkt-vor-strich}) and 
            evaluating the expression inside the parentheses first yields 
            \[\left(2^4\right)^3=16^3=4096\] 
            Following rule \ref{rule:potenzgesetze} about powers of powers gives the same result:
            \[\left(2^4\right)^3=2^{(4 \cdot3)}=2^12=4096 \]
            \textbf{Without} parentheses we interpret powers from the top down, so
            \[2^{4^3}=2^{(4^3)}=2^64\neq \left(2^4\right)^3 \]
            }
       
   \end{itemize}
 \end{tabs*}
\end{example}


\begin{quickcheckcontainer}

\randomquickcheckpool{1}{4}  % vier verschiedene Terme mit randomisierten Zahlen
%1
\begin{quickcheck}
		\type{input.number}
		\begin{variables}
			\randint[Z]{a}{-1}{5}
			\randint[Z]{b}{1}{3}
			\randint[Z]{c}{2}{4}
			\randint[Z]{d}{1}{5}
		    \function{f}{a-b*c+d}
		    \function[calculate]{loes}{a-b*c+d}
		\end{variables}
		
			\text{\lang{de}{Berechnen Sie den folgenden Term:}\lang{en}{Evaluate the following expression:}\\
				  $\var{a}-\var{b}\cdot \var{c}+\var{d}=$\ansref}
		
		
		\begin{answer}
			\solution{loes}
		\end{answer}
	\end{quickcheck}
%2
\begin{quickcheck}
		\type{input.number}
		\begin{variables}
			\randint[Z]{a}{-1}{5}
			\randint[Z]{b}{1}{3}
			\randint[Z]{c}{2}{4}
			\randint[Z]{d}{1}{5}
		    \function[calculate]{loes}{a-b*(c+d)}
		\end{variables}
		
			\text{\lang{de}{Berechnen Sie den folgenden Term:}\lang{en}{Evaluate the following expression:}\\
				  $\var{a}-\var{b}\cdot (\var{c}+\var{d})=$\ansref}
		
		\begin{answer}
			\solution{loes}
		\end{answer}
	\end{quickcheck}
%3
\begin{quickcheck}
		\type{input.number}
		\begin{variables}
			\randint[Z]{a}{-1}{5}
			\randint[Z]{b}{1}{3}
			\randint[Z]{c}{2}{4}
			\randint[Z]{d}{1}{5}
		    \function[calculate]{loes}{(a-b)*c+d}
		\end{variables}
		
			\text{\lang{de}{Berechnen Sie den folgenden Term:}\lang{en}{Evaluate the following expression:}\\
				  $(\var{a}-\var{b})\cdot \var{c}+\var{d}=$\ansref}

		
		\begin{answer}
			\solution{loes}
		\end{answer}
	\end{quickcheck}
%4
\begin{quickcheck}
		\type{input.number}
		\begin{variables}
			\randint[Z]{a}{-1}{5}
			\randint[Z]{b}{1}{3}
			\randint[Z]{c}{2}{4}
			\randint[Z]{d}{1}{5}
		    \function[calculate]{loes}{a-(b*c+d)}
		\end{variables}
		
			\text{\lang{de}{Berechnen Sie den folgenden Term:}\lang{en}{Evaluate the following expression:}\\
				  $\var{a}-(\var{b}\cdot \var{c}+\var{d})=$\ansref}

			
		\begin{answer}
			\solution{loes}
		\end{answer}
	\end{quickcheck}

\end{quickcheckcontainer}

\section{\lang{de}{Rechengesetze und Termumformungen}\lang{en}{Rearranging expressions}}\label{sec:rechengesetze}

\lang{de}{
Terme können zuweilen sehr komplex sein. Deshalb ist man häufig bestrebt, sie so weit wie möglich zu vereinfachen. 
Will man beispielsweise für einen Term ohne Variablen dessen Wert bestimmen, so kann durch geschickte vorherige 
Umformung des Terms der Rechenaufwand oft deutlich verringert werden. 
\\
Für die Vereinfachung von Termen gibt es neben der bereits beschriebenen 
\emph{Klammer-vor-Potenz-vor-Punkt-vor-Strich-Regel } (\ref{rule:punkt-vor-strich}) und den 
\emph{Potenzgesetzen} (\ref{rule:potenzgesetze}) verschiedene weitere Umformungsregeln zu beachten,
die wir teilweise auch schon benutzt haben. 
Dabei werden Termumformungen stets durch ein Gleichheitszeichen \glqq{}$=$\grqq{} gekennzeichnet.
}
\lang{en}{
Expressions can be very long and complicated, so we often need to simplify them. 
For example, if we want to calculate the value of an expression that does not contain any variables, 
we can often do so more easily after simplifying it.
\\
We introduce some more rules for manipulating expressions besides the \emph{order of operations} (\ref{rule:punkt-vor-strich}) and \emph{power laws} (\ref{rule:potenzgesetze}) explained above. 
An equals sign '$=$' is used between expressions that have the same value.
}

%%%%%%%%%%%%%%%%%%%%%%%%%%%%%%%%%%%%%%%%%%%%%%%%%%%%%%%%%%%
% Die Warnung ist an dieser Stelle unverständlich - setzt die Kenntnis der Regeln für Bruchterme voraus
%%%%%%%%
% \begin{block}[warning]
% Bei Umformungen mit Variablen ist darüber hinaus zu beachten, dass diese Gleichheiten nur f"ur die 
% Werte von Variablen gelten, bei denen beide Seiten definiert sind!
% \end{block}
%

\begin{rule}[\lang{de}{Rechengesetze}\lang{en}{Properties of addition and multiplication}]\label{rule:rechengesetze}
\lang{de}{F\"ur die Addition und die Multiplikation reeller Zahlen $a,b$ und $c \;$ gelten:}
\lang{en}{For addition and multiplication of real numbers $a,b$ and $c \;$ we have:} \\


\\	\begin{table}[\class{layout}  \cellaligns{lccc}]
		
		& \notion{\lang{de}{Kommutativgesetz:}\lang{en}{Commutativity:}} 
        &  $\quad a+b =b+a \quad$ & \lang{de}{und}\lang{en}{and} & 
           $\quad a \cdot b = b \cdot a$ \\
	
	\end{table}       
\\	\begin{table}[\class{layout}  \cellaligns{lccc}]
	
    	& \notion{\lang{de}{Assoziativgesetz:}\lang{en}{Associativity:}} 
        &  $\quad (a+b)+c =a+(b+c)\quad$  & \lang{de}{und}\lang{en}{and} & 
           $\quad (a\cdot b)\cdot c = a\cdot (b\cdot c)$ \\ 
	
	\end{table}        
\\	\begin{table}[\class{layout}  \cellaligns{lccc}]
	
        & \notion{\lang{de}{Distributivgesetz:}\lang{en}{Distributivity:}}
		&  $\quad a \cdot (b + c) = a \cdot b+ a \cdot c \quad$ & \lang{de}{und}\lang{en}{and} &
           $\quad (b + c)\cdot a = b\cdot a +c\cdot a $ \\ 		
	
	\end{table}

\end{rule}

\begin{remark}
    \lang{de}{
    Aus dem Distributivgesetz folgt:
      \[ (a + b) \cdot (c + d) = a \cdot (c + d) + b \cdot (c + d) 
              = a \cdot c + a \cdot d + b \cdot c + b \cdot  d \]
      Die Termumformung, von links nach rechts betrachtet, nennt man auch 
      \emph{\notion{\glqq{}Ausmultiplizieren\grqq{}}}. 
      In der umgekehrten Richtung spricht man von 
      \emph{\notion{\glqq{}Ausklammern\grqq{}}} oder auch 
      \emph{\notion{\glqq{}Faktorisieren\grqq{}}.}
    }
    \lang{en}{
    Distributivity gives us:
      \[ (a + b) \cdot (c + d) = a \cdot (c + d) + b \cdot (c + d) 
              = a \cdot c + a \cdot d + b \cdot c + b \cdot  d \]
      Such removal of parentheses via manipulation of an expression, as is 
      done here from left to right, is called \emph{\notion{'expanding parentheses'}}. 
      In the other direction, the process is called \emph{\notion{'factorising'}}.
    }  
\end{remark}  

\lang{de}{
Kommutativ- und Assoziativgesetz gelten \textbf{nicht} für die Subtraktion und die Division. 
Wandelt man aber die Subtraktion in eine Addition um, indem man den \textit{Subtrahenden} 
mit negativem Vorzeichen addiert, also $a-b=a+(-b)$, so können die Regeln doch angewandt werden. 
Wichtig ist dabei nur, dass man das Vorzeichen immer 
\textit{\glqq{}mitnimmt\grqq{}}.
}
\lang{en}{
Commutativity and associativity do \textbf{not} apply to subtraction or division. A subtraction 
can however be written as an addition by changing the sign of the term being subtracted, so 
$a-b=a+(-b)$. Then these rules can still be applied, remembering to not seperate a term from its 
negative sign.
}
\begin{example}
   \begin{itemize}  
    \item \lang{de}{
    Betrachen wir den Term $\, 2-3.$ Wenn wir das Minuszeichen als Vorzeichen an die $3$
    \textit{\glqq{}binden\grqq{}} und hierdurch die Subtraktion als Addition betrachten,
    ist der Term kommutativ, denn \[2+(-3)=-1=(-3)+2,\]
    Betrachten wir den Term als Subtraktion, stellen wir fest \[2-3=-1\neq 1= 3-2.\]
    Die Subtraktion selbst ist also \textbf{nicht} kommutativ.
    }
    \lang{en}{
    Consider the expression $\, 2-3.$ If we consider the minus sign as a negative sign of the $3$, 
    and view the expression as an addition, we have commutativity: \[2+(-3)=-1=(-3)+2.\]
    Clearly \[2-3=-1\neq 1= 3-2,\] so subtraction itself is \textbf{not} commutative.
    }

    \item \lang{de}{
    Die Subtraktion ist auch nicht assoziativ, denn
          \[(2-4)-6=-2-6=-8 \neq 4=2-(-2)=2-(4-6).\]
     Betrachtet man aber die Subtraktionen durch \textit{\glqq{}Binden\grqq{}} 
     des Vorzeichens als Additionen, ist das Assoziativgesetz anwendbar, denn es gilt
          \[(2-4)-6=(2+(-4))+(-6)=(-2)+(-6)=-8 \neq -8=2+(-10)=2+(-4)+(-6).\]
    }
    \lang{en}{
    Subtraction is also not associative, as 
    \[(2-4)-6=-2-6=-8 \neq 4=2-(-2)=2-(4-6).\]
    Again, viewing the subtaction as an addition by $-4$, we have associativity: 
    \[(2-4)-6=(2+(-4))+(-6)=(-2)+(-6)=-8 \neq -8=2+(-10)=2+(-4)+(-6)).\]
    }

    \item \lang{de}{
    Das folgende Beispiel zeigt, dass die Division nicht assoziativ ist.
          \[(24:4):2=6:2=3 \neq 12=24:2=24:(4:2)\]
    }
    \lang{en}{
    The following example shows that division is not associative.
    \[(24:4):2=6:2=3 \neq 12=24:2=24:(4:2)\]
    }
   \end{itemize}  
\end{example}

\begin{remark}
\lang{de}{Das Distributivgesetz gilt auch für die Subtraktion statt der Addition, also}
\lang{en}{Distributivity holds for subtraction as well as addition, so}
          \[
          a \cdot (b - c) = a \cdot b - a \cdot c \quad \text{\lang{de}{und}\lang{en}{and}}
           \quad (b - c)\cdot a = b\cdot a - c\cdot a 
           \]
\begin{showhide}
\lang{de}{
Wie zuvor erläutert, gilt zum einen $\, b-c=b+(-c).$ 
Zudem gilt wegen $-c=(-1)\cdot c\,$ unter Verwendung des Kommutativ- 
und des Assoziativgesetzes.
}
\lang{en}{
As mentioned earlier, $\, b-c=b+(-c).$ Furthermore, $-c=(-1)\cdot c\,$, 
by communitativity and associativity.
}
\[a\cdot (-c)=a\cdot ((-1)\cdot c)=(-1)\cdot (a \cdot c)=-a \cdot c.\]
\end{showhide}

\lang{de}{Das Distributivgesetz gilt aber nicht für die Division anstelle der Multiplikation.}
\lang{en}{Distributivity however does not hold for division.}   %But for division by a non-zero number you can consider multiplication by a reciprocal? - Niccolo
\end{remark}

\begin{example}
 \begin{tabs*}[\initialtab{0}]
 \tab{\lang{de}{Anwendung der Rechengesetze}\lang{en}{Manipulating expressions}}
% myblue: \#4169E1
% myorange: \#D2691E 
\lang{de}{Wir vereinfachen den Term}\lang{en}{We simplify the expression}
\[  -x \cdot (x-z) \cdot 2 - (-2x) \cdot (5-z)  - 2 \cdot (7-x) \cdot (x-3).\]
\lang{de}{
In den ersten beiden Schritten der Termumformung wenden wir zunächst das Kommutativgesetz (KG) und anschließend das 
Distributivgesetz (DG) an, um den Faktor $\textcolor{\#D2691E}{(-2x)}$ auszuklammern:
}
\lang{en}{
In the first two steps of the manipulation we use commutativity and distributivity to expand the factor 
$\textcolor{\#D2691E}{(-2x)}$:
}

 \begin{align*}
  && \quad \textcolor{\#D2691E}{(-x)}\cdot (x-z) \cdot \textcolor{\#D2691E}{2}  - \textcolor{\#D2691E}{(-2x)} \cdot (5-z) - 2 \cdot (7-x) \cdot (x-3)\\ 
  &\underset{(KG)}{=}& \quad \textcolor{\#D2691E}{(-2x)}\cdot (x-z)  - \textcolor{\#D2691E}{(-2x) } \cdot (5-z) - 2 \cdot (7-x) \cdot (x-3) \\ 
  &\underset{(DG)}{=}& \quad  \textcolor{\#D2691E}{(-2x)}\cdot \underbrace{((x-z)-(5-z))}_{=(x-5)} - 2 \cdot (7-x) \cdot (x-3) \\   
 \end{align*}

\lang{de}{
Im nächsten Schritt wenden wir das Distributivgesetz erneut an, diesmal jedoch zum Ausmultiplizieren des orangenen und des blauen Terms:
}
\lang{en}{
In the next step we once more make use of distributivity, this time to expand the orange and blue terms:
}
 \[
 \begin{mtable}[\class{layout} ]
  &=& \textcolor{\#D2691E}{(-2x)}\cdot \textcolor{\#D2691E}{(x-5)} &-& \textcolor{\#4169E1}{2} \cdot \textcolor{\#4169E1}{(7-x)} \cdot \textcolor{\#4169E1}{(x-3)} \\ 
  & \underset{(DG)}{=}&  \textcolor{\#D2691E}{-2x^2}+\textcolor{\#D2691E}{10x} &+& \textcolor{\#4169E1}{(-2)} \cdot \textcolor{\#4169E1}{(7x+3x-21-x^2)}\\
  & \underset{(DG)}{=}& \textcolor{\#D2691E}{-2x^2}+\textcolor{\#D2691E}{10x} &+& \textcolor{\#4169E1}{(2x^2 - 20x + 42)}\\
 \end{mtable}
 \]

\lang{de}{
Im letzten Schritt sortieren wir die Summanden durch Anwendung des Kommutativgesetzes (KG) und des Assoziativgesetzes (AG)
so, dass wir gleiche Variablen bzw. Variablen mit gleichen Potenzen zusammenbringen:
}
\lang{en}{
In the final step we gather the terms of the the sum using commutativity and associativity, bringing together variables with the same exponents:
}

 \[
 \begin{mtable}[\class{layout} ]
  & \underset{(KG), (AG)}{=}&  (\textcolor{\#D2691E}{-2x^2} + \textcolor{\#4169E1}{2x^2}) &+& (\textcolor{\#D2691E}{10x} +\textcolor{\#4169E1}{(-20x)}) &+& \textcolor{\#4169E1}{42} \\
  & \underset{(DG)}{=}& (-2+2) \cdot x^2 &+& (10-20) \cdot x &+& 42\\
  &=& 0 &+& (-10) \cdot x &+& 42\\
  &\underset{(KG)}{=}& 42 - 10x &&&&\\
 \end{mtable}
 \]
\end{tabs*}
\end{example}


\begin{quickcheck}
		\type{input.number}
		\begin{variables}
			\randint[Z]{a}{-1}{5}
			\randint[Z]{b}{1}{3}
			\randint[Z]{c}{2}{4}
			\randint[Z]{d}{1}{5}
            \randint[Z]{g}{1}{5}
		    
			\function[calculate]{loes1}{a*c}
			\function[calculate]{loes2}{d-b*c}
            \function[calculate]{loes3}{g*c-d}
		\end{variables}
			
			\text{\lang{de}{Vereinfachen Sie den folgenden Term m"oglichst weit:}\lang{en}{Simplify the following expression as far as possible:} \\
			$(\var{a} - \var{b} x + \var{g} y)\cdot \var{c}+\var{d}\cdot (x-y)=$\ansref + \ansref $\cdot x + $\ansref $\cdot y$}
		
		
		\begin{answer}
			\solution{loes1}
		\end{answer}
		\begin{answer}
			\solution{loes2}
		\end{answer}
        \begin{answer}
			\solution{loes3}
		\end{answer}
\end{quickcheck}


\lang{de}{Mit Hilfe des \lref{rule:rechengesetze}{Distributivgesetzes} lassen sich des Weiteren
auch die sogenannten \emph{binomischen Formeln} herleiten.}
\lang{en}{Thanks to \lref{rule:rechengesetze}{distributivity}, some \emph{identities} for expanding so-called \emph{binomials} can be derived.}   %Is there a proper name for these? - Niccolo

\begin{rule}[\lang{de}{Binomische Formeln}\lang{en}{Binomial identities}]\label{rule:binomische_formeln}

\lang{de}{Für alle reellen Zahlen $a$ und $b$ gelten:}\lang{de}{For all real numbers $a$ and $b$ we have}


\\	\begin{table}[\class{layout} \cellaligns{llcl}]
		
		& \notion{1. \lang{de}{binomische Formel:}\lang{en}{Square of a binomial (sum):}} &$\quad (a+b)^2 $&$ =a^2+2ab+b^2 $\\
             

		& \notion{2. \lang{de}{binomische Formel:}\lang{en}{Square of a binomial (subtract):}} &$\quad(a-b)^2 $&$ =a^2-2ab+b^2$ \\ 
        

        & \notion{3. \lang{de}{binomische Formel:}\lang{en}{'Difference of two squares':}} &$\quad (a+b)(a-b) \; $&$ =a^2-b^2 $  \\
		
	
	\end{table}
    
    
\end{rule}
 
\begin{tabs*}[\initialtab{0}]
\tab{\lang{de}{Herleitung}\lang{en}{Derivation}}\label{ausmultipliziert}

\lang{de}{
Nach \lref{def:potenz}{Definition der Potenz} ist
}
\lang{en}{By the \lref{def:potenz}{definition of powers},
}

\[(a+b)^2 = (a+b)(a+b).\]

\lang{de}{
Multipliziert man diesen Term nach dem Distributivgesetz aus, so erhält man
}
\lang{en}{
Expanding this expression by distributivity gives
}

\[ (a+b)(a+b) = a \cdot a + a \cdot b + b \cdot a + b \cdot  b =a^2 + 2ab + b^2.\]

\lang{de}{
Ersetzt man nun $b$ durch $-b$, folgt direkt
}
\lang{en}{
Replace $b$ with $-b$ to obtain
}
\[(a-b)^2=(a+ (-b))^2 = a^2+2a(-b)+(-b)^2=a^2-2ab+b^2 \]

\lang{de}{
und durch Ausmultiplizieren von $(a+b)(a+ (-b))$ erhält man entsprechend
}
\lang{en}{
and by expanding $(a+b)(a+ (-b))$ we have
}

\[ a \cdot a + a \cdot (-b) + b \cdot a + b \cdot  (-b) =a^2 -ab + ab - b^2=a^2-b^2.\]

\end{tabs*}

% Beispiele

\begin{example}[\lang{de}{Anwendung der binomischen Formeln}\lang{en}{Applying the above 'binomial identities'}]
% \begin{tabs*} [\initialtab{0}]
% \tab{Anwendung der Binomischen Formeln}
  \begin{itemize}
  \item $32^2=(30+2)^2=30^2+2\cdot 30\cdot 2+2^2=900+120+4=1024$
  \item $39^2=(40-1)^2=40^2-2\cdot 40\cdot 1+1^2=1600-80+1=1521$
  \item $32\cdot 28=(30+2)\cdot (30-2)=30^2-2^2=900-4=896$ 
  \item $(2x+3)^2=(2x)^2+2\cdot 2x\cdot 3+3^2=4x^2+12x+9$
  \item $(x+1)^2-(x-1)^2=(x^2+2x+1)-(x^2-2x+1)$\\
        $\phantom{(x+1)^2-(x-1)^2}=x^2+2x+1-x^2+2x-1$\\
        $\phantom{(x+1)^2-(x-1)^2}=2x+2x=4x$
  \item $(x+y+4)^2=((x+y)+4)^2$\\
        $\phantom{(x+y+4)^2}=(x+y)^2+2\cdot(x+y)\cdot 4+4^2$\\
        $\phantom{(x+y+4)^2}=x^2+2xy+y^2+8(x+y)+16$\\
        $\phantom{(x+y+4)^2}=x^2+2xy+y^2+8x+8y+16$
  \item $(2x+4)(2x-4)=(2x)^2-4^2=4x^2-16$
  \end{itemize}

% \end{tabs*}
\end{example}
%%%%%%%%%%%%%%%%%%%%%%%%%%%%%555%%%%%%%%%%
\begin{quickcheckcontainer}

\randomquickcheckpool{1}{3}  
%1
\begin{quickcheck}
		\type{input.number}
		\begin{variables}
			\randint[Z]{a}{1}{4}
			\randint[Z]{b}{1}{5}

            \function[calculate]{t1}{2*a*b}
            \function[calculate]{t2}{a^2}
			\function[calculate]{loes1}{a}
			\function[calculate]{loes2}{b}
            \function[calculate]{loes3}{b^2}
		\end{variables}
			
			\text{\lang{de}{Vervollständigen Sie die folgende binomische Formel mit passenden natürlichen Zahlen:}
            \lang{en}{Complete the following equation with the correct natural numbers:} \\
			$($ \ansref $ \cdot x +$ \ansref $)^2= \var{t2} \cdot x^2 + \var{t1} \cdot x +$ \ansref}
		
		
		\begin{answer}
			\solution{loes1}
		\end{answer}
		\begin{answer}
			\solution{loes2}
		\end{answer}
        \begin{answer}
			\solution{loes3}
		\end{answer}
\end{quickcheck}

%2
\begin{quickcheck}
		\type{input.number}
		\begin{variables}
			\randint[Z]{a}{1}{4}
			\randint[Z]{b}{-5}{-1}

			\function[calculate]{t1}{-2*a*b}
			\function[calculate]{loes1}{b}
			\function[calculate]{loes2}{a^2}
            \function[calculate]{loes3}{b^2}
		\end{variables}
			
			\text{\lang{de}{Vervollständigen Sie die folgende binomische Formel:}
            \lang{en}{Complete the following equation:} \\
			$(\var{a} +$ \ansref $\cdot y)^2=$ \ansref $ - \var{t1} \cdot y +$ \ansref $\cdot y^2$}
		
		
		\begin{answer}
			\solution{loes1}
		\end{answer}
		\begin{answer}
			\solution{loes2}
		\end{answer}
        \begin{answer}
			\solution{loes3}
		\end{answer}
\end{quickcheck}

%3
\begin{quickcheck}
		\type{input.number}
		\begin{variables}
			\randint[Z]{a}{1}{4}
			\randint[Z]{b}{1}{5}

            \function[calculate]{t1}{b^2}
            \function[calculate]{t2}{a^2}
			\function[calculate]{loes1}{b}
			\function{loes2}{a*z}
            \function[calculate]{loes3}{b}
		\end{variables}
			
			\text{\lang{de}{Vervollständigen Sie die folgende binomische Formel mit passenden natürlichen Zahlen:}
            \lang{en}{Complete the following equation with the correct natural numbers:} \\
			$( \var{a} z +$ \ansref $) \cdot ($\ansref$-$\ansref$) = \var{t2} z^2 - \var{t1} $}
				
		\begin{answer}
			\solution{loes1}
		\end{answer}
		\begin{answer}
			\solution{loes2}
		\end{answer}
        \begin{answer}
			\solution{loes3}
		\end{answer}
\end{quickcheck}

\end{quickcheckcontainer}
%%%%%%%%%%%%%%%%%%%%%%%%%%%%%555%%%%%%%%%%

\lang{de}{Wichtig ist auch die Anwendung der binomischen Formeln zur Faktorisierung, 
also quasi rückwärts angewandt, um eine Summe in ein Produkt umzuwandeln.}
\lang{en}{
The binomial identities above are often also used to factorise an already-expanded expression, applying them in the other direction, to turn a sum into a product.
} \\


\begin{example}
  \begin{itemize}
	\item $26^2-4^2=(26+4)\cdot (26-4)=30\cdot 22=660$ \\
    \item $4x^2+4x+1=(2x)^2+2\cdot 2x\cdot 1+1^2=(2x+1)^2$ \\
    \item $(x+1)^2-(x-1)^2=\big( (x+1)+(x-1)\big)\big((x+1)-(x-1)\big)=2x\cdot 2=4x$\\
  \end{itemize}
\end{example}

\lang{de}{Das erste Beispiel zeigt, wie durch die rückwärtige Anwendung der 3. binomischen Formel die 
    Berechnung eines reinen Zahlen-Terms vereinfacht werden kann.\\ 
    Die Faktorisierung von Termen wird ansonsten häufig zur Vereinfachung (\emph{Kürzung}) von 
    sogenannten \emph{Bruchtermen} genutzt, also Brüchen, die im Zähler oder im Nenner statt einer ganzen 
    Zahl einen Term enthalten. Die Regeln, wie man mit Bruchtermen rechnet, werden im Kapitel 
    \ref[bruchrechnung][ Bruchrechnung ]{add} behandelt werden.\\
    }
\lang{en}{The first example shows how the third of the identities (the difference of two squares) can be 
    used to simplify a numerical expression.\\
    Factorisation of expressions is often used to simplify fractions that have an expression other than a 
    number in their numerator or denominator. The rules regarding these are introduced in 
    \ref[bruchrechnung][ Manipulating fractions ]{add}.\\
}

\begin{example}
 \begin{tabs*}[\initialtab{0}]
 \tab{\lang{de}{Vereinfachen eines Bruchterms}\lang{en}{Simplifying a fraction}}
  \lang{de}{Wir betrachten den folgenden Bruchterm:}\lang{en}{Consider the following fraction:}
  \[\frac{(x^2+4xy+4y^2) \cdot (4x-8y)}{x^2-4y^2}
  \]

  \lang{de}{
  Im Zähler erkennen wir die 1. binomische Formel:
  }
  \lang{en}{
  In the numerator we recognise the first binomial identity:
  }
  \[(x^2+4xy+4y^2) = (x+2y)^2.\]

  \lang{de}{
  Der Nenner lässt sich mit der 3. binomischen Formel faktorisieren:
  }
  \lang{en}{
  The numerator can be factorised using the difference of two squares rule,
  }
 \[ x^2-4y^2 = (x+2y)\cdot (x-2y).\]
  
  \lang{de}{Damit erhalten wir:}\lang{en}{Hence we have:}
  
  $\quad\displaystyle \frac{(x^2+4xy+4y^2) \cdot (4x-8y)}{x^2-4y^2}=\frac{(x+2y)^2 \cdot 4(x-2y)}{(x+2y)\cdot (x-2y)}$\\ \\
  
  $\quad\displaystyle =\frac{(x+2y) \cdot \cancel{(x+2y)} \cdot 4 \cdot \cancel{(x-2y)}}{\cancel{(x+2y)}\cdot \cancel{(x-2y)}}=4 \cdot (x+2y)$


 \end{tabs*}
\end{example}




\section{\lang{de}{Summenzeichen und Produktzeichen}\lang{en}{Summation and product symbols}}

\lang{de}{
Für Summen mit vielen Summanden ist es unpraktisch, jeden einzelnen Summanden aufzuschreiben. 
Man sucht also nach einer abkürzenden Schreibweise für solche Summen.
Dies ist jedoch nur dann möglich, wenn die Folge der Summanden einer Regel 
unterliegt, d.\,h. wenn man aus einzelnen aufeinanderfolgenden Summanden mittels dieser Regel 
die weiteren Summanden herleiten kann. Betrachten wir hierzu die folgenden Beispiele:
}
\lang{en}{
For sums of particularly many numbers, it is impractical to write down every one of these numbers. 
If the sequence being summed follows a rule, we can use this to express the sum concisely and 
unambiguously.
}

\begin{example} \label{ex:summen}

\begin{enumerate}[alph]
  \item $\; 1+2+3+\ldots+98+99+100 
        \quad$ \lang{de}{ist die Summe aller natürlichen Zahlen von $1$ bis $100$.}
               \lang{en}{is the sum of all natural numbers from $1$ to $100$.}
  \item $\; 1+3+5+7+9+11+\ldots+23
        \;$ \lang{de}{ist die Summe aller ungeraden Zahlen von  $1$ bis $23$.}
            \lang{en}{is the sum of all odd numbers from $1$ to $23$.}
  \item $\; 1+\frac{1}{2}+\frac{1}{4}+\frac{1}{8}+\frac{1}{16}+\ldots+\frac{1}{256}
        \quad$ \lang{de}{ist die Summe aller Potenzen von $\frac{1}{2}$ mit Exponenten von $0$ bis $8$.}
               \lang{en}{is the sum of all powers of $\frac{1}{2}$ with exponents from $0$ to $8$.}
\end{enumerate}

%
% Wie wir leicht erkennen können, sind die Summanden
%
% \begin{block}
% \begin{enumerate}
%  \item[im 1. Beispiel] alle natürlichen Zahlen von $1$ bis $100$,
%  \item[im 2. Beispiel] alle ungeraden Zahlen von  $1$ bis $23$ und
%  \item[im 3. Beispiel] alle Potenzen von $\frac{1}{2}$ mit Exponenten von $0$ bis $8$. 
% \end{enumerate}
% \end{block}
%
\lang{de}{
Jede dieser Summen ist durch eine Regel beschrieben.
L\"asst sich eine solche Regel zudem als mathematische Formel darstellen, 
erhalten wir hiermit die folgende n\"utzliche Kurzform für die jeweilige Summe, die 
schlie\"slich die allgemeine Definition des \emph{Summenzeichens $\, \sum$\,} motiviert:
}
\lang{en}{
Each of these sums is described by a rule. As these rules happen to be expressible in 
mathematical notation, we obtain the following useful shorthand for each sum, which motivates 
the more general definition of the \emph{summation symbol $\, \sum$\,}.
}

\begin{enumerate}[alph]
    \item $\;
     \begin{mtable}[\cellaligns{lcl}]     
      \sum^{100}_{k=1}k &\coloneq& 1+2+3+\ldots+98+99+100 \\
       \end{mtable}
       $
      
\\    \item $\;
     \begin{mtable}[\cellaligns{lcl}]     
      \sum^{11}_{l=0}(2l+1) &\coloneq& 1+(2+1)+(2\cdot 2+1)+\ldots+(2\cdot 11+1) \\
         &=& 1+3+5+\ldots+23 \\
       \end{mtable}
       $
          
\\    \item $\;
     \begin{mtable}[\cellaligns{lcl}]
        %\sum^{8}_{j=0}\left(\frac{1}{2}\right)^j  &\coloneq&  \left(\frac{1}{2}\right)^0+
        %\left(\frac{1}{2}\right)^1+\left(\frac{1}{2}\right)^2+\ldots+\left(\frac{1}{2}\right)^8 \\
        %&=& 1+\frac{1}{2}+\frac{1}{4}+\ldots+\frac{1}{256} \\
       \end{mtable}
       $
\end{enumerate}

\end{example}

\begin{definition}[\lang{de}{Summenzeichen}\lang{en}{Summation symbol}] \label{def:summenzeichen}
 \lang{de}{F"ur $m,n \in \Nzero$ mit $\, m \leq n$ 
         % und beliebige Zahlen $\,x_m,x_{m+1}, \ldots,x_{n-1},x_n\,$
           definieren wir}
            
\lang{en}{For $m,n \in \Nzero$ with $m \leq n$,
        % and numbers \nowrap{$x_m,x_{m+1},\ldots,x_{n-1},x_n\,$}
          we define}
          
  \[\quad \sum^n_{k=m}{a_k}\,\coloneq\,a_m+a_{m+1}+\ldots+a_{n-1}+a_n. \]

\lang{de}{
Wir nennen $k$ den \notion{\emph{Summations-}} oder auch \emph{\notion{Laufindex}}, $m$ die 
\emph{\notion{untere}}, $n$ die \emph{\notion{obere Summationsgrenze}} und für 
$k=m, \ldots , n$ sind $a_k$ die \emph{\notion{Summanden.}}
}
\lang{en}{
We call $k$ the \notion{\emph{index}}, $m$ the \notion{\emph{lower bound}} and $n$ the 
\notion{\emph{upper bound}} of the sum. $a_k$ are the \emph{\notion{summands}} for $k=m, \ldots , n$.
}


\end{definition}

\begin{remark}
  \begin{itemize}[bullet]
  
  \item
  \lang{de}{$\,\Sigma\,$ ist der griechische Gro"sbuchstabe \emph{Sigma}, der dem Deutschen \emph{\glqq S\grqq} entspricht. Er wird symbolisch als 
    Abk"urzung f"ur $\Sigma\,umme$ verwendet.}
  \lang{en}{$\sum$ is the capital Greek letter Sigma (in English S), a shorthand for $\mbox{$\sum\text{um.}$}$}

\item 
    \lang{de}{Im Spezialfall $\,n=m\,$ gilt: $\quad \sum^m_{k=m}{a_k}=a_m.$}
    \lang{en}{In the special case of $\,n=m\,$ we have $\quad \sum^m_{k=m}{a_k}=a_m.$}
%   \item 
%   \lang{de}{F"ur $m > n$ definieren wir $\sum^n_{k=m}{x_k}\coloneq 0.$ }
%   \lang{en}{If $m > n$ one defines $\quad\sum^n_{k=m}{x_k}\coloneq 0.$}

\item 
    \lang{de}{Das Summenzeichen kann auch für $m > n$ definiert werden. In diesem Fall erhalten
    wir die \notion{leere Summe} (es ist nichts zu summieren) und der Wert der Summe ist definiert als $0$.}
    \lang{en}{The summation symbol can also be defined for $m > n$. In this case we obtain the 
    \notion{empty sum} (as there is nothing to sum) and we define the value of the sum to be $0$.}
 
  \item
    \lang{de}{Als Summationsindex kann, anstelle von $k, \,$ jeder beliebige Buchstabe verwendet werden.
     "Ublicherweise werden Buchstaben aus der Mitte des Alphabets gew"ahlt, wie z.\,B.
     $j,\;k,\;l,\;\ldots\;$ \\
     Entsprechend schreibt man}
    \lang{en}{Any letter may be used as the index of the sum, not just $k$. By convention, letters from the 
     middle of the alphabet are used, such as $j,\;k,\;l,\;\ldots\;$ \\
     Correspondingly we write}
    $\, \sum^n_{k=m}{a_k}=\sum^n_{j=m}{a_j}=\sum^n_{l=m}{a_l}=\ldots$
  
  \item
    \lang{de}{
    Die Summanden $a_k$ können Zahlen sein oder beliebige andere Objekte, die addiert werden können, z.\,B.
    }
    \lang{en}{
    The summands $a_k$ can be numbers or any other objects that can be summed, for example
    }
    \begin{itemize}
        \item  \lang{de}{Vektoren oder Matrizen einer festen Gr"o"se oder }
               \lang{en}{vectors or matrices of a given size}
        \item  \lang{de}{Funktionsterme in Abhängigkeit vom Summationsindex $k$.}
               \lang{en}{functions of the summation index $k$}
      \end{itemize}
          \\
          \lang{de}{
          In der Literatur findet man deshalb auch die Schreibweise $a(k)$ statt $a_k\;$ und
          }
          \lang{en}{
          Thanks to the final point, in the literature one also finds the notation $a(k)$ instead of $a_k\;$, and
          }
          \[ \sum^n_{k=m}{a(k)}=a(m)+a(m+1)+\ldots+a(n-1)+a(n) \]
%               Im konkreten Beispiel $\,a(k)\,\coloneq\,k^2:$     
%              \[\sum^n_{k=m}{k^2}=m^2+(m+1)^2+\;\ldots\;+(n-1)^2+n^2 \] 
             
%    \lang{en}{The \notion{summands} $x_k$ need not be numbers. They can be any objects which can be added like functions,
%    matrices of a given size etc.}

  \end{itemize}
\end{remark}

\begin{example}
    
  \begin{itemize}

    \item $\displaystyle \sum^n_{k=m}{k^2}=m^2+(m+1)^2+\;\ldots\;+(n-1)^2+n^2 \quad$  für $\; a(k)\,\coloneq\,k^2$     
\\
    \item $\displaystyle \sum^20_{j=18}{s}=s+s+s$ \\\\
          \lang{de}{
          Man beachte, dass hier der Funktionsterm $a(j)=s\,$ unabhängig von $j$ 
          und somit konstant vom Wert $s$ ist.
          }
          \lang{en}{
          Observe that here the function $a(j)=s\,$ is not dependent on $j$, and so is constant.
          } \\
\\
    \item  $\displaystyle \sum^2_{i=0}{\frac{i^3 \cdot (i+1)}{2}}=\frac{0^3 \cdot (0+1)}{2}+\frac{1^3 \cdot (1+1)}{2}+\frac{2^3 \cdot (2+1)}{2} $
\\
    \item  $\displaystyle \sum^5_{k=1}{\frac{(-1)^{k+1}}{k}}=\frac{1}{1}-\frac{1}{2}+\frac{1}{3}-\frac{1}{4}+\frac{1}{5} $
\\
    \item \lang{de}{
        Im einleitenden Beispiel \ref{ex:summen} lassen sich die Summanden wie folgt als 
        Funktionsterme darstellen:
        }
        \lang{en}{
        In example \ref{ex:summen}, the summands could be expressed as functions as follows:
        }
        \begin{enumerate}[alph]
          \item $a_k = a(k) \,\coloneq\, k$
          \item $a_l = a(l) \,\coloneq\,(2l-1)$
          \item $a_j = a(j) \,\coloneq\,\left(\frac{1}{2}\right)^j$ 
        \end{enumerate} 
\end{itemize}



\end{example}

\begin{quickcheck}
		\type{input.number}
		\begin{variables}
			\randint[Z]{a}{-1}{5}
			\randint[Z]{b}{-2}{2}
			\randint[Z]{j0}{1}{3}
			\randint{i}{3}{5}
			\function[calculate]{j1}{j0+i}
			\function[calculate]{jj}{j0+1}
		    \function{w1}{a*j0^b}
		    \function{w2}{a*jj^b}
		    \function{w3}{a*j1^b}
		\end{variables}
		
			\text{\lang{de}{Wofür steht die Summe}\lang{en}{Write out the sum} $\displaystyle \sum^{\var{j1}}_{j=\var{j0}} {\var{a}\cdot j^{\var{b}}}$?\\
				  \ansref $+$\ansref$+\ldots +$\ansref}
			
		\begin{answer}
			\solution{w1}
		\end{answer}
			
		\begin{answer}
			\solution{w2}
		\end{answer}
			
		\begin{answer}
			\solution{w3}
		\end{answer}
	\end{quickcheck}

\lang{de}{
Die verkürzte Darstellung einer Summe mit Hilfe des Summenzeichens ist in der Regel nicht eindeutig. 
Insbesondere kann es Unterschiede bei den Summationsgrenzen und bei der Verwendung des Summationsindex 
geben, die jedoch durch eine sogenannte \emph{Indexverschiebung} ausgleichbar sind.
}
\lang{en}{
The representation of a sum using the summation symbol is not unique. For example, the same 
sum can be expressed using different indices. We call this \emph{shifting the summation index}.
}

% \begin{tabs*}[\initialtab{0}]  \label{Indexverschiebung}
% \tab{\notion{Indexverschiebung}}
    
 \begin{algorithm}[\lang{de}{Indexverschiebung}\lang{en}{Shifting the index}] \label{Indexverschiebung}
 % myorange: \#D2691E 
  \lang{de}{
  Durch eine \emph{Indexverschiebung} kann man die Darstellung einer Summe verändern, ohne 
  dass sich der Wert der Summe verändert. 
  Hierbei werden die Summationsgrenzen um eine feste Zahl $\textcolor{\#D2691E}{+z} \; (z \in \Z)\,$ 
  verschoben, während gleichzeitig der Summationsindex um $\textcolor{\#D2691E}{-z} \,$ verändert wird, 
  was beim Ausschreiben der Summe die Verschiebung wieder aufhebt:
  }
  \lang{en}{
  By \emph{shifting the index} of a sum, we can change the way we write it without changing its value. 
  To do this, we shift the summation bounds by a fixed amount $\textcolor{\#D2691E}{+z} \; (z \in \Z)\,$, 
  whilst at the same time changing the summation index by $\textcolor{\#D2691E}{-z} \,$, which keeps the 
  summands the same:
  }

  \begin{alignat}{1}

  \sum^n_{k=m}{a(k)} &= \sum^{n \textcolor{red}{+z}}_{k=m \textcolor{\#D2691E}{+z}}{a(k\textcolor{\#D2691E}{-z})}\\
  &\\
  a(m)+a(m+1)+  \ldots + a(n) &= 
  a((m \textcolor{\#D2691E}{+z})\textcolor{\#D2691E}{-z})+a((m+1 \textcolor{\#D2691E}{+z})\textcolor{\#D2691E}{-z})
          +  \ldots + a((n \textcolor{\#D2691E}{+z})\textcolor{\#D2691E}{-z}) 
  \end{alignat}
 \end{algorithm}
 %Could be typeset nicer? - Niccolo

\\ \textbf{\lang{de}{Beispiel:}\lang{en}{Example:}}  
 $\quad \sum^4_{k=0}{2^{k}} =\sum^{4\textcolor{\#D2691E}{+1}}_{k=0\textcolor{\#D2691E}{+1}}{2^{k\textcolor{\#D2691E}{-1}}} 
 = \sum^5_{k=1}{2^{k-1}}$ 
 
 % = 2^0 + 2^1 + 2^2 + 2^3 + 2^4$
 
% \end{tabs*}


\\

\lang{de}{
Ähnlich wie für Summen gibt es auch für Produkte eine abkürzende Produktschreibweise, 
die analog zur Summenschreibweise funktioniert. Der Unterschied ist, dass, anstatt zu summieren, 
multipliziert wird und anstelle des Sigmazeichens $\, \Sigma\,$ ein großes Pi $\, \Pi\,$ als symbolische
Abkürzung für das $\, \Pi\,rodukt$ verwendet wird.
}
\lang{en}{
There is also an efficient way to express products of numbers, which works analogously to the summation 
symbol. The difference is that it denotes a product rather than a sum, and that its symbol is not an uppercase sigma 
$\, \Sigma,$ but an uppercase pi $\, \Pi\,$, for $\, \Pi$roduct.
}


 \begin{definition}[\lang{de}{Produktzeichen}\lang{en}{Product symbol}] \label{def:produktzeichen}
  \lang{de}{Es seien $m,n \in \Nzero$ mit $ m \leq n$. 
%           und  $a_m,a_{m+1},\ldots,a_{n-1},a_n$ beliebige Objekte, die multiplizierbar sind.
           \\
  Wir definieren} 
  \lang{en}{Consider $m,n \in \Nzero $ with $ m \leq n$ 
  %         and numbers  $a_m,a_{m+1},\ldots,a_{n-1},a_n$.
  \\
  We define:}
  \[\quad \prod^n_{k=m}{a_k}\coloneq a_m\cdot a_{m+1}\cdot a_{m+2}\cdot\ldots \cdot a_{n-1}\cdot a_n.\]
  
\end{definition}


\begin{remark}
  \begin{itemize}
    \item \lang{de}{
          Wir nennen auch hier $k$ den \notion{Laufindex}, $m$ die \notion{untere}, $n$ die 
          \notion{obere Grenze} und $a_k$ (oder auch $a(k)$) ist ein Funktionsterm bezüglich 
          der Laufvariablen $k$.
          }
          \lang{en}{
          We again call $k$ the \notion{index}, $m$ the \notion{lower bound} and $n$ the \notion{upper bound}. 
          $a_k$ (or $a(k)$) is a function in the index $k$.
          } \\    
    
    \item \lang{de}{
          Für $m=n$ gilt: $\quad \prod^m_{k=m}{a_k}=a_m.$ 
          }
          \lang{en}{
          For $m=n$ we have: $\quad \prod^m_{k=m}{a_k}=a_m.$
          } \\
    
    \item \lang{de}{
          Für $n<m$ bezeichnet man das Produkt als das \notion{leere Produkt} und definiert  
          $\quad\prod^n_{k=m}{a_k}\coloneq 1.$ 
          }
          \lang{en}{
          If $n<m$ we call the product the \notion{empty product} and define 
          $\quad\prod^n_{k=m}{a_k}\coloneq 1.$ 
          } \\
    
  \end{itemize}
\end{remark}
  
\begin{example} \label{ex:produktzeichen}
  \begin{itemize}

    \item $\displaystyle \prod^5_{k=1}{k^2}=1^2\cdot 2^2\cdot 3^2 \cdot 4^2\cdot 5^2 \;$   \\\\    

    \item $\displaystyle \prod^20_{j=18}{p}=p\cdot p\cdot p=p^3 $ \\\\     

    \item  $\displaystyle \prod^2_{i=0}{\frac{i^3 \cdot (i+1)}{2}}=\frac{0^3 \cdot (0+1)}{2}\cdot \frac{1^3 \cdot (1+1)}{2}\cdot \frac{2^3 \cdot (2+1)}{2}=\frac{0}{2}\cdot \frac{2}{2}\cdot \frac{24}{2}$ \\\\

    \item  $\displaystyle \prod^5_{k=1}{\frac{(-1)^{k+1}}{k}}=\frac{1}{1}\cdot \frac{(-1)}{2}\cdot \frac{1}{3}\cdot \frac{(-1)}{4}\cdot \frac{1}{5}$ \\

  \end{itemize}
\end{example}

%%% Video K.M.
%
\lang{de}{
Weitere Beschreibungen und Beispiele zum Summen- und Produktzeichen finden Sie im folgenden Video. 
Dabei sei darauf hingewiesen, dass die Definition von Summen- und Produktzeichen im Video 
allgemeiner gehalten ist, als die Definitionen \ref{def:summenzeichen} und \ref{def:produktzeichen}. 
Anstelle einer kontinuierlichen Abfolge natürlicher Zahlen für den Laufindex wird in der Definition im Video
eine sogenannte \emph{endliche Indexmenge} zum Durchzählen der Summanden bzw. Faktoren verwendet.   
\\
Man schreibt zum Beispiel $\displaystyle \sum_{k \in \{2;4;5\}}{k^2}=2^2+4^2+5^2 = 45. \,$   
Entspricht die endliche Indexmenge einer kontinuierlichen Abfolge natürlicher Zahlen, dann 
gilt wiederum
\\
$\displaystyle \sum_{k \in \{2;3;4;5\}}{k^2}=\sum^5_{k=2}{k^2}.$ 
\floatright{\href{https://api.stream24.net/vod/getVideo.php?id=10962-2-10951&mode=iframe&speed=true}
{\image[75]{00_video_button_schwarz-blau}}}\\
\\
}
\lang{en}{
Note that as long as the meaning is clear, it is also acceptable to sum over a finite set such as 
$\displaystyle \sum_{k \in \{2;4;5\}}{k^2}=2^2+4^2+5^2 = 45\;$ or 
$\;\displaystyle \sum_{k \in \{2;3;4;5\}}{k^2}=\sum^5_{k=2}{k^2}$. 
The same applies to the product notation.
}
  
\end{visualizationwrapper}

\end{content}