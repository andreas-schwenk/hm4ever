\documentclass{mumie.problem.gwtmathlet}
%$Id$
\begin{metainfo}
 \name{
  \lang{de}{A06: Bruchrechnung}
  }
  \begin{description} 
 This work is licensed under the Creative Commons License Attribution 4.0 International (CC-BY 4.0)   
 https://creativecommons.org/licenses/by/4.0/legalcode 

    \lang{de}{...}
  \end{description}
  \corrector{system/problem/GenericCorrector.meta.xml}
  \begin{components}
    \component{js_lib}{system/problem/GenericMathlet.meta.xml}{gwtmathlet}
  \end{components}
  \begin{links}
  \end{links}
  \creategeneric
\end{metainfo}

\begin{content}

\begin{block}[annotation]
	Im Ticket-System: \href{https://team.mumie.net/issues/22291}{Ticket 22291}
\end{block}

\begin{block}[annotation]
	Rechnen mit Brüchen
\end{block}

\usepackage{mumie.genericproblem}

\lang{de}{
	\title{A06: Bruchrechnung}
}

\begin{problem}
%
    \randomquestionpool{1}{2}       % Doppelbrüche
    \randomquestionpool{3}{6}       % Addition/Subtraktion
    \randomquestionpool{7}{8}       % Ordnungen
    \randomquestionpool{9}{11}      % Division

%
%%%%%%%%%%%%%%%%%%%%%%%%%%%%%%%%%%%%%%%%%%%%%%%%%%%%%%%%%%%%%%%%%%%%%%%%%%%%%%%%%%%%%%%%%%%%%%%%%%%%%%%
% Q1 Doppelbrüche
	\begin{question}

			\text{Berechnen Sie die folgenden Doppelbrüche und kürzen Sie ihr Ergebnis vollständig. \\
			$\frac{\var{fao}}{\var{fau}} = $\ansref, $\qquad \frac{\var{fco}}{\var{fcu}} = $\ansref. 
			}
			
%			\explanation{Erklärung}
				
		\type{input.number} 
        \field{rational}
		
		\begin{variables}
			
			\randint{a}{1}{20}
			\randint{b}{1}{20}
			\randint{c}{1}{10}
			\randint{d}{1}{9}
			\randint{m}{3}{6}
			\randint{n}{2}{3}
			\randint{p}{2}{5}
			\randint{q}{2}{5}
			\function[normalize]{fa}{a*c / d}
			\function[normalize]{fb}{(a/b)/(n*a/b)}
			\function[normalize]{fc}{ ((m*a)/b) / (a/b) }
			\function[normalize]{fd}{((a)/(n*d)) / ((q*a)/(p*d)) }
			
			\function[normalize]{fao}{a*c}
			\function[normalize]{fbo}{a/b}
			\function[normalize]{fco}{ m*a/b }
			\function[normalize]{fdo}{a/(n*d) }
			
			\function[normalize]{fau}{ d}
			\function[normalize]{fbu}{n*a/b}
			\function[normalize]{fcu}{ a/b }
			\function[normalize]{fdu}{(q*a)/(p*d) }
			
			\randadjustIf{a,b}{a=b}
			\randadjustIf{c,d}{c=d}
			
		\end{variables}
		
		 
		\begin{answer}
%			\Large\text{$\frac{\var{fao}}{\var{fau}} = $}
			\solution{fa}
		\end{answer}
		
		\begin{answer}
%			\text{$\frac{\var{fco}}{\var{fcu}} = $}
			\solution{fc}
		\end{answer}
		
	\end{question}

% Q2 Doppelbrüche
	\begin{question}

			\text{Berechnen Sie die folgenden Doppelbrüche und kürzen Sie ihr Ergebnis vollständig. \\
			$\frac{\var{fbo}}{\var{fbu}} = $\ansref, $\qquad \frac{\var{fdo}}{\var{fdu}} = $\ansref. 
			}
			
%			\explanation{Erklärung}
				
		\type{input.number} 
        \field{rational}
		
		\begin{variables}
			
			\randint{a}{1}{20}
			\randint{b}{1}{20}
			\randint{c}{1}{10}
			\randint{d}{1}{9}
			\randint{m}{3}{6}
			\randint{n}{2}{3}
			\randint{p}{2}{5}
			\randint{q}{2}{5}
			\function[normalize]{fa}{a*c / d}
			\function[normalize]{fb}{(a/b)/(n*a/b)}
			\function[normalize]{fc}{ ((m*a)/b) / (a/b) }
			\function[normalize]{fd}{((a)/(n*d)) / ((q*a)/(p*d)) }
			
			\function[normalize]{fao}{a*c}
			\function[normalize]{fbo}{a/b}
			\function[normalize]{fco}{ m*a/b }
			\function[normalize]{fdo}{a/(n*d) }
			
			\function[normalize]{fau}{ d}
			\function[normalize]{fbu}{n*a/b}
			\function[normalize]{fcu}{ a/b }
			\function[normalize]{fdu}{(q*a)/(p*d) }
			
			\randadjustIf{a,b}{a=b}
			\randadjustIf{c,d}{c=d}
			
		\end{variables}
		
		 
		\begin{answer}
%			\text{$\frac{\var{fbo}}{\var{fbu}} = $}
			\solution{fb}
		\end{answer}
		
		\begin{answer}
%			\text{$\frac{\var{fdo}}{\var{fdu}} = $}
			\solution{fd}
		\end{answer}
		
		
	\end{question}
%
%
%%%%%%%%%%%%%%%%%%%%%%%%%%%%%%%%%%%%%%%%%%%%%%%%%%%%%%%%%%%%%%%%%%%%%%%%%%%%%%%%%%%%%%%%%%%%%%%%%%%%%%%
% Q1 Addition/Subtraktion
    \begin{question}
      \text{Bestimmen Sie $x$, so dass 
      $\frac{\var{a1}}{\var{b1}}+\frac{\var{a2}}{\var{b2}}=\frac{x}{\var{ee}}$ erfüllt ist.} 

      \explanation{}

      \type{input.number} 
      \precision{3}
      \field{real}

      \begin{variables}
        \randint[Z]{a1}{1}{3}
        \randint[Z]{a2}{1}{3}
        \randadjustIf{a2}{a1=a2}
        \randint[Z]{b}{1}{4}
        \drawFromSet{p1}{1,3,5}
        \drawFromSet{p2}{2,7}
        
%        \function[calculate]{d}{2-a}
        \function[calculate]{d}{a1*p2+a2*p1}
        \function[calculate]{ee}{p1*p2*b}
        \function[calculate]{x}{d}
%        \function[calculate]{a1}{a+1}
        \function[calculate]{b1}{p1*b}
        \function[calculate]{b2}{p2*b}
      \end{variables}

      \begin{answer}
        \text{$x=$}
        \solution{x}
      \end{answer}
    \end{question}
%    
% Q2 Addition/Subtraktion
    \begin{question}
      \text{Bestimmen Sie $z$, so dass 
      $\frac{\var{a1}}{\var{b1}}-\frac{\var{a2}}{\var{b2}}=\frac{z}{\var{ee}}$ erfüllt ist.} 

      \explanation{}

      \type{input.number} 
      \precision{3}
      \field{real}

      \begin{variables}
        \randint[Z]{a1}{1}{3}
        \randint[Z]{a2}{1}{3}
        \randadjustIf{a2}{a1=a2}
        \randint[Z]{b}{1}{4}
        \drawFromSet{p1}{1,3,5}
        \drawFromSet{p2}{2,7}
       
%        \function[calculate]{d}{2-a}
        \function[calculate]{d}{a1*p2-a2*p1}
        \function[calculate]{ee}{p1*p2*b}
        \function[calculate]{z}{d}
%        \function[calculate]{a1}{a+1}
        \function[calculate]{b1}{p1*b}
        \function[calculate]{b2}{p2*b}
      \end{variables}

      \begin{answer}
        \text{$z=$}
        \solution{z}
      \end{answer}
    \end{question}
%    
% Q3 Addition/Subtraktion
    \begin{question}
      \text{Finden Sie $x$ mit  
      $\frac{\frac{\var{a}}{\var{b}}}{\frac{\var{c}}{\var{d}}}+\frac{\var{f}}{\var{e}}=\frac{x}{\var{e}}$.}

      \explanation{}

      \type{input.number} 
      \precision{3}
      \field{real}

      \begin{variables}
        \number{a}{3}
        \number{b}{4}
        \number{c}{2}
        \number{d}{3}
        \number{e}{8}
        \randint[Z]{f}{1}{4}
        \function[calculate]{x}{9+f}
      \end{variables}

      \begin{answer}
        \text{$x =$ }
        \solution{x}
      \end{answer}
    \end{question}
%    
% Q4 Addition/Subtraktion
    \begin{question}
      \text{Bestimmen Sie $x$, so dass 
      $\frac{\var{a}}{3}+\frac{\var{b}}{5}-\frac{\var{c}}{6}=\frac{x}{\var{d}}$ erfüllt ist.}

      \explanation{}

      \type{input.number} 
      \precision{3}
      \field{real}

      \begin{variables}
        \number{a}{5}
        \randint[Z]{b}{1}{4}
        \randint[Z]{c}{1}{5}
        \number{d}{30}
        \function[calculate]{x}{(a*10)+(6*b)-(c*5)}
      \end{variables}

      \begin{pool}
        \begin{variables}
          \number{a}{1}
        \end{variables}
        \begin{variables}
          \number{a}{2}
        \end{variables}
      \end{pool}

      \begin{answer}
        \text{$x =$ }
        \solution{x}
      \end{answer}
    \end{question}

%
%%%%%%%%%%%%%%%%%%%%%%%%%%%%%%%%%%%%%%%%%%%%%%%%%%%%%%%%%%%%%%%%%%%%%%%%%%%%%%%%%%%%%%%%%%%%%%%%%%%%%%%
% Q1 Ordnungen
	\begin{question}

			\text{Ordnen Sie die Brüche $\, \frac{\var{n2}}{\var{p2}}, \, \frac{\var{n3}}{\var{p3}} \,$ 
                und $\, \, \frac{\var{n1}}{\var{p1}} \,$ der Größe nach. Kürzen Sie die Brüche vor der Eingabe
                soweit wie möglich.\\
            \\
            
			$\qquad $\ansref $\; \leq \; $\ansref $\; \leq \; $\ansref. 
			}
			
			\explanation{Bringen Sie die Brüche zunächst auf den Hauptnenner, um sie dann 
                         anhand der Zähler zu ordnen.}
				
        \type{input.number}
        \field{rational}
		
		\begin{variables}
			
			\randint[Z]{k}{-1}{1}
            \drawFromSet{p1}{1,2,3}
            \drawFromSet{p2}{5,7}
            \randint{p3}{7}{11}
            
            \function[calculate]{p4}{p1*p2}
            \randadjustIf{p3}{p3=p1 OR p3=p2 OR p3=p4}

            \randint{n0}{0}{4}
			\randint{n1}{5}{10}
            
            \function[calculate, normalize]{n2}{n1-n0}
			\randint{n3}{1}{10}
            \randadjustIf{n3}{n3>n2}
            
            \function[calculate]{hn}{p1*p2*p3}
			\function[calculate]{z1}{n1*p2*p3}          % da p1 < p2 und n2 <= n1 ist n2*p1 < n1*p2
			\function[calculate]{z2}{n2*p1*p3}          % => z2 < z1
			\function[calculate]{z3}{n3*p1*p2}          % da n3 <= n2 und p3/p2 >= 1 ist n3*p2 <= n2*p3
                                                        % => z3 <= z2
			\function[normalize]{f1}{n1/p1}
			\function[normalize]{f2}{n2/p2}
			\function[normalize]{f3}{n3/p3}
						
		\end{variables}
		 
		\begin{answer}
			\solution{f3}
		\end{answer}

		\begin{answer}
			\solution{f2}
		\end{answer}
        
		\begin{answer}
			\solution{f1}
		\end{answer}
		
	\end{question}

% Q2 Ordnungen
	\begin{question}

			\text{Ordnen Sie die Brüche $\, -\frac{\var{n2}}{\var{p2}}, \, -\frac{\var{n3}}{\var{p3}} \,$ 
                und $\, \, -\frac{\var{n1}}{\var{p1}} \,$ der Größe nach. Kürzen Sie die Brüche vor der Eingabe
                soweit wie möglich.\\

            \\
			$\qquad $\ansref $\; \geq \; $\ansref $\; \geq \; $\ansref. 
			}
			
			\explanation{Bringen Sie die Brüche zunächst auf den Hauptnenner, um sie dann 
                         anhand der Zähler zu ordnen.}
				
        \type{input.number}
        \field{rational}
		
		\begin{variables}
			
			\randint[Z]{k}{-1}{1}
            \drawFromSet{p1}{1,2,3}
            \drawFromSet{p2}{5,7}
            \randint{p3}{7}{11}
            
            \function[calculate]{p4}{p1*p2}
            \randadjustIf{p3}{p3=p1 OR p3=p2 OR p3=p4}

            \randint{n0}{0}{4}
			\randint{n1}{5}{10}
            
            \function[calculate, normalize]{n2}{n1-n0}
			\randint{n3}{1}{10}
            \randadjustIf{n3}{n3>n2}
            
            \function[calculate]{hn}{p1*p2*p3}
			\function[calculate]{z1}{n1*p2*p3}          % da p1 < p2 und n2 <= n1 ist n2*p1 < n1*p2
			\function[calculate]{z2}{n2*p1*p3}          % => z2 < z1
			\function[calculate]{z3}{n3*p1*p2}          % da n3 <= n2 und p3/p2 >= 1 ist n3*p2 <= n2*p3
                                                        % => z3 <= z2
			\function[normalize]{f1}{-n1/p1}
			\function[normalize]{f2}{-n2/p2}
			\function[normalize]{f3}{-n3/p3}
						
		\end{variables}
		 
		\begin{answer}
			\solution{f3}
		\end{answer}

		\begin{answer}
			\solution{f2}
		\end{answer}
        
		\begin{answer}
			\solution{f1}
		\end{answer}
		
	\end{question}
%
%
%%%%%%%%%%%%%%%%%%%%%%%%%%%%%%%%%%%%%%%%%%%%%%%%%%%%%%%%%%%%%%%%%%%%%%%%%%%%%%%%%%%%%%%%%%%%%%%%%%%%%%%
% Q1 Division
    \begin{question}
      \text{Finden Sie  $x$ mit 
      $\frac{\var{a}}{\var{b}}:   \frac{\var{d}}{\var{c}}=\frac{\var{ee}}{x}$.}

      \explanation{}

      \type{input.number} 
      \precision{3}
      \field{real}

      \begin{variables}
        \randint[Z]{a}{1}{20}
        \randint[Z]{b}{3}{40}
        \randadjustIf{a,b}{a=b}
        \randint[Z]{c}{2}{20}
        \randint[Z]{d}{2}{40}
        \randadjustIf{c,d}{c=d}
        \function[calculate]{f}{b*d}
        \function[calculate]{ee}{a*c}
      \end{variables}

      \begin{answer}
        \text{$x =$ }
        \solution{f}
      \end{answer}
    \end{question}
    
% Q2 Division    
    \begin{question}
      \text{Bestimmen Sie $a$, so dass 
      $\left(\frac{\var{a}}{\var{b1}}-\frac{\var{a1}}{\var{b2}}\right)
      :   \frac{\var{d1}}{\var{b3}}
      =\frac{a}{\var{d1}}$ erfüllt ist.}
      
      \explanation{}

      \type{input.number} 
      \precision{3}
      \field{real}

      \begin{variables}
        \randint[Z]{a}{1}{2}
        \randint[Z]{b}{1}{7}
        \randadjustIf{a,b}{a=b}
        \function[calculate]{d}{(8*a+3)*(b+1)}
        \function[calculate]{d1}{3-a}
        \function[calculate]{ee}{a-2}
        \function[calculate]{x}{d}
        \function[calculate]{a1}{a+1}
        \function[calculate]{b1}{2*b}
        \function[calculate]{b2}{3*b}
        \function[calculate]{b3}{6*b}
      \end{variables}

      \begin{answer}
        \text{$a =$ }
        \solution{ee}
      \end{answer}
    \end{question}

% Q3 Division
    \begin{question}
      \text{Finden Sie $x$, so dass 
      $\left(\frac{\var{a}}{\var{b}}:   \frac{\var{d}}{\var{d1}}\right)
      \cdot \frac{\var{b1}}{\var{a1}}
      =\frac{x}{\var{d3}}$ erfüllt ist.}
      
      \explanation{}

      \type{input.number} 
      \precision{3}
      \field{real}

      \begin{variables}
        \randint[Z]{a}{2}{6}
        \randint[Z]{b}{7}{15}
        \randint[Z]{d}{1}{7}
        \function[calculate]{a1}{3*a}
        \function[calculate]{d1}{d+1}
        \function[calculate]{d2}{2*(d+1)}
        \function[calculate]{d3}{3*d}
        \function[calculate]{b1}{2*b}
      \end{variables}

      \begin{answer}
        \text{$x =$ }
        \solution{d2}
      \end{answer}
    \end{question}



%
%
\end{problem}

\embedmathlet{gwtmathlet}

\end{content}
