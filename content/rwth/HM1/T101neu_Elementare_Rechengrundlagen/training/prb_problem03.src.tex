\documentclass{mumie.problem.gwtmathlet}
%$Id$
\begin{metainfo}
 \name{
  \lang{de}{A03: Intervalle}
  }
  \begin{description} 
 This work is licensed under the Creative Commons License Attribution 4.0 International (CC-BY 4.0)   
 https://creativecommons.org/licenses/by/4.0/legalcode 

    \lang{de}{...}
  \end{description}
  \corrector{system/problem/GenericCorrector.meta.xml}
  \begin{components}
    \component{js_lib}{system/problem/GenericMathlet.meta.xml}{gwtmathlet}
  \end{components}
  \begin{links}
  \end{links}
  \creategeneric
\end{metainfo}

\begin{content}

\begin{block}[annotation]
	Im Ticket-System: \href{https://team.mumie.net/issues/22285}{Ticket 22285}
\end{block}

\begin{block}[annotation]
	Intervalle, Ordnungen, Brüche, Beträge
\end{block}

\usepackage{mumie.genericproblem}

\lang{de}{
	\title{A03: Intervalle}
}

\begin{problem}
%
  \randomquestionpool{1}{1}
  \randomquestionpool{2}{4}
  \permutequestions

% Q1
  \begin{question}
    \text{In welchem der folgenden Intervalle liegt die Zahl $\var{a}$? \\
        W"ahlen Sie alle richtigen Antworten aus.}

    \explanation{Sie haben nicht alle Intervalle ausgewählt, die $\var{a}$ enthalten.}

    \permutechoices{1}{3}

    \type{mc.multiple}
    \field{rational}

    \begin{variables}
      %\randrat{a}{9}{20}{2}{6}
      \randint{a1}{9}{20}
      \randint{a2}{2}{6}
      \randint{c}{3}{7}
      \randint{d}{-2}{4}
      \randint{f}{0}{1}
      \function[calculate]{l1}{floor(a-c)}
      \function[calculate]{r1}{floor(a+d)}
      \function[calculate]{l2}{floor(a)-f+1}
      \function[calculate]{r2}{floor(a)+c}
      \function[calculate]{l3}{floor(a)-2}
      \function[calculate]{r3}{floor(a+d+0.5)+1}
      \function[normalize]{a}{a1/a2}
      \function[calculate]{t}{floor(a)}
      \randadjustIf{a2}{a=t}                        % make sure that 'a' is not an integer
      \randadjustIf{c,d}{a>r1 AND a<=l2 AND a>=r3}  % make sure there is a true solution
    \end{variables}

    \begin{choice}
      \text{$[\var{l1};\var{r1}]$}
      \solution{compute}
    \iscorrect{a}{<=}{r1}
    \end{choice}
    \begin{choice}
      \text{$(\var{l2};\var{r2}]$}
      \solution{compute}
      \iscorrect{a}{>}{l2}
    \end{choice}
    \begin{choice}
      \text{$[\var{l3};\var{r3})$}
      \solution{compute}
      \iscorrect{a}{<}{r3}
    \end{choice}
  \end{question}
%
% Q2
%
  \begin{question}
  
    \type{input.interval}
    \field{rational}

    \text{Schreiben Sie die folgende Menge in Intervallschreibweise 
        bzw. als Vereinigung von Intervallen und prüfen Sie, ob es sich 
        dabei um offene, halboffene oder geschlossene Intervalle handelt. 
        Ändern Sie diese ggf. durch anklicken.
        (Schreiben Sie "`\textit{infinity}"' für $\infty$.) \\
        }
%        
    \begin{variables}
      \randint{a1}{1}{6}
      \randint{a2}{1}{6}
      \function[normalize]{a}{(a1/a2)^2}

      \randint{b1}{1}{10}
      \randint{b2}{1}{10}
      \function[normalize]{b}{b1/b2}     
      \function{f}{(x-b)^2}

      \randint{c1}{1}{10}
      \randint{c2}{1}{10}
      \function[normalize]{c}{(c1/c2)^2}
      
      \function[normalize]{l1}{-c1/c2}
      \function[normalize]{r1}{c1/c2}

      \function[normalize]{l2}{-(a1/a2)+b}
      \function[normalize]{r2}{(a1/a2)+b}
    \end{variables}
%
    \begin{answer}
          \text{$\{x \in \R \, \vert \, x^2 \geq \var{c}\}=$}    
          \allowIntervalUnionsForInput
          \solution{(-infinity;l1],[r1;infinity)}
          \explanation{Prüfen Sie die Intervallgrenzen und beachten Sie, 
                        dass $\,\sqrt{x^2}=\abs{x}\,$ ist.}
    \end{answer}
\\

    \begin{answer}
          \text{$\{x \in \R \, \vert \, \var{f} < \var{a}\}=$}
          \allowIntervalUnionsForInput
          \solution{(l2;r2)}
          \explanation{Prüfen Sie die Intervallgrenzen und beachten Sie, 
                       dass $\,\sqrt{\var{f}}=\abs{x-\var{b}}\,$ ist.}
    \end{answer}
%    

  \end{question}
%
%
% Q3
  \begin{question}
  
    \type{input.interval}
    \field{rational}

    \text{Schreiben Sie die folgende Menge in Intervallschreibweise 
        bzw. als Vereinigung von Intervallen und prüfen Sie, ob es sich 
        dabei um offene, halboffene oder geschlossene Intervalle handelt. 
        Ändern Sie diese ggf. durch anklicken.
        (Schreiben Sie "`\textit{infinity}"' für $\infty$.) \\
        }
%
    \begin{variables}
      \randint{a1}{1}{6}
      \randint{a2}{1}{6}
      \function[normalize]{a}{(a1/a2)^2}

      \randint{b1}{1}{10}
      \randint{b2}{1}{10}
      \function[normalize]{b}{b1/b2}     
      \function{f}{(x+b)^2}

      \randint{c1}{1}{10}
      \randint{c2}{1}{10}
      \function[normalize]{c}{(c1/c2)^2}
      
      \function[normalize]{l1}{-c1/c2}
      \function[normalize]{r1}{c1/c2}

      \function[normalize]{l2}{-(a1/a2)-b}
      \function[normalize]{r2}{(a1/a2)-b}
    \end{variables}
%
    \begin{answer}
          \text{$\{x \in \R \, \vert \, x^2 \leq \var{c}\}=$}    
          \allowIntervalUnionsForInput
          \solution{[l1;r1]}
          \explanation{Prüfen Sie die Intervallgrenzen und beachten Sie, 
                        dass $\,\sqrt{x^2}=\abs{x}\,$ ist.}
    \end{answer}
\\

    \begin{answer}
          \text{$\{x \in \R \, \vert \, \var{f} > \var{a}\}=$}
          \allowIntervalUnionsForInput
          \solution{(-infinity;l2),(r2;infinity)}
          \explanation{Prüfen Sie die Intervallgrenzen und beachten Sie, 
                       dass $\,\sqrt{\var{f}}=\abs{x+\var{b}}\,$ ist.}
    \end{answer}
%    
  \end{question}
%
% Q4
%
  \begin{question}
  
    \type{input.interval}
    \field{rational}
%
    \text{Schreiben Sie die folgende Menge in Intervallschreibweise 
        bzw. als Vereinigung von Intervallen und prüfen Sie, ob es sich 
        dabei um offene, halboffene oder geschlossene Intervalle handelt. 
        Ändern Sie diese ggf. durch anklicken.
        (Schreiben Sie "`\textit{infinity}"' für $\infty$.) \\
        }
%
    \begin{variables}
      \randint{a1}{1}{6}
      \randint{a2}{1}{6}
      \function[normalize]{a}{(a1/a2)^2}

      \randint{b1}{1}{10}
      \randint{b2}{1}{10}
      \function[normalize]{b}{b1/b2}     
      \function{f}{(x-b)^2}

      \randint{c1}{1}{10}
      \randint{c2}{1}{10}
      \function[normalize]{c}{(c1/c2)^2}
      
      \function[normalize]{l1}{-c1/c2}
      \function[normalize]{r1}{c1/c2}

      \function[normalize]{l2}{-(a1/a2)+b}
      \function[normalize]{r2}{(a1/a2)+b}
    \end{variables}
%
    \begin{answer}
          \text{$\{x \in \R \, \vert \, x^2 < \var{c}\}=$ }    
          \allowIntervalUnionsForInput
          \solution{(l1;r1)}
          \explanation{Prüfen Sie die Intervallgrenzen und beachten Sie, 
                        dass $\,\sqrt{x^2}=\abs{x}\,$ ist.}
    \end{answer}
\\

    \begin{answer}
          \text{$\{x \in \R \, \vert \, \var{f} \leq \var{a}\}=$}
          \allowIntervalUnionsForInput
          \solution{[l2;r2]}
          \explanation{Prüfen Sie die Intervallgrenzen und beachten Sie, 
                       dass $\,\sqrt{\var{f}}=\abs{x-\var{b}}\,$ ist.}
    \end{answer}
%    
  \end{question}

%
%
\end{problem}

\embedmathlet{gwtmathlet}

\end{content}
