\documentclass{mumie.problem.gwtmathlet}
%$Id$
\begin{metainfo}
 \name{
  \lang{de}{A13: Lineare Gleichungssysteme}
  }
  \begin{description} 
 This work is licensed under the Creative Commons License Attribution 4.0 International (CC-BY 4.0)   
 https://creativecommons.org/licenses/by/4.0/legalcode 

    \lang{de}{...}
  \end{description}
  \corrector{system/problem/GenericCorrector.meta.xml}
  \begin{components}
    \component{js_lib}{system/problem/GenericMathlet.meta.xml}{gwtmathlet}
  \end{components}
  \begin{links}
  \end{links}
  \creategeneric
\end{metainfo}

\begin{content}

\begin{block}[annotation]
	Im Ticket-System: \href{https://team.mumie.net/issues/22338}{Ticket 22338}
\end{block}

\begin{block}[annotation]
	Lösen lineare Gleichungssysteme unter Verwendung des Additions- oder des Einsetzungsverfahrens
\end{block}

\usepackage{mumie.genericproblem}

\lang{de}{
	\title{A13: Lineare Gleichungssysteme}
}

\begin{problem}

	\randomquestionpool{1}{3}
%
% Q1 - lgs2
%		
	\begin{question}
		\text{Lösen Sie das folgende lineare Gleichungssystems nach einem der bekannten Verfahren.\\
              \\
              \begin{displaymath}
              \begin{mtable}[\cellaligns{crcrcr}]
                \text{(I)} &\qquad \var{a}x&+&\var{b}y&=&\var{r1}\\
                \text{(II)}&\qquad        x&-&\var{md}y&=&\var{r2}
              \end{mtable}
              \end{displaymath}
              \\
            }
        
        \explanation{Verwenden Sie das \textit{Einsetzungsverfahren}, indem Sie die Gleichung (I) 
                    nach $x$ auflösen und anschließend in (II) einsetzen. Alternativ können Sie auch
                    das \textit{Additionsverfahren} anwenden, indem Sie die Gleichung (II) mit 
                    $\var{ma}$ multiplizieren und anschließend (I) und (II) addieren.  
                    In beiden Fällen kommt in der resultierenden Gleichung $x$ nicht mehr vor.
                    }
                    
        \type{input.number}
       
        \begin{variables}
            \randint[Z]{a}{-10}{10}
            \randint[Z]{b}{0}{10}
            \randint[Z]{d}{-10}{0}
            \randadjustIf{a,b,d}{a*d-b=0 OR a*a=1 OR b*b=1 OR d*d=1 OR abs(b)=abs(d)}
            
            \function[calculate]{ma}{-a}
            \function[calculate]{md}{-d}
            
            \randint{s}{-5}{5}
            \randint{t}{-5}{5}
            
            \function[calculate]{r1}{a*s+b*t} 
            \function[calculate]{r2}{s+d*t}
            
        \end{variables}

        \begin{answer}
              \text{$x=$}
              \solution{s}
        \end{answer}
        
        \begin{answer}
        	  \text{$y=$}
              \solution{t} 
        \end{answer}
        
    \end{question}
%
% Q2 - lgs2
%
	\begin{question}
    
		\text{Lösen Sie das folgende lineare Gleichungssystems nach einem der bekannten Verfahren.\\
              \\
              \begin{displaymath}
              \begin{mtable}[\cellaligns{crcrcr}]
                \text{(I)} &\qquad \var{a}x&-&\var{mb}y&=&\var{r1}\\
                \text{(II)}&\qquad \var{c}x&+&\var{d}y&=&\var{r2}
              \end{mtable}
              \end{displaymath}
              \\
            }
        
        \explanation{Am einfachsten multiplizieren Sie die Gleichung (I) mit $\var{q}$ und addieren 
                    anschließend die beiden Gleichungen gemäß dem \textit{Additionsverfahren.} 
                    In der resultierenden Gleichung kommt dann $y$ nicht mehr vor.
                    }
    
        \type{input.number}
       
        \begin{variables}
            \randint[Z]{a}{-10}{10}
            \randint[Z]{b}{-5}{0}
            \randint[Z]{c}{-10}{10}
            \randint[Z]{q}{2}{3}
            \randadjustIf{a,b,c,q}{-a*q*b-b*c=0 OR a*a=1 OR c*c=1 OR b*b=1 OR abs(a)=abs(c)}
            
            
            \function[calculate]{mb}{-b}
            \function[calculate]{d}{-q*b}
            
            \randint{s}{-5}{5}
            \randint{t}{-5}{5}
            
            \function[calculate]{r1}{a*s+b*t} 
            \function[calculate]{r2}{c*s+d*t}
        \end{variables}

        \begin{answer}
              \text{$x=$}
              \solution{s}
        \end{answer}
        
        \begin{answer}
        	  \text{$y=$}
              \solution{t} 
        \end{answer}
        
    \end{question}
%
% Q3 - lgs3
%
	\begin{question}
		\text{Lösen Sie das folgende lineare Gleichungssystems nach einem der bekannten Verfahren.\\
        \\
        \begin{displaymath}
        \begin{mtable}[\cellaligns{crcrcr}]
          \text{(I)} &\qquad \var{a}x&+&\var{b}y&=&\var{r1}\\
          \text{(II)}&\qquad \var{c}x&-&\var{md}y&=&\var{r2}
        \end{mtable}
        \end{displaymath}
        \\
        }
        
        \explanation{Multiplizieren Sie zum Beispiel die Gleichung (I) mit $\var{mc}$ 
                   und (II) mit $\var{a}$ und addieren Sie anschließend die beiden 
                   Gleichungen gemäß dem \textit{Additionsverfahren.} 
                   In der resultierenden Gleichung kommt dann $x$ nicht mehr vor.}
        
        \type{input.number}
       
        \begin{variables}
            \randint[Z]{a}{-10}{10}
            \randint[Z]{b}{0}{10}
            \randint[Z]{c}{-10}{10}
            \randint[Z]{d}{-10}{0}
            
            \function[calculate]{ma}{-a}
            \function[calculate]{mc}{-c}
            \function[calculate]{md}{-d}
            
            \function[calculate]{h1}{floor(a/c)-(a/c)}
            \function[calculate]{h2}{floor(c/a)-(c/a)}
            \function[calculate]{h3}{floor(b/d)-(b/d)}
            \function[calculate]{h4}{floor(d/b)-(d/b)}
 
            \randadjustIf{a,b,c,d}{a*d-b*c=0 OR h1=0 OR h2=0 OR h3=0 OR h4=0}
            
            \randint{s}{-5}{5}
            \randint{t}{-5}{5}
            
            \function[calculate]{r1}{a*s+b*t} 
            \function[calculate]{r2}{c*s+d*t}
        \end{variables}

        \begin{answer}
              \text{$x=$}
              \solution{s}
        \end{answer}
        
        \begin{answer}
        	  \text{$y=$}
              \solution{t} 
        \end{answer}
        
    \end{question}

%
%
\end{problem}

\embedmathlet{gwtmathlet}

\end{content}
