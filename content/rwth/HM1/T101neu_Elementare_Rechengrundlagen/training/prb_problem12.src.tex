\documentclass{mumie.problem.gwtmathlet}
%$Id$
\begin{metainfo}
 \name{
  \lang{de}{A12: Quadratische Gleichungen}
  }
  \begin{description} 
 This work is licensed under the Creative Commons License Attribution 4.0 International (CC-BY 4.0)   
 https://creativecommons.org/licenses/by/4.0/legalcode 

    \lang{de}{...}
  \end{description}
  \corrector{system/problem/GenericCorrector.meta.xml}
  \begin{components}
    \component{js_lib}{system/problem/GenericMathlet.meta.xml}{gwtmathlet}
  \end{components}
  \begin{links}
  \end{links}
  \creategeneric
\end{metainfo}

\begin{content}

\begin{block}[annotation]
	Im Ticket-System: \href{https://team.mumie.net/issues/22337}{Ticket 22337}
\end{block}

\begin{block}[annotation]
	Normalform bestimmen und quadratische Gleichung lösen
\end{block}

\usepackage{mumie.genericproblem}

\lang{de}{
	\title{A12: Quadratische Gleichungen}
}

\begin{problem}

\randomquestionpool{1}{2}
\randomquestionpool{3}{4}
%
% Q1
%
\begin{question}

\begin{variables}
	\randint[Z]{a}{-5}{5}
	\randint[Z]{l}{-5}{5}
	\randadjustIf{a,l}{a=l OR a=1 OR l=1}    
	
	\randint[Z]{b0}{-5}{5}
	\randint[Z]{m0}{-5}{5}
   	\function[calculate]{b}{b0*(a-l)}
	\function[calculate]{m}{m0*(a-l)}
	
	\randint[Z]{c0}{-5}{5}
	\randint[Z]{n0}{-5}{5}
   	\function[calculate]{c}{c0*(a-l)}
	\function[calculate]{n}{n0*(a-l)}    
	
	\randint[Z]{d0}{-5}{5}
	\randint[Z]{i0}{-5}{5}
   	\function[calculate]{d}{d0*(a-l)}
	\function[calculate]{i}{i0*(a-l)}        
	
	\randint[Z]{ee0}{-5}{5}
	\randint[Z]{j0}{-5}{5}
   	\function[calculate]{ee}{ee0*(a-l)}
	\function[calculate]{j}{j0*(a-l)}    
	
    \function[calculate]{disk}{(c-n-2*(a*b-l*m))^2-4*(a-l)*(a*b^2-l*m^2+c*d-n*i+ee-j)}
	\randadjustIf{b0,c0,d0,ee0,m0,n0,i0,j0,a,l}{a=l OR disk<=0}

% quadratische Gleichung
	\function{q}{a*(x-b)^2+c*(x+d)+ee}
	\function{z}{l*(x-m)^2+n*(x+i)+j}

% Lösungen	
	\function[calculate,2]{sa}{(-(c-n-2*a*b+2*l*m)+sqrt(disk))/(2*a-2*l)}
	\function[calculate,2]{sb}{(-(c-n-2*a*b+2*l*m)-sqrt(disk))/(2*a-2*l)}

% Normalform
    \function{norm}{1/(a-l)}
    \function[expand, normalize]{fn}{norm*(q-z)}
	\function[calculate,2]{sp}{(c-2*(a*b-l*m)-n)/(a-l)}
	\function[calculate,2]{sq}{(a*b^2+c*d+ee-l*m^2-n*i-j)/(a-l)}
    
	
\end{variables}

	\type{input.number}
	\field{real} 
	\precision{2}
	    \text{
	    Überführen Sie zunächt die quadratische Gleichung   \\
        \\
        $\, \var{q}=\var{z} \,$ \\
        \\
        in die Normalform und bestimmen Sie anschließend ihre Lösungen $x_1$ und $x_2$.\\
	    (Geben Sie die gerundeten Ergebnisse auf zwei Nachkommastellen an.) \\
        \\
        Die Normalform ist $\; x^2+\,($\ansref$)\,x\,+\,($\ansref$)\,=0\,$  \\
        \\
        und die Lösungen sind $\, x_1 =$\ansref  und $\,x_2 =$\ansref .
    }
    
    \explanation{Ihre Eingabe ist noch unvollständig oder enthält Fehler.}
              
     \permuteAnswers{3, 4}
    \begin{answer}
	    \solution{sp}
    \end{answer}
    \begin{answer}
	    \solution{sq}
    \end{answer}

    \begin{answer}
%	    \text{$x_1 =$}
	    \solution{sa}
    \end{answer}
    
    \begin{answer}
%	    \text{$x_2 =$}
	    \solution{sb}
    \end{answer}
    
\end{question}
%
% Q2
%
\begin{question}

\begin{variables}
	\randint[Z]{a}{-5}{5}
	\randint[Z]{l}{-5}{5}
	\randadjustIf{a,l}{a=l OR a=1 OR l=1}    
	
	\randint{b0}{-5}{5}
	\randint{m0}{-5}{5}
   	\function[calculate]{b}{b0*(a-l)}
	\function[calculate]{m}{m0*(a-l)}
	
	\randint{c0}{-5}{5}
	\randint{n0}{-5}{5}
   	\function[calculate]{c}{c0*(a-l)}
	\function[calculate]{n}{n0*(a-l)}    
	
	\randint{d0}{-5}{5}
	\randint{i0}{-5}{5}
   	\function[calculate]{d}{d0*(a-l)}
	\function[calculate]{i}{i0*(a-l)}        
	
	\randint{ee0}{-5}{5}
	\randint{j0}{-5}{5}
   	\function[calculate]{ee}{ee0*(a-l)}
	\function[calculate]{j}{j0*(a-l)}    
	
    \function[calculate]{disk}{(c-n-2*(a*b-l*m))^2-4*(a-l)*(a*b^2-l*m^2+c*d-n*i+ee-j)}
	\randadjustIf{b0,c0,d0,ee0,m0,n0,i0,j0,a,l}{a=l OR disk<=0}

% quadratische Gleichung
	\function[normalize]{q}{a*(x-b)^2+c*(x+d)+ee}
	\function[normalize]{z}{l*(x-m)^2+n*(x+i)+j}

% Lösungen	
	\function[calculate,2]{sa}{(-(c-n-2*a*b+2*l*m)+sqrt(disk))/(2*a-2*l)}
	\function[calculate,2]{sb}{(-(c-n-2*a*b+2*l*m)-sqrt(disk))/(2*a-2*l)}

% Normalform
    \function{norm}{1/(a-l)}
    \function[expand, normalize]{fn}{norm*(q-z)}
	\function[calculate,2]{sp}{(c-2*(a*b-l*m)-n)/(a-l)}
	\function[calculate,2]{sq}{(a*b^2+c*d+ee-l*m^2-n*i-j)/(a-l)}
    
	
\end{variables}

	\type{input.number}
	\field{real} 
	\precision{2}
	    \text{
	    Überführen Sie zunächt die quadratische Gleichung   \\
        \\
        $\, \var{q}=\var{z} \,$ \\
        \\
        in die Normalform und bestimmen Sie anschließend ihre Lösungen $x_1$ und $x_2$.\\
	    (Geben Sie die gerundeten Ergebnisse auf zwei Nachkommastellen an.) \\
        \\
        Die Normalform ist $\; x^2+\,($\ansref$)\,x\,+\,($\ansref$)\,=0\,$  \\
        \\
        und die Lösungen sind $\, x_1 =$\ansref  und $\,x_2 =$\ansref .
    }
    
    \explanation{Ihre Eingabe ist noch unvollständig oder enthält Fehler.}
    
    \permuteAnswers{3, 4}
    \begin{answer}
	    \solution{sp}
    \end{answer}
    \begin{answer}
	    \solution{sq}
    \end{answer}

    \begin{answer}
%	    \text{$x_1 =$}
	    \solution{sa}
    \end{answer}
    
    \begin{answer}
%	    \text{$x_2 =$}
	    \solution{sb}
    \end{answer}
    
\end{question}
%
% Q3 -Anzahl Lösungen
%
\begin{question}

\begin{variables}
	\randint[Z]{a}{2}{5}
	
	\randint{b0}{-5}{5}
	\randint{m0}{-5}{5}
   	\function[calculate]{b}{b0*(a-1)}
	\function[calculate]{m}{m0*(a-1)}
	
	\randint{c0}{-5}{5}
	\randint{n0}{-5}{5}
   	\function[calculate]{c}{c0*(a-1)}
	\function[calculate]{n}{n0*(a-1)}    
	
	\randint{d0}{-5}{5}
	\randint{i0}{-5}{5}
   	\function[calculate]{d}{d0*(a-1)}
	\function[calculate]{i}{i0*(a-1)}        
			
    \function[calculate]{disk}{(c-n-2*(a*b-m))^2-4*(a-1)*(a*b^2-m^2+d-i)}

    \begin{switch}
        \begin{case}{disk<0}
          \number{sol}{0}
      \end{case}
      \begin{case}{disk=0}
        \number{sol}{1}
      \end{case}
      \begin{default}
         \number{sol}{2}
      \end{default} 
   \end{switch}       

% quadratische Gleichung
	\function[normalize]{q}{a*(x-b)^2+c*x+d}
	\function[normalize]{z}{(x-m)^2+n*x+i}

% Lösungen	
	\function[calculate,2]{sa}{(-(c-n-2*a*b+2*m)+sqrt(disk))/(2*a-2)}
	\function[calculate,2]{sb}{(-(c-n-2*a*b+2*m)-sqrt(disk))/(2*a-2)}

% Normalform
    \function{norm}{1/(a-1)}
    \function[expand, normalize]{fn}{norm*(q-z)}
	\function[calculate,2]{sp}{(c-2*(a*b-m)-n)/(a-1)}
	\function[calculate,2]{sq}{(a*b^2+d-m^2-i)/(a-1)}   
	
\end{variables}

	\type{input.number}
	\field{rational} 
	\precision{2}
    
    \text{Geben Sie an, wieviele Lösungen die folgende quadratische Gleichung
           $\, \var{q}=\var{z} \,$ besitzt.
          }
          
    \explanation{Überführen Sie die Gleichung in die Normalform $\,x^2+px+q=0\,$ und prüfen Sie 
                dann, ob die Diskriminante $(\frac{p}{2})^2-q$ größer, gleich oder kleiner Null ist.}
              
    \begin{answer}
        \text{Die Anzahl der Lösungen ist: $\,$}
	    \solution{sol}
    \end{answer}
        
\end{question}
%
%
% Q4 -Anzahl Lösungen
%
\begin{question}

\begin{variables}
	\randint[Z]{a}{1}{5}
	\randint[Z]{b}{-5}{5}
	\randint[Z]{c}{-5}{5}
	\randint{d}{-5}{5}    
	
	\function[calculate]{p}{-2*a*b+2*c}
	\function[calculate]{q}{a*b^2+c^2}    
				
    \function[calculate]{disk}{d/a}

    \begin{switch}
        \begin{case}{disk<0}
          \number{sol}{0}
      \end{case}
      \begin{case}{disk=0}
        \number{sol}{1}
      \end{case}
      \begin{default}
         \number{sol}{2}
      \end{default} 
   \end{switch}       

% quadratische Gleichung
	\function[normalize]{v}{(a+1)*x^2+p*x+q}    % = a(x-b)^2 + (x+c)^2
	\function[normalize]{z}{(x+c)^2+d}          % = (x+c)^2 + d
	
\end{variables}

	\type{input.number}
	\field{rational} 
	\precision{2}
    
    \text{Geben Sie an, wieviele Lösungen die quadratische Gleichung 
           $\, \var{v}=\var{z} \,$ besitzt.
          }
          
    \explanation{Überführen Sie die Gleichung in die Normalform $\,x^2+px+q=0\,$ und prüfen Sie 
                dann, ob die Diskriminante $(\frac{p}{2})^2-q$ größer, gleich oder kleiner Null ist.}
              
    \begin{answer}
        \text{Die Anzahl der Lösungen ist: $\,$}
	    \solution{sol}
    \end{answer}
        
\end{question}
%
\end{problem}

\embedmathlet{gwtmathlet}

\end{content}
