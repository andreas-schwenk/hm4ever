\documentclass{mumie.problem.gwtmathlet}
%$Id$
\begin{metainfo}
 \name{
  \lang{de}{A07: Bruchterme}
  }
  \begin{description} 
 This work is licensed under the Creative Commons License Attribution 4.0 International (CC-BY 4.0)   
 https://creativecommons.org/licenses/by/4.0/legalcode 

    \lang{de}{...}
  \end{description}
  \corrector{system/problem/GenericCorrector.meta.xml}
  \begin{components}
    \component{js_lib}{system/problem/GenericMathlet.meta.xml}{gwtmathlet}
  \end{components}
  \begin{links}
  \end{links}
  \creategeneric
\end{metainfo}

\begin{content}
\begin{block}[annotation]
	Im Ticket-System: \href{https://team.mumie.net/issues/22305}{Ticket 22305}
\end{block}

\begin{block}[annotation]
	Vereinfachen von Bruchtermen
\end{block}

\usepackage{mumie.genericproblem}

\lang{de}{
	\title{A07: Bruchterme}
}

\begin{problem}
%
%
  \randomquestionpool{1}{2}
  \randomquestionpool{3}{4}
  \randomquestionpool{5}{6}
  \permutequestions
%
% Q1 
%
  \begin{question}
    \text{Vereinfachen Sie den folgenden Ausdruck:
    $\frac{\var{b1}}{\var{f1}} +\frac{\var{b2}}{\var{f2}}$.
    }
    
    \explanation{Bringen Sie beide Brüche auf den Hauptnenner und addieren Sie anschließend die Zähler.}

    \type{input.function}
    \field{real}

    \begin{variables}
      \randint{a1}{1}{4}
      \randint{b1}{1}{4}
      \randint{c1}{1}{5}
      \randint{a2}{1}{4}
      \randint{b2}{1}{4}
      \randint{c2}{1}{5}
      \function[normalize]{f1}{c1*x-a1}
      \function[normalize]{f2}{c2*x+a2}
      \function[normalize]{z}{b1*f2+b2*f1}
      \function[normalize]{n}{f1*f2}
      \function[normalize]{s}{z/n}
    \end{variables}

    \begin{answer}
      \text{Berechnen Sie erst den Nenner: }
      \solution{n}
      \inputAsFunction{x}{gn}
    \end{answer}    
    
    \begin{answer}
      \text{Berechnen Sie jetzt den Zähler:}
      \solution{z}
      \inputAsFunction{x}{gz}
      \checkFuncForZero{gz/gn-s}{-10}{10}{100}
    \end{answer}
  \end{question}
  
%
% Q2 
%
  \begin{question}
    \text{Vereinfachen Sie den folgenden Ausdruck:
    $\frac{\var{b1}}{\var{f1}} -\frac{\var{b2}}{\var{f2}}$.
    }
    
    \explanation{Bringen Sie beide Brüche auf den Hauptnenner und addieren Sie anschließend die Zähler.}

    \type{input.function}
    \field{real}

    \begin{variables}
      \randint{a1}{1}{4}
      \randint{b1}{1}{4}
      \randint{c1}{1}{5}
      \randint{a2}{1}{4}
      \randint{b2}{1}{4}
      \randint{c2}{1}{5}
      \function[normalize]{f1}{c1*x-a1}
      \function[normalize]{f2}{c2*x+a2}
      \function[normalize]{z}{b1*f2-b2*f1}
      \function[normalize]{n}{f1*f2}
      \function[normalize]{s}{z/n}  
    \end{variables}

    \begin{answer}
      \text{Berechnen Sie zuerst den Nenner: }
      \solution{n}
      \inputAsFunction{x}{gn}
    \end{answer}  
    
    \begin{answer}
      \text{Berechnen Sie jetzt den Zähler: }
      \solution{z}
      \inputAsFunction{x}{gz}
      \checkFuncForZero{gz/gn-s}{-10}{10}{100}
    \end{answer}
  \end{question}


%%%%%%%%%%%%%%%%%%%%%%%%%%%%%%%%%%%%%%%%%%%%%%%%%%%%%%%%%%%%%%%%%%%%%%%%%%%%%%%%%%%%%%%%%%%%%%%%%%%%%%%%%%%%%%%%%%%%%%%%%

%
% Q3
%
  \begin{question}
    \text{Kürzen Sie den Bruchterm 
    $\quad \frac{\var{fz}}{\var{fn}} \quad$ so weit wie möglich. 
    }
    
   \explanation{Finden Sie gemeinsame Faktoren in Zähler und Nenner, die Sie durch Anwendung 
               des Distributivgesetzes ausklammern. Kürzen Sie anschließend den Bruch.}
    
    \type{input.function}
    \field{real}

    \begin{variables}
      \randint{a}{1}{8}
      \randint{n}{1}{3}
      \randint{m}{1}{5}
      \randadjustIf{m}{m=n}
      
      \function[calculate]{b}{n*a}
      \function[calculate]{c}{m*a}
      \function[normalize]{fz}{a*x^2-b*x}
      \function[normalize]{fn}{a*x*y+c*x}
      \function[normalize]{gz}{x-n}
      \function[normalize]{gn}{y+m}
    \end{variables}  
    
    \begin{answer}
      \text{Zähler des gekürzten Bruchterms ist: $\quad$}
      \solution{gz}
      \checkAsFunction{x,y}{-10}{10}{100}
    \end{answer}

    \begin{answer}
      \text{Nenner des gekürzten Bruchterms ist: $\quad$}
      \solution{gn}
      \checkAsFunction{x,y}{-10}{10}{100}
    \end{answer}  

  \end{question}
  

%
% Q4
%
  \begin{question}
    \text{Kürzen Sie den Bruchterm 
    $\quad \frac{\var{fz}}{\var{fn}} \quad$ so weit wie möglich. 
    }
    
    \explanation{Finden Sie gemeinsame Faktoren in Zähler und Nenner, die Sie durch Anwendung 
                 des Distributivgesetzes ausklammern. Kürzen Sie anschließend den Bruch.}
    
    \type{input.function}
    \field{real}

    \begin{variables}
      \randint{a}{1}{8}
      \randint[Z]{n1}{-2}{2}
      \randint{n2}{1}{3}
      \randint[Z]{m1}{-3}{3}
      \randint{m2}{1}{5}
      \randadjustIf{m2}{m2=n2}
      
      \function[calculate]{b1}{n1*a}
      \function[calculate]{b2}{n2*a}
      \function[calculate]{c1}{m1*a}      
      \function[calculate]{c2}{m2*a}
      \function[normalize]{fz}{b1*x^2+b2*x}
      \function[normalize]{fn}{c1*x*y-c2*x}
      \function[normalize]{gz}{n1*x+n2}
      \function[normalize]{gn}{m1*y-m2}
    \end{variables}  
    
    \begin{answer}
      \text{Zähler des gekürzten Bruchterms ist: $\quad$}
      \solution{gz}
      \checkAsFunction{x,y}{-10}{10}{100}
    \end{answer}

    \begin{answer}
      \text{Nenner des gekürzten Bruchterms ist: $\quad$}
      \solution{gn}
      \checkAsFunction{x,y}{-10}{10}{100}
    \end{answer}  

  \end{question}

%%%%%%%%%%%%%%%%%%%%%%%%%%%%%%%%%%%%%%%%%%%%%%%%%%%%%%%%%%%%%%%%%%%%%%%%%%%%%%%%%%%%%%%%%%%%%%%%%%%%%%%%%%%%%%%%%%%%%%%%%

%
% Q5
%
  \begin{question}
    \text{Kürzen Sie den Bruchterm 
    $\quad \frac{\var{fz}}{\var{fn}} \quad$ so weit wie möglich. 
    }
    
    \explanation{Faktorisieren Sie Zähler und Nenner durch Anwendung der binomischen Formeln 
                 und kürzen Sie anschließend den Bruch.}
    
    \type{input.function}
    \field{real}

    \begin{variables}
      \drawFromSet{a}{1,5,7}
      \drawFromSet{b}{2,3,4,9}
      
      \function[calculate]{a2}{a^2}
      \function[calculate]{b2}{b^2}
      \function[calculate]{c}{2*a*b}
      \function[normalize]{fz}{a2*x^2-b2*y^2}
      \function[normalize]{fn}{a2*x^2-c*x*y+b2*y^2}
      \function[normalize]{gz}{a*x+b*y}
      \function[normalize]{gn}{a*x-b*y}
    \end{variables}  
    
    \begin{answer}
      \text{Zähler des gekürzten Bruchterms ist: $\quad$}
      \solution{gz}
      \checkAsFunction{x,y}{-10}{10}{100}
    \end{answer}

    \begin{answer}
      \text{Nenner des gekürzten Bruchterms ist: $\quad$}
      \solution{gn}
      \checkAsFunction{x,y}{-10}{10}{100}
    \end{answer}  

  \end{question}


%
% Q6
%
  \begin{question}
    \text{Kürzen Sie den Bruchterm 
    $\quad \frac{\var{fz}}{\var{fn}} \quad$ so weit wie möglich. 
    }
    
    \explanation{Faktorisieren Sie Zähler und Nenner durch Anwendung der binomischen Formeln 
                 und/oder des Distributivgesetzes und kürzen Sie anschließend den Bruch.}
    
    \type{input.function}
    \field{real}

    \begin{variables}
      \drawFromSet{a}{5,7,11}
      \drawFromSet{b}{2,3,4,9}
      
      \function[calculate]{ab}{a*b}
      \function[calculate]{b2}{b^2}
      \function[normalize]{fz}{a*z+ab}
      \function[normalize]{fn}{z^2-b2}
      \function[normalize]{gz}{a}
      \function[normalize]{gn}{z-b}
    \end{variables}  
    
    \begin{answer}
      \text{Zähler des gekürzten Bruchterms ist: $\quad$}
      \solution{gz}
      \checkAsFunction{z}{-10}{10}{100}
    \end{answer}

    \begin{answer}
      \text{Nenner des gekürzten Bruchterms ist: $\quad$}
      \solution{gn}
      \checkAsFunction{z}{-10}{10}{100}
    \end{answer}  

  \end{question}


%
%
\end{problem}

\embedmathlet{gwtmathlet}

\end{content}
