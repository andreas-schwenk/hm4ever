\documentclass{mumie.problem.gwtmathlet}
%$Id$
\begin{metainfo}
 \name{
  \lang{de}{A08: Aussagenlogik}
  }
  \begin{description} 
 This work is licensed under the Creative Commons License Attribution 4.0 International (CC-BY 4.0)   
 https://creativecommons.org/licenses/by/4.0/legalcode 

    \lang{de}{...}
  \end{description}
  \corrector{system/problem/GenericCorrector.meta.xml}
  \begin{components}
    \component{js_lib}{system/problem/GenericMathlet.meta.xml}{gwtmathlet}
  \end{components}
  \begin{links}
  \end{links}
  \creategeneric
\end{metainfo}

\begin{content}

\begin{block}[annotation]
	Im Ticket-System: \href{https://team.mumie.net/issues/22307}{Ticket 22307}
\end{block}

\begin{block}[annotation]
	Formalisierung und Verknüpfung sprachlich formulierter Aussagen
\end{block}

\usepackage{mumie.genericproblem}

\lang{de}{
	\title{A08: Aussagenlogik}
}

\begin{problem}
%
%
    \randomquestionpool{1}{5}
% Q1    
    \begin{question}
      \text{Man betrachte die Aussagen \\
      $A$: \textit{Tim isst Eis.}\\
      $B$: \textit{Tom isst Eis.}\\
      $C$: \textit{Es ist heiß.}\\
      $D$: \textit{Es regnet.}\\

      Wie setzt sich die Aussage\\ \textit{Wenn es heiß ist und nicht regnet, 
      essen Tim und Tom Eis.}\\
      aus den Aussagen $A$ bis $D$ zusammen?}
      
      \explanation{}

      \begin{variables}
      \end{variables}

      \permutechoices{1}{4}
      \type{mc.unique}
      \field{real}
      \precision{3}

      \begin{choice}
        \text{$C\wedge \neg D \Rightarrow A\wedge B$}
        \solution{true}  
      \end{choice}      
      \begin{choice}
        \text{$\neg (C\wedge D) \Rightarrow A\wedge B$}
        \solution{false}  
      \end{choice}      
      \begin{choice}
        \text{$A\wedge B\Rightarrow C\wedge \neg D$}
        \solution{false}  
      \end{choice}      
      \begin{choice}
        \text{$A\wedge B\Rightarrow \neg (C\wedge  D)$}
        \solution{false}
      \end{choice}      
    \end{question}
    
% Q2  
    \begin{question}
      \text{Man betrachte die Aussagen \\
      $A$: \textit{Tim isst Eis.}\\
      $B$: \textit{Tom isst Eis.}\\
      $C$: \textit{Es ist heiß.}\\
      $D$: \textit{Es regnet.}\\

      Wie setzt sich die Aussage\\ \textit{Tim und Tom essen kein Eis, wenn es regnet.}\\
      aus den Aussagen $A$ bis $D$ zusammen?}
      
      \explanation{Die Aussage lässt sich auch so formulieren:\\
      \textit{Wenn es regnet, dann isst Tim kein Eis und Tom isst kein Eis.}}
      
      \begin{variables}
      \end{variables}

      \permutechoices{1}{4}
      \type{mc.unique}
      \field{real}
      \precision{3}

      \begin{choice}
        \text{$D \Rightarrow \neg A\wedge \neg B$}
        \solution{true}  
      \end{choice}      
      \begin{choice}
        \text{$\neg A \wedge \neg B \Rightarrow D$}
        \solution{false}  
      \end{choice}      
      \begin{choice}
        \text{$D\Rightarrow \neg(A\wedge B)$}
        \solution{false}  
      \end{choice}      
      \begin{choice}
        \text{$\neg (A\wedge B)\Rightarrow D$}
        \solution{false}
      \end{choice}      
    \end{question}

% Q3
    \begin{question}
      \text{Man betrachte die Aussagen \\
      $A$: \textit{Tim isst Eis.}\\
      $B$: \textit{Tom isst Eis.}\\
      $C$: \textit{Es ist heiß.}\\
      $D$: \textit{Es regnet.}\\

      Wie setzt sich die Aussage\\ \textit{Tim isst immer dann Eis, wenn es heiß ist
      und nicht regnet.}\\
      aus den Aussagen $A$ bis $D$ zusammen?}
      
      \explanation{Es besteht hier eine \textit{genau dann, wenn} Beziehung.}
      
      \begin{variables}
      \end{variables}

      \permutechoices{1}{4}
      \type{mc.unique}
      \field{real}
      \precision{3}

      \begin{choice}
        \text{$A\Leftrightarrow C\wedge \neg D$}
        \solution{true}  
      \end{choice}      
      \begin{choice}
        \text{$\neg D\vee C \Leftrightarrow A$}
        \solution{false}  
      \end{choice}      
      \begin{choice}
        \text{$C\wedge \neg D\Rightarrow A$}
        \solution{false}  
      \end{choice}      
      \begin{choice}
        \text{$\neg (C\wedge D)\Leftrightarrow A$}
        \solution{false}
      \end{choice}  
    \end{question}

% Q4
    \begin{question}
      \text{Man betrachte die Aussagen \\
      $A$: \textit{Felix hat ein grünes Fahrrad.}\\
      $B$: \textit{Alex hat ein grünes Fahrrad.}\\
      $C$: \textit{Laura hat ein rotes Fahrrad.}\\
      Wie setzt sich die Aussage\\ \textit{Weder Felix, noch Alex haben ein grünes Fahrrad, 
      aber Laura hat ein rotes.}\\
      aus den Aussagen $A$ bis $C$ zusammen?}
      
      \explanation{}

      \begin{variables}
      \end{variables}

      \permutechoices{1}{4}
      \type{mc.unique}
      \field{real}
      \precision{3}

      \begin{choice}
        \text{$(\neg A \wedge \neg B)\wedge C$}
        \solution{true}  
      \end{choice}      
      \begin{choice}
        \text{$(\neg A \vee \neg B)\wedge C$}
        \solution{false}  
      \end{choice}      
      \begin{choice}
        \text{$\neg (A\wedge B)\wedge C$}
        \solution{false}  
      \end{choice}      
      \begin{choice}
        \text{$\neg (A\wedge B)\vee C$}
        \solution{false}
      \end{choice}      
    \end{question}

% Q5
    \begin{question}
      \text{Man betrachte die Aussagen \\
      $A$: \textit{Felix hat ein grünes Fahrrad.}\\
      $B$: \textit{Alex hat ein grünes Fahrrad.}\\
      $C$: \textit{Laura hat ein rotes Fahrrad.}\\
      Wie setzt sich die Aussage\\ \textit{Entweder hat Laura ein rotes  Fahrrad und weder Felix,
      noch Alex haben ein grünes, oder Felix und Alex haben beide ein grünes Fahrrad, aber 
      Laura kein rotes.}\\
      aus den Aussagen $A$ bis $C$ zusammen?}
      
      \lang{de}{\explanation{In diesem Fall schließen sich die "`oder"'-Alternativen sowieso 
      aus, weshalb "`entweder, oder"' hier gleich "`oder"' ist.}}
      
      \begin{variables}
      \end{variables}

      \permutechoices{1}{4}
      \type{mc.unique}
      \field{real}
      \precision{3}

      \begin{choice}
        \text{$(C\wedge \neg A \wedge \neg B)\vee (A\wedge B\wedge \neg C)$}
        \solution{true}  
      \end{choice}      
      \begin{choice}
        \text{$(C\wedge \neg (A\vee B))\vee ((A\vee B)\wedge \neg C)$}
        \solution{false}  
      \end{choice}      
      \begin{choice}
        \text{$(C\vee \neg C) \wedge ((\neg A\wedge \neg B)\vee (A\wedge B))$}
        \solution{false}  
      \end{choice}      
      \begin{choice}
        \text{$(C\wedge \neg (A\wedge B))\vee ((A\wedge B)\wedge \neg C)$}
        \solution{false}
      \end{choice}      
    \end{question}


%
%
\end{problem}

\embedmathlet{gwtmathlet}

\end{content}
