\documentclass{mumie.problem.gwtmathlet}
%$Id$
\begin{metainfo}
 \name{
  \lang{de}{A11: Lineare Gleichungen (TA)}
  }
  \begin{description} 
 This work is licensed under the Creative Commons License Attribution 4.0 International (CC-BY 4.0)   
 https://creativecommons.org/licenses/by/4.0/legalcode 

    \lang{de}{...}
  \end{description}
  \corrector{system/problem/GenericCorrector.meta.xml}
  \begin{components}
    \component{js_lib}{system/problem/GenericMathlet.meta.xml}{gwtmathlet}
  \end{components}
  \begin{links}
  \end{links}
  \creategeneric
\end{metainfo}

\begin{content}
\begin{block}[annotation]
	Im Ticket-System: \href{https://team.mumie.net/issues/22336}{Ticket 22336}
\end{block}

\begin{block}[annotation]
	Anwendungen zu linearen und quadratischen GLeichungen
\end{block}

\usepackage{mumie.genericproblem}

\lang{de}{
	\title{A11: Lineare Gleichungen}
}

\begin{problem}

%
  \randomquestionpool{1}{6}
%

% Question 1

\begin{question}

\begin{variables}
	\randint{a}{20}{50}
	\randint{b}{1}{30}
	\randadjustIf{a,b}{3b>=2a} % 2a>3b => a>b und sl=sb-b=1/5*(2a-3b)>0
	
	\function[calculate, 2]{sc}{(a+b)/5}	
    \function[calculate, 2]{sb}{2*sc}
	\function[calculate, 2]{sl}{sb-b}
\end{variables}

	\type{input.number}
	\field{real}
    \precision{2}
    \text{
	    Die Geschwister Ben, Carl und Laura gehen zusammen einkaufen. Sie geben zusammen $\var{a}$ € aus.
        Ben gibt dabei doppelt soviel aus wie Carl und Laura gibt $\var{b}$ € weniger aus als Ben. 
	    Welchen Betrag gibt jeder einzelne von ihnen aus?\\
        \\
        Ben gibt \ansref €, $\,$ Carl \ansref € $\,$ und Laura \ansref € aus.}
        
    \explanation{Zu Lösen ist die lineare Gleichung $x+ \frac{1}{2} x+ (x-\var{b})=\var{a}$.}   
    
    \begin{answer}
	    \solution{sb}
    \end{answer}
    \begin{answer}
	    \solution{sc}
    \end{answer}
    \begin{answer}
	    \solution{sl}
    \end{answer}

\end{question}

% Question 2

\begin{question}

\begin{variables}
	\randint{a}{50}{150}
	\randint{b}{3}{5}
	
	\function[calculate, normalize]{s}{2*a*b/(b-2)}	

\end{variables}

	\type{input.number}
	\field{rational}
    \text{
	    Marc und Lea machen eine dreitägige Radtour. Am ersten Tag legen sie $50$ \% der Gesamtstrecke 
        zurück. Am zweiten Tag schaffen sie $\var{a}$. Für den letzten Tag bleibt dann noch ein $\var{b}$-tel 
        der Gesamtstrecke zu fahren. Wie lang ist die gesamte Strecke?\\
        \\
        Insgesamt fahren die beiden an den drei Tagen $\,$\ansref km.}
        
    \explanation{Zu Lösen ist die lineare Gleichung $\; \frac{1}{2} x+ \var{a} + \frac{1}{\var{b}} x = x $.}   
    
    \begin{answer}
	    \solution{s}
    \end{answer}

\end{question}

% Question 3

\begin{question}

\begin{variables}
	\randint{a}{60}{120}
    \randint{d}{10}{30}
	\randint{b}{5}{15}
    \randadjustIf{b}{b>=d}
 
	\function[calculate, normalize]{cd}{d-b}    % Alter von David vor b Jahren
    \function[calculate, normalize]{co}{a-d-b}	% Alter der Großmutter vor b Jahren
	\function[calculate, normalize]{c}{co/cd}
	\function[calculate, normalize]{sd}{d}
	\function[calculate, normalize]{so}{a-d}	    

\end{variables}

	\type{input.number}
	\field{rational}
    \text{
	    David und seine Großmutter sind zusammen $\var{a}$ Jahre alt. Vor $\var{b}$ Jahren war seine Großmutter
        $\var{c}$ mal so alt wie David. Wie alt sind die beiden heute?\\
        \\
        David ist \ansref Jahre alt und seine Großmutter ist \ansref .}
        
    \explanation{Zu Lösen ist die lineare Gleichung $\; \var{c} (x-\var{b}) = (\var{a}-x)-\var{b}$.}   
    
    \begin{answer}
	    \solution{sd}
    \end{answer}
    \begin{answer}
	    \solution{so}
    \end{answer}

\end{question}

% Question 4

\begin{question}

\begin{variables}
	\randint{a}{10}{15}                                 % Geschwindikeit Marie
    \drawFromSet{d}{5,15}
    \function[calculate, normalize]{b}{a+d}             % Geschwindikeit Mutter 
 	\function[calculate, normalize]{sh}{1/4 * a/(b-a)}  % Fahrtzeit der Mutter in h
	\function[calculate, normalize]{sm}{sh * 60}	    % Fahrtzeit der Mutter in min   

\end{variables}

	\type{input.number}
	\field{rational}
    \text{
	    Marie fährt mit ihrem Fahrrad mit (konstant) $\var{a}$ km/h zur Schule. Als ihre Mutter 
        bemerkt, dass Marie ihren Sportbeutel vergessen hat, fährt sie mit einem e-Bike hinterher, 
        um Marie den Sportbeutel zu bringen. Die Mutter bricht genau $15$ min später auf und fährt mit
        einer (konstanten) Geschwindigkeit von $\var{b}$ km/h.  
        Wie lange braucht sie, bis sie ihre Tochter eingeholt hat? (Geben Sie die Zeit in Minuten an.)\\
        \\
        Es dauert $\,$\ansref min, bis die Mutter Marie eingeholt hat.}
        
    \explanation{Zu Lösen ist die lineare Gleichung 
                $\; (\frac{1}{4}$h $\,+ x\,$h $) \cdot \var{a} \,$km/h $\,= x\,$h $\,\cdot \var{b} \,$km/h.\\
                Sie beschreibt links die gefahrene Strecke bis zum Treffpunkt aus Sicht von Marie und
                rechts des Gleichheitszeichens dieselbe Strecke aus Sicht der Mutter. 
               }   
    
    \begin{answer}
	    \solution{sm}
    \end{answer}

\end{question}

% Question 5

\begin{question}

\begin{variables}
	\randint[Z]{a}{-50}{50}
	\randint[Z]{b}{-50}{50}
%    \randadjustIf{b}{b<a}

    \function[calculate, normalize]{sum}{a+b}
    \function[calculate, normalize]{sn}{-a-b}
 	\function[calculate, normalize]{p}{a*b} 

\end{variables}

	\type{input.number}
	\field{rational}
    \text{
	    Zerlegen Sie die Zahl $\var{sum}$ in zwei Summanden, deren Produkt $\var{p}$ ergibt. \\
        \\
        Die gesuchten Summanden sind $x_1=$\ansref und $x_2=$\ansref .}
        
    \explanation[sum<0 AND p<0]
                {Zu Lösen ist die quadratische Gleichung $x^2 + \var{sn} x + (\var{p}) =0$.\\
                Für die Lösungen $x_1$ und $x_2$ gilt nach dem Satz von Viëta 
                $\var{sum}=(x_1+x_2)\,$ und $\var{p}=x_1 \cdot x_2$.
                }  
    \explanation[sum<0 AND p>=0]
                {Zu Lösen ist die quadratische Gleichung $x^2 + \var{sn} x + \var{p} =0$.\\
                Für die Lösungen $x_1$ und $x_2$ gilt nach dem Satz von Viëta 
                $\var{sum}=(x_1+x_2)\,$ und $\var{p}=x_1 \cdot x_2$.
                }   
    \explanation[sum>=0 AND p<0]
                {Zu Lösen ist die quadratische Gleichung $x^2 - \var{sum} x + (\var{p}) =0$.\\
                Für die Lösungen $x_1$ und $x_2$ gilt nach dem Satz von Viëta 
                $\var{sum}=(x_1+x_2)\,$ und $\var{p}=x_1 \cdot x_2$.
                }   
    \explanation[sum>=0 AND p>=0]
                {Zu Lösen ist die quadratische Gleichung $x^2 - \var{sum} x + \var{p} =0$.\\
                Für die Lösungen $x_1$ und $x_2$ gilt nach dem Satz von Viëta 
                $\var{sum}=(x_1+x_2)\,$ und $\var{p}=x_1 \cdot x_2$.
                }   
                
%
   \permuteAnswers{1, 2}
%   
    \begin{answer}
	    \solution{a}
    \end{answer}
    \begin{answer}
	    \solution{b}
    \end{answer}

\end{question}

% Question 6

\begin{question}

\begin{variables}
	\randint[Z]{a}{0}{35}
	\randint[Z]{b}{1}{10}

    \function[calculate, normalize]{s}{2*(2*a+b)}
 	\function[calculate, normalize]{f}{a*(a+b)} 

\end{variables}

	\type{input.number}
	\field{rational}
    \text{
	    Ein rechteckiges Grundstück mit einer Fläche von $\var{f}$ qm ist von einem
        Zaun umgeben. Die längere Seite des Grundstücks ist $\var{b}$ m länger
        als die kürzere Seite. Wie lang ist der Zaun?  \\
        \\
        Der Zaun hat eine Länge von \ansref m.}
        
    \explanation{Die positive Lösung der quadratischen Gleichung $x^2 + \var{b} x - \var{f} =0$ liefert 
                die Länge der kürzeren Grundstücksseite.
                }  
%   
    \begin{answer}
	    \solution{s}
    \end{answer}

\end{question}


\end{problem}

\embedmathlet{gwtmathlet}

\end{content}
