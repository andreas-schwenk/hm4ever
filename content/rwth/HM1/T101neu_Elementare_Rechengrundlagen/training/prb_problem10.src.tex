\documentclass{mumie.problem.gwtmathlet}
%$Id$
\begin{metainfo}
  \name{
    \lang{de}{A10: Aussageformen bewerten}
    \lang{en}{Problem 10}
  }
  \begin{description} 
 This work is licensed under the Creative Commons License Attribution 4.0 International (CC-BY 4.0)   
 https://creativecommons.org/licenses/by/4.0/legalcode 

    \lang{de}{Die Beschreibung}
    \lang{en}{description}
  \end{description}
  \corrector{system/problem/GenericCorrector.meta.xml}
  \begin{components}
    \component{js_lib}{system/problem/GenericMathlet.meta.xml}{mathlet}
  \end{components}
  \begin{links}
  \end{links}
  \creategeneric
\end{metainfo}
\begin{content}

\begin{block}[annotation]
	Im Ticket-System: \href{https://team.mumie.net/issues/22332}{Ticket 22332}
\end{block}

\begin{block}[annotation]
    Bewertung von Formeln und Verknüpfungen
\end{block}

\usepackage{mumie.genericproblem}

\lang{de}{\title{A10: Aussageformen bewerten}}
\lang{en}{\title{Problem 10}}

\begin{problem}

	\randomquestionpool{1}{5}
%
% Q1	
	\begin{question}
   \text{Welche Zusammenhänge bestehen zwischen den Aussagen $A$, $B$ und $C$ für jede
   reelle Zahl $x$?\\
   $A:\,  x=\var{b}$\\
   $B:\,  \var{f}=0$\\
   $C:\,  x=\var{a}$
   }
   
   \begin{variables}
   \randint[Z]{a}{-2}{2}
   \function[normalize]{b}{-a}
   \function[normalize]{f}{x^2+2*a*x+a^2}
   \end{variables}
    \permutechoices{1}{6}
      \type{mc.multiple}
      \field{real}
      \precision{3}
  	\explanation{Beachten Sie bei der Bewertung der Formeln: Die einzige Lösung der Gleichung $\var{f}=0$ ist $x=\var{b}$.}
 
    \begin{choice}
        \text{$A\Rightarrow B$}
          \solution{true}  
        \end{choice}      
    \begin{choice}
        \text{$C\Rightarrow B$}
          \solution{false}  
        \end{choice}      
    \begin{choice}
        \text{$B\Rightarrow A$}
          \solution{true}  
        \end{choice}      
    \begin{choice}
        \text{$B\Rightarrow (A\vee C)$}
          \solution{true}
        \end{choice}     
    \begin{choice}
        \text{$A\Leftrightarrow \neg C$}
          \solution{false}
        \end{choice}
        \begin{choice}
        \text{$\neg B\Rightarrow \neg A$}
          \solution{true}
        \end{choice}         
  \end{question}

%
% Q2	  
  \begin{question}
   \text{Welche Zusammenhänge bestehen zwischen den Aussagen $A$, $B$ und $C$ für jede
   reelle Zahl $x$?\\
   $A:\,  \var{b}=\var{r}$\\
   $B:\,  \var{f}=\var{d}$\\
   $C:\,  \var{b}=-\var{r}$
   }
   
   \begin{variables}
   \randint[Z]{a}{-2}{2}
   \function[normalize]{b}{x+a}
   \function[normalize]{amin}{-a}
   \function[normalize]{f}{x^2+2*a*x+a^2}
   \randint{r}{1}{4}
   \function[calculate]{d}{r^2}
   \end{variables}
    \permutechoices{1}{6}
      \type{mc.multiple}
      \field{real}
      \precision{3}
    \explanation{Beachten Sie bei der Bewertung der Formeln: Die Lösungen der Gleichung $\var{f}=\var{d}$ sind $x=\var{amin}+\var{r}$ und
    $x=\var{amin}-\var{r}$.}
 
    \begin{choice}
        \text{$A\Rightarrow B$}
          \solution{true}  
        \end{choice}      
    \begin{choice}
        \text{$C\Rightarrow B$}
          \solution{true}  
        \end{choice}      
    \begin{choice}
        \text{$B\Rightarrow C$}
          \solution{false}  
        \end{choice}      
    \begin{choice}
        \text{$B\Rightarrow (A\vee C)$}
          \solution{true}
        \end{choice}     
    \begin{choice}
        \text{$A\Leftrightarrow \neg C$}
          \solution{false}
        \end{choice}
        \begin{choice}
        \text{$\neg B\Rightarrow \neg A$}
          \solution{true}
        \end{choice}  
  \end{question}
  
%
% Q3 (nochmal die gleiche Frage wie Q2, um dieser weniger Gewicht zu verleihen.)
% 
 \begin{question}
   \text{Welche Zusammenhänge bestehen zwischen den Aussagen $A$, $B$ und $C$ für jede
   reelle Zahl $x$?\\
   $A:\,  \var{b}=\var{r}$\\
   $B:\,  \var{f}=\var{d}$\\
   $C:\,  \var{b}=-\var{r}$
   }
   
   \begin{variables}
   \randint[Z]{a}{-2}{2}
   \function[normalize]{b}{x+a}
   \function[normalize]{amin}{-a}
   \function[normalize]{f}{x^2+2*a*x+a^2}
   \randint{r}{1}{4}
   \function[calculate]{d}{r^2}
   \end{variables}
    \permutechoices{1}{6}
      \type{mc.multiple}
      \field{real}
      \precision{3}
      
	  \explanation{Beachten Sie bei der Bewertung der Formeln: Die Lösungen der Gleichung $\var{f}=\var{d}$ sind $x=\var{amin}+\var{r}$ und
    	$x=\var{amin}-\var{r}$.}
 
    \begin{choice}
        \text{$A\Rightarrow B$}
          \solution{true}  
        \end{choice}      
    \begin{choice}
        \text{$C\Rightarrow B$}
          \solution{true}  
        \end{choice}      
    \begin{choice}
        \text{$B\Rightarrow C$}
          \solution{false}  
        \end{choice}      
    \begin{choice}
        \text{$B\Rightarrow (A\vee C)$}
          \solution{true}
        \end{choice}     
    \begin{choice}
        \text{$A\Leftrightarrow \neg C$}
          \solution{false}
        \end{choice}
        \begin{choice}
        \text{$\neg B\Rightarrow \neg A$}
          \solution{true}
        \end{choice}  
         
  \end{question}

%
% Q4            
   \begin{question}
   \text{Welche Zusammenhänge bestehen zwischen den Aussagen $A$, $B$ und $C$ für jede
   reelle Zahl $x$?\\
   $A:\,  \var{b}=\var{r}$\\
   $B:\,  \var{f}=\var{d}$\\
   $C:\,  \var{b}=-\var{r}$
   }
   
   \begin{variables}
   \randint[Z]{a}{-2}{2}
   \function[normalize]{b}{x+a}
   \function[normalize]{f}{(x+a)^3}
   \randint{r}{1}{3}
   \function[calculate]{d}{r^3}
   \end{variables}
    \permutechoices{1}{6}
      \type{mc.multiple}
      \field{real}
      \precision{3}
  
 \explanation{Beachten Sie bei der Bewertung der Formeln: Wenn die dritten Potenzen zweier Zahlen gleich sind, müssen die Zahlen selbst
 gleich sein.}
    \begin{choice}
        \text{$A\Rightarrow B$}
          \solution{true}  
        \end{choice}      
    \begin{choice}
        \text{$C\Rightarrow B$}
          \solution{false}  
        \end{choice}      
    \begin{choice}
        \text{$B\Rightarrow A$}
          \solution{true}  
        \end{choice}      
    \begin{choice}
        \text{$(A\vee C)\Rightarrow B$}
          \solution{false}
        \end{choice}     
    \begin{choice}
        \text{$A\Leftrightarrow \neg C$}
          \solution{false}
        \end{choice}
        \begin{choice}
        \text{$\neg B\Rightarrow \neg A$}
          \solution{true}
        \end{choice}  
         
  \end{question}
%
% Q5 (nochmal die gleiche Frage wie Q4, um dieser weniger Gewicht zu verleihen.)
% 
  \begin{question}
   \text{Welche Zusammenhänge bestehen zwischen den Aussagen $A$, $B$ und $C$ für jede
   reelle Zahl $x$?\\
   $A:\,  \var{b}=\var{r}$\\
   $B:\,  \var{f}=\var{d}$\\
   $C:\,  \var{b}=-\var{r}$
   }
   
   \begin{variables}
   \randint[Z]{a}{-2}{2}
   \function[normalize]{b}{x+a}
   \function[normalize]{f}{(x+a)^3}
   \randint{r}{1}{3}
   \function[calculate]{d}{r^3}
   \end{variables}
    \permutechoices{1}{6}
      \type{mc.multiple}
      \field{real}
      \precision{3}
  
 \explanation{Beachten Sie bei der Bewertung der Formeln: Wenn die dritten Potenzen zweier Zahlen gleich sind, müssen die Zahlen selbst
 gleich sein.}
 
    \begin{choice}
        \text{$A\Rightarrow B$}
          \solution{true}  
        \end{choice}      
    \begin{choice}
        \text{$C\Rightarrow B$}
          \solution{false}  
        \end{choice}      
    \begin{choice}
        \text{$B\Rightarrow A$}
          \solution{true}  
        \end{choice}      
    \begin{choice}
        \text{$(A\vee C)\Rightarrow B$}
          \solution{false}
        \end{choice}     
    \begin{choice}
        \text{$A\Leftrightarrow \neg C$}
          \solution{false}
        \end{choice}
        \begin{choice}
        \text{$\neg B\Rightarrow \neg A$}
          \solution{true}
        \end{choice}  
         
  \end{question}
\end{problem}

\embedmathlet{mathlet}
\end{content}