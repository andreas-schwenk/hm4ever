
%$Id:  $
\documentclass{mumie.article}
%$Id$
\begin{metainfo}
  \name{
    \lang{de}{Überblick: Rechengrundlagen}
    \lang{en}{overview: }
  }
  \begin{description} 
 This work is licensed under the Creative Commons License Attribution 4.0 International (CC-BY 4.0)   
 https://creativecommons.org/licenses/by/4.0/legalcode 

    \lang{de}{Beschreibung}
    \lang{en}{}
  \end{description}
  \begin{components}
  \end{components}
  \begin{links}
\link{generic_article}{content/rwth/HM1/T101neu_Elementare_Rechengrundlagen/g_art_content_05_loesen_gleichungen_und_lgs.meta.xml}{content_05_loesen_gleichungen_und_lgs}
\link{generic_article}{content/rwth/HM1/T101neu_Elementare_Rechengrundlagen/g_art_content_04_aussagen_aequivalenzumformungen.meta.xml}{content_04_aussagen_aequivalenzumformungen}
\link{generic_article}{content/rwth/HM1/T101neu_Elementare_Rechengrundlagen/g_art_content_03_bruchrechnung.meta.xml}{content_03_bruchrechnung}
\link{generic_article}{content/rwth/HM1/T101neu_Elementare_Rechengrundlagen/g_art_content_02_rechengrundlagen_terme.meta.xml}{content_02_rechengrundlagen_terme}
\link{generic_article}{content/rwth/HM1/T101neu_Elementare_Rechengrundlagen/g_art_content_01_zahlenmengen.meta.xml}{content_01_zahlenmengen}
\end{links}
  \creategeneric
\end{metainfo}
\begin{content}
\begin{block}[annotation]
	Im Ticket-System: \href{https://team.mumie.net/issues/30148}{Ticket 30148}
\end{block}


\begin{block}[annotation]
Im Entstehen: Überblicksseite für Kapitel Rechengrundlagen
\end{block}

\usepackage{mumie.ombplus}
\ombchapter{1}
\lang{de}{\title{Überblick: Rechengrundlagen}}
\lang{en}{\title{Overview: Foundations of mathematics}}



\begin{block}[info-box]
\lang{de}{\strong{Inhalt}}
\lang{en}{\strong{Contents}}


\lang{de}{
    \begin{enumerate}%[arabic chapter-overview]
   \item[1.1]  \link{content_01_zahlenmengen}{Mengen und Zahlenmengen}
   \item[1.2] \link{content_02_rechengrundlagen_terme}{Elementares Rechnen und Termumformungen}
   \item[1.3] \link{content_03_bruchrechnung}{Rechnen mit Brüchen und Bruchtermen}
   \item[1.4] \link{content_04_aussagen_aequivalenzumformungen}{Aussagen und deren Umformungen, Äquivalenzumformungen}
   \item[1.5] \link{content_05_loesen_gleichungen_und_lgs}{Lösen elementarer Gleichungen und Gleichungssysteme}
     \end{enumerate}
}
\lang{en}{
    \begin{enumerate}%[arabic chapter-overview]
   \item[1.1]  \link{content_01_zahlenmengen}{Sets and sets of numbers}
   \item[1.2] \link{content_02_rechengrundlagen_terme}{Writing and rearranging expressions}
   \item[1.3] \link{content_03_bruchrechnung}{Fraction arithmetic}
   \item[1.4] \link{content_04_aussagen_aequivalenzumformungen}{Propositional logic}
   \item[1.5] \link{content_05_loesen_gleichungen_und_lgs}{Solving elementary equations and systems of equations}
     \end{enumerate}
}

\end{block}

\begin{zusammenfassung}

\lang{de}{Dieses Kapitel behandelt die grundlegensten Objekte der Mathematik.
Wir beginnen mit Mengen, Mengenrelationen und -operationen sowie den Zahlenmengen $\N$, $\Z$, $\Q$ und $\R$ und Intervallen.
\\
Dazu kommen die Rechenoperationen Addition, Subtraktion, Multiplikation und Division zusammen mit 
der \glqq Punkt vor Strich\grqq{}-Regel und dem oft gebrauchten Begriff des Terms.
Wir begegnen sinnvollen abkürzenden Schreibweisen wie den Potenzen, dem Summen- und dem Produktsymbol.
Darauf folgen die Rechengesetze
Assoziativität, Kommutativität und Distributivität sowie die binomischen Formeln.
\\
Rechenoperationen mit Brüchen (Kürzen, Addieren, Multiplizieren, Dividieren) widmen wir insbesondere unsere Aufmerksamkeit.
\\
Diese Begriffe werden Sie mit größter Wahrscheinlichkeit allesamt aus der Schule kennen.
Wir wiederholen sie, um sie ein für alle Mal ordentlich aufzuschreiben und Ihnen die Möglichkeit zu geben, ggf. Lücken zu schließen.
Alle diese Begriffe werden im Folgenden ohne großes Erwähnen vorausgesetzt. 
Viele dieser Konzepte werden Ihnen im weiteren Verlauf in weit abstrakterer Form wiederbegegnen.
Nutzen Sie also die letzte Möglichkeit, zum Beispiel das Bruchrechnen zu üben.
\\
Allem mathematischen Formulieren liegt die Aussagenlogik zugrunde. 
Wir geben einen Einblick, um Ihnen das elementarste logische Handwerkszeug mit auf den Weg zu geben.  
Insbesondere wahre und falsche Aussagen sowie Implikationen und Äquivalenzen müssen unterschieden werden, 
um sogar einfachste Termumformungen konsequent durchführen zu können.
\\
Als erste Anwendung lösen wir elementare Gleichungen und erste lineare Gleichungssysteme.
}
\lang{en}{
This chapter deals with the most foundational objects in mathematics. We begin with 
sets, set relationships and set operations. We introduce the sets $\N$, $\Z$, $\Q$ and 
$\R$, and the notion of an interval.
\\
Also covered are the operations of addition, subtraction, multiplication and division, 
together with the order of operations and the definition of an expression. Convenient 
notation is introduced for powers, sums and products, and rules for calculation such 
as associativity, commutativity and distributivity, exemplified by some binomial 
identities.
\\
Special attention is paid to fractions and how they can be simplified, summed, 
multiplied and divided.
\\
It is likely that these definitions and techniques are already mostly familiar from 
school. We repeat them for completeness, and to fill any possible gaps in knowledge.
All of these concepts will be used in the following chapters without necessarily 
refering to them directly. Many of them will be developed into more abstract versions. 
This chapter is an opportunity for a student to ensure they are confident with the 
basics of arithmetic.
\\
Mathematical statements, relationships and proofs are expressed using propositional 
logic. We give an overlook of propositional logic, in order to start building a 
toolset for dealing with mathematics rigorously. Particularly, this means being able 
to decide the the truth or falsity of propositions such as implications and 
equivalences, which is crucial for even the most basic manipulation of expressions and 
equations.
\\
As the first application of the above, we look at solving elementary equations and 
systems of linear equations.
}


\end{zusammenfassung}

\begin{block}[info]\lang{de}{\strong{Lernziele}}
                   \lang{en}{\strong{Learning Goals}} 
\begin{itemize}[square]
\item \lang{de}{Sie bilden Teil-, Vereinigungs- und Komplementärmengen in abstrakten Zusammenhängen ebenso wie in den bekannten Zahlenmengen.}
      \lang{en}{Being able to represent subsets, intersections, unions and complements 
      for abstract sets and for known sets of numbers.}
\item \lang{de}{Sie benutzen selbstverständlich die Grundrechenarten und Rechenregeln.}
      \lang{en}{Being able to use operators and the rules associated with them 
      appropriately.}
\item \lang{de}{Sie addieren, multiplizieren und dividieren und kürzen auch sehr komplizierte Brüche.}
      \lang{en}{Being able to add, multiply, divide and simplify even complex 
      fractions.}
\item \lang{de}{Sie kennen das Assoziativitäts-, Kommutativitäts- und Distributivitätsgesetz.}
      \lang{en}{Understanding associativity, commutativity and distributivity.}
\item \lang{de}{Sie formen Terme äquivalent um, benutzen die binomischen Formeln und sind mit der Notation durch Summen- und Produktzeichen vertraut.}
      \lang{en}{Being able to manipulate and expand expressions, familiarity with 
      the sum and product notations.}
\item \lang{de}{Sie negieren logische Aussagen, kennen Implikation und Äquivalenz und füllen einfache Wahrheitstafeln aus.}
      \lang{en}{Being able to negate logical propositions, understand implication and 
      equivalence, and fill out truth tables.}
\item \lang{de}{Sie formen Terme äquivalent um und kennen wichtige Umformungen, die keine Äquivalenzumformungen sind.}
      \lang{en}{Being familiar with some common ways to manipulate equations without 
      changing their solution sets.}
\item \lang{de}{Sie lösen lineare und quadratische Gleichungen in den reellen Zahlen sowie kleine lineare Gleichungssysteme.}
      \lang{en}{Being able to solve linear and quadratic equations in the real 
      numbers, and systems of linear equations.}
\end{itemize}
\end{block}




\end{content}
