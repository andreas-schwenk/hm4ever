%$Id:  $
\documentclass{mumie.article}
%$Id$
\begin{metainfo}
  \name{
    \lang{de}{Aussagenlogik und Äquivalenzumformungen}    
    \lang{en}{Propositional logic}
  }
  \begin{description} 
 This work is licensed under the Creative Commons License Attribution 4.0 International (CC-BY 4.0)   
 https://creativecommons.org/licenses/by/4.0/legalcode 

    \lang{de}{Beschreibung}
    \lang{en}{Description}
  \end{description}
  \begin{components}
    \component{generic_image}{content/rwth/HM1/images/g_img_00_video_button_schwarz-blau.meta.xml}{00_video_button_schwarz-blau}
  \end{components}
  \begin{links}
    \link{generic_article}{content/rwth/HM1/T101neu_Elementare_Rechengrundlagen/g_art_content_01_zahlenmengen.meta.xml}{content_01_zahlenmengen}
  \end{links}
  \creategeneric
\end{metainfo}
\begin{content}

\usepackage{mumie.ombplus}
\ombchapter{1}
\ombarticle{4}

\usepackage{mumie.genericvisualization}

\begin{visualizationwrapper}

\title{\lang{de}{Aussagenlogik und "Aquivalenzumformungen}\lang{en}{Propositional logic}}
 
\begin{block}[annotation]
  Aussagen, Folgerungen und Äquivalenz von Aussagen, Äquivalenzumformungen von (Un-)Gleichungen
  
\end{block}
%
% ursprüngliche Version: T101_Rechengrundlagen/content_02_aussagen
%
% \begin{block}[annotation]
%  Im Ticket-System: \href{http://team.mumie.net/issues/8970}{Ticket 8970}\\
% \end{block}
%
%Ticket neu:
%
\begin{block}[annotation]
	Im Ticket-System: \href{https://team.mumie.net/issues/19193}{Ticket 19193}
\end{block}

\begin{block}[info-box]
\tableofcontents
\end{block}

\section{\lang{de}{Aussagen und Aussageformen}\lang{en}{Propositions}} \label{sec:aussagen}
%
% Motivation
%
\lang{de}{
Die \emph{Aussagenlogik} ist ein Teilgebiet der Logik, das auch in der Mathematik Anwendung findet.
Hier befasst sie sich konkret mit der Verknüpfung mathematischer Aussagen, denen Wahrheitswerte 
zugeordnet werden. Mit Hilfe der \emph{Aussagenlogik} lassen sich auch Sachverhalte aus der
realen Welt formalisieren, um dann nach bestimmten Regeln Schlussfolgerungen ziehen zu können. 
Dies wird anhand des folgenden Beispiels im Laufe des Kapitels veranschaulicht.
}
\lang{en}{
\emph{Propositional logic is a subfield of logic which has applications in mathematics. It is about 
combining mathematical statements which are assigned truth-values. With the help of 
\emph{propositional logic}, even difficult real-world problems can be formalised, so that 
inferences can be made using a set of rules. This is demonstrated throughout this chapter using the 
following example.
}
}

%
% Motivationsbeispiel
%
\begin{block}
\begin{example}[\textit{\lang{de}{Wer sagt die Wahrheit?}\lang{en}{Who is telling the truth?}}] \label{ex:motivation}
\lang{de}{
Lukas lebt in einer Vierer-WG zusammen mit Julia, Marie und Lara. Am Wochenende 
erwartet er Besuch und hat hierfür einen Kasten Bier besorgt. Am Samstag muss er leider
feststellen, dass nur noch eine volle Flasche Bier in seinem Kasten ist. 
Seine Mitbewohnerinnen leugnen ausnahmslos, das Bier genommen zu haben. Da außer ihnen
niemand in der Wohnung war, muss also mindestens eine von ihnen lügen. Lukas kennt die drei 
Mitbewohnerinnen noch nicht sehr gut, er weiß nur, dass jede von ihnen entweder 
grundsätzlich die Wahrheit sagt oder grundsätzlich lügt. Deshalb befragt er sie nochmal 
einzeln, woraufhin Julia sagt, dass Marie lügt, Marie sagt, dass Lara lügt und Lara sagt,
dass Julia und Marie beide lügen. Kann Lukas nun herausfinden, wer das Bier genommen hat?
}
\lang{en}{
Lukas lives with his roommates Julia, Marie and Lara. On the weekend he is having a party, 
and he has bought a crate of beers. On Saturday he lifts the crate and notices that it feels lighter 
than before; he checks and finds that most of the bottles in the crate are now empty. His roommates 
all deny having taken any of the beer, but since nobody else had been in their apartment, at least 
one of them must be lying. Lukas does not yet know any of his roommates very well, but he knows that 
each one of them either always tells the truth or always lies. Upon questioning each of 
them individually, Julia says that Marie is lying, Marie says that Lara is lying, and Lara says 
that both Julia and Marie are lying. Can Lukas deduce who took the beer?
}
\end{example}
\end{block}

\lang{de}{
Wir werden im Folgenden sehen, wie sich Lukas' Problem formalisieren und mithilfe der 
\emph{Aussagenlogik} lösen lässt. Hierzu definieren wir zunächst, was eine Aussage im 
mathematischen Sinne ist.
}
\lang{en}{
In this chapter we will see how Lukas' problem can be formalised and solved with the help of 
\emph{propositional logic}. We first define a 'proposition' in the mathematical sense.
}

\begin{definition}[\lang{de}{Aussage}\lang{en}{Proposition}] \label{def:aussagen}
\lang{de}{
Eine \notion{(mathematische) Aussage} ist eine sprachlich oder symbolisch formulierte Behauptung,
          der auf eindeutige Weise ein Wahrheitswert, nämlich \textbf{wahr (W)} oder \textbf{falsch (F)}
          zugeordnet werden kann. 
         }
\lang{en}{
A \textit{(mathematical) proposition} is a statement, expressed either in words or symbols, that can be 
unambiguously assigned a truth-value, i.e. a statement that is either \textbf{true (T)} or 
\textbf{false (F)}
}
\end{definition}

\lang{de}{
  Im rein sprachlichen Gebrauch werden die Begriffe \textit{"`Aussage"' \,} und \textit{"`Behauptung"' \,} 
  oft synonym verwendet, unabhängig davon, ob es sich dabei um eine \emph{(mathematische) Aussage} im Sinne 
  der vorstehenden Definition \ref{def:aussagen} handelt. Zur besseren Abgrenzung der Begrifflichkeiten werden 
  wir im Folgenden stets zunächst von einer \textit{"`Behauptung"' \,} sprechen, solange noch 
  nicht nachgewiesen ist, ob ihr ein eindeutiger Wahrheitswert zugeordnet werden kann.
  Für eine \emph{"`Aussage"'} ist dann gemäß Definition \ref{def:aussagen} stets die
  \textit{prinzipielle} Nachprüfbarkeit der zugrundeliegenden \textit{"`Behauptung"'} essenziell.
}
\lang{en}{
Informally we often use the terms \textit{'proposition'} and \textit{'statement'} interchangably, not always 
referring to a \emph{(mathematical) proposition} in the sense of definition \ref{def:aussagen}. To remove 
ambiguity, we henceforth use \textit{'statement'} whenever it has not yet been shown whether a truth-value can 
be assigned. Hence it is verifiability that distinguishes a \textit{'statement'} from a \emph{proposition}.
}

\begin{example} \label{ex:aussagen}
  \begin{itemize}
  \item \lang{de}{
  So ist beispielsweise die Behauptung "`Es regnet genau jetzt draußen im Garten."' eine Aussage, denn durch einen Blick 
  aus dem Fenster können wir prüfen, ob die Aussage wahr oder falsch ist. Aber auch, wenn wir uns in einem 
  fensterlosen Keller befinden, haben wir keinen Zweifel, dass der Satz "`Es regnet genau jetzt draußen im Garten."' prinzipiell 
  überprüfbar ist. Unverzichtbar für die eindeutige Bestimmung des Wahrheitswert ist jedoch die konkrete Angabe von Zeit und Ort in 
  dieser Aussage, denn es kann zum gleichen Zeitpunkt an einem anderen Ort nicht regnen, ebenso kann es zwei 
  Minuten später im Garten nicht mehr regnen.
  }
  \lang{en}{
  For example, the statement 'It is currently raining in Lukas' garden.' is a proposition, as we can decide whether 
  it is true or false by simply looking at the garden. However, even if we were nowhere near the garden, the 
  statement 'It is currently raining in Lukas' garden.' is still clearly either true or false. Specifying the time 
  and location here is essential for making it a proposition, as there are other places at the same time 
  where it is not raining, and in Lukas' garden it sometimes rains and sometimes does not. In this case, the 
  statement is specific enough that it can reasonably be assigned a truth-value.
  }
  
  \item \lang{de}{
  Die Behauptung "`Sechs Richtige im Lotto!"' hingegen ist \textbf{keine} Aussage, da nicht klar spezifiziert ist,
  um wen oder was es geht, und ob die "`Sechs Richtigen"' "`getippt"' oder "`knapp verfehlt"' wurden. 
  Durch das fehlende Verb in dem Satz ist es also nicht möglich, hier einen eindeutigen Wahrheitswert zuzuordnen.
  }
  \lang{en}{
  The statement 'Six matching lottery numbers!' on the other hand is \textbf{not} a proposition, as it is not clearly 
  specified whose lottery ticket the numbers match with, or whether they \textit{almost} match or \textit{fully} 
  match. The sentence has no verb either, meaning it cannot be assigned a truth-value.
  }
        
  \begin{block}[tip]
     \lang{de}{
     Für Behauptungen in Form von sprachlich formulierten Sätzen gilt die \textbf{Faustregel}, dass in der 
     Regel nur ein vollständig formulierter Satz mit Subjekt, Prädikat und (optional) 
     Objekt eine Aussage im mathematischen Sinne gemäß \ref{def:aussagen} sein kann.
     }
     \lang{en}{
     As a rule of thumb, statements expressed using natural language must be gramatically correct sentences, 
     with a subject and predicate, and possibly an object, in order to be considered a mathematical proposition 
     as in definition \ref{def:aussagen}.
     }
  \end{block}

    \item \lang{de}{
        Für eine Aussage ist es wichtig, dass ihr ein Wahrheitswert zugeordnet werden \emph{kann}, 
        in dem Sinne, dass dies grundsätzlich möglich ist.
        Welcher Wahrheitswert ihr zugeordnet wird, ist aber stark vom Kontext abhängig. Zum Beispiel wird die Aussage
          \textit{"`Diese Aussage ist falsch."'} je nach Zusammenhang mit 
          \textbf{wahr} oder \textbf{falsch} belegt werden.
        }
        \lang{en}{
        It is important that a truth-value \textbf{can} be assigned to a proposition. The value itself is 
        highly dependent on context. For example, the proposition 
          \textit{'this proposition is false'} 
        can be assigned to be either be \textbf{'true'} or \textbf{'false'} depending on the context.
        }

    \item \lang{de}{
    Ähnliches gilt für mathematisch formulierte Behauptungen, die Variablen enthalten, wie z.\,B. $\, x^2=1.$ \\
    Diese Behauptung wird erst dann zu einer Aussage, wenn ergänzend konkrete Werte für die Variable $x$ spezifiziert 
    sind. So gilt innerhalb der reellen Zahlen:
    \\
    \begin{itemize}[arrow]
      \item "`Für alle $x \in \{-1,1\} \;\;$  gilt $\ x^2=1$ ."'$\quad $  eine \textbf{wahre} Aussage,\\
      \item "`Für alle $x \in [-1,1] \quad$ gilt $\ x^2=1$ ."'$\quad $  eine \textbf{falsche} Aussage.\\
    \end{itemize}
    }
    \lang{en}{
    A mathematical statement containing variables, such as $\, x^2=1$, only becomes a proposition when it is 
    specified what values are taken by the variable $x$. For example, 
    \\
    \begin{itemize}[arrow]
      \item 'For all $x \in \{-1,1\} \;\;$  we have $\ x^2=1$ .'$\quad $  A \textbf{true} proposition,\\
      \item 'For all $x \in [-1,1] \quad$ we have $\ x^2=1$ .'$\quad $  A \textbf{false} proposition.\\
    \end{itemize}
    }

  \end{itemize}
\end{example}


\begin{remark}\label{rem:aussagen}
    
    \begin{itemize}
        \item \lang{de}{
        Enthält eine mathematisch formulierte Behauptung Variablen, so kann man ihren Wahrheitswert
       nur in Abhängigkeit von diesen Variablen bestimmen. 
       Die Behauptung erfüllt also per se nicht die geforderte 
        Eigenschaft einer \emph{Aussage}. Erst durch die Belegung der Variablen mit konkreten Werten, wie im 
        vorstehenden Beispiel, wird die Behauptung zu einer Aussage im Sinne der Definition \ref{def:aussagen}. 
        }
        \lang{en}{
        If a mathematical statement contains variables, its truth-value can only be determined depending on those 
        variables. The statement only becomes a proposition in the sense of definition \ref{def:aussagen} once the 
        values of the variables are specified.
        }
        
        \item \lang{de}{
        Auch mathematische Terme wie zum Beispiel $x^2+1$ oder auch $2+5$ sind keine \emph{Aussagen,}
        da sie weder \textbf{wahr} noch \textbf{falsch} sein können. 
        Sprachlich würde dem etwa \glqq Blume\grqq{} oder \glqq Haus\grqq{} entsprechen.
         Ausdrücke mit Termen können nur dann Aussagen 
        im Sinne der Definition \ref{def:aussagen} sein, wenn sie mehrere Terme oder auch Terme und 
        Mengen enthalten, die durch Relationszeichen wie z.\,B. $=$, $>$, $<$, $\geq$, $\leq$, $\neq$, $\in, \ldots\quad$ in Beziehung
        gesetzt werden. Zum Beipiel ist
        \\
        \begin{itemize}[arrow]
          \item $2+5>3 \quad $  eine \textbf{wahre} Aussage,\\
          \item $2+5=6 \quad $  eine \textbf{falsche} Aussage,\\
          \item "`Für $x=1\;$ ist $\; x^2=1 $."'$\quad $  eine \textbf{wahre} Aussage.\\
        \end{itemize}
        }
        \lang{en}{
        Mathematical expressions such as $x^2+1$ or $2+5$ are not propositions either, as they can neither be 
        \textbf{true} nor \textbf{false}. Analogously, neither 'flower' nor 'house' can hold a truth-value. Such 
        expressions can only become propositions in the sense of definition \ref{def:aussagen} if they contain a relation 
        such as $=$, $>$, $<$, $\geq$, $\leq$, $\neq$, $\in, \ldots\quad$ that compares two or more expressions. For 
        example, 
        \\
        \begin{itemize}[arrow]
          \item $2+5>3 \quad $  is a \textbf{true} proposition,\\
          \item $2+5=6 \quad $  is a \textbf{false} proposition,\\
          \item `For $x=1\;$ we have $\; x^2=1 $.'$\quad $  is a \textbf{true} proposition.\\
        \end{itemize}
        }

\end{itemize}       
\end{remark}



%    \lang{de}{In Definition \ref{def:aussagen} ist die \textit{prinzipielle} 
%    Nachprüfbarkeit gemeint.
%   So ist der Satz "`Es regnet."' eine Aussage, denn durch einen Blick aus 
%    dem Fenster können wir prüfen, ob die Aussage wahr oder falsch ist. 
%    Aber auch, wenn wir uns in einem fensterlosen Keller befinden, haben 
%    wir keinen Zweifel, dass der Satz "`Es regnet."' prinzipiell überprüfbar 
%    ist.
%    Auch}
%    \begin{align}
%    x^2=1 \label{eq-II.1.01}
%    \end{align}
%    \lang{de}{ist eine Aussage; für $x = 1$ und $x = -1$ ist \ref{eq-II.1.01} wahr, 
%    und für alle anderen reellen Zahlen ist \ref{eq-II.1.01} falsch - aber 
%    stets ist \ref{eq-II.1.01} entweder wahr oder falsch 
%    (und nicht "`weder wahr noch falsch"').\\
%    \\
%    Zum Verständnis des Begriffs der Aussage ist es nützlich, Beispiele 
%    von Sätzen zu geben, die \textit{keine} Aussagen sind: "`Ein Auto"' 
%    ist keine Aussage, "`Sechs Richtige im Lotto"' ist keine Aussage, 
%    und auch ein einzelner Term,}
%   \begin{align}
%   \frac{x^2}{x^2+1} \label{eq-II.1.02}
%    \end{align}
%    \lang{de}{ist keine Aussage. An diesem Beispiel erkennt man die Bedeutung der 
%    Gleichheits- und Ungleichheitszeichen $=,\; <,\; >,\; \leq$ und $\geq$: 
%    Erst wenn Terme durch diese (und noch viele weitere) Zeichen miteinander 
%    in Beziehung gesetzt werden, entstehen Aussagen. Als Faustregel gilt, dass 
%    ein \textit{ganzer deutscher Satz} meist eine Aussage im mathematischen Sinn ist.}



\begin{quickcheck}
		\type{input.number}
		\begin{variables}
			\randint{y}{1}{5}
			\function[calculate]{l}{y*(y-1)}
			\function[calculate]{u}{y*(y+1)}
		\end{variables}
		
			\text{\lang{de}{
      F"ur welchen Wert der Variable $x$ ist die Aussage\\ 
			\emph{$x \;$ ist eine natürliche Zahl mit $\quad \var{l}< x^2\leq \var{u}$.}\\
			eine wahre Aussage?\\
			F"ur $x=$\ansref.
      }
      \lang{en}{
      For which value of the variable $x$ is the proposition\\
      \emph{$x \;$ is a natural number with $\quad \var{l}< x^2\leq \var{u}$.}\\
      a true proposition?\\
      For $x=$\ansref.
      }}
		
		\begin{answer}
			\solution{y}
		\end{answer}
	\end{quickcheck}

%%%%%%
% NEU
%%%%%%

%
% Motivationsbeispiel
%
%
  
\begin{block} 
\lang{de}{\notion{1. Schritt zur Lösung von \ref{ex:motivation}{$\,$} \textit{Wer sagt die Wahrheit?} }}
\lang{en}{\notion{First step to solve \ref{ex:motivation}{$\,$} \textit{Who is telling the truth?}}}
\\  
    \lang{de}{
    Schauen wir uns noch einmal die Situation in der Vierer-WG an. 
    Zur Prüfung der Sachverhalte suchen wir zunächst die im Text enthaltenen Behauptungen,
    die sich als % \emph{(mathematische) Aussagen} 
    \emph{Aussagen} im Sinne der Definition \ref{def:aussagen}
    formulieren lassen, denen also eindeutig
    ein Wahrheitswert zugeordnet werden kann.
    }
    \lang{en}{
    Let us remind ourselves of the situation in Lukas' apartment, and look for the statements made in the example 
    text that can be formulated as \emph{propositions} in the sense of definition \ref{def:aussagen}, and assigned 
    a truth-value.
    }


 \begin{incremental}

  \step %1
  \lang{de}{
  Die Tatsache, dass jede Mitbewohnerin von Lukas entweder grundsätzlich die Wahrheit sagt
  oder grundsätzlich lügt, lässt sich durch die folgenden Aussagen darstellen:
  \begin{enumerate}
      \item[$A_1:$]  "`Julia sagt die Wahrheit."' \\
      \item[$A_2:$]  "`Marie sagt die Wahrheit."' \\
      \item[$A_3:$]  "`Lara sagt die Wahrheit."'  \\
  \end{enumerate}
  % Dies sind Aussagen im mathematischen Sinne, 
  Dies sind \emph{Aussagen} im Sinne der Definition \ref{def:aussagen},
  denn jede der Aussagen ist entweder wahr oder falsch. Eine konkrete Festlegung von Zeit und Ort, wie in 
  Beispiel \ref{ex:aussagen} erwähnt, spielt in dieser Aussage aufgrund der Voraussetzung keine Rolle, denn
  sagt beispielsweise Julia die Wahrheit, tut sie dies ja grundsätzlich, somit ist
  $A_1$ \textbf{wahr}. Wenn aber Julia lügt, tut sie auch dies grundsätzlich und in diesem 
  Fall ist $A_1$ \textbf{falsch}.
  }
  \lang{en}{
  The fact that each of Lukas' roommates either always tells the truth or always lies can be expressed as:
  \begin{enumerate}
      \item[$A_1:$]  'Julia tells the truth.' \\
      \item[$A_2:$]  'Marie tells the truth.' \\
      \item[$A_3:$]  'Lara tells the truth.'  \\
  \end{enumerate}
  These are \emph{propositions} in the sense of definition \ref{def:aussagen}, as each one is either true 
  or false. A specified time and place like in example \ref{ex:aussagen} is not necessary here. This is because 
  we are given as a fact that if Julia tells the truth, she always does so, making $A_1$ \textbf{true.} If she 
  lies, she always does so, making $A_1$ \textbf{false.} 
  }
   
  
  \step %2
  \lang{de}{
  Die einzelnen Angaben der Mitbewohnerinnen, "`Julia sagt, dass Marie lügt"', "`Marie sagt, 
  dass Lara lügt"' und "`Lara sagt, dass Julia und Marie lügen"' sind weitere Behauptungen,
  die jedoch alleinstehend noch keine sinnvollen Aussagen im mathematischen Sinne ergeben.
  Sie müssen in logischem Zusammenhang zu den vorstehenden Aussagen $A_1$, $A_2$ 
  und $A_3$ betrachtet werden, um zur Lösung der Fragestellung zu verhelfen.\\ 
  Wie man logische Zusammenhänge formalisieren und auswerten kann, werden wir nachfolgend
  weiter untersuchen.
  }
  \lang{en}{
  Each of the roommates' claims, 'Julia says that Marie is lying', 'Marie says that Lara is lying' and 'Lara says 
  that both Julia and Marie are lying', need to logically connected to the propositions $A_1$, $A_2$ and 
  $A_3$ before they are useable mathematical propositions. We now look at how logical relationships can be formalised 
  and evaluated.
  }
  
 \end{incremental}
\end{block}

\lang{de}{
Neben den Aussagen und ihren Wahrheitswerten spielen auch Verknüpfungen eine wichtige Rolle in der \emph{Aussagenlogik}.
% Man betrachtet dabei Aussagen, die sich aus einzelnen Teilaussagen zusammensetzen, wobei jede Teilaussage selbst eine Aussage 
% im Sinne der Definition \ref{def:aussagen} ist. 
\\
Beispiele:
}
\lang{en}{
Logical connectives play as important a role in \emph{propositional logic} as propositions and their truth-values do.
\\
Examples:
}
\begin{itemize}
\item \lang{de}{
Die Aussage "`Wenn die Sonne scheint, ist es hell."' setzt sich mittels einer Wenn-Dann-Verknüpfung 
aus den (Teil-)Aussagen "`Die Sonne scheint."' und "`Es ist hell."' zusammen.
}
\lang{en}{
The proposition 'If the sun is shining, it is bright.' is created using an 'implication' connective between the smaller propositions 'The sun is shining.' and 'It is bright.'.
}
\item \lang{de}{
Die Aussage "`Die Sonne scheint und es ist hell."' ergibt sich aus einer Und-Verknüpfung der beiden Teil-Aussagen.
}
\lang{en}{
The proposition 'The sun is shining and it is bright.' is created using an 'and' connective between the two smaller 
propositions.
}
\end{itemize}

\lang{de}{
Diese Zusammenhänge wollen wir im Folgenden mathematisch formalisieren. Um hierbei die Zuordnung der Wahrheitswerte bei 
Verknüpfungen zu abstrahieren und nicht vom konkreten Inhalt der verknüpften Aussage oder vom Kontext abzuleiten, verwendet man
statt des Begriffs \emph{Aussage} oft auch die Bezeichnung \emph{Aussagenvariable} (als Platzhalter für eine konkrete 
Aussage).
}
\lang{en}{
We will now express these relationships more formally. In order to abstract away from the need to assign a truth-value 
to a proposition depending on the specific statement it is making, we often use 'propositional variables' in place of 
propositions.
}

\begin{definition}[\lang{de}{Objekte der Aussagenlogik}\lang{en}{Objects in propositional logic}]\label{def:objekte_aussagenlogik}

\lang{de}{
Die zentralen Objekte der Aussagenlogik sind
}
\lang{en}{
The main objects in propositional logic are
}
  \begin{itemize}
    \item \lang{de}{
    \notion{Aussagen} oder \notion{Variablen} (kurz für \emph{Aussagenvariablen}), welche meist mit 
    $A, B, C,\ldots$ bezeichnet werden,
    }
    \lang{en}{
    \notion{Propositions} or \notion{variables} (short for \emph{propositional variables}), often denoted by 
    $A, B, C,\ldots$.
    }
    \item \lang{de}{
    \notion{Wahrheitswerte} (auch \emph{Konstanten} genannt) "`wahr"' und "`falsch"',
        bezeichnet mit $W$ und $F$.
    }
    \lang{en}{
    \notion{Truth-values} (also called \emph{constants}) 'true' and 'false', denoted by $T$ and $F$, and
    }
    \item \lang{de}{
    \emph{Logische Verknüpfungen} zwischen Aussagen, auch \notion{Junktoren} genannt. Diese sind 
    \begin{enumerate} 
      \item[$\mathbf{\neg}$] \notion{Negation}, gelesen \emph{"`nicht"'},
      \item[$\mathbf{\wedge}$] \notion{Konjunktion}, gelesen \emph{"`und"'},
      \item[$\mathbf{\vee}$] \notion{Disjunktion}, gelesen \emph{"`oder"'},
      \item[$\mathbf{\Rightarrow}$] \notion{Implikation}, gelesen \emph{"`wenn ..., dann ..."'},
      \item[$\mathbf{\Leftrightarrow}$] \notion{Äquivalenz}, gelesen \emph{"`genau dann ..., wenn ..."'}.
    \end{enumerate}
    }
    \lang{en}{
    \emph{Logical connectives} between propositions, also called \notion{logical operators} and \notion{sentential connectives}. These are the following:
    \begin{enumerate} 
      \item[$\mathbf{\neg}$] \notion{Negation}, said \emph{'not'},
      \item[$\mathbf{\wedge}$] \notion{Conjunction}, said \emph{'and'},
      \item[$\mathbf{\vee}$] \notion{Disjunction}, said \emph{'or'},
      \item[$\mathbf{\Rightarrow}$] \notion{Implication}, said \emph{'if ..., then ...'},
      \item[$\mathbf{\Leftrightarrow}$] \notion{Equivalence}, said \emph{'... if and only if ...''}.
    \end{enumerate}
    }
\end{itemize}
\lang{de}{
    Eine Formel aus \emph{Aussagen/Variablen}, \emph{Wahrheitswerten} und \emph{Junktoren} wird auch \notion{Aussageform} 
    oder einfach \notion{Formel} genannt und im Folgenden % als Abgrenzung zu \emph{(Aussage-)Variablen} 
    mit $\mathbf{\mathcal{A}, \mathcal{B}, \mathcal{C},\ldots}$ bezeichnet.
}
\lang{en}{
A formula constructed from \emph{propositions/variables}, \emph{truth-values} and \emph{connectives} is also called a 
\notion{propositional formula}, but we often just use \notion{formula} to mean the same. In the following we use $\mathbf{\mathcal{A}, \mathcal{B}, \mathcal{C},\ldots}$ for formulas.
}

\end{definition}

\begin{remark}\label{rem:formeln}
    
  \begin{itemize}
  
    \item \lang{de}{
        Eine Variable $A$ selbst ist bereits eine \emph{Formel}, die einfach nur aus der 
        Variablen $A$ besteht. Eine solche Formel bezeichnet man als \notion{einfache} oder 
        auch als \notion{atomare Formel.}
        }
        \lang{en}{
        A variable $A$ is a \emph{formula} on its own, called an \notion{atomic formula} because it contains no 
        logical connectives.
        }
      
    \item \lang{de}{
        Für alle \emph{Variablen} $A$ und $B\,$ lassen sich weitere \emph{Formeln} durch einen
        schrittweisen Prozess von logischen Verknüpfungen definieren, zum Beispiel im 
        ersten Schritt
       \begin{enumerate}
        \item[$\; \neg A$]  die Negation einer Variablen $A$, gelesen 'nicht $A$',
        \item[$A\wedge B$]  die Konjunktion von $A$ und $B$, gelesen '$A$ und $B$', 
        \item[$A\vee B$]    die Disjunktion von $A$ und $B$, gelesen '$A$ oder $B$',
       \end{enumerate}
       und durch weitere Verknüfungen
       \begin{enumerate}
        \item[$ \neg (A \wedge (\neg B) )\;$] die Negation von '$A$ und nicht $B$',
        \item[$(\neg B)\Rightarrow A \;$] 'wenn nicht $B$, dann $A$',
        \item[$A \Leftrightarrow (\neg B)\,$] '$A$ genau dann, wenn nicht $B$'.
       \end{enumerate}
       }
       \lang{en}{
        With any \emph{variables} $A$ and $B\,$ we can construct more \emph{formulas} using logical connectives:
       \begin{enumerate}
        \item[$\; \neg A$]  the negation of a variable $A$, said 'not $A$',
        \item[$A\wedge B$]  the conjunction of $A$ and $B$, said '$A$ and $B$', 
        \item[$A\vee B$]    the disjunction of $A$ and $B$, said '$A$ or $B$',
       \end{enumerate}
       and by further combining these,
       \begin{enumerate}
        \item[$ \neg (A \wedge (\neg B) )\;$] the negation of '$A$ and not $B$',
        \item[$(\neg B)\Rightarrow A \;$] 'if not $B$, then $A$',
        \item[$A \Leftrightarrow (\neg B)\,$] '$A$ if and only if not $B$'.
       \end{enumerate}
       }
      
  \end{itemize}
\end{remark}

%
\begin{example}
 \lang{de}{
    Wir stehen am Fenster, blicken in den Garten und beschreiben die aktuelle Wettersituation. 
    Dabei steht die Variable $A\;$ konkret für die Aussage "`Die Sonne scheint."' und $B\;$ für "`Es regnet."'.
    Dann liefern die Formeln die folgenden neuen Aussagen:
    }
\lang{en}{
We stand at a window, look into the garden and describe the current weather. Let the variable $A\;$ represent the 
proposition 'The sun is shining.' and $B\;$ represent 'It is raining'. Then the above formulas represent the following 
propositions:
}
    
 \begin{enumerate}
      \item[$\neg A$] 
        \lang{de}{"`Die Sonne scheint nicht."'}
        \lang{en}{'The sun is not shining'}
      \item[$A\wedge B$] 
        \lang{de}{"`Die Sonne scheint und es regnet."'}
        \lang{en}{'The sun is shining and it is raining.'}
      \item[$A\vee B$]   
        \lang{de}{"`Die Sonne scheint oder es regnet."'}
        \lang{en}{'The sun is shining or it is raining.'} 
      \item[$\neg (A\wedge \neg B)$] 
        \lang{de}{"`Es ist nicht wahr, dass die Sonne scheint und es nicht regnet."'}
        \lang{en}{'It is not true that the sun is shining and it is not raining.'}
      \item[$(\neg B)\Rightarrow A$] 
        \lang{de}{"`Wenn es nicht regnet, dann scheint die Sonne."'}
        \lang{en}{'If it is not raining, then the sun is shining.'}
      \item[$A \Leftrightarrow (\neg B)\,$]  
        \lang{de}{"`Die Sonne scheint genau dann, wenn es nicht regnet."'}
        \lang{en}{'The sun is shining if and only if it is not raining.'}            
    \end{enumerate} 
\end{example}
%
%%% Video K.M.
%
\lang{de}{
Bevor Sie anhand des folgenden Kurztests überprüfen können, ob Sie die Bedeutung
mathematischer Aussagen, deren logische Verknüpfung und Bewertung bereits beherrschen, 
bietet Ihnen das folgende Video nocheinmal eine Zusammenfassung des bisher Erlernten: 
\\
\floatright{\href{https://api.stream24.net/vod/getVideo.php?id=10962-2-10743&mode=iframe&speed=true}
{\image[75]{00_video_button_schwarz-blau}}}\\
\\
}
\lang{en}{
Keep in mind the above example when doing the question below.
}

%
%%%%%%%%%%%%%%%%%%%%%%%%%%%%%%%%%%%%%%%%%%%%%%%%%%%%%%%%%%%%%%%%%%%%%%%%%
%%
%   Quickcheck mit randomquestionpool und mc !?!?!?!??!?
%%
%%%%%%%%%%%%%%%%%%%%%%%%%%%%%%%%%%%%%%%%%%%%%%%%%%%%%%%%%%%%%%%%%%%%%%%%%

\begin{quickcheckcontainer}
 
\randomquickcheckpool{1}{6}

%1a
\begin{quickcheck}
        \text{\lang{de}{
              Welche der folgenden \emph{"`wenn..., dann..."'}-Formen entspricht dem
              umgangssprachlichen Satz \notion{"`Ich gehe jeden Samstag ins Kino"'}?
              } 
              \lang{en}{
              Which of the following implications corresponds to the colloquial sentence 
              \notion{'I go to the cinema every Saturday'}?
              } \\
             }
   \begin{choices}{multiple}

    \begin{choice}
      \text{\lang{de}{Wenn Samstag ist, dann gehe ich ins Kino.}
            \lang{en}{If it is Saturday, then I go to the cinema.}}
      \solution{true}
    \end{choice}
    
    \begin{choice}
      \text{\lang{de}{Wenn ich ins Kino gehe, ist Samstag.}
            \lang{en}{If I go to the cinema, then it is Saturday.}}
      \solution{false}
    \end{choice}    

    \begin{choice}
      \text{\lang{de}{Wenn nicht Samstag ist, dann gehe ich nicht ins Kino.}
            \lang{en}{If it is not Saturday, then I do not go to the cinema.}}
      \solution{false}
    \end{choice}

    \begin{choice}
      \text{\lang{de}{Wenn Samstag ist, dann gehe ich nicht ins Kino.}
            \lang{en}{If it is Saturday, then I do not go to the cinema.}}
      \solution{false}
    \end{choice}

  \end{choices}         
            
\end{quickcheck}

%1b
\begin{quickcheck}
        \text{\lang{de}{Wir formalisieren den umgangssprachlichen Satz \notion{"`Ich gehe jeden Samstag ins Kino"'.}\\
              Wie lautet die zugehörige logische Formel unter Verwendung der Aussagen 
              $A$: "`Es ist Samstag"' und $B$: "`Ich gehe ins Kino"'.
              }
              \lang{en}{
              We formalise the colloquial sentence \notion{'I go to the cinema every Saturday'}. 
              What is the corresponding propositional formula if we take $A$ to mean the proposition 
              'It is Saturday', and $B$ to mean 'I go to the cinema'.
              }\\
            }
   \begin{choices}{unique}

    \begin{choice}
      \text{$A \Rightarrow \neg B$}
      \solution{false}
    \end{choice}

    \begin{choice}
      \text{$A \Rightarrow B$}
      \solution{true}
    \end{choice}

    \begin{choice}
      \text{$\neg A \Rightarrow \neg B$}
      \solution{false}
    \end{choice}

    \begin{choice}
      \text{$B \Rightarrow A$}
      \solution{false}
    \end{choice}

  \end{choices}         
            
\end{quickcheck}
 
%2a
\begin{quickcheck}
        \text{\lang{de}{
              Welche der folgenden \emph{"`wenn..., dann..."'}-Formen entspricht dem
              umgangssprachlichen Satz \notion{"`Ich gehe nur samstags ins Kino"'}?
              }
              \lang{en}{
              Which of the following implications corresponds to the colloquial sentence 
              \notion{'I go to the cinema only on Saturdays'}?
              } \\
            }
   \begin{choices}{multiple}

    \begin{choice}
      \text{\lang{de}{Wenn Samstag ist, dann gehe ich ins Kino.}
            \lang{en}{If it is Saturday, then I go to the cinema.}}
      \solution{false}
    \end{choice}
    
    \begin{choice}
      \text{\lang{de}{Wenn ich ins Kino gehe, ist Samstag.}
            \lang{en}{If I go to the cinema, then it is Saturday.}}
      \solution{true}
    \end{choice}    

    \begin{choice}
      \text{\lang{de}{Wenn nicht Samstag ist, dann gehe ich nicht ins Kino.}
            \lang{en}{If it is not Saturday, then I do not go to the cinema.}}
      \solution{true}
    \end{choice}

    \begin{choice}
      \text{\lang{de}{Wenn Samstag ist, dann gehe ich nicht ins Kino.}
            \lang{en}{If it is Saturday, then I do not go to the cinema.}}
      \solution{false}
    \end{choice}

  \end{choices}         
            
\end{quickcheck}
 
%2b
\begin{quickcheck}
        \text{\lang{de}{
              Wir formalisieren den umgangssprachlichen Satz \notion{"`Ich gehe nur samstags ins Kino"'.}\\
              Wie lautet die zugehörige logische Formel unter Verwendung der Aussagen 
              $A=$"`Es ist Samstag"' und $B=$"`Ich gehe ins Kino"'.
              }
              \lang{en}{
              We formalise the colloquial sentence \notion{'I go to the cinema only on Saturdays'}. 
              What is the corresponding propositional formula if we take $A$ to mean the proposition 
              'It is Saturday', and $B$ to mean 'I go to the cinema'.
              }\\
            }
   \begin{choices}{multiple}

    \begin{choice}
      \text{$A \Rightarrow \neg B$}
      \solution{false}
    \end{choice}
    
    \begin{choice}
      \text{$B \Rightarrow A$}
      \solution{true}
    \end{choice}    

    \begin{choice}
      \text{$A \Rightarrow B$}
      \solution{false}
    \end{choice}

    \begin{choice}
      \text{$\neg A \Rightarrow \neg B$}
      \solution{true}
    \end{choice}

  \end{choices}         
            
\end{quickcheck}
 
%3a
\begin{quickcheck}
        \text{\lang{de}{
              Welche der folgenden \emph{"`wenn..., dann..."'}-Formen entspricht dem
              umgangssprachlichen Satz \notion{"`Samstags gehe ich nie ins Kino"'}?}
              \lang{en}{
              Which of the following implications corresponds to the colloquial sentence 
              \notion{'I never go to the cinema on a Saturday'}?
              }\\
            }
   \begin{choices}{multiple}

    \begin{choice}
      \text{\lang{de}{Wenn Samstag ist, dann gehe ich ins Kino.}
            \lang{en}{If it is Saturday, then I go to the cinema.}}
      \solution{false}
    \end{choice}

    \begin{choice}
      \text{\lang{de}{Wenn nicht Samstag ist, dann gehe ich nicht ins Kino.}
            \lang{en}{If it is not Saturday, then I do not go to the cinema.}}
      \solution{false}
    \end{choice}

    \begin{choice}
      \text{\lang{de}{Wenn Samstag ist, dann gehe ich nicht ins Kino.}
            \lang{en}{If it is Saturday, then I do not go to the cinema.}}
      \solution{true}
    \end{choice}

  \end{choices} 
  
\end{quickcheck}
 
%3b
\begin{quickcheck}
        \text{\lang{de}{Wir formalisieren den umgangssprachlichen Satz \notion{"`Samstags gehe ich nie ins Kino"'.}\\
              Wie lautet die zugehörige logische Formel unter Verwendung der Aussagen 
              $A=$"`Es ist Samstag"' und $B=$"`Ich gehe ins Kino"'.
              }
              \lang{en}{
              We formalise the colloquial sentence \notion{'I never go to the cinema on a Saturday'}. 
              What is the corresponding propositional formula if we take $A$ to mean the proposition 
              'It is Saturday', and $B$ to mean 'I go to the cinema'.
              }\\
            }
   \begin{choices}{unique}

    \begin{choice}
      \text{$A \Rightarrow \neg B$}
      \solution{true}
    \end{choice}

    \begin{choice}
      \text{$\neg A \Rightarrow \neg B$}
      \solution{false}
    \end{choice}

    \begin{choice}
      \text{$A \Rightarrow B$}
      \solution{false}
    \end{choice}

  \end{choices} 
  
\end{quickcheck}

\end{quickcheckcontainer}

%
%%%%%%
% NEU
%%%%%%

%
% Motivationsbeispiel
%
%

\begin{block} 
  \notion{\lang{de}{2. Schritt zur Lösung von \ref{ex:motivation}{$\,$} \textit{Wer sagt die Wahrheit?}}
          \lang{en}{Second step to solve \ref{ex:motivation}{$\,$} \textit{Who is telling the truth?}}}\\
  
\lang{de}{
    Wir werden nun die noch nicht formalisierten Behauptungen der WG-Bewohnerinnen mit den in Schritt 1
    spezifizierten Aussagen $A_1, A_2$ und $A_3$ in Zusammenhang bringen und hieraus logische  
    Aussageformen gemäß Definition {$\,$}\ref{def:objekte_aussagenlogik}{$\,$} ableiten. 
    }
\lang{en}{
    We will now formally express the claims by Lukas' roommates using the propositions $A_1, A_2$ 
    and $A_3$ defined in the first step, using the connectives from definition {$\,$}\ref{def:objekte_aussagenlogik}{$\,$}.
}

 \begin{incremental}

  \step %1 
   \lang{de}{
   Betrachten wir zunächst die Behauptung "`Julia sagt, dass Marie lügt"' in logischem Zusammenhang mit der Aussage 
   "`Julia sagt die Wahrheit"', dann erhalten wir eine logische \emph{"`genau dann, wenn"'}$-$Verknüpfung der 
   Aussagen $A_1$ (Julia sagt die Wahrheit) und $\neg A_2$ (Marie sagt nicht die Wahrheit), denn genau dann, wenn
   Julia die Wahrheit sagt, lügt Marie und wenn Marie lügt, sagt Julia die Wahrheit. Es gilt also die Formel
   }
   \lang{en}{
   Let us express the statement 'Julia says that Marie is lying' using the propositions $A_1$ ('Julia tells 
   the truth'), $\neg A_2$ ('Marie does not tell the truth') and the logical connective 'if and only 
   if'. According to the claim, if Julia tells the truth, then Marie must really be lying; if Marie is 
   lying, then Julia was truthful in her claim, i.e. Julia tells the truth. That is to say, 
   \emph{'Julia tells the truth if and only if Marie does not tell the truth'}:
   }
      \[\mathcal{B}_1: \qquad A_1 \Leftrightarrow (\neg A_2).\] 
   
   \lang{de}{
   Ebenso liefert die Behauptung "`Marie sagt, dass Lara lügt"' in logischem Zusammenhang mit der Aussage 
   "`Marie sagt die Wahrheit"' die Formel
   }
   \lang{en}{
   Similarly, the statement 'Marie says that Lara is lying' can be expressed using the propositions $A_2$ 
   ('Marie tells the truth') and $\neg A_3$ ('Lara does not tell the truth') as
   }
   \[\mathcal{B}_2: \qquad A_2 \Leftrightarrow (\neg A_3).\] 
   
   \lang{de}{
   Bringen wir schließlich noch die Behauptung "`Lara sagt, dass Julia und Marie lügen"' in logischen Zusammenhang mit der 
   Aussage "`Lara sagt die Wahrheit"', bedeutet dies, wenn Lara die Wahrheit sagt, lügt sowohl Julia als auch 
   Marie (und umgekehrt). Es gilt also die \emph{"`genau dann, wenn"'}$-$Formel
   }
   \lang{en}{
   Finally, the statement 'Lara says that Julia and Marie are lying' needs $A_1$, $A_2$ and $A_3$ to 
   be expressed. If Lara is telling the truth, both Julia \emph{and} Marie are lying, and if both 
   Julia and Marie are lying, then Lara is telling the truth, i.e.
   }
    \[\mathcal{B}_3: \qquad A_3 \Leftrightarrow (\neg A_1 \wedge \neg A_2)\]
 
  \step %2
  \lang{de}{
   Um nun zu einem abschließenden Ergebnis zu kommen, müssen wir diese drei Aussageformen $\mathcal{B}_1,
   \mathcal{B}_2\,$ und $\, \mathcal{B}_3$ auf ihre Wahrheitswerte prüfen und herausfinden, unter
   welchen Bedingungen alle drei Aussageformen \emph{wahr} sind. 
   Schauen wir uns nun zunächst an, wie wir hierzu weiter vorgehen können.
  }
  \lang{en}{
  To finally reach a result and determine which roommates tell the truth, we have to find under which 
  conditions all three propositional formulas $\mathcal{B}_1, \mathcal{B}_2\,$ and $\, \mathcal{B}_3$ 
  are true. Next we examine how to proceed with this.
  }

 \end{incremental}
\end{block}

\lang{de}{
    Ordnet man einer Variablen $A$ einen Wahrheitswert \notion{W} $\,(=\,$ wahr$)$ oder 
    \notion{F} $\,(=\,$ falsch$)$ zu, so spricht man von einer \emph{Bewertung der Variablen}. 
    Ordnet man allen Variablen einer Formel Wahrheitswerte
    zu, dann spricht man von einer \emph{Bewertung der Formel} oder \emph{Bewertung der Aussageform.}
    Da man eine Variable auch als \emph{einfache (atomare) Formel} betrachten kann,
    lässt sich die Definiton \ref{def:aussagen} der \emph{Aussage} wie folgt mathematisch formal 
    vereinfachen:
}
\lang{en}{
Assigning a variable a truth-value \notion{T} $\,(=\,$ true$)$ or \notion{F} $\,(=\,$ false$)$ is 
called \emph{evaluating} the variable. Doing this to every variable in a formula is hence called 
\emph{evaluating} the propositional formula, in the same way that expressions can be evaluated. As a 
variable can be considered an \emph{atomic formula}, we can formally simplify the definition 
\ref{def:aussagen} of a \emph{proposition}.
}
 
\begin{definition}[\lang{de}{Aussage als bewertete Aussageform}
                   \lang{en}{Propositions as evaluated propositional formulas}] \label{def:aussagen_af}
        \lang{de}{
        Eine \notion{Aussage} ist eine \notion{bewertete Aussageform/Formel}, 
        also eine \emph{Aussageform / Formel} mit einem fest zugeordneten Wahrheitswert 
        \textbf{wahr (W)} oder \textbf{falsch (F)}.
        }
        \lang{en}{
        A \notion{proposition} is an \notion{evaluated propositional formula}, that is, a 
        \emph{propositional formula} with a fixed truth-value (\textbf{true} or \textbf{false}).
        }

\end{definition}
    

\begin{remark}\label{rem:wahrheitstafeln}
\lang{de}{
Die aus einen einzelnen \emph{Variablen} bzw. \emph{atomaren Formeln} zusammengesetzen \emph{Aussageformen} 
werden somit erst dann zu \emph{Aussagen}, wenn wir ihnen einen Wahrheitswert fest zuordnen können.
Dieser wird je nach Bewertung der auftretenden Variablen unterschiedlich ausfallen. Man spricht in diesem Zusammenhang 
        auch von den \notion{verschiedenen Belegungen der Aussageform}.
\\
        Um Wahrheitswerte einer Aussageformen für verschiedene Belegungen systematisch zu belegen, 
        verwendet man sogenannte \notion{Wahrheitstafeln}, die in den linken Spalten alle 
        möglichen Bewertungskombinationen der einzelnen \emph{Variablen} (Teilaussagen) enthalten, und in weiteren Spalten 
        rechts daneben die hieraus abzuleitenden Bewertungen der \emph{Formeln}, die diese 
        Variablen enthalten.
}
\lang{en}{
A \emph{propositional formula} containing \emph{variables} or smaller \emph{atomic formulas} only 
becomes a proposition when its truth-value is fixed to be either true or false. This value is 
dependent on the values of the variables in the formula.
\\
To systematically find the different values that a propositional formula can take, so-called 
\notion{truth tables} are employed. The left columns of a truth table contain all possible values 
of the individual variables, and to their right are columns containing values of formulas containing 
these variables.
}
\end{remark}  


%
%%% Video K.M. *** NEU: Video 10744 ersetzen durch 11279 (Aussagenlogik_2_a) ***
%
\lang{de}{
Wir betrachten nachfolgend die möglichen Bewertungen der elementaren \emph{Formeln} aus 
Definition \ref{def:objekte_aussagenlogik} und ihre systematische Darstellung mit Hilfe 
von Wahrheitstafeln.
\\\\
Eine anschauliche Erläuterung hierzu finden Sie zum Teil auch in 
folgendem Video, ebenso wie einige Anwendungsbeispiele:
\\
\floatright{\href{https://api.stream24.net/vod/getVideo.php?id=10962-2-11279&mode=iframe&speed=true}
{\image[75]{00_video_button_schwarz-blau}}}\\
}
%
\lang{en}{
Next we consider the connectives from definition \ref{def:objekte_aussagenlogik}, and how the values 
of the most elementary formulas containing each one can be systematically represented using truth 
tables.
}
\\         

        
         
 \begin{enumerate}
    \item[\notion{Negation}]\label{ex:negation}
      \lang{de}{
      Die Wahrheitstafel verdeutlicht, dass durch die \emph{Negation} einer Aussage
      $A$ ihr Wahrheitswert genau umgekehrt wird.
      }
      \lang{en}{
      The truth table means that \emph{negation} of a proposition $A$ switches its value.
      }

      \begin{table}[\cellaligns{ccc}] % {c||c}
        \head 
        $A$ && $\neg A$ 
        \body 
        \lang{de}{W}\lang{en}{T}\lang{zh}{T}\lang{fr}{$\text{V}$} && \lang{de}{F}\lang{en}{F}\lang{zh}{F}\lang{fr}{$\text{F}$} \\
        \lang{de}{F}\lang{en}{F}\lang{zh}{F}\lang{fr}{$\text{F}$} && \lang{de}{W}\lang{en}{T}\lang{zh}{T}\lang{fr}{$\text{V}$}
      \end{table}
          
    \item[\notion{\lang{de}{Konjunktion}\lang{en}{Conjugation}}]\label{ex:konjunktion}
      \lang{de}{
      Die durch die \emph{Konjunktion} oder auch \emph{"`UND"'}$-$Verknüpfung zweier (Teil-)Aussagen gebildete Aussageform wird nur dann zu
      einer wahren Aussage, wenn beide Teilaussagen wahr sind.
      }
      \lang{en}{
      A propositional formula made of two propositions connected by a \emph{conjunction} or 
      \emph{'AND'} connective is true precisely if both propositions are true, and false otherwise.
      }

      \begin{table}[\cellaligns{cccc}] % {c|c||c}
        \head
        $A$ & $B$ && $A\wedge B$ 
        \body
        \lang{de}{W}\lang{en}{T}\lang{zh}{T}\lang{fr}{$\text{V}$} & \lang{de}{W}\lang{en}{T}\lang{zh}{T}\lang{fr}{$\text{V}$} && \lang{de}{W}\lang{en}{T}\lang{zh}{T}\lang{fr}{$\text{V}$} \\
        \lang{de}{W}\lang{en}{T}\lang{zh}{T}\lang{fr}{$\text{V}$} & \lang{de}{F}\lang{en}{F}\lang{zh}{F}\lang{fr}{$\text{F}$} && \lang{de}{F}\lang{en}{F}\lang{zh}{F}\lang{fr}{$\text{F}$} \\
        \lang{de}{F}\lang{en}{F}\lang{zh}{F}\lang{fr}{$\text{F}$} & \lang{de}{W}\lang{en}{T}\lang{zh}{T}\lang{fr}{$\text{V}$} && \lang{de}{F}\lang{en}{F}\lang{zh}{F}\lang{fr}{$\text{F}$} \\
        \lang{de}{F}\lang{en}{F}\lang{zh}{F}\lang{fr}{$\text{F}$} & \lang{de}{F}\lang{en}{F}\lang{zh}{F}\lang{fr}{$\text{F}$} && \lang{de}{F}\lang{en}{F}\lang{zh}{F}\lang{fr}{$\text{F}$}
      \end{table}
      
    \item[\notion{\lang{de}{Disjunktion}\lang{en}{Disjunction}}]\label{ex:disjunktion}
      \lang{de}{
      Die durch die \emph{Disjunktion}  oder auch \emph{"`ODER"'}$-$Verknüpfung zweier (Teil-)Aussagen gebildete Aussageform ist genau dann 
      eine wahre Aussage, wenn mindestens eine der beiden Teilaussagen wahr ist.
      }
      \lang{en}{
      A propositional formula made of two propositions connected by a \emph{disjunction} or 
      \emph{'OR'} connective is true if at least one of the propositions is true, and false otherwise.
      }

      \begin{table}[\cellaligns{cccc}] % {c|c||c}
        \head
        $A$ & $B$ && $A\vee B$ 
        \body
        \lang{de}{W}\lang{en}{T}\lang{zh}{T}\lang{fr}{$\text{V}$} & \lang{de}{W}\lang{en}{T}\lang{zh}{T}\lang{fr}{$\text{V}$} && \lang{de}{W}\lang{en}{T}\lang{zh}{T}\lang{fr}{$\text{V}$} \\
        \lang{de}{W}\lang{en}{T}\lang{zh}{T}\lang{fr}{$\text{V}$} & \lang{de}{F}\lang{en}{F}\lang{zh}{F}\lang{fr}{$\text{F}$} && \lang{de}{W}\lang{en}{T}\lang{zh}{T}\lang{fr}{$\text{V}$} \\
        \lang{de}{F}\lang{en}{F}\lang{zh}{F}\lang{fr}{$\text{F}$} & \lang{de}{W}\lang{en}{T}\lang{zh}{T}\lang{fr}{$\text{V}$} && \lang{de}{W}\lang{en}{T}\lang{zh}{T}\lang{fr}{$\text{V}$} \\
        \lang{de}{F}\lang{en}{F}\lang{zh}{F}\lang{fr}{$\text{F}$} & \lang{de}{F}\lang{en}{F}\lang{zh}{F}\lang{fr}{$\text{F}$} && \lang{de}{F}\lang{en}{F}\lang{zh}{F}\lang{fr}{$\text{F}$}
      \end{table}

    \item[\notion{\lang{de}{Äquivalenz}\lang{en}{Equivalence}}]\label{ex:equivalence}
       \lang{de}{
       Die \emph{Äquivalenz} $A \Leftrightarrow B$ zweier (Teil-)Aussagen $A$ und $B$ ist genau dann 
       wahr, wenn $A$ und $B$ den gleichen Wahrheitsgehalt haben, also beide falsch oder 
       beide wahr sind.
       }
       \lang{en}{
       A propositional formula made of two propositions connected by an \emph{equivalence} or 
       \emph{'IF AND ONLY IF'} or \emph{biconditional} connective is true precisely if both propositions 
       have the same value (both are true or both are false), and false otherwise.
       }

      \lang{de}{
      \begin{table}[\cellaligns{cccc}] 
        \head
        $A$ & $B$ && $A \Leftrightarrow B$ 
        \body
        W & W && W \\
        W & F && F \\
        F & W && F \\
        F & F && W
      \end{table}}
      \lang{en}{
      \begin{table}[\cellaligns{cccc}] 
        \head
        $A$ & $B$ && $A \Leftrightarrow B$ 
        \body
        T & T && T \\
        T & F && F \\
        F & T && F \\
        F & F && T
      \end{table}} 

    \item[\notion{\lang{de}{Implikation}\lang{en}{Implication}}]\label{ex:implication}
       \lang{de}{
       Eine sehr wichtige Verknüpfung ist die Implikation $A \Rightarrow B$ zweier (Teil-)Aussagen 
       $A$ und $B$. Sie ist genau dann falsch ist, wenn $A$ zwar wahr, aber $B$ falsch ist. 
       }   %Sie ist genau dann falsch ist - typo? - Niccolo
       \lang{en}{
       An implication $A \Rightarrow B$ between two propositions $A$ and $B$ is false precisely if 
       $A$ is true, but $B$ is false. Otherwise $A \Rightarrow B$ is true.
       }

      \lang{de}{
      \begin{table}[\cellaligns{cccc}] 
        \head
        $A$ & $B$ && $A \Rightarrow B$ 
        \body
        W & W && W \\
        W & F && F \\
        F & W && W \\
        F & F && W
      \end{table}}
      \lang{en}{
      \begin{table}[\cellaligns{cccc}] 
        \head
        $A$ & $B$ && $A \Rightarrow B$ 
        \body
        T & T && T \\
        T & F && F \\
        F & T && T \\
        F & F && T
      \end{table}}

       \lang{de}{
       Die \emph{Implikation}, aber auch die \emph{Äquivalenz} schauen wir uns im folgenden 
       Abschnitt \ref{sec:folgerung} noch genauer an.
       }
       \lang{en}{
       We further examine \emph{implication} and \emph{equivalence} in section \ref{sec:folgerung}.
       }

 \end{enumerate}

 \lang{de}{
 Besonders hilfreich sind die Wahrheitstafeln zur Bestimmung der Wahrheitswerte für komplexere
 Formeln. Man zerlegt hierzu die komplexe Formel dann in Teil-Formeln mit möglichst elementaren 
 Verknüpfungen und fügt in der Wahrheitstafel für jede Teil-Formel eine neue Spalte ein.
 }
 \lang{en}{
 Truth tables are especially helpful for evaluating the truth-values of more complex formulas. This 
 is because long formulas can be decomposed into smaller ones so that each sub-formula can be 
 evaluated step by step, column by column.
 }
 
 \begin{example}
  \begin{enumerate}[alph]
   \item \lang{de}{
         Wir betrachten die Formel $\neg (A\wedge \neg B)$ und deren Wahrheitswerte mit Hilfe einer 
         Wahrheitstafel:
         \begin{table}[\cellaligns{cccccc}] 
          \head
          $A$ & $B$ && $\neg B$ & $A\wedge (\neg B)$ & $\neg (A\wedge \neg B)$
          \body
          W & W && F &  F & W \\
          W & F && W &  W & F \\
          F & W && F &  F & W \\
          F & F && W &  F & W
        \end{table} 
        Die Formel $\neg (A\wedge \neg B)$ ist also nur dann falsch, wenn $A$ wahr und $B$ falsch ist.
        }
        \lang{en}{
        We evaluate the formula $\neg (A\wedge \neg B)$ using a truth table:
        \begin{table}[\cellaligns{cccccc}] 
          \head
          $A$ & $B$ && $\neg B$ & $A\wedge (\neg B)$ & $\neg (A\wedge \neg B)$
          \body
          T & T && F &  F & T \\
          T & F && T &  T & F \\
          F & T && F &  F & T \\
          F & F && T &  F & T
        \end{table}
        The formula $\neg (A\wedge \neg B)$ is therefore only false if $A$ is true and $B$ is false.
        }
     
%
%%% Video K.M.
%
    \item \lang{de}{
          In dem folgenden Video untersuchen wir die Formeln \[(A \vee B)\wedge C \; \text{und}  
          \; A \vee (B \wedge C ),\] die sich nur durch die Klammersetzung unterscheiden.  
          Die Verknüpfung der Variablen erfolgt also in unterschiedlicher Reihenfolge. Hat dies
          Einfluss auf den jeweiligen Wahrheitswert der Formel?
          
%QS        \floatright
           \center{\href{https://api.stream24.net/vod/getVideo.php?id=10962-2-10960&mode=iframe&speed=true}
          {\image[75]{00_video_button_schwarz-blau}}}\\
          }
          \lang{en}{
          Completing a truth table for the formula $\; (A \vee B)\wedge C \;$ and 
          another for the formula $\; A \vee (B \wedge C )$ reveals that the placement of 
          the parentheses here is important, i.e. these two formulas are not the same.
          }
          \\ 
      
% 

  \end{enumerate}
 \end{example}

%
%%%%%%
% NEU
%%%%%%

%
% Motivationsbeispiel
%
%

\begin{block} 
  \notion{\lang{de}{3. Schritt zur Lösung von \ref{ex:motivation}{$\,$} \textit{Wer sagt die Wahrheit?}}
          \lang{en}{Third step to solve \ref{ex:motivation}{$\,$} \textit{Who is telling the truth?}}}\\
  
\lang{de}{
          Mithilfe einer Wahrheitstafel können wir nun die im 2. Schritt gefundenen Aussageformen 
          $\mathcal{B}_1, \mathcal{B}_2 \,$ und $\, \mathcal{B}_3$ auf ihre Wahrheitswerte prüfen, 
          indem wir zunächst die enthaltenen Aussagen-Variablen $A_1, A_2 \,$ und $\, A_3$ mit allen 
          möglichen Wahrheitswert-Belegungskombinationen bewerten. \\
          Die Lösung unseres Problems können wir dann aus der Zeile ablesen, in der alle drei
          Aussageformen  $\mathcal{B}_1, \mathcal{B}_2 \,$ und $\, \mathcal{B}_3$ den Wahrheitswert 
          \notion{W} $\,(=\,$ wahr$)$ haben.
}
\lang{en}{
Using a truth table we can now evaluate the propositional formulas $\mathcal{B}_1, \mathcal{B}_2 \,$ and $\, \mathcal{B}_3$ that we found in the second step. We evaluate the formulas using all possible 
values of the variables $A_1, A_2 \,$ and $\, A_3$. The solution to our problem is then the row 
in which all three formulas $\mathcal{B}_1, \mathcal{B}_2 \,$ and $\, \mathcal{B}_3$ have the value 
\notion{T} $\,(=\,$ true$)$.
}
 \begin{incremental}

  \step %1 
     \lang{de}{
     \begin{table}[\cellaligns{ccccccccccc}] 
      \head
      $A_1$ & $A_2$ & $A_3$ &&  $\neg A_1$ & $\neg A_2$ & $\neg A_1 \wedge \neg A_2$ & $\neg A_3$ 
                             & $A_1 \Leftrightarrow (\neg A_2)$ 
                             & $A_2 \Leftrightarrow (\neg A_3)$                             
                             & $A_3 \Leftrightarrow (\neg A_1 \wedge \neg A_2)$
      \body
      W & W & W && F & F & F & F & F & F & F  \\
      W & W & F && F & F & F & W & F & W & W \\
      W & F & W && F & W & F & F & W & W & F \\
      W & F & F && F & W & F & W & W & F & W \\      
      F & W & W && W & F & F & F & W & F & F \\
      \notion{F} & \notion{W} & \notion{F} && W & F & F & W & \notion{W} & \notion{W} & \notion{W} \\
      F & F & W && W & W & W & F & F & W & W \\
      F & F & F && W & W & W & W & F & F & F        
    \end{table}}
    \lang{en}{
    \begin{table}[\cellaligns{ccccccccccc}] 
      \head
      $A_1$ & $A_2$ & $A_3$ &&  $\neg A_1$ & $\neg A_2$ & $\neg A_1 \wedge \neg A_2$ & $\neg A_3$ 
                             & $A_1 \Leftrightarrow (\neg A_2)$ 
                             & $A_2 \Leftrightarrow (\neg A_3)$                             
                             & $A_3 \Leftrightarrow (\neg A_1 \wedge \neg A_2)$
      \body
      T & T & T && F & F & F & F & F & F & F  \\
      T & T & F && F & F & F & T & F & T & T \\
      T & F & T && F & T & F & F & T & T & F \\
      T & F & F && F & T & F & T & T & F & T \\      
      F & T & T && T & F & F & F & T & F & F \\
      \notion{F} & \notion{T} & \notion{F} && T & F & F & T & \notion{T} & \notion{T} & \notion{T} \\
      F & F & T && T & T & T & F & F & T & T \\
      F & F & F && T & T & T & T & F & F & F        
    \end{table}}
 
  \step %2
  \lang{de}{
   Aus der Tabelle können wir nun ablesen, dass unsere Aussageformen $\mathcal{B}_1, \mathcal{B}_2 \,$ und $\, \mathcal{B}_3$
   genau dann \notion{wahr} sind, wenn 
   \begin{itemize}
   \item die Aussage $A_1\,$ \notion{falsch} ist, wenn also Julia lügt, und
   \item die Aussage $A_3\,$ \notion{falsch} ist, also Lara ebenfalls lügt, und
   \item die Aussage $A_2\,$ \notion{wahr} ist, also Marie die Wahrheit sagt.
   \end{itemize}
   \\
   Damit haben wir nun ermittelt, dass Julia und Lara grundsätzlich lügen und nur Marie die Wahrheit 
   sagt. Julia und Lara haben also das Bier von Lukas genommen. 
   }
   \lang{en}{
   From the truth table it is clear that the formulas $\mathcal{B}_1, \mathcal{B}_2 \,$ and 
   $\, \mathcal{B}_3$ are all \notion{true} if and only if
   \begin{itemize}
   \item the proposition $A_1\,$ is \notion{false}, so Julia is lying, and
   \item the proposition $A_3\,$ is \notion{false}, so Lara is also lying, and
   \item the proposition $A_2\,$ is \notion{true}, so Marie is telling the truth.
   \end{itemize}
   }

 \end{incremental}
\end{block}

%
%%% Video K.M. *** NEU: Aussagenlogik_2_b (11280)***
%
\lang{de}{
    Betrachten wir die Wahrheitswerte zu folgenden einfachen Formeln:
\\
      \begin{table}[\align{c}\cellaligns{ccccc}] 
        \head
        $A$ && $\neg A$ & $A \vee (\neg A)$ & $A \wedge (\neg A)$
        \body
        W && F & W & F \\
        F && W & W & F \\
      \end{table} 
\\
    Wir stellen fest, dass die Formel $A \vee (\neg A)$ immer wahr, 
    die Formel $A \wedge (\neg A)$ hingegen immer falsch ist, 
    egal welchen Wahrheitswert man der Variablen $A$ zuordnet.
    \\
    Diese Besonderheit wird in folgendem Video genauer untersucht:
\\      
          \floatright{\href{https://api.stream24.net/vod/getVideo.php?id=10962-2-11280&mode=iframe&speed=true}
          {\image[75]{00_video_button_schwarz-blau}}}\\
          \\ 
    Zusammenfassend halten wir fest
}
\lang{en}{
Consider the truth table:
      \begin{table}[\align{c}\cellaligns{ccccc}] 
        \head
        $A$ && $\neg A$ & $A \vee (\neg A)$ & $A \wedge (\neg A)$
        \body
        T && F & T & F \\
        F && T & T & F \\
      \end{table}
\\
The formula $A \vee (\neg A)$ is always true, and the formula $A \wedge (\neg A)$ is always false. 
This motivates the following definition.
}

    
    \begin{definition}[\lang{de}{Tautologie und Widerspruch}\lang{en}{Tautologies and contradictions}] \label{def:tautologie_widerspruch}
     \begin{itemize}
        \item \lang{de}{
              Eine Formel, die für alle Bewertungen \emph{wahr} ist, heißt 
              \emph{Tautologie} oder \emph{allgemeingültige Formel}.
              }
              \lang{en}{
              A formula that is \emph{true} for all values of its variables 
              is called a \emph{tautology}.
              }
        \item \lang{de}{
              Eine Formel, die für alle Bewertungen \emph{falsch} ist, heißt 
              \emph{Widerpruch}, \emph{Kontradiktion} oder \emph{unerfüllbare Formel}.
              }
              \lang{en}{
              A formula that is \emph{false} for all values of its variables 
              is called an \emph{unsatisfiable formula}.
              }
     \end{itemize}
    \end{definition} 
               
   \begin{remark} \label{rem:tautologie_widerspruch}
     \begin{itemize}
        \item \lang{de}{Die Negation einer Tautologie ist stets ein Widerspruch, und }
              \lang{en}{Negating a tautology gives an unsatisfiable formula.}
        \item \lang{de}{die Negation eines Widerspruchs ist stets eine Tautologie.}
              \lang{en}{Negating an unsatisfiable formula gives a tautology.}
     \end{itemize}
    \end{remark} 

%         

%%%%%%%%%%%%%%%%%%
%
% Weitere Vertiefung der Aussagenlogik, wie 
%   - die Herleitung und Anwendung von Logikregeln (Distributivgesetz, de Morgan, etc) 
%   - die Definition weiterer Begriffe wie z.B. die "Tautologie", der "Widerspruch"
%
%   werden hier nicht weiter vertieft ..., obwohl im OMB+ teilweise vorhanden!!!
%
%%%%%%%%%%%%%%%%%%

% Wir werden die Aussagenlogik an dieser Stelle nicht weiter vertiefen, sondern uns nachfolgend mit ihrer Anwendung
% in der Mathematik beschäftigen. 

\lang{de}{
Wir haben nun die grundlegenden Definitionen über Aussagen, ihre logischen Verknüpfungen und Bewertungen
kennengelernt 
%
%%% Video K.M.
%
und beenden diesen Abschnitt mit einem letzten kleinen Beispiel, das uns noch
einmal verdeutlicht, wie hilfreich logisches Verständnis im realen Leben sein kann: 
\\
\floatright{\href{https://api.stream24.net/vod/getVideo.php?id=10962-2-10922&mode=iframe&speed=true}
{\image[75]{00_video_button_schwarz-blau}}}
\\\\\\
Im Folgenden werden wir uns nun mit der Anwendung dieser sogenannten \emph{"`Objekte der 
Aussagenlogik"' \,} in der Mathematik beschäftigen.
}
%

\lang{en}{
In the following section we will consider how some of these logical constructions are applied to 
mathematics.
}

\section{\lang{de}{Folgerung/Implikation und Äquivalenz}
         \lang{en}{Implication and equivalence}}\label{sec:folgerung}

\lang{de}{
Die Aussagenlogik spielt eine sehr große Rolle in der Mathematik und kommt in verschiedensten 
Teilgebieten der Mathematik zur Anwendung.
}
\lang{en}{
Propositional logic plays a very large role in mathematics and can be applied to various fields.
}

\begin{itemize}
    \item \lang{de}{
        Zum Beispiel findet man logische Verknüpfungen von Aussagen häufig bei der Definition und 
        Beschreibung von Mengen und \ref[content_01_zahlenmengen][Mengenoperationen]{def:mengenoperationen}.
        So lässt sich beispielsweise die Vereinigung zweier Mengen $M$ und $N$ durch die logische Formel  
        \[x \in M\cup N \Leftrightarrow  (x\in M \vee x\in N) \]
        beschreiben. Man schreibt deshalb auch $M\cup N =\{x\mid x\in M \vee x\in N\}$ in der Kenntnis, dass für 
        alle Elemente $x$ der Menge $M\cup N$ die Disjunktion $(x\in M \vee x\in N)$ eine wahre Aussagen ist.
    \\
        Analog gilt
        \begin{itemize}                    
            \item   für den Schnitt von $M$ und $N$: $\; x \in M\cap N \Leftrightarrow (x\in M \wedge x\in N)$, \\
                    man schreibt deshalb auch $\; M\cap N =\{x\mid x\in M \wedge x\in N\} \,$, und
            \item   für die Differenz von $M$ und $N$: $\; x \in M\setminus N \Leftrightarrow (x\in M \wedge \neg (x\in N))$, \\
                    man schreibt deshalb auch $\; M\setminus N =\{x\mid x\in M \wedge \neg (x\in N)\}.$   
        \end{itemize} 
        }
        \lang{en}{
        For example, propositions and connectives are often used when describing sets and 
        \ref[content_01_zahlenmengen][set operations]{def:mengenoperationen}. For example, the union 
        of two sets $M$ and $N$ can be described by the logical formula 
        \[x \in M\cup N \Leftrightarrow  (x\in M \vee x\in N). \]
        Using this, we can write $M\cup N =\{x\mid x\in M \vee x\in N\}$, as $x$ being an element of 
        $M\cup N$ is equivalent to the disjunction $(x\in M \vee x\in N)$ being true.
        \\
        Analogously,
        \begin{itemize}                    
            \item   for the intersection of $M$ and $N$: $\; x \in M\cap N \Leftrightarrow (x\in M \wedge x\in N)$, \\
                    so we write $\; M\cap N =\{x\mid x\in M \wedge x\in N\} \,$, and
            \item   for the difference of $M$ and $N$: $\; x \in M\setminus N \Leftrightarrow (x\in M \wedge \neg (x\in N))$, \\
                    so we write $\; M\setminus N =\{x\mid x\in M \wedge \neg (x\in N)\}.$   
        \end{itemize}
        }

    \item \lang{de}{
          Von besonderer Bedeutung in der Mathematik ist auch die \emph{Äquivalenz \,} als \emph{\, "`... genau dann, wenn ..."'}
          $-$Verknüpfung bei der Umformung von Gleichungen und Ungleichungen. Auf diese sogenannten 
          \emph{Äquivalenzumformungen\,} werden wir in Abschnitt \ref{sec:aequivalenz} noch weiter eingehen.
          }
          \lang{en}{
          Also especially important in mathematics is \emph{equivalence}, often written as 
          \emph{'if and only if'}. This will be looked at further in section \ref{sec:aequivalenz}.
          }
        
    \item \lang{de}{
          Die \emph{Implikation\,} spielt unter anderem in der mathematischen Beweisführung eine wichtige Rolle.
        Mathematische \emph{Sätze} oder \emph{Theoreme} basieren meist auf einer \emph{Voraussetzung} $\,A$, 
        aus der dann eine \emph{Behauptung}  $\,B\,$ folgt. Es handelt sich hierbei also um eine Aussage vom 
        Typ \textit{"`wenn $A$ gilt, dann gilt auch $B$."'} und enstspricht
        somit der logischen Struktur einer \emph{Implikation} oder auch \emph{Folgerung}.
        }
        \lang{en}{
        \emph{Implication} plays an important role in writing mathematical proofs, amongst other 
        things. Mathematical \emph{results} and \emph{theorems} often depend on a \emph{condition} 
        $\,A$ from which a statement $\,B\,$ follows. This is represented by a proposition 
        \textit{'if $A$ holds, then $B$ holds'}, a logical \emph{implication}.
        }
\end{itemize}

\lang{de}{
Aufgrund ihrer besonderen Bedeutung werden wir die \emph{Implikation} und die \emph{Äquivalenz} 
noch etwas genauer betrachten. Hierzu definieren wir zunächst beide logischen Verknüpfungen allgemein 
für Aussageformen (Formeln) und untersuchen anschließend in den Beispielen \ref{ex:implikation} und 
\ref{ex:aqiv} ihre Wahrheitswerte in Abhängigkeit von allen möglichen Belegungen der Formeln.
}
\lang{en}{
Due to their particular importance, we consider \emph{implication} and \emph{equivalence} further. 
Firstly we introduce their general definitions for propositional formulas, and then work through 
examples \ref{ex:implikation} and \ref{ex:aqiv}.
}

\begin{definition}[\lang{de}{Implikation/Folgerung}\lang{en}{Implication}] \label{def:implikation}

\lang{de}{
  Für zwei Aussageformen $\mathcal{A}$ und $\mathcal{B}\,$ wird die \notion{\emph{Implikation (Folgerung)}}
\begin{center}  
  \emph{\,"`Wenn $\mathcal{A}$ wahr ist, dann ist auch $\mathcal{B}$ wahr."'\,}
\end{center}
ausgedrückt durch die Schreibweise
 
	\begin{table}[\class{layout} \align{c} \cellaligns{llclcl}]
		
		& $\mathcal{A} \Rightarrow \mathcal{B} \quad$
        &  gelesen als &$\quad$ \emph{"`aus $\mathcal{A}$ folgt $\mathcal{B}$"'} 
        & $\;$ oder &$\;$ \emph{"`$\mathcal{A}$ impliziert $\mathcal{B}$"'} 
        \\
        
   oder $\quad$ & $\mathcal{B} \Leftarrow \mathcal{A} \quad$
        &  gelesen als &$\quad$\emph{"`$\mathcal{B}$ folgt aus $\mathcal{A}$"'} &&\\
	
	\end{table}       
  
 Dabei heißt $\mathcal{A}$ die \notion{Prämisse} und $\mathcal{B}$ die \notion{Konklusion} der Folgerung.
}
\lang{en}{
For two propositional formulas $\mathcal{A}$ and $\mathcal{B}\,$, the \notion{\emph{implication}} 
\begin{center}  
  \emph{\,'If $\mathcal{A}$ is true, then $\mathcal{B}$ is true.'\,}
\end{center}
is expressed as 
\begin{table}[\class{layout} \align{c} \cellaligns{llclcl}]
		& $\mathcal{A} \Rightarrow \mathcal{B} \quad$
        & said &$\;$ \emph{'$\mathcal{A}$ only if $\mathcal{B}$'} 
        & $\;$ or &$\;$ \emph{'$\mathcal{A}$ implies $\mathcal{B}$'} 
        \\
        or $\quad$ & $\mathcal{B} \Leftarrow \mathcal{A} \quad$
        & said &$\;$ \emph{'$\mathcal{B}$ if $\mathcal{A}$'}
        & $\;$ or &$\;$ \emph{'$\mathcal{B}$ follows from $\mathcal{A}$'} \\
	\end{table}
}



%  Die Folgerung $\; \mathcal{A} \Rightarrow \mathcal{B}\;$ ist also selbst eine Aussage und sie ist 
%  genau dann wahr, wenn $\mathcal{A}$ und auch $\mathcal{B}$ beide wahr sind, oder wenn $\mathcal{A}$ falsch 
%  ist, wie in der folgenden Tabelle, die man auch \emph{Wahrheitstafel} nennt, zu entnehmen.

  \lang{de}{
  Die Folgerung \textbf{$\; \mathcal{A} \Rightarrow \mathcal{B}\;$} ist per Definition selbst wieder 
  eine Aussageform, die genau dann zu einer \emph{wahren} Aussage wird, wenn entweder 
  $\mathcal{A}$ und $\mathcal{B}$ beide \emph{wahr} sind, oder wenn $\mathcal{A}$ 
  \emph{falsch} ist.
  }
  \lang{en}{
  The implication \textbf{$\; \mathcal{A} \Rightarrow \mathcal{B}\;$} is defined to be a propositional 
  formula that is a true proposition precisely either if $\mathcal{A}$ and $\mathcal{B}$ are both 
  \emph{true}, or if $\mathcal{A}$ is \emph{false}.
  }
\end{definition}  
\lang{de}{
  Die Folgerung $\; \mathcal{A} \Rightarrow \mathcal{B}\;$ genügt also der folgenden \emph{Wahrheitstafel}:
  \begin{table}[\align{c}\cellaligns{cccc}] 
    \head                %[\background{rgb(220,205,100)}]
    $\mathcal{A}$ & $\mathcal{B}$ && $\mathcal{A}\Rightarrow \mathcal{B}$ 
    \body
    W & W && W \\
    W & F && F \\
    F & W && W \\
    F & F && W 
  \end{table}
}
\lang{en}{
The implication $\; \mathcal{A} \Rightarrow \mathcal{B}\;$ satisfies the following \emph{truth table}. 
    \begin{table}[\align{c}\cellaligns{cccc}] 
    \head                %[\background{rgb(220,205,100)}]
    $\mathcal{A}$ & $\mathcal{B}$ && $\mathcal{A}\Rightarrow \mathcal{B}$ 
    \body
    T & T && T \\
    T & F && F \\
    F & T && T \\
    F & F && T 
  \end{table}
}
  
\begin{remark}  
  \lang{de}{
  Besonders hervorzuheben ist also die Tatsache, dass die Implikation $\;\mathcal{A} \Rightarrow \mathcal{B}\;$
  immer wahr ist, wenn die Prämisse $\mathcal{A}$ falsch ist. Dabei spielt es keine Rolle, welchen Wahrheitswert 
  die Konklusion $\mathcal{B}$ hat.\\
  Sprachlich sehen wir das ein, wenn wir in den Konjunktiv wechseln: Wenn $\mathcal{A}$ wahr \emph{wäre},
  dann \emph{wäre} auch $\mathcal{B}$ wahr.
  }
  \lang{en}{
  Especially noteworthy is the fact that the implication $\;\mathcal{A} \Rightarrow \mathcal{B}\;$ is 
  always true if the premise $\mathcal{A}$ is false, regardless of the value of $\mathcal{B}$. \\
  This can be aligned with our linguistic intuition by rephrasing it as: if $\mathcal{A}$ were true, 
  then $\mathcal{B}$ would be true. %Does this help explain the vacuous truth in English? - Niccolo
  }
\end{remark}


\begin{example}\label{ex:implikation}
%%%%%%%%%%%%%%%%%%%%%%%%%%%%%
% Das 1. Beispiel wurde gestrichen, da es nicht ganz schlüssig ist und wir uns einleitend fest-
% gelegt haben, dass wir uns nun mit der Anwendung der Aussagenlogik in der Mathematik beschäftigen 
% werden ...
%%%%%%%%%%%%%%%%%%%%%%%%%%%%%
% \begin{enumerate}[alph]
%  \item Betrachten wir die beiden Aussagen $A:$ \textit{"`Es regnet."'} und $B:$ \textit{"`Die Straße ist nass."'}.\\
%      Dann lautet die Folgerung $\,A \Rightarrow B$:
%      \begin{center}
%      \textit{"`Wenn es regnet, dann ist die Straße nass."'}
%      \end{center}
%      Diese Aussage wäre genau dann falsch, wenn es regnen würde und die Straße dabei trocken bliebe. Da die Straße bei 
%      Regen aber immer nass wird, kann man diese Konstellation ($A=\,$W und $B=\,$F) ausschließen. 
%      Wir können also sagen, die Aussage \textit{"`Wenn es regnet, dann ist die Straße nass."'} ist, unabhängig davon, 
%      was für ein Wetter herrscht, immer wahr, denn wenn es regnet, also die Aussage $A$ wahr ist, ist auch die Straße nass, d.h. die
%      Aussage $B$ auch wahr. Regnet es jedoch nicht, so ist die Aussage $A$ falsch und die Folgerung per Definition wahr.
%      \\
%      Anders verhält es sich für die umgekehrte Folgerung $B \Rightarrow A$.
%      Die Straße könnte nass sein, auch wenn es nicht regnet. In diesem Fall wäre die Folgerung $B \Rightarrow A\,$ falsch,
%      weil die Prämisse $B$ wahr ist, aber die Konklusion $A$ falsch.
%
%  \item Betrachten wir nun die beiden Aussagen $A$: "`$x = 1$"' und $B$: "`$x^2 = 1$"'\\

\lang{de}{
    Wir betrachten die beiden Formeln $\;\mathcal{A}:\,$ "`$x = 1$"' und $\;\mathcal{B}:\,$ "`$x^2 = 1$"',
    deren Wahrheitswerte jeweils abhängig sind vom Wert der Variablen $x$.\\
    
    Der Wahrheitswert der Implikation $\mathcal{A} \Rightarrow \mathcal{B}$ hingegen ist unabhängig vom 
    tatsächlichen Wert von $x$ immer \emph{wahr}, denn wenn $\mathcal{A}\,$ \emph{wahr\,} ist, gilt $x=1\,$ 
    und folglich $\;x^2=1^2=1$. Also ist in diesem Fall $\mathcal{B}\,$ \emph{wahr} und kann niemals \emph{falsch} sein.\\
    Wenn $\mathcal{A}\,$ \emph{falsch \,} ist, spielt der von $x$ abhängige Wahrheitswert 
    von $\mathcal{B}$ keine Rolle, da die Implikation $\mathcal{A} \Rightarrow \mathcal{B}$ aufgrund der \emph{falschen} Prämisse
    per Definition \emph{wahr} ist.\\
    
%      
%    $\mathcal{B}: x^2=1$ kann sowohl \emph{wahr} als auch \emph{falsch} sein, denn \\
%                 &&  für $x=-1 \;$ ist $x^2 = (-1)^2=1$, also ist $\mathcal{B}\,$ \emph{wahr} und \\
%                 &&  für $x=4 \;$ ist $x^2 = 4^2=16 \neq 1$, also ist $\mathcal{B}\,$ \emph{falsch}.\\
%
     Wir fassen die möglichen Bewertungs-Konstellationen für die Formeln $\mathcal{A}\,$ und $\mathcal{B}\,$, und somit
     die möglichen Belegungen für $\, \mathcal{A} \Rightarrow \mathcal{B}$ in einer Wahrheitstafel zusammen:
}
\lang{en}{
Consider the two formulas $\;\mathcal{A}:\,$ '$x = 1$' and $\;\mathcal{B}:\,$ '$x^2 = 1$', whose 
truth-values are both dependent on the value of the variable $x$.
\\
On the other hand, the truth-value of the implication $\mathcal{A} \Rightarrow \mathcal{B}$ is 
always \emph{true}, for all values of $x$. This is because when $\mathcal{A}\,$ is 
\emph{true}, $x=1\,$ and so $\;x^2=1^2=1$, and $\mathcal{B}\,$ is \emph{true}.
\\
If $\mathcal{A}\,$ is \emph{false}, the $x$-dependent truth-value of $\mathcal{B}$ is not relevant 
to the truth-value of the implication $\mathcal{A} \Rightarrow \mathcal{B}$ by its truth-table definition.
\\
The one case for which the implication could be false by its truth-table definition is if 
$\;\mathcal{A}\,$ were true and $\;\mathcal{B}\,$ false, but this is not possible here for any $x\in\R$.
}

\lang{de}{
   \begin{table}[\align{c}\cellaligns{cccc}] 
    \head             
    $\mathcal{A}:\,x = 1$ & $\mathcal{B}:\,x^2 = 1$ && $\mathcal{A} \Rightarrow \mathcal{B}$ 
    \body
    W & W && W \\
    W & F && \textit{nicht möglich}\\
    F & W && W \\
    F & F && W 
  \end{table}
}
\lang{en}{
   \begin{table}[\align{c}\cellaligns{cccc}] 
    \head             
    $\mathcal{A}:\,x = 1$ & $\mathcal{B}:\,x^2 = 1$ && $\mathcal{A} \Rightarrow \mathcal{B}$ 
    \body
    T & T && T \\
    T & F && \textit{not possible}\\
    F & T && T \\
    F & F && T 
  \end{table}
}

\lang{de}{
  Hier können wir nun in der rechten Spalte direkt ablesen, dass die Implikation $\mathcal{A} \Rightarrow \mathcal{B},$ 
  unabhängig vom tatsächlichen Wert von $x$, immer \emph{wahr} ist.
  
  Die umgekehrte Folgerung $\mathcal{B} \Rightarrow \mathcal{A}$ gilt für diese Formeln allerdings im Allgemeinen
  nicht, 
%  d.h. sie ist nicht für alle möglichen Berwertungs-Konstellationen für $\mathcal{B}\,$ und $\mathcal{A}\,$ \emph{wahr}, 
  denn für $x = -1\;$ ist die Aussage $\mathcal{B}\,$ \emph{wahr, \,} da $(-1)^2 = 1, \;$ jedoch die Aussage $\mathcal{A}\,$ 
  \emph{falsch, \,}  da $-1 \neq 1$. Somit erhalten wir folgende Bewertungen:
}
\lang{en}{
From the rightmost column we see that the implication $\mathcal{A} \Rightarrow \mathcal{B},$ is always 
\emph{true} independently from the value of $x$.
\\
The converse implication $\mathcal{B} \Rightarrow \mathcal{A}$ does not hold in general for these 
formulas, as for $x = -1\;$ the statement $\mathcal{B}\,$ is \emph{true} as $(-1)^2 = 1, \;$ even 
though the proposition $\mathcal{A}\,$ is \emph{false}. This gives rise to the following values:
}

\lang{de}{
   \begin{table}[\align{c}\cellaligns{cccc}] 
    \head               
    $\mathcal{B}:\,x^2 = 1$ & $\mathcal{A}:\,x = 1$ && $\mathcal{B} \Rightarrow \mathcal{A}$ 
    \body
    W & W && W \\
    W & F && \textcolor{\#D2691E}{\textbf{F}} \\
    F & W && W \\
    F & F && W 
  \end{table}
}
\lang{en}{
   \begin{table}[\align{c}\cellaligns{cccc}] 
    \head               
    $\mathcal{B}:\,x^2 = 1$ & $\mathcal{A}:\,x = 1$ && $\mathcal{B} \Rightarrow \mathcal{A}$ 
    \body
    T & T && T \\
    T & F && \textcolor{\#D2691E}{\textbf{F}} \\
    F & T && T \\
    F & F && T 
  \end{table}
}

\lang{de}{
  Die Implikation $\mathcal{B} \Rightarrow \mathcal{A}\,$ gilt also nicht allgemein, d.h. sie wird nicht bei 
  jeder Belegung zu einer \emph{wahren} Aussage. Man schreibt in diesem Fall auch $\mathcal{B} \not\Rightarrow \mathcal{A}$.
}
\lang{en}{
The implication $\mathcal{B} \Rightarrow \mathcal{A}\,$ therefore does not hold in general, as it is not 
evaluated as \emph{true} for every possible value of the variables. This can be expressed as $\mathcal{B} \not\Rightarrow \mathcal{A}$.
}

% \end{enumerate}
\end{example}
%
%%% Video K.M. *** NEU: Aussagenlogik_4_a ***
%
\lang{de}{
In folgendem Video wird die Definition der Implikation noch einmal wiederholt und die möglichen 
Bewertungen dieser Verknüfung anhand weiterer Beispiele erläutert.
\\
\floatright{\href{https://api.stream24.net/vod/getVideo.php?id=10962-2-10746&mode=iframe&speed=true}
{\image[75]{00_video_button_schwarz-blau}}}\\\\
}
\lang{en}{
\\
}
%
  
\begin{remark} \label{rem:gueltige_folgerung}
\lang{de}{
  In der Mathematik ist man hauptsächlich an solchen \emph{Implikationen / Folgerungen} % (und \emph{Äquivalenzen})
  interessiert, deren Wahrheitswert unabhängig von der konkreten Belegung der Variablen (in Prämisse und Konklusion) 
  immer \emph{wahr} ist, d.h. die immer glütig sind.
}
\lang{en}{
In mathematics we are primarily interested in those \emph{implications} that are always true, 
regardless of the value of any variables (both in the premise and in the conclusion).
}
\end{remark}

\begin{quickcheck}
		\field{rational}
		\type{input.number}
		\begin{variables}
			\randint{t}{2}{5}
			\function[calculate]{t2}{t^2}	
			\randint{z}{0}{1}
			\function[normalize]{a}{z*t*x+(1-z^2)*t^2*x^2}
			\function[normalize]{c}{z*t^2*x^2+(1-z^2)*t*x}
		\end{variables}

		\text{\lang{de}{Wir betrachten die Formeln}\lang{en}{We consider the formulas}\\ 
			\[ \mathcal{A}:\;  \var{a}    >1 \quad\quad 
               \mathcal{B}:\; -\var{t}x   <-1 \quad\quad 
               \mathcal{C}:\;  \var{c}    >1. \]

        \lang{de}{Welche der Folgerungen ist % \emph{nicht\,} 
        für alle reellen Zahlen $x$ wahr?}
        \lang{en}{Which of the implications is true for all real $x$?}}

        \begin{choices}{multiple}
               %  1)
                    \begin{choice}
                      \text{$\mathcal{A} \Rightarrow \mathcal{B}$}
                      \solution[compute]{z=1}
                    \end{choice}
                    
                %  2)
                    \begin{choice}
                      \text{$\mathcal{A} \Rightarrow \mathcal{C}$}
                      \solution[compute]{z=1}
                    \end{choice}
                    
                %  3)                    
                    \begin{choice}
                      \text{$\mathcal{C} \Rightarrow \mathcal{A}$}
                      \solution[compute]{z=0}
                    \end{choice}
                    
        \end{choices} 
         
		\explanation{\lang{de}{
  Für negative Zahlen $x$ sind $\,(\var{t}x>1) \,$ und $\,(-\var{t}x<-1) \,$ falsche Aussagen. 
        Die Aussage $\, (\var{t2}x^2>1) \,$ ist jedoch wahr, wenn zum Beispiel $x<-1$ ist, denn dann ist $x^2>1$ und 
        damit auch $\var{t2}x^2>1.$\\
		Die Folgerung $\, (\var{t2}x^2>1 \Rightarrow \var{t}x>1) \,$ wird daher für $x<-1$ zu einer falschen Aussage und 
        ist somit \textbf{nicht} für alle rellen Zahlen wahr.
        }
        \lang{en}{
        For negative $x$ the propositions $\,(\var{t}x>1) \,$ and $\,(-\var{t}x<-1) \,$ are false. The 
        proposition $\, (\var{t2}x^2>1) \,$ is however true for e.g. $x<-1$, as then $x^2>1$ and hence 
        $\var{t2}x^2>1.$
        \\
        The implication $\, (\var{t2}x^2>1 \Rightarrow \var{t}x>1) \,$ hence becomes a false proposition 
        for $x<-1$ and so \textbf{does not hold} for all real $x$.
        }}

\end{quickcheck}

%%%%%%%%%%%%%%%%%%%%%%%%%%%%%%%%%%%%%%%%%%%%%%%%%%%%%%%%%%%%%
%   Alte Version ohne Multiple Choice
%%%%%%%%%%%%%%%%%%%%%%%%%%%%%%%%%%%%%%%%%%%%%%%%%%%%%%%%%%%%%
%
%\begin{quickcheck}
%		\field{rational}
%		\type{input.number}
%		\begin{variables}
%			\randint{t}{2}{5}
%			\function[calculate]{t2}{t^2}	
%			\randint{z}{-1}{1}
%			\function[calculate]{y}{z+2}
%			\function[normalize]{x1}{z*(z-1)*A/2+(1-z^2)*C+z*(z+1)*B/2}
%			\function[normalize]{x2}{z*(z-1)*B/2+(1-z^2)*A+z*(z+1)*C/2}
%			\function[normalize]{x3}{z*(z-1)*C/2+(1-z^2)*B+z*(z+1)*A/2}
%		\end{variables}
%
%
%
%			\text{Wir betrachten die Aussagen\\ 
%				\[ A:\; \var{t}x>1 \quad\quad B:\; -\var{t}x<-1 \quad\quad C:\; \var{t2}x^2>1. \] 
%			Welche der folgenden Folgerungen ist \emph{nicht} für alle reellen Zahlen $x$ wahr?
%			\[ (1): \var{x3}\Rightarrow \var{x1} \quad\quad (2): \var{x1}\Rightarrow \var{x2} \quad\quad (3): \var{x2}\Rightarrow \var{x3} \]
%			Die Folgerung mit der Nummer \ansref.}
%		% Folgerung C=>A ist nicht immer richtig. Variablen so gesetzt, dass Lösung =y ist.
%		\begin{answer}
%			\solution{y}
%		\end{answer}
%		\explanation{Für negative Zahlen $x$ sind $A$ und $B$ falsche Aussagen. Die Aussage $C$ ist
%		jedoch wahr, wenn zum Beispiel $x<-1$ ist, weil dann $x^2>1$ und damit auch $\var{t2}x^2>1$ ist.\\
%		Die Folgerung $\, C\Rightarrow A$ ist daher nicht immer wahr.
%		}
%	\end{quickcheck}

\begin{definition}[\lang{de}{Äquivalenz}\lang{en}{Equivalence}] \label{def:aequvalenz}
    \lang{de}{
         Zwei Aussageformen $\mathcal{A}$ und $\mathcal{B}$ heißen \notion{äquivalent} 
         oder \notion{gleichwertig}, wenn gilt
% \\
%            \begin{table}[\class{layout} \align{c} \cellaligns{lll}]		
%                & falls $\; \mathcal{B}\,$ aus $\,\mathcal{A}\,$ folgt,
%                & $\;$ d.\,h. $ \, \mathcal{A} \Rightarrow \mathcal{B},\quad$ \\
%                & und $\; \mathcal{A}\,$ aus $\,\mathcal{B}\,$ folgt,
%                & $\;$ d.\,h. $ \, \mathcal{B} \Rightarrow \mathcal{A}.$\\       
%             \end{table} 
%        In diesem Fall sagt man auch 
              \begin{center}  
              \emph{\,"`$\mathcal{A}$ ist genau dann wahr, wenn $\mathcal{B}$ wahr ist."'\,}
              \end{center}
              
        Man schreibt 
                \[\mathcal{A} \Leftrightarrow \mathcal{B}.\]  % \label{eq-II.1.04}
        
        Die Äquivalenz $\mathcal{A} \Leftrightarrow \mathcal{B}$ ist per Definition selbst wieder 
        eine Aussageform, die genau dann zu einer wahren Aussage wird, wenn beide Aussageformen
        $\mathcal{A}$ und $\mathcal{B}$ \emph{gleichwertig} sind, d.\,h. entweder beide sind 
        \emph{wahr} oder beide sind \emph{falsch}.
        }
    \lang{en}{
        Two propositions $\mathcal{A}$ and $\mathcal{B}$ are called \notion{equivalent} if
              \begin{center}  
              \emph{\,'$\mathcal{A}$ is true if and only if $\mathcal{B}$ is true.'\,}
              \end{center}
        We write
              \[\mathcal{A} \Leftrightarrow \mathcal{B}.\]
        The equivalence $\mathcal{A} \Leftrightarrow \mathcal{B}$ is itself by defintion a propositional 
        formula that becomes a true proposition precisely if $\mathcal{A}$ and $\mathcal{B}$ have the 
        same truth value. That is, $\mathcal{A} \Leftrightarrow \mathcal{B}$ is true if $\mathcal{A}$ 
        and $\mathcal{B}$ are both \emph{true} or both \emph{false}, else it is \emph{false}.
    }
\end{definition}

\lang{de}{
Die Äquivalenz $\mathcal{A} \Leftrightarrow \mathcal{B}$ genügt also der folgenden Wahrheitstafel:
}
\lang{en}{
The equivalence $\mathcal{A} \Leftrightarrow \mathcal{B}$ gives the following truth table:
}
\lang{de}{
  \begin{table}[\align{c}\cellaligns{cccc}] 
    \head                %[\background{rgb(220,205,100)}]
    $\mathcal{A}$ & $\mathcal{B}$ && $\mathcal{A} \Leftrightarrow \mathcal{B}$ 
    \body
    W & W && W \\
    W & F && F \\
    F & W && F \\
    F & F && W 
  \end{table}
}
\lang{en}{
  \begin{table}[\align{c}\cellaligns{cccc}] 
    \head                %[\background{rgb(220,205,100)}]
    $\mathcal{A}$ & $\mathcal{B}$ && $\mathcal{A} \Leftrightarrow \mathcal{B}$ 
    \body
    T & T && T \\
    T & F && F \\
    F & T && F \\
    F & F && T 
  \end{table}
}
%
%%% Video K.M.
%
\lang{de}{
In folgendem Video wird die Äquivalenz von Aussageformen noch einmal
erläutert und anhand einiger wichtiger Anwendungsbeispiele weiter vertieft:
\\
\floatright{\href{https://api.stream24.net/vod/getVideo.php?id=10962-2-10745&mode=iframe&speed=true}
{\image[75]{00_video_button_schwarz-blau}}}\\\\
}
\lang{en}{
\\
}     

%         

  \begin{remark}\label{rem:aquiv-ist-doppelte-implikation}
    \lang{de}{
     Vergleicht man die Wahrheitstafeln der Implikationen $\mathcal{A} \Rightarrow \mathcal{B}$ und 
     $\mathcal{B} \Rightarrow \mathcal{A}$ mit der Wahrheitstafel von $\mathcal{A} \Leftrightarrow \mathcal{B}$, 
     so stellt man fest, dass die Äquivalenz $\mathcal{A} \Leftrightarrow \mathcal{B}$ genau dann wahr ist, wenn
     beide Implikationen wahr sind.
     }
     \lang{en}{
     Comparing the truth tables of the implications $\mathcal{A} \Rightarrow \mathcal{B}$ and 
     $\mathcal{B} \Rightarrow \mathcal{A}$ with that that of $\mathcal{A} \Leftrightarrow \mathcal{B}$, 
     it is clear that the equivalence $\mathcal{A} \Leftrightarrow \mathcal{B}$ is true if and only if 
     both implications are true.
     }
    
    \begin{table}[\align{c}\cellaligns{cccccc}] % {c|c||c|c|c }
      \head[\background{rgb(150,200,150)}]   %[\background{rgb(220,205,100)}]
      $\mathcal{A}$ & $\mathcal{B}$ && $\mathcal{A} \Rightarrow \mathcal{B}$ &  $\mathcal{B} \Rightarrow \mathcal{A}$ & $\mathcal{A} \Leftrightarrow \mathcal{B}$ 
      \body
      \lang{de}{W}\lang{en}{T}\lang{zh}{T}\lang{fr}{$\text{V}$} & \lang{de}{W}\lang{en}{T}\lang{zh}{T}\lang{fr}{$\text{V}$} && \lang{de}{W}\lang{en}{T}\lang{zh}{T}\lang{fr}{$\text{V}$} & \lang{de}{W}\lang{en}{T}\lang{zh}{T}\lang{fr}{$\text{V}$} & \lang{de}{W}\lang{en}{T}\lang{zh}{T}\lang{fr}{$\text{V}$} \\
      \lang{de}{W}\lang{en}{T}\lang{zh}{T}\lang{fr}{$\text{V}$} & \lang{de}{F}\lang{en}{F}\lang{zh}{F}\lang{fr}{$\text{F}$} && \lang{de}{F}\lang{en}{F}\lang{zh}{F}\lang{fr}{$\text{F}$} & \lang{de}{W}\lang{en}{T}\lang{zh}{T}\lang{fr}{$\text{V}$} & \lang{de}{F}\lang{en}{F}\lang{zh}{F}\lang{fr}{$\text{F}$} \\
      \lang{de}{F}\lang{en}{F}\lang{zh}{F}\lang{fr}{$\text{F}$} & \lang{de}{W}\lang{en}{T}\lang{zh}{T}\lang{fr}{$\text{V}$} && \lang{de}{W}\lang{en}{T}\lang{zh}{T}\lang{fr}{$\text{V}$} & \lang{de}{F}\lang{en}{F}\lang{zh}{F}\lang{fr}{$\text{F}$} & \lang{de}{F}\lang{en}{F}\lang{zh}{F}\lang{fr}{$\text{F}$} \\
      \lang{de}{F}\lang{en}{F}\lang{zh}{F}\lang{fr}{$\text{F}$} & \lang{de}{F}\lang{en}{F}\lang{zh}{F}\lang{fr}{$\text{F}$} && \lang{de}{W}\lang{en}{T}\lang{zh}{T}\lang{fr}{$\text{V}$} & \lang{de}{W}\lang{en}{T}\lang{zh}{T}\lang{fr}{$\text{V}$} & \lang{de}{W}\lang{en}{T}\lang{zh}{T}\lang{fr}{$\text{V}$}
    \end{table}

    \lang{de}{
    Dies wird benutzt, wenn man zeigen m"ochte, dass zwei Aussageformen $\mathcal{A}$ und $\mathcal{B}$ 
    äquivalent sind. Man beweist dann, dass sowohl die Implikation $\mathcal{A} \Rightarrow \mathcal{B}$ als auch die Implikation 
    $\mathcal{B} \Rightarrow \mathcal{A}$ wahr ist.
    }
    \lang{en}{
    Hence if we want two formulas $\mathcal{A}$ and $\mathcal{B}$ to be equivalent, it is enough to 
    prove that both $\mathcal{A} \Rightarrow \mathcal{B}$ and $\mathcal{B} \Rightarrow \mathcal{A}$ are 
    always true.
    }
     \end{remark}
  
  \begin{example} \label{ex:aqiv}
   \begin{enumerate}[alph]
%
%%% Video K.M.
%   
    \item \lang{de}{
          Schauen wir uns hierzu zunächst ein Beispiel mit detaillierten Erklärungen im Video an:
          \\
%QS        \floatright
           \center{\href{https://api.stream24.net/vod/getVideo.php?id=10962-2-10747&mode=iframe&speed=true}
      {\image[75]{00_video_button_schwarz-blau}}}\\
      }
      \lang{en}{
      The above remark essentially says that $\mathcal{A} \Leftrightarrow \mathcal{B}$ is the same 
      formula as 
      \[(\mathcal{A} \Rightarrow \mathcal{B}) \wedge (\mathcal{B} \Rightarrow \mathcal{A}).\]
      }
      \\         
%   
    \item \lang{de}{
    In diesem zweiten Beispiel seien nun $\;\mathcal{A}:\,$ "`$-1 \leq x \leq 1$"' und $\;\mathcal{B}:\,$ "`$x^2 \leq 1$"'.\\
    
    Wir untersuchen die Formeln $\, \mathcal{A}$ und $\mathcal{B}$ auf \emph{Äquivalenz}.\\
    
    Dazu stellen wir zunächst fest, dass die Implikation $\mathcal{A} \Rightarrow \mathcal{B}\,$ 
    unabhängig vom tatsächlichen Wert von $x\in\R$ immer \emph{wahr} ist, also unabhängig davon, ob $\mathcal{A}$ wahr oder falsch ist:
    }
    \lang{en}{
    Let $\;\mathcal{A}:\,$ '$-1 \leq x \leq 1$' and $\;\mathcal{B}:\,$ '$x^2 \leq 1$'.
    \\
    We check the formulas $\, \mathcal{A}$ and $\mathcal{B}$ to see if they are \emph{equivalent}.
    \\
    To do this, we show that the implication $\mathcal{A} \Rightarrow \mathcal{B}\,$ is true for all 
    $x\in\R$:
    }
 %
 %  Hinweis: das Label zu "\item" in den eckigen Klammern wird in der Vorschau nicht dargestellt, 
 %           aber in der Testumgebung erscheint es korrekt. Die "Tabellen"-Variante finktioniert nicht
 %           wegen ungünstiger Zeilenumrüche.
 %
    \begin{itemize}
      \item \lang{de}{
              Ist $\mathcal{A}\,$ \emph{wahr}, 
%              also $-1 \leq x \leq 1,\,$ dann gilt entweder $0\leq x\leq 1$ und somit $0=0\cdot x\leq x\cdot x\leq x\leq 1$, 
%              oder $0<-x\leq 1$    , also $0=0\cdot(-x)\leq (-x)\cdot (-x)\leqq -x\leq 1$.
              also $-1 \leq x \leq 1,\,$ dann gilt \\
              entweder $\; 0\leq x\leq 1 \,$ und somit $\, 0=0\cdot x\leq x\cdot x\leq 1\cdot 1 =1$, \\
              oder $\quad 0<-x\leq 1,\,$  also $\, 0=0\cdot(-x) < (-x)\cdot (-x)\leq 1\cdot 1 =1$. \\            
              Es folgt insbesondere       $\;x^2 \leq 1$. 
              Also ist $\mathcal{B}\,$ für jedes $x$ \emph{wahr}, für das $\mathcal{A}$ wahr ist.\\
              }
              \lang{en}{
              If $\mathcal{A}\,$ is \emph{true}, that is to say $-1 \leq x \leq 1\,$, then either \\
              $\quad 0\leq x\leq 1 \,$ and so $\, 0=0\cdot x\leq x\cdot x\leq 1\cdot 1 =1$, or \\
              $\quad 0<-x\leq 1,\,$ and so $\, 0=0\cdot(-x) < (-x)\cdot (-x)\leq 1\cdot 1 =1$.
              \\
              In particular, $\;x^2 \leq 1$.
              Hence $\mathcal{B}\,$ is \emph{true} for each $x$ that $\mathcal{A}$ is true for.
              \\
              }
      \item \lang{de}{
           Ist $\mathcal{A}\,$ hingegen \emph{falsch,}
           dann ist entweder $1<x$ oder $1<-x$. \\
           Somit ist entweder $1<1\cdot x<x\cdot x=x^2$, oder 
           $1<1\cdot (-x)<(-x)\cdot (-x)=x^2$. 
          Also ist hier auch $\mathcal{B}$ falsch.\\
      Wir hätten hier übrigens auch von der Definition Gebrauch machen können,
      dass die Implikation $\mathcal{A}\Rightarrow\mathcal{B}$ stets wahr ist, wenn
       $\mathcal{A}$ falsch ist, unabhängig vom Wahrheitswert von $\mathcal{B}$. 
      Unsere Argumentation unterstreicht noch einmal, dass diese Definition sinnvoll ist.
      }
      \lang{en}{
      If, on the other hand, $\mathcal{A}\,$ is \emph{false}, then either $1<x$ or $1<-x$. Hence either \\
      $\quad 1<1\cdot x<x\cdot x=x^2$ or \\
      $\quad 1<1\cdot (-x)<(-x)\cdot (-x)=x^2$. \\
      Hence $\mathcal{B}$ is false. Note that instead, we could have made use of the definition of 
      implication to tell that $\mathcal{A}\Rightarrow\mathcal{B}$ is always \emph{true} when 
      $\mathcal{A}$ is \emph{false}.
      }
            % dann spielt der Wahrheitswert von $\mathcal{B}$ keine Rolle, da eine Implikation per Definition 
            % immer \emph{wahr} ist, wenn ihre Prämisse \emph{falsch} ist.  
    \end{itemize}   
    
%    \begin{table}[\class{layout} \cellaligns{ll}] 
%        ist $\mathcal{A}\,$ \emph{wahr}, & also $-1 \leq x \leq 1,\,$ dann gilt $\;x^2 \leq 1$. \\
%                               & Somit ist also auch $\mathcal{B}\,$ \emph{wahr} und kann für diesen 
%                               Fall nie \emph{falsch} sein. \\
%       ist $\mathcal{A}\,$ \emph{falsch,} $\;$ & dann spielt der Wahrheitswert von $\mathcal{B}$ keine Rolle, 
%                                da aus \emph{falsch} per Definition immer \emph{wahr} folgt.
%    \end{table}
    \lang{de}{
    Ebenso ist die Implikation $\,\mathcal{B} \Rightarrow \mathcal{A}\,$ unabhängig vom tatsächlichen Wert von $x$ 
    immer \emph{wahr}, denn:
    }
    \lang{en}{
    Furthermore, the implication $\,\mathcal{B} \Rightarrow \mathcal{A}\,$ is always true, regardless 
    of the value of $x$, as:
    }
    
    \begin{itemize} 
      \item \lang{de}{
            Ist $\mathcal{B}\,$ \emph{wahr},
            also $\, x^2 \leq 1,\,$ dann ist auch $\;-1 \leq x \leq 1$. \\
            Also ist auch $\mathcal{A}\,$ \emph{wahr} für alle $x$, für die $\mathcal{B}$ wahr ist. \\
            }
            \lang{en}{
            If $\mathcal{B}\,$ is \emph{true}, that is to say $\, x^2 \leq 1\,$, then 
            $\;-1 \leq x \leq 1$.\\
            Hence $\mathcal{A}\,$ is \emph{true} for each $x$ that $\mathcal{B}$ is true for.
            }
      \item \lang{de}{
            Ist $\mathcal{B}\,$ \emph{falsch,}
            dann ist wiederum  die Implikation $\,\mathcal{B} \Rightarrow \mathcal{A}\,$ per Definition 
            immer \emph{wahr}, unabhängig vom Wahrheitswert von $\mathcal{A}$.
            }
            \lang{en}{
            If $\mathcal{B}\,$ is \emph{false}, then by definition the implication 
            $\,\mathcal{B} \Rightarrow \mathcal{A}\,$ is always true, regardless of the truth-value 
            of $\mathcal{A}$.
            }
     \end{itemize} 
   
%     \begin{table}[\class{layout} \align{c} \cellaligns{ll}] 
%     ist $\mathcal{B}\,$ \emph{wahr}, & also $x^2 \leq 1,\,$ dann ist $\;-1 \leq x \leq 1$. \\
%                               & Somit ist also auch $\mathcal{A}\,$ \emph{wahr} und kann für diesen 
%                               Fall nie \emph{falsch} sein. \\
%       ist $\mathcal{B}\,$ \emph{falsch,} $\;$ & dann spielt der Wahrheitswert von $\mathcal{A}$ keine Rolle, 
%                                da aus \emph{falsch} per Definition immer \emph{wahr} folgt.
%      \end{table} 

\lang{de}{
  Nun übertragen wir diese Ergebnisse in eine Wahrheitstafel und überprüfen Bemerkung \ref{rem:aquiv-ist-doppelte-implikation}, laut der
  die Wahrheitswerte für die Äquivalenz $\mathcal{A} \Leftrightarrow \mathcal{B}$ aus denen der Implikationen $\,\mathcal{A} \Rightarrow \mathcal{B}\,$ 
  und $\,\mathcal{B} \Rightarrow \mathcal{A}$ folgen. 
}
\lang{en}{
Now we transfer these results into a truth table and confirm remark 
\ref{rem:aquiv-ist-doppelte-implikation} using the truth-values for the equivalence 
$\mathcal{A} \Leftrightarrow \mathcal{B}$ and the implications $\,\mathcal{A} \Rightarrow \mathcal{B}\,$ 
and $\,\mathcal{B} \Rightarrow \mathcal{A}$.
}

\lang{de}{
   \begin{table}[\align{c}\cellaligns{cccccc}] 
    \head             
    $\mathcal{A}:\,-1 \leq x \leq 1$ & $\mathcal{B}:\,x^2 \leq 1$ && $\mathcal{A} \Rightarrow \mathcal{B}$ 
            & $\mathcal{B} \Rightarrow \mathcal{A}$ & $\mathcal{A} \Leftrightarrow \mathcal{B}$
    \body
    W & W && W & W & W \\
 %   W & F && \textit{nicht relevant} & W & \textit{nicht relevant} \\
 %   F & W && W & \textit{nicht relevant} & \textit{nicht relevant}\\
    W & F && \colspan{3} \textit{tritt hier nicht auf}\\
    F & W && \colspan{3} \textit{tritt hier nicht auf}\\
    F & F && W & W & W
  \end{table}
}
\lang{en}{
   \begin{table}[\align{c}\cellaligns{cccccc}] 
    \head             
    $\mathcal{A}:\,-1 \leq x \leq 1$ & $\mathcal{B}:\,x^2 \leq 1$ && $\mathcal{A} \Rightarrow \mathcal{B}$ 
            & $\mathcal{B} \Rightarrow \mathcal{A}$ & $\mathcal{A} \Leftrightarrow \mathcal{B}$
    \body
    T & T && T & T & T \\
 %   W & F && \textit{nicht relevant} & W & \textit{nicht relevant} \\
 %   F & W && W & \textit{nicht relevant} & \textit{nicht relevant}\\
    T & F && \colspan{3} \textit{not possible for $\,x\in\R$}\\
    F & T && \colspan{3} \textit{not possible for $\,x\in\R$}\\
    F & F && T & T & T
  \end{table}
}

\lang{de}{
  Aus der Tabelle lesen wir ab, dass in diesem Beispiel die Formel $\:\mathcal{A} \Leftrightarrow \mathcal{B}$ unabhängig von
  den Wahrheitswerten von $\mathcal{A}$ und $\mathcal{B}$ immer wahr ist.
  Das heißt, die Äquivalenz $\,-1 \leq x \leq 1\Leftrightarrow x^2 \leq 1$ ist immer \emph{wahr}, unabhängig vom
  tatsächlichen Wert von $x\in\R$.
}
\lang{en}{
From the truth table we see that in this example, the propositional formula 
$\:\mathcal{A} \Leftrightarrow \mathcal{B}$ 
is always \emph{true}, and not dependent on the values of $\mathcal{A}$ and $\mathcal{B}$. 
This means that the equivalence $\,-1 \leq x \leq 1\Leftrightarrow x^2 \leq 1$ is \emph{true} for all 
values of $x\in\R$.
}
  \end{enumerate}
 \end{example}  

%  \begin{block}[warning]
%    \lang{de}{Man sollte die Äquivalenz von Aussagen nicht mit der Äquivalenz von Formeln verwechseln.
%    Bei der Äquivalenz von Aussagen haben die Aussagen einen festgelegten Wahrheitswert,  
%    % welcher aber dem Leser vielleicht nicht bekannt ist. 
%    auch wenn dieser möglicherweise nicht bekannt ist.
%    Die Äquivalenz drückt dann aus, dass die Wahrheitswerte der beiden Aussagen gleich sind.
%    Findet man also den Wahrheitswert der einen Aussage heraus, kennt man damit auch den der anderen Aussage.}
%    \lang{en}{One should not mix up the equivalence of statements with the equivalence of formulas. In the
%    equivalence of statements, the statements have a fixed truth value which might be unknown to the reader. The equivalence expresses that the truth values of both statements are the same. If one
%    determines the truth value of one of the statements, one also knows the truth value of the other 
%    statement.}
%  \end{block}

\lang{de}{
Ebenso wie bei den \emph{Implikationen} (siehe Bemerkung \ref{rem:gueltige_folgerung}) sind auch
bei den \emph{Äquivalenzen} in der Mathematik meist diejenigen Aussageformen (Formeln) von Interesse, 
die unabhängig von konkreten Werten der enthaltenen Variablen immer gültig sind, d.\,h. 
deren Wahrheitswert immer wahr ist. Dies ist zum Beispiel von grundlegender Bedeutung 
für \textit{Äquivalenzumformungen}, wie wir im folgenden Abschnitt sehen werden.
}
\lang{en}{
Much like with \emph{implications} (see remark \ref{rem:gueltige_folgerung}), the interesting 
\emph{equivalences} in mathematics are usually those that are \emph{true}, or 'hold', for every possible 
value that can be assigned to the variables in the formula. For example, the \emph{equivalence 
transformations} seen in the next sections rely on these always-true equivalences.
}


\section{\lang{de}{Äquivalenzumformungen}
         \lang{en}{Equivalence transformations}}\label{sec:aequivalenz}

%  \lang{de}{\textit{Äquivalenzumformungen} wandeln Aussagen, welche von Variablen abhängen (können),
%  in Aussagen um, die unabhängig vom Wert der Variablen zur ursprünglichen Aussage äquivalent sind.\\ 
%  So ist die Aussage $\mathcal{A}$: "`$3y - 3x = 9x$"' stets
%  gleichwertig  mit $\mathcal{B}$: "`$3y = 12x$"' oder auch mit $\mathcal{C}$: "`$y = 4x$"', 
%  denn die Werte für $x$ und $y$, für die die Aussage $\mathcal{A}$, $\mathcal{B}$ 
%  oder $\mathcal{C}$ jeweils wahr ist (d.h. die jeweilige 
%  Gleichung erfüllt ist), ist für alle drei Gleichungen gleich. \\ \\

\lang{de}{Äquivalenzumformungen wandeln Formeln, die von Variablen abhängen (können), in neue
Formeln um, ohne dass sich ihr Wahrheitswert verändert. Die neue Formel ist also, unabhängig vom 
Wert eventueller Variablen, äquivalent zur ursprünglichen Formel.
}
\lang{en}{
Equivalence transformations are used to change an equation that (may) involve variables without changing 
the equation's truth-values. The changed equation is therefore equivalent to the original one for all 
values of the variables.
}
\\ 

\begin{example}
  \lang{de}{
  Betrachten wir beispielsweise die Formeln 
  }
  \lang{en}{
  Consider the equations
  }
  \begin{enumerate}
      \item[ ] $\mathcal{A}: \;$   $3y - 3x = 9x,$
      \item[ ] $\mathcal{B}: \;$   $3y = 12x \;$ \lang{de}{und}\lang{en}{and}
      \item[ ] $\mathcal{C}: \;$   $y = 4x.$
  \end{enumerate}

  \lang{de}{
  Wir stellen fest, dass die Variablenbelegung für $x$ und $y$, die die einzelnen Gleichungen erfüllen und
  somit die jeweiligen Formeln $\mathcal{A}$, $\mathcal{B}$ oder $\mathcal{C}$ zu wahren Aussagen machen,
  für alle drei Gleichungen identisch sind. Das bedeutet, dass die drei Formeln (unabhängig vom Wert der 
  Variablen $x$ und $y$) gleichwertig sind. Also gilt nach Definition \ref{def:aequvalenz} $\,$
  }
  \lang{en}{
  We notice that the exact same values for $x$ and $y$ satisfy all three equations, that is to say, 
  the equations are true propositions for the exact same values of $x$ and $y$. Hence by definition 
  \ref{def:aequvalenz},
  }

  \[\mathcal{A} \Leftrightarrow \mathcal{B} \Leftrightarrow \mathcal{C} \, ( \Leftrightarrow \mathcal{A} ).\]

  \lang{de}{
  Wir stellen weiterhin fest, dass sich jede der drei Gleichungen durch Anwendung mathematischer Operationen in eine
  der anderen umformen lässt, zum Beispiel
  }
  \lang{en}{
  We point out that each of the three equations can be transformed into any of the other three 
  equations, for example
  }

  \begin{itemize}
      \item  \lang{de}{entsteht $\mathcal{B}$ aus $\mathcal{A}$ durch Addition von $3x$ auf beiden Seiten der Gleichung und }
             \lang{en}{$\mathcal{A}$ can be turned into $\mathcal{B}$ by adding $3x$ to both sides 
             of the equality, and}
      \item  \lang{de}{$\mathcal{C}$ entsteht aus $\mathcal{B}$ durch Division beider Seiten der Gleichung durch $3.$}
             \lang{en}{$\mathcal{B}$ can be turned into $\mathcal{C}$ by dividing both sides of the 
             equality by $3$.}
  \end{itemize}

  \lang{de}{
  Da sich der Wahrheitswert der Formeln durch diese Umformungen nicht verändert, nennt man sie auch
  \emph{Äquivalenzumformungen}. Die Gleichwertigkeit der Formeln wird dabei durch Äquivalenzzeichen 
  gekennzeichnet. Zur besseren Nachvollziehbarkeit der Umformungen wird die jeweils angewandte 
  mathematische Operation meist rechts der ursprünglichen Formel, getrennt durch einen senkrechten 
  Strich, vermerkt:
  }
  \lang{en}{
  As the truth-values of the equations for each $x$, $y$ do not change through these transformations, 
  we call them \emph{equivalence transformations}. An equivalence symbol can be used to represent this. 
  For clarity, often each transformation is explicitly written to the right of the equation that is 
  being transformed, seperated by a vertical line:
  }
  
  \begin{align*}
  &&\;					3y-3x	&= 9x		&\quad		&\vert +3x&\\
  &\Leftrightarrow&\; 	3y  	&=12x		&\quad		&\vert \div 3&\\
  &\Leftrightarrow&\;\; 	y		&=4x		&\quad		&&
  \end{align*}
  
\end{example}

\lang{de}{
Wie das vorstehende Beispiel zeigt, lassen sich mathematische \emph{Gleichungen} oder \emph{Ungleichungen} mit Variablen, betrachtet
als logische Formeln, durch geeignete mathematische Operationen so umformen, dass die neue (umgeformte) Formel bei gleicher 
Variablenbelegung denselben Wahrheitswert hat wie die ursprüngliche Formel. Das bedeutet, dass die Lösungsmenge einer
mathematischen \emph{(Un-)Gleichung} nach solchen \textit{äquivalenten Umformungen} unverändert bleibt. Aufgrund dieser 
Eigenschaft sind \textit{Äquivalenzumformungen} in der Mathematik die wichtigste Methode zum Vereinfachen und Lösen mathematischer 
\emph{(Un-)Gleichungen} mit einer oder mehreren Variablen.
}
\lang{en}{
As is shown by the above example, mathematical \emph{equalities} or \emph{inequalities} containing 
variables can be transformed by certain mathematical operations to give new equations with the same 
truth-values for each value of the variables. This means that the set of solutions of the 
\emph{(in-)equality} is unchanged by these \textit{equivalence transformations}. Due to this, 
\textit{equivalence transformations} are the most important method for simplifying and solving 
mathematical \emph{(in-)equalities} containing one or more variables.
}




\lang{de}{
Dabei ist allerdings zu beachten, dass nicht jede mathematische Operation zu einer äquivalenten Umformung einer Gleichung 
oder Ungleichung führt. 
Gültige Umformungen sind:
}
\lang{en}{
It is important to note that not every mathematical operation leads to an equivalence transformation 
when applied to an equation. Valid equivalence transformations are:
}


\begin{tabs*}\label{aequivalenzumformungen}
\tab{\lang{de}{Äquivalenzumformungen für Gleichungen}\lang{en}{Equivalence transformations for equations}}
\begin{itemize}
\item \lang{de}{Vertauschen der Gleichungsseiten:\\ $a = b \Leftrightarrow b = a$;}
      \lang{en}{Swapping sides of an equation $a = b \Leftrightarrow b=a$;}
\item \lang{de}{das Gleiche zu beiden Seiten addieren oder von beiden Seiten subtrahieren:\\
$a = b \Leftrightarrow a \pm c = b \pm c$;}
      \lang{en}{adding or subtracting the same value to (from) both sides,\\
$a = b \Leftrightarrow a \pm c = b \pm c$;}
\item \lang{de}{beide Seiten mit einem Faktor $m \neq 0$ (darf nicht Null sein!) 
multiplizieren:\\ 
$a = b \Leftrightarrow m \cdot a = m \cdot b$;}
      \lang{en}{multiplying both sides of an equation by a non-zero value $m \neq 0$, $a = b \Leftrightarrow m \cdot a = m \cdot b$;}
\item \lang{de}{beide Seiten durch einen Divisor $m \neq 0$ (darf nicht Null sein!)
dividieren:\\
$a=b \Leftrightarrow \displaystyle\frac{a}{m} = \displaystyle\frac{b}{m}$.}
      \lang{en}{dividing both sides by a non-zero value $m \neq 0$,\\
$a=b \Leftrightarrow \displaystyle\frac{a}{m} = \displaystyle\frac{b}{m}$.}
\end{itemize}
\tab{\lang{de}{Äquivalenzumformungen für Ungleichungen}\lang{en}{Equivalence transformations for inequalities}}
\lang{de}{
Für Ungleichungen hat man die gleichen Äquivalenzumformungen wie für Gleichungen, man muss jedoch darauf
achten, dass sich in einigen Fällen das Vergleichszeichen \glqq{}umdreht\grqq{}:
}
\lang{en}{
For inequalities the same equivalence transformations hold as for equalities, but the direction of the 
inequality sign can change when swapping sides and when multiplying or dividing by a negative value:
}\\
\begin{itemize}
\item \lang{de}{Vertauschen der Seiten:\\ $a > b \Leftrightarrow b \textcolor{\#D2691E}{<} a$ bzw. $a < b \Leftrightarrow b \textcolor{\#D2691E}{>} a$;}
      \lang{en}{Swapping sides  $a > b \Leftrightarrow b<a$, and $a < b \Leftrightarrow b>a$;}
\item \lang{de}{das Gleiche zu beiden Seiten addieren oder von beiden das Gleiche subtrahieren:\\
$a > b \Leftrightarrow a \pm c > b \pm c$ bzw. $a < b \Leftrightarrow a \pm c < b \pm c$;}
\lang{en}{adding or subtracting the same value to (from) both sides,\\
$a > b \Leftrightarrow a \pm c > b \pm c$, and $a < b \Leftrightarrow a \pm c < b \pm c$;}
\item \lang{de}{beide Seiten mit einem positiven Faktor $m > 0$  multiplizieren:\\ 
$a > b \Leftrightarrow m \cdot a > m \cdot b$ bzw. $a < b \Leftrightarrow m \cdot a < m \cdot b$;}
\lang{en}{multiplying both sides of an equation by a positive value $m > 0$, 
$a > b \Leftrightarrow m \cdot a > m \cdot b$, and $a < b \Leftrightarrow m \cdot a < m \cdot b$;}
\item \lang{de}{beide Seiten mit einem negativen Faktor $m < 0$  multiplizieren:\\
bzw. $a < b \Leftrightarrow m \cdot a \textcolor{\#D2691E}{>} m \cdot b$;}
\lang{en}{multiplying both sides of an equation by a negative value $m < 0$, 
$a > b \Leftrightarrow m \cdot a \textcolor{\#D2691E}{<} m \cdot b$, 
and $a < b \Leftrightarrow m \cdot a \textcolor{\#D2691E}{>} m \cdot b$;}
\item \lang{de}{beide Seiten durch einen positiven Divisor $m > 0$ dividieren:\\
$a>b \Leftrightarrow \displaystyle\frac{a}{m} > \displaystyle\frac{b}{m}$ bzw. $a<b \Leftrightarrow \displaystyle\frac{a}{m} < \displaystyle\frac{b}{m}$;}
\lang{en}{dividing both sides by a positive value $m > 0$,\\
$a>b \Leftrightarrow \displaystyle\frac{a}{m} > \displaystyle\frac{b}{m}$, and $a<b \Leftrightarrow \displaystyle\frac{a}{m} < \displaystyle\frac{b}{m}$;}
\item \lang{de}{beide Seiten durch einen negativen Divisor $m < 0$ dividieren:\\
$a>b \Leftrightarrow \displaystyle\frac{a}{m} \textcolor{\#D2691E}{<} \displaystyle\frac{b}{m}$ bzw. 
$a<b \Leftrightarrow \displaystyle\frac{a}{m} \textcolor{\#D2691E}{>} \displaystyle\frac{b}{m}$;}
\lang{en}{dividing both sides by a negative value $m < 0$,\\
$a>b \Leftrightarrow \displaystyle\frac{a}{m} \textcolor{\#D2691E}{<} \displaystyle\frac{b}{m}$, and 
$a<b \Leftrightarrow \displaystyle\frac{a}{m} \textcolor{\#D2691E}{>} \displaystyle\frac{b}{m}$;}
\end{itemize}
\end{tabs*}


\begin{block}[warning]\label{proposition.warning.1}
\lang{de}{Potenzieren (insbesondere Quadrieren) und Wurzelziehen sind im Allgemeinen keine Äquivalenzumformungen!}
\lang{en}{In general, taking a power, taking a root, and squaring an equation are not equivalence transformations!}
\end{block}

\lang{de}{Wir schauen uns weitere Beispiele für Äquivalenzumformungen an und verzichten ab sofort auf die Bezeichnungen 
$\mathcal{A}$, $\mathcal{B}$, etc. für die  \emph{(Un-)Gleichungen} als logische Formeln.}


\lang{en}{From now on we will omit the letters $\mathcal{A}$, $\mathcal{B}$, and
$\mathcal{C}$ as identifiers for the statements.}

\begin{example}\label{proposition.example.2}

\lang{de}{1. Seiten vertauschen:}
\lang{en}{1. Swapping sides}
\begin{align*}
&&\quad					3x+12	&\;=	4		&\\
&\Leftrightarrow&\quad  	4		&\;=	3x+12	&
\end{align*}
\lang{de}{Bei Ungleichungen muss das Vergleichszeichen \glqq{}umgedreht\grqq{} werden:}
\lang{en}{In inequalities, swapping sides also involves changing the direction of the inequality sign:}
\begin{align*}
&&\quad					3x+12	&\;\leq	4		&\\
&\Leftrightarrow&\quad		4	&\;\geq	3x+12	&
\end{align*}

\lang{de}{2. Gleiche Zahl addieren oder subtrahieren:}
\lang{en}{2. Adding (or subtracting) the same number to (from) both sides}
\begin{align*}
&&\quad					    	3x+12		&\;=4		&  		&\quad\vert  -3& \\
&\Leftrightarrow&\quad		3x+12-3		&\;=4-3	&		&&\\
&\Leftrightarrow&\quad		3x+9		&\;=1	&		&&
\end{align*}

\lang{de}{3. Mit gleicher Zahl $\neq0$ multiplizieren oder dadurch dividieren:}
\begin{align*}
&&\quad			    		3x +12			&\;=4				&			&\quad\vert  \lang{de}{:}\lang{en}{/}(-3)&\\
&\Leftrightarrow&\quad	(3x)\lang{de}{:}\lang{en}{/}(-3) +12\lang{de}{:}\lang{en}{/}(-3)	&\;=4\lang{de}{:}\lang{en}{/}(-3)	&	&&\\
&\Leftrightarrow&\quad	-x-4			&\;=-\frac{4}{3}	&			&\quad\vert  +4&\\
&\Leftrightarrow&\quad	-x				&\;=-\frac{4}{3}+4  &			&&\\
&\Leftrightarrow&\quad	-x				&\;=\frac{8}{3}	    &			&\quad\vert  \cdot(-1)&\\
&\Leftrightarrow&\quad	 x				&\;=-\frac{8}{3}     &			&\quad\vert  \cdot 3 &\\
&\Leftrightarrow&\quad	 3x				&\;=-8               &			&&
\end{align*}
\lang{de}{
Bei Ungleichungen muss das Vergleichszeichen \glqq{}umgedreht\grqq{} werden, sofern mit einer negativen
Zahl multipliziert oder durch eine negative Zahl dividiert wird:
}
\lang{en}{
Inequality signs have to be reversed when both sides are being multiplied or divided by a negative 
number:
}
\begin{align*}
&&\quad			    		3x +12			&\;<4				&		&\quad\vert  \lang{de}{:}\lang{en}{/}(-3)&\\
&\Leftrightarrow&\quad	(3x)\lang{de}{:}\lang{en}{/}(-3) +12\lang{de}{:}\lang{en}{/}(-3)	&\;>4\lang{de}{:}\lang{en}{/}(-3)	&	&&\\
&\Leftrightarrow&\quad	-x-4				&\;>-\frac{4}{3}&			&\quad\vert  +4&\\
&\Leftrightarrow&\quad	-x				&\;>-\frac{4}{3}+4  &			&&\\
&\Leftrightarrow&\quad	-x				&\;>\frac{8}{3}	    &			&\quad\vert  \cdot (-1)&\\
&\Leftrightarrow&\quad	 x				&\;<-\frac{8}{3}    &			&\quad\vert  \cdot 3 &\\
&\Leftrightarrow&\quad	 3x				&\;<-8              &			&&
\end{align*}

\end{example}

\lang{de}{
Wie bereits erwähnt, werden Äquivalenzumformungen zur Lösung von Gleichungen und Ungleichungen verwendet. 
Als \emph{"`Lösung"'} einer (Un-) Gleichungen bezeichnet man dabei (reelle) Zahlen,
die eingesetzt für die Variablen in der (Un-) Gleichungen, diese zu einer wahren Aussage machen.
Ziel der äquivalenten Umformungen ist es nun, diese Variablen in der (Un-) Gleichung so weit wie möglich 
zu isolieren, um so die Lösung der (Un-) Gleichung leichter bestimmen zu können. Im Idealfall lässt sich die 
Lösung nach dem letzten Umformungsschritt direkt ablesen. Gibt es mehrere verschiedene Lösungen, so bezeichnet
man die Menge aller Lösungen als die \emph{"`Lösungsmenge"'} der (Un-)Gleichung und schreibt $\mathbb{L}.$
}
\lang{en}{
As has been mentioned, equivalence transformations are used to find solutions to equations and 
inequalities. We define the \emph{solution} of an \emph{(in-)equality} as the real numbers that, when 
substituted into the variables of the \emph{(in-)equality}, make it \emph{true}. 
The aim of equivalence transformations is that the solutions of an \emph{(in-)equality} are easier to 
find. Ideally, a solution can be directly read after all of the transformations have taken place. If 
there is more than one solution, we call the set of all solutions the \emph{solution set} of the 
equation and denote it by $\mathbb{L}$.
}

\begin{example}\label{proposition.example.3}
\begin{enumerate}[alph]
\item \lang{de}{
      Wir bestimmen die Lösung % die Lösungsmenge $\mathbb{L}$ 
      der folgenden Gleichung:
      }
      \lang{en}{
      We determine the solution of the following equation:
      }
\begin{align*}
&&\quad					5(3x-5)+2x	&=7(3-2x)+16		&		&&\\
&\Leftrightarrow&\quad 	15x-25+2x	&=21-14x+16			&		&&\\
&\Leftrightarrow&\quad 	17x-25		&=37-14x			&		&\vert +14x&\\
&\Leftrightarrow&\quad 	31x-25		&=37				& 		&\vert +25&\\
&\Leftrightarrow&\quad 	31x			&=62				&		&\vert \lang{de}{:}\lang{en}{/}31&\\
&\Leftrightarrow&\quad 	x			&=\frac{62}{31}		&		&&\\
&\Leftrightarrow&\quad 	x			&=2 				&		&&
\end{align*}

\lang{de}{
Die Ausgangsgleichung $\, 5(3x-5)+2x=7(3-2x)+16\,$ wurde äquivalent umgeformt zur 
Gleichung $\, x=2\,$. Man sagt auch, sie wurde \emph{nach $x$ "`aufgelöst'"}.
% Die Lösungsmenge 
Die Lösung $\, x=2 \,$ lässt sich nun direkt ablesen und liefert die sogenannte 
\emph{"`Lösungsmenge'"}, die wie folgt dargestellt wird
}
\lang{en}{
The original equation $\, 5(3x-5)+2x=7(3-2x)+16\,$ is transformed into the equivalent equation 
$\, x=2\,$. This is also called \emph{'rearranging for $x$'}.
The solution $\, x=2 \,$ can simply be read from the transformed equation, giving the solution set
}
 \[\mathbb{L}=\{\,x\in\R\,|\,x = 2 \,\}=\{2\}.	\]
 
 \item \lang{de}{Wir bestimmen die Lösungsmenge $\mathbb{L}$ der folgenden Ungleichung:}
       \lang{en}{We determine the solution set $\mathbb{L}$ of the following inequality:}
\begin{align*}
&&\quad					5x-4	    &\geq 2x+2  		&		&\quad\vert -2x+4&\\
&\Leftrightarrow&\quad 	3x      	&\geq 6 			&		&\quad\vert  \lang{de}{:}\lang{en}{/}(3)&\\
&\Leftrightarrow&\quad 	x			&\geq 2 			&		&&
\end{align*}

\lang{de}{
Die Ausgangsungleichung $\, 5x-4 \geq 2x+2 \,$ wurde äquivalent umgeformt zur Ungleichung $\, x \geq 2,$ sie
wurde also ebenfalls \emph{nach $x$ "`aufgelöst'"}. Hieraus lässt sich ablesen, dass alle reellen Zahlen
$x$, die größer oder gleich $2$ sind, die Ungleichung lösen. Die Lösung der Ungleichung wird daher 
auch dargestellt als die \emph{"`Lösungsmenge'"}
}
\lang{en}{
The original inequality $\, 5x-4 \geq 2x+2 \,$ is transformed into the equivalent inequality 
$\, x \geq 2$, that is, it is \emph{rearranged for $x$}. From this we simply read that all real numbers 
$x$ that are greater than or equal to $2$ solve the inequality. The \emph{solution set} of the 
inequality is therefore
}
 \[\mathbb{L}=\{\,x\in\R\,|\,x \geq 2 \,\}=[2;\infty).	\]
 
\end{enumerate}

\end{example}

% \lang{de}{Äquivalenzumformungen und Folgerungen lassen sich auch kombinieren. 
% Durch die Schreibweise wie in Gleichung (5) lässt sich gut verfolgen, 
% welche der Aussagen Folgerung anderer Aussagen sind, z.B.}
% \lang{en}{Equivalence transformations and implications can also be combined. 
% Using the same style as in Equation (5) we can easily follow which statements are a result
% of implications from other statements, e.g.}
% \begin{align}
% &&					3y-3x	&=9x		&			&\vert +3x		&			&(8)\label{eq-II.1.08}&\\
% &\Leftrightarrow&	3y 		&=12x		&			&\vert \lang{de}{:}\lang{en}{/}3		&			&(9)\label{eq-II.1.09}&\\
% &\Leftrightarrow&	y		&=4x		&			&\vert(\cdot)^2	&		 	&(10)\label{eq-II.1.10}&\\
% &\Rightarrow&		y^2 	&=16x^2		&			&\vert -5		& 			&(11)\label{eq-II.1.11}&\\
% &\Leftrightarrow&	y^2-5 	&=16x^2-5.	& 			&				&			&(12)\label{eq-II.1.12}&
% \end{align}
% 
% \lang{de}{Beachte, dass Gleichung (11) nur eine Folgerung ($\Rightarrow$) aus Gleichung (10), aber 
% nicht gleichwertig ($\Leftrightarrow$) ist, denn in (10) müssen $x$ und $y$ dasselbe 
% Vorzeichen haben, während dies in (11) keine Rolle spielt.}
% \lang{en}{Note that Equation (11) is simply an implication ($\Rightarrow$) of Equation (10), but not 
% an equivalent one ($\Leftrightarrow$), since in (10) $x$ and $y$ need to have the same sign, which doesn't play a role in (11) anymore.}


\begin{quickcheck}
		\field{rational}
		\type{input.number}
		\begin{variables}
			\randint{a}{2}{4}
			\randint{b}{1}{5}
			\randint{c}{-2}{2}
			\randint{d}{2}{5}
			\function[calculate]{m}{a+d}
			\function[calculate]{n}{c+b}
			\function[calculate]{loes}{(b+c)/(a+d)}	
		\end{variables}
		
%			\text{Die Gleichung $\var{a}x-\var{b}=\var{c}-\var{d}x$ ist äquivalent zur Gleichung\\
%			 $x= $\ansref.}
    \text{\lang{de}{Lösen Sie die folgende Gleichung nach $x$ auf.}
          \lang{en}{Rearrange the following equality for $x$.}\\
    $\var{a}x-\var{b}=\var{c}-\var{d}x \quad\Leftrightarrow\quad x= $ \ansref.}
		
		\begin{answer}
			\solution{loes}
		\end{answer}
		\explanation{\lang{de}{Mittels Äquivalenzumformungen erhält man}
                 \lang{en}{Using equivalence transformations we get}
		\begin{align*}
		  & \var{a}x-\var{b} &=\var{c}-\var{d}x &\quad & \vert +\var{d}x \\
		  \Leftrightarrow&\quad \var{a}x+\var{d}x-\var{b} &=\var{c} &\quad & \vert +\var{b} \\
		  \Leftrightarrow&\quad  \var{m}x &= \var{n} &\quad & \vert : \var{m} \\
		  \Leftrightarrow&\quad x&= \frac{\var{n}}{\var{m}} && %  =\var{loes} &&
		\end{align*}
		}
	\end{quickcheck}


\begin{quickcheck}
  \type{input.number}
  \field{rational}
  \begin{variables}
    \randint{num1}{2}{3}
    \randint{b1}{4}{5}
    \randint{c1}{6}{9}
    \function[calculate]{f1}{(c1-b1)/num1}
    \randint{b2}{2}{5}
    \function[calculate]{b2m}{-b2}
    \function[calculate]{b22}{b2*b2}

  \end{variables}
  \text{\lang{de}{Lösen Sie die folgenden Ungleichungen nach $x$ auf.} 
        \lang{en}{Rearrange the following inequality for $x$.}\\
  $\var{num1}\,x+\var{b1}<\var{c1}\quad\Leftrightarrow\quad x<$ \ansref , \\
%  $\frac{x}{\var{b2m}} +\var{b2}\geq 0 \quad\Leftrightarrow\quad x \leq$ \ansref.}}
  $\displaystyle\Big(-\frac{x}{\var{b2}}\Big) +\var{b2}\geq 0 \quad\Leftrightarrow\quad x \leq$ \ansref.}
   \begin{answer}
    \solution{f1}
  \end{answer}
  \begin{answer}
    \solution{b22}
  \end{answer}
\end{quickcheck}
%No explanation for this quickcheck? - Niccolo





\end{visualizationwrapper}
\end{content}
