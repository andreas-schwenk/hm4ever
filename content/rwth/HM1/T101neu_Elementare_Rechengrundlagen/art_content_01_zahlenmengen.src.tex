%$Id:  $
\documentclass{mumie.article}
%$Id$
\begin{metainfo}
  \name{
    \lang{de}{Mengen und Zahlenmengen}
    \lang{en}{Sets and sets of numbers}
   }
  \begin{description} 
 This work is licensed under the Creative Commons License Attribution 4.0 International (CC-BY 4.0)   
 https://creativecommons.org/licenses/by/4.0/legalcode 

    \lang{de}{Beschreibung}
    \lang{en}{Description}
  \end{description}
  \begin{components}
    \component{generic_image}{content/rwth/HM1/images/g_img_00_video_button_schwarz-blau.meta.xml}{00_video_button_schwarz-blau}
    \component{generic_image}{content/rwth/HM1/images/g_tkz_T101_IntervalExample_B.meta.xml}{T101_IntervalExample_B}
    \component{generic_image}{content/rwth/HM1/images/g_tkz_T101_IntervalExample_A.meta.xml}{T101_IntervalExample_A}
    \component{generic_image}{content/rwth/HM1/images/g_tkz_T101_IntervalExample_C.meta.xml}{T101_IntervalExample_C}
    \component{generic_image}{content/rwth/HM1/images/g_tkz_T101_OpenInterval.meta.xml}{T101_OpenInterval}
    \component{generic_image}{content/rwth/HM1/images/g_tkz_T101_ClosedInterval.meta.xml}{T101_ClosedInterval}
    \component{generic_image}{content/rwth/HM1/images/g_tkz_T101_RightOpenInterval.meta.xml}{T101_RightOpenInterval}
    \component{generic_image}{content/rwth/HM1/images/g_tkz_T101_LeftOpenInterval.meta.xml}{T101_LeftOpenInterval}
    \component{generic_image}{content/rwth/HM1/images/g_tkz_T101_NumberLine_D.meta.xml}{T101_NumberLine_D}
    \component{generic_image}{content/rwth/HM1/images/g_tkz_T101_NumberLine_C.meta.xml}{T101_NumberLine_C}
    \component{generic_image}{content/rwth/HM1/images/g_tkz_T101_NumberLine_B.meta.xml}{T101_NumberLine_B}
    \component{generic_image}{content/rwth/HM1/images/g_tkz_T101_NumberLine_A.meta.xml}{T101_NumberLine_A}
    \component{generic_image}{content/rwth/HM1/images/g_tkz_T101_NumberSets.meta.xml}{T101_NumberSets}
    \component{generic_image}{content/rwth/HM1/images/g_tkz_T101_VennQuickCheck.meta.xml}{T101_VennQuickCheck}
    \component{generic_image}{content/rwth/HM1/images/g_tkz_T101_VennComplement.meta.xml}{T101_VennComplement}
    \component{generic_image}{content/rwth/HM1/images/g_tkz_T101_VennSubset.meta.xml}{T101_VennSubset}
    \component{generic_image}{content/rwth/HM1/images/g_tkz_T101_VennDifference_B.meta.xml}{T101_VennDifference_B}
    \component{generic_image}{content/rwth/HM1/images/g_tkz_T101_VennDifference_A.meta.xml}{T101_VennDifference_A}
    \component{generic_image}{content/rwth/HM1/images/g_tkz_T101_VennIntersection.meta.xml}{T101_VennIntersection}
    \component{generic_image}{content/rwth/HM1/images/g_tkz_T101_VennUnion.meta.xml}{T101_VennUnion}
    \component{js_lib}{system/media/mathlets/GWTGenericVisualization.meta.xml}{gwtviz}
  \end{components}
  \begin{links}
    \link{generic_article}{content/rwth/HM1/T201neu_Vollstaendige_Induktion/g_art_content_01_indirekter_widerspruchsbeweis.meta.xml}{content_01_indirekter_widerspruchsbeweis}
    \link{generic_article}{content/rwth/HM1/T101neu_Elementare_Rechengrundlagen/g_art_content_01_zahlenmengen.meta.xml}{content_01_zahlenmengen}
    \link{generic_article}{content/rwth/HM1/T101neu_Elementare_Rechengrundlagen/g_art_content_02_rechengrundlagen_terme.meta.xml}{content_02_rechengrundlagen_terme}
    \link{generic_article}{content/rwth/HM1/T202_Reelle_Zahlen_axiomatisch/g_art_content_04_koerperaxiome.meta.xml}{content_04_koerperaxiome}
    \link{generic_article}{content/rwth/HM1/T207_Intervall_Schachtelung/g_art_content_21_intervalle.meta.xml}{intervalle}
  %  \link{generic_article}{content/rwth/HM1/T203_komplexe_Zahlen/g_art_content_08_algebraische_darstellung.meta.xml}{komplexe_Zahlen}
  \end{links}
  \creategeneric
\end{metainfo}
\begin{content}

%
\usepackage{mumie.ombplus}
\ombchapter{1}
\ombarticle{1}

\usepackage{mumie.genericvisualization}

\begin{visualizationwrapper}

\lang{de}{\title{Mengen und Zahlenmengen}}
\lang{en}{\title{Sets and sets of numbers}}
 
\begin{block}[annotation]
  Dieser erste Abschnitt ist neu. Er wiederholt einige grundlegende Definitionen und Konventionen, 
  die im folgenden Abschnitt bei der Einführung der elementaren Rechengrundlagen verwendet und 
  vorausgesetzt werden. Es beinhaltet Teile des ursprünglichen Kapitels 2.1 (T201_content_01_natürliche_zahlen), 
  ergänzt umd die wichtigsten Grundlagen der Mengenlehre. 
  
  Insbesondere werden in diesem Grundlagenkapitel Konventionen definiert bzw. festgelegt, die für den 
  gesamten HM1 gültig sind, u.a.
  \begin{itemize}
        \item \ref{def:menge} Der Begriff der Menge
        \item \ref{def:mengenrelationen} Die Definition von Mengenrelationen und Mengenoperationen                
        \item \ref{def:zahlenmengen} Die Definition der Zahlenmengen $\N$, $\Z$, $\Q$, $\R$    
        \item \ref{notation:menge_relle_zahlen} Notationen für spezielle Teilmengen von $\R$ und $\Q$  
        \item \ref{def:betrag} Anordnung von Zahlen und Beträge
        \item \ref{def:intervall} Die Einführung von Intervallen
      
    \end{itemize}    
  
\end{block}

\begin{block}[annotation]
	Im Ticket-System: \href{https://team.mumie.net/issues/19189}{Ticket 19189}
\end{block}


\begin{block}[info-box]
\tableofcontents
\end{block}

% Motivation
\lang{de}{
  In diesem ersten Kapitel befassen wir uns mit den \textbf{elementaren Rechengrundlagen}  
  der Mathematik. Hierzu erinnern wir uns zunächst an die aus der Schulmathematik bekannten Mengen
  der natürlichen, ganzen, rationalen und reellen Zahlen  
  (den \ref[content_01_zahlenmengen][Zahlenmengen]{sec:zahlenmengen})  und ihre grundlegenden Eigenschaften. 
  Im nachfolgenden Abschnitt wiederholen wir die 
  \ref[content_02_rechengrundlagen_terme][Grundrechenarten]{sec:grundrechenarten}
   $+$, $-$, $\cdot$ und $:$ und die damit verbundenen Rechenregeln. Ziel ist hierbei jedoch nur eine 
   Auffrischung der mathematischen Grundlagen. Wer sich intensiver damit beschäftigen möchte, dem sei  
   das Modul
   \href{https://www.ombplus.de/ombplus/link/ElemenRechneMengenZahlen/MengenZahlenGrundr}{"IA Elementares Rechnen: Mengen und Zahlen"}
   des \emph{OMB+} empfohlen.
}
\lang{en}{
  In this first chapter we look at some \textbf{foundations of mathematics}, reviewing the sets of natural numbers, integers, rational numbers and real 
  numbers (\ref[content_01_zahlenmengen]['sets of numbers']{sec:zahlenmengen}), which the reader may already be familiar with, and their 
  elementary properties.
  In the following section we review the \ref[content_02_rechengrundlagen_terme][basic operations]{sec:grundrechenarten} $+$, $-$, $\cdot$ and $:$, and the rules associated with them. The aim is to revise some
  mathematical foundations which the reader may already have encountered. For those who want to cover these foundations more thoroughly, the \emph{OMB+} module 
  \href{https://www.ombplus.de/ombplus/link/ElemenRechneMengenZahlen/MengenZahlenGrundr}{'IA Elementary Calculus: Sets and Numbers'} 
  is recommended.
}
   

\section{\lang{de}{Mengen}\lang{en}{Sets}} \label{sec:mengen}

\lang{de}{Wir definieren zunächst den Begriff der \emph{Menge}. Dieser Begriff wird in unserem Alltag vielfältig 
          verwendet. So spricht man beispielsweise von der Menge der Gabeln in der Besteckschublade oder von einer bestimmten Menge 
          Mehl als Zutat für einen Kuchenteig. Die Mathematik erfordert jedoch eine klare und exakte Definition für
          den Mengenbegriff. Der Mathematiker Georg Cantor (1845 - 1918), der als Begründer der \emph{naiven 
          Mengenlehre} gilt, hat als einer der Ersten den Begriff der \emph{Menge} \, wie folgt definiert:}
\lang{en}{We first provide a definition of a set. We use this term in several contexts in our day to day lives, referring for example to a set of 
          cutlery, or a set of rules. However, we require a clear and precise definition if we are to use it in mathematics. The mathematician
          Georg Cantor (1845 - 1918), who first developed \emph{naïve set theory}, was one of the first to define a set as follows:}

\begin{definition}[\lang{de}{Menge}\lang{en}{Set}] \label{def:menge}
\lang{de}{Eine \emph{\notion{Menge} $M$} ist eine Zusammenfassung von bestimmten, unterscheidbaren Objekten unserer 
          Anschauung oder unseres Denkens zu einem Ganzen. Die Objekte dieser Zusammenfassung nennt man 
          \emph{die \notion{Elemente} von $M$}. Ist $x$ ein Element der Menge $M$, so schreibt man
          \[x\in M\] 
          (sprich: \emph{\glqq x ist Element von M\grqq} oder \emph{\glqq x liegt in M\grqq}); andernfalls schreibt man
          \[x\notin M\] 
          (sprich: \emph{\glqq x ist kein Element von M\grqq} oder \emph{\glqq x liegt nicht in M\grqq)}.}
\lang{en}{A \emph{\notion{set} $M$} is a collection of definite, distinguishable objects of perception or thought, concieved as a whole.
          The objects in the collection are called \emph{the \notion{elements} (or \notion{members}) of $M$}. If $x$ is an element of $M$, we write
          \[x\in M\] 
          (said: \emph{'x is an element of M'}, or \emph{'x is in M'}); otherwise we write
          \[x\notin M\] 
          (said: \emph{'x is not an Element of M'}, or \emph{'x is not in M')}.}
\end{definition}

\lang{de}{Diese anschauliche Definition ist zwar aus mathematischer Sicht nicht sehr präzise, aber für die Einführung 
          der Zahlenmengen zunächst ausreichend.}
\lang{en}{This elegant description may not be very mathematically precise, but it is sufficient for the following introduction to sets of numbers.
}

%%
%%% Video K.M. *** NEU: Video 10859 gekürzt ***
%
%          Etwas formeller und genauer werden Mengen in folgendem Video definiert und beschreiben.
%          % inklusive eines kurzen Ausblicks auf Intervalle, die wir jedoch erst später in Abschnitt 
%          % \ref{sec:intervall} einführen werden. 
%
%          \floatright{\href{https://api.stream24.net/vod/getVideo.php?id=10962-2-10859&mode=iframe&speed=true}
%          {\image[75]{00_video_button_schwarz-blau}}}\\
%          \\
%          
%
\lang{de}{Eine mathematisch präzise Definition erhält man allerdings erst in Verbindung mit einer 
          \emph{axiomatischen Beschreibung}, worauf wir hier allerdings nicht weiter eingehen werden.}
\lang{en}{A mathematically precise definition can be obtained using an \emph{axiomatic definition}, but we do not go into this further here.}
\\
          
%          zum Beispiel für die Menge der reellen Zahlen,
%          erhalten wir erst in Verbindung mit einer \emph{axiomatischen Beschreibung} \, der reellen Zahlen im 
%          Kapitel \ref[content_04_koerperaxiome][Charakterisierung der reellen Zahlen]{sec:axiome}. 
%          

\begin{remark}
   \lang{de}{Mengen können auf verschiedene Weisen beschrieben werden. Es gibt folgende \textbf{Darstellungsformen für Mengen}:}
   \lang{en}{A set can be described in various ways, some of which are given below.}
   \begin{itemize}
        \item \lang{de}{
              Eine Menge kann in einer \textbf{aufzählenden} Schreibweise dargestellt werden, 
              bei der die Elemente der Menge zwischen geschweifte Klammern gesetzt und durch Semikolons getrennt werden. 
              So beschreibt beispielsweise \[  M:=\{a; b; c\}  \] die Menge, die die Buchstaben $a,b,c$ enthält. 
              Offensichtlich gilt $a\in M$ und $d\notin M$.
              
              Dabei steht das Zeichen  $:= \,$  für \emph{\glqq wird definiert durch \grqq} und bedeutet hier, dass $M$ als Symbol für die 
              Menge $\{a; b; c\}$ festgelegt wird. Der Doppelpunkt steht dabei immer auf der zu definierenden Seite.\\
              
              (Hinweis: In der Literatur findet man auch Mengen mit Komma als Trennzeichen zwischen Elementen.
              In diesem Kurs wird ausschließlich das Semikolon als Trennzeichen verwendet.)
              }
              \lang{en}{
              A set can be explicitly written, or \textbf{enumerated}, by writing its elements inside a pair of braces 
              ('curly brackets'), seperated by semicolons. For example, \[  M:=\{a; b; c\}  \] describes the set containing the letters $a,b,c$.
              Clearly $a\in M$ and $d\notin M$.
              \\
              The symbol $:= \,$ used above can be read as \emph{'is defined by'}, and means that $M$ is to be used as a name for the set 
              $\{a; b; c\}$. The colon is always on the side which is being defined.
              \\
              (Important: in the literature, commas are also used to seperate elements of a set. In this course however, only semicolons are used for 
              this purpose.)
              }
              
        \item \lang{de}{
              Wenn eine in aufzählender Schreibweise dargestellte Menge sehr viele oder gar unendlich viele Elemente 
              enthält und aus der Nennung einzelner Elemente die übrigen Elemente absolut klar und zweifelsfrei erkennbar 
              sind, kann die Beschreibung durch Verwendung von Auslassungspunkten vereinfacht dargestellt werden, z.\,B. 
              
              \begin{center} $\quad \{3; 4; 5;\ldots; 15\} \; $ für die Menge $\; \{3; 4; 5; 6; 7; 8; 9; 10; 11; 12; 13; 14; 15\} $ \end{center}
              oder  
              \begin{center} $\{2; 4; 6; 8; 10; 12; 14; \ldots \}\quad $ für die Menge aller geraden Zahlen. \end{center}
              }
              \lang{en}{
              If a set has many or infinitely many elements, and if it is absolutely clear from giving only some of its elements what the set is,
              we can replace some elements with an ellipsis. We write for example
              
              \begin{center} $\quad \{3; 4; 5;\ldots; 15\} \; $ instead of $\; \{3; 4; 5; 6; 7; 8; 9; 10; 11; 12; 13; 14; 15\} $, \end{center}
              or we write
              \begin{center} $\{2; 4; 6; 8; 10; 12; 14; \ldots \}\quad $ for the set of all even numbers larger than $0$. \end{center}
              }
   
        \item \lang{de}{
              Eine andere Form ist die \textbf{beschreibende} Darstellung einer Menge. Dabei werden die Elemente der Menge
              durch ihre charakteristischen Eigenschaften beschrieben, die hinter einem senkrechten Trennstrich \glqq $\mid$\grqq
              angegeben werden, z.\,B.
              \[  M:=\{ h \;  \text{ist ein Haus} \mid  h \; \text{hat mehr als 2 Etagen} \}  \]
              bedeutet \glqq $M$ ist die Menge aller Häuser, die mehr als 2 Etagen haben\grqq.
              
              Manchmal wird statt des Trennstrichs auch ein Doppelpunkt \glqq $:$\grqq verwendet:
              \[  M:=\{ h \;  \text{ist ein Haus} :  h \; \text{hat mehr als 2 Etagen} \}.  \]
              }
              \lang{en}{
              A set can also be written \textbf{descriptively}, by giving its properties after a pipe symbol '|', which means \emph{'such that'}.
              For example,
              \[  M:=\{ h \;  \text{is a house} \mid  h \; \text{has more than two floors} \}  \]
              means 'define $M$ as the set of all houses with more than two floors'.

              Sometimes a colon '$:$' is used instead of a pipe symbol:
              \[  M:=\{ h \;  \text{is a house} :  h \; \text{has more than two floors} \}.  \]
              }
              
        \item \lang{de}{
              Eine Menge, die keine Elemente enthält, heißt \textbf{leere Menge} und wird mit $\emptyset$ oder $\{\}$ bezeichnet.
              }
              \lang{en}{
              The set which does not contain any elements is called the \textbf{empty set}, and is denoted by $\emptyset$ or $\{\}$.
              }
           
     \end{itemize}    
\end{remark} 

%
%%% Video K.M. *** NEU: Video 10859 gekürzt ***
%
\lang{de}{
          Im folgenden Video werden Mengen noch etwas vereinfachter, mit Blick auf ihre 
          Verwendung in der Mathematik, beschrieben. \\
          \floatright{\href{https://api.stream24.net/vod/getVideo.php?id=10962-2-10859&mode=iframe&speed=true}
          {\image[75]{00_video_button_schwarz-blau}}}\\
          \\\\
Neben den verschiedenen Darstellungsformen sind auch die Relationen zwischen Mengen von Bedeutung.
}
%
\lang{en}{It is also important to be able to express the relationships between sets.}

\begin{definition}[\lang{de}{Mengenrelationen}\lang{en}{Relationships between sets}] \label{def:mengenrelationen}
\lang{de}{
  Es seien $A$ und $B$ zwei Mengen. \\
  
  $A$ heißt \emph{\notion{Teilmenge} von $B$}, wenn jedes Element von $A$ auch Element von $B$ ist. 
  $B$ bezeichnet man in diesem Fall als \emph{\notion{Obermenge} von $A$}. Man schreibt: \\

  \[\quad A\subseteq B.\]
  
  Der Operator \glqq$ \subseteq $\grqq  verdeutlicht, dass auch die Gleichheit $A=B$ 
  möglich ist. Wenn $A$ eine Teilmenge von $B$ ist und $A$ und $B$ nicht gleich sind, 
  so bezeichnet man $A$ auch als \emph{\notion{echte Teilmenge} von $B$} und schreibt:

  \begin{center}
   $\quad A\subsetneq B \quad$ oder auch einfach $\quad A\subset B.$
  \end{center}

  Die Teilmengenbeziehung wird auch \emph{\notion{Inklusion}} genannt.}
\lang{en}{
  Let $A$ and $B$ be two sets. \\

  $A$ is called a \emph{\notion{subset} of $B$} if every element of $A$ is also an element of $B$.
  $B$ is then called a \emph{\notion{superset} of $A$}. We write: \\

  \[\quad A\subseteq B.\]

  The operator '$ \subseteq $' means that it is possible for the two sets to be equal, i.e. $A=B$. \\
  If $A$ is a subset of $B$ and $A$ is not equal to $B$, we call $A$ a \emph{\notion{strict subset} of $B$} and write:

  \begin{center}
   $\quad A\subsetneq B \quad$ or even just $\quad A\subset B.$
  \end{center}

  The relationship between a set and its subset is also called \emph{\notion{set inclusion}}.
}
 
\end{definition}

\lang{de}{\textbf{Hinweise: }}
\lang{en}{\textbf{Tips: }}
\begin{itemize}
    \item \lang{de}{Statt $\subseteq$ und $\subsetneq$ sind auch die Symbole 
        $\subseteqq$ und $\subsetneqq$ gebräuchlich.}
          \lang{en}{Sometimes $\subseteqq$ and $\subsetneqq$ are used instead
          of $\subseteq$ and $\subsetneq$.}
    \item \lang{de}{Für $A$ \emph{\notion{nicht} Teilmenge von $B$}
        schreibt man $\; A \not\subseteq B.$}
          \lang{en}{$\; A \not\subseteq B$ means $A$ is \emph{\notion{not} 
          a subset of $B$}.}
    \item \lang{de}{Für $A$ \emph{\notion{nicht} echte Teilmenge von $B$}
        schreibt man $\; A \not\subset B.$}
          \lang{en}{$\; A \not\subset B$ means $A$ is \emph{\notion{not} 
          a strict subset of $B$}.}
\end{itemize}


\begin{definition}[\lang{de}{Mengenoperationen}\lang{en}{Set operations}] \label{def:mengenoperationen}

  \lang{de}{Für zwei Mengen $A$ und $B$ definiert man }
  \lang{en}{For the sets $A$ and $B$ we define }

  \begin{itemize}
  
    \item
    \lang{de}{
    die \emph{\notion{Vereinigung} $A\cup B$ von $A$ und $B$} als die Menge, die alle Elemente
    von $A$ und alle Elemente von $B$ enthält: 
    \[
      A\cup B :=\{x\mid x\in A \text{ oder }\; x\in B\},
    \]
    }
    \lang{en}{
    the \emph{\notion{union} $A\cup B$ of $A$ and $B$} as the set containing all elements of $A$ and all elements of $B$:
    \[
      A\cup B :=\{x\mid x\in A \text{ or }\; x\in B\},
    \]
    }

    \item
    \lang{de}{
    den \emph{\notion{Durchschnitt} $A\cap B$ von $A$ und $B$} als die Menge aller Elemente,
    die sowohl in $A$ als auch in $B$ liegen:
    \[
      A\cap B :=\{x\mid x\in A \text{ und }\; x\in B\},
    \]
    }
    \lang{en}{
    the \emph{\notion{intersection} $A\cap B$ of $A$ and $B$} as the set containing all elements that are in both $A$ and $B$:
    \[
      A\cap B :=\{x\mid x\in A \text{ and }\; x\in B\},
    \]
    }

    \item
    \lang{de}{
    und die \emph{\notion{Differenz(menge)} $A\setminus B\,$} (\glqq $A$ ohne $B$\grqq) als die Menge der Elemente in $A$,
    die nicht in $B$ liegen:  
    \[
      A\setminus B :=\{x\mid x\in A \text{ und }\; x\notin B\}.
    \]
%
    Gilt zusätzlich $\, B\subseteq A, \;$ dann nennt man diese Differenzmenge auch das
    \emph{\notion{Komplement}} von $\,B\,$ in $\,A\,$ und schreibt 
    \[ \complement_A(B):=A\setminus B . \]
    }
    \lang{en}{
    and the \emph{\notion{difference} $A\setminus B\,$} ('$B$ subtracted from $A$') as the set of elements of $A$ that are not elements of $B$:
    \[
      A\setminus B :=\{x\mid x\in A \text{ and }\; x\notin B\}.
    \]
    If we also have $\, B\subseteq A, \;$ then this difference is also called the \emph{\notion{complement}} of $\,B\,$ in $\,A\,$ and we write 
    \[ \complement_A(B):=A\setminus B . \]
    }
    
\\
    \item
    \lang{de}{
    Die \emph{\notion{Produktmenge} $A \times B$ von $A$ und $B \,$} ist die Menge aller \emph{geordneten Paare $(x,y)$}
    mit $x \in A$ und $y \in B$. Man schreibt: \\
    \[
      A \times B :=\{(x,y) \mid x\in A \text{ und }\; y \in B\}.
    \]
    Man bezeichnet $A \times B$ auch als \emph{\notion{kartesisches Produkt} von $A$ und $B$.}
    }
    \lang{en}{
    The \emph{\notion{Cartesian product} $A \times B$ of $A$ and $B \,$} is the set of all \emph{ordered pairs $(x,y)$}
    with $x \in A$ and $y \in B$. We write: \\
    \[
      A \times B :=\{(x,y) \mid x\in A \text{ and }\; y \in B\}.
    \]
    }
    \end{itemize}

\end{definition}

% \lang{de}{In obiger Definition ist bei \textit{$x\in A$ oder $x\in B$} \textbf{kein} \glqq{}entweder oder\grqq{}, sondern ein
% \glqq logisches Oder\grqq gemeint: $x$ ist in $A$ oder in $B$ oder in beiden
% Mengen (gleichzeitig) enthalten. Ebenso ist bei \textit{$x\in A$ und
%  $x\in B$} als \glqq logisches Und\grqq gemeint: $x$ ist sowohl in $A$ als
% auch in $B$ enthalten.}         
%

\lang{de}{
Mengenrelationen und Mengenoperationen lassen sich zum Beispiel durch sogenannte \emph{Venn-Diagramme} grafisch veranschaulichen, wie
in nachfolgendem Video und in Beispiel \ref{bsp:venndiagramm} dargestellt.
%
%%% Video K.M. *** NEU: 11264 (Mengen_Kartoffeln) ***
%
          \floatright{\href{https://api.stream24.net/vod/getVideo.php?id=10962-2-11264&mode=iframe&speed=true}
          {\image[75]{00_video_button_schwarz-blau}}}\\     
%
}
\lang{en}{Relationships between sets and set operations can be visualised using \emph{Venn diagrams}, which are used in example \ref{bsp:venndiagramm}.}

\begin{example} \label{bsp:venndiagramm}
  \begin{tabs*}[\initialtab{0}]
  \tab{\lang{de}{Mengenoperationen, -relationen und Venn-Diagramme}\lang{en}{Set operations, relationships between sets and Venn diagrams}}


    \lang{de}{Für die Mengen $\; A=\{1;2;3;4\},\;$ $B=\{3;4;5\}\;$ und $\; C=\{2;3\}\;$ gilt:}
    \lang{en}{For the sets $\; A=\{1;2;3;4\},\;$ $B=\{3;4;5\}\;$ and $\; C=\{2;3\}\;$ we have the following:}


  \begin{enumerate}
   \item \lang{de}{Die Vereinigung von $A$ und $B \;$ ist die Menge $\;  A\cup B= \{ 1; 2; 3; 4; 5 \} .$}
         \lang{en}{The union of $A$ and $B \;$ is the set $\;  A\cup B= \{ 1; 2; 3; 4; 5 \} .$}
  
    \begin{figure}
    \image{T101_VennUnion}
    \end{figure}
%
   \item \lang{de}{Der Durchschnitt von $A$ und $B \;$ ist die Menge $\;  A\cap B= \{ 3; 4\}. $}
         \lang{en}{The intersection of $A$ and $B \;$ is the set $\;  A\cap B= \{ 3; 4\}. $}
 
    \begin{figure}
    \image{T101_VennIntersection}
    \end{figure}
%        
   \item \lang{de}{Die Differenz von $A$ ohne $B \;$ ist die Menge $\; A\setminus B=\{ 1; 2\}.$}
         \lang{en}{$B$ subtracted from $A \;$ is the set $\; A\setminus B=\{ 1; 2\}.$}
    
    \begin{figure}
    \image{T101_VennDifference_A}
    \end{figure}
%       
   \item \lang{de}{Die Differenz $B$ ohne $A \;$ ist die Menge $\; B\setminus A=\{ 5\}.$}
         \lang{en}{$A$ subtracted from $B \;$ is the set $\; B\setminus A=\{ 5\}.$}
    
    \begin{figure}
    \image{T101_VennDifference_B}
    \end{figure}
%        
  \item \lang{de}{$C \;$ ist Teilmenge von $A$, aber nicht von $B: \;$ $\; C \subseteq A \quad \text{und} \quad C  \not\subseteq B.$}
        \lang{en}{$C \;$ is a subset of $A$, but not of $B: \;$ $\; C \subseteq A \quad \text{and} \quad C  \not\subseteq B.$}
    
    \begin{figure}
    \image{T101_VennSubset}
    \end{figure}
% 
  \item \lang{de}{Das Komplement von $C\,$ in $A\,$ ist $A\setminus C=\{ 1;4\}$.}
        \lang{en}{The complement of $C\,$ in $A\,$ is $A\setminus C=\{ 1;4\}$.}
    
    \begin{figure}
    \image{T101_VennComplement}
    \end{figure}    
  \end{enumerate}
 \end{tabs*} 
\end{example}

%
\lang{de}{Aus den Definitionen \ref{def:mengenrelationen} und \ref{def:mengenoperationen} lassen sich folgende Aussagen über Mengen ableiten:}
\lang{en}{From definitions \ref{def:mengenrelationen} and \ref{def:mengenoperationen} we get the following statements about sets:}

\begin{itemize}
     \item \lang{de}{Für jede Menge $M$ gilt:}
           \lang{en}{For every set $M$,}
        
        \begin{itemize}    
         \item[a)] $M \subseteq M$,
         \item[b)] $M \cap M = M = M \cup M$,
         \item[c)] $M \setminus M = \emptyset$.
        \end{itemize}
%
      \item \lang{de}{Der Durchschnitt zweier Mengen ist eine (gemeinsame) Teilmenge der beiden Mengen.}
            \lang{en}{The intersection of two sets is a subset of both of them.}
%
      \item \lang{de}{Folglich ist die leere Menge $\emptyset$ Teilmenge einer jeden Menge (denn es gilt $A \cap \emptyset = \emptyset $).}
            \lang{en}{Thus the empty set $\emptyset$ is a subset of every set (as $A \cap \emptyset = \emptyset $).}
%
      \item \lang{de}{Die Vereinigung zweier Mengen ist eine (gemeinsame) Obermenge der beiden Mengen.}
            \lang{en}{The union of two sets is a superset of both of them.}
%
      \item \lang{de}{Ist $T$ eine Teilmenge von $P$, so ist auch das Komplement $ P\setminus T \;$ Teilmenge von $P\;$ 
            und es gilt: \[ (P\setminus T ) \cap T = \emptyset\quad \text{und} \quad (P\setminus T ) \cup T = P.\] 
            Zwei Mengen mit leerem Durchschnitt wie hier $T$ und $P \setminus T$ bezeichnet man auch 
            als \emph{\notion{disjunkte} Mengen.}}
            \lang{en}{If $T$ is a subset of $P$, then the complement $ P\setminus T \;$ is a subset of $P\;$, satisfying:
            \[ (P\setminus T ) \cap T = \emptyset\quad \text{and} \quad (P\setminus T ) \cup T = P.\]
            Two sets with empty intersection (for example $T$ and $P \setminus T$) are also called \emph{\notion{disjoint} sets.}}
%
\end{itemize}

%%%
% Kurztzest zu Mengen
%
%%%%


% \begin{tabs*}[\initialtab{0}]
% \tab{\lang{de}{Mengenoperationen zum Üben}}

\begin{quickcheckcontainer}

\randomquickcheckpool{1}{4}  % vier verschiedene Terme mit randomisierten Zahlen

%1
\begin{quickcheck}
    
    \lang{de}{\text{Gegeben seien die Mengen $\; A=\{-1;1;3;4;6;7\},$ $\;B=\{3;7;8\}, \;$ 
          $C=\{-4;1;3;5\}\;$ und ihre Venn-Darstellung: 
          \begin{center}
          \image{T101_VennQuickCheck}
          \end{center}  
          Welche der folgenden Aussagen treffen zu? 
  		  }}
    \lang{en}{\text{Consider the sets $\; A=\{-1;1;3;4;6;7\},$ $\;B=\{3;7;8\}, \;$ 
          $C=\{-4;1;3;5\}\;$ and their Venn diagram representation: 
          \begin{center}
          \image{T101_VennQuickCheck}
          \end{center}  
          Which of the following statements hold? 
  		  }}
     
    \begin{variables}                             % r1=0 und r2=0
      \randint{r1}{0}{1}
      \randint{r2}{0}{2}
      \function[calculate]{a}{1}            
      \function[calculate]{b}{3}             
      \function[calculate]{c}{3}             
      \function[calculate]{d}{7}             
    \end{variables}


    \begin{choices}{multiple}
    
      \begin{choice}
      \text{$B\cap C= \emptyset$}
      \solution{false}
		\end{choice}
		
      \begin{choice}
      \text{$A\setminus (C\cup B)=\{-1; 4; 6\}$ }
      \solution{true}
		\end{choice}
		
      \begin{choice}
      \text{$C\cap A=\{\var{a}; \var{b} \}$ }
      \solution{true}
		\end{choice}

      \begin{choice}
      \text{$A\setminus C=\{4; 6; \var{d}; -1 \}$ }
      \solution{true}
		\end{choice}

      \begin{choice}
      \text{$(B\cap C)\subseteq A$}
      \solution{true}
		\end{choice}
		
      \begin{choice}
      \text{$\var{c}\in (B\setminus C)$ }
      \solution{false}
		\end{choice}
     
  \end{choices}{multiple} 

\end{quickcheck}

%2
\begin{quickcheck}

    \lang{de}{\text{Gegeben seien die Mengen $\; A=\{-1;1;3;4;6;7\}, \;B=\{3;7;8\}, \; 
          C=\{-4;1;3;5\}\;$ und ihre Venn-Darstellung: 
          \begin{center}
          \image{T101_VennQuickCheck}
          \end{center}  
          Welche der folgenden Aussagen treffen zu? 
  		  }}
    \lang{en}{\text{Consider the sets $\; A=\{-1;1;3;4;6;7\}, \;B=\{3;7;8\}, \; 
          C=\{-4;1;3;5\}\;$ and their Venn diagram representation: 
          \begin{center}
          \image{T101_VennQuickCheck}
          \end{center}  
          Which of the following statements hold? 
  		  }}


    \begin{variables}                             % r1=1,2 und r2=0
      \randint{r1}{1}{2}
      \randint{r2}{0}{0}
      \function[calculate]{a}{1+r1*3}             % = 4 oder 7           
      \function[calculate]{b}{2+r1*3}             % = 5 oder 8       
      \function[calculate]{c}{3}             
      \function[calculate]{d}{8}            
    \end{variables}

    \begin{choices}{multiple}
    
      \begin{choice}
      \text{$B\cap C= \{3\}$}
      \solution{true}
		\end{choice}
        
      \begin{choice}
      \text{$\var{a}\in (A\setminus C)$ }
      \solution{true}
		\end{choice}

      \begin{choice}
      \text{$A\setminus (C\cup B)=\{-1; 4; 6; 7\}$ }
      \solution{false}
		\end{choice}
		
      \begin{choice}
      \text{$B\cap A=\{\var{a}; \var{b} \}$ }
      \solution{false}
		\end{choice}

      \begin{choice}
      \text{$A\setminus C=\{4; 6; \var{b}; -1 \}$ }
      \solution{false}
		\end{choice}

      \begin{choice}
      \text{$\var{c}\in (B\setminus C)$ }
      \solution{false}
		\end{choice}
     
  \end{choices}{multiple} 

\end{quickcheck}

%3
\begin{quickcheck}

    \lang{de}{\text{Gegeben seien die Mengen $\; A=\{-1;1;3;4;6;7\}, \;B=\{3;7;8\}, \; 
          C=\{-4;1;3;5\}\;$ und ihre Venn-Darstellung: 
          \begin{center}
          \image{T101_VennQuickCheck}
          \end{center}  
          Welche der folgenden Aussagen treffen zu? 
  		  }}
    \lang{en}{\text{Consider the sets $\; A=\{-1;1;3;4;6;7\}, \;B=\{3;7;8\}, \; 
          C=\{-4;1;3;5\}\;$ and their Venn diagram representation: 
          \begin{center}
          \image{T101_VennQuickCheck}
          \end{center}  
          Which of the following statements hold? 
  		  }}

   	     
    \begin{variables}                             % r1=0 und r2=1,2
      \randint{r1}{0}{0}
      \randint{r2}{1}{2}
      \function[calculate]{a}{1+r1*3}             % = 1
      \function[calculate]{b}{3+r1*2}             % = 3
      \function[calculate]{c}{3+5*r2-3*r2*(r2-1)} % = 8 oder 7
      \function[calculate]{d}{7}            
    \end{variables}

    \begin{choices}{multiple}
 		
      \begin{choice}
      \text{$C\cap A=\{\var{a}; \var{b} \}$ }
      \solution{true}
		\end{choice}
		
      \begin{choice}
      \text{$A\setminus (C\cup B)=\{-1; 4; 6\}$ }
      \solution{true}
		\end{choice}

      \begin{choice}
      \text{$A\setminus C=\{4; 6; \var{d}; -1 \}$ }
      \solution{true}
		\end{choice}
   
      \begin{choice}
      \text{$A\cap C \cap B= \{1; 3; 7\}$}
      \solution{false}
		\end{choice}

      \begin{choice}
      \text{$(B\cap C)\subseteq A$}
      \solution{true}
		\end{choice}
		
      \begin{choice}
      \text{$\var{c}\in (B\setminus C)$ }
      \solution{true}
		\end{choice}
     
  \end{choices}{multiple} 

\end{quickcheck}

%4
\begin{quickcheck}

    \lang{de}{\text{Gegeben seien die Mengen $\; A=\{-1;1;3;4;6;7\}, \;B=\{3;7;8\}, \; 
          C=\{-4;1;3;5\}\;$ und ihre Venn-Darstellung: 
          \begin{center}
          \image{T101_VennQuickCheck}
          \end{center}  
          Welche der folgenden Aussagen treffen zu? 
  		  }}
    \lang{en}{\text{Consider the sets $\; A=\{-1;1;3;4;6;7\}, \;B=\{3;7;8\}, \; 
          C=\{-4;1;3;5\}\;$ and their Venn diagram representation: 
          \begin{center}
          \image{T101_VennQuickCheck}
          \end{center}  
          Which of the following statements hold? 
  		  }}

   	     
    \begin{variables}                             % r1=1,2 und  r2=1,2
      \randint{r1}{1}{2}
      \randint{r2}{1}{2}
      \function[calculate]{a}{1+r1*3}             % = 4 oder 7           
      \function[calculate]{b}{3+r1*2}             % = 5 oder 7
      \function[calculate]{c}{3+5*r2-3*r2*(r2-1)} % = 8 oder 7            
      \function[calculate]{d}{8}            
    \end{variables}

    \begin{choices}{multiple}
    
      \begin{choice}
      \text{$A\setminus (C\cap B)=\{-1; 1; 4; 6\}$ }
      \solution{false}
		\end{choice}
		
      \begin{choice}
      \text{$\var{a} \in (A \setminus C) \}$ }
      \solution{true}
		\end{choice}

      \begin{choice}
      \text{$A\setminus C=\{4; 6; \var{d}; -1 \}$ }
      \solution{false}
		\end{choice}

        
      \begin{choice}
      \text{$B\cap C= \emptyset$}
      \solution{false}
		\end{choice}
				
      \begin{choice}
      \text{$\var{c}\in (B\setminus C)$ }
      \solution{true}
		\end{choice}

      \begin{choice}
      \text{$(B\cap C)\subseteq A$}
      \solution{true}
		\end{choice}     
  \end{choices}{multiple} 

\end{quickcheck}
\end{quickcheckcontainer}
% \end{tabs*}
%%%
\section{\lang{de}{Zahlenmengen}\lang{en}{Sets of numbers}} \label{sec:zahlenmengen}

\lang{de}{Wir werden nun im Folgenden die bekannten Mengen der natürlichen, der ganzen, der rationalen und der reellen Zahlen 
          als Beispiele für \emph{Mengen} im Sinne der obigen Definition \ref{def:menge} einführen:}
\lang{en}{We will now introduce the natural numbers, integers, rational numbers and real numbers as examples of \emph{sets} in the sense of the above definition \ref{def:menge}.}

\begin{definition}[Zahlenmengen]  \label{def:zahlenmengen}
   \lang{de}{Die \emph{Menge der \notion{nat\"urlichen Zahlen ohne $\; 0$}} ist  
            \[\mathbb{N}:=\{1; 2; 3; 4;\ldots\}.\]
            Nimmt man die Null hinzu, wird dies durch den Index $0$ gekennzeichnet.
            Die \emph{Menge der \notion{nat\"urlichen Zahlen mit $\; 0$}} ist somit
            \[\mathbb{N}_0:=\mathbb{N} \cup \{0\}=\{0; 1; 2; 3; 4;\ldots\}.\]
   }
   \lang{en}{The \emph{set of \notion{natural numbers without $0$}} is  
            \[\mathbb{N}:=\{1; 2; 3; 4;\ldots\}.\]
            If we include zero, we add an index $0$ to the notation.
            The \emph{set of \notion{natural numbers with $0$}} is therefore
            \[\mathbb{N}_0:=\mathbb{N} \cup \{0\}=\{0; 1; 2; 3; 4;\ldots\}.\]
   }

  \lang{de}{Die \emph{Menge der \notion{ganzen Zahlen}} ist 
            \[\mathbb{Z}:=\mathbb{N}_0 \cup \{ -1; -2; -3; \ldots\}=\{\ldots; -2; -1; 0; 1; 2; \ldots\}.\] 
  }
  \lang{en}{The \emph{set of \notion{integers}} is 
            \[\mathbb{Z}:=\mathbb{N}_0 \cup \{ -1; -2; -3; \ldots\}=\{\ldots; -2; -1; 0; 1; 2; \ldots\}.\] 
  }

  \lang{de}{Die \emph{Menge der \notion{rationalen Zahlen}} ist 
            \[\mathbb{Q}:=\Big\{\,\frac{p}{q}\,\Big|\,p,q\in\mathbb{Z},\~q\neq 0\,\Big\}.\] 
            Hierbei nennt man $p$ den \strong{Z"ahler} und $q$ den \strong{Nenner} des \strong{Bruches} $\, \frac{p}{q}.$
  }
  \lang{en}{The \emph{set of \notion{rational numbers}} is 
            \[\mathbb{Q}:=\Big\{\,\frac{p}{q}\,\Big|\,p,q\in\mathbb{Z},\~q\neq 0\,\Big\}.\]
            We call $p$ the \strong{numerator} and $q$ the \strong{denominator} of the \strong{fraction} $\, \frac{p}{q}.$
  }
  
  
\end{definition}

\lang{de}{
Wir sehen, dass jede natürliche Zahl zugleich eine ganze Zahl ist. Die Menge der ganzen Zahlen ist jedoch eine \emph{echte Obermenge} zur
Menge der natürlichen Zahlen, da beispielsweise $-7\in \Z$, aber $-7\notin \N$ gilt. Ferner ist jede ganze, und damit auch jede natürliche
Zahl $z$ aufgrund ihrer Darstellung als Bruch \strong{$z=\frac{z}{1}$} zugleich eine rationale Zahl.
}
\lang{en}{
We see that every natural number is also an integer. The set of natural numbers is therefore a \emph{strict subset} of the set of integers, as there are elements of the integers which are not natural numbers (i.e. $-7\in \Z$ but $-7\notin \N$).
Furthermore, every integer $z$ can be expressed as a fraction \strong{$z=\frac{z}{1}$}, so every integer is rational.
}

\begin{center}
\image{T101_NumberSets}
\end{center} 


\begin{showhide}[\lang{de}{\textbf{Das Euler-Diagramm der Zahlenbereiche}}\lang{en}{\textbf{The Euler diagram of numbers}}]	
\lang{de}{
Bei diesem Mengendiagramm handelt es sich um ein sogenanntes \emph{Euler-Diagramm}, benannt nach dem 
Schweizer Mathematiker Leonard Euler. Euler-Diagramme unterscheiden sich von Venn-Diagrammen dadurch, 
dass in Euler-Diagrammen nur solche Überschneidungen zu sehen sind, in denen es tatsächlich auch gemeinsame 
Elemente der sich schneidenden Mengen gibt. In Venn-Diagramme können auch Überschneidungen dargestellt sein, 
die keine Elemente enthalten, die also quasi leer sind.
}
\lang{en}{
This diagram is called an Euler diagram, named after the Swiss mathematician Leonard Euler. Euler diagrams are different from
Venn diagrams as overlapping ovals in the former only occur between sets which actually share elements. In Venn diagrams however, overlapping ovals can represent an empty intersection.
}
\end{showhide}

\lang{de}{Es gelten also die Inklusionen $ \N\subseteq \N_0\subseteq \Z\subseteq \Q.$}
\lang{en}{We therefore have the inclusion chain $ \N\subseteq \N_0\subseteq \Z\subseteq \Q.$}


\\
\lang{de}{
Die ganzen Zahlen (und damit auch die natürlichen Zahlen) kann man sich als Punkte 
auf der sogenannten \emph{Zahlengerade} veranschaulichen.
Von der Null als Referenzpunkt liegen dann in immer gleichen Abständen nach rechts die 
nat"urlichen Zahlen $1$, $2$, $3$ usw., und in immer gleichen Abständen nach links die zugehörigen
negativen Zahlen $-1$, $-2$, $-3$ usw.
}
\lang{en}{
Integers (and thus natural numbers) can be visualised as points on a so-called \emph{number line}.
Using zero as a central point, the natural numbers $1$, $2$, $3$ etc. lie to its right, evenly spaced on the line. 
Likewise, the negative numbers $-1$, $-2$, $-3$ etc. lie to the left.
}

\begin{center}
\image{T101_NumberLine_A}
\end{center}

\lang{de}{
Auch die rationalen Zahlen können als Punkte auf der Zahlengerade 
veranschaulicht werden. Zum Beispiel erhält man die Zahl $\frac{3}{5}$ (\glqq Drei durch Fünf\grqq oder \glqq Dreif"unftel\grqq), 
wenn man die Strecke, die von Null nach Drei geht, in fünf Teile gleicher Länge teilt.
}
\lang{en}{
Rational numbers can also be visualised as points on a number line. For example, one obtains the number $\frac{3}{5}$ ('three fifths') 
by dividing the line segment from zero to three into five segments of equal length.
}

\begin{center}
\image{T101_NumberLine_B}
\end{center}

\lang{de}{
Alternativ kann man auch die Abschnitte zwischen zwei benachbarten ganzen Zahlen in jeweils f"unf gleiche St"ucke teilen.
Dann ist das St"uck von einem Teilungspunkt zum nächsten immer ein F"unftel, und die Zahl $\frac{3}{5}$ liegt
also rechts von der Null beim dritten Teilungspunkt.
}
\lang{en}{
Alternatively, the distance between two consecutive numbers can be divided into five equal sections, so that the number $\frac{3}{5}$
lies between the third and fourth section to the right of zero.
}

\begin{center}
\image{T101_NumberLine_C}
\end{center}

\lang{de}{
Als Ergebnis der Division zweier ganzer Zahlen ist eine rationale Zahl als Dezimalzahl 
entweder abbrechend oder periodisch, d.\,h. sie hat entweder endlich viele Nachkommastellen, oder sie hat unendlich
viele Nachkommastellen mit einer festen Folge von Ziffern, die sich immer wiederholt. 
\\
Aufgrund dieser Eigenschaft können die rationalen Zahlen die Zahlengerade nicht lückenlos ausfüllen, denn auch 
Zahlen mit einer nicht-periodischen unendlichen Dezimaldarstellung sind Bestandteil der Zahlengerade.
\\
Diese Zahlen mit einer nicht-periodischen unendlichen Dezimaldarstellung, die also nicht als rationale Zahl
$\frac{p}{q}\,$ mit $\,p,q\in\mathbb{Z},\,q\neq 0\,$ dargestellt werden k\"onnen, nennt man 
\emph{\notion{irrationale Zahlen}.}
\\
Ein Beispiel hierfür ist die positive Zahl $\sqrt{2}$ (sprich: \glqq Wurzel aus 2\grqq), deren Quadrat gleich $2$ ist.
}
\lang{en}{
The result of division of an integer by another integer is a rational number, which has either a finite or repeating decimal representation.
That is to say, it can either be written as a decimal ending in infinite zeros, or as as a decimal which eventually repeats itself, endlessly
looping over a sequence of one or more digits.
\\
Due to this, rational numbers alone are not enough to completely fill the number line. After all, numbers which have a non-periodic infinite 
decimal representation are also part of the number line, and cannot be obtained by dividing an integer by another integer. These numbers are hence 
not rational, as they cannot be written in the form $\frac{p}{q}\,$ with $\,p,q\in\mathbb{Z},\,q\neq 0\,$. We call them irrational numbers.
\\
An example of an irrational number is the positive number $\sqrt{2}$ (often said 'root two'), which gives $2$ when squared.
}

%
% liegt auf der Zahlengeraden, ist keine rationale Zahl. 
%
% Beweis
%
% \begin{tabs*}[\initialtab{0}]
% \tab{Beweis dass $\sqrt{2}$ nicht rational ist}

\begin{proof*}[\lang{de}{Beweis dass $\sqrt{2}$ nicht rational ist}\lang{en}{Proof that $\sqrt{2}$ is not rational}] \label{proof:sqrt2}
\begin{showhide}

\lang{de}{
Dass $\sqrt{2}$ keine rationale Zahl ist, kann man leicht durch einen sogenannten
 \ref[content_01_indirekter_widerspruchsbeweis][Widerspruchseweis]{sec:indirekterbeweis} 
zeigen. Hierzu nimmt man zunächst das Gegenteil an, nämlich $\sqrt{2}\in\mathbb{Q}$, und führt diese Annahme zu einem Widerspruch. 
\\
Aus der Annahme folgt nach Definition \ref{def:zahlenmengen}, \, dass es ganze Zahlen $p$ und $q$ \, ($q \neq 0$) gibt, so dass 
}
\lang{en}{
We can easily prove that $\sqrt{2}$ is not rational using a so-called \ref[content_01_indirekter_widerspruchsbeweis][proof by contradiction]{sec:indirekterbeweis}. We start by assuming that $\sqrt{2}$ is in fact rational, i.e. $\sqrt{2}\in\mathbb{Q}$, and deriving a contradiction from this.
\\
From this assumption, using definition \ref{def:zahlenmengen}, there must exist integers $p$ and $q$ \, ($q \neq 0$) such that 
}

\[\sqrt{2}= \frac{p}{q}\] 

\lang{de}{
gilt. Dabei seien $p$ und $q$ so gewählt, dass sie keinen gemeinsamen Teiler haben, der Bruch $\frac{p}{q}$ also vollständig gekürzt ist.
Man bezeichnet $p$ und $q$ dann auch als \emph{teilerfremd}. Da $\sqrt{2}$ die positive Zahl ist, deren Quadrat gleich $2$ ist, folgt:
}
\lang{en}{
holds. We can choose $p$ and $q$ so that they have no common divisor, so the fraction $\frac{p}{q}$ is reduced. We say then that $p$ and $q$ are
\emph{coprime}, or \emph{relatively prime}. As $\sqrt{2}$ is defined as the positive number whose square is $2$, we have
}

\[2= \left( \frac{p}{q} \right) ^2=\frac{p^2}{q^2}\]

\lang{de}{
und damit \center{$2 \cdot q^2=p^2.$}
\\
$p^2$ ist also eine gerade Zahl, was wiederum nur möglich ist, wenn auch $p$ selbst eine gerade Zahl ist.
Somit gibt es eine natürliche Zahl $n \in \mathbb{N}$, so dass $p=2\cdot n$ gilt. Hieraus folgt:
}
\lang{en}{
and thus \center{$2 \cdot q^2=p^2.$}
\\
$p^2$ is therefore an even number, which is of course only possible if $p$ itself is even.
Hence there exists a natural number $n \in \mathbb{N}$ such that $p=2\cdot n$. Therefore
}

\[2 \cdot q^2 = p^2 = (2n)^2 = 4\cdot n^2= 2 \cdot (2n^2), \]

\lang{de}{
also $q^2=2n^2$. \, Folglich sind $q^2$ und damit auch $q$ ebenfalls gerade Zahlen.
\\
Insbesondere sind also $q$ und $p$ beide durch $2$ teilbar, was im Widerspruch zu unserer 
anf\"anglichen Annahme steht, dass $p$ und $q$ teilerfremd sind.
\\
Damit ist gezeigt, dass $\sqrt{2}$ nicht rational sein kann.
}
\lang{en}{
so $q^2=2n^2$. Hence $q^2$ and so also $q$ are even numbers.
\\
However, this means that both $p$ and $q$ are divisible by $2$, which contradicts our initial assumption that 
$p$ and $q$ are coprime.
\\
We have shown by contradiction that $\sqrt{2}$ cannot be rational. It is therefore an irrational number.
}

\end{showhide}
\end{proof*}
%
% \end{tabs*}
%
\lang{de}{
Tats\"achlich sind die meisten Wurzeln wie $\sqrt{3}$, $\sqrt{5}$ oder $\sqrt{7}$ irrationale Zahlen, 
ebenso die Kreiszahl $\pi = 3,14\ldots$ oder die Eulersche Zahl $e = 2,71\ldots$.
}
\lang{en}{
In fact, most square roots such as $\sqrt{3}$, $\sqrt{5}$ and $\sqrt{7}$ are irrational, as are the mathematical 
constants $\pi = 3.14\ldots$ and $e = 2.71\ldots$.
}

\begin{definition}[Reelle Zahlen] \label{introduction_R}
	\lang{de}{
	Nimmt man die irrationalen Zahlen zu den rationalen hinzu - nimmt man also alle Zahlen der Zahlengerade -
	so erh\"alt man die \emph{\notion{reellen Zahlen}}.
  }
  \lang{en}{
  Combining the rational and irrational numbers, one obtains the set of all numbers on the number line, which we call the  
  set of \emph{\notion{real numbers}}.
  }
	
	\begin{center}
	\image{T101_NumberLine_D}
	\end{center}

  \lang{de}{
	Die Menge der reellen Zahlen wird mit \notion{$\R$} bezeichnet. 
	}
  \lang{en}{
  The set of real numbers is denoted by the symbol \notion{$\R$}.
  }
\end{definition}

\lang{de}{
Neben den natürlichen, den ganzen und den rationalen Zahlen gibt es weitere spezielle 
Teilmengen der reellen Zahlen, die in der Mathematik von Bedeutung sind und im weiteren Verlauf 
häufig benötigt werden.
}
\lang{en}{
Besides the natural numbers, integers and rational numbers, there are other useful subsets of the real numbers.
}

\begin{remark} \label{notation:menge_relle_zahlen}
 \lang{de}{
 Für die folgenden besonderen Teilmengen von $\R$ gibt es spezielle Notationen, die in der Literatur 
 nicht immer einheitlich sind. Deshalb legen wir für diesen Kurs folgende Notation fest:
 }
 \lang{en}{
 For the following subsets of $\R$ we have special notation, which is not consistent across all literature. For this 
 reason, we use the following notation on this course:
 }

 \lang{de}{
 \begin{table}[\class{layout}]
  $\qquad \R^\ast \;$ & bezeichnet die Menge der reellen Zahlen ohne Null, also  $\, \R\setminus \{ 0 \}$, \\
  $\qquad \R_+ \;$    & bezeichnet die Menge der positiven reellen Zahlen (ohne Null), \\ 
  %  & & \text{(also die reellen Zahlen, die auf der Zahlengeraden rechts von $0$ liegen)} \\
  $\qquad \R_-   \;$  &bezeichnet die Menge der negativen reellen Zahlen (ohne Null), \\ 
  %  & & \text{(also die reellen Zahlen, die auf der Zahlengeraden links von $0$ liegen)} \\
  $\qquad \R_{\geq 0} \; $ & bezeichnet die \emph{nicht-negativen} reellen Zahlen, also $\, \R_+ \cup \{0\},$ \\
  $\qquad \R_{\leq 0} \;$ & bezeichnet die \emph{nicht-positiven} reellen Zahlen, also $\, \R_- \cup \{0\}.$
 \end{table}
 }
 \lang{en}{
 \begin{table}[\class{layout}]
  $\qquad \R^\ast \;$ & denotes the real numbers without zero, i.e.  $\, \R\setminus \{ 0 \}$, \\
  $\qquad \R_+ \;$    & denotes the set of positive real numbers (without zero), \\ 
  %  & & \text{(i.e. the real numbers which lie to the right of zero on the number line)} \\
  $\qquad \R_-   \;$  & denotes the set of negative real numbers (without zero), \\ 
  %  & & \text{(i.e. the real numbers which lie to the left of zero on the number line)} \\
  $\qquad \R_{\geq 0} \; $ & denotes the \emph{non-negative} real numbers, i.e. $\, \R_+ \cup \{0\},$ \\
  $\qquad \R_{\leq 0} \;$ & denotes the \emph{non-positive} real numbers, i.e. $\, \R_- \cup \{0\}.$
 \end{table}
 }

 \lang{de}{
 Ebenso werden die entsprechenden Teilmengen der rationalen Zahlen bezeichnet mit
 $\Q^\ast$ \, für $\Q\setminus \{ 0 \},$ $\Q_+$ , $\Q_-$ , $\Q_{\geq 0} $ und $\Q_{\leq 0}$.
 }
 \lang{en}{
 The corresponding subsets of the rational numbers are denoted by 
 $\Q^\ast$ , $\Q_+$ , $\Q_-$ , $\Q_{\geq 0} $ and $\Q_{\leq 0}$.
 }
\end{remark}
%%%%%%%%%%%%%%%%%%%%%%%%%%%%%%%%%%%%%%%%%%%%%%%%%%%%%%%%%%%%%%%%%%%%%%%%%%%%%%%%%%%%%%%%%%%%%%%%%
%
% NEU: Anordnung der Zahlen, Vorzeichen und Betrag
%
\section{\lang{de}{Anordnung von Zahlen und Betrag}\lang{en}{Ordering numbers and magnitude}}
\lang{de}{
Bei der Beschreibung der Zahlenmengen haben wir anhand des Zahlenstrahls gesehen, dass man die 
natürlichen, ganzen, rationalen und reellen Zahlen ihrer Größe nach anordnen kann.
}
\lang{en}{
In our description of the sets of numbers on the number line, we see that the natural numbers, integers, 
rational numbers and real numbers can be ordered by size.
}

\begin{definition}[\lang{de}{Anordnung von Zahlen}\lang{en}{Ordering numbers}]\label{def:ordering}
\lang{de}{Für beliebige reelle Zahlen $a$ und $b$ gilt:}
\lang{en}{For all real numbers $a$ and $b$ we have:}
\begin{enumerate}
\item \lang{de}{
    Eine Zahl $\,a\,$ ist kleiner als eine Zahl $\,b$, wenn $\,a\,$ auf der Zahlengeraden 
    weiter links liegt als $\,b$. Man schreibt
    \begin{center}
        $a < b \quad $ (bzw. $\;a \leq b$, wenn $\,a\,$ kleiner oder gleich $\,b\,$ ist).
    \end{center}
    }
    \lang{en}{
    A number $\,a\,$ is less than a number $\,b$ if $\,a\,$ is to the left of $\,b$ on the 
    number line. We write
    \begin{center}
        $a < b \quad $ (or $\;a \leq b$ if $\,a\,$ is less than or equal to $\,b\,$).
    \end{center}
    }
    
 \item \lang{de}{
    Eine Zahl  $\,a\,$ ist größer als eine Zahl  $\,b$, wenn  $\,a\,$ auf der Zahlengeraden 
    weiter rechts liegt als $\,b$. Man schreibt
    \begin{center}
        $a > b \quad $ (bzw. $\;a \geq b$, wenn $\,a\,$ größer oder gleich $\,b\,$ ist).
    \end{center}
    }
    \lang{en}{
    A number $\,a\,$ is greater than a number $\,b$ if $\,a\,$ is to the right of $\,b$ on the 
    number line. We write
    \begin{center}
        $a > b \quad $ (or $\;a \geq b$ if $\,a\,$ is greater than or equal to $\,b\,$).
    \end{center}
    }
\end{enumerate}
\end{definition}

\lang{de}{
Jede Zahl (natürlich, ganz, rational oder reell) hat ein \emph{Vorzeichen} und einen \emph{Betrag}. 
Eine Zahl, die auf dem rechten Abschnitt der Zahlengeraden liegt, hat ein \emph{positives Vorzeichen} 
(das Vorzeichen „$+$“). Eine Zahl auf der linken Seite der Zahlengeraden hat ein 
\emph{negatives Vorzeichen} (das Vorzeichen „$-$“). Die Null ist ihre eigene Gegenzahl, d.\,h.$\, 0=-0\,$
und das Vorzeichen kann frei gewählt werden.
}
\lang{en}{
Every real number has a \emph{sign} and a \emph{magnitude}. 
Numbers to the right of zero on the number line have a \emph{positive sign} ($+$), and numbers to the left 
of zero on the number line have a \emph{negative sign} ($-$). Zero is the exception, and can be written with 
either sign ($\, 0=-0\,$).
}

\begin{definition}[\lang{de}{Betrag}\lang{en}{Magnitude}] \label{def:betrag}
\lang{de}{
Der \notion{\emph{Betrag}} einer Zahl $a$ (wir schreiben $|a|$) ist der Abstand von $\,a\,$ zur Null
auf der Zahlengeraden. In einer Formel ausgedrückt bedeutet das:
}
\lang{en}{
The \notion{\emph{magnitude}} of a real number $a$, also known as its \notion{\emph{absolute value}}, is written as 
$|a|$ and is the distance from $a$ to zero on the number line. Formally, we define:
}
 \[
      {\abs{a}} = 
      \begin{cases}
        a  & \text{falls} \;  a \geq 0,\\
        -a & \text{falls} \;  a < 0.
      \end{cases}
 \]
\end{definition}

\begin{example} \label{ex:betrag}
\lang{de}{
Mithilfe der folgenden Applikation können Sie das Verhalten einer Zahl $a$ und ihres Betrags $|a|$ 
auf der Zahlengeraden beobachten:
}
\lang{en}{
Using the following application we can see the behaviour of a number $a$ and its absolute value $|a|$ 
on the number line.
}

\begin{genericGWTVisualization}[600][600]{gwtviz}
      
      \text{
      \lang{de}{W\"ahlen Sie eine Zahl $a$ zwischen $-10$ und $10$. Beobachten Sie $a$
      und ihren Betrag $|a|$ auf der Zahlengeraden.}
      \lang{en}{Choose a number $a$ between $-10$ and $10$, and observe $a$ and its absolute 
      value $|a|$ on the number line.}
      }
      
        \begin{canvas}
          \plotSize{500,80}
          \plotLeft{-11}
          \plotRight{11}
          \plot[numberLine,noToolbar]{p1,p2}
          \slider{slider}
        \end{canvas}
      
      \text{\IFELSE{var(n) >= 0}{
      \lang{de}{Die Zahl $a$ is positiv oder Null. Somit ist $a=|a|$ und auf der 
      Zahlengeraden durch einen grünen Punkt dargestellt.}
      \lang{en}{$a$ is either positive or zero. Hence $a=|a|$ is represented by a single green point 
      on the number line.}
      }{
      \lang{de}{Die Zahl $a$ ist negativ. Somit ist  
      $a = -|a|$. Die Zahl $a = \var{a}$ ist durch einen orangenen Punkt auf der negativen Halbachse 
      dargestellt. Der Betrag $|a|=\var{abs}$ ist auf der positiven Halbachse als blauer Punkt 
      dargestellt.}
      \lang{en}{$a$ is negative. Hence $a = -|a|$. The orange point represents $a = \var{a}$ and the 
      blue point represents $|a|=\var{abs}$.}
      }} 
      
      
      \begin{variables} 
        \number[editable]{n}{real}{-1}
        \number{a}{real}{var(n)}
        \slider{slider}{n}{-10}{10}
        \function{f}{real}{|var(n)|}
        \number{abs}{real}{abs(var(n))}
        \point{p1}{real}{var(n),0}
        \point{p2}{real}{var(abs),0}
        \circle{p3}{real}{var(p2),0.5}
      \end{variables}

      \label{n}{$a= $}

      \color{p1}{#CC6600}
      \color{p2}{#0066CC}

    \end{genericGWTVisualization}
\end{example} 

%%%%%%%%%%%%%%%%%%%%%%%%%%%%%%%%%%%%%%%%%%%%%%%%%%%%%%%%%%%%%%%%%%%%%%%%%%%%%%%%%%%%%%%%%%%%%%%%%

\section{\lang{de}{Intervalle}\lang{en}{Intervals}} \label{sec:intervall}

\lang{de}{
Weitere spezielle Teilmengen der reellen Zahlen sind die \emph{Intervalle}. 
% Referenz auf Kapitel 2.7 ?  \ref[intervalle][Intervalle]{section.intervals}
Sie beschreiben einen zusammenhängenden Abschnitt zwischen zwei Zahlen auf der Zahlengerade (oder $\pm\infty$). 
Dabei wird unterschieden, ob die \emph{begrenzenden Zahlen} oder auch \emph{Randpunkte} dazugehören oder nicht. 
}
\lang{en}{
\emph{Intervals} are another type of special subset of the real numbers. 
They describe the set of every real number between two numbers on the number line (or $\pm\infty$). 
A distinction is made between those intervals which include these bounding numbers and those which do not. 
So-called 'half-open' intervals include only one of these bounding numbers and exclude the other.
}

\begin{definition}[\lang{de}{Intervalle}\lang{en}{Intervals}] \label{def:intervall}
	\lang{de}{
	Es seien $a,b \in \R$ mit $a < b. \;$  Dann heißen 
  }
  \lang{en}{
  Let $a,b \in \R$ satisfy $a < b. \;$ Then we define
  }
    $\qquad(a;b):=\{\,x\in\R\,|\,a< x< b\,\}\qquad$ \notion{\lang{de}{offenes Intervall}\lang{en}{open interval}}, \\ 
    $\qquad$ (\lang{de}{$a$ und $b$ gehören nicht zum Intervall}\lang{en}{$a$ and $b$ do not belong to the interval})
      \begin{center}
      \image{T101_OpenInterval}
      \end{center}
      \\
      
    $\qquad[a;b]:=\{\,x\in\R\,|\,a\leq x\leq b\,\}\qquad$ \notion{\lang{de}{abgeschlossenes Intervall}\lang{en}{closed interval}}, \\ 
    $\qquad$ (\lang{de}{$a$ und $b$ gehören zum Intervall}\lang{en}{$a$ and $b$ belong to the interval})
      \begin{center}      
      \image{T101_ClosedInterval}
      \end{center}  
      \\
      
    $\qquad[a;b):=\{\,x\in\R\,|\,a\leq x < b\,\}\qquad$ \notion{\lang{de}{rechtsoffenes Intervall}\lang{en}{right-open interval}}, \\ 
    $\qquad$(\lang{de}{$a$ gehört zum Intervall, $b$ nicht}\lang{en}{$a$ belongs to the interval, but not $b$})
      \begin{center}
      \image{T101_RightOpenInterval}
      \end{center}  
      \\
            
    $\qquad(a;b]:=\{\,x\in\R\,|\,a < x\leq b\,\}\qquad$ \notion{\lang{de}{linksoffenes Intervall}\lang{en}{left-open interval}}. \\ 
    $\qquad$ (\lang{de}{$b$ gehört zum Intervall, $a$ nicht}\lang{en}{$b$ belongs to the interval, but not $a$})
      \begin{center}
      \image{T101_LeftOpenInterval}
      \end{center}  

%    Man bezeichnet diese Intervalle auch als \emph{endliche Intervalle}.
   
\end{definition}

\begin{remark} \label{rem:intervall}
    \begin{itemize}
    
    \item  \lang{de}{Jedes der vorstehend definierten Intervalle ist durch die reellen Zahlen $a$ und $b$ begrenzt.
           Man bezeichnet sie deshalb als \emph{beschränkte Intervalle}.}
           \lang{en}{All of the above intervals are bounded by real numbers $a$ and $b$, and are called \emph{bounded intervals}.}
     
    \item  \lang{de}{In Anlehnung daran definiert man für beliebige $a,b \in \R$}
           \lang{en}{We therefore define for any $a,b \in \R$,}
         \begin{itemize}
            \item \lang{de}{\emph{nach oben unbeschränkte Intervalle}}\lang{en}{\emph{unbounded-above intervals}}
            \[
             \begin{mtable}[\cellaligns{ccl}]
               (a;\infty)   &:=& \{\,x\in\R\,|\, x>a \,\}, \\
               [a;\infty)   &:=& \{\,x\in\R\,|\,x \geq a \,\}, \\
             \end{mtable}
            \]
            \item \lang{de}{\emph{nach unten unbeschränkte Intervalle}}\lang{en}{\emph{unbounded-below intervals}}
             \[
             \begin{mtable}[\cellaligns{ccl}]
               (-\infty;b)  &:=& \{\,x\in\R\,|\,x< b\,\}, \\
               (-\infty;b]  &:=& \{\,x\in\R\,|\,x \leq b\,\}, \\ 
             \end{mtable}             
            \]
            \item \lang{de}{\emph{das beidseitig unbeschränkte Intervall}}\lang{en}{the set of real numbers represented as an interval}
             \[
             \begin{mtable}[\cellaligns{ccl}]
               (-\infty;\infty) &:=& \R.
            \end{mtable}           
            \]
          \end{itemize}
           \lang{de}{
           Hierbei wird die fehlende \glqq{}obere Grenze\grqq{} durch $\, \infty$ (sprich: \emph{unendlich}) 
           bzw. die fehlende \glqq{}untere Grenze\grqq{} durch $\, -\infty$ (sprich: \emph{minus unendlich} oder 
           \emph{negativ unendlich}) dargestellt. Da $\infty$ und $-\infty$ keine Zahlen sind, steht dort 
           immer eine runde Klammer.
           }
           \lang{en}{
           Here we use $\, \infty$ ('\emph{infinity}') to mean that no upper bound exists, and $\, -\infty$ 
           ('\emph{minus infinity}' or '\emph{negative infinity}') to mean that no lower bound exists. As $\infty$ and $-\infty$ 
           are not numbers, we always use rounded parentheses with these.
           }
          
       \item \lang{de}{
            Speziell bei abgeschlossenen Intervallen ist neben $a<b$ auch $a=b$ zulässig: Das Intervall
            $[a;a] = \{ a \}$ ist die einpunktige Menge, die nur aus $a$ besteht.
            Ein solches Intervall nennt man auch ein \notion{ausgeartetes Intervall}, 
            wohingegen man als \notion{nicht-ausgeartete Intervalle} solche bezeichnet,
            die (wie in \ref{def:intervall} definiert) mehr als ein Element enthalten.
            }
            \lang{en}{
            Closed intervals $[a;b]$ permit $a=b$, unlike open intervals $(a;b)$ which require $a<b$. The interval $[a;a] = \{ a \}$ is a set containing 
            a single element, $a$. We call this a \notion{degenerate interval}, and we call intervals containing more than one element 
            \notion{non-degenerate intervals}.
            }
    \end{itemize}
    
% \begin{tabs*}[\initialtab{0}]
%   \tab{Anmerkung zu weiteren Notationen}
\begin{showhide}[\textbf{\lang{de}{Anmerkung zu weiteren Notationen}\lang{en}{Note about notation}}]	
  \lang{de}{
  Während die Notation zu abgeschlossenen Intervallen in der Literatur als $[a;b]$ einheitlich ist, gibt
  es für rechts-, links- bzw. offene Intervalle auch eine weitere Notation, nämlich anstelle der runden Klammer die nach außen 
  geöffnete eckige Klammer, d.\,h. das rechtsoffene Intervall von $a$ bis $b$ wird auch mit $[a;b[$ bezeichnet, das linksoffene
  mit $]a;b]$ und das offene Intervall von $a$ bis $b$ mit $]a;b[$.\\
  In diesem Kurs wird aber durchgängig die Notation mit runden Klammern verwendet.}
  \lang{en}{
  Whilst the notation for closed intervals is consistently $[a;b]$ in the literature, there is another notation sometimes used for 
  half-open and open intervals, which uses outwards-facing square brackets instead of parentheses. For example, the left-open interval 
  from $a$ to $b$ might be written as $]a;b]$, and the open interval from $a$ to $b$ as $]a;b[$.
  However, in this course we use parentheses.
  }
\end{showhide}
\end{remark}

\begin{example}
  \begin{tabs*}[\initialtab{0}]
  \tab{\lang{de}{Intervalle}\lang{en}{Intervals}}
	
		\lang{de}{Rechtsoffenes Intervall}\lang{en}{Right-open interval}
		$\qquad \big[-\frac{3}{2}\lang{de}{;}\lang{en}{,}\frac{3}{2}\big)=\big\{\,x\in\R\,\big|\,-\frac{3}{2}\leq x<\frac{3}{2}\,\big\}$
		\begin{center}
			\image{T101_IntervalExample_A}
		\end{center}
		
		\lang{de}{Linksoffenes Intervall}\lang{en}{Left-open interval}
		$\qquad (-2\lang{de}{;}\lang{en}{,}1]=\{\,x\in\R\,|\,-2< x\leq 1\,\}$
		\begin{center}
			\image{T101_IntervalExample_B}
		\end{center}
    \end{tabs*}
\end{example}

%%%
% Kurztzest zu Intervallen
%
%%%%
\begin{quickcheck}
		\type{input.interval}
        \field{rational}
				
		\text{\lang{de}{Welche der folgenden Mengen beschreibt die auf der Zahlengerade in blau dargestellte
              Teilmenge von $\R$?}
          \lang{en}{Which of the following subsets of $\R$ is shown by the blue section of the number line?}
           \begin{center}
                 \image{T101_IntervalExample_C} 
            \end{center}                   
            }      

    \begin{variables}                   
      \randint{r1}{1}{2}
    \end{variables}
    
    \begin{choices}{multiple}

       \begin{choice}
          \text{$\Big\{\,-\frac{1}{2} ; \, 3\,\Big\}$}
		\solution{false}
	\end{choice}
               
  \begin{choice}
     \text{$\bigg(-\frac{1}{2} ; \, 3\bigg]$}
	\solution{false}
\end{choice}
                
        \begin{choice}
            \text{$\Big\{\,x\in\R\,\Big|\,-\frac{1}{2}\leq x\leq 3\,\Big\}$}
			\solution{true}
		\end{choice}
        
        \begin{choice}
            \text{$\bigg[-\frac{1}{2} ;  \, 3\bigg]$}
			\solution{true}
		\end{choice}
        
        \begin{choice}
            \text{$\Big\{\,x\in\R\,\Big|\,-\frac{1}{2} < x\leq 3\,\Big\}$}
			\solution{false}
		\end{choice}
       
    \end{choices}{multiple}
           			
	\end{quickcheck}

%%%%


\end{visualizationwrapper}

\end{content}

