%$Id:  $
\documentclass{mumie.article}
%$Id$
\begin{metainfo}
  \name{
    \lang{de}{Lösen elementarer Gleichungen und Gleichungssysteme}
    \lang{en}{Solving elementary equations and systems of equations}
  }
  \begin{description} 
 This work is licensed under the Creative Commons License Attribution 4.0 International (CC-BY 4.0)   
 https://creativecommons.org/licenses/by/4.0/legalcode 

    \lang{de}{Beschreibung}
    \lang{en}{Description}
  \end{description}
  \begin{components}
    \component{generic_image}{content/rwth/HM1/images/g_img_00_video_button_schwarz-blau.meta.xml}{00_video_button_schwarz-blau}      
    \component{js_lib}{system/media/mathlets/GWTGenericVisualization.meta.xml}{mathlet1}
  \end{components}
  \begin{links}
   \link{generic_article}{content/rwth/HM1/T101neu_Elementare_Rechengrundlagen/g_art_content_01_zahlenmengen.meta.xml}{content_01_zahlenmengen}
   \link{generic_article}{content/rwth/HM1/T101neu_Elementare_Rechengrundlagen/g_art_content_02_rechengrundlagen_terme.meta.xml}{link-01-terme}  
   \link{generic_article}{content/rwth/HM1/T101neu_Elementare_Rechengrundlagen/g_art_content_03_bruchrechnung.meta.xml}{bruchrechnung}   
   \link{generic_article}{content/rwth/HM1/T101neu_Elementare_Rechengrundlagen/g_art_content_04_aussagen_aequivalenzumformungen.meta.xml}{content_04_aussagen_aequivalenzumformungen}  
   \link{generic_article}{content/rwth/HM1/T101neu_Elementare_Rechengrundlagen/g_art_content_05_loesen_gleichungen_und_lgs.meta.xml}{content_05_loesen_gleichungen_und_lgs}
   \link{generic_article}{content/rwth/HM1/T112neu_Lineare_Gleichungssysteme/g_art_content_40_lineare_gleichungssysteme.meta.xml}{lineare_gleichungssysteme}
   \link{generic_article}{content/rwth/HM1/T112neu_Lineare_Gleichungssysteme/g_art_content_41_gauss_verfahren.meta.xml}{gauss-verfahren}
  \end{links}
  \creategeneric
\end{metainfo}

\begin{content}

%
% 
% In diesem Modul sind zusammengefasst Teile aus ursprünglich
%   
%  1. T102_Einfache_Funktionen,_grundlegende_Begriffe/art_content_04_lineare_funktionen {Ticket 8973}
%  2. T102_Einfache_Funktionen,_grundlegende_Begriffe/g_art_content_07_nullstellen_quadratischer_funktionen {Ticket 8975}
%
% und das komplette Modul
%
%  3. T111_Matrizen,_lineare_Gleichungssysteme/art_content_38b_adhoc_methoden {Ticket 9378}
%
% 
% Ticket neu: 
%
\begin{block}[annotation]
	Im Ticket-System: \href{https://team.mumie.net/issues/21146}{Ticket 21146}
\end{block}

\usepackage{mumie.ombplus}
\ombchapter{1}
\ombarticle{5}
\usepackage{mumie.genericvisualization}
\begin{visualizationwrapper}

\lang{de}{\title{Lösen elementarer Gleichungen und Gleichungssysteme}}
\lang{en}{\title{Solving elementary equations and systems of equations}}
 
\begin{block}[annotation]
  In diesem Kapitel sind die Verfahren zur Lösung elementarer Gleichungen und 
  Gleichungssysteme zusammengefasst.
  Die Verfahren zur Lösung linearer und quadratischer Gleichungen wurden zuvor 
  im Rahmen der Nullstellenberechnung linearer bzw. quadratischer Funktionen beschrieben.
  Die ad hoc Methoden zur Lösung einfacher linearer Gleichungssysteme mit maximal
  zwei oder drei Gleichungen und Unbekannten wurden aus Kapitel 12 in das 
  Grundlagenkapitel vorgezogen.
\end{block}


\begin{block}[info-box]
\tableofcontents
\end{block}

%
% Motivation
%
\lang{de}{
Das Lösen von Gleichungen ist eine wichtige mathematische Fähigkeit,
die in fast allen Teilgebieten der Mathematik zur Anwendung kommt. Abhängig 
vom Typ einer Gleichung gibt es hierzu viele verschiedene Lösungsansätze 
und Verfahren, die zumeist auf der Anwendung geeigneter 
\ref[content_04_aussagen_aequivalenzumformungen][Äquivalenzumformungen]{aequivalenzumformungen} 
basieren. Ziel der Umformungen ist dabei jeweils das Isolieren der Variablen zur Bestimmung
der \emph{Lösung} der Gleichung bzw. ihrer \emph{Lösungsmenge}.
}
\lang{en}{
Solving equations is an important skillset in mathematics, used in almost every field. 
Depending on the type of equation, there are various approaches and methods to solve them, most of which 
are based on
\ref[content_04_aussagen_aequivalenzumformungen][equivalence transformations]{aequivalenzumformungen}. 
The aim of these transformations is often to isolate the variables and make the \emph{solution} or 
\emph{solution set} of the equation easier to find.
}

%
%%% Video K.M.
%
\lang{de}{
Ein wichtiger Aspekt bei der Bestimmung der Lösungsmenge einer Gleichung ist die Festlegung einer
sogenannten \emph{Grundmenge}, in der die Lösungen enthalten sein sollen. 
Dies wird in nachstehendem Video veranschaulicht:
\\
%QS  \floatright
     \center{\href{https://api.stream24.net/vod/getVideo.php?id=10962-2-10811&mode=iframe&speed=true}
    {\image[75]{00_video_button_schwarz-blau}}}\\
}
\lang{en}{
An important aspect of determining the solution set of an equation is determining a \emph{'base set'} 
which the solutions may belong to. This is often $\Z$, $\Q$ or $\R$.
}
\\
%

\lang{de}{
Im folgenden Abschnitt werden die elementaren Verfahren zur Lösung von \emph{"`linearen"'} 
und \emph{"`quadratischen Gleichungen"'} sowie einfache Methoden zur Lösung 
\emph{"`linearer Gleichungssysteme"'} eingeführt.
}
\lang{en}{
In the following section we introduce the basic methods for solving \emph{'linear equations'} and 
\emph{'quadratic equations'}, and for solving \emph{'linear systems of equations'}.
}
%
%%% Ergänzung zum Video K.M.
%
\lang{de}{
Als \emph{Grundmenge} wird hierbei, soweit nicht anders spezifiziert, 
die Menge der reellen Zahlen (bzw. \ref[content_05_loesen_gleichungen_und_lgs][n-Tupel]{def:lgs} aus reellen Zahlen)
zugrundegelegt. 
}
\lang{en}{
Unless otherwise specified, our \emph{base set} will be the set of real numbers $\R$ (or the set of 
\ref[content_05_loesen_gleichungen_und_lgs][n-tuples]{def:lgs} of real numbers, $\R^n$).
}
%

\section{\lang{de}{Lösen linearer Gleichungen}\lang{en}{Linear equations}}\label{sec:linear}

\begin{definition}[\lang{de}{Lineare Gleichung}\lang{en}{Linear equation}] \label{def:lin_gleichung}
\lang{de}{
Eine Gleichung der Form $bx+c=0$, wobei $b$ und $c$ reelle Zahlen mit $b \neq 0$ sind und
$x$ die gesuchte reelle \emph{Unbekannte} ist, heißt \emph{\notion{lineare Gleichung (in $x$)}.}
\\
% Die Begriffe \emph{"`Variable"'} und  \emph{"`Unbekannte"'}  werden hier synonym verwendet.
%
Die \emph{Unbekannte} $x$ wird synonym auch als \emph{Variable} bezeichnet. Die konstanten reellen
Zahlen $b$ und $c$ nennt man \emph{Koeffizienten}. 
}
\lang{en}{
An equation of the form $bx+c=0$, where $b$ and $c$ are real numbers with $b \neq 0$ and $x$ is the 
\emph{indeterminate}, is called a \notion{\emph{linear equation (in $x$)}}.
\\
The \emph{indeterminate} $x$ is also called a \emph{variable}. The constant real numbers $b$ and $c$ 
are called \emph{coefficients}.
}
\end{definition}

\lang{de}{
Generell bezeichnet man konstante Faktoren vor einer \emph{"`Unbekannten"' } $x$,
wie auch vor Potenzen dieser Unbekannten (also $x^0(=1)$, $x^1 (=x)$,
$x^2, \ldots\,$), als \emph{"`Koeffizienten"' }.
}
\lang{en}{
Constant factors in front of an \emph{'indeterminate'} $x$ or any its powers $x^0(=1)$, $x^1 (=x)$, 
$x^2, \ldots\,$ are always called \emph{'coefficients'}.
}

\begin{algorithm}[\lang{de}{Lösung linearer Gleichungen}
                  \lang{en}{Solving linear equations}] \label{alg:lin_gleichung}
  \lang{de}{
  Jede lineare Gleichung lässt sich durch elementare
  \ref[content_04_aussagen_aequivalenzumformungen][Äquivalenzumformungen]{aequivalenzumformungen} 
  nach der Variablen $x$ auflösen:
  }
  \lang{en}{
  Every linear equation can be rearranged for the variable $x$ using basic 
  \ref[content_04_aussagen_aequivalenzumformungen][equivalence transformations]{aequivalenzumformungen};
  }

  \begin{align*}
  &&\quad	                bx+c \,&=\,& 0 \quad &\vert -c \\
  &\Leftrightarrow &\quad  bx     &=\,& -c \quad &\vert \lang{de}{:}\lang{en}{/}\lang{fr}{/}b \; (\neq 0)\\
  &\Leftrightarrow &\quad  x      &=\,& - \frac{c}{b}&\\
  \end{align*}

  \lang{de}{
  Im letzten Schritt ist die Variable $x$ isoliert und wir können sehen, 
  dass eine lineare Gleichung in $x$ immer die \notion{eindeutige} Lösung 
  $x=- \frac{c}{b}$ besitzt. Die Lösungsmenge der Gleichung stellt sich daher wie folgt dar
  }
  \lang{en}{
  In the final step, the variable $x$ is isolated and we see that a linear equation in $x$ always has 
  the \emph{unique} solution $x=- \frac{c}{b}$. The solution set of the equation is hence 
  }
  \begin{align*}
  \mathbb{L}= \{x \in \R \mid -\frac{c}{b} \}= \{ -\frac{c}{b} \}. \\
  \end{align*}
\end{algorithm}

\begin{example}
\lang{de}{
Wir schauen uns noch einmal die Gleichung $\;5(3x-5)+2x	= 7(3-2x)+16\;$ aus
dem \ref[content_04_aussagen_aequivalenzumformungen][Kapitel zu Äquivalenzumformungen]{proposition.example.3}
an. Auch diese Gleichung bezeichnet man als \emph{linear}, denn sie lässt sich durch
Äquivalenzumformungen in eine lineare Gleichung gemäß Definition \ref{def:lin_gleichung}
überführen:
}
\lang{en}{
We once more look at the equation $\;5(3x-5)+2x	= 7(3-2x)+16\;$ from the 
\ref[content_04_aussagen_aequivalenzumformungen][chapter on equivalence transformations]{proposition.example.3}. 
This equation is also considered to be \emph{linear}, as it can be manipulated using equivalence 
transformations into the form given by definition \ref{def:lin_gleichung}:
}

\begin{align*}
&&\quad					5(3x-5)+2x	&=7(3-2x)+16		&		&&\\
&\Leftrightarrow&\quad 	15x-25+2x	&=21-14x+16			&		&&\\
&\Leftrightarrow&\quad 	17x-25		&=37-14x			&		&\vert +14x&\\
&\Leftrightarrow&\quad 	31x-25		&=37				& 		&\vert -37&\\
&\Leftrightarrow&\quad 	31x	-62		&=0				    &		&&
\end{align*}

\lang{de}{
Die Lösungsmenge ist dann 
}
\lang{en}{
The solution set is then 
}
$\quad \mathbb{L}= \{ \frac{62}{31} \} = \{ 2 \}.$

\end{example}
%
%%% Video K.M. *** NEU: 11272 ((Un)Gleichungen2_neu) ***
%
\lang{de}{
Eine allgemeiner gehaltene Beschreibung zur Lösung linearer Gleichungen finden Sie
in folgendem Video. Hier wird zusätzlich untersucht, welche Lösungsmenge man erhält,
wenn der Koeffizient vor der Variablen $x$ Null ist oder wenn die \emph{Grundmenge}
%(hier auch \emph{Grundgesamtheit} genannt) 
eingeschränkt wird.
\\
\floatright{\href{https://api.stream24.net/vod/getVideo.php?id=10962-2-11272&mode=iframe&speed=true}
{\image[75]{00_video_button_schwarz-blau}}}\\
\\
}
\lang{en}{
If the coefficient $b$ of the variable $x$ is zero in a linear equation, the equation becomes $c=0$, 
which is no longer dependent on $x$. It is either true for all $x$ if $c$ really is zero, or false for 
all $x$ if $c$ was some non-zero number. That is, the solution set is $\mathbb{L} = \R$ if $c=0$, and 
$\mathbb{L} = \emptyset$ if $c\neq0$.
}
\\
%

\lang{de}{
Das Charakteristische an einer \emph{linearen Gleichung} ist, dass $1$ die 
höchste Potenz der Variablen $x$ ist. Eine Gleichung der Form $ax^2+bx=c$  
mit $a, b, c \in \R$ und $a \neq 0$ ist demnach nicht linear,
da die höchste Potenz der Variablen $x$ hier $2$ ist. Eine solche Gleichung 
bezeichnet man als \emph{quadratische Gleichung}. 
}
\lang{en}{
An identifying feature of a \emph{linear equation} is that $1$ is the highest power of the variable 
$x$. An equation of the form $ax^2+bx=c$ with $a, b, c \in \R$ and $a \neq 0$ is not linear, as the 
highest power of $x$ here is $2$. This is called a \emph{quadratic equation}.
}

\section{\lang{de}{Lösen quadratischer Gleichungen}\lang{en}{Quadratic equations}}\label{sec:quadratic}

%
%%% Video K.M. *** NEU: 11273 ((Un-)Gleichungen_3_a) ***
% 
%
%  \floatright{\href{https://api.stream24.net/vod/getVideo.php?id=10962-2-11273&mode=iframe&speed=true}
%  {\image[75]{00_video_button_schwarz-blau}}}\\
%  \\
%  \\
%


\begin{definition}[\lang{de}{Quadratische Gleichung und Normalform}
                   \lang{en}{Quadratic equation and leading coefficient}]\label{quadratic.definition.2}

    \lang{de}{
    Eine Gleichung in $x$ der Form \[ax^2+bx+c=0\] mit den reellen Koeffizienten $a (\neq 0)$, 
    $b$ und $c$, heißt \notion{\emph{quadratische Gleichung}}.
    \\
    Speziell bezeichnet man quadratische Gleichungen der Form \[x^2+px+q=0\]
    mit $p,q \in \mathbb{R}$ als \notion{\emph{quadratische Gleichung in Normalform}}.
    Dabei sind die Bezeichnungen $p$ und $q$ als Koeffizienten für die Normalform natürlich 
    willkürlich, haben sich aber im deutschen und englischen Sprachraum durchgesetzt.
    }

    \lang{en}{
    An equation in $x$ of the form \[ax^2+bx+c=0\] with real coefficients $a (\neq 0)$, 
    $b$ and $c$ is called a \notion{\emph{quadratic equation}}.
    \\
    Quadratic equations of the form \[x^2+px+q=0\] with $p,q \in \mathbb{R}$ are called 
    \notion{\emph{monic quadratic equations}}, or more descriptively, \notion{\emph{quadratic 
    equations with leading coefficient $1$}}, as the coefficient of the highest power of $x$, that is 
    $x^2$, is $1$. The letters $p$ and $q$ are of course arbitrary, but are often used to denote 
    coefficients of a quadratic equation in this form.
    }

\end{definition} 

\lang{de}{
Eine quadratische Gleichung in Normalform ist in der Regel einfacher zu lösen 
als eine allgemeine quadratische Gleichung. Wie das nachfolgende Verfahren 
zeigt, lässt sich jedoch jede quadratische Gleichung äquivalent in eine Normalform 
überführen.
}
\lang{en}{
A quadratic equation with leading coefficient $1$ is, as a rule, easier to solve than a general 
quadratic equation. Having said this, it is shown in the following method how every quadratic equation 
can be manipulated to have leading coefficient $1$.
}

\begin{algorithm}[\lang{de}{Überführung quadratischer Gleichungen in Normalform}
                  \lang{en}{Getting a leading coefficient $1$}] \label{alg:normalform}
  \lang{de}{
  Nach Division durch $\,a\,$ erhält man für jede quadratische Gleichung $\,ax^2+bx+c=0\,$ die quadratische 
  Gleichung $\,x^2+\frac{b}{a}x+\frac{c}{a}=0\,$ in Normalform mit $\, p=\frac{b}{a} \,$ und $\, q=\frac{c}{a}.\;$
  Da $a \neq 0\,$ ist, handelt es sich hierbei um eine 
  \ref[content_04_aussagen_aequivalenzumformungen][Äquivalenzumformung]{sec:aequivalenz}, also hat die Normalform 
  die gleiche Lösungsmenge wie die ursprüngliche quadratische Gleichung.
  }
  \lang{en}{
  Any quadratic equation $\,ax^2+bx+c=0\,$ can simply be divided by $\,a\,$ on both sides to obtain 
  $\,x^2+\frac{b}{a}x+\frac{c}{a}=0\,$, which has leading coefficient $1$. As $a \neq 0\,$, this is 
  an \ref[content_04_aussagen_aequivalenzumformungen][equivalence transformation]{sec:aequivalenz}, so 
  the new equation has the same solution set as the original.
  }
\end{algorithm}

\begin{example}\label{ex:normalform}
\textit{\lang{de}{Allgemeine Form und Normalform}\lang{en}{Changing the leading coefficient to $1$}}
              %
              % Das Beispiel ist nicht mehr erforderlich, nach der vorstehenden Bemerkung
              % in Kombination mit dem anschließenden Quickcheck.
              %
              % \begin{align*}
              % 2x^2-26x-44&=&0 \quad \vert :2\\
              % \Leftrightarrow \quad x^2-13x-22&=&0
              % \end{align*}
              %
              % Die quadratischen Gleichung $2x^2-26x-44=0$ hat also dieselbe Lösungsmenge 
              % wie die Normalform $x^2-13x-22=0$. Folglich genügt es, zur Bestimmung der Lösungsmenge 
              % die Normalform zu lösen.
              % \end{example}

\begin{quickcheckcontainer}
\randomquickcheckpool{1}{1}
\begin{quickcheck}
		\field{rational}
		\type{input.number}
		\begin{variables}
			\randint[Z]{a}{-3}{3}
			\randint[Z]{b}{1}{4}
			\randint{c}{-4}{4}
		    \function[normalize]{f}{a*x^2+b*x+c}
			\function[calculate]{p}{b/a}
			\function[calculate]{q}{c/a}
		\end{variables}
		
			\text{\lang{de}{Bestimmen Sie die Normalform der quadratischen Gleichung $\var{f}=0$.}
            \lang{en}{Transform the quadratic equation $\var{f}=0$ into its monic form.} \\
			 $x^2 +($\ansref $)x +($\ansref $)= 0$}
		
		\begin{answer}
			\solution{p}
		\end{answer}
		\begin{answer}
			\solution{q}
		\end{answer}
	\end{quickcheck}
\end{quickcheckcontainer}
\end{example}


\lang{de}{
Wir werden nun die folgenden drei Methoden zur Lösung quadratischer 
Gleichungen behandeln:
}
\lang{en}{
We will now introduce three methods for solving quadratic equations.
}

\begin{enumerate}
    \item \lang{de}{die \ref[content_05_loesen_gleichungen_und_lgs][quadratische Ergänzung]{alg:quadr_erg},}
          \lang{en}{\ref[content_05_loesen_gleichungen_und_lgs][Completing the square]{alg:quadr_erg},}
    \item \lang{de}{die \ref[content_05_loesen_gleichungen_und_lgs][p-q-Formel]{rule:pqFormel},}
          \lang{en}{\ref[content_05_loesen_gleichungen_und_lgs][The p-q formula]{alg:quadr_erg},}
    \item \lang{de}{den \ref[content_05_loesen_gleichungen_und_lgs][Satz von Viëta]{thm:Satz_von_Vieta}.}
          \lang{en}{\ref[content_05_loesen_gleichungen_und_lgs][Vieta's formulas]{thm:Satz_von_Vieta}.}
\end{enumerate}

\lang{de}{
Wir können uns dabei, aufgrund des Verfahrens \ref{alg:normalform}, ohne 
Einschränkung der Allgemeingültigkeit auf die Betrachtung \emph{quadratischer 
Gleichungen in Normalform} beschränken.
}
\lang{en}{
We can, thanks to algorithm \ref{alg:normalform}, suppose that every quadratic equation is 
\emph{monic} before any of the following algorithms are applied.
}

%%%%%%%%%%%%%%%%%%%%%%%%%%%%%%%%%%%%%%%%%%%%%%%%%%%%%%%%%%%%%%%%%

\begin{algorithm}[\lang{de}{quadratische Ergänzung}\lang{en}{Completing the square}] \label{alg:quadr_erg}

  \lang{de}{
  Gegeben sei eine quadratische Gleichung in Normalform $x^2+px+q=0$ mit $p, q \in \R$.
  \\
  Die Summe $\;x^2+px\;$ kann durch Addition einer Konstanten zu einem vollständigen Quadrat
  ergänzt werden. Anschließend kann die 
  \ref[link-01-terme][erste binomische Formel]{rule:binomische_formeln} angewandt werden:
  }
  \lang{en}{
  Consider the monic quadratic equation $x^2+px+q=0$ with $p, q \in \R$.
  \\
  The sum $\;x^2+px\;$ can be fully expressed as an (expanded) square by adding a constant. Specifically, 
  it is the \ref[link-01-terme][square of a binomial]{rule:binomische_formeln}:
  }
  
    \begin{align*}
         &\; x^2+px &\\
        =&\; \underbrace{x^2+2 \frac{p}{2}x+ \Big{(}\frac{p}{2} \Big{)}^2} %_{=}  
            &-\Big{(}\frac{p}{2} \Big{)}^2\\
        =&\; \phantom{x^2+}\Big{(}x+\frac{p}{2}\Big{)}^2 &-\Big{(}\frac{p}{2} \Big{)}^2
    \end{align*}

  \lang{de}{
  Dieses Verfahren nennt man \emph{\notion{quadratische Ergänzung}}. Es wird 
  wie folgt zur äquivalenten Umformung von quadratischen Gleichungen verwendet:
  }
  \lang{en}{
  This procedure is called \notion{\emph{completing the square}}, as it involves adding a constant to 
  turn an expression into a square. It is applied to quadratic equations as an equivalence 
  transformation:
  }

  \begin{align*}
  &\quad x^2+px+q &=&\; 0 &\qquad \vert -q\\
   \Leftrightarrow &\quad x^2+px &=&\; -q \quad &\qquad \vert +\Big{(}\frac{p}{2} \Big{)}^2\\
   \Leftrightarrow &\quad \underbrace{x^2+2 \frac{p}{2}x+ \Big{(}\frac{p}{2} \Big{)}^2} %_{=} 
                            &=&\; -q+\Big{(}\frac{p}{2} \Big{)}^2&\\ %\qquad \vert \, \text{1.\, binomische \, Formel}\\
   \Leftrightarrow &\quad \phantom{x^2+}\Big{(}x+\frac{p}{2}\Big{)}^2 &=&\; -q+\Big{(}\frac{p}{2} \Big{)}^2 &
  \end{align*}

   \lang{de}{
   Die letzte Gleichung in dieser Umformung wird dann als die \emph{\notion{quadratisch ergänzte
   Form }} $\,$ von $\; x^2+px+q=0\,$ bezeichnet. Sie ermöglicht eine Auflösung 
   nach der Unbekannten $x$, sofern die rechte Seite der Gleichung größer oder gleich Null ist. 
   Man unterscheidet daher die folgenden drei Fälle:
   }
   \lang{en}{
   The final equation is called the \notion{\emph{completed square form}} of $\; x^2+px+q=0\,$. Provided 
   the right hand side is greater than or equal to zero, it lets us solve the equation for $x\in\R$. We 
   distinguish between the folloving three cases:
   }
                                                        
 \\	\begin{table}[\class{layout} \cellvaligns{mm}]
		& \colspan{2} 1. $\; $ \lang{de}{Falls}\lang{en}{If} 
    $\,-q+\Big{(}\frac{p}{2} \Big{)}^2 < 0 \;$ \lang{de}{gilt}\lang{en}{holds}, \\
        & \phantom{ 1. $\; $ \lang{de}{Falls}\lang{en}{If}} & 
        \lang{de}{ist die Gleichung nicht lösbar, da der quadratische Term auf der linken 
                      Seite der Gleichung für kein reelles $x$ negativ sein kann.}
        \lang{en}{the equation is not solvable, as the square on the left side of the equation 
        cannot be negative for any real $x$.} \\
				
		& \colspan{2} 2. $\; $ \lang{de}{Falls}\lang{en}{If} 
    $\,-q+\Big{(}\frac{p}{2} \Big{)}^2 = 0 \;$ \lang{de}{gilt}\lang{en}{holds}, \\
        & \phantom{ 2. $\; $ \lang{de}{Falls}\lang{en}{If}} & 
        \lang{de}{gibt es genau eine Lösung, nämlich $x=-\frac{p}{2}\,$.} 
        \lang{en}{there is precisely one solution, namely $x=-\frac{p}{2}\,$.} \\
		
    & \colspan{2} 3. $\; $ \lang{de}{Falls}\lang{en}{If} 
    $\,-q+\Big{(}\frac{p}{2} \Big{)}^2 > 0 \;$ \lang{de}{gilt}\lang{en}{holds}, \\
        & \phantom{ 3. $\; $ \lang{de}{Falls}\lang{en}{If}} & 
        \lang{de}{kann auf beiden Seiten der Gleichung die Wurzel gezogen werden. In diesem
                      Fall gibt es zwei Lösungen für $x$.}
        \lang{en}{there are precisely two solutions, as both the positive and the negative square root of 
        the right hand side can be considered.}

\begin{showhide}[\buttonlabels{\lang{de}{Zeige Erklärung}\lang{en}{Show explanation}}
                              {\lang{de}{Verstecke Erklärung}\lang{en}{Hide explanation}}]

\lang{de}{
Überträgt man die \ref[link-01-terme][Definition der Wurzel]{def:wurzel} auf Terme, so gilt
}
\lang{en}{
Applying the \ref[link-01-terme][definition of a square root]{def:wurzel} to the left hand side gives:
}
\[
\displaystyle{\sqrt{\Big{(}x+\frac{p}{2}\Big{)}^2}=\abs{x+\frac{p}{2}},}
\]
\lang{de}{
denn wegen der Variablen $x$ in dem Term können wir nur durch den Betrag sicherstellen, dass 
das Ergebnis der Wurzel positiv ist.
Gemäß der \ref[content_01_zahlenmengen][Definition von Beträgen]{def:betrag} ist
}
\lang{en}{
as due to the variable $x$ in the epression, the square root may otherwise be positive or negative.
The \ref[content_01_zahlenmengen][definition of the absolute value]{def:betrag} gives
}

\[\abs{x+\frac{p}{2}}\ =
    \begin{cases}
        \Big{(}x+\frac{p}{2}\Big{)} & \text{falls} \;  \Big{(}x+\frac{p}{2}\Big{)} \geq 0\\
        -\Big{(}x+\frac{p}{2}\Big{)} & \text{falls} \;   \Big{(}x+\frac{p}{2}\Big{)} < 0.
    \end{cases}
\]
\lang{de}{
Dies führt zu zwei Lösungen für unsere Gleichung.
}
\lang{en}{
This provides two solutions to our equation.
}
%, und zwar eine mit $\, x < -\frac{p}{2}\,$  
% und die zweite mit $\, x > -\frac{p}{2}\,$ (da in diesem Fall $\,x \neq -\frac{p}{2}\,$).
\end{showhide}                     
                      
	    \begin{align*}
        &\quad \Big{(}x+\frac{p}{2}\Big{)}^2 \;=\; -q+\Big{(}\frac{p}{2} \Big{)}^2 &\quad \vert \sqrt{\ldots}\\
        & &\\ 
       
        \Leftrightarrow &\quad x +\frac{p}{2} =  + \sqrt{\Big{(}\frac{p}{2} \Big{)}^2-q} \quad \text{oder}
                    &\quad x +\frac{p}{2} = - \sqrt{\Big{(}\frac{p}{2} \Big{)}^2-q}
        \end{align*}
    
	\end{table}



%
%       Dies gilt nur unter Beachtung des Vorzeichens
%
%         \begin{align*}
%          \Leftrightarrow &\quad  x_{1,2} +\frac{p}{2}\; = \; \pm{} \sqrt{\Big{(}\frac{p}{2} \Big{)}^2-q} &\quad \vert - \frac{p}{2} \\
%          \Leftrightarrow &\quad  x_1 \; = \; -\frac{p}{2} + \sqrt{\Big{(}\frac{p}{2} \Big{)}^2 -q}\quad  \text{und} 
%                           \quad  x_2 \; = \; -\frac{p}{2} - \sqrt{\Big{(}\frac{p}{2} \Big{)}^2 -q}&
%
%         \end{align*}
%
\end{algorithm}  


\begin{example}\label{ex:quadr_erg}
    \lang{de}{
    Wir lösen die quadratische Gleichung (in Normalform) $x^2+8x+7=0$ mittels \emph{quadratischer Ergänzung:}
    }
    \lang{en}{
    We solve the monic quadratic equation $x^2+8x+7=0$ by \emph{completing the square}:
    }
	\begin{align*}
		&					&\quad	x^2+8x+7			&\, =\,0					&	&			&\\
		&\Leftrightarrow	&\quad	x^2+8x+4^2-4^2+7	&\, =\,0					&	&			&\\
		&\Leftrightarrow	&\quad	(x+4)^2-16+7		&\, =\,0					&	&			&\\
		&\Leftrightarrow	&\quad	(x+4)^2-9			&\, =\,0					&	&\quad \vert +9	&\\
		&\Leftrightarrow	&\quad	(x+4)^2				&\, =\,9					&	&\quad \vert \sqrt{\ldots} &\\
		&\Leftrightarrow    &\quad	x+4					&\, =\,{}\pm{}\sqrt{9}	&	&			&\\
		&\Leftrightarrow	&\quad	x =\,-4+3=-1 		&\,	\text{ oder }
		&            		 \quad	x			    	\, =\,-4-3=-7				&			&\\
	\end{align*}
    \lang{de}{
    Es gibt also zwei Lösungen und die Lösungsmenge ist
    }
    \lang{en}{
    Hence there are two solutions, and the solution set is
    }
  	\begin{align*}
        &\quad	\mathbb{L} 			&\, =\,\{-1;\;-7\}.		&	&			&
	\end{align*}
\end{example}

\begin{quickcheck}
		\field{rational}
		\type{input.number}
		\begin{variables}
			\number{a}{1}			
			\randint{b}{-4}{4}
			\randint{c}{-4}{4}
		    \function[normalize]{f}{a*x^2+b*x+c}
			\function[calculate]{d}{b/(2*a)}
			\function[calculate]{h}{-c+b^2/(4*a^2)}
			\function[calculate]{anz}{sign(h)+1}
		\end{variables}
		
			\text{\lang{de}{Bestimmen Sie für die Gleichung $\var{f}=0$ die quadratisch ergänzte Form:\\
		        $(x+$\ansref$)^2=$\ansref.\\ 
		        Die Gleichung besitzt damit \ansref reelle Lösung(en).}
            \lang{en}{Determine the monic form of the equation $\var{f}=0$:\\
            $(x+$\ansref$)^2=$\ansref.\\
            The equation then has \ansref distinct real solutions.
            }}
		       
		\begin{answer}
			\solution{d}
		\end{answer}
		\begin{answer}
			\solution{h}
		\end{answer}
		\begin{answer}
			\solution{anz}
		\end{answer}
	\end{quickcheck}

\lang{de}{
Die Auflösung der \emph{quadratisch ergänzten Form$\,$}  
$\; \Big{(}x+\frac{p}{2}\Big{)}^2 = -q+\Big{(}\frac{p}{2} \Big{)}^2 \;$ 
nach $x$ gemäß dem in \ref{alg:quadr_erg} beschriebenen
Verfahren liefert die folgende allgemeine Lösungsformel für quadratische 
Gleichungen in Normalform.
}
\lang{en}{
Solving the \emph{monic quadratic equation} 
$\; \Big{(}x+\frac{p}{2}\Big{)}^2 = -q+\Big{(}\frac{p}{2} \Big{)}^2 \;$ 
for $x$ as in algorithm \ref{alg:quadr_erg} immediately gives us the following 
formula for solving quadratic equations.
}

% \begin{align*}   
%                  &\quad  \Big{(}x+\frac{p}{2}\Big{)}^2 = \Big{(}\frac{p}{2} \Big{)}^2 -q  & \\
%  \Leftrightarrow &\quad  x_{1,2} +\frac{p}{2}\; = \; \pm{} \sqrt{\Big{(}\frac{p}{2} \Big{)}^2-q} &\quad \vert - \frac{p}{2} \\
%  \Leftrightarrow &\quad  x_{1,2}  \; = \; -\frac{p}{2} \; \pm{}  \sqrt{\Big{(}\frac{p}{2} \Big{)}^2 -q}\quad  & 
% \end{align*}

%%%%%%%%%%%%%%%%%%%%%%%%%%%%%%%%%%%%%%%%%%%%%%%%%%%%%%%%%%%%%%%%%
\begin{rule}[\lang{de}{p-q-Formel}\lang{en}{The p-q formula}] \label{rule:pqFormel}

    \lang{de}{
    Die Lösungen einer quadratischen Gleichung $\;x^2+px+q=0\;$ in Normalform sind gegeben durch
    die sogenannte \emph{\notion{p-q-Formel}}
    }
    \lang{en}{
    The solutions of a monic quadratic equation $\;x^2+px+q=0\;$ are given by the so-called 
    \notion{\emph{p-q formula}}
    }
    \[x_{1,2}=-\frac{p}{2}{}\pm{}\sqrt{\Big{(}\frac{p}{2}\Big{)}^2-q}.\]

    \lang{de}{
    Dabei gibt der Term unter der Wurzel, die sogenannte \emph{\notion{Diskriminante}}
    }
    \lang{en}{
    The expression under the square root, called the \notion{\emph{discriminant}}
    }
    \[D:=\Big{(}\frac{p}{2}\Big{)}^2-q, \]
    \lang{de}{
    Aufschluss über die Anzahl der Lösungen der quadratischen Gleichung. 
    Man unterscheidet, analog zu \ref{alg:quadr_erg}, die folgenden drei Fälle:
    }
    \lang{en}{
    gives information about the number of real solutions of the quadratic equation.
    As in \ref{alg:quadr_erg}, a distinction is made between the following three cases:
    }
    
    
    \begin{align*}
    1. \quad D &\,<\,& 0 \quad \Rightarrow \quad &\text{
    \lang{de}{Die quadratische Gleichung hat keine Lösung, also}
    \lang{en}{The quadratic equation has no real solutions, so}}\\
    &&& \mathbb{L}= \emptyset .  \\         
    2. \quad D &\,=\,& 0 \quad \Rightarrow \quad &\text{
    \lang{de}{Die quadratische Gleichung hat eine Lösung, nämlich}
    \lang{en}{The quadratic equation has a single real solution,}}\\
    &&& \mathbb{L}=\{-\frac{p}{2}\} \\  
    3. \quad D &\,>\,& 0 \quad \Rightarrow \quad &\text{
    \lang{de}{Die quadratische Gleichung hat zwei Lösungen,}
    \lang{en}{The quadratic equation has two real solutions,}}\\
    &&& \mathbb{L}=\{-\frac{p}{2}-\sqrt{\Big{(}\frac{p}{2}\Big{)}^2-q}; \: -\frac{p}{2}+\sqrt{\Big{(}\frac{p}{2}\Big{)}^2-q}\} \\  
    \end{align*}

\end{rule}  

%
%%% Video K.M. *** NEU: 11273 ((Un-)Gleichungen_3_a) ***
%
\lang{de}{
Das folgende Video beschreibt noch einmal die Herleitung der p-q-Formel und ihre Verwendung zur Lösung
quadratischer Gleichungen.
\\
\floatright{\href{https://api.stream24.net/vod/getVideo.php?id=10962-2-11273&mode=iframe&speed=true}
{\image[75]{00_video_button_schwarz-blau}}}\\\\
}
\lang{en}{
\\
}
%


\begin{example} \label{ex:pqFormel}
    \lang{de}{
    Wir berechnen die Lösung der quadratischen Gleichung $\; x^2-6x+5=0 \;$ mithilfe
    der \emph{p-q-Formel}
    }
    \lang{en}{
    We calculate the solution of the quadratic equation $\; x^2-6x+5=0 \;$ using the 
    \emph{p-q formula}
    }
    \begin{align*}
    x_{1,2}&\;=&\; 3{}\pm{}\sqrt{9-5}\\
    \Rightarrow \quad x_1&\;=&\; 3-2=1 \; \text{\lang{de}{und}\lang{en}{and}} \; x_2=3+2=5    
    \end{align*}

    \lang{de}{Die Lösungsmenge ist}\lang{en}{The solution set is} $\quad \mathbb{L} =\{1;\;5\}.$
\end{example}

\begin{quickcheckcontainer}
\randomquickcheckpool{1}{1}
\begin{quickcheck}
		\field{rational}
		\type{input.number}
		\begin{variables}
			\randint[Z]{a}{-3}{3}
			\randint[Z]{b}{1}{4}
			\randint{c}{-4}{4}
		    \function[normalize]{f}{a*x^2+b*x+c}
			\function[calculate]{p}{b/a}
			\function[calculate]{q}{c/a}
			\function[calculate]{l1}{-p/2}
			\function[calculate]{l2}{(p/2)^2}
			\function[calculate]{l3}{q}			
		\end{variables}
		
			\text{\lang{de}{Die p-q-Formel für Lösungen der quadratischen Gleichung $\var{f}=0$ lautet}
            \lang{en}{The p-q formula for solving the quadratic equation $\var{f}=0$ is}\\
			 $x_{1,2}=$\ansref $\pm \sqrt{}($\ansref $-$\ansref $)$.}
		
		\begin{answer}
			\solution{l1}
		\end{answer}
		\begin{answer}
			\solution{l2}
		\end{answer}
		\begin{answer}
			\solution{l3}
		\end{answer}

	\end{quickcheck}
\end{quickcheckcontainer}

\begin{remark} \label{rem:mitternachtsformel}
\begin{itemize}
  \item \lang{de}{In der Mathematik wird häufig auch der Term $p^2-4q$ als \textit{Diskriminante} in  
    der p-q-Formel bezeichnet. Der Grund dafür ist eine andere Schreibweise der p-q-Formel,
    in der ein Faktor $\frac{1}{2}$ aus der Gleichung ausgeklammert wurde:
    }
    \lang{en}{
    In other literature, $p^2-4q$ is often called the discriminant, due to the 
    existence of a different (but equivalent) form of the p-q formula in which a 
    factor $\frac{1}{2}$ is taken out of the expression:
    }
    \begin{align*}
    x_{1,2}= \frac{1}{2}\Big{(}-p \pm \sqrt{p^2-4q}\Big{)}.
    \end{align*}

  \item \lang{de}{
    Für allgemeine quadratische Gleichungen der Form $\,ax^2+bx+c=0\,$ (mit $a \neq 0$)
    gibt es eine ähnliche Lösungsformel, die sogenannte \emph{\notion{Mitternachtsformel}}
    }
    \lang{en}{
    For a general quadratic equation of the form $\,ax^2+bx+c=0\,$ (with $a \neq 0$), 
    there is a more general version of the p-q formula, called the 
    \notion{\emph{quadratic formula}},
    }
    \[
    x_{1,2}= \frac{-b \pm \sqrt{b^2-4ac}}{2a}.
    \]
    \begin{showhide}[\buttonlabels{
    \lang{de}{Zeige weitere Details zur Mitternachtsformel}
    \lang{en}{Show further information about the quadratic formula}}{
    \lang{de}{Verstecke weitere Details zur  Mitternachtsformel}
    \lang{en}{Hide further information about the quadratic formula}}]
    \lang{de}{
    Ebenso wie bei der \emph{p-q-Formel} gibt auch in der \emph{Mitternachtsformel}
    die \emph{Diskriminante}, also der Term unter der Wurzel, Aufschluss über die Anzahl 
    der Lösungen der quadratischen Gleichung.
    }
    \lang{en}{
    Just like the \emph{p-q formula}, the \emph{quadratic formula} also has a 
    \emph{discriminant} - the expression under the square root sign that determines 
    the number of solutions to the quadratic equation.
    }

      \begin{enumerate}
        \item \lang{de}{
              Für $\,b^2-4ac<0 \,$  besitzt die quadratische Gleichung keine Lösung, 
                                    also $\mathbb{L}= \emptyset$.
              }
              \lang{en}{
              If $\,b^2-4ac<0 \,$   the quadratic equation has no real solutions, so 
                                    $\mathbb{L}= \emptyset$.
              }
        \item \lang{de}{
              Für $\,b^2-4ac=0 \,$  besitzt die quadratische Gleichung eine Lösung, 
                                    nämlich $\mathbb{L}=\{-\frac{b}{2a}\}$.
              }
              \lang{en}{
              If $\,b^2-4ac=0 \,$   the quadratic equation has one real solution, so 
                                    $\mathbb{L}=\{-\frac{b}{2a}\}$.
              }
        \item \lang{de}{
              Für $\,b^2-4ac>0 \,$  besitzt die quadratische Gleichung zwei Lösungen:
              $\mathbb{L}=\{\frac{-b - \sqrt{b^2-4ac}}{2a}; \frac{-b + \sqrt{b^2-4ac}}{2a}\}$.
              }
              \lang{en}{
              If $\,b^2-4ac>0 \,$ the quadratic equation has two real solutions, so 
              $\mathbb{L}=\{\frac{-b - \sqrt{b^2-4ac}}{2a}; \frac{-b + \sqrt{b^2-4ac}}{2a}\}$.
              }
      \end{enumerate}

    \end{showhide}

  \item \lang{de}{
        Generell gilt, dass eine quadratische Gleichung maximal 
        zwei reelle Lösungen hat.
        }
        \lang{en}{
        In general, any quadratic equation has a maximum of two real solutions.
        }
\end{itemize}
\end{remark}

%%%%%%%%
\lang{de}{
Weitere Möglichkeiten zur Lösung quadratischer Gleichungen (in Normalform)
liefert der \emph{\notion{Satz von Viëta}}, der auf der sogenannten \emph{Linearfaktorzerlegung} 
 basiert. Als \emph{Linearfaktor} bezeichnet man dabei
einen Faktor der Form $(x-a)$, wobei $a$ eine reelle Zahl ist.
}
\lang{en}{
A further method for solving (monic) quadratic equations is derived from 
\notion{\emph{Vieta's formulas}}, which come from finding the linear factors of 
a quadratic equation. A \emph{linear factor} in this case is defined to be a factor 
of the form $(x-a)$, where $a$ is a real number.
}

\begin{theorem}[\lang{de}{Satz von Viëta / Linearfaktorzerlegung}
                \lang{en}{Vieta's formulas / linear factors}] \label{thm:Satz_von_Vieta}
  \lang{de}{
  Es sei $x^2+px+q=0$ eine quadratische Gleichung in Normalform und es
  seien $x_1$ und $x_2$ die Lösungen dieser Gleichung. Dann gilt
  }
  \lang{en}{
  Let $x^2+px+q=0$ be a monic quadratic equation with solutions $x_1$ and $x_2$. 
  Then
  }
  \begin{align*}
      x^2+px+q = (x-x_1)\cdot(x-x_2),
  \end{align*}
  \lang{de}{
  d.\,h. die Normalform $x^2+px+q$ ist zerlegbar in die \emph{Linearfaktoren} $(x-x_1)$
  und $(x-x_2)$.
  \\
  Zwischen den Koeffizienten $p$ und $q$ und den Lösungen $x_1$ und $x_2$ gelten dabei
  die folgenden Beziehungen:
  }
  \lang{en}{
  so the equation $x^2+px+q$ can be decomposed into its \emph{linear factors} 
  $(x-x_1)$ and $(x-x_2)$.
  \\
  The following relationships exist between the coefficients $p$ and $q$ and the 
  solutions $x_1$ and $x_2$:
  }

    \begin{align*}
      &p=-(x_1+x_2),\\
      &q=x_1 \cdot x_2.
    \end{align*}

  \lang{de}{
  Hat die quadratische Gleichung nur eine Lösung, so setzt man $x_1=x_2$.
  }
  \lang{en}{
  If the quadratic equation only has one solution, we simply set $x_1=x_2$.
  }
\end{theorem}

% Dabei bezeichnet man $(x-x_1)\cdot(x-x_2)$ als die \emph{\notion{Linearfaktorzerlegung}} von $x^2+px+q$.


\begin{proof*}[\lang{de}{Beweis des Satzes von Viëta}\lang{en}{Proof of Vieta's formulas}]
\begin{showhide}
\lang{de}{
Multipliziert man das Produkt der Linearfaktoren $(x-x_1)\cdot(x-x_2)\,$ aus, so erhält 
man eine quadratische Gleichung in Normalform mit dem Koeffizienten $\, -(x_1+x_2)\,$ vor
$\,x\,$ und dem konstanten Summanden $\, x_1 \cdot x_2$, der formal dem Koeffizienten vor
$\, x^0(=1) \,$ entspricht:
}
\lang{en}{
Multiply out the product $(x-x_1)\cdot(x-x_2)\,$ to obtain a quadratic equation with 
an $x$-coefficient of $\, -(x_1+x_2)\,$ and a constant of $\, x_1 \cdot x_2$, which 
is formally the coefficient of $\, x^0(=1) \,$.
}
\begin{align*}
    (x-x_1)(x-x_2)\, &= x^2 &\,- x_1 x -x x_2 &\,+ x_1 x_2  \\
                     &= x^2 &\,+ \underbrace{(-x_1-x_2)}_{=p}x &\,+ \underbrace{x_1 x_2}_{=q}\\
                     &= x^2 &\,+\phantom{(-x_1} px &\,+\phantom{x_1}q
\end{align*}

\lang{de}{
Die Formeln von Viëta ergeben sich nun aus dem sogenannten \emph{\notion{Koeffizientenvergleich}},
der besagt, dass zwei quadratische Gleichungen identisch sind, wenn sie jeweils gleiche 
Koeffizienten vor $x^2$, $x^1$ und $x^0$ haben. In diesem Fall gilt also $-(x_1+x_2)=p$ und $x_1 \cdot x_2=q$.
}
\lang{en}{
Vieta's formulas are now obtained by the principle of 
\notion{\emph{comparing coefficients}}, which states that two quadratic equations 
are identical if they have the same coefficients of $x^2$, $x^1$ and $x^0$. In this 
case, we have $-(x_1+x_2)=p$ and $x_1 \cdot x_2=q$.
}

\end{showhide}
\end{proof*}

\lang{de}{
Dieser Satz hat im Wesentlichen drei wichtige Anwendungen, von denen zwei zur Lösung
quadratischer Gleichungen in Normalform genutzt werden können:
}
\lang{en}{
This statement has three important applications, of which two are helpful for solving 
monic quadratic equations:
}
\begin{enumerate}

  \item \lang{de}{
  Ist eine der beiden Lösungen von $x^2+px+q=0\;$ (z.\,B. $x_1$) 
  schon bekannt, so lässt sich die zweite Lösung bequem mittels der Formeln von Viëta 
  berechnen, z.\,B. durch $\,x_2=-p-x_1\,$ oder, falls $x_1\ne 0$, auch durch $x_2=q/x_1$.
  }
  \lang{en}{
  If we know one of the solutions of $x^2+px+q=0\;$ (say $x_1$), the second 
  solution can easily be found using one of Vieta's formulas. For example we have
  $\,x_2=-p-x_1\,$, or if $x_1\ne 0$, we have $x_2=q/x_1$.
  }

  \item \lang{de}{
  Vermutet man, dass die Lösungen einer Gleichung $x^2+px+q=0$ mit ganzen Zahlen $p$ und $q$ auch 
  ganzzahlig sind, so kann man die Lösungen leicht raten. In diesem Fall müssen
  nämlich wegen $x_1 \cdot x_2=q$ beide Lösungen Teiler der Zahl $q$ sein.
  }
  \lang{en}{
  If we suspect that the solutions of a quadratic $x^2+px+q=0$ with integer $p$ and 
  $q$ are integers themselves, then we may be able to guess them. This is because 
  $x_1 \cdot x_2=q$ so both solutions have to divide $q$, and then their sum can 
  be checked to see if $-(x_1+x_2)=p$.
  }

  \item \lang{de}{
  Sind die beiden Lösungen $x_1$ und $x_2$ bekannt (weil man sie z.\,B. mit
  der p-q-Formel schon berechnet hat), so kann man den Term $x^2+px+q$ 
  durch den Term $(x-x_1)(x-x_2)$ ersetzen, was in größeren Ausdrücken möglicherweise weitere Termvereinfachungen 
  zulässt, etwa beim \ref[bruchrechnung][Rechnen mit Bruchtermen]{add}.
  }
  \lang{en}{
  If two solutions $x_1$ and $x_2$ are known (we may have found them with the 
  p-q formula or quadratic equation), then the expression $x^2+px+q$ can be replaced 
  by the expression $(x-x_1)(x-x_2)$. This form of the quadratic may be useful in 
  longer expressions, where it could for example lead to a 
  \ref[bruchrechnung][fraction being simplified]{add}.
  }
%
%%% Video K.M. *** NEU: 11275 (Un)Gleichungen_4a_Faktorisierung ***
% 
\lang{de}{
  Man spricht in diesem Zusammenhang auch von einer \emph{Faktorisierung} der quadratischen Gleichung bzw.
  ihrer \emph{Zerlegung in Linearfaktoren}.
\\
  \floatright{\href{https://api.stream24.net/vod/getVideo.php?id=10962-2-11275&mode=iframe&speed=true}
  {\image[75]{00_video_button_schwarz-blau}}}\\
}
\lang{en}{
We call this a \emph{factorisation} or a \emph{decomposition into linear factors}.
}
  \\
  \\
%

\end{enumerate}


\begin{example}\label{ex:vieta_2lsg}
\lang{de}{
Wir lösen die Gleichung $x^2 - x - 6 = 0$ mit Hilfe des Satz von Viëta.
Hierzu lesen wir zunächst $\,p\,$ und $\,q\,$ aus der Gleichung ab und erhalten
}
\lang{en}{
We solve the equation $x^2 - x - 6 = 0$ using Vieta's formulas. 
Firstly we read $\,p\,$ and $\,q\,$ from the equation and obtain
}

  \begin{align*}
    &p=-1 &\; \text{\lang{de}{und}\lang{en}{and}}\\
    &q=-6. &
  \end{align*}

\lang{de}{
Wir suchen also Werte für $x_1$ und $x_2$, für die gilt:
}
\lang{en}{
We need values for $x_1$ and $x_2$ that satisfy:
}

  \begin{align*}
    &p=-(x_1 + x_2) &\; = -1    &\; \text{\lang{de}{und}\lang{en}{and}}\\
    &q= x_1 \cdot x_2 &\; = -6.  &
  \end{align*}

\lang{de}{
Da $\,q=-6<0\,$ ist, erkennen wir, dass entweder $\,x_1\,$ oder $\,x_2\,$ negativ sein muss.
Durch Ausprobieren finden wir heraus, dass
}
\lang{en}{
As $\,q=-6<0\,$, we recognise that either $\,x_1\,$ or $\,x_2\,$ must be negative. 
By trial and error, we find that
}

  \begin{align*}
    &3 + (-2) &\; = 1        &\; \text{\lang{de}{und}\lang{en}{and}}\\
    &3 \cdot (-2) &\; = -6.  &
  \end{align*}

\lang{de}{
Die ermittelten Werte sind daher $x_1=3$ und $x_2=-2$, folglich ist die 
Lösungsmenge der quadratischen Gleichung
}
\lang{en}{
satisfy Vieta's formulas. The solutions are therefore $x_1=3$ und $x_2=-2$, so the solution set is
}
\[\mathbb{L} =\{-2;3\}.\]

 \begin{showhide}[\buttonlabels{\lang{de}{Zeige Rückrechnung}
                                \lang{en}{Show reverse calculation}}{
                                \lang{de}{Verstecke Rückrechnung}
                                \lang{en}{Hide reverse calculation}}]
  \lang{de}{
  Ausgehend von bereits bekannten Lösungen $x_1=3$ und $x_2=-2$, lässt sich hieraus mittels 
  der Formeln von Viëta auch die quadratische Gleichung wieder
  bestimmen.
  \\
  Rückrechnung:
  }
  \lang{en}{
  Starting with only the known solutions $x_1=3$ and $x_2=-2$, Vieta's formulas can 
  be used to reconstruct the original quadratic equation:
  }
  \begin{align*}
      &p =-(x_1 + x_2) =-(3 -2) = -1,\\
      &q = x_1 \cdot x_2 = 3\cdot (-2) = -6.
  \end{align*}
  \lang{de}{
  Einsetzen in die Normalform ergibt
  }
  \lang{en}{
  Substituting this $p$ and $q$ into the general monic form of a quadratic equation 
  yields
  }
  
  \[x^2 - x - 6 = 0.\]
 \end{showhide}
\end{example}


\begin{example}\label{ex:vieta_1lsg}

\lang{de}{
Es sei $\quad x^2+4x+4=0$. Wir haben also
}
\lang{en}{
Let $\quad x^2+4x+4=0$. Hence
}

  \begin{align*}
    p = 4,\\
    q = 4.
  \end{align*}

\lang{de}{
Da $q$ positiv ist, müssen $x_1$ und $x_2$ beide negativ oder beide positiv sein.
\\
Durch Ausprobieren erhalten wir
}
\lang{en}{
As $q$ is positive, $x_1$ and $x_2$ must either both be negative or both be positive. 
\\
Through trial and error we obtain
}

  \begin{align*}
    p = -((-2)+(-2)) = 4,\\
    q =  (-2)\cdot (-2) = 4.
  \end{align*}

\lang{de}{
Es gilt hier $x_1=x_2=-2$. Die Gleichung $x^2+4x+4=0$ hat also nur eine Lösung,
}
\lang{en}{
Thus $x_1=x_2=-2$. The equation $x^2+4x+4=0$ hence has only one solution,
}

\[\mathbb{L} =\{-2\}.\]

 \begin{showhide}[\buttonlabels{\lang{de}{Zeige Rückrechnung}
                                \lang{en}{Show reverse calculation}}{
                                \lang{de}{Verstecke Rückrechnung}
                                \lang{en}{Hide reverse calculation}}]
\lang{de}{
Die Lösungen sind
}
\lang{en}{
The solutions are
}
  \begin{align*}
  x_1 = -2 \quad \text{\lang{de}{und}\lang{en}{and}} \quad x_2 = -2.\\
  \end{align*}
\lang{de}{
Mit den Formeln von Viëta kriegen wir
}
\lang{en}{
Using Vieta's formulas we obtain
}
  \begin{align*}
        p &\; =-(x_1 + x_2) = -((-2)+(-2)) = 4, \\
  q &= x_1 \cdot x_2 = (-2)\cdot(-2) = 4.
  \end{align*}
  \lang{de}{
  Einsetzen in die Normalform ergibt
  }
  \lang{en}{
  Substituting these into the general form of a monic quadratic equation yields
  }
  \begin{align*}
   x^2+4x+4 = 0.
  \end{align*}
 \end{showhide}
\end{example}



\begin{example}\label{ex:vieta_0lsg}

\lang{de}{
Es sei $x^2-2x+3=0$, d.\,h.
}
\lang{en}{
Let $x^2-2x+3=0$, so
}

\begin{align*}
  p = -2,\\
  q = 3.
\end{align*}

\lang{de}{
Durch Ausprobieren lassen sich keine Lösungen für $x_1$ und $x_2$ finden, die 
als Summe $\,2\,$ und als Produkt $\,3\,$ ergeben. Da die Vorgehensweise des Ausprobierens
zum Auffinden nicht-ganzzahliger Lösungen für $x_1$ und $x_2$ nicht geeignet ist, verwenden wir
für die weiteren Untersuchungen die p-q-Formel:
}
\lang{en}{
In this case, trial and error does not yield any integer solutions $x_1$ und $x_2$ 
that sum to $\,2\,$ and have product $\,3\,$. As this is unsuccessful in finding 
integers, we apply the p-q formula:
}

\[
    x_{1,2} = -1{}\pm{}\sqrt{1-3}.
\]

\lang{de}{
Die Diskriminante ist $-2$, also $<0$, also gibt tatsächlich keine Lösung für diese quadratische Gleichung.
Es gilt daher
}
\lang{en}{
The discriminant is $-2$, a negative number, so there really is no solution to this 
quadratic equation. Hence
}
\[\mathbb{L} =\emptyset.\]

\end{example}

	\begin{quickcheck}
		\field{rational}
		\type{input.number}
		\begin{variables}
			\randint[Z]{x1}{-3}{3}
			\randint[Z]{x2}{-5}{5}
		    \function[expand, normalize]{f}{(x-x1)*(x-x2)}
			\function[calculate]{nx1}{-x1}
			\function[calculate]{nx2}{-x2}
		\end{variables}
		
			\text{\lang{de}{
      Bestimmen Sie die ganzzahligen Nullstellen der Funktion $f(x)=\var{f}$
			durch Raten. Geben Sie dabei die kleinere zuerst an: $x_1=$\ansref und $x_2=$\ansref.\\
			Die Linearfaktorzerlegung von $\var{f}$ ist damit:\\
			$\var{f}=(x+$\ansref$)(x+$\ansref$)$.			
			}
      \lang{en}{
      Determine the integer $x$-values for the function $f(x)=\var{f}$ by guessing. 
      Enter the smaller one first: $x_1=$\ansref und $x_2=$\ansref.\\
      The factorisation of $\var{f}$ is therefore:\\
      $\var{f}=(x+$\ansref$)(x+$\ansref$)$.
      }}
		       
		\begin{answer}
			\solution{x1}
		\end{answer}
		\begin{answer}
			\solution{x2}
		\end{answer}
		\begin{answer}
			\solution{nx1}
		\end{answer}
		\begin{answer}
			\solution{nx2}
		\end{answer}
	\end{quickcheck}
%\end{supplement}

%%%%%%%%%%%%%%%%%%%%%%%%%%%%%%%%%%%%%%%%%%%%%%%%%%%%%%%%%%%%%%%%%%%%%%%%%%%%%%%%%%%%%%%%%%%%%%


\section{\lang{de}{Lösen linearer Gleichungssysteme}\lang{en}{Systems of linear equations}}\label{sec:lgs}

\lang{de}{
Häufig erfordern mathematische Probleme die \emph{gleichzeitige} Lösung \emph{mehrerer} 
linearer Gleichungen mit \emph{mehreren} Variablen, also Gleichungen der Form
}
\lang{en}{
Often mathematical problems require solutions that \emph{simultaneously} satisfy 
several linear equations with \emph{multiple} variables, of the form 
}
\[a_1 x_1 + a_2 x_2+ \ldots + a_n x_n=b\]
\lang{de}{
mit reellen Koeffizienten $a_1, \ldots, a_n$, %alle $\neq 0$  
in denen jede der Variablen $x_1, \ldots, x_n$ die Potenz $1$ hat.
}
\lang{en}{
with real coefficients $a_1, \ldots, a_n$ and where every variable 
$x_1, \ldots, x_n$ has power $1$.
}

\begin{definition}[\lang{de}{Lineares Gleichungssystem}
                   \lang{en}{Linear systems}] \label{def:lgs}
\lang{de}{
Ein System von linearen Gleichungen, die alle zugleich
erfüllt sein sollen, wird \notion{\emph{lineares Gleichungssystem (LGS)}} genannt.
\\
Eine \emph{Lösung} eines linearen Gleichungssystems mit $n$ Unbekannten ist ein
$n$-Tupel $(x_1; x_2; \ldots; x_n)$ bestehend aus $n$ reellen Zahlen $x_1$, 
$x_2$, $\ldots$, $x_n$, die \notion{alle} Gleichungen des Systems erfüllen.
}
\lang{en}{
A collection of linear equations that must all be satisfied at once by a solution 
is called a \notion{\emph{system of linear equations}}, often shortened to 
\notion{\emph{linear system}}.
\\
A \emph{solution} to a system of linear equations with $n$ invariants is an $n$-tuple 
$(x_1; x_2; \ldots; x_n)$ consisting of $n$ real numbers $x_1$, $x_2$, $\ldots$, 
$x_n$ that satisfy $all$ equations in the system.
}
\end{definition}

% Wie bereits in Definition \ref{def:lin_gleichung} erwähnt, werden auch hier die Begriffe "Variable"
% und "{Unbekannte}" synonym verwendet.
       
  \begin{tabs*}[\initialtab{0}] 
  \tab{\lang{de}{Erl\"auterung zum Begriff $n$-Tupel}
       \lang{en}{Definition of an $n$-tuple}}
    \lang{de}{
    Zwei Zahlen $(x_1; x_2)$ bezeichnet man als Paar, \\
    drei Zahlen $(x_1; x_2; x_3)$ als Tripel, \\
    vier Zahlen $(x_1; x_2; x_3; x_4)$ als Quadrupel, \\
    fünf Zahlen $(x_1; x_2; x_3; x_4; x_5)$ als Quintupel \mbox{usw.} \\
    Ist die Anzahl der Zahlen durch eine natürliche Zahl $n$ gegeben, 
    die nicht konkret bekannt ist, so spricht man bei $(x_1; x_2; \ldots; x_n)$ 
    von einem $n$\textit{-Tupel}.
    \\
    Dabei ist zu beachten, dass bei einem $n$-Tupel, im Gegensatz zu einer 
    $n$-elementigen Menge, die \emph{Reihenfolge der Zahlen} eine Rolle spielt. 
    So sind zum Beispiel die Mengen $\,\{1;5;4\}\,$ und $\,\{5;1;4\}\,$ gleich,
    die Tripel $\,(1;5;4)\,$ und $\,(5;1;4)\,$ sind jedoch verschieden.
    }
    \lang{en}{
    A $2$-tuple $(x_1; x_2)$ is called a double or a pair, \\
    A $3$-tuple $(x_1; x_2; x_3)$ is called a treble or a triplet, \\
    A $4$-tuple $(x_1; x_2; x_3; x_4)$ is called a quad or a quartet, \\
    A $5$-tuple $(x_1; x_2; x_3; x_4; x_5)$ is called a pentuplet, \mbox{etc.} \\
    Of course an $n$-tuple is $(x_1; x_2; x_3;... ;x_{n-1}; x_n)$, and this term is 
    also used whenever the number of entries is unknown.
    \\
    Attention must be paid to the distinction between an $n$-tuple and a set 
    containing $n$ elements. The \emph{order} of the entries of a tuple matters - 
    the triplets $\,(1;5;4)\,$ and $\,(5;1;4)\,$ are distinct, as their elements 
    are in different orders.
    }
  \end{tabs*}
\lang{de}{
Bei kleineren LGS heißen die Variablen häufig auch z.\,B. $x$, $y$, $z$, $\ldots \,$ 
statt $\,x_1$, $x_2$, $x_3$, $\ldots$. In dem Fall erfolgt die Angabe der Lösung 
als $n$-Tupel in alphabetischer Reihenfolge.
\\
Üblich ist im Allgemeinen auch, dass die Gleichungen in einem LGS nummeriert sind, 
so wie in den nachfolgenden Beispielen. Dies dient der besseren Referenzierbarkeit
auf einzelne Gleichungen.
}
\lang{en}{
In smaller linear systems, the variables are often called $x$, $y$, $z$, $\ldots \,$ 
instead of $\,x_1$, $x_2$, $x_3$, $\ldots$. In that case a tuple representing 
variable values does so in alphabetical order.
\\
It is typical for the equations in a system to be numbered, as they are in the 
following example. This is done so that any of the equations may be unambiguously 
referred to.
}

\begin{example} \label{ex:bsp_lgs_1}
%\textbf{Beispiel 1}
%

  \begin{displaymath}
  \begin{mtable}[\cellaligns{crcrcr}]
  \text{(I)}&\quad  3x&-&4y&=&2\\
  \text{(II)}&\quad 2x&+&3y&=&7
  \end{mtable}
  \end{displaymath}
\lang{de}{
Dies ist ein lineares Gleichungssystem mit zwei Gleichungen und zwei Unbekannten 
$x$ und $y$. Das Zahlenpaar $(x;y)=(2; 1)$ erfüllt sowohl die Gleichung \textbf{(I)} 
als auch die Gleichung \textbf{(II)} und ist somit eine Lösung für dieses LGS.
}
\lang{en}{
This is a linear system with two equations and two variables $x$ and $y$. The tuple 
$(x;y)=(2; 1)$ satisfies both equation \textbf{(I)} and equation \textbf{(II)}, thus 
is a solution to this system.
}


%  Wir suchen zunächst eine Lösung der Gleichung \textbf{(I)}
%  und lösen diese hierzu gemäß dem Verfahren \ref{alg:lin_gleichung} nach $x$ auf.
%  \begin{align*}
%      &&\quad					3x-4y	&=2		                    &		&\vert +4y&\\
%      &\Leftrightarrow&\quad 	3x 	    &=2 + 4y		            &		&\vert :3&\\
%      &\Leftrightarrow&\quad 	x		&=\frac{2}{3}+\frac{4}{3} y	&		&&
%    \end{align*}
%  Setzen wir nun für $y$ zum Beispiel den Wert $1$ ein, so erhalten wir $x=2$. 

\end{example}


\begin{example}\label{ex:bsp_lgs_2}
%\textbf{Beispiel 2}
%
\begin{displaymath}
\begin{mtable}[\cellaligns{ccrcrcrcr}]
\text{(I)}&&x_{1}&+&2x_{2}&+&x_{3}&=&4\\
\text{(II)}&&x_{1}&-&x_{2}&+&\frac{3}{2}x_{3}&=&-7\\
\text{(III)}&\qquad-&4x_{1}&+&2x_{2}&&&=&-2
\end{mtable}
\end{displaymath}
\lang{de}{
Es handelt sich um ein lineares Gleichungssystem mit drei Gleichungen und drei Unbekannten.
Das Zahlentripel $(x_1;x_2;x_3)=(2; 3; -4)$ ist eine Lösung des linearen Gleichungssystems, 
da für $\,x_{1}=2$, $x_{2}=3$ und $x_{3}=-4$ alle drei Gleichungen erfüllt sind.
}
\lang{en}{
This is a linear system with three equations and three variables. The tuple 
$(x_1;x_2;x_3)=(2; 3; -4)$ is a solution of the linear system, as all three 
equations are satisfied if we set $\,x_{1}=2$, $x_{2}=3$ and $x_{3}=-4$.
}
\end{example}

\begin{remark} \label{rem:lgs}
\lang{de}{
Ein lineares Gleichungssystem hat 
% ähnlich wie eine lineare Gleichung (s. \ref{alg:lin_gleichung}), 
entweder \emph{genau eine Lösung},
wie in den beiden vorstehenden Beispielen \ref{ex:bsp_lgs_1} und \ref{ex:bsp_lgs_2},
oder es besitzt \emph{keine} oder \emph{unendlich viele} Lösungen, wie das Beispiel \ref{ex:nicht-eindeutig-loesbar} 
oder auch das folgende Video zeigt. 
%
%%% Video K.M.
% 
\center{\href{https://api.stream24.net/vod/getVideo.php?id=10962-2-10840&mode=iframe&speed=true}
{\image[75]{00_video_button_schwarz-blau}}}\\
%
Eine mathematische Erklärung hierzu ist in dem vertiefenden Kapitel zu 
\link{lineare_gleichungssysteme}{Linearen Gleichungssystemen} zu finden.
}
\lang{en}{
A linear system has either \emph{precisely one solution}, like in the previous 
examples \ref{ex:bsp_lgs_1} and \ref{ex:bsp_lgs_2}, or it has \emph{no solutions}, 
or it has \emph{infinitely many solutions}, like in the following example 
\ref{ex:nicht-eindeutig-loesbar}.
\\
A mathematical explanation for this can be found in the more advanced chapter on 
\link{lineare_gleichungssysteme}{systems of linear equations}.
}
\end{remark}


\begin{example}\label{ex:nicht-eindeutig-loesbar}
\begin{tabs*}[\initialtab{0}]
%
  \tab{\lang{de}{LGS ohne Lösung}\lang{en}{Linear system with no solutions}}
  \[ \begin{mtable}[\cellaligns{crcrcr}]
  \text{(I)}&\qquad 2 \cdot  x & - & 4 \cdot  y & = & 10 \\
  \text{(II)}&-3 \cdot  x & + & 6\cdot y & =  & -10
  \end{mtable} \]
  \lang{de}{
  L"ost man die erste Gleichung nach $x$ auf, erh"alt man $x=5+2y$.
  \\
  Einsetzen in die zweite Gleichung ergibt:
  }
  \lang{en}{
  Rearranging the first equation for $x$ yields $x=5+2y$.
  \\
  Substituting this into the second equation gives:
  }
  
  \[
  \begin{mtable}[\cellaligns{crcr}]
  & -3 \cdot  (5+2y)  +  6\cdot y & =  & -10 \\
  \Leftrightarrow & \quad -15-6y +6y &=& -10\\
  \Leftrightarrow & \quad  -15 &=& -10 
  \end{mtable}\]

  \lang{de}{
  Diese Gleichung ist nie erf"ullt. Es gibt also kein Zahlenpaar $(x;y)$, das das Gleichungssystem erfüllt, d.\,h.
  }
  \lang{en}{
  This equation is never satisfied, so there does not exist a tuple $(x;y)$ that 
  satisfies both equation, that is
  }
  \[ \mathbb{L}=\{ \}. \]
%
  \tab{\lang{de}{LGS mit unendlich vielen Lösungen}
       \lang{en}{Linear systems with infinitely many solutions}}
  \[  
  \begin{mtable}[\cellaligns{crcrcr}]
  \text{(I)}&\qquad 2 \cdot  x & - & 4 \cdot  y & = & 10 \\
  \text{(II)}&-3 \cdot  x & + & 6\cdot y & =  & -15
  \end{mtable} \]
  \lang{de}{
  L"ost man die erste Gleichung nach $x$ auf, erh"alt man
  }
  \lang{en}{
  Rearranging the first equation for $x$ yields
  }
      \begin{displaymath}
           \text{(I*)}  \qquad x=5+2y.  
      \end{displaymath}

  \lang{de}{
  Einsetzen in die zweite Gleichung ergibt:
  }
  \lang{en}{
  Substituting this into the second equation gives:
  }
  \[\begin{mtable}[\cellaligns{crcr}]
  & -3 \cdot  (5+2y)  +  6\cdot y & =  & -15 \\
  \Leftrightarrow & \quad -15-6y +6y &=& -15\\
  \Leftrightarrow & \quad  -15 &=& -15 
  \end{mtable}\]

  \lang{de}{
  Diese Gleichung ist immer erf"ullt. Es bleibt also nur die Bedingung $\, x=5+2y\,$ aus 
  \textbf{(I*)}, d.\,h.
  für jede beliebige reelle Zahl $y$ ist das Paar $(5+2y;y)$ eine Lösung 
  des linearen Gleichungssystems. Die Lösungsmenge ist somit
  }
  \lang{en}{
  This equation is always satisfied. Hence the only condition needed for our $(x;y)$ 
  to satisfy both equations is $\, x=5+2y\,$ from \textbf{(I*)}. Therefore for any 
  real number $y$, the tuple $(5+2y;y)$ is a solution of the linear system. The 
  solution set is therefore 
  }
  % \[ \mathbb{L}=\left\{ \left(\begin{smallmatrix} 5+2y \\ y \end{smallmatrix}\right) \mid\, y\in \R \right\}
  % = \left\{ \begin{pmatrix}5\\ 0 \end{pmatrix}+r\cdot \begin{pmatrix} 2 \\ 1 \end{pmatrix} \mid\, r\in \R \right\} \] 

  \[ \mathbb{L}=\left\{ (5+2y;y) \mid\, y\in \R \right\}. \]

\end{tabs*}
\end{example}

\lang{de}{
Zur Bestimmung der Lösungsmenge eines linearen Gleichungssystems gibt es, wie wir in den Beispielen und dem Video
bereits erkennbar, verschiedene Ansätze und Verfahren, die wir uns nachfolgend noch einmal genauer anschauen werden.
Dabei beschränken wir uns im Rahmen dieses Kapitels zunächst auf die Einführung der folgenden elementaren 
Lösungsverfahren:
}
\lang{en}{
As we have seen in the examples, there are several approaches and methods for solving 
systems of linear equations, which we will look at more closely now. 
In this chapter the methods that we will examine are the following:
}

\begin{itemize}
    \item \lang{de}{\ref[content_05_loesen_gleichungen_und_lgs][Additionsverfahren]{alg:additionsverfahren},}
          \lang{en}{\ref[content_05_loesen_gleichungen_und_lgs][addition method]{alg:additionsverfahren},}
    \item \lang{de}{\ref[content_05_loesen_gleichungen_und_lgs][Einsetzungsverfahren]{alg:einsetzungsverfahrene},}
          \lang{en}{\ref[content_05_loesen_gleichungen_und_lgs][substitution method]{alg:einsetzungsverfahrene} and}
    \item \lang{de}{\ref[content_05_loesen_gleichungen_und_lgs][Gleichsetzungsverfahren]{alg:gleichsetzungsverfahren}.}
          \lang{en}{\ref[content_05_loesen_gleichungen_und_lgs][equality method]{alg:gleichsetzungsverfahren}.}
\end{itemize}

\lang{de}{
Bei allen drei Verfahren handelt es sich um Methoden, um das lineare 
Gleichungssystem schrittweise auf eine lineare Gleichung (mit nur einer Variablen) zu reduzieren, die dann 
gemäß \ref{alg:lin_gleichung} gelöst werden kann. Diese Verfahren eignen sich gut zur Lösung kleiner 
linearer Gleichungssysteme. Deshalb verwenden wir zur Einführung der drei Verfahren lineare Gleichungssysteme 
mit 2 Gleichungen und 2 Unbekannten, die zudem genau eine Lösung haben, und erläutern anhand eines 
Beispiels mit 3 Gleichungen und 3 Unbekannten, wie das Verfahren bei größeren LGS anzuwenden ist.
Für größere Gleichungssysteme und solche, in denen die Anzahl der Gleichungen von der Anzahl der Variablen
abweicht, gibt es allgemeinere und effektivere Lösungsverfahren, wie zum Beispiel das 
\ref[gauss-verfahren][Gauß-Verfahren]{ezu}, die später im vertiefenden Kapitel über 
Linearen Gleichungssysteme eingeführt werden.
}
\lang{en}{
Each of these three methods aims to reduce a system of linear equations into a single 
linear equation in one variable. This can then be solved using algorithm 
\ref{alg:lin_gleichung}. The methods are only practical for solving small linear 
systems, so systems with two equations, two variables and a single solution are used 
to introduce them. Examples are then given using systems with three equations and 
three variables. 
There are more general and efficient algorithms for solving larger systems of linear 
equations, and those with a different number of equations from the number of 
variables. For example, there is \ref[gauss-verfahren][Gaussian elimination]{ezu}, 
an algorithm covered in the more advanced chapter about linear systems.
}
\\

% \section{\lang{de}{Das Additionsverfahren}\lang{en}{The Addition Method (the Elimination Method)}} \label{sec:additionsverfahren}
\lang{de}{
Zur Beschreibung der Lösungsverfahren betrachten wir das LGS aus \ref{ex:bsp_lgs_1}.
}
\lang{en}{
To introduce the methods for solving linear systems, we use the system from example 
\ref{ex:bsp_lgs_1}.
}

\begin{algorithm}[\lang{de}{Additionsverfahren}
                  \lang{en}{Addition method}] \label{alg:additionsverfahren}

\begin{displaymath}
  \begin{mtable}[\cellaligns{crcrcr}]
    \text{(I)}&\qquad3x&-&4y&=&2\\
    \text{(II)}&\qquad2x&+&3y&=&7
  \end{mtable}
  \end{displaymath}

%    Die Zielsetzung bei der Anwendung des Additionsverfahrens ist also die Eliminierung einer der beiden
%    Variablen und die Reduktion um eine Gleichungen, um die zweite Variable als Lösung einer linearen
%    Gleichung zu bestimmen.

\begin{enumerate}
\item[\notion{\lang{de}{1. Schritt}\lang{en}{First step}}]
    \lang{de}{
    Zuerst wird, falls erforderlich, eine oder beide Gleichungen jeweils mit einer 
    Zahl $c\neq0\,$ so multipliziert, dass nach anschließender Addition der beiden Gleichungen 
    (mindestens) eine Variable nicht mehr vorkommt.
    \\
    In unserem Beispiel multiplizieren wir etwa die Gleichung \textbf{(I)} mit $\,-2\,$ 
    und die Gleichung \textbf{(II)} mit $\,3\,$, so dass die neuen
    Koeffizienten vor der Variablen $\,x\,$ aufaddiert $0$ ergeben.
    }
    \lang{en}{
    Firstly, if it is necessary, we multiply one or both of the equations by a number 
    $c\neq0\,$ such that subsequently summing the two equations will cancel out (at 
    least) one of the variables.
    \\
    In our example, we multiply equation \textbf{(I)} by $\,-2\,$ and equation 
    \textbf{(II)} by $\,3\,$, so that the new coefficients of the variable $x$ sum to 
    $0$.
    }
      \begin{displaymath}
      \begin{mtable}[\cellaligns{ccrcrcrl}]
      \text{(I)}&\phantom{\qquad-}&3x&-&4y&=&\phantom{-}2&\qquad|\,\cdot(-2)\\
      \text{(II)}&&2x&+&3y&=&7&\qquad|\,\cdot3
      \end{mtable}
      \end{displaymath}
      \begin{center}
      ________________________________________________
      \end{center}
      \begin{displaymath}
      \begin{mtable}[\cellaligns{ccrcrcrl}]
      \phantom{\text{(I)}}&\qquad-&6x&+&8y&=&-4&\phantom{\qquad|\,\cdot(-2)}\\
      \phantom{\text{(II)}}&&6x&+&9y&=&21&
      \end{mtable}
      \end{displaymath}
    \lang{de}{
    Die Addition der beiden Gleichungen führt nunmehr zur Elimination der 
    Variablen $x$.
    }
    \lang{en}{
    Summing the two equations now eliminates the variable $x$.
    }

\item[\notion{\lang{de}{2. Schritt}\lang{en}{Second step}}]
    \lang{de}{
    Die Addition der beiden Gleichungen erfolgt, indem jeweils die Terme rechts und die Terme 
    links des Gleichheitszeichens addiert werden. Die dadurch entstandene neue Gleichung ist, aufgrund der Elimination
    von $x$, eine \emph{lineare Gleichung} in der Variablen $y$ und wird als solche gemäß \ref{alg:lin_gleichung} gelöst:
    }
    \lang{en}{
    Summing the two equations means adding the left side of the new equation 
    \textbf{(II)} to the left side of the new equation \textbf{(I)}, and adding the 
    right side of the new equation \textbf{(II)} to the right side of the new 
    equation \textbf{(I)}. The result is a \emph{linear equation} in $y$, and will 
    be solved using algorithm \ref{alg:lin_gleichung}:
    }
    

        \begin{displaymath}
        17y=17 \quad \Leftrightarrow \quad y=1
        \end{displaymath}
      
\item[\notion{\lang{de}{3. Schritt}\lang{en}{Third step}}]
    \lang{de}{
    Das Ergebnis $y=1$ wird nun in eine beliebige der beiden Ausgangsgleichungen 
    eingesetzt, um hieraus die Variable $\,x\,$ zu bestimmen. $y=1$ eingesetzt in
    Gleichung \textbf{(I)} ergibt
    }
    \lang{en}{
    The solution $y=1$ can now be substituted into either of the two original 
    equations to find which value for $\,x\,$ they hold for. $y=1$ substituted into 
    equation \textbf{(I)} yields
    }
    
        \begin{displaymath}
        3x-4\cdot1=2 \quad \Leftrightarrow \quad x=2. 
        \end{displaymath}

        \lang{de}{
        Das Einsetzen in die Gleichung \textbf{(II)} liefert selbstverständlich dasselbe Ergebnis.        
        Somit ist das Zahlenpaar $(2; 1)$ als eindeutige Lösung des LGS bestimmt und es gilt
        }
        \lang{en}{
        Substituting it into equation \textbf{(II)} yields the same solution. Thus 
        the tuple $(2; 1)$ is the unique solution of the linear system, that is,
        }
        \begin{displaymath}
        \mathbb{L}=\{(2; 1)\}.   
        \end{displaymath}

%\end{table} 
\end{enumerate}
\end{algorithm}


\textbf{\lang{de}{Anmerkungen:}\lang{en}{Notes:}}
  \begin{itemize}

    \item \lang{de}{
        Bei der Umformung in Schritt 1 gibt es viele Möglichkeiten, um zum 2. Schritt zu gelangen.
        }
        \lang{en}{
        There are several ways to get from the first to the second step.
        }
      \begin{itemize}    
         \item \lang{de}{
             Um die Variable $x$ zu eliminieren, kann auch einfach die Gleichung 
             \textbf{(I)} mit $\,-\frac{2}{3}\,$ multipliziert werden.
             }
             \lang{en}{
             To eliminate the variable $x$ we can also multiply equation \textbf{(I)} 
             by $\,-\frac{2}{3}\,$.
             }
             \begin{showhide}
             \begin{enumerate}

             \item \lang{de}{
                 Dann erhält man als Ausgangspunkt für Schritt 2 das folgende LGS:
                 }
                 \lang{en}{
                 Then we obtain the following system for the second step:
                 }
                  \begin{displaymath}
                  \begin{mtable}[\cellaligns{ccrcrcrl}]
                    \text{(I)}&\qquad-&2x&+&\frac{8}{3}y&=&-\frac{4}{3}&\phantom{\qquad|\,\cdot(-2)}\\
                    \text{(II)}&&2x&+&3y&=&7&
                  \end{mtable}
                  \end{displaymath}
 
              \item \lang{de}{
                  Die Addition der beiden Gleichungen führt nun zu einer Gleichung,
                  in der $x$ nicht mehr vorkommt:
                  }
                  \lang{en}{
                  Summing the two equations now leads to an equation with no terms in 
                  $x$.
                  }
              \begin{displaymath}
              \frac{17}{3}y=\frac{17}{3} \quad \Leftrightarrow \quad y=1
              \end{displaymath}
 
              \item \lang{de}{
                  Die Bestimmung von $x$ erfolgt dann analog zu Schritt 3 in \ref{alg:additionsverfahren}
                  und wir erhalten wieder $\; \mathbb{L}=\{(2;{,} 1)\}.$
                  }
                  \lang{en}{
                  The value for $x$ that solves the equation can then be found 
                  anagolously to step 3 in algorithm \ref{alg:additionsverfahren}.
                  }
            \end{enumerate}
            \end{showhide}

      \item \lang{de}{
          Man kann auch die Variable $y$ statt $x$ eliminieren, indem in Schritt 1 die Gleichung 
          \textbf{(II)} mit $\frac{4}{3}$ multipliziert wird.
          }
          \lang{en}{
          We could also eliminate the variable $y$ instead of $x$, by multiplying 
          equation \textbf{(II)} by $\frac{4}{3}$ in the first step.
          }
         
        \begin{showhide}

        % \begin{example}
        % \textbf{Beispiel 4}
        %
        \begin{enumerate}
         \item \lang{de}{
          Wir starten also mit der Multiplikation der Gleichung \textbf{(II)} mit $\frac{4}{3}$:
          }
          \lang{en}{
          We begin by multiplying equation \textbf{(II)} by $\frac{4}{3}$:
          }
          \begin{displaymath}
          \begin{mtable}[\cellaligns{crcrccl}]
          \text{(I)}&\qquad3x&-&4y&=&\,\,\,2\,\,\,&\\
          \text{(II)}&\qquad2x&+&3y&=&\,\,\,7\,\,\,&\qquad|\,\cdot\frac{4}{3}
          \end{mtable}
          \end{displaymath}
          \begin{center}
          ___________________________________________
          \end{center}
          \begin{displaymath}
          \begin{mtable}[\cellaligns{crcrccl}]
          \phantom{\text{(I)}}&\qquad3x&-&4y&=&2&\phantom{\qquad|\,\cdot\frac{4}{3}}\\
          \phantom{\text{(II)}}&\frac{8}{3}x&+&4y&=&\frac{28}{3}&
          \end{mtable}
          \end{displaymath}
          \item
          \lang{de}{
          Die Addition der beiden Gleichungen f\"uhrt nun zu einer Gleichung, in der $y$ nicht mehr vorkommt:
          }
          \lang{en}{
          Summing the two equations now leads to an equation with no terms in $y$.
          }
          \begin{displaymath}
          \frac{17}{3}x=\frac{34}{3} \quad \Leftrightarrow \quad x=2
          \end{displaymath}
          \item \lang{de}{
          Wir setzen das Ergebnis in eine der Ausgangsgleichungen ein und bestimmen daraus $y$:
          }
          \lang{en}{
          We substitute the result into one of the original equations and determine 
          $y$:
          }
          \begin{displaymath}
          3\cdot2-4y=2 \quad \Leftrightarrow \quad y=1
          \end{displaymath}
          \lang{de}{
          Die Lösungsmenge ist also abermals
          }
          \lang{en}{
          The solution set is again
          }
          \begin{displaymath}
          \mathbb{L}=\{(2; 1)\}.   
          \end{displaymath}
        \end{enumerate}
        \end{showhide}
 
    \item \lang{de}{
        Sofern beide Gleichungen eines LGS zu einer Variablen gleiche Koeffizienten haben, wird
        eine der beiden Gleichungen mit $(-1)\,$ multipliziert und man startet direkt mit dem 2. Schritt.
        Dies entspricht quasi einer Subtraktion statt einer Addition im 2. Schritt.
        }
        \lang{en}{
        If two equations in a linear system have the same coefficient for one 
        variable, we may simply multiply one of the equations by $(-1)\,$ and begin 
        with the second step. This is equivalent to subtracting one equation from the 
        other.
        }
    \end{itemize}

    \item \lang{de}{
        Zur Lösung \glqq größerer\grqq linearer Gleichungssysteme mit dem Additionsverfahren erfolgt die Addition in 
        Schritt 2 so, dass \textbf{eine} ausgewählte Gleichung zu jeder der anderen Gleichungen addiert wird und 
        hierdurch in allen Gleichungen \textbf{eine} ausgewählte Variable eliminiert wird. Die Schritte 1 und 2 werden
        dann so oft wiederholt, bis das LGS auf eine lineare Gleichung mit nur noch einer Variablen reduziert ist.
        \\
        Schauen wir uns das beispielhaft an dem linearen Gleichungssystem 
        mit 3 Gleichungen und 3 Unbekannten aus Beipiel \ref{ex:bsp_lgs_2} an.
        }
        \lang{en}{
        For solving 'larger' systems of linear equations with the addition method, 
        a single equation is chosen to add to all of the other equations. If they 
        each have been multiplied by appropriate constants, a variable can be 
        eliminated from every equation but one. Repeating this (essentially the 
        first and second steps) yields new equations with progressively fewer 
        variables. Once there is an equation with only one variable, we can solve 
        this and then begin substituting back into the other equations to find the 
        values of the other variables.
        \\
        Let us look at an example linear system with three equations and three 
        variables.
        }

      \begin{example}[\lang{de}{LGS mit 3 Gleichungen und 3 Unbekannten}
                      \lang{en}{Linear system with 3 equations and 3 variables}] \label{ex:additionsverfahren_3_3}
%      \begin{tabs*}[\initialtab{0}] 
%      \tab{\lang{de}{Beispiel zur Lösung eines LGS mit 3 Gleichungen und 3 Unbekannten}\lang{en}{Example}}
      \begin{showhide}

        \begin{displaymath}
        \begin{mtable}[\cellaligns{ccrcrcrcr}]
        \text{(I)}&&x_{1}&+&2x_{2}&+&x_{3}&=&4\\
        \text{(II)}&&x_{1}&-&x_{2}&+&\frac{3}{2}x_{3}&=&-7\\
        \text{(III)}&\qquad-&4x_{1}&+&2x_{2}&&&=&-2
        \end{mtable}
        \end{displaymath}   
         \begin{enumerate}
             \item \lang{de}{
             Wir multiplizieren die Gleichungen \textbf{(I)} und \textbf{(II)} jeweils mit $4$,
             um anschließend die Gleichung \textbf{(III)} jeweils einmal zu \textbf{(I)} und zu \textbf{(II)} 
             zu addieren:
             }
             \lang{en}{
             We multiply both equations \textbf{(I)} und \textbf{(II)} by $4$ and 
             then add equation \textbf{(III)} to both \textbf{(I)} and \textbf{(II)}:
             }
             \begin{displaymath}
              \begin{mtable}[\cellaligns{ccrcrcrcr}]
              \text{(I)}&&4x_{1}&+&8x_{2}&+&4x_{3}&=&16\\
              \text{(II)}&&4x_{1}&-&4x_{2}&+&6x_{3}&=&-28\\
              \text{(III)}&\qquad-&4x_{1}&+&2x_{2}&&&=&-2
              \end{mtable}
             \end{displaymath}   
             \begin{center}
              _____________________________________________________
             \end{center}
             \begin{displaymath}
              \begin{mtable}[\cellaligns{ccrcrcrcr}]
              \text{(IV)}&=&\text{(I)}+\text{(III)}&\quad&10x_{2}&+&4x_{3}&=&14\\
              \text{(V)}&=&\text{(II)}+\text{(III)}&\quad-&2x_{2}&+&6x_{3}&=&-30
              \end{mtable}
             \end{displaymath}
             \lang{de}{
             In diesem ersten Schritt haben wir $x_1$ eliminiert und das Gleichungssystem auf
             ein LGS mit 2 Unbekannten ($x_2$ und $x_3$) und 2 Gleichungen (\textbf{(IV)} und \textbf{(V)})
             reduziert. Dieses neue LGS können wir nun gemäß dem in \ref{alg:additionsverfahren} 
             beschriebenen Verfahren lösen.
             }
             \lang{en}{
             In this first step we have eliminated $x_1$ and reduced the linear 
             system into one with two variables ($x_2$ and $x_3$) and two equations 
             (\textbf{(IV)} and \textbf{(V)}). This new linear system can now be 
             solved using algorithm \ref{alg:additionsverfahren}.
             }
             
             \item \lang{de}{
             Wir multiplizieren daher die Gleichung \textbf{(IV)} mit $-\frac{3}{2}$ und erhalten das folgende 
              kleinere LGS:
              }
              \lang{en}{
              We multiply equation \textbf{(IV)} by $-\frac{3}{2}$ and obtain the 
              smaller linear system:
              }
             \begin{displaymath}
              \begin{mtable}[\cellaligns{ccrcrcrcr}]
                &&&\qquad-&15x_{2}&-&6x_{3}&=&-21\\
                &&&\qquad-&2x_{2}&+&6x_{3}&=&-30
              \end{mtable}
             \end{displaymath}  
             
             \item \lang{de}{
             Anschließend addieren wir die beiden Gleichungen, um $x_3$ zu eliminieren
             und nach $x_2$ aufzulösen:
             }
             \lang{en}{
             Finally we sum the two equations to eliminate $x_3$ and solve for $x_2$:
             }
              \begin{displaymath}
             -17\cdot x_2=-51 \quad \Leftrightarrow \quad x_2=3
             \end{displaymath} 
             
             \item \lang{de}{
                Der Wert $x_2=3$ eingesetzt in \textbf{(IV)} oder \textbf{(V)} liefert $x_3=-4$ 
                und diese beiden Ergebnisse eingesetzt in \textbf{(I)}, \textbf{(II)} 
                oder \textbf{(III)} liefern $x_1=2$. Wir erhalten so das Zahlentripel $(2;3;-4)$
                als eindeutige Lösung des LGS und schließlich
                }
                \lang{en}{
                The value $x_2=3$ substituted into \textbf{(IV)} or \textbf{(V)} 
                yields $x_3=-4$, and substituting both of these into \textbf{(I)}, 
                \textbf{(II)} or \textbf{(III)} gives $x_1=2$. We thus obtain the 
                triplet $(2;3;-4)$ as the unique solution of the linear system.
                }
          \begin{displaymath}
          \mathbb{L}=\{(2;3;-4)\}.   
          \end{displaymath}
          
         \end{enumerate}
%        \end{tabs*}          
        \end{showhide}
        \end{example}
        
    \item \lang{de}{
    Enthält eine der Gleichungen von Anfang an nur eine Variable, kann man das LGS ohne Addition
    lösen. Beim LGS
    } % Only if it has two equations and not more - Niccolo
    \lang{en}{
    If one of the equations contains only one variable from the beginning, a linear 
    system with two equations can be solved without any further additions. For the 
    linear system 
    }
      \begin{displaymath}
      \begin{mtable}[\cellaligns{crcrcr}]
      \text{(I)}&\qquad3x&-&4y&=&2\\
      \text{(II)}&&&17y&=&17
      \end{mtable}
      \end{displaymath}
    \lang{de}{
    löst man die zweite Gleichung direkt nach $y$ auf und setzt diesen Wert anschließend 
    in die erste Gleichung ein, um den zugehörigen Wert für $x$ zu berechnen. Hierfür wird das 
    \emph{Einsetzungsverfahren} benutzt, das nun als Nächstes beschrieben wird.
    }
    \lang{en}{
    we immediately solve the second equation for $y$ and substitute this value into 
    the first equation to obtain the corresponding value for $x$.
    }
    \end{itemize}

% \section{\lang{de}{Das Einsetzungsverfahren und das Gleichsetzungsverfahren}\lang{en}{The Substitution and Equality Methods}}\label{sec:einsetzungsverfahren}
% \lang{de}{Ein weiteres Verfahren zum L\"osen eines LGS ist das \textit{Einsetzungsverfahren}. Wir gehen das Verfahren wieder schrittweise durch.}

\begin{algorithm}[\lang{de}{Einsetzungsverfahren}
                  \lang{en}{Substitution method}] \label{alg:einsetzungsverfahren}

    \begin{displaymath}
    \begin{mtable}[\cellaligns{crcrcr}]
      \text{(I)}&\qquad3x&-&4y&=&2\\
      \text{(II)}&\qquad2x&+&3y&=&7
    \end{mtable}
    \end{displaymath}

  \begin{enumerate}
   \item[\notion{\lang{de}{1. Schritt}\lang{en}{First step}}] \lang{de}{
        Zunächst wird eine der beiden Gleichungen nach einer der beiden Variablen
        aufgelöst. Wir wählen in unserem Beispiel die Gleichung \textbf{(II)}, lösen diese 
        nach $x$ auf und benennen sie zur weiteren Referenzierung mit \textbf{(II*)}.
        }
        \lang{en}{
        Firstly we rearrange one of the two equations for one of the variables. In 
        the example we choose equation \textbf{(II)} and rearrange it for $x$, 
        calling the rearranged equation \textbf{(II*)}.
        }
        \begin{displaymath}
          \text{(II*)}  \qquad x=\frac{7}{2}-\frac{3}{2}y  \qquad \qquad
        \end{displaymath}
     
    \item[\notion{\lang{de}{2. Schritt}\lang{en}{Second step}}] \lang{de}{
        Nun setzt man den aufgelösten Term in die jeweils andere 
        Gleichung ein, um diese zu lösen:
        }
        \lang{en}{
        Now we substitute the rearranged expression into the other equation, and 
        solve this for the other variable:
        }
        
        \begin{displaymath}
        3\cdot\Big(\frac{7}{2}-\frac{3}{2}y\Big)-4y=2 \quad \Leftrightarrow \quad y=1 
        \end{displaymath} 
        \lang{de}{
        \textbf{(II*)} eingesetzt in Gleichung \textbf{(I)} liefert also $y=1$.
        }
        \lang{en}{
        \textbf{(II*)} substituted into the equation \textbf{(I)} yields $y=1$.
        }
        
   \item[\notion{\lang{de}{3. Schritt}\lang{en}{Third step}}] \lang{de}{
        Der Wert der anderen Variablen wird bestimmt, indem man das Ergebnis aus dem 2. Schritt
        in die aufgelöste Gleichung aus Schritt 1 einsetzt. In unserem Beispiel setzen wir daher nun
        $y=1$ in die Gleichung \textbf{(II*)} ein und erhalten:
        }
        \lang{en}{
        The values of the other variable is determined by substituting the result 
        from the second step into the rearranged equation from the first step. In 
        this example we substitute $y=1$ into equation \textbf{(II*)} and obtain:
        }
          \begin{displaymath}
            x=\frac{7}{2}-\frac{3}{2}\cdot1=2
          \end{displaymath}
        \lang{de}{
        Die Lösungsmenge ist also  
        }
        \lang{en}{
        The solution set is therefore 
        }
        $\quad \mathbb{L}=\{(2; 1)\}.$
      
  \end{enumerate}
\end{algorithm}


\lang{de}{\textbf{Anmerkungen:}}\lang{en}{\textbf{Notes:}}
\begin{itemize}

%   Überflüssig ?
%  \item Selbstverständlich kann man in Schritt 1 auch eine der beiden Gleichungen nach
%      $y$ auflösen, um in Schritt 2 zunächst den Wert für $x$ zu bestimmen.
        
  \item \lang{de}{
        Ähnlich wie beim Additionsverfahren ist zur Lösung \glqq größerer\grqq linearer Gleichungssysteme mit dem 
        Einsetzungsverfahren eine Wiederholung der Schritte 1 und 2 so oft erforderlich, bis das LGS auf 
        eine lineare Gleichung mit nur einer Variablen reduziert ist.
        \\
        Wir wenden das \emph{Einsetzungsverfahren} nachfolgend auf das lineare Gleichungssystem aus 
        Beipiel \ref{ex:bsp_lgs_2} an.
        }
        \lang{en}{
        Similarly to the addition method, for larger systems the first and second 
        steps need to be repeated until the linear system is reduced to one with a 
        single variable.
        \\
        We now apply the \emph{substitution method} to the linear system from example 
        \ref{ex:bsp_lgs_2}.
        }
        
      \begin{example}[\lang{de}{LGS mit 3 Gleichungen und 3 Unbekannten}
                      \lang{en}{Linear system with 3 solutions and 3 variables}] \label{ex:einsetzungsverfahren_3_3}
%      \begin{tabs*}[\initialtab{0}] 
%      \tab{\lang{de}{Beispiel zur Lösung eines LGS mit 3 Gleichungen und 3 Unbekannten}\lang{en}{Example}}
      \begin{showhide}
      \begin{displaymath}
      \begin{mtable}[\cellaligns{ccrcrcrcr}]
      \text{(I)}&&x_{1}&+&2x_{2}&+&x_{3}&=&4\\
      \text{(II)}&&x_{1}&-&x_{2}&+&\frac{3}{2}x_{3}&=&-7\\
      \text{(III)}&\qquad-&4x_{1}&+&2x_{2}&&&=&-2
      \end{mtable}
      \end{displaymath}   
       \begin{enumerate}
         \item \lang{de}{
             Wir wählen die Gleichung \textbf{(I)} und lösen diese nach $x_3$ auf:
             }
             \lang{en}{
             We choose equation \textbf{(I)} to solve for $x_3$:
             }
             \begin{displaymath}
              \text{(I*)} \qquad  x_3=4-x_1-2x_2 
             \end{displaymath}   
           
          \item \lang{de}{
            Der Term auf der rechten Seite wird nun in den beiden anderen Gleichungen
            \textbf{(II)} und \textbf{(III)} für $x_3$ eingesetzt:
            }
            \lang{en}{
            The expression on the right hand side is now substituted for $x_3$ into 
            the other two equations, \textbf{(II)} and \textbf{(III)}.
            }
            \begin{displaymath}
            \begin{mtable}[\cellaligns{ccrcrcrcr}]
            \text{(II*)}&&x_{1}&-&x_{2}&+&\frac{3}{2} \cdot (4-x_1-2x_2)&=&-7\\
            \text{(III*)}&\qquad-&4x_{1}&+&2x_{2}&&&=&-2
            \end{mtable}
            \end{displaymath}   

            \lang{de}{
            Wir vereinfachen die Gleichung \textbf{(II*)} und erhalten das folgende 
            LGS mit 2 Gleichungen und den 2 Unbekannten $x_1$ und $x_2$:
            }
            \lang{en}{
            We simplify equation \textbf{(II*)} and obtain the following linear 
            system with 2 equations and the 2 unknowns $x_1$ and $x_2$:
            }
            \begin{displaymath}
            \begin{mtable}[\cellaligns{ccrcrcrcr}]
            \text{(II*)}&&x_{1}&+&8x_{2}&&&=&26\\
            \text{(III*)}&\qquad-&4x_{1}&+&2x_{2}&&&=&-2
            \end{mtable}
            \end{displaymath}   

            \lang{de}{
             Dieses neue LGS können wir nun gemäß dem in \ref{alg:einsetzungsverfahren} 
             beschriebenen Verfahren lösen.
             }
             \lang{en}{
             This new linear system can now be solved using algorithm 
             \ref{alg:einsetzungsverfahren}.
             }

           \item \lang{de}{
            Lösen wir also zum Beispiel die Gleichung \textbf{(III*)} nach $x_2$ auf:
            }
            \lang{en}{
            As an example we rearrange equation \textbf{(III*)} for $x_2$:
            }
             \begin{displaymath}
              \text{(III**)} \qquad x_2=-1 +2x_1  
             \end{displaymath}
             \lang{de}{
             Das Ergebnis setzen wir für $x_2$ in \textbf{(II*)} ein:
             }
             \lang{en}{
             The result for $x_2$ is then substituted into \textbf{(II*)}:
             }
             \begin{displaymath}
               x_{1} + 8 \cdot (-1+2x_1) = 26 \quad \Leftrightarrow \quad x_1=2
            \end{displaymath}
            \lang{de}{
            Wir erhalten damit eine Lösung für $x_1$, die eingesetzt in \textbf{(III**)}
            den Wert $x_2=-1+2 \cdot 2=3\,$ liefert. Einsetzen der Ergebnisse für $x_1$ und $x_2$ 
            in \textbf{(I*)} liefert schließlich
            }
            \lang{en}{
            We hence obtain a solution for $x_1$ that yields $x_2=-1+2 \cdot 2=3\,$ 
            when substituted into \textbf{(III**)}. Finally, substituting the 
            solutions for $x_1$ and $x_2$ into \textbf{(I*)} gives
            }
            \begin{displaymath}
              x_3=4-2-2\cdot 3=-4.
             \end{displaymath}
             \lang{de}{
            Das Einsetzungsverfahren liefert uns also als eindeutige Lösung das Zahlentripel
            $(2;3;-4)$ und somit die Lösungsmenge
            }
            \lang{en}{
            The substitution method gives us a unique solution $(2;3;-4)$ and hence 
            the solution set
            }
            \begin{displaymath}
                \mathbb{L}=\{(2;3;-4)\}.   
            \end{displaymath}

       \end{enumerate}
%     \end{tabs*}
     \end{showhide}
     \end{example}
 %%%%%%%%%%%%%%%%%%%%%%%%%%%%%%%%%%%%%%%%%%%%%%%%%%%%%%%%%%%%%%%%%

  \item \lang{de}{
        Bei einem Gleichungssystem wie im folgenden Beispiel kommt man natürlich auch mit weniger
        Wiederholungsschritten aus, da man die zweite Gleichung direkt nach $y$ auflösen kann:
        }
        \lang{en}{
        In a linear system such as the one used in the following example, fewer 
        repetitions are needed because the second equation can be immediately 
        solved for $y$:
        }

\begin{displaymath}
        \begin{mtable}[\cellaligns{crcrcrcr}]
            \text{(I)}&&&y&-&2z&=&7\\
            \text{(II)}&&-&3y&&&=&-6\\
            \text{(III)}&\qquad3x&-&2y&&&=&-1
        \end{mtable}
    \end{displaymath}        
        
  \begin{showhide}[\buttonlabels{\lang{de}{Zeige Lösung}\lang{en}{Show solution}}
                                {\lang{de}{Verstecke Lösung}\lang{en}{Hide solution}}]
   	\lang{de}{
    Es handelt sich um ein Gleichungssystem mit drei Gleichungen und drei Variablen. 
    Da die zweite Gleichung nur eine Unbekannte enth\"alt, kann man daraus den Wert f\"ur $y$ 
    bestimmen, nämlich 
     \begin{displaymath}
         \text{(II*)}  \qquad y=2.  
    \end{displaymath}     
	Setzen wir diesen Wert gemäß des Einsetzungsverfahrens in \textbf{(I)} und in 
  \textbf{(III)} ein, so erhalten wir
    daraus die Werte der anderen beiden Variablen. 
	Die L\"osung ist eindeutig und gegeben durch das Zahlentripel $\left(1;2;-\frac{5}{2}\right)$, also}
	\lang{en}{
  This example has a system of equations with three equations and three variables. 
	Because the second equation only has a single unknown, we can determine the value for $y$, here $y=2$.
	If we substitute this value into \textbf{(I)} and \textbf{(III)}, we get the values of the other two variables.
	The solution is unique and given by the triplet $\left(1;2;-\frac{5}{2}\right)$.
 }
	
	\begin{displaymath}
		\mathbb{L}=\left\{\left(1;2;-\frac{5}{2}\right)\right\}.
	\end{displaymath}
  \end{showhide} 
 %%%%%%%%%%%%%%%%%%%%%%%%%%%%%%%%%%%%%%%%%%%%%%%%%%%%%%%%%%%%%%%%%

  \item \lang{de}{
      Ein Vorteil des Einsetzungsverfahrens ist, dass es im Gegensatz zum 
      Additionsverfahren auch bei einigen nichtlinearen Gleichungssystemen 
          anwendbar ist.
      }
      \lang{en}{
      An advantage of the substitution method is that unlike the addition method, it 
      is applicable to systems of non-linear equations.
      }

    \begin{example}[\lang{de}{nichtlineares Gleichungssystems}
                    \lang{en}{System of non-linear equations}]
%      \begin{tabs*}[\initialtab{0}] 
%      \tab{\lang{de}{Beispiel zur Lösung eines nichtlinearen Gleichungssystems}\lang{en}{Example}}
    \begin{showhide}
        \begin{displaymath}
        \begin{mtable}[\cellaligns{crcrcrcr}]
          \text{(I)}&\qquad x^2&+&2x&+&y&=&0\\
          \text{(II)}&&&x&+&y&=&-2
        \end{mtable}
        \end{displaymath}

      \lang{de}{
      Dieses Gleichungssystem ist nichtlinear, da in der ersten Gleichung ein quadratischer Term in $x$ vorkommt.
      Wir wenden das Einsetzungsverfahren an, um das Gleichungssystem zu lösen.
      }
      \lang{en}{
      This system is non-linear, as in the first equation there appears a quadratic 
      term in $x$. We apply the substitution method in order to solve the system of 
      equations.
      }
      \begin{enumerate}
        \item \lang{de}{Die zweite Gleichung wird nach $y$ aufgelöst:}
              \lang{en}{The second equation is solved for $y$:}
          \begin{displaymath}
          y=-x-2
          \end{displaymath}

        \item \lang{de}{
        Der Term auf der rechten Seite wird für $y$ in Gleichung \textbf{(I)} eingesetzt.
        Daraus bestimmt man den Wert der Variablen $x$:
        }
        \lang{en}{
        The expression on the right hand side is substituted for $y$ in equation 
        \textbf{(I)}. From this we get the value of the variable $x$:
        }
          \begin{displaymath}
          x^2+2x+(-x-2)=x^2+x-2=0 \quad \Leftrightarrow \quad (x=1 \quad \text{oder} \quad x=-2)
          \end{displaymath}
        \item \lang{de}{
          Mit der aufgel\"osten Gleichung aus Schritt 1 lassen sich die Werte für 
          $y$ berechnen:
          }
          \lang{en}{
          With the solved equation from step 1, the values of $y$ can be calculated.
          }
          \begin{displaymath}
          \begin{mtable}[\cellaligns{rcrcrcr}]
            x&=&1\qquad&\Rightarrow&\qquad y&=&-3,\\
            x&=&-2\qquad&\Rightarrow&\qquad y&=&0.
          \end{mtable}
          \end{displaymath}
          Und wir erhalten als Lösungsmenge 
          \begin{displaymath}
              \mathbb{L}=\{(1, -3),(-2, 0)\}.
          \end{displaymath}
          \lang{de}{
          Da das Gleichungssystem nichtlinear ist, kann es auch genau zwei Lösungen 
          besitzen.
          }
          \lang{en}{
          As the system is non-linear, it can also have precisely two solutions.
          }

      \end{enumerate}
%      \end{tabs*}
      \end{showhide}
      \end{example}
\end{itemize}

\lang{de}{
Ein weiteres Verfahren, das mit dem Einsetzungsverfahren eng verwandt ist,
ist das \emph{Gleichsetzungsverfahren}. 
Auch dieses Verfahren ist nicht nur bei linearen Gleichungssystemen anwendbar.
}
\lang{en}{
Another method that is closely related to substitution is what we call here the 
\emph{equality method}. This is also applicable to non-linear systems.
}

\begin{algorithm}[\lang{de}{Gleichsetzungsverfahren}
                  \lang{en}{Equality method}]\label{alg:gleichsetzungsverfahren}
% Wir verwenden hier wieder das lineare Gleichungssystem aus Beispiel \ref{ex:bsp_lgs_1}
 \begin{displaymath}
  \begin{mtable}[\cellaligns{crcrcr}]
   \text{(I)}&\qquad3x&-&4y&=&2\\
    \text{(II)}&\qquad2x&+&3y&=&7
  \end{mtable}
 \end{displaymath}
% um dieses mithilfe des Gleichsetzungsverfahrens zu lösen. 

\begin{enumerate}
  \item[\notion{\lang{de}{1. Schritt}\lang{en}{First step}}] 
    \lang{de}{
    Zunächst werden beide Gleichungen nach derselben Variablen aufgelöst. 
    Wir wählen hier beispielhaft die Variable $x$.
    }
    \lang{en}{
    Firstly we rearrange both equations for the same variable. In this example, we 
    choose $x$.
    }
    \begin{displaymath}
    \begin{mtable}[\cellaligns{crcccr}]
        \text{(I*)}&\quad x&=&\frac{2}{3}&+&\frac{4}{3}y\\
        \text{(II*)}&\quad x&=&\frac{7}{2}&-&\frac{3}{2}y
    \end{mtable}
    \end{displaymath}
  \item[\notion{\lang{de}{2. Schritt}\lang{en}{Second step}}] 
    \lang{de}{
    Die beiden aufgelösten Terme auf der rechten Seite werden gleichgesetzt und 
    diese Gleichung wird schließlich gelöst:
    }
    \lang{en}{
    The two rearranged equations on the right hand side are equated and this new 
    equation is then solved:
    }
    \begin{displaymath}
    \frac{2}{3}+\frac{4}{3}y=\frac{7}{2}-\frac{3}{2}y \quad \Leftrightarrow \quad y=1
    \end{displaymath}
  \item[\notion{\lang{de}{3. Schritt}\lang{en}{Third step}}] 
    \lang{de}{
    Dieses Ergebnis wird anschließend in eine der beiden Gleichungen  
    \textbf{(I*)} oder \textbf{(II*)} eingesetzt, um daraus den Wert für die zweite 
    Variable zu ermitteln. Wir wählen hier  \textbf{(I*)} und setzen $y=1$ ein:
    }
    \lang{en}{
    This result is finally substituted into equation \textbf{(I*)} or \textbf{(II*)} 
    to get the value for the second variable. Here we choose \textbf{(I*)} and 
    substitute into it $y=1$.
    }
    \begin{displaymath}
     x=\frac{2}{3}+\frac{4}{3}\cdot1=2
    \end{displaymath}
    \lang{de}{
    Wir haben hier die Gleichung \textbf{(I*)} verwendet, aber auch die zweite 
    Gleichung \textbf{(II*)} liefert für $y=1$ dasselbe Resultat. Die Lösungsmenge 
    ist daher
    }
    \lang{en}{
    We have changed equation \textbf{(I*)} here, but the second equation 
    \textbf{(II*)} yields the same result for $y=1$. The solution set is therefore
    }
    \begin{displaymath}
    \mathbb{L}=\{(2; 1)\}.
    \end{displaymath}
\end{enumerate}

\end{algorithm}

% \lang{de}{\textbf{Anmerkungen:}}\lang{en}{\textbf{Notes:}}
%  \begin{itemize}
%
%    \item In Schritt 1 des Verfahrens kann man auch beide Gleichungen nach $y$ 
%        aufzulösen, um in Schritt 2 zunächst den Wert für die Variable $x$ 
%        zu bestimmen.
%
%    \item Mit Hilfe der oben vorgestellten Verfahren kann man u. U. auch 
%        einfache lineare Gleichungssysteme lösen, die mehr als zwei Variablen 
%        oder Gleichungen enthalten.                
%  \end{itemize}
%
%%%%%%%%%%%%%%%%%%%%%%%%%%%%%%%%%%%%%%%%%%%%%%%%%%%%%%%%%%%%%%%%%%%%%%%%%%%%%%%%%%%%%%%%%%%%%%%%%%%%%%%%%%
% Der nachfolgende Abschnitt wurde in Zshg. mit Viedeo-Einbau vorgezogen (zu Bemerkung \label{rem:lgs}) 
%%%%%%%%%%%%%%%%%%%%%%%%%%%%%%%%%%%%%%%%%%%%%%%%%%%%%%%%%%%%%%%%%%%%%%%%%%%%%%%%%%%%%%%%%%%%%%%%%%%%%%%%%%
%
% Bisher haben wir in diesem Kapitel nur Beispiele von linearen Gleichungssystemen betrachtet, die
% eine eindeute Lösung hatten. Im Gegensatz zu linearen Gleichungen gemäß Definition \ref{def:lin_gleichung} 
% sind aber lineare Gleichungssysteme
% % wie in \ref{rem:lgs} bereits erwähnt, 
% nicht immer eindeutig lösbar. Sie können auch unlösbar sein oder auch unendlich viele Lösungen haben. 


% \begin{example}%\label{ex:nicht-eindeutig-loesbar}
% \begin{tabs*}
%
%  \tab{LGS ohne Lösung}
%  \[ \begin{mtable}[\cellaligns{crcrcr}]
%  \text{(I)}&\qquad 2 \cdot  x & - & 4 \cdot  y & = & 10 \\
%  \text{(II)}&-3 \cdot  x & + & 6\cdot y & =  & -10
%  \end{mtable} \]
%  L"ost man die erste Gleichung nach $x$ auf, erh"alt man $x=5+2y$.
%
%  Einsetzen in die zweite Gleichung ergibt:
%  \[
%  \begin{mtable}[\cellaligns{crcr}]
%  & -3 \cdot  (5+2y)  +  6\cdot y & =  & -10 \\
%  \Leftrightarrow & \quad -15-6y +6y &=& -10\\
%  \Leftrightarrow & \quad  -15 &=& -10 
%  \end{mtable}\]
%
%  Diese Gleichung ist nie erf"ullt. Es gibt also kein Zahlenpaar $(x;y)$, das das Gleichungssystem erfüllt, d.\,h.
%  \[ \mathbb{L}=\{ \}. \]
%
%  \tab{LGS mit unendlich vielen Lösungen}
%  \[  
 % \begin{mtable}[\cellaligns{crcrcr}]
%  \text{(I)}&\qquad 2 \cdot  x & - & 4 \cdot  y & = & 10 \\
%  \text{(II)}&-3 \cdot  x & + & 6\cdot y & =  & -15
%  \end{mtable} \]
%  L"ost man die erste Gleichung nach $x$ auf, erh"alt man 
%      \begin{displaymath}
%           \text{(I*)}  \qquad x=5+2y.  
%      \end{displaymath}
%
%
%  Einsetzen in die zweite Gleichung ergibt:
%  \[\begin{mtable}[\cellaligns{crcr}]
%  & -3 \cdot  (5+2y)  +  6\cdot y & =  & -15 \\
%  \Leftrightarrow & \quad -15-6y +6y &=& -15\\
%  \Leftrightarrow & \quad  -15 &=& -15 
%  \end{mtable}\]
%
%  Diese Gleichung ist immer erf"ullt. Es bleibt also nur die Bedingung $\, x=5+2y\,$ aus 
%  \textbf{(I*)}, d.\,h.
%  für jede beliebige reelle Zahl $y$ ist das Paar $(5+2y;y)$ eine Lösung 
%  des linearen Gleichungssystems. Die Lösungsmenge ist somit
%  % \[ \mathbb{L}=\left\{ \left(\begin{smallmatrix} 5+2y \\ y \end{smallmatrix}\right) \mid\, y\in \R \right\}
%  % = \left\{ \begin{pmatrix}5\\ 0 \end{pmatrix}+r\cdot \begin{pmatrix} 2 \\ 1 \end{pmatrix} \mid\, r\in \R \right\} \] 
%
%  \[ \mathbb{L}=\left\{ (5+2y;y) \mid\, y\in \R \right\}. \]
%
% \end{tabs*}
% \end{example}

\end{visualizationwrapper}

\end{content}
