\documentclass{mumie.element.exercise}
%$Id$
\begin{metainfo}
  \name{
    \lang{de}{Ü06: Bruchrechnung}
    \lang{en}{Ex06: Fractions}
  }
  \begin{description} 
 This work is licensed under the Creative Commons License Attribution 4.0 International (CC-BY 4.0)   
 https://creativecommons.org/licenses/by/4.0/legalcode 

    \lang{de}{Rechnen mit Brüchen}
    \lang{en}{Evaluating expressions containing fractions}
  \end{description}
  \begin{components}
  \end{components}
  \begin{links}
    \link{generic_article}{content/rwth/HM1/T101neu_Elementare_Rechengrundlagen/g_art_content_03_bruchrechnung.meta.xml}{content_03_bruchrechnung}
    \link{generic_article}{content/rwth/HM1/T101neu_Elementare_Rechengrundlagen/g_art_content_02_rechengrundlagen_terme.meta.xml}{content_02_rechengrundlagen_terme}
  \end{links}
  \creategeneric
\end{metainfo}
\begin{content}
\begin{block}[annotation]
	Im Ticket-System: \href{https://team.mumie.net/issues/21976}{Ticket 21976}
\end{block}

\begin{block}[annotation]
  Rechnen mit Brüchen
     
\end{block}

\usepackage{mumie.ombplus}

\title{
  \lang{de}{Ü06: Bruchrechnung}
  \lang{en}{Ex06: Fractions}
}


\lang{de}{Berechnen Sie folgende Terme ohne Verwendung eines Taschenrechners:}
\lang{en}{Evaluate the following expressions without using a calculator:}
\begin{table}[\class{items}]
  \nowrap{a) $\, \frac{5}{6}-\frac{1}{4}\cdot \frac{2}{3}$} &
  \nowrap{b) $\, \big( \frac{6}{5}+ \frac{2}{3}\big)\cdot \frac{3}{2}$}\\
  \nowrap{c) $\, \big(\frac{1}{2}-\frac{1}{3}\big):\big(\frac{1}{2}+\frac{1}{3}\big)$} &
  \nowrap{d) $\, \frac{3}{5}-\big(7 \cdot \big(\frac{12}{42}-\frac{14}{70} \big) 
                     -4 \cdot \big(\frac{3}{21}-\frac{23}{92} \big)\big) $}

\end{table}

\begin{tabs*}[\initialtab{0}\class{exercise}]
  \tab{
  \lang{de}{Antworten}
  \lang{en}{Answers}
  }
\begin{table}[\class{items}]
    \nowrap{a) $\, \frac{2}{3}$}  & \nowrap{b) $\, \frac{14}{5}$} \\
    \nowrap{c)$\, \frac{1}{5}$} & \nowrap{d) $\, -\frac{3}{7}$}
  \end{table}

  \tab{
  \lang{de}{Lösung a)}
  \lang{en}{Solution a)}
  }
  
  \begin{incremental}[\initialsteps{1}]
    \step 
    \lang{de}{
    Wegen der \ref[content_02_rechengrundlagen_terme][Klammer- vor Potenz- vor Punkt- vor Strich-Regel]{rule:punkt-vor-strich}
    wird zunächst multipliziert und anschließend subtrahiert. Beim Multiplizieren müssen einfach die 
    jeweiligen Zähler miteinander multipliziert werden und die jeweiligen Nenner:
    }
    \lang{en}{
    Following the \ref[content_02_rechengrundlagen_terme][order of operations]{rule:punkt-vor-strich}, 
    first we need to multiply and then we subtract.
    When multiplying, the respective numerators and denominators need to be multiplied together:
    }
    \[ \frac{1}{4}\cdot \frac{2}{3}=\frac{1\cdot 2}{4\cdot 3}=\frac{1\cdot \cancel{2}}{\cancel{2}\cdot 2\cdot 3}
    =\frac{1}{6}. \]
   \lang{de}{Also gilt:}
   \lang{en}{Hence:}
    \[     \frac{5}{6}-\textcolor{#0066CC}{\frac{1}{4}\cdot \frac{2}{3}} = \frac{5}{6}-\textcolor{#0066CC}{\frac{1}{6}}. \]

    \step 
    \lang{de}{
    Da beide Br"uche schon die gleichen Nenner haben, m"ussen bei der Subtraktion lediglich die
    Zähler subtrahiert werden, also
    }
    \lang{en}{
    As both fractions have the same denominator, we may simply subtract the numerators, so
    }
    \[ \frac{5}{6}-\frac{1}{6} =\frac{5-1}{6}=\frac{4}{6}=\frac{\cancel{2} \cdot 2}{\cancel{2} \cdot 3}. \]
    \lang{de}{K"urzt man nun noch mit dem Faktor $2$, erhält man als Ergebnis $\frac{2}{3}$.}
    \lang{en}{Simplifying by cancelling by the factor $2$ yields the answer $\frac{2}{3}$.}
        
  \end{incremental}

 
\tab{\lang{de}{Lösung b)}\lang{en}{Solution for b)}}
  
  \begin{incremental}[\initialsteps{1}]
    \step 
    \lang{de}{
    Wegen der \ref[content_02_rechengrundlagen_terme][Klammer- vor Potenz- vor Punkt- vor Strich-Regel]{rule:punkt-vor-strich}
    wird zuerst die Addition in der Klammer ausgeführt. Hierzu müssen zunächst die Brüche auf den
    gleichen Nenner gebracht und dann die neuen Zähler addiert werden:
    }
    \lang{en}{
    Following the \ref[content_02_rechengrundlagen_terme][order of operations]{rule:punkt-vor-strich}, 
    we firstly evaluate the sum in the parentheses. In order to do this, both fractions must have the 
    same denominator, then their numerators may be added:
    }
        
    \begin{equation*} \frac{6}{5}+\frac{2}{3}=\frac{6\cdot 3}{5\cdot 3}+\frac{5\cdot 2}{5\cdot 3}=\frac{18+10}{15}
    =\frac{28}{15}.
    \end{equation*}
    
     \step \lang{de}{Anschließend sind die zwei Brüche zu multiplizieren und das Ergebnis zu k"urzen:}
     \lang{en}{Finally we multiply the two fractions and simplify the result:}
    
     \[ \textcolor{#0066CC}{\big( \frac{6}{5}+ \frac{2}{3}\big)}\cdot \frac{3}{2} 
       =\textcolor{#0066CC}{\frac{28}{15}}\cdot \frac{3}{2}=\frac{28\cdot 3}{15\cdot 2}=\frac{14\cdot 1}{5\cdot 1}=\frac{14}{5}.\]
     
    
  \end{incremental}

\tab{\lang{de}{Alternative Lösung b)}\lang{en}{Alternative solution for b)}}
    
  \begin{incremental}[\initialsteps{1}]
    \step \lang{de}{
    Anstatt zuerst die Addition auszuführen, kann man auch gemäß dem Distributivgesetz 
    ausmultiplizieren und anschließend die einzelnen Produkte berechnen:
    }
    \lang{en}{
    Instead of performing the addition first, we could have also used the distributive property to 
    multiply out the terms and then calculate their individual products:
    }
    \[ \big( \frac{6}{5}+\frac{2}{3} \big)\cdot \frac{3}{2}=\frac{6}{5}\cdot \frac{3}{2}+ \frac{2}{3}\cdot \frac{3}{2}
  =\frac{\overbrace{6}^{=\cancel{2} \cdot 3}\cdot 3}{5\cdot \cancel{2}}+\frac{2\cdot 3}{3\cdot 2}
  =\frac{3\cdot 3}{5\cdot 1}+1=\frac{9}{5}+1 .\]
  
  \step \lang{de}{
    Für die Summe schließlich müssen die Brüche auf den
    gleichen Nenner gebracht und dann die neuen Zähler addiert werden:
    }
    \lang{en}{
    In order to add the two terms together they must have the same denominator. After they have been 
    expanded or simplified to have the same denominator, their numerators can be added together:
    }
    \[ \frac{9}{5}+1=\frac{9}{5}+\frac{5}{5}=\frac{9+5}{5}=\frac{14}{5}.\]
    \lang{de}{
  	\emph{Anmerkung:} Im Vergleich zur anderen Lösung sieht man, dass die Rechnungen hier einfacher 
    waren, weil sich durch das Multiplizieren einiges kürzen lie"s. Es war also durchaus 
  	sinnvoll, das Distributivgesetz anzuwenden. Im Allgemeinen ist es jedoch besser, zuerst 
  	die Addition auszuführen und anschließend zu multiplizieren.
    }
    \lang{en}{
  	\emph{Remark:} The calculations here were simpler than those in the other solutions. However, it 
    was only luck that the terms simplified nicely, making it useful to use the distributive 
    property. In general, it is better to evaluate the sum in the parentheses first.
    }

	
  \end{incremental}
  %... other tabs

\tab{\lang{de}{Lösung c)}\lang{en}{Solution for c)}}
    
  \begin{incremental}[\initialsteps{1}]
    \step \lang{de}{
    Man könnte hier auf die Idee kommen, die dritte binomische Formel
    anzuwenden. Diese gilt jedoch nur bei Produkten, nicht aber bei Quotienten! Es
    müssen also zunächst die Differenz und die Summe innerhalb der Klammern 
    berechnet werden:
    }
    \lang{en}{
    Here it is perhaps tempting to apply the third binomial formula. However, this only works for 
    products, and not for quotients! Thus we must first evaluate the difference and the sum within 
    the parentheses:
    }
    \[ \frac{1}{2}-\frac{1}{3}=\frac{3}{6}-\frac{2}{6}=\frac{1}{6} \qquad
    \text{\lang{de}{und}\lang{en}{and}}
    \qquad \frac{1}{2}+\frac{1}{3}=\frac{3}{6}+\frac{2}{6}=\frac{5}{6}.\]

    
    \step 
    \lang{de}{Man dividiert durch einen Bruch, indem man mit dem Kehrbruch multipliziert. Daher gilt}
    \lang{en}{To divide by a fraction, we multiply by its reciprocal. Hence}
    \[ \textcolor{#0066CC}{\big(\frac{1}{2}-\frac{1}{3}\big)}:\textcolor{#00CC00}{\big(\frac{1}{2}+\frac{1}{3}\big)}
    =\textcolor{#0066CC}{\frac{1}{6}}:\textcolor{#00CC00}{\frac{5}{6}}=\frac{1}{6}\cdot \frac{6}{5}=\frac{\,1\cdot \cancel{6}}{\cancel{6}\cdot 5}=\frac{1}{5}.\]

    
 \end{incremental}
  %... other tabs

\tab{\lang{de}{Lösung d)}\lang{en}{Solution for d)}}
    
  \begin{incremental}[\initialsteps{1}]
    \step 
    \lang{de}{
    Bevor wir den Term unter Anwendung der 
    \ref[content_02_rechengrundlagen_terme][Klammer- vor Potenz- vor Punkt- vor Strich-Regel]{rule:punkt-vor-strich}
    oder alternativ des 
    \ref[content_02_rechengrundlagen_terme][Distributivgesetzes]{rule:rechengesetze}
    ausrechnen, kürzen wir die Brüche in den inneren Klammern soweit wie möglich bzw.
    so weit es sinnvol ist. Dabei ist unbedingt zu beachten, dass das Kürzen aus der 
    Summe, wie z.B.
    }
    \lang{en}{
    Before we evaluate the expression using 
    \ref[content_02_rechengrundlagen_terme][order of operations]{rule:punkt-vor-strich} 
    or alternatively using the 
    \ref[content_02_rechengrundlagen_terme][distributivity law]{rule:rechengesetze}, 
    we simplify the fractions in the innermost parentheses as much as possible or as much as is 
    convenient. It is important to not cancel across different terms of a sum, i.e.
    }
    \[
    \frac{12}{42}-\frac{14}{70} = \frac{12}{3 \cdot \cancel{14}}-\frac{\cancel{14}}{70}
                              \neq \frac{12}{3}-\frac{1}{70},
    \]
    \lang{de}{
    nicht zulässig ist. Wir kürzen also z.B. nach vorheriger 
    \ref[content_03_bruchrechnung][Primfaktorzerlegung]{ggT} zur Bestimmung des ggT,
    wobei wir gleich im Auge behalten, dass für die anschließende Addition oder Subtraktion
    ein gemeinsamer Nenner benötigt wird. Deshalb kürzen wir den zweiten Bruch nicht durch 
    $\, 7$, sondern erweitern stattdessen den ersten Bruch um $\,5\,$ auf den gemeinsamen
    Hauptnenner.
    }
    \lang{en}{
    as this is not allowed. Instead, we can use the 
    \ref[content_03_bruchrechnung][prime factorisation]{ggT} of the numerator and denominator of each 
    fraction, and keep in mind that we require two fractions to have the same denominator if we want 
    to add or subtract them. For this reason, we do not cancel the second fraction by $\, 7$, instead 
    multipllying the numerator and denominator of the first fraction by $5$ to attain a common 
    denominator.
    }
    \[
    \frac{12}{42}-\frac{14}{70} 
    =\frac{2 \cdot \cancel{2} \cdot \cancel{3}}{\cancel{2} \cdot \cancel{3} \cdot 7}
    -\frac{\cancel{2}\cdot 7}{\cancel{2} \cdot 5 \cdot 7} 
    = \frac{2}{7} - \frac{7}{5 \cdot 7}
    = \frac{2\cdot 5}{7\cdot 5} - \frac{7}{5 \cdot 7}= \frac{3}{35} 
    \]
    \lang{de}{Mit dem zweiten Klammerterm verfahren wir analog:}
    \lang{en}{Similarly, for the second inner parentheses we obtain:}
    \[
    \frac{3}{21}-\frac{23}{92} 
    =\frac{\cancel{3}}{\cancel{3} \cdot 7} - \frac{\cancel{23}}{4 \cdot \cancel{23}} 
    = \frac{1}{7} - \frac{1}{4}
    = \frac{4}{7\cdot 4} - \frac{7}{4 \cdot 7}= -\frac{3}{28} 
    \]    
    
    \step 
    \lang{de}{Die Ergebnisse der berechneten Klammerterme setzen wir nun in den Ausgangsterm ein}
    \lang{en}{These evaluated expressions can now be substituted into the original expression}
    \[
      \frac{3}{5}-\big(7 \cdot \textcolor{#0066CC}{\big(\frac{12}{42}-\frac{14}{70} \big)} 
                     -4 \cdot \textcolor{#00CC00}{\big(\frac{3}{21}-\frac{23}{92} \big)}\big)
     =\frac{3}{5}-\big(7 \cdot \textcolor{#0066CC}{\frac{3}{35}} 
                     -4 \cdot \textcolor{#00CC00}{\big(-\frac{3}{28}\big)}\big)
    \] 
    \lang{de}{
    und rechnen diesen unter Beachtung der \emph{Klammer- vor Potenz- vor Punkt- vor Strich-Regel} 
    $\,$ schrittweise weiter aus.
    }
    \lang{en}{
    and we may now use the \emph{order of operations} to simplify the expression further.
    }
    \[
      \frac{3}{5} -\big(7 \cdot \frac{3}{35}-4 \cdot \big(-\frac{3}{28}\big)\big)
     =\frac{3}{5} - \big(\frac{3}{5} + \frac{3}{7}\big)
     =-\frac{3}{7}
    \] 

  \end{incremental}
  %... other tabs
\end{tabs*}
\end{content}