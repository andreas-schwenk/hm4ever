\documentclass{mumie.element.exercise}
%$Id$
\begin{metainfo}
  \name{
    \lang{de}{Ü11: Lineare Gleichungen}
    \lang{en}{Ex11: Linear equations}
  }
  \begin{description} 
 This work is licensed under the Creative Commons License Attribution 4.0 International (CC-BY 4.0)   
 https://creativecommons.org/licenses/by/4.0/legalcode 

    \lang{de}{Lösen linearer Gleichungen}
    \lang{de}{Solving linear equations}
  \end{description}
  \begin{components}
  \end{components}
  \begin{links}
\link{generic_article}{content/rwth/HM1/T101neu_Elementare_Rechengrundlagen/g_art_content_02_rechengrundlagen_terme.meta.xml}{content_02_rechengrundlagen_terme}
\link{generic_article}{content/rwth/HM1/T101neu_Elementare_Rechengrundlagen/g_art_content_05_loesen_gleichungen_und_lgs.meta.xml}{content_05_loesen_gleichungen_und_lgs}
\end{links}
  \creategeneric
\end{metainfo}
\begin{content}
\begin{block}[annotation]
	Im Ticket-System: \href{https://team.mumie.net/issues/21981}{Ticket 21981}
\end{block}

 \begin{block}[annotation]
%
   Übung zu linearen Gleichungen und deren Lösung
   \\
   Teile dieser Aufgabe entstammen aus dem OMB+ \\
   (Kapitel II Gleichungen in einer Unbekannten) 
%
\end{block}

  \title{
    \lang{de}{Ü11: Lösen linearer Gleichungen}
    \lang{en}{Ex11: Linear equations}
  }
 
%
% Aufgabenstellung
%
\begin{enumerate}[alph]
%
  \item \lang{de}{
        Prüfen Sie, ob es sich bei den folgenden Gleichungen um  
        \ref[content_05_loesen_gleichungen_und_lgs][lineare Gleichungen]{def:lin_gleichung}
        der Form $bx+c=0$ mit $b,c \in \R$ und $b \neq 0$ handelt und bestimmen Sie ihre Lösungsmenge.
        }
        \lang{en}{
        Check whether the following equations are 
        \ref[content_05_loesen_gleichungen_und_lgs][linear equations]{def:lin_gleichung} 
        of the form $bx+c=0$ with $b,c \in \R$ and $b \neq 0$, and determine their solution sets.
        }
        
    \begin{table}[\class{items}]
     \nowrap{i.  $\; 8x+17=81$}  &                              
     \nowrap{ii. $\; -7(-x-3)=(-6+21x)\cdot \frac{1}{3} $}\\    
     \nowrap{iii.$\; \big((x+3)2+4\big)5-20=40$}&               
     \nowrap{iv. $\; -2-2x=-(x+1)^2+(x+1)(x-1)$} \\                     
    \end{table}
%
  \item \lang{de}{Stellen Sie Gleichungen zu den Textaufgaben auf und lösen Sie diese.}
        \lang{en}{Express the following problems as equations and solve these.}
    
    \begin{enumerate}[roman]
      \item \lang{de}{
        Eine Wandergruppe wandert 2 Tage. Am 1. Tag legen sie ein Viertel 
        der Gesamtstrecke und zusätzlich 2 km zurück. Am 2. Tag legen sie die 
        Hälfte der Gesamtstrecke und zusätzlich 1 km zurück.\\
        Wie lang ist die zurückgelegte Gesamtstrecke?\\
        }
        \lang{en}{
        A group of hikers walks for two days. On the first day they they cover a quarter of the 
        total distance and an additional $2$km. On the second day they cover half of the total 
        distance and an additional $1$km.\\
        What is the total distance of their hike?\\
        }
        
      \item \lang{de}{
        Vier Freunde fahren zusammen in Urlaub. Sie teilen sich die 
        Benzinkosten für die Fahrt. Für die Hinfahrt zahlt jeder 2,40 €. 
        Auf der Rückfahrt sind sie nur noch zu dritt, es fallen aber die
        selben Benzinkosten an, wie für die Hinfahrt. Wie viel kostet die 
        Rückfahrt pro Person?
        }
        \lang{en}{
        Four friends go on holiday together. They share the cost of fuel for the journey. On the 
        way there each one pays $2.40$€. Only three of them return in the car, with the same total 
        cost of petrol for the journey split between them. How much does the journey back cost each 
        person?
        }
    \end{enumerate}

\end{enumerate}

%
% Lösungen
%
  \begin{tabs*}[\initialtab{0}\class{exercise}]
  
    \tab{\lang{de}{Antworten}\lang{en}{Answers}}
      \begin{enumerate}[alph]
        \item 
         \begin{enumerate}[roman]
          \item $\mathbb{L}= \{ 8 \}$
          \item \lang{de}{
                Die Gleichung lässt sich nicht in die Form $bx+c=0$ mit $b \neq 0$ umformen und
                wir für keine reelle Zahl $x$ zu einer wahren Aussage. Es gilt daher 
                }
                \lang{en}{
                The equation cannot be written in the form $bx+c=0$ with $b \neq 0$ and is not true 
                for any real number $x$.
                }
                $\mathbb{L}= \emptyset.$
          \item $\mathbb{L}= \{ 1 \}$
          \item \lang{de}{
                Die Gleichung lässt sich nicht in die Form $bx+c=0$ mit $b \neq 0$ umformen, ist
                aber unabhängig von $x$ immer erfüllt. Es gilt daher $\mathbb{L}= \R.$
                }
                \lang{en}{
                The equation cannot be written in the form $bx+c=0$ with $b \neq 0$, but is always 
                true independently of the value of $x$. Hence $\mathbb{L}= \R$.
                }
         \end{enumerate}
        
        \item 
         \begin{enumerate}[roman]
            \item \lang{de}{Die Gesamtstrecke ist $12$ Kilometer lang.}
                  \lang{en}{The total distance hiked is $12$ kilometers.}
            \item \lang{de}{Für die Rückfahrt zahlt jeder 3,20 €.}
                  \lang{en}{On the journey back, each person pays $3.20$€.}
          \end{enumerate}
     \end{enumerate}

     
    \tab{\lang{de}{Lösung a) i.}\lang{en}{Solution for a) i.}} 
     \begin{incremental}[\initialsteps{1}]
      \step 
        \lang{de}{
        Subtrahiert man auf beiden Seiten der Gleichung $81$, 
        erhält man $8x-64=0$, also eine lineare Gleichung gemäß 
        der gewünschten Form. Der Leitkoeffizient ist hier $b=8 \neq 0.$
        }
        \lang{en}{
        Subtracting $81$ from both sides of the equation yields $8x-64=0$, which is in the desired 
        form. The leading coefficient here is $b=8 \neq 0$.
        }
      \step
        \lang{de}{Wir lösen die Gleichung weiter nach $x$ auf}
        \lang{en}{We rearrange the equation for $x$}
        \[
        \begin{mtable}[\cellaligns{crcll}]
            & 8x-64  &\,=\,& 0 \quad &\vert +64 \\
         \Leftrightarrow &\qquad 8x  &\,=\,& 64 \quad &\vert :8 \\
         \Leftrightarrow & \qquad x &\,=\,& 8  &
        \end{mtable}
        \]

      \lang{de}{Die Lösung ist also $x = 8$.}
      \lang{en}{The solution is therefore $x = 8$.}
 
  \end{incremental}
   
    \tab{\lang{de}{Lösung a) ii.}\lang{en}{Solution for a) ii.}} 

        \lang{de}{
        Wir lösen zunächst die Klammern rechts und links des Gleichheitszeichens 
        nach dem Distributivgesetz auf:
        }
        \lang{en}{
        Firstly we expand the parentheses on both sides of the equation using the distributivity law:
        }
        \[
        \begin{mtable}[\cellaligns{crcll}]
                         &       -7(-x-3) &\,=\,& (-6+21x)\cdot \frac{1}{3} &\\
         \Leftrightarrow &\qquad 7x + 21 &\,=\,& -2 + 7x & \vert +2\\
         \Leftrightarrow &\qquad  7x + 23 &\,=\,&  7x & \vert -7x\\
         \Leftrightarrow &\qquad    23 &\,=\,& 0 &
         \end{mtable}
        \]
        \lang{de}{
        Die Äquivalenzumformung der Gleichung führt zu der falschen Aussage $\,23=0$.
        Das bedeutet, dass auch die Ausgangsgleichung für keine reelle Zahl $x$ lösbar
        ist. Daher ist die Lösungsmenge $\mathbb{L}= \emptyset.$
        }
        \lang{en}{
        These equivalence transformations of the equation lead to the false statement $\,23=0$. 
        This means that the original equation is also not satisfied by any real number $x$. 
        Therefore the solution set is $\mathbb{L}= \emptyset$.
        }
        
   
    \tab{\lang{de}{Lösung a) iii.}\lang{en}{Solution for a) iii.}} 
     \begin{incremental}[\initialsteps{1}]
      \step \lang{de}{Zuerst werden die Klammerterme von innen nach außen aufgelöst:}
        \lang{en}{Firstly we expand the parentheses, starting with the innermost ones:}
        \[
        \begin{mtable}[\cellaligns{crcll}]
                         &       \big((x+3)2+4\big)\cdot 5-20 &\,=\,& 40 &\\
         \Leftrightarrow &\qquad \big(2x+6+4\big)\cdot 5-20 &\,=\,& 40 &\\
         \Leftrightarrow &\qquad \big(2x+10\big)\cdot 5-20 &\,=\,& 40 &\\
         \Leftrightarrow &\qquad 10x+50-20 &\,=\,& 40 &\\
         \Leftrightarrow &\qquad 10x+30 &\,=\,& 40 &
         \end{mtable}
        \]
        \lang{de}{
        Subtrahiert man auf beiden Seiten der Gleichung die Zahl $40$, so erhält man
        mit \[10x-10=0 \] eine lineare Gleichung der gewünschten Form. Der 
        Leitkoeffizient ist $b=10 \neq 0.$
        }
        \lang{en}{
        If we subtract $40$ from both sides of the equation we obtain a linear equation \[10x-10=0 \] 
        in the desired form. The leading coefficient is $b=10 \neq 0$.
        }
      
     \step \lang{de}{
        Anschließend wird die Variable $x$ isoliert, d.h. die Gleichung wird nach $x$ aufgelöst:
        }
        \lang{en}{
        Finally we rearrange for the variable $x$:
        }
        \[
        \begin{mtable}[\cellaligns{crcll}]
                         &\qquad 10x+30 &\,=\,& 40 &\qquad \vert -30\\
         \Leftrightarrow &\qquad   10x &\,=\,& 10 &\qquad \vert :10\\
         \Leftrightarrow &\qquad     x &\,=\,& 1 &
         \end{mtable}
        \]
        \lang{de}{Die Lösung ist also $x = 1$.}
        \lang{en}{Hence the solution is $x = 1$.}
     \end{incremental}
   
    \tab{\lang{de}{Lösung a) iv.}\lang{en}{Solution for a) iv.}} 

        \lang{de}{
        Um zu prüfen, ob es sich bei der vorliegenden Gleichung um eine lineare 
        Gleichung handelt, lösen wir zunächst die 
        \ref[content_02_rechengrundlagen_terme][Binome]{rule:binomische_formeln}
        rechts des Gleichheitszeichens auf. Für den ersten Summanden verwenden wir 
        die 1. binomische Formel und für den 2. Summanden die 3. binomische Formel:
        }
        \lang{en}{
        In order to check whether the equation is in fact linear, we expand the products of 
        \ref[content_02_rechengrundlagen_terme][binomials]{rule:binomische_formeln} 
        on the right-hand side. For the first we apply the 1st binomial formula, and for the 
        second we apply the 3rd binomial formula:
        }
        \[
        \begin{mtable}[\cellaligns{crcll}]
                         &\qquad -2-2x &\,=\,& -(x+1)^2+(x+1)(x-1) &\\
         \Leftrightarrow &\qquad -2-2x &\,=\,&  -(x^2+2x+1)+(x^2-1)&\\
         \end{mtable}
        \]
        \lang{de}{
        Anschließend lösen wir die Klammern auf und sortieren die Summanden nach 
        den Potenzen von $x$:
        }
        \lang{en}{
        Finally we rearrange the terms in order of their powers of $x$:
        }
        \[
        \begin{mtable}[\cellaligns{crcll}]
                         &\qquad -2-2x &\,=\,& -(x^2+2x+1)+(x^2-1)  &\\
         \Leftrightarrow &\qquad -2-2x &\,=\,&  -x^2 + x^2 -2x -1-1 &\\
         \Leftrightarrow &\qquad -2-2x &\,=\,&  -2x -2              &\vert +2x+2 \\
         \Leftrightarrow &\qquad 0 \cdot x &\,=\,& 0 &  
        \end{mtable}
        \]
        \lang{de}{
        Da sich die Summanden mit $x^2$ aufheben, ist die Gleichung zwar \emph{linear}, 
        aber mit Leitkoeffizient $\,b=0.\,$ Sie entspricht also nicht der gewünschten Form
        und ist daher auch nicht eindeutig lösbar.
        \\\\
        Im Gegensatz zu Teilaufgabe ii. ist das Ergebnis der Äquivalenzumformung hier jedoch
        eine wahre Aussage, nämlich $\,0=0$. Das bedeutet, dass die Ausgangsgleichung 
        für jede beliebige reelle Zahl $x$ ebenfalls wahr ist. Somit ist die Lösungsmenge $\mathbb{L}= \R.$
        }
        \lang{en}{
        As the terms in $x^2$ cancel each other out, the equation is indeed \emph{linear} with 
        leading coefficient $\,b=0.\,$ It therefore does not have the desired form and does not 
        have a unique solution.
        \\\\
        As opposed to question ii. the result of the equivalence transformations here is a true 
        proposition, $\,0=0$. This means that the original equation is true for every real number 
        $x$. Thus the solution set is $\mathbb{L}= \R$.
        }
   
   
    \tab{\lang{de}{Lösung b) i.}\lang{en}{Solution for b) i.}}
     \begin{incremental}[\initialsteps{1}]
      \step \lang{de}{
        Es wird nach der Gesamtstrecke gefragt, daher lassen sich die einzelnen 
        Abschnitte pro Tag in einer Gleichung über die Gesamtstrecke $x$ zusammenfassen:
        }
        \lang{en}{
        We wish to find the total distance hiked, so we may denote this by $x$ and note that this 
        equals the sum of the distances travelled on each day:
        }
      \step 
        \[
        \begin{mtable}[\cellaligns{crcll}]
            &(\frac{1}{4}x+2)+(\frac{1}{2}x+1) &\,=\,& x &\\
         \Leftrightarrow &\qquad \frac{3}{4}x+3    &\,=\,& x \quad &\vert -\frac{3}{4}x\\
         \Leftrightarrow & \qquad 3 &\,=\,& \frac{1}{4}x \quad &\vert \cdot 4\\
         \Leftrightarrow & \qquad 12 &\,=\,& x &
        \end{mtable}
        \]
        \lang{de}{Die Gesamtstrecke beträgt also 12 Kilometer.}
        \lang{en}{The total distance travelled is thus 12 kilometers.}

   \end{incremental}
   
    \tab{\lang{de}{Lösung b) ii.}\lang{en}{Solution for b) ii.}} 
        \lang{de}{
        Gesucht sind die Kosten $k$ pro Person für die Rückfahrt, wenn die 
        Gesamtkosten für die Rückfahrt den Gesamtkosten für die Hinfahrt 
        entsprechen.
        }
        \lang{en}{
        We wish to find the cost $k$ per person on the journey back, if the total cost of the journey 
        back is equal to the total cost of the journey there.
        }
        
        \[
        \begin{mtable}[\cellaligns{crcll}]
                         &  3\cdot k &\,=\,& 4 \cdot 2,4  & \vert :3\\
         \Leftrightarrow &\qquad  k &\,=\,& \displaystyle{\frac{9,6}{3}} \quad &\\
         \Leftrightarrow & \qquad k &\,=\,& 3,2 \quad &
        \end{mtable}
        \]
 
        \lang{de}{Für die Rückfahrt zahlt jeder der drei Freunde 3,20 €.}
        \lang{en}{Hence for the journey back each of the three friends pays 3.20€.}
   
  \end{tabs*}


\end{content}

