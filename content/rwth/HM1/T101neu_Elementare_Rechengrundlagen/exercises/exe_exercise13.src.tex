\documentclass{mumie.element.exercise}
%$Id$
\begin{metainfo}
  \name{
    \lang{de}{Ü13: Lineare Gleichungssysteme}
    \lang{en}{Ex13: Linear systems}
  }
  \begin{description} 
 This work is licensed under the Creative Commons License Attribution 4.0 International (CC-BY 4.0)   
 https://creativecommons.org/licenses/by/4.0/legalcode 

    \lang{de}{Lösung eines linearen Gleichungssystems}
    \lang{en}{Solving a linear system}
  \end{description}
  \begin{components}
  \end{components}
  \begin{links}
  \end{links}
  \creategeneric
\end{metainfo}
\begin{content}
\begin{block}[annotation]
	Im Ticket-System: \href{https://team.mumie.net/issues/21983}{Ticket 21983}
\end{block}

\begin{block}[annotation]
	Lineare Gleichungssysteme
\end{block}

\title{
  \lang{de}{Ü13: Lineare Gleichungssysteme}
  \lang{en}{Ex13: Linear systems}
}

%
% Aufgabe
%
\lang{de}{Prüfen Sie, ob das jeweils angegebene Zahlenpaar eine Lösung des linearen Gleichungssystems}
\lang{en}{Check whether each of the following pairs $(x;y)$ is a solution to the linear system}
  \begin{displaymath} 
  \begin{mtable}[\cellaligns{ccrcrcr}]
  \text{(I)}&\qquad&3x&+&4y&=&8\\
  \text{(II)}&\qquad&-2x&+&3y&=&-11
  \end{mtable}
  \end{displaymath}   
\lang{de}{ist.}
\lang{en}{}

\begin{table}[\class{items}]
  \nowrap{a) $(4; -1)$} & \nowrap{b) $(-4; 5)$}\\
  \nowrap{c) $(1; -3)$} & \nowrap{d) $(2; 2)$}
\end{table}

%
% Lösungen
%

\begin{tabs*}[\initialtab{0}\class{exercise}]
  \tab{
  \lang{de}{Antwort}
  \lang{en}{Answer}
  }
  \begin{table}[\class{items}]
    \nowrap{a) $\;$ \lang{de}{Ja}\lang{en}{Yes} $\quad$} & 
    \nowrap{b) $\;$ \lang{de}{Nein}\lang{en}{No} $\quad$}\\
    \nowrap{c) $\;$ \lang{de}{Nein}\lang{en}{No} $\quad$} & 
    \nowrap{d)  $\;$ \lang{de}{Nein}\lang{en}{No} $\quad$}
  \end{table}

  \tab{
  \lang{de}{Lösung}
  \lang{en}{Solution}
  }
  \begin{incremental}[\initialsteps{1}]
    \step \lang{de}{
    Um zu prüfen, ob ein Zahlenpaar eine Lösung des linearen Gleichungssystems ist, 
    müssen die Zahlen für $x$ und $y$ in die beiden Gleichungen eingesetzt werden 
    und geschaut werden, ob die Gleichungen dann erfüllt sind, also eine wahre Aussage 
    herauskommt.
    }
    \lang{en}{
    To test whether a pair of numbers is a solution to the linear system, we substitute the two 
    numbers for $x$ and $y$ in both equations, and see if the equations are then satisfied. That is, 
    we see whether this yields a true proposition.
    }
     
    \step \lang{de}{
    Setzt man beispielsweise das Zahlenpaar aus a) in die erste Gleichung ein, so erhält man
    \[3\cdot 4+4\cdot(-1)=12-4=8.\]
    Es erfüllt also diese Gleichung. Ebenso erfüllt das Zahlenpaar aus b) die erste Gleichung.
    \\\\
    Beim Einsetzen des Zahlenpaars aus c) in die erste Gleichung kommt man auf 
    \[3\cdot 1+4\cdot(-3)=3-12=-9\neq 8.\] Es erfüllt also die erste Gleichung nicht.
    Auch das Zahlenpaar aus d) erfüllt die erste Gleichung nicht. 
    Somit scheiden die Zahlenpaare aus c) und d) als mögliche Lösung für da Gleichungsssystem aus.
    }
    \lang{en}{
    For example, if we substitute the numbers in a) into the first equation, we obtain 
    \[3\cdot 4+4\cdot(-1)=12-4=8.\]
    They therefore satisfy this equation. Likewise, the pair of numbers in b) satisfies the first 
    equation.
    \\\\
    If we substitute the numbers from c) into the first equation we get 
    \[3\cdot 1+4\cdot(-3)=3-12=-9\neq 8.\]
    Hence these do not satisfy the equation, nor does the pair of numbers in d). We can thus 
    remove c) and d) from consideration as solutions of the linear system.
    }
       
    \step \lang{de}{
    Setzt man das Zahlenpaar aus a) nun in die zweite Gleichung ein, so erhält man 
    \[-2\cdot 4+3\cdot(-1)=-8-3=-11.\] Es erfüllt also auch diese Gleichung und ist 
    somit eine Lösung des linearen Gleichungssystems.
    \\\\
    Mit dem Einsetzen des Zahlenpaars aus b) in die zweite Gleichung kommt man auf 
    \[-2\cdot(-4)+3\cdot 5=8+15=23 \neq -11.\] Es erfüllt die zweite Gleichung
    also nicht und ist somit auch keine Lösung des linearen Gleichungssystems.
    \\\\
    (Das Zahlenpaar aus c) würde zwar die zweite Gleichung erfüllen, ist aber ja schon als Lösung ausgeschlossen worden.\\
    Das Zahlenpaar aus d) erfüllt auch die zweite Gleichung nicht.)
    }
    \lang{en}{
    Substituting the pair of numbers in a) into the second equation yields 
    \[-2\cdot 4+3\cdot(-1)=-8-3=-11.\] They therefore also satisfy this equation, and a) is hence a 
    solution to the linear system.
    \\\\
    Substituting the pair of numbers in b) into the second equation yields 
    \[-2\cdot(-4)+3\cdot 5=8+15=23 \neq -11.\]
    These therefore do not satisfy the second equation, and are therefore not a solution to the 
    linear system.
    }
    
    \step 
    \lang{de}{Bemerkung:}
    \lang{en}{Remark:}
    \begin{itemize}
      \item \lang{de}{
            Das Zahlenpaar aus c) würde zwar die zweite Gleichung erfüllen, ist aber ja schon
            als Lösung ausgeschlossen worden.
            }
            \lang{en}{
            The numbers in c) would actually satisfy the second equation, but cannot be a solution 
            because they do not satisfy the first.
            }
      \item \lang{en}{
            Das Zahlenpaar aus d) erfüllt auch die zweite Gleichung nicht.
            }
            \lang{en}{
            The numbers from d) satisfy neither equation.
            }
    \end{itemize}
  \end{incremental}
  
    \tab{
  \lang{de}{Hinweis}
  \lang{en}{Note}
  }
    \lang{de}{
    Alternativ könnte man zum Lösen dieser Aufgabe auch das LGS mit einem der 
    bekannten Verfahren lösen und anschließend die Lösung mit den angebenen 
    Zahlenpaaren vegleichen.
    }
    \lang{en}{
    Alternatively we may solve the linear system and compare the solutions to the given pairs of 
    numbers. This is more labour intensive than simply substituting each pair into the equations 
    in this case.
    }
    
\end{tabs*}
\end{content}