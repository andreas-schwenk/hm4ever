\documentclass{mumie.element.exercise}
%$Id$
\begin{metainfo}
  \name{
    \lang{de}{Ü05: Summen-/Produktformel}
    \lang{en}{Ex05: Sum and product notation}
  }
  \begin{description} 
 This work is licensed under the Creative Commons License Attribution 4.0 International (CC-BY 4.0)   
 https://creativecommons.org/licenses/by/4.0/legalcode 

    \lang{de}{Summen- und Produktformeln}
    \lang{en}{Sum and product notation}
  \end{description}
  \begin{components}
  \end{components}
  \begin{links}
    \link{generic_article}{content/rwth/HM1/T101neu_Elementare_Rechengrundlagen/g_art_content_02_rechengrundlagen_terme.meta.xml}{content_02_rechengrundlagen_terme}
  \end{links}
  \creategeneric
\end{metainfo}
\begin{content}
\begin{block}[annotation]
	Im Ticket-System: \href{https://team.mumie.net/issues/21975}{Ticket 21975}
\end{block}

\begin{block}[annotation]
  Übung zur Berechnung von Summenformeln und Herleitung von Produktformeln 
     
\end{block}

\usepackage{mumie.ombplus}
\title{
\lang{de}{Ü05: Summen-/Produktformel}
\lang{en}{Ex05: Sum and product notation}
}

\begin{enumerate}
  \item \lang{de}{Geben Sie jeweils den Zahlenwert der folgenden Summen an:}
  \lang{en}{Evaluate the following sums (give their numerical value):}
  \begin{table}[\class{items}]

      \nowrap{a) $\, \displaystyle\sum_{j=1}^3 j (j + 1)$} & 
      \nowrap{b) $\, \displaystyle\sum_{n=1}^1 (3 n + 7)$} \\
      \nowrap{c) $\, \displaystyle\sum_{k=1}^{101} 3$} &
      \nowrap{d) $\, \displaystyle\sum_{k=1}^5 \left( \frac{1}{k} - \frac{1}{k + 1} \right)$}
    \end{table}

  \item \lang{de}{Schreiben Sie die folgenden Produkte mit Hilfe des Produktzeichens:}
  \lang{en}{Express the following products using the product notation:}
  \begin{table}[\class{items}]

      \nowrap{a) $\, 4 \cdot 7 \cdot 10 \cdot 13$} \\ 
      \nowrap{b) $\, \frac{2}{3} \cdot \frac{4}{5} \cdot \frac{6}{7} \cdot \frac{8}{9}$} \\
      \nowrap{c) $\, 9 \cdot 16 \cdot 25 \cdot 36 \cdot 49$}
    \end{table}
 \end{enumerate} 

  

\begin{tabs*}[\initialtab{0}\class{exercise}]
  \tab{
  \lang{de}{Antworten}
  \lang{en}{Answers}
  }
\begin{table}[\class{items}]

    \nowrap{1.a) $\; 20, \qquad$ b) $\; 10, \qquad$ c) $\; 303, \qquad$ d)  $\; \frac{5}{6}$}  \\
    
    \nowrap{2.a) $\, \displaystyle\prod_{j=1}^4 (3 j + 1)$} \\ 
    \nowrap{2.b) $\, \displaystyle\prod_{j=1}^4 \frac{2 j}{2 j + 1}$} \\
    \nowrap{2.c) $\, \displaystyle\prod_{j=3}^7 \, j^2$}

  \end{table}
\lang{de}{
\textbf{Bemerkung:}
Die Lösungen für das Produktzeichen (2.a)-c)) sind nicht eindeutig. Es könnte z.\,B. eine 
\emph{Indexverschiebung} vorgenommen werden, die den Bereich des Laufindex verändert.
}
\lang{en}{
\textbf{Remark:} 
The solutions for (2.a),b),c)) are not unique. For example, a \emph{shift of the index} may be 
performed, changing the domain of the index of the product.
}

  \tab{\lang{de}{Lösung 1.a)}\lang{en}{Solution for 1.a)}}
  
  \begin{incremental}[\initialsteps{1}]
    \step 
    \lang{de}{
    Die Summe läuft über $j=1, 2, 3$. Das heißt, wir setzen diese Werte für $j$ ein und schreiben 
    zunächst die Summe aus, um sie anschließend zu berechnen:
    }
    \lang{en}{
    The sum runs over  $j=1, 2, 3$. That means that we can substitute each of these values for $j$ 
    and write the sum manually.
    }
    \step
    \begin{equation*}\sum_{j=1}^3 j(j+1) = 1(1+1)+2(2+1)+3(3+1) = 2+ 2\cdot 3 + 3\cdot 4 = 2+6+12 =20.
    \end{equation*} 
    
  \end{incremental}
  

  \tab{\lang{de}{Lösung 1.b)}\lang{en}{Solution for 1.b)}}
  
  \begin{incremental}[\initialsteps{1}]
    \step 
    \lang{de}{
    Die Summe läuft von $n=1$ bis $1$. Das bedeutet, dass sie nur aus einem einzigen Summanden 
    besteht, den wir durch Einsetzen von $n=1$ erhalten.
    }
    \lang{en}{
    The sum runs from $n=1$ to $1$. That means that it consists of only a single summand, which we 
    obtain by setting $n=1$.
    }
     \step
     \lang{de}{Also}
     \lang{en}{So}
    
    \begin{equation*}
     \sum_{n=1}^1 (3n +7) = 3\cdot 1 +7 = 10. 
    \end{equation*}
    
    
  \end{incremental}
  
 
 \tab{\lang{de}{Lösung 1.c)}\lang{en}{Solution for 1.c)}}
  
  \begin{incremental}[\initialsteps{1}]
    \step 
    \lang{de}{
    Der Laufindex $k$, der von $1$ bis $101$ läuft, kommt in der Summe nicht als Variable vor, d.\,h. 
    jeder Summand ist gleich, nämlich $3$.
    }
    \lang{en}{
    The index $k$ runs from $1$ to $10$, but does not appear in the sum as a variable. Therefore 
    every summand is $3$.
    }
     
     \step
     \lang{de}{Damit erhält man}
     \lang{en}{Hence we get}
    \begin{equation*}
       \sum\limits_{k=1}^{101} 3 = \underbrace{3+3+\cdots+3}_{101-\text{mal}} = 101\cdot 3 = 303.
    \end{equation*}

\end{incremental}

\tab{\lang{de}{Lösung 1.d)}\lang{en}{Solution for 1.d)}}
  
  \begin{incremental}[\initialsteps{1}]
    \step 
    \lang{de}{Setzen wir in die Summanden die Werte $k=1, \ldots , 5$ ein, so erhalten wir}
    \lang{en}{Substituting into the summand the values $k=1, \ldots , 5$ yields}
     
    

  \begin{equation*}
  	\sum_{k=1}^5 \left(\frac{1}{k}-\frac{1}{k+1}\right)= \left(\frac{1}{1}  \textcolor{#00CC00}{ - \frac{1}{2}}  \right) +
  	\left( \textcolor{#00CC00}{\frac{1}{2}}  \textcolor{#0066CC}{- \frac{1}{3}} \right) +
  	\left( \textcolor{#0066CC}{\frac{1}{3}}  \textcolor{#CC6600}{- \frac{1}{4}} \right) +
    \left( \textcolor{#CC6600}{\frac{1}{4}}  \textcolor{#CC00CC}{- \frac{1}{5}} \right)+
	\left( \textcolor{#CC00CC}{\frac{1}{5}} - {\frac{1}{6}} \right) = 1-\frac{1}{6} =\frac{5}{6}.
  \end{equation*}

	\lang{de}{
  Hinweis: Nebeneinanderstehende Summanden heben sich gegenseitig weg, sodass nur der erste
	und der letzte Summand übrig bleiben. Summen dieser Form nennt man \textit{Teleskopsummen}.
	\\\\
  Es gibt aber noch einen alternativen Weg, diese Summe zu berechnen.
  }
  \lang{en}{
  Note that adjacent summands cancel each other out, so that only the first and the last terms 
  remain. Sums of the form are called \textit{telescoping sums}.
  \\\\
  There is however an alternative method for evaluating this sum.
  }
    \step
    \lang{de}{
    Nach dem \ref[content_02_rechengrundlagen_terme][Kommutativgesetz]{rule:rechengesetze}
    und dem Assoziativgesetz können wir die Summe in zwei Summen aufteilen, nämlich
    }
    \lang{en}{
    By the \ref[content_02_rechengrundlagen_terme][commutativity law]{rule:rechengesetze} and the 
    associativity law we can split the sum into two different sums:
    }
  \begin{equation*}
  	\sum_{k=1}^5 \left(\frac{1}{k}-\frac{1}{k+1}\right)= 
    \left(\sum_{k=1}^5 \frac{1}{k}\right) - \left(\sum_{k=1}^5\frac{1}{k+1}\right)
  \end{equation*} 
  \step
    \lang{de}{
    Dann ziehen den ersten Summanden aus der ersten Summe heraus und verschieben 
    den Index um 1. Das ergibt
    }
    \lang{en}{
    Then we remove the first summand from the first sum and shift the index by $1$:
    }
  \begin{equation*}
  	\sum_{k=1}^5 \frac{1}{k}
    = \frac{1}{1} + \sum_{k=2}^5 \frac{1}{k}    
    = 1 + \sum_{k=1}^4 \frac{1}{k+1}
  \end{equation*} 
   \lang{de}{Bei der zweiten Summe ziehen wir den letzten Summanden heraus und erhalten}
   \lang{en}{We remove the last summand from the second sum and obtain:}
  \begin{equation*}
  	\sum_{k=1}^5\frac{1}{k+1}
    = \left(\sum_{k=1}^4\frac{1}{k+1}\right) + \frac{1}{5+1}  
    = \sum_{k=1}^4 \frac{1}{k+1} + \frac{1}{6}
  \end{equation*} 
  
  \step 
   \lang{de}{Insgesamt führt dies zu folgender Rechnung}
   \lang{en}{The full sum is therefore equal to:}
  \begin{eqnarray*}
  	    \sum_{k=1}^5 \left(\frac{1}{k}-\frac{1}{k+1}\right)
    &=&  \left(\sum_{k=1}^5 \frac{1}{k}\right) - \left(\sum_{k=1}^5\frac{1}{k+1}\right)\\
    &=& \left(1 + \sum_{k=1}^4 \frac{1}{k+1}\right) - \left(\sum_{k=1}^4 \frac{1}{k+1} + \frac{1}{6}\right)\\
    &=& 1- \frac{1}{6} \; + \cancel{\left( \sum_{k=1}^4 \frac{1}{k+1}\right)} -\cancel{\left( \sum_{k=1}^4 \frac{1}{k+1}\right)}\\
    &=& \frac{5}{6}
  \end{eqnarray*} 
  
    
\end{incremental}

%%%%%%%% Teil 2.

  \tab{\lang{de}{Lösung 2.a)}\lang{en}{Solution for 2.a)}}
  
  \begin{incremental}[\initialsteps{1}]
    \step 
    \lang{de}{
    Wir stellen fest, dass die Differenz zweier aufeinanderfolgender Faktoren stets $3$ ist.
  	Außerdem erhalten wir die Zahl $4$ als $1+3$, analog dann $7=1+2\cdot 3$ usw.
    }
    \lang{en}{
    We note that the difference between consecutive factors is $3$. 
    Hence can write $4$ as $1+3$, $7=1+2\cdot 3$ etc.
    }
    \step
    \lang{de}{Damit ist eine mögliche Schreibweise für das angegebene Produkt}
    \lang{en}{Therefore a possible way to write the product is}
    \begin{equation*}
 4\cdot 7\cdot 10\cdot 13= \prod_{j=1}^4 (3 j + 1).
\end{equation*}
    
  \end{incremental}
  

  \tab{\lang{de}{Lösung 2.b)}\lang{en}{Solution for 2.b)}}
  
  \begin{incremental}[\initialsteps{1}]
    \step 
    \lang{de}{
    In den Zählern werden die ersten vier geraden natürlichen Zahlen aufmultipliziert, welche in der 
    Form $2\cdot n$ für eine natürliche Zahl $n$ geschrieben werden können. So ist beispielsweise 
    $2=2\cdot 1$ und $4=2\cdot 2$ usw. In den Nennern werden vier ungerade Zahlen aufmultipliziert, 
    welche in der Form $2\cdot m +1$ (für eine natürliche Zahl $m$) geschrieben werden können. Es ist 
    zum Beispiel $3=2\cdot 1 +1$ und $5=2\cdot 2 +1$.
    }
    \lang{en}{
    In the numerators, the first four even natural numbers are being multiplied. These are of the 
    form $2\cdot n$ for a natural number $n$, i.e. $2=2\cdot 1$ and $4=2\cdot 2$. In the 
    denominators, four odd numbers are being multiplied. These are of the form $2\cdot m +1$ for 
    a natural number $m$, i.e. $3=2\cdot 1 +1$ and $5=2\cdot 2 +1$.
    }
     \step
     \lang{de}{Damit erhalten wir eine mögliche Schreibweise für das vorliegende Produkt durch}
     \lang{en}{Hence we have a way of representing each factor of the product using an index, and}
    
    \begin{equation*}
	\prod_{j=1}^4 \frac{2 j}{2 j + 1}.
	\end{equation*}
    
    
  \end{incremental}
  
 
 \tab{\lang{de}{Lösung 2.c)}\lang{en}{Solution for 2.c)}}
  
  \begin{incremental}[\initialsteps{1}]
    \step 
    \lang{de}{
    Es liegt ein Produkt von \textit{Quadratzahlen} vor, also von Zahlen der Form $a^2$ für natürliche
    Zahlen $a$. Beginnend beim Quadrat der Zahl $3$ multipliziert man aufsteigend bis zum
  	Quadrat der Zahl $7$ alle zwischenliegenden Quadratzahlen auf.
    }
    \lang{en}{
    This is a product of \textit{square numbers}, i.e. numbers of the form $a^2$ for a natural number 
    $a$. It is the product of the square of $3$ with the square of $7$ and every other square in 
    between.
    }
    
     
     \step
     \lang{de}{Damit ist}
     \lang{en}{Hence}
    \begin{equation*}
 	    9 \cdot 16 \cdot 25 \cdot 36 \cdot 49 = \prod_{j=3}^7 \, j^2.
	\end{equation*}
    
    \lang{de}{Diese Lösung ist aber nicht eindeutig.}
    \lang{en}{Again, this solution is not unique.}
    \step
    \lang{de}{
    Durch eine sogenannte \emph{Indexverschiebung} kann der Bereich des Laufindex
    verändert werden. Man erhält zum Beispiel
    }
    \lang{en}{
    Using a so-called \emph{index shift} we can change the domain of the index, for example
    }
    \begin{equation*}
 	    9 \cdot 16 \cdot 25 \cdot 36 \cdot 49 = \prod_{j=1}^5 \, (j+2)^2.
	\end{equation*}

    
\end{incremental}

\end{tabs*}
\end{content}