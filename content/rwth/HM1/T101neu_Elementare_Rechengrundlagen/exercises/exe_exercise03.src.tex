\documentclass{mumie.element.exercise}
%$Id$
\begin{metainfo}
  \name{
    \lang{de}{Ü03: Rechenregeln}
    \lang{en}{Ex03: Order of operations}
  }
  \begin{description} 
 This work is licensed under the Creative Commons License Attribution 4.0 International (CC-BY 4.0)   
 https://creativecommons.org/licenses/by/4.0/legalcode 

    \lang{de}{Klammer- vor Potenz- vor Punkt- vor Strichrechnung}
    \lang{en}{Order of operations}
  \end{description}
  \begin{components}
  \end{components}
  \begin{links}
     \link{generic_article}{content/rwth/HM1/T101neu_Elementare_Rechengrundlagen/g_art_content_02_rechengrundlagen_terme.meta.xml}{content_02_rechengrundlagen_terme}
  \end{links}
  \creategeneric
\end{metainfo}
\begin{content}
\begin{block}[annotation]
	Im Ticket-System: \href{https://team.mumie.net/issues/21973}{Ticket 21973}
\end{block}

\begin{block}[annotation]
Übung zur Anwendung der Klammer- vor Potenz- vor Punkt- vor Strich-Regel
und der Potenzregeln 
\end{block}

\usepackage{mumie.ombplus}

\title{
  \lang{de}{Ü03: Rechenregeln}
  \lang{en}{Ex03: Order of operations}
}

\lang{de}{Berechnen Sie die folgenden Ausdr\"ucke ohne Verwendung eines Taschenrechners:}
\lang{en}{Evaluate the following expressions without using a calculator:}

\begin{table}[\class{items}]
  \nowrap{a) $\; 2-3+2-1-5$}
  & \nowrap{b) $\; 2\cdot 3+(-2)\cdot 3+2\cdot(-3)+5\cdot6$}\\
   \nowrap{c) $\; -4+(-3): 3+2\cdot(-3)+(-9)\cdot11$} 
  & \nowrap{d) $\; 2 - (3 + 4 \cdot (-2)) $}\\
    \nowrap{e) $\; 5 \cdot 3^{8-2 \cdot 3}$}
  & \nowrap{f) $\; \Big(6^{2^3} \cdot ((-2^2)+2\cdot5)^2 \Big) \cdot \Big(\frac{1}{3}\Big)^10 $}
  
\end{table}

\begin{tabs*}[\initialtab{0}\class{exercise}]
  \tab{\lang{de}{Antworten}\lang{en}{Answers}}
\begin{table}[\class{items}]

    \nowrap{a) $\ -5$} & \nowrap{b) $\ 24$} \\
    \nowrap{c) $\ -110$} &  \nowrap{d) $\ 7$} \\
    \nowrap{e) $\ 45$} &  \nowrap{f) $\ 1024$}
  \end{table}

  \tab{\lang{de}{Lösung a)}\lang{en}{Solution for a)}}
    \lang{de}{
    Der Term enthält ausschließlich Additionen und Subtraktionen. In diesem Fall werten wir die 
    Rechenoperationen im Ausdruck einfach von links nach rechts der Reihe nach aus:
    }
    \lang{en}{
    The expression only contains addition and subtraction. In this case, we calculate each operation 
    from left to right as they appear:
    }
    \begin{equation*}\textcolor{#00CC00}{2-3}+2-1-5=\textcolor{#00CC00}{-1+2}-1-5=\textcolor{#00CC00}{1-1}-5=0-5=-5.
    \end{equation*} 
    
 
  \tab{\lang{de}{Lösung b)}\lang{en}{Solution for b)}}
    \lang{de}{
    Der Ausdruck enthält sowohl Additionen als auch Multiplikationen. In diesem Fall werden 
    zun\"achst die Multiplikationen berechnet:
    }
    \lang{en}{
    The expression contains addition and multiplication. In this case, we calculate the 
    multiplication terms first:
    } 
    \begin{equation*}
  \textcolor{#00CC00}{2\cdot 3}+\textcolor{#CC6600}{(-2)\cdot 3}+\textcolor{#0066CC}{2\cdot(-3)}+5\cdot6= 
   \textcolor{#00CC00}{6}+\textcolor{#CC6600}{(-6)}+\textcolor{#0066CC}{(-6)}+30\,.
    \end{equation*} 
     
    \lang{de}{
    Es verbleiben nur Additionen und Subtraktionen, also werten wir von links nach rechts aus und 
    erhalten
    }
    \lang{en}{
    The remaining expression only contains addition and subtraction operations, so we can evaluate 
    from left to right and obtain:}
    \begin{equation*}\textcolor{#00CC00}{6-6}-6+30=\textcolor{#00CC00}{0-6}+30=-6+30=24\,.
    \end{equation*} 
    
    \lang{de}{Das Ergebnis ist also}
    \lang{en}{The answer is therefore} 
    \begin{equation*}2\cdot 3+(-2)\cdot 3+2\cdot(-3)+5\cdot6=24\,.
    \end{equation*}
    

\tab{\lang{de}{Lösung c)}\lang{en}{Solution for c)}}
    \lang{de}{
    Weil sowohl Additionen als auch Multiplikationen und Divisionen vorkommen, 
    werden zuerst die Multiplikationen und die Divisionen ausgewertet:
    }
    \lang{en}{
    As this expression contains additions, multiplications and divisions, we first evaluate the 
    multiplication and division operators:
    }

    \begin{equation*}
  -4+\textcolor{#00CC00}{(-3): 3}+\textcolor{#CC6600}{2\cdot(-3)}+\textcolor{#0066CC}{(-9)\cdot11}=
  -4+\textcolor{#00CC00}{(-1)}+\textcolor{#CC6600}{(-6)}+\textcolor{#0066CC}{(-99)}\,.
    \end{equation*}

    \lang{de}{
    Der verbleibende Term enthält nur noch Additionen und Subtraktionen, 
    also wird das Ergebnis von links nach rechts berechnet:
    }
    \lang{en}{
    The remaining expression only contains additions and subtractions, so we can evaluate from left 
    to right and obtain:
    }
     
    \begin{equation*}
    \textcolor{#00CC00}{-4+(-1)}+(-6)+(-99)=\textcolor{#00CC00}{-5+(-6)}+(-99)=-11+(-99)=-110\,.
    \end{equation*} 
     
  
  

  \tab{\lang{de}{Lösung d)}\lang{en}{Solution for d)}}
    \lang{de}{
    Zunächst wird der Ausdruck in der Klammer berechnet. Innerhalb der Klammer wird addiert und 
    multipliziert. Wegen \glqq Punkt vor Strich\grqq wird erst die Multiplikation ausgeführt und 
    dann addiert:
    }
    \lang{en}{
    First, the expression in the parentheses is calculated. Within the brackets there is a sum and a 
    product. The multiplication is carried out first and then the addition:
    }
    \begin{equation*}
    2 - (3 + \textcolor{#00CC00}{4 \cdot (-2)}) = 2 - \textcolor{#0066CC}{(3 + (-8))} = 2 - (-5) = 2 + 5 = 7.
    \end{equation*}
    \lang{de}{
    Wir erinnern daran, dass man zwei aufeinander folgende Minuszeichen durch ein Pluszeichen 
    ersetzen kann.
    }
    \lang{en}{
    Recall that we can replace two adjacent minus signs with a plus sign.
    }
   
  
    \tab{\lang{de}{Lösung e)}\lang{en}{Solution for e)}}
    \lang{de}{
    Die Potenz wird vor der Multiplikation mit der Zahl $5$ berechnet. Um die Potenz auszurechnen, 
    gehen wir \glqq von oben nach unten\grqq vor und berechnen zuerst den Exponenten. Dort gilt 
    außerdem \glqq Punkt vor Strich\grqq:
    }
    \lang{en}{
    The power is calculated before multiplying by the number $5$. To evaluate the power, we go from 
    top to bottom and first evaluate the exponent. Within the exponent, we evaluate the 
    multiplication before the addition:
    }
    \begin{equation*}
      5 \cdot 3^{8-\textcolor{#00CC00}{2\cdot 3}} = 5 \cdot 3^{\textcolor{#0066CC}{8-6}} = 5 \cdot \textcolor{#CC6600}{3^2} = 5 \cdot 9 = 45.
    \end{equation*}         
 
    \tab{\lang{de}{Lösung f)}\lang{en}{Solution for f)}}
    \lang{de}{
    Zunächst wird der Ausdruck in der Klammer berechnet unter Berücksichtigung der 
    \ref[content_02_rechengrundlagen_terme][Klammer- vor Potenz- vor Punkt- vor Strich-Regel]{rule:punkt-vor-strich}
    Dabei ist zu beachten, dass das Minuszeichen in $\,-2^2\,$ als Vorzeichen zur \emph{gesamten} 
    Potenz gehört, und nicht zur Basis $\,2\,$, d.h. $\,-2^2=-(2^2) \neq (-2)^2.$
    }
    \lang{en}{
    Firstly we evaluate the expression in the parentheses, starting with the innermost, as required 
    by the \ref[content_02_rechengrundlagen_terme][order of operations]{rule:punkt-vor-strich}. 
    We must keep in mind that the minus sign in $\,-2^2\,$ is the sign of the \emph{power}, rather 
    than of the \emph{base} $\,2\,$, i.e. $\,-2^2=-(2^2) \neq (-2)^2$.
    }
    
    \begin{equation*}
      \Big(6^{(2^3)} \cdot \textcolor{#0066CC}{((-2^2)+2\cdot5)}^2 \Big) \cdot \Big(\frac{1}{3}\Big)^10
     = \Big(6^{(2^3)} \cdot \textcolor{#0066CC}{((-4)+10)}^2 \Big) \cdot \Big(\frac{1}{3}\Big)^10
     = \Big(6^{(2^3)} \cdot \textcolor{#0066CC}{6}^2 \Big) \cdot \Big(\frac{1}{3}\Big)^10
    \end{equation*}         

    \lang{de}{
    Der verbleibende Ausdruck enthält dann nur noch Potenzen und Multiplikationen. Bei der Berechnung der Potenzen 
    gehen wir wieder \glqq von oben nach unten\grqq vor. Bevor wir jedoch die Potenzen ausrechnen, um sie 
    anschließend zu multiplizieren, prüfen wir zunächst, ob sich die Multiplikationen der Potenzen durch Anwendung der
    \ref[content_02_rechengrundlagen_terme][Potenzregeln]{rule:potenzgesetze} vereinfachen lassen, indem wir Produkte 
    von Potenzen mit gleicher Basis oder Produkte von Potenzen mit gleichen Exponenten zusammenfassen. Dies erspart 
    das Rechnen mit großen Zahlen.
    }
    \lang{en}{
    The remaining expression only contains powers and multiplication. When evaluating the powers we 
    once more go from the top to the bottom. Having evaluated $2^3$, before evaluating $6^8$ we may 
    use a \ref[content_02_rechengrundlagen_terme][power law]{rule:potenzgesetze} to combine it with 
    the other power of $6$. This strategy avoids working with large numbers for as long as possible.
    }

    \begin{equation*}
       \textcolor{#CC6600}{\Big(6^{(2^3)} \cdot 6^2 \Big)} \cdot \Big(\frac{1}{3}\Big)^10
     = \textcolor{#CC6600}{\Big(6^{8} \cdot 6^2 \Big)} \cdot \Big(\frac{1}{3}\Big)^10 
     = \textcolor{#CC6600}{\Big(6^{8+2} \Big)} \cdot \Big(\frac{1}{3}\Big)^10
%     = \textcolor{red}{ 6^10 } \cdot \Big(\frac{1}{3}\Big)^10
     = \textcolor{#00CC00}{6^{10}  \cdot \Big(\frac{1}{3}\Big)^10}
     = \textcolor{#00CC00}{\Big( 6 \cdot \frac{1}{3}\Big)^10} = 2^10=1024
    \end{equation*}         
     


  %... other tabs

\end{tabs*}
\end{content}