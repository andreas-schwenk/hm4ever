\documentclass{mumie.element.exercise}
%$Id$
\begin{metainfo}
  \name{
     \lang{de}{Ü08: Aussagenlogik}
     \lang{en}{Ex08: Propositional logic}
  }
  \begin{description} 
 This work is licensed under the Creative Commons License Attribution 4.0 International (CC-BY 4.0)   
 https://creativecommons.org/licenses/by/4.0/legalcode 

    \lang{de}{Dies ist eine Übung zu Aussagen}
    \lang{en}{Propositions}
  \end{description}
  \begin{components}
  \end{components}
  \begin{links}
    \link{generic_article}{content/rwth/HM1/T101neu_Elementare_Rechengrundlagen/g_art_content_04_aussagen_aequivalenzumformungen.meta.xml}{content_04_aussagen_aequivalenzumformungen}
    \link{generic_article}{content/rwth/HM1/T101neu_Elementare_Rechengrundlagen/g_art_content_01_zahlenmengen.meta.xml}{content_01_zahlenmengen}
  \end{links}
  \creategeneric
\end{metainfo}
\begin{content}
\begin{block}[annotation]
	Im Ticket-System: \href{https://team.mumie.net/issues/21978}{Ticket 21978}
\end{block}

  \begin{block}[annotation]
    Neue Übung zu den Grundbegriffen der Aussagenlogik:\\
    Aussagen, logischen Verknüpfungen und Aussageformen
  \end{block}

  \title{
    \lang{de}{Ü08: Aussagenlogik}
    \lang{en}{Ex08: Propositional logic}
  }
 
  \begin{enumerate}[alph]
   \item \lang{de}{
        Handelt es sich bei folgenden Ausdrücken um Aussagen? Wenn ja, nennen Sie
        den Wahrheitswert.
        \begin{table}[\class{items}]
          \nowrap{i.  $\;$ "`Hunde sind Pflanzen."' } &
          \nowrap{ii. $\;$ "`Morgen scheint die Sonne."' }\\
          \nowrap{iii.$\;$ "`$8\,$ ist durch $\,2\,$ teilbar."'} &
          \nowrap{iv. $\;$ "`$1 + x >0$"'}
        \end{table}
        }
        \lang{en}{
        Which of the following statements are propositions? For the ones that are, what is their 
        truth-value?
        \begin{table}[\class{items}]
          \nowrap{i.  $\;$ 'Dogs are plants.' } &
          \nowrap{ii. $\;$ 'Tomorrow it will be sunny.' }\\
          \nowrap{iii.$\;$ ''$8\,$ is divisible by $\,2$.'} &
          \nowrap{iv. $\;$ '$1 + x >0$'}
        \end{table}
        }
      
    \item \lang{de}{
         Formalisieren Sie den folgenden Ausdruck im Sinne der 
         \ref[content_04_aussagen_aequivalenzumformungen][Aussagenlogik.{}]{def:objekte_aussagenlogik}
         \\\\
        \begin{table}[\class{items}]
          \nowrap{\textit{"`Heike spielt mit Armin und mit Bettina oder mit keinem von beiden."'}}
        \end{table}
         \\\\
         Bestimmen Sie hierzu die Teilaussagen, die darin enthalten sind, sowie die
         logischen Verknüfungen, die diese miteinander verbinden?
         }
         \lang{en}{
         Formalise the following statement in the sense of 
         \ref[content_04_aussagen_aequivalenzumformungen][propositional logic]{def:objekte_aussagenlogik}.
        \begin{table}[\class{items}]
          \nowrap{\textit{'Heike plays with Armin and with Bettina, or with neither of them.'}}
        \end{table}
         Determine the sub-propositions contained in the formalised version, and the logical 
         operators that relate them.
         }
   
    \item \lang{de}{
        Formulieren Sie mit den in b) identifizierten Teilaussagen
      \begin{table}[\class{items}]
        $A\,:\quad$ \textit{"`Heike spielt mit Armin."'} \\
        $B\,:\quad$ \textit{"`Heike spielt mit Bettina."'}
      \end{table}
        die folgende Aussageform in einem "`möglichst einfachen"' Satz.
        }
        \lang{en}{
        Use the sub-propositions found in b)
      \begin{table}[\class{items}]
        $A\,:\quad$ \textit{'Heike plays with Armin'} \\
        $B\,:\quad$ \textit{'Heike plays with Bettina'}
      \end{table}
        to express in sentence-form (as clearly as possible) the proposition
        }
        \[(\neg A \wedge B) \vee (\neg B \wedge A)\]   
          
  \end{enumerate}
  
  \begin{tabs*}[\initialtab{0}\class{exercise}]
    \tab{\lang{de}{Lösung a)}\lang{en}{Solution for a)}}

      \lang{de}{
      \begin{table}[\class{items}]
        \nowrap{i.  $\;$ \textit{"`Hunde sind Pflanzen."'} 
                    $\;$ ist eine \notion{\emph{falsche}} Aussage. } \\
        \nowrap{ii. $\;$ \textit{"`Morgen scheint die Sonne."' }
                    $\;$ ist \notion{\emph{keine}} Aussage, da der Wahrheitswert nicht 
                    bestimmbar ist.}\\
        \nowrap{iii.$\;$ \textit{"`$8\,$ ist durch $\,2\,$ teilbar."' }
                    $\;$ ist eine \notion{\emph{wahre}} Aussage. } \\
        \nowrap{iv. $\;$ \textit{"`$1 + x >0$"' }
                    $\;$ ist \notion{\emph{keine}} Aussage, da $\,x\,$ unbekannt ist. }
      \end{table}
      }
      \lang{en}{
      \begin{table}[\class{items}]
        \nowrap{i.  $\;$ \textit{'Dogs are plants.'} 
                    $\;$ is a \notion{\emph{false}} proposition. } \\
        \nowrap{ii. $\;$ \textit{'Tomorrow it will be sunny.'}
                    $\;$ is \notion{\emph{not}} a valid proposition, as its truth-value cannot be 
                    determined.}\\
        \nowrap{iii.$\;$ \textit{'$8\,$ is divisible by $\,2$.'}
                    $\;$ is a \notion{\emph{true}} proposition.} \\
        \nowrap{iv. $\;$ \textit{'$1 + x >0$' }
                    $\;$ is \notion{\emph{not}} a valid proposition, as $\,x\,$ is unknown.}
      \end{table}
      }


    \tab{\lang{de}{Lösung b)}\lang{en}{Solution for b)}}
     \begin{incremental}[\initialsteps{1}]
      \step 
      \lang{de}{
      Die sogenannten \notion{\emph{Ojekte der Aussage $\,$}} sind gegeben durch die 
      \notion{\emph{Teilaussagen}}
      \\\\
      \begin{table}[\class{items}]
        $A\,:\quad$ \textit{"`Heike spielt mit Armin."'} \\
        $B\,:\quad$ \textit{"`Heike spielt mit Bettina."'}
      \end{table}
      }
      \lang{en}{
      The sub-propositions are 
      \\\\
      \begin{table}[\class{items}]
        $A\,:\quad$ \textit{'Heike plays with Armin'} \\
        $B\,:\quad$ \textit{'Heike plays with Bettina'}
      \end{table}
      }
      
      \step 
      \lang{de}{
      Die gesamte Aussage sagt aus, dass Heike entweder mit Armin \textit{und} mit 
      Bettina spielt, \textit{oder} dass Heike mit keinem von beiden, also 
      \textit{nicht} mit Armin \textit{und} auch \textit{nicht} mit Bettina spielt.
      Die Teilaussagen sind also verbunden durch die \notion{\emph{Junktoren}} 
      \emph{"`Und"'} $\,\wedge\,$,  \emph{"`Oder"'} $\, \vee \,$ und die \emph{Negation}
      $\, \neg \,$.  Formal also:
      }
      \lang{en}{
      The complete proposition states that Heike either plays with Armin \textit{and} with Bettina, 
      \textit{or} Heinke plays with neither of them, i.e. \textit{not} with Armin \textit{and} 
      \textit{not} with Bettina. The statements are therefore related by \notion{\emph{conjunction}} 
      and \notion{\emph{disjunction}} operators, that is, \emph{'and'} and \emph{'or'} operators, 
      as well as the \emph{negation} operator $\, \neg \,$. Formally, the statement is written as 
      follows:
      
      }
      
      \[ (A \wedge B) \vee (\neg A \wedge \neg B) \] 
      
     \end{incremental}
     
    \tab{\lang{de}{Lösung c)}\lang{en}{Solution for c)}}
      
      \lang{de}{
      Für die Aussagen
      \begin{table}[\class{items}]
        $A\,:\quad$ \textit{"`Heike spielt mit Armin."'} $\;$ und\\
        $B\,:\quad$ \textit{"`Heike spielt mit Bettina."'}
      \end{table}
      bedeuten die Verknüpfungen in Worte gefasst:
      \begin{table}[\class{items}]
        $\neg A \wedge B \,:\quad$ \textit{"`Heike spielt nicht mit Armin, aber mit Bettina."'}  $\;$ und\\
        $\neg B \wedge A \,:\quad$ \textit{"`Heike spielt nicht mit Bettina, aber mit Armin."'} 
      \end{table}
      und schließlich $\; (\neg A \wedge B) \vee (\neg B \wedge A) \,: $
      \\\\
      \textit{"`Heike spielt nicht mit Armin, aber mit Bettina,
      oder Heike spielt nicht mit Bettina, aber mit Armin."'}
      \\\\
      Kürzer ausgedrückt:
      \\\\
      \textit{"`Heike spielt mit Armin oder Bettina, aber nicht mit beiden."'}
      }
      \lang{en}{
      Given the propositions
      \begin{table}[\class{items}]
        $A\,:\quad$ \textit{'Heike plays with Armin.'} $\;$ and\\
        $B\,:\quad$ \textit{'Heike plays with Bettina.'}
      \end{table}
      the propositions in the parentheses are 
      \begin{table}[\class{items}]
        $\neg A \wedge B \,:\quad$ \textit{'Heike does not play with Armin, but does play with Bettina.'}  $\;$ and\\
        $\neg B \wedge A \,:\quad$ \textit{'Heike does not play with Bettina, but does play with Armin.'} 
      \end{table}
      and the full proposition $\; (\neg A \wedge B) \vee (\neg B \wedge A) \,$ is:
      \\\\
      \textit{'Heike does not play with Armin, but does play with Bettina, or Heike does not play 
      with Bettina, but does play with Armin.'}
      \\\\
      Written in a simpler way:
      \\\\
      \textit{'Heike plays with Armin or Bettina, but not with both.'}
      }

  \end{tabs*}

\end{content}

