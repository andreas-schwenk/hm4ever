\documentclass{mumie.element.exercise}
%$Id$
\begin{metainfo}
  \name{
    \lang{de}{Ü07: Bruchterme}
    \lang{en}{Ex07: Fractional expressions}
  }
  \begin{description} 
 This work is licensed under the Creative Commons License Attribution 4.0 International (CC-BY 4.0)   
 https://creativecommons.org/licenses/by/4.0/legalcode 

    \lang{de}{Bruchterme vereinfachen}
    \lang{en}{Simplifying expressions containing fractions of variables}
  \end{description}
  \begin{components}
  \end{components}
  \begin{links}
\link{generic_article}{content/rwth/HM1/T101neu_Elementare_Rechengrundlagen/g_art_content_02_rechengrundlagen_terme.meta.xml}{content_02_rechengrundlagen_terme}
\end{links}
  \creategeneric
\end{metainfo}
\begin{content}

\begin{block}[annotation]
	Im Ticket-System: \href{https://team.mumie.net/issues/21977}{Ticket 21977}
\end{block}

\begin{block}[annotation]
  Bruchterme vereinfachen
     
\end{block}


\usepackage{mumie.ombplus}

\title{
  \lang{de}{Ü07: Bruchrterme}
  \lang{en}{Ex07: Fractional expressions}
}


\lang{de}{Vereinfachen Sie die folgende Terme soweit wie möglich:}
\lang{en}{Simplify the following expressions as much as possible:}

\begin{table}[\class{items}]
  \nowrap{a) $\, \displaystyle{\big(\frac{1}{x}+\frac{1}{x^2}\big): \frac{1}{x}}$} &
  \nowrap{b) $\, \displaystyle{\Big(-b+a-\frac{b^2}{a-b} \Big) \cdot  \Big(-\frac{1}{a}+\frac{1}{2b-a} \Big)}$}\\
  \nowrap{c) $\, \displaystyle{\Big(\frac{x-2y}{y^2} + \frac{1}{x}\Big) : \Big(\frac{x-y}{xy^2}\Big) }$} &
  \nowrap{d) $\, \displaystyle{\Big(\frac{7x^2+35x}{7xy+14x^2} \Big) + \Big(\frac{y^2-x^2}{(4x+2y)\cdot(x+y)} \Big)}$}\\
\end{table}

\begin{tabs*}[\initialtab{0}\class{exercise}]
  \tab{\lang{de}{Antworten}\lang{en}{Answers}}
\begin{table}[\class{items}]
    \nowrap{a) $\, \displaystyle{\frac{x+1}{x}} $} &
    \nowrap{b) $\, \displaystyle{-2}          $} \\
    \nowrap{c) $\, \displaystyle{x-y}         $} &
    \nowrap{d) $\, \displaystyle{\frac{x +y + 10}{2 \cdot (2x+y)}} $}
  \end{table}

  \tab{\lang{de}{Lösung a)}\lang{en}{Solution for a)}}
  
    \lang{de}{
    Beim Rechnen mit Bruchtermen geht man genauso vor wie bei Brüchen. 
    Man berechnet zuerst die Summe in der Klammer, wofür die
    beiden Brüche erst auf einen gemeinsamen Nenner gebracht werden müssen:
    }
    \lang{en}{
    Simplifying these expressions is done in a very similar way to simplifying expressions 
    that do not contain variables. We firstly evaluate the sum in the parentheses, for 
    which we require the two fractions to have the same denominator:
    }
    \[ \frac{1}{x}+\frac{1}{x^2}=\frac{x}{x^2}+\frac{1}{x^2}=\frac{x+1}{x^2}.\]
    \lang{de}{Man dividiert durch einen Bruch, indem mit seinem Kehrwert multipliziert wird:}
    \lang{en}{We divide by a fraction by multiplying by its reciprocal:}
    \[\big(\frac{1}{x}+\frac{1}{x^2}\big): \frac{1}{x}=\frac{x+1}{x^2}\cdot \frac{x}{1}
    =\frac{(x+1)x}{x^2\cdot 1}=\frac{(x+1)\cdot \cancel{x}}{x\cdot \cancel{x}\cdot 1}
    =\frac{x+1}{x}.\]
 
\tab{\lang{de}{Lösung b)}\lang{en}{Solution for b)}}
  
  \begin{incremental}[\initialsteps{1}]
    \step 
    \lang{de}{
    Die Unbekannten in einem Term können mit beliebigen Buchstaben des Alphabets 
    bezeichnet werden, so in diesem Fall mit $\,a$\, und $\,b\,$. Wir bringen zunächst 
    wieder die Summanden in den Klammern jeweils auf ihren Hauptnenner und fassen dann 
    die Zähler-Summe gemäß den Regeln für Termumformungen so weit wie möglich zusammen.
    }
    \lang{en}{
    The variables in an expression may be expressed using any letter of the alphabet, in this case 
    with $\,a$\, and $\,b\,$. Firstly we write the sums in the parentheses with a common denominator, 
    and combine as many terms as possible in the resulting sum of the numerators.
    }
    \begin{eqnarray*}
         \Big(-b+a-\frac{b^2}{a-b} \Big) \cdot  \Big(\frac{(-1)}{a}+\frac{1}{2b-a} \Big)
     &=& \frac{-b\cdot (a-b) + a\cdot (a-b) - b^2}{a-b} \cdot \frac{-(2b-a) + a}{a \cdot (2b-a)}\\
     &=& \frac{-ba + b^2 + a^2-ab - b^2}{a-b} \cdot  \frac{-2b + 2a}{2ab-a^2} \\
     &=& \frac{a^2 -2ab \cancel{+ b^2} \cancel{- b^2}}{a-b} \cdot  \frac{-2b + 2a}{2ab-a^2}\\
     &=& \frac{a^2 -2ab}{a-b} \cdot  \frac{-2b + 2a}{2ab-a^2}
    \end{eqnarray*}  

    \lang{de}{
    Nun suchen wir nach gemeinsamen Faktoren in Zähler und Nenner, um den Bruchterm 
    so weit wie möglich zu kürzen.
    }
    \lang{en}{
    Now we look for common factors in the numerator and denominator, in order to simplify the 
    fraction as much as possible.
    }
   
    \step
    
    \begin{eqnarray*}
        
        \displaystyle{\frac{(a^2 -2ab) \cdot (-2b + 2a)}{(a-b)\cdot (2ab-a^2)} }        
     &=& \displaystyle{\frac{(-1)\cdot (2ab-a^2) \; \cdot \; 2 \cdot (a-b)}{(a-b)\cdot (2ab-a^2)} }\\
     &=& -2 \cdot \frac{\cancel{(2ab-a^2)}\cdot \cancel{(a-b)} }{\cancel{(a-b)} \cdot \cancel{(2ab-a^2)}} \\
     &=& -2 
     \end{eqnarray*}
 
        
  \end{incremental}


\tab{\lang{de}{Lösung c)}\lang{en}{Solution for c)}}
    
  \begin{incremental}[\initialsteps{1}]
    \step 
        \lang{de}{
        Wir bringen zunächst den Term in der Klammer auf den Hauptnenner und 
        multiplizieren das Ergebnis mit dem Kehrwert des Terms im Divisor.
        }
        \lang{en}{
        Firstly we write the sum in the parentheses with a common denominator, then we 
        multiply the result with the reciprocal of the fraction being divided by,
        }
       \begin{eqnarray*}
          \Big(\frac{x-2y}{y^2} + \frac{1}{x}\Big) : \Big( \frac{x-y}{xy^2} \Big)
        &=& \Big(\frac{x \cdot (x-2y) + y^2}{xy^2} \Big) \cdot \Big( \frac{xy^2}{x-y} \Big)\\
        &=& \Big(\frac{x^2 -2xy + y^2}{xy^2} \Big) \cdot \Big( \frac{xy^2}{x-y} \Big)
        \end{eqnarray*}
    
    \step 
      \lang{de}{
      Der Zähler des ersten Bruchterms ist mittels der 
      \ref[content_02_rechengrundlagen_terme][2. binomischen Formel]{rule:binomische_formeln} 
      faktorisierbar. Zudem kann der Zähler des zweiten Bruchterms gegen den Nenner des 
      des ersten Bruchterms gekürzt werden. Wir erhalten daher
      }
      \lang{en}{
      The numerator of the first fraction is, using the second 
      \ref[content_02_rechengrundlagen_terme][binomial formula]{rule:binomische_formeln}, 
      factorisable. Furthermore, the numerator of the second fraction can be cancelled with the 
      denominator of the first fraction, giving us
      }
        \begin{equation*}
          \frac{x^2 -2xy + y^2}{\cancel{xy^2}} \cdot \frac{\cancel{xy^2}}{x-y} \\
        = \frac{(x-y)^2}{(x-y)}=x-y
        \end{equation*}
    
 \end{incremental}

\tab{\lang{de}{Lösung d)}\lang{en}{Solution for d)}}
  
  \begin{incremental}[\initialsteps{1}]
    \step 
    \lang{de}{
    Wir betrachten zunächst den Term in der ersten Klammer. Um diesen zu vereinfachen, müssen wir
    Zähler und Nenner jeweils in geeignete Faktoren zerlegen, man nennt das auch \emph{"`faktorisieren"'}, 
    da man aus Summen bekanntermaßen nicht kürzen darf. In diesem Fall nutzen wir das Distributivgesetz, 
    um möglichst viele gleiche Faktoren aus den Summen in Zähler und Nenner auszuklammern, um
    schließlich hierdurch zu kürzen.
    }
    \lang{en}{
    We firstly consider the expression in the first set of parentheses. To simplify this, we must 
    first write its numerator and denominator in terms of their factors, i.e. \emph{'factorise'} 
    them. Currently they are both sums, and we cannot cancel by anything that does not divide both 
    summands. We use the distributivity law to seperate the sums in the numerator and denominator 
    into their factors.
    }
    \[
      \frac{7x^2+35x}{7xy+14x^2}=\frac{7x \cdot (x + 5)}{7x \cdot (y+2x)}=
      \frac{\cancel{7x} \cdot (x + 5)}{\cancel{7x} \cdot (y+2x)}=\frac{x + 5}{y+2x}
    \]

    \step 
    \lang{de}{
    Bei dem Term in der zweiten Klammer ist der Nenner bereits faktorisiert. 
    Zur Faktorisierung des Zählers verwenden wir die 
    \ref[content_02_rechengrundlagen_terme][3. binomische Formel.]{rule:binomische_formeln}
    }
    \lang{en}{
    The expression in the second set of parentheses already has a factorised denominator. To 
    factorise the numerator, we apply the third 
    \ref[content_02_rechengrundlagen_terme][binomial formula]{rule:binomische_formeln}.
    }
   \[
      \frac{y^2-x^2}{(4x+2y)\cdot(x+y)}=\frac{(y+x) \cdot (y-x)}{(4x+2y)\cdot(x+y)}
    \]
    \lang{de}{Wir können nun durch den Faktor $\,(x+y)\,$ kürzen und erhalten}
    \lang{en}{We may now cancel by the factor $\,(x+y)\,$ and obtain}
   \[
      \frac{\cancel{(x+y)} \cdot (y-x)}{(4x+2y)\cdot \cancel{(x+y)}}
      =\frac{y-x}{4x+2y}
    \]

    \step 
    \lang{de}{
    Setzten wir die berechneten vereinfachten Klammer-Terme in den Ausgangsterm ein,
    so erhalten wir
    }
    \lang{en}{
    Substituting the simplified fractions into the original expression yields
    }
    \[
      \Big(\frac{7x^2+35x}{7xy+14x^2} \Big) + \Big(\frac{4x+2y}{x^2-y^2} \Big)
      =\frac{x + 5}{y+2x} + \frac{y-x}{4x+2y }
    \]
    \lang{de}{Wir bringen nun beide Summanden auf den Hauptnenner und kommen zu folgendem Ergebnis}
    \lang{en}{Now we write both terms with a common denominator and arrive at the following result:}
    \[
      \frac{2 \cdot (x + 5) + (y-x)}{2 \cdot (2x+y)}
      =\frac{x +y + 10}{2 \cdot (2x+y)}
    \]
        
  \end{incremental}


  %... other tabs
\end{tabs*}
\end{content}