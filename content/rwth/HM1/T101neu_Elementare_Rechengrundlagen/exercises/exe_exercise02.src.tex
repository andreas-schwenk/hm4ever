\documentclass{mumie.element.exercise}
%$Id$
\begin{metainfo}
  \name{
     \lang{de}{Ü02: Intervalle}
     \lang{en}{Ex02: Intervals}
  }
  \begin{description} 
 This work is licensed under the Creative Commons License Attribution 4.0 International (CC-BY 4.0)   
 https://creativecommons.org/licenses/by/4.0/legalcode 

    \lang{de}{Dies ist eine Übung zu Intervallen, Anordnungen}
    \lang{de}{This exercise is on intervals}
  \end{description}
  \begin{components}
    \component{generic_image}{content/rwth/HM1/images/g_tkz_T101_Exercise02_G.meta.xml}{T101_Exercise02_G}
    \component{generic_image}{content/rwth/HM1/images/g_tkz_T101_Exercise02_F.meta.xml}{T101_Exercise02_F}
    \component{generic_image}{content/rwth/HM1/images/g_tkz_T101_Exercise02_E.meta.xml}{T101_Exercise02_E}
    \component{generic_image}{content/rwth/HM1/images/g_tkz_T101_Exercise02_D.meta.xml}{T101_Exercise02_D}
    \component{generic_image}{content/rwth/HM1/images/g_tkz_T101_Exercise02_C.meta.xml}{T101_Exercise02_C}
    \component{generic_image}{content/rwth/HM1/images/g_tkz_T101_Exercise02_B.meta.xml}{T101_Exercise02_B}
    \component{generic_image}{content/rwth/HM1/images/g_tkz_T101_Exercise02_A.meta.xml}{T101_Exercise02_A}
  \end{components}
  \begin{links}
    \link{generic_article}{content/rwth/HM1/T101neu_Elementare_Rechengrundlagen/g_art_content_02_rechengrundlagen_terme.meta.xml}{content_02_rechengrundlagen_terme}
    \link{generic_article}{content/rwth/HM1/T101neu_Elementare_Rechengrundlagen/g_art_content_01_zahlenmengen.meta.xml}{content_01_zahlenmengen}
  \end{links}
  \creategeneric
\end{metainfo}
\begin{content}
\begin{block}[annotation]
	Im Ticket-System: \href{https://team.mumie.net/issues/21972}{Ticket 21972}
\end{block}

  \begin{block}[annotation]
%
   Übung zu Intervallen, Anordnungen und Beträgen, Betragstermen und Wurzeltermen

  \end{block}

\title{
    \lang{de}{Ü02: Intervalle}
    \lang{en}{Ex02: Intervals}
  }
 
%
% Aufgabenstellung
%
\begin{enumerate}[alph] 

  \item \lang{de}{
        Was ist der Unterschied zwischen der Menge $\,\{2; 5\}\,$ und dem Intervall $\,[2; 5)$?
        }
        \lang{en}{
        What is the difference between the set $\,\{2; 5\}\,$ and the interval $\,[2; 5)$?
        }
  
  \item \lang{de}{
        Schreiben Sie die folgenden Mengen in Intervallschreibweise (bzw. als Vereinigung von 
        Intervallen) und zeichnen Sie sie auf einem Zahlenstrahl ein:
        }
        \lang{en}{
        Write the following sets using interval notation (or as a union of intervals) and represent 
        them on a number line:
        }
     \begin{enumerate}      
        \item[(i)] $A=\{x \in \R \mid \abs{x-1} \leq 1 \}$
        \item[(ii)] $B=\{x \in \R \mid x^2 \geq 9 \}$
     \end{enumerate} 
     
   \item \lang{de}{
         Ist der Durchschnitt zweier abgeschlossener Intervalle wieder ein Intervall?
         Wie verhält es sich mit der Vereinigung zweier Intervalle?
         }
         \lang{en}{
         Is the intersection of two closed intervals also an interval?
         What about the union of two intervals?
         }
           
%
% Lösungen
%
  
\end{enumerate}   
  \begin{tabs*}[\initialtab{0}\class{exercise}]
  
    \tab{\lang{de}{Lösung a)}\lang{en}{Solution for a)}}
        \lang{de}{
        Die Menge $\, \{2; 5\} \,$ besteht nur aus zwei Zahlen, nämlich der Zahl $2$ und der Zahl $5$.
        \\\\
        Das \ref[content_01_zahlenmengen][Intervall]{def:intervall} $\,[2; 5)\,$ entspricht 
        definitionsgemäß der Menge $\, \{\,x\in\R\,|\,2 \leq x <5 \,\}\,$ und enthält daher 
        \emph{alle} Zahlen zwischen $2$ und $5$ einschließlich der Zahl $2$, jedoch ohne die $5$.
        }
        \lang{en}{
        The set $\, \{2; 5\} \,$ contains only two numbers, namely $2$ and $5$.
        \\\\
        The \ref[content_01_zahlenmengen][interval]{def:intervall} $\,[2; 5)\,$ is defined to be the 
        set $\, \{\,x\in\R\,|\,2 \leq x <5 \,\}\,$ and thus contains \emph{all} numbers between $2$ 
        and $5$, including $2$ but not including $5$.
        }
     

    \tab{\lang{de}{Lösung b)}\lang{en}{Solution for b)}}
      \begin{incremental}[\initialsteps{1}]
       \step 
        %\begin{enumerate}[roman]       
         %
      \textbf{(i)} 
            \lang{de}{
            Wir erinnern daran, dass der Absatnd zweier reeller Zahlen $a, b$ gegeben wird durch 
            $|b-a|$. Die Menge $A=\{x \in \R \mid \abs{x-1} \leq 1 \}$ ist somit genau die Menge der 
            Zahlen, die von der Zahl $1$ höchstens den Abstand eins haben.  
            }
            \lang{en}{
            Recall that the 'distance' bwtween two real numbers $a, b$ on the number line is given by 
            $|b-a|$. The set $A=\{x \in \R \mid \abs{x-1} \leq 1 \}$ is therefore exactly the set of 
            all numbers which have at most distance $1$ from the number $1$.
            }
         
         \\
        \begin{center}
            \image{T101_Exercise02_A}
        \end{center}
        
        \lang{de}{
        Dies liefert die \ref[content_01_zahlenmengen][Intervalldarstellung]{rem:intervall} 
        $\; A=[0;2]$.
        }
        \lang{en}{
        Hence $A$ can be represented as an \ref[content_01_zahlenmengen][interval]{rem:intervall} 
        $\; A=[0;2]$.
        }
        \step
        \lang{de}{
        Natürlich könnte man hier auch die Betragsungleichung auflösen, um zum selben Ergebnis zu kommen:
         Für den \ref[content_01_zahlenmengen][Betrag]{def:intervall} $\,\abs{x-1}\,$ gilt per Definition
         }
         \lang{en}{
         An alternative method of finding this is by solving the absolute value equation, which 
         brings us to the same solution. The 
         \ref[content_01_zahlenmengen][absolute value]{def:intervall} $\,\abs{x-1}\,$ is defined to be
         }
         \[
            {\abs{x-1}} = 
            \begin{cases}
            \, (x-1) & \text{\lang{de}{falls}\lang{en}{if}} \; 
            (x-1) \geq 0 \quad (\text{\lang{de}{also}\lang{en}{i.e.}} \; x \geq 1 )\\
              -(x-1) & \text{\lang{de}{falls}\lang{en}{if}} \; 
              (x-1) < 0    \quad (\text{\lang{de}{also}\lang{en}{i.e.}} \; x < 1 ).
            \end{cases}
         \]
         \lang{de}{Daher folgt:}
         \lang{en}{Hence:}
         \[
            {\abs{x-1} \leq 1} \Leftrightarrow 
            \begin{cases}
            \, (x-1) \leq 1 & \Leftrightarrow \; x \leq 2 \; \wedge \;   x \geq 1\\
              -(x-1) \leq 1 & \Leftrightarrow \; 0 \leq x \; \wedge \;   x < 1 ,
            \end{cases}
        \]
        \[
            \Rightarrow \; A=\{x \in \R \mid 0 \leq x <1 \; \vee \; 1 \leq x \leq 2 \}=\{x \in \R \mid 0 \leq x \leq 2 \}.
        \]
  
      \step

        \textbf{(ii)}
         \lang{de}{
         Schauen wir uns nun die Menge $\;B=\{x \in \R \mid x^2 \geq 9 \}\,$ an.
         Die \ref[content_02_rechengrundlagen_terme][Wurzel]{def:wurzel}
         aus $\,x^2\,$ ist definiert als die eindeutige nichtnegative Zahl, 
         die mit sich selbst multipliziert wieder $\,x^2\,$ ergibt. Daher gilt
         $\, \sqrt{x^2}=\abs{x}\;$ und wir können auch schreiben
         }
         \lang{en}{
         Now let us consider the set $\;B=\{x \in \R \mid x^2 \geq 9 \}\,$.
         The \ref[content_02_rechengrundlagen_terme][square root]{def:wurzel} of 
         $\,x^2\,$ is defined to be the unique non-negative number that, when 
         multiplied by itself, gives $\,x^2\,$. Hence $\, \sqrt{x^2}=\abs{x}\;$ 
         and we can write
         }
         \[ B=\{x \in \R \mid \abs{x} \geq 3 \}.\]
     \step 
     \lang{de}{
     Die Menge $B$ besteht somit aus genau den Zahlen, die mindestens den Abstand drei zur Null haben:
     }
     \lang{en}{
     The set $B$ thus contains exactly the numbers which have at least distance $3$ from $0$ on the 
     number line:
     }
      \\
      
        \begin{center}
            \image{T101_Exercise02_B}
        \end{center}
    \\

%         Analog zur Teilaufgabe i. lösen wir nun den Betrag in der beschreibenden 
%         Darstellung der Menge $\,B\,$ auf und erhalten 
%          \[
%             {\abs{x}} \geq 3 \Leftrightarrow
%             \begin{cases}
%             \; x \geq \; 3 & \quad \text{falls} \;  x \geq 0 \\
%             \; x \leq   -3 & \quad \text{falls} \;   x < 0,  
%             \end{cases}
%          \]
%          \[\Rightarrow \quad  B=\{x \in \R \mid x \leq -3 \;(<0)\; \vee x \geq 3 \; (\geq 0) \}. \]
     \lang{de}{
     Die Menge $B$ ist also die Verenigung zweier 
     \ref[content_01_zahlenmengen][einseitig unbeschränkter Intervalle]{rem:intervall},
     }
     \lang{en}{
     The set $B$ is therefore the union of two 
     \ref[content_01_zahlenmengen][unbounded intervals]{rem:intervall},
     }
     \[B= \{x \in \R \mid x \leq -3 \}\; \cup \; \{x \in \R \mid x \geq 3 \}= (-\infty;-3] \cup [3;\infty).\]
     \lang{de}{Weil es eine "`Lücke"' zwischen $-3$ und $3$ gibt, ist $B$ kein Intervall.}
     \lang{en}{Because there is a 'gap' between $-3$ and $3$, $B$ is not an interval.}
     \end{incremental}

    \tab{\lang{de}{Lösung c)}\lang{en}{Solution for c)}} 
     \begin{incremental}[\initialsteps{1}]
      \step 
        \lang{de}{
        Betrachten wir zwei beliebige abgeschlossene Intervalle $\, [a;b] \,$ und $\, [c;d] \,$
        mit $\, a, b, c, d \in \R$, $a \leq b$ und $c \leq d$. 
%         Außerdem müssen
%         $\, d \geq a\,$ und $\, b \geq c $ sein, da andernfalls 
%         $\; [a;b] \cap [c;d] = \emptyset .$
        Das Aussehen von Durchschnitt und Vereinigung dieser Intervalle hängt von der Lage der 
        Intervallgrenzen zueinander auf dem Zahlenstrahl ab:
        }
        \lang{en}{
        Consider two arbitrary closed intervals $\, [a;b] \,$ and $\, [c;d] \,$ with 
        $\, a, b, c, d \in \R$, $a \leq b$ and $c \leq d$. 
        The appearance of the intersection and union of these intervals depends on the relationship 
        between the interval endpoints on the number line:
        }
     \step
           \begin{enumerate}
           \item[\lang{de}{1. Fall:}\lang{en}{Case 1:}]
        \lang{de}{Es gilt entweder $d<a$ oder $b<c$.}
        \lang{en}{Either $d<a$ or $b<c$.}
        \\\\
         \begin{center}
            \image{T101_Exercise02_C}
        \end{center}
        \\
       \lang{de}{
       Hier ist der Durchschnitt leer, $[a;b]\cap[c;d]=\emptyset$. 
       Die Vereinigung $[a;b]\cup[c;d]$ ist kein Intervall, denn es gibt Zahlen zwischen den beiden Teilintervallen. 
       (Im Fall $d<a$ ist zum Beispiel die Mitte $\frac{d+a}{2}$ zwischen $d$ und $a$ eine solche, anlog $\frac{b+c}{2}$ im Fall $b<c$.)
       }
       \lang{en}{
       The intersection is empty in this case, $[a;b]\cap[c;d]=\emptyset$. 
       The union $[a;b]\cup[c;d]$ is not an interval, as there exist numbers between the two 
       subintervals. 
       (For example, if $d<a$ then the point $\frac{d+a}{2}$ between $d$ and $a$ is such a number, 
       and if $b<c$, the number $\frac{b+c}{2}$ is between $b$ and $c$.)
       }
      
%         Unter diesen Bedingungen gibt es vier verschiedene Fälle zu betrachten.
%         Dabei fixieren wir gedanklich das Intervall $\, [a;b] \,$ auf dem 
%         Zahlenstrahl und untersuchen in Relation hierzu die verschiedenen Lagen
%         des Intervalls $\, [c;d].$
%         \\
          \end{enumerate}
    \step 
       \begin{enumerate} 
       \item[\lang{de}{2. Fall:}\lang{en}{Case 2:}]
         \lang{de}{
         Es gilt $c\leq a,b\leq d$ oder $a\leq c,d\leq b$, das heißt ein Intervall ist ein 
         Teilintervall des anderen.
         }
         \lang{en}{
         Either $c\leq a,b\leq d$ or $a\leq c,d\leq b$. 
         One interval is contained in the other interval in this case.
         }
                        \\\\
                       \begin{center}
                         \image{T101_Exercise02_D}
                      \end{center}
                      
      
%         $\,c\,$ liegt außerhalb des Intervalls $\, [a;b],$ also $\, c < a,$\\
%                        und $\,d\,$ liegt innerhalb, also $\, a \leq d \leq b \,$ \\
 
%                       \\
%                        \begin{center}
%                           \image[300]{intervall_c1}
%                       \end{center}
%                       \\
                      \\
                       \begin{center}
                         \image{T101_Exercise02_E}
                      \end{center}
                      \\
         \lang{de}{
         In diesem Fall ist der Duchschnitt der beiden Intervall das innere Intervall und die Vereinigung das äußere.
         Z.B. ist für $c\leq a,b\leq d$ also $[a;b]\cap[c;d]=[a;b]$ und $[a;b]\cup[c;d]=[c;d]$.
         }
         \lang{en}{
         In this case the intersection of the two intervals is the inner interval, and the union is 
         the outer interval. For example, if $c\leq a,b\leq d$ then $[a;b]\cap[c;d]=[a;b]$ and 
         $[a;b]\cup[c;d]=[c;d]$.
         }
         \end{enumerate}
    \step 
       \begin{enumerate}
        \item[\lang{de}{3. Fall:}\lang{en}{Case 3:}]
             \lang{de}{
             Genau eine Intervallgrenze liegt im anderen Intervall, es gilt also entweder 
             $a\leq c\leq b<d$ oder $c<a\leq d\leq b$. Also entweder:
             }
             \lang{en}{
             Exactly one of the endpoints of each interval lies in the other interval. 
             In this case we have either $a\leq c\leq b<d$ or $c<a\leq d\leq b$, so either:
             }
                       
                       \\
                       \begin{center}
                          \image{T101_Exercise02_G}
                      \end{center}
             \lang{de}{Hier ist $[a;b]\cap[c;d]=[c;b]$ und $[a;b]\cup [c;d]=[a;d]$. Oder:}
             \lang{en}{Here we have $[a;b]\cap[c;d]=[c;b]$ and $[a;b]\cup [c;d]=[a;d]$. Or:}
                      
                      \\
                       \begin{center}
                          \image{T101_Exercise02_F}
                      \end{center}
             \lang{de}{Hier ist $[a;b]\cap[c;d]=[a;d]$ und $[a;b]\cup [c;d]=[c;b]$.}
             \lang{en}{Here we have $[a;b]\cap[c;d]=[a;d]$ and $[a;b]\cup [c;d]=[c;b]$.}
      

\end{enumerate}
\lang{de}{
Zusammenfassend bemerken wir also, dass der Durchschnitt abgeschlossener Intervalle entweder wieder 
ein Intervall ist oder leer. Ist der Durchschnitt leer, dann ist die Vereinigung kein Intervall 
(sondern Vereinigung unzusammenhängender Intervalle). Ist der Durchschnitt ein Intervall, dann auch 
die Vereinigung.
}
\lang{en}{
To conclude we remark that the intersection of two intervals is either an interval itself, or empty. 
If the intersection is empty, then the union is not an interval (rather it is the union of 
disjoint intervals). If the intersection is an interval, then the union is also an interval.
}
   \end{incremental}
  \end{tabs*}
  
\end{content}

