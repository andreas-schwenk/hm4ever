\documentclass{mumie.element.exercise}
%$Id$
\begin{metainfo}
  \name{
    \lang{de}{Ü14: Lineare Gleichungssysteme}
    \lang{en}{Ex14: Linear Systems}
  }
  \begin{description} 
 This work is licensed under the Creative Commons License Attribution 4.0 International (CC-BY 4.0)   
 https://creativecommons.org/licenses/by/4.0/legalcode 

    \lang{de}{Hier die Beschreibung}
    \lang{en}{Solving linear systems}
  \end{description}
  \begin{components}
  \end{components}
  \begin{links}
  \end{links}
  \creategeneric
\end{metainfo}
\begin{content}
\begin{block}[annotation]
	Im Ticket-System: \href{https://team.mumie.net/issues/21984}{Ticket 21984}
\end{block}

\begin{block}[annotation]
	Lösen linearer Gleichungssysteme mit Additions- und Einsetzungsverfahren
\end{block}

\title{
  \lang{de}{Ü14: Lineare Gleichungssysteme}
  \lang{en}{Ex14: Linear Systems}
}

%
% Aufgabe
%
\lang{de}{Lösen Sie die linearen Gleichungssysteme mit dem Additions- oder dem Einsetzungsverfahren.}
\lang{en}{Solve the following linear systems using either the addition or substitution method.}

\begin{table}[\class{items}]
  \nowrap{a) 
    \begin{displaymath}
    \begin{mtable}[\cellaligns{ccrcrcr}]
    \text{(I)}&\qquad&x&+&3y&=&7\\
    \text{(II)}&\qquad&-x&-&2y&=&-5
    \end{mtable}
    \end{displaymath}
  }\\
  \nowrap{b) 
    \begin{displaymath}
    \begin{mtable}[\cellaligns{ccrcrcr}]
    \text{(I)}&\qquad&7x&-&6y&=&27\\
    \text{(II)}&\qquad&5x&-&4y&=&19
    \end{mtable}
    \end{displaymath}
  }
\end{table}
%
%
% Lösungen
%
\begin{tabs*}[\initialtab{0}\class{exercise}]
%
  \tab{
  \lang{de}{Antwort}
  \lang{en}{Answer}
  }
    \begin{table}[\class{items}]
      \nowrap{a) $\mathbb{L}=\{(1\lang{de}{;}\lang{en}{,} 2)\}$ }\\
      \nowrap{b) $\mathbb{L}=\{(3\lang{de}{;}\lang{en}{,} -1)\}$}
    \end{table}
%
  \tab{\lang{de}{Lösung a) mit AV}\lang{en}{Solution for a) with addition method}}
  
  \begin{incremental}[\initialsteps{1}]
    \step \lang{de}{
      Um dieses lineare Gleichungssystem mit dem \emph{Additionsverfahren} zu lösen 
      genügt es, die beiden Gleichungen ohne vorherige Änderung zu addieren.
      }
      \lang{en}{
      To solve this linear system with the \emph{addition method}, we may simply add the two 
      equations together without any modifications.
      }
       \[
       \text{(I)} +\text{(II)}= (1-1)\cdot x+ (3-2)\cdot y =7-5 \; \Leftrightarrow \; y=2
       \]
      \lang{de}{
      In der resultierenden Gleichung kommt dann $x$ nicht mehr vor und wir erhalten
      $\,y=2\,$. Damit steht ein Teil der Lösung bereits fest.
      }
      \lang{en}{
      The variable $x$ does not appear in the resulting equation, and we obtain $\,y=2\,$. Hence 
      we have part of our solution.
      }
    
    \step \lang{de}{
      Setzt man nun in der ursprünglichen ersten Gleichung (I) den Wert $\,2\,$ für $\,y\,$ ein, 
      so kann man dadurch $x$ bestimmen.
      }
      \lang{en}{
      If we now substitute the value $\,2\,$ für $\,y\,$ into the original equation $(I)$ we may 
      determine the value for $x$.
      }
        \[ x + 3\cdot 2=7 \; \Leftrightarrow \; x=1 \]
      \lang{de}{Damit steht die Lösung des linearen Gleichungssystems fest.}
      \lang{en}{Now we have the whole solution to the linear system.}
    
    \step \lang{de}{
      Die Lösung des linearen Gleichungssystems ist also das Zahlenpaar $\,(1,2)\,$ und es gilt
      }
      \lang{en}{
      The unique solution of the linear system is therefore the tuple $\,(1,2)\,$ and hence
      }
      \[\mathbb{L}=\{(1;2)\} \]    
    
    \end{incremental}
%
  \tab{\lang{de}{Lösung a) mit EV}\lang{en}{Solution for a) with substitution method}}
  
  \begin{incremental}[\initialsteps{1}]
    \step \lang{de}{
      Bei diesem linearen Gleichungssystem bietet sich auch die Anwendung des 
      \emph{Einsetzungsverfahrens} an, da für beide Gleichungen die Auflösung 
      nach $\,x\,$ sehr einfach ist.
      }
      \lang{en}{
      The \emph{substitution method} is also suitable for solving this linear system, as either 
      equation can easily be rearranged for $\,x\,$.
      }
    
    \step \lang{de}{
        Wir wählen die Gleichung (I) und lösen diese im ersten Schritt nach $\,x\,$ auf.
        Wir erhalten die neue Gleichung
        }
        \lang{en}{
        We choose equation $(I)$ and rearrange it for $\,x\,$ to obtain the equation
        }
        \[ \text{(I*)} \qquad x=7-3y\]
    
    \step \lang{de}{
        Nun setzen wir im 2. Schritt den Term $\,7-3y\,$ aus der Gleichung (I*) für $\,x\,$ 
        in die ursprüngliche zweite Gleichung (II) ein und erhalten
        }
        \lang{en}{
        Now we substitute the expression $\,7-3y\,$ for $\,x\,$ from equation $(I*)$ into equation 
        $(II)$ and obtain
        }
        \[ -(7-3y) - 2y =-5 \; \Leftrightarrow \; -7+ (3-2)y=-5 \; \Leftrightarrow \; y=2\]
        \lang{de}{Mit steht der erste Teil der Lösung bereits fest.}
        \lang{en}{We hence have the first part of our solution.}
        
    \step \lang{de}{
      Einsetzen von $y$ in die Gleichung (I*) ergibt $x=7-3 \cdot 2=1.$ 
      Hiermit steht auch der zweite Teil der Lösung und schließlich die Gesamtlösung fest.
      Sie besteht hier aus dem Zahlenpaar $(1; 2)$.
      }
      \lang{en}{
      Substituting $y$ into equation $(I*)$ gives $x=7-3 \cdot 2=1$. 
      Thus we have the complete solution, which consists of the tuple $(1; 2)$.
      }
    
    \end{incremental}

%
  \tab{\lang{de}{Lösung b)}\lang{en}{Solution for b)}}
  
  \begin{incremental}[\initialsteps{1}]
    \step \lang{de}{
    Um dieses lineare Gleichungssystem mit dem \emph{Additionsverfahren} zu lösen, kann man beispielsweise die erste
    Gleichung mit $-5$ und die zweite Gleichung mit $7$ multiplizieren und dann beide resultierenden Gleichungen addieren. Dies führt auf eine Gleichung, 
    die $x$ nicht mehr enthält.
      }
      \lang{en}{
      To solve this linear system with the \emph{addition method}, we may for example multiply the 
      first equation by $-5$ and the second equation by $7$, then add the two resulting equations 
      together. This leads to an equation that no longer contains $x$.
      }
      \begin{displaymath}
      \begin{mtable}[\cellaligns{ccrcrcrl}]
      \text{(I)}&\phantom{\qquad-}&7x&-&6y&=&\phantom{-}27&\qquad|\,\cdot(-5)\\
      \text{(II)}&&5x&-&4y&=&19&\qquad|\,\cdot7
      \end{mtable}
      \end{displaymath}
      \begin{center}
      ________________________________________________
      \end{center}
      \begin{displaymath}
      \begin{mtable}[\cellaligns{ccrcrcrl}]
      \phantom{\text{(I)}}&\qquad-&35x&+&30y&=&-135&\phantom{\qquad|\,\cdot(-2)}\\
      \phantom{\text{(II)}}&&35x&-&28y&=&133&
      \end{mtable}
      \end{displaymath}

    
    \step \lang{de}{
    Durch Ausführung des ersten Schritts, die Addition der beiden Gleichungen, entsteht die 
    Gleichung $\,2y=-2\,$, woraus $\,y=-1\,$ folgt. Damit steht ein Teil der Lösung bereits fest.
    }
    \lang{en}{
    The first step yields the equation $\,2y=-2\,$, from which we get $\,y=-1\,$. Hence we have part 
    of our solution.
    }

    \step \lang{de}{
    Setzt man nun in der ursprünglichen ersten Gleichung (I) den Wert $\,-1\,$ für $\,y\,$ ein, 
    so kann man dadurch $x$ bestimmen.
    }
    \lang{en}{
    If we now substitute the value $\,-1\,$ for $\,y\,$ into equation $(I)$, we can determine the 
    value for $x$.
    }
    \[ 7x-6\cdot (-1)=27 \; \Leftrightarrow \; 7x=21 \; \Leftrightarrow \; x=3  \]
    
    \step \lang{de}{
    Die Lösung des linearen Gleichungssystems ist also das Zahlenpaar $\,(3,-1)\,$ und es gilt
    }
    \lang{en}{
    The unique solution of the linear system is therefore the tuple $\,(3,-1)\,$ and we have
    }
    \[\mathbb{L}=\{(3;-1)\} \]    
    
  \end{incremental}  
%  
  \tab{\lang{de}{Hinweis}\lang{en}{Remark}}
    \lang{de}{
    Beachten Sie, dass es auch andere Möglichkeiten gibt, die linearen Gleichungssysteme mit dem 
    Additions- oder Einsetzungsverfahren zu lösen. Die aufgeführten Lösungswege sind nur eine Variante.
    }
    \lang{en}{
    Note that there are other ways to solve the linear systems using the addition or substitution 
    methods. We have only shown one way to determine the solution set.
    }
    
\end{tabs*}
\end{content}