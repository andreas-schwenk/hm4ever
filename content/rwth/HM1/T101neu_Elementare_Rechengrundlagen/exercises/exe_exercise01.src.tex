\documentclass{mumie.element.exercise}
%$Id$
\begin{metainfo}
  \name{
     \lang{de}{Ü01: Mengenoperationen}
     \lang{en}{Ex01: Set operations}
  }
  \begin{description} 
 This work is licensed under the Creative Commons License Attribution 4.0 International (CC-BY 4.0)   
 https://creativecommons.org/licenses/by/4.0/legalcode 

    \lang{de}{Mengenoperationen}
    \lang{en}{Set operations}
  \end{description}
  \begin{components}
    \component{generic_image}{content/rwth/HM1/images/g_tkz_T101_Exercise01.meta.xml}{T101_Exercise01}
  \end{components}
  \begin{links}
    \link{generic_article}{content/rwth/HM1/T101neu_Elementare_Rechengrundlagen/g_art_content_01_zahlenmengen.meta.xml}{content_01_zahlenmengen}
  \end{links}
  \creategeneric
\end{metainfo}
\begin{content}
\begin{block}[annotation]
	Im Ticket-System: \href{https://team.mumie.net/issues/21971}{Ticket 21971}
\end{block}

 \begin{block}[annotation]
   Übung zu Mengenoperationen, Mengendarstellungen und Zahlenmengen    
\end{block}

  \title{
    \lang{de}{Ü01: Mengenoperationen}
    \lang{en}{Ex01: Set operations}
  }
 

  \lang{de}{Von der Menge}
  \lang{en}{Consider the set}
  \[ N= \{ 1; 2; \ldots; 10 \} \]
  \lang{de}{werden die Teilmengen}
  \lang{en}{and the two subsets}
  \[ A= \{ 1; 3; 5; 7; 8; 9 \} \quad 
  \text{\lang{de}{und}\lang{en}{and}} \quad 
  B=\{ x \in N \mid x\text{\lang{de}{ ist durch}\lang{en}{ is divisible by}} 
  \, 3 \, \text{\lang{de}{teilbar}\lang{en}{}} \} \]
%    \[ A= \{ 1; 3; 5; 7; 8; 9 \} \quad \text{und} \quad B=\{ 3; 4; 6; 8; 10\} \]
  \lang{de}{betrachtet.}
  \lang{en}{of $N$.}
  \begin{enumerate}[alph]
  \item \lang{de}{
        Schreiben Sie die Menge $\,B\,$ in eine aufzählende Darstellungform um und
        bestimmen Sie anschließend die Schnittmenge und die Vereinigungsmenge der 
        Mengen $\,A\,$ und $\,B.$
        }
        \lang{en}{
        Write the set $\,B\,$ in increasing order and determine the intersection and union of the 
        sets $\,A\,$ and $\,B$.
        }
  \item \lang{de}{
        Bestimmen Sie die Komplemente der Teilmengen $A$, $B$, $A\cup B$ und 
        $A\cap B$ in $N$. Welche Beziehung besteht zwischen diesen Komplementen?
        }
        \lang{en}{
        Determine the complement of the subsets $A$ and $B$, $A\cup B$ and $A\cap B$ in $N$. 
        What is the relationship between these complements?
        }
  \end{enumerate}
  
  \begin{tabs*}[\initialtab{0}\class{exercise}]
    \tab{\lang{de}{Antworten}\lang{en}{Answers}}
      \begin{enumerate}[alph]
        \item $B=\{ 3; 6; 9\}, \;$ 
              $A\cap B=\{3; 9 \}\;$ \lang{de}{und}\lang{en}{and} 
              $\; A\cup B=\{ 1; 3; 5; 6; 7; 8; 9 \}\;$
        \item \lang{de}{Die Komplemente sind}\lang{en}{The complements are}
          \begin{eqnarray*}
              \complement_N(A) &=& \{2; 4; 6; 10\}, \\
              \complement_N(B) &=& \{1; 2; 4; 5; 7; 8; 10 \}, \\
              \complement_N(A\cup B) &=& \{2; 4; 10 \}, \\
              \complement_N(A\cap B) &=& \{1; 2; 4; 5; 6; 7; 8; 10 \}
          \end{eqnarray*}
    
        \item \lang{de}{
              Das Komplement $\complement_N(A\cup B)$ ist gerade der Durchschnitt von
              $\complement_N(A)$ und $\complement_N(B)$ und das Komplement 
              $\complement_N(A\cap B)$ ist die Vereinigung von $\complement_N(A)$ 
              und $\complement_N(B)$, d.h.
              }
              \lang{en}{
              The complement $\complement_N(A\cup B)$ is the intersection of
              $\complement_N(A)$ and $\complement_N(B)$, and the complement 
              $\complement_N(A\cap B)$ is the union of $\complement_N(A)$ 
              and $\complement_N(B)$, i.e.
              }
              \begin{eqnarray*}
                \complement_N(A\cup B) &=& \complement_N(A) \cap \complement_N(B) \\
                \complement_N(A\cap B) &=& \complement_N(A) \cup \complement_N(B)
              \end{eqnarray*}
      \end{enumerate}

    \tab{\lang{de}{Lösung a)}\lang{en}{Solution for a)}}
     \begin{incremental}[\initialsteps{1}]
      \step \lang{de}{
        Die Menge $N$ enthält alle \emph{natürlichen Zahlen} von $1$ bis $10$. Darunter 
        sind $3, 6\,$ und $\, 9\,$ die Zahlen, die durch $\,3\,$ teilbar sind. Daher
        ist \[B=\{ 3; 6; 9\}.\]
        }
        \lang{en}{
        The set $N$ contains all \emph{natural numbers} from $1$ to $10$. Within this, $3, 6\,$ and 
        $\, 9\,$ are the numbers that are divisible by $\,3$. Hence \[B=\{ 3; 6; 9\}.\]
        }

       \step \lang{de}{
        Für den Durchschnitt und die Vereinigung der Mengen $A$ und $B$ gilt 
        \ref[content_01_zahlenmengen][definitionsgemäß]{def:mengenoperationen}
        }
        \lang{en}{
        The intersection and union of the sets $A$ and $B$  
        \ref[content_01_zahlenmengen][are defined to be]{def:mengenoperationen}
        }

        \begin{eqnarray*} 
         A\cap B=\{x\mid x\in A \wedge x\in B\} &=& \{3; 9 \}\; 
         \text{\lang{de}{und}\lang{en}{and}} \\
         A\cup B=\{x\mid x\in A \vee x\in B\} &=& \{ 1; 3; 5; 6; 7; 8; 9 \}.
        \end{eqnarray*}
     \end{incremental}
     
    \tab{\lang{de}{Lösung b)}\lang{en}{Solution for b)}} 
     \begin{incremental}[\initialsteps{1}]
      \step \lang{de}{
        Das \ref[content_01_zahlenmengen][Komplement]{def:mengenoperationen} einer
        Teilmenge besteht aus genau den Elementen, die in der umgebenden Obermenge 
        liegen, aber nicht in der Teilmenge selbst. Daher sind
        }
        \lang{en}{
        The \ref[content_01_zahlenmengen][complement]{def:mengenoperationen} of a subset contains 
        exactly those elements that belong to the larger set, but not to the subset itself. Hence
        }
        \begin{align*}
         \complement_N(A) &\;=\;& N \setminus \{ 1; 3; 5; 7; 8; 9 \} &\;=\;&  \{2; 4; 6; 10\},\\
         \complement_N(B) &\;=\;& N \setminus \{ 3; 6; 9\} &\;=\;& \{1; 2; 4; 5; 7; 8; 10\},\\
         \complement_N(A\cap B) &\;=\;& N \setminus \{3; 9\} &\;=\;& \{1; 2; 4; 5; 6; 7; 8; 10 \} 
         \quad \text{\lang{de}{und}\lang{en}{and}}\\
         \complement_N(A\cup B) &\;=\;& N \setminus \{1; 3; 4; 5; 6; 7; 8; 9; 10\}&\;=\;& \{2; 4; 10 \}.
        \end{align*}
        
     \step \lang{de}{
       Wenn man sich die Komplemente genau anschaut, kann man erkennen, dass 
       $\complement_N(A\cup B)$ genau der Durchschnitt von  $\complement_N(A)$ 
       und $\complement_N(B)$ ist, und $\complement_N(A\cap B)$ genau deren Vereinigung
       ist, denn
       }
       \lang{en}{
       If we carefully look at the complements, we can recognise that $\complement_N(A\cup B)$ is 
       exactly the intersection of $\complement_N(A)$ and $\complement_N(B)$, and that 
       $\complement_N(A\cap B)$ is exactly their union:
       }
       
    \begin{align*}
      \complement_N(A) \cap \complement_N(B) = \{2; 4; 6; 10\} \cap \{1; 2; 4; 5; 7; 8; 10\} &\,=\,& \{2; 4; 10 \} =  \complement_N(A\cup B) \quad \text{\lang{de}{und}\lang{en}{and}}\\
      \complement_N(A) \cup \complement_N(B) = \{2; 4; 6; 10\} \cup \{1; 2; 4; 5; 7; 8; 10\} &\,=\,& \{1; 2; 4; 5; 6; 7; 8; 10 \} =  \complement_N(A\cap B) .
    \end{align*}
    
    \step \lang{de}{
    Dies ist nicht nur in dem vorliegenden Beispiel der Fall, sondern sogar allgemein, wie man sich 
    mit Hilfe eines Mengendiagramms klar machen kann:
    }
    \lang{en}{
    This is not only the case in the above example, it is true in general. This can be illustrated 
    by the following diagram:
    }
    
    \begin{center}
      \image{T101_Exercise01}
    \end{center}

    \lang{de}{
    $\complement_N(A)$ ist der Bereich in $N$ ohne $A$ und $\complement_N(B)$ ist der Bereich 
    in $N$ ohne $B$.
    Der Durchschnitt beider Bereiche ist genau der Bereich in $N$, der außerhalb von $A$ und außerhalb 
    von $B$ liegt,
    also das Komplement von $A\cup B$.\\
    Die Vereinigung beider Bereiche ist der Bereich in $N$, der außerhalb von $A$ oder außerhalb 
    von $B$ liegt, der also nicht von beiden Bereichen gleichzeitig bedeckt ist. Das ist also gleich 
    dem Komplement von $A\cap B$.
    }
    \lang{en}{
    $\complement_N(A)$ is the region in $N$ having removed $A$, and $\complement_N(B)$ is the region 
    in $N$ having removed $B$. 
    The intersection of the two regions is exactly the region in $N$ not including either $A$ or $B$, 
    which is the complement of $A\cup B$.\\
    The union of the two regions is exactly the region in $N$ that is either not in $A$ or not in 
    $B$, that is, the set of points that are not contained in both $A$ and $B$. That is by definition 
    the complement of $A\cap B$.
    }

   \end{incremental}
  \end{tabs*}

\end{content}

