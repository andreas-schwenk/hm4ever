\documentclass{mumie.element.exercise}
%$Id$
\begin{metainfo}
  \name{
    \lang{de}{Ü10: Implikation und Äquivalenz}
    \lang{en}{Ex10: Implication and equivalence}
  }
  \begin{description} 
 This work is licensed under the Creative Commons License Attribution 4.0 International (CC-BY 4.0)   
 https://creativecommons.org/licenses/by/4.0/legalcode 

    \lang{de}{Implikation und Äquivalenz}
    \lang{en}{Implication and equivalence}
  \end{description}
  \begin{components}
  \end{components}
  \begin{links}
  \end{links}
  \creategeneric
\end{metainfo}
\begin{content}
\begin{block}[annotation]
	Im Ticket-System: \href{https://team.mumie.net/issues/21980}{Ticket 21980}
\end{block}

  \begin{block}[annotation]
%
    Übung zu Äquivalenzumformungen und Implikationen
%
  \end{block}

\title{
\lang{de}{Ü10: Implikation und Äquivalenz}
\lang{en}{Ex10: Implication and equivalence}
}

\lang{de}{Man betrachte die Aussagen}
\lang{en}{Consider the propositions}
\begin{align*}
A&:& \ x=2, \\
B&:& \ x^2=4\quad \text{\lang{de}{und}\lang{en}{and}}\\
C&:& \ x-3=1-x.
\end{align*}
\lang{de}{
Welche Implikationen bestehen zwischen diesen Aussagen? \\
Welche Aussagen sind äquivalent?
}
\lang{en}{
Which implications hold between these propositions? \\
Which of these propositions are equivalent?
}


\begin{tabs*}[\initialtab{0}\class{exercise}]
  \tab{\lang{de}{Antwort}\lang{en}{Answer}} 
  	\lang{de}{
    $A$ und $C$ sind äquivalent, und jede einzelne der beiden impliziert $B$. Aber $B$ impliziert
  	weder $A$ noch $C$. In Formeln:
    }
    \lang{en}{
    $A$ and $C$ are equivalent, and each of the implies $B$. However, $B$ implies neither $A$ nor 
    $C$. In formulas,
    }
  	\[  A\Leftrightarrow C, \quad A\Rightarrow B, \quad C\Rightarrow B. \]
  
   \tab{\lang{de}{Lösung}\lang{en}{Solution}}
  
  \begin{incremental}[\initialsteps{1}]
 \step 
  \lang{de}{
  Dass $A$ sowohl $B$ als auch $C$ impliziert, sieht man direkt durch Einsetzen:
  Wenn $A$ gilt, d.\,h. $x$ gleich $2$ ist, dann gilt $x^2=2^2=4$. Also ist $B$ wahr.
  \\\\
  Außerdem gilt dann $x-3=2-3=-1$ und $1-x=1-2=-1$, d.\,h. auch $C$ ist wahr. Insgesamt 
  gilt also
  }
  \lang{en}{
  The fact that $A$ implies both $B$ and $C$ is easily seen by substitution: 
  if $A$ is true, that is if $x=2$, then we have $x^2=2^2=4$. Hence $B$ is true.
  \\\\
  We also have $x-3=2-3=-1$ and $1-x=1-2=-1$, so $C$ is also true. Hence in general we have
  }
  \[  A\Rightarrow B \quad  \text{und} \quad A\Rightarrow C. \]
  
  \step 
  \lang{de}{
  Dass $B$ die Aussage $A$ \emph{nicht} impliziert, sieht man dadurch, dass die Gleichung $x^2=4$
  nicht nur für $x=2$, sondern auch für $x=-2$ erfüllt ist. Die Aussage $A$ kann also falsch sein, 
  auch wenn $B$ wahr ist, daher
  }
  \lang{en}{
  The fact that $B$ does \emph{not} imply $A$ is seen by considering that $x=-2$ also satisfies the 
  equation $x^2=4$. This means that the statement $A$ may be false even if $B$ is true, so
  }
  \[  B \cancel{\Rightarrow} A. \]
  
 \step 
 \lang{de}{Die Aussage $A$ und die Aussage $C$ sind in der Tat äquivalent:}
 \lang{en}{Proposition $A$ is equivalent to proposition $C$:}
  \begin{align*}
	&&					\quad x-3 \, &= \, 1-x \quad	&& \vert  +3& \\
&\Leftrightarrow  &  		\quad x \, &= \, 4-x \quad 	&& \vert	 +x& \\
&\Leftrightarrow  & 		\quad 2x \, &= \, 4 \quad 	&& \vert	 :2& \\
&\Leftrightarrow  &  		\quad x \, &= \, 2 \,	& &&\\
\end{align*}

 \lang{de}{Insbesondere ist damit auch gezeigt, dass $C$ die Aussage $A$ impliziert, also}
 \lang{en}{In particular, we have shown here that $C$ implies $A$, so}
 \[  C \Rightarrow A. \]
 
  \lang{de}{
  Weil $A$ die Aussage $B$ impliziert und $A$ und $C$ äquivalent sind, wissen wir 
  außerdem, dass $C$ die Aussage $B$ impliziert, also
  }
  \lang{en}{
  Because $A$ implies $B$ and $A$ is equivalent to $C$, we know that $C$ also implies $B$,
  }
  \[  C \Rightarrow B. \]
  
  \lang{de}{
  \textbf{Bemerkung:} Für die Äquivalenz von $A$ und $C$ wurden Äquivalenzumformungen von Gleichungen 
  gemacht, also Umformungen, die man wieder rückgängig machen kann (z.\,B. durch Subtraktion von $3$, 
  um von der zweiten Zeile zur ersten zu gelangen). Um aus der Aussage $A$ die Aussage $B$ zu 
  folgern, wurde quadriert, was nicht rückgängig gemacht werden kann, da $x^2=4$ für $x=2$ und auch 
  für $x=-2$ gilt (s.o.), und deshalb keine Äquivalenzumformung ist.
  }
  \lang{en}{
  \textbf{Remark:} Showing the equivalence of $A$ and $C$ corresponds to applying equivalence 
  transformations (which are invertible) to get from one equation to the other, for example by 
  subtracting $3$ to get from the second row to the first. To get from $A$ to $B$ however, we square 
  both sides, which is not a invertible because $x^2=4$ holds for both $x=2$ and $x=-2$.
  }
  
  \end{incremental}

 

\end{tabs*}
\end{content}