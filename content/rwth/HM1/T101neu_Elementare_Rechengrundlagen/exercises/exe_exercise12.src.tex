\documentclass{mumie.element.exercise}
%$Id$
\begin{metainfo}
  \name{
    \lang{de}{Ü12: Quadratische Gleichungen}
    \lang{en}{Ex12: Quadratic equations}
  }
  \begin{description} 
 This work is licensed under the Creative Commons License Attribution 4.0 International (CC-BY 4.0)   
 https://creativecommons.org/licenses/by/4.0/legalcode 

    \lang{de}{Lösen quadratischer Gleichungen}
    \lang{en}{Solving quadratic equations}
  \end{description}
  \begin{components}
  \end{components}
  \begin{links}
\link{generic_article}{content/rwth/HM1/T101neu_Elementare_Rechengrundlagen/g_art_content_05_loesen_gleichungen_und_lgs.meta.xml}{content_05_loesen_gleichungen_und_lgs}
\end{links}
  \creategeneric
\end{metainfo}
\begin{content}
\begin{block}[annotation]
	Im Ticket-System: \href{https://team.mumie.net/issues/21982}{Ticket 21982}
\end{block}

%
 \begin{block}[annotation]
%
   Neue Übung zur quadratischen Gleichungen mit Bestimmung der Normalform sowie
   der Anwendung quadratischer Ergänzung, pq-Formel,Linearfaktorzerlegung und 
   dem Satz von Viëta.
%
\end{block}

  \title{
    \lang{de}{Ü12: Quadratische Gleichungen}
    \lang{en}{Ex12: Quadratic equations}
  }
 
%
% Aufgabenstellung
%
\begin{enumerate}[alph]
  \item \lang{de}{
  Lösen Sie die folgenden Gleichungen mit einem oder verschiedenen
  beliebigen Lösungsverfahren.
  }
  \lang{en}{
  Solve the following equations using whichever method is most suitable.
  }
        
    \begin{table}[\class{items}]
     \nowrap{i.   $\;  2x^2+8x-10=32 \quad$}      &
     \nowrap{ii.   $\;  -24-x=6x^2+23x+12$}  \\
     \nowrap{iii.   $\; -5x\cdot (2x-3)=0 \quad$}   &
     \nowrap{iv.  $\; -0,5x^2+2=x^2-3x+3,5$}           
    \end{table}
%
     \item \lang{de}{
        Die Quadrate zweier Zahlen, die auf dem Zahlenstrahl den gleichen
        Abstand zu der Zahl 20 haben, haben die Summe 832. Stellen Sie eine Gleichung
        auf und berechnen Sie diese Zahlen.
        }
        \lang{en}{
        Consider two numbers which are the same distance from the number $20$ on the number 
        line. If the sum of their squares is $832$, what are the numbers? Express this problem as an 
        equation and find a solution using an appropriate method.
        }
\end{enumerate}
%
% Lösungen
%
  \begin{tabs*}[\initialtab{0}\class{exercise}]
  
    \tab{\lang{de}{Antworten}\lang{en}{Answers}}
      \begin{enumerate}[alph]
        \item 
         \begin{table}[\class{items}]
           \nowrap{i.   $\; \mathbb{L}= \{ -7;3 \} \quad$} &
           \nowrap{ii.  $\; \mathbb{L}= \emptyset $} \\
           \nowrap{iii. $\; \mathbb{L}= \{ 0 ; \frac{3}{2} \} \quad$} &
           \nowrap{iv.  $\; \mathbb{L}= \{ 1 \} $} \\
        \end{table}
        
        \item \lang{de}{Die gesuchten Zahlen sind 16 und 24.}
              \lang{en}{The two numbers are $16$ and $24$.}

      \end{enumerate}

     
    \tab{\lang{de}{Lösung a) i.}\lang{en}{Solution for a) i.}} 
     \begin{incremental}[\initialsteps{1}]
      \step \lang{de}{
        Zuerst wird die Gleichung normiert, indem wir sie durch den Leitkoeffizienten 
        $\,2 \,$ teilen.
        }
        \lang{en}{
        Firstly we normalise the equation by dividing it by its leading coefficient $\,2$.
        }
        \[
        \begin{mtable}[\cellaligns{crcll}]
                         &  2x^2 + 8x - 10 &\,=\,& 32   & \vert :2\\
         \Leftrightarrow &\qquad  x^2 + 4x - 5 &\,=\,& 16  \quad &
        \end{mtable}
        \]
     \step \lang{de}{
        Lösung mit \ref[content_05_loesen_gleichungen_und_lgs][quadratischer Ergänzung:]{alg:quadr_erg}
        \\\\
        Hieraus lässt sich die Gleichung nun durch Anwendung der 
        \emph{quadratischen Ergänzung} lösen. 
        Hierzu ergänzt man den quadratischen Term durch 
        $\,\textcolor{#0066CC}{+\big(\frac{4}{2}\big)^2 - \big(\frac{4}{2}\big)^2=+4-4},\,$
        wobei $4$ der Koeffizient vor $x$ ist.
        }
        \lang{en}{
        Solution by \ref[content_05_loesen_gleichungen_und_lgs][completing the square]{alg:quadr_erg}:
        \\\\
        From here we may solve the equation by \emph{completing the square}. To do this, we add 
        $\,\textcolor{#0066CC}{+\big(\frac{4}{2}\big)^2 - \big(\frac{4}{2}\big)^2=+4-4},\,$ to the 
        equation, which artificially allows us to factorise the left-hand side. Note that $4$ is used 
        here because it is the coefficient of $x$.
        }
        
        \[
        \begin{mtable}[\cellaligns{crcll}]
                         &          x^2 + 4x - 5 &\,=\,& 16  \quad & \\
         \Leftrightarrow &\qquad    x^2 + 4x \textcolor{#0066CC}{+4-4} - 5 &\,=\,& 16 &\vert \text{1. bin. Formel} \\
         \Leftrightarrow &\qquad   (x+2)^2 -4 - 5 &\,=\,& 16  &\vert +9 \\
         \Leftrightarrow &\qquad   (x+2)^2  &\,=\,& 25  &\vert \sqrt{...}
        \end{mtable}
        \]
        \lang{de}{
        Im nächsten Schritt ist zu beachten, dass die Wurzel aus einem quadratischen Term sowohl 
        positiv, als auch negativ sein kann.
        }
        \lang{en}{
        In the next step we must take care that the square root of a squared expression can be either 
        positive or negative.
        }
        \[
        \begin{mtable}[\cellaligns{crcll}]
                         &         (x+2)^2  &\,=\,& 25    &\vert \sqrt{...} \\
         \Leftrightarrow &\qquad    x+2     &\,=\,& \pm 5 &
        \end{mtable}
        \]
        \lang{de}{Es gibt folglich zwei Lösungen, nämlich $\;x_1=-7 \;$ und $\; x_2=3.$}
        \lang{en}{There are hence two solutions, namely $\;x_1=-7 \;$ and $\; x_2=3$.}
        \\
        
    \step \lang{de}{
        Lösung mit \ref[content_05_loesen_gleichungen_und_lgs][pq-Formel:]{rule:pqFormel}
        \\\\
        Alternativ kann die quadratische Gleichung auch mit der \emph{pq-Formel}
        gelöst werden. Hierzu muss sie jedoch zunächst in einem weiteren Schritt noch
        in ihre Normalform gebrachtt werden.
        }
        \lang{en}{
        Solution using the \ref[content_05_loesen_gleichungen_und_lgs][p-q formula]{rule:pqFormel}:
        \\\\
        Alternatively we may solve the quadratic equation using the \emph{p-q formula}. In order to 
        do this, it must be brought into a more standard form than in the previous method.
        }
        \[
        \begin{mtable}[\cellaligns{crcll}]
                         &          x^2 + 4x - 5 &\,=\,& 16  \quad & \vert -16 \\
         \Leftrightarrow &\qquad    x^2 + 4x - 21 &\,=\,& 0  &
        \end{mtable}
        \]              
        \lang{de}{
        Aus der Normalform können die Parameter $\,p=4\,$ und $\,q=-21\,$ dann direkt
        abgelesen und in die pq-Formel eingesetzt werden.
        }
        \lang{en}{
        We may immediately read the parameters $\,p=4\,$ and $\,q=-21\,$ from the normal form of the 
        equation, and substitute these into the  p-q formula.
        }
        \[
        \begin{mtable}[\cellaligns{crcll}]
             &  x_{1,2} &\,=\,&-\frac{4}{2} \pm \sqrt{\big(\frac{4}{2}\big)^2-(-21)} & \\
         \Leftrightarrow &\qquad    x_{1,2} &\,=\,&-2 \pm \sqrt{25} & \\
         \Leftrightarrow &\qquad    x_1=-2 -5=-7  &\,\wedge\,& x_2=-2+5=3  &
        \end{mtable}
        \]
        \lang{de}{Dies Lösungen sind $\;x_1=-7 \;$ und $\; x_2=3.$}
        \lang{en}{The solutions are $\;x_1=-7 \;$ and $\; x_2=3$.}

   \end{incremental}

    \tab{\lang{de}{Lösung a) ii.}\lang{en}{Solution for a) ii.}} 
     \begin{incremental}[\initialsteps{1}]
      \step \lang{de}{
        Zuerst bringen wir die Gleichung durch geeignete Äquivalenzumformungen in ihre
        Normalform.
        }
        \lang{en}{
        Firstly we bring the equation into its normal form using some equivalence transformations.
        }
        \[
        \begin{mtable}[\cellaligns{crcll}]
                         &  -24-x &\,=\,& 6x^2+23x+12     & \vert +24+x\\
         \Leftrightarrow &\qquad  0 &\,=\,& 6x^2+24x+36   & \vert :6 \\
         \Leftrightarrow &\qquad  0 &\,=\,& x^2+4x+6    &
        \end{mtable}
        \]
     \step \lang{de}{
        Lösung mit \ref[content_05_loesen_gleichungen_und_lgs][quadratischer Ergänzung:]{alg:quadr_erg}
        \\\\
        Diese Gleichung können wir nun durch Anwendung der 
        \emph{quadratischen Ergänzung} lösen. 
        Wir ergänzen hierzu den quadratischen Term durch 
        $\,\textcolor{#0066CC}{+\big(\frac{4}{2}\big)^2 - \big(\frac{4}{2}\big)^2=+4-4},\,$
        wobei $4$ der Koeffizient vor $x$ ist.
        }
        \lang{en}{
        Solution by \ref[content_05_loesen_gleichungen_und_lgs][completing the square]{alg:quadr_erg}:
        \\\\
        This equation can now be solved by \emph{completing the square}. To do this, we add 
        $\,\textcolor{#0066CC}{+\big(\frac{4}{2}\big)^2 - \big(\frac{4}{2}\big)^2=+4-4},\,$
        to the equation, which artificially allows us to factorise the left-hand side. Note that $4$ 
        is used here because it is the coefficient of $x$.
        }
        \[
        \begin{mtable}[\cellaligns{crcll}]
                         &          x^2 + 4x + 6 &\,=\,& 0  \quad & \\
         \Leftrightarrow &\qquad    x^2 + 4x \textcolor{#0066CC}{+4-4} +6 &\,=\,& 0 &\vert \text{1. bin. Formel} \\
         \Leftrightarrow &\qquad   (x+2)^2 +2  &\,=\,& 0  &\vert -2 \\
         \Leftrightarrow &\qquad   (x+2)^2  &\,=\,& -2  &\vert \sqrt{...}
        \end{mtable}
        \]
        \lang{de}{
        Da die Quadratwurzel nie negativ sein kann, ist die Gleichung nicht lösbar.
        Es gilt also  $\; \mathbb{L}= \emptyset $
        }
        \lang{en}{
        As squares are always positive, the square root is not defined for a negative argument, and 
        the equation has no solutions. That is, $\; \mathbb{L}= \emptyset$.
        }

    \step \lang{de}{
        Lösung mit \ref[content_05_loesen_gleichungen_und_lgs][pq-Formel:]{rule:pqFormel}
        \\\\
        Dasselbe Ergebnis erhlaten wir natürlich auch mit der \emph{pq-Formel}.      
        Aus der Normalform lesen wir die Parameter $\,p=4\,$ und $\,q=6\,$ ab und 
        setzen diese in die pq-Formel ein.
        }
        \lang{en}{
        Solution using the \ref[content_05_loesen_gleichungen_und_lgs][p-q formula]{rule:pqFormel}:
        \\\\
        The same result can be obtained using the \emph{p-q formula}. From the normal form of the 
        equation we read the parameters $\,p=4\,$ and $\,q=6\,$ and substitute these into the p-q 
        formula.
        }
        \[
        \begin{mtable}[\cellaligns{crcll}]
             &  x_{1,2} &\,=\,&-\frac{4}{2} \pm \sqrt{\big(\frac{4}{2}\big)^2-6)} & \\
         \Leftrightarrow &\qquad    x_{1,2} &\,=\,&-2 \pm \sqrt{-2} & 
        \end{mtable}
        \]
        \lang{de}{Die Diskriminante ist $-2 <0$, folglich hat die Gleichung keine Lösung.}
        \lang{en}{The discriminant is $-2 <0$, so the equation has no solutions.}

   \end{incremental}

    \tab{\lang{de}{Lösung a) iii.}\lang{en}{Solution for a) iii.}} 
     \begin{incremental}[\initialsteps{1}]
      \step \lang{de}{
        Die Gleichung $\,-5x\cdot (2x-3)=0\,$ sieht schon fast aus wie eine 
        Linearfaktorzerlegung, allerdings sind Linearfaktoren von der Form $\,(x-c),\,$
        der Koeffizient vor dem $x$ ist $1$. 
        \\\\
        Wir multiplizieren den quadratischen Term also erst einmal aus und 
        bringen die Gleichung in ihre Normalform.
        }
        \lang{en}{
        The equation $\,-5x\cdot (2x-3)=0\,$ already seems to be factorised, although for the 
        sake of practice we choose to expand it and factorise it again to obtain a linear factor. 
        with a coefficient $1$ of $x$.
        \\\\
        We expand the quadratic and bring the equation into a normal form.
        }
        \[
        \begin{mtable}[\cellaligns{crcll}]
                         &  -5x\cdot (2x-3) &\,=\,& 0   & \\
         \Leftrightarrow &\qquad -10x^2+15x &\,=\,& 0   & \vert :(-10)\\
         \Leftrightarrow &\qquad  x^2-1\lang{de}{,}\lang{en}{.}5x &\,=\,& 0    & 
        \end{mtable}
        \]
     \step \lang{de}{
        Da die Gleichung keinen konstanten Summanden (ohne $x$) enthält, können wir $x$ 
        nun wieder ausklammern und erhalten somit tatsächlich die Linearfaktorzerlegung
        dieser Gleichung, nämlich
        }
        \lang{en}{
        As the equation contains no constant terms (terms with no $x$), we may factor out $x$ and 
        obtain a linear factor with coefficient $1$ of $x$ as desired:
        }
        \[
        x(x-1,5)=(x-0)(x-1\lang{de}{,}\lang{en}{.}5)=0
        \]
     \step \lang{de}{
        Nach dem
        \ref[content_05_loesen_gleichungen_und_lgs][Satz von Viëta]{thm:Satz_von_Vieta}
        lassen sich hieraus die Nullstellen der Gleichung ablesen. 
        \\
        Es sind $\;x_1=0\,$. und $\,x_2=1,5.$
        }
        \lang{en}{
        By \ref[content_05_loesen_gleichungen_und_lgs][Vieta's formula]{thm:Satz_von_Vieta} we 
        may read the solutions of the equations directly from here.
        \\
        These are $\;x_1=0\,$ and $\,x_2=1.5$.
        }

   \end{incremental}


    \tab{\lang{de}{Lösung a) iv.}\lang{en}{Solution for a) iv.}} 
     \begin{incremental}[\initialsteps{1}]
      \step \lang{de}{
        Zuerst bringen wir die Gleichung durch geeignete Äquivalenzumformungen in ihre
        Normalform.
        }
        \lang{en}{
        Firstly we bring the equation into normal form using some equivalence transformations.
        }
        \[
        \begin{mtable}[\cellaligns{crcll}]
                         &  -0,5x^2+2 &\,=\,& x^2-3x+3,5    & \vert +0,5x^2 -2\\
         \Leftrightarrow &\qquad  0 &\,=\,& 1,5x^2-3x+1,5   & \vert :1,5 \\
         \Leftrightarrow &\qquad  0 &\,=\,& x^2-2x+1        & \vert \,2. \text{bin. Formel} \\
         \Leftrightarrow &\qquad  0 &\,=\,& (x-1)^2    &
        \end{mtable}
        \]
     \step \lang{de}{
        Lösung mit \ref[content_05_loesen_gleichungen_und_lgs][quadratischer Ergänzung:]{alg:quadr_erg}
        \\\\
        Nach dem Verfahren zur \emph{quadratischen Ergänzung} können wir aus 
        der Gleichung $\,(x-1)^2=0 \,$ sofort die Lösung ablesen, nämlich $\,x=1\,$
        und es gibt nur genau diese eine Lösung. 
        Es gilt also $\; \mathbb{L}= \{1\} .$\\
        }
        \lang{en}{
        Solution by \ref[content_05_loesen_gleichungen_und_lgs][completing the square]{alg:quadr_erg}:
        \\\\
        By the \emph{completing the square} method we know that the solution $\,x=1\,$ may be 
        immediately read from the equation $\,(x-1)^2=0$. This is the only solution, and 
        $\; \mathbb{L}= \{1\}$.\\
        }


    \step \lang{de}{
        Lösung mit dem \ref[content_05_loesen_gleichungen_und_lgs][Satz von Viëta:]{thm:Satz_von_Vieta}
        \\\\
        Wir können die Gleichung aber auch schreiben als
        }
        \lang{en}{
        Solution using \ref[content_05_loesen_gleichungen_und_lgs][Vieta's formula]{thm:Satz_von_Vieta}:
        \\\\
        We may also write the equation as
        }
        \[ 0= (x-1)^2 =(x-1)(x-1)\]
        \lang{de}{
        und haben damit die Linearfaktorzerlegung dieser quadratischen Gleichung 
        gegeben. Nach dem \emph{Satz von Viëta:} 
        lassen sich hieraus die Nullstellen ablesen und da beide Linearfaktoren gleich sind,
        gibt es nur die eine Nullstelle: $\,x=1.$
        }
        \lang{en}{
        and hence have the linear factorisation of this quadratic equation. By \emph{Vieta's 
        formula}, we may read the solutions directly from this form, and as the two linear factors 
        are the same, there is a single solution $\,x=1$.
        }
   \end{incremental}

    \tab{\lang{de}{Lösung b)}\lang{en}{Solution for b)}}
        \lang{de}{
        Gesucht sind zwei Zahlen, die zur Zahl $20$ den gleichen Abstand $a$ haben
        und deren Quadrate summiert die Zahl $\,832\,$ ergeben. 
        Dies führt zu folgender quadratischer Gleichung:
        }
        \lang{en}{
        We wish to find numbers which both have the same distance from $20$, and whose squares add 
        together to give $\,832$. 
        This can be expressed using the following quadratic equation:
        }
       
        \[
        \begin{mtable}[\cellaligns{crcll}]
                         &  (20-a)^2 + (20+a)^2 &\,=\,& 832 & \\
         \Leftrightarrow &\qquad  400-40a+a^2 + 400+40a+a^2 &\,=\,& 832 &\\
         \Leftrightarrow &\qquad  800 +2 a^2  &\,=\,& 832   & \vert -800\\
         \Leftrightarrow &\qquad  2 a^2  &\,=\,& 32         & \vert :2\\
         \Leftrightarrow & \qquad   a^2  &\,=\,& 16         &\vert \sqrt{...} \\
         \Leftrightarrow & \qquad   a   &\,=\,& \pm 4  &
        \end{mtable}
        \]
 
        \lang{de}{
        Die quadratische Gleichung hat also zwei Lösungen, nämlich $-4$ und $4$,
        die den Abstand der gesuchten Zahlen zur $20$ beschreiben. Es gilt
        folglich:
        }
        \lang{en}{
        This quadratic equation has two solutions, $-4$ and $4$, which describe the distance from 
        $20$ of the desired numbers. Hence
        }
        \[
          \begin{mtable}[\cellaligns{cccc}]
            &  x_1=20-4=20+(-4)     &\;\wedge\;&    x_2=20+4=20-(-4)   \\
                 
           \Leftrightarrow &\qquad  x_1=16  &\;\wedge\;& x_2=24   
           \end{mtable}
        \]
        \lang{de}{Die gesuchten Zahlen sind $16$ und $24$.}
        \lang{en}{The desired numbers are $16$ and $24$.}
        
  \end{tabs*} 
\end{content}

