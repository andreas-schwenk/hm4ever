%$Id:  $
\documentclass{mumie.article}
%$Id$
\begin{metainfo}
  \name{
    \lang{de}{Kettenregel}
    \lang{en}{The chain rule}
  }
  \begin{description} 
 This work is licensed under the Creative Commons License Attribution 4.0 International (CC-BY 4.0)   
 https://creativecommons.org/licenses/by/4.0/legalcode 

    \lang{de}{Beschreibung}
    \lang{en}{Description}
  \end{description}
  \begin{components}
    \component{generic_image}{content/rwth/HM1/images/g_img_00_Videobutton_schwarz.meta.xml}{00_Videobutton_schwarz}
    \component{js_lib}{system/media/mathlets/GWTGenericVisualization.meta.xml}{mathlet1}
  \end{components}
  \begin{links}
    \link{generic_article}{content/rwth/HM1/T104_weitere_elementare_Funktionen/g_art_content_14_potenzregeln.meta.xml}{power-rules}
  \end{links}
  \creategeneric
\end{metainfo}
\begin{content}
\usepackage{mumie.ombplus}
\ombchapter{6}
\ombarticle{2}
\usepackage{mumie.genericvisualization}

\begin{visualizationwrapper}

\title{\lang{de}{Kettenregel}\lang{en}{The chain rule}}
 
\begin{block}[annotation]
  übungsinhalt
  
\end{block}
\begin{block}[annotation]
  Im Ticket-System: \href{http://team.mumie.net/issues/9032}{Ticket 9032}\\
\end{block}

\begin{block}[info-box]
\tableofcontents
\end{block}

\section{\lang{de}{Verkettung von zwei Funktionen}\lang{en}{Composition of two functions}}
\\
\lang{de}{
Die Verkettung $\,f\circ g$ (sprich \nowrap{"$f$ Kringel $g$" oder \nowrap{"$f$ nach $g$"})} zweier Funktionen $f$ und $g$ 
entsteht durch das Einsetzen der Funktionsvorschrift der Funktion $g$ als Argument in die Funktion $f$. 
\\
% Einfache Verkettungen haben Sie bereits im Kapitel über die \link{Transformationen}{Transformationen} kennengelernt.
% Dort wurden horizontale Verschiebungen \nowrap{$\,(\;f(x) \rightarrow f(x + a)$,} also \nowrap{$g(x)=x + a\,)\,$}
% bzw. horizontale Streckungen und Stauchungen \nowrap{$\,(\;f(x) \rightarrow f(a \cdot x)$,} $g(x)=a \cdot x\,)\,$ besprochen.
% \\
Alle Funktionswerte der in die Funktion $f$ eingesetzten Funktion $g$ müssen 
im Definitionsbereich der Funkton $f$ liegen. Ist dies für alle Argumente der eingesetzten
Funktion $g$ der Fall, so ist der Definitionsbereich der verketteten Funktion 
$f \circ g$ gleich dem Definitionsbereich der Funktion $g$,
im Allgemeinen kann er aber kleiner sein.
}
\lang{en}{
The composition of two functions $f$ and $g$, $\;f \circ g\;$ (pronounced \nowrap{'f composed 
with g' or \nowrap{'f of g'})} comes about by using the output of $g$ as the argument of the 
function $f$. 
\\
% We've already encountered simple compositions in the chapter on \link{Transformationen}{transformations}. 
% There, we discussed the horizontal shift \nowrap{$\,(\;f(x) \rightarrow f(x + a)$,} i.e. \nowrap{$g(x)=x + a\,)\,$}
% as well as the horizontal stretchings and contractions \nowrap{$\,(\;f(x) \rightarrow f(a \cdot x)$,} $g(x)=a \cdot x\,)\,$.
% \\
All of the function values of the function $g$ must be in the domain of the function $f$. For this 
reason, the domain of the composed function $f \circ g$ is either the domain of the function $g$, 
or the largest subset of this which maps into the domain of $f$:
}

\begin{definition}[\lang{de}{Verkettung von Funktionen}\lang{en}{Composition of functions}] \label{addition.definition.1}
  \lang{de}{Seien $f : D_f \to \R$ und $g : D_g \to \R$ zwei Funktionen. Wir setzen}
  \lang{en}{Let $f : D_f \to \R$ and $g : D_g \to \R$ be two functions. We define}
   \begin{align*}
    (f \circ g)(x) &=& f\left(g(x)\right),
   \end{align*}
  \lang{de}{
  wobei $ x \in D_{f \circ g} \; = \; \big\{ x \in D_g \; \big| \; g(x) \in D_f \big\}$. 
  \\\\
  Man nennt $f \circ g$ die \notion{\emph{Verkettung}} von $f$ und $g$. Die Funktion $f$ wird oft 
  \notion{\emph{äußere Funktion}} und $g$ \notion{\emph{innere Funktion}} genannt.}
  \lang{en}{
  where $ x \in D_{f \circ g} \; = \; \big\{ x \in D_g \; \big| \; g(x) \in D_f \big\}$. 
  \\\\
  The function $f$ is often referred to as the \emph{outer function} and $g$ as the \emph{inner 
  function}.
  }
\end{definition}

\lang{de}{
In der nachfolgenden Visualisierung können Sie zwei Funktionen miteinander verketten und den Graphen 
darstellen. Achten Sie hierbei auf die Besonderheiten der Definitionsbereiche.
}
\lang{en}{
In the following visualisation you can compose two functions and see the graphs. Pay special 
attention to the domain of the composed function.
} 
\begin{example}
\lang{de}{
Hier können Sie die Verkettung der Funktionen $\,f(y)=\sin(y)$, $\exp(y)$ und $\sqrt{y}$ mit 
beliebigen Funktionen $g(x)$ anschauen.
}
\lang{en}{
Here you can visualise the composition of the functions $\,f(y)=\sin(y)$, $\exp(y)$, and $\sqrt{y}$ with any chosen function $g(x)$.
}
\begin{tabs*}[\initialtab{0}] %\class{exercise}
\tab{$(f\circ g)(x) = \sin(g(x))$}

\begin{genericGWTVisualization}[480][550]{mathlet1}
\title{$(f \circ g)(x) = \sin(g(x))$}
	\begin{variables}
	
	\function{F1}{real}{sin(x)}
	\function[editable]{F2}{rational}{x^2/3}
	\function{F22}{rational}{var(F2)}
	\function{G}{real}{sin(var(F22))}
	\function{G1}{real}{var(G)}
	\end{variables}
	
	\label{F1}{\textcolor{blue}{f}}
	\label{G}{\textcolor{#CC6600}{f\circ g}}
	\label{F22}{\textcolor{#00CC00}{g}}
	
	% COLOR:
	
	\color{G}{#CC6600}
	\color[0.5]{F1}{BLUE}
	\color[0.5]{F22}{#00CC00}
	
		
	\begin{canvas}
	\plotSize{500}
	\plotLeft{-10}
	\plotRight{10}
	\plot[coordinateSystem]{F1,F22,G}
	\end{canvas}
	\lang{de}{\text{
  Die Grafik zeigt die Funktion \textcolor{blue}{$f(y)=\sin(y)$},\\
  sowie die Funktion \textcolor{#00CC00}{$g(x)=$}$\var{F2}$\\
	und deren Komposition \textcolor{#CC6600}{$(f\circ g)(x) = \sin(g(x))=$}$\var{G1}$.\\
	Sie können für $g(x)$ andere Funktionen eingeben, z.B.  $g(x) = x^2,x^3, e^x, 
  \sqrt{x}=sqrt(x)\ldots $.
  }}
  \lang{en}{\text{
  The graph shows the function \textcolor{blue}{$f(y)=\sin(y)$},\\
  the function \textcolor{#00CC00}{$g(x)=$}$\var{F2}$\\
	and their composition \textcolor{#CC6600}{$(f\circ g)(x) = \sin(g(x))=$}$\var{G1}$.\\
	Try replacing $g(x)$ with another function, e.g.  $g(x) = x^2,x^3, e^x, 
  \sqrt{x}=sqrt(x)\ldots $.
  }}
 
\end{genericGWTVisualization}
\tab{$(f\circ g)(x) = \exp(g(x))$}
\begin{genericGWTVisualization}[480][550]{mathlet1}
\title{$(f \circ g)(x) = \exp(g(x))$}
	\begin{variables}
	
	\function{F1}{real}{exp(x)}
	\function[editable]{F2}{real}{-x^2}
	\function{F22}{real}{var(F2)}
	\function{G}{real}{exp(var(F22))}
	\function{G1}{real}{var(G)}	
	\end{variables}
	
	\label{F1}{\textcolor{blue}{f}}
	\label{G}{\textcolor{#CC6600}{f\circ g}}
	\label{F22}{\textcolor{#00CC00}{g}}
	
	% COLOR:
	
	\color{G}{#CC6600}
	\color[0.5]{F1}{BLUE}
	\color[0.5]{F22}{#00CC00}
	
		
	\begin{canvas}
	\plotSize{500}
	\plotLeft{-2}
	\plotRight{2}
	\plot[coordinateSystem]{F1,F22,G}
	\end{canvas}
	\lang{de}{\text{
  Die Grafik zeigt die Funktion \textcolor{blue}{$f(y)=\exp(y)$},\\
  sowie die Funktion \textcolor{#00CC00}{$g(x)=$}$\var{F2}$\\
	und deren Komposition \textcolor{#CC6600}{$(f\circ g)(x) = \exp(g(x))=$}$\var{G1}$.\\
	Sie können für $g(x)$ andere Funktionen eingeben, z.B.  $g(x) = x^2,x^3, \sqrt{x}=sqrt(x), 
  1/x, \ldots $.
  }}
  \lang{en}{\text{
  The graph shows the function \textcolor{blue}{$f(y)=\exp(y)$},\\
  the function \textcolor{#00CC00}{$g(x)=$}$\var{F2}$\\
	and their composition \textcolor{#CC6600}{$(f\circ g)(x) = \exp(g(x))=$}$\var{G1}$.\\
	Try replacing $g(x)$ with another function, e.g. $g(x) = x^2,x^3, \sqrt{x}=sqrt(x), 
  1/x, \ldots $.
  }}
\end{genericGWTVisualization}

\tab{$(f\circ g)(x) = \sqrt{g(x)}$}
\begin{genericGWTVisualization}[480][550]{mathlet1}
\title{$(f \circ g)(x) = \sqrt{g(x)}$}
	\begin{variables}
	
	\function{F1}{real}{sqrt(x)}
	\function[editable]{F2}{real}{x+2}
	\function{F22}{real}{var(F2)}
	\function{G}{real}{sqrt(var(F2))}
	\function{G1}{real}{var(G)}		
	\end{variables}
	
	\label{F1}{\textcolor{blue}{f}}
	\label{G}{\textcolor{#CC6600}{$f\circ g$}}
	\label{F22}{\textcolor{#00CC00}{g}}
	
	% COLOR:
	
	\color{G}{#CC6600}
	\color[0.5]{F1}{BLUE}
	\color[0.5]{F22}{#00CC00}
	
	\begin{canvas}
	\plotSize{500}
	\plotLeft{-10}
	\plotRight{10}
	\plot[coordinateSystem]{F1,F22,G}
	\end{canvas}
	\lang{de}{\text{
  Die Grafik zeigt die Funktion \textcolor{blue}{$f(y)=\sqrt{y}$},\\
  sowie die Funktion \textcolor{\#00CC00}{$g(x)=$}$\var{F2}$\\
	und deren Komposition \textcolor{#CC6600}{$(f\circ g)(x) = \sqrt{g(x)}=$}$\var{G1}$.\\
	Sie können für $g(x)$ andere Funktionen eingeben, z.B.  $g(x) = x^2,x^3, e^x, 1/x, \ldots $.
  }}
  \lang{en}{\text{
  The graph shows the function \textcolor{blue}{$f(y)=\sqrt{y}$},\\
  the function \textcolor{\#00CC00}{$g(x)=$}$\var{F2}$\\
	and their composition \textcolor{#CC6600}{$(f\circ g)(x) = \sqrt{g(x)}=$}$\var{G1}$.\\
	Try replacing $g(x)$ with another function, e.g. $g(x) = x^2,x^3, e^x, 1/x, \ldots $.
  }}
\end{genericGWTVisualization}
\end{tabs*}
\end{example}

\begin{quickcheck}
		\field{rational}
		\type{input.function}
		\begin{variables}
			\randint{k}{1}{2}   %Zufallsvariable für Auswahl
			\function[calculate]{d1}{-(k-2)}  % "Dirac"-funktionen
			\function[calculate]{d2}{(k-1)}

			\randint[Z]{a}{-2}{2}
			\randint[Z]{b}{1}{4}
			\randint{c}{1}{4}
			\randint[Z]{d}{-4}{4}


      		\function[expand,normalize]{f1}{x^2+a*x+b}
      		\function[expand,normalize]{f2}{c*x+d}
      		\function[expand,normalize]{f12}{(c*x+d)^2+a*(c*x+d)+b}
      		\function[expand,normalize]{f21}{c*(x^2+a*x+b)+d}

			\function[expand,normalize]{f}{d1*f1+d2*f2}
			\function[expand,normalize]{g}{d1*f2+d2*f1}
      		\function[expand,normalize]{fg}{d1*f12+d2*f21}
		\end{variables}

			\text{\lang{de}{
      Bestimmen Sie die Verkettung $f\circ g$ der Funktionen $f(x)=\var{f}$ und $g(x)=\var{g}$.\\
			Es ist $(f\circ g)(x)= $\ansref.
      }
      \lang{en}{
      Determine the composition $f\circ g$ of the functions $f(x)=\var{f}$ und $g(x)=\var{g}$.\\
      It is $(f\circ g)(x)= $\ansref.
      }}

     \begin{answer}
          \solution{fg}
          \checkAsFunction{x}{-2}{2}{20}
      \end{answer}
	\end{quickcheck}


\section{\lang{de}{Ableitung der Verkettung von Funktionen}
         \lang{en}{Differentiating composite functions}}

\lang{de}{Die Ableitung der Verkettung zweier Funktionen führt auf die Kettenregel.}
\lang{en}{The differentiation of the composition of two functions is described by chain rule.}
\begin{rule}[Kettenregel]\label{rule:kettenregel}
%\href{https://www.hm-kompakt.de/video?watch=513}{\image[75]{00_Videobutton_schwarz}}\\
%\textit{\lang{de}{Kettenregel:}\lang{en}{The Chain Rule}}\\
\lang{de}{
Sind $f$ und $g$ differenzierbare Funktionen, so gilt
\[(f\circ g)'(x)=f'(g(x))\cdot g'(x).\]
\floatright{\href{https://www.hm-kompakt.de/video?watch=513}{\image[75]{00_Videobutton_schwarz}}}\\\\
}
\lang{en}{
If $f$ and $g$ are differentiable functions, then
\[(f\circ g)'(x)=f'(g(x))\cdot g'(x).\]
}

%\textit{\lang{de}{Kettenregel:}\lang{en}{The Chain Rule}}\\
\end{rule}

\lang{de}{
Man bildet also $f'$ an der Stelle $g(x)$ sowie $g'$ an der Stelle $x$ und multipliziert diese 
beiden Werte miteinander.
}
\lang{en}{
We evaluate the derivative $f'$ of $f$ and evaluate it at $g(x)$, then multiply the result by the 
derivative $g'$ of $g$, evaluated at $x$.
} 




\begin{example}

\lang{de}{
Wir berechnen als Beispiel die Ableitung der Funktion $\sin(x^2)$. Setzen wir als \"{a}u{\ss}ere 
Funktion $f(y)=\sin y$ und als innere Funktion $g(x)=x^2$, so ist $f(g(x))=\sin(x^2)$. Wir 
k\"{o}nnen also die Kettenregel benutzen. Zun\"{a}chst ist $f'(y)=\cos y$. F\"{u}r $y=g(x)=x^2$ 
ergibt das:
}
\lang{en}{
As an example, we can calculate the derivative of the function $\sin(x^2)$. First consider an outer 
function be $f(y)=\sin y$ and an inner function be $g(x)=x^2$, so that $f(g(x))=\sin(x^2)$. Since 
this is the composition of two functions, we can use the chain rule here. We know that 
$f'(y)=\cos y$, so as $y=g(x)=x^2$,
}
\[f'(g(x))=\cos (g(x))=\cos(x^2).\]
\lang{de}{Nun m\"{u}ssen wir noch die Ableitung von $g$ berechnen. Wir erhalten:}
\lang{en}{Now we calculate the derivative of $g$:}
\[g'(x)=2x.\]
\lang{de}{Nach der Kettenregel ist dann die Ableitung von $f(g(x))=\sin(x^2)$:}
\lang{en}{According to the chain rule, the derivative of $f(g(x))=\sin(x^2)$ is:}
\[(f\circ g)'(x)=f'(g(x))\cdot g'(x)=2x\cos (x^2).\]
\lang{de}{
Wenn man ein wenig Erfahrung beim Ableiten von Funktionen gesammelt hat, braucht man nat\"{u}rlich 
nicht mehr diese vielen kleinen Einzelschritte zu gehen! Die Rechnung kann man in etwa auf die 
letzte Zeile des Beispieles reduzieren.
}
\lang{en}{
Once we have enough experience differentiating functions, we no longer need to perform each of these 
small steps individually. The calculation can be shortened and the last line can be written down 
in one step, having cited the chain rule.
}

\end{example}




\begin{quickcheck}
		\field{rational}
		\type{input.function}
		\begin{variables}
			\randint{k}{1}{4}	% Zufallsvariable für Auswahl:
			\function[calculate]{d1}{-(k-2)*(k-3)*(k-4)/6}  % "Dirac"-funktionen
			\function[calculate]{d2}{(k-1)*(k-3)*(k-4)/2}
			\function[calculate]{d3}{-(k-1)*(k-2)*(k-4)/2}
			\function[calculate]{d4}{(k-1)*(k-2)*(k-3)/6}

			\randint{l}{1}{2}   %Zufallsvariable für Auswahl
			\function[calculate]{l1}{-(l-2)}  % "Dirac"-funktionen
			\function[calculate]{l2}{(l-1)}

			\function{h0}{x^2+1}
			\function[normalize]{h}{l1*x+l2*h0}

			\function[normalize]{f1}{sin(h)}
			\function[normalize]{f2}{cos(h)}
			\function[normalize]{f3}{exp(h)}
			\function[normalize]{f4}{ln(h)}
			\function[normalize]{f}{d1*f1+d2*f2+d3*f3+d4*f4}
			\function[expand,normalize]{df}{d1*f2+d2*(-f1)+d3*f3+d4*(1/h)}  % Ableitung von f an der Stelle h

			% g=h0°f oder g=f°h0 mit f in fi
			\function[expand,normalize]{g}{l2*f+l1*(f^2+1)}
			\function[expand,normalize]{dg}{l2*df*2*x+l1*(2*df*f)}

		\end{variables}

			\text{\lang{de}{
        Bestimmen Sie mit der Kettenregel die Ableitung der Funktion $f(x)=\var{g}$.\\
				Die Ableitung ist $f'(x)=$ \ansref.
        }
        \lang{en}{
        Using the chain rule, determine the derivative of the function $f(x)=\var{g}$.\\
        The derivative is $f'(x)=$ \ansref.
        }}

     \begin{answer}
          \solution{dg}
%		  \allowForInput[false]{)}
          \checkAsFunction{x}{-2}{2}{20}
      \end{answer}
\end{quickcheck}

% 
% 
% 	\begin{genericGWTVisualization}[550][1000]{mathlet1}
% 		\begin{variables}
% 			\randint{randomA}{1}{2}
% 
% 			\point[editable]{P}{rational}{var(randomA),var(randomA)}
% 		\end{variables}
% 		\color{P}{BLUE}
% 		\label{P}{$\textcolor{BLUE}{P}$}
% 
% 		\begin{canvas}
% 			\plotSize{300}
% 			\plotLeft{-3}
% 			\plotRight{3}
% 			\plot[coordinateSystem]{P}
% 		\end{canvas}
% 		\text{Der Punkt hat die Koordinaten $(\var{P}[x],\var{P}[y])$.}
% 	    	\end{genericGWTVisualization}

\end{visualizationwrapper}


\end{content}