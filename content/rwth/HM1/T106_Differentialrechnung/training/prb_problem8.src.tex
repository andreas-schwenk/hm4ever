\documentclass{mumie.problem.gwtmathlet}
%$Id$
\begin{metainfo}
  \name{
    \lang{de}{A08: 1. und 2. Ableitung}
    \lang{en}{problem_8}
  }
  \begin{description} 
 This work is licensed under the Creative Commons License Attribution 4.0 International (CC-BY 4.0)   
 https://creativecommons.org/licenses/by/4.0/legalcode 

    \lang{de}{}
    \lang{en}{}
  \end{description}
  \corrector{system/problem/GenericCorrector.meta.xml}
  \begin{components}
    \component{js_lib}{system/problem/GenericMathlet.meta.xml}{mathlet}
  \end{components}
  \begin{links}
  \end{links}
  \creategeneric
\end{metainfo}
\begin{content}
\usepackage{mumie.ombplus}
\usepackage{mumie.genericproblem}


\lang{de}{
	\title{A08: 1. und 2. Ableitung}
}
\lang{en}{
	\title{Problem 8}
}

\begin{block}[annotation]
      
\end{block}
\begin{block}[annotation]
  Im Ticket-System: \href{http://team.mumie.net/issues/9473}{Ticket 9473}
\end{block}


\begin{problem}
	
\randomquestionpool{1}{3}
\randomquestionpool{4}{6}
\randomquestionpool{7}{9}
	
%Frage 1 von 9
\begin{question}
	\lang{de}{
		\text{Gegeben $f(x) = \var{f}$. Berechnen Sie $f'(x)$ sowie $f''(x)$ und geben Sie Ihre Ergebnisse ein.}
		\explanation{Hier bietet es sich an die Kettenregel zweimal anzuwenden. Die innere Funktion besteht aus einem Polynom ersten Grades und die äußere aus einem Monom $x^n$.}
	}
	\lang{en}{
		\text{Let $f(x) = \var{f}$. Calculate $f'(x)$ and $f''(x)$ and input your results below.}
	}
	\begin{variables}
        \randint{a}{2}{5}
		\randint{b}{3}{6}
		\function{f}{(x-a)^b}
		\derivative[normalize]{f_1}{f}{x}      
        \derivative[normalize]{f_2}{f_1}{x} 
	\end{variables}	
	\type{input.function}
	\begin{answer}
		\text{ $f'(x) = $}
		\solution{f_1}
		\checkAsFunction{x}{-10}{10}{100}		
	\end{answer}
	\begin{answer}
		\text{ $f''(x) = $}
		\solution{f_2}
		\checkAsFunction{x}{-10}{10}{100}		
	\end{answer}
\end{question}
	
%Frage 2 von 9
\begin{question}
	\lang{de}{
		\text{Gegeben $f(x) = \var{f}$. Berechnen Sie $f'(x)$ sowie $f''(x)$ und geben Sie Ihre Ergebnisse ein.}
		\explanation{Die innere Funtkion besteht aus einem Polynom zweiten Grades und die äußere ist die Sinus-Funktion. Für die erste Ableitung benötigt man die Kettenregel und erhält ein Produkt aus einem Monom und der Sinus-Funktion.
        Deshalb benötigt man für die zweite Abletiung eine Kombination aus Ketten- und Produktregel.}
	}
	\lang{en}{
		\text{Let $f(x) = \var{f}$. Calculate $f'(x)$ and $f''(x)$ and input your results below.}
	}
	\begin{variables}
        \randint{a}{2}{5}
		\randint{b}{3}{6}
		\function{f}{sin(x^2+1)}
		\derivative[normalize]{f_1}{f}{x}      
        \derivative[normalize]{f_2}{f_1}{x} 
	\end{variables}
	\type{input.function}
	\begin{answer}
		\text{ $f'(x) = $}
		\solution{f_1}
		\checkAsFunction{x}{-10}{10}{100}		
	\end{answer}
	\begin{answer}
		\text{ $f''(x) = $}
		\solution{f_2}
		\checkAsFunction{x}{-10}{10}{100}		
	\end{answer}
\end{question}
	
%Frage 3 von 9
\begin{question}
	\lang{de}{
		\text{Gegeben $f(x) = \var{f}$. Berechnen Sie $f'(x)$ sowie $f''(x)$ und geben Sie Ihre Ergebnisse ein.}
		\explanation{Auch hier muss zunächst die Kettenregel verwendet werden. Die innere Funktion ist der Cosinus und die äußere ein Monom zweiten Grades.
        Für die zweite Ableitung benötig man dann die Produktregel.}
	}
	\lang{en}{
		\text{Let $f(x) = \var{f}$. Calculate $f'(x)$ and $f''(x)$ and input your results below.}
	}
	\begin{variables}
        \randint{a}{2}{5}
		\randint{b}{3}{6}
		\function{f}{(cos(x))^2}
		\derivative[normalize]{f_1}{f}{x}      
        \derivative[normalize]{f_2}{f_1}{x} 
	\end{variables}
	\type{input.function}
	\begin{answer}
		\text{ $f'(x) = $}
		\solution{f_1}
		\checkAsFunction{x}{-10}{10}{100}		
	\end{answer}
	\begin{answer}
		\text{ $f''(x) = $}
		\solution{f_2}
		\checkAsFunction{x}{-10}{10}{100}		
	\end{answer}
\end{question}
	
%Frage 4 von 9
\begin{question}
	\lang{de}{
		\text{Gegeben $f(x) = \var{f}$. Berechnen Sie $f'(x)$ sowie $f''(x)$ und geben Sie Ihre Ergebnisse ein.}
		\explanation{Zunächst wenden wir die Produktrgel an. Der erste Term des Produktes 
        ist allerdings eine Verkettung eines Polynoms $(x-a)^n$ und eines Monoms $x^m$ weshalb hier noch die Kettenregel
        benötigt wird. Für die zweite Ableitung wird ebenfalls eine Kombination aus Ketten- und Produktregel benötigt.
      }
	}
	\lang{en}{
		\text{Let $f(x) = \var{f}$. Calculate $f'(x)$ and $f''(x)$ and input your results below.}
	}
	\begin{variables}
        \randint{a}{2}{5}
		\randint{b}{3}{6}
		\randint{d}{3}{6}
		\function{f}{(x-a)^b*x^d}
		\derivative[normalize]{f_1}{f}{x}      
        \derivative[normalize]{f_2}{f_1}{x} 
	\end{variables}
	\type{input.function}
	\begin{answer}
		\text{ $f'(x) = $}
		\solution{f_1}
		\checkAsFunction{x}{-10}{10}{100}		
	\end{answer}
	\begin{answer}
		\text{ $f''(x) = $}
		\solution{f_2}
		\checkAsFunction{x}{-10}{10}{100}		
	\end{answer}
\end{question}
	
%Frage 5 von 9
\begin{question}
	\lang{de}{
		\text{Gegeben $f(x) = \var{f}$. Berechnen Sie $f'(x)$ sowie $f''(x)$ und geben Sie Ihre Ergebnisse ein.}
		\explanation{Hier haben wir ein Produkt aus einer Sinus-Funktion und einem Monom. Allerdings ist die Sinus-Funktion auch einem Monom verkettet,
        weshalb dann noch die Kettenregel benötigt wird. Für die zweite Ableitung wird ebenfalls eine Kombinatio aus Ketten- und Produktregel für die zwei Termen benötigt.}
	}
	\lang{en}{
		\text{Let $f(x) = \var{f}$. Calculate $f'(x)$ and $f''(x)$ and input your results below.}
	}
	\begin{variables}
        \randint{a}{2}{5}
		\randint{b}{3}{6}
		\randint{d}{3}{6}
		\function{f}{sin(x^b)*x^a}
		\derivative[normalize]{f_1}{f}{x}      
        \derivative[normalize]{f_2}{f_1}{x} 
	\end{variables}
	\type{input.function}
	\begin{answer}
		\text{ $f'(x) = $}
		\solution{f_1}
		\checkAsFunction{x}{-10}{10}{100}		
	\end{answer}
	\begin{answer}
		\text{ $f''(x) = $}
		\solution{f_2}
		\checkAsFunction{x}{-10}{10}{100}		
	\end{answer}
\end{question}

%Frage 6 von 9
\begin{question}
	\lang{de}{
		\text{Gegeben $f(x) = \var{f}$. Berechnen Sie $f'(x)$ sowie $f''(x)$ und geben Sie Ihre Ergebnisse ein.}
		\explanation{Für die Ableitung dieses Produktes aus einem Monom und der mit einem Monom verketteten Exponentialfunktion
        wird eine Kombination aus Produkt- und Kettenregel benötigt. Auch für die zweite Ableitung muss für beide Terme dies wiederholt werden.}
	}
	\lang{en}{
		\text{Let $f(x) = \var{f}$. Calculate $f'(x)$ and $f''(x)$ and input your results below.}
	}
	\begin{variables}
        \randint{a}{2}{5}
		\randint{b}{3}{6}
		\function{f}{x^a*e^(x^b)}
		\derivative[normalize]{f_1}{f}{x}      
        \derivative[normalize]{f_2}{f_1}{x} 
	\end{variables}
	\type{input.function}
	\begin{answer}
		\text{ $f'(x) = $}
		\solution{f_1}
		\checkAsFunction{x}{-10}{10}{100}		
	\end{answer}
	\begin{answer}
		\text{ $f''(x) = $}
		\solution{f_2}
		\checkAsFunction{x}{-10}{10}{100}		
	\end{answer}
\end{question}
		
%Frage 7 von 9
\begin{question}
	\lang{de}{
		\text{Gegeben $f(x) = \var{f}$. Berechnen Sie $f'(x)$ sowie $f''(x)$ und geben Sie Ihre Ergebnisse ein.}
		\explanation{Bei dieser Aufgabe wird im ersten Schritt die Kettenregel benötigt. 
        Die innere Funktion ist ein Polynom der Form $x^m + a$ und die äußere ist der 
        natürliche Logarithmus. Für die zweite Ableitung benötigt man die Quotientenregel.}
	}
	\lang{en}{
		\text{Let $f(x) = \var{f}$. Calculate $f'(x)$ and $f''(x)$ and input your results below.}
	}
	\begin{variables}
        \randint{a}{2}{5}
		\randint{b}{3}{6}
		\function[normalize]{f}{ln(x^(2*a)+b)}
		\derivative[normalize]{f_1}{f}{x}      
        \derivative[normalize]{f_2}{f_1}{x} 
	\end{variables}
	\type{input.function}
	\begin{answer}
		\text{ $f'(x) = $}
		\solution{f_1}
		\checkAsFunction{x}{0.1}{10}{100}		
	\end{answer}
	\begin{answer}
		\text{ $f''(x) = $}
		\solution{f_2}
		\checkAsFunction{x}{0.1}{10}{100}		
	\end{answer}
\end{question}

%Frage 8 von 9
\begin{question}
	\lang{de}{
		\text{Gegeben $f(x) = \var{f}$. Berechnen Sie $f'(x)$ sowie $f''(x)$ und geben Sie Ihre Ergebnisse ein.}
		\explanation{Die innere Funktion besteht aus einem Polynom und die äußere Funktion ist die Wurzelfunktion.
        Für die erste Ableitung wird deshalb die Kettenregel benötigt. 
        Die Ableitung der Wurzelfunktion ist $\frac{1}{2 \cdot \sqrt{x}}$. Um die zweite Ableitung zu bestimmen kann die Quotientenregel
        in Kombination mit der Kettenregel verwendet werden.}
	}
	\lang{en}{
		\text{Let $f(x) = \var{f}$. Calculate $f'(x)$ and $f''(x)$ and input your results below.}
	}
	\begin{variables}
	   	\randint{a}{2}{5}
	   	\randint{b}{3}{6}
		\function[normalize]{f}{sqrt(x^(2*a)+b)}
		\derivative[normalize]{f_1}{f}{x}      
	    \derivative[normalize]{f_2}{f_1}{x} 
	\end{variables}	
	\type{input.function}
	\begin{answer}
		\text{ $f'(x) = $}
		\solution{f_1}
		\checkAsFunction{x}{0.1}{10}{100}		
	\end{answer}
	\begin{answer}
		\text{ $f''(x) = $}
		\solution{f_2}
		\checkAsFunction{x}{0.1}{10}{100}		
	\end{answer}
\end{question}

%Frage 9 von 9
\begin{question}
	\lang{de}{
		\text{Gegeben $f(x) = \var{f}$. Berechnen Sie $f'(x)$ sowie $f''(x)$ und geben Sie Ihre Ergebnisse ein.}
		\explanation{Hier bietet sich die Quotientenregel an, da hier ein Quotient zweier Polynome abzuleiten ist.}
	}
	\lang{en}{
		\text{Let $f(x) = \var{f}$. Calculate $f'(x)$ and $f''(x)$ and input your results below.}
	}
	\begin{variables}
		\randint{a}{2}{5}
    	\randint{b}{3}{6}
		\function[normalize]{f}{x/(x^(2*a)+b)}
		\derivative[normalize]{f_1}{f}{x}      
        \derivative[normalize]{f_2}{f_1}{x} 
	\end{variables}
	\type{input.function}
	\begin{answer}
		\text{ $f'(x) = $}
		\solution{f_1}
		\checkAsFunction{x}{0.1}{10}{100}		
	\end{answer}
	\begin{answer}
		\text{ $f''(x) = $}
		\solution{f_2}
		\checkAsFunction{x}{0.1}{10}{100}		
	\end{answer}
\end{question}

\end{problem}

\embedmathlet{mathlet}
\end{content}