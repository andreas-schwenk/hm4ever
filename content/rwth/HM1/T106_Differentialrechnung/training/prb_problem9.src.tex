\documentclass{mumie.problem.gwtmathlet}
%$Id$
\begin{metainfo}
  \name{
    \lang{de}{A09: zweite Ableitung}
    \lang{en}{problem_9}
  }
  \begin{description} 
 This work is licensed under the Creative Commons License Attribution 4.0 International (CC-BY 4.0)   
 https://creativecommons.org/licenses/by/4.0/legalcode 

    \lang{de}{}
    \lang{en}{}
  \end{description}
  \corrector{system/problem/GenericCorrector.meta.xml}
  \begin{components}
    \component{js_lib}{system/problem/GenericMathlet.meta.xml}{mathlet}
  \end{components}
  \begin{links}
  \end{links}
  \creategeneric
\end{metainfo}
\begin{content}
\usepackage{mumie.ombplus}
\usepackage{mumie.genericproblem}


\lang{de}{
	\title{A09: zweite Ableitung}
}
\lang{en}{
	\title{Problem 9}
}

\begin{block}[annotation]
      
\end{block}
\begin{block}[annotation]
  Im Ticket-System: \href{http://team.mumie.net/issues/9474}{Ticket 9474}
\end{block}


\begin{problem}
	
\randomquestionpool{1}{3}
\randomquestionpool{4}{5}
\randomquestionpool{6}{8}
\randomquestionpool{9}{9}
	
%Frage 1 von 9
\begin{question}
	\lang{de}{
		\text{Gegeben $f(x) = \var{ida}x^3-\var{idb}x^2+\var{idc}x+\var{idd}$. Berechnen Sie $f''(x)$ und geben Sie Ihr Ergebnis ein.}
		\explanation{}
	}
	\lang{en}{
		\text{Let $f(x) = \var{ida}x^3-\var{idb}x^2+\var{idc}x+\var{idd}$. Calculate $f''(x)$ and input your answer below.}
	}
	\begin{variables}
        \randint[Z]{a}{2}{5}
		\randint[Z]{b}{3}{6}
		\randint[Z]{c}{2}{7}
		\randint[Z]{d}{1}{12}
		\function[calculate]{ida}{a}
		\function[calculate]{idb}{b}
		\function[calculate]{sida}{6*a}
		\function[calculate]{zidb}{2*b}
		\function[calculate]{idc}{c}
		\function[calculate]{idd}{d}
	    \function{g}{sida*x-zidb}
	\end{variables}	
	\type{input.function}
	\begin{answer}
		\text{ $f''(x) = $}
		\solution{g}
		\checkAsFunction{x}{-10}{10}{100}		
	\end{answer}
\end{question}

%Frage 2 von 9
\begin{question}
	\lang{de}{
		\text{Gegeben $f(x) = \cos(\var{idb}x)+\sin(\var{idd}x)$. Berechnen Sie $f''(x)$ und geben Sie Ihr Ergebnis ein.}
		\explanation{}
	}
	\lang{en}{
		\text{Let $f(x) = \cos(\var{idb}x)+\sin(\var{idd}x)$. Calculate $f''(x)$ and input your answer below.}
	}
	\begin{variables}
        \randint[Z]{a}{2}{5}
		\randint[Z]{b}{3}{6}
		\randint[Z]{c}{2}{7}
		\randint[Z]{d}{1}{12}
		\function[calculate]{ida}{a}
		\function[calculate]{idb}{b}
		\function[calculate]{vfe}{-1*b^2}
		\function[calculate]{vfz}{d^2}
		\function[calculate]{idc}{c}
		\function[calculate]{idd}{d} 
		\function{g}{vfe*cos(idb*x)-vfz*sin(idd*x)}
	\end{variables}
	\type{input.function}
	\begin{answer}
		\text{ $f''(x) = $}
		\solution{g}
		\checkAsFunction{x}{-10}{10}{100}		
	\end{answer}
\end{question}

%Frage 3 von 9
\begin{question}
	\lang{de}{
		\text{Gegeben $f(x) = e^{x^{\var{idn}}}$. Berechnen Sie $f''(x)$ und geben Sie Ihr Ergebnis ein.}
		\explanation{}
	}
	\lang{en}{
		\text{Let $f(x) = e^{x^{\var{idn}}}$. Calculate $f''(x)$ and input your answer below.}
	}
	\begin{variables}
        \randint[Z]{n}{3}{6}
		\function[calculate]{idn}{n}
		\function[calculate]{vfe}{n-1}
		\function{g}{x^(n-2)*n*exp(x^n)*(vfe+n*x^n)}
	\end{variables}
	\type{input.function}
	\begin{answer}
		\text{ $f''(x) = $}
		\solution{g}
		\checkAsFunction{x}{-10}{10}{100}		
	\end{answer}
\end{question}

%Frage 4 von 9
\begin{question}
	\lang{de}{
		\text{Gegeben $f(x) =\var{ida}x^3-\var{vfe}x^2+\var{idd}$. Wo besitzt $f$ eine Wendestelle?}
		\explanation{$f$ hat eine Wendestelle in $x$, wenn $f''(x)=0$ gilt und $f''$ in $x$ das Vorzeichen wechselt.}
	}
	\lang{en}{
		\text{Let $f(x) =\var{ida}x^3-\var{vfe}x^2+\var{idd}$. Where does $f$ have an inflection point?}
		\explanation{$f$ has an inflection point at $x$-values where both $f''(x)=0$ and where $f''$ changes sign.}
	}
	\begin{variables}
        \randint[Z]{a}{2}{4}
		\randint[Z]{b}{3}{6}
		\randint[Z]{c}{1}{5}
		\randint[Z]{d}{1}{7}
		\function[calculate]{vfe}{3*c*a}
		\function[calculate]{ida}{a}
		\function[calculate]{idd}{d}
	\end{variables}
	\type{input.number}
	\begin{answer}
  		\lang{de}{\text{ $f$ hat eine Wendestelle in:}}
  		\lang{en}{\text{$f$ has an inflection point at $x=$}}
		\solution{c}
	\end{answer}
\end{question}

%Frage 5 von 9
\begin{question}
	\lang{de}{
		\text{Gegeben $f(x) =\ln(\var{qa}x^2+\var{qb})$. In welchem $x>0$ besitzt $f$ eine Wendestelle?}
		\explanation{$f$ hat eine Wendestelle in $x$, wenn $f''(x)=0$ gilt und $f''$ in $x$ das Vorzeichen wechselt.}
	}
	\lang{en}{
		\text{Let $f(x) =\ln(\var{qa}x^2+\var{qb})$. At what $x>0$ does $f$ have an inflection point?}
		\explanation{$f$ has an inflection point at $x$-values where both $f''(x)=0$ and where $f''$ changes sign.}
	}
	\begin{variables}
        \randint[Z]{a}{2}{3}
	 	\randint[Z]{c}{1}{3}
		\function[calculate]{b}{c*a}
		\function[calculate]{qa}{a^2}
		\function[calculate]{qb}{c^2*a^2}
	\end{variables}
	\type{input.number}
	\begin{answer}
    	\lang{de}{\text{ $f$ hat eine Wendestelle in:}}
      	\lang{en}{\text{$f$ has an inflection point at $x=$}}
		\solution{c}
	\end{answer}
\end{question}
	
%Frage 6 von 9	
\begin{question}
	\lang{de}{
		\text{Gegeben $f(x) =\var{f}$. 
		Bestimmen Sie die Sattelstellen $x_1\leq x_2$\\ und die weitere Wendestelle $x_3$ von $f$. }
		\explanation{$f$ hat eine Wendestelle in $x$, wenn $f''(x)=0$ gilt und $f''$ in $x$ das Vorzeichen wechselt.\\
		Gilt zus\"{a}tzlich noch $f'(x)=0$, so liegt in $x$ eine Sattelstelle vor.}
	}
	\lang{en}{
		\text{Let $f(x)=\var{f}$.
		Find the saddle points $x_1\leq x_2$ and an additional inflection point $x_3$ of $f$.}
		\explanation{$f$ has an inflection point at $x$-values where both $f''(x)=0$ and where $f''$ changes sign.
		If in addition $f'(x)=0$ at that point, then $f$ has a saddle point there.}
	}
	\begin{variables}
	    \randint[Z]{m}{2}{4}
	    \randint[Z]{b}{1}{5}
	    \function[calculate]{vfe}{m^4}
	    \function[calculate]{vfz}{2*m^2}
		\function{f}{1/5*x^5-vfz/3*x^3+vfe*x+b}
	    \number{loes}{0}
	   	\function{loesz}{-m}
	   	\function{loesd}{m}
	\end{variables}
	\type{input.number}
	\begin{answer}
		\text{ $x_1$=}
		\solution{loesz}	
	\end{answer}
	\begin{answer}
		\text{ $x_2$=}
		\solution{loesd}	
	\end{answer}
	\begin{answer}
		\text{ $x_3$=}
		\solution{loes}	
	\end{answer}
\end{question}

%Frage 7 von 9
\begin{question}
	\lang{de}{
		\text{Gegeben $f(x) =\var{f}$. 
		Bestimmen Sie die Sattelstelle $x_1$\\ und die weitere Wendestelle $x_2$ von $f$. }
		\explanation{$f$ hat eine Wendestelle in $x$, wenn $f''(x)=0$ gilt und $f''$ in $x$ das Vorzeichen wechselt.\\
		Gilt zus\"{a}tzlich noch $f'(x)=0$, so liegt in $x$ eine Sattelstelle vor.}
	}
	\lang{en}{
		\text{Let $f(x)=\var{f}$.
		Find the saddle point $x_1$ and an additional inflection point $x_2$ of $f$.}
		\explanation{$f$ has an inflection point at $x$-values where both $f''(x)=0$ and where $f''$ changes sign.
		If in addition $f'(x)=0$ at that point, then $f$ has a saddle point there.}
	}
	\begin{variables}
	    \randint[Z]{a}{1}{3}
	    \function{s}{2*a-1}
	    \randint[Z]{b}{1}{5}
        \function[calculate, 2]{vfe}{9*s^2/2}
        \function[calculate, 2]{vfz}{2*s}
		\function{f}{0.25*x^4-vfz*x^3+vfe*x^2+b}
		\function{loesz}{6*a-3}
		\function{loesd}{2*a-1}
	\end{variables}
	\type{input.number}
	\begin{answer}
		\text{ $x_1$=}
		\solution{loesz}	
	\end{answer}
	\begin{answer}
		\text{ $x_2$=}
		\solution{loesd}	
	\end{answer}
\end{question}

%Frage 8 von 9
\begin{question}
	\lang{de}{
		\text{Gegeben $f(x) =\var{f}$. 
		Bestimmen Sie die Sattelstelle $x_1$\\ und die weitere Wendestelle $x_2$ von $f$. }
		\explanation{$f$ hat eine Wendestelle in $x$, wenn $f''(x)=0$ gilt und $f''$ in $x$ das Vorzeichen wechselt.\\
		Gilt zus\"{a}tzlich noch $f'(x)=0$, so liegt in $x$ eine Sattelstelle vor.}
	}
	\lang{en}{
		\text{Let $f(x)=\var{f}$.
		Find the saddle point $x_1$ and an additional inflection point $x_2$ of $f$.}
		\explanation{$f$ has an inflection point at $x$-values where both $f''(x)=0$ and where $f''$ changes sign.
		If in addition $f'(x)=0$ at that point, then $f$ has a saddle point there.}
	}
	\begin{variables}
		\randint[Z]{a}{1}{2}
        \function{s}{2*a}
        \randint[Z]{b}{1}{5}
	    \function[calculate, 2]{vfe}{18*a^2}
	    \function[calculate, 2]{vfz}{2*s}
    	\function{f}{0.25*x^4-vfz*x^3+vfe*x^2+b}  
		\function{loesz}{6*a}
		\function{loesd}{2*a}
	\end{variables}
	\type{input.number}
	\begin{answer}
		\text{ $x_1$=}
		\solution{loesz}
	\end{answer}
	\begin{answer}
		\text{ $x_2$=}
		\solution{loesd}
	\end{answer}	
\end{question}

%Frage 9 von 9
\begin{question}
	\lang{de}{
		\text{Gegeben $f(x) =\var{idb}x(x-\var{ida})^3$. Wo besitzt $f$ eine Sattelstelle?\\ Tipp: Leiten Sie den Term $(x-\var{ida})^3$ 
		mit der Kettenregel ab! }
		\explanation{$f$ besitzt in $x$ eine Sattelstelle, wenn $f'(x)=f''(x)=0$ gilt und zus\"{a}tzlich $f''$ in $x$ das Vorzeichen wechselt.}
	}
	\lang{en}{
		\text{Let $f(x) =\var{idb}x(x-\var{ida})^3$. Where does $f$ have a saddle point?\\
		Hint: Take the derivative of the term $(x-\var{ida})^3$ using the chain rule!}
		\explanation{$f$ has a saddle point at $x$-values where $f'(x)=f''(x)=0$ and where $f''$ changes sign.}
	}
	\begin{variables}
        \randint[Z]{a}{2}{3}
        \randint[Z]{b}{2}{7}
		\function[calculate]{ida}{a}  		
		\function[calculate]{idb}{b}
	\end{variables}
	\type{input.number}
	\begin{answer}
    	\lang{de}{\text{$f$ hat eine Sattelstelle in:}}
       	\lang{en}{\text{$f$ has a saddle point at $x=$}}
		\solution{a}	
	\end{answer}
\end{question}

\end{problem}

\embedmathlet{mathlet}

\end{content}