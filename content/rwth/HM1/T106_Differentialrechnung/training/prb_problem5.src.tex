\documentclass{mumie.problem.gwtmathlet}
%$Id$
\begin{metainfo}
  \name{
    \lang{de}{A05: Ableitungsregeln}
    \lang{en}{problem_5}
  }
  \begin{description} 
 This work is licensed under the Creative Commons License Attribution 4.0 International (CC-BY 4.0)   
 https://creativecommons.org/licenses/by/4.0/legalcode 

    \lang{de}{}
    \lang{en}{}
  \end{description}
  \corrector{system/problem/GenericCorrector.meta.xml}
  \begin{components}
    \component{js_lib}{system/problem/GenericMathlet.meta.xml}{mathlet}
  \end{components}
  \begin{links}
  \end{links}
  \creategeneric
\end{metainfo}
\begin{content}
\usepackage{mumie.ombplus}
\usepackage{mumie.genericproblem}


\lang{de}{
	\title{A05: Ableitungsregeln}
}
\lang{en}{
	\title{Problem 5}
}

\begin{block}[annotation]
      
\end{block}
\begin{block}[annotation]
  Im Ticket-System: \href{http://team.mumie.net/issues/9470}{Ticket 9470}
\end{block}

\begin{problem}	

\randomquestionpool{1}{2}
\randomquestionpool{3}{5}
\randomquestionpool{6}{8}
%begin-cosh
%Frage 1 von 8
\begin{question}
	\lang{de}{
		\text{Verwenden Sie die Kettenregel, um die Ableitung der Funktion\\
		$h(x) = f\circ g (x) = f(g(x)) = \var{h} ,\quad x \in\R$ zu bestimmen.}
	}
	\lang{en}{
		\text{Use the chain rule to find the derivative of the function \\
		$h(x) = f\circ g(x) = f(g(x)) = \var{h} , \quad x \in\R$.}
	}
 	\begin{variables}
 	    \randint{n}{2}{9}
 	    \randint{l}{1}{9}
 	    \function[calculate]{p}{2*n}
  		\function{f}{cos(y)}
 		\function{g}{n*x^2+l}
 		\function{h}{cos(n*x^2+l)}
 	   	\function{df}{-sin(y)}
   		\function{dg}{p*x}
    	\function{Dfg}{-p*x*sin(n*x^2+l)}
    	\function{dfg}{-sin(n*x^2+l)}
 	\end{variables}
    
    \explanation{Beachte, dass hier die Cosinus-Funktion (äußere Funktion) mit einem Polynom
    (innere Funktion) verkettet wurde. Die Ableitung von $\cos(x)$ ist $-\sin(x)$. Die Kettenregel lautet $(f\circ g)'(x)=f'(g(x))\cdot g'(x)$.}
    
	\type{input.function}
 	\field{real}
 	\begin{answer}
      	\lang{de}{\text{Äußere Funktion: $f(y) = $}}
      	\lang{en}{\text{Outer function: $f(y) = $}}
 		\solution{f}
 		\inputAsFunction{y}{f_s0}
 	\end{answer}
 	\begin{answer}
      	\lang{de}{\text{Innere Funktion: $g(x) = $}}
      	\lang{en}{\text{Inner function: $g(x) = $}}
 		\solution{g}
 		\inputAsFunction{x}{g_s0}
 		\checkFuncForZero{f_s0[g_s0] - h}{-10}{10}{100}
		\score{0.2}
 	\end{answer}
 	\begin{answer}
      	\lang{de}{\text{Äußere Ableitung: $f\prime(y) = $}}
      	\lang{en}{\text{Outer derivative: $f\prime(y) = $}}
 		\solution{df}
 		\inputAsFunction{y}{f_s1}
 		\checkFuncForZero{f_s1 - D[f_s0]}{-10}{10}{100}
 		\score{0.2}
 	\end{answer}
	\begin{answer}
      	\lang{de}{\text{Einsetzen: $\quad f\prime(g(x)) = $}}
      	\lang{en}{\text{Substitute: $\quad f\prime(g(x)) = $}}
 		\solution{dfg}
 		\inputAsFunction{x}{f_s2}
 		\checkFuncForZero{f_s2 - f_s1[g_s0]}{-10}{10}{100}
 		\score{0.2}
 	\end{answer}
	\begin{answer}
      	\lang{de}{\text{Innere Ableitung: $g\prime(x) = $}}
      	\lang{en}{\text{Inner derivative: $g\prime(x) = $}}
 		\solution{dg}
 		\inputAsFunction{x}{g_s1}
 		\checkFuncForZero{g_s1 - D[g_s0]}{-10}{10}{100}
 		\score{0.2}
 	\end{answer}
	\begin{answer}
      	\lang{de}{\text{Ableitung der Komposition: $(f \circ g)\prime(x) = $}}
      	\lang{en}{\text{Derivative of the composition: $(f\circ g)\prime(x) = $}}
 		\solution{Dfg}
 		\checkAsFunction{x}{-10}{10}{100}
 		\score{0.2}
 	\end{answer}
\end{question}
                 	
%Frage 2 von 8
\begin{question}
	\lang{de}{
		\text{Verwenden Sie die Kettenregel, um die Ableitung der Funktion\\
		$h(x) = f\circ g (x) = f(g(x)) = \var{h} ,\quad x \in\R$ zu bestimmen.}
	}
	\lang{en}{
		\text{Use the chain rule to find the derivative of the function \\
		$h(x) = f\circ g(x) = f(g(x)) = \var{h} , \quad \in\R$.}
	}
 	\begin{variables}
 	\randint{n}{2}{6}
 	    \randint{l}{1}{9}
 	    \function[calculate]{p}{4*n}
  		\function{f}{ln(y)}
 		\function{g}{n*x^4+l}
 		\function{h}{ln(n*x^4+l)}
 	   	\function{df}{1/y}
   		\function{dg}{p*x^3}
    	\function{Dfg}{p*x^3/(n*x^4+l)}
    	\function{dfg}{1/(n*x^4+l)}
 	\end{variables}
    
    \explanation{Bei der innren Funktion handelt es sich um ein Polynom. Die äußere Funktion ist der natürlich Logarithmus. Die Kettenregel lautet 
    $(f\circ g)'(x)=f'(g(x))\cdot g'(x)$.}
    
	\type{input.function}
 	\field{real}
 	\begin{answer}
      	\lang{de}{\text{Äußere Funktion: $f(y) = $}}
      	\lang{en}{\text{Outer function: $f(y) = $}}
 		\solution{f}
 		\inputAsFunction{y}{f_s0}
 	\end{answer}
 	\begin{answer}
      	\lang{de}{\text{Innere Funktion: $g(x) = $}}
      	\lang{en}{\text{Inner function: $g(x) = $}}
 		\solution{g}
 		\inputAsFunction{x}{g_s0}
 		\checkFuncForZero{f_s0[g_s0] - h}{1.1}{10}{100}
		\score{0.2}
 	\end{answer}
 	\begin{answer}
      	\lang{de}{\text{Äußere Ableitung: $f\prime(y) = $}}
      	\lang{en}{\text{Outer derivative: $f\prime(y) = $}}
 		\solution{df}
 		\inputAsFunction{y}{f_s1}
 		\checkFuncForZero{f_s1 - D[f_s0]}{1.1}{10}{100}
 		\score{0.2}
 	\end{answer}
	\begin{answer}
      	\lang{de}{\text{Einsetzen: $\quad f\prime(g(x)) = $}}
      	\lang{en}{\text{Substitute: $\quad f\prime(g(x)) = $}}
 		\solution{dfg}
 		\inputAsFunction{x}{f_s2}
 		\checkFuncForZero{f_s2 - f_s1[g_s0]}{1.1}{10}{100}
 		\score{0.2}
 	\end{answer}
	\begin{answer}
      	\lang{de}{\text{Innere Ableitung: $g\prime(x) = $}}
      	\lang{en}{\text{Inner derivative: $g\prime(x) = $}}
 		\solution{dg}
 		\inputAsFunction{x}{g_s1}
 		\checkFuncForZero{g_s1 - D[g_s0]}{1.1}{10}{100}
 		\score{0.2}
 	\end{answer}
	\begin{answer}
      	\lang{de}{\text{Ableitung der Komposition: $(f \circ g)\prime(x) = $}}
      	\lang{en}{\text{Derivative of the composition: $(f\circ g)\prime(x) = $}}
 		\solution{Dfg}
 		\checkAsFunction{x}{1.1}{10}{100}
 		\score{0.2}
 	\end{answer}
\end{question}
          
%ende-cosh    
%Frage 3 von 8
\begin{question}
	\lang{de}{
		\text{Gegeben sei $f(x) = (\var{p})\cdot \var{h}$. Berechnen Sie $f'(x)$ und geben Sie Ihr Ergebnis ein.}
		\explanation{}
	}
	\lang{en}{
		\text{Let $f(x) = (\var{p})\cdot \var{h}$. Find $f'(x)$ and input your answer below.}
	}
	\begin{variables}
        \randint[Z]{a}{2}{10}
		\randint[Z]{b}{2}{10}
 		\randint[Z]{m}{3}{7}
		\function[calculate]{s}{2*m}
	    \function{p}{a*x+b}
		\function{h}{ln(x^s+1)}
 		\derivative[normalize]{f_1}{p}{x}	       
 		\derivative[normalize]{f_3}{h}{x}
 		\function{sol}{f_1*h+p*f_3}
 		\functionNormalize{sol}  
	\end{variables}
    
    \explanation{Hier führt eine Kombination der Produkt- und Kettenregel zum Ziel. 
    Der erste Term im Produkt ist ein Polynom. Der zweite Term besteht aus einer Verkettung 
    des natürlichen Logarithmus mit einem Polynom. Für die Ableitung des zweiten Terms wird 
    deshalb die Kettenregel benötigt.}
    
	\type{input.function}
	\begin{answer}
		\text{ $f'(x) = $}
		\solution{sol}
		\checkAsFunction{x}{-10}{10}{100}		
	\end{answer}
\end{question}
	
%Frage 4 von 8   
\begin{question}
	\lang{de}{
		\text{Gegeben sei $f(x) = \var{p}\cdot \var{q}$. Berechnen Sie $f'(x)$ und geben Sie Ihr Ergebnis ein.}
		\explanation{}
	}
	\lang{en}{
		\text{Let $f(x) = \var{p}\cdot \var{q}$. Find $f'(x)$ and input your answer below. }
	}
	\begin{variables}
        \randint[Z]{a}{2}{10}
		\randint[Z]{b}{2}{10}
 		\randint[Z]{d}{3}{7}
 		\randint[Z]{m}{3}{7}
 		\randint[Z]{n}{3}{7}
		\function[calculate]{s}{2*m}
	    \function{p}{(a*x+b)^s}
		\function{q}{cos(x^n)}
 		\derivative[normalize]{f_1}{p}{x}	    
 		\derivative[normalize]{f_2}{q}{x}	    
 		\function{sol}{f_1*q+p*f_2}
 		\functionNormalize{sol} 
	\end{variables}
    
    \explanation{Hier führt eine Kombination der Produkt- und Kettenregel zum Ziel. 
    Der erste Term im Produkt ist ein Polynom. Der zweite Term besteht aus einer Verkettung 
    der Cosinus-Funktion mit einem Polynom. Für die Ableitung des zweiten Terms wird 
    deshalb die Kettenregel benötigt.}
    
    
	\type{input.function}
	\begin{answer}
		\text{ $f'(x) = $}
		\solution{sol}
		\checkAsFunction{x}{-5}{-0.2}{100}		
	\end{answer}
\end{question}
	
%Frage 5 von 8   
\begin{question}
	\lang{de}{
		\text{Gegeben sei $f(x) = \var{q}\cdot \var{h}$. Berechnen Sie $f'(x)$ und geben Sie Ihr Ergebnis ein.}
		\explanation{}
	}
	\lang{en}{
		\text{Let $f(x) = \var{q}\cdot \var{h}$. Find $f'(x)$ and input your answer below.}
	}
	\begin{variables}
 		\randint[Z]{d}{3}{7}
 		\randint[Z]{m}{3}{7}
 		\randint[Z]{n}{3}{7}
		\function[calculate]{s}{2*m}
		\function{h}{ln(x^s+1)}
		\function{q}{sin(x)}  
 		\derivative[normalize]{f_2}{q}{x}	    
 		\derivative[normalize]{f_3}{h}{x}
 		\function{sol}{f_2*h+q*f_3}
 		\functionNormalize{sol}  
	\end{variables}
    
    \explanation{Hier führt eine Kombination der Produkt- und Kettenregel zum Ziel. 
    Der erste Term im Produkt ist die Sinus-Funktion. Der zweite Term besteht aus einer Verkettung 
    des natürlichen Logarithmus mit einem Polynom. Für die Ableitung des zweiten Terms wird 
    deshalb die Kettenregel benötigt.}
    
    
    
	\type{input.function}
	\begin{answer}
		\text{ $f'(x) = $}
		\solution{sol}
		\checkAsFunction{x}{0.01}{10}{100}		
	\end{answer}
\end{question}
	
%Frage 6 von 8
\begin{question}
	\lang{de}{
		\text{Gegeben sei $f(x) = (\var{p})\cdot \var{q}+\var{h}$. Berechnen Sie $f'(x)$ und geben Sie Ihr Ergebnis ein.}
		\explanation{}
	}
	\lang{en}{
		\text{Let $f(x) = (\var{p})\cdot \var{q}+\var{h}$. Find $f'(x)$ and input your answer below.}
	}
	\begin{variables}
        \randint[Z]{a}{2}{10}
		\randint[Z]{b}{2}{10}
 		\randint[Z]{d}{3}{7}
 		\randint[Z]{l}{3}{7}
 		\randint[Z]{m}{3}{7} 
	    \function{p}{a*x+b}
		\function{q}{(cos(x))^l}
		\function{h}{e^(x^m)}
 		\derivative[normalize]{f_1}{p}{x}	    
 		\derivative[normalize]{f_2}{q}{x}	    
 		\derivative[normalize]{f_3}{h}{x}
 		\function{sol}{f_1*q+p*f_2+f_3}
 		\functionNormalize{sol}  
	\end{variables}
    
    \explanation{Hier führt eine Kombination der Produkt- und Kettenregel zum Ziel. 
    Der erste Term im Produkt ist ein Polynom. Der zweite Term des Produktes besteht aus einer Verkettung 
    der Cosinus-Funktion mit einem Monom $x^n$. Für die Ableitung des zweiten Terms wird 
    deshalb die Kettenregel benötigt. Der additive dritte Term ist eine Verkettung der Exponentialfunktion mit einem Monom $x^m$.}
    
    
    
	\type{input.function}
	\begin{answer}
		\text{ $f'(x) = $}
		\solution{sol}
		\checkAsFunction{x}{-10}{10}{100}		
	\end{answer}
\end{question}
	
%Frage 7 von 8
\begin{question}
	\lang{de}{
		\text{Gegeben sei $f(x) = \var{h}\cdot \var{q}+\var{p}$. Berechnen Sie $f'(x)$ und geben Sie Ihr Ergebnis ein.}
		\explanation{}
	}
	\lang{en}{
		\text{Let $f(x) = \var{h}\cdot \var{q}+\var{p}$. Find $f'(x)$ and input your answer below.}
	}
	\begin{variables}
        \randint[Z]{a}{2}{10}
		\randint[Z]{b}{2}{10}
 		\randint[Z]{d}{3}{7}
 		\randint[Z]{l}{3}{7}
 		\randint[Z]{m}{3}{7}
		\function[calculate]{s}{2*m}
		\function{p}{a*x+b}
		\function{q}{x^l}
		\function{h}{ln(x^s+1)}
 		\derivative[normalize]{f_1}{h}{x}	    
 		\derivative[normalize]{f_2}{q}{x}	    
 		\derivative[normalize]{f_3}{p}{x}
 		\function{sol}{f_1*q+h*f_2+f_3}
 		\functionNormalize{sol} 
	\end{variables}
	\type{input.function}
	\begin{answer}
		\text{ $f'(x) = $}
		\solution{sol}
		\checkAsFunction{x}{-10}{10}{100}	
        
        \explanation{Hier muss eine Kombination aus Kette- und Produktregel angewandt werden. Der erste Term ist ein Produkt zwischen einem Polynom und dem natürlichen Logarithmus
        , wobei der Logarithmus mit einem Polynom verkettet wurde.}
        
        
        
        
	\end{answer}
\end{question} 

%Frage 8 von 8
\begin{question}
	\lang{de}{
		\text{Gegeben sei $f(x) = (\var{p})\cdot \var{q}+\var{h}$. Berechnen Sie $f'(x)$ und geben Sie Ihr Ergebnis ein.}
		\explanation{}
	}
	\lang{en}{
		\text{Let $f(x) = (\var{p})\cdot \var{q}+\var{h}$. Find $f'(x)$ and input your answer below.}
	}
	\begin{variables}
        \randint[Z]{a}{2}{10}
		\randint[Z]{b}{2}{10}
 		\randint[Z]{d}{3}{7}
 		\randint[Z]{l}{3}{7}
 		\randint[Z]{m}{3}{7}
	    \function{p}{a*x+b}
		\function{h}{(sin(x))^l}
		\function{q}{e^(x^m)}
 		\derivative[normalize]{f_1}{p}{x}	    
 		\derivative[normalize]{f_2}{q}{x}	    
 		\derivative[normalize]{f_3}{h}{x}
 		\function{sol}{f_1*q+p*f_2+f_3}
 		\functionNormalize{sol}
	\end{variables}
	\type{input.function}
	\begin{answer}
		\text{ $f'(x) = $}
		\solution{sol}
		\checkAsFunction{x}{-10}{10}{100}	
        \explanation{Hier muss eine Kombination aus Kette- und Produktregel angewandt werden. Der erste Term ist ein Produkt zwischen einem Polynom und einer Exponetialfunktion
        , wobei die Exponentialfunktion mit einem Monom $x^m$ verkettet wurde. Der zweite Term ist eine Verkettung eines Monoms $x^n$ mit der Sinus-Funktion.}
        
	\end{answer}
\end{question}
	
	
\end{problem}	

\embedmathlet{mathlet}
\end{content}