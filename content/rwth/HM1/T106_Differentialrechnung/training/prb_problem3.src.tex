\documentclass{mumie.problem.gwtmathlet}
%$Id$
\begin{metainfo}
  \name{
    \lang{de}{A03: Komposition}
    \lang{en}{problem_3}
  }
  \begin{description} 
 This work is licensed under the Creative Commons License Attribution 4.0 International (CC-BY 4.0)   
 https://creativecommons.org/licenses/by/4.0/legalcode 

    \lang{de}{}
    \lang{en}{}
  \end{description}
  \corrector{system/problem/GenericCorrector.meta.xml}
  \begin{components}
    \component{js_lib}{system/problem/GenericMathlet.meta.xml}{mathlet}
  \end{components}
  \begin{links}
  \end{links}
  \creategeneric
\end{metainfo}
\begin{content}
\usepackage{mumie.ombplus}
\usepackage{mumie.genericproblem}


\lang{de}{
	\title{A03: Komposition}
}
\lang{en}{
	\title{Problem 3}
}

\begin{block}[annotation]
      
\end{block}
\begin{block}[annotation]
  Im Ticket-System: \href{http://team.mumie.net/issues/9489}{Ticket 9489}
\end{block}

\begin{problem}
 
 
 
 
 \begin{variables}
    	 \randint{c1}{1}{2}
    	 \function[calculate]{g1}{dirac(c1-1)}
    	 \function[calculate]{g2}{dirac(c1-2)}
    	 \function[normalize]{hf}{g1*ln(x)+g2*e^x}
    	 \function[normalize]{hg}{g1*e^x+g2*ln(x)}
    	 \function[normalize]{hgf}{g1*e^(ln(x))+g2*ln(e^x)}
    	 
    	 \randint{c2}{1}{3}
    	 \function[calculate]{g3}{dirac(c2-1)}
    	 \function[calculate]{g4}{dirac(c2-2)}
    	 \function[calculate]{g5}{dirac(c2-3)}
    	 \function{p1}{g3*hf+g4*hg+g5*hgf}
    	 \function{p2}{g3*hg+g4*hgf+g5*hf}
    	 \function{p3}{g3*hgf+g4*hf+g5*hg}
    	 
    	 \randint{c3}{1}{6}
    	 \function[calculate]{g6}{dirac(c3-1)}
    	 \function[calculate]{g7}{dirac(c3-2)}
    	 \function[calculate]{g8}{dirac(c3-3)}
    	 \function[calculate]{g9}{dirac(c3-4)}
    	 \function[calculate]{g10}{dirac(c3-5)}
    	 \function[calculate]{g11}{dirac(c3-6)}
    	 \function[normalize]{jf}{(g6+g7)*x^2+(g8+g9)*e^x+(g10+g11)*cos(x)}
    	 \function[normalize]{jg}{(g8+g10)*x^2+(g6+g11)*e^x+(g7+g9)*cos(x)}
    	 \function[normalize]{jgf}{g6*(e^(x^2))+g7*cos(x^2)+g8*(e^x)^2+g9*cos(e^x)+g10*cos(x)^2+g11*e^(cos(x))}
    	 
    	 \randint{c4}{1}{3}
    	 \function[calculate]{g12}{dirac(c4-1)}
    	 \function[calculate]{g13}{dirac(c4-2)}
    	 \function[calculate]{g14}{dirac(c4-3)}
    	 \function{p4}{g12*jf+g13*jg+g14*jgf}
    	 \function{p5}{g12*jg+g13*jgf+g14*jf}
    	 \function{p6}{g12*jgf+g13*jf+g14*jg}
    	     	 
    	\end{variables} 
      

	\randomquestionpool{1}{2}
 
    \begin{question}
    	 \text{Bestimmen Sie den Definitionsbereich von $g_1 \circ f_1$ für die nachfolgend angegebenen Funktionen und ordnen Sie die farbigen Funktionsgraphen jeweils richtig zu.\\
    	 Seien $f_1(x)=\var{hf}$ und $g_1(x)=\var{hg}$.}
    	 
    	 
    	 \plotF{1}{p1} % % the function a1 is defined below in 'variables' in the usual way
    	 \plotFrom{1}{0} % % and is plotted starting from 0.0
   		 \plotTo{1}{2} % % and ending in 1.0 , it's a quarter of a circle
    	\plotColor{1}{blue} % % colored blu
    	\plotF{2}{p2} % % the function a1 is defined below in 'variables' in the usual way
    	\plotFrom{2}{0} % % and is plotted starting from 0.0
    	\plotTo{2}{2} % % and ending in 1.0 , it's a quarter of a circle
    	\plotColor{2}{red} % % colored red
    	\plotF{3}{p3} % % the function a1 is defined below in 'variables' in the usual way
    	\plotFrom{3}{0} % % and is plotted starting from 0.0
    	\plotTo{3}{2} % % and ending in 1.0 , it's a quarter of a circle
    	\plotColor{3}{green} % % colored magenta
    	\plotLeft{0} % % defines the canvas bound left
    	\plotRight{2} % % and right
    	\plotSize{400}		
    	
    \type{input.generic}
    \begin{answer}
        \text{Der blaue Funktionsgraph gehört zu der Funktion:}
    	\type{mc.multiple}
		\permutechoices{1}{3}
		\begin{choice}
  			\lang{de}{\text{$f_1$}}
			\solution{compute}
			\iscorrect{g3}{=}{1}	
		\end{choice}
		\begin{choice}
  			\lang{de}{\text{$g_1$}}
			\solution{compute}
			\iscorrect{g4}{=}{1}		
		\end{choice}
		\begin{choice}
  			\lang{de}{\text{ $g_1 \circ f_1$}}
  			\solution{compute}
  			\iscorrect{g5}{=}{1}			
		\end{choice}
	\end{answer}
	\begin{answer}
		\text{Der richtige Definitionsbereich von $g_1 \circ f_1$ ist:}
		\type{mc.multiple}
		\permutechoices{1}{6}
		\begin{choice}
  			\lang{de}{\text{$\mathbb{R}$}}
			\solution{compute}
			\iscorrect{g2}{=}{1}	
		\end{choice}
		\begin{choice}
  			\lang{de}{\text{$(0,\infty)$}}
			\solution{compute}
			\iscorrect{g1}{=}{1}		
		\end{choice}
		\begin{choice}
  			\lang{de}{\text{$(-\infty,0)$}}
  			\solution{false}			
		\end{choice}
		\begin{choice}
  			\lang{de}{\text{$(0,1)$}}
  			\solution{false}			
		\end{choice}
		\begin{choice}
  			\lang{de}{\text{$(-1,1)$}}
  			\solution{false}			
		\end{choice}
		\begin{choice}
  			\lang{de}{\text{$(-1,0)$}}
  			\solution{false}			
		\end{choice}
          \explanation{Der Definitionsbereich einer Verkettung von Funktionen $f(g(x))$ besteht aus allen Punkte des Definitionsbereichs der inneren Funktion $g(x)$, die auch im Definitionsbereich der äußeren Funktion $f(x)$ liegen.}
	\end{answer}
    
	\end{question}
	
%Zweiter Fragenpart
	\begin{question}
    	 \text{Bestimmen Sie den Definitionsbereich von $g_2 \circ f_2$ für die nachfolgend angegebenen Funktionen und ordnen Sie die farbigen Funktionsgraphen jeweils richtig zu.\\
    	 Seien $f_2(x)=\var{jf}$ und $g_2(x)=\var{jg}$.}
    	 
    	 
    	 \plotF{1}{p4} % % the function a1 is defined below in 'variables' in the usual way
    	 \plotFrom{1}{-2} % % and is plotted starting from 0.0
   		 \plotTo{1}{2} % % and ending in 1.0 , it's a quarter of a circle
    	\plotColor{1}{blue} % % colored blu
    	\plotF{2}{p5} % % the function a1 is defined below in 'variables' in the usual way
    	\plotFrom{2}{-2} % % and is plotted starting from 0.0
    	\plotTo{2}{2} % % and ending in 1.0 , it's a quarter of a circle
    	\plotColor{2}{red} % % colored red
    	\plotF{3}{p6} % % the function a1 is defined below in 'variables' in the usual way
    	\plotFrom{3}{-2} % % and is plotted starting from 0.0
    	\plotTo{3}{2} % % and ending in 1.0 , it's a quarter of a circle
    	\plotColor{3}{green} % % colored magenta
    	\plotLeft{-2} % % defines the canvas bound left
    	\plotRight{2} % % and right
    	\plotSize{400}		
    	
    \type{input.generic}
    \begin{answer}
        \text{Der blaue Funktionsgraph gehört zu der Funktion:}
    	\type{mc.multiple}
		\permutechoices{1}{3}
		\begin{choice}
  			\lang{de}{\text{$f_2$}}
			\solution{compute}
			\iscorrect{g12}{=}{1}	
		\end{choice}
		\begin{choice}
  			\lang{de}{\text{$g_2$}}
			\solution{compute}
			\iscorrect{g13}{=}{1}		
		\end{choice}
		\begin{choice}
  			\lang{de}{\text{ $g_2 \circ f_2$}}
  			\solution{compute}
  			\iscorrect{g14}{=}{1}			
		\end{choice}
	\end{answer}
	\begin{answer}
		\text{Der richtige Definitionsbereich von $g_2 \circ f_2$ ist:}
		\type{mc.multiple}
		\permutechoices{1}{6}
		\begin{choice}
  			\lang{de}{\text{$\mathbb{R}$}}
			\solution{true}	
		\end{choice}
		\begin{choice}
  			\lang{de}{\text{$(0,\infty)$}}
			\solution{false}		
		\end{choice}
		\begin{choice}
  			\lang{de}{\text{$(-\infty,0)$}}
  			\solution{false}			
		\end{choice}
		\begin{choice}
  			\lang{de}{\text{$(0,1)$}}
  			\solution{false}			
		\end{choice}
		\begin{choice}
  			\lang{de}{\text{$(-1,1)$}}
  			\solution{false}			
		\end{choice}
		\begin{choice}
  			\lang{de}{\text{$(-1,0)$}}
  			\solution{false}			
		\end{choice}
 \explanation{Der Definitionsbereich einer Verkettung von Funktionen $f(g(x))$ besteht aus allen Punkte des Definitionsbereichs der inneren Funktion $g(x)$, die auch im Definitionsbereich der äußeren Funktion $f(x)$ liegen.}
	\end{answer}
	\end{question}
	
\end{problem}

\embedmathlet{mathlet}



\end{content}