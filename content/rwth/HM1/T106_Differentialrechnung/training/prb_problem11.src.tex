\documentclass{mumie.problem.gwtmathlet}
%$Id$
\begin{metainfo}
  \name{
    \lang{de}{A11: Kritische Stellen}
    \lang{en}{problem_11}
  }
  \begin{description} 
 This work is licensed under the Creative Commons License Attribution 4.0 International (CC-BY 4.0)   
 https://creativecommons.org/licenses/by/4.0/legalcode 

    \lang{de}{}
    \lang{en}{}
  \end{description}
  \corrector{system/problem/GenericCorrector.meta.xml}
  \begin{components}
    \component{js_lib}{system/problem/GenericMathlet.meta.xml}{mathlet}
  \end{components}
  \begin{links}
  \end{links}
  \creategeneric
\end{metainfo}
\begin{content}
\usepackage{mumie.ombplus}
\usepackage{mumie.genericproblem}


\lang{de}{
	\title{A11: Kritische Stellen}
}
\lang{en}{
	\title{Problem 11}
}

\begin{block}[annotation]
      
\end{block}
\begin{block}[annotation]
  Im Ticket-System: \href{http://team.mumie.net/issues/9476}{Ticket 9476}
\end{block}


\begin{problem}

\begin{variables}
	\randint[Z]{s}{-1}{1}
    \randint{c}{2}{8}
    \randint{a}{2}{4}
	\randint{q}{1}{9}
	\function[calculate]{v}{c^2}
	\function[calculate]{d}{c*q}
	\function[calculate]{z}{s*a}
	\function[calculate]{w}{d^2}
	\function[calculate]{exte}{-q}
	\function[calculate]{extz}{q}
	\function{f}{(z*x)/(v*x^2+w)} 
\end{variables}

%Frage 1 von 4
 \begin{question}
	\lang{de}{
		\text{Gegeben ist die Funktion $f(x)=\var{f}$. Wo hat $f$ kritische Stellen?}
	    \explanation{Um die kritischen Stellen zu bestimmen benötigt man die Nullstellen der ersten Ableitung.
        Für die Bestimmung der ersten Ableitung bietet sich
        hier die Quotientenregel an, da $f(x)$ ein Quotient zweier Polynome ist. Anschließend muss die so erhaltene erste Ableitung gleich Null
        gesetzt und nach $x$ aufgelöst werden.}
	}
	\lang{en}{
		\text{Let $f(x)=\var{f}$. Where does $f$ have critical points?}
	}
	\type{input.number}
	\begin{answer}
  		\lang{de}{\text{ Die kleinere der kritischen Stellen von $f$ ist $x_0=$}}
   		\lang{en}{\text{The smaller of the critical points of $f$ is $x_0=$}}
		\solution{exte}
	\end{answer}		
	\begin{answer}
  		\lang{de}{\text{ Die gr\"{o}{\ss}ere der kritischen Stellen von $f$ ist $x_1=$}}
   		\lang{en}{\text{The larger of the critical points of $f$ is $x_1=$}}
		\solution{extz}
	\end{answer}
\end{question}

%Frage 2 von 4
\begin{question}
    \lang{de}{
		\text{Liegen in diesen kritischen Stellen $x_0$ und $x_1$ Extrema vor? Falls ja, welcher Art sind sie?}
		\explanation{Um zu prüfen, ob Maxima oder Minima vorliegen, kann man 
        prüfen, ob $f'(x)$ in den kritischen Stellen das Vorzeichen wechselt. Dazu reicht es, den Zähler 
        der ersten Ableitung zu betrachten, da der Nenner immer positiv ist. Eine andere Möglichkeit wäre es, die zweite Ableitung zu bilden und zu schauen, 
        ob sie in den kritischen Stellen positiv oder negativ ist.}
	}
	\lang{en}{
		\text{Are there extrema at the points $x_0$ and $x_1$? If so, which type of extrema are they?}
	}
	\type{mc.multiple}
	\begin{choice}
    	\lang{de}{\text{In $x_0$ liegt kein Extremum vor.}}
		\lang{en}{\text{There is no extremum at $x_0$.}}
  		\solution{false}
	\end{choice}
	\begin{choice}
   		\lang{de}{\text{In $x_0$ liegt ein lokales Minimum vor.}}
   		\lang{en}{\text{There is a local minimum at $x_0$.}}
   		\solution{compute}
 		\iscorrect{s}{=}{1}
 	\end{choice}
 	\begin{choice}
 		\lang{de}{\text{In $x_0$ liegt ein lokales Maximum vor.}}
   		\lang{en}{\text{There is a local maximum at $x_0$.}}	
   		\solution{compute}
 		\iscorrect{s}{=}{-1}
 	\end{choice}
 	\begin{choice}
    	\lang{de}{\text{In $x_1$ liegt kein Extremum vor.}}
		\lang{en}{\text{There is no extremum at $x_1$.}}
  		\solution{false}
	\end{choice}
	\begin{choice}
   		\lang{de}{\text{In $x_1$ liegt ein lokales Minimum vor.}}
   		\lang{en}{\text{There is a local minimum at $x_1$.}}
   		\solution{compute}
 		\iscorrect{s}{=}{-1}
 	\end{choice}
 	\begin{choice}
 		\lang{de}{\text{In $x_1$ liegt ein lokales Maximum vor.}}
   		\lang{en}{\text{There is a local maximum at $x_1$.}}	
   		\solution{compute}
 		\iscorrect{s}{=}{1}
 	\end{choice}
\end{question}
	

\begin{variables}
	\randint[Z]{t}{-1}{1}
    \randint{j}{1}{3}
    \randint{p}{2}{5}
	\randint{l}{1}{9}
	\function[calculate]{u}{t*2*j^2}
	\function[calculate]{k}{p*j}
	\function[calculate]{m}{-t*k^2}
	\function[calculate]{ee}{-p/2}
	\function[calculate]{ed}{p/2}
	\number{result}{0}
	\function{g}{u*x^4+m*x^2+l} 
\end{variables}

%Frage 3 von 4
\begin{question}
	\lang{de}{
		\text{Gegeben ist die Funktion $f(x)=\var{g}$. Welche sind die kritische Stellen $x_0$, $x_1$, $x_2$ von $f$ mit $x_0<x_1<x_2$?}
	    \explanation{Um die kritischen Stellen zu bestimmen, benötigt man die Nullstellen der ersten Ableitung. 
        Anschließend muss die so erhaltene erste Ableitung gleich Null gesetzt und nach $x$ aufgelöst werden.}
	}
	\lang{en}{
		\text{Let $f(x)=\var{g}$. What are the critical points $x_0$, $x_1$, $x_2$ of $f$ where $x_0<x_1<x_2$?}
	}
	\type{input.number}
	\begin{answer}
		\text{$x_0=$}
		\solution{ee}
	\end{answer}		
	\begin{answer}
		\text{$x_1=$}
		\solution{result}			
	\end{answer}
	\begin{answer}
		\text{$x_2=$}
		\solution{ed}
	\end{answer}
\end{question}
	 
%Frage 4 von 4
\begin{question}
	\lang{de}{
		\text{Liegen in diesen kritischen Stellen $x_0$, $x_1$ und $x_2$ Extrema vor? Falls ja, welcher Art sind sie?}
		\explanation{Um zu prüfen, ob Maxima oder Minima vorliegen, kann man 
        prüfen, ob $f'(x)$ in den kritischen Stellen das Vorzeichen wechselt. Eine andere Möglichkeit
        wäre es, die zweite Ableitung zu bilden und zu schauen, 
        ob sie in den kritischen Stellen positiv oder negativ ist. }
	}
	\lang{en}{
		\text{Are there extrema at the critical points $x_0$, $x_1$, and $x_2$? If so, what type of extrema are they?}
	}
	\type{mc.multiple}
	\begin{choice}
    	\lang{de}{\text{In $x_0$ liegt kein Extremum vor.}}
		\lang{en}{\text{There is no extremum at $x_0$.}}
  		\solution{false}
	\end{choice}
	\begin{choice}
   		\lang{de}{\text{In $x_0$ liegt ein lokales Minimum vor.}}
   		\lang{en}{\text{There is a local minimum at $x_0$.}}	
   		\solution{compute}
 		\iscorrect{t}{=}{1}
 	\end{choice}
 	\begin{choice}
 		\lang{de}{\text{In $x_0$ liegt ein lokales Maximum vor.}}
   		\lang{en}{\text{There is a local maximum at $x_0$.}}	
   		\solution{compute}
 		\iscorrect{t}{=}{-1}
 	\end{choice}
 	\begin{choice}
    	\lang{de}{\text{In $x_1$ liegt kein Extremum vor.}}
		\lang{en}{\text{There is no extremum at $x_1$.}}
  		\solution{false}
	\end{choice}
	\begin{choice}
   		\lang{de}{\text{In $x_1$ liegt ein lokales Minimum vor.}}
  		\lang{en}{\text{There is a local minimum at $x_1$.}}
   		\solution{compute}
 		\iscorrect{t}{=}{-1}
 	\end{choice}
 	\begin{choice}
 		\lang{de}{\text{In $x_1$ liegt ein lokales Maximum vor.}}
   		\lang{en}{\text{There is a local maximum at $x_1$.}}	
   		\solution{compute}
 		\iscorrect{t}{=}{1}
 	\end{choice}
 	\begin{choice}
    	\lang{de}{\text{In $x_2$ liegt kein Extremum vor.}}
		\lang{en}{\text{There is no extremum at $x_2$.}} 
  		\solution{false}
	\end{choice}
	\begin{choice}
   		\lang{de}{\text{In $x_2$ liegt ein lokales Minimum vor.}}
   		\lang{en}{\text{There is a local minimum at $x_2$.}}	
   		\solution{compute}
 		\iscorrect{t}{=}{1}
 	\end{choice}
 	\begin{choice}
 		\lang{de}{\text{In $x_2$ liegt ein lokales Maximum vor.}}
 		\lang{en}{\text{There is a local maximum at $x_2$.}}
   		\solution{compute}
 		\iscorrect{t}{=}{-1}
 	\end{choice}
\end{question}


\end{problem}

\embedmathlet{mathlet}




\end{content}