
%$Id:  $
\documentclass{mumie.article}
%$Id$
\begin{metainfo}
  \name{
    \lang{de}{Überblick: Differentialrechnung}
    \lang{en}{Overview: Differentiation}
  }
  \begin{description} 
 This work is licensed under the Creative Commons License Attribution 4.0 International (CC-BY 4.0)   
 https://creativecommons.org/licenses/by/4.0/legalcode 

    \lang{de}{Beschreibung}
    \lang{en}{Description}
  \end{description}
  \begin{components}
  \end{components}
  \begin{links}
\link{generic_article}{content/rwth/HM1/T106_Differentialrechnung/g_art_content_23_kurvendiskussion.meta.xml}{content_23_kurvendiskussion}
\link{generic_article}{content/rwth/HM1/T106_Differentialrechnung/g_art_content_22_extremstellen.meta.xml}{content_22_extremstellen}
\link{generic_article}{content/rwth/HM1/T106_Differentialrechnung/g_art_content_21_kettenregel.meta.xml}{content_21_kettenregel}
\link{generic_article}{content/rwth/HM1/T106_Differentialrechnung/g_art_content_20_ableitung_als_tangentensteigung.meta.xml}{content_20_ableitung_als_tangentensteigung}
\end{links}
  \creategeneric
\end{metainfo}
\begin{content}
\begin{block}[annotation]
	Im Ticket-System: \href{https://team.mumie.net/issues/30143}{Ticket 30143}
\end{block}
\begin{block}[annotation]
Copy of : /home/mumie/checkin/content/rwth/HM1/T107_Integralrechnung/art_T107_overview.src.tex
\end{block}



\begin{block}[annotation]
Im Entstehen: Überblicksseite für Kapitel Differentialrechnung
\end{block}

\usepackage{mumie.ombplus}
\ombchapter{1}
\title{\lang{de}{Überblick: Differentialrechnung}\lang{en}{Overview: Differentiation}}



\begin{block}[info-box]
\lang{de}{\strong{Inhalt}}
\lang{en}{\strong{Contents}}


\lang{de}{
    \begin{enumerate}%[arabic chapter-overview]
   \item[6.1] \link{content_20_ableitung_als_tangentensteigung}{Ableitung und Ableitungsformel}
   \item[6.2] \link{content_21_kettenregel}{Kettenregel}
   \item[6.3] \link{content_22_extremstellen}{Extremstellen}
   \item[6.4] \link{content_23_kurvendiskussion}{Kurvendiskussion}
     \end{enumerate}
}
\lang{en}{
    \begin{enumerate}%[arabic chapter-overview]
   \item[6.1] \link{content_20_ableitung_als_tangentensteigung}{Differentiation}
   \item[6.2] \link{content_21_kettenregel}{The chain rule}
   \item[6.3] \link{content_22_extremstellen}{Monotonicity and extrema}
   \item[6.4] \link{content_23_kurvendiskussion}{Curves}
     \end{enumerate}
} %lang

\end{block}

\begin{zusammenfassung}

\lang{de}{
Dieses Kapitel nimmt das aus der Schule bekannte Konzept der Ableitung auf. 
Die Ableitung einer Funktion in einem Punkt beschreibt, wenn sie denn existiert, deren Änderung an 
dieser Stelle. Anschaulich bedeutet das, dass man an dieser Stelle genau eine Tangente an den 
Funktionsgraphen legen kann, deren Steigung gerade die Ableitung ist.
\\\\
Wir listen die Ableitungsfunktionen der meisten elementaren Funktionen auf und geben mit Summen-, 
Produkt- und Kettenregel die wichtigsten Bildungsgesetze für Ableitungen zusammengesetzter Funktionen.
\\\\
Wir definieren Funktionseigenschaften wie Monotonie, Minima und Maxima. Diese Eigenschaften werden 
bei differenzierbaren Funktionen von ihrer Ableitung beeinflusst. Mit Hilfe der zweiten Ableitung 
können Extrema, Sattelstellen und die Krümmung einer Funktion dingfest gemacht werden.
\\\\
Damit können in einer Kurvendiskussion die Eigenschaften einer Funktion so gut beschrieben werden, 
dass ihr Graph skizziert werden kann.
}
\lang{en}{
This chapter begins with the concept of derivatives, and how to determine them (differentiation). 
The derivative of a function at a point, if it exists, describes the slope of the graph at that 
point. Graphically this corresponds to the gradient of the tangent to the graph at that point.
\\\\
We list the derivatives of the most commonly used functions and cover the sum, product and chain 
rule, used for differentiating combinations of functions.
\\\\
We define properties of a function such as monotonicity, minima and maxima. For differentiable 
functions, these properties are closely related to the derivative. Employing also the second 
derivative, we can rigorously define concavity, turning points and inflection points, and distinguish 
between extrema.
\\\\
The graph of a function can often be relatively accurately sketched, knowing only the properties of a 
given function listed above.
}


\end{zusammenfassung}

\begin{block}[info]\lang{de}{\strong{Lernziele}}
\lang{en}{\strong{Learning Goals}} 
\begin{itemize}[square]
\item \lang{de}{
      Sie stellen Tangentengleichungen an Graphen differenzierbarer Funktionen auf.
      }
      \lang{en}{
      Being able to derive the definition of the derivative of a function at a point using the limit 
      of a gradient of line approaching the tangent ('from first principles'). Knowing then the 
      definition of a differentiable function.
      }
\item \lang{de}{Sie kennen die Ableitungsfunktionen der bisher diskutieren elementaren Funktionen.}
      \lang{en}{Knowing the derivatives of the aforementioned elementary functions.}
\item \lang{de}{
      Sie wenden die Ableitungsregeln für Summen, Produkte, Quotienten und Verkettungen sicher an.
      }
      \lang{en}{
      Being able to apply the rules for differentiation for sums, products, quotients and composite 
      functions.
      }
\item \lang{de}{
      Sie führen Kurvendiskussionen durch, wenden also notwendige und hinreichende Kriterien an, um 
      eine Funktion auf Extrem-, Sattel- und Wendestellen zu untersuchen und ihr Monotonie- wie 
      Krümmungsverhalten zu charakterisieren.
      }
      \lang{en}{
      Being able to determine the extrema, turning points and inflection points of a function and 
      classify these, as well as its monotony and concavity in given intervals.
      }
\item \lang{de}{Sie skizzieren Funktionsgraphen auf Grundlage einer Kurvendiskussion.}
      \lang{en}{Being able to sketch functions using the aforementioned properties.}
\end{itemize}
\end{block}




\end{content}
