%$Id:  $
\documentclass{mumie.article}
%$Id$
\begin{metainfo}
  \name{
    \lang{de}{Ableitung und Ableitungsformeln}
    \lang{en}{Differentiation}
  }
  \begin{description} 
 This work is licensed under the Creative Commons License Attribution 4.0 International (CC-BY 4.0)   
 https://creativecommons.org/licenses/by/4.0/legalcode 

    \lang{de}{Beschreibung}
    \lang{en}{Description}
  \end{description}
  \begin{components}
    \component{generic_image}{content/rwth/HM1/images/g_tkz_T106_Tangent_B.meta.xml}{T106_Tangent_B}
    \component{generic_image}{content/rwth/HM1/images/g_tkz_T106_Tangent_A.meta.xml}{T106_Tangent_A}
    \component{js_lib}{system/media/mathlets/GWTGenericVisualization.meta.xml}{mathlet1}
    \component{generic_image}{content/rwth/HM1/images/g_img_00_video_button_schwarz-blau.meta.xml}{00_video_button_schwarz-blau}
  \end{components}
  \begin{links}
    \link{generic_article}{content/rwth/HM1/T104_weitere_elementare_Funktionen/g_art_content_14_potenzregeln.meta.xml}{power-rules}
  \end{links}
  \creategeneric
\end{metainfo}
\begin{content}
\usepackage{mumie.ombplus}
\ombchapter{6}
\ombarticle{1}
\usepackage{mumie.genericvisualization}

\begin{visualizationwrapper}

\title{\lang{de}{Ableitung und Ableitungsformeln}\lang{en}{Differentiation}}
 
\begin{block}[annotation]
  übungsinhalt
  
\end{block}
\begin{block}[annotation]
  Im Ticket-System: \href{http://team.mumie.net/issues/9031}{Ticket 9031}\\
\end{block}
\begin{block}[info-box]
\tableofcontents
\end{block}

%\begin{block}[info-box]
\lang{de}{
Die Differentialrechnung stellt Werkzeuge zur Verfügung, mit denen das Änderungsverhalten von 
Funktionen "im Kleinen" untersucht werden kann. Dabei können beispielsweise lokale Extrema (Hoch- 
und Tiefpunkte) ermittelt werden. 
Auch eine lineare Approximation nicht-linearer Funktionen ist dadurch lokal möglich.
}
\lang{en}{
Differentiation is a way of finding the rate of change of a function over 'arbitrarily small' parts 
of the domain. It can for example be used to determine local extrema (maximum and minimum points) of 
a function. It can also be used for constructing, for example, linear approximations to non-linear 
functions.
}
%\end{block}

\section{
	\lang{de}{Die Ableitung als Tangentensteigung}
	\lang{en}{The derivative as the gradient of the tangent}\label{sec:ableitung}
}

% In der Analysis interessiert man sich häufig für Eigenschaften von Kurven wie Extremstellen, Steigungs- oder Kr\"{u}mmungsverhalten. Oder
% man m\"{o}chte einen Funktionswert an einer Stelle n\"{a}herungsweise bestimmen, ohne ihn konkret auszurechnen. Bei all diesen Problemen
% ist die sogenannte Ableitung einer Funktion n\"{u}tzlich.\\
\lang{de}{
Die Ableitung einer Funktion $f$ an einer Stelle $x_0$ soll die Steigung der Tangente an den Graphen von $f$ im Punkt $(x_0;f(x_0))$ angeben.
Dies ist zugleich ein Ma{\ss} f\"{u}r die Steigung von $f$ in $x_0$.\\
Diese Tangente ist eine Gerade, die den Graphen im Punkt $P_0=(x_0;f(x_0))$ \textit{berührt}.
Um rechnerisch zu verstehen, was dies bedeutet, betrachtet man zunächst Sekanten des Graphen durch den Punkt $P_0=(x_0;f(x_0))$, also solche Geraden,
 die den Graphen der Funktion außer in $P_0$ noch in mindestens einem weiteren Punkt schneidet.
Die Steigung der Sekante des Graphen von $f$ durch die Punkte $P_0=(x_0;f(x_0))$ und $P=(x_0+h;f(x_0+h))$ bestimmt man \"{u}ber
das Steigungsdreieck. Die Steigung entspricht also dem Quotienten der Differenzen auf der $y$-Achse und der $x$-Achse und berechnet sich zu
}
\lang{en}{
The derivative of a function $f$ at a pont $x_0$ can be thought of as the gradient of the tangent 
of the graph of $f$ at the point $(x_0;f(x_0))$. This is essentially a measure of the slope of $f$ 
at $x_0$. \\
The tangent is a specific line which \textit{touches} the line at $P_0=(x_0;f(x_0))$, and its 
gradient is found using a process called \emph{differentiation from first principles}, as follows. 
Consider the secants of the graph of $f$ through the point $P_0=(x_0;f(x_0))$, that is, those lines 
that intersect with the graph in at least one point other than $P_0$. The gradient of the secant of 
the graph of $f$ through the points $P_0=(x_0;f(x_0))$ and $P=(x_0+h;f(x_0+h))$ is found by dividing 
the change in the $y$-axis by the change in the $x$-axis,
}
\[\frac{f(x_0+h)-f(x_0)}{x_0+h-x_0}=\frac{f(x_0+h)-f(x_0)}{h}.\]\label{steigung_sekante}

\begin{center}
\image{T106_Tangent_A}
\end{center}

\lang{de}{
Macht man $h$ nun betragsmäßig immer kleiner, so erkennt man an dem folgenden Schaubild, dass sich 
die so entstandenen Sekanten immer weiter der Tangente an die Kurve von $f$ im Punkt $(x_0;f(x_0))$ n\"{a}hern, und auch der Quotient $\frac{f(x_0+h)-f(x_0)}{h}$
(also die Steigung der jeweiligen Sekante) n\"{a}hert sich der Steigung der Tangente an.\\
}
\lang{en}{
Suppose we now choose progressively smaller $h$, and note from the diagram that the secant 
approaches the tangent of the curve at the point $(x_0;f(x_0))$, and that the quotient 
$\frac{f(x_0+h)-f(x_0)}{h}$ (the gradient of the secant) approaches the gradient of the tangent.\\
}

\begin{center}
\image{T106_Tangent_B}
\end{center}

\lang{de}{
In der folgenden Visualisierung ist \nowrap{$f(x) = x^2 - |x|+1 $} und \nowrap{$P = (x_0; f(x_0))=(1; 1)$.} 
Verändern Sie die $x$-Koordinate des Punktes $Q$ und lassen Sie so $Q$ gegen $P$ laufen. Überzeugen Sie sich,
dass es eine Tangente an den Graphen von $f$ im Punkt $P$ gibt und dass die Ableitung, d.h. die Steigung der Tangente,
den Wert $1$ hat.
}
\lang{en}{
In the following visualisation, we show \nowrap{$f(x) = x^2 - |x|+1 $} and 
\nowrap{$P = (x_0; f(x_0))=(1; 1)$}. Change the $x$-coordinate of $Q$ and in order to make it 
approach $P$, and convince yourself that there exists a tangent to the graph of $f$ at the point 
$P$ and that the derivative of $f$ at $P$ (the gradient of the tangent) has value $1$.
}

\begin{genericGWTVisualization}[550][650]{mathlet1}
\lang{de}{\title{Tangente im Punkt $P$}}
\lang{en}{\title{The tangent at the point $P$}}
\begin{variables}

	\point{P}{rational}{1,1}
	\point[editable]{Q0}{rational}{1.5,0}
	\point{R}{rational}{0,0}
	\point{x0}{rational}{1,0}
	\function{f}{rational}{x^2 - abs(x) + 1}
	\number{Qx}{rational}{var(Q0)[x]}
	\number{h}{rational}{var(Qx)-1}
	\number{Qy}{rational}{(var(Qx))^2 - abs(var(Qx)) + 1}
	\point{Q}{rational}{var(Qx),var(Qy)}
	\point{Q1}{rational}{var(Qx)-dirac(var(h)),var(Qy)-dirac(var(h))}
	\line{s}{rational}{var(P),var(Q1)}
	\line{t}{rational}{var(R),var(P)}
	
\end{variables}
 \color{Q}{#0066CC}
 \color{Qx}{#0066CC}
 \color{Qy}{#0066CC}
 \color{P}{#CC6600}
 \color{x0}{#CC6600}
 \color{s}{#0066CC}
 \color{f}{#CC6600}
 \color{h}{#0066CC}
 \label{P}{@2d[$\textcolor{#CC6600}{P}$]}
 \label{x0}{@2d[$\textcolor{#CC6600}{x_0}$]}
 \label{Q}{@2d[$\textcolor{#0066CC}{Q}$]}
 \label{f}{@2d[$\textcolor{#CC6600}{f}$]}
	\begin{canvas}
	\plotSize{450}
	\plotLeft{-1.5}
	\plotRight{2.5}
	\plot[coordinateSystem]{f,Q,P,x0,s,t,Q0}
	\end{canvas}
\lang{de}{\text{\textcolor{#CC6600}{Orangener} Graph: \textcolor{#CC6600}{$f(x) = x^2 -|x| + 1$}, fester Punkt: \textcolor{#CC6600}{$P= (x_0;f(x_0))=(1; 1)$}.}}
\lang{en}{\text{\textcolor{#CC6600}{Orange} graph: \textcolor{#CC6600}{$f(x) = x^2 -|x| + 1$}, fixed point: \textcolor{#CC6600}{$P= (x_0,f(x_0))=(1, 1)$}.}}
\lang{de}{\text{Ziehen Sie mit der Maus den schwarzen Punkt entlang der $x$-Achse,}}
\lang{en}{\text{Use the mouse to move around the black point along the $x$-axis}}
\lang{de}{\text{um $\textcolor{#0066CC}{x=x_0+h = 1+}\var{h}$ zu verändern. Die \textcolor{#0066CC}{blaue} Sekante ist}}
\lang{en}{\text{in order to vary $\textcolor{#0066CC}{x=x_0+h=1+}\var{h}$. The \textcolor{#0066CC}{blue} secant is}}
\lang{de}{\text{die Gerade durch die Punkte \textcolor{#CC6600}{$P$} und 
$\textcolor{#0066CC}{Q = (x;f(x))= (}\var{Qx};\var{Qy}\textcolor{#0066CC}{)}$.}}
\lang{en}{\text{the line through the point \textcolor{#CC6600}{$P$} and 
$\textcolor{#0066CC}{Q = (x,f(x))= (}\var{Qx},\var{Qy}\textcolor{#0066CC}{)}$.}}
\lang{de}{\text{Wenn $\textcolor{#CC6600}{P}=\textcolor{#0066CC}{Q}$, d.h. $\textcolor{#0066CC}{h=0}$, ist die Sekante nicht definiert, aber}}
\lang{en}{\text{If $\textcolor{#CC6600}{P}=\textcolor{#0066CC}{Q}$, i.e. $\textcolor{#0066CC}{h=0}$, the secant is not defined, however}}
\lang{de}{\text{die Sekanten nähern sich für $\textcolor{#0066CC}{h}$ gegen Null der schwarzen Tangente.}}
\lang{en}{\text{the secant approximates the black tangent line as $\textcolor{#0066CC}{h}$ approaches $0$.}}


\end{genericGWTVisualization}
%There seems to be neither a black point nor a black tangent line on the graph - Niccolo

\lang{de}{
In der zweiten Visualisierung geht es um den Ausnahmefall, dass es keine Tangente und keine 
Ableitung gibt. Dazu betrachten wir die Funktion \nowrap{$f(x) = x^2 - |x|+1$} diesmal im Punkt 
\nowrap{$P = (x_0; f(x_0))=(0;1)$.} Überzeugen Sie sich, dass die Endlage der Sekante
durch $P$ und $Q$ unterschiedlich ist, je nachdem ob wir die $x-$Koordinate des Punktes $Q$ von der 
Seite der positiven Zahlen gegen $x_0=0$ gehen lassen oder von der Seite der negativen Zahlen. 
Nähert man sich von links der Stelle $x_0  = 0$ an, so hat die Tangente dort die Steigung $+1$. Eine 
Annäherung von rechts liefert die Steigung $-1$.
}
\lang{en}{
The following second visualization deals with an exception: a point where there is no tangent to the 
graph and hence no derivative. Here, again we'll consider the function \nowrap{$f(x) = x^2 - |x|+1$} 
but this time at the point \nowrap{$P = (x_0, f(x_0))=(0,1)$.} Convince yourself that the secant is 
different depending on which direction the point $Q$ approaches the point $x_0=0$ (whether from the 
negative or the positive side).
}


\begin{genericGWTVisualization}[550][650]{mathlet1}
\lang{de}{\title{Keine Tangente im Punkt $P$}}
\lang{en}{\title{No tangent at the point $P$}}
\begin{variables}
	\point{P}{rational}{0,1}
	\point[editable]{Q0}{rational}{1.5,0}
	\function{f}{rational}{x^2 - abs(x)+1}
	\number{Qx}{rational}{var(Q0)[x]}
	\number{h}{rational}{var(Qx)}
	\number{Qy}{rational}{(var(Qx))^2 - abs(var(Qx))+1}
	\point{Q}{rational}{var(Qx),var(Qy)}
	\point{P1}{rational}{1000*dirac(var(h)),1}
	\point{Q1}{rational}{var(Qx)-1000*dirac(var(h)),var(Qy)-10*dirac(var(h))}
	\line{s}{rational}{var(P1),var(Q1)}
\end{variables}

 \color{Q}{#0066CC}
 \color{Qx}{#0066CC}
 \color{Qy}{#0066CC}
 \color{P}{#CC6600}
 \color{s}{#0066CC}
 \color{f}{#CC6600}
 \color{h}{BLUE}
 \label{P}{@2d[$\textcolor{#CC6600}{P}$]}
 \label{Q}{@2d[$\textcolor{BLUE}{Q}$]}
 \label{f}{@2d[$\textcolor{#CC6600}{f}$]}
	\begin{canvas}
	\plotSize{450}
	\plotLeft{-2}
	\plotRight{2}
	\plot[coordinateSystem]{f,Q,P,s,Q0}
	\end{canvas}
\lang{de}{\text{\textcolor{#CC6600}{Orangener} Graph: \textcolor{#CC6600}{$f(x) = x^2 -|x|+1$}, fester Punkt: \textcolor{#CC6600}{$P= (x_0; f(x_0))= (0; 1)$}.}}
\lang{en}{\text{\textcolor{#CC6600}{Orange} graph: \textcolor{#CC6600D}{$f(x) = x^2 -|x|+1$}, fixed point: \textcolor{#CC6600}{$P= (x_0, f(x_0))= (0, 1)$}.}}
\lang{de}{\text{Ziehen Sie mit der Maus den schwarzen Punkt entlang der $x$-Achse,}}
\lang{en}{\text{Using the mouse, move the black point along the $x$-axis}}
\lang{de}{\text{um $\textcolor{#0066CC}{x=x_0+h = h =} \var{h}$ zu verändern. Die \textcolor{#0066CC}{blaue} Sekante ist}}
\lang{en}{\text{in order to vary $\textcolor{#0066CC}{x=x_0+h = h =} \var{h}$. The \textcolor{#0066CC}{blue} secant is}}
\lang{de}{\text{die Gerade durch die Punkte \textcolor{#CC6600}{$P$} und
$\textcolor{#0066CC}{Q = (x; f(x)) = (}\var{Qx}; \var{Qy}\textcolor{#0066CC}{)}$.}}
\lang{en}{\text{the line through the points \textcolor{#CC6600}{$P$} and
$\textcolor{#0066CC}{Q = (x, f(x)) = (}\var{Qx}, \var{Qy}\textcolor{#0066CC}{)}$.}}
\lang{de}{\text{Wenn $\textcolor{#CC6600}{P}=\textcolor{#0066CC}{Q}$, d.h. $\textcolor{#0066CC}{h=0}$, ist die Sekante nicht definiert.}}
\lang{en}{\text{If $\textcolor{#CC6600}{P}=\textcolor{#0066CC}{Q}$, i.e. $\textcolor{#0066CC}{h=0}$, the secant is not defined.}}


\end{genericGWTVisualization}

\lang{de}{
Wenn sich wie in der ersten Visualisierung die Sekante einer Tangente und damit die Steigung der 
Sekante der Steigung der Tangente ann\"{a}hert, soll $f$ in $x_0$ differenzierbar hei{\ss}en.
}
\lang{en}{
If, like in the first visualization, the secant line gets closer to the tangent, and hence the 
gradient of the secant approaches the gradient of the tangent, we call $f$ differentiable at $x_0$.
}


\begin{definition}%\textit{Definition:}\\
\lang{de}{
Eine Funktion $f$ ist \notion{\emph{in $x_0$ differenzierbar}} (oder auch ableitbar),
falls ein eindeutiges $c\in\R$ existiert, so dass die Steigung der Sekante
}
\lang{en}{
A function $f$ is called \textit{differentiable} at $x_0$ if there exists a unique $c\in\R$ such 
that the gradient
}
\[\frac{f(x_0+h)-f(x_0)}{h}\;\;\;\;\;\;(h\neq 0)\]
\lang{de}{
gegen $c$ geht f\"{u}r $h$ gegen Null.
Dieses $c$ bezeichnet man dann als die \textit{Ableitung von $f$ in $x_0$} und schreibt
}
\lang{en}{
of the secant approaches $c$ as $h$ goes to $0$.
This number $c$ is called the \textit{derivative of $f$ at $x_0$} and we write
}
\begin{equation}\label{diffbarkeit}\label{diffbarkeitz}
c=:f'(x_0).
\end{equation}
\lang{de}{
Obigen Quotienten nennt man \notion{\emph{Differenzenquotient}}.
\\\\
Wichtig: Beachten Sie, dass $c=+\infty$ und $c=-\infty$ ausgeschlossen sind!
}
\lang{en}{
Note: It is important to note that $c=+\infty$ and $c=-\infty$ are not permissible gradients!
}
\end{definition}

\begin{theorem}\label{thm:steigung_tangente}
\lang{de}{
Sei $f$ an der Stelle $x_0$ differenzierbar. Dann ist $c=f'(x_0)$ aus Gleichung $(1)$
die Steigung der Tangente an den Graphen von $f$ im Punkt $(x_0;f(x_0))$.\\
Die Tangente $T(x)$ an den Graphen von $f$ im Punkt $(x_0;f(x_0))$ hat also die Form 
}
\lang{en}{
Let $f$ be differentiable at the point $x_0$. Then $c=f'(x_0)$ from equation $(1.1)$
is the gradient of the tangent of the graph of $f$ at the point $(x_0,f(x_0))$.\\
The tangent $T(x)$ of the graph of $f$ at the point $(x_0,f(x_0))$ has the form
}
\[T(x)=f'(x_0)\cdot(x-x_0)+f(x_0).\]
\end{theorem}

\begin{quickcheck}
		\field{rational}
		\type{input.number}
		\begin{variables}
			\randint[Z]{a}{-2}{1}
			\randint[Z]{b}{-2}{1}
			\randint{c}{-2}{2}
			\function[normalize]{f}{x^3+a*x^2+b*x+c}
			\randint[Z]{p}{0}{8}
			\function[calculate]{x0}{p/4}
			\function[calculate]{y0}{x0^3+a*x0^2+b*x0+c}
			\function[calculate]{m}{3*x0^2+2*a*x0+b}
		\end{variables}
		
			\text{\lang{de}{
      Wir betrachten die Funktion $f(x)=\var{f}$ an der Stelle $x_0=\var{x0}$.\\
			Bestimmen Sie mit Hilfe der folgenden Visualisierung (bei der Sie die Koeffizienten von $f$ und die Stelle
			$x_0$ ändern können) den Wert der Ableitung von $f$ an der Stelle $\var{x0}$.\\
			$f'(\var{x0})=$\ansref.
      }
      \lang{en}{
      Consider the function $f(x)=\var{f}$ at the point $x_0=\var{x0}$.\\
      Determine, using the following visualisation (for which the coefficients of $f$ and the point 
      $x_0$ can be modified), the value of the derivative of $f$ at the point $\var{x0}$.\\
      $f'(\var{x0})=$\ansref.
      }}
		
		\begin{answer}
			\solution{m}
		\end{answer}
		\explanation{\lang{de}{
    Der Ableitungswert $f'(\var{x0})$ ist die Steigung der Tangente am Punkt $(\var{x0};\var{y0})$.
    }
    \lang{en}{
    The value of the derivative $f'(\var{x0})$ is the gradient of the tangent to $f$ at the point 
    $(\var{x0};\var{y0})$.
    }}
	\end{quickcheck}

	\begin{genericGWTVisualization}[550][1000]{mathlet1}
		\begin{variables}
			\number[editable]{a}{rational}{1}
			\number[editable]{b}{rational}{1}
			\number[editable]{c}{rational}{1}
			\number[editable]{x0}{rational}{0}
			\function{f}{rational}{x^3+var(a)*x^2+var(b)*x+var(c)}
			\number{y0}{rational}{var(x0)^3+var(a)*var(x0)^2+var(b)*var(x0)+var(c)}
			\number{m}{rational}{3*var(x0)^2+2*var(a)*var(x0)+var(b)}
			\number{q0}{rational}{var(y0)-var(m)*var(x0)}
			\point{P}{rational}{var(x0),var(y0)}
			\point{Q}{rational}{var(x0)+1,var(y0)}
			\point{R}{rational}{var(x0)+1,var(y0)+var(m)}
			\line{T}{rational}{var(P),var(R)}
			\segment{s1}{rational}{var(P),var(Q)}
			\segment{s2}{rational}{var(Q),var(R)}
		\end{variables}
% 		\color{P}{BLUE}
% 		\label{P}{$\textcolor{BLUE}{P}$}
		\color{s1}{#00CC00}
		\color{s2}{#00CC00}
		\color{T}{#0066CC}
%		\label{s1}{$\textcolor{BLUE}{1}$}
%		\label{s2}{$\textcolor{BLUE}{\var{m}}$}
		\begin{canvas}
			\plotSize{300}
			\plotLeft{-3}
			\plotRight{3}
			\plot[coordinateSystem]{f,T,P, s1,s2}
		\end{canvas}
		\lang{de}{\text{
    Die Funktion $f(x)=x^3+\var{a}\cdot x^2+\var{b}\cdot x+\var{c}$ hat an der Stelle $x_0=\var{x0}$
		die \textcolor{#0066CC}{Tangente $T$} mit der Gleichung 
    \textcolor{#0066CC}{$T(x)=\var{m}x+\var{q0}$}.
    }}
    \lang{en}{\text{
    The function $f(x)=x^3+\var{a}\cdot x^2+\var{b}\cdot x+\var{c}$ has the 
    \textcolor{#0066CC}{tangent $T$} with equation \textcolor{#0066CC}{$T(x)=\var{m}x+\var{q0}$} at 
    the point $x_0=\var{x0}$.
    }}
	    	\end{genericGWTVisualization}

\lang{de}{
Das nachfolgende Video führt ebenfalls in die Differentiation ein und beinhaltet einige Beispiele.   \\
\floatright{\href{https://api.stream24.net/vod/getVideo.php?id=10962-2-10766&mode=iframe&speed=true}{\image[75]{00_video_button_schwarz-blau}}}\\
\\\\
}
\lang{en}{}
\section{\lang{de}{Wichtige Ableitungsformeln}\lang{en}{Important derivatives}}

\begin{definition}%\textit{Definition:}\\
\label{def:differenzierbar}

\lang{de}{
Ist $f$ in jedem Punkt des Definitionsbereiches differenzierbar, so nennt man $f$ \notion{\emph{differenzierbar}}. 
Da man dann in jedem Punkt $x$ des Definitionsbereiches 
$f'(x)$ bilden kann, kann man $f'$ auch als Funktion auffassen, die einem Punkt $x_0$ die\notion{\emph{Ableitung}} von $f$ in $x_0$ 
zuordnet.
}
\lang{en}{
If a function $f$ is differentiable at every point of its domain we call $f$ \notion{\emph{differentiable}}. In this case, since the function $f$ has a derivative
$f'(x)$ at every point in its domain, we can think of $f'$ as a function that gives us the derivative of the function $f$ at any point $x_0$.
}
\end{definition}

\begin{remark}
\begin{enumerate}
\item \lang{de}{
      Die Ableitung wird zur Unterscheidung von h\"{o}heren Ableitungen (wie der 
      sp\"{a}ter eingef\"{u}hrten zweiten Ableitung) auch als \textit{erste 
      Ableitung} bezeichnet.
      }
      \lang{en}{
      In particular, the derivative we just defined is called the first derivative, in order to 
      distinguish it from higher derivatives (such as the second derivative, introduced later on).
      }
\item \lang{de}{
      Da $f'(x_0)$ auch als Grenzwert des Differenzenquotienten berechnet werden kann, schreibt man 
      statt $f'(x_0)$ manchmal auch $\frac{df}{dx}(x_0)$. Statt $f'$ schreibt man auch 
      $\frac{df}{dx}$, falls $f$ von $x$ abh\"{a}ngig ist.
      }
      \lang{en}{
      Because $f'(x_0)$ is also the limit of a change in the function value divided by a change in 
      $x$-coordinate, instead of writing $f'(x_0)$ we sometimes write $\frac{df}{dx}(x_0)$. Instead 
      of $f'$ we can also write $\frac{df}{dx}$ if $f$ is dependent on $x$.
      }
\item \lang{de}{
      Die Ableitung $f'(x_0)$ wird manchmal auch als 
      $\lim\limits_{\Delta x\rightarrow 0}\frac{\Delta y}{\Delta x}$ geschrieben. 
      Hier ist dann $\Delta x=h$ und $\Delta y=f(x_0+h)-f(x_0)$. Das $\Delta$ (Delta) steht 
      allgemein f\"{u}r eine Differenz, $\Delta x$ f\"{u}r die Differenz auf der $x$-Achse, 
      $\Delta y$ f\"{u}r die Differenz auf der $y$-Achse.
      %Alternativ schreibt man dann auch $\frac{ dy}{dx}$ f\"{u}r $f'$. <-- This comment was already included in item 2.
      }
      \lang{en}{The derivative $f'(x_0)$ is sometimes written 
      $\lim\limits_{\Delta x\rightarrow 0}\frac{\Delta y}{\Delta x}$. 
      In this case $\Delta x=h$ and $\Delta y=f(x_0+h)-f(x_0)$. The $\Delta$ (the uppercase greek 
      letter delta) usually denotes a difference, $\Delta x$ stands for a distance on the 
      $x$-axis (i.e. a difference in the $x$-coordinate), $\Delta y$ for a distance on the $y$-axis 
      (i.e. a difference in the $y$-coordinate). 
      %Alternatively, we can also write $\frac{ dy}{dx}$ for $f'$.
      }
\end{enumerate} 
\end{remark}


\lang{de}{
Die folgenden, in einer Tabelle zusammengefassten Ableitungen sollten Sie kennen.
}
\lang{en}{
The following is a short table summarizing some derivatives of common functions.
}
\begin{rule}\label{rule:elementare_abl}
\begin{align*}
\underline{\text{\lang{de}{Funktion}\lang{en}{Function}}\; f(x)}&\hspace{20pt}&  
\underline{\text{\lang{de}{Ableitung}\lang{en}{Derivative}}\; f'(x)}&&
\underline{\text{\lang{de}{Bedingung an }\lang{en}{Conditions on }}\, x}\\
c \;(c\in\R) &&0&&\\
x^n\;( n\in\N)&&nx^{n-1}&&\\
x^n\;(n\in\Z, n<0)&&nx^{n-1}&& x\neq 0\\
\sqrt{x}
&&\frac{1}{2\sqrt{x}}&& x>0\\
e^x&&e^x&&\\
\ln(x)&&\frac{1}{x}&&x>0\\
a^x \;(a>0)&& a^x \cdot ln (a)&&\\
\sin(x)&&\cos(x)&&\\
\cos(x)&&-\sin(x)&&\\
\end{align*}
\end{rule}

\section{\lang{de}{Ableitungsregeln}\lang{en}{Rules for differentiation}}\label{sec:abl_regeln}

\lang{de}{
Summen differenzierbarer Funktionen sind differenzierbar. Wie ihre Ableitung berechnet wird, 
beschreibt die Summenregel.
}
\lang{en}{
The sum of differentiable functions is also differentiable. This is called the 'sum rule'.
}


\begin{rule}[\lang{de}{Summenregel}\lang{en}{Sum rule}]\label{rule:additiv}
\lang{de}{F\"{u}r differenzierbare Funktionen $f$ und $g$ ist}
\lang{en}{Given two differentiable functions $f$ and $g$,}
\[(f+g)'(x)=f'(x)+g'(x).\] 
\lang{de}{Damit ist auch jede endliche Summe von differenzierbaren Funktionen $f_1, f_2, \ldots , f_n$ wieder differenzierbar:}
\lang{en}{Hence any finite sum of differentiable functions $f_1, f_2, \ldots ,f_n$ is also 
differentiable:}
\[(f_1+f_2+\ldots f_n)'=f_1'+f_2'+\ldots + f_n'.\]
\end{rule}



\begin{example}%\textit{Beispiel:}\\
\lang{de}{
Wir berechnen die Ableitung der Funktion $f(x)=x^3+x+1$. Diese Funktion ist die Summe der 
Funktionen $x^3$, $x$ und der konstanten Funktion $1$. Nach der Summenregel ist
}
\lang{en}{
We can calculate the derivative of the function $f(x)=x^3+x+1$. This function is the sum of the 
functions $x^3$, $x$, and the constant function $1$. According to the sum rule,
}
\[f'(x)=3x^2+1.\]
\end{example}


\lang{de}{
Multipliziert man eine differenzierbare Funktion $f$ mit einer reellen Zahl $c$, so ist die 
Produktfunktion differenzierbar. Die Ableitung berechnet man zu
}
\lang{en}{If we multiply a differentiable function $f$ by a real number $c$, then the product is 
differentiable. The derivative is then
}
\[(c\cdot f)'(x)=\lim_{h\rightarrow 0}\frac{c\cdot f(x+h)-c \cdot f(x)}{h}=c\cdot \lim_{h\rightarrow 0}\frac{f(x+h)-f(x)}{h}=c\cdot f'(x).\]
\lang{de}{Die Faktorregel fasst diesen Sachverhalt zusammen.}
\lang{en}{Summarizing you get the so called constant rule.}


\begin{rule}[\lang{de}{Faktorregel}\lang{en}{The constant rule}]\label{rule:const_factor}
\lang{de}{F\"{u}r eine differenzierbare Funktion $f$ und $c\in\R$ ist }
\lang{en}{For any $c\in\R$ and differentiable $f$, }
\[(c\cdot f)'(x)=c\cdot f'(x).\] 

\end{rule}

% \lang{de}{Zur Ergänzung ein Rap über die Faktorregel.}
% \begin{center}
%   \lang{de}{\iframe[400][225][S]{https://www.stream24.net/vod/getVideo.php?id=10962-1-5528&mode=iframe}}
% \end{center}

\lang{de}{
Wie Produkte differenzierbarer Funktionen abgeleitet werden, beschreibt die Produktregel.
}
\lang{en}{
The product rule describes how we take the derivative of the product of differentiable functions.
}

\begin{rule}[\lang{de}{Produktregel} \lang{en}{The product rule}]\label{rule:produktregel}
 \lang{de}{Sind $f$ und $g$ differenzierbare Funktionen, so ist \ }
 \lang{en}{If $f$ and $g$ are two differentiable functions, then  }
 \[(f\cdot g)'(x)=f'(x)\cdot g(x)+f(x)\cdot g'(x).\]

 \end{rule}
 
\begin{proof*}[\lang{de}{Beweis der Produktregel}\lang{en}{Proof of the product rule}]
\begin{incremental}[\initialsteps{0}]
\step
\begin{eqnarray*}
(f\cdot g)'(x)&=&\lim_{h\rightarrow 0}\frac{(f\cdot g)(x+h)-(f\cdot g)(x)}{h}\\
&=&\lim_{h\rightarrow 0}\frac{f(x+h)\cdot g(x+h)-f(x)\cdot g(x)}{h}\\
&=&\lim_{h\rightarrow 0}\frac{f(x+h)\cdot g(x+h)-f(x)\cdot g(x+h)+f(x)\cdot g(x+h)-f(x)\cdot g(x)}{h}\\
&=&\lim_{h\rightarrow 0}\frac{(f(x+h)-f(x))\cdot g(x+h)+f(x) \cdot (g(x+h) -g(x))}{h}\\
&=&\lim_{h\rightarrow 0} \bigg(\frac{f(x+h)- f(x)}{h}\cdot g(x+h)+f(x)\cdot \frac{g(x+h) -g(x)}{h}\bigg)\\
&=&
f'(x)\cdot g(x)+f(x)\cdot g'(x).
\end{eqnarray*}
\end{incremental}
\end{proof*}
 
 
 
 
 
 
 
 
 
 
 
%  \lang{de}{Zur Ergänzung ein Rap über die Produktregel.}
% \begin{center}
%   \lang{de}{\iframe[400][225][S]{https://www.stream24.net/vod/getVideo.php?id=10962-1-5524&mode=iframe}}
% \end{center}
%\lang{de}{Zur Herleitung der Produktregel klicken Sie bitte} 
% This wasn't consistent with the way this was used in the past.

 \begin{example}%\textit{Beispiel:}\\
\lang{de}{
Wir berechnen als Beispiel die Ableitung der Funktion $F(x)=\frac{1}{x^2}\cdot \sin x$. Diese ist 
das Produkt $F(x)=f(x)\cdot g(x)$ der differenzierbaren Funktionen $f(x)=\frac{1}{x^2}$ und 
$g(x)=\sin x$. Die Ableitung von $f(x)$ ist $-\frac{2}{x^3}$, die Ableitung von $g(x)$ ist $\cos x$. 
Dann ist
}
\lang{en}{
As an example we can calculate the derivative of the function $F(x)=\frac{1}{x^2}\cdot \sin x$, which is the product $F(x)=f(x)\cdot g(x)$ 
of the differentiable functions 
$f(x)=\frac{1}{x^2}$ and $g(x)=\sin x$. The derivative of $f(x)$ is $-\frac{2}{x^3}$ and the derivative of $g(x)$ is $\cos x$. Hence 
}
\[F'(x)=f'(x)\cdot g(x)+f(x)\cdot g'(x)=-\frac{2}{x^3}\cdot\sin x+\frac{1}{x^2}\cdot \cos x.\]

\end{example}
\begin{remark}%\textit{Bemerkung:}\\
\lang{de}{Im vorigen Abschnitt ist dargelegt, dass die Ableitung jeder konstanten Funktion $0$ ist. Nach der Produktregel ist also, 
falls $g(x)=c$ konstant ist:
\[(c\cdot f)'(x)=(g\cdot f)'(x)=g'(x)\cdot f(x)+g(x)\cdot f'(x)=0 \cdot f(x)+c \cdot f'(x)=c \cdot f'(x).\] Die Faktorregel ist also ein 
Spezialfall der Produktregel.}
\lang{en}{In the earlier section we stated that the derivative of every constant function is $0$. According to the product rule, 
if we treat $g(x)=c$ as a constant function:
\[(c\cdot f)'(x)=(g\cdot f)'(x)=g'(x)\cdot f(x)+g(x)\cdot f'(x)=0 \cdot f(x)+c \cdot f'(x)=c \cdot f'(x).\] From this example we can
see that the constant rule is simply a special case of the product rule.}
\end{remark}



\begin{quickcheckcontainer}
\randomquickcheckpool{1}{1}
\begin{quickcheck}
		\field{rational}
		\type{input.function}
		\begin{variables}
			\randint[Z]{a}{-2}{2}
			\randint[Z]{b}{1}{4}
			\randint{l}{1}{4}
			\function[calculate]{l1}{l-1}
			
			\randint{k}{1}{4}	% Zufallsvariable zum Vertauschen:
			\function[calculate]{d1}{-(k-2)*(k-3)*(k-4)/6}  % "Dirac"-funktionen
			\function[calculate]{d2}{(k-1)*(k-3)*(k-4)/2}
			\function[calculate]{d3}{-(k-1)*(k-2)*(k-4)/2}
			\function[calculate]{d4}{(k-1)*(k-2)*(k-3)/6}
			
			\function[normalize]{f1}{sin(x)}
			\function[normalize]{f2}{cos(x)}
			\function[normalize]{f3}{exp(x)}
			\function[normalize]{f4}{ln(x)}

			\function[expand,normalize]{g}{d1*f1+d2*f2+d3*f3+d4*f4}						
			\function[expand,normalize]{dg}{d1*f2+d2*(-f1)+d3*f3+d4*(1/x)}

			\function[normalize]{p}{a*x^l+b}
			\function[normalize]{dp}{l*a*x^(l-1)}
		    \function[normalize]{f}{p*g}
			\function[normalize]{df}{dp*g+p*dg}
		\end{variables}

			\text{\lang{de}{
      Bestimmen Sie mit Hilfe der Formeln und der Ableitungsregeln die Ableitung der
			Funktion $f(x)=\var{f}$.\\ Die Ableitungsfunktion ist: $f'(x)=$ \ansref.
      }
      \lang{en}{
      Using the table of common derivatives and the rules of differentiation, determine the 
      derivative of the function $f(x)=\var{f}$.\\ The derivative is: $f'(x)=$ \ansref.
      }}

		\begin{answer}
			\solution{df}
			\checkAsFunction{x}{-2}{2}{20}
		\end{answer}
		\explanation{\lang{de}{
    Mit der Produktregel ist zunächst $f'(x)=(\var{p})'\cdot \var{g}+(\var{p})\cdot (\var{g})'$.\\
		Für $(\var{p})'$ rechnet man:\\
		$(\var{p})'=\var{a}\cdot (x^{\var{l}})'+ (\var{b})'=\var{a}\cdot \var{l}\cdot x^{\var{l1}}+0=\var{dp}$\\
		und $(\var{g})'=\var{dg}$ mit obiger Formel. Also insgesamt:\\
		$f'(x)=\var{df}$.
    }
    \lang{en}{
    Firstly we apply the product rule, giving us $f'(x)=(\var{p})'\cdot 
      \var{g}+(\var{p})\cdot (\var{g})'$.\\
    For $(\var{p})'$ we calculate:\\
    $(\var{p})'=\var{a}\cdot (x^{\var{l}})'+ (\var{b})'=\var{a}\cdot 
      \var{l}\cdot x^{\var{l1}}+0=\var{dp}$\\
    and $(\var{g})'=\var{dg}$ by the table of derivatives. Combining these,\\
    $f'(x)=\var{df}$.
    }}
		
	\end{quickcheck}
\end{quickcheckcontainer}
% 
% 
% 	\begin{genericGWTVisualization}[550][1000]{mathlet1}
% 		\begin{variables}
% 			\randint{randomA}{1}{2}
% 
% 			\point[editable]{P}{rational}{var(randomA),var(randomA)}
% 		\end{variables}
% 		\color{P}{BLUE}
% 		\label{P}{$\textcolor{BLUE}{P}$}
% 
% 		\begin{canvas}
% 			\plotSize{300}
% 			\plotLeft{-3}
% 			\plotRight{3}
% 			\plot[coordinateSystem]{P}
% 		\end{canvas}
% 		\text{Der Punkt hat die Koordinaten $(\var{P}[x],\var{P}[y])$.}
% 	    	\end{genericGWTVisualization}

\end{visualizationwrapper}

\begin{rule}[\lang{de}{Quotientenregel} \lang{en}{The Quotient Rule}]\label{rule:quotient_regel}
 \lang{de}{
 Sind $f : I \to \R$ und $g : I \to \R$ differenzierbare Funktionen mit $g(x) \neq 0$ für alle 
 $x \in I$. Dann ist $\frac{f}{g}$ differenzierbar und 
 }
 \lang{en}{
 If $f : I \to \R$ and $g : I \to \R$ are two differentiable functions and $g(x) \neq 0$ for all 
 $x \in I$, then
 }
 \[\left( \frac{f}{g} \right)^'(x)= \frac{f'(x) \cdot g(x) - f(x) \cdot g'(x)}{(g(x))^2}. \]
 \end{rule}
 
 
 \begin{example}
 \begin{enumerate}
\item \lang{de}{
Wir berechnen die Ableitung der Funktion $f(x) = e^{-x} = \frac{1}{e^x}$. Mit der Quotientenregel 
erhalten wir
}
\lang{en}{
We calculate the derivative of the function $f(x) = e^{-x} = \frac{1}{e^x}$. By the quotient rule,
}
\[\left(\frac{1}{e^x}\right)^' = \frac{0 \cdot e^x - 1 \cdot e^x}{(e^x)^2} = \frac{-e^x}{(e^x)^2} = \frac{-1}{e^x} = -e^{-x}.\]

\item \lang{de}{
Es soll die Ableitung von $f(x) = \frac{2x^2 + 3}{2x +1}$ für $x > 0$ berechnet werden. Mit der Quotientenregel erhalten wir
}
\lang{en}{
We calculate the derivative of the function $f(x) = \frac{2x^2 + 3}{2x +1}$ for $x > 0$. By the 
quotient rule,
}
\[ f'(x) = \frac{(4x) \cdot (2x +1) - (2) \cdot (2x^2 +3)}{(2x +1)^2} = \frac{8x^2 + 4x - 4x^2 -6}{(2x+1)^2} = \frac{4x^2 + 4x -6}{(2x+1)^2}. \]
\end{enumerate}
\end{example}

\end{content}