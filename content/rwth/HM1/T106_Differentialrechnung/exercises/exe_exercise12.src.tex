\documentclass{mumie.element.exercise}
%$Id$
\begin{metainfo}
  \name{
    \lang{de}{Ü12: Maximierungsaufgabe}
    \lang{en}{Exercise 12}
  }
  \begin{description} 
 This work is licensed under the Creative Commons License Attribution 4.0 International (CC-BY 4.0)   
 https://creativecommons.org/licenses/by/4.0/legalcode 

    \lang{de}{Hier die Beschreibung}
    \lang{en}{}
  \end{description}
  \begin{components}
  \component{generic_image}{content/rwth/HM1/images/g_tkz_T106_Exercise12.meta.xml}{T106_Exercise12}
  \end{components}
  \begin{links}
  \end{links}
  \creategeneric
\end{metainfo}
\begin{content}

\title{
\lang{de}{Ü12: Maximierungsaufgabe}
\lang{en}{Exercise 12}
}
  \begin{block}[annotation]
  Im Ticket-System: \href{http://team.mumie.net/issues/9462}{Ticket 9462}
\end{block}


\lang{de}{Maximieren Sie den Umfang $U$ eines Rechtecks, 
das auf der $x$-Achse mit einer Ecke im Ursprung liegt, und dessen gegen\"{u}berliegende Ecke auf der Parabel $-0,25x^2+4$ liegt. Die Grundseite soll 
auf der positiven $x$-Achse liegen. }
\lang{en}{Maximize the perimeter, $P$, of a rectangle that lies on the $x$-axis with a corner at the origin and the opposite corner on the parabola $-0.25x^2+4$. 
The base should be on the positive $x$-axis.}
\begin{center}
\image{T106_Exercise12}
\end{center}
\\


\begin{tabs*}[\initialtab{0}\class{exercise}]
  \tab{
  \lang{de}{Antwort}
  \lang{en}{Answer}
  }
  \lang{de}{Der maximale Umfang $U$ ist $U=10$.}
  \lang{en}{The maximum perimeter is $P=10$. }
 
  \tab{
  \lang{de}{L\"{o}sung}
  \lang{en}{Solution}
  }
  
  \begin{incremental}[\initialsteps{1}]
    \step \lang{de}{Zuerst m\"{u}ssen wir die Funktion finden, die wir maximieren wollen. Es soll der maximale Wert von $2x+2y$ bestimmt werden, wenn $x$ die Breite des Rechtecks (abgetragen auf der $x$-Achse) 
und $y$ die H\"{o}he des Rechtecks ist (abgetragen auf der $y$-Achse). Da dann $y=-0,25x^2+4$ ist, k\"{o}nnen wir $y$ in der Formel f\"{u}r den Umfang 
ersetzen: wir m\"{u}ssen also den maximalen Wert von $2x+2\cdot (-0,25x^2+4)=-0,5x^2+2x+8$  bestimmen. Dabei soll nat\"{u}rlich weiterhin $x\geq 0$ 
gelten, und da die Parabel die $x$-Achse im Punkt $(4;0)$ schneidet, ist auch $x\leq 4$. Setzen wir also $f(x)=-0,5x^2+2x+8$ 
f\"{u}r $x\in [0;4]$, so wollen wir den maximalen Wert von $f$ bestimmen. }
\lang{en}{First, we need to find the function that we want to maximize. It should be determined by the maximum value of $2x+2y$, where $x$ is the width of the rectangle (plotted along the $x$-axis) and $y$ 
is the height (plotted along the $y$-axis).
Since $y=-0.25x^2+4$, we can replace $y$ in the equation for the perimeter. Therefore, we can determine the maximum value of $2x+2\cdot (-0.25x^2+4)=-0.5x^2+2x+8$. Of course, we should continue to apply $x\geq 0$, 
and since the parabola cuts the $x$-axis at the point $(4,0)$, we can also apply $x\leq 4$. So we set $f(x)=-0.5x^2+2x+8$ for $x\in [0,4]$ and determine the maximum value of $f$. }

\step \lang{de}{Wir suchen zuerst in dem Intervall $(0;4)$ 
nach einem Maximum.}
\lang{en}{First, we look for a maximum in the interval $(0,4)$.}
 
\step \lang{de}{An einer solchen Stelle muss $f'(x)=0$ gelten. Wir berechnen also $f'(x)$:}
\lang{en}{At such points, we need $f'(x)=0$. So we calculate $f'(x)$:}
\[f'(x)=-x+2.\]
\step \lang{de}{Dann ist also $f'(2)=0$, und es gibt keine weiteren Nullstellen von $f'$. Damit besitzt $f$ in $(0;4)$ nur die kritische Stelle $x=2$.}
\lang{en}{Then $f'(2)=0$, and there are no more roots of $f'$. Therefore, in the interval $(0,4)$, $f$ has only the critical point $x=2$.}

 \step \lang{de}{Da $f''(x)=-1$ konstant und $f''(2)=-1<0$ ist, liegt in $x=2$ ein 
lokales Maximum vor. }
\lang{en}{Since $f''(x)=-1$ is constant and $f''(2)=-1<0$, there is a local maximum at $x=2$.}

 \step\lang{de}{Da $f(0)=8$ und $f(4)=8$, aber $f(2)=10$, wird in $x=2$ der maximale Wert von $f$ auf $[0;4]$ angenommen. Also ist der 
 maximale Umfang $U=10$. }
 \lang{en}{Since $f(0)=8$ and $f(4)=8$, but $f(2)=10$, the maximum value of $f$ in $[0,4]$ is at $x=2$. So the maximum perimeter is $P=10$.}

    

   
  
  \end{incremental}


\end{tabs*}

\end{content}