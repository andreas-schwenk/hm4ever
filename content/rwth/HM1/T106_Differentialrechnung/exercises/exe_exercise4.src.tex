\documentclass{mumie.element.exercise}
%$Id$
\begin{metainfo}
  \name{
    \lang{de}{Ü04: Ableitungsregeln}
    \lang{en}{Exercise 4}
  }
  \begin{description} 
 This work is licensed under the Creative Commons License Attribution 4.0 International (CC-BY 4.0)   
 https://creativecommons.org/licenses/by/4.0/legalcode 

    \lang{de}{Hier die Beschreibung}
    \lang{en}{}
  \end{description}
  \begin{components}
  \end{components}
  \begin{links}
  \end{links}
  \creategeneric
\end{metainfo}
\begin{content}

\title{
\lang{de}{Ü04: Ableitungsregeln}
}
 \begin{block}[annotation]
  Im Ticket-System: \href{http://team.mumie.net/issues/9458}{Ticket 9458}
\end{block}
 
 
 
\lang{de}{

Berechnen Sie die Ableitungen der folgenden Funktionen mit der Produkt- bzw. Quotientenregel.

\begin{table}[\class{items}]
\nowrap{a) $f(x)=x \cdot e^x$} \\
\nowrap{b) $f(t) = t^2 \cdot \sin(t)$} \\
\nowrap{c) $f(a) = (a^2 -1)(1+a^2)$}\\
\nowrap{d) $f(z) = z\cdot\sqrt{z} $}\\
\nowrap{e) $ f(x) = x \cdot \ln(x) $}\\
\nowrap{f) $ f(\omega) = \cos(\omega) \cdot \tan(\omega) $}\\
\nowrap{g) $ f(x) = \frac{x^2 +2x}{3x +1} $}\\
\nowrap{h) $f(x) = \frac{1 }{x^2 + x +1} $}\\
\nowrap{i) $f(s) = \frac{s^2 + 4s +5}{s^3} $}\\
\nowrap{j) $f(x) = \frac{\sqrt[3]{x} }{x} $}\\
\nowrap{k) $f(s) = \frac{\ln(s)}{s} $  }\\
\nowrap{l) $f(w) = \frac{\tan(w)}{\sin(w)} $ }\\
\end{table}



\begin{tabs*}[\initialtab{0}\class{exercise}]
  
   \tab{\lang{de}{Lösungsvideo a) - l)}}
    \youtubevideo[500][300]{vwilDEhSoX0}\\
  
 \end{tabs*}
 }
  
\end{content}


