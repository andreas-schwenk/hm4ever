\documentclass{mumie.element.exercise}
%$Id$
\begin{metainfo}
  \name{
    \lang{de}{Ü02: Steigung}
    \lang{en}{Exercise 2}
  }
  \begin{description} 
 This work is licensed under the Creative Commons License Attribution 4.0 International (CC-BY 4.0)   
 https://creativecommons.org/licenses/by/4.0/legalcode 

    \lang{de}{Hier die Beschreibung}
    \lang{en}{}
  \end{description}
  \begin{components}
\component{generic_image}{content/rwth/HM1/images/g_tkz_T106_Exercise02.meta.xml}{T106_Exercise02}
\end{components}
  \begin{links}
  \end{links}
  \creategeneric
\end{metainfo}
\begin{content}
\title{
  \lang{de}{Ü02: Steigung}
  \lang{en}{Exercise 2}
}





\begin{block}[annotation]
  TODO: Konzeption 2
\end{block}

\begin{block}[annotation]
  Im Ticket-System: \href{http://team.mumie.net/issues/9102}{Ticket 9102}
\end{block}

\lang{de}{
\begin{table}[\class{items}]
a) Gegeben sei die Funktion $f(x)=x^{2}$. Berechnen Sie die Steigung der Sekante
\[\Delta(h):=\frac{f(x+h)-f(x)}{h}\,.\]
Was erhalten Sie für $\Delta(h)$, wenn $h$ gegen 0 geht? Vergleichen Sie ihr Ergebnis mit den Ableitungsformeln.\\
b) Wir betrachten nun eine Funktion $f$, welche die Eigenschaft $f(-x)=f(x)$ für alle $x\in\R$ hat. Man sagt in diesem Fall, dass $f$ symmetrisch zur $y$-Achse ist.
Begründen Sie anschaulich anhand des Steigungsdreiecks, dass dann $f'(-x)=-f'(x)$ gilt für alle $x\in\R$.
\end{table}
}

\begin{center}
\image{T106_Exercise02}
\end{center}

\begin{tabs*}[\initialtab{0}\class{exercise}]
  \tab{
  \lang{de}{Lösung a)}
  \lang{en}{Solution}
  }
  
  \begin{incremental}[\initialsteps{1}]
    \step \lang{de}{Es sei $x\in\R$ und $h\in\R\backslash\{0\}$. Dann gilt nach der ersten binomischen Formel
\[\Delta(h)=\frac{(x+h)^{2}-x^{2}}{h}=\frac{x^{2}+2hx+h^{2}-x^{2}}{h}=\frac{2hx+h^{2}}{h}=2x+h\,.\]}
	\step \lang{de}{Für $h\to 0$ geht also $\Delta(h)$ gegen den Ausdruck $2x$. Dies entspricht genau der Formel für die Ableitung der Potenzfunktion  $x\mapsto x^{n}$ im Falle $n=2$.}
  \end{incremental}
  
  \tab{
  \lang{de}{Lösung b)}
  \lang{en}{Solution}
  }
  
  \begin{incremental}[\initialsteps{1}]
    \step \lang{de}{Es sei $x_{0}\in\R$ und $h\neq 0$. Das Steigungsdreieck $\Delta_{1}$ durch die Punkte 
\[(-x_{0};f(-x_{0})),(-x_{0}+h;f(-x_{0})),(-x_{0}+h;f(-x_{0}+h))\] 
soll mit dem Steigungsdreieck $\Delta_{2}$ durch die Punkte  
\[(x_{0};f(x_{0})),(x_{0}+h;f(x_{0})),(x_{0}+h;f(x_{0}+h))\] 
verglichen werden. Wegen der Achsensymmetrie von $f$ lassen sich die Eckpunkte von $\Delta_{1}$ schreiben als 
\[(-x_{0};f(x_{0})),(-x_{0}+h;f(x_{0})),(-x_{0}+h;f(x_{0}-h))\,.\]}
	\step \lang{de}{Damit ergibt sich für $f$ an der Stelle $-x_{0}$ die Steigung
\[\frac{f(x_{0}-h)-f(x_{0})}{h}=-\frac{f(x_{0}+h')-f(x_{0})}{h'}\,,\]
wenn man $h'=-h$ substituiert.}
	\step \lang{de}{Man beachte, dass mit $h$ auch $h'$ gegen 0 strebt. Daran sieht man, dass $f'(-x_{0})=-f'(x_{0})$ gilt.}
   \end{incremental}

\end{tabs*}


\end{content}