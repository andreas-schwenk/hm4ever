\documentclass{mumie.element.exercise}
%$Id$
\begin{metainfo}
  \name{
    \lang{de}{Ü13: Minimum}
    \lang{en}{Exercise 13}
  }
  \begin{description} 
 This work is licensed under the Creative Commons License Attribution 4.0 International (CC-BY 4.0)   
 https://creativecommons.org/licenses/by/4.0/legalcode 

    \lang{de}{Hier die Beschreibung}
    \lang{en}{}
  \end{description}
  \begin{components}
  \component{generic_image}{content/rwth/HM1/images/g_tkz_T106_Exercise13.meta.xml}{T106_Exercise13}
  \end{components}
  \begin{links}
  \end{links}
  \creategeneric
\end{metainfo}
\begin{content}

\title{
\lang{de}{Ü13: Minimum}
\lang{en}{Exercise 13}
}
 
   \begin{block}[annotation]
  Im Ticket-System: \href{http://team.mumie.net/issues/9463}{Ticket 9463}
\end{block}


 

\lang{de}{Finden Sie den minimalen Funktionswert der Funktion $f(x)=x^3-\frac{3}{2}x^2+1$ auf dem Intervall $[-1;2]$. Wo wird dieser angenommen?}
\lang{en}{Find the minimum value of the function $f(x)=x^3-\frac{3}{2}x^2+1$ in the interval $[-1,2]$. Where does this occur?}

\begin{tabs*}[\initialtab{0}\class{exercise}]
  \tab{
  \lang{de}{Antwort}
  \lang{en}{Answer}
  }
  \lang{de}{Der kleinste Funktionswert ist $-\frac{3}{2}$ und wird in $-1$ angenommen.}
  \lang{en}{The smallest value of the function is $-\frac{3}{2}$ and it occurs when $x=-1$.}
 
  \tab{
  \lang{de}{L\"{o}sung}
  \lang{en}{Solution}
  }
  
  \begin{incremental}[\initialsteps{1}]
    \step \lang{de}{Um den kleinsten Funktionswert zu bestimmen, m\"{u}ssen wir nach globalen Minima suchen. 
    Wir suchen zuerst die station\"{a}ren Stellen im Intervall $(-1;2)$. Dazu berechnen wir $f'(x)$:}
    \lang{en}{To determine the smallest value of the function, we need to find the global minimum.
    First, we look for the critical points in the interval $(-1,2)$. Therefore, we calculate $f'(x)$:}
\[f'(x)=3x^2-3x=3x(x-1).\]
\step \lang{de}{Wir setzen $f'(x)=0$, also}
\lang{en}{We set $f'(x)=0$, and so}
\[3x(x-1)=0.\] 
\step \lang{de}{$f'(x)$ hat die beiden Nullstellen $0$ und $1$, diese liegen auch beide im Intervall $(-1;2)$. 
Nur in $x=0$ und $x=1$ k\"{o}nnen also im Intervall $(-1;2)$ Minima vorliegen. Nun berechnen wir die zweite Ableitung:}
\lang{en}{$f'(x)$ has two roots, $x=0$ and $x=1$, which both lie in the interval $(-1,2)$.
So, minima can only exist at $x=0$ and $x=1$ in the interval $(-1,2)$. Now, we calculate the second derivative: }
\step \[f''(x)=6x-3.\]
\step\lang{de}{ Wir berechnen $f''(0)=-3<0$. Damit liegt in $0$ ein lokales Maximum vor und kein Minimum.}
\lang{en}{We calculate $f''(0)=-3<0$. Therefore, there is a local maximum at $x=0$ (not a local minimum).}
\step\lang{de}{ Weiter ist $f''(1)=3>0$. Damit hat $f$ in $1$ ein lokales Minimum. Der Funktionswert ist $f(1)=\frac{1}{2}$.}
\lang{en}{Furthermore, $f''(1)=3>0$. Therefore, $f$ has a local minimum at $x=1$. The value of the function at that point is $f(1)=\frac{1}{2}$.}
\step\lang{de}{ Um zu \"{u}berpr\"{u}fen, ob dies auch tats\"{a}chlich der kleinste Funktionswert auf dem Intervall $[-1;2]$ ist, 
m\"{u}ssen wir noch $f(1)$ mit $f(-1)$ und $f(2)$ vergleichen.}
\lang{en}{To verify whether this is in fact the smallest value of the function in $[-1,2]$, we still need to compare it with $f(-1)$ and $f(2)$. }
\step \lang{de}{$f(-1)=-\frac{3}{2}<\frac{1}{2}$, $f(2)=3>\frac{1}{2}$. 
Damit wird der kleinste Funktionswert von $f$ auf $[-1;2]$ in $x=-1$ angenommen, und der kleinste Funktionswert ist $-\frac{3}{2}$. }
\lang{en}{$f(-1)=-\frac{3}{2}<\frac{1}{2}$, $f(2)=3>\frac{1}{2}$.
Therefore, in the interval $[-1,2]$, the smallest value of the function $f$ occurs at $x=-1$, and the smallest value is $-\frac{3}{2}$.}
\step
\begin{center}
\image{T106_Exercise13}
\end{center}

\lang{de}{Wie man im Bild erkennen kann, wird das globale Minimum in $x=-1$ angenommen. 
In $x=0$ liegt ein lokales Maximum vor.}
\lang{en}{As you can see from the picture, the global minimum occurs at $x=-1$ and a local maximum occurs at $x=0$.}
    

   
  
  \end{incremental}


\end{tabs*}
\end{content}