\documentclass{mumie.element.exercise}
%$Id$
\begin{metainfo}
  \name{
    \lang{de}{Ü06: Monotonie}
    \lang{en}{Exercise 6}
  }
  \begin{description} 
 This work is licensed under the Creative Commons License Attribution 4.0 International (CC-BY 4.0)   
 https://creativecommons.org/licenses/by/4.0/legalcode 

    \lang{de}{Hier die Beschreibung}
    \lang{en}{}
  \end{description}
  \begin{components}
  \end{components}
  \begin{links}
  \end{links}
  \creategeneric
\end{metainfo}
\begin{content}
\title{
  \lang{de}{Ü06: Monotonie}
  \lang{en}{Exercise 6}
}





\begin{block}[annotation]
  TODO: Konzeption 2
\end{block}

\begin{block}[annotation]
  Im Ticket-System: \href{http://team.mumie.net/issues/9107}{Ticket 9107}
\end{block}
\lang{de}{Betrachten Sie die Funktion $f(x)=x^4+\frac{4}{3}x^3-4x^2+2$. Bestimmen Sie die Intervalle der Monotonie von $f$.}
\lang{en}{Consider the function $f(x)=x^4+\frac{4}{3}x^3-4x^2+2$. Determine the intervals of monotonicity of $f$.}


\begin{tabs*}[\initialtab{0}\class{exercise}]
  \tab{
  \lang{de}{Antwort}
  \lang{en}{Answer}
  }
\lang{de}{$f$ ist auf den Intervallen $(-\infty;-2)$ und $(0;1)$ streng monoton fallend, und auf $(-2;0)$ sowie $(1;\infty)$ streng monoton steigend.}
\lang{en}{$f$ is strictly monotonically decreasing in the intervals $(-\infty,-2)$ and $(0,1)$ and strictly monotonically increasing in $(-2,0)$ and $(1,\infty)$.}

  \tab{
  \lang{de}{L\"{o}sung}
  \lang{en}{Solution}
  }
  \begin{incremental}[\initialsteps{1}]
  \step \lang{de}{Wir berechnen zuerst $f'(x)$.}
  \lang{en}{First we calculate $f'(x)$.}
  \[f'(x)=4x^3+4x^2-8x.\]
  
  
  \lang{de}{Wir versuchen nun, die Nullstellen von $f'$ zu finden.}
  \lang{en}{Now we attempt to find the roots of $f'$.}
  
  
   \step \lang{de}{Offensichtlich ist $x=0$ eine Nullstelle. Wir klammern $x$ aus:}
   \lang{en}{Clearly, $x=0$ is a root. We factorise out $x$:}
   \[f'(x)=4x(x^2+x-2).\]
  \step \lang{de}{Die beiden anderen Nullstellen $x=-2$ und $x=1$ bestimmt man mit der $pq$-Formel. Damit ist dann}
  \lang{en}{The other two roots $-2$ and $1$ can be determined using the quadratic formula. With this we get}
  \[f'(x)=4x(x-1)(x+2).\]
  \step \lang{de}{F\"{u}r $x\in (-\infty;-2)$ oder $x\in (0;1)$ ist $f'(x)<0$, und damit ist $f$ dort streng monoton fallend.  }
  \lang{en}{For $x\in (-\infty,-2)$ or $x\in (0,1)$ we have $f'(x)<0$, and so $f$ is strictly monotonically decreasing there.}
  \step \lang{de}{F\"{u}r $x\in (-2;0)$ sowie $x\in(1;\infty)$ ist $f'(x)>0$, und damit ist $f$ dort streng monoton steigend.}
  \lang{en}{For $x\in (-2,0)$ or $x\in(1,\infty)$ we have $f'(x)>0$, and so $f$ is strictly monotonically increasing there.}
   


   
  
  \end{incremental}


\end{tabs*}

\end{content}