\documentclass{mumie.element.exercise}
%$Id$
\begin{metainfo}
  \name{
    \lang{de}{Ü09: Wendestellen}
    \lang{en}{Exercise 9}
  }
  \begin{description} 
 This work is licensed under the Creative Commons License Attribution 4.0 International (CC-BY 4.0)   
 https://creativecommons.org/licenses/by/4.0/legalcode 

    \lang{de}{}
    \lang{en}{here the description}
  \end{description}
  \begin{components}
  \end{components}
  \begin{links}
  \end{links}
  \creategeneric
\end{metainfo}
\begin{content}

\title{
\lang{de}{Ü09: Wendestellen}
\lang{en}{Exercise 9}
}

\begin{block}[annotation]
  Im Ticket-System: \href{http://team.mumie.net/issues/9459}{Ticket 9459}
\end{block}


\lang{de}{Bestimmen Sie die zweite Ableitung und die Wendestellen der Funktion $f(x)=3x^5-5x^4+2x+1$.}
\lang{en}{Determine the second derivative and the inflection points of the function $f(x)=3x^5-5x^4+2x+1$.}

\begin{tabs*}[\initialtab{0}\class{exercise}]
  \tab{
  \lang{de}{Antwort}
  \lang{en}{Answer}
  }
  \lang{de}{$f''(x)=60x^2(x-1)$. $f$ besitzt eine Wendestelle in $1$. $f$ besitzt keine weiteren Wendestellen.}
  \lang{en}{$f''(x)=60x^2(x-1)$. $f$ has only one inflection point which is at $1$.}
 
  \tab{
  \lang{de}{L\"{o}sung}
  \lang{en}{Solution}
  }
  
  \begin{incremental}[\initialsteps{1}]
    
\step \lang{de}{Wir berechnen $f'(x)$:}
\lang{en}{We calculate $f'(x)$:}
\step\[f'(x)=15x^4-20x^3+2.\]
\step \lang{de}{Wir berechnen $f''(x)$:}
\lang{en}{We calculate $f''(x)$:}
\[f''(x)=60x^3-60x^2=60x^2(x-1).\] 
\step \lang{de}{Nun bestimmen wir die Wendestellen. Daf\"{u}r setzen wir $f''(x)=0$ an. }
\lang{en}{Now we determine the inflection points. Therefore, we set $f''(x)=0$.}
\step \lang{de}{$f''$ hat die beiden Nullstellen $x=0$ und $x=1$. }
\lang{en}{$f''(x)$ has two roots $0$ and $1$.}

\step \lang{de}{Da $x^2\geq 0$ f\"{u}r alle $x\in\R$ und $x-1<0$ f\"{u}r alle $x<1$, ist $f''(x)=60x^2(x-1)\leq 0$ 
f\"{u}r alle $x<1$. Damit liegt in $x=0$ kein Vorzeichenwechsel von $f''$ vor, und folglich ist $x=0$ keine Wendestelle von $f$.}
\lang{en}{Since $x^2\geq 0$ for all $x\in\R$ and $x-1<0$ for all $x<1$ then $f''(x)=60x^2(x-1)\leq 0$ for all $x<1$. 
Therefore, there is no change of sign of $f''$ at $0$ and it follows that $0$ is not an inflection point of $f$.}
 \step \lang{de}{Da aber $60x^2(x-1)< 0$ f\"{u}r alle $0<x<1$ und $60x^2(x-1)> 0$ f\"{u}r alle $x>1$, 
 liegt in $x=1$ ein Vorzeichenwechsel von $f''$ und damit eine Wendestelle von $f$ vor.}
 \lang{en}{Since $60x^2(x-1)< 0$ for all $0<x<1$ and $60x^2(x-1)> 0$ for all $x>1$, there exists a change of sign of $f''$ and, therefore, an inflection point of $f$. }
   
  
  \end{incremental}


\end{tabs*}
\end{content}