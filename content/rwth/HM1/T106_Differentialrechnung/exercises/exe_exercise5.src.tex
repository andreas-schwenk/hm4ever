\documentclass{mumie.element.exercise}
%$Id$
\begin{metainfo}
  \name{
    \lang{de}{Ü05: Funktionsgraph}
    \lang{en}{Exercise 5}
  }
  \begin{description} 
 This work is licensed under the Creative Commons License Attribution 4.0 International (CC-BY 4.0)   
 https://creativecommons.org/licenses/by/4.0/legalcode 

    \lang{de}{Hier die Beschreibung}
    \lang{en}{}
  \end{description}
  \begin{components}
  \component{generic_image}{content/rwth/HM1/images/g_tkz_T106_Exercise05_D.meta.xml}{T106_Exercise05_D}
  \component{generic_image}{content/rwth/HM1/images/g_tkz_T106_Exercise05_C.meta.xml}{T106_Exercise05_C}
  \component{generic_image}{content/rwth/HM1/images/g_tkz_T106_Exercise05_B.meta.xml}{T106_Exercise05_B}
  \component{generic_image}{content/rwth/HM1/images/g_tkz_T106_Exercise05_A.meta.xml}{T106_Exercise05_A}
  \end{components}
  \begin{links}
  \end{links}
  \creategeneric
\end{metainfo}
\begin{content}
\title{
  \lang{de}{Ü05: Funktionsgraph}
  \lang{en}{Exercise 5}
}





\begin{block}[annotation]
  TODO: Konzeption 2
\end{block}

\begin{block}[annotation]
  Im Ticket-System: \href{http://team.mumie.net/issues/9105}{Ticket 9105}
\end{block}

\lang{de}{Die Ableitung einer Funktion $f$ soll das folgende Vorzeichen besitzen: 

\begin{table}
 Intervall   &$(-\infty;-1)$&$-1$&$(-1;1)$&$1$&$(1;3)$&$3$&$(3;\infty)$\\
$f'$& $\;\;\;\;\;\;+$&$\;\,0$&$\;\;\;\;-$&$0$&$\;\;\,+$&$0$&$\;\;\;+$\\
\end{table} }

\lang{en}{The derivative of a function $f$ should have the following sign:
\begin{table}
 Interval   &$(-\infty,-1)$&$-1$&$(-1,1)$&$1$&$(1,3)$&$3$&$(3,\infty)$\\
$f'$& $\;\;\;\;\;\;+$&$\;\,0$&$\;\;\;\;-$&$0$&$\;\;\,+$&$0$&$\;\;\;+$\\
\end{table} }



\lang{de}{Skizzieren Sie einen m\"{o}glichen Graphen von $f$.}
\lang{en}{Sketch a possible graph of $f$.}



\begin{tabs*}[\initialtab{0}\class{exercise}]
 
  \tab{
  \lang{de}{L\"{o}sung}
  \lang{en}{Solution}
  }
  
  \begin{incremental}[\initialsteps{1}]
   

 
    \step
    
    \lang{de}{ $f$ ist auf $(-\infty;-1)$ monoton wachsend, 
    da die Ableitung dort ein positives Vorzeichen hat. Au{\ss}erdem ist $f'(-1)=0$, was bedeutet, dass in $x=-1$ die Tangente an den Graphen 
    die Steigung $0$ hat.
    Damit verl\"{a}uft $f$ auf $(-\infty;-1]$ ungef\"{a}hr wie in diesem Bild:}
    \lang{en}{Since the slope is positive in the interval $(-\infty,-1)$, $f$ is monotonically increasing. In addition, $f'(-1)=0$ i.e. at $x=-1$, the tangent on the graph has a slope of $0$.
    Therefore, the graph of $f$ in the interval $(-\infty,-1]$ should be roughly of the following form:}\\
    \begin{center}
    \image{T106_Exercise05_A}
    \end{center}
    \step 
    \lang{de}{Im Intervall $(-1;1)$ ist $f$ monoton fallend, da die Ableitung negativ ist. In $x=1$ ist die Tangente wieder wegen $f'(1)=0$ horizontal.}
    \lang{en}{Since the slope is negative in the interval $(-1,1)$, $f$ is monotonically decreasing. The tangent is horizontal again at $x=1$ because $f'(1)=0$.}
    \begin{center}
   \image{T106_Exercise05_B}
    \end{center}
     \step 
     \lang{de}{Auf dem Intervall $(1;3)$ steigt $f$ wegen des positiven Vorzeichens der Ableitung wieder, $x=3$ ist wegen $f'(3)=0$ eine station\"{a}re Stelle.}
     \lang{en}{The slope is positive again in the interval $(1,3)$, so $f$ increases. Since $f'(3)=0$, $x=3$ is a stationary point.}\\
      \begin{center}
      \image{T106_Exercise05_C}
      \end{center}
      \step 
      \lang{de}{Da $f'(x)$ ein positives Vorzeichen hat f\"{u}r $x>3$, steigt $f$ auf $(3;\infty)$ wieder monoton.}
      \lang{en}{Since $f'(x)$ is positive for $x>3$, $f$ continues to increase in the interval $(3,\infty)$.}
      \begin{center}
      \image{T106_Exercise05_D}
      \end{center}
      
     \lang{de}{ Die oben beschriebene Ableitung könnte z.B. $f'(x)=(x+1)(x-1)(x-3)^2$ lauten.}
     \lang{en}{The derivative described above could be, for example, $f'(x)=(x+1)(x-1)(x-3)^2$.}
    

    

   
  
  \end{incremental}


\end{tabs*}
\end{content}