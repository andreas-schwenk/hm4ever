\documentclass{mumie.element.exercise}
%$Id$
\begin{metainfo}
  \name{
    \lang{de}{Ü03: Kettenregel}
    \lang{en}{Exercise 3}
  }
  \begin{description} 
 This work is licensed under the Creative Commons License Attribution 4.0 International (CC-BY 4.0)   
 https://creativecommons.org/licenses/by/4.0/legalcode 

    \lang{de}{Hier die Beschreibung}
    \lang{en}{}
  \end{description}
  \begin{components}
  \end{components}
  \begin{links}
  \end{links}
  \creategeneric
\end{metainfo}
\begin{content}
\title{
  \lang{de}{Ü03: Kettenregel}
  \lang{en}{Exercise 3}
}





\begin{block}[annotation]
  TODO: Konzeption 2
\end{block}

\begin{block}[annotation]
  Im Ticket-System: \href{http://team.mumie.net/issues/9103}{Ticket 9103}
\end{block}





\lang{de}{



Berechnen Sie die Ableitungen der folgenden Funktionen mit Hilfe der Kettenregel.}

\begin{table}[\class{items}]
\nowrap{a) $f(x)=\ln(x^{2}+1)$} \\
\nowrap{b) $f(x)=\exp(\sin(2x))$} \\
\nowrap{c) $f(x)=\cos^{2}(\ln(x+1))$}
\end{table}

Bestimmen Sie die Ableitung der folgenden Funktionen mit Hilfe der Produktregel.

\begin{table}[\class{items}]
\nowrap{d)  $f(x)=xe^{x}$.} \\
\nowrap{e)  $f(x)=x^{4}+\ln(x)\cdot \sqrt{x}$.} \\
\nowrap{f)  $f(x)=\sin(x)\cos(x)$.}
\end{table}



\begin{tabs*}[\initialtab{0}\class{exercise}]
 \tab{
  \lang{de}{Antwort}
  \lang{en}{Answer}
  }
  \begin{table}[\class{items}]
\nowrap{a) $f'(x)=\frac{2x}{x^{2}+1}$}\\
\nowrap{b) $f'(x)=2\exp(\sin(2x))\cos(2x)$}\\
\nowrap{c) $f'(x) =\frac{-2}{x+1}\cos(\ln(x+1))\sin(\ln(x+1))$}\\
\nowrap{d) $f'(x)=e^{x}(1+x)$} \\
\nowrap{e) $f'(x)=4x^{3}+\frac{\sqrt{x}}{x}+\frac{\ln(x)}{2\sqrt{x}}$} \\
\nowrap{f) $f'(x)=\cos^{2}(x)-\sin^{2}(x)$}
\end{table}

\tab{
  \lang{de}{Lösung a)}
  \lang{en}{Solution}
  }
  
  Die Funktion $f$ ist auf ganz $\R$ definiert. Mit Hilfe der Kettenregel erhalten wir als Ableitung
\[f'(x)=\ln'(x^{2}+1)\cdot (x^{2}+1)'=\frac{2x}{x^{2}+1}\,.\]

\tab{
\lang{de}{Lösung b)}
}
Die Funktion $f$ ist auf ganz $\R$ definiert. Mit Hilfe der Kettenregel erhalten wir als Ableitung
\[f'(x)=\exp'(\sin(2x))\cdot \sin'(2x)\cdot (2x)'=\exp(\sin(2x))\cos(2x)\cdot 2=2\exp(\sin(2x))\cos(2x)\,.\]
  
  \tab{
	\lang{de}{Lösung c)}
	}
	Die Funktion $f$ ist überall dort definert, wo $x+1$ postiv ist. Daher lautet der Definitionsbereich von $f$ genau $(-1,\infty)$.  Mit Hilfe der Kettenregel erhalten wir als Ableitung
\begin{align*}
   f'(x)&=&&2\cos(\ln(x+1))\cdot \cos'(\ln(x+1))\cdot(\ln'(x+1))\cdot (x+1)'&\\
&=&&2\cos(\ln(x+1))(-\sin(\ln(x+1)))\cdot\frac{1}{x+1}&\\
&=&&\frac{-2}{x+1}\cos(\ln(x+1))\sin(\ln(x+1))\,.&
  \end{align*}
  
  \tab{
  \lang{de}{Lösung d)}
  \lang{en}{Solution}
  }
  
  \begin{incremental}[\initialsteps{1}]
    \step \lang{de}{Wir verwenden die Ableitungsformeln. Die Funktion $f$ lässt sich schreiben als Produkt $f(x)=u(x)v(x)$ mit $u(x)=x$ und $v(x)=e^{x}$.}
    \step \lang{de}{Die Produktregel besagt nun, dass $f'(x)=u'(x)v(x)+u(x)v'(x)$ für alle $x\in\R$ gilt. Somit bekommen wir als Ableitung
\[f'(x)=e^{x}+xe^{x}=e^{x}(1+x)\,.\]}
  \end{incremental}
  
  \tab{
  \lang{de}{Lösung e)}
  \lang{en}{Solution}
  }
  
  \begin{incremental}[\initialsteps{1}]
    \step \lang{de}{Wir verwenden wieder die Produktregel mit $u(x)=\ln(x)$ und $v(x)=\sqrt{x}$ für alle $x>0$.}
    \step \lang{de}{Aus der Summen- und Faktorregel der Ableitung ergibt sich somit 
\[f'(x)=\frac{d}{dx}(x^{4})+u'(x)v(x)+u(x)v'(x)=4x^{3}+\frac{\sqrt{x}}{x}+\frac{\ln(x)}{2\sqrt{x}}\]
für alle $x>0$ als Ableitung.}
  \end{incremental}  

 \tab{
  \lang{de}{Lösung f)}
  \lang{en}{Solution}
  }
  
 \lang{de}{Wir setzen $u(x)=\sin(x)$ und $v(x)=\cos(x)$. Aus der Produktregel erhalten wir mit Hilfe der Ableitungsformeln für Sinus und Cosinus
\[f'(x)=u'(x)v(x)+u(x)v'(x)=\cos(x)\cos(x)+\sin(x)[-\sin(x)]=\cos^{2}(x)-\sin^{2}(x)\]
für alle $x\in\R$.} 

   
\end{tabs*}

\end{content}