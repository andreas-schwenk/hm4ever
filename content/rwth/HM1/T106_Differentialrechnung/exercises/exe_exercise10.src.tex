\documentclass{mumie.element.exercise}
%$Id$
\begin{metainfo}
  \name{
    \lang{de}{Ü10: Kurvendiskussion}
    \lang{en}{Exercise 10}
  }
  \begin{description} 
 This work is licensed under the Creative Commons License Attribution 4.0 International (CC-BY 4.0)   
 https://creativecommons.org/licenses/by/4.0/legalcode 

    \lang{de}{Hier die Beschreibung}
    \lang{en}{}
  \end{description}
  \begin{components}
  \end{components}
  \begin{links}
  \end{links}
  \creategeneric
\end{metainfo}
\begin{content}

\title{
\lang{de}{Ü10: Kurvendiskussion}
\lang{en}{Exercise 10}
}
 
 \begin{block}[annotation]
  Im Ticket-System: \href{http://team.mumie.net/issues/9460}{Ticket 9460}
\end{block}

\lang{de}{
Es sei $ f : \R \setminus \{0\} \to \R$, $x \mapsto \frac{e^x}{x}$.

a) Berechnen Sie Nullstellen, Extremstellen und Wendestellen von $f$.\\
b) Geben Sie das Verhalten von $f$ an den Rändern des Definitionsbereiches und der Definitionslücke an.\\
c) Skizzieren Sie grob den Funktionsgraph von $f$ auf Grund der Informationen aus a) und b).
}



\begin{tabs*}[\initialtab{0}\class{exercise}]
  \tab{
  \lang{de}{Antwort}
  \lang{en}{Answer}
  }
  \lang{de}{
  a) Es gibt keine Nullstellen. $f$ hat bei $x=1$ eine Minimalstelle, aber keine Wendestellen.\\
  b) $\lim_{x \to -\infty} f(x) = 0$, $\lim_{x \to \infty} f(x) = \infty$, $\lim_{x \to 0^{-}}f(x) = - \infty$,
  $\lim_{x \to 0^{+}} f(x) = \infty$.\\
  c) siehe Video.\\
  
  
  }
  
  
  \lang{en}{There exists a local maximum at $x=2$. There are no more local extrema. }
 
  \tab{\lang{de}{Lösungsvideo a) - c)}}
  \youtubevideo[500][300]{QOBLhb2TQFg}\\
 

   

\end{tabs*}
 
\end{content}