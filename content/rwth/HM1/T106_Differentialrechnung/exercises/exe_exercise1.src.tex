\documentclass{mumie.element.exercise}
%$Id$
\begin{metainfo}
  \name{
    \lang{de}{Ü01: Tangente}
    \lang{en}{Exercise 1}
  }
  \begin{description} 
 This work is licensed under the Creative Commons License Attribution 4.0 International (CC-BY 4.0)   
 https://creativecommons.org/licenses/by/4.0/legalcode 

    \lang{de}{Hier die Beschreibung}
    \lang{en}{}
  \end{description}
  \begin{components}
  \end{components}
  \begin{links}
  \end{links}
  \creategeneric
\end{metainfo}
\begin{content}
\title{
  \lang{de}{Ü01: Tangente}
  \lang{en}{Exercise 1}
}





\begin{block}[annotation]
  TODO: Konzeption 2
\end{block}

\begin{block}[annotation]
  Im Ticket-System: \href{http://team.mumie.net/issues/9104}{Ticket 9104}
\end{block}



\lang{de}{
\begin{table}[\class{items}]
a) Bestimmen Sie die Gleichung der Tangente für die Funktion $f(x)=mx+c$ für beliebig gewählte $c,m\in\R$ an der Stelle $x_{0}\in\R$. Was fällt Ihnen auf? \\
b) Bestimmen Sie die Gleichung der Tangente für die Funktion $f(x)=x^{2}-4x+6$ an der Stelle $x_{0}=1$. \\
c) Für welche Punkte auf der Parabel $f(x) = x^2$ führt die Tangente an $f$ durch den Punkt $(1,-3)$?
\end{table}
}


\begin{tabs*}[\initialtab{0}\class{exercise}]
  \tab{
  \lang{de}{Lösung a)}
  \lang{en}{Solution}
  }
  
  \begin{incremental}[\initialsteps{1}]
    \step \lang{de}{Die Ableitung von $f$ an der Stelle $x_{0}$ ist nach den Ableitungsregeln $f'(x_{0})=m$. 
    Also ist die Steigung der Tangente an der Stelle $x_{0}$ gegeben durch $f'(x_{0})=m$.}
    \step \lang{de}{Somit erhalten wir die Tangentengleichung
\[T(x)=f'(x_{0})(x-x_{0})+f(x_{0})=m(x-x_{0})+(mx_{0}+c)=mx+c\,.\]
Also stimmt die Tangente an $f$ an jeder Stelle $x_{0}$ mit der Funktionsvorschrift von $f$ überein.}    
  \end{incremental}

\tab{
	\lang{de}{Lösung b)}
  	\lang{en}{Solution}
  	}
  
  \begin{incremental}[\initialsteps{1}]
    \step \lang{de}{Nach den Ableitungsregeln gilt $f'(x)=2x-4$ für alle $x\in\R$. Daher erhalten wir $f'(x_{0})=-2$.}
    \step \lang{de}{Aus der Gleichung $f(x_{0})=f(1)=3$ bekommen wir somit die Gleichung der Tangenten als
\[T(x)=f'(x_{0})(x-x_{0})+f(x_{0})=-2(x-1)+3=-2x+5\,.\]}
  \end{incremental}
  
    \tab{\lang{de}{Lösungsvideo c)}}
  \youtubevideo[500][300]{e4ZHVG_B2qU}\\
  
\end{tabs*}

\end{content}