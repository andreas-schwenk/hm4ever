\documentclass{mumie.problem.gwtmathlet}
%$Id$
\begin{metainfo}
  \name{
    \lang{de}{A04: Faktorisierung}
    \lang{en}{Exercise 4}
  }
  \begin{description} 
 This work is licensed under the Creative Commons License Attribution 4.0 International (CC-BY 4.0)   
 https://creativecommons.org/licenses/by/4.0/legalcode 

    \lang{de}{Die Beschreibung}
    \lang{en}{}
  \end{description}
  \corrector{system/problem/GenericCorrector.meta.xml}
  \begin{components}
    \component{js_lib}{system/problem/GenericMathlet.meta.xml}{gwtmathlet}
  \end{components}
  \begin{links}
  \end{links}
  \creategeneric
\end{metainfo}
\begin{content}

\usepackage{mumie.genericproblem}

\lang{de}{
	\title{A04: Faktorisierung}
}

\lang{en}{
	\title{Exercise 4}
}

\begin{block}[annotation]
	 
\end{block}
\begin{block}[annotation]
	Im Ticket-System: \href{http://team.mumie.net/issues/9338}{Ticket 9338}
\end{block}


\begin{problem}
	\begin{variables}
			\randint{a0}{-5}{5}
            \randint{a1}{-5}{5}
            \randint{a2}{-5}{5}
            \randint[Z]{a3}{-5}{5}
            \randint[Z]{x0}{-2}{2}
            \function{g0}{a3*x^3+a2*x^2+a1*x+a0}
            \function[expand, normalize]{p}{(x-x0)*g0}
            \function[expand, normalize]{b}{x-x0}
	\end{variables}
	\begin{question}
		\lang{de}{
			\text{Gegeben sei das Polynom $p(x)=\var{p}$ mit Nullstelle $x_0= \var{x0}$.
			Bestimmen Sie $g(x)$, so dass $p(x)= (\var{b}) \cdot g(x)$.}
			\explanation{Polynomdivision}
		}
		
		\type{input.function}
		\begin{answer}
			\text{$g(x) =$}
			\solution{g0}
            \checkAsFunction{x}{-10}{10}{5}
		\end{answer}
		
	\end{question}
	
\end{problem}

\embedmathlet{gwtmathlet}

\end{content}