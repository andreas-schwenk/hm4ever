\documentclass{mumie.problem.gwtmathlet}
%$Id$
\begin{metainfo}
  \name{
    \lang{de}{A02: Funktionsgrenzwerte}
    \lang{en}{Exercise 2}
  }
  \begin{description} 
 This work is licensed under the Creative Commons License Attribution 4.0 International (CC-BY 4.0)   
 https://creativecommons.org/licenses/by/4.0/legalcode 

    \lang{de}{}
    \lang{en}{}
  \end{description}
  \corrector{system/problem/GenericCorrector.meta.xml}
  \begin{components}
    \component{js_lib}{system/problem/GenericMathlet.meta.xml}{gwtmathlet}
  \end{components}
  \begin{links}
  \end{links}
  \creategeneric
\end{metainfo}
\begin{content}

\usepackage{mumie.genericproblem}

 \lang{de}{
 	 \title{A02: Funktionsgrenzwerte}
  }
  \lang{en}{
  	\title{Exercise 2} 
  }
  \begin{block}[annotation]
	 
\end{block}
\begin{block}[annotation]
	Im Ticket-System: \href{http://team.mumie.net/issues/9337}{Ticket 9337}
\end{block}

\begin{problem}
	\randomquestionpool{1}{3}

%1. Möglichkeit
	\begin{question}
	
	\begin{variables}
      		\randint{a0}{-10}{10}
            \randint[Z]{a1}{-10}{10}
            \randint{a2}{-10}{10}
            \randint{a3}{-10}{10}
            \randint{a4}{-10}{10}
            \randint[Z]{a5}{-10}{10}
            \function[expand, normalize]{f1}{a5*x^5+a4*x^4+ a3*x^3+a2*x^2+a1*x+a0}
            \function{s1}{a5*x^5}
            \function{s2}{a1*x+a0}
	\end{variables}
	
	\precision{4}
		\lang{de}{
			\text{Gegeben ist die Funktion $p(x) = \var{f1}$.}
		}
		
		\type{input.function}
		\begin{answer}
			\text{$p(x)$ verhält sich für $x \to \pm \infty$ wie }
			\solution{s1}
			\checkAsFunction{x}{-10}{10}{100}
		\end{answer}
		\begin{answer}
			\text{$p(x)$ verhält sich für $x \to 0$ wie }
			\solution{s2}
			\checkAsFunction{x}{-10}{10}{100}
		\end{answer}
        \explanation{Für das Verhalten $x \to \pm \infty$ muss man das Monom höchsten Grades betrachten. Für das Verhalten $x \to 0$ betrachten wir den Term niedrigster Ordnung.}
	\end{question}

%2.Möglichkeit
\begin{question}
	
	\begin{variables}
      		\randint{a0}{-10}{10}
            \randint[Z]{a2}{-10}{10}
            \randint{a3}{-10}{10}
            \randint{a4}{-10}{10}
            \randint[Z]{a5}{-10}{10}
            \function[expand, normalize]{f1}{a5*x^5+a4*x^4+ a3*x^3+a2*x^2+a0}
            \function{s1}{a5*x^5}
            \function{s2}{a2*x^2+a0}
	\end{variables}
	
	\precision{4}
		\lang{de}{
			\text{Gegeben ist die Funktion $p(x) = \var{f1}$.}
		}
		
		\type{input.function}
		\begin{answer}
			\text{$p(x)$ verhält sich für $x \to \pm \infty$ wie }
			\solution{s1}
			\checkAsFunction{x}{-10}{10}{100}
		\end{answer}
		\begin{answer}
			\text{$p(x)$ verhält sich für $x \to 0$ wie }
			\solution{s2}
			\checkAsFunction{x}{-10}{10}{100}
            \explanation{Für das Verhalten $x \to \pm \infty$ muss man das Monom höchsten Grades betrachten. Für das Verhalten $x \to 0$ betrachten wir den Term niedrigster Ordnung.}
		\end{answer}
	\end{question}

%3.Möglichkeit
\begin{question}
	
	\begin{variables}
      		\randint{a0}{-10}{10}
            \randint[Z]{a3}{-10}{10}
            \randint{a4}{-10}{10}
            \randint[Z]{a5}{-10}{10}
            \function[expand, normalize]{f1}{a5*x^5+a4*x^4+ a3*x^3+a0}
            \function{s1}{a5*x^5}
            \function{s2}{a3*x^3+a0}
	\end{variables}
	
	\precision{4}
		\lang{de}{
			\text{Gegeben ist die Funktion $p(x) = \var{f1}$.}
		}
		
		\type{input.function}
		\begin{answer}
			\text{$p(x)$ verhält sich für $x \to \pm \infty$ wie }
			\solution{s1}
			\checkAsFunction{x}{-10}{10}{100}
		\end{answer}
		\begin{answer}
			\text{$p(x)$ verhält sich für $x \to 0$ wie }
			\solution{s2}
			\checkAsFunction{x}{-10}{10}{100}
            \explanation{Für das Verhalten $x \to \pm \infty$ muss man das Monom höchsten Grades betrachten. Für das Verhalten $x \to 0$ betrachten wir den Term niedrigster Ordnung.}
		\end{answer}
	\end{question}

\end{problem}

\embedmathlet{gwtmathlet}

\end{content}