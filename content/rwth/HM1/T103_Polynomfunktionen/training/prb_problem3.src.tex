\documentclass{mumie.problem.gwtmathlet}
%$Id$
\begin{metainfo}
  \name{
    \lang{de}{A03: Horner-Schema}
    \lang{en}{}
  }
  \begin{description} 
 This work is licensed under the Creative Commons License Attribution 4.0 International (CC-BY 4.0)   
 https://creativecommons.org/licenses/by/4.0/legalcode 

    \lang{de}{Beschreibung}
    \lang{en}{}
  \end{description}
  \corrector{system/problem/GenericCorrector.meta.xml}
  \begin{components}
    \component{js_lib}{system/problem/GenericMathlet.meta.xml}{mathlet}
  \end{components}
  \begin{links}
  \end{links}
  \creategeneric
\end{metainfo}
\begin{content}
\usepackage{mumie.genericproblem}


\lang{de}{
	\title{A03: Horner-Schema}
}
\lang{en}{
	\title{Problem 3}
}

\begin{block}[annotation]
	 
\end{block}
\begin{block}[annotation]
	Im Ticket-System: \href{http://team.mumie.net/issues/9085}{Ticket 9085}
\end{block}



\begin{problem}

%Frage 1 von 7
\begin{question}
	\type{input.number}
	\field{rational}
	\precision{3}

	\begin{variables}
      		\randint{a0}{-5}{5}
            \randint{a1}{-5}{5}
            \randint{a2}{-5}{5}
            \randint{a3}{-5}{5}
            \randint[Z]{a4}{-5}{5}
            \randint[Z]{x0}{-2}{2}
            \function[expand, normalize]{f1}{a4*x^4+ a3*x^3+a2*x^2+a1*x+a0}
            \function{s11}{x0*a4}
            \function{s12}{x0*(a3+x0*a4)}
            \function{s13}{x0*(a2+x0*(a3+x0*a4))}
            \function{s14}{x0*(a1+x0*(a2+x0*(a3+x0*a4)))}
            \function{s21}{a3+ s11}
            \function{s22}{a2+s12}
            \function{s23}{a1+s13}
            \function{s24}{a0+s14}
	\end{variables}
    \lang{de}{
\text{Gegeben sei das Polynom $p(x)=\var{f1}$. Bestimmen Sie $p(\var{x0})$ mit Hilfe 
des Horner-Schemas: \\
\phantom{xxxxxi}  $\vert$  \ansref  \ansref  \ansref  \ansref  \ansref  \\ 
\ansref $\vert$ \phantom{xxxxx}  \ansref  \ansref  \ansref  \ansref  \\ 
\phantom{xxxxxi}  $\overline{\phantom{ixxxxxxxxxxxxxxxxxxxxxxx}}$ \\		
\phantom{xxxxxi} \phantom{|} \ansref  \ansref  \ansref  \ansref  \ansref  
\\
Es ist also $p(\var{x0})=$\ansref
}}

%\head \phantom{xxxx} & \phantom{xxxx} &\phantom{xxxx}  &\phantom{xxxx} &\phantom{xxxx} &\phantom{xxxx} & \\
%\body &$\Bigg|$& \ansref & \ansref & \ansref & \ansref & \ansref\\
%\ansref &$\Bigg|$&       & \ansref & \ansref & \ansref & \ansref\\
%		 & &\colspan{5}$\overline{\phantom{xxxxxxxxxxxxxxxxxxxxxx}}$ \\		
%	\foot &&  \ansref & \ansref & \ansref & \ansref & \ansref\\            

		\begin{answer}\solution{a4}\end{answer}
        \explanation{Schreiben Sie die Koeffizienten in die erste Zeile. Auch der Koeffizient Null muss berücksichtigt werden. Orientieren Sie sich an den gelösten Beispielen.}
		\begin{answer}\solution{a3}\end{answer}
		\begin{answer}\solution{a2}\end{answer}
		\begin{answer}\solution{a1}\end{answer}
		\begin{answer}\solution{a0}\end{answer}
		\begin{answer}\solution{x0}\end{answer}
		\begin{answer}\solution{s11}\end{answer}
		\begin{answer}\solution{s12}\end{answer}
		\begin{answer}\solution{s13}\end{answer}
		\begin{answer}\solution{s14}\end{answer}
		\begin{answer}\solution{a4}\end{answer}
		\begin{answer}\solution{s21}\end{answer}
 		\begin{answer}\solution{s22}\end{answer}
 		\begin{answer}\solution{s23}\end{answer}
 		\begin{answer}\solution{s24}\end{answer}
		\begin{answer}\solution{s24}\end{answer}
\end{question}
\end{problem}


\embedmathlet{mathlet}

\end{content}