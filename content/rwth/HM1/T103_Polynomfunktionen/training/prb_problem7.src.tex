\documentclass{mumie.problem.gwtmathlet}
%$Id$
\begin{metainfo}
  \name{
    \lang{de}{A07: rationale Funktion}
    \lang{en}{Exercise 7}
  }
  \begin{description} 
 This work is licensed under the Creative Commons License Attribution 4.0 International (CC-BY 4.0)   
 https://creativecommons.org/licenses/by/4.0/legalcode 

    \lang{de}{Die Beschreibung}
    \lang{en}{}
  \end{description}
  \corrector{system/problem/GenericCorrector.meta.xml}
  \begin{components}
    \component{js_lib}{system/problem/GenericMathlet.meta.xml}{gwtmathlet}
  \end{components}
  \begin{links}
  \end{links}
  \creategeneric
\end{metainfo}
\begin{content}

\usepackage{mumie.genericproblem}

\lang{de}{
	\title{A07: rationale Funktion}
}

\lang{en}{
	\title{Exercise 7}
}

\begin{block}[annotation]
	 
\end{block}
\begin{block}[annotation]
	Im Ticket-System: \href{http://team.mumie.net/issues/9341}{Ticket 9341}
\end{block}


\begin{problem}
	\begin{variables}
			\randint{a}{-5}{5}
            
            \function{u}{x+a}
            \function[normalize]{v}{2*x+3+2*a}
            \function[normalize]{r}{(x^3+a*x^2+3)/(x^2-2)}
            
	\end{variables}
	\begin{question}
		\type{input.function}
		\lang{de}{\text{Schreiben Sie die folgende gebrochen rationale Funktion als Summe eines Polynoms und einer echt gebrochen rationalen Funktion:\\ $r(x)=\var{r}$.\\ Es ist $r(x)=p(x)+\frac{q(x)}{x^2-2}$.}
		\explanation{Polynomdivision}}
		\begin{answer}
			\text{$p(x)=$}
			\solution{u}
			\checkAsFunction{x}{-10}{10}{100}
		\end{answer}
		\begin{answer}
			\text{$q(x)=$}
			\solution{v}
			\checkAsFunction{x}{-10}{10}{100}
		\end{answer}
		
	\end{question}
	
\end{problem}

\embedmathlet{gwtmathlet}

\end{content}