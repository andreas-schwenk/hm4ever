\documentclass{mumie.problem.gwtmathlet}
%$Id$
\begin{metainfo}
  \name{
    \lang{de}{A05: Division mit Rest}
    \lang{en}{Exercise 5}
  }
  \begin{description} 
 This work is licensed under the Creative Commons License Attribution 4.0 International (CC-BY 4.0)   
 https://creativecommons.org/licenses/by/4.0/legalcode 

    \lang{de}{Die Beschreibung}
    \lang{en}{}
  \end{description}
  \corrector{system/problem/GenericCorrector.meta.xml}
  \begin{components}
    \component{js_lib}{system/problem/GenericMathlet.meta.xml}{gwtmathlet}
  \end{components}
  \begin{links}
  \end{links}
  \creategeneric
\end{metainfo}
\begin{content}

\usepackage{mumie.genericproblem}

\lang{de}{
	\title{A05: Division mit Rest}
}

\lang{en}{
	\title{Exercise 5}
}

\begin{block}[annotation]
	 
\end{block}
\begin{block}[annotation]
	Im Ticket-System: \href{http://team.mumie.net/issues/9339}{Ticket 9339}
\end{block}


\begin{problem}
	\begin{variables}
			\randint{a0}{-5}{5}
            \randint{a1}{-5}{5}
            \randint{a2}{-5}{5}
            \randint{a3}{-5}{5}
            \randint{a4}{-5}{5}
            
            \randint{b0}{-5}{5}
            \randint{b1}{-5}{5}
            \randint{b2}{-5}{5}
            
            \function[expand, normalize]{f}{x^5+a4*x^4+a3*x^3+a2*x^2+a1*x+a0}
            \function[expand, normalize]{g}{x^3+b2*x^2+b1*x+b0}
            
            \function{a}{a4-b2}
            \function{b}{a3-b1-b2*a}
            \function{c}{a2-b0-b1*a-b2*b}
            \function{d}{a1-b0*a-b1*b}
            \function{h}{a0-b*b0}
            
            \function[expand, normalize]{q}{x^2+a*x+b}
            \function[expand, normalize]{r}{c*x^2+d*x+h}
            
	\end{variables}
	\begin{question}
	\type{input.function}
		\lang{de}{\text{Gegeben seien die Polynome $f(x)=\var{f}$ und $g(x)= \var{g}$.\\
		Bestimmen Sie Polynome $q(x)$ und $r(x)$, so dass $f(x)= q(x)\cdot g(x) +r(x)$.
        }
		\explanation{Polynomdivision}
		}
		\begin{answer}
			\text{$q(x) =$}
			\solution{q}
			\checkAsFunction{x}{-10}{10}{100}
		\end{answer}
		\begin{answer}
			\text{$r(x) =$}
			\solution{r}
			\checkAsFunction{x}{-10}{10}{100}
		\end{answer}
		
	\end{question}
	
\end{problem}

\embedmathlet{gwtmathlet}

\end{content}