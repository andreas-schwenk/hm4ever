\documentclass{mumie.problem.gwtmathlet}
%$Id$
\begin{metainfo}
  \name{
    \lang{de}{A06: rationale Funktion}
    \lang{en}{Exercise 6}
  }
  \begin{description} 
 This work is licensed under the Creative Commons License Attribution 4.0 International (CC-BY 4.0)   
 https://creativecommons.org/licenses/by/4.0/legalcode 

    \lang{de}{Die Beschreibung}
    \lang{en}{}
  \end{description}
  \corrector{system/problem/GenericCorrector.meta.xml}
  \begin{components}
    \component{js_lib}{system/problem/GenericMathlet.meta.xml}{gwtmathlet}
  \end{components}
  \begin{links}
  \end{links}
  \creategeneric
\end{metainfo}
\begin{content}

\usepackage{mumie.genericproblem}

\lang{de}{
	\title{Aufgabe 6}
}

\lang{en}{
	\title{Exercise 6}
}

\begin{block}[annotation]
	 
\end{block}
\begin{block}[annotation]
	Im Ticket-System: \href{http://team.mumie.net/issues/9340}{Ticket 9340}
\end{block}


\begin{problem}
    \randomquestionpool{1}{2}
	
	\begin{question}
        \begin{variables}
			\randint{a}{-5}{-2}
            \randint{b}{-1}{5}
            \randint{c}{-1}{5}
            \function[expand,normalize]{q}{(x-b)*(x-c)}
            \function[normalize]{r}{(x-a)/q}     
	    \end{variables}
		\type{input.finite-number-set}
		\field{real}
		\lang{de}{\text{Bestimmen Sie die Definitionslücken von $r(x)=\var{r}$.\\ 
        Tragen Sie die Definitionslücken und Pole ein.}
        \explanation{Die Definitionslücken sind die Nullstellen des Nenners.
        Die Nullstellen des Nenners, die sich nicht mit Nullstellen des Zählers wegheben, sind auch Pole.
    }}
		\begin{answer}
			\text{Menge der Definitionslücken von $r$:}
			\solution{b,c}
		\end{answer}
		\begin{answer}
			\text{Menge der Polstellen von $r$:}
			\solution{b,c}
		\end{answer}
		
	\end{question}
	\begin{question}
        \begin{variables}
			\randint{a}{-5}{5}
            \randint{b}{-5}{5}
            \function[expand,normalize]{q}{(x-a)*(x-b)}
            \function[normalize]{r}{(x-a)/q}     
	    \end{variables}
		\type{input.finite-number-set}
		\field{real}
		\lang{de}{\text{Bestimmen Sie die Definitionslücken von $r(x)=\var{r}$.\\ 
        Tragen Sie die Definitionslücken und Pole ein.}
        \explanation{Die Definitionslücken sind die Nullstellen des Nenners.
        Die Nullstellen des Nenners, die sich nicht mit Nullstellen des Zählers wegheben, sind auch Pole.
    }}
		\begin{answer}
			\text{Menge der Definitionslücken von $r$:}
			\solution{a,b}
		\end{answer}
		\begin{answer}
			\text{Menge der Polstellen von $r$:}
			\solution{b}
		\end{answer}
		
	\end{question}
\end{problem}

\embedmathlet{gwtmathlet}

\end{content}