
%$Id:  $
\documentclass{mumie.article}
%$Id$
\begin{metainfo}
  \name{
    \lang{de}{Überblick: Polynome und rationale Funktionen}
    \lang{en}{Overview: Polynomials and rational functions}
  }
  \begin{description} 
 This work is licensed under the Creative Commons License Attribution 4.0 International (CC-BY 4.0)   
 https://creativecommons.org/licenses/by/4.0/legalcode 

    \lang{de}{Beschreibung}
    \lang{en}{Description}
  \end{description}
  \begin{components}
  \end{components}
  \begin{links}
\link{generic_article}{content/rwth/HM1/T103_Polynomfunktionen/g_art_content_10_polynomdivision.meta.xml}{content_10_polynomdivision}
\link{generic_article}{content/rwth/HM1/T103_Polynomfunktionen/g_art_content_09_polynome.meta.xml}{content_09_polynome}
\end{links}
  \creategeneric
\end{metainfo}
\begin{content}
\begin{block}[annotation]
	Im Ticket-System: \href{https://team.mumie.net/issues/30146}{Ticket 30146}
\end{block}


\begin{block}[annotation]
Im Entstehen: Überblicksseite für Kapitel Polynome und rationale Funktionen
\end{block}

\usepackage{mumie.ombplus}
\ombchapter{1}
\lang{de}{\title{Überblick: Polynome und rationale Funktionen}}
\lang{en}{\title{Overview: Polynomials and rational functions}}



\begin{block}[info-box]
\lang{de}{\strong{Inhalt}}
\lang{en}{\strong{Contents}}


\lang{de}{
    \begin{enumerate}%[arabic chapter-overview]
   \item[3.1] \link{content_09_polynome}{Polynome}
   \item[3.2] \link{content_10_polynomdivision}{Polynomdivision und rationale Funktionen}
     \end{enumerate}
}
\lang{en}{
    \begin{enumerate}%[arabic chapter-overview]
   \item[3.1] \link{content_09_polynome}{Polynomials}
   \item[3.2] \link{content_10_polynomdivision}{Polynomial division and rational functions}
     \end{enumerate}
} %lang

\end{block}

\begin{zusammenfassung}

\lang{de}{
Polynome bilden eine sehr wichtige Klasse von Funktionen. Sie treten sehr häufig in Anwendungen auf, 
ihre Eigenschaften sind einfach aus der Funktionsvorschrift bestimmbar, ihre Werte einfach berechenbar.
\\
Wir zeigen verschiedene Darstellungen von Polynomen, zeigen die Bedeutung ihrer Nullstellen und fakorisieren sie.
Die Polynomdivision ist dafür ein wichtiges Mittel, das auch für spätere Anwendungen wichtig ist. 
Sie führt uns weiter zu den rationalen Funktionen.
\\
Mit dem Horner-Schema geben wir eine effiziente Methode an, Funktionswerte von Polynomen zu bestimmen.
}
\lang{en}{
Polynomials are an incredibly important class of functions. They appear often and with important 
applications throughout mathematics. Their properties are easily determined from their coefficients, 
and they are quick to evaluate.
\\
In this chapter we demonstrate different representations of polynomials, introduce the meaning beind 
their roots, and factorise them. 
Polynomial long division is an important algorithm by which this is done, and is also important for 
later applications. It directly leads to the topic of rational functions.
\\
Horner's method is an efficient method for evaluating polynomials, and can also be used to help 
factorise them.
}


\end{zusammenfassung}

\begin{block}[info]\lang{de}{\strong{Lernziele}}
                   \lang{en}{\strong{Learning Goals}} 
\begin{itemize}[square]
\item \lang{de}{
Sie wissen, was (mehrfache) Nullstellen von Polynomen sind und welche Eigenschaften der Funktionsterm durch sie bekommt.
}
\lang{en}{
To know the definition of (multiple) roots of a polynomial, and their use in representing the 
function.
}
\item \lang{de}{Sie kennen den Begriff \textit{Potenzfunktion}.}
      \lang{en}{Knowing the definition of the \textit{$n$th power} function.}
\item \lang{de}{Sie faktorisieren Polynome mit Hilfe der Polynomdivision.}
      \lang{en}{Being able to factorise polynomials using long division.}
\item \lang{de}{Sie werten Polynome mit Hilfe des Horner-Schemas aus.}
      \lang{en}{Being able to evaluate polynomials using Horner's method.}
\item \lang{de}{Sie bestimmen Definitionsbereiche rationaler Funktionen.}
      \lang{en}{Being able to identify the maximal domain of a rational function.}
\end{itemize}
\end{block}




\end{content}
