\documentclass{mumie.element.exercise}
%$Id$
\begin{metainfo}
  \name{
    \lang{de}{Ü05: Polynomdivision}
    \lang{en}{Exercise 5}
  }
  \begin{description} 
 This work is licensed under the Creative Commons License Attribution 4.0 International (CC-BY 4.0)   
 https://creativecommons.org/licenses/by/4.0/legalcode 

    \lang{de}{Hier die Beschreibung}
    \lang{en}{}
  \end{description}
  \begin{components}
  \end{components}
  \begin{links}
  \end{links}
  \creategeneric
\end{metainfo}
\begin{content}
\title{
\lang{de}{Ü05: Polynomdivision}
\lang{en}{Exercise 5}
}
\begin{block}[annotation]
	Im Ticket-System: \href{http://team.mumie.net/issues/9274}{Ticket 9274}
\end{block}
 
\lang{de}{}
\begin{table}[\class{items}]
    Zerlegen Sie die folgenden Polynome soweit es geht in lineare und quadratische Faktoren \\ 
    
     \nowrap{a) $f(x) = x^4-2x^3 +2x^2 -2x +1$,}\\
     \nowrap{b) $f(x) = x^3 +x^2-4x -4$,}\\
     \nowrap{c) $h(y) = -2y^3 -8y^2-6y $,}\\
     \nowrap{d) $g(a) = a^4 - a^2 -12$,}\\
     
     Bestimmen Sie das Ergebnis der Division mit Rest von $f(x) : g(x)$ für \\
  \nowrap{e) $f(x)=x^{3}+x^{2}+4x-1$, $g(x)=x^{2}+1$, }\\
   \nowrap{f) $f(x)=3x^{4}+x^{2}-3$, $g(x)=x-3$. }\\
  
\end{table}

\begin{tabs*}[\initialtab{0}\class{exercise}]
  \tab{
  \lang{de}{Antwort}
  }
\begin{table}[\class{items}]

   \nowrap{a) $f(x)=(x-1)^2 (x^2-1)$} & \nowrap{b) $f(x) = (x+2)(x-2)(x+1)$}
   & \nowrap{c) $h(y) = -2y (y+1)(y+3)$} & \nowrap{d) $g(a)= (a^{2} + 3)(a-2)(a+2)$} & \nowrap{e) $x+1+\frac{3x-2}{x^2+1}$} &
   \nowrap{f) $3x^3+9x^2+28x+84+\frac{249}{x-3}$}
  \end{table}
  
  
  \tab{\lang{de}{Lösungsvideo a) -d)}}	
    \youtubevideo[500][300]{b5twjIZg6mg}\\
  

  \tab{
  \lang{de}{Lösung e)}}
Wir führen eine Polynomdivision aus, um die Polynome $q(x)$ und $r(x)$ zu ermitteln mit 
\[f(x)=q(x)g(x)+r(x)\,,\] 
so dass Grad$(r) < $Grad$(g) =2$:
\begin{table}[\class{item} \cellaligns{rrrrrrr}]  
(& &$x^3$ & $+$&$x^2$& $+4x$&$-1$& $):(x^2+1)=x+1+\frac{3x-2}{x^2+1}$\\
&$-$&$(x^3$ & & &$+x)$ & &\\
& \colspan{5} \hrule & & \\
& & &  & $ x^2$&$+3x$&$-1$ &  \\
& & & $-$&$x^2$ & & $-1$ & \\
& & & \colspan{4} \hrule &\\
& & &  & & $3x$&$-2$ & 
\end{table}
Also ist $q(x)=x+1$ und $r(x)=3x-2$.

  \tab{
  \lang{de}{Lösung f)}
  }
  Wir führen eine Polynomdivision aus, um die Polynome $q(x)$ und $r(x)$ zu ermitteln mit 
\[f(x)=q(x)g(x)+r(x)\,,\] 
so dass Grad$(r) < $Grad$(g) =1$:

\begin{table}[\class{item} \cellaligns{rrrrrrr}]
$($&$3x^4$ & & $x^2$ & &$-3$&$):(x-3)=3x^3+9x^2+28x+84+\frac{249}{x-3}$\\
& $-(3x^4$&$-9x^3)$ & & & & \\
& \colspan{2} \hrule & & & & \\
& & $9x^3$&$+x^2$ & & & \\
& & $-9x^3$&$+27x^2$ & & & \\
&& \colspan{2} \hrule & & & \\
& & & $28x^2$ & & & \\
& & & $-28x^2$&$+84x$ & &\\
& & & \colspan{2} \hrule & & \\
& & & & $84x$&$-3$& \\
& & & & $-84x$&$+252$ & \\
& & & & \colspan{2} \hrule & \\
& & & & & $249$&
\end{table}


Also ist $q(x)=3x^{3}+9x^{2}+28x+84$ und $r(x)=249$.

    


\end{tabs*}


\end{content}