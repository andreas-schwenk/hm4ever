\documentclass{mumie.element.exercise}
%$Id$
\begin{metainfo}
  \name{
    \lang{de}{Ü07: Polynomdivision}
    \lang{en}{Exercise 7}
  }
  \begin{description} 
 This work is licensed under the Creative Commons License Attribution 4.0 International (CC-BY 4.0)   
 https://creativecommons.org/licenses/by/4.0/legalcode 

    \lang{de}{}
    \lang{en}{}
  \end{description}
  \begin{components}
  \end{components}
  \begin{links}
  \end{links}
  \creategeneric
\end{metainfo}
\begin{content}

\title{
\lang{de}{Ü07: Polynomdivision}
\lang{en}{Exercise 7}
}

\begin{block}[annotation]
	Im Ticket-System: \href{http://team.mumie.net/issues/9280}{Ticket 9280}
\end{block}

\lang{de}{Schreiben Sie die gebrochen rationale Funktion $r(x)=\frac{x^{3}+3x^{2}+x+4}{x-5}$ als eine Summe eines Polynoms und einer echt gebrochen rationalen Funktion. }

\begin{tabs*}[\initialtab{0}\class{exercise}]
  \tab{
  \lang{de}{Antwort}
  }
$x^2+8x+41+\frac{209}{x-5}$

  \tab{
  \lang{de}{Lösung}}
  
  \begin{incremental}[\initialsteps{1}]
    \step 
    \lang{de}{Wir führen die gewünschte Zerlegung mittels Polynomdivision durch.}
    \step \lang{de}{
   \begin{table}[\class{item} \cellaligns{rrrrrrr}]
   $($&$x^3$&$+3x^2$&$+x$&$+4$&$):(x-5)=x^2+8x+41+\frac{209}{x-5}$ \\
   & $-x^3$&$+5x^2$ & & & \\
   & \colspan{2} \hrule & & & \\
   & & $8x^2$&$+x$ & & \\
   & & $-8x^2$&$+40x$ & & \\
   & & \colspan{2} \hrule & &\\
   & & & $41x$&$+4$&  \\
   & & & $-41x$&$+205$& \\
   & & & \colspan{2} \hrule & \\
   & & & & $209$ &
   \end{table}
     }
    \step \lang{de}{Somit lässt sich $r(x)$ für alle $x\in\R\backslash\{5\}$ schreiben als
\[r(x)=x^{2}+8x+41+\frac{209}{x-5}\,.\]}
  \end{incremental}

\end{tabs*}

\end{content}