\documentclass{mumie.element.exercise}
%$Id$
\begin{metainfo}
  \name{
    \lang{de}{Ü02: Funktionsgrenzwerte}
    \lang{en}{Exercise 2}
  }
  \begin{description} 
 This work is licensed under the Creative Commons License Attribution 4.0 International (CC-BY 4.0)   
 https://creativecommons.org/licenses/by/4.0/legalcode 

    \lang{de}{Hier die Beschreibung}
    \lang{en}{}
  \end{description}
  \begin{components}
  \end{components}
  \begin{links}
  \end{links}
  \creategeneric
\end{metainfo}
\begin{content}

\title{
\lang{de}{Ü02: Funktionsgrenzwerte}
}

\begin{block}[annotation]
	Im Ticket-System: \href{http://team.mumie.net/issues/9273}{Ticket 9273}
\end{block}
 
\lang{de}{Bestimmen Sie das Verhalten der folgenden Polynome für $x\to\pm \infty$ und für $x\to 0$.}
\begin{table}[\class{items}]
  \nowrap{a) $p(x)=3x^{3}+2x^2+x$}\\
  \nowrap{b) $p(x)=4x^{4}+2x^{2}+3x+1$}
\end{table}


\begin{tabs*}[\initialtab{0}\class{exercise}]
  \tab{
  \lang{de}{Lösung a)}}
  Die Funktion verhält sich für $x\to\pm\infty$ wie $f(x)=3x^{3}$ und für $x\to 0$ wie $g(x)=x$.

  \tab{
  \lang{de}{Lösung b)}
  }
  Die Funktion verhält sich für $x\to\pm\infty$ wie $f(x)=4x^{4}$ und für $x\to 0$ wie $g(x)=3x+1$.
    \tab{
  \lang{de}{Erklärung}
  }
Es sei $p(x)=a_{n}x^{n}+a_{n-1}x^{n-1}+\ldots a_{1}x+a_{0}$ mit $a_{n}\neq 0$. 
Die Funktion $p$ verhält sich für $x\to\pm\infty$ wie $f(x)=a_{n}x^{n}$. Ist $m\in\N$ die kleinste Zahl mit der Eigenschaft $m\leq n$ und $a_{m}\neq 0$, so verhält sich $p$ für $x\to 0$ wie die Funktion $g(x)=a_{m}x^{m}+a_{0}$.  
  

\end{tabs*}

\end{content}