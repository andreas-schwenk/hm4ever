\documentclass{mumie.element.exercise}
%$Id$
\begin{metainfo}
  \name{
    \lang{de}{Ü03: Horner-Schema}
    \lang{en}{Exercise 3}
  }
  \begin{description} 
 This work is licensed under the Creative Commons License Attribution 4.0 International (CC-BY 4.0)   
 https://creativecommons.org/licenses/by/4.0/legalcode 

    \lang{de}{Hier die Beschreibung}
    \lang{en}{}
  \end{description}
  \begin{components}
  \end{components}
  \begin{links}
  \end{links}
  \creategeneric
\end{metainfo}
\begin{content}

\title{
\lang{de}{Ü03: Horner-Schema}
\lang{en}{Exercise 3}
}
\begin{block}[annotation]
	Im Ticket-System: \href{http://team.mumie.net/issues/9084}{Ticket 9084}
\end{block}
 
\lang{de}{}
\begin{table}[\class{items}]
  \nowrap{a) Berechnen Sie den Wert von $p(x)=2x^{4}-2x^{2}+3x+1$ an der Stelle $x=1$ mit Hilfe des Horner-Schemas.}\\
   \nowrap{b) Berechnen Sie den Wert von $p(x)=3x^{5}+4x^{4}+3x^{2}+5$ an der Stelle $x=2$ mit Hilfe des Horner-Schemas.}
\end{table}

\begin{tabs*}[\initialtab{0}\class{exercise}]
  \tab{
  \lang{de}{Antwort}
  }
\begin{table}[\class{items}]

    \nowrap{a) $4$} & \nowrap{b) $177$}
  \end{table}

  \tab{
  \lang{de}{Lösung a)}}
Wir verwenden das Horner-Schema und übertragen die Koeffizienten von $p$:

\begin{table}[\class{items}]
 & $\Big|$ &  $2$ & $0$ &$-2$ & $3$ & $1$ \\
$1$ & $\Big|$ &  & $2$ & $2$ & $0$ & $3$ \\
\colspan{7} \hrule \\
  & $\Big|$ &  $2$ & $2$ & $0$ & $3$ & $4$
  \end{table} 

Also ist $p(1)=4$.

  \tab{
  \lang{de}{Lösung b)}
  }
Wir verwenden das Horner-Schema und übertragen die Koeffizienten von $p$:
\begin{table}[\class{items}]
 & $\Big|$ &  $3$ & $4$ &$0$ & $3$ & $0$ & $5$ \\
$2$ & $\Big|$ &  & $6$ & $20$ & $40$ & $86$ & $172$ \\
\colspan{8} \hrule \\
  & $\Big|$ &  $3$ & $10$ & $20$ & $43$ & $86$ & $177$
  \end{table} 

Also ist $p(2)=177$.

\end{tabs*}


\end{content}