\documentclass{mumie.element.exercise}
%$Id$
\begin{metainfo}
  \name{
    \lang{de}{Ü01: Grad und Leitkoeffizient}
    \lang{en}{Exercise 1}
  }
  \begin{description} 
 This work is licensed under the Creative Commons License Attribution 4.0 International (CC-BY 4.0)   
 https://creativecommons.org/licenses/by/4.0/legalcode 

    \lang{de}{}
    \lang{en}{}
  \end{description}
  \begin{components}
  \end{components}
  \begin{links}
  \end{links}
  \creategeneric
\end{metainfo}
\begin{content}

\title{
\lang{de}{Ü01: Grad und Leitkoeffizient}
\lang{en}{Exercise 1}
}

\begin{block}[annotation]
	Im Ticket-System: \href{http://team.mumie.net/issues/9083}{Ticket 9083}
\end{block}

\lang{de}{Bestimmen Sie den Grad und den Leitkoeffizienten der folgenden Polynome:}
\begin{table}[\class{items}]
  \nowrap{a) $p(x)=x^{2}+2x+3$} \\
  \nowrap{b) $p(x)=5x^{4}+3x^{2}+2$} \\
   \nowrap{c) $p(x)=(2x^{3}+1)\cdot (3x+2)$}
 
\end{table}

\begin{tabs*}[\initialtab{0}\class{exercise}]
  \tab{
  \lang{de}{Lösung a)}
  \lang{en}{Solution a)}}
  Das Polynom hat den Grad 2 und $a_{2}=1$ ist der Leitkoeffizient.

  \tab{
  \lang{de}{Lösung b)}
  \lang{en}{Solution b)}
  }
  Das Polynom hat den Grad 4 und $a_{4}=5$ ist der Leitkoeffizient.

  \tab{
  \lang{de}{Lösung c)}
  \lang{en}{Solution c)}
  }
  \begin{incremental}[\initialsteps{1}]
    \step \lang{de}{In diesem Fall ist $p$ nicht in Normalform gegeben. Wir multiplizieren daher zunächst aus.}
    \step \lang{de}{Wir erhalten:
\[p(x)=2x^{3}\cdot (3x+2 )+1 \cdot (3x+2 )=6x^{4}+4x^{3}+3x+2\,.\] }
    \step \lang{de}{Also hat $p$ den Grad 4 und der Leitkoeffizient ist $a_{4}=6$.}
    % ...
    
  \end{incremental}

  %... other tabs

\end{tabs*}

\end{content}