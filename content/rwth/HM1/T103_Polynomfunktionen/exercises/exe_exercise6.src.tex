\documentclass{mumie.element.exercise}
%$Id$
\begin{metainfo}
  \name{
    \lang{de}{Ü06: rationale Funktion}
    \lang{en}{Exercise 6}
  }
  \begin{description} 
 This work is licensed under the Creative Commons License Attribution 4.0 International (CC-BY 4.0)   
 https://creativecommons.org/licenses/by/4.0/legalcode 

    \lang{de}{Hier die Beschreibung}
    \lang{en}{}
  \end{description}
  \begin{components}
  \end{components}
  \begin{links}
  \end{links}
  \creategeneric
\end{metainfo}
\begin{content}

\title{
\lang{de}{Ü06: rationale Funktion}
\lang{en}{Exercise 6}
}
\begin{block}[annotation]
	Im Ticket-System: \href{http://team.mumie.net/issues/9279}{Ticket 9279}
\end{block}


\begin{table}[\class{items}]
  a) Bestimmen Sie die Pole der gebrochen rationalen Funktion
\[r(x)=\frac{x^{3}+x^{2}+x}{x^{2}-6x+9}.\]\\
  b) Gegeben sei die gebrochen rationale Funktion $r(x)$. Bestimmen Sie die Definitionslücken und Polstellen von $r$. Wie verhält sich $r$ an den Singularitäten?
\[r(x)=\frac{(x-1)^{2}}{x^{2}-1}.\]
 c) Skizzieren Sie den Funktionsgraphen von 
 \[f(x) = \frac{1}{x+2} + \frac{-2}{x-3}. \]
 d) Skizzieren Sie den Funktionsgraphen von
 \[ g(x) = \frac{1}{(x+1)^2} - \frac{1}{x-2}.\]
    
\end{table}

\begin{tabs*}[\initialtab{0}\class{exercise}]
  \tab{
  \lang{de}{Lösung a)}}
  
  \begin{incremental}[\initialsteps{1}]
    \step 
    \lang{de}{ Die Definitionslücken sind die Nullstellen des Nenners von $r$.}  
    \step \lang{de}{Nach der zweiten binomischen Formel gilt 
\[x^{2}-6x+9=(x-3)^{2}\,.\]}
    \step \lang{de}{Faktorisiert man den Zähler ebenfalls, so erhält man beispielsweise mit der $p$-$q$-Formel:
\[x^{3}+x^{2}+x=x(x^{2}+x+1)\,.\]}
	\step \lang{de}{Nach quadratischer Ergänzung sieht erhält man für den zweiten Faktor
\[x^{2}+x+1=(x+1/2)^{2}+3/4\,.\]}
	\step \lang{de}{Also hat der zweite Faktor keine reelle Nullstelle und der Zähler hat als einzige Nullstelle $x=0$. Daher strebt $r(x)$ für $x\to 3$ gegen unendlich und es liegt eine Polstelle bei $x=3$ vor.}
  \end{incremental}

  \tab{
  \lang{de}{Lösung b)}
  }
  \begin{incremental}[\initialsteps{1}]
     \step \lang{de}{Durch Faktorisieren können wir den Nenner von $r$ umschreiben als 
\[x^{2}-1=(x+1)(x-1)\]
gemäß der dritten binomischen Formel.}
     \step \lang{de}{Die Definitionslücken von $r$ sind daher $1$ und $-1$.}
     \step \lang{de}{Wir können die Funktion $r$ somit vereinfachend durch den Ausdruck 
\[r(x)=\frac{x-1}{x+1}\]
beschreiben für alle $x\in \R\backslash\{-1,1\}$.}
	\step \lang{de}{Für $x\to -1$ wird $|r(x)|$ beliebig groß. Daher weist der Graph von $r$ an dieser Stelle einen Pol auf.}
	\step \lang{de}{An der zweiten Singularität $x=1$ sieht das Verhalten dagegen anders aus. Für $x\to 1$ strebt $r(x)$ gegen den Wert 0.}
	
	\lang{de}{Man sieht daran, dass der Graph einer gebrochen rationalen Funktion an ihren Singularitäten nicht notwendigerweise gegen unendlich strebt.}
    \lang{de}{Da der Graph in der Nähe von $1$ gegen Null strebt, liegt dort kein Pol vor.}
     % ...
  \end{incremental}

 \tab{\lang{de}{Lösungsvideo c) und d)}}	
    \youtubevideo[500][300]{63-HxIEhwx8}\\



\end{tabs*}

\end{content}