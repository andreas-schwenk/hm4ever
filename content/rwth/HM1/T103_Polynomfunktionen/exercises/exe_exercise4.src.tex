\documentclass{mumie.element.exercise}
%$Id$
\begin{metainfo}
  \name{
    \lang{de}{Ü04: Horner-Schema}
    \lang{en}{Exercise 4}
  }
  \begin{description} 
 This work is licensed under the Creative Commons License Attribution 4.0 International (CC-BY 4.0)   
 https://creativecommons.org/licenses/by/4.0/legalcode 

    \lang{de}{Hier die Beschreibung}
    \lang{en}{}
  \end{description}
  \begin{components}
  \end{components}
  \begin{links}
  \end{links}
  \creategeneric
\end{metainfo}
\begin{content}
\title{
	\lang{de}{Ü04: Horner-Schema}
	\lang{en}{Exercise 4} 														
}

\begin{block}[annotation]
	
\end{block}
\begin{block}[annotation]
	Im Ticket-System: \href{http://team.mumie.net/issues/9086}{Ticket 9086}
\end{block}

\begin{table}[\class{items}]
 	a) Es sei $f(x)=x^{3}+x^{2}-5x-2$. Bestimmen Sie mit Hilfe des Horner-Schemas ein Polynom $g(x)$ und eine reelle Zahl $a$ mit der Eigenschaft $f(x)=(x-2)g(x)+a$.\\
   	b) Es sei  $f(x)=x^{4}-3x^{3}-9x^{2}-10x+25$. Bestimmen Sie mit Hilfe des Horner-Schemas ein Polynom $g(x)$ und eine reelle Zahl $a$ mit der Eigenschaft $f(x)=(x-5)g(x)+a$.
\end{table}
\begin{tabs*}[\initialtab{0}\class{exercise}]
  \tab{
  \lang{de}{Antwort}
  }
\begin{table}[\class{items}]

    \nowrap{a) $a=0$ und $g(x)=x^2+3x+1$} & \nowrap{b) $a=0$ und $g(x)=x^3+2x^2+x-5$}
  \end{table}
  
  \tab{
  \lang{de}{Lösung a)}
  }
\begin{incremental}[\initialsteps{1}] 
\step Wir verwenden das Horner-Schema:
\begin{table}[\class{items}]
  & $\Big|$ & $1$ & $1$ & $-5$ & $-2$ \\
2 & $\Big|$ &  & $2$ & $6$ & $2$ \\
\colspan{6} \hrule \\
  & $\Big|$ & $1$ & $3$ & $1$ & $0$
  \end{table}
  
\step Also ist $2$ eine Nullstelle von $f$. Daher ist $a=0$ und wir lesen die Koeffizienten von $g$ aus der letzten Zeile des Horner Schemas ab, so dass
$g(x)=x^{2}+3x+1$.
\end{incremental}

  \tab{
  \lang{de}{Lösung b)}
  }
  
  \begin{incremental}[\initialsteps{1}]
\step Wir verwenden das Horner-Schema:
\begin{table}[\class{items}]
  & $\Big|$ & $1$ & $-3$ &$-9$ & $-10$ & $25$ \\
5 & $\Big|$ &  & $5$ & $10$ & $5$ & $-25$ \\
\colspan{7} \hrule \\
  & $\Big|$ & $1$ & $2$ & $1$ & $-5$ & $0$
  \end{table}
  
\step Also ist $5$ eine Nullstelle von $f$. Daher ist $a=0$ und wir lesen die Koeffizienten von $g$ aus der letzten Zeile des Horner Schemas ab, so dass
$g(x)=x^{3}+2x^{2}+x-5$.
  \end{incremental}
  \end{tabs*}



\end{content}