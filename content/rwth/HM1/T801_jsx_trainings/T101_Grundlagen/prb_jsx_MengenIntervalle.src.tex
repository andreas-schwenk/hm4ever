\documentclass{mumie.problem.gwtmathlet}
%$Id$
\begin{metainfo}
  \name{
    \lang{de}{MengenIntervalle}
    \lang{en}{}
  }
  \begin{description} 
 This work is licensed under the Creative Commons License Attribution 4.0 International (CC-BY 4.0)   
 https://creativecommons.org/licenses/by/4.0/legalcode 

    \lang{de}{Beschreibung}
    \lang{en}{}
  \end{description}
  \corrector{system/problem/GenericCorrector.meta.xml}
  \begin{components}
    \component{js_lib}{system/problem/GenericMathlet.meta.xml}{mathlet}
  \end{components}    
  \begin{links}
  \end{links}
  \creategeneric
\end{metainfo}
%  
\begin{content}
\begin{block}[annotation]
	Im Ticket-System: \href{https://team.mumie.net/issues/32293}{Ticket 32293}
\end{block}
\begin{block}[annotation]
Copy of : /home/mumie/checkin/content/playground/HM4mint-Testaufgaben/T101neu_Elementare_Rechengrundlagen/prb_jsx_MengenIntervalle.src.tex
\end{block}

\begin{block}[annotation]
(B. Guyet)\\
Copy of : /home/mumie/checkin/content/playground/BarbaraGuyet/JSX-Aufgaben/prb_jsx_MengenIntervalle.src.tex
\end{block}


\begin{block}[annotation]
Intervalle: Lösung grafisch ableiten, Menge umschreiben als Intervall
\end{block}

\usepackage{mumie.genericproblem}
\lang{de}{
\title{Intervallschreibweise von Mengen}
}
\begin{visualizationwrapper}


\begin{genericJSXVisualization}
	\begin{variables}
 
        \point[editable]{G1}{real}{0,8}   % blauer Punkt obere Grenze
        \point[editable]{G2}{real}{0,-8}  % blauer Punkt untere Grenze

        \function{g1}{real}{G1[y]}
        \function{g2}{real}{G2[y]}
        \function{h}{real}{0}             % x-Achse

        \pointOnCurve[editable]{P}{real}{var(h)}{-1} % roter Punkt für linke Grenze (bewegt sich auf x-Achse)
    		\pointOnCurve[editable]{Q}{real}{var(h)}{2}  % roter Punkt für rechte Grenze (bewegt sich auf x-Achse)
      
        \point{A}{real}{P[x],1}
        \point{B}{real}{Q[x],1}
        \line{l1}{real}{var(A), var(P)}
        \line{l2}{real}{var(B), var(Q)}

        \number{l}{real}{G2[y]}  % untere Grenze (blau)
        \number{r}{real}{G1[y]}  % obere Grenze(blau)
        \number{lx}{real}{P[x]}  % linke Grenze (rot)
        \number{rx}{real}{Q[x]}  % rechte Grenze(rot)
        \segment{s}{real}{P,Q}
        \problem{f}{real}   % Funktion f ist global auf problem-Ebene definiert, 
                            % nicht auf question-Ebene      
	\end{variables}

\label{P}{lGr}
\label{Q}{rGr}
\label{f}{f}
\color{f}{green}

\color{g1}{BLUE}
\color{G1}{BLUE}
\color{g2}{BLUE}
\color{G2}{BLUE}
\label{g1}{oGr}
\label{g2}{uGr}

\color[0.7]{l1}{blue}
\color[0.7]{l2}{blue}
\color[0.7]{g1}{blue}
\color[0.7]{g2}{blue}
\color{P}{red}
\color{Q}{red}
\color{s}{red}
\answer{g2}{1,1}  % wird in Q1, Antwort 1 geprüft
\answer{g1}{1,2}  % wird in Q1, Antwort 2 geprüft
\answer{lx}{1,3}  % wird in Q1, Antwort 3 geprüft
\answer{rx}{1,4}  % wird in Q1, Antwort 4 geprüft



	\begin{canvas}
        \snapToGrid{0.05,0.1}
        \plotSize{500,500}
        \plotLeft{-10}
        \plotRight{10}
        \plot[coordinateSystem,showPointCoords]{G1,G2,P,Q,f,g1,g2,l1,l2,s}
	\end{canvas}

\end{genericJSXVisualization}

\end{visualizationwrapper}


\begin{problem}

    \begin{variables}
          \randint{c}{1}{3}
          % Definition aller Variablen außerhalb der Switch-Umgebung !
          % (wiki: "Every variable in a switch environment must have a default definition")
          % c=1
            \randint{r1}{1}{10}
            \randint{l1}{-10}{-1}
            \randint[Z]{d1}{-2}{2}
            \randint[Z]{m}{2}{5}
          % c=2
            \randint{zr}{4}{12}
            \randint{nr}{1}{4}
            \randint{l2}{-7}{0}
            \randint[Z]{d2}{-2}{2}
          % c=3
            \randint[Z]{q}{0}{3}
            \randint[Z]{p}{-3}{0}
            \randint[Z]{d3}{-5}{5}
          %  
           \begin{switch}
             \begin{case}{c=1}      %f=m*x
                  \function[calculate]{r}{r1}
                  \function[calculate]{l}{l1}
                  \function[calculate]{lx}{l/m+d1}
                  \function[calculate]{rx}{r/m+d1}                  
                  \function{f}{m(x-d1)} 
              \end{case}
              \begin{case}{c=2}    %f=|x|
%                  \function[calculate]{r}{(zr/nr)^2}
                  \function[calculate]{r}{zr/nr}
                  \function[calculate]{l}{l2}
                  \function[calculate]{rx}{r+d2}
                  \function[calculate]{lx}{-r+d2}                  
                  \function{f}{|x-d2|}
              \end{case}
              \begin{default}    %f=x^3
                  \function[normalize]{r}{q*q*q}
                  \function[normalize]{l}{p*p*p} 
                  \function[calculate]{rx}{q+d3}
                  \function[calculate]{lx}{p+d3}
                  \function{f}{(x-d3)^3}
              \end{default}
            \end{switch}
     \end{variables}
%
 \randomquestionpool{1}{1}
 \randomquestionpool{2}{3}
 \showQuestionLabels{no}
%
% Q1 grafische Bestimmung der Intervallgrenzen 
%
  \begin{question}

    \field{rational}
    \correctorprecision{1}

     \text{Beschreiben Sie grafisch den Bereich aller Werte $x\in \R$, für
           die $f(x)=\var{f}$ zwischen $\var{l}$ und $\var{r}$ liegt.\\
           Verschieben Sie hierzu die blauen Begrenzungslinien mithilfe der 
           Punkte auf den Koordinatenachsen so, dass der gesuchte Bereich in
           rot dargestellt ist.
          } 
      \explanation[NOT[correct(ans_1)] OR NOT[correct(ans_2)]]{Der vorgegebene Bereich für $f(x)$
                                 wird durch die blauen Begrenzungslinien noch nicht korrekt eingegrenzt.}
      \explanation[correct(ans_1) AND correct(ans_2) AND NOT[correct(ans_3)] AND correct(ans_4)]{Die untere (bzw. linke) Grenze für $x$ ist in der Grafik nicht korrekt dargestellt.}
      \explanation[correct(ans_1) AND correct(ans_2) AND NOT[correct(ans_4)] AND correct(ans_3)]{Die obere (bzw. rechte) Grenze für $x$ ist in der Grafik nicht korrekt gesetzt.}
      \explanation[correct(ans_1) AND correct(ans_2) AND NOT[correct(ans_4)] AND NOT[correct(ans_3)]]{Die Grenzen für $x$ (rote Punkte) sind in der Grafik nicht korrekt gesetzt.}
%          
%\debug[c,lx,rx,l,r] 
%
          \begin{answer}
            \type{graphics.function}
            \solution{l}
            \checkAsFunction[1E-1]{}{-1}{1}{10}
          \end{answer}
          %
          \begin{answer}
            \type{graphics.function}
            \solution{r}
            \checkAsFunction[1E-1]{}{-1}{1}{10}
          \end{answer}
%
          \begin{answer}
            \type{graphics.function}
            \solution{lx}
            \checkAsFunction[1E-1]{}{-1}{1}{10}
          \end{answer}
          %
          \begin{answer}
            \type{graphics.function}
            \solution{rx}
            \checkAsFunction[1E-1]{}{-1}{1}{10}
          \end{answer}
%
  \end{question}
%
% Q2 Umschreibung der Lösungsmenge in Intervallform (geschlossenes Intervall)
%
  \begin{question}

    \field{rational}
    \correctorprecision{1}
    \begin{variables}
           \begin{switch}
             \begin{case}{c=2}      %f=|x|
                  \string{cond}{$\var{f} \le \var{r}$}
              \end{case}
              \begin{default}    %f=mx+b oder f=x^3
                  \string{cond}{$\var{l} \le \var{f} \le \var{r}$}
              \end{default}
            \end{switch}
     \end{variables}
%

     \text{Leiten Sie nun aus der Grafik die Intervalldarstellung für die Lösungsmenge ab 
           und beachten Sie dabei die Vorgaben für die Randwerte.
          }      
%          
%\debug[c,cond,lx,rx]   
% 
          \begin{answer}
          \type{input.interval}
              \text{$\mathbb{L}=\{x \in \R \vert$ \var{cond} $\}=$}          
              %\allowIntervalUnionsForInput
              \checkAsFunction[1E-1]{}{-1}{1}{10}
              \solution{[lx;rx]}
              \explanation{Die Intervallgrenzen sind nicht korrekt. Haben Sie die Randwerte mit berücksichtigt?}
          \end{answer}
%
  \end{question}

%
% Q3 Umschreibung der Lösungsmenge in Intervallform (offenes Intervall)
%
  \begin{question}

    \field{rational}
    \correctorprecision{1}
    \begin{variables}
           \begin{switch}
             \begin{case}{c=2}      %f=|x|
                  \string{cond}{$\var{f} < \var{r}$}
              \end{case}
              \begin{default}    %f=mx+b oder f=x^3
                  \string{cond}{$\var{l} < \var{f} < \var{r}$}
              \end{default}
            \end{switch}
     \end{variables}
%

     \text{Leiten Sie nun aus der Grafik die Intervalldarstellung für die Lösungsmenge unter
           Berücksichtigung der Vorgaben für die Randwerte ab:
          }      
%          
%\debug[c,cond,lx,rx]   
% 
          \begin{answer}
          \type{input.interval}
              \text{$\mathbb{L}=\{x \in \R \vert$ \var{cond} $\}=$}          
              %\allowIntervalUnionsForInput
              checkAsFunction[1E-1]{}{-1}{1}{10}
              \solution{(lx;rx)}
              \explanation{Die Intervallgrenzen sind nicht korrekt. Sollten die Randwerte enthalten Sein?}
          \end{answer}
%
  \end{question}


\end{problem}

\embedmathlet{mathlet}

\end{content}
