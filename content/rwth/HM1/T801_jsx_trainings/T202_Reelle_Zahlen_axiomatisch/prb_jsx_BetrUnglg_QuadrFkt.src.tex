\documentclass{mumie.problem.gwtmathlet}
%$Id$
\begin{metainfo}
  \name{
    \lang{de}{Name}
    \lang{en}{}
  }
  \begin{description} 
 This work is licensed under the Creative Commons License Attribution 4.0 International (CC-BY 4.0)   
 https://creativecommons.org/licenses/by/4.0/legalcode 

    \lang{de}{Beschreibung}
    \lang{en}{}
  \end{description}
  \corrector{system/problem/GenericCorrector.meta.xml}
  \begin{components}
    \component{js_lib}{system/problem/GenericMathlet.meta.xml}{mathlet}
  \end{components}    
  \begin{links}
  \end{links}
  \creategeneric
\end{metainfo}
%  
\begin{content}
\begin{block}[annotation]
	Im Ticket-System: \href{https://team.mumie.net/issues/32493}{Ticket 32493}
\end{block}
\begin{block}[annotation]
Copy of : /home/mumie/checkin/content/playground/HM4mint-Testaufgaben/T202_Reelle_Zahlen_axiomatisch/prb_jsx_BetrUnglg_QuadrFkt.src.tex
\end{block}

\begin{block}[annotation]
(B. Guyet)\\
Copy of : /home/mumie/checkin/content/playground/BarbaraGuyet/JSX-Aufgaben/prb_jsx_BetrUnglg_QuadrFkt.src.tex
\end{block}

\begin{block}[annotation]
Grapfische Lösung einer Betragsungleichung mit einfacher quadratischer Funktion im Betrag.
\end{block}

\usepackage{mumie.genericproblem}
\lang{de}{
\title{Betragsungleichung grafisch lösen}
}
\begin{visualizationwrapper}


\begin{genericJSXVisualization}
	\begin{variables}
    		\point[editable]{S}{real}{2,3}
        \function{f0}{real}{abs((x-S[x])^2+S[y])}
        \function{f}{real}{(x-S[x])^2+S[y]}
        \point[editable]{P}{real}{0,5}
        \function{g}{real}{P[y]}
        \function{xaxis}{real}{0*x}
        \pointOnCurve{W}{real}{xaxis}{-4}
        \parametricFunction{v}{real}{W[x]+0*t,t,0,22}
	\end{variables}
\color{f0}{RED}
\color[0.7]{f}{RED}
\color{S}{RED}
\label{S}{S}
\color[0.7]{g}{GREEN}
\color{P}{GREEN}
%\label{P}{P}
\label{g}{g}
\color{W}{GREEN}
\color[0.7]{v}{GREEN}

	\begin{canvas}
        \snapToGrid{0.1,0.1}
        \plotSize{500,500}
        \plotLeft{-25}
        \plotRight{25}
        \plot[coordinateSystem,showPointCoords]{S,P,f0,f,g,W,v}
	\end{canvas}

\end{genericJSXVisualization}

\end{visualizationwrapper}


\begin{problem}

% Betragsungleichung (neu)
% 
  \randomquestionpool{1}{4} % (hier zunächst nur die Fälle mit |...| <= od >= b, 
                            % ggf. noch duplizieren für |...| < od > b
                            % oder mit "case" arbeiten)
%
%%%%%%%%%%%%%%%%%%%%%%%%%%%%%%%%%%%%%%%%%%%%%%%%%
%
% Q1  |(x+c)^2-a|<=b und a<b => 1 Intervall
%
  \begin{question}
  
    \type{input.interval}
    \field{real}
    \precision{2}

    \text{Bestimmen Sie grafisch alle Lösungen $x \in \R$ der Ungleichung     
        $\;  \abs{(\var{term})^2 -\var{a}} \leq \var{b}. $ \\\\
        
        Verschieben Sie hierzu den roten Punkt $S$ in den Scheitelpunkt 
        des Funktionsgraphen $f(x)=(\var{term})^2 -\var{a}$. Durch Spiegelung
        an der $x-$Achse entsteht dabei automatisch $\abs{f(x)}=\abs{(\var{term})^2 -\var{a}}$. \\
        
        \textit{Tipp: Verwenden Sie die grünen Hilfslinien und Punkte zur Bestimmung und zum Ablesen
                der Intervallgrenzen für die Lösungsmenge.}
        }
%
        \explanation{Prüfen Sie die Intervallgrenzen und beachten Sie, ob die gesuchten
         Funktionswerte von $\abs{f(x)}$ oberhalb oder unterhalb der Begrenzungsgeraden $g$ liegen sollen.}
%        
    \begin{variables}
    %Für den Fall a<b
      \drawFromSet{qp}{4,9,16,25} % qp=(a+b)^2
      \randint{a}{1}{10}
      \randint[Z]{c}{-10}{10}      
      \function[normalize]{term}{x+c}
      \function[calculate,2]{b}{qp-a}
      \randadjustIf{a}{a>=b OR b<=0}
    %Intervallgrenzen
      \function[normalize]{l1}{-sqrt(a+b)-c}
      \function[normalize]{r1}{sqrt(a+b)-c}

    \end{variables}
%
%\debug[qp,c,a,b,l1,r1]
%
    \begin{answer}
          \text{$\mathbb{L}=$}    
          \allowIntervalUnionsForInput
          \solution{[l1;r1]}
    \end{answer}
%
  \end{question}

%
% Q2  |(x+c)^2-a|>=b und a<b => 2 Intervalle
%
  \begin{question}
  
    \type{input.interval}
    \field{real}
    \precision{2}

    \text{Bestimmen Sie grafisch alle Lösungen $x \in \R$ der Ungleichung     
        $\;  \abs{(\var{term})^2 -\var{a}} \leq \var{b}. $ \\\\

        Verschieben Sie hierzu den roten Punkt $S$ in den Scheitelpunkt 
        des Funktionsgraphen $f(x)=(\var{term})^2 -\var{a}$. Durch Spiegelung
        an der $x-$Achse entsteht dabei automatisch $\abs{f(x)}=\abs{(\var{term})^2 -\var{a}}$. \\
        
        \textit{Tipp: Verwenden Sie die grünen Hilfslinien und Punkte zur Bestimmung und zum Ablesen
                der Intervallgrenzen für die Lösungsmenge.}
        }
%
        \explanation{Prüfen Sie die Intervallgrenzen und beachten Sie, ob die gesuchten
         Funktionswerte von $\abs{f(x)}$ oberhalb oder unterhalb der Begrenzungsgeraden $g$ liegen sollen.}
%        
    \begin{variables}
    %Für den Fall a<b
      \drawFromSet{qp}{4,9,16,25} % qp=(a+b)
      \randint{a}{1}{10}
      \randint[Z]{c}{-10}{10}
      \function[normalize]{term}{x+c}      
      \function[calculate,2]{b}{qp-a}
      \randadjustIf{a}{a>=b OR b<=0}
    %Intervallgrenzen
      \function[normalize]{l1}{-sqrt(a+b)-c}
      \function[normalize]{r1}{sqrt(a+b)-c}

    \end{variables}
%
%\debug[qp,qm,a,b,l1,r1]
%
    \begin{answer}
          \text{$\mathbb{L}=$}    
          \allowIntervalUnionsForInput
          \solution{(-infty;l1],[r1;infty)}
    \end{answer}
%
  \end{question}

%
% Q3  |(x+c)^2-a|<=b und a>b => 2 Intervalle
%
  \begin{question}
  
    \type{input.interval}
    \field{real}
    \precision{2}

    \text{Bestimmen Sie grafisch alle Lösungen $x \in \R$ der Ungleichung     
        $\;  \abs{(\var{term})^2 -\var{a}} \leq \var{b}. $ \\\\

        Verschieben Sie hierzu den roten Punkt $S$ in den Scheitelpunkt 
        des Funktionsgraphen $f(x)=(\var{term})^2 -\var{a}$. Durch Spiegelung
        an der $x-$Achse entsteht dabei automatisch $\abs{f(x)}=\abs{(\var{term})^2 -\var{a}}$. \\
        
        \textit{Tipp: Verwenden Sie die grünen Hilfslinien und Punkte zur Bestimmung und zum Ablesen
                der Intervallgrenzen für die Lösungsmenge.}

        }
%
        \explanation{Prüfen Sie die Intervallgrenzen und beachten Sie, ob die gesuchten
         Funktionswerte von $\abs{f(x)}$ oberhalb oder unterhalb der Begrenzungsgeraden $g$ liegen sollen.}
%        
    \begin{variables}
    %Für den Fall a>b    
      \drawFromSet{qp}{4,9,16,25} % qp=(a+b)
      \drawFromSet{qm}{1,4,9,16}  % qm=(a-b)
      \randadjustIf{qp,qm}{qp<=qm}
      \randint[Z]{c}{-10}{10}
      \function[normalize]{term}{x+c}            
      \function[calculate,2]{a}{(qp+qm)/2}
      \function[calculate,2]{b}{(qp-qm)/2}
    %Intervallgrenzen
      \function[normalize]{l1}{-sqrt(a+b)-c}
      \function[normalize]{r1}{-sqrt(a-b)-c}

      \function[normalize]{l2}{sqrt(a-b)-c}
      \function[normalize]{r2}{sqrt(a+b)-c}
    \end{variables}
%
%\debug[qp,qm,a,b,l1,r1,l2,r2]
%
    \begin{answer}
          \text{$\mathbb{L}=$}    
          \allowIntervalUnionsForInput
          \solution{[l1;r1],[l2;r2]}
    \end{answer}
%
  \end{question}

%
% Q4  |(x+c)^2-a|>=b und a>b => 3 Intervalle
%
  \begin{question}
  
    \type{input.interval}
    \field{real}
    \precision{2}

    \text{Bestimmen Sie grafisch alle Lösungen $x \in \R$ der Ungleichung     
        $\;  \abs{(\var{term})^2 -\var{a}} \geq \var{b}. $ \\\\

        Verschieben Sie hierzu den roten Punkt $S$ in den Scheitelpunkt 
        des Funktionsgraphen $f(x)=(\var{term})^2 -\var{a}$. Durch Spiegelung
        an der $x-$Achse entsteht dabei automatisch $\abs{f(x)}=\abs{(\var{term})^2 -\var{a}}$. \\
        
        \textit{Tipp: Verwenden Sie die grünen Hilfslinien und Punkte zur Bestimmung und zum Ablesen
                der Intervallgrenzen für die Lösungsmenge.}
        }
%
        \explanation{Prüfen Sie die Intervallgrenzen und beachten Sie, ob die gesuchten
         Funktionswerte von $\abs{f(x)}$ oberhalb oder unterhalb der Begrenzungsgeraden $g$ liegen sollen.}
%        
    \begin{variables}
    %Für den Fall a>b    
      \drawFromSet{qp}{4,9,16,25} % qp=(a+b)
      \drawFromSet{qm}{1,4,9,16}  % qm=(a-b)
      \randadjustIf{qp,qm}{qp<=qm}
      \randint[Z]{c}{-10}{10}
      \function[normalize]{term}{x+c}                  
      \function[calculate,2]{a}{(qp+qm)/2}
      \function[calculate,2]{b}{(qp-qm)/2}
    %Intervallgrenzen
      \function[normalize]{r1}{-sqrt(a+b)-c}
      \function[normalize]{l2}{-sqrt(a-b)-c}
      \function[normalize]{r2}{sqrt(a-b)-c}
      \function[normalize]{l3}{sqrt(a+b)-c}

    \end{variables}
%
%\debug[qp,qm,a,b,r1,l2,r2, l3]
%
    \begin{answer}
          \type{input.interval}
          \text{$\mathbb{L}=$}    
          \allowIntervalUnionsForInput
          \solution{(-infty;r1],[l2;r2],[l3;infty)}
    \end{answer}
%
  \end{question}

  
\end{problem}

\embedmathlet{mathlet}

\end{content}
