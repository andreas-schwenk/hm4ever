\documentclass{mumie.problem.gwtmathlet}
%$Id$
\begin{metainfo}
  \name{
    \lang{en}{Problem 3.2}
    \lang{de}{Quadratische Funktionen: Scheitelform bestimmen und Parabel anpassen}
  }
  \begin{description} 
 This work is licensed under the Creative Commons License Attribution 4.0 International (CC-BY 4.0)   
 https://creativecommons.org/licenses/by/4.0/legalcode 

    \lang{en}{...}
    \lang{de}{...}
  \end{description}
  \corrector{system/problem/GenericCorrector.meta.xml}
  \begin{components}
    \component{js_lib}{system/problem/GenericMathlet.meta.xml}{gwtmathlet}
  \end{components}
  \begin{links}
  \end{links}
  \creategeneric
  \begin{taxonomy}
        \difficulty{3}
        \usage{FH Aachen, Mathematische Grundlagen}
        \objectives{analyze,understand}
        \topic{analysis1/functions/graph_of_function}
  \end{taxonomy}
\end{metainfo}
\begin{content}
\begin{block}[annotation]
	Im Ticket-System: \href{https://team.mumie.net/issues/32291}{Ticket 32291}
\end{block}
\begin{block}[annotation]
Copy of : /home/mumie/checkin/content/playground/HM4mint-Testaufgaben/T102neu_Einfache_Reelle_Funktionen/prb_jsx_QuadFunk_LFZ.src.tex
\end{block}


\begin{block}[annotation]
(B. Guyet)\\
Abwandlung der Aufgabe: \\
/home/mumie/checkin/content/playground/BarbaraGuyet/JSX-Aufgaben/prb_jsx_QuadFunk_SPF.src.tex
\\
Weiterentwicklung basierend auf Kopien von: \\
/home/mumie/checkin/content/playground/Andreas_Maurischat/grafischeAufgaben/prb_QuadFunk_A3.3.src.tex
\\
/home/mumie/checkin/content/pool/fh_aachen/MGI/Quadratische_Funktionen/prb_QuadFunk_A3.3.src.tex
\end{block}

\begin{block}[annotation]
 Bestimmung der Nullstellen einer quadratischen Funktion nach Viëta; Umformung der Funktion in
 die faktorisierte Form (Linearfaktorzerlegung) und in die Scheitelpunktform; Grafische Bestimmung
 der Lage der Parabel unter Verwendung einer der Nullstellen und des Scheitelpunkts.
\end{block}

\usepackage{mumie.genericproblem}

\begin{visualizationwrapper}


\begin{genericJSXVisualization}
	\begin{variables}
  		\point[editable]{S}{real}{0,0}
  		\point[editable]{N}{real}{1,1}
        \function{f}{real}{(N[y]-S[y])/((N[x]-S[x])*(N[x]-S[x]))*(x-S[x])*(x-S[x])+S[y]}
        \number{sx}{real}{S[x]}
        \number{sy}{real}{S[y]}
        \number{nx}{real}{N[x]}
        \number{ny}{real}{N[y]}
    \end{variables}
\color{f}{RED}
\label{S}{S}
\label{N}{N}
\label{f}{f}
\answer{sx}{3,1}
\answer{sy}{3,2}
\answer{nx}{3,3}
\answer{ny}{3,4}

	\begin{canvas}
        \snapToGrid{0.05,0.05}
        \plotSize{400,400}
        \plotLeft{-5}
        \plotRight{5}
        \plot[coordinateSystem,showPointCoords]{S,N,f}
	\end{canvas}

\end{genericJSXVisualization}

\end{visualizationwrapper}

   \begin{problem} 
%   
    \begin{variables}
	    \randint[Z]{a}{-3}{3}
	    \randint{b}{1}{4}
	    \randint{c}{1}{4}
	    \randint{d}{1}{4}
      \randadjustIf{b,d}{b=d OR b/a=d/c} 
     % Quadr. Funktion f mit Linearfaktorzerlegung f1*f2
      \function{ff}{(c*x+d)*(a*x+b)} 
      \function[expand, normalize]{f}{ff}
     %  Nullstellen von f (es müssen 2 Nullstellen sein, sonst N=S)  
	    \function[calculate]{ns1}{-b/a}
	    \function[calculate]{ns2}{-d/c}
      \function[calculate]{nx0}{ns2}      
      \substitute[calculate]{ny0}{ff}{x}{nx0}     
     % faktorisierte Form
      \function[calculate, normalize]{m}{a*c}     % Steckungsfaktor der Parabel
      \function[normalize]{f1}{x-ns1}
      \function[normalize]{f2}{x-ns2}
	   % Scheitelpunktform der Parabel       
      \function[calculate, normalize]{xs}{-(b*c+a*d)/(2*a*c)}      % x-Koordinate des Scheitelpunkts
      \function[calculate, normalize]{ys}{b*d-(b*c+a*d)^2/(4*a*c)} % y-Koordinate des Scheitelpunkts
      \function[normalize]{f0}{m*(x-xs)^2+ys}  % Scheitelpunktform der Parabel
    \end{variables}
% (a)         
  \begin{question}
     \text{Die quadratische Funktion $f:\, \R \rightarrow \R$, $\;f(x)= \var{f}\,$ 
          hat eine Nullstellen in $\, x_1=\var{ns1}$. \\
          Bestimmen Sie die Linearfaktorzerlegung von $f$ und schreiben 
          Sie $f$ in die faktorisierte Form um:\\
          $f(x)=$ \ansref $\cdot (\var{f1}) \cdot ($ \ansref $)$.\\
          \\
          \textit{Geben alle Zahlen als (gekürzte) Brüche oder ganze Zahlen an.}}
%
    
     \field{rational}
% \debug[m,ff,f1,f2,f0,nx0,ny0,ns1,ns2,xs,ys]
      
     \begin{answer}
       \type{input.number}
       \solution{m}
     \end{answer} 
     \begin{answer}
        \type{input.function}
        \solution{f2}
        \inputAsFunction{x}{ant1}
        \checkStringsForRelation{count(x,ant1)=1 AND equal(ant1,f2)}
        \checkAsFunction{x}{-1}{1}{10}
       \end{answer} 
       \explanation[NOT[equal(ans_1,m)] AND [equal(ans_2,f2)]]{Der Vorfaktor ist nicht korrekt.}
       \explanation[NOT[equal(ans_1,m)] AND NOT[equal(ans_2,f2)]]{Klammern Sie zuerst den richtigen 
                   Vorfaktor aus und verwenden Sie den Satz von Viëta zur Berechnung der zweiten 
                   Nullstelle von $f$.}
       \explanation[[equal(ans_1,m)] AND NOT[equal(ans_2,f2)]]{Verwenden Sie den Satz von Viëta 
                   zur Berechnung der zweiten Nullstelle von $f$.}
     \end{question}         
% (b)         
     \begin{question}
     \text{Bringen Sie den Funktionsterm von $f$ in Scheitelpunktform:\\
     $f(x)=$ \ansref.
     \\
     \textit{Geben alle Zahlen als (gekürzte) Brüche oder ganze Zahlen an.}}
     
     \field{rational}
     
       \begin{answer}
        \type{input.function}
        \solution{f0}
        \inputAsFunction{x}{ant1}
        \checkStringsForRelation{count(x,ant1)=1 AND equal(ant1,f0)}
%        \checkAsFunction{x}{-1}{1}{10}
        \checkFuncForZero[5e-1]{(ant1-f0)}{-1}{1}{100}
       \end{answer} 
       \explanation[count(x,ans_1)!=1]{Das ist keine Scheitelpunktform.}
       \explanation[count(x,ans_1)=1]{Leider falsch. Klammern Sie zuerst den Vorfaktor $\var{m}$ aus und 
                   machen Sie dann eine quadratische Ergänzung.}
     \end{question}
% (c)      
     \begin{question}
       \begin{variables}
        \earlierAnswer{f}{2,1}
       \end{variables}          
       \precision{1}
          \type{input.generic}
          \field{real}
               \text{Verschieben Sie anschließend in der Grafik den Punkt $S$ in den 
               zuvor berechneten Scheitelpunkt der Parabel und den Punkt $N$ in eine
               der beiden Nullstellen, so dass der Funktionsgraph der Funktion $f(x)$ 
               entsteht.}        
     \begin{answer}
      \type{graphics.function}
      \solution{xs}
      \inputAsFunction{}{xsx}
     \end{answer}          
     \begin{answer}
      \type{graphics.function}
      \solution{ys}
      \inputAsFunction{}{ysy}
      \checkFuncForZero[1e-1]{f[xsx]-ysy}{1}{2}{2}
     \end{answer}                    
     \begin{answer}
      \type{graphics.function}
      \solution{nx0}
      \inputAsFunction{x}{nx}
      \checkFuncForZero[2e-1]{(nx-ns1)*(nx-ns2)}{-1}{1}{100}
     \end{answer}     
     \begin{answer}
      \type{graphics.function}
      \solution{ny0}
      \inputAsFunction{}{ny}
      \checkFuncForZero[1e-1]{f[nx]-ny}{1}{2}{2}
     \end{answer}
     \explanation[NOT[equal(ans_1,xs)] AND NOT[equal(ans_2,ys)]]{Der Scheitelpunkt ist falsch.}
     \explanation[NOT[equal(ans_1,xs)] AND [equal(ans_2,ys)]]{Der Scheitelpunkt hat die richtige $y$-Koordinate, 
     aber nicht die richtige $x$-Koordinate.}
     \explanation[NOT[equal(ans_2,ys)] AND [equal(ans_1,xs)]]{Der Scheitelpunkt hat die richtige $x$-Koordinate, 
     aber nicht die richtige $y$-Koordinate.}
     \explanation[[equal(ans_1,xs)] AND [equal(ans_2,ys)]]{Der Scheitelpunkt ist richtig, aber der Punkt $N$ 
     stimmt noch nicht.}
   \end{question}

   
 \end{problem}

\embedmathlet{gwtmathlet}

\end{content}
