\documentclass{mumie.problem.gwtmathlet}
%$Id$
\begin{metainfo}
  \name{
    \lang{en}{Problem 3.2}
    \lang{de}{Quadratische Funktionen: Scheitelform bestimmen und Parabel anpassen}
  }
  \begin{description} 
 This work is licensed under the Creative Commons License Attribution 4.0 International (CC-BY 4.0)   
 https://creativecommons.org/licenses/by/4.0/legalcode 

    \lang{en}{...}
    \lang{de}{...}
  \end{description}
  \corrector{system/problem/GenericCorrector.meta.xml}
  \begin{components}
    \component{js_lib}{system/problem/GenericMathlet.meta.xml}{gwtmathlet}
  \end{components}
  \begin{links}
  \end{links}
  \creategeneric
  \begin{taxonomy}
        \difficulty{3}
        \usage{FH Aachen, Mathematische Grundlagen}
        \objectives{analyze,understand}
        \topic{analysis1/functions/graph_of_function}
  \end{taxonomy}
\end{metainfo}
\begin{content}
\begin{block}[annotation]
	Im Ticket-System: \href{https://team.mumie.net/issues/32294}{Ticket 32294}
\end{block}
\begin{block}[annotation]
Copy of : /home/mumie/checkin/content/playground/HM4mint-Testaufgaben/T102neu_Einfache_Reelle_Funktionen/prb_jsx_QuadFunk_SPF.src.tex
\end{block}


\begin{block}[annotation]
(B. Guyet)\\
Weiterentwicklung basierend auf Kopien von: \\
/home/mumie/checkin/content/playground/Andreas_Maurischat/grafischeAufgaben/prb_QuadFunk_A3.3.src.tex
\\
/home/mumie/checkin/content/pool/fh_aachen/MGI/Quadratische_Funktionen/prb_QuadFunk_A3.3.src.tex
\end{block}

\begin{block}[annotation]
 Bestimmung einiger Parabeleigenschaften und verschiedener Punkten auf der Parabel, insbesondere des
 Scheitelpunkts. Grafische Bestimmung der Lage der Parabel unter Verwendung der berechneten Punkte.
\end{block}

\usepackage{mumie.genericproblem}

\begin{visualizationwrapper}


\begin{genericJSXVisualization}
	\begin{variables}
  		\point[editable]{S}{real}{0,0}
		\point[editable]{P}{real}{1,1}
        \function{f}{real}{(P[y]-S[y])/((P[x]-S[x])*(P[x]-S[x]))*(x-S[x])*(x-S[x])+S[y]}
        \number{sx}{real}{S[x]}
        \number{sy}{real}{S[y]}
        \number{px}{real}{P[x]}
        \number{py}{real}{P[y]}
    \end{variables}
\color{f}{RED}
\label{S}{S}
\label{P}{P}
\label{f}{f}
\answer{sx}{3,1}
\answer{sy}{3,2}
\answer{px}{3,3}
\answer{py}{3,4}

	\begin{canvas}
        \snapToGrid{0.05,0.05}
        \plotSize{400,400}
        \plotLeft{-5}
        \plotRight{5}
        \plot[coordinateSystem,showPointCoords]{S,P,f}
	\end{canvas}

\end{genericJSXVisualization}

\end{visualizationwrapper}

   \begin{problem} 
    \begin{variables}
              \randint[Z]{mz}{-10}{10}
              \randint{mn}{1}{5}
              \function[calculate]{m}{mz/mn}  % Steckungsfaktor der Parabel
              \randint{ys}{-3}{3}             % y-Koordinate des Scheitelpunkts
              \randint{xs}{-3}{3}             % x-Koordinate des Scheitelpunkts
              \randadjustIf{mz}{m=1 AND xs=0 AND ys=0}
            % Scheitelpunktform der Parabel
              \function{f0}{m*(x-xs)^2+ys}          % Scheitelpunktform der Parabel
              \function[expand, normalize]{f}{f0}  % expandiert zu ax^2+bx+c                
              \function[calculate]{px0}{xs+1}      
              \substitute[calculate]{py0}{f0}{x}{px0}                     
            % Parameter der Parabel f(x)=ax^2+bx+c
              \function[calculate, normalize]{a}{m}
              \function[calculate, normalize]{b}{-2*m*xs}
              \function[calculate, normalize]{c}{m*xs^2+ys}
            % Anzahl Nullstellen              
              \function[calculate, normalize]{D}{-4*m*ys}      % Diskriminante Mitternachtsformel b^2-4*a*c
              \randint{nst}{0}{2}    % Anzahl Nullstellen
              \begin{switch}  
               \begin{case}{D<0}
                \number{nst}{0}
               \end{case} 
               \begin{case}{D=0}
                \number{nst}{1}
               \end{case} 
               \begin{default}
                \number{nst}{2}
                \end{default}         
              \end{switch}       
         \end{variables}
% (a)         
     \begin{question}
     \text{Gegeben ist die quadratische Funktion $f:\, \R \rightarrow \R$,  $\; f(x)=\var{f}$.\\ 
           Geben Sie an, wie oft der Funktionsgraph von 
           $f$ die $x-$Achse schneidet: \ansref ,\\
           an welcher Stelle er die $y-$Achse schneidet: $\;y_0=$ \ansref \\
           und berechnen Sie den Funktionswert von $f$ an der Stelle $x_1=\var{px0}$.
           \\
           $\; f(\var{px0})=$\ansref 
           }
     \field{rational}
% \debug[a,b,c,nst,xs,ys,px0,py0]
      \begin{answer}
        \type{input.number}
        \solution{nst}
        \explanation{Die Anzahl der Nullstellen kann mithilfe der Diskriminante 
                    aus der Mitternachtsformel % $\,(D=b^2-4ac)\,$ 
                    bestimmt werden. }
      \end{answer}   
      \begin{answer}
        \type{input.function}
        \solution{c}
        \checkAsFunction[5E-1]{}{-1}{1}{10}
        \explanation{Die Schnittstelle mit der $y-$Achse entspricht dem 
                    Funktionswert von $f$ an der Stelle $x_0=0$.}
       \end{answer}
       \begin{answer}
        \type{input.function}
        \solution{py0}
        \checkAsFunction[5E-1]{}{-1}{1}{10}
        \explanation{Das ist nicht $\,f(\var{px0})$. }
       \end{answer}

     \end{question}         
% (b)         
     \begin{question}
     \text{Bringen Sie den Funktionsterm von $f$ in Scheitelpunktform:\\
     $f(x)=$ \ansref.}
     
     \field{rational}
     
       \begin{answer}
        \type{input.function}
        \solution{f0}
        \inputAsFunction{x}{ant1}
        \checkStringsForRelation{count(x,ant1)=1 AND equal(ant1,f0)}
        \checkAsFunction{x}{-1}{1}{10}
       \end{answer} 
       \explanation[count(x,ans_1)!=1]{Das ist keine Scheitelpunktform.}
       \explanation[count(x,ans_1)=1]{Leider falsch. Klammern Sie zuerst den Vorfaktor $\var{m}$ aus und 
                   machen Sie dann eine quadratische Ergänzung.}
     \end{question}
% (c)      
     \begin{question}
       \begin{variables}
        \earlierAnswer{f}{2,1}
       \end{variables}          
       \precision{1}
          \type{input.generic}
          \field{real}
               \text{Verschieben Sie anschließend in der Grafik den Punkt $S$ in den Scheitelpunkt
               der Parabel und den Punkt $P$ in einen beliebigen anderen der zuvor berechneten Punkte,
               so dass der Funktionsgraph der Funktion $f(x)$ entsteht.}        
     \begin{answer}
      \type{graphics.number}
      \solution{xs}
     \end{answer}          
     \begin{answer}
      \type{graphics.number}
      \solution{ys}
     \end{answer}                    
     \begin{answer}
      \type{graphics.function}
      \solution{px0}
      \inputAsFunction{}{px}
     \end{answer}
     \begin{answer}
      \type{graphics.function}
      \solution{py0}
      \inputAsFunction{}{py}
      \checkFuncForZero[1e-1]{f[px]-py}{1}{2}{2}
     \end{answer}
     \explanation[NOT[equal(ans_1,xs)] AND NOT[equal(ans_2,ys)]]{Der Scheitelpunkt ist falsch.}
     \explanation[NOT[equal(ans_1,xs)] AND [equal(ans_2,ys)]]{Der Scheitelpunkt hat die richtige $y$-Koordinate, 
     aber nicht die richtige $x$-Koordinate.}
     \explanation[NOT[equal(ans_2,ys)] AND [equal(ans_1,xs)]]{Der Scheitelpunkt hat die richtige $x$-Koordinate, 
     aber nicht die richtige $y$-Koordinate.}
     \explanation[[equal(ans_1,xs)] AND [equal(ans_2,ys)]]{Der Scheitelpunkt ist richtig, aber der Punkt $P$ 
     stimmt noch nicht.}
   \end{question}

   
 \end{problem}

\embedmathlet{gwtmathlet}

\end{content}
