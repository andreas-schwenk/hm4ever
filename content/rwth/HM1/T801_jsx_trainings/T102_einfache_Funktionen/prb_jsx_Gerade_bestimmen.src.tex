\documentclass{mumie.problem.gwtmathlet}
%$Id$
\begin{metainfo}
  \name{
    \lang{en}{...}
    \lang{de}{A03 + JSX: Lage einer Geraden}
  }
  \begin{description} 
 This work is licensed under the Creative Commons License Attribution 4.0 International (CC-BY 4.0)   
 https://creativecommons.org/licenses/by/4.0/legalcode 

    \lang{en}{...}
    \lang{de}{Bestimmung der Parameter einer Geraden}
  \end{description}
  \corrector{system/problem/GenericCorrector.meta.xml}
  \begin{components}
    \component{js_lib}{system/problem/GenericMathlet.meta.xml}{gwtmathlet}
  \end{components}
  \begin{links}
  \end{links}
  \creategeneric
\end{metainfo}
\begin{content}
\begin{block}[annotation]
	Im Ticket-System: \href{https://team.mumie.net/issues/32290}{Ticket 32290}
\end{block}
\begin{block}[annotation]
Copy of : /home/mumie/checkin/content/playground/HM4mint-Testaufgaben/T102neu_Einfache_Reelle_Funktionen/prb_jsx_Gerade_bestimmen.src.tex
\end{block}

%

\begin{block}[annotation]
(B. Guyet)\\
Basis ist Problem 3:\\
Copy of : /home/mumie/checkin/content/rwth/HM1/T102neu_Einfache_Reelle_Funktionen/training/prb_problem3.src.tex
\\
Ergänzt um Grapfik aus: \\
/home/mumie/checkin/content/playground/Andreas_Maurischat/grafischeAufgaben/prb_Linear_A1.1.src.tex
\end{block}


\usepackage{mumie.genericproblem}

\begin{block}[annotation]
Weiterentwicklung der Aufgabe 3:\\
 Bestimmung der Parameter und der Nullstelle einer Geraden aus 
 gegebener Geradengleichung 
        
\end{block}

\lang{de}{\title{A03 + JSX: Lage einer Geraden}}
\lang{en}{\title{Problem 3}}

\begin{visualizationwrapper}


\begin{genericJSXVisualization}
	\begin{variables}
  		\point[editable]{Q}{integer}{2,1}
  		\point[editable]{P}{integer}{1,1}
      \number{m}{rational}{(P[y]-Q[y])/(P[x]-Q[x])}
      \function{g}{real}{m*(x-Q[x])+Q[y]}
	\end{variables}
\color{g}{RED}
\answer{g}{3,1}

	\begin{canvas}
        \snapToGrid{0.05,0.05}
        \plotSize{400,400}
        \plotLeft{-5}
        \plotRight{5}
        \plot[coordinateSystem,showPointCoords]{Q,P,g}
	\end{canvas}

\end{genericJSXVisualization}

\end{visualizationwrapper}

\\
Geben Sie im folgenden alle Zahlen als Brüche oder ganze Zahlen an und kürzen Sie Brüche soweit wie möglich.

\begin{problem}


%++++++++++++++++++_Frage_1_von_3_+++++++++++++++++++++++++++++	

 
      \begin{variables}
          \randint[Z]{p}{-10}{10}
          \drawFromSet{q}{2,4,5}
          \randint[Z]{cc}{-5}{5} 
          \function[calculate]{c}{cc*p}              	       	  	   	
          \function[calculate]{b}{-(c/q)}  
          \function[calculate]{m}{-(p/q)}
          \function[calculate]{nst}{-(b/m)}
          \function[normalize]{f}{p * x + q * y + c}
          \function[normalize]{g}{m * x + b}   	           	        	
      \end{variables}

  \begin{question}  
  
      \text{Überführen Sie die lineare Gleichung $\,\var{f}= 0\,$
            in die Punkt-Steigungsform einer Geraden. 
          }
            
        \type{input.function}
        \field{rational} 
 
%++++++++++++++++++++++ANSWERS++++++++++++++++++++++++    

      
	\begin{answer}
        \text{$\qquad g: y=mx+b=$}
        \solution{g}
        \checkAsFunction{x}{-10}{10}{100}
        \explanation{Die Geradengleichung erhält man durch Aufl"osen von $\var{f}= 0$ nach $y$.}
   \end{answer}

 \end{question}		

%++++++++++++++++++_Frage_2_von_3_+++++++++++++++++++++++++++++	

  \begin{question}  
  
      \text{Beschreiben Sie die Lage dieser Geraden anhand der
            charakteristischen Eigenschaften, die sich aus der 
            Geradengleichung ableiten lassen.
      }


    \type{input.number} %input.text %input.cases.function %input.finite-number-set %input.interval %...http://team.mumie.net/projects/support/wiki/DifferentAnswerType
    \field{rational} 
    \precision{2}
           

 	\begin{answer}
       \text{$\quad$ Die Gerade hat die Steigung $\,m=$}
       \solution{m}
       \explanation{Die Steigung der Geraden kann direkt aus der Geradengleichung abgelesen werden.}        
   \end{answer}
%
 	\begin{answer}
        \text{$\quad$ Sie schneidet die $y-$Achse in $\,y_0=$}
        \solution{b}       
        \explanation{Die Schnittstelle mit der $y-$Achse entspricht dem $y-$Achsenabschnitt $b$ 
        der Geraden und kann daher direkt aus der Geradengleichung abgelesen werden.}        
    \end{answer}
%
 	\begin{answer}
       \text{$\quad$ und schneidet die $x-$Achse in $\,x_0=$}
       \solution{nst}
       \explanation[NOT equal(ans, nst)]{Die Schnittstelle mit der $x-$Achse entspricht der Nullstelle der Geraden. 
        Sie berchnet sich aus $\,-\frac{b}{m}.$}        
    \end{answer}
      
\end{question}		

%++++++++++++++++++_Frage_3_von_3_+++++++++++++++++++++++++++++	
%
% grafischen Teil
%
\begin{question}
    \field{rational}

    \lang{de}{
         \text{Verschieben Sie in der Grafik die eingezeichneten Punkte so, dass der Funktionsgraph 
               Gerade $g$ entsteht.}}        
    \begin{answer}
          \type{graphics.function}
          \solution{g}
          \checkAsFunction[2e-1]{x}{-10}{10}{100}
    \end{answer}
    \explanation{Wählen Sie zum Beispiel den Schnittpunkt mit der $y$-Achse als einen Punkt und 
            den Schnittpunkt mit der $x$-Achse als den zweiten Punkt.}
\end{question}

%######################################################QUESTION_END


\end{problem}

\embedmathlet{gwtmathlet}                                         

\end{content}
