\documentclass{mumie.problem.gwtmathlet}
%$Id$
\begin{metainfo}
  \name{
    \lang{en}{Problem 1.3}
    \lang{de}{Lineare Gleichung grafisch lösen}
  }
  \begin{description} 
 This work is licensed under the Creative Commons License Attribution 4.0 International (CC-BY 4.0)   
 https://creativecommons.org/licenses/by/4.0/legalcode 

    \lang{en}{...}
    \lang{de}{...}
  \end{description}
  \corrector{system/problem/GenericCorrector.meta.xml}
  \begin{components}
    \component{js_lib}{system/problem/GenericMathlet.meta.xml}{gwtmathlet}
  \end{components}
  \begin{links}
  \end{links}
  \creategeneric
  \begin{taxonomy}
        \difficulty{2}
        \usage{FH Aachen, Mathematische Grundlagen}
        \objectives{analyze,apply}
        \topic{analysis1/equations/one_variable/eq_one_linear}
  \end{taxonomy}
\end{metainfo}
\begin{content}
\begin{block}[annotation]
	Im Ticket-System: \href{https://team.mumie.net/issues/32298}{Ticket 32298}
\end{block}
\begin{block}[annotation]
Copy of : /home/mumie/checkin/content/playground/HM4mint-Testaufgaben/T102neu_Einfache_Reelle_Funktionen/prb_jsx_Geraden_Schnittpkt.src.tex
\end{block}


\begin{block}[annotation]
(B. Guyet)\\
Copy of : /home/mumie/checkin/content/playground/Andreas_Maurischat/grafischeAufgaben/prb_Linear_A1.3.src.tex
/home/mumie/checkin/content/pool/fh_aachen/MGI/Lineare_Funktionen/prb_Linear_A1.3.src.tex
\\
mit leicht veränderten Parametern.
\end{block}

\begin{block}[annotation]
Grafische Bestimmung des Schnittpunkts zweier Geraden
\end{block}

\usepackage{mumie.genericproblem}

\begin{visualizationwrapper}


\begin{genericJSXVisualization}
	\begin{variables}
% 
    \question{1}{m1}{real}  % Steigung der Geraden f
    \question{1}{m2}{real}  % Steigung der Geraden g
    
    \point[editable]{S}{real}{2,0} % bel. Punkt auf der roten Geraden mit Steigung von f
    \point[editable]{A}{real}{0,0} % bel. Punkt auf der blauen Geraden mit Steigung von g

    \function{f}{real}{m1*(x-S[x])+S[y]}
    \function{g}{real}{m2*(x-A[x])+A[y]}
%
% alt (Geraden f und g ohne Steigung vorgegeben. Dabei entsteht Problem, dass die Geraden parallel
%      zur y-Achse (mit unendlicher Steigung) gezeichnet werden können. Das führt zu hartem JSX-Fehler.)
%      \point[editable]{S}{real}{2,0}
%  		 \point[editable]{P}{real}{4,1}
%      \line[editable]{fl}{real}{S, P}
%     % \number{m1}{rational}{(P[y]-S[y])/(P[x]-S[x])}
%     % \function{f}{real}{m1*(x-S[x])+S[y]}
%      \point[editable]{A}{real}{0,0}
%  		 \point[editable]{B}{real}{2,2}
%      \line[editable]{gl}{real}{A, B}
%     % \number{m2}{rational}{(B[y]-A[y])/(B[x]-A[x])}
%     % \function{g}{real}{m2*(x-A[x])+A[y]}
%
      \point[editable]{W}{real}{-4,-2} % Hilfspunkt zum Ablesen des Schnittpunkts von f und g
	\end{variables}
% 
\color{f}{RED}
\color{S}{RED}
% \color{P}{RED}
\color{g}{BLUE}
\color{A}{BLUE}
% \color{B}{BLUE}
\color{W}{green}
%
\answer{f}{1,1}
\answer{g}{1,2}
%
	\begin{canvas}
        \snapToGrid{0.05,0.05}
        \plotSize{500,500}
        \plotLeft{-6.5}
        \plotRight{6.5}
        \plot[coordinateSystem,showPointCoords]{S,A,f,g,W}
	\end{canvas}
%
\end{genericJSXVisualization}

\end{visualizationwrapper}

     \begin{problem} 
          \begin{question}
          \field{rational}
          \correctorprecision{1}
          \begin{variables}
                    \randint{N1}{-5}{5}
                    \drawFromSet{D1}{2,4}
                    \function[calculate]{m1}{N1/D1}
                    \randint{N2}{-5}{5}
                    \drawFromSet{D2}{2,4}
                    \function[calculate]{m2}{N2/D2}
                    \randadjustIf{N2}{|m1|=|m2|}
                    \randint[Z]{z}{-5}{5}
                    \randint{w}{-5}{5}
                    \number{d}{2}
                    %\function[calculate]{x1}{z/d}
                    \function[calculate]{x1}{z}
                    \function[calculate]{y1}{w/d}
                    \function[expand,normalize]{f}{m1*(x-x1)+y1}
                    \function[expand,normalize]{g}{m2*(x-x1)+y1}
               \end{variables}
          \lang{de}{
               \text{Bestimmen Sie grafisch den Schnittpunkt der beiden Geraden $\;f(x) = \var{f}\;$
                     und $\;g(x) = \var{g}$.\\\\
                     Verschieben Sie hierzu zunächst den roten und den blauen Punkt in der Grafik so, 
                     dass die Funktionsgraphen der beiden Geraden entstehen. 
                     Lesen Sie anschließend die Lösung in der Grafik unter Verwendung des grünen Punktes ab.}}               
% \debug[f,g,m1,m2,x1,y1]

          \begin{answer}
                \type{graphics.function}
                \solution{f}
                \inputAsFunction{x}{ant1}
                \checkFuncForZero[2e-1]{(ant1-f)*(ant1-g)}{-1}{1}{100}
          \end{answer}
          \begin{answer}
                \type{graphics.function}
                \solution{g}
                \inputAsFunction{x}{ant2}
                % zweiter Teil der Bedingung prüft, ob die Steigungen der Antworten verschieden sind.
                \checkFuncForZero[2e-1]{|(ant2-f)*(ant2-g)|+dirac(|D[ant1,x]-D[ant2,x]|)}{-1}{1}{100}
          \end{answer}
          \begin{answer}
            \type{input.number}
            \text{Die Geraden schneiden sich in $x_s=$} % x-Koordinate des Schnittpunkts
            \solution{x1}
           \end{answer}
          \begin{answer}
            \type{input.number}
            \text{und es ist $f(x_s)=g(x_s)=$}  % y-Koordinate des Schnittpunkts
            \solution{y1}
          \end{answer}
          \explanation[NOT[equal(ans_1,f)]]{Die rote Gerade ist falsch eingezeichnet.}
          \explanation[NOT[equal(ans_2,g)]]{Die blaue Gerade ist falsch eingezeichnet.}
          \explanation[[equal(ans_1,f) AND equal(ans_2,g)] AND [NOT edited(ans_3) OR NOT edited(ans_4)]]
                      {Die Koordinaten des Schnittpunkts sind noch nicht vollständig eingegeben.}
          \explanation[[equal(ans_1,f) AND equal(ans_2,g)] AND [edited(ans_3) AND edited(ans_4)] AND [NOT correct(ans_3)]]
                      {Die $x$-Koordinate des Schnittpunkts ist falsch.}
          \explanation[[equal(ans_1,f) AND equal(ans_2,g)] AND [edited(ans_3) AND edited(ans_4)] AND [NOT correct(ans_4)]]
                      {Die $y$-Koordinate des Schnittpunkts ist falsch.}
% alt:
%          \explanation[equal(ans_1,ans_2)]{Sie haben zweimal die gleiche Gerade eingezeichnet.}
%          \explanation[NOT[equal(ans_1,ans_2)] AND NOT [equal(ans_1,f) OR equal(ans_1,g)]]{Die rote Gerade ist falsch eingezeichnet.}
%          \explanation[NOT[equal(ans_1,ans_2)] AND NOT [equal(ans_2,f) OR equal(ans_2,g)]]{Die blaue Gerade ist falsch eingezeichnet.}
%          \explanation[[[equal(ans_1,g) AND equal(ans_2,f)] OR [equal(ans_1,f) AND equal(ans_2,g)]] AND [NOT edited(ans_3) OR NOT edited(ans_4)]]
%                      {Die Koordinaten des Schnittpunkts sind noch nicht vollständig eingegeben.}
%          \explanation[[[equal(ans_1,g) AND equal(ans_2,f)] OR [equal(ans_1,f) AND equal(ans_2,g)]] AND [edited(ans_3) AND edited(ans_4)] AND [NOT correct(ans_3)]]
%                      {Die $x$-Koordinate des Schnittpunkts ist falsch.}
%          \explanation[[[equal(ans_1,g) AND equal(ans_2,f)] OR [equal(ans_1,f) AND equal(ans_2,g)]] AND [edited(ans_3) AND edited(ans_4)] AND [NOT correct(ans_4)]]
%                      {Die $y$-Koordinate des Schnittpunkts ist falsch.}                      
%
          \end{question}
     \end{problem}

\embedmathlet{gwtmathlet}

\end{content}
