\documentclass{mumie.problem.gwtmathlet}
%$Id$ 
\begin{metainfo}
  \name{
  \lang{en}{...}
  \lang{de}{SQ02: Bestimmung des Quadrats einer komplexen Zahl}
  \lang{zh}{...}
  \lang{fr}{...}
  }
  \begin{description} 
 This work is licensed under the Creative Commons License Attribution 4.0 International (CC-BY 4.0)   
 https://creativecommons.org/licenses/by/4.0/legalcode 

    \lang{en}{...}
    \lang{de}{...}
    \lang{zh}{...}
    \lang{fr}{...}
  \end{description}
  \corrector{system/problem/GenericCorrector.meta.xml}
   \begin{components}
      \component{js_lib}{system/problem/GenericMathlet.meta.xml}{gwtmathlet}
   \end{components}
  \begin{links}
  \end{links}
  \creategeneric
\end{metainfo}
\begin{content}
\begin{block}[annotation]
Copy of : /home/mumie/checkin/content/playground/HM4mint-Testaufgaben/T203_komplexe_Zahlen/prb_qseq_komplexe_zahl_quadrieren.src.tex
\end{block}

\begin{block}[annotation]
Im Ticket-System
\href{https://team.mumie.net/issues/32296}{Ticket 32296}
\end{block}

\begin{block}[annotation]
This is a question sequence problem for squaring of a complex number. 
The adaptive path with all intermediate step tasks is supposed to have the structure given by the graphics in the ticket.
\\
\textbf{Hinweis:} Technisch funktioniert derzeit noch nicht der Rücksprung aus Q3/4, Q8/9 und Q10/11 in die 
editierbare Graphik in Q1, deshalb gilt die Aufgabe aktuell mit  korrekter Beantwortung dieser Fragen 
als vollständig gelöst.
\end{block}

\title{SQ02: Bestimmung des Quadrats einer komplexen Zahl}

\usepackage{mumie.genericproblem}

\begin{visualizationwrapper}
  \begin{genericJSXVisualization}
    \begin{variables}
        \question{1}{x}{real}
        \question{1}{y}{real}
        %%%
        \point{z1}{real}{x,y}
        \vector{v1}{real}{z1}
        \point[editable]{R}{real}{2,1}
        \vector{v2}{real}{R}
        %%%
        \number{Rx}{real}{R[x]}
        \number{Ry}{real}{R[y]}  
    \end{variables}
    
    \color{v1}{#000077} %quite dark blue
    \color{z1}{#000077} %quite dark blue
    \color{R}{GREEN}
    \color{v2}{GREEN}
    \answer{Rx}{1,1}
    \answer{Ry}{1,2}
  
    \begin{canvas}
      %\snapToGrid{0.05,0.05}
  		\plotSize{300,300}
  		\plotLeft{-9.5}
  		\plotRight{9.5}
  		\plot[coordinateSystem,showPointCoords]{z1,v1,R,v2}
  	\end{canvas}
  \end{genericJSXVisualization}  
\end{visualizationwrapper}  

\begin{problem}
% globale Variablen
  \begin{variables}
    \randint{x}{-3}{3}
    \randint[Z]{y}{-3}{3} % y=0 darf nicht sein bei der Berechnung von phi 
    
  % kartesische Darstellung
    \function[normalize]{z}{x+y*i}
    \function[calculate]{z2x}{x*x - y*y}
    \function[calculate]{z2y}{2*x*y}
    \function[normalize]{z2}{z2x + z2y*i}

  % Polarkoordinaten
    \function[calculate]{lz2}{x^2 + y^2}
    \function{lengthz}{sqrt(lz2)}
    %
    \begin{switch}
      \begin{case}{x=0}    % Re(z)=0
        \function[calculate, display]{argz}{y/abs(y)*pi/2}
      \end{case}
      \begin{case}{x<0}     % Re(z)<0 
        \function[calculate, display]{argz}{arctan(y/x) + y/abs(y)*pi}
      \end{case}      
      \begin{default}       % Re(z)>0 
        \function[calculate, display]{argz}{arctan(y/x)}
      \end{default}
    \end{switch}
    
  % Multiplikation in Polardarstellung
    \function[calculate, display]{lengthz2}{lengthz^2}
    \function[calculate, display]{argzz}{2*argz}
    %
    \begin{switch}
      \begin{case}{argzz>pi}    
        \function[calculate, display]{argz2}{argzz-2*pi}
      \end{case}
      \begin{case}{argzz<=-pi}     
        \function[calculate, display]{argz2}{argzz+2*pi}
      \end{case}      
      \begin{default}       
        \function[calculate, display]{argz2}{argzz}
      \end{default}
    \end{switch}
    %    
    % \function[normalize]{z2p}{lengthz^2 *(cos(2*argz)+i*sin(2*argz))}
    % \function[calculate]{z2px}{lengthz^2 * cos(2*argz)}
    % \function[calculate]{z2py}{lengthz^2 * sin(2*argz)}
  %     
    % \function[calculate]{px}{(x^2+y^2)*cos(2*acos(x/(x^2+y^2)^0.5)*y/abs(y))}  %r^2cos(2*phi)
    % \function[calculate]{py}{(x^2+y^2)*sin(2*acos(x/(x^2+y^2)^0.5)*y/abs(y))}  %r^2sin(2*phi)
    

  \end{variables}



% Q1
  \begin{question}
  \correctorprecision{1}
  \precision{1}
    \text{Bestimmen Sie das Quadrat $z^2$ der komplexen Zahl zu $z=\var{z}$ und ziehen 
          Sie den grünen Punkt in die Koordinaten von $z^2$ in der Gaußschen Zahlenebene.
          Beachten Sie, dass Sie hierzu die Größe (durch Klick auf +/-) sowie die Lage 
          des Koordiantensystem (durch Klick auf die Pfeile) geeignet anpassen können.}
           
          
   \explanation{Die Koordinaten des grünen Punkts entsprechen nicht $z^2$.}
   \followupQuestion[NOT edited OR NOT correct(ans_1) OR NOT correct(ans_2)]{2}  % falls die Eingabe fehlerhaft ist

 %\debug[z2,z2x,z2y,lengthz,argz]

   \begin{answer}
      \type{graphics.function}
      \solution{z2x}
      \checkAsFunction[1e-1]{x}{-5}{5}{100}
     \end{answer}
    \begin{answer}
      \type{graphics.function}
      \solution{z2y}
      \checkAsFunction[1e-1]{x}{-5}{5}{100}
     \end{answer} 
   \end{question}   

% Q2
  \begin{question}
  
    \text{Sie haben $z^2$ nicht korrekt bestimmt oder wissen nicht, wie Sie hier vorgehen können? 
          Dann wählen Sie zunächst einen der folgenden Rechenwege zur Bestimmung der Lage von $z^2$ 
          in der Gaußschen Zahlenebene:
        }
% \debug[px,py]
% \explanation{Der Winkel $\phi$ von $z$ muss verdoppelt werden, der Betrag $r$ muss quadriert werden. }

      \type{mc.unique}

      \followupQuestion[equalChoice(1)]{3}
      \followupQuestion[equalChoice(2)]{5}
      
     \explanation[NOT edited]{Sie haben noch keine Auswahl getroffen.}
      %1
      \begin{choice}
          \text{Einfaches Ausmultiplizieren von $z \cdot z \,$ in kartesischer Form.}
          \solution{true}
      \end{choice}      
      %2
      \begin{choice}
          \text{Verwendung der Polarkoordinaten von $z$.}
          \solution{true}
      \end{choice}      
  \end{question} 
  
% Q3
  \begin{question}
    \text{Berechnen Sie die Koordinaten von $z^2$ durch Multiplikation der komplexen Zahl $z=\var{z}$
          mit sich selbst und geben Sie das Ergebnis als Zahlenpaar an: \\\\

          $z^2=(\var{z}) \cdot (\var{z})=($ \ansref $;$ \ansref $)$.
        }
            
    \type{input.number}
    \field{complex}
    
%       \followupQuestion[correct(ans_1) AND correct(ans_2)]{12}  % falls die Eingabe korrekt ist
        \followupQuestion[NOT edited OR NOT correct(ans_1) OR NOT correct(ans_2)]{4}  % falls die Eingabe fehlerhaft ist

    \begin{answer}
      \solution{z2x}
      \explanation{Der Realteil von $z^2$ ist nicht korrekt.}
    \end{answer}
    \begin{answer}
      \solution{z2y}
      \explanation{Der Imainärteil von $z^2$ ist nicht korrekt.}
    \end{answer} 
  \end{question} 
   
% Q4
  \begin{question}
    \text{Berechnen Sie die Koordinaten von $z^2$ durch Multiplikation der komplexen Zahl $z=\var{z}$
          mit sich selbst und geben Sie das Ergebnis als Zahlenpaar an: \\\\

          $z^2=(\var{z}) \cdot (\var{z})=($ \ansref $;$ \ansref $)$.
        }
    \type{input.number}
    \field{complex}

%      \followupQuestion[correct(ans_1) AND correct(ans_2)]{12}  % falls die Eingabe korrekt ist    
       \followupQuestion[NOT edited OR NOT correct(ans_1) OR NOT correct(ans_2)]{3}  % falls die Eingabe fehlerhaft ist

    \begin{answer}
      \solution{z2x}
      \explanation{Der Realteil von $z^2$ ist nicht korrekt.}
    \end{answer}
    \begin{answer}
      \solution{z2y}
      \explanation{Der Imainärteil von $z^2$ ist nicht korrekt.}
    \end{answer} 
  \end{question} 
  
% Q5 (Polarkoord.)
   \begin{question} 

      \type{input.function}
  		\field{real}
      \displayprecision{5} 
        
       \text{Berechnen Sie die Polarzerlegung, also die Länge $r$ und den Phasenwinkel $\phi \in (-\pi;\pi]$
            der komplexen Zahl $z = \var{z}$. Sie können Ihr Ergebnis als Formel eingeben oder berechnet 
            als Dezimalzahl auf 2 Nachkommastellen gerundet.\\
       $r = $ \ansref\\
       $\phi= $ \ansref\\
       }

       \followupQuestion[correct(ans_1) AND correct(ans_2)]{7}  % falls die Eingabe korrekt ist
       \followupQuestion[NOT edited OR NOT correct(ans_1) OR NOT correct(ans_2)]{6}  % falls die Eingabe fehlerhaft ist
%\debug[x,y,z2,lengthz,argz,lengthz2,argz2]

	\begin{answer} %1
		\solution{lengthz}
		\allowForInput[false]{sin cos exp}
		\checkAsFunction[1E-1]{x}{-10}{10}{10}
    \explanation{Die Länge wurde falsch berechnet.}
	\end{answer}
    
	\begin{answer} %1
		\solution{argz}
		\allowForInput[false]{sin cos exp}
    \checkAsFunction[1E-1]{x}{-10}{10}{10}
    \explanation{Das Argument wurde falsch berechnet oder liegt nicht Hauptbereich $(-\pi;\pi]$.}
	\end{answer}
\end{question} 

% Q6 (Polarkoord.)
  \begin{question} 

      \type{input.function}
  		\field{real}
      \displayprecision{5} 
        
       \text{Berechnen Sie die Polarzerlegung, also die Länge $r$ und den Phasenwinkel $\phi \in (-\pi;\pi]$
            der komplexen Zahl $z = \var{z}$. Sie können Ihr Ergebnis als Formel eingeben oder berechnet 
            als Dezimalzahl auf 2 Nachkommastellen gerundet.\\
       $r = $ \ansref\\
       $\phi = $ \ansref\\
       }

       \followupQuestion[correct(ans_1) AND correct(ans_2)]{7}  % falls die Eingabe korrekt ist
       \followupQuestion[NOT edited OR NOT correct(ans_1) OR NOT correct(ans_2)]{6}  % falls die Eingabe fehlerhaft ist

	\begin{answer} %1
		\solution{lengthz}
		\allowForInput[false]{sin cos exp}
		\checkAsFunction[1E-1]{x}{-10}{10}{10}
    \explanation{Die Länge wurde falsch berechnet.}
	\end{answer}
    
	\begin{answer} %1
		\solution{argz}
		\allowForInput[false]{sin cos exp}
    \checkAsFunction[1E-1]{x}{-10}{10}{10}
    \explanation{Das Argument wurde falsch berechnet oder liegt nicht Hauptbereich $(-\pi;\pi]$.}
	\end{answer}
\end{question} 

% Q7
  \begin{question}
  
    \text{Mit Hilfe der Polardarstellung von $z$ können Sie nun zwischen den folgenden zwei Wegen wählen:}

      \type{mc.unique}

      \followupQuestion[equalChoice(1)]{8}
      \followupQuestion[equalChoice(2)]{10}
      
     \explanation[NOT edited]{Sie haben noch keine Auswahl getroffen.}
      %1
      \begin{choice}
          \text{Sie Bestimmen die Koordinaten von $z^2$ graphisch nach den Regeln zur 
                Multiplikation in Polardarstellung.}
          \solution{true}
      \end{choice}  
      %3
      \begin{choice}
          \text{Sie Berechnen $z^2$ unter Verwendung der Eulerscher Form von $z$.}
          \solution{true}
      \end{choice}         
  \end{question} 

% Q8: Multiplikation in Polardarstellung
   \begin{question} 
   
      \type{input.function}
  		\field{real}
      \displayprecision{2} 
        
       \text{Aus den Polarkoordinaten von $z$, Länge $|z|=\var{lengthz}$ und Phasenwinkel 
             $\phi=$arg$(z)=\var{argz}$, lassen sich nach den Regeln zur Multiplikation in 
             Polardarstellung die Länge sowie das Argument von $z^2=z\cdot z$ in $(-\pi;\pi]$ bestimmen.\\\\
             $|z^2| = $ \ansref\\
             arg$(z^2) = $ \ansref\\\\
             Mit der Länge $|z^2|$ von $z^2$ und dem Phasenwinkel arg$(z^2)$ können nun
             die Koordinaten von $z^2$ in der Gaußschen Zahlenebene graphisch bestimmt werden.
             }

%       \followupQuestion[correct(ans_1) AND correct(ans_2)]{12}  % falls die Eingabe korrekt ist
       \followupQuestion[NOT edited OR NOT correct(ans_1) OR NOT correct(ans_2)]{9}  % falls die Eingabe fehlerhaft ist
%\debug[x,y,z2,lengthz,argz,lengthz2,argzz,pi,argz2]

	\begin{answer} %1
		\solution{lengthz2}
		\allowForInput[false]{* sin cos exp}
		\checkAsFunction[1E-1]{x}{-10}{10}{10}
    \explanation{Die Länge von $z^2$ entspricht dem Quadrat der Länge von $z$.}
	\end{answer}
    
	\begin{answer} %1
		\solution{argz2}
		\allowForInput[false]{* sin cos exp}
    \checkAsFunction[1E-1]{x}{-10}{10}{10}
    \explanation{Der Phasenwinkel von $z^2$ entspricht dem Zweifachen des Phasenwinkels von $z, \;$
                 ggf. um $2\pi$ verschoben in den Hauptbereich $(-\pi;\pi]$.}    
	\end{answer}
\end{question} 

% Q9: Multiplikation in Polardarstellung
   \begin{question} 
   
      \type{input.function}
  		\field{real}
      \displayprecision{2} 
        
       \text{Aus den Polarkoordinaten von $z$, Länge $|z|=\var{lengthz}$ und Phasenwinkel 
             $\phi=$arg$(z)=\var{argz}$, lassen sich nach den Regeln zur Multiplikation in 
             Polardarstellung die Länge sowie das Argument von $z^2=z\cdot z$ in $(-\pi;\pi]$ bestimmen.\\\\
             $|z^2| = $ \ansref\\
             arg$(z^2) = $ \ansref\\\\
             Mit der Länge $|z^2|$ von $z^2$ und dem Phasenwinkel arg$(z^2)$ können nun
             die Koordinaten von $z^2$ in der Gaußschen Zahlenebene graphisch bestimmt werden.
             }

%       \followupQuestion[correct(ans_1) AND correct(ans_2)]{12}  % falls die Eingabe korrekt ist
       \followupQuestion[NOT edited OR NOT correct(ans_1) OR NOT correct(ans_2)]{8}  % falls die Eingabe fehlerhaft ist

	\begin{answer} %1
		\solution{lengthz2}
		\allowForInput[false]{* sin cos exp}
		\checkAsFunction[1E-1]{x}{-10}{10}{10}
    \explanation{Die Länge von $z^2$ entspricht dem Quadrat der Länge von $z$.}
	\end{answer}
    
	\begin{answer} %1
		\solution{argz2}
		\allowForInput[false]{* sin cos exp}
    \checkAsFunction[1E-1]{x}{-10}{10}{10}
    \explanation{Der Phasenwinkel von $z^2$ entspricht dem Zweifachen des Phasenwinkels von $z, \;$
                 ggf. um $2\pi$ verschoben in den Hauptbereich $(-\pi;\pi]$.}    
	\end{answer}

\end{question} 

% Q10: Eulerscher Form

   \begin{question} 
   
      \type{input.function}
  		\field{real}
      \displayprecision{2} 
        
       \text{Aus den Polarkoordinaten von $z\;$ ($r=\var{lengthz}$ und $\phi=\var{argz}$)
             erhalten wir die Eulersche Form von $z\; =r\cdot e^{i \cdot \phi}$. \\\\
             
             Berechnen Sie nun $z^2$ in der Eulerschen Form: \\

             $z^2=(r\cdot e^{i\cdot \phi})^2=$\ansref $\cdot \exp(i\cdot $ \ansref $)$ \\\\
            
             und leiten sie hieraus die Länge $|z^2|=$\ansref sowie den Phasenwinkel 
             arg$(z^2)= $\ansref von $z^2$ direkt ab. Mit diesen Größen lassen sich die
             die Koordinaten von $z^2$ in der Gaußschen Zahlenebene graphisch bestimmen.
             }

%       \followupQuestion[correct(ans_1) AND correct(ans_2)]{12}  % falls die Eingabe korrekt ist
       \followupQuestion[NOT edited OR NOT correct(ans_1) OR NOT correct(ans_2)]{11}  % falls die Eingabe fehlerhaft ist

 	\begin{answer} %1
		\solution{lengthz2}
		\checkAsFunction[1E-1]{x}{-10}{10}{10}
    \explanation{Die Länge von $z^2$ entspricht dem Quadrat der Länge von $z$.}
	\end{answer}
    
	\begin{answer} %2
		\solution{argz2}
    \checkAsFunction[1E-1]{x}{-10}{10}{10}
    \explanation{Der Phasenwinkel von $z^2$ entspricht dem Zweifachen des Phasenwinkels von $z, \;$
                 ggf. um $2\pi$ verschoben in den Hauptbereich $(-\pi;\pi]$.}    
	\end{answer}

 	\begin{answer} %3
		\solution{lengthz2}
		\allowForInput[false]{sqrt * ^}
		\checkAsFunction[1E-1]{x}{-10}{10}{10}
    \explanation{Die Länge von $z^2$ entspricht dem Quadrat der Länge von $z$.}
	\end{answer}
    
	\begin{answer} %4
		\solution{argz2}
		\allowForInput[false]{* +}
    \checkAsFunction[1E-1]{x}{-10}{10}{10}
    \explanation{Der Phasenwinkel von $z^2$ entspricht dem Zweifachen des Phasenwinkels von $z, \;$
                 ggf. um $2\pi$ verschoben in den Hauptbereich $(-\pi;\pi]$.}    
	\end{answer}

\end{question} 

% Q11: Eulerscher Form

   \begin{question} 
   
      \type{input.function}
  		\field{real}
      \displayprecision{2} 
        
       \text{Aus den Polarkoordinaten von $z\;$ ($r=\var{lengthz}$ und $\phi=\var{argz}$)
             erhalten wir die Eulersche Form von $z\; =r\cdot e^{i \cdot \phi}$. \\\\
             
             Berechnen Sie nun $z^2$ in der Eulerschen Form: \\

             $z^2=(r\cdot e^{i\cdot \phi})^2=$\ansref $\cdot \exp(i\cdot $ \ansref $)$ \\\\
            
             und leiten sie hieraus die Länge $|z^2|=$\ansref sowie den Phasenwinkel 
             arg$(z^2)= $\ansref von $z^2$ direkt ab. Mit diesen Größen lassen sich die
             die Koordinaten von $z^2$ in der Gaußschen Zahlenebene graphisch bestimmen.
             }

%       \followupQuestion[correct(ans_1) AND correct(ans_2)]{12}  % falls die Eingabe korrekt ist
       \followupQuestion[NOT edited OR NOT correct(ans_1) OR NOT correct(ans_2)]{10}  % falls die Eingabe fehlerhaft ist

 	\begin{answer} %1
		\solution{lengthz2}
		\checkAsFunction[1E-1]{x}{-10}{10}{10}
    \explanation{Die Länge von $z^2$ entspricht dem Quadrat der Länge von $z$.}
	\end{answer}
    
	\begin{answer} %2
		\solution{argz2}
    \checkAsFunction[1E-1]{x}{-10}{10}{10}
    \explanation{Der Phasenwinkel von $z^2$ entspricht dem Zweifachen des Phasenwinkels von $z, \;$
                 ggf. um $2\pi$ verschoben in den Hauptbereich $(-\pi;\pi]$.}    
	\end{answer}

 	\begin{answer} %3
		\solution{lengthz2}
		\allowForInput[false]{sqrt * ^}
		\checkAsFunction[1E-1]{x}{-10}{10}{10}
    \explanation{Die Länge von $z^2$ entspricht dem Quadrat der Länge von $z$.}
	\end{answer}
    
	\begin{answer} %4
		\solution{argz2}
		\allowForInput[false]{* +}
    \checkAsFunction[1E-1]{x}{-10}{10}{10}
    \explanation{Der Phasenwinkel von $z^2$ entspricht dem Zweifachen des Phasenwinkels von $z, \;$
                 ggf. um $2\pi$ verschoben in den Hauptbereich $(-\pi;\pi]$.}    
	\end{answer}

\end{question} 
  
\end{problem}
 
  \embedmathlet{gwtmathlet}
  
\end{content}