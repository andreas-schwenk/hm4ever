\documentclass{mumie.problem.gwtmathlet}
%$Id$
\begin{metainfo}
  \name{
  \lang{en}{...}
  \lang{de}{Kreis komplex}
  \lang{zh}{...}
  \lang{fr}{...}
  }
  \begin{description} 
 This work is licensed under the Creative Commons License Attribution 4.0 International (CC-BY 4.0)   
 https://creativecommons.org/licenses/by/4.0/legalcode 

    \lang{en}{...}
    \lang{de}{...}
    \lang{zh}{...}
    \lang{fr}{...}
  \end{description}
  \corrector{system/problem/GenericCorrector.meta.xml}
  \begin{components}
    \component{js_lib}{system/problem/GenericMathlet.meta.xml}{gwtmathlet}
  \end{components}
  \begin{links}
  \end{links}
  \creategeneric
\end{metainfo}
\begin{content}
\begin{block}[annotation]
	Im Ticket-System: \href{https://team.mumie.net/issues/32288}{Ticket 32288}
\end{block}
\begin{block}[annotation]
Copy of : /home/mumie/checkin/content/playground/HM4mint-Testaufgaben/T203_komplexe_Zahlen/prb_jsx_Kreis_komplex.src.tex
\end{block}

\begin{block}[annotation]
(Dagmar Schumacher)\\
Copy of : /home/mumie/checkin/content/playground/Dagmar_Schumacher/prb_jsx_Kreis_komplex.src.tex
\end{block}

  \usepackage{mumie.genericproblem}

\begin{visualizationwrapper}

\title{Kreis in der komplexen Ebene}
\\

\begin{genericJSXVisualization}
  \begin{variables}
      
      \point[editable]{q}{real}{-2.5,-1}
      \point[editable]{R}{real}{2,2}
      \circle[editable]{c1}{real}{R,q}
      %%%
      \number{Rx}{real}{R[x]}
      \number{Ry}{real}{R[y]}
      \number{g}{real}{(R[x]-q[x])^2+(R[y]-q[y])^2}

  \end{variables}
%  \text{Sie können den Mittelpunkt $M$ und den Punkt $P$ auf der Kreislinie verschieben.}
  \color{R}{#000077} %quite dark blue
  \color{q}{#000077} %quite dark blue
  \label{R}{M}
  \label{q}{P}
  \answer{Rx}{1,1}
  \answer{Ry}{1,2}
  \answer{g}{1,3}
  	\begin{canvas}
    \snapToGrid{0.1,0.1}
		\plotSize{300,300}
		\plotLeft{-3.5}
		\plotRight{3.5}
		\plot[coordinateSystem,showPointCoords]{q,R,c1}
	\end{canvas}

\end{genericJSXVisualization}

\end{visualizationwrapper}
 
  
 \begin{problem}

    \begin{question}
     \begin{variables}
        \drawFromSet{r2}{1,4,9,16}
        \randint{a}{-2}{2}
        \function{b}{-2}{2}
        \function[normalize]{z0}{a+b*i}
        \function[normalize]{zb}{z-z0}
      \end{variables} 
      \text{Stellen Sie in der Gaußschen Zahlenebene den Kreis dar, 
            der durch folgende Gleichung beschrieben wird:\\
      $|\var{zb}|^2=\var{r2}$
      \\
      \textit{Hinweis: Sie können den Mittelpunkt $M$ und den Punkt $P$ 
        auf der Kreislinie verschieben. }
      }     
      \begin{answer}
        \type{graphics.number}
        \solution{a}
      \end{answer} 
      \begin{answer}
        \type{graphics.number}
        \solution{b}
      \end{answer}
      \begin{answer}
        \type{graphics.number}
        \solution{r2}
      \end{answer}       

    \explanation[NOT[correct(ans_1)] OR NOT[correct(ans_2)]]{Der Mittelpunkt des Kreises ist nicht korrekt platziert.}
    \explanation[NOT[correct(ans_3)]]{Sie haben den Radius falsch berechnet.}
    
    \end{question}       
  \end{problem}

  \embedmathlet{gwtmathlet}
  

  
\end{content}