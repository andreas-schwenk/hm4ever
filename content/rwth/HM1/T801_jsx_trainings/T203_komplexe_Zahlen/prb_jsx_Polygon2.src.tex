\documentclass{mumie.problem.gwtmathlet}
%$Id$
\begin{metainfo}
  \name{
  \lang{en}{...}
  \lang{de}{Polygon2}
  \lang{zh}{...}
  \lang{fr}{...}
  }
  \begin{description} 
 This work is licensed under the Creative Commons License Attribution 4.0 International (CC-BY 4.0)   
 https://creativecommons.org/licenses/by/4.0/legalcode 

    \lang{en}{...}
    \lang{de}{...}
    \lang{zh}{...}
    \lang{fr}{...}
  \end{description}
  \corrector{system/problem/GenericCorrector.meta.xml}
  \begin{components}
    \component{js_lib}{system/problem/GenericMathlet.meta.xml}{gwtmathlet}
  \end{components}
  \begin{links}
  \end{links}
  \creategeneric
\end{metainfo}
\begin{content}
\begin{block}[annotation]
	Im Ticket-System: \href{https://team.mumie.net/issues/32289}{Ticket 32289}
\end{block}
\begin{block}[annotation]
Copy of : /home/mumie/checkin/content/playground/HM4mint-Testaufgaben/T203_komplexe_Zahlen/prb_jsx_Polygon2.src.tex
\end{block}

\begin{block}[annotation]
(Dagmar Schumacher)\\
Copy of : /home/mumie/checkin/content/playground/Dagmar_Schumacher/prb_jsx_Polygon2.src.tex
\end{block}

  \usepackage{mumie.genericproblem}
  
\title{Polygon 2 in der komplexen Ebene}
\\

  \begin{visualizationwrapper}
    \begin{genericJSXVisualization}
      \begin{variables}

        \question{1}{r}{real}
        
        \point{A1}{real}{0,r}
        \point{B1}{real}{r,0}
        \point[editable]{B2}{real}{r-0.5,0.3}
        \point[editable]{A2}{real}{0.3,r-0.5}
        \segment[editable]{A}{real}{var(A2),var(A1)}
        \segment[editable]{B}{real}{var(B2),var(B1)}
        
        \number{a2x}{real}{A2[x]}
        \number{a2y}{real}{A2[y]}
        \number{b2x}{real}{B2[x]}
        \number{b2y}{real}{B2[y]}
      \end{variables}

        \answer{a2x}{1,1}
        \answer{a2y}{1,2}
        \answer{b2x}{1,3}
        \answer{b2y}{1,4}
        
        \color{v1}{#000077} %quite dark blue
        \color{z1}{#000077} %quite dark blue
      
       \begin{canvas}
        \snapToGrid{0.1,0.1}
        \plotSize{500,500}
        \plotLeft{-5.5}
        \plotRight{5.5}
        \plot[coordinateSystem,showPointCoords]{A,B,A2,B2}
      \end{canvas}
 
    \end{genericJSXVisualization}
  \end{visualizationwrapper}

  \begin{problem}
    \begin{variables}
      \randint{r}{1}{5}
      \number{null}{0}
      \function{mr}{-r}
      
    \end{variables}
    \begin{question}
    \text{Markieren Sie in der obigen Grafik die Funktion \\ 
    $f(z)=\max\{|\Re(z)|,|\Im(z)|\}=\var{r}$ \\
    für den 1. Quadranten, indem Sie die zwei blauen Punkte verschieben.\\
    %\max\{\vert x\vert, \vert y\vert\}<r\}
    }
    \begin{answer} %A2
      \type{graphics.number}
      \solution{r}
     \end{answer}
    \begin{answer}
      \type{graphics.number}
      \solution{r}
     \end{answer} 
     \begin{answer} %B2
      \type{graphics.number}
      \solution{r}
     \end{answer}
    \begin{answer}
      \type{graphics.number}
      \solution{r}
     \end{answer}
    \explanation[NOT[correct(ans_1)] OR NOT[correct(ans_2)]]{Die Polygonkante am Punkt $(0;\var{r})$ ist nicht korrekt markiert.}
    \explanation[NOT[correct(ans_3)] OR NOT[correct(ans_4)]]{Die Polygonkante am Punkt $(\var{r};0)$ ist nicht korrekt markiert.}    
   \end{question}
   
   \begin{question}
      \text{Nun soll die Funktion auf der ganzen Gaußschen 
      Zahlenebene betrachtet werden:\\
      $f(z)=\max\{|\Re(z)|,|\Im(z)|\}=\var{r}$\\
      Geben Sie für das entstehende Polygon die Eckpunkte im 2., 3. und 4.
      Quadranten an.\\

      $P_2=($\ansref ;\ansref$)$\\
      $P_3=($\ansref ;\ansref$)$\\
      $P_4=($\ansref ;\ansref$)$\\

      \textit{Bei der Eingabe von Brüchen ist zu beachten, dass diese soweit wie möglich zu kürzen sind!}\\
      }
      \begin{answer}
        \type{input.number}
        \solution{mr}
      \end{answer}
     \begin{answer}
        \type{input.number}
        \solution{r}
      \end{answer} 
      \begin{answer}
        \type{input.number}
        \solution{mr}
      \end{answer}
     \begin{answer}
        \type{input.number}
        \solution{mr}
      \end{answer}
      \begin{answer}
        \type{input.number}
        \solution{r}
      \end{answer}
     \begin{answer}
        \type{input.number}
        \solution{mr}
      \end{answer} 
 
  \explanation[NOT[correct(ans_1)] OR NOT[correct(ans_2)]]{Der Polygon-Eckpunkt $P_2$ im 2. Quadraten ist falsch.}
  \explanation[NOT[correct(ans_3)] OR NOT[correct(ans_4)]]{Der Polygon-Eckpunkt $P_3$ im 3. Quadraten ist falsch.}
  \explanation[NOT[correct(ans_5)] OR NOT[correct(ans_6)]]{Der Polygon-Eckpunkt $P_4$ im 4. Quadraten iat falsch.}
  
  \end{question}   
  \end{problem}
  
  \embedmathlet{gwtmathlet}
  
\end{content}