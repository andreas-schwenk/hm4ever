\documentclass{mumie.problem.gwtmathlet}
%$Id$
\begin{metainfo}
  \name{
  \lang{en}{...}
  \lang{de}{ProduktKomplex}
  \lang{zh}{...}
  \lang{fr}{...}
  }
  \begin{description} 
 This work is licensed under the Creative Commons License Attribution 4.0 International (CC-BY 4.0)   
 https://creativecommons.org/licenses/by/4.0/legalcode 

    \lang{en}{...}
    \lang{de}{...}
    \lang{zh}{...}
    \lang{fr}{...}
  \end{description}
  \corrector{system/problem/GenericCorrector.meta.xml}
  \begin{components}
    \component{js_lib}{system/problem/GenericMathlet.meta.xml}{gwtmathlet}
  \end{components}
  \begin{links}
  \end{links}
  \creategeneric
\end{metainfo}
\begin{content}
\begin{block}[annotation]
	Im Ticket-System: \href{https://team.mumie.net/issues/32297}{Ticket 32297}
\end{block}
\begin{block}[annotation]
Copy of : /home/mumie/checkin/content/playground/HM4mint-Testaufgaben/T203_komplexe_Zahlen/prb_jsx_produkt_komplex.src.tex
\end{block}

\begin{block}[annotation]
(Dagmar Schuacher)\\
Copy of : /home/mumie/checkin/content/playground/Dagmar_Schumacher/prb_produkt_komplex_jsx.src.tex
\end{block}

  \usepackage{mumie.genericproblem}
  \begin{visualizationwrapper}

\title{Produkt zweier komplexer Zahlen}
\\
\begin{genericJSXVisualization}
  \begin{variables}
      \question{1}{x1}{real}
      \question{1}{y1}{real}
      \question{1}{x2}{real}
      \question{1}{y2}{real}
      %%%
      \point{z1}{real}{x1,y1}
      \point{z2}{real}{x2,y2}
      \vector{v1}{real}{z1}
      \vector{v2}{real}{z2}
      \point[editable]{R}{real}{-2.5,-1}
      %%%
      \number{Rx}{real}{R[x]}
      \number{Ry}{real}{R[y]}
  \end{variables}
  
  \color{v1}{#000077} %quite dark blue
  \color{v2}{#000077} %quite dark blue
  \label{z1}{z1}
  \label{z2}{z2}
  \color{R}{GREEN}
  \label{R}{z}
  \answer{Rx}{1,1}
  \answer{Ry}{1,2}

  	\begin{canvas}
    \snapToGrid{0.1,0.1}
		\plotSize{300,300}
		\plotLeft{-3.5}
		\plotRight{3.5}
		\plot[coordinateSystem,showPointCoords]{z1,z2,v1,v2,R}
	\end{canvas}

\end{genericJSXVisualization}

\end{visualizationwrapper}

\begin{problem}

  \begin{question}
  \field{complex}
  %debug
  \begin{variables}
    \randint{x1}{-1}{1}
    \randint[Z]{y1}{-1}{1} % y=0 darf nicht sein bei der Berechnung von phi 
    \randint{x2}{-1}{1}
    \randint[Z]{y2}{-1}{1}
    \randadjustIf{x2,y2}{x1=x2 AND y1=y2}
    \function[normalize]{z1}{x1+y1*i}
    \function[normalize]{z2}{x2+y2*i}
    \function{r1}{(x1^2+y1^2)^0.5}
    \function{r2}{(x2^2+y2^2)^0.5}
    \function{re12}{r1*r2}
    \function{arg1}{acos(x1/(x1^2+y1^2)^0.5)*y1/abs(y1)}
    \function{arg2}{acos(x2/(x2^2+y2^2)^0.5)*y2/abs(y2)}
    \function{arge}{arg1+arg2}
    \function{pex}{re12*cos(arge)}
    \function{pey}{re12*sin(arge)}
  \end{variables}
  
  \correctorprecision{1}
  \precision{1}
%  \text{Bestimmen Sie graphisch das Produkt $z=z_1\cdot z_2$ der beiden komplexen 
%        Zahlen $z_1=\var{z1}$ und $z_2=\var{z2}$. \\
%        Ziehen Sie dazu den grünen Punkt auf die Zahl $z$.} 
%
  \text{Bestimmen Sie graphisch das Produkt $z=z_1\cdot z_2$ der beiden komplexen 
        Zahlen $z_1=\var{z1}$ und $z_2=\var{z2}$ unter Verwendung ihrer Polarkoordinaten.
        Leiten Sie hierzu, falls möglich, die Polarkoordinaten (Winkel und Länge) von $z_1$ 
        und $z_2$ aus der Graphik ab, um hieraus Winkel und Länge ihres Produktes $z$
        zu berechnen. Markieren Sie das Ergebnis für $z$ in der Graphik mit dem grünen Punkt.} 
  
  \explanation{Multiplikation in Polarkoordinaten heißt, die Winkel von $z_1$ und $z_2$ müssen
              addiert und ihre Längen $(|z_1|$ und $|z_2|)$ müssen multipliziert werden. }
   \begin{answer}
      \type{graphics.number}
      \solution{pex}
     \end{answer}
    \begin{answer}
      \type{graphics.number}
      \solution{pey}
     \end{answer} 
   \end{question}     
\end{problem}
 
  \embedmathlet{gwtmathlet}
  
\end{content}