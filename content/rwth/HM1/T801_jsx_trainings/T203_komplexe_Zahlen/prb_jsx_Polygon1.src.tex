\documentclass{mumie.problem.gwtmathlet}
%$Id$
\begin{metainfo}
  \name{
  \lang{en}{...}
  \lang{de}{Polygon1}
  \lang{zh}{...}
  \lang{fr}{...}
  }
  \begin{description} 
 This work is licensed under the Creative Commons License Attribution 4.0 International (CC-BY 4.0)   
 https://creativecommons.org/licenses/by/4.0/legalcode 

    \lang{en}{...}
    \lang{de}{...}
    \lang{zh}{...}
    \lang{fr}{...}
  \end{description}
  \corrector{system/problem/GenericCorrector.meta.xml}
  \begin{components}
    \component{js_lib}{system/problem/GenericMathlet.meta.xml}{gwtmathlet}
  \end{components}
  \begin{links}
  \end{links}
  \creategeneric
\end{metainfo}
\begin{content}
\begin{block}[annotation]
	Im Ticket-System: \href{https://team.mumie.net/issues/32292}{Ticket 32292}
\end{block}
\begin{block}[annotation]
Copy of : /home/mumie/checkin/content/playground/HM4mint-Testaufgaben/T203_komplexe_Zahlen/prb_jsx_Polygon1.src.tex
\end{block}

\begin{block}[annotation]
(Dagmar Schumacher)\\
Copy of : /home/mumie/checkin/content/playground/Dagmar_Schumacher/prb_jsx-Polygon1.src.tex
\end{block}

  \usepackage{mumie.genericproblem}

\title{Polygon 1 in der komplexen Ebene}
\\

  \begin{visualizationwrapper}
    \begin{genericJSXVisualization}
      \begin{variables}

        \question{1}{r}{real}
        
        \point{A1}{real}{0,r}
        \point{B1}{real}{-r,0}
        \point{C1}{real}{0,-r}
        \point{D1}{real}{r,0}
        \point[editable]{B2}{real}{-r,0.5}
        \point[editable]{C2}{real}{-0.5,-r}
        \point[editable]{A2}{real}{0.5,r}
        \point[editable]{D2}{real}{r,-0.5}
        \segment[editable]{A}{real}{var(A2),var(A1)}
        \segment[editable]{B}{real}{var(B2),var(B1)}
        \segment[editable]{C}{real}{var(C2),var(C1)}
        \segment[editable]{D}{real}{var(D2),var(D1)}

        \number{a2x}{real}{A2[x]}
        \number{a2y}{real}{A2[y]}
        \number{b2x}{real}{B2[x]}
        \number{b2y}{real}{B2[y]}
        \number{c2x}{real}{C2[x]}
        \number{c2y}{real}{C2[y]}
        \number{d2x}{real}{D2[x]}
        \number{d2y}{real}{D2[y]}
      \end{variables}

        \answer{a2x}{1,1}
        \answer{a2y}{1,2}
        \answer{b2x}{1,3}
        \answer{b2y}{1,4}
        \answer{c2x}{1,5}
        \answer{c2y}{1,6}
        \answer{d2x}{1,7}
        \answer{d2y}{1,8}
        
        \color{v1}{#000077} %quite dark blue
        \color{z1}{#000077} %quite dark blue
      
       \begin{canvas}
        \snapToGrid{0.1,0.1}
        \plotSize{500,500}
        \plotLeft{-5.5}
        \plotRight{5.5}
        \plot[coordinateSystem,showPointCoords]{A,B,C,D,A2,B2,C2,D2}
      \end{canvas}
 
    \end{genericJSXVisualization}
  \end{visualizationwrapper}

  \begin{problem}
    \begin{variables}
      \randint{r}{1}{5}
      \number{null}{0}
      \function{mr}{-r}
      
    \end{variables}
    \begin{question}
    \text{Skizzieren Sie in der obigen Grafik das Polygon, das durch 
    $|\Re(z)|+|\Im(z)|=\var{r}$ festgelegt ist, indem Sie die vier blauen Punkte 
    in ihrem jeweiligen Quadranten verschieben.\\ 
    \textit{Starten Sie Ihre Überlegungen am besten im ersten
    Quadranten mit $\Re(z)>0,\Im(z)>0$.}
    }
    \begin{answer} %A2
      \type{graphics.number}
      \solution{r}
     \end{answer}
    \begin{answer}
      \type{graphics.number}
      \solution{null}
     \end{answer} 
     \begin{answer} %B2
      \type{graphics.number}
      \solution{null}
     \end{answer}
    \begin{answer}
      \type{graphics.number}
      \solution{r}
     \end{answer}
     \begin{answer} %C2
      \type{graphics.number}
      \solution{mr}
     \end{answer}
    \begin{answer}
      \type{graphics.number}
      \solution{null}
     \end{answer}  
    \begin{answer} %D2
      \type{graphics.number}
      \solution{null}
     \end{answer}
    \begin{answer}
      \type{graphics.number}
      \solution{mr}
     \end{answer}

    \explanation[NOT[correct(ans_1)] OR NOT[correct(ans_2)]]{Polygonzug im 1. Quadranten fehlerhaft.}
    \explanation[NOT[correct(ans_3)] OR NOT[correct(ans_4)]]{Polygonzug im 2. Quadranten fehlerhaft.}
    \explanation[NOT[correct(ans_5)] OR NOT[correct(ans_6)]]{Polygonzug im 3. Quadranten fehlerhaft.}
    \explanation[NOT[correct(ans_7)] OR NOT[correct(ans_8)]]{Polygonzug im 4. Quadranten fehlerhaft.}
              
   \end{question} 
  \end{problem}
  
  \embedmathlet{gwtmathlet}
  
\end{content}