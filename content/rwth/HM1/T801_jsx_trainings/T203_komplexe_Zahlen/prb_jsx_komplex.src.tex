\documentclass{mumie.problem.gwtmathlet}
%$Id$
\begin{metainfo}
  \name{
  \lang{en}{...}
  \lang{de}{QuadratKomplex}
  \lang{zh}{...}
  \lang{fr}{...}
  }
  \begin{description} 
 This work is licensed under the Creative Commons License Attribution 4.0 International (CC-BY 4.0)   
 https://creativecommons.org/licenses/by/4.0/legalcode 

    \lang{en}{...}
    \lang{de}{...}
    \lang{zh}{...}
    \lang{fr}{...}
  \end{description}
  \corrector{system/problem/GenericCorrector.meta.xml}
   \begin{components}
     \component{js_lib}{system/problem/GenericMathlet.meta.xml}{gwtmathlet}
   \end{components}
  \begin{links}
  \end{links}
  \creategeneric
\end{metainfo}
\begin{content}
\begin{block}[annotation]
	Im Ticket-System: \href{https://team.mumie.net/issues/32299}{Ticket 32299}
\end{block}
\begin{block}[annotation]
Copy of : /home/mumie/checkin/content/playground/HM4mint-Testaufgaben/T203_komplexe_Zahlen/prb_jsx_komplex.src.tex
\end{block}

\begin{block}[annotation]
(Dagmar Schumacher)\\
Copy of : /home/mumie/checkin/content/playground/Dagmar_Schumacher/prb_komplexjsx.src.tex
\end{block}

  \usepackage{mumie.genericproblem}
  
  \begin{visualizationwrapper}

\title{Quadrat einer komplexen Zahl}

\begin{genericJSXVisualization}
  \begin{variables}
      \question{1}{x}{real}
      \question{1}{y}{real}
      %%%
      \point{z1}{real}{x,y}
      \vector{v1}{real}{z1}
      \point[editable]{R}{real}{-2.5,-1}
      %\vector[editable]{v2}{real}{R} 
      %%%
      \number{Rx}{real}{R[x]}
      \number{Ry}{real}{R[y]}
  \end{variables}
  
  \color{v1}{#000077} %quite dark blue
  \color{z1}{#000077} %quite dark blue
  \color{R}{GREEN}
  \label{z1}{z}
  \label{R}{z_quadrat}
  \answer{Rx}{1,1}
  \answer{Ry}{1,2}

  	\begin{canvas}
    \snapToGrid{0.1,0.1}
		\plotSize{300,300}
		\plotLeft{-3.5}
		\plotRight{3.5}
		\plot[coordinateSystem,showPointCoords]{z1,v1,R}
	\end{canvas}

\end{genericJSXVisualization}

\end{visualizationwrapper}

\begin{problem}
  \begin{question}
  \begin{variables}
    \randint{x}{-1}{1}
    \randint[Z]{y}{-1}{1} % y=0 darf nicht sein bei der Berechnung von phi 
    \function[calculate]{px}{(x^2+y^2)*cos(2*acos(x/(x^2+y^2)^0.5)*y/abs(y))}  %r^2cos(2*phi)
    \function[calculate]{py}{(x^2+y^2)*sin(2*acos(x/(x^2+y^2)^0.5)*y/abs(y))}  %r^2sin(2*phi)
    \function[normalize]{z}{x+y*i}   
  \end{variables}
  \correctorprecision{1}
  \precision{1}
    \text{Berechnen Sie graphisch zu $z=\var{z}$ das Quadrat $z^2$, indem Sie 
          den grünen Punkt an die Koordinaten von $z^2$ in der Gaußschen
          Zahlenebene ziehen.}
% \debug[px,py]  
  \explanation{Betrachten Sie die Polarkoordinaten-Darstellung von 
              $z=|z|\cdot(cos(\phi)+i \cdot sin(\phi))$ zur graphischen Bestimmung von $z^2$. 
              Wie verändert sich der Winkel $\phi$ und welche Länge hat $z^2$?
           %  Der Winkel $\phi$ von $z$ muss verdoppelt werden, der Betrag $r$ muss quadriert werden. 
              }
   \begin{answer}
      \type{graphics.number}
      \solution{px}
     \end{answer}
    \begin{answer}
      \type{graphics.number}
      \solution{py}
     \end{answer} 
   \end{question}     
\end{problem}

 \embedmathlet{gwtmathlet}
  
\end{content}