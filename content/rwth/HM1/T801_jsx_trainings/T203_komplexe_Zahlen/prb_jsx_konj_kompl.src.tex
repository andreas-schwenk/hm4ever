\documentclass{mumie.problem.gwtmathlet}
%$Id$
\begin{metainfo}
  \name{
  \lang{en}{...}
  \lang{de}{Konjugiert komplex}
  \lang{zh}{...}
  \lang{fr}{...}
  }
  \begin{description} 
 This work is licensed under the Creative Commons License Attribution 4.0 International (CC-BY 4.0)   
 https://creativecommons.org/licenses/by/4.0/legalcode 

    \lang{en}{...}
    \lang{de}{...}
    \lang{zh}{...}
    \lang{fr}{...}
  \end{description}
  \corrector{system/problem/GenericCorrector.meta.xml}
  \begin{components}
    \component{js_lib}{system/problem/GenericMathlet.meta.xml}{gwtmathlet}
  \end{components}
  \begin{links}
  \end{links}
  \creategeneric
\end{metainfo}
\begin{content}
\begin{block}[annotation]
	Im Ticket-System: \href{https://team.mumie.net/issues/32295}{Ticket 32295}
\end{block}
\begin{block}[annotation]
Copy of : /home/mumie/checkin/content/playground/HM4mint-Testaufgaben/T203_komplexe_Zahlen/prb_jsx_konj_kompl.src.tex
\end{block}

\begin{block}[annotation]
(Dagmar Schumacher)\\
Copy of : /home/mumie/checkin/content/playground/Dagmar_Schumacher/prb_konj_kompl_jsx.src.tex
\end{block}

\begin{block}[annotation]
Copy of : /home/mumie/checkin/content/playground/Dagmar_Schumacher/prb_komplex2jsx.src.tex
\end{block}

  \usepackage{mumie.genericproblem}

  \begin{visualizationwrapper}

\title{konjugiert komplexe Zahl}

\begin{genericJSXVisualization}
  \begin{variables}
      \question{1}{a}{real}
      \question{1}{b}{real}
      %%%
      \point{z}{real}{a,b}
      \label{z}{z}
      \label{R}{z_konjugiert}
      \point[editable]{R}{real}{-2.5,-1}
      %%%
      \number{Rx}{real}{R[x]}
      \number{Ry}{real}{R[y]}

      

  \end{variables}
  
  %\color{v1}{#000077} %quite dark blue
  \color{z}{RED} 
  \color{R}{GREEN}
  \answer{Rx}{1,1}
  \answer{Ry}{1,2}

  	\begin{canvas}
    \snapToGrid{0.1,0.1}
		\plotSize{300,300}
		\plotLeft{-6.5}
		\plotRight{6.5}
		\plot[coordinateSystem,showPointCoords]{z,R}
	\end{canvas}

\end{genericJSXVisualization}

\end{visualizationwrapper}

\begin{problem}
  \begin{variables}
    \randint[Z]{a}{-5}{5}
    \randint[Z]{b}{-5}{5}
    %\function{zquer}{a-b*i}
    \function{re}{a}
    \function{im}{-b}
    \function[normalize]{z}{a+b*i}
  \end{variables}
  
  \begin{question}
  \correctorprecision{1}
  \precision{1}
  
  \text{Zeichnen Sie zu $z=\var{z}$  % $z=\var{a}+(\var{b})\,i$ 
  die konjugiert komplexe Zahl $\bar{z}$ ein, indem Sie den grünen Punkt in der Gaußschen 
  Zahlenebene verschieben.}

  \explanation{Der grüne Punkt liegt nicht exakt in $\bar{z}$.}
  
   \begin{answer}
      \type{graphics.number}
      \solution{re}
     \end{answer}
    \begin{answer}
      \type{graphics.number}
      \solution{im}
     \end{answer} 
%     \explanation{Beim Konjugieren ändert sich nur der Imaginärteil.}
   \end{question} 

   \begin{question}
      \text{Geben Sie den Real- und Imaginärteil von $\bar{z}$ an:\\
      $\text{Re}\,(\bar{z})=$\ansref\\
      $\text{Im}\,(\bar{z})=$\ansref\\
      }
     \begin{answer}
       \type{input.number}
       \solution{re}
     \end{answer}
     \begin{answer}
       \type{input.number}
       \solution{im}
     \end{answer} 
   \explanation[NOT[correct(ans_1)]]{Beim Konjugieren ändert sich nur der Imaginärteil.}
   \explanation[NOT[correct(ans_2)]]{Konjugieren entspricht Spiegelung an der x-Achse.}
%   \explanation[correct(ans_1)]{Haben Sie den grünen Punkt genommen?}
  \end{question}

\end{problem}
  \embedmathlet{gwtmathlet}
  
\end{content}