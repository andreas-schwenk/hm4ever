\documentclass{mumie.problem.gwtmathlet}
%$Id$
\begin{metainfo}
  \name{
    \lang{de}{A02 + JSX: Verbindungsvektoren}
    \lang{en}{problem_2}
  }
  \begin{description} 
 This work is licensed under the Creative Commons License Attribution 4.0 International (CC-BY 4.0)   
 https://creativecommons.org/licenses/by/4.0/legalcode 

  
  \end{description}
  \corrector{system/problem/GenericCorrector.meta.xml}
  \begin{components}
    \component{js_lib}{system/problem/GenericMathlet.meta.xml}{mathlet}
  \end{components}
  \begin{links}
  %  \link{generic_article}{content/rwth/HM1/T108_Vektorrechnung/g_art_content_27_vektoren.meta.xml}{content_27_vektoren}
  \end{links}
  \creategeneric
\end{metainfo}
\begin{content}
\begin{block}[annotation]
	Im Ticket-System: \href{https://team.mumie.net/issues/32491}{Ticket 32491}
\end{block}
\begin{block}[annotation]
Copy of : /home/mumie/checkin/content/playground/HM4mint-Testaufgaben/T108_Vektorrechnung/prb_jsx_Verbindungsvektoren.src.tex
\end{block}


\begin{block}[annotation]
(B. Guyet)\\
Erweiterung der Aufgabe: \\
/home/mumie/checkin/content/rwth/HM1/T108_Vektorrechnung/training/prb_problem_2.src.tex
\\
um einen grafischen Teil aus:\\
/home/mumie/checkin/content/playground/Andreas_Maurischat/grafischeAufgaben/prb_Vektor_A12.2.src.tex
\\
/home/mumie/checkin/content/pool/fh_aachen/MGI/Vektorrechnung_Grundlagen/prb_Vektor_A12.2.src.tex
\end{block}

\usepackage{mumie.ombplus}
\usepackage{mumie.genericproblem}

\lang{de}{\title{A02 + JSX: Verbindungsvektoren}}
\lang{en}{\title{Problem 2}}

\begin{block}[annotation]
	Punkte, Vektoren und Verbindunsvektoren \\
  (a) ist neu, Teil (b) ist (unverändert) Problem 2 aus Kapitel 1.8 Vektorrechnung
\end{block}


\begin{visualizationwrapper}


\begin{genericJSXVisualization}
	\begin{variables}
        \question{1}{px}{rational}
        \question{1}{py}{rational}
        \question{1}{qx}{rational}
        \question{1}{qy}{rational}
        \point{P0}{rational}{0,0}
        \point{Q0}{rational}{0,0}
        \point{P}{rational}{px,py}
        \point{Q}{rational}{qx,qy}
        \affine{vp}{rational}{var(P0),var(P)}
        \affine{vq}{rational}{var(Q0),var(Q)}
        \point[editable]{Pv}{rational}{-3,0}
        \point[editable]{Qv}{rational}{-2,1}
        \affine[editable]{v}{rational}{var(Pv),var(Qv)}
        \number{vx}{rational}{v[x]}
        \number{vy}{rational}{v[y]}
	\end{variables}

\label{P}{P}
\label{Q}{Q}
\label{v}{v}
\color{P}{blue}
\color{Q}{blue}
\color[0.7]{vp}{blue}
\color[0.7]{vq}{blue}
\color{Pv}{RED}
\color{Qv}{RED}
\color{v}{RED}
\answer{vx}{1,1}
\answer{vy}{1,2}

	\begin{canvas}
        \snapToGrid{0.25,0.25}
        \plotSize{400,400}
        \plotLeft{-6.5}
        \plotRight{6.5}
        \plotTop{6.5}
        \plotBottom{-6.5}
        \plot[coordinateSystem]{vp,vq,Pv,Qv,P,Q,v}
	\end{canvas}

\end{genericJSXVisualization}

\end{visualizationwrapper}


\begin{problem}
%Question 1 of 2  Verbindungsvektor im \R^2 grafisch bestimmen
\begin{question} 
          \correctorprecision{2}
          \field{rational}
          \begin{variables}
               \randint{a0}{-8}{8}
               \randint{b0}{-8}{4}
               \randint{c0}{-8}{8}
               \randint{d0}{-4}{8}
               \randadjustIf{a0,c0}{|a0|+|b0|<4 OR |c0|+|d0|<4 OR |b0-d0|+|a0-c0|<4 OR a0*c0<0}
               \function[calculate]{px}{a0/2}
               \function[calculate]{py}{b0/2}
               \function[calculate]{qx}{c0/2}
               \function[calculate]{qy}{d0/2}
               \function[calculate]{vx0}{qx-px}
               \function[calculate]{vy0}{qy-py}
               \pmatrix[calculate]{vv}{vx0\\vy0}
               \pmatrix[calculate]{vvm}{-vx0\\-vy0}
               \function{eins}{1}
               \function{fal1x}{px-qx}
               \function{fal1y}{py-qy}
          \end{variables}
          \lang{de}{
               \text{In der Grafik sind in blau die Punkte $P= (\var{px};\var{py})$ und $Q= (\var{qx};\var{qy})$ 
                 mit ihren Orstvektoren dargestellt. 
                 Bestimmen Sie grafisch den Verbindungsvektor $\vec{PQ}$ von $P$ nach $Q$. Platzieren Sie 
                 hierzu zunächst Anfangs- und Endpunkt des roten Vektors so, dass dieser den Verbindungsvektor 
                 $\vec{PQ}$ darstellt. \\
                 Bestimmen Sie im zweiten Schritt die Koordinaten des Verbindungsvektors $\vec{PQ}$ grafisch. 
                 Verschieben Sie hierzu den roten Vektor $v=\vec{PQ}$ parallel, ohne seine Länge oder Orientierung 
                 zu verändern, so dass sein Anfangspunkt im Koordinatenursprung liegt. Lesen Sie schließlich die 
                 Koordinaten von $\vec{PQ}$ aus dem Endpunkt des verschobenen Vektors $v$ ab. }}        
          
          \begin{answer}
            \type{graphics.number}
            \solution{vx0}
          \end{answer}
          \begin{answer}
            \type{graphics.number}
            \solution{vy0}
          \end{answer}
          \explanation[edited AND [equal(ans_1,eins) AND equal(ans_2,eins)]]{Markieren Sie den Verbindungsvektor $\vec{PQ}$ mit dem roten Vektor.}
          \explanation[equal(ans_1,fal1x) AND equal(ans_2,fal1y)]{Der rote Vektor markiert den Verbindungsvektor von $Q$ nach $P$ statt von $P$ nach $Q$.}
          \explanation[edited AND NOT[equal(ans_1,vx0) AND equal(ans_2,vy0)] AND NOT[equal(ans_1,eins) AND equal(ans_2,eins)] AND NOT[equal(ans_1,fal1x) AND equal(ans_2,fal1y)]]
                      {Der rote Vektor in der Grapfik ist nicht $\vec{PQ}$. Bitte überprüfen Sie,  
                       ob Sie diesen im ersten Schritt korrekt platziert haben.}

          \begin{answer}
            \type{input.matrix}
            \text{$\vec{PQ}=$}
            \solution{vv}
            \format{2}{1}
            \explanation[not edited AND [equal(ans_1,vx0) AND equal(ans_2,vy0)]]{Bitte geben Sie noch die Koordinaten des Vektors $\vec{PQ}$ an.}
            \explanation[equal(ans_3,vvm)]{Die eingegebenen Koordinaten sind die des Verbindungsvektors von $Q$ nach $P$.}
            \explanation[edited AND NOT equal(ans_3,vvm)]{Die Koordinaten des Vektors $\vec{PQ}$ sind falsch.}
          \end{answer}

\end{question}

%Question 2 of 2 Verbindungsvektoren im \R^3 rechnerisch bestimmen
\begin{question} 
	\lang{de}{ 
  		\text{Beantworten Sie folgende Fragen zu den Punkten $P = (\var{Px};\var{Py};\var{Pz})$, $Q = (\var{Qx};\var{Qy};\var{Qz})$ 
        und $R = (\var{Rx};\var{Ry};\var{Rz})$.}

		\explanation{Verbindungsvektoren zwischen zwei Punkten berechnet man nach der Regel
%        \ref[content_27_vektoren][Regel zu Verbindungsvektoren]{Verbindungsvektor}          % Link aus einer Explanation
%                                                                                            % ist derzeit nicht möglich!   
         "Endpunkt minus Anfangspunkt". Das bedeutet also, dass beispielsweise für den Vektor 
          von $Q = (q_1; q_2; q_3)$ nach $R = (r_1; r_2; r_3)$ gilt: 
          $ \overrightarrow{QR}=\left(\begin{smallmatrix} r_1-q_1 \\ r_2-q_2 \\ r_3-q_3 \end{smallmatrix} \right).$
          \\
          Die beiden anderen Verbindungsvektoren berechnet man analog.}     
    }
  	\lang{en}{
  		\text{Answer the following questions about the points $P = (\var{Px},\var{Py},\var{Pz})$, $Q = (\var{Qx},\var{Qy},\var{Qz})$, and $R = (\var{Rx},\var{Ry},\var{Rz})$.}
  		\explanation{The vector from $P = (P_x,P_y,P_z)$ to $Q = (Q_x,Q_y,Q_z)$ is $\vec{PQ} = \begin{pmatrix} Q_x - P_x \\ Q_y - P_y \\ Q_z - P_z\end{pmatrix}$. 
  		The length of a vector $\vec{v}$ is $\abs{\vec{v}} = \sqrt{v_x^2+v_y^2+v_z^2}$.}
  	}
	\begin{variables}
		\randint{Px}{-5}{5}
		\randint{Py}{-5}{5}
		\randint{Pz}{-5}{5}
		\randint{Qx}{-5}{5}
		\randint{Qy}{-5}{5}
		\randint{Qz}{-5}{5}
		\randint{Rx}{-5}{5}
		\randint{Ry}{-5}{5}
		\randint{Rz}{-5}{5}				    		
		\randint[Z]{h}{-1}{1}
		\randint[Z]{f}{-1}{1}
		\randint[Z]{g}{-1}{1}				    		
		\function[calculate]{vx}{h*(Qx-Px)}
		\function[calculate]{vy}{h*(Qy-Py)}
		\function[calculate]{vz}{h*(Qz-Pz)}
		\function[calculate]{lv}{sqrt(vx^2+vy^2+vz^2)}
		\function[calculate]{wx}{f*(Rx-Px)}
		\function[calculate]{wy}{f*(Ry-Py)}
		\function[calculate]{wz}{f*(Rz-Pz)}
		\function[calculate]{lw}{sqrt(wx^2+wy^2+wz^2)}
		\function[calculate]{ux}{g*(Rx-Qx)}
		\function[calculate]{uy}{g*(Ry-Qy)}
		\function[calculate]{uz}{g*(Rz-Qz)}
		\function[calculate]{lu}{sqrt(ux^2+uy^2+uz^2)}
	\end{variables}
	\permutechoices{1}{3}
	\type{mc.yesno}
	\begin{choice}
		\lang{de}{\text{Der Vektor von $P$ nach $Q$ ist gegeben durch $\begin{pmatrix} \var{vx} \\ \var{vy} \\ \var{vz} \end{pmatrix}$.}}
		\lang{en}{\text{The vector from $P$ to $Q$ is given by $\begin{pmatrix} \var{vx} \\ \var{vy} \\ \var{vz} \end{pmatrix}$.}}
		\solution{compute}
		\iscorrect{h}{=}{1}
	\end{choice}
	\begin{choice}
		\lang{de}{\text{Der Vektor von $Q$ nach $R$ ist gegeben durch $\begin{pmatrix} \var{ux} \\ \var{uy} \\ \var{uz} \end{pmatrix}$.}}
		\lang{en}{\text{The vector from $Q$ to $R$ is given by $\begin{pmatrix} \var{ux} \\ \var{uy} \\ \var{uz} \end{pmatrix}$.}}
		\solution{compute}
		\iscorrect{g}{=}{1}
	\end{choice}
	\begin{choice}
		\lang{de}{\text{Der Vektor von $P$ nach $R$ ist gegeben durch $\begin{pmatrix} \var{wx} \\ \var{wy} \\ \var{wz} \end{pmatrix}$.}}
		\lang{en}{\text{The vector from $P$ to $R$ is given by $\begin{pmatrix} \var{wx} \\ \var{wy} \\ \var{wz} \end{pmatrix}$.}}
		\solution{compute}
		\iscorrect{f}{=}{1}
	\end{choice}
\end{question}

\end{problem}

\embedmathlet{mathlet}

\end{content}