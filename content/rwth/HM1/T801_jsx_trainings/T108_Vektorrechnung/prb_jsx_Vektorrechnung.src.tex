\documentclass{mumie.problem.gwtmathlet}
%$Id$
\begin{metainfo}
  \name{
    \lang{en}{}
    \lang{de}{Grafische Bestimmung von Vektorsummen}
  }
  \begin{description} 
 This work is licensed under the Creative Commons License Attribution 4.0 International (CC-BY 4.0)   
 https://creativecommons.org/licenses/by/4.0/legalcode 

    \lang{en}{...}
    \lang{de}{...}
  \end{description}
  \corrector{system/problem/GenericCorrector.meta.xml}
  \begin{components}
    \component{js_lib}{system/problem/GenericMathlet.meta.xml}{gwtmathlet}
  \end{components}
  \begin{links}
  \end{links}
  \creategeneric
  \begin{taxonomy}
  \end{taxonomy}
\end{metainfo}
\begin{content}
\begin{block}[annotation]
	Im Ticket-System: \href{https://team.mumie.net/issues/32492}{Ticket 32492}
\end{block}
\begin{block}[annotation]
Copy of : /home/mumie/checkin/content/playground/HM4mint-Testaufgaben/T108_Vektorrechnung/prb_jsx_Vektorrechnung.src.tex
\end{block}

\begin{block}[annotation]
(B. Guyet)\\
Copy of : /home/mumie/checkin/content/playground/BarbaraGuyet/JSX-Aufgaben/prb_jsx_Vektorrechnung01.src.tex
\end{block}


\begin{block}[annotation]
 Bestimmung des Mittelpunkts und der Verlängerung einer Strecke mittels Vektorrechnung
\end{block}

\usepackage{mumie.genericproblem}
\lang{de}{
\title{Vektorrechnung zur Bestimmung des Mittelpunkts einer Strecke}
}

\begin{visualizationwrapper}


\begin{genericJSXVisualization}
	\begin{variables}
       % variables fetched from question 1
        \question{1}{a}{rational}
        \question{1}{b}{rational}
        \question{1}{c}{rational}
        \question{1}{d}{rational}

       % 
        \point{P}{rational}{a,b}
        \point{Q}{rational}{c,d}
        \vector{vp}{rational}{var(P)}
        \vector{vq}{rational}{var(Q)}
%        \affine{vpq}{rational}{var(P),var(Q)}
        \segment{vpq}{real}{var(P),var(Q)}
        
      % bewegliche Vektoren zum Rechnen
        \point[editable]{A1}{rational}{-5,-6}
        \point[editable]{A2}{rational}{-4,-5}
        \point[editable]{B1}{rational}{-6,-6}
        \point[editable]{B2}{rational}{-5,-5}
        \affine[editable]{va}{rational}{var(A1),var(A2)}
        \affine[editable]{vb}{rational}{var(B1),var(B2)}

       % Ergebnis-Vektoren
        \point[editable]{M}{rational}{-1,2}
        \vector[editable]{vm}{rational}{var(M)}
        \number{mx}{rational}{M[x]}
        \number{my}{rational}{M[y]}
        \point[editable]{S}{rational}{-2,1}
        \vector[editable]{vs}{rational}{var(S)}
        \number{sx}{rational}{S[x]}
        \number{sy}{rational}{S[y]}

	\end{variables}

\label{P0}{P0}
\label{Q0}{Q0}
\label{P}{P}
\label{Q}{Q}
%\label{vp}{v}
%\label{vq}{w}
\color{P}{BLUE}
\color{Q}{BLUE}
\color[0.7]{vp}{BLUE}
\color[0.7]{vq}{BLUE}

\color[0.7]{A1}{GREEN}
\color[0.7]{A2}{GREEN}
\color[0.7]{B1}{GREEN}
\color[0.7]{B2}{GREEN}
\color[0.7]{va}{GREEN}
\color[0.7]{vb}{GREEN}

% Ergebnis-Vektoren
\label{M}{M}
%\label{vm}{m}
\color{M}{RED}
\color[0.7]{vm}{RED}
\answer{mx}{1,1}  % mx is used as solution for question 1, answer 1
\answer{my}{1,2}  % my is used as solution for question 1, answer 2
\label{S}{S}
%\label{vs}{s}
\color{S}{RED}
\color[0.7]{vs}{RED}
\answer{sx}{1,3}  % sx is used as solution for question 1, answer 3
\answer{sy}{1,4}  % sy is used as solution for question 1, answer 4

	\begin{canvas}
        \snapToGrid{0.25,0.25}
        \plotSize{400,400}
        \plotLeft{-6.5}
        \plotRight{6.5}
        \plotTop{6.5}
        \plotBottom{-6.5}
        \plot[coordinateSystem]{vpq,vp,vq,P,Q,A1,A2,B1,B2,va,vb,M,vm,S,vs}
	\end{canvas}

\end{genericJSXVisualization}

\end{visualizationwrapper}

     \begin{problem} 

          \begin{question}
          \correctorprecision{2}
          \field{rational}
          \begin{variables}
               \randint{a}{-5}{5}
               \randint{b}{-4}{1}
               \randint{c}{-5}{5}
               \randint{d}{-1}{4}
               \randadjustIf{a,c,d}{a=c OR b=d OR |a|+|b|<2 OR |c|+|d|<2 OR |b+d|+|a+c|<4 OR |b-d|+|a-c|<4 OR |b*c-a*d|<2}
               \function{mx}{1/2*(a+c)}  % mx=px+1/2(qx-px)
               \function{my}{1/2*(b+d)}  % my=py+1/2(qy-py)
               \function{sx}{mx+c-a}     % sx=px+3/2(qx-px)=mx+(qx-px)
               \function{sy}{my+d-b}     % sy=py+3/2(qy-py)=my+(qy-py)
          \end{variables}

               \text{Gegeben seien die Punkte $P=(\var{a};\var{b})$ und $Q=(\var{c};\var{d})$. 
                     Bestimmen Sie grafisch den Punkt $M$, der die Strecke $\bar{PQ}$ halbiert. 
                     Bestimmen Sie anschließend den Punkt $S$ so, dass die Strecke $\bar{PS}$ 
                     durch die Punkte $M$ und $Q$ jeweils gedrittelt wird. \\\\
                    \textit{Tipp: Nutzen Sie die Regeln der Vektorrechnung und verwenden Sie als
                    Hilfsmittel die grünen Verktoren, die in Lage, Länge und Richtung beliebig 
                    veränderbar sind.}
                    }
%    
%\debug[a,b,c,d,mx,my,sx,sy]
% 
          \begin{answer}
            \type{graphics.number}
            \solution{mx}
          \end{answer}
          \begin{answer}
            \type{graphics.number}
            \solution{my}
          \end{answer}
          \begin{answer}
            \type{graphics.number}
            \solution{sx}
          \end{answer}
          \begin{answer}
            \type{graphics.number}
            \solution{sy}
          \end{answer}

          \explanation[Not [edited]]{Ziehen Sie die roten Punkts $M$ und $S$ jeweils an die grafisch 
                bestimmte Position im Koordinatensystem.}
          \explanation[edited AND [ans_1 != mx OR ans_2 != my]]{Bilden Sie zur Bestimmung von $M$ 
                die Summe $\vec{OP} + \vec{OQ}$. Verwenden Sie die grünen Hilfsvektoren, um 
                die Vektoraddition grafisch darzustellen. Es entsteht ein Parallelogramm mit dem  
                Summenvektor und der Strecke $\bar{PQ}$ als Diagonalen, die einander bekanntermaßen halbieren.}
          \explanation[edited AND equal(mx,ans_1) AND  equal(my,ans_2) AND[ans_3 != sx OR ans_4 != sy]]{Den
                Punkt $S$ erreichen Sie grafisch zum Beispiel über die Summe $\vec{OM} + \vec{PQ}$ oder über
                die Summe $\vec{OP} + \vec{PQ} + \vec{PM}$. Verwenden Sie die grünen Hilfsvektoren, 
                um die Vektoraddition grafisch darzustellen.}
    \end{question}
   \end{problem}

\embedmathlet{gwtmathlet}

\end{content}
