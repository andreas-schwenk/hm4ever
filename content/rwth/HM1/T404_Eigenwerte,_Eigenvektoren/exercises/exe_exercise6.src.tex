\documentclass{mumie.element.exercise}
%$Id$
\begin{metainfo}
  \name{
    \lang{en}{Ex06: Principal axes}
    \lang{de}{Ü06: Hauptachsen}
  }
  \begin{description} 
 This work is licensed under the Creative Commons License Attribution 4.0 International (CC-BY 4.0)   
 https://creativecommons.org/licenses/by/4.0/legalcode 

    \lang{en}{...}
    \lang{de}{...}
  \end{description}
  \begin{components}
  \end{components}
  \begin{links}
  \end{links}
  \creategeneric
\end{metainfo}
\begin{content}
\title{\lang{de}{Ü06: Hauptachsen}\lang{en}{Ex06: Principal axes}}
\begin{block}[annotation]
	Im Ticket-System: \href{https://team.mumie.net/issues/22656}{Ticket 22656}
\end{block}
\lang{de}{
Gegeben ist die quadratische Gleichung
\[7x^2+8xy+y^2=1,\]
die einen Kegelschnitt im $\R^2$ beschreibt.
Bestimmen Sie die symmetrische Matrix $A\in\M(2;\R)$ so, dass
\[7x^2+8xy+y^2=1\:\Leftrightarrow\: \begin{pmatrix} x&y\end{pmatrix}\cdot A\cdot\begin{pmatrix} x\\y\end{pmatrix}=1.\]
Transformieren Sie diese Gleichung auf Hauptachsenform und bestimmen Sie die Richtungen der Hauptachsen.
Welche Art Kegelschnitt wird beschrieben?}

\lang{en}{
Given is the quadratic equation
\[7x^2+8xy+y^2=1,\]
which describes a conic section in $\R^2$.
Determine the symmetric matrix $A\in\M(2;\R)$ such, that
\[7x^2+8xy+y^2=1\:\Leftrightarrow\: \begin{pmatrix} x&y\end{pmatrix}\cdot A\cdot\begin{pmatrix} x\\y\end{pmatrix}=1.\]
Transform this equation into principal axes form and determine the directions of the principal axes.
What type of conic section is described?}

%##################################################ANTWORTEN_TEXT
\begin{tabs*}[\initialtab{0}\class{exercise}]
%++++++++++++++++++++++++++++++++++++++++++START_TAB_X
  \tab{\lang{de}{   Lösung    }\lang{en}{   Solution    }}
For determining the symmetric matrix $A=\left(\begin{smallmatrix}a_{11}&a_{12}\\a_{12}&a_{22}\end{smallmatrix}\right)$,
we expand the right equation (by multiplying)
\[
1=\begin{pmatrix} x&y\end{pmatrix}\cdot A\cdot\begin{pmatrix} x\\y\end{pmatrix}\\
=a_{11}x^2+2a_{12}xy+a_{22}y^2.
\]
The result of a coefficient comparision with the first equation is
$a_{11}=7$, $a_{12}=\frac{8}{2}=4$, $a_{22}=1$.
So the matrix $A$ is
\[A=\begin{pmatrix} 7&4\\4&1\end{pmatrix}.\]
For the principal axes transformation we need the eigenvalues of the matrix. So we determine its characteristic
polynomial
\[\det\big(\begin{pmatrix}7-t&4\\4&1-t\end{pmatrix}\big)=t^2-8t-9.\]
The zeros of this characteristic polynomials are $\lambda_1=9$ and $\lambda_2=-1$.
(We may use the$pq$-formula for finding those, if we do not see them right away.)
Those are the eigenvalues of $A$. So the principal axes form is
\[1=\begin{pmatrix} x&y\end{pmatrix}\cdot\begin{pmatrix} 9&0\\0&-1\end{pmatrix}\cdot\begin{pmatrix} x\\y\end{pmatrix}=9x^2-y^2\]
with respect of the basis $B$ consisting of the normalised eigenvectors $v_1$ and $v_2$ for the eigenvalues $\lambda_1$ respectively $\lambda_2$.\\
The numbering of the eigenvalues is random. Likewise 
$1=\begin{pmatrix} x&y\end{pmatrix}\cdot\begin{pmatrix} -1&0\\0&9\end{pmatrix}\cdot\begin{pmatrix} x\\y\end{pmatrix}=-x^2+9y^2$
is a principal axes form, for the transformed basis $\tilde{B}$, so $v_2$, $v_1$.
\\
Because one eigenvalues is positive and the other is negative, the quadratic equation described a hyperbola.
\\
The directions of the principal axes are determined by the eigenvectors $v_1$ and $v_2$. So let us calculate those.
\[Av_1=\lambda_1v_1\:\Leftrightarrow\:\begin{pmatrix}-2&4\\4&-8\end{pmatrix}v_1=\begin{pmatrix}0\\0\end{pmatrix}\:
\Leftrightarrow v_1\in\R\cdot \begin{pmatrix}2\\1\end{pmatrix},\]
as well as
\[Av_2=\lambda_2v_2\:\Leftrightarrow\:\begin{pmatrix}8&4\\4&2\end{pmatrix}v_2=\begin{pmatrix}0\\0\end{pmatrix}\:
\Leftrightarrow v_2\in\R\cdot \begin{pmatrix}1\\-2\end{pmatrix}.\]
If we want to give normalised eigenvectors, so $|\!| v_j|\!|=1$,then the eigenvectors are unambiguous (exept for the sign)
\[v_1=\frac{\pm 1}{\sqrt{5}}\begin{pmatrix}2\\1\end{pmatrix} \quad\text{ und }\quad v_2=\frac{\pm 1}{\sqrt{5}}\begin{pmatrix}1\\-2\end{pmatrix}.\]
 
\end{tabs*}
\end{content}

