\documentclass{mumie.element.exercise}
%$Id$
\begin{metainfo}
  \name{
    \lang{de}{Ü01: Eigenvektor}
    \lang{en}{Ex01: Eigenvector}
  }
  \begin{description} 
 This work is licensed under the Creative Commons License Attribution 4.0 International (CC-BY 4.0)   
 https://creativecommons.org/licenses/by/4.0/legalcode 

    \lang{de}{}
    \lang{en}{}
  \end{description}
  \begin{components}
  \end{components}
  \begin{links}
  \end{links}
  \creategeneric
\end{metainfo}
\begin{content}
\usepackage{mumie.ombplus}

\title{\lang{de}{Ü01: Eigenvektor} lang{en}{Ex01: Eigenvector}}

\begin{block}[annotation]
  Im Ticket-System: \href{http://team.mumie.net/issues/11621}{Ticket 11621}
\end{block}

%######################################################FRAGE_TEXT
\lang{de}{ Gegeben sind die folgenden reellen $(3\times 3)$-Matrizen
\[ \begin{mtable}[\cellaligns{llll}]
A & =\begin{pmatrix}2 & 0 & -2\\ 5 & 4& -1\\ 1 & 2 & -2\end{pmatrix}, 
\qquad & B&= \begin{pmatrix} -3 &2& 2\\ 2&4 & 0\\ 4&4& -1 \end{pmatrix},  \\
C & =\begin{pmatrix} 1 & 6& 2\\ -2&9 & 3 \\ 4&2& -1 \end{pmatrix} ,
 \qquad & D&= \begin{pmatrix}1 & 4& 1\\ 2& 5& 0\\  3& -2 & -1 \end{pmatrix}.
\end{mtable} \]
Zu welchen dieser Matrizen ist der Vektor $v=\left(\begin{smallmatrix}
2 \\ -1\\ 4 \end{smallmatrix}\right)$ ein Eigenvektor? 

Was sind die zugehörigen Eigenwerte? }


\lang{en}{ Given are the following $(3\times 3)$-matrices
\[ \begin{mtable}[\cellaligns{llll}]
A & =\begin{pmatrix}2 & 0 & -2\\ 5 & 4& -1\\ 1 & 2 & -2\end{pmatrix}, 
\qquad & B&= \begin{pmatrix} -3 &2& 2\\ 2&4 & 0\\ 4&4& -1 \end{pmatrix},  \\
C & =\begin{pmatrix} 1 & 6& 2\\ -2&9 & 3 \\ 4&2& -1 \end{pmatrix} ,
 \qquad & D&= \begin{pmatrix}1 & 4& 1\\ 2& 5& 0\\  3& -2 & -1 \end{pmatrix}.
\end{mtable} \]
For which of the matrices is the vector $v=\left(\begin{smallmatrix}
2 \\ -1\\ 4 \end{smallmatrix}\right)$ an eigenvector? 

What are the corresponding eigenvalues? }

%##################################################ANTWORTEN_TEXT
\begin{tabs*}[\initialtab{0}\class{exercise}]

%++++++++++++++++++++++++++++++++++++++++++START_TAB_X
  \tab{\lang{de}{    Antwort    }  \lang{en}{    Answer   }}
  \begin{incremental}[\initialsteps{1}]
  
  	 %----------------------------------START_STEP_X
    \step 
    \lang{de}{   Der Vektor $v$ ist
\begin{itemize}
\item Eigenvektor von $A$ zum Eigenwert $\lambda=-2$,
\item Eigenvektor von $B$ zum Eigenwert $\lambda=0$,
\item kein Eigenvektor von $C$ und
\item Eigenvektor von $D$ zum Eigenwert $\lambda=1$.
\end{itemize}    }
\lang{en}{   The vector $v$ is
\begin{itemize}
\item eigenvector of $A$ for the eigenvalue $\lambda=-2$,
\item eigenvector of $B$ for the eigenvalue $\lambda=0$,
\item no eigenvector of $C$ and
\item eigenvector of $D$ for the eigenvalue $\lambda=1$.
\end{itemize}    }
  	 %------------------------------------END_STEP_X
 
  \end{incremental}
  %++++++++++++++++++++++++++++++++++++++++++++END_TAB_X
  
  
  %++++++++++++++++++++++++++++++++++++++++++START_TAB_X
  \tab{\lang{de}{    Lösung zu $A$    }\lang{en}{    Solution for $A$    }}
  \begin{incremental}[\initialsteps{1}]
  
  	 %----------------------------------START_STEP_X
    \step 
    \lang{de}{   Der Vektor $v$ ist ein Eigenvektor von $A$, wenn $A\cdot v$ ein Vielfaches von $v$ ist, also
\[  A\cdot v= \begin{pmatrix}2 & 0 & -2\\ 5 & 4& -1\\ 1 & 2 & -2\end{pmatrix}\cdot \begin{pmatrix}
2 \\ -1\\ 4 \end{pmatrix} =\begin{pmatrix} -4\\ 2 \\-8\end{pmatrix} =-2\cdot \begin{pmatrix}
2 \\ -1\\ 4 \end{pmatrix}. \]
Also ist $v$ Eigenvektor von $A$, denn $v$ wird bei Multiplikation mit $A$ um den Faktor $\lambda=-2$ skaliert.
Dieser Faktor ist der zugehörige Eigenwert $\lambda=-2$.    }
\lang{en}{   The vector $v$ is an eigenvector of $A$, if $A\cdot v$ is a multiplie of $v$, so
\[  A\cdot v= \begin{pmatrix}2 & 0 & -2\\ 5 & 4& -1\\ 1 & 2 & -2\end{pmatrix}\cdot \begin{pmatrix}
2 \\ -1\\ 4 \end{pmatrix} =\begin{pmatrix} -4\\ 2 \\-8\end{pmatrix} =-2\cdot \begin{pmatrix}
2 \\ -1\\ 4 \end{pmatrix}. \]
So $v$ is eigenvector of $A$, because the multiplication with $A$ scales $v$ with the factor $\lambda=-2$.
This factor is the corresponding eigenvalue $\lambda=-2$.    }
  	 %------------------------------------END_STEP_X
 
  \end{incremental}
  %++++++++++++++++++++++++++++++++++++++++++++END_TAB_X


  %++++++++++++++++++++++++++++++++++++++++++START_TAB_X
  \tab{\lang{de}{    Lösung zu $B$    } \lang{en}{    Solution of $B$    }}
  \begin{incremental}[\initialsteps{1}]
  
  	 %----------------------------------START_STEP_X
    \step 
    \lang{de}{   Um die Frage zu beantworten, müssen wir auch hier das Produkt $B\cdot v$ berechnen, also
\[ B\cdot v=  \begin{pmatrix} -3 &2& 2\\ 2&4 & 0\\ 4&4& -1 \end{pmatrix}\cdot \begin{pmatrix}
2 \\ -1\\ 4 \end{pmatrix} = \begin{pmatrix} 0\\ 0 \\ 0\end{pmatrix}=0\cdot  \begin{pmatrix}
2 \\ -1\\ 4 \end{pmatrix}. \]
Also ist $v$ Eigenvektor von $B$ und der zugehörige Eigenwert ist $\lambda=0$.     }
\lang{en}{   To answer the question, we must calculate $B\cdot v$, so
\[ B\cdot v=  \begin{pmatrix} -3 &2& 2\\ 2&4 & 0\\ 4&4& -1 \end{pmatrix}\cdot \begin{pmatrix}
2 \\ -1\\ 4 \end{pmatrix} = \begin{pmatrix} 0\\ 0 \\ 0\end{pmatrix}=0\cdot  \begin{pmatrix}
2 \\ -1\\ 4 \end{pmatrix}. \]
So $v$ is eigenvector of $B$ and the corresponding eigenvalue is $\lambda=0$.     }
  	 %------------------------------------END_STEP_X
 
  \end{incremental}
  %++++++++++++++++++++++++++++++++++++++++++++END_TAB_X


  %++++++++++++++++++++++++++++++++++++++++++START_TAB_X
  \tab{\lang{de}{    Lösung zu $C$    } \lang{en}{    Solution for $C$    }}
  \begin{incremental}[\initialsteps{1}]
  
  	 %----------------------------------START_STEP_X
    \step 
    \lang{de}{   Wir berechnen wieder das Produkt $C\cdot v$, also
\[ C\cdot v =\begin{pmatrix} 1 & 6& 2\\ -2&9 & 3 \\ 4&2& -1 \end{pmatrix}\cdot \begin{pmatrix}
2 \\ -1\\ 4 \end{pmatrix} = \begin{pmatrix} 4\\ -1\\ 2\end{pmatrix}. \]
Auch wenn dieser Vektor Ähnlichkeit mit dem Vektor $v$ hat, nämlich im Vergleich zu ihm nur in der ersten und dritten Komponente vertauscht ist, ist er dennoch kein Vielfaches von $v$.\\

Also ist der Vektor $v$ kein Eigenvektor von $C$.
    }
    \lang{en}{   We calculate again the product $C\cdot v$, so
\[ C\cdot v =\begin{pmatrix} 1 & 6& 2\\ -2&9 & 3 \\ 4&2& -1 \end{pmatrix}\cdot \begin{pmatrix}
2 \\ -1\\ 4 \end{pmatrix} = \begin{pmatrix} 4\\ -1\\ 2\end{pmatrix}. \]
This vector may look simular to the vector $v$, only the first and the third component are switched, but it is still no multiple of $v$.\\

So the vector $v$ is no eigenvector of $C$.
    }
  	 %------------------------------------END_STEP_X
 
  \end{incremental}
  %++++++++++++++++++++++++++++++++++++++++++++END_TAB_X
  %++++++++++++++++++++++++++++++++++++++++++START_TAB_X
  \tab{\lang{de}{    Lösung zu $D$    }  \lang{en}{    Solution for $D$    }}
  \begin{incremental}[\initialsteps{1}]
  
  	 %----------------------------------START_STEP_X
    \step 
    \lang{de}{  
Wir berechnen wieder das Produkt $D\cdot v$, also
\[ D\cdot v=\begin{pmatrix}1 & 4& 1\\ 2& 5& 0\\  3& -2 & -1 \end{pmatrix}\cdot \begin{pmatrix}
2 \\ -1\\ 4 \end{pmatrix} = \begin{pmatrix} 2 \\ -1\\ 4 \end{pmatrix}=1\cdot \begin{pmatrix} 2 \\ -1\\ 4 \end{pmatrix}. \]

Also ist $v$ Eigenvektor von $D$ und der zugehörige Eigenwert ist $\lambda=1$. 
    }

    \lang{en}{  
We calculate again the product $D\cdot v$, so
\[ D\cdot v=\begin{pmatrix}1 & 4& 1\\ 2& 5& 0\\  3& -2 & -1 \end{pmatrix}\cdot \begin{pmatrix}
2 \\ -1\\ 4 \end{pmatrix} = \begin{pmatrix} 2 \\ -1\\ 4 \end{pmatrix}=1\cdot \begin{pmatrix} 2 \\ -1\\ 4 \end{pmatrix}. \]

The vector $v$ is eigenvector of $D$ and the corresponding eigenvalue is $\lambda=1$. 
    }
  	 %------------------------------------END_STEP_X
 
  \end{incremental}
  %++++++++++++++++++++++++++++++++++++++++++++END_TAB_X

%#############################################################ENDE
\end{tabs*}
\end{content}