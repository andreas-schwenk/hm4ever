\documentclass{mumie.element.exercise}
%$Id$
\begin{metainfo}
  \name{
    \lang{de}{Ü03: Eigenwert, -vektoren}
    \lang{en}{Ex03: Eigenvalues, eigenvectors}
  }
  \begin{description} 
 This work is licensed under the Creative Commons License Attribution 4.0 International (CC-BY 4.0)   
 https://creativecommons.org/licenses/by/4.0/legalcode 

    \lang{de}{}
    \lang{en}{}
  \end{description}
  \begin{components}
  \end{components}
  \begin{links}
  \end{links}
  \creategeneric
\end{metainfo}
\begin{content}
\usepackage{mumie.ombplus}

\title{\lang{de}{Ü03: Eigenwert, -vektoren}\lang{en}{Ex03: Eigenvalues, eigenvectors}}

\begin{block}[annotation]
  Im Ticket-System: \href{http://team.mumie.net/issues/11623}{Ticket 11623}
\end{block}

%######################################################FRAGE_TEXT
\lang{de}{ 
Wir betrachten die Matrix
\[ A=\begin{pmatrix}
        -2  & 5  &  -3  \\   
        1  &  -1  &  0  \\
         3  & -6  & 3  \\
     \end{pmatrix}\in M(3;\R). \]
Bestimmen Sie alle Eigenwerte von $A$ und die zugehörigen Eigenvektoren.
}

\lang{en}{ 
We consider the matrix
\[ A=\begin{pmatrix}
        -2  & 5  &  -3  \\   
        1  &  -1  &  0  \\
         3  & -6  & 3  \\
     \end{pmatrix}\in M(3;\R). \]
Determine all eigenvalues of $A$ and the corresponding eigenvectors.
}

%##################################################ANTWORTEN_TEXT
\begin{tabs*}[\initialtab{0}\class{exercise}]
%++++++++++++++++++++++++++++++++++++++++++START_TAB_X
  \tab{\lang{de}{    Antwort  }\lang{en}{    Answer  }}
  \begin{incremental}[\initialsteps{1}]
  
  	 %----------------------------------START_STEP_X
    \step 
    \lang{de}{

Die Eigenwerte von $A$ sind $0$, $\sqrt{3}$ und $-\sqrt{3}$.

Die Eigenvektoren zu $0$ sind $r\cdot \left( \begin{smallmatrix}
1 \\ 1 \\ 1
\end{smallmatrix}\right)$ mit $r\in \R\setminus \{0\}$.

Die Eigenvektoren zu $\sqrt{3}$ sind $r\cdot \left( \begin{smallmatrix}
-1-\frac{1}{3}\sqrt{3} \\ -\frac{1}{3}\sqrt{3} \\ 1
\end{smallmatrix}\right)$ mit $r\in \R\setminus \{0\}$.

Die Eigenvektoren zu $-\sqrt{3}$ sind $r\cdot \left( \begin{smallmatrix}
-1+\frac{1}{3}\sqrt{3} \\ +\frac{1}{3}\sqrt{3} \\ 1
\end{smallmatrix}\right)$ mit $r\in \R\setminus \{0\}$. 
  }

\lang{en}{

The eigenvalues of $A$ are $0$, $\sqrt{3}$ and $-\sqrt{3}$.

The eigenvectors for $0$ are $r\cdot \left( \begin{smallmatrix}
1 \\ 1 \\ 1
\end{smallmatrix}\right)$ with $r\in \R\setminus \{0\}$.

The eigenvectors for $\sqrt{3}$ are $r\cdot \left( \begin{smallmatrix}
-1-\frac{1}{3}\sqrt{3} \\ -\frac{1}{3}\sqrt{3} \\ 1
\end{smallmatrix}\right)$ with $r\in \R\setminus \{0\}$.

The eigenvectors for $-\sqrt{3}$ are $r\cdot \left( \begin{smallmatrix}
-1+\frac{1}{3}\sqrt{3} \\ +\frac{1}{3}\sqrt{3} \\ 1
\end{smallmatrix}\right)$ with $r\in \R\setminus \{0\}$. 
  }
  	 %------------------------------------END_STEP_X
 
  \end{incremental}
  %++++++++++++++++++++++++++++++++++++++++++++END_TAB_X
  
%++++++++++++++++++++++++++++++++++++++++++START_TAB_X
  \tab{\lang{de}{    Lösung zu Eigenwerte  }\lang{en}{    Solution for the eigenvalues  }}
  \begin{incremental}[\initialsteps{1}]
  
  	 %----------------------------------START_STEP_X
    \step 
    \lang{de}{   Da die Eigenwerte genau die Nullstellen des charakteristischen Polynoms von $A$ sind, müssen wir zunächst dieses bestimmen:
\begin{eqnarray*}
p_A(t)&=&\det(A-t\cdot E_3)=
\det \Big( \begin{pmatrix}
-2-t & 5 & -3\\
1 & -1-t & 0 \\
3 & -6 & 3-t \\
\end{pmatrix} \Big) \\
& =&(-2-t)(-1-t)(3-t)+0+18-(-9)(-1-t)-0-5(3-t)\\
& =&(t^2+3t+2)(3-t)+18-9-9t-15+5t\\
& =&3t^2+9t+6-t^3-3t^2-2t-6-4t\\
& =&-t^3+3t = -t(t^2-3)\\ &=& -t(t-\sqrt{3})(t+\sqrt{3})
\end{eqnarray*}

Die Nullstellen von $p_A(t)$ und damit die Eigenwerte von $A$ sind also
\[ \lambda_1=0, \quad \lambda_2=\sqrt{3}, \quad \lambda_3=-\sqrt{3}.  \]    }

\lang{en}{ Since the eigenvalues are exactly the zeros of the characteristic polynomial of $A$,
we need to determine that:
\begin{eqnarray*}
p_A(t)&=&\det(A-t\cdot I_3)=
\det \Big( \begin{pmatrix}
-2-t & 5 & -3\\
1 & -1-t & 0 \\
3 & -6 & 3-t \\
\end{pmatrix} \Big) \\
& =&(-2-t)(-1-t)(3-t)+0+18-(-9)(-1-t)-0-5(3-t)\\
& =&(t^2+3t+2)(3-t)+18-9-9t-15+5t\\
& =&3t^2+9t+6-t^3-3t^2-2t-6-4t\\
& =&-t^3+3t = -t(t^2-3)\\ &=& -t(t-\sqrt{3})(t+\sqrt{3})
\end{eqnarray*}

The zeros of $p_A(t)$ and therefore the eigenvalues of $A$ are
\[ \lambda_1=0, \quad \lambda_2=\sqrt{3}, \quad \lambda_3=-\sqrt{3}.  \]    }
  	 %------------------------------------END_STEP_X
 
  \end{incremental}
  %++++++++++++++++++++++++++++++++++++++++++++END_TAB_X
  
  
  %++++++++++++++++++++++++++++++++++++++++++START_TAB_X
  \tab{\lang{de}{    Lösung zu Eigenvektoren    }\lang{en}{    Solution for the eigenvectors    }}
  \begin{incremental}[\initialsteps{1}]
  
  	 %----------------------------------START_STEP_X
    \step 
    \lang{de}{   Die Eigenvektoren zu einem Eigenwert $\lambda$ sind ja genau die nicht-trivialen Lösungen des Gleichungssystems
\[  (A-\lambda E_3)\cdot x=0.\]

Für $\lambda_1=0$ berechnet man (wobei wir die rechte Seite, welche ja immer gleich $0$ ist, weglassen)

\begin{eqnarray*}
& &
\begin{pmatrix}
-2 & 5 & -3\\
1 & -1 & 0 \\
3 & -6 & 3 
\end{pmatrix}  \begin{matrix} \updownarrow \\ \phantom{1} \\ \end{matrix} \\

&\rightsquigarrow&
\begin{pmatrix}
1 & -1 & 0 \\
-2 & 5 & -3\\
3 & -6 & 3 
\end{pmatrix}  \begin{matrix} \phantom{1}\\  /  +2\cdot \text{(I)}\\  /  -1\cdot \text{(I)} \end{matrix} \\
&\rightsquigarrow&
\begin{pmatrix}
1 & -1 & 0 \\
0 & 3 & -3\\
0 & -1 & 1 
\end{pmatrix} \begin{matrix} \phantom{1}\\  \phantom{1} \\  /  +\frac{1}{3}\cdot \text{(I)} \end{matrix} \\
&\rightsquigarrow&
\begin{pmatrix}
1 & -1 & 0 \\
0 & 3 & -3\\
0 & 0 & 0
\end{pmatrix}\begin{matrix}  \phantom{1} \\ / \cdot \frac{1}{3} \\  \phantom{1}  \end{matrix} \\
&\rightsquigarrow&
\begin{pmatrix}
1 & -1 & 0 \\
0 &  1 & -1\\
0 & 0 & 0
\end{pmatrix} \begin{matrix}/  +1\cdot \text{(II)}\\  \phantom{1} \\  \phantom{1} \end{matrix} \\
&\rightsquigarrow&
\begin{pmatrix}
1 & 0 & -1 \\
0 & 1 & -1\\
0 & 0 & 0
\end{pmatrix}
\end{eqnarray*}
Die Eigenvektoren zum Eigenwert $0$ sind also die
Vektoren
\[  r\cdot  \begin{pmatrix}
1\\ 1 \\ 1
\end{pmatrix}\quad \text{mit }r\in \R\setminus { \{0\} }. \]
    }


\lang{en}{   The eigenvectors for an eigenvalue $\lambda$ are the nontrivial solutions of the linear system
\[  (A-\lambda I_3)\cdot x=0.\]

For $\lambda_1=0$ we calculate (we omit the right side, since it is always equal to $0$)

\begin{eqnarray*}
& &
\begin{pmatrix}
-2 & 5 & -3\\
1 & -1 & 0 \\
3 & -6 & 3 
\end{pmatrix}  \begin{matrix} \updownarrow \\ \phantom{1} \\ \end{matrix} \\

&\rightsquigarrow&
\begin{pmatrix}
1 & -1 & 0 \\
-2 & 5 & -3\\
3 & -6 & 3 
\end{pmatrix}  \begin{matrix} \phantom{1}\\  /  +2\cdot \text{(I)}\\  /  -1\cdot \text{(I)} \end{matrix} \\
&\rightsquigarrow&
\begin{pmatrix}
1 & -1 & 0 \\
0 & 3 & -3\\
0 & -1 & 1 
\end{pmatrix} \begin{matrix} \phantom{1}\\  \phantom{1} \\  /  +\frac{1}{3}\cdot \text{(I)} \end{matrix} \\
&\rightsquigarrow&
\begin{pmatrix}
1 & -1 & 0 \\
0 & 3 & -3\\
0 & 0 & 0
\end{pmatrix}\begin{matrix}  \phantom{1} \\ / \cdot \frac{1}{3} \\  \phantom{1}  \end{matrix} \\
&\rightsquigarrow&
\begin{pmatrix}
1 & -1 & 0 \\
0 &  1 & -1\\
0 & 0 & 0
\end{pmatrix} \begin{matrix}/  +1\cdot \text{(II)}\\  \phantom{1} \\  \phantom{1} \end{matrix} \\
&\rightsquigarrow&
\begin{pmatrix}
1 & 0 & -1 \\
0 & 1 & -1\\
0 & 0 & 0
\end{pmatrix}
\end{eqnarray*}
The eigenvectors for the eigenvalue $0$ are also the vectors
\[  r\cdot  \begin{pmatrix}
1\\ 1 \\ 1
\end{pmatrix}\quad \text{with }r\in \R\setminus { \{0\} }. \]
    }


    
    \step 
    \lang{de}{ 

Für den Eigenwert $\lambda_2=\sqrt{3}$ berechnet man
\begin{eqnarray*}
& &
\begin{pmatrix}
-2-\sqrt{3} & 5 & -3 \\
1 & -1-\sqrt{3} & 0 \\
3 & -6 & 3- \sqrt{3} 
\end{pmatrix}  \begin{matrix} \updownarrow \\ \phantom{1} \end{matrix} \\
&\rightsquigarrow &
\begin{pmatrix}
1 & -1-\sqrt{3} & 0 \\
-2-\sqrt{3} & 5 & -3 \\
3 & -6 & 3- \sqrt{3} 
\end{pmatrix}  \begin{matrix} \phantom{1}\\  /  +(2+\sqrt{3})\cdot \text{(I)}\\  /  -3\cdot \text{(I)} \end{matrix} \\
&\rightsquigarrow & 
\begin{pmatrix}
1 & -1-\sqrt{3} & 0 \\
0 & -3\sqrt{3} & -3 \\
0 & -3+3\sqrt{3} & 3-\sqrt{3} 
\end{pmatrix}  \begin{matrix} \phantom{1} \\  \cdot (-\frac{1}{9}\sqrt{3}) \\   \phantom{1} \end{matrix} \\
&\rightsquigarrow & 
\begin{pmatrix}
1 & -1-\sqrt{3} & 0 \\
0 & 1 &  \frac{1}{3}\sqrt{3} \\
0 & -3+3\sqrt{3} & 3-\sqrt{3} 
\end{pmatrix}  \begin{matrix} \phantom{1}\\   \phantom{1}\\  /  +(3-3\sqrt{3})\cdot \text{(II)} \end{matrix} \\
&\rightsquigarrow &
\begin{pmatrix}
1 & -1-\sqrt{3} & 0 \\
0 & 1 &  \frac{1}{3}\sqrt{3} \\
0 & 0 & 0
\end{pmatrix} \begin{matrix}/  +(1+\sqrt{3})\cdot \text{(II)} \\   \phantom{1}\\   \phantom{1} \end{matrix} \\
&\rightsquigarrow &
\begin{pmatrix}
1 & 0 &  \frac{1}{3}\sqrt{3}+1 \\
0 & 1 &  \frac{1}{3}\sqrt{3} \\
0 & 0 & 0
\end{pmatrix}
\end{eqnarray*} }


\lang{en}{ 

For the eigenvalue $\lambda_2=\sqrt{3}$ we calculate
\begin{eqnarray*}
& &
\begin{pmatrix}
-2-\sqrt{3} & 5 & -3 \\
1 & -1-\sqrt{3} & 0 \\
3 & -6 & 3- \sqrt{3} 
\end{pmatrix}  \begin{matrix} \updownarrow \\ \phantom{1} \end{matrix} \\
&\rightsquigarrow &
\begin{pmatrix}
1 & -1-\sqrt{3} & 0 \\
-2-\sqrt{3} & 5 & -3 \\
3 & -6 & 3- \sqrt{3} 
\end{pmatrix}  \begin{matrix} \phantom{1}\\  /  +(2+\sqrt{3})\cdot \text{(I)}\\  /  -3\cdot \text{(I)} \end{matrix} \\
&\rightsquigarrow & 
\begin{pmatrix}
1 & -1-\sqrt{3} & 0 \\
0 & -3\sqrt{3} & -3 \\
0 & -3+3\sqrt{3} & 3-\sqrt{3} 
\end{pmatrix}  \begin{matrix} \phantom{1} \\  \cdot (-\frac{1}{9}\sqrt{3}) \\   \phantom{1} \end{matrix} \\
&\rightsquigarrow & 
\begin{pmatrix}
1 & -1-\sqrt{3} & 0 \\
0 & 1 &  \frac{1}{3}\sqrt{3} \\
0 & -3+3\sqrt{3} & 3-\sqrt{3} 
\end{pmatrix}  \begin{matrix} \phantom{1}\\   \phantom{1}\\  /  +(3-3\sqrt{3})\cdot \text{(II)} \end{matrix} \\
&\rightsquigarrow &
\begin{pmatrix}
1 & -1-\sqrt{3} & 0 \\
0 & 1 &  \frac{1}{3}\sqrt{3} \\
0 & 0 & 0
\end{pmatrix} \begin{matrix}/  +(1+\sqrt{3})\cdot \text{(II)} \\   \phantom{1}\\   \phantom{1} \end{matrix} \\
&\rightsquigarrow &
\begin{pmatrix}
1 & 0 &  \frac{1}{3}\sqrt{3}+1 \\
0 & 1 &  \frac{1}{3}\sqrt{3} \\
0 & 0 & 0
\end{pmatrix}
\end{eqnarray*} }


\step 
    \lang{de}{
    
Die Eigenvektoren zum Eigenwert $\sqrt{3}$ sind also die
Vektoren
\[  r\cdot  \begin{pmatrix}
-1-\frac{1}{3}\sqrt{3} \\ -\frac{1}{3}\sqrt{3} \\ 1
\end{pmatrix}\quad \text{mit }r\in \R\setminus { \{0\} }. \]


Ganz genauso berechnet man die Eigenvektoren zum Eigenwert $-\sqrt{3}$. Man muss nur an
jeder Stelle $\sqrt{3}$ durch $-\sqrt{3}$ ersetzen. Also sind die Eigenvektoren zu $-\sqrt{3}$ die
Vektoren
\[  r\cdot  \begin{pmatrix}
-1+\frac{1}{3}\sqrt{3} \\ +\frac{1}{3}\sqrt{3} \\ 1
\end{pmatrix}\quad \text{mit }r\in \R\setminus { \{0\} }. \]
    }

\lang{en}{
    
The eigenvectors for the eigenvalue $\sqrt{3}$ are the vectors
\[  r\cdot  \begin{pmatrix}
-1-\frac{1}{3}\sqrt{3} \\ -\frac{1}{3}\sqrt{3} \\ 1
\end{pmatrix}\quad \text{with }r\in \R\setminus { \{0\} }. \]


Similarly we calculate the eigenvectors for the eigenvalue $-\sqrt{3}$. 
We only need to subsitute for each spot $\sqrt{3}$ through $-\sqrt{3}$. So the eigenvectors for $-\sqrt{3}$ 
are the vectors
\[  r\cdot  \begin{pmatrix}
-1+\frac{1}{3}\sqrt{3} \\ +\frac{1}{3}\sqrt{3} \\ 1
\end{pmatrix}\quad \text{mit }r\in \R\setminus { \{0\} }. \]
    }

  	 %------------------------------------END_STEP_X
 
  \end{incremental}
  %++++++++++++++++++++++++++++++++++++++++++++END_TAB_X

%#############################################################ENDE
\end{tabs*}
\end{content}