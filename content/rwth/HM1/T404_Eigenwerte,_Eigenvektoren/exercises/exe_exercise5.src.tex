\documentclass{mumie.element.exercise}
%$Id$
\begin{metainfo}
  \name{
    \lang{de}{Ü05: Eigenwerte, -vektoren}
    \lang{en}{Ex05: Eigenvalues, eigenvectors}
  }
  \begin{description} 
 This work is licensed under the Creative Commons License Attribution 4.0 International (CC-BY 4.0)   
 https://creativecommons.org/licenses/by/4.0/legalcode 

    \lang{de}{}
    \lang{en}{}
  \end{description}
  \begin{components}
  \end{components}
  \begin{links}
  \end{links}
  \creategeneric
\end{metainfo}
\begin{content}
\usepackage{mumie.ombplus}

\title{\lang{de}{Ü05: Eigenwerte, -vektoren} \lang{en}{Ex05: Eigenvalues, eigenvectors}}

\begin{block}[annotation]
  Im Ticket-System: \href{http://team.mumie.net/issues/11625}{Ticket 11625}
\end{block}

%######################################################FRAGE_TEXT
\lang{de}{ \begin{enumerate}[a)]

\item a) Von der Matrix \[A=\left(\begin{smallmatrix} 11/17& 75/17\\
75/17 & 91/17\end{smallmatrix}\right) \in M(2;\R) \] ist schon
bekannt, dass sie den Vektor $v_1=\left(\begin{smallmatrix}3\\ 5 \end{smallmatrix}\right)$ als
Eigenvektor besitzt.\\

Bestimmen Sie eine Basis des $\R^2$ aus Eigenvektoren von $A$, sowie die Eigenwerte von $A$.

\item b) Von der Matrix 
\[ B=\begin{pmatrix} 15/7 & 4/7 & 2/7\\
 4/7 &30/7 & 8/7\\
 2/7 & 8/7 &18/7 \end{pmatrix}~ \in M(3;\R) \]
ist schon bekannt, dass sie die Eigenvektoren
\[ w_1=\left( \begin{smallmatrix}
2 \\ -1 \\ 1 \end{smallmatrix}\right)\quad \text{und}\quad w_2=\left( \begin{smallmatrix}
1 \\ 4 \\ 2 \end{smallmatrix}\right)\]
besitzt.\\

Bestimmen Sie eine Basis des $\R^3$ aus Eigenvektoren von $B$, sowie die Eigenwerte von $B$.
\end{enumerate} }


\lang{en}{ \begin{enumerate}[a)]

\item a) We know, that the matrix \[A=\left(\begin{smallmatrix} 11/17& 75/17\\
75/17 & 91/17\end{smallmatrix}\right) \in M(2;\R) \] 
has the vector $v_1=\left(\begin{smallmatrix}3\\ 5 \end{smallmatrix}\right)$ as its eigenvector.\\

Determine a basis of $\R^2$ consisting of eigenvectors of $A$, such as the eigenvalues of $A$.

\item b) We know, that the matrix 
\[ B=\begin{pmatrix} 15/7 & 4/7 & 2/7\\
 4/7 &30/7 & 8/7\\
 2/7 & 8/7 &18/7 \end{pmatrix}~ \in M(3;\R) \]
has the eigenvectors
\[ w_1=\left( \begin{smallmatrix}
2 \\ -1 \\ 1 \end{smallmatrix}\right)\quad \text{and}\quad w_2=\left( \begin{smallmatrix}
1 \\ 4 \\ 2 \end{smallmatrix}\right)\].\\

Determine a basis of $\R^3$ consisting of eigenvectors of $B$ and also the eigenvalues of $B$.
\end{enumerate} }

%##################################################ANTWORTEN_TEXT
\begin{tabs*}[\initialtab{0}\class{exercise}]

%++++++++++++++++++++++++++++++++++++++++++START_TAB_X
  \tab{\lang{de}{    Antwort    } \lang{en}{    Answer    }}
  \begin{incremental}[\initialsteps{1}]
  
  	 %----------------------------------START_STEP_X
    \step 
    \lang{de}{a)   Zur Matrix $A$ ist eine Basis des $\R^2$ aus Eigenvektoren gegeben durch die beiden Vektoren
\[ v_1=\left(\begin{smallmatrix}3\\ 5 \end{smallmatrix}\right)\quad \text{und} \quad
v_2=\left(\begin{smallmatrix}5\\ -3 \end{smallmatrix}\right). \]
Der Eigenwert zu $v_1$ ist $\lambda_1=8$, der Eigenwert zu $v_2$ ist $\lambda_2=-2$.\\

~\\
b) Zur Matrix $B$ ist eine Basis des $\R^3$ aus Eigenvektoren gegeben durch die drei  
Vektoren
\[ w_1=\left( \begin{smallmatrix}
2 \\ -1 \\ 1 \end{smallmatrix}\right),\quad w_2=\left( \begin{smallmatrix}
1 \\ 4 \\ 2 \end{smallmatrix}\right)\quad \text{und} \quad w_3=\left( \begin{smallmatrix}
2 \\ 1 \\ -3 \end{smallmatrix}\right).\]
Der Eigenwert zu $w_2$ ist $\lambda_2=5$, die Eigenwerte zu $w_1$ und $w_3$ sind
jeweils $\lambda_1=\lambda_3=2$.\\

\textit{Bemerkung:} Wie immer sind die genannten Eigenvektoren nicht eindeutig. Stets kann man
die Vektoren durch Vielfache $\neq 0$ ersetzen und erhält wieder eine Basis aus Eigenvektoren.\\
Im Fall der Matrix $B$ gibt es sogar noch mehr Möglichkeiten, da zwei Eigenwerte gleich sind.
Das heißt, jede Linearkombination von $w_1$ und $w_3$ ist wieder ein Eigenvektor zum Eigenwert $2$
und man könnte auch $w_3$ zum Beispiel durch $w_3+3w_1$ o.ä. ersetzen.
    }


\lang{en}{a)   A basis of $\R^2$ consisting of eigenvectors of the matrix $A$ is given with the following vectors
\[ v_1=\left(\begin{smallmatrix}3\\ 5 \end{smallmatrix}\right)\quad \text{und} \quad
v_2=\left(\begin{smallmatrix}5\\ -3 \end{smallmatrix}\right). \]
The eigenvalue for $v_1$ is $\lambda_1=8$, the eigenvalue for $v_2$ is $\lambda_2=-2$.\\

~\\
b) For the matrix $B$ a basis of $\R^3$ consisting of eigenvectors is given by the three vectors
\[ w_1=\left( \begin{smallmatrix}
2 \\ -1 \\ 1 \end{smallmatrix}\right),\quad w_2=\left( \begin{smallmatrix}
1 \\ 4 \\ 2 \end{smallmatrix}\right)\quad \text{und} \quad w_3=\left( \begin{smallmatrix}
2 \\ 1 \\ -3 \end{smallmatrix}\right).\]
The eigenvalue for $w_2$ is $\lambda_2=5$, the eigenvalues for $w_1$ and $w_3$ are each $\lambda_1=\lambda_3=2$.\\

\textit{Remark:} Like always, the given eigenvectors are not unambiuous. We can always
choose a multiple (unequal to zero) of the eigenvectors and get again a basis of eigenvectors.\\
For the matrix $B$ there are even more options, because two eigenvalues are equal.
That means, that every linear combination of $w_1$ and $w_3$ is again an eigenvector for the eigenvalue $2$
and we could choose $w_3+3w_1$ instead of $w_3$.
    }
  	 %------------------------------------END_STEP_X
 
  \end{incremental}
  %++++++++++++++++++++++++++++++++++++++++++++END_TAB_X
  
  
  %++++++++++++++++++++++++++++++++++++++++++START_TAB_X
  \tab{\lang{de}{    Lösung a)    }\lang{en}{    Solution a)    }}
  \begin{incremental}[\initialsteps{1}]
  
  	 %----------------------------------START_STEP_X
    \step 
    \lang{de}{   Eine Möglichkeit wäre, nach den allgemeinen Methoden zur Bestimmung von 
	Eigenwerten und Eigenvektoren zunächst das charakteristische
Polynom von $A$ zu bestimmen, anschließend dessen Nullstellen, welche ja die Eigenwerte von $A$ sind,
und zuletzt durch Lösen von Gleichungssystemen Eigenvektoren zu berechnen.

Da die Matrix jedoch symmetrisch und reell ist, und ein Eigenvektor schon bekannt ist, lässt sich der
Rechenaufwand stark verringern:

Da $A$ eine symmetrische reelle Matrix ist, gibt es eine Basis des $\R^2$ aus zueinander
orthogonalen Eigenvektoren. Wählen wir also einen Vektor, der senkrecht zu $v_1$ ist, z.B.
\[  v_2=\begin{pmatrix} 5\\ -3\end{pmatrix}, \]
so ist dies automatisch ein Eigenvektor, da es nur diese eine senkrechte Richtung zu $v_1$ gibt.

Die Eigenwerte berechnen wir jetzt direkt, indem wir die Produkte $Av_1$ und $Av_2$ berechnen
\[  Av_1=\begin{pmatrix} 11/17& 75/17\\ 75/17 & 91/17 \end{pmatrix}\cdot 
\begin{pmatrix} 3\\ 5 \end{pmatrix} =\begin{pmatrix} (33+375)/17 \\ (225+455)/17
\end{pmatrix} =\begin{pmatrix} 24\\ 40 \end{pmatrix}=8\cdot \begin{pmatrix} 3\\ 5 \end{pmatrix},\]
\[  Av_2=\begin{pmatrix} 11/17& 75/17\\ 75/17 & 91/17 \end{pmatrix}\cdot 
\begin{pmatrix} 5\\ -3 \end{pmatrix} =\begin{pmatrix} (55-225)/17 \\ (375-273)/17
\end{pmatrix} =\begin{pmatrix} -10 \\ 6 \end{pmatrix}=-2\cdot \begin{pmatrix} 5\\ -3 \end{pmatrix}.\]
Der Eigenwert zu $v_1$ ist also $\lambda_1=8$ und der Eigenwert zu $v_2$ ist $\lambda_2=-2$.
    }

\lang{en}{ We could use the general method for the determination 
of eigenvalues and eigenvectors. In this method we determine the zeros of the characteristic polynomial of $A$,
which are the eigenvalues of $A$ and then we solve the linear systems for determining the eigenvectors.

Since the matrix is symmetric and real and we already know one eigenvector,
we are able to reduce the calctulation effort:

Because $A$ is a symmetric real matrix, there exists a basis of $\R^2$ consisting of
perpendicular eigenvectors. If we choose vector, orthogonal to $v_1$ ist, e.g.
\[  v_2=\begin{pmatrix} 5\\ -3\end{pmatrix}, \]
this vector is automatically an eigenvector, because there is only this direction perpendicular to $v_1$.

We calculate the eigenvalues right away by calculating the products $Av_1$ and $Av_2$:
\[  Av_1=\begin{pmatrix} 11/17& 75/17\\ 75/17 & 91/17 \end{pmatrix}\cdot 
\begin{pmatrix} 3\\ 5 \end{pmatrix} =\begin{pmatrix} (33+375)/17 \\ (225+455)/17
\end{pmatrix} =\begin{pmatrix} 24\\ 40 \end{pmatrix}=8\cdot \begin{pmatrix} 3\\ 5 \end{pmatrix},\]
\[  Av_2=\begin{pmatrix} 11/17& 75/17\\ 75/17 & 91/17 \end{pmatrix}\cdot 
\begin{pmatrix} 5\\ -3 \end{pmatrix} =\begin{pmatrix} (55-225)/17 \\ (375-273)/17
\end{pmatrix} =\begin{pmatrix} -10 \\ 6 \end{pmatrix}=-2\cdot \begin{pmatrix} 5\\ -3 \end{pmatrix}.\]
The eigenvalue of $v_1$ is $\lambda_1=8$ and the eigenvalue of $v_2$ is $\lambda_2=-2$.
    }
  	 %------------------------------------END_STEP_X
 
  \end{incremental}
  %++++++++++++++++++++++++++++++++++++++++++++END_TAB_X


  %++++++++++++++++++++++++++++++++++++++++++START_TAB_X
  \tab{\lang{de}{    Lösung b)    }\lang{en}{    Solution b)    }}
  \begin{incremental}[\initialsteps{1}]
  
  	 %----------------------------------START_STEP_X
    \step 
    \lang{de}{   Eine Möglichkeit wäre, nach den allgemeinen Methoden zur Bestimmung von 
	Eigenwerten und Eigenvektoren zunächst das charakteristische
Polynom von $B$ zu bestimmen, anschließend dessen Nullstellen, welche ja die Eigenwerte von $B$ sind,
und zuletzt durch Lösen von Gleichungssystemen Eigenvektoren zu berechnen.
Schon allein die Berechnung des charakteristischen Polynoms ist bei diesen Zahlen aufwändig.\\


Da die Matrix jedoch symmetrisch und reell ist, und zwei Eigenvektoren schon bekannt sind, lässt sich
 der Rechenaufwand stark verringern:

Da $B$ eine symmetrische reelle Matrix ist, gibt es eine Basis des $\R^3$ aus zueinander
orthogonalen Eigenvektoren. Wählen wir also einen Vektor, der senkrecht zu $w_1$ und $w_2$ ist, z.B.
deren Vektorprodukt
\[  w=w_1\times w_2=\begin{pmatrix}2 \\ -1 \\ 1\end{pmatrix}\times \begin{pmatrix}
1 \\ 4 \\ 2 \end{pmatrix} = \begin{pmatrix} -1\cdot 2-1\cdot 4\\ 1\cdot 1-2\cdot 2\\ 2\cdot 4-(-1)\cdot 1\end{pmatrix}=\begin{pmatrix} -6\\ -3\\ 9
\end{pmatrix}, \]
oder auch
$w_3=-\frac{1}{3}\cdot w=\left(\begin{smallmatrix} 2 \\ 1 \\ -3 \end{smallmatrix}\right)$,
so ist dies automatisch ein Eigenvektor, da es nur diese eine senkrechte Richtung zu $w_1$
und $w_2$ gibt.

Die Eigenwerte berechnen wir jetzt direkt, indem wir die Produkte $Bw_1$, $Bw_2$ und $Bw_3$ berechnen
\[  Bw_1=\begin{pmatrix} 15/7 & 4/7 & 2/7\\
 4/7 &30/7 & 8/7\\
 2/7 & 8/7 &18/7 \end{pmatrix}\cdot \begin{pmatrix}2 \\ -1 \\ 1\end{pmatrix} =
 \begin{pmatrix} (30-4+2)/7 \\ (8-30+8)/7 \\ (4-8+18)/7 \end{pmatrix}
 =\begin{pmatrix}
 4 \\ -2\\ 2 \end{pmatrix}=2\cdot \begin{pmatrix}2 \\ -1 \\ 1\end{pmatrix}, \]
\[  Bw_2=\begin{pmatrix} 15/7 & 4/7 & 2/7\\
 4/7 &30/7 & 8/7\\
 2/7 & 8/7 &18/7 \end{pmatrix}\cdot \begin{pmatrix}1 \\ 4 \\ 2\end{pmatrix} =
 \begin{pmatrix} (15+16+4)/7 \\ (4+120+16)/7 \\ (2+32+36)/7 \end{pmatrix}
 =\begin{pmatrix}
 5 \\ 20\\ 10 \end{pmatrix}=5\cdot \begin{pmatrix}1 \\ 4 \\ 2\end{pmatrix}, \]
\[  Bw_3=\begin{pmatrix} 15/7 & 4/7 & 2/7\\
 4/7 &30/7 & 8/7\\
 2/7 & 8/7 &18/7 \end{pmatrix}\cdot \begin{pmatrix}2 \\ 1 \\ -3\end{pmatrix} =
 \begin{pmatrix} (30+4-6)/7 \\ (8+30-24)/7 \\ (4+8-54)/7 \end{pmatrix}
 =\begin{pmatrix}
 4 \\ 2\\ -6 \end{pmatrix}=2\cdot \begin{pmatrix}2 \\ 1 \\ -3\end{pmatrix}. \]
Der Eigenwert zu $w_2$ ist also $\lambda_2=5$, die Eigenwerte zu $w_1$ und $w_3$ sind
jeweils $\lambda_1=\lambda_3=2$.\\    }


 \lang{en}{ We could use the general method for the determination 
of eigenvalues and eigenvectors. In this method we determine the zeros of the characteristic polynomial of $A$,
which are the eigenvalues of $A$ and then we solve the linear systems for determining the eigenvectors.
But for those numbers even the calculation of the characteristic polynomial is time-consuming.\\


Since the matrix is symmetric and real, and we already know two eigenvectors,
we are able to reduce the calculation effort:

$B$ is a real symmetric matrix and therefore it exists a basis of $\R^3$, which consists of pairwise perpendicular
eigenvectors. So if we choose a vector, which is orthogonal to $w_1$ and $w_2$, e.g.
their vectorproduct
\[  w=w_1\times w_2=\begin{pmatrix}2 \\ -1 \\ 1\end{pmatrix}\times \begin{pmatrix}
1 \\ 4 \\ 2 \end{pmatrix} = \begin{pmatrix} -1\cdot 2-1\cdot 4\\ 1\cdot 1-2\cdot 2\\ 2\cdot 4-(-1)\cdot 1\end{pmatrix}=\begin{pmatrix} -6\\ -3\\ 9
\end{pmatrix}, \]
or
$w_3=-\frac{1}{3}\cdot w=\left(\begin{smallmatrix} 2 \\ 1 \\ -3 \end{smallmatrix}\right)$,
then this vector is automatically an eigenvector, because there is only one direction possible, that is
perpendicular to $w_1$
and $w_2$.

Now we calculate the eigenvalues by calculating the products $Bw_1$, $Bw_2$ and $Bw_3$
\[  Bw_1=\begin{pmatrix} 15/7 & 4/7 & 2/7\\
 4/7 &30/7 & 8/7\\
 2/7 & 8/7 &18/7 \end{pmatrix}\cdot \begin{pmatrix}2 \\ -1 \\ 1\end{pmatrix} =
 \begin{pmatrix} (30-4+2)/7 \\ (8-30+8)/7 \\ (4-8+18)/7 \end{pmatrix}
 =\begin{pmatrix}
 4 \\ -2\\ 2 \end{pmatrix}=2\cdot \begin{pmatrix}2 \\ -1 \\ 1\end{pmatrix}, \]
\[  Bw_2=\begin{pmatrix} 15/7 & 4/7 & 2/7\\
 4/7 &30/7 & 8/7\\
 2/7 & 8/7 &18/7 \end{pmatrix}\cdot \begin{pmatrix}1 \\ 4 \\ 2\end{pmatrix} =
 \begin{pmatrix} (15+16+4)/7 \\ (4+120+16)/7 \\ (2+32+36)/7 \end{pmatrix}
 =\begin{pmatrix}
 5 \\ 20\\ 10 \end{pmatrix}=5\cdot \begin{pmatrix}1 \\ 4 \\ 2\end{pmatrix}, \]
\[  Bw_3=\begin{pmatrix} 15/7 & 4/7 & 2/7\\
 4/7 &30/7 & 8/7\\
 2/7 & 8/7 &18/7 \end{pmatrix}\cdot \begin{pmatrix}2 \\ 1 \\ -3\end{pmatrix} =
 \begin{pmatrix} (30+4-6)/7 \\ (8+30-24)/7 \\ (4+8-54)/7 \end{pmatrix}
 =\begin{pmatrix}
 4 \\ 2\\ -6 \end{pmatrix}=2\cdot \begin{pmatrix}2 \\ 1 \\ -3\end{pmatrix}. \]
So the eigenvalue of $w_2$ is $\lambda_2=5$, the eigenvalues for $w_1$ and $w_3$ are each $\lambda_1=\lambda_3=2$.\\    }
  	 %------------------------------------END_STEP_X
 
  \end{incremental}
  %++++++++++++++++++++++++++++++++++++++++++++END_TAB_X


%#############################################################ENDE
\end{tabs*}
\end{content}