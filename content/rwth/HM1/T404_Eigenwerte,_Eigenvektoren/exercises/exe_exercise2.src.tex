\documentclass{mumie.element.exercise}
%$Id$
\begin{metainfo}
  \name{
    \lang{de}{Ü02: char. Polynom, Eigenwerte}
    \lang{en}{Ex02: Characteristic polynomial, eigenvalues}
  }
  \begin{description} 
 This work is licensed under the Creative Commons License Attribution 4.0 International (CC-BY 4.0)   
 https://creativecommons.org/licenses/by/4.0/legalcode 

    \lang{de}{}
    \lang{en}{}
  \end{description}
  \begin{components}
  \end{components}
  \begin{links}
  \end{links}
  \creategeneric
\end{metainfo}
\begin{content}
\usepackage{mumie.ombplus}

\title{\lang{de}{Ü02: char. Polynom, Eigenwerte} \lang{en}{Ex02: Characteristic polynomial, eigenvalues}}

\begin{block}[annotation]
  Im Ticket-System: \href{http://team.mumie.net/issues/11622}{Ticket 11622}
\end{block}

%######################################################FRAGE_TEXT
\lang{de}{ 
Bestimmen Sie jeweils  das charakteristische Polynom der folgenden Matrizen, sowie deren komplexe
Eigenwerte.
\[  A=\begin{pmatrix}
2 & -1 \\ 3 & 0\end{pmatrix}\in M(2;\C), \qquad B=\begin{pmatrix}-2 & -5 & 2\\
 7& 10& -6\\  11 &13 &-9\end{pmatrix}\in M(3;\C) \]
\[ \text{und}\qquad C=\begin{pmatrix}
2 & -1 & 1 & i \\ 3 & 0 & i & 1 \\ 0&0& 2& 1\\ 0&0& 1& -2\end{pmatrix}\in M(4;\C). \] }

\lang{en}{ 
Determine the characteristic polynomial for each of the following matrices and also their complex eigenvalues.
\[  A=\begin{pmatrix}
2 & -1 \\ 3 & 0\end{pmatrix}\in M(2;\C), \qquad B=\begin{pmatrix}-2 & -5 & 2\\
 7& 10& -6\\  11 &13 &-9\end{pmatrix}\in M(3;\C) \]
\[ \text{und}\qquad C=\begin{pmatrix}
2 & -1 & 1 & i \\ 3 & 0 & i & 1 \\ 0&0& 2& 1\\ 0&0& 1& -2\end{pmatrix}\in M(4;\C). \] }

%##################################################ANTWORTEN_TEXT
\begin{tabs*}[\initialtab{0}\class{exercise}]

%++++++++++++++++++++++++++++++++++++++++++START_TAB_X
  \tab{\lang{de}{    Antwort    } \lang{en}{    Answer    }}
  \begin{incremental}[\initialsteps{1}]
  
  	 %----------------------------------START_STEP_X
    \step 
    \lang{de}{Lösungsübersicht:   \begin{itemize}
\item Die charakteristischen Polynome sind
\begin{eqnarray*}
p_A(t) &=& t^2-2t+3, \\
p_B(t) &=& -t^3-t^2+t+1, \\
p_C(t) &=& (t^2-2t+3)(t^2-5) = t^4-2t^3-2t^2+10t-15.
\end{eqnarray*}
\item Die Eigenwerte von $A$ sind $1+i\sqrt{2}$ und $1-i\sqrt{2}$.
\item Die Eigenwerte von $B$ sind $1$ und $-1$.
\item Die Eigenwerte von $C$ sind $1+i\sqrt{2}$, $1-i\sqrt{2}$, $\sqrt{5}$ und $-\sqrt{5}$.
\end{itemize}    }

    \lang{en}{Overview of the solution:   \begin{itemize}
\item The characteristic polynomials are
\begin{eqnarray*}
p_A(t) &=& t^2-2t+3, \\
p_B(t) &=& -t^3-t^2+t+1, \\
p_C(t) &=& (t^2-2t+3)(t^2-5) = t^4-2t^3-2t^2+10t-15.
\end{eqnarray*}
\item The eigenvalues of $A$ are $1+i\sqrt{2}$ and $1-i\sqrt{2}$.
\item The eigenvalues of $B$ are $1$ and $-1$.
\item The eigenvalues of $C$ are $1+i\sqrt{2}$, $1-i\sqrt{2}$, $\sqrt{5}$ and $-\sqrt{5}$.
\end{itemize}    }
  	 %------------------------------------END_STEP_X
 
  \end{incremental}
  %++++++++++++++++++++++++++++++++++++++++++++END_TAB_X
  
  
  %++++++++++++++++++++++++++++++++++++++++++START_TAB_X
  \tab{\lang{de}{    Lösung zu $A$    } \lang{en}{    Solution of $A$    }}
  \begin{incremental}[\initialsteps{1}]
  
  	 %----------------------------------START_STEP_X
    \step 
    \lang{de}{   Das charakteristische Polynom für Matrix $A$ ist
    $p_A(t) = \det(A-t\cdot E_2)$.
    Dieses können wir mit der Formel für Determinanten
    von $(2\times 2)$-Matrizen berechnen
\begin{eqnarray*}
 p_A(t) &=& \det(A-t\cdot E_2) =\det \Big( \begin{pmatrix} 2-t & -1 \\ 3 & -t 
 \end{pmatrix}\Big) \\
 &=& (2-t)\cdot (-t)-(-1)\cdot 3 =t^2-2t+3.
\end{eqnarray*}
Die Eigenwerte von $A$ sind die Nullstellen von $p_A(t)$.

Für die Diskrimiante der $p$-$q$-Formel gilt hier
$(\frac{p}{2})^2-q=(\frac{2}{2})^2-3=-2<0$.
Also sind die Nullstellen komplexwertig
\[  \lambda^2-2\lambda+3=0 \Leftrightarrow \lambda_{1,2}=- (-1)\pm i\sqrt{2}=1\pm i\sqrt{2}.\]
    }


       \lang{en}{   The characteristic polynomial of the matrix $A$ is
    $p_A(t) = \det(A-t\cdot I_2)$.
    We can calculate this with help of the formula for the determinants of $(2\times 2)$-matrices
\begin{eqnarray*}
 p_A(t) &=& \det(A-t\cdot I_2) =\det \Big( \begin{pmatrix} 2-t & -1 \\ 3 & -t 
 \end{pmatrix}\Big) \\
 &=& (2-t)\cdot (-t)-(-1)\cdot 3 =t^2-2t+3.
\end{eqnarray*}
The eigenvalues of $A$ are the zeros of $p_A(t)$.

For the discriminant of the $p$-$q$-formula holds
$(\frac{p}{2})^2-q=(\frac{2}{2})^2-3=-2<0$.
So the complex-valued zeros are
\[  \lambda^2-2\lambda+3=0 \Leftrightarrow \lambda_{1,2}=- (-1)\pm i\sqrt{2}=1\pm i\sqrt{2}.\]
    }
  	 %------------------------------------END_STEP_X
 
  \end{incremental}
  %++++++++++++++++++++++++++++++++++++++++++++END_TAB_X


  %++++++++++++++++++++++++++++++++++++++++++START_TAB_X
  \tab{\lang{de}{    Lösung zu $B$    }  \lang{en}{    Solution for $B$    }}
  \begin{incremental}[\initialsteps{1}]
  
  	 %----------------------------------START_STEP_X
    \step 
    \lang{de}{   Das charakteristische Polynom 
    für Matrix $B$ ist 
    $p_B(t) = \det(B-t\cdot E_3)$.
    Für die Berechnung können wir hier am einfachsten die Sarrus-Formel 
    für Determinanten von $(3\times 3)$-Matrizen anwenden
\begin{eqnarray*}
 p_B(t) &=& \det(B-t\cdot E_2) =\det \Big( \begin{pmatrix}-2-t & -5 & 2\\
 7& 10-t& -6\\  11 &13 &-9-t\end{pmatrix}\Big) \\
 &=& (-2-t)(10-t)(-9-t)+330+182 -22(10-t)-(-78)(-2-t)-(-35)(-9-t) \\
 &=& (-20-8t+t^2)(-9-t) +512 -220+22t-156-78t -315-35t \\
 &=& 180+92t-t^2-t^3 +\, 292 -56t -471 -35t =-t^3-t^2+t+1.
\end{eqnarray*}
Die Eigenwerte von $B$ sind die Nullstellen von $p_B(t)$. 
Da es für Polynome dritten Grades jedoch keine einfache Formel für die
Nullstellenberechnung gibt, muss zunächst eine Nullstelle "`geraten"' werden. 

Man sieht schnell, dass
$p_B(1)=-1-1+1+1=0$ ist, weshalb $\lambda_1=1$ eine Nullstelle ist. 
Per Polynomdivision (welche wir hier nicht explizit angeben) kann man dann
den Faktor $(t-1)$ abspalten
\[   p_B(t)=-t^3-t^2+t+1=(t-1)(-t^2-2t-1)=-(t-1)(t^2+2t+1) =-(t-1)(t+1)^2, \]

Für die letzte Umformung haben wir die erste binomische Formel verwendet.
Als einzige weitere Nullstelle von $p_B(t)$ bekommt man also die Nullstelle 
von $(t+1)^2$, welche $-1$ ist.

Die Eigenwerte von $B$ sind also $-1$ und $1$.

\textit{Bemerkung:} Hier ist 
%Man sagt manchmal auch, 
$-1$ ein "doppelter Eigenwert" von $B$, 
da $-1$ eine doppelte Nullstelle des charakteristischen Polynoms $p_B(t)$ ist.
    }
    
    \lang{en}{   The characteristic polynomial of the matrix $B$ is
    $p_B(t) = \det(B-t\cdot I_3)$.
    We use the rule of Sarrus for determinants of $(3\times 3)$-matrices to determine the charactersistic polynomial
\begin{eqnarray*}
 p_B(t) &=& \det(B-t\cdot I_2) =\det \Big( \begin{pmatrix}-2-t & -5 & 2\\
 7& 10-t& -6\\  11 &13 &-9-t\end{pmatrix}\Big) \\
 &=& (-2-t)(10-t)(-9-t)+330+182 -22(10-t)-(-78)(-2-t)-(-35)(-9-t) \\
 &=& (-20-8t+t^2)(-9-t) +512 -220+22t-156-78t -315-35t \\
 &=& 180+92t-t^2-t^3 +\, 292 -56t -471 -35t =-t^3-t^2+t+1.
\end{eqnarray*}
The eigenvalues of $B$ are the zeros of $p_B(t)$. 
Since there is no easy formula for the calculation of zeros of a polynomial with degree,
we need to \glqq guess \grqq the first zero.

We see quickly, that
$p_B(1)=-1-1+1+1=0$ holds, which is why $\lambda_1=1$ is a zero of the polynomial. 
With help of polynomial division (not explicitly given here) we may seperate the factor $(t-1)$
\[   p_B(t)=-t^3-t^2+t+1=(t-1)(-t^2-2t-1)=-(t-1)(t^2+2t+1) =-(t-1)(t+1)^2, \]

For the last transformation we used the first binomial formula.
The only other zero of $p_B(t)$ is zero of $(t+1)^2$, which is $-1$.

So the eigenvalues of $B$ are $-1$ and $1$.

\textit{Remark:} Here is 
%Man sagt manchmal auch, 
$-1$ a "double eigenvalue" of $B$, 
because $-1$ is a double zero of the characteristic polynomial $p_B(t)$.
    }
  	 %------------------------------------END_STEP_X
 
  \end{incremental}
  %++++++++++++++++++++++++++++++++++++++++++++END_TAB_X


  %++++++++++++++++++++++++++++++++++++++++++START_TAB_X
  \tab{\lang{de}{    Lösung zu $C$    }\lang{en}{    Solution for $C$    }}
  \begin{incremental}[\initialsteps{1}]
  
  	 %----------------------------------START_STEP_X
    \step 
    \lang{de}{   Das charakteristische Polynom für Matrix $C$ ist  
\[ p_C(t) = \det(C-t\cdot E_4) =\det \bigg( \begin{pmatrix}
2-t & -1 & 1 & i \\ 3 & -t & i & 1 \\ 0&0& 2-t& 1\\ 0&0& 1& -2-t\end{pmatrix} \bigg). \]
In diesem Fall können wir die Determinantenformel für Block-Matrizen verwenden, 
weil die untere linke $(2\times 2)$-Teilmatrix
nur Nullen als Einträge hat. Also ist
\begin{eqnarray*}
  \det(C-t\cdot E_4) &=&\det \bigg( \begin{pmatrix}
2-t & -1 & 1 & i \\ 3 & -t & i & 1 \\ 0&0& 2-t& 1\\ 0&0& 1& -2-t\end{pmatrix} \bigg) \\
&=& \det \big( \begin{pmatrix}2-t & -1 \\ 3 & -t\end{pmatrix} \big)\cdot \det \big(  \begin{pmatrix} 2-t& 1\\
1& -2-t\end{pmatrix} \big) \\
&=& \big( (2-t)(-t)+3 \big)\cdot \big( (2-t)(-2-t)-1 \big) = (t^2-2t+3)(t^2-5) \\
&=&  t^4-2t^3-2t^2+10t-15.
\end{eqnarray*}

Um die Eigenwerte von $C$ zu berechnen, welche ja die Nullstellen von $p_C(t)$ sind, sollte man nicht die
letzte angegebene Form des charakteristischen Polynoms verwenden, sondern die teilfaktorisierte Form
\[  p_C(t)= (t^2-2t+3)(t^2-5). \]
Um nämlich die Nullstellen von $p_C(t)$ zu bestimmen, muss man dann lediglich die Nullstellen der quadratischen
Faktoren bestimmen. Die von $t^2-2t+3$ sind $\lambda_1=1+ i\sqrt{2}$ und $\lambda_2=1- i\sqrt{2}$ (siehe Lösung für $A$) und die von $t^2-5$ sind offensichtlich $\lambda_3=\sqrt{5}$ und $\lambda_4=-\sqrt{5}$.

Damit haben wir alle vier Eigenwerte von $C$ bestimmt.    }

\lang{en}{   The characteristic polynomial for the matrix $C$ is  
\[ p_C(t) = \det(C-t\cdot I_4) =\det \bigg( \begin{pmatrix}
2-t & -1 & 1 & i \\ 3 & -t & i & 1 \\ 0&0& 2-t& 1\\ 0&0& 1& -2-t\end{pmatrix} \bigg). \]
In this case we use the determinant formular for block matrices, 
because the lower left $(2\times 2)$-sub matrix
has only zeros as its entries. So it is
\begin{eqnarray*}
  \det(C-t\cdot I_4) &=&\det \bigg( \begin{pmatrix}
2-t & -1 & 1 & i \\ 3 & -t & i & 1 \\ 0&0& 2-t& 1\\ 0&0& 1& -2-t\end{pmatrix} \bigg) \\
&=& \det \big( \begin{pmatrix}2-t & -1 \\ 3 & -t\end{pmatrix} \big)\cdot \det \big(  \begin{pmatrix} 2-t& 1\\
1& -2-t\end{pmatrix} \big) \\
&=& \big( (2-t)(-t)+3 \big)\cdot \big( (2-t)(-2-t)-1 \big) = (t^2-2t+3)(t^2-5) \\
&=&  t^4-2t^3-2t^2+10t-15.
\end{eqnarray*}

We should not use the last given form of the characteristic polynomial to determine the eigenvalues of $C$ (which are the zeros of $p_C(t)$).
\[  p_C(t)= (t^2-2t+3)(t^2-5). \]
We simply need to determine the zeros of the quadratic factors.
The zeros of $t^2-2t+3$ are $\lambda_1=1+ i\sqrt{2}$ and $\lambda_2=1- i\sqrt{2}$ (see solution for $A$) and the zeros of $t^2-5$ are obviously $\lambda_3=\sqrt{5}$ and $\lambda_4=-\sqrt{5}$.

Therefore we have found all of the four eigenvalues of $C$.    }
  	 %------------------------------------END_STEP_X
 
  \end{incremental}
  %++++++++++++++++++++++++++++++++++++++++++++END_TAB_X


%#############################################################ENDE
\end{tabs*}
\end{content}