\documentclass{mumie.element.exercise}
%$Id$
\begin{metainfo}
  \name{
    \lang{de}{Ü04: Symmetrie}
    \lang{en}{Ex04: Symmetry}
  }
  \begin{description} 
 This work is licensed under the Creative Commons License Attribution 4.0 International (CC-BY 4.0)   
 https://creativecommons.org/licenses/by/4.0/legalcode 

    \lang{de}{}
    \lang{en}{}
  \end{description}
  \begin{components}
  \end{components}
  \begin{links}
  \end{links}
  \creategeneric
\end{metainfo}
\begin{content}
\usepackage{mumie.ombplus}

\title{\lang{de}{Ü04: Symmetrie}  \lang{en}{Ex04: Symmetry}}

\begin{block}[annotation]
  Im Ticket-System: \href{http://team.mumie.net/issues/11624}{Ticket 11624}
\end{block}

%######################################################FRAGE_TEXT
\lang{de}{ 
Welche der folgenden Matrizen sind symmetrisch, welche nicht?
\[ A= \begin{pmatrix} 2 & -1 \\ -1 & 0 \end{pmatrix},\quad
B=  \begin{pmatrix} 1&2& 3\\ 0 & 2 & 2\\ -1 & 0 &1 \end{pmatrix},\quad
C=\begin{pmatrix} 1 & i \\ -i & 2 \end{pmatrix},\quad D=\begin{pmatrix}
1 & 0 & 0 & 0\\ 0&2&0&0\\ 0&0&3&0\\ 0&0&0&4\end{pmatrix}.\] }

\lang{en}{ 
Which of the following matrices are symmetric, which are not?
\[ A= \begin{pmatrix} 2 & -1 \\ -1 & 0 \end{pmatrix},\quad
B=  \begin{pmatrix} 1&2& 3\\ 0 & 2 & 2\\ -1 & 0 &1 \end{pmatrix},\quad
C=\begin{pmatrix} 1 & i \\ -i & 2 \end{pmatrix},\quad D=\begin{pmatrix}
1 & 0 & 0 & 0\\ 0&2&0&0\\ 0&0&3&0\\ 0&0&0&4\end{pmatrix}.\] }

%##################################################ANTWORTEN_TEXT
\begin{tabs*}[\initialtab{0}\class{exercise}]

%++++++++++++++++++++++++++++++++++++++++++START_TAB_X
  \tab{\lang{de}{    Antwort    }\lang{en}{    Answer    }}
  \begin{incremental}[\initialsteps{1}]
  
  	 %----------------------------------START_STEP_X
    \step 
    \lang{de}{   Die Matrizen $A$ und $D$ sind symmetrisch, die Matrizen $B$ und $C$ nicht.    }
    \lang{en}{   The matrices $A$ and $D$ are symmetric, the matrices $B$ and $C$ are not.    }
  	 %------------------------------------END_STEP_X
 
  \end{incremental}
  %++++++++++++++++++++++++++++++++++++++++++++END_TAB_X
  
  
  %++++++++++++++++++++++++++++++++++++++++++START_TAB_X
  \tab{\lang{de}{    Lösung zu $A$    }\lang{en}{    Solution for $A$    }}
  \begin{incremental}[\initialsteps{1}]
  
  	 %----------------------------------START_STEP_X
    \step 
    \lang{de}{   Matrizen sind symmetrisch, wenn sie gleich ihrer Transponierten sind.\\

Beim Transponieren von $(2\times 2)$-Matrizen wird lediglich
der Eintrag an Stelle $(1,2)$ mit dem Eintrag an Stelle $(2,1)$
vertauscht. 

Da die beiden Einträge in dieser Aufgabe gleich sind, ist
die $(2\times 2)$-Matrix 
$A=\left( \begin{smallmatrix}2 & -1 \\ -1 & 0\end{smallmatrix}\right) 
\in M(2;\R)$ 
also symmetrisch.    }
\lang{en}{   Matrices are symmetric, when they are equal to its transpose.\\

While transposing a $(2\times 2)$-matrix we only need to swap the entry $(1,2)$ with the entry $(2,1)$. 

Since the entries in this exercise are the same, the $(2\times 2)$-matrix 
$A=\left( \begin{smallmatrix}2 & -1 \\ -1 & 0\end{smallmatrix}\right) 
\in M(2;\R)$ 
is symmetric.    }
  	 %------------------------------------END_STEP_X
 
  \end{incremental}
  %++++++++++++++++++++++++++++++++++++++++++++END_TAB_X


  %++++++++++++++++++++++++++++++++++++++++++START_TAB_X
  \tab{\lang{de}{    Lösung zu $B$      }\lang{en}{    Solution for $B$      }}
  \begin{incremental}[\initialsteps{1}]
  
  	 %----------------------------------START_STEP_X
    \step 
    \lang{de}{   Matrizen sind symmetrisch, wenn sie gleich ihrer Transponierten sind.\\

Für die Matrix $B$
gilt
\[ B^T=
\left( \begin{smallmatrix}1&2& 3\\ 0 & 2 & 2\\ -1 & 0 &1\end{smallmatrix}\right)^T 
= \begin{pmatrix} 1& 0 & -1 \\ 2 & 2 & 0\\ 3 & 2 & 1\end{pmatrix} \neq B. \]
Also ist die Matrix $B$ nicht symmetrisch.

\textit{Bemerkung:} Würde man $B$ an der Gegendiagonalen spiegeln (also den $(1,1)$-Eintrag mit dem $(3,3)$-Eintrag vertauschen etc.), würde wieder die Matrix $B$ herauskommen. Dies hat aber für die Symmetrie einer
Matrix keine Bedeutung.
    }

      \lang{en}{   Matrices are symmetric, when they are equal to its transpose.\\

For a matrix $B$
holds
\[ B^T=
\left( \begin{smallmatrix}1&2& 3\\ 0 & 2 & 2\\ -1 & 0 &1\end{smallmatrix}\right)^T 
= \begin{pmatrix} 1& 0 & -1 \\ 2 & 2 & 0\\ 3 & 2 & 1\end{pmatrix} \neq B. \]
So the matrix $B$ is not symmetric.

\textit{Remark:} If we would reflect the matrix along the counter diagonal of $B$, (so exchange the $(1,1)$-entry with the $(3,3)$-entry etc.), 
we would end up with the matrix $B$. But this has no meaning for the symmetry of a matrix.
    }
  	 %------------------------------------END_STEP_X
 
  \end{incremental}
  %++++++++++++++++++++++++++++++++++++++++++++END_TAB_X


  %++++++++++++++++++++++++++++++++++++++++++START_TAB_X
  \tab{\lang{de}{    Lösung zu $C$      }\lang{en}{    Solution for $C$      }}
  \begin{incremental}[\initialsteps{1}]
  
  	 %----------------------------------START_STEP_X
    \step 
    \lang{de}{   
Die Transponierte zu $C$ ist
\[  C^T=\begin{pmatrix} 1 & i \\ -i & 2 \end{pmatrix}^T=\begin{pmatrix} 1 & -i \\ i & 2 \end{pmatrix}. \]
Da diese verschieden von $C$ ist, ist die Matrix $C$ nicht symmetrisch.

\textit{Nebenbemerkung:} Wie man an der Rechnung sieht, ist die Transponierte $C^T$ identisch mit der
komplex konjugierten Matrix zu $C$ (wenn man also alle Einträge durch ihre komplex konjugierten Einträge
ersetzt). 

Diese Eigenschaft ist für komplexe Matrizen sogar wichtiger, als symmetrisch zu sein. 
Man nennt komplexe Matrizen, deren Transponierte gleich der Konjugierten ist,
\notion{Hermitsche Matrizen.}
    }

   \lang{en}{   
The transpose of $C$ is
\[  C^T=\begin{pmatrix} 1 & i \\ -i & 2 \end{pmatrix}^T=\begin{pmatrix} 1 & -i \\ i & 2 \end{pmatrix}. \]
Because of $C^t\neq C$, the matrix $C$ is not symmetric.

\textit{Remark:} The calculation shows, that the transpose $C^T$ is identical to the complex
conjugate matrix of $C$ (so if we substitute each entry by its complex conjugate). 

This property is for complex matrices even more important than the symmetry.
We call complex matrices, if their transpose is equal to the conjugate,
\notion{Hermitian matrices.}
    }
  	 %------------------------------------END_STEP_X
 
  \end{incremental}
  %++++++++++++++++++++++++++++++++++++++++++++END_TAB_X

  %++++++++++++++++++++++++++++++++++++++++++START_TAB_X
  \tab{\lang{de}{    Lösung zu $D$      }\lang{en}{    Solution for $D$      }}
  \begin{incremental}[\initialsteps{1}]
  
  	 %----------------------------------START_STEP_X
    \step 
    \lang{de}{   
Beim Transponieren der Matrix  werden lediglich die Nullen oberhalb der Diagonalen mit den Nullen unterhalb
der Diagonalen vertauscht:
\[
D^T=\left( \begin{smallmatrix} 1 & 0 & 0 & 0\\ 0&2&0&0\\ 0&0&3&0\\ 0&0&0&4
\end{smallmatrix}\right)^T = \left( \begin{smallmatrix} 1 & 0 & 0 & 0\\ 0&2&0&0\\ 0&0&3&0\\ 0&0&0&4
\end{smallmatrix}\right) = D
\]

Die Matrix $D$ ist also (wie jede Diagonalmatrix) symmetrisch.
    }

    \lang{en}{   
While transposing the matrix we only swap the zeros above the diagonal with the zeros below the diagonal:
\[
D^T=\left( \begin{smallmatrix} 1 & 0 & 0 & 0\\ 0&2&0&0\\ 0&0&3&0\\ 0&0&0&4
\end{smallmatrix}\right)^T = \left( \begin{smallmatrix} 1 & 0 & 0 & 0\\ 0&2&0&0\\ 0&0&3&0\\ 0&0&0&4
\end{smallmatrix}\right) = D
\]

So the matrix $D$ is (as every diagonal matrix) symmetric.
    }
  	 %------------------------------------END_STEP_X
 
  \end{incremental}
  %++++++++++++++++++++++++++++++++++++++++++++END_TAB_X

%#############################################################ENDE
\end{tabs*}
\end{content}