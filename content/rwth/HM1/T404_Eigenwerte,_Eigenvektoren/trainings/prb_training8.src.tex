\documentclass{mumie.problem.gwtmathlet}
%$Id$
\begin{metainfo}
  \name{
    \lang{en}{P08: Principal axes}
    \lang{de}{A08: Hauptachsen}
  }
  \begin{description} 
 This work is licensed under the Creative Commons License Attribution 4.0 International (CC-BY 4.0)   
 https://creativecommons.org/licenses/by/4.0/legalcode 

    \lang{en}{...}
    \lang{de}{...}
  \end{description}
  \corrector{system/problem/GenericCorrector.meta.xml}
  \begin{components}
    \component{js_lib}{system/problem/GenericMathlet.meta.xml}{gwtmathlet}
  \end{components}
  \begin{links}
  \end{links}
  \creategeneric
\end{metainfo}
\begin{content}

\begin{block}[annotation]
	Im Ticket-System: \href{https://team.mumie.net/issues/22657}{Ticket 22657}
\end{block}
\usepackage{mumie.genericproblem}
\lang{de}{\title{A08: Hauptachsen}}
\lang{en}{\title{P08: Principal axes}}
\begin{problem}
% Hauptachsentransformation
     \begin{question}
          \begin{variables}
              \randint{a1}{3}{5}
              \randint[Z]{b1}{-2}{2}
              \function{a}{a1}
              \function{b}{b1}
              \function{zero}{0}
              \function[calculate]{bbetrag}{|b|}
% Matrix A entsteht aus diag(a,b) durch Konjugieren mit der orthogonalen Matrix (1/25)*\begin{pmatrix}3&-4\\4&3\end{pmatrix}
              \matrix[calculate]{aa}{(9*a+16*b)/25&(-12*a+12*b)/25\\
                                    (-12*a+12*b)/25&(16*a+9*b)/25}
          \end{variables}     
%           \type{input.function} 
     \type{input.generic}
     \field{rational}
%     \displayprecision{3}
%     \correctorprecision{2}
     \lang{de}{\text{(a) Bringen Sie die Matrix $A=\var{aa}$ auf  Hauptachsenform $C$.\\ \\
     \begin{table}[\class{layout}]
      $C=\quad$ & 
      \begin{table}[\class{pmatrix}] \ansref & \ansref\\ \ansref &\ansref \end{table}  \end{table} 
     }}
      \lang{en}{\text{(a) Transform the matrix $A=\var{aa}$ into principal axes form $C$.\\ \\
     \begin{table}[\class{layout}]
      $C=\quad$ & 
      \begin{table}[\class{pmatrix}] \ansref & \ansref\\ \ansref &\ansref \end{table}  \end{table} 
     }}
          \begin{answer}
          \type{input.function}
          %\field{rational}
           \solution{a}
            \inputAsFunction{}{ans1}
            \checkFuncForZero{(ans1-a)*(ans1-b)}{1}{2}{10}
            
          \end{answer}     
          \begin{answer}
          \type{input.function}
         % \field{rational}
           \solution{zero}
            \inputAsFunction{}{ans3}
            \checkAsFunction{}{1}{2}{10}
            %\checkFuncForZero{ans3-zero}{1}{2}{10}
          \end{answer}     
          \begin{answer}
          \type{input.function}
          %\field{rational}
           \solution{zero}
            \inputAsFunction{}{ans4}
            \checkFuncForZero{ans4-zero}{1}{2}{10}
            %\explanation[NOT[equals(ans4,0)] OR NOT[equals(ans3,0)]]{Die Hauptachsenform einer Matrix ist eine Diagonalmatrix.}
            %\checkAsFunction{}{1}{2}{10}
          \end{answer}     
          \begin{answer}
          \type{input.function}
          %\field{rational}
           \solution{b}
            \inputAsFunction{}{ans2}
            \checkFuncForZero[1E-2]{|(ans2-a)*(ans2-b)|+dirac(ans1-ans2)}{1}{2}{10}
            
          %\explanation{Die Hauptachsengestalt beschreibt die quadratische Form $q_A$ bezüglich einer Basis aus Eigenvektoren.
          %              Die Matrix $C$ ist also eine Diagonalmatrix, deren Diagonalglieder die Eigenwerte von $A$ sind.}
          \lang{de}{
          \explanation[NOT[equal(ans4,0)] OR NOT[equal(ans3,0)]]{Zu (a): Die Hauptachsenform einer Matrix ist eine Diagonalmatrix.}
          \explanation[equal(ans4,0) AND equal(ans3,0) AND [NOT[equal(ans1,a) OR equal(ans1,b)]] AND [NOT[equal(ans2,a) OR equal(ans2,b)]]]{Zu (a): Sie haben die Eigenwerte nicht korrekt bestimmt.}
          \explanation[equal(ans4,0) AND equal(ans3,0) AND [NOT[equal(ans1,a) OR equal(ans1,b)]] AND [equal(ans2,a) OR equal(ans2,b)]]{Zu (a): Sie haben nur den zweiten Eigenwert korrekt bestimmt.}
          \explanation[equal(ans4,0) AND equal(ans3,0) AND [equal(ans1,a) OR equal(ans1,b)] AND [NOT[equal(ans2,a) OR equal(ans2,b)]]]{Zu (a): Sie haben nur den  ersten Eigenwert korrekt bestimmt.}
   }
   \lang{en}{
          \explanation[NOT[equal(ans4,0)] OR NOT[equal(ans3,0)]]{Zu (a): The principal axes form of a matrix is a diagonal matrix.}
          \explanation[equal(ans4,0) AND equal(ans3,0) AND [NOT[equal(ans1,a) OR equal(ans1,b)]] AND [NOT[equal(ans2,a) OR equal(ans2,b)]]]{Zu (a): You did not determine the eigenvalues correctly.}
          \explanation[equal(ans4,0) AND equal(ans3,0) AND [NOT[equal(ans1,a) OR equal(ans1,b)]] AND [equal(ans2,a) OR equal(ans2,b)]]{Zu (a): You only determined the second eigenvalues correctly.}
          \explanation[equal(ans4,0) AND equal(ans3,0) AND [equal(ans1,a) OR equal(ans1,b)] AND [NOT[equal(ans2,a) OR equal(ans2,b)]]]{Zu (a): You only determined the first eigenvalues correctly.}
   }
   \end{answer} 
   \begin{answer}
   \lang{de}{
   \text{(b) Welche Art Kegelschnitt beschreibt die Gleichung $\begin{pmatrix}x&y\end{pmatrix}\cdot A\cdot\begin{pmatrix}x\\y\end{pmatrix}=5$?}
   \type{mc.unique}        
        \permutechoices{1}{4}
 
        \begin{choice}
            \text{Ellipse}
            \solution{compute}
            \iscorrect{b}{>}{0}
        \end{choice}
 
        \begin{choice}
            \text{Parabel}
            \solution{false}
        \end{choice}
 
        \begin{choice}
            \text{Hyperbel}
            \solution{compute}
            \iscorrect{b}{<}{0}
        \end{choice}
 
        \begin{choice}
           \text{Geradenpaar}
           \solution{false}
        \end{choice}
  
    \explanation[edited(ans) AND [equal(b,bbetrag)]]{Zu (b): Weil die Eigenwerte beide positiv sind und ebenfalls $5>0$,
                                handelt es sich um eine Ellipse.}
    \explanation[edited(ans) AND NOT[equal(b,bbetrag)]]{Zu (b): Weil einer der Eigenwerte  positiv ist und der andere negativ, handelt es sich um eine Hyperbel.}                         
    }

     \lang{en}{
   \text{(b) What type of conic section is described by the equation $\begin{pmatrix}x&y\end{pmatrix}\cdot A\cdot\begin{pmatrix}x\\y\end{pmatrix}=5$?}
   \type{mc.unique}        
        \permutechoices{1}{4}
 
        \begin{choice}
            \text{Ellipse}
            \solution{compute}
            \iscorrect{b}{>}{0}
        \end{choice}
 
        \begin{choice}
            \text{Parabola}
            \solution{false}
        \end{choice}
 
        \begin{choice}
            \text{Hyperbola}
            \solution{compute}
            \iscorrect{b}{<}{0}
        \end{choice}
 
        \begin{choice}
           \text{Pair of linea}
           \solution{false}
        \end{choice}
  
    \explanation[edited(ans) AND [equal(b,bbetrag)]]{(b): Since bboth eigenvalues are positive and also $5>0$,
                                the equation describes an ellipse.}
    \explanation[edited(ans) AND NOT[equal(b,bbetrag)]]{(b): Since the eigenvalues ahve opposite signs, we have a hyperbola.}                         
    }
    \end{answer} 
          \end{question}   
       \end{problem}
     
     
\embedmathlet{gwtmathlet}

\end{content}
