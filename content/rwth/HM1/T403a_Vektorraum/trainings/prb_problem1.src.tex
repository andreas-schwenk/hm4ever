\documentclass{mumie.problem.gwtmathlet}
%$Id$
\begin{metainfo}
  \name{
    \lang{de}{A01: Koordinatendarstellung}
    \lang{en}{P01: Coordinate representation}
  }
  \begin{description} 
 This work is licensed under the Creative Commons License Attribution 4.0 International (CC-BY 4.0)   
 https://creativecommons.org/licenses/by/4.0/legalcode 

    \lang{de}{Beschreibung}
    \lang{en}{}
  \end{description}
  \corrector{system/problem/GenericCorrector.meta.xml}
  \begin{components}
    \component{js_lib}{system/problem/GenericMathlet.meta.xml}{gwtmathlet}
  \end{components}
  \begin{links}
  \end{links}
  \creategeneric
\end{metainfo}
\begin{content}
\begin{block}[annotation]
	Im Ticket-System: \href{https://team.mumie.net/issues/14365}{Ticket 14365}
\end{block}
\usepackage{mumie.genericproblem}

\lang{de}{
	\title{A01: Koordinatendarstellung}
}
\lang{en}{
	\title{P01: Coordinate representation}
}

\begin{block}[annotation]
	Im Ticket-System: \href{http://team.mumie.net/issues/}{Ticket }
\end{block}



\begin{problem}

    
\begin{question}

	\begin{variables}
		\randint[Z]{a}{-9}{9}                   % a=0 ist im Command radiant[Z] ausgeschlossen
        \randint{b}{-5}{5}
        \randint[Z]{c}{-5}{5}
        \function[expand]{p}{-a*x+1}            
        \function[expand]{q}{x+a}
        \function[expand, normalize]{r}{b*x+c}  % "normalize" bewirkt:  falls b=0, wird 
                                                %  für \var{r} c statt 0x+c ausgegeben
		

			\pmatrix[calculate]{aa}{
  			1/(-a^2-1)*(a*b-c)\\ 1/(-a^2-1)*(-b-a*c)
      	}


	\end{variables}

	\type{input.matrix}
	\displayprecision{3}
    \correctorprecision{2}
    \field{real}
    
    \lang{de}{
	    \text{
	    Bestimmen Sie die Koordinatendarstellung von $\var{r}$ 
        bezüglich der Basis $\{\var{p},\var{q} \}.$
        }       
        \explanation{ 
        Die Koordinatendarstellung $\begin{pmatrix} \lambda_1 \\ \lambda_2 \end{pmatrix}$ von $\var{r}$ 
        bezüglich der Basis $\{\var{p},\var{q} \}$ ergibt sich aus der Gleichung

         $\lambda_1 \cdot (\var{p}) + \lambda_2 \cdot (\var{q}) = \var{r}$
         und einem Koeffizientenvergleich.
        }
        }
      \lang{en}{
	    \text{
	    Determine the coordinate representation of $\var{r}$ 
        with respect of the basis $\{\var{p},\var{q} \}.$
        }       
        \explanation{ 
        The coordinate representation $\begin{pmatrix} \lambda_1 \\ \lambda_2 \end{pmatrix}$ of $\var{r}$ 
        with respect of the basis $\{\var{p},\var{q} \}$ results from the equation

         $\lambda_1 \cdot (\var{p}) + \lambda_2 \cdot (\var{q}) = \var{r}$
         a comparison of the coefficients.
        }
        }
    
    \begin{answer}
	    \solution{aa}
      \format{2}{1}
	\end{answer}
    
\end{question}

\end{problem}


\embedmathlet{gwtmathlet}

\end{content}
