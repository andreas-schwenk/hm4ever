\documentclass{mumie.problem.gwtmathlet}
%$Id$
\begin{metainfo}
  \name{
    \lang{de}{A05: Orthogonalbasen / ONB}
    \lang{en}{P05: Orthogonal basis und orthonormal basis}
  }
  \begin{description} 
 This work is licensed under the Creative Commons License Attribution 4.0 International (CC-BY 4.0)   
 https://creativecommons.org/licenses/by/4.0/legalcode 

    \lang{de}{Beschreibung}
    \lang{en}{}
  \end{description}
  \corrector{system/problem/GenericCorrector.meta.xml}
  \begin{components}
    \component{js_lib}{system/problem/GenericMathlet.meta.xml}{gwtmathlet}
  \end{components}
  \begin{links}
  \end{links}
  \creategeneric
\end{metainfo}
\begin{content}
\begin{block}[annotation]
	Im Ticket-System: \href{https://team.mumie.net/issues/14280}{Ticket 14280}
\end{block}
\usepackage{mumie.genericproblem}
\usepackage{mumie.ombplus}

\lang{de}{
	\title{A05: Orthogonalbasen / Orthonormalbasen}
}
\lang{en}{
	\title{P05: Orthogonal basis and orthonormal basis}
}

\begin{block}[annotation]
	Im Ticket-System: \href{http://team.mumie.net/issues/}{Ticket }
\end{block}

\lang{de}{
\begin{problem}   
     \randomquestionpool{1}{4}
%     \randomquestionpool{1}{2}
%     \randomquestionpool{3}{4}
     %Q1
     \begin{question}
     
       \text{Gegeben seien die Vektoren $v_1=\var{a}, v_2= \var{b}, v_3=\var{c} \in \R^3$. 
       Wir betrachten den Vektorraum $\R^3$ zusammen mit dem Standard-Skalarprodukt. 
       Wählen Sie alle korrekten Antworten aus.
       
     %Test Q1: det=$\var{det}$, m=$\var{m}$, norm=$\var{norm}$, orth=$\var{orthog}$
       
       }
   
       \permutechoices{1}{4}
       \type{mc.yesno}
       \field{real}
       
        \begin{variables}
             \randint{m}{0}{4}
             \begin{switch}
                \begin{case}{m=0}
                    \function{x}{0}
                    \function{y}{-1}
                \end{case}
                \begin{case}{m=1}
                    \function{x}{1/2}
                    \function{y}{-sqrt(3)/2}
                \end{case}
                \begin{case}{m=3}
                    \function{x}{sqrt(3)/2}
                    \function{y}{-1/2}
                \end{case}
                \begin{case}{m=4}
                    \function{x}{1}
                    \function{y}{0}
                \end{case}
                \begin{default}%m=2
                    \function{x}{sqrt(2)/2}
                    \function{y}{-sqrt(2)/2}
                \end{default}
             \end{switch}
             
             \randint[Z]{z}{-1}{2}
             \pmatrix{a}{0 \\ 1 \\ 0}
             \pmatrix{b}{x \\ 0 \\ y}
             \pmatrix{c}{z \\ 0 \\ 0}
             \function[calculate]{det}{y*z}
             \function[calculate]{norm}{abs(z)}
             \function[calculate]{orthog}{(m+1)*norm}%nur orthonormal, falls m=0 und z=+-1
         \end{variables}
%
% Erklärung bei Fehler 
% keine Basis

                
                \explanation[[det=0] AND [equalChoice(111?)]]{
                $\{v_1;v_2;v_3\}$ bildet keine Basis und somit auch keine Orthogonal- und keine Orthonormalbasis von $\R^3$.
                }
                \explanation[[det=0] AND [equalChoice(101?)]]{
                $\{v_1;v_2;v_3\}$ bildet keine Basis und somit auch keine Orthonormalbasis von $\R^3$.
                }
                \explanation[[det=0] AND [equalChoice(110?)]]{
                $\{v_1;v_2;v_3\}$ bildet keine Basis und somit auch keine Orthogonalbasis von $\R^3$.
                }
                \explanation[[det=0] AND [equalChoice(100?)]]{
                $\{v_1;v_2;v_3\}$ bildet keine Basis von $\R^3$.
                }
                \explanation[[det=0] AND [equalChoice(011?)]]{
                $\{v_1;v_2;v_3\}$ bildet keine Orthogonal- und auch keine Orthonormalbasis von $\R^3$.
                }
                \explanation[[det=0] AND [equalChoice(001?)]]{
                $\{v_1;v_2;v_3\}$ bildet keine Orthonormalbasis von $\R^3$.
                }
                \explanation[[det=0] AND [equalChoice(010?)]]{
                $\{v_1;v_2;v_3\}$ bildet keine Orthogonalbasis von $\R^3$.
                }
% Basis aber nicht orthogonal 
                \explanation[[equalChoice(011?)] AND NOT [det=0] AND NOT [m=0]]{   
                $\{v_1;v_2;v_3\}$ bildet eine Basis, aber keine Orthogonal- und damit auch keine Orthonormalbasis von $\R^3$.
                }
                \explanation[[equalChoice(010?)] AND NOT [det=0] AND NOT [m=0]]{   
                $\{v_1;v_2;v_3\}$ bildet eine Basis, aber keine Orthogonalbasis von $\R^3$.
                }
                \explanation[[equalChoice(001?)] AND NOT [det=0] AND NOT [m=0] AND NOT [orthog=1]]{   
                $\{v_1;v_2;v_3\}$ bildet eine Basis, aber keine Orthonormalbasis von $\R^3$.
                }
                \explanation[[equalChoice(111?)] AND NOT [det=0] AND NOT [m=0] AND NOT [orthog=1]]{   
                $\{v_1;v_2;v_3\}$ bildet keine Orthogonal- und damit auch keine Orthonormalbasis von $\R^3$.
                }
                \explanation[[equalChoice(110?)] AND NOT [det=0] AND NOT [m=0]]{   
                $\{v_1;v_2;v_3\}$ bildet keine Orthogonalbasis von $\R^3$.
                }
                \explanation[[equalChoice(101?)] AND NOT [det=0] AND NOT [m=0] AND NOT [orthog=1]]{   
                $\{v_1;v_2;v_3\}$ bildet keine Orthogonalbasis von $\R^3$.
                }
                \explanation[[equalChoice(000?)] AND NOT [det=0] AND NOT [m=0] AND NOT [orthog=1]]{   
                $\{v_1;v_2;v_3\}$ bildet eine Basis von $\R^3$.
                } 
% Basis orthogonal
                \explanation[[equalChoice(001?)] AND NOT [det=0] AND [m=0] AND NOT [orthog=1]]{   
                $\{v_1;v_2;v_3\}$ bildet eine Orthogonalbasis, also auch eine Basis von $\R^3$ aber keine Orthonormalbasis.
                }                
% Basis orthonormal 
                \explanation[[[equalChoice(0???)] OR [equalChoice(?0??)] OR [equalChoice(??0?)]] AND NOT [det=0] AND [m=0] AND [orthog=1]]{ 
                $\{v_1;v_2;v_3\}$ bildet eine Orthonormalbasis, also insbesondere auch
                eine Orthogonalbasis und eine Basis von $\R^3$.
                }
% Vektoren normiert?
                \explanation[[equalChoice(???0)] AND [norm=1]]{   
                Die Vektoren $v_1, v_2, v_3$ sind alle bez\"uglich des Standard-Skalarprodukts normiert.
                }                                              
                \explanation[[equalChoice(???1)] AND NOT [norm=1]]{
                Der Vektor $v_3$ ist nicht normiert.
                % bez\"uglich des Standard-Skalarprodukts normiert.
                }
% Ende
%
        \begin{choice}
              \text{$\{v_1;v_2;v_3\}$ bildet eine Basis von $\R^3$.}
              \solution{compute}
              \iscorrect{det}{!=}{0}
        \end{choice}
         \begin{choice}
               \text{$\{v_1;v_2;v_3\}$ bildet eine Orthogonalbasis von $\R^3$.}
               \solution{compute}
               \iscorrect{m}{=}{0}
         \end{choice}
         \begin{choice}
               \text{$\{v_1;v_2;v_3\}$ bildet eine Orthonormalbasis von $\R^3$.}
               \solution{compute}
               \iscorrect{orthog}{=}{1}
         \end{choice}
         \begin{choice}
               \text{Die Vektoren $v_1, v_2, v_3$ sind alle bezüglich des Standard-Skalarprodukts normiert,
               d.h. $\langle v_i,v_i \rangle = 1$ für $i=1,2,3$.}
               \solution{compute}     
               \iscorrect{norm}{=}{1}
         \end{choice}
    \end{question}
    %Q2
    \begin{question}
       \text{Gegeben seien die Vektoren $v_1=\var{a}, v_2= \var{b}, v_3=\var{c} \in \R^3$. 
       Wir betrachten den Vektorraum $\R^3$ zusammen mit dem Standard-Skalarprodukt. 
       Wählen Sie alle korrekten Antworten aus.
       
     %Test Q2: det=$\var{det}$, m=$\var{m}$, norm=$\var{norm}$, orth=$\var{orthog}$
       
       }     

       \permutechoices{1}{4}
       \type{mc.yesno}
       \field{real}
       
        \begin{variables}
             \randint{m}{0}{4}
             \begin{switch}
                \begin{case}{m=0}
                    \function{x}{0}
                    \function{y}{-1}
                \end{case}
                \begin{case}{m=1}
                    \function{x}{1/2}
                    \function{y}{-sqrt(3)/2}
                \end{case}
                \begin{case}{m=3}
                    \function{x}{sqrt(3)/2}
                    \function{y}{-1/2}
                \end{case}
                \begin{case}{m=4}
                    \function{x}{1}
                    \function{y}{0}
                \end{case}
                \begin{default}%m=2
                    \function{x}{sqrt(2)/2}
                    \function{y}{-sqrt(2)/2}
                \end{default}
             \end{switch}
             
             \randint[Z]{z}{-1}{2}
             \pmatrix{a}{0 \\ 0 \\ 1}
             \pmatrix{b}{0 \\ z \\ 0}
             \pmatrix{c}{y \\ 0 \\ x}
             %\function[calculate]{det}{x*z}  dies ist falsch
             \function[calculate]{det}{-y*z}
             \function[calculate]{norm}{abs(z)}
             \function[calculate]{orthog}{(m+1)*norm}%nur orthonormal, falls m=4 und z=+-1
         \end{variables}
%
% Erklärung bei Fehler 
% keine Basis
                \explanation[[det=0] AND [equalChoice(111?)]]{
                $\{v_1;v_2;v_3\}$ bildet keine Basis und somit auch keine Orthogonal- und keine Orthonormalbasis von $\R^3$.
                }
                \explanation[[det=0] AND [equalChoice(101?)]]{
                $\{v_1;v_2;v_3\}$ bildet keine Basis und somit auch keine Orthonormalbasis von $\R^3$.
                }
                \explanation[[det=0] AND [equalChoice(110?)]]{
                $\{v_1;v_2;v_3\}$ bildet keine Basis und somit auch keine Orthogonalbasis von $\R^3$.
                }
                \explanation[[det=0] AND [equalChoice(100?)]]{
                $\{v_1;v_2;v_3\}$ bildet keine Basis von $\R^3$.
                }
                \explanation[[det=0] AND [equalChoice(011?)]]{
                $\{v_1;v_2;v_3\}$ bildet keine Orthogonal- und auch keine Orthonormalbasis von $\R^3$.
                }
                \explanation[[det=0] AND [equalChoice(001?)]]{
                $\{v_1;v_2;v_3\}$ bildet keine Orthonormalbasis von $\R^3$.
                }
                \explanation[[det=0] AND [equalChoice(010?)]]{
                $\{v_1;v_2;v_3\}$ bildet keine Orthogonalbasis von $\R^3$.
                }
% Basis aber nicht orthogonal 
                \explanation[[equalChoice(011?)] AND NOT [det=0] AND NOT [m=0]]{   
                $\{v_1;v_2;v_3\}$ bildet eine Basis, aber keine Orthogonal- und damit auch keine Orthonormalbasis von $\R^3$.
                }
                \explanation[[equalChoice(010?)] AND NOT [det=0] AND NOT [m=0]]{   
                $\{v_1;v_2;v_3\}$ bildet eine Basis, aber keine Orthogonalbasis von $\R^3$.
                }
                \explanation[[equalChoice(001?)] AND NOT [det=0] AND NOT [m=0] AND NOT [orthog=1]]{   
                $\{v_1;v_2;v_3\}$ bildet eine Basis, aber keine Orthonormalbasis von $\R^3$.
                }
                \explanation[[equalChoice(111?)] AND NOT [det=0] AND NOT [m=0] AND NOT [orthog=1]]{   
                $\{v_1;v_2;v_3\}$ bildet keine Orthogonal- und damit auch keine Orthonormalbasis von $\R^3$.
                }
                \explanation[[equalChoice(110?)] AND NOT [det=0] AND NOT [m=0]]{   
                $\{v_1;v_2;v_3\}$ bildet keine Orthogonalbasis von $\R^3$.
                }
                \explanation[[equalChoice(101?)] AND NOT [det=0] AND NOT [m=0] AND NOT [orthog=1]]{   
                $\{v_1;v_2;v_3\}$ bildet keine Orthogonalbasis von $\R^3$.
                }
                \explanation[[equalChoice(000?)] AND NOT [det=0] AND NOT [m=0] AND NOT [orthog=1]]{   
                $\{v_1;v_2;v_3\}$ bildet eine Basis von $\R^3$.
                } 
% Basis orthogonal
                \explanation[[equalChoice(001?)] AND NOT [det=0] AND [m=0] AND NOT [orthog=1]]{   
                $\{v_1;v_2;v_3\}$ bildet eine Orthogonalbasis, also auch eine Basis von $\R^3$ aber keine Orthonormalbasis.
                }                
% Basis orthonormal 
                \explanation[[[equalChoice(0???)] OR [equalChoice(?0??)] OR [equalChoice(??0?)]] AND NOT [det=0] AND [m=0] AND [orthog=1]]{ 
                $\{v_1;v_2;v_3\}$ bildet eine Orthonormalbasis, also insbesondere auch
                eine Orthogonalbasis und eine Basis von $\R^3$.
                }
% Vektoren normiert?
                \explanation[[equalChoice(???0)] AND [norm=1]]{   
                Die Vektoren $v_1, v_2, v_3$ sind alle bez\"uglich des Standard-Skalarprodukts normiert.
                }                                              
                \explanation[[equalChoice(???1)] AND NOT [norm=1]]{
                Der Vektor $v_2$ ist nicht normiert.
                % bez\"uglich des Standard-Skalarprodukts normiert.
                }
% Ende
%
 
        \begin{choice}
              \text{$\{v_1;v_2;v_3\}$ bildet eine Basis von $\R^3$.}
              \solution{compute}
              \iscorrect{det}{!=}{0}
        \end{choice}
         \begin{choice}
               \text{$\{v_1;v_2;v_3\}$ bildet eine Orthogonalbasis von $\R^3$.}
               \solution{compute}
               \iscorrect{m}{=}{0}
         \end{choice}
         \begin{choice}
               \text{$\{v_1;v_2;v_3\}$ bildet eine Orthonormalbasis von $\R^3$.}
               \solution{compute}
               \iscorrect{orthog}{=}{1}
         \end{choice}
         \begin{choice}
               \text{Die Vektoren $v_1;v_2;v_3$ sind alle bezüglich des Standard-Skalarprodukts normiert,
               d.h. $\langle v_i,v_i \rangle = 1$ für $i=1,2,3$.}
               \solution{compute}     
               \iscorrect{norm}{=}{1}
         \end{choice}
    \end{question}
    
    %-----------------------
    %Q3
    \begin{question}
       \text{Gegeben seien die Vektoren $v_1=\var{a}, v_2= \var{b}, v_3=\var{c} \in \R^3$. 
       Wir betrachten den Vektorraum $\R^3$ zusammen mit dem Standard-Skalarprodukt. 
       Wählen Sie alle korrekten Antworten aus.
     %Test  Teil Q3
       }
       \permutechoices{1}{4}
       \type{mc.yesno}
       \field{real}
  
        \begin{variables}
             \randint{m}{0}{4}
             \begin{switch}
                \begin{case}{m=0}
                    \function{x}{0}
                    \function{y}{-1}
                    \function{y1}{1}
                \end{case}
                \begin{case}{m=1}
                    \function{x}{1/2}
                    \function{y}{sqrt(3)/2}
                    \function{y1}{-sqrt(3)/2}
                \end{case}
                \begin{case}{m=3}
                    \function{x}{sqrt(3)/2}
                    \function{y}{1/2}
                    \function{y1}{-1/2}
                \end{case}
                \begin{case}{m=4}
                    \function{x}{1}
                    \function{y}{0}
                    \function{y1}{0}
                \end{case}
                \begin{default}%m=2
                    \function{x}{sqrt(2)/2}
                    \function{y}{sqrt(2)/2}
                    \function{y1}{-sqrt(2)/2}
                \end{default}
             \end{switch}
             
             \randint[Z]{z}{-2}{2}
             \pmatrix{a}{z \\ 0 \\ 0}
             \pmatrix{b}{0 \\ x \\ y}
             \pmatrix{c}{0 \\ y1 \\ x}
             \function[calculate]{norm}{abs(z)}
         \end{variables}
%
% Erklärung bei Fehler 
%
                \explanation[[equalChoice(010?)] AND NOT [norm=1]]{
                $\{v_1;v_2;v_3\}$ bildet auch eine Basis von $\R^3$.
                }
                \explanation[[equalChoice(100?)] AND NOT [norm=1]]{
                $\{v_1;v_2;v_3\}$ bildet auch eine Orthogonalbasis von $\R^3$.
                }
                \explanation[[equalChoice(000?)] AND NOT [norm=1]]{
                $\{v_1;v_2;v_3\}$ bildet eine Basis und auch eine Orthogonalbasis von $\R^3$.
                }

                \explanation[[equalChoice(000?)] AND [norm=1]]{   
                $\{v_1;v_2;v_3\}$ bildet eine Orthonormalbasis und somit auch eine Orthogonalbasis
                und eine Basis von $\R^3$.
                }
                \explanation[[equalChoice(100?)] AND [norm=1]]{   
                $\{v_1;v_2;v_3\}$ bildet eine Orthonormalbasis und somit auch eine Orthogonalbasis
                von $\R^3$.
                }
                \explanation[[equalChoice(010?)] AND [norm=1]]{   
                $\{v_1;v_2;v_3\}$ bildet eine Orthonormalbasis und somit auch eine Basis von $\R^3$.
                }
                \explanation[[equalChoice(001?)] AND [norm=1]]{   
                $\{v_1;v_2;v_3\}$ bildet auch eine Orthogonalbasis und eine Basis von $\R^3$.
                }
                \explanation[[equalChoice(101?)] AND [norm=1]]{   
                $\{v_1;v_2;v_3\}$ bildet auch eine Orthogonalbasis von $\R^3$.
                }
                \explanation[[equalChoice(011?)] AND [norm=1]]{   
                $\{v_1;v_2;v_3\}$ bildet auch eine Basis von $\R^3$.
                }
                \explanation[[equalChoice(110?)] AND [norm=1]]{   
                $\{v_1;v_2;v_3\}$ bildet eine Orthonormalbasis von $\R^3$.
                }
                
                \explanation[[equalChoice(001?)] AND NOT [norm=1]]{   
                $\{v_1;v_2;v_3\}$ bildet eine Basis und auch eine Orthogonalbasis, aber
                keine Orthonormalbasis von $\R^3$.
                }
                \explanation[[equalChoice(011?)] AND NOT [norm=1]]{   
                $\{v_1;v_2;v_3\}$ bildet eine Basis, aber keine Orthonormalbasis von $\R^3$.
                }
                \explanation[[equalChoice(101?)] AND NOT [norm=1]]{   
                $\{v_1;v_2;v_3\}$ bildet eine Orthogonalbasis, aber keine Orthonormalbasis von $\R^3$.
                }
                \explanation[[equalChoice(111?)] AND NOT [norm=1]]{   
                $\{v_1;v_2;v_3\}$ bildet keine Orthonormalbasis von $\R^3$.
                }

                \explanation[[equalChoice(???0)] AND [norm=1]]{   
                Die Vektoren $v_1, v_2, v_3$ sind alle bez\"uglich des Standard-Skalarprodukts normiert.
                }                                              
                \explanation[[equalChoice(???1)] AND NOT [norm=1]]{
                Der Vektor $v_1$ ist nicht normiert.
                % bez\"uglich des Standard-Skalarprodukts normiert.
                }
% Ende
 
        \begin{choice}
              \text{$\{v_1;v_2;v_3\}$ bildet eine Basis von $\R^3$.}
              \solution{true}
        \end{choice}
         \begin{choice}
               \text{$\{v_1;v_2;v_3\}$ bildet eine Orthogonalbasis von $\R^3$.}
               \solution{true}
         \end{choice}
         \begin{choice}
               \text{$\{v_1;v_2;v_3\}$ bildet eine Orthonormalbasis von $\R^3$.}
               \solution{compute}
               \iscorrect{norm}{=}{1}
         \end{choice}
         \begin{choice}
               \text{Die Vektoren $v_1;v_2;v_3$ sind alle bezüglich des Standard-Skalarprodukts normiert,
               d.h. $\langle v_i,v_i \rangle = 1$ für $i=1,2,3$.}
               \solution{compute}     
               \iscorrect{norm}{=}{1}
         \end{choice}
    \end{question}
    
    %Q4
    \begin{question}
       \text{Gegeben seien die Vektoren $v_1=\var{a}, v_2= \var{b}, v_3=\var{c} \in \R^3$. 
       Wir betrachten den Vektorraum $\R^3$ zusammen mit dem Standard-Skalarprodukt. 
       Wählen Sie alle korrekten Antworten aus.
     %Test  Teil Q4
       }
       \permutechoices{1}{4}
       \type{mc.yesno}
       \field{real}
  
        \begin{variables}
             \randint{m}{0}{4}
             \begin{switch}
                \begin{case}{m=0}
                    \function{x}{0}
                    \function{y}{-1}
                    \function{y1}{1}
                \end{case}
                \begin{case}{m=1}
                    \function{x}{1/2}
                    \function{y}{sqrt(3)/2}
                    \function{y1}{-sqrt(3)/2}
                \end{case}
                \begin{case}{m=3}
                    \function{x}{sqrt(3)/2}
                    \function{y}{1/2}
                    \function{y1}{-1/2}
                \end{case}
                \begin{case}{m=4}
                    \function{x}{1}
                    \function{y}{0}
                    \function{y1}{0}
                \end{case}
                \begin{default}%m=2
                    \function{x}{sqrt(2)/2}
                    \function{y}{sqrt(2)/2}
                    \function{y1}{-sqrt(2)/2}
                \end{default}
             \end{switch}
             
             \randint[Z]{z}{-2}{2}
             \pmatrix{a}{x \\ 0 \\ y}
             \pmatrix{b}{0 \\ z \\ 0}
             \pmatrix{c}{y1 \\ 0 \\ x}
             \function[calculate]{norm}{abs(z)}
         \end{variables}
%
% Erklärung bei Fehler 
%
                \explanation[[equalChoice(010?)] AND NOT [norm=1]]{
                $\{v_1;v_2;v_3\}$ bildet auch eine Basis von $\R^3$.
                }
                \explanation[[equalChoice(100?)] AND NOT [norm=1]]{
                $\{v_1;v_2;v_3\}$ bildet auch eine Orthogonalbasis von $\R^3$.
                }
                \explanation[[equalChoice(000?)] AND NOT [norm=1]]{
                $\{v_1;v_2;v_3\}$ bildet eine Basis und auch eine Orthogonalbasis von $\R^3$.
                }

                \explanation[[equalChoice(000?)] AND [norm=1]]{   
                $\{v_1;v_2;v_3\}$ bildet eine Orthonormalbasis und somit auch eine Orthogonalbasis
                und eine Basis von $\R^3$.
                }
                \explanation[[equalChoice(100?)] AND [norm=1]]{   
                $\{v_1;v_2;v_3\}$ bildet eine Orthonormalbasis und somit auch eine Orthogonalbasis
                von $\R^3$.
                }
                \explanation[[equalChoice(010?)] AND [norm=1]]{   
                $\{v_1;v_2;v_3\}$ bildet eine Orthonormalbasis und somit auch eine Basis von $\R^3$.
                }
                \explanation[[equalChoice(001?)] AND [norm=1]]{   
                $\{v_1;v_2;v_3\}$ bildet auch eine Orthogonalbasis und eine Basis von $\R^3$.
                }
                \explanation[[equalChoice(101?)] AND [norm=1]]{   
                $\{v_1;v_2;v_3\}$ bildet auch eine Orthogonalbasis von $\R^3$.
                }
                \explanation[[equalChoice(011?)] AND [norm=1]]{   
                $\{v_1;v_2;v_3\}$ bildet auch eine Basis von $\R^3$.
                }                
                \explanation[[equalChoice(110?)] AND [norm=1]]{   
                $\{v_1;v_2;v_3\}$ bildet eine Orthonormalbasis von $\R^3$.
                }
                
                \explanation[[equalChoice(001?)] AND NOT [norm=1]]{   
                $\{v_1;v_2;v_3\}$ bildet eine Basis und auch eine Orthogonalbasis, aber
                keine Orthonormalbasis von $\R^3$.
                }
                \explanation[[equalChoice(011?)] AND NOT [norm=1]]{   
                $\{v_1;v_2;v_3\}$ bildet eine Basis, aber keine Orthonormalbasis von $\R^3$.
                }
                \explanation[[equalChoice(101?)] AND NOT [norm=1]]{   
                $\{v_1;v_2;v_3\}$ bildet eine Orthogonalbasis, aber keine Orthonormalbasis von $\R^3$.
                }
                \explanation[[equalChoice(111?)] AND NOT [norm=1]]{   
                $\{v_1;v_2;v_3\}$ bildet keine Orthonormalbasis von $\R^3$.
                }

                \explanation[[equalChoice(???0)] AND [norm=1]]{   
                Die Vektoren $v_1, v_2, v_3$ sind alle bez\"uglich des Standard-Skalarprodukts normiert.
                }                                              
                \explanation[[equalChoice(???1)] AND NOT [norm=1]]{
                Der Vektor $v_2$ ist nicht normiert.
                % bez\"uglich des Standard-Skalarprodukts normiert.
                }
% Ende


        \begin{choice}
              \text{$\{v_1;v_2;v_3\}$ bildet eine Basis von $\R^3$.} 
              \solution{true}
        \end{choice}
         \begin{choice}
               \text{$\{v_1;v_2;v_3\}$ bildet eine Orthogonalbasis von $\R^3$.}
               \solution{true}
         \end{choice}
         \begin{choice}
               \text{$\{v_1;v_2;v_3\}$ bildet eine Orthonormalbasis von $\R^3$.}
               \solution{compute}
               \iscorrect{norm}{=}{1}
         \end{choice}
         \begin{choice}
               \text{Die Vektoren $v_1;v_2;v_3$ sind alle bezüglich des Standard-Skalarprodukts normiert,
               d.h. $\langle v_i,v_i \rangle = 1$ für $i=1,2,3$.}
               \solution{compute}     
               \iscorrect{norm}{=}{1}
         \end{choice}
    \end{question}
    
\end{problem}}





\lang{en}{
\begin{problem}   
     \randomquestionpool{1}{4}
%     \randomquestionpool{1}{2}
%     \randomquestionpool{3}{4}
     %Q1
     \begin{question}
     
       \text{Given are the vectors $v_1=\var{a}, v_2= \var{b}, v_3=\var{c} \in \R^3$. 
       We consider the vector space $\R^3$ with the standard scalar product. 
       Choose the correct answers.
       
     %Test Q1: det=$\var{det}$, m=$\var{m}$, norm=$\var{norm}$, orth=$\var{orthog}$
       
       }
   
       \permutechoices{1}{4}
       \type{mc.yesno}
       \field{real}
       
        \begin{variables}
             \randint{m}{0}{4}
             \begin{switch}
                \begin{case}{m=0}
                    \function{x}{0}
                    \function{y}{-1}
                \end{case}
                \begin{case}{m=1}
                    \function{x}{1/2}
                    \function{y}{-sqrt(3)/2}
                \end{case}
                \begin{case}{m=3}
                    \function{x}{sqrt(3)/2}
                    \function{y}{-1/2}
                \end{case}
                \begin{case}{m=4}
                    \function{x}{1}
                    \function{y}{0}
                \end{case}
                \begin{default}%m=2
                    \function{x}{sqrt(2)/2}
                    \function{y}{-sqrt(2)/2}
                \end{default}
             \end{switch}
             
             \randint[Z]{z}{-1}{2}
             \pmatrix{a}{0 \\ 1 \\ 0}
             \pmatrix{b}{x \\ 0 \\ y}
             \pmatrix{c}{z \\ 0 \\ 0}
             \function[calculate]{det}{y*z}
             \function[calculate]{norm}{abs(z)}
             \function[calculate]{orthog}{(m+1)*norm}%nur orthonormal, falls m=0 und z=+-1
         \end{variables}
%
% Erklärung bei Fehler 
% keine Basis

                
                \explanation[[det=0] AND [equalChoice(111?)]]{
                $\{v_1;v_2;v_3\}$ is not a basis and therefore neither a orthogonal basis nor a orthonormal basis of$\R^3$.
                }
                \explanation[[det=0] AND [equalChoice(101?)]]{
                $\{v_1;v_2;v_3\}$ is not a basis and therefore no orthonormal basis of $\R^3$.
                }
                \explanation[[det=0] AND [equalChoice(110?)]]{
                $\{v_1;v_2;v_3\}$ is not a basis and therefore not an orthogonal basis $\R^3$.
                }
                \explanation[[det=0] AND [equalChoice(100?)]]{
                $\{v_1;v_2;v_3\}$ is not a basis of $\R^3$.
                }
                \explanation[[det=0] AND [equalChoice(011?)]]{
                $\{v_1;v_2;v_3\}$ is neither an orthogonal nor an orthonormal basis of $\R^3$.
                }
                \explanation[[det=0] AND [equalChoice(001?)]]{
                $\{v_1;v_2;v_3\}$ is not an orthonormal basis of $\R^3$.
                }
                \explanation[[det=0] AND [equalChoice(010?)]]{
                $\{v_1;v_2;v_3\}$ is not an orthogonal basis of $\R^3$.
                }
% Basis aber nicht orthogonal 
                \explanation[[equalChoice(011?)] AND NOT [det=0] AND NOT [m=0]]{   
                $\{v_1;v_2;v_3\}$ is a basis, but not an orthogonal and therefore not an orthonormal basis of $\R^3$.
                }
                \explanation[[equalChoice(010?)] AND NOT [det=0] AND NOT [m=0]]{   
                $\{v_1;v_2;v_3\}$ forms a basis, but not an orthognal basis of $\R^3$.
                }
                \explanation[[equalChoice(001?)] AND NOT [det=0] AND NOT [m=0] AND NOT [orthog=1]]{   
                $\{v_1;v_2;v_3\}$ forms a basis, but not an orthonormal basis of $\R^3$.
                }
                \explanation[[equalChoice(111?)] AND NOT [det=0] AND NOT [m=0] AND NOT [orthog=1]]{   
                $\{v_1;v_2;v_3\}$ is not an orthogonal and therefore not an orthonormal basis of $\R^3$.
                }
                \explanation[[equalChoice(110?)] AND NOT [det=0] AND NOT [m=0]]{   
                $\{v_1;v_2;v_3\}$ is not an orthogonal basis of $\R^3$.
                }
                \explanation[[equalChoice(101?)] AND NOT [det=0] AND NOT [m=0] AND NOT [orthog=1]]{   
                $\{v_1;v_2;v_3\}$ is not an orthgonal basis of $\R^3$.
                }
                \explanation[[equalChoice(000?)] AND NOT [det=0] AND NOT [m=0] AND NOT [orthog=1]]{   
                $\{v_1;v_2;v_3\}$ forms a basis of $\R^3$.
                } 
% Basis orthogonal
                \explanation[[equalChoice(001?)] AND NOT [det=0] AND [m=0] AND NOT [orthog=1]]{   
                $\{v_1;v_2;v_3\}$ is a orthogonal basis, so also a basis of $\R^3$ but not an orthonormal basis.
                }                
% Basis orthonormal 
                \explanation[[[equalChoice(0???)] OR [equalChoice(?0??)] OR [equalChoice(??0?)]] AND NOT [det=0] AND [m=0] AND [orthog=1]]{ 
                $\{v_1;v_2;v_3\}$ forms a orthonormal basis, so especially an orthogonal basis and a basis of $\R^3$.
                }
% Vektoren normiert?
                \explanation[[equalChoice(???0)] AND [norm=1]]{   
                The vectors $v_1, v_2, v_3$ are normalised with respect of the standard basis.
                }                                              
                \explanation[[equalChoice(???1)] AND NOT [norm=1]]{
                The vecotr is $v_3$ not normalised.
                % bez\"uglich des Standard-Skalarprodukts normiert.
                }
% Ende
%
        \begin{choice}
              \text{$\{v_1;v_2;v_3\}$ forms a basis of $\R^3$.}
              \solution{compute}
              \iscorrect{det}{!=}{0}
        \end{choice}
         \begin{choice}
               \text{$\{v_1;v_2;v_3\}$ is an orthogonal basis of von $\R^3$.}
               \solution{compute}
               \iscorrect{m}{=}{0}
         \end{choice}
         \begin{choice}
               \text{$\{v_1;v_2;v_3\}$ is an orthonormal basis of $\R^3$.}
               \solution{compute}
               \iscorrect{orthog}{=}{1}
         \end{choice}
         \begin{choice}
               \text{The vectors $v_1, v_2, v_3$ are all normalised with respect to the standard basis,
               so $\langle v_i,v_i \rangle = 1$ for $i=1,2,3$.}
               \solution{compute}     
               \iscorrect{norm}{=}{1}
         \end{choice}
    \end{question}
    %Q2
    \begin{question}
       \text{Given are the vectors $v_1=\var{a}, v_2= \var{b}, v_3=\var{c} \in \R^3$. 
       We consider the vector space $\R^3$ with the standard scalar product. 
       Choose all the correct answers.
       
     %Test Q2: det=$\var{det}$, m=$\var{m}$, norm=$\var{norm}$, orth=$\var{orthog}$
       
       }     

       \permutechoices{1}{4}
       \type{mc.yesno}
       \field{real}
       
        \begin{variables}
             \randint{m}{0}{4}
             \begin{switch}
                \begin{case}{m=0}
                    \function{x}{0}
                    \function{y}{-1}
                \end{case}
                \begin{case}{m=1}
                    \function{x}{1/2}
                    \function{y}{-sqrt(3)/2}
                \end{case}
                \begin{case}{m=3}
                    \function{x}{sqrt(3)/2}
                    \function{y}{-1/2}
                \end{case}
                \begin{case}{m=4}
                    \function{x}{1}
                    \function{y}{0}
                \end{case}
                \begin{default}%m=2
                    \function{x}{sqrt(2)/2}
                    \function{y}{-sqrt(2)/2}
                \end{default}
             \end{switch}
             
             \randint[Z]{z}{-1}{2}
             \pmatrix{a}{0 \\ 0 \\ 1}
             \pmatrix{b}{0 \\ z \\ 0}
             \pmatrix{c}{y \\ 0 \\ x}
             %\function[calculate]{det}{x*z}  dies ist falsch
             \function[calculate]{det}{-y*z}
             \function[calculate]{norm}{abs(z)}
             \function[calculate]{orthog}{(m+1)*norm}%nur orthonormal, falls m=4 und z=+-1
         \end{variables}
%
% Erklärung bei Fehler 
% keine Basis
                \explanation[[det=0] AND [equalChoice(111?)]]{
                $\{v_1;v_2;v_3\}$ does not form a basis and therefore neither an orthogonal nor an orthonormal basis of $\R^3$.
                }
                \explanation[[det=0] AND [equalChoice(101?)]]{
                $\{v_1;v_2;v_3\}$ is not a basis and therefore not an orthonormal basis of $\R^3$.
                }
                \explanation[[det=0] AND [equalChoice(110?)]]{
                $\{v_1;v_2;v_3\}$ does not form a basis and therefore not an orthogonal basis of $\R^3$.
                }
                \explanation[[det=0] AND [equalChoice(100?)]]{
                $\{v_1;v_2;v_3\}$ bildet keine Basis von $\R^3$.
                }
                \explanation[[det=0] AND [equalChoice(011?)]]{
                $\{v_1;v_2;v_3\}$ is neither an orthogonal nor an orthonormal basis of $\R^3$.
                }
                \explanation[[det=0] AND [equalChoice(001?)]]{
                $\{v_1;v_2;v_3\}$ is not a orthonormal basis of $\R^3$.
                }
                \explanation[[det=0] AND [equalChoice(010?)]]{
                $\{v_1;v_2;v_3\}$ is not an orthogonal basis of $\R^3$.
                }
% Basis aber nicht orthogonal 
                \explanation[[equalChoice(011?)] AND NOT [det=0] AND NOT [m=0]]{   
                $\{v_1;v_2;v_3\}$ forms a basis, but not an orthogonal and thereofore not an orthonormal basis of $\R^3$.
                }
                \explanation[[equalChoice(010?)] AND NOT [det=0] AND NOT [m=0]]{   
                $\{v_1;v_2;v_3\}$ is a bsis, but not an orthogonal basis of $\R^3$.
                }
                \explanation[[equalChoice(001?)] AND NOT [det=0] AND NOT [m=0] AND NOT [orthog=1]]{   
                $\{v_1;v_2;v_3\}$ forms a basis, but not an orthonormal basis of $\R^3$.
                }
                \explanation[[equalChoice(111?)] AND NOT [det=0] AND NOT [m=0] AND NOT [orthog=1]]{   
                $\{v_1;v_2;v_3\}$ is neither a orthogonal nor an orthonormal basis of $\R^3$.
                }
                \explanation[[equalChoice(110?)] AND NOT [det=0] AND NOT [m=0]]{   
                $\{v_1;v_2;v_3\}$ does not form a orthogonal basis of $\R^3$.
                }
                \explanation[[equalChoice(101?)] AND NOT [det=0] AND NOT [m=0] AND NOT [orthog=1]]{   
                $\{v_1;v_2;v_3\}$ does not form an orthogonal basis of $\R^3$.
                }
                \explanation[[equalChoice(000?)] AND NOT [det=0] AND NOT [m=0] AND NOT [orthog=1]]{   
                $\{v_1;v_2;v_3\}$ forms a basis of $\R^3$.
                } 
% Basis orthogonal
                \explanation[[equalChoice(001?)] AND NOT [det=0] AND [m=0] AND NOT [orthog=1]]{   
                $\{v_1;v_2;v_3\}$ is an orthogonal basis, so also a basis of $\R^3$ but not an orthonormal basis.
                }                
% Basis orthonormal 
                \explanation[[[equalChoice(0???)] OR [equalChoice(?0??)] OR [equalChoice(??0?)]] AND NOT [det=0] AND [m=0] AND [orthog=1]]{ 
                $\{v_1;v_2;v_3\}$ is an orthonormal basis, so also an orthogonal basis and a basis of $\R^3$.
                }
% Vektoren normiert?
                \explanation[[equalChoice(???0)] AND [norm=1]]{   
                The vectors $v_1, v_2, v_3$ are all normalised with respect of the standard scalar product.
                }                                              
                \explanation[[equalChoice(???1)] AND NOT [norm=1]]{
                The vector $v_2$ is not normalised.
                % bez\"uglich des Standard-Skalarprodukts normiert.
                }
% Ende
%
 
        \begin{choice}
              \text{$\{v_1;v_2;v_3\}$ is a basis of $\R^3$.}
              \solution{compute}
              \iscorrect{det}{!=}{0}
        \end{choice}
         \begin{choice}
               \text{$\{v_1;v_2;v_3\}$ is an orthogonal basis of $\R^3$.}
               \solution{compute}
               \iscorrect{m}{=}{0}
         \end{choice}
         \begin{choice}
               \text{$\{v_1;v_2;v_3\}$ is an orthonormal basis of $\R^3$.}
               \solution{compute}
               \iscorrect{orthog}{=}{1}
         \end{choice}
         \begin{choice}
               \text{The vectors $v_1;v_2;v_3$ are normalised with respect of the standard basis,
               so $\langle v_i,v_i \rangle = 1$ for $i=1,2,3$.}
               \solution{compute}     
               \iscorrect{norm}{=}{1}
         \end{choice}
    \end{question}
    
    %-----------------------
    %Q3
    \begin{question}
       \text{Given are the vectors $v_1=\var{a}, v_2= \var{b}, v_3=\var{c} \in \R^3$. 
       We consider the vector space $\R^3$ with the standard scalar product. 
       Choose all the right answers.
     %Test  Teil Q3
       }
       \permutechoices{1}{4}
       \type{mc.yesno}
       \field{real}
  
        \begin{variables}
             \randint{m}{0}{4}
             \begin{switch}
                \begin{case}{m=0}
                    \function{x}{0}
                    \function{y}{-1}
                    \function{y1}{1}
                \end{case}
                \begin{case}{m=1}
                    \function{x}{1/2}
                    \function{y}{sqrt(3)/2}
                    \function{y1}{-sqrt(3)/2}
                \end{case}
                \begin{case}{m=3}
                    \function{x}{sqrt(3)/2}
                    \function{y}{1/2}
                    \function{y1}{-1/2}
                \end{case}
                \begin{case}{m=4}
                    \function{x}{1}
                    \function{y}{0}
                    \function{y1}{0}
                \end{case}
                \begin{default}%m=2
                    \function{x}{sqrt(2)/2}
                    \function{y}{sqrt(2)/2}
                    \function{y1}{-sqrt(2)/2}
                \end{default}
             \end{switch}
             
             \randint[Z]{z}{-2}{2}
             \pmatrix{a}{z \\ 0 \\ 0}
             \pmatrix{b}{0 \\ x \\ y}
             \pmatrix{c}{0 \\ y1 \\ x}
             \function[calculate]{norm}{abs(z)}
         \end{variables}
%
% Erklärung bei Fehler 
%
                \explanation[[equalChoice(010?)] AND NOT [norm=1]]{
                $\{v_1;v_2;v_3\}$ forms a basis of $\R^3$.
                }
                \explanation[[equalChoice(100?)] AND NOT [norm=1]]{
                $\{v_1;v_2;v_3\}$ forms an orthogonal basis of $\R^3$.
                }
                \explanation[[equalChoice(000?)] AND NOT [norm=1]]{
                $\{v_1;v_2;v_3\}$ is a basis and also an orthogonal basis of $\R^3$.
                }

                \explanation[[equalChoice(000?)] AND [norm=1]]{   
                $\{v_1;v_2;v_3\}$ is an orthonormal basis and therefore also orthogonal basis and basis of $\R^3$.
                }
                \explanation[[equalChoice(100?)] AND [norm=1]]{   
                $\{v_1;v_2;v_3\}$ forms an orthonormal basis and therefore also an orthogonal basis of $\R^3$.
                }
                \explanation[[equalChoice(010?)] AND [norm=1]]{   
                $\{v_1;v_2;v_3\}$ is an orthonormal basis and therefore also basis of $\R^3$.
                }
                \explanation[[equalChoice(001?)] AND [norm=1]]{   
                $\{v_1;v_2;v_3\}$ is an orthogonal basis and a basis of $\R^3$.
                }
                \explanation[[equalChoice(101?)] AND [norm=1]]{   
                $\{v_1;v_2;v_3\}$ forms an orthogonal basis of $\R^3$.
                }
                \explanation[[equalChoice(011?)] AND [norm=1]]{   
                $\{v_1;v_2;v_3\}$ forms a basis of $\R^3$.
                }
                \explanation[[equalChoice(110?)] AND [norm=1]]{   
                $\{v_1;v_2;v_3\}$ is an orthonormal basis of $\R^3$.
                }
                
                \explanation[[equalChoice(001?)] AND NOT [norm=1]]{   
                $\{v_1;v_2;v_3\}$ forms a basis and also an orthogonal basis, but not an orthonormal basis of $\R^3$.
                }
                \explanation[[equalChoice(011?)] AND NOT [norm=1]]{   
                $\{v_1;v_2;v_3\}$ is basis, but not an orthonormal basis of $\R^3$.
                }
                \explanation[[equalChoice(101?)] AND NOT [norm=1]]{   
                $\{v_1;v_2;v_3\}$ forms an orthogonal basis, but not an orthonormal basis of $\R^3$.
                }
                \explanation[[equalChoice(111?)] AND NOT [norm=1]]{   
                $\{v_1;v_2;v_3\}$ does not form an orthonormal basis of $\R^3$.
                }

                \explanation[[equalChoice(???0)] AND [norm=1]]{   
                The vectors $v_1, v_2, v_3$ are all normalised with respect of the standard basis.
                }                                              
                \explanation[[equalChoice(???1)] AND NOT [norm=1]]{
                The vector $v_1$  is not normalised.
                % bez\"uglich des Standard-Skalarprodukts normiert.
                }
% Ende
 
        \begin{choice}
              \text{$\{v_1;v_2;v_3\}$ forms a basis of $\R^3$.}
              \solution{true}
        \end{choice}
         \begin{choice}
               \text{$\{v_1;v_2;v_3\}$ is an orthogonal basis of $\R^3$.}
               \solution{true}
         \end{choice}
         \begin{choice}
               \text{$\{v_1;v_2;v_3\}$ forms an orthonormal basis of $\R^3$.}
               \solution{compute}
               \iscorrect{norm}{=}{1}
         \end{choice}
         \begin{choice}
               \text{The vectors $v_1;v_2;v_3$ are all normalised with respect to the standard scalar product,
               so $\langle v_i,v_i \rangle = 1$ for $i=1,2,3$.}
               \solution{compute}     
               \iscorrect{norm}{=}{1}
         \end{choice}
    \end{question}
    
    %Q4
    \begin{question}
       \text{Given are the vectors $v_1=\var{a}, v_2= \var{b}, v_3=\var{c} \in \R^3$. 
       We consider the vector space $\R^3$ with the standard scalar product. 
       Choose all the correct answers.
     %Test  Teil Q4
       }
       \permutechoices{1}{4}
       \type{mc.yesno}
       \field{real}
  
        \begin{variables}
             \randint{m}{0}{4}
             \begin{switch}
                \begin{case}{m=0}
                    \function{x}{0}
                    \function{y}{-1}
                    \function{y1}{1}
                \end{case}
                \begin{case}{m=1}
                    \function{x}{1/2}
                    \function{y}{sqrt(3)/2}
                    \function{y1}{-sqrt(3)/2}
                \end{case}
                \begin{case}{m=3}
                    \function{x}{sqrt(3)/2}
                    \function{y}{1/2}
                    \function{y1}{-1/2}
                \end{case}
                \begin{case}{m=4}
                    \function{x}{1}
                    \function{y}{0}
                    \function{y1}{0}
                \end{case}
                \begin{default}%m=2
                    \function{x}{sqrt(2)/2}
                    \function{y}{sqrt(2)/2}
                    \function{y1}{-sqrt(2)/2}
                \end{default}
             \end{switch}
             
             \randint[Z]{z}{-2}{2}
             \pmatrix{a}{x \\ 0 \\ y}
             \pmatrix{b}{0 \\ z \\ 0}
             \pmatrix{c}{y1 \\ 0 \\ x}
             \function[calculate]{norm}{abs(z)}
         \end{variables}
%
% Erklärung bei Fehler 
%
                \explanation[[equalChoice(010?)] AND NOT [norm=1]]{
                $\{v_1;v_2;v_3\}$ forms also a basis of $\R^3$.
                }
                \explanation[[equalChoice(100?)] AND NOT [norm=1]]{
                $\{v_1;v_2;v_3\}$ forms also an orthogonal basis of $\R^3$.
                }
                \explanation[[equalChoice(000?)] AND NOT [norm=1]]{
                $\{v_1;v_2;v_3\}$ forms a basis and also na orthogonal basis of $\R^3$.
                }

                \explanation[[equalChoice(000?)] AND [norm=1]]{   
                $\{v_1;v_2;v_3\}$ forms an orthonormal basis and therefore also
                an orthogonalbasis and a basis of $\R^3$.
                }
                \explanation[[equalChoice(100?)] AND [norm=1]]{   
                $\{v_1;v_2;v_3\}$ forms an orthonormal basis and therefore also an orthogonal basis of
                von $\R^3$.
                }
                \explanation[[equalChoice(010?)] AND [norm=1]]{   
                $\{v_1;v_2;v_3\}$ is an orthonormal basis and therefore also a basis of $\R^3$.
                }
                \explanation[[equalChoice(001?)] AND [norm=1]]{   
                $\{v_1;v_2;v_3\}$ ist also an orthogonal basis and a basis of $\R^3$.
                }
                \explanation[[equalChoice(101?)] AND [norm=1]]{   
                $\{v_1;v_2;v_3\}$ is also an orthogonal basis of $\R^3$.
                }
                \explanation[[equalChoice(011?)] AND [norm=1]]{   
                $\{v_1;v_2;v_3\}$ forms also a basis of $\R^3$.
                }                
                \explanation[[equalChoice(110?)] AND [norm=1]]{   
                $\{v_1;v_2;v_3\}$ forms an orthonormal basis of $\R^3$.
                }
                
                \explanation[[equalChoice(001?)] AND NOT [norm=1]]{   
                $\{v_1;v_2;v_3\}$ forms a basis and also an orthogonal basis, but not an orthonormal basis of $\R^3$.
                }
                \explanation[[equalChoice(011?)] AND NOT [norm=1]]{   
                $\{v_1;v_2;v_3\}$ forms a basis, but not an orthonormal basis of $\R^3$.
                }
                \explanation[[equalChoice(101?)] AND NOT [norm=1]]{   
                $\{v_1;v_2;v_3\}$ forms an orthogonal basis, but not an orthonormal basis of $\R^3$.
                }
                \explanation[[equalChoice(111?)] AND NOT [norm=1]]{   
                $\{v_1;v_2;v_3\}$ does not form an orthonormal basis of $\R^3$.
                }

                \explanation[[equalChoice(???0)] AND [norm=1]]{   
                The vectors $v_1, v_2, v_3$ are all normalised with respect of the standard scalar product.
                }                                              
                \explanation[[equalChoice(???1)] AND NOT [norm=1]]{
                The vector $v_2$ is not normalised.
                % bez\"uglich des Standard-Skalarprodukts normiert.
                }
% Ende


        \begin{choice}
              \text{$\{v_1;v_2;v_3\}$ forms a basis of $\R^3$.} 
              \solution{true}
        \end{choice}
         \begin{choice}
               \text{$\{v_1;v_2;v_3\}$ forms an orthogonal basis of $\R^3$.}
               \solution{true}
         \end{choice}
         \begin{choice}
               \text{$\{v_1;v_2;v_3\}$ forms an orthonormal basis of $\R^3$.}
               \solution{compute}
               \iscorrect{norm}{=}{1}
         \end{choice}
         \begin{choice}
               \text{The vectors $v_1;v_2;v_3$ are all normalised respect to the standard basis,
               so $\langle v_i,v_i \rangle = 1$ for $i=1,2,3$.}
               \solution{compute}     
               \iscorrect{norm}{=}{1}
         \end{choice}
    \end{question}
    
\end{problem}}


\embedmathlet{gwtmathlet}

\end{content}