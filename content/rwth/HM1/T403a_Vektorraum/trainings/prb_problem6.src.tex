\documentclass{mumie.problem.gwtmathlet}
%$Id$
\begin{metainfo}
  \name{
    \lang{de}{A06: Orthogonalisierung}
    \lang{en}{P06: Gram-Schmidt process for orthogonalisation}
  }
  \begin{description} 
 This work is licensed under the Creative Commons License Attribution 4.0 International (CC-BY 4.0)   
 https://creativecommons.org/licenses/by/4.0/legalcode 

    \lang{de}{Beschreibung}
    \lang{en}{}
  \end{description}
  \corrector{system/problem/GenericCorrector.meta.xml}
  \begin{components}
    \component{js_lib}{system/problem/GenericMathlet.meta.xml}{gwtmathlet}
  \end{components}
  \begin{links}
  \end{links}
  \creategeneric
\end{metainfo}
\begin{content}
\usepackage{mumie.genericproblem}

\lang{de}{
	\title{A06: Gram-Schmidt Orthogonalisierung}
}
\lang{en}{
	\title{P06: Gram-Schmidt process for orthogonalisation}
}

\begin{block}[annotation]
	Im Ticket-System: \href{http://team.mumie.net/issues/14524}{Ticket 14524}
\end{block}


\begin{problem}
   \randomquestionpool{1}{4}
   \randomquestionpool{5}{5}


%Q1
%FS: Nun mit Permutationen programmiert
  \begin{question}
    \begin{variables}
        \randint[Z]{a}{-3}{3}
        \randint[Z]{b}{-2}{2}
        \randint{perm}{0}{1}
        \function[calculate]{ab}{4*a}
        \function[calculate]{ca}{-12*a/25}
        \function[calculate]{cb}{9*a/25}
        
        %Test:
        %\function[calculate]{ca}{-12*a/5}
        %\function[calculate]{cb}{-11*a/5}
        
        \function[calculate]{cd}{sqrt((ca)^2+(cb)^2+b^2)}
        \function[calculate]{da}{ca/cd}
        \function[calculate]{db}{cb/cd}
        \function[calculate]{dc}{b/cd}
        
        \function[calculate]{aaa}{3*perm+4*(1-perm)}
        \function[calculate]{aab}{3*(1-perm)+4*perm}
        \function[calculate]{saa}{aaa/5}
        \function[calculate]{sab}{aab/5}
        \function[calculate]{baa}{a*(1-perm)}
        \function[calculate]{bab}{a*perm}
        \function[calculate]{sba}{da*perm+db*(1-perm)}
        \function[calculate]{sbb}{db*perm+da*(1-perm)}
        
            \pmatrix{aa}{aaa\\0\\aab}
            \pmatrix{bb}{baa\\b\\bab}
            \pmatrix{sa}{saa \\ 0 \\ sab}
%            \matrix{sb1}{ca \\ b \\  cb} % Hilfswert für Fehlermeldung
%            \matrix{sb2}{cb \\ b \\  ca} % Hilfswert für Fehlermeldung
            \pmatrix{sb}{sba \\ dc \\ sbb}
        \end{variables}

      \lang{de}{
      \text{Wir betrachten den zweidimensionalen Untervektorraum von $\R^3$ mit der Basis $\left\{ \var{aa};\var{bb} \right\}$. 
      Konstruieren Sie eine Orthonormalbasis $\{u_1;u_2\}$ hiervon bezüglich des Standard-Skalarprodukts mit 
      Hilfe des Orthogonalisierungsverfahren von Gram-Schmidt. Runden Sie entweder die Vektoreinträge auf zwei 
      Nachkommastellen oder geben Sie die Werte genau an. (Um Wurzelterme einzutragen, nutzen Sie 'sqrt()'.)}}
       \lang{en}{
      \text{We consider the vector subspace of $\R^3$ with dimension two and basis $\left\{ \var{aa};\var{bb} \right\}$. 
      Construct an orthonormal basis $\{u_1;u_2\}$ of the given basis with respect to the standard basis with the help
      of the Gram-Schmidt-process for orthogonalisation. Round the vector entries to two decimal places or give the precise values.
      (Use 'sqrt()' for roots.)}}

%% Test      
%      $u'_1=\var{aa}, u_1=\var{sa}, u'_2=\var{sb1},$ wenn $perm=\var{perm}=1$ sonst $u'_2=\var{sb2}, u_2=\var{sb}$
%    }
%
%   Eine bedingte Erklärung ist für Matrizen/Vektoren (type{input.matrix}) derzeit leider nicht möglich,
%   eine allgemeine Erklärung, die über das, was bereits in der Aufgabenstellung hinaus geht, ist nicht sinnvoll
%
%      \explanation[NOT equalAnswer(ans_1, sa)]{Der Vektor $u_1$ wurde falsch berechnet.}
%      \explanation[equalAnswer[ans_1, aa]]{$u_1$ ist nicht normiert.}
%      \explanation[NOT equalAnswer(ans_2, sb)]{Der Vektor $u_2$ wurde falsch berechnet.}
%      \explanation[[equalAnswer[ans_2, sb1]] OR [equalAnswer[ans_2, sb2]]]{$u_2$ ist nicht normiert.}
%       
        \type{input.matrix}
        \field{real}
        \precision{2}
        \begin{answer}
            \text{$u_1=$}
            \solution{sa}
            \format{3}{1}
        \end{answer}
        \begin{answer}
            \text{$u_2=$}
            \solution{sb}
            \format{3}{1}
        \end{answer}
  \end{question}
%Q2
  \begin{question}
    \begin{variables}
        \randint[Z]{a}{-3}{3}
        \randint[Z]{b}{-2}{2}
        \randint{perm}{0}{1}
        \function[calculate]{ab}{4*a}
        \function[calculate]{ca}{-12*a/25}
        \function[calculate]{cb}{9*a/25}
        \function[calculate]{cd}{sqrt(ca^2+cb^2+b^2)}
        \function[calculate]{da}{ca/cd}
        \function[calculate]{db}{cb/cd}
        \function[calculate]{dc}{b/cd}
        
        \function[calculate]{aaa}{3*perm+4*(1-perm)}
        \function[calculate]{aab}{3*(1-perm)+4*perm}
        \function[calculate]{saa}{aaa/5}
        \function[calculate]{sab}{aab/5}
        \function[calculate]{baa}{a*(1-perm)}
        \function[calculate]{bab}{a*perm}
        \function[calculate]{sba}{da*perm+db*(1-perm)}
        \function[calculate]{sbb}{db*perm+da*(1-perm)}

    
            \pmatrix{aa}{aaa\\aab\\0}
            \pmatrix{bb}{baa\\bab\\b}
            \pmatrix{sa}{saa \\ sab \\0}
            \pmatrix{sb}{sba \\ sbb \\ dc}
        \end{variables}
      \lang{de}{  
      \text{Wir betrachten den zweidimensionalen Untervektorraum von $\R^3$ mit der Basis $\left\{ \var{aa};\var{bb} \right\}$. 
      Konstruieren Sie eine Orthonormalbasis $\{u_1;u_2\}$ hiervon bezüglich des Standard-Skalarprodukts mit Hilfe 
      des Orthogonalisierungsverfahren von Gram-Schmidt. Runden Sie entweder die Vektoreinträge auf zwei Nachkommastellen 
      oder geben Sie die Werte genau an. (Um Wurzelterme einzutragen, nutzen Sie 'sqrt()'.)}}
      \lang{en}{
      \text{We consider the vector subspace of $\R^3$ dimension two and basis $\left\{ \var{aa};\var{bb} \right\}$. 
      Construct an orthonormal basis $\{u_1;u_2\}$ of the given basis with respect to the standard basis with the help
      of the Gram-Schmidt-process for orthogonalisation. Round the vector entries to two decimal places or give the precise values.
      (Use 'sqrt()' for roots.)}}
%      \explanation{
%      zu (b)
%      }
        \type{input.matrix}
        \field{real}
        \precision{2}
        \begin{answer}
            \text{$u_1=$}
            \solution{sa}
            \format{3}{1}
        \end{answer}
        \begin{answer}
            \text{$u_2=$}
            \solution{sb}
            \format{3}{1}
        \end{answer}
  \end{question}
%Q3  
  \begin{question}
    \begin{variables}
        \randint[Z]{a}{-3}{3}
        \randint[Z]{b}{-2}{2}
        \randint{perm}{0}{1}
        \function[calculate]{ab}{4*a}
        \function[calculate]{ca}{-12*a/25}
        \function[calculate]{cb}{9*a/25}
        \function[calculate]{cd}{sqrt(ca^2+cb^2+b^2)}
        \function[calculate]{da}{ca/cd}
        \function[calculate]{db}{cb/cd}
        \function[calculate]{dc}{b/cd}
        
        \function[calculate]{aaa}{3*perm+4*(1-perm)}
        \function[calculate]{aab}{3*(1-perm)+4*perm}
        \function[calculate]{saa}{aaa/5}
        \function[calculate]{sab}{aab/5}
        \function[calculate]{baa}{a*(1-perm)}
        \function[calculate]{bab}{a*perm}
        \function[calculate]{sba}{da*perm+db*(1-perm)}
        \function[calculate]{sbb}{db*perm+da*(1-perm)}

            \pmatrix{aa}{0 \\aaa\\aab}
            \pmatrix{bb}{b \\baa\\bab}
            \pmatrix{sa}{0 \\ saa \\ sab}
            \pmatrix{sb}{dc \\ sba \\ sbb}
        \end{variables}

      \lang{de}{
      \text{Wir betrachten den zweidimensionalen Untervektorraum von $\R^3$ mit der Basis $\left\{ \var{aa};\var{bb} \right\}$. 
      Konstruieren Sie eine Orthonormalbasis $\{u_1;u_2\}$ hiervon bezüglich des Standard-Skalarprodukts mit 
      Hilfe des Orthogonalisierungsverfahren von Gram-Schmidt. Runden Sie entweder die Vektoreinträge auf zwei 
      Nachkommastellen oder geben Sie die Werte genau an. (Um Wurzelterme einzutragen, nutzen Sie 'sqrt()'.)}}
      \lang{en}{
      \text{We consider the vector subspace of $\R^3$ with dimension two and basis $\left\{ \var{aa};\var{bb} \right\}$. 
      Construct an orthonormal basis $\{u_1;u_2\}$ of the given basis with respect to the standard basis with the help
      of the Gram-Schmidt-process for orthogonalisation. Round the vector entries to two decimal places or give the precise values.
      (Use 'sqrt()' for roots.)}}
%      \explanation{
%      zu (c)
%      }
        \type{input.matrix}
        \field{real}
        \precision{2}
        \begin{answer}
            \text{$u_1=$}
            \solution{sa}
            \format{3}{1}
        \end{answer}
        \begin{answer}
            \text{$u_2=$}
            \solution{sb}
            \format{3}{1}
        \end{answer}
  \end{question}    
%Q4
  \begin{question}
    \begin{variables}
        \randint{a}{-3}{3}
        \randint[Z]{b}{-2}{2}
        \randint[Z]{c}{-1}{1}
        \function[calculate]{pa}{c*2}
        \function{spa}{pa/3}
        \function[calculate]{da}{a-1/9*a-2/9*b*c}
        \function[calculate]{db}{-2/9*a-4/9*b*c}
        \function[calculate]{dc}{b-2/9*a*c-4/9*b}
        \function[calculate]{normnenner}{sqrt(da^2+db^2+dc^2)}
        \function[calculate]{sba}{da/normnenner}
        \function[calculate]{sbb}{db/normnenner}
        \function[calculate]{sbc}{dc/normnenner}
        
            \pmatrix{aa}{1 \\2 \\pa}
            \pmatrix{bb}{a \\0 \\ b}
            \pmatrix{sa}{1/3 \\ 2/3 \\ spa}
            \pmatrix{sb}{sba \\ sbb \\ sbc}
        \end{variables}

       \lang{de}{
      \text{Wir betrachten den zweiimensionalen Untervektorraum von $\R^3$ mit der Basis $\left\{ \var{aa};\var{bb} \right\}$. 
      Konstruieren Sie eine Orthonormalbasis $\{u_1;u_2\}$ hiervon bezüglich des Standard-Skalarprodukts mit Hilfe 
      des Orthogonalisierungsverfahren von Gram-Schmidt. Runden Sie entweder die Vektoreinträge auf zwei Nachkommastellen 
      oder geben Sie die Werte genau an. (Um Wurzelterme einzutragen, nutzen Sie 'sqrt()'.)}}
      \lang{en}{
      \text{We consider the vector subspace of $\R^3$ with dimension two and basis $\left\{ \var{aa};\var{bb} \right\}$. 
      Construct an orthonormal basis $\{u_1;u_2\}$ of the given basis with respect to the standard basis with the help
      of the Gram-Schmidt-process for orthogonalisation. Round the vector entries to two decimal places or give the precise values.
      (Use 'sqrt()' for roots.)}}
%      \explanation{
%      zu (d)
%      }
        \type{input.matrix}
        \field{real}
        \precision{2}
        \begin{answer}
            \text{$u_1=$}
            \solution{sa}
            \format{3}{1}
        \end{answer}
        \begin{answer}
            \text{$u_2=$}
            \solution{sb}
            \format{3}{1}
        \end{answer}
  \end{question}

%Q5
  \begin{question}
    \begin{variables}
        \randint{a}{-3}{3}
        \randint[Z]{b}{-2}{2}
        \randint[Z]{c}{-1}{1}
        \randint{d}{-5}{-1}
        \randint[Z]{f}{-3}{3}
        \randint[Z]{g}{-2}{2}
        
        %\randint{perm}{0}{1}
    
    %Kommentar: Die Lösung kann vermutlich einfacher berechnet werden...
    \function[calculate]{aaa}{f}
    \function[calculate]{aab}{g}
    \randadjustIf{a,c}{aaa*c-a*aab = 0}

    \function[calculate]{normeins}{sqrt(f^2+g^2)}

    \function[calculate]{saa}{aaa/normeins}
    \function[calculate]{sab}{aab/normeins}

        \function[calculate]{da}{a-(a*saa+c*sab)*saa}
        %\function[calculate]{db}{0}
        \function[calculate]{dc}{c-(a*saa+c*sab)*sab}
        \function[calculate]{normnenner}{sqrt(da^2+dc^2)}
        \function[calculate]{sca}{da/normnenner}
        %\function[calculate]{scb}{db/normnenner}
        \function[calculate]{scc}{dc/normnenner}
        
            \pmatrix{aa}{aaa \\0 \\aab}
            \pmatrix{bb}{0 \\ d \\ 0}
            \pmatrix{cc}{a \\ b \\ c}
            
            \pmatrix{sa}{saa \\ 0 \\ sab}
            \pmatrix{sb}{0\\ -1\\ 0}
            \pmatrix{sc}{sca \\ 0 \\ scc}
        \end{variables}

      \lang{de}{
      \text{Wir betrachten den Vektorraum $\R^3$ mit der Basis $\left\{ \var{aa};\var{bb};\var{cc} \right\}$. 
      Konstruieren Sie eine Orthonormalbasis $\{u_1;u_2;u_3\}$ hiervon bezüglich des Standard-Skalarprodukts 
      mit Hilfe des Orthogonalisierungsverfahren von Gram-Schmidt. Runden Sie entweder die Vektoreinträge auf 
      zwei Nachkommastellen oder geben Sie die Werte genau an. (Um Wurzelterme einzutragen, nutzen Sie 'sqrt()'.)}}
      \lang{en}{
      \text{We consider the vector space of $\R^3$ with the basis $\left\{ \var{aa};\var{bb};\var{cc} \right\}$. 
      Construct an orthonormal basis $\{u_1;u_2;u_3\}$ of the given basis with respect to the standard basis with the help
      of the Gram-Schmidt-process for orthogonalisation. Round the vector entries to two decimal places or give the precise values.
      (Use 'sqrt()' for roots.)}}
%      \explanation{
%      zu (e)
%      }
        \type{input.matrix}
        \field{real}
        \precision{2}
        \begin{answer}
            \text{$u_1=$}
            \solution{sa}
            \format{3}{1}
        \end{answer}
        \begin{answer}
            \text{$u_2=$}
            \solution{sb}
            \format{3}{1}
        \end{answer}
        \begin{answer}
            \text{$u_3=$}
            \solution{sc}
            \format{3}{1}
        \end{answer}
  \end{question}
    
\end{problem}


\embedmathlet{gwtmathlet}

\end{content}