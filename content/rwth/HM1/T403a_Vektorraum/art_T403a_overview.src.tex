%$Id:  $
\documentclass{mumie.article}
%$Id$
\begin{metainfo}
  \name{
    \lang{de}{Überblick: Vektorräume und lineare Abbildungen}
    \lang{en}{Overview: Vector spaces and linear maps}
  }
  \begin{description} 
 This work is licensed under the Creative Commons License Attribution 4.0 International (CC-BY 4.0)   
 https://creativecommons.org/licenses/by/4.0/legalcode 

    \lang{de}{Beschreibung}
    \lang{en}{}
  \end{description}
  \begin{components}
  \end{components}
  \begin{links}
\link{generic_article}{content/rwth/HM1/T403a_Vektorraum/g_art_content_10c_Orthogonalbasen.meta.xml}{content_10c_Orthogonalbasen}
\link{generic_article}{content/rwth/HM1/T403a_Vektorraum/g_art_content_10b_lineare_abb.meta.xml}{content_10b_lineare_abb}
\link{generic_article}{content/rwth/HM1/T403a_Vektorraum/g_art_content_10a_vektorraum.meta.xml}{content_10a_vektorraum}
\end{links}
  \creategeneric
\end{metainfo}
\begin{content}
\begin{block}[annotation]
	Im Ticket-System: \href{https://team.mumie.net/issues/30115}{Ticket 30115}
\end{block}


\begin{block}[annotation]
Im Entstehen: Überblicksseite für Kapitel Vektorräume und lineare Abbildungen
\end{block}

\usepackage{mumie.ombplus}
\ombchapter{1}
\lang{de}{\title{Überblick: Vektorräume und lineare Abbildungen}}
\lang{en}{\title{Overview: Vector spaces and linear maps}}



\begin{block}[info-box]
\lang{de}{\strong{Inhalt}}
\lang{en}{\strong{Contents}}


\lang{de}{
    \begin{enumerate}%[arabic chapter-overview]
   \item[4.1] \link{content_10a_vektorraum}{Allgemeine Vektorräume}
   \item[4.2] \link{content_10b_lineare_abb}{Lineare Abbildungen}
   \item[4.3]\link{content_10c_Orthogonalbasen}{Orthogonalbasen, Orthonormalbasen}
  \end{enumerate}
} %lang
\lang{en}{
    \begin{enumerate}%[arabic chapter-overview]
   \item[4.1] \link{content_10a_vektorraum}{General vector space}
   \item[4.2] \link{content_10b_lineare_abb}{Linear maps}
   \item[4.3]\link{content_10c_Orthogonalbasen}{Orthogonal bases, orthonormal bases}
  \end{enumerate}
}

\end{block}

\begin{zusammenfassung}
\lang{de}{
Vektorräume bilden eine essentielle Grundlage für die Theorie der linearen Algebra. Im Kurs sind uns Vektorräume sogar bereits mehrfach begegnet. Nun geben wir eine exakte Definition an.
Aus der Linearkombination von Vektoren führen wir Erzeugendensysteme ein, betrachten die lineare (Un-)Abhängigkeit von Vektoren und gehen der Frage nach, was eine Basis ist.
Wir befassen uns dann mit linearen Abbildungen die wir zunächst exemplarisch und  anschaulich im $\R^2$ durch Drehungen, Spiegelungen und Streckungen untersuchen.
Wir stellen fest, dass man Koordinaten auch in anderen Basen als der Standardbasis angeben kann. Zur Umrechnung bedienen wir uns des Basiswechsels.
Am Ende des Kapitels untersuchen wir, wie man eine gegebene Menge linear unabhängiger Vektoren zu einer Basis ergänzen kann.
}
\lang{en}{
Vectorspace build the essential basis for the theory of linear algebra. We already came across vector spaces in our course.
Now we give an exact definition. We introduce generating sets based on linear combination, consider linearly (in-)dependence
and answer the question of what a basis is.\newline
We discuss linear maps exemplary and graphic using the example of rotation, reflection and extension in $\R^2$.
We see, that we can give coordinates in bases other than the standard basis. For the transformation we use the basis transformation.
At the end of the chapter we discuss, how we can complement a given set of linearly independent vectors to a basis.}

\end{zusammenfassung}

\begin{block}[info]\lang{de}{\strong{Lernziele}}
\lang{en}{\strong{Learning Goals}} 
\begin{itemize}[square]
\item \lang{de}{Sie kennen die Eigenschaften eines Vektorraums.} \lang{en}{Knowing the characteristics of a vector space.}
\item \lang{de}{Sie entscheiden, ob gegebene Mengen eine Basis bilden.} \lang{en}{Being able to decide if given sets are a basis.}
\item \lang{de}{Sie bestimmen den Koordinatenvektor bezüglich gegebener Basen.} \lang{en}{Being able to determine the coordinate vector with respect of given bases.}
\item \lang{de}{Sie führen Basiswechsel und Koordinatentransformationen durch.} \lang{en}{Being able to perfom basis and coordinate transformations.}
\item \lang{de}{Sie wenden das Orthogonalisierungsverfahren von Gram-Schmidt zur Bestimmung einer Orthonormalbasis an.}
\lang{en}{Being able to perfom the Gramd-Schmidt process for orthogonalisation for determining a orthonormal basis.}
\end{itemize}
\end{block}



\end{content}
