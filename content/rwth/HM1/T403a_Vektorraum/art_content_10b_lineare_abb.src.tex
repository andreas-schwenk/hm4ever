\documentclass{mumie.article}
%$Id$
\begin{metainfo}
  \name{
    \lang{de}{Lineare Abbildungen}
    \lang{en}{}
  }
  \begin{description} 
 This work is licensed under the Creative Commons License Attribution 4.0 International (CC-BY 4.0)   
 https://creativecommons.org/licenses/by/4.0/legalcode 

    \lang{de}{}
    \lang{en}{}
  \end{description}
  \begin{components}
    \component{generic_image}{content/rwth/HM1/images/g_tkz_T403a_Basis_A.meta.xml}{T403a_Basis_A}
    \component{generic_image}{content/rwth/HM1/images/g_tkz_T403a_Basis_B.meta.xml}{T403a_Basis_B}
    \component{generic_image}{content/rwth/HM1/images/g_tkz_T403a_Basis_C.meta.xml}{T403a_Basis_C}
    \component{generic_image}{content/rwth/HM1/images/g_tkz_T403a_LinearMaps.meta.xml}{T403a_LinearMaps}
    \component{generic_image}{content/rwth/HM1/images/g_tkz_T403a_Rotation.meta.xml}{T403a_Rotation}
    \component{generic_image}{content/rwth/HM1/images/g_img_00_video_button_schwarz-blau.meta.xml}{00_video_button_schwarz-blau}        
    \component{js_lib}{system/media/mathlets/GWTGenericVisualization.meta.xml}{mathlet1}
  \end{components}
  \begin{links}
    \link{generic_article}{content/rwth/HM1/T401_Matrizenrechnung/g_art_content_02_matrizenmultiplikation.meta.xml}{content_02_matrizenmultiplikation}
    \link{generic_article}{content/rwth/HM1/T306_Reelle_Quadratische_Matrizen/g_art_content_15_inverse_matrix.meta.xml}{inverse_matrix}
    \link{generic_article}{content/rwth/HM1/T403a_Vektorraum/g_art_content_10a_vektorraum.meta.xml}{vektorraum-basis}
  \end{links}
  \creategeneric
\end{metainfo}
\begin{content}
\begin{block}[annotation]
	Im Ticket-System: \href{https://team.mumie.net/issues/}{Ticket }
\end{block}
\usepackage{mumie.ombplus}
\ombchapter{4}
\ombarticle{2}
\usepackage{mumie.genericvisualization}

\begin{visualizationwrapper}


\lang{de}{\title{Lineare Abbildungen}}
\lang{en}{\title{Linear maps}}
  
\begin{block}[annotation]
  Im Ticket-System: \href{http://team.mumie.net/issues/13355}{Ticket 13355}\\
\end{block}

\begin{block}[info-box]
\tableofcontents
\end{block}

\section{\lang{de}{Lineare Abbildungen} \lang{en}{Linear maps}}

\begin{definition}[\lang{de}{Lineare Abbildungen} \lang{en}{Linear maps}] \label{def_lin_abb}
\lang{de}{
 Gegeben seien zwei Vektorräume $V,W$ über einem Körper $\K$. Eine Funktion $f: V \to W$ heißt \emph{lineare Abbildung}, falls für alle $u,v \in V$ und alle $\alpha \in \K$ gilt:}
\lang{en}{
 Given are two vector spaces $V,W$ over a field $\K$. A function $f: V \to W$ is called \emph{linear maps}, if for all $u,v \in V$ and all $\alpha \in \K$ holds:}
\[
  f(u+v) = f(u) + f(v)
 \]
 \lang{de}{und} \lang{en}{and}
 \[
  f(\alpha v) = \alpha f(v).
 \]
\end{definition}

\begin{example}
\begin{tabs}
\tab{\lang{de}{Matrixmultiplikation} \lang{en}{Matrix multiplication}}
\lang{de}{
Die Multiplikation mit einer $(m\times n)$-Matrix $A$ ist eine lineare Abbildung:
\[  f:\mathbb{K}^n\to \mathbb{K}^m, \left( \begin{smallmatrix}
x_1\\ x_2\\ \vdots \\ x_n
\end{smallmatrix}\right) \mapsto A\cdot  \left( \begin{smallmatrix}
x_1\\ x_2\\ \vdots \\ x_n
\end{smallmatrix}\right) .\]
Wie wir leicht mit den Eigenschaften der Matrizenrechnung verifizieren können, gilt
\[
 f(u+v)= A \cdot (u+v) = A \cdot u + A \cdot v = f(u) + f(v) 
\]
und
\[
  f(\alpha u)= A \cdot (\alpha u)= \alpha \cdot A \cdot u = \alpha f(u)
\]
für alle $u,v \in \K^n$ und $\alpha \in \K$.}
\lang{en}{
Multiplication with a $(m\times n)$-matrix $A$ is a linear map:
\[  f:\mathbb{K}^n\to \mathbb{K}^m, \left( \begin{smallmatrix}
x_1\\ x_2\\ \vdots \\ x_n
\end{smallmatrix}\right) \mapsto A\cdot  \left( \begin{smallmatrix}
x_1\\ x_2\\ \vdots \\ x_n
\end{smallmatrix}\right) .\]
With the help of the characteristics of matrix multiplication we can easily verify, that it holds:
\[
 f(u+v)= A \cdot (u+v) = A \cdot u + A \cdot v = f(u) + f(v) 
\]
and
\[
  f(\alpha u)= A \cdot (\alpha u)= \alpha \cdot A \cdot u = \alpha f(u)
\]
for all $u,v \in \K^n$ and $\alpha \in \K$.}

\tab{$\R$}
\lang{de}{
Für den Vektorraum $\R$ gilt, dass die linearen Abbildungen $f:\R \to \R$ alle von der Form $f(x)=c\cdot x$ 
mit einem beliebigen $c \in \R$ sind.

Zunächst zeigen wir, dass für beliebige $c \in \R$ die Funktion $f:\R \to \R, f(x)=c\cdot x$ 
die Eigenschaften einer linearen Abbildung gemäß Definition \ref{def_lin_abb} erfüllt:
\[
 f(x+y) = c\cdot (x+y) = cx + cy = f(x) + f(y) 
\]
und
\[
  f(\alpha x)= c \cdot (\alpha x)= \alpha (cx) = \alpha f(x)
\]
für alle $x,y \in \R$ und $\alpha \in \R$.}

\lang{en}{
Für den Vektorraum $\R$ gilt, dass die linearen Abbildungen $f:\R \to \R$ alle von der Form $f(x)=c\cdot x$ 
mit einem beliebigen $c \in \R$ sind.

Zunächst zeigen wir, dass für beliebige $c \in \R$ die Funktion $f:\R \to \R, f(x)=c\cdot x$ 
die Eigenschaften einer linearen Abbildung gemäß Definition \ref{def_lin_abb} erfüllt:
\[
 f(x+y) = c\cdot (x+y) = cx + cy = f(x) + f(y) 
\]
und
\[
  f(\alpha x)= c \cdot (\alpha x)= \alpha (cx) = \alpha f(x)
\]
für alle $x,y \in \R$ und $\alpha \in \R$.}

\lang{de}{
Betrachten wir nun umgekehrt eine beliebige lineare Abbildung $f:\R \to \R$ im Vektorraum $\R$ 
mit $x \mapsto f(x)$, so gilt aufgrund der Eigenschaften von linearen Abbildungen:
\[ f(x)= f(x \cdot 1)\overset{x\in\R}{=}x\cdot f(1).\]

Wir haben hierbei $x\in \R$ als einen Skalar interpretiert. 
Wählen wir nun $c:=f(1)\in\R,$ so erhalten wir nunmehr als Funktionsvorschrift für $f$:\[f(x)=c \cdot x.\]}

\lang{en}{
For the vector space $\R$ holds, that the linear maps $f:\R \to \R$ are all of the form $f(x)=c\cdot x$ 
with any $c \in \R$.

First of all we check, that the functions $f:\R \to \R, f(x)=c\cdot x$ fufills the characteristics of a linear
maps according to definition \ref{def_lin_abb} for any $c\in\R$:
\[
 f(x+y) = c\cdot (x+y) = cx + cy = f(x) + f(y) 
\]
and
\[
  f(\alpha x)= c \cdot (\alpha x)= \alpha (cx) = \alpha f(x)
\]
for all $x,y \in \R$ und $\alpha \in \R$.

If we have a look at any linear map $f:\R \to \R$ in the vector space $\R$ 
with $x \mapsto f(x)$, because of the characteristics of linear maps holds:
\[ f(x)= f(x \cdot 1)\overset{x\in\R}{=}x\cdot f(1).\} \]

Here we interpret $x\in\R$ as a skalar. 
Now we set $c:=f(1)\in\R,$ and receive the function specification $f$:\[f(x)=c \cdot x.\]}

\end{tabs}
\end{example}

\begin{block}[warning] 
\lang{de}{
In der Schulmathematik bezeichnet man Funktionen $f(x)=ax+b$ oft als lineare Funktionen.
Sie sind aber keine linearen Abbildung. Die linearen Abbildungen müssen nämlich alle $f(0)=0$ erfüllen.
Dies sieht man am einfachsten, wenn man $\alpha=0$ in der Definition wählt.
Es müsste dann nämlich gelten:}
\lang{en}{
In school mathematics, functions like $f(x)=ax+b$ are often called linear functions. But they are not,
since all linear maps need to fulfill $f(0)=0$. This is easy to unterstand, if we choose $\alpha=0$ 
in the definition.
Because then it would have to hold:}
\[
f(0)= f(0 \cdot x)= 0 \cdot f(x) = 0.
\]
\end{block}


\section{\lang{de}{Abbildungsmatrizen} \lang{en}{Transformation matrix}}\label{sec:abbildungsmatrizen}
\lang{de}{Wir haben gesehen, dass die Matrixmultiplikation eine lineare Abbildung ist.
Nun zeigen wir, dass sich jede lineare Abbildung als Matrixmultiplikation interpretieren lässt.}
\lang{en}{
We already know, that matrix multiplication is a linear map. Now we will see, that every linear map can be interpret
as matrix multiplication.}
%%% Video K.M. -  1.Teil des Videos 10829 = 11284 Koordinaten_3a
%
\lang{de}{\floatright{\href{https://api.stream24.net/vod/getVideo.php?id=10962-2-11284&mode=iframe&speed=true}
{\image[75]{00_video_button_schwarz-blau}}}}\\
\\


\lang{de}{Sei nun $f : V \to W$ eine lineare Abbildung zwischen den endlich-erzeugten
$\K$-Vektorräumen $V$ mit Basis $B= \{v_1; . . .; v_n\}$ und $W$ mit Basis $C = \{w_1 ; . . .; w_m \}$.
Wir nehmen nun ein Element $v = \sum_{j=1}^n \alpha_j v_j$ mit passenden $\alpha_j \in \K$ her. Dann gilt
\[
 f(v) = f\left( \sum_{j=1}^n \alpha_j v_j \right) =  \sum_{j=1}^n \alpha_j f(v_j).
\]
Wir können also das Abbildungsverhalten einer linearen Abbildung $f$ eindeutig beschreiben durch das Abbildungsverhalten der Basisvektoren $v_j$, $j=1,...n$.}
\lang{en}{Let $f : V \to W$ be a linear map between the finitely generated $\K$-vector spaces $V$ with basis $B= \{v_1; . . .; v_n\}$ and $W$ with basis $C = \{w_1 ; . . .; w_m \}$.
Now we use an element $v = \sum_{j=1}^n \alpha_j v_j$ with a suitable $\alpha_j \in \K$. Then it holds
\[
 f(v) = f\left( \sum_{j=1}^n \alpha_j v_j \right) =  \sum_{j=1}^n \alpha_j f(v_j).
\]
So we can describe a linear map $f$ uniquely by the mapping the basis vectors $v_j$, $j=1,...n$.}

\lang{de}{
Nun kann man $f(v_j)$ natürlich auch als Linearkombination der Basiselemente $w_k$ von $W$ darstellen.
Wir schreiben $f(v_j) = \sum_{i=1}^m a_{ij} w_i$. Dann heißt $A= (a_{ij})$ die \emph{Abbildungsmatrix} bezüglich der Basen $B$ und $C$. Wir schreiben auch $A= {\,}_Cf_B$.}

\lang{en}{
Now we can represent $f(v_j)$ also as a linear combination of the basis elements $w_k$ of $W$.
We note $f(v_j) = \sum_{i=1}^m a_{ij} w_i$. Then $A= (a_{ij})$ is called the \emph{transformation map} regarding the bases $B$ and $C$. We also write $A= {\,}_Cf_B$.}

\begin{theorem}[\lang{de}{Eindeutigkeit der Abbildungsmatrix} \lang{en}{Uniqueness of the transformation matrix}] \label{abb_matrix}
\lang{de}{Eine lineare Abbildung $f : V \to W$ zwischen endlich-erzeugten Vektorr\"aumen $V$ und $W$ \"uber $\K$ ist nach Festlegung von Basen $B$ von $V$ und $C$
von $W$ durch die entsprechende Abbildungsmatrix $A = {\,}_Cf_B \in M(m,n;\K)$ eindeutig festgelegt.}
\lang{en}{A linear map $f : V \to W$ between the finitely generated vector spaces $V$ and $W$ over $\K$
is after choosing the bases $B$ of $V$ und $C$ of $W$ unique specified by the corresponding transformation matrix $A = {\,}_Cf_B \in M(m,n;\K)$.}
\end{theorem}


\begin{block}[warning]
\lang{de}{Die Notation für die Abbildungsmatrix ist nicht einheitlich. Es gibt hier viele verschiedene Notationen. Eine
weitere häufig benutzte Notation ist $M(f,B,C)$.}
\lang{den}{The notation for the transformation matrix is not uniform. It exist several different notations. Another
commonly used notation is $M(f,B,C)$.}
\end{block}

\lang{de}{Auf den Vorteil der in Satz \ref{abb_matrix} gewählten Notation gehen wir im nächsten Abschnitt (Regeln \ref{Basiswechsel} und \ref{regel_bw}) ein.}
\lang{en}{We will discuss the advantage of the in theorem \ref{abb_matrix} chosen notation in the next section (rules \ref{Basiswechsel} and \ref{regel_bw}).}

\begin{remark}
\lang{de}{
In der $j$-ten Spalte der Abbildungsmatrix ${\,}_Cf_B$ befindet sich der Vektor
$\begin{pmatrix} a_{1j} \\ \vdots \\ a_{mj} \end{pmatrix}$. \\
Dieser ist der Koordinatenvektor von $f(v_j)$ bezüglich der Basis $C$, wobei $B=\{v_1;\ldots ;v_n\}$ die Basis des Definitionsbereiches ist.}
\lang{en}{
The $j$th coloumn of the transformation matrix ${\,}_Cf_B$ is the vector
$\begin{pmatrix} a_{1j} \\ \vdots \\ a_{mj} \end{pmatrix}$. \\
This is the coordinate vector of $f(v_j)$ regarding the basis $C$, whereat $B=\{v_1;\ldots ;v_n\}$ is the basis of the domain of definition.}
\end{remark}

%%% Video K.M. 
%
\lang{de}{
\begin{example}
Das Beispiel im Video verdeutlicht, dass die Abbildungsmatrix zu einer linearen Abbildung $f: V \to W$ maßgelblich
von der Festlegung der Basen für die Vektorräume $V$ und $W$ abhängt.

\center{\href{https://api.stream24.net/vod/getVideo.php?id=10962-2-10830&mode=iframe&speed=true}
{\image[75]{00_video_button_schwarz-blau}}}\\
\\
\end{example}}

\begin{example}\label{bsp:lin_abb}
\lang{de}{Gegeben sei die lineare Abbildung}
\lang{en}{Given is a linear map}
 \[
 f: \R^3 \to \R^2, \begin{pmatrix}x \\ y\\ z \end{pmatrix} \mapsto \begin{pmatrix}-2x-3y+2z \\x-y+2z\end{pmatrix}.
 \]

\\
\begin{incremental}

\step %1
\lang{de}{
Bevor wir die Abbildungsmatrix bestimmen, überzeugen wir uns kurz davon, dass $f$ tatsächlich eine
 lineare Abbildung ist.}
 \lang{en}{
Before we determine the transformation matrix, we check, that $f$ really is a linear map.}

 \lang{de}{
 Es gilt
 \[
 f(\begin{pmatrix}x \\ y\\ z \end{pmatrix}+\begin{pmatrix}\tilde{x} \\ \tilde{y}\\ \tilde{z} \end{pmatrix})
 = f(\begin{pmatrix}x+\tilde{x} \\ y+\tilde{y}\\ z+\tilde{z} \end{pmatrix})
 = \begin{pmatrix}-2(x+\tilde{x})-3(y+\tilde{y})+2(z+\tilde{z}) \\(x+\tilde{x})-(y+\tilde{y})+2(z+\tilde{z})\end{pmatrix}
 = \begin{pmatrix}-2x-3y+2z \\x-y+2z\end{pmatrix}+\begin{pmatrix}-2\tilde{x} -3\tilde{y} +2\tilde{z} \\\tilde{x}-\tilde{y} +2\tilde{z}\end{pmatrix}
 = f(\begin{pmatrix}x \\ y\\ z \end{pmatrix})+f(\begin{pmatrix}\tilde{x} \\ \tilde{y}\\ \tilde{z} \end{pmatrix})
 \]
 und
 \[
 f(\alpha \begin{pmatrix}x \\ y\\ z \end{pmatrix}) 
 = f(\begin{pmatrix} \alpha x \\ \alpha y\\ \alpha z \end{pmatrix})
 = \begin{pmatrix}-2\alpha x-3\alpha y+2\alpha z \\\alpha x-\alpha y+2\alpha z\end{pmatrix}
 = \begin{pmatrix}\alpha(-2x-3y+2z) \\\alpha(x-y+2z)\end{pmatrix}
 = \alpha \cdot \begin{pmatrix}-2x-3y+2z \\x-y+2z\end{pmatrix} = \alpha f( \begin{pmatrix}x \\ y\\ z \end{pmatrix}) .
 \]}

  \lang{en}{
 It holds
 \[
 f(\begin{pmatrix}x \\ y\\ z \end{pmatrix}+\begin{pmatrix}\tilde{x} \\ \tilde{y}\\ \tilde{z} \end{pmatrix})
 = f(\begin{pmatrix}x+\tilde{x} \\ y+\tilde{y}\\ z+\tilde{z} \end{pmatrix})
 = \begin{pmatrix}-2(x+\tilde{x})-3(y+\tilde{y})+2(z+\tilde{z}) \\(x+\tilde{x})-(y+\tilde{y})+2(z+\tilde{z})\end{pmatrix}
 = \begin{pmatrix}-2x-3y+2z \\x-y+2z\end{pmatrix}+\begin{pmatrix}-2\tilde{x} -3\tilde{y} +2\tilde{z} \\\tilde{x}-\tilde{y} +2\tilde{z}\end{pmatrix}
 = f(\begin{pmatrix}x \\ y\\ z \end{pmatrix})+f(\begin{pmatrix}\tilde{x} \\ \tilde{y}\\ \tilde{z} \end{pmatrix})
 \]
 and
 \[
 f(\alpha \begin{pmatrix}x \\ y\\ z \end{pmatrix}) 
 = f(\begin{pmatrix} \alpha x \\ \alpha y\\ \alpha z \end{pmatrix})
 = \begin{pmatrix}-2\alpha x-3\alpha y+2\alpha z \\\alpha x-\alpha y+2\alpha z\end{pmatrix}
 = \begin{pmatrix}\alpha(-2x-3y+2z) \\\alpha(x-y+2z)\end{pmatrix}
 = \alpha \cdot \begin{pmatrix}-2x-3y+2z \\x-y+2z\end{pmatrix} = \alpha f( \begin{pmatrix}x \\ y\\ z \end{pmatrix}) .
 \]}

\step %2

\lang{de}{
Wir wollen nun die Abbildungsmatrix $A={\,}_Cf_B$ bestimmen mit den Basen $B=  \textcolor{gray}{\underbrace{\textcolor{black}{\left\{\begin{pmatrix}
 1 \\ 0 \\ 0 \end{pmatrix}; \begin{pmatrix} 0 \\ 1 \\ 0 \end{pmatrix}; \begin{pmatrix} 1 \\ 1  \\ 1\end{pmatrix}\right\}}}_{\textcolor{gray}{=\left\{v_1;v_2;v_3\right\}}}}$ des $\R^3$
 und $C=\textcolor{gray}{\underbrace{\textcolor{black}{\left\{ \begin{pmatrix} 1\\ 0 \end{pmatrix}; \begin{pmatrix} 1\\ 1\end{pmatrix}\right\}}}_{\textcolor{gray}{=\left\{w_1;w_2\right\}}}}$ des $\R^2$.}
\lang{en}{
We will now determine the transformation matrix $A={\,}_Cf_B$ with the bases $B=  \textcolor{gray}{\underbrace{\textcolor{black}{\left\{\begin{pmatrix}
 1 \\ 0 \\ 0 \end{pmatrix}; \begin{pmatrix} 0 \\ 1 \\ 0 \end{pmatrix}; \begin{pmatrix} 1 \\ 1  \\ 1\end{pmatrix}\right\}}}_{\textcolor{gray}{=\left\{v_1;v_2;v_3\right\}}}}$ des $\R^3$
 und $C=\textcolor{gray}{\underbrace{\textcolor{black}{\left\{ \begin{pmatrix} 1\\ 0 \end{pmatrix}; \begin{pmatrix} 1\\ 1\end{pmatrix}\right\}}}_{\textcolor{gray}{=\left\{w_1;w_2\right\}}}}$ des $\R^2$.}


\lang{de}{
 Wir bestimmen also zunächst $f(\begin{pmatrix}1 \\ 0 \\ 0\end{pmatrix})\textcolor{gray}{=f(v_1)}$
 \[
 f(\begin{pmatrix}1 \\ 0 \\ 0\end{pmatrix})= \begin{pmatrix}-2\cdot 1 -3 \cdot 0 +2 \cdot 0 \\ 1 \cdot 1 - 1 \cdot 0 +2 \cdot 0\end{pmatrix} = 
 \begin{pmatrix} -2 \\ 1\end{pmatrix}\textcolor{gray}{=f(v_1)}
 \]
 Den Vektor auf der rechten Seite müssen wir nun als Koordinatenvektor bezüglich der Basis $C$ schreiben  \, 
 (wie im  
 \ref[vektorraum-basis][Beispiel zu $\R^2 $]{koord_vec} im vorherigen Kapitel)\textcolor{gray}{, d.h. als} \textcolor{gray}{$f(v_1)=\sum_{i=1}^{m=2}a_{i1}w_i=a_{11}w_1+a_{21}w_2$.}}
 \lang{en}{
Therefore we determine $f(\begin{pmatrix}1 \\ 0 \\ 0\end{pmatrix})\textcolor{gray}{=f(v_1)}$
 \[
 f(\begin{pmatrix}1 \\ 0 \\ 0\end{pmatrix})= \begin{pmatrix}-2\cdot 1 -3 \cdot 0 +2 \cdot 0 \\ 1 \cdot 1 - 1 \cdot 0 +2 \cdot 0\end{pmatrix} = 
 \begin{pmatrix} -2 \\ 1\end{pmatrix}\textcolor{gray}{=f(v_1)}
 \]
 We need to write the vector on the right side as a coordinate vector regarding the basis $C$  \, 
 (like we have done it in the
 \ref[vektorraum-basis][example for $\R^2 $]{koord_vec} in the previous chapter)\textcolor{gray}{, so als} \textcolor{gray}{$f(v_1)=\sum_{i=1}^{m=2}a_{i1}w_i=a_{11}w_1+a_{21}w_2$.}}


 \lang{de}{
 Es gilt
 \[
 \begin{pmatrix} -2 \\ 1\end{pmatrix} = \textcolor{gray}{\underbrace{\textcolor{#CC6600}{-3}}_{=a_{11}}} \cdot \textcolor{gray}{\underbrace{\textcolor{black}{\begin{pmatrix} 1 \\ 0 \end{pmatrix}}}_{=w_1}}+\textcolor{gray}{\underbrace{\textcolor{#CC6600}{1}}_{=a_{21}}} \cdot \textcolor{gray}{\underbrace{\textcolor{black}{\begin{pmatrix} 1 \\ 1 \end{pmatrix}}}_{=w_2}}
 \]
 Der Koordinatenvektor von $\begin{pmatrix} -2 \\ 1\end{pmatrix}$ bezüglich $C$ ist somit gegeben durch 
 $\textcolor{#CC6600}{\begin{pmatrix}-3 \\ 1 \end{pmatrix}}$.
 Dies ist dann schon die erste Spalte der Matrix $A$.}
 \lang{en}{
 It holds
 \[
 \begin{pmatrix} -2 \\ 1\end{pmatrix} = \textcolor{gray}{\underbrace{\textcolor{#CC6600}{-3}}_{=a_{11}}} \cdot \textcolor{gray}{\underbrace{\textcolor{black}{\begin{pmatrix} 1 \\ 0 \end{pmatrix}}}_{=w_1}}+\textcolor{gray}{\underbrace{\textcolor{#CC6600}{1}}_{=a_{21}}} \cdot \textcolor{gray}{\underbrace{\textcolor{black}{\begin{pmatrix} 1 \\ 1 \end{pmatrix}}}_{=w_2}}
 \]
 The coordinate vector of $\begin{pmatrix} -2 \\ 1\end{pmatrix}$ regarding $C$ is given by 
 $\textcolor{#CC6600}{\begin{pmatrix}-3 \\ 1 \end{pmatrix}}$.
 So this is the first column of the matrix $A$.}

\lang{de}{
 Nun verfahren wir mit der zweiten Spalte von $A$ genauso fort.
 Wir bestimmen $f(\begin{pmatrix} 0 \\ 1 \\ 0 \end{pmatrix})\textcolor{gray}{=f(v_2)}$:
 \[
 f(\begin{pmatrix} 0 \\ 1 \\ 0 \end{pmatrix}) = \begin{pmatrix}-2 \cdot 0 - 3 \cdot 1 + 2 \cdot 0 \\ 1 \cdot 0 - 1 \cdot 1 + 2 \cdot 0 \end{pmatrix}
 = \begin{pmatrix}-3 \\ -1 \end{pmatrix}\textcolor{gray}{=f(v_2)}
 \]
 Wir schreiben den Vektor auf der rechten Seite bezüglich der Basis $C$\textcolor{gray}{, d.h. als} \textcolor{gray}{$f(v_2)=\sum_{i=1}^{m=2}a_{i2}w_i=a_{12}w_1+a_{22}w_2$.}
 \[
 \begin{pmatrix}-3 \\ -1 \end{pmatrix} = \textcolor{gray}{\underbrace{\textcolor{#0066CC}{-2}}_{=a_{12}}} \cdot \textcolor{gray}{\underbrace{\textcolor{black}{\begin{pmatrix} 1 \\ 0 \end{pmatrix}}}_{=w_1}}\textcolor{gray}{\underbrace{\textcolor{#0066CC}{-1}}_{=a_{22}}} \cdot \textcolor{gray}{\underbrace{\textcolor{black}{\begin{pmatrix} 1 \\ 1 \end{pmatrix}}}_{=w_2}}
 \]
 Dann ist also die zweite Spalte von $A$ gegeben durch
 $\textcolor{#0066CC}{\begin{pmatrix}-2 \\ -1 \end{pmatrix}}$.}

 \lang{en}{
 We proceed with the second column of $A$ the same way.
 We determine $f(\begin{pmatrix} 0 \\ 1 \\ 0 \end{pmatrix})\textcolor{gray}{=f(v_2)}$:
 \[
 f(\begin{pmatrix} 0 \\ 1 \\ 0 \end{pmatrix}) = \begin{pmatrix}-2 \cdot 0 - 3 \cdot 1 + 2 \cdot 0 \\ 1 \cdot 0 - 1 \cdot 1 + 2 \cdot 0 \end{pmatrix}
 = \begin{pmatrix}-3 \\ -1 \end{pmatrix}\textcolor{gray}{=f(v_2)}
 \]
 We write the vector of the right side regarding the basis $C$\textcolor{gray}{, so as} \textcolor{gray}{$f(v_2)=\sum_{i=1}^{m=2}a_{i2}w_i=a_{12}w_1+a_{22}w_2$.}
 \[
 \begin{pmatrix}-3 \\ -1 \end{pmatrix} = \textcolor{gray}{\underbrace{\textcolor{#0066CC}{-2}}_{=a_{12}}} \cdot \textcolor{gray}{\underbrace{\textcolor{black}{\begin{pmatrix} 1 \\ 0 \end{pmatrix}}}_{=w_1}}\textcolor{gray}{\underbrace{\textcolor{#0066CC}{-1}}_{=a_{22}}} \cdot \textcolor{gray}{\underbrace{\textcolor{black}{\begin{pmatrix} 1 \\ 1 \end{pmatrix}}}_{=w_2}}
 \]
 Then the second column von $A$ is given by
 $\textcolor{#0066CC}{\begin{pmatrix}-2 \\ -1 \end{pmatrix}}$.}

 \lang{de}{
 Schließlich bestimmen wir die dritte Spalte von $A$:
 \[
 f(\begin{pmatrix} 1 \\ 1 \\ 1 \end{pmatrix}) = \begin{pmatrix}-2 \cdot 1 - 3 \cdot 1 + 2 \cdot 1 \\ 1 \cdot 1 - 1 \cdot 1 + 2 \cdot 1 \end{pmatrix}
 = \begin{pmatrix}-3 \\ 2 \end{pmatrix}\textcolor{gray}{=f(v_3)}
 \]
 Der Vektor auf der rechten Seite wird nun bezüglich der Basis $C$ ausgedrückt\textcolor{gray}{, d.h. als} \textcolor{gray}{$f(v_3)=\sum_{i=1}^{m=2}a_{i3}w_i=a_{13}w_1+a_{23}w_2$.}
 \[
 \begin{pmatrix}-3 \\ 2 \end{pmatrix} = \textcolor{gray}{\underbrace{\textcolor{#00CC00}{-5}}_{=a_{13}}} \cdot \textcolor{gray}{\underbrace{\textcolor{black}{\begin{pmatrix} 1 \\ 0 \end{pmatrix}}}_{=w_1}}+\textcolor{gray}{\underbrace{\textcolor{green}{2}}_{=a_{23}}} \cdot \textcolor{gray}{\underbrace{\textcolor{black}{\begin{pmatrix} 1 \\ 1 \end{pmatrix}}}_{=w_2}}
 \]
 Damit ist die Abbildungsmatrix $A$ gegeben durch
 \[
A={\,}_Cf_B=\begin{pmatrix}\textcolor{#CC6600}{-3} & \textcolor{#0066CC}{-2} & \textcolor{#00CC00}{-5} \\ \textcolor{#CC6600}{1} & \textcolor{#0066CC}{-1} & \textcolor{#00CC00}{2}\end{pmatrix}\textcolor{gray}{=\begin{pmatrix}a_{11}& a_{12}& a_{13}\\a_{21} & a_{22} & a_{23}\end{pmatrix}}
 \]}
 \lang{en}{
Eventually we determine the third column of $A$:
 \[
 f(\begin{pmatrix} 1 \\ 1 \\ 1 \end{pmatrix}) = \begin{pmatrix}-2 \cdot 1 - 3 \cdot 1 + 2 \cdot 1 \\ 1 \cdot 1 - 1 \cdot 1 + 2 \cdot 1 \end{pmatrix}
 = \begin{pmatrix}-3 \\ 2 \end{pmatrix}\textcolor{gray}{=f(v_3)}
 \]
 We note the vector on the right side now regarding the basis $C$ \textcolor{gray}{, so as} \textcolor{gray}{$f(v_3)=\sum_{i=1}^{m=2}a_{i3}w_i=a_{13}w_1+a_{23}w_2$.}
 \[
 \begin{pmatrix}-3 \\ 2 \end{pmatrix} = \textcolor{gray}{\underbrace{\textcolor{#00CC00}{-5}}_{=a_{13}}} \cdot \textcolor{gray}{\underbrace{\textcolor{black}{\begin{pmatrix} 1 \\ 0 \end{pmatrix}}}_{=w_1}}+\textcolor{gray}{\underbrace{\textcolor{green}{2}}_{=a_{23}}} \cdot \textcolor{gray}{\underbrace{\textcolor{black}{\begin{pmatrix} 1 \\ 1 \end{pmatrix}}}_{=w_2}}
 \]
 Therefore the transformations matrix $A$ is given by
 \[
A={\,}_Cf_B=\begin{pmatrix}\textcolor{#CC6600}{-3} & \textcolor{#0066CC}{-2} & \textcolor{#00CC00}{-5} \\ \textcolor{#CC6600}{1} & \textcolor{#0066CC}{-1} & \textcolor{#00CC00}{2}\end{pmatrix}\textcolor{gray}{=\begin{pmatrix}a_{11}& a_{12}& a_{13}\\a_{21} & a_{22} & a_{23}\end{pmatrix}}
 \]}
\end{incremental}
\end{example}

%%%%%%%%%%%%%%%%%%%%%%%%%%%%%%%%%%%%%%%%%%%%%%%%%%%

\begin{example} \label{ex:lin_abb_drehung}
\lang{de}{
Im $\R^2 \,$ lassen sich einige lineare Abbidungen gut veranschaulichen.}
\lang{en}{
In $\R^2 \,$ we can easily illustrate some linear maps.}

  \begin{enumerate}
  
    \item \lang{de}{So ist beispielsweise eine Drehung um den Winkel $\alpha$ gegen den Uhrzeigersinn darstellbar durch die lineare Abbildung
     \[
     f_{\alpha}: \R^2 \to \R^2, \begin{pmatrix} x \\ y \end{pmatrix} \mapsto \begin{pmatrix} \cos(\alpha) & -\sin(\alpha) \\\sin(\alpha) & \cos(\alpha)\end{pmatrix} \cdot \begin{pmatrix} x \\ y \end{pmatrix},
     \]

    mit der Abbildungsmatrix  
     $
     A(\alpha)={\,}\begin{pmatrix} \cos(\alpha) & -\sin(\alpha) \\ \sin(\alpha) & \cos(\alpha)\end{pmatrix}
     $}

     \lang{en}{A rotation counterclockwiese through the angle $\alpha$ can be represented by the linear map
     \[
     f_{\alpha}: \R^2 \to \R^2, \begin{pmatrix} x \\ y \end{pmatrix} \mapsto \begin{pmatrix} \cos(\alpha) & -\sin(\alpha) \\\sin(\alpha) & \cos(\alpha)\end{pmatrix} \cdot \begin{pmatrix} x \\ y \end{pmatrix},
     \]

    with the transformation matrix  
     $
     A(\alpha)={\,}\begin{pmatrix} \cos(\alpha) & -\sin(\alpha) \\ \sin(\alpha) & \cos(\alpha)\end{pmatrix}
     $}

    \\
%    \begin{showhide}[\buttonlabels{Zeige Beispiel}{Verstecke Beispiel}]
    
    \lang{de}{Betrachten wir den konkreten Fall $\alpha= \frac{\pi}{2}$, dann lautet die Abbildungsmatrix 
     $
     A\left(\frac{\pi}{2}\right)={\,}\begin{pmatrix} 0 & -1 \\ 1 & 0 \end{pmatrix}.
     $
    Sie bildet einen beliebigen Vektor $\begin{pmatrix} x \\ y \end{pmatrix} \in \R^2 \,$ auf den Vektor 
    $\; \begin{pmatrix} 0 & -1 \\ 1 & 0\end{pmatrix} \cdot \begin{pmatrix} x \\ y \end{pmatrix}= \begin{pmatrix} -y \\ x \end{pmatrix}$ ab,
    was genau einer Drehung des Vektors um $\frac{\pi}{2}$ gegen den Uhrzeigersinn entspricht, so zum Beispiel
    \[
       \begin{pmatrix} 4 \\ 2 \end{pmatrix}{\,} \mapsto {\,} \begin{pmatrix} 0 & -1 \\ 1 & 0\end{pmatrix} \cdot \begin{pmatrix} 4 \\ 2 \end{pmatrix}= \begin{pmatrix} -2 \\ 4 \end{pmatrix}
    \]}

    \lang{en}{For the concrete case $\alpha= \frac{\pi}{2}$, the transformationmatrix is 
     $
     A\left(\frac{\pi}{2}\right)={\,}\begin{pmatrix} 0 & -1 \\ 1 & 0 \end{pmatrix}.
     $
    The matrix maps any vector $\begin{pmatrix} x \\ y \end{pmatrix} \in \R^2 \,$ onto the vector
    $\; \begin{pmatrix} 0 & -1 \\ 1 & 0\end{pmatrix} \cdot \begin{pmatrix} x \\ y \end{pmatrix}= \begin{pmatrix} -y \\ x \end{pmatrix}$,
    which is exactly a counterclockwise rotation of the vector through the angle $\frac{\pi}{2}$, for example
    \[
       \begin{pmatrix} 4 \\ 2 \end{pmatrix}{\,} \mapsto {\,} \begin{pmatrix} 0 & -1 \\ 1 & 0\end{pmatrix} \cdot \begin{pmatrix} 4 \\ 2 \end{pmatrix}= \begin{pmatrix} -2 \\ 4 \end{pmatrix}
    \]}

        \begin{center}
         \image{T403a_Rotation}
        \end{center}  
 
%    \end{showhide}
    
    \item \lang{de}{Weitere Beispiele für lineare Abbildungen im $\R^2$ sind} \lang{en}{More examples for linear maps in $\R^2$ are}
    
    \begin{enumerate}
      \item \lang{de}{die \textcolor{#00CC00}{Spiegelung an der x-Achse}, gegeben durch die Abbildungsmatrix  $ \begin{pmatrix}  1 & 0 \\ 0 & -1 \end{pmatrix}.$ 
            \\ Der Vektor $(4;2)$ wird hiermit abgebildet auf
           \[
           \begin{pmatrix} 4 \\ 2 \end{pmatrix}{\,} \mapsto {\,} \begin{pmatrix} 1 & 0 \\ 0 & -1 \end{pmatrix} \cdot \begin{pmatrix} 4 \\ 2 \end{pmatrix}
           = \textcolor{#00CC00}{\begin{pmatrix} 4 \\ -2 \end{pmatrix}}
           \]}
           \lang{en}{the \textcolor{#00CC00}{reflection along the $x$-axis}, given by the transformation matrix $ \begin{pmatrix}  1 & 0 \\ 0 & -1 \end{pmatrix}.$ 
            \\ The vectors $(4;2)$ is mapped to
           \[
           \begin{pmatrix} 4 \\ 2 \end{pmatrix}{\,} \mapsto {\,} \begin{pmatrix} 1 & 0 \\ 0 & -1 \end{pmatrix} \cdot \begin{pmatrix} 4 \\ 2 \end{pmatrix}
           = \textcolor{#00CC00}{\begin{pmatrix} 4 \\ -2 \end{pmatrix}}
           \]}

      
      \item \lang{de}{die \textcolor{#0066CC}{Spiegelung an der y-Achse}, gegeben durch die Abbildungsmatrix  $ \begin{pmatrix} -1 & 0 \\ 0 &  1 \end{pmatrix}.$
            \\ Der Vektor $(4;2)$ wird hiermit abgebildet auf
           \[
           \begin{pmatrix} 4 \\ 2 \end{pmatrix}{\,} \mapsto {\,} \begin{pmatrix} -1 & 0 \\ 0 & 1 \end{pmatrix} \cdot \begin{pmatrix} 4 \\ 2 \end{pmatrix}
           = \textcolor{#0066CC}{\begin{pmatrix} -4 \\ 2 \end{pmatrix}}
           \]}
           \lang{en}{the \textcolor{#0066CC}{reflection along the $y$-axis}, given b the transformation matrix  $ \begin{pmatrix} -1 & 0 \\ 0 &  1 \end{pmatrix}.$
            \\ The vector $(4;2)$ is mapped to
           \[
           \begin{pmatrix} 4 \\ 2 \end{pmatrix}{\,} \mapsto {\,} \begin{pmatrix} -1 & 0 \\ 0 & 1 \end{pmatrix} \cdot \begin{pmatrix} 4 \\ 2 \end{pmatrix}
           = \textcolor{#0066CC}{\begin{pmatrix} -4 \\ 2 \end{pmatrix}}
           \]}


      \item \lang{de}{eine \textcolor{#CC6600}{Streckung der y-Koordinate} um den Faktor $\lambda \in \R$, 
            gegeben durch die Abbildungsmatrix $ \begin{pmatrix} 1 & 0 \\ 0 & \lambda \end{pmatrix}.$
            \\ Der Vektor $(4;2)$ wird für $\lambda = 2$ hiermit abgebildet auf
           \[
           \begin{pmatrix} 4 \\ 2 \end{pmatrix}{\,} \mapsto {\,} \begin{pmatrix} 1 & 0 \\ 0 & 2 \end{pmatrix} \cdot \begin{pmatrix} 4 \\ 2 \end{pmatrix}
           = \textcolor{#CC6600}{\begin{pmatrix} 4 \\ 4 \end{pmatrix}}
           \]}
           \lang{en}{a \textcolor{#CC6600}{extension of the $y$-coordinate} iwith the factor $\lambda \in \R$, 
            given by the transformation matrix $ \begin{pmatrix} 1 & 0 \\ 0 & \lambda \end{pmatrix}.$
            \\ For $\lambda = 2$ the vector $(4;2)$ is mapped to
           \[
           \begin{pmatrix} 4 \\ 2 \end{pmatrix}{\,} \mapsto {\,} \begin{pmatrix} 1 & 0 \\ 0 & 2 \end{pmatrix} \cdot \begin{pmatrix} 4 \\ 2 \end{pmatrix}
           = \textcolor{#CC6600}{\begin{pmatrix} 4 \\ 4 \end{pmatrix}}
           \]}


          \begin{center}
           \image{T403a_LinearMaps}
          \end{center}  

    \end{enumerate}
  \end{enumerate}


\end{example}

%%%%%%%%%%%%%%%%%%%%%%%%%%%%%%%%%%%%%%%%%%%%%%%%%%%%

\begin{example}
\lang{de}{
Wir betrachten den Vektorraum $P_n$ der reellen Polynomfunktionen $p$ mit $\text{grad} (p)\leq n$. \\
Der sogenannte Differentialoperator
\[
D: P_n \to P_{n-1}, p \mapsto p\prime,
\]
der jedem Polynom aus $ P_n $ seine erste Ableitung $p\prime $ zuordnet.}
\lang{en}{
We consider the vector space $P_n$, which consists of the real polynomial functions $p$ with $\text{grad} (p)\leq n$. \\
The so-called differential operator
\[
D: P_n \to P_{n-1}, p \mapsto p\prime,
\]
maps each polynomial function in $ P_n $ to its first derivative $p\prime $.}

\\
\begin{incremental}

\step %1
\lang{de}{
Der Diffentialoperator $D$ ist eine lineare Abbildung, 
denn nach den Ableitungsregeln gilt für je zwei Polynome $p$ und $q$ 
\[
D(p+q)=(p+q)\prime =p\prime +q\prime = D(p)+D(q).
\]
Außerdem ist 
\[ 
D(\alpha p)=(\alpha p)\prime = \alpha p\prime=\alpha D(p).
\]}
\lang{en}{
The differential operator $D$ is a linear map, 
because with the differential rule holds for zwo polynomial functions $p$ and $q$ 
\[
D(p+q)=(p+q)\prime =p\prime +q\prime = D(p)+D(q).
\]
Futhermore it is 
\[ 
D(\alpha p)=(\alpha p)\prime = \alpha p\prime=\alpha D(p).
\]}

\step %2

\lang{de}{Für Polynomfunktionen können wir zudem genau angeben, wie die Ableitung aussieht.
Ist 
\[p(x)=\sum_{j=0}^n a_j x^j, \] dann ist \[p'(x)=\sum_{j=0}^{n-1} (j+1) \cdot a_{j+1} x^j.\]

Wir wollen die Abbildungsmatrix bezüglich der Standardbasis $S:= \{p_0(x)=x^0=1; p_1(x)=x^1=x; ...; p_n(x)=x^n \}$ 
auf dem Raum der Polynomfunktionen bestimmen.
Dazu müssen wir also das Abbildungsverhalten der Polynomfunktionen $p_0(x),...,p_n(x)$ betrachten.
Es ist}

\lang{en}{We can give the exact form of the derivative of polynomial functions.
Is
\[p(x)=\sum_{j=0}^n a_j x^j, \] then it is \[p'(x)=\sum_{j=0}^{n-1} (j+1) \cdot a_{j+1} x^j.\]

We want to determine the transformation matrix for the standard basis $S:= \{p_0(x)=x^0=1; p_1(x)=x^1=x; ...; p_n(x)=x^n \}$ 
of the vector space of polynomial functions.
Therefore we discuss how the polynomial functions $p_0(x),...,p_n(x)$ are mapped.
It is}

\begin{align*}
    D(p_0)(x) &= p_0'(x)= 0 \\
    D(p_1)(x) &= p_1'(x)=1 \\
    D(p_2)(x) &= p_2'(x)=2x \\
    & \vdots \\
    D(p_n)(x) &= p_n'(x)=n x^{n-1}
\end{align*}

\lang{de}{Diese Funktionen müssen wir nun als Koordinatenvektoren bezüglich der Standardbasis des Vektorraums $P_{n-1}$ schreiben.
Wir betrachten ein paar Beispiele:
\[
D(p_0)(x)= 0 = \textcolor{#CC6600}{0} \cdot 1 + \textcolor{#CC6600}{0} \cdot x + ... + \textcolor{#CC6600}{0} \cdot x^{n-1}
\]
Deshalb ist der Koordinatenvektor von $D(p_0)$ gegeben durch $\textcolor{#CC6600}{\begin{pmatrix}0 \\ 0 \\ \vdots \\ 0 \end{pmatrix}}$.}
\lang{en}{We need to write those functions as coordinate vectore regarding to standard basis of the vector space $P_{n-1}$.
We consider some examples:
\[
D(p_0)(x)= 0 = \textcolor{#CC6600}{0} \cdot 1 + \textcolor{#CC6600}{0} \cdot x + ... + \textcolor{#CC6600}{0} \cdot x^{n-1}
\]
Therefore is the coordinate vector of $D(p_0)$ given by $\textcolor{#CC6600}{\begin{pmatrix}0 \\ 0 \\ \vdots \\ 0 \end{pmatrix}}$.}

\lang{de}{
\[
D(p_1)(x)= 1 = \textcolor{#0066CC}{1} \cdot 1 + \textcolor{#0066CC}{0} \cdot x + ... + \textcolor{#0066CC}{0} \cdot x^{n-1}
\]
Deshalb ist der Koordinatenvektor von $D(p_1)$ gegeben durch $\textcolor{#0066CC}{\begin{pmatrix} 1 \\ 0 \\ \vdots \\ 0 \end{pmatrix}}$.}
\lang{en}{
\[
D(p_1)(x)= 1 = \textcolor{#0066CC}{1} \cdot 1 + \textcolor{#0066CC}{0} \cdot x + ... + \textcolor{#0066CC}{0} \cdot x^{n-1}
\]
Therefore the coordinate vector of $D(p_1)$ is given by $\textcolor{#0066CC}{\begin{pmatrix} 1 \\ 0 \\ \vdots \\ 0 \end{pmatrix}}$.}

\lang{de}{Dies führt man nun fort bis zum letzten Polynom:
\[
D(p_n)(x)= n x^{n-1} = \textcolor{#00CC00}{0} \cdot 1 + \textcolor{#00CC00}{0} \cdot x + ... + \textcolor{#00CC00}{n} \cdot x^{n-1}
\]
Deshalb ist der Koordinatenvektor von $D(p_n)$ gegeben durch $\textcolor{#00CC00}{\begin{pmatrix}0 \\ \vdots \\ 0 \\ n \end{pmatrix}}$.}
\lang{en}{We continue likes this until the last polynomial function:
\[
D(p_n)(x)= n x^{n-1} = \textcolor{#00CC00}{0} \cdot 1 + \textcolor{#00CC00}{0} \cdot x + ... + \textcolor{#00CC00}{n} \cdot x^{n-1}
\]
Therefore the coordinate vector of $D(p_n)$ is given by $\textcolor{#00CC00}{\begin{pmatrix}0 \\ \vdots \\ 0 \\ n \end{pmatrix}}$.}



\lang{de}{
Die Abbildungsmatrix ist damit gegeben durch}
\lang{en}{
The transformation matrix is given by}
\[
A={\,}_SD_S=\begin{pmatrix} \textcolor{#CC6600}{0} & \textcolor{#0066CC}{1} & 0 & 0 & \ldots & \textcolor{#00CC00}{0}
\\ \textcolor{#CC6600}{0} & \textcolor{#0066CC}{0} & 2 & 0 & \ldots & \textcolor{#00CC00}{0}
\\ \textcolor{#CC6600}{0} & \textcolor{#0066CC}{0} & 0 & 3 & & \textcolor{#00CC00}{0}
\\ \vdots & \vdots & \vdots &  & \ddots  & \vdots
\\ \textcolor{#CC6600}{0} & \textcolor{#0066CC}{0} & 0 & 0 & \ldots & \textcolor{#00CC00}{n}
\end{pmatrix}.
\]
\end{incremental}

%%% Video K.M. -  2.Teil des Videos 10829 = 11285 Koordinaten_3b
%
\lang{de}{\floatright{\href{https://api.stream24.net/vod/getVideo.php?id=10962-2-11285&mode=iframe&speed=true}
{\image[75]{00_video_button_schwarz-blau}}}}\\
\\


\end{example}

\section{\lang{de}{Basiswechsel und Koordinatentransformation} \lang{en}{Basis transformation and coordinate transformation}} \label{sec:basiswechsel}

\lang{de}{In Beispiel \ref{bsp:lin_abb} war es nicht sehr schwer, die Vektoren bezüglich der Basis $C$ zu schreiben.
In höheren Dimensionen jedoch wird das Bestimmen der Koordinatenvektoren schwierig. Um in solchen Fällen
den Koordinatenvektor zu bestimmen, bedienen wir uns folgender Tatsache:}
\lang{en}{In example \ref{bsp:lin_abb} is was easy to note the vectors with respect of the basis $C$.
But in higher dimensions it gets more difficult to determine the coordinate vectors.
For those cases, we may use the following fact:}

\begin{rule}[\lang{de}{Basiswechsel} \lang{en}{Basis transformation}] \label{Basiswechsel}
\lang{de}{Es seien $V,W,U$ drei $\K$-Vektorräume und $f:V \to W$ und $g: W \to U$ lineare Abbildungen. \\
 Desweiteren seien $B$ eine Basis von $V$, $C$ eine Basis von $W$ und $D$ eine Basis von $U$. Dann gilt
 \[
  {\,}_D (g \circ f)_B = {\,}_D{g}_C \cdot _C{f}_B.
 \]}
 \lang{en}{Let $V,W,U$ be three $\K$-vector spaces and $f:V \to W$ and $g: W \to U$ linear maps. \\
 Furthermore is $B$ a basis of $V$, $C$ a basis of $W$ and $D$ a basis of $U$. Then holds
 \[
  {\,}_D (g \circ f)_B = {\,}_D{g}_C \cdot _C{f}_B.
 \]}
\end{rule}

\begin{proof*}[\lang{de}{Beweis} \lang{en}{Proof}]
% \begin{block}[explanation]
\lang{de}{
Seien $B = \{v_1 ; . . . ; v_n \}$, $C=\{ w_1;...;w_m \}$ und $D=\{ u_1;...;u_l\}.$ 
Wir betrachten nun die Spalte $i$ der Matrix ${\,}_D (g \circ f)_B.$
Dies entspricht der Koordinatendarstellung von $(g\circ f)(v_j)$ bezüglich der Basis $D.$}
\lang{en}{
Let be $B = \{v_1 ; . . . ; v_n \}$, $C=\{ w_1;...;w_m \}$ and $D=\{ u_1;...;u_l\}.$  
We consider now the column $i$ of the matrix ${\,}_D (g \circ f)_B.$
This corresponds to the coordinate representation of $(g\circ f)(v_j)$ with respect of the basis $D.$}

\lang{de}{Nun l\"asst sich $f(v_j)$ schreiben als
\[
f(v_j)= \sum_{i=1}^m a_{ij} w_i
\]
wobei $A=(a_{ij})$ die Abbildungsmatrix ${\,}_C{f}_B$ ist.  Da $g$ eine lineare Abbildung ist, gilt 
\[
(g\circ f)(v_j) = g(\sum_{i=1}^m a_{ij} w_i)= \sum_{i=1}^m a_{ij} g(w_i).
\]
Nun l\"asst sich auch $g(w_i)$ schreiben als

\[g(w_i)=\sum_{k=1}^l \tilde{a}_{ki} u_k,
\]
wobei $\tilde{A}=(\tilde{a}_{ki})$ die Abbildungsmatrix von ${\,}_D{g}_C$ ist. Daher folgt
\[
(g\circ f)(v_j) = \sum_{i=1}^m a_{ij} \cdot g(w_i)
    = \sum_{i=1}^m a_{ij} \cdot \left( \sum_{k=1}^l \tilde{a}_{ki} u_k \right)
    = \sum_{i=1}^m \sum_{k=1}^l a_{ij} \tilde{a}_{ki} u_k 
    = \sum_{k=1}^l \left( \sum_{i=1}^m  a_{ij} \tilde{a}_{ki}\right) u_k
\]}
\lang{en}{Now we can write $f(v_j)$ as
\[
f(v_j)= \sum_{i=1}^m a_{ij} w_i
\]
with the transformation matrix $A=(a_{ij})$ of ${\,}_C{f}_B$.  Since $g$ is a linear map, it holds 
\[
(g\circ f)(v_j) = g(\sum_{i=1}^m a_{ij} w_i)= \sum_{i=1}^m a_{ij} g(w_i).
\]
Now we can also write $g(w_i)$ as

\[g(w_i)=\sum_{k=1}^l \tilde{a}_{ki} u_k,
\]
with the tranformation matrix $\tilde{A}=(\tilde{a}_{ki})$ of ${\,}_D{g}_C$. Therefore concludes
\[
(g\circ f)(v_j) = \sum_{i=1}^m a_{ij} \cdot g(w_i)
    = \sum_{i=1}^m a_{ij} \cdot \left( \sum_{k=1}^l \tilde{a}_{ki} u_k \right)
    = \sum_{i=1}^m \sum_{k=1}^l a_{ij} \tilde{a}_{ki} u_k 
    = \sum_{k=1}^l \left( \sum_{i=1}^m  a_{ij} \tilde{a}_{ki}\right) u_k
\]}

\lang{de}{
Die Behauptung der Regel ist nun äquivalent dazu, dass der Eintrag $(k,i)$ der 
Matrix ${\,}_D{g}_C \cdot _C{f}_B$ gleich $\sum_{i=1}^m  a_{ij} \tilde{a}_{ki}$.
Dies ist richtig, wenn man sich an die 
\ref[content_02_matrizenmultiplikation][Regel für Matrixmultiplikation]{sec:matrix-matrix-mult}
 zurück erinnert.}
 \lang{en}{
The claim of the rule s equivalent to the thesis that, the entry $(k,i)$ of the matrix ${\,}_D{g}_C \cdot _C{f}_B$ 
is equal to $\sum_{i=1}^m  a_{ij} \tilde{a}_{ki}$.
This is correct, since we know the 
\ref[content_02_matrizenmultiplikation][rules for matrix multiplication]{sec:matrix-matrix-mult}.}
% \end{block}
\end{proof*}

\lang{de}{Aus Regel \ref{Basiswechsel} lässt sich nun einiges folgern. Als erste Anwendung beschäftigen wir uns mit Koordinatenwechseln.}
\lang{en}{There are several things we can conclude from rule \ref{Basiswechsel}. The first application we want to look at is the coordinate transformation.}

\lang{de}{
Bezüglich einer Basis $B = \{v_1 ; . . . ; v_n \}$ von $\K^n$ besitze der Vektor $v \in \K^n$ 
die Koordinatendarstellung \[\begin{pmatrix} b_{1} \\ \vdots \\ b_{n} \end{pmatrix},\] 
wobei
$
 v= \sum_{j=1}^n b_j v_j
$
gelte.}
\lang{en}{
With respect to a basis $B = \{v_1 ; . . . ; v_n \}$ of $\K^n$ the vector $v \in \K^n$ 
has the coordinate respresentation \[\begin{pmatrix} b_{1} \\ \vdots \\ b_{n} \end{pmatrix},\] 
whereat
$
 v= \sum_{j=1}^n b_j v_j
$
holds.}

\lang{de}{
Gehen wir zu einer neuen Basis $C = \{w_1 ; . . . ; w_n\}$ über, in der $v$ sich linear darstellen lässt
durch $v = \sum_{j=1}^n c_j w_j ,$
so besitzt derselbe Vektor bezüglich der neuen Basis $C$ die Koordinatendarstellung 
\[\begin{pmatrix} c_{1} \\ \vdots \\ c_{n} \end{pmatrix}.\]}
\lang{en}{
If we transform to a new basis $C = \{w_1 ; . . . ; w_n\}$, in which $v$ can be displayed as $v = \sum_{j=1}^n c_j w_j ,$
the coordinate representation of the same vector with respect to the new basis $C$ is 
\[\begin{pmatrix} c_{1} \\ \vdots \\ c_{n} \end{pmatrix}.\]}

\lang{de}{
Nun können wir die Identitätsabbildung $id$ auf einem beliebigen Vektorraum $V$, d.h. $id: \, V \to V, v \mapsto v$ als 
lineare Abbildung wählen und $B=\{v_1;...;v_n\}$ und $C=\{w_1;...;w_n\}$ zwei Basen des Vektorraums $V$.
Dann nennen wir die zugehörige Abbildungsmatrix ${\,}_C T_B \, (= {\,}_C id_B)$ auch \emph{Basistransformation} von der Basis $B$ in die Basis $C$. 
In dieser Abbildungsmatrix lässt sich dann in der Spalte $j$ die Koordinatendarstellung des Basisvektors $v_j$ 
bezüglich der Basis $C$ ablesen.}
\lang{en}{
We may utilise the identity map $id$ on any vector space, i.e. $id: \, V \to V, v \mapsto v$ as
as liner map. Furthermore may $B=\{v_1;...;v_n\}$ and $C=\{w_1;...;w_n\}$ be two bases of the vector space $V$.
Then we call the corresponding transformation matrix ${\,}_C T_B \, (= {\,}_C id_B)$ \emph{basis transformation} from the basis $B$ to the basis $C$. 
In this transformations matrix we find the coordine representation of the basis vector $v_j$ with respect to the basis $C$ in the $j$th column.}

%

\begin{example} \label{bsp:basiswechsel}

 \lang{de}{Wir betrachten die Identitätsabbildung auf dem $\R^2$
 \[
 id: \R^2 \to \R^2, \begin{pmatrix}x \\ y \end{pmatrix} \mapsto \begin{pmatrix}x \\ y\end{pmatrix}.
 \]
 die, wie man leicht erkennen kann, eine lineare Abbildung ist. Hiermit bestimmen wir nun die \emph{Basistransformation} 
 ${\,}_C T_S$ von der Standardbasis $S$ des $\R^2$ 
 in die Basis $C=\left\{ \begin{pmatrix} 1\\ 0 \end{pmatrix}; \begin{pmatrix} 1\\ 1\end{pmatrix}\right\}$ des $\R^2$, \, 
 die der Abbildungsmatrix ${\,}_C id_S$ entspricht.}
 \lang{en}{We consider the identity map of $\R^2$
 \[
 id: \R^2 \to \R^2, \begin{pmatrix}x \\ y \end{pmatrix} \mapsto \begin{pmatrix}x \\ y\end{pmatrix}.
 \]
 which is a linear map, as we can easily check. Here we determine the \emph{basis transformation} 
 ${\,}_C T_S$ of the standard basis $S$ of $\R^2$ 
 into the basis $C=\left\{ \begin{pmatrix} 1\\ 0 \end{pmatrix}; \begin{pmatrix} 1\\ 1\end{pmatrix}\right\}$ of $\R^2$, \, 
 which corresponds to the transformation map ${\,}_C id_S$.}

\\
%
% Klappfunktion entfernt, da Beispiel mit Anschauung wichtig zum Verständnis
%
%\begin{incremental}

%\step %1

 \lang{de}{Aus 
 \[ id(\begin{pmatrix} 1\\ 0 \end{pmatrix})=\begin{pmatrix} 1\\ 0 \end{pmatrix}
    = \textcolor{#00CCCC}{1} \cdot \begin{pmatrix} 1\\ 0 \end{pmatrix} + \textcolor{#00CCCC}{0} \cdot \begin{pmatrix} 1\\ 1\end{pmatrix}
 \]
 und
 \[id(\begin{pmatrix} 0\\ 1 \end{pmatrix})=\begin{pmatrix} 0\\ 1 \end{pmatrix}
    = \textcolor{#CC00CC}{-1} \cdot \begin{pmatrix} 1\\ 0 \end{pmatrix} + \textcolor{#CC00CC}{1} \cdot \begin{pmatrix} 1\\ 1\end{pmatrix} 
 \] 
 ergibt sich die Abbildungsmatrix (mit der Koordinatendarstellung des Basisvektors $v_j$ 
bezüglich der Basis $C$ in der Spalte $j\,$)
 \[
 {\,}_C id_S= \begin{pmatrix} \textcolor{#00CCCC}{1} & \textcolor{#CC00CC}{-1} \\ \textcolor{#00CCCC}{0} & \textcolor{#CC00CC}{1} \end{pmatrix}={\,}_C T_S.
 \]}

 \lang{en}{The transformationmatrix ${\,}_C id_S $ consists of  
 \[ id(\begin{pmatrix} 1\\ 0 \end{pmatrix})=\begin{pmatrix} 1\\ 0 \end{pmatrix}
    = \textcolor{#00CCCC}{1} \cdot \begin{pmatrix} 1\\ 0 \end{pmatrix} + \textcolor{#00CCCC}{0} \cdot \begin{pmatrix} 1\\ 1\end{pmatrix}
 \]
 and
 \[id(\begin{pmatrix} 0\\ 1 \end{pmatrix})=\begin{pmatrix} 0\\ 1 \end{pmatrix}
    = \textcolor{#CC00CC}{-1} \cdot \begin{pmatrix} 1\\ 0 \end{pmatrix} + \textcolor{#CC00CC}{1} \cdot \begin{pmatrix} 1\\ 1\end{pmatrix}. 
 \] 
The $j$th column represents the coordinate representation of the basis vector $v_j$ with respect of the basis $C$.
The transformation matrix is then
 \[
 {\,}_C id_S= \begin{pmatrix} \textcolor{#00CCCC}{1} & \textcolor{#CC00CC}{-1} \\ \textcolor{#00CCCC}{0} & \textcolor{#CC00CC}{1} \end{pmatrix}={\,}_C T_S.
 \]}

%%%%%%%%%%
%\\
%\step %2

\lang{de}{Mit der Basistransformation kann nun jeder beliebige Vektor $\begin{pmatrix} x\\ y \end{pmatrix} \in \R^2$ aus der Standardbasis $S$ in seine
Koordinatendarstellung bezüglich der Basis $C$ umgerechnet werden. Betrachten wir als Beispiel den Vektor $\textcolor{#00CC00}{\begin{pmatrix} 4\\ 2 \end{pmatrix}}$.
Dieser hat bzgl. der Standardbasis die Koordinatendarstellung $\textcolor{#0066CC}{\begin{pmatrix} 4\\ 2 \end{pmatrix}}$, denn es gilt:
\[
   \textcolor{#00CC00}{\begin{pmatrix} 4\\ 2 \end{pmatrix}}=\textcolor{#0066CC}{4} \cdot \begin{pmatrix} 1\\ 0 \end{pmatrix} + \textcolor{#0066CC}{2} \cdot \begin{pmatrix} 0\\ 1 \end{pmatrix}
\]
Graphisch dargestellt, in dem von der Standardbasis $S$ aufgespannten Koordinatensystem, erhalten wir
    \begin{center}
     \image{T403a_Basis_A}
    \end{center}  }

\lang{en}{The basis transformation transforms any vector $\begin{pmatrix} x\\ y \end{pmatrix} \in \R^2$ from the standard basis $S$ 
into its coordinate representation regarding the basies $C$. Let us consider the vector $\textcolor{#00CC00}{\begin{pmatrix} 4\\ 2 \end{pmatrix}}$ as an example.
The coordinate representation with respect to the standard basis is $\textcolor{#0066CC}{\begin{pmatrix} 4\\ 2 \end{pmatrix}}$, because it holds:
\[
   \textcolor{#00CC00}{\begin{pmatrix} 4\\ 2 \end{pmatrix}}=\textcolor{#0066CC}{4} \cdot \begin{pmatrix} 1\\ 0 \end{pmatrix} + \textcolor{#0066CC}{2} \cdot \begin{pmatrix} 0\\ 1 \end{pmatrix}
\]
In the coordinate system, set up by the standard basis $S$, we have graphically represented
    \begin{center}
     \image{T403a_Basis_A} %Sprache anpassen!!
    \end{center} }

%\step %3

\lang{de}{
Mithilfe der Transformationsmatrix ${\,}_C T_S \,$ zum Basiswechsel berechnen wir nun die Koodinatendarstellung
des Vektors $ \textcolor{#00CC00}{\begin{pmatrix} 4\\ 2 \end{pmatrix}}$ bzgl. der Basis $C=\left\{ \begin{pmatrix} 1\\ 0 \end{pmatrix}; \begin{pmatrix} 1\\ 1\end{pmatrix}\right\}$
\[
   {\,}_C T_S \, \cdot \begin{pmatrix} 4\\ 2 \end{pmatrix}=\begin{pmatrix} 1 & -1 \\ 0 & 1 \end{pmatrix}\cdot \begin{pmatrix} 4\\ 2 \end{pmatrix}
   = \textcolor{#CC6600}{\begin{pmatrix} 2\\ 2 \end{pmatrix}}
\]
Es ist also 
\[
   \textcolor{#00CC00}{\begin{pmatrix} 4\\ 2 \end{pmatrix}}=\textcolor{#CC6600}{2} \cdot \begin{pmatrix} 1\\ 0 \end{pmatrix} + \textcolor{#CC6600}{2} \cdot \begin{pmatrix} 1\\ 1 \end{pmatrix},
\] 
was man auch grapisch ablesen kann
    \begin{center}
     \image{T403a_Basis_B}
    \end{center}}   

\lang{en}{
With the help of the transformation matrix ${\,}_C T_S \,$ of this basis transformation we may now calculate
the coordinate representation of the vector $ \textcolor{#00CC00}{\begin{pmatrix} 4\\ 2 \end{pmatrix}}$
regarding the basis $C=\left\{ \begin{pmatrix} 1\\ 0 \end{pmatrix}; \begin{pmatrix} 1\\ 1\end{pmatrix}\right\}$
\[
   {\,}_C T_S \, \cdot \begin{pmatrix} 4\\ 2 \end{pmatrix}=\begin{pmatrix} 1 & -1 \\ 0 & 1 \end{pmatrix}\cdot \begin{pmatrix} 4\\ 2 \end{pmatrix}
   = \textcolor{#CC6600}{\begin{pmatrix} 2\\ 2 \end{pmatrix}}
\]
So we have 
\[
   \textcolor{#00CC00}{\begin{pmatrix} 4\\ 2 \end{pmatrix}}=\textcolor{#CC6600}{2} \cdot \begin{pmatrix} 1\\ 0 \end{pmatrix} + \textcolor{#CC6600}{2} \cdot \begin{pmatrix} 1\\ 1 \end{pmatrix},
\] 
which we can also read from the graphic.
    \begin{center}
     \image{T403a_Basis_B}
    \end{center}}   

%\step %4
\lang{de}{
Die graphische Darstellung in dem von der Basis $C$ aufgespannten Koordinatensystem ist dann

    \begin{center}
     \image{T403a_Basis_C}
    \end{center}} 

\lang{en}{
The graphic representation of the coordinate system, set up by the basis $C$, is now

    \begin{center}
     \image{T403a_Basis_C} %Sprache anpassen!
    \end{center}} 
%\end{incremental}

\end{example}
%%%%%%%%%
\\
\begin{example}
\lang{de}{
 Wir betrachten nun noch einmal die lineare Abbildung aus Beispiel \ref{bsp:lin_abb}:
 \[
 f: \R^3 \to \R^2, \begin{pmatrix}x \\ y\\ z \end{pmatrix} \mapsto \begin{pmatrix}-2x-3y+2z \\x-y+2z\end{pmatrix}.
 \]}

 \lang{en}{
 We consider again the linear map from example \ref{bsp:lin_abb}:
 \[
 f: \R^3 \to \R^2, \begin{pmatrix}x \\ y\\ z \end{pmatrix} \mapsto \begin{pmatrix}-2x-3y+2z \\x-y+2z\end{pmatrix}.
 \]}
\\
\begin{incremental}
 
\step %1
\lang{de}{
 Bezüglich der Standardbasis des $\R^2=\left\{ \begin{pmatrix}1 \\ 0 \end{pmatrix}; \begin{pmatrix}0 \\ 1 \end{pmatrix} \right\}$ 
 lassen sich die Berechnungen zunächst einfach darstellen:
 \[
  f(\begin{pmatrix}1 \\ 0 \\ 0\end{pmatrix})=\begin{pmatrix}-2\cdot 1 -3 \cdot 0 +2 \cdot 0 \\ 1 \cdot 1 - 1 \cdot 0 +2 \cdot 0\end{pmatrix} 
  = \begin{pmatrix} -2 \\ 1\end{pmatrix}
  = \textcolor{#CC6600}{-2} \cdot \begin{pmatrix} 1 \\ 0 \end{pmatrix} + \textcolor{#CC6600}{1} \cdot \begin{pmatrix} 0 \\ 1 \end{pmatrix}
 \]
 \[
 f(\begin{pmatrix} 0 \\ 1 \\ 0 \end{pmatrix}) = \begin{pmatrix}-2 \cdot 0 - 3 \cdot 1 + 2 \cdot 0 \\ 1 \cdot 0 - 1 \cdot 1 + 2 \cdot 0 \end{pmatrix}
 = \begin{pmatrix}-3 \\ -1 \end{pmatrix}
 = \textcolor{#0066CC}{-3} \cdot \begin{pmatrix} 1 \\ 0 \end{pmatrix} + (\textcolor{#0066CC}{-1}) \cdot \begin{pmatrix} 0 \\ 1 \end{pmatrix}

 \]
 \[
 f(\begin{pmatrix} 1 \\ 1 \\ 1 \end{pmatrix}) = \begin{pmatrix}-2 \cdot 1 - 3 \cdot 1 + 2 \cdot 1 \\ 1 \cdot 1 - 1 \cdot 1 + 2 \cdot 1 \end{pmatrix}
 = \begin{pmatrix}-3 \\ 2 \end{pmatrix}
 = \textcolor{#00CC00}{-3} \cdot \begin{pmatrix} 1 \\ 0 \end{pmatrix} + \textcolor{#00CC00}{2} \cdot \begin{pmatrix} 0 \\ 1 \end{pmatrix}
 \]
 
 Damit ist
 \[
 {\,}_S f_B = \begin{pmatrix} \textcolor{#CC6600}{-2} & \textcolor{#0066CC}{-3} & \textcolor{#00CC00}{-3}\\ \textcolor{#CC6600}{1} & \textcolor{#0066CC}{-1} & \textcolor{#00CC00}{2}\end{pmatrix}.
 \]}

 \lang{en}{
 Regarding the standard basis of $\R^2=\left\{ \begin{pmatrix}1 \\ 0 \end{pmatrix}; \begin{pmatrix}0 \\ 1 \end{pmatrix} \right\}$ 
 we can easily display our calculations:
 \[
  f(\begin{pmatrix}1 \\ 0 \\ 0\end{pmatrix})=\begin{pmatrix}-2\cdot 1 -3 \cdot 0 +2 \cdot 0 \\ 1 \cdot 1 - 1 \cdot 0 +2 \cdot 0\end{pmatrix} 
  = \begin{pmatrix} -2 \\ 1\end{pmatrix}
  = \textcolor{#CC6600}{-2} \cdot \begin{pmatrix} 1 \\ 0 \end{pmatrix} + \textcolor{#CC6600}{1} \cdot \begin{pmatrix} 0 \\ 1 \end{pmatrix}
 \]
 \[
 f(\begin{pmatrix} 0 \\ 1 \\ 0 \end{pmatrix}) = \begin{pmatrix}-2 \cdot 0 - 3 \cdot 1 + 2 \cdot 0 \\ 1 \cdot 0 - 1 \cdot 1 + 2 \cdot 0 \end{pmatrix}
 = \begin{pmatrix}-3 \\ -1 \end{pmatrix}
 = \textcolor{#0066CC}{-3} \cdot \begin{pmatrix} 1 \\ 0 \end{pmatrix} + (\textcolor{#0066CC}{-1}) \cdot \begin{pmatrix} 0 \\ 1 \end{pmatrix}

 \]
 \[
 f(\begin{pmatrix} 1 \\ 1 \\ 1 \end{pmatrix}) = \begin{pmatrix}-2 \cdot 1 - 3 \cdot 1 + 2 \cdot 1 \\ 1 \cdot 1 - 1 \cdot 1 + 2 \cdot 1 \end{pmatrix}
 = \begin{pmatrix}-3 \\ 2 \end{pmatrix}
 = \textcolor{#00CC00}{-3} \cdot \begin{pmatrix} 1 \\ 0 \end{pmatrix} + \textcolor{#00CC00}{2} \cdot \begin{pmatrix} 0 \\ 1 \end{pmatrix}
 \]
 
 With that it is
 
 \[
 {\,}_S f_B = \begin{pmatrix} \textcolor{#CC6600}{-2} & \textcolor{#0066CC}{-3} & \textcolor{#00CC00}{-3}\\ \textcolor{#CC6600}{1} & \textcolor{#0066CC}{-1} & \textcolor{#00CC00}{2}\end{pmatrix}.
 \]}

\step %2

\lang{de}{Nach vorstehender Regel \ref{Basiswechsel} und gilt nun für die lineare Abbildung $f$ mit den Basen 

\\
$B= \left\{ \begin{pmatrix}
 1 \\ 0 \\ 0 \end{pmatrix}; \begin{pmatrix} 0 \\ 1 \\ 0 \end{pmatrix}; \begin{pmatrix} 1 \\ 1  \\ 1\end{pmatrix}\right\}$ von $\R^3$
 und $C=\left\{ \begin{pmatrix} 1\\ 0 \end{pmatrix}; \begin{pmatrix} 1\\ 1\end{pmatrix}\right\}$ von $\R^2$: 

\[{\,}_C{f}_B = {\,}_C (id \circ f)_B = {\,}_C{id}_S \cdot _S{f}_B= {\,}_CT_S \cdot _S{f}_B \]

\[
   = \begin{pmatrix} 1 & -1 \\ 0 & 1 \end{pmatrix} \cdot \begin{pmatrix} -2 & -3 & -3 \\ 1 & -1 & 2 \end{pmatrix}
   = \begin{pmatrix} -3 & -2 & -5 \\ 1 & -1 & 2 \end{pmatrix}.
 \]
 Wir erhalten also das gleiche Ergebnis wie in Beispiel \ref{bsp:lin_abb}.}

 \lang{en}{With rule \ref{Basiswechsel} now holds for the linear map $f$ with the bases 

\\
$B= \left\{ \begin{pmatrix}
 1 \\ 0 \\ 0 \end{pmatrix}; \begin{pmatrix} 0 \\ 1 \\ 0 \end{pmatrix}; \begin{pmatrix} 1 \\ 1  \\ 1\end{pmatrix}\right\}$ von $\R^3$
 and $C=\left\{ \begin{pmatrix} 1\\ 0 \end{pmatrix}; \begin{pmatrix} 1\\ 1\end{pmatrix}\right\}$ von $\R^2$: 

\[{\,}_C{f}_B = {\,}_C (id \circ f)_B = {\,}_C{id}_S \cdot _S{f}_B= {\,}_CT_S \cdot _S{f}_B \]

\[
   = \begin{pmatrix} 1 & -1 \\ 0 & 1 \end{pmatrix} \cdot \begin{pmatrix} -2 & -3 & -3 \\ 1 & -1 & 2 \end{pmatrix}
   = \begin{pmatrix} -3 & -2 & -5 \\ 1 & -1 & 2 \end{pmatrix}.
 \]
 So we receive the same result as in example \ref{bsp:lin_abb}.}

\end{incremental}

\end{example}
%%%%%%%%%%%%%%%

\lang{de}{Desweiteren ergeben sich folgende nützliche Konsequenzen,
%%% Video K.M. -  11286 = Schnitt aus Videos 10831
%
die auch das Beispiel in folgendem Video zeigt:


\floatright{\href{https://api.stream24.net/vod/getVideo.php?id=10962-2-11286&mode=iframe&speed=true}
{\image[75]{00_video_button_schwarz-blau}}}}\\
\\


\begin{rule}[\lang{de}{Regeln zum Basiswechsel} \lang{en}{Rules for the basis transformation}] \label{regel_bw}
\lang{de}{Gegeben seien ein Vektorraum $V$ mit zwei Basen $B$ und $C$, 
ein Vektorraum $W$ mit Basen $D$ und $G$,
ein Vektor $v \in V$ und eine lineare Abbildung $\phi: V \to W$. Dann gilt}
\lang{en}{Given are a vector space $V$ with two bases $B$ and $C$,
a vector space $W$ with basis $D$ and $G$, a vector $v\in V$ and a linear map $\phi: V \to W$.
Then holds}
\begin{enumerate}
  \item ${\,}_C v = {\,}_{C} {T}_B \cdot {\,}_B v$
  \item ${\,}_C {T}_B = \left({\,}_B {T}_{C}\right)^{-1}$
  \item ${\,}_{G}\phi_C = {\,}_{G}{T}_D \cdot  {\,}_D\phi_B \cdot {\,}_B{T}_C$
\end{enumerate}

\end{rule}

\lang{de}{An den beiden Regeln lässt sich nun der Vorteil der hier verwendeten Notation erkennen:}
\lang{de}{Those rules point out the advantage of the used notation:}

\lang{de}{
Um zwei Abbildungsmatrizen sinnvoll miteinander multiplizieren zu können oder einen Koordinatenwechsel richtig auszuführen,
müssen die ''aufeinander treffenden'' Basen übereinstimmen, d.h.
\[
{\,}_D g_{\textcolor{#0066CC}{C}} \cdot {\,}_{\textcolor{#0066CC}{C}} f_B \text{ und } {\,}_{\textcolor{#00CC00}{C}} v = {\,}_{\textcolor{#00CC00}{C}}{T}_{\textcolor{#0066CC}{B}} \cdot {\,}_{\textcolor{#0066CC}{B}} v.
\]
Eine Multiplikation
\[
{\,}_{G} g_{\textcolor{#CC6600}{D}} \cdot {\,}_{\textcolor{#CC6600}{C}} f_B
\]
ergibt \textcolor{#CC6600}{keinen} Sinn.}
\lang{en}{
To multiply two transformation matrices or to perform the coordinate transformation right,
the ''meeting''' bases must match, i.e.
\[
{\,}_D g_{\textcolor{#0066CC}{C}} \cdot {\,}_{\textcolor{#0066CC}{C}} f_B \text{ und } {\,}_{\textcolor{#00CC00}{C}} v = {\,}_{\textcolor{#00CC00}{C}}{T}_{\textcolor{#0066CC}{B}} \cdot {\,}_{\textcolor{#0066CC}{B}} v.
\]
A multiplication
\[
{\,}_{G} g_{\textcolor{#CC6600}{D}} \cdot {\,}_{\textcolor{#CC6600}{C}} f_B
\]
does \textcolor{#CC6600}{not} make sense.}

\lang{de}{
Diese Überlegung zeigt auch, warum es hier sinnvoll ist, die Eingangsbasis auf der rechten Seite zu notieren.
Als weitere Eselsbrücke kann man nennen, dass man standardmäßig $f(x)$ schreibt. Das Funktionsargument $x$ steht rechts der Abbildung.}
\lang{en}{
This consideration shows, why it is reasonable, to note the starting basis on the right side.
Here is a little trick to remember the right order: We usually note $f(x)$. The argument $x$ is on the right side of the map.}

\lang{de}{In den Beispielen \ref{bsp:lin_abb} \, und \ref{bsp:basiswechsel} \, konnte man schon erkennen, dass es deutlich einfacher ist,
die Abbildungsmatrix $A$ bezüglich der Standardbasis zu bestimmen. Da man in diesem
Fall den Vektor $f(v_1)$ direkt in die erste Spalte von $A$ hineinschreiben kann, den Vektor $f(v_2)$ 
in die zweite Spalte, usw.,
falls $\{v_1;v_2;...;v_n\}$ eine Basis der Vektorraums des Definitionsbereiches ist.}
\lang{en}{We already saw in the examples \ref{bsp:lin_abb} \, and \ref{bsp:basiswechsel}, that it is easier
to determine the transformation matrix $A$ regarding to the standard basis.
In this case we can note the vector $f(v_1)$ in the first column of $A$, the vector $f(v_2)$ 
in the second column, and so on, if $\{v_1;v_2;...;v_n\}$ is a basis of the vector space
of the domain of definition.}

\lang{de}{
Ein Basiswechsel von einer beliebigen Basis in die Standardbasis ist genauso einfach zu bestimmen.
Man schreibt die Basisvektoren der beliebig gewählten Basis einfach in der richtigen Reihenfolge in die 
Basistransformationsmatrix.
Mit Hilfe der Regeln können wir nun die Abbildungsmatrizen für beliebig komplizierte Basen
berechnen. Der rechenintensive Schritt ist nun das Invertieren der Transformationsmatrix.}
\lang{en}{
A basis tranformation vom any basis to the standard basis is also
easy to determine. We note the basis vectors of the random basis in the right order into the
basis transformation matrix. With help of the rules we may calculate the transformation matrix for
any complicated bases. The inverting of the transformation matrix comes with a lot of calculations.}


\begin{example}

 \lang{de}{Wir betrachten nun noch einmal die lineare Abbildung aus Beispiel \ref{bsp:lin_abb}:
 \[
 f: \R^3 \to \R^2, \begin{pmatrix}x \\ y\\ z \end{pmatrix} \mapsto \begin{pmatrix}-2x-3y+2z \\x-y+2z\end{pmatrix}.
 \]
 und bestimmen, wie in Beispiel \ref{bsp:basiswechsel}, die Abbildungsmatrix $A={\,}_Cf_B$ mit den Basen 
 
 \\
 $B= \left\{ \begin{pmatrix}
 1 \\ 0 \\ 0 \end{pmatrix}; \begin{pmatrix} 0 \\ 1 \\ 0 \end{pmatrix}; \begin{pmatrix} 1 \\ 1  \\ 1\end{pmatrix}\right\}$ von $\R^3$
 und $C=\left\{ \begin{pmatrix} 1\\ 0 \end{pmatrix}; \begin{pmatrix} 1\\ 1\end{pmatrix}\right\}$.}
 \lang{en}{We consider again the linear map from example \ref{bsp:lin_abb}:
 \[
 f: \R^3 \to \R^2, \begin{pmatrix}x \\ y\\ z \end{pmatrix} \mapsto \begin{pmatrix}-2x-3y+2z \\x-y+2z\end{pmatrix}.
 \]
 and determine, like in example \ref{bsp:basiswechsel}, the tranformation matrix $A={\,}_Cf_B$ with the bases 
 
 \\
 $B= \left\{ \begin{pmatrix}
 1 \\ 0 \\ 0 \end{pmatrix}; \begin{pmatrix} 0 \\ 1 \\ 0 \end{pmatrix}; \begin{pmatrix} 1 \\ 1  \\ 1\end{pmatrix}\right\}$ von $\R^3$
 and $C=\left\{ \begin{pmatrix} 1\\ 0 \end{pmatrix}; \begin{pmatrix} 1\\ 1\end{pmatrix}\right\}$.}

\\ 
\begin{incremental}
 \step %1
 
 \lang{de}{Zur weiteren Vereinfachung wenden wir nun die letze Regel \ref{regel_bw} noch an:
 \[
 {\,}_C f_B = {\,}_C {T}_S \cdot {\,}_S f_B = ({\,}_S {T}_C)^{-1} \cdot {\,}_S f_B
 \] 
 Der Vorteil ist, dass sich auch hier bezüglich der Standardbasis die Basistransformation von $C$ sofort hinschreiben lässt:
 \[
 {\,}_S {T}_C = \begin{pmatrix} 1& 1\\ 0 & 1\end{pmatrix}.
 \]
 In der ersten Spalte steht der Koordinatenvektor des ersten Basiselements von $C$ (bezüglich der Standardbasis) 
 und in der zweiten Spalte der Vektor des zweiten Basiselements von $C$.}
 \lang{en}{For further simplification we will apply the last rule \ref{regel_bw}:
 \[
 {\,}_C f_B = {\,}_C {T}_S \cdot {\,}_S f_B = ({\,}_S {T}_C)^{-1} \cdot {\,}_S f_B
 \] 
 The benefit is, that we can note the basis transformation regarding the standard basis $C$ right away:
 \[
 {\,}_S {T}_C = \begin{pmatrix} 1& 1\\ 0 & 1\end{pmatrix}.
 \]
 In the first column we have the coordinate vector of the first basis element of $C$ (standard basis) and in the second
 column the vector of the second element of $C$.}
 
 \step %2
 
 \lang{de}{Das Berechnen der Abbildungsmatrix läuft nun auf das Invertieren dieser Matrix hinaus.
 Die Inverse bestimmen wir mit der
 \ref[inverse_matrix][Formel für $2 \times 2$ Matrizen]{rule:inverse-2x2}, die aus der Cramerschen Regel folgt:
 \[
 ({\,}_S {T}_C)^{-1} = \frac{1}{1\cdot 1 - 1 \cdot 0} \cdot \begin{pmatrix}1 & -1 \\ 0 & 1\end{pmatrix}
 \]}
 \lang{en}{To calculate the transformation matrix we need to invert {\,}_S {T}_C)^{-1}.
 We determine the inverse with the
 \ref[inverse_matrix][formula for $2 \times 2$ matrix]{rule:inverse-2x2}, which results from Cramer's rule:
 \[
 ({\,}_S {T}_C)^{-1} = \frac{1}{1\cdot 1 - 1 \cdot 0} \cdot \begin{pmatrix}1 & -1 \\ 0 & 1\end{pmatrix}
 \]}

 \step %3
 
 \lang{de}{Schließlich erhalten wir mit
 \[
 {\,}_C f_B = \begin{pmatrix}1 & -1 \\ 0 & 1\end{pmatrix} \cdot \begin{pmatrix} -2 & -3 & -3\\ 1 & -1 & 2\end{pmatrix}
 = \begin{pmatrix} -2 -1 & -3+1 & -3-2 \\ 1 & -1 & 2 \end{pmatrix} 
 = \begin{pmatrix} -3 & -2 & -5 \\ 1 & -1 & 2\end{pmatrix},
 \]
 das gleiche Ergebnis wie in \ref{bsp:lin_abb} \, bzw. \ref{bsp:basiswechsel} \,.}

 \lang{en}{Finally we end up with
 \[
 {\,}_C f_B = \begin{pmatrix}1 & -1 \\ 0 & 1\end{pmatrix} \cdot \begin{pmatrix} -2 & -3 & -3\\ 1 & -1 & 2\end{pmatrix}
 = \begin{pmatrix} -2 -1 & -3+1 & -3-2 \\ 1 & -1 & 2 \end{pmatrix} 
 = \begin{pmatrix} -3 & -2 & -5 \\ 1 & -1 & 2\end{pmatrix},
 \]
 the same result as in \ref{bsp:lin_abb} \, rather \ref{bsp:basiswechsel} \,.}
 
 \end{incremental}
 
\end{example}


\end{visualizationwrapper}



\end{content}