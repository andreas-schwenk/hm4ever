\documentclass{mumie.article}
%$Id$
\begin{metainfo}
  \name{
    \lang{de}{Allgemeine Vektorräume}
    \lang{en}{General vector spaces}
  }
  \begin{description} 
 This work is licensed under the Creative Commons License Attribution 4.0 International (CC-BY 4.0)   
 https://creativecommons.org/licenses/by/4.0/legalcode 

    \lang{de}{}
    \lang{en}{}
  \end{description}
  \begin{components}
    \component{generic_image}{content/rwth/HM1/images/g_tkz_T403a_Subspace.meta.xml}{T403a_Subspace}
    \component{generic_image}{content/rwth/HM1/images/g_img_00_video_button_schwarz-blau.meta.xml}{00_video_button_schwarz-blau}      
    \component{js_lib}{system/media/mathlets/GWTGenericVisualization.meta.xml}{mathlet1}
    \component{generic_image}{content/rwth/HM1/images/g_img_T403a_untervektorraum.meta.xml}{untervektorraum}    
  \end{components}
  \begin{links}
    \link{generic_article}{content/rwth/HM1/T108_Vektorrechnung/g_art_content_27_vektoren.meta.xml}{Vektoren}    
    \link{generic_article}{content/rwth/HM1/T108_Vektorrechnung/g_art_content_29_linearkombination.meta.xml}{Vektorrechnung}  
    \link{generic_article}{content/rwth/HM1/T108_Vektorrechnung/g_art_content_30_basen_eigenschaften.meta.xml}{basen}
    \link{generic_article}{content/rwth/HM1/T401_Matrizenrechnung/g_art_content_02_matrizenmultiplikation.meta.xml}{matrix-mult}
    \link{generic_article}{content/rwth/HM1/T402_Lineare_Gleichungssysteme/g_art_content_04_lgs.meta.xml}{lgs}
    \link{generic_article}{content/rwth/HM1/T202_Reelle_Zahlen_axiomatisch/g_art_content_04_koerperaxiome.meta.xml}{koerperaxiome}
  \end{links}
  \creategeneric
\end{metainfo}
\begin{content}

\usepackage{mumie.ombplus}
\ombchapter{4}
\ombarticle{1}
\usepackage{mumie.genericvisualization}

\begin{visualizationwrapper}


\lang{de}{\title{Allgemeine Vektorräume}}
\lang{en}{\title{General vector spaces}}
 
 
\begin{block}[annotation]
  Im Ticket-System: \href{http://team.mumie.net/issues/13439}{Ticket 13439}\\
\end{block}


\begin{block}[info-box]
\tableofcontents
\end{block}

\section{\lang{de}{Allgemeiner Begriff des Vektorraums} \lang{en}{General concept of the vector space}}

\lang{de}{
Vektorräume sind uns schon mehrmals in diesem Kurs begegnet. Bevor wir auf die bereits bekannten 
Beispiele für Vektorräume nochmals hinweisen, definieren wir zunächst den allgemeinen Begriff des 
Vektorraums.}
\lang{en}{
We already came across vector spaces several times. Before we refer to the already known 
examples for vector spaces, we will define the general concept of the vector space.}


%%% Video K.M.
% 
\lang{de}{
Die hierzu relevanten Grundlagen zu \ref[Vektoren][Vektoren im $\R^n$]{sec:1_1_Vektoren} und den
\ref[Vektorrechnung][Rechenregeln]{Vektorraum_R_n} für diese Vektoren sind in folgendem Video
noch einmal zusammengefasst:

\floatright{\href{https://api.stream24.net/vod/getVideo.php?id=10962-2-10929&mode=iframe&speed=true}
{\image[75]{00_video_button_schwarz-blau}}}}\\
\\
%

\begin{definition}[\lang{de}{Vektorraum} \lang{en}{Vector space}] \label{def:Allg_VR}
   \lang{de}{
    Es sei $\K$ ein \ref[koerperaxiome][Körper]{sec:axiome}. Eine Menge $V$ mit zwei Verkn\"upfungen}
      \lang{en}{
    Let $\K$ be a \ref[koerperaxiome][field]{sec:axiome}. A set $V$ with two operations}
   % \\
    \[
    \begin{mtable}[\align{l}] 
    \begin{mtable}
      +\, :\, &\, \vectorspace{V} \, \times \, \vectorspace{V} &\to &\vectorspace{V}\\
      &({u},{v}) &\mapsto &{u} + {v}\quad               
    \end{mtable}&\qquad \text{\notion{(\lang{en}{Addition}\lang{de}{Addition})}}\\
    \begin{mtable}
    &&&\\
         \cdot\, :\,  &\, \K \, \times \, \vectorspace{V} &\to     &\vectorspace{V}  \\
           &(\alpha,{v})          &\mapsto &\alpha \cdot {v}\quad              
    \end{mtable}& \qquad \text{\notion{(\lang{en}{Multiplication with a scalar}\lang{de}{Multiplikation mit einem Skalar})}}
    \end{mtable}
    \]

\lang{de}{ 
	 hei"st \emph{$\K$-Vektorraum } und die Elemente aus ${V}$ hei"sen
    \emph{Vektoren}, 
    falls die folgenden Rechengesetze erf\"ullt sind.}
   \lang{en}{
    is called a 
    \emph{(real) vector space}, and the elements of ${V}$ are called \emph{vectors}, if the following calculating rules apply.}
    \lang{de}{F"ur die Addition gilt:}
    \lang{en}{For addition holds:}
    \begin{enumerate}
    \item \lang{de}{Das Kommutativgesetz: ${u} + {v} = {v} + {u}$ f\"ur alle ${u},{v} \in \vectorspace{V}$,}
          \lang{en}{The commutative property: ${u} + {v} = {v} + {u}$ for all ${u},{v} \in \vectorspace{V}$,}
    \item \lang{de}{Das Assoziativgesetz: \nowrap{$ {u} + ({v} + {w}) = ({u} + {v}) + {w}$ f\"ur alle
      ${u},{v},{w} \in \vectorspace{V} $.}}
      \lang{en}{The associative property: \nowrap{$ {u} + ({v} + {w}) = ({u} + {v}) + {w}$ for all
      ${u},{v},{w} \in \vectorspace{V} $.}}
    \item \lang{de}{Es existiert ein Nullvektor ${0} \in \vectorspace{V}$, sodass $ {v} + {0} = {v}$ 
    f\"ur alle ${v} \in \vectorspace{V} $.
    Diesen Nullvektor nennt man das \emph{neutrale Element}  bez\"uglich der Addition.}
    \lang{en}{It exists a zero vector ${0} \in \vectorspace{V}$, such that $ {v} + {0} = {v}$ 
    for all ${v} \in \vectorspace{V} $. \\ This zero vector is called the \emph{identity element} of addition.}
    \item \lang{de}{\nowrap{Zu jedem $ {v} \in \vectorspace{V}$ existiert ein sogenanntes \emph{inverses Element}  $-{v} \in \vectorspace{V}$, \\ 
    sodass    ${v} + (-{v}) = {0}$.}}
    \lang{en}{\nowrap{For every $ {v} \in \vectorspace{V}$ exists a \emph{inverse element}  $-{v} \in \vectorspace{V}$, \\ 
    such that ${v} + (-{v}) = {0}$.}}
    \end{enumerate}
    \lang{de}{F"ur die Multiplikation mit Skalaren gilt:}
    \lang{en}{For multiplication with scalars holds:}
    \begin{enumerate}
    \item \lang{de}{\nowrap{$ \alpha \cdot  (\beta \cdot  {v}) = (\alpha \beta) \cdot  {v}$ f\"ur alle ${v} \in
      \vectorspace{V} $ und $ \alpha, \beta \in \K $}}
      \lang{en}{\nowrap{$ \alpha \cdot  (\beta \cdot  {v}) = (\alpha \beta) \cdot  {v}$ for all ${v} \in
      \vectorspace{V} $ and $ \alpha, \beta \in \K $}}
    \item \lang{de}{\nowrap{$ 1 \cdot  {v} = {v}$ f\"ur alle ${v} \in \vectorspace{V} $}}
    \lang{en}{\nowrap{$ 1 \cdot  {v} = {v}$ for all ${v} \in \vectorspace{V} $}}
    \end{enumerate}
    \lang{de}{Zwischen der Addition und der Multiplikation mit Skalaren gelten die Distributivgesetze:}
    \lang{en}{Between addition and multiplication with scalars hold the distributive properties:}
    \begin{enumerate}
    \item \lang{de}{\nowrap{$ \alpha \cdot  ({u} + {v}) = \alpha \cdot  {u} + \alpha \cdot  {v}$ f\"ur alle
      ${u}, {v} \in \vectorspace{V}$ und $\alpha \in \K$}}
      \lang{en}{\nowrap{$ \alpha \cdot  ({u} + {v}) = \alpha \cdot  {u} + \alpha \cdot  {v}$ for all
      ${u}, {v} \in \vectorspace{V}$ und $\alpha \in \K$}}
    \item \lang{de}{\nowrap{$ (\alpha + \beta) \cdot  {v} = \alpha \cdot  {v} + \beta \cdot  {v}$ f\"ur alle
      ${v} \in \vectorspace{V}$ und $\alpha , \beta \in \K $}}
      \lang{en}{\nowrap{$ (\alpha + \beta) \cdot  {v} = \alpha \cdot  {v} + \beta \cdot  {v}$ f\"ur alle
      ${v} \in \vectorspace{V}$ und $\alpha , \beta \in \K $}}
    \end{enumerate}
\end{definition}

\begin{remark}
   \begin{itemize}
   \item \lang{de}{Statt Vektorraum wird manchmal auch der Begriff \emph{linearer Raum} verwendet.}
   \lang{en}{Sometimes we say \emph{linear space} instead of vector space.}
   \item \lang{de}{Aus den obigen Eigenschaften folgt, dass $\vectorspace{V}\ne\emptyset$, $\vectorspace{V}$
         also nicht leer ist, da mindestens der Nullvektor ${0}$ enthalten sein muss. Eine Erläuterung 
         hierzu finden Sie in Beispiel 2) des nachstehenden Videos.}
         \lang{en}{ It concludes from the above properties, that $\vectorspace{V}\ne\emptyset$, $\vectorspace{V}$ is not empty,
         because at least the zero vector ${0}$ is contained.}
     \end{itemize}
\end{remark} 

\lang{de}{Es folgen nun einige Beispiele von bekannten Mengen, die einen Vektorraum bilden.

%%% Video K.M.
% 
\floatright{\href{https://api.stream24.net/vod/getVideo.php?id=10962-2-10930&mode=iframe&speed=true}
{\image[75]{00_video_button_schwarz-blau}}}}\\
\\

\begin{example} \label{ex:vektorräume}
\begin{enumerate}
   \item \lang{de}{Wie bereits bei der \ref[Vektorrechnung][Vektorrechnung im $\R^n$]{Vektorraum_R_n} \,
    gesehen, gelten die oben angegebenen Rechenregeln insbesondere auch für die Spaltenvektoren des $\R^n$ (für jedes $n> 0$). \\ 
    Die Vektorräume der Spaltenvektoren im $\R^n $ (über $\K = \R $) sind also Spezialfälle des hier allgemein definierten 
    Begriffs des Vektorraums.}
    \lang{en}{As we have already seen for \ref[Vektorrechnung][vector calculations in $\R^n$]{Vektorraum_R_n} \,, 
    the above calculating rules hold especially for the column vectors of $\R^n$ (for every $n>0$.)\\
    The vector spaces of the column vectors in $\R^n$ (over $\K =\R$) are special cases of the general defined concept
    of vector space.}

   \item \lang{de}{Spaltenvektoren können auch auf gleiche Weise über jedem anderen Körper definiert werden (also z.B. über $\K=\Q$ oder $\K=\C$).
   Dann bildet die Menge aller Spaltenvektoren
   \[
    \K^n := \{ \begin{pmatrix}
                x_1 \\ \vdots \\ x_n
               \end{pmatrix} | \, x_1,...,x_n \in \K \}

   \]
   einen $\K$-Vektorraum.}
   \lang{en}{Column vectors can be defined analogously over any field (e.g. over $\K=\Q$ or $\K=\C$).
   Then the set of all column vectors
   \[
    \K^n := \{ \begin{pmatrix}
                x_1 \\ \vdots \\ x_n
               \end{pmatrix} | \, x_1,...,x_n \in \K \}

   \]
   is a $\K$-vector space.}
   \item \lang{de}{Die Menge der Polynomfunktionen (über den rationalen, reellen oder komplexen Zahlen) mit der üblichen Addition und Skalar-Multiplikation
    mit rationalen, reellen bzw. komplexen Zahlen 
   \begin{align*}
    (p_1 + p_2)(x) &\coloneq& p_1(x) + p_2(x) \\
    (\alpha\, p)(x) &\coloneq& \alpha\, p(x)
  \end{align*}
   bildet einen Vektorraum (s. Beispiel 3) in obigem Video). Der Nullvektor ist dabei das Nullpolynom $p_0$ mit $p_0(x)=0$
   für alle $x\in \K$ (mit $\K=\Q,\R,\C$).}
   \lang{en}{The set of polynomial functions (over the rational, real or complex numbers) with the known addition and scalar multiplication
    with rational, real or complex numbers 
   \begin{align*}
    (p_1 + p_2)(x) &\coloneq& p_1(x) + p_2(x) \\
    (\alpha\, p)(x) &\coloneq& \alpha\, p(x)
  \end{align*}
  is a vector space. Die zero vector is the zero-polynomial $p_0$ with $p_0(x)=0$
   for all $x\in \K$ (with $\K=\Q,\R,\C$).}
   
   \item \lang{de}{Auch die Menge aller reellen Funktionen $f:\R\to \R$ bildet einen $\R$-Vektorraum. 
   Die Addition und Skalar-Multiplikation ist die Gleiche wie im Fall der Polynomfunktionen und auch 
   hier ist der Nullvektor die Nullfunktion $f_0$, für die $f_0(x)=0$ für alle $x\in \R$ gilt.}
   \lang{de}{The set of the real functions $f:\R\to \R$ is also a $\R$-vector space. Addition and scalar-multiplication
   are defined like we defined it for the polynomial functions. The zero-vector is again the zero-function $f_0$ with $f_0(x)=0$
   for all $x\in\R$.}
   
   \item \lang{de}{Genauso bildet die Menge aller stetigen bzw. differenzierbarer Funktionen $f:I \to \R$ (für ein beliebiges, aber festes, nicht-leeres Intervall $I$) 
   einen reellen Vektorraum (s. vorstehendes Video). Die Nullfunktion übernimmt wieder die Rolle des Nullvektors.}
   \lang{en}{The set of continuous (differentiable) functions $f:I \to \R$ (for an arbitrary but fixed non-empty interval $I$)
   is a real vector space. The zero-function is again the zero-vector.}
   
   \item \lang{de}{Die Menge aller $(m \times n)$ Matrizen über einem Körper $\K$ bildet ebenfalls einen Vektorraum. 
   Sie erfüllen, wie ein Vergleich mit den \ref[matrix-mult][Rechenregeln für die Matrizen-Multiplikation]{rule:rechenregeln} aus den vorherigen Abschnitten zeigt, 
   alle Eigenschaften eines Vektorraums. 
   Der Nullvektor ist hier die Nullmatrix (also die Matrix, deren Einträge alle gleich $0$ sind). 
   Die Addition auf diesem Vektorraum ist die bekannte Matrixaddition.}
   \lang{en}{The set of all $(m \times n)$ matrices over a field $\K$ is also a vector space. 
   They fulfill all the characteristics of a vector space. Therefor compare with the \ref[matrix-mult][calculating rules for matrix-multiplication]{rule:rechenregeln} from the before section. 
   The zero-vector is the zero-matrix (all entries equal to $0$). The addition in this vector space is the usual matrix addition.}
 \end{enumerate}
\end{example}   

%%% Video K.M.
%
\lang{de}{
In folgendem Video werden weitere spezielle Beispiele für Vektorräume betrachtet, nämlich solche,
die Teilmenge eines Vektorraums sind:

\floatright{\href{https://api.stream24.net/vod/getVideo.php?id=10962-2-10931&mode=iframe&speed=true}
{\image[75]{00_video_button_schwarz-blau}}}}\\
\\
 
\begin{definition}[\lang{de}{Untervektorraum} \lang{en}{Vector subspace}] \label{def:Unter_VR}

  \lang{de}{
  Sei $V$ ein $\K$-Vektorraum entsprechend der Definition \ref{def:Allg_VR}
  und sei $U$ eine \emph{nichtleere} Teilmenge von $V$, die bezüglich der für $V$ definierten 
  Verknüfpungen \emph{abgeschlossen} ist, d.h.}
  \lang{en}{
  Let $V$ be a $\K$-vector space, according the definiton \ref{def:Allg_VR}, and $U$ a 
   \emph{non-empty} subset of $V$, which is \emph{closed} with respect to the operations defined for $V$, namely}
  
    \begin{itemize}
    \item \lang{de}{\nowrap{f\"ur alle Vektoren ${u}, {v} \in \vectorspace{U}\;$ ist auch die Summe $\; {u} + {v} \in \vectorspace{U}$}  und}
     \lang{en}{\nowrap{for all vectors ${u}, {v} \in \vectorspace{U}\;$ is also the sum $\; {u} + {v} \in \vectorspace{U}$}  and}
    \item \lang{de}{\nowrap{f\"ur alle ${v} \in \vectorspace{U}$ und beliebiges $\, \alpha \in \K \;$ ist auch $\; \alpha \cdot  {v} \in \vectorspace{U}.$ }}
    \lang{en}{\nowrap{for all ${v} \in \vectorspace{U}$ and any $\, \alpha \in \K \;$ is also $\; \alpha \cdot  {v} \in \vectorspace{U}.$ }}
    \end{itemize}

    \lang{de}{Dann bezeichnet man $\vectorspace{U}$ als \emph{Untervektorraum} oder auch \emph{linearer Unterraum} von $\vectorspace{V}$.}
    \lang{en}{Then we call $\vectorspace{U}$ \emph{vector subspace} or \emph{linear subspace} of $\vectorspace{V}$.}

\end{definition}

\begin{remark}
   \lang{de}{Sei $V$ ein $\K$-Vektorraum entsprechend der Definition \ref{def:Allg_VR}}
   \lang{en}{Let $V$ be a $\K$-vector space according to the definition \ref{def:Allg_VR}}
   
 \begin{itemize}
   \item \lang{de}{Ein \emph{Untervektorraum} $\vectorspace{U}$ von $\vectorspace{V}$ ist mit den für $\vectorspace{V}$ definierten
   Verknüpfungen selbst wieder ein \emph{$\K$-Vektorraum. }}
   \lang{en}{A \emph{vector subspace} $\vectorspace{U}$ of $\vectorspace{V}$ is with the operations defined for $\vectorspace{V}$ 
   a  \emph{$\K$-vector space} itself.}
   \item \lang{de}{Folgende Teilmengen von $\vectorspace{V}$ erfüllen stets die Eigenschaften eines \emph{Untervektorraums}:
    \begin{itemize}
      \item die Menge $V$ selbst  und
      \item die Menge $\{0\}$, die nur aus dem \emph{Nullvektor} besteht
    \end{itemize}}
     \lang{en}{The following subsets of $\vectorspace{V}$ always fulfill the characteristics of a \emph{vector subspace}:
    \begin{itemize}
      \item the set $V$ itself  and
      \item the set $\{0\}$, that consists of ony the \emph{zero-vector}
    \end{itemize}}
 \end{itemize}
\end{remark} 

\begin{example}
 \begin{enumerate}   
     
   \item \lang{de}{Betrachten wir den Vektorraum $\R^3$. Dann bildet die, durch den Koordinatenursprung $O=(0;0;0)$ 
    und die weiteren zwei Punkte $\;P=(1;1;1)\,$ und $\,R=(1;0;0)\,$ bestimmte Ebene
    \[
    E: \begin{pmatrix}x\\y\\z\end{pmatrix}=\lambda \cdot \vec{OP} + \mu \cdot \vec{OR}
    =\lambda \cdot\begin{pmatrix}1\\1\\1\end{pmatrix}+ \mu \cdot \begin{pmatrix}1\\0\\0\end{pmatrix}
    \] 
    
        \begin{center}
         \image{T403a_Subspace}
        \end{center}  
    
    einen Untervektorraum des $\R^3$ mit der Basis $\{\begin{pmatrix}1\\1\\1\end{pmatrix};\begin{pmatrix}1\\0\\0\end{pmatrix}\}$.}

    \lang{en}{We consider the vector space $\R^3$. Then the origin of ordinates $O=(0;0;0)$ and the two points
     $\;P=(1;1;1)\,$ und $\,R=(1;0;0)\,$ define the plane
    \[
    E: \begin{pmatrix}x\\y\\z\end{pmatrix}=\lambda \cdot \vec{OP} + \mu \cdot \vec{OR}
    =\lambda \cdot\begin{pmatrix}1\\1\\1\end{pmatrix}+ \mu \cdot \begin{pmatrix}1\\0\\0\end{pmatrix}
    \].
    
        \begin{center}
         \image{T403a_Subspace}
        \end{center}  
    
    $E$ is a vector subspace of $\R^3$ with the basis $\{\begin{pmatrix}1\\1\\1\end{pmatrix};\begin{pmatrix}1\\0\\0\end{pmatrix}\}$.}

\lang{de}{
\begin{showhide}[\buttonlabels{Zeige Nachweis für Untervektorraum}{Verstecke Nachweis für Untervektorraum}]
  Da $E \subseteq \R^3$ und $E \neq \emptyset$, bleibt nur die \emph{Abgeschlossenheit} von $E$ zu zeigen. \\
  Seien $u$ und $v$ beliebige Vektoren in $E$, dann gibt es $\lambda_u, \lambda_v, \mu_u, \mu_v \in \R$ mit
  \[ 
  \begin{mtable}
    u= \lambda_u \cdot \begin{pmatrix}1\\1\\1\end{pmatrix}+ \mu_u \cdot \begin{pmatrix}1\\0\\0\end{pmatrix}\\
    v= \lambda_v \cdot \begin{pmatrix}1\\1\\1\end{pmatrix}+ \mu_v \cdot \begin{pmatrix}1\\0\\0\end{pmatrix}
  \end{mtable}
  \] 
  Dann ist 
  \[
  u+v= (\lambda_u + \lambda_v) \cdot \begin{pmatrix}1\\1\\1\end{pmatrix}+ (\mu_u + \mu_v) \cdot \begin{pmatrix}1\\0\\0\end{pmatrix} \in E 
  \]
  und für beliebiges $\alpha \in \R$ auch
  \[
   \alpha \cdot u = (\alpha \cdot \lambda_u) \cdot \begin{pmatrix}1\\1\\1\end{pmatrix}+ (\alpha \cdot \mu_u) \cdot \begin{pmatrix}1\\0\\0\end{pmatrix} \in E 
  \]
  
  
\end{showhide}}



\lang{en}{
\begin{showhide}[\buttonlabels{Show verification of the vector subspace}{Hide verification of the vector subspace}]
Since $E \subseteq \R^3$ and $E \neq \emptyset$, we only need to verify the \emph{closure} of $E$. \\
  Let $u$ and $v$ be any vectors in $E$, it exist $\lambda_u, \lambda_v, \mu_u, \mu_v \in \R$ with
  \[ 
  \begin{mtable}
    u= \lambda_u \cdot \begin{pmatrix}1\\1\\1\end{pmatrix}+ \mu_u \cdot \begin{pmatrix}1\\0\\0\end{pmatrix}\\
    v= \lambda_v \cdot \begin{pmatrix}1\\1\\1\end{pmatrix}+ \mu_v \cdot \begin{pmatrix}1\\0\\0\end{pmatrix}
  \end{mtable}
  \] 
  Then it holds
  \[
  u+v= (\lambda_u + \lambda_v) \cdot \begin{pmatrix}1\\1\\1\end{pmatrix}+ (\mu_u + \mu_v) \cdot \begin{pmatrix}1\\0\\0\end{pmatrix} \in E 
  \]
  and for any $\alpha \in \R$ also
  \[
   \alpha \cdot u = (\alpha \cdot \lambda_u) \cdot \begin{pmatrix}1\\1\\1\end{pmatrix}+ (\alpha \cdot \mu_u) \cdot \begin{pmatrix}1\\0\\0\end{pmatrix} \in E 
  \]
  
  
\end{showhide}}


   \item \label{ex:LGS_UnterVR}
     \lang{de}{Wie im \ref[lgs][Kapitel über Lineare Gleichungssystem (in Teil 3b)]{rule-lgs_raum} gesehen, bildet die Lösungsmenge $\mathbb{L}$ eines homogenen LGS $Ax=0$ (mit $m$ Gleichungen und $n$ Variablen) einen Vektorraum.
     Dies gilt sogar, falls es nur die triviale Lösung $x=0$ gibt. Daher bildet auch $\mathbb{L}=\{ 0 \}$ ein Vektorraum.
     Im Fall, dass $A$ die Nullmatrix ist, ist $\mathbb{L}= \K^n$. In allen anderen Fällen ist der Vektorraum $\mathbb{L}$ eine Teilmenge des Vektorraums $\K^n$. 
     Wir sprechen deshalb auch von einem \emph{Untervektorraum}.}
     \lang{en}{We have already seen in the \ref[lgs][chapter about systems of linear equations (in Part 3b)]{rule-lgs_raum}, 
     that the solutions set $\mathbb{L}$ of a homogeneous linear system $Ax=0$ (with $m$ equations and $n$ variable) form a vector space.
     This holds even, if the only solution is the trivial solution $x=0$. Therefore also $\mathbb{L}=\{ 0 \}$ forms a vector space.
     In the case, that $A$ is the zero-matrix, the solution set is $\mathbb{L}= \K^n$. In every other case the vector space $\mathbb{L}$ is a subset of the vector space $\K^n$. 
     Therefore we can talk about a \emph{vector subspace}.}
 \end{enumerate}
\end{example}

\lang{de}{Im Grundlagen-Teil haben wir die Vektoren immer mit einem Vektorpfeil versehen, um den Unterschied zwischen Punkten und Vektoren zu verdeutlichen.
In diesem allgemeinen Kontext verzichten wir auf diese Notation und schreiben Vektoren ohne Vektorpfeile.}
\lang{en}{In the basic-part we always denoted vectore with an arrow to illustrate the difference between points and vectors.
In this general context we will omit this notation and write vectors without an arrow.}


\section{\lang{de}{Linearkombinationen} \lang{en}{Linear combinations}}

  \begin{definition}[\lang{de}{Linearkombinationen} \lang{en}{Linear combinations}]\label{def:linearkombination}
    \lang{de}{Es sei $\K$ ein Körper und $\vectorspace{V}$ ein $\K$-Vektorraum. Seien
    ${v}_1, \ldots , {v}_k$
    Vektoren des Vektorraums $\vectorspace{V}$ "uber $\K$.}
    \lang{de}{Ein Vektor der Form
    \begin{displaymath}
      \alpha_1\, {v}_1 + \ldots +\alpha_k\, {v}_k
    \end{displaymath}
    mit Koeffizienten
    $\alpha_1, \ldots , \alpha_k \in  \K$
    hei"st \notion{Linearkombination} von ${v}_1,  \ldots,{v}_k$.}

    \lang{en}{Let $\K$ be a field and $\vectorspace{V}$ a $\K$-vector space. Let
    ${v}_1, \ldots , {v}_k$
    be vectors in $\vectorspace{V}$ over $\K$.}
    \lang{en}{A vector of the form
    \begin{displaymath}
      \alpha_1\, {v}_1 + \ldots +\alpha_k\, {v}_k
    \end{displaymath}
    with coefficients
    $\alpha_1, \ldots , \alpha_k \in  \K$
    is called \notion{linear combination} of ${v}_1,  \ldots,{v}_k$.}
  \end{definition}

\lang{de}{Diese Definition ist eine Verallgemeinerung der entsprechenden \ref[Vektorrechnung][Definition im $\R^n$]{sec:lin-comb}.}
\lang{en}{This definitions is a generalisation of the corresponding \ref[Vektorrechnung][definition in $\R^n$]{sec:lin-comb}. }

\begin{remark}
 \begin{itemize}
   \item \lang{de}{Eine Linearkombination ist eine \emph{endliche} Summe von Vektoren.}
   \lang{en}{A linear combination is a \emph{finite} sum of vectors.}

%%% Video K.M.
%    
   \item \lang{de}{Eine Menge von Linearkombinationen aus Vektoren ${v}_1, \ldots , {v}_k$ eines Vektorraums $\vectorspace{V}$
         $\{ \alpha_1\, {v}_1 + \ldots +\alpha_k\, {v}_k \mid \alpha_1, \ldots , \alpha_k \in  \K \}$
       erzeugt einen Untervektorraum von $\vectorspace{V}$. \\
         Diesen bezeichnet man auch als \emph{linearen Spann}
         oder \emph{lineare Hülle} von ${v}_1, \ldots , {v}_k$.} 
         \lang{en}{A set of linear combinations of vectors ${v}_1, \ldots , {v}_k$ in $\vectorspace{V}$
         $\{ \alpha_1\, {v}_1 + \ldots +\alpha_k\, {v}_k \mid \alpha_1, \ldots , \alpha_k \in  \K \}$
           generates a vector subspace of $\vectorspace{V}$. \\
         This is also called \emph{linear span}
         or \emph{linear hull} of ${v}_1, \ldots , {v}_k$.} 

\lang{de}{\floatright{\href{https://api.stream24.net/vod/getVideo.php?id=10962-2-10932&mode=iframe&speed=true}
{\image[75]{00_video_button_schwarz-blau}}}}\\
\\
         
 \end{itemize}
\end{remark}

\section{\lang{de}{Erzeugendensysteme} \lang{en}{Generating set}}

\lang{de}{Wir verallgemeinern nun den Begriff des Erzeugendensystems. 
Dieser wurde für den Spezialfall $\R^n$ bereits im \link{basen}{Grundlagenteil zur Vektorrechnung} definiert.}
\lang{en}{Now we generalise the concept of the generating set. We already intruced it for the special case of $\R^n$ in the part about \link{basen}{basic vector calculations}.}

\begin{definition}[\lang{de}{Erzeugendensystem} \lang{en}{Generating system}]\label{def:erzeugendensystem}
\lang{de}{Eine (endliche) Menge von Vektoren $\{ {v}_1;\ldots;{v}_l \}$ in einem Vektorraum $\vectorspace{V}$ hei"st \emph{(endliches) Erzeugendensystem von $\vectorspace{V},$}
wenn jedes Element von $\vectorspace{V}$ sich als Linearkombination der Vektoren ${v}_1,\ldots,{v}_l$ schreiben lässt.}
\lang{en}{A (finite) set of vectors $\{ {v}_1;\ldots;{v}_l \}$ in a vector space $\vectorspace{V}$ is called \emph{(finite) generating set of $\vectorspace{V},$}
if every element of $\vectorspace{V}$ can be written as a linear combinations of the vectors ${v}_1,\ldots,{v}_l$.}
\end{definition}

%%% Video K.M. - Teil 1 von Video 10933, neu 11270
%  
\lang{de}{
Dabei muss die \emph{"`Erzeugung"'} eines Vektors ${v} \in \vectorspace{V}$ als Linearkombination
aus den Vektoren ${v}_1,\ldots,{v}_l$ nicht eindeutig sein, ebenso wenig besitzt ein $\K$-Vektorraum $\vectorspace{V}$
ein eindeutiges \emph{Erzeugendensystem}. Dies wird in folgendem Video verdeutlicht. \\
(Bemerkung: In den Videos in diesem Kapitel wird für Erzeugendensysteme die Notation 
$\langle {v}_1,\ldots,{v}_l\rangle_{\K}$ verwendet.)}
\lang{en}{The \emph{"`generation"'} of a vector ${v}\in\vectorspace{V}$ as a linear combination of the vectors ${v}_1,\ldots,{v}_l$
does not necessarily need to be unique. There is also no unique \emph{generating set} for a $\K$-vector space $\vectorspace{V}$.}

\lang{de}{\floatright{\href{https://api.stream24.net/vod/getVideo.php?id=10962-2-11270&mode=iframe&speed=true}
{\image[75]{00_video_button_schwarz-blau}}}}\\
\\
         

\begin{definition}[\lang{de}{Endlichkeit von Vektorräumen} \lang{en}{Finitude of vector spaces}]
 \lang{de}{Ein Vektorraum $\vectorspace{V}$, für den ein endliches Erzeugendensystem $\{ v_1; ...; v_l\}$ existiert, heißt \emph{endlich erzeugt} (oder auch \emph{endlich dimensional}).}
 \lang{en}{A vector space $\vectorspace{V}$, with a finite generating set $\{ v_1; ...; v_l\}$, is called \emph{finitely generated} (or \emph{finite dimensional}).}
\end{definition}

\lang{de}{Nicht jeder Vektorraum ist endlich erzeugt, wie die folgenden Beispiele zeigen.}
\lang{en}{Not every vector space is finitely generated, as the following examples will show.}

\begin{example}
%TODO
\begin{tabs*}
\tab{$\R^n$}
\lang{de}{
Die Standardvektoren im $\R^n$
\[ e_1=\begin{pmatrix} 1\\ 0\\ \vdots \\ 0\end{pmatrix},\quad e_2=\begin{pmatrix} 0\\ 1\\ \vdots \\ 0\end{pmatrix}
,\quad \ldots \quad e_n=\begin{pmatrix} 0\\ \vdots \\ 0\\ 1\end{pmatrix} \]
bilden ein Erzeugendensystem des $\R^n$, da jeder Vektor $\begin{pmatrix} \alpha_1\\ \vdots \\ \alpha_n\end{pmatrix}$ geschrieben 
werden kann als
\[ \begin{pmatrix} \alpha_1\\ \vdots \\ \alpha_n \end{pmatrix} = \alpha_1 \cdot e_1 + \alpha_2 \cdot e_2 +\ldots +\alpha_n\cdot e_n, \]
%  \[ \vec{e}_1=\begin{pmatrix} 1\\ 0\\ \vdots \\ 0\end{pmatrix},\quad \vec{e}_2=\begin{pmatrix} 0\\ 1\\ \vdots \\ 0\end{pmatrix}
%  ,\quad \ldots \quad \vec{e}_n=\begin{pmatrix} 0\\ \vdots \\ 0\\ 1\end{pmatrix} \]
%  bilden ein Erzeugendensystem des $\R^n$, da jeder Vektor $\begin{pmatrix} v_1\\ \vdots \\ v_n\end{pmatrix}$ geschrieben 
%  werden kann als
%  \[ \begin{pmatrix} v_1\\ \vdots \\ v_n\end{pmatrix} = v_1\cdot \vec{e}_1 +v_2\cdot \vec{e}_2 +\ldots +v_n\cdot \vec{e}_n, \]
d.h. jeder Vektor ist Linearkombination der Standardvektoren.}
\lang{en}{
The standard vectors of $\R^n$
\[ e_1=\begin{pmatrix} 1\\ 0\\ \vdots \\ 0\end{pmatrix},\quad e_2=\begin{pmatrix} 0\\ 1\\ \vdots \\ 0\end{pmatrix}
,\quad \ldots \quad e_n=\begin{pmatrix} 0\\ \vdots \\ 0\\ 1\end{pmatrix} \]
form a generating set of $\R^n$, because every vector $\begin{pmatrix} \alpha_1\\ \vdots \\ \alpha_n\end{pmatrix}$ 
can be written as
\[ \begin{pmatrix} \alpha_1\\ \vdots \\ \alpha_n \end{pmatrix} = \alpha_1 \cdot e_1 + \alpha_2 \cdot e_2 +\ldots +\alpha_n\cdot e_n, \]
d.h. Every vector is a linear combination of the standard vectors.}

\tab{\lang{de}{Polynomfunktionen} \lang{en}{Polynomial functions}}
\lang{de}{
Jedes Polynom $p$ von Grad $\leq n$ lässt sich als die folgende Summe schreiben:
\[
p(x) = a_0 + a_1 x + a_2 x^2 + \ldots + a_n x^n = \sum_{j=0}^n a_j x^j
\]
Die sogenannten Monome $x^j$, $j=0,...,n$ erzeugen den Vektorraum der Polynome mit Grad $\leq n,$
denn jedes Polynom kann als Linearkombination von den Monomen $x^j$ geschrieben werden.}
\lang{en}{Every polynomial $p$ of degree $\leq n$ can be writting as the following sum:
\[
p(x) = a_0 + a_1 x + a_2 x^2 + \ldots + a_n x^n = \sum_{j=0}^n a_j x^j
\]
The so-called monomials $x^j$, $j=0,...,n$ generate the vector space of the polynomials with degree $\leq n,$
because every polynomial can also be seen as a linear combination of monomials $x^j$.}

\lang{de}{Betrachten wir allerdings den Raum aller % (beispielsweise reellen) 
Polynome, so können wir $x^{n+1}$ nicht
als Linearkombination aus $1, x, x^2,...,x^n$ schreiben. Da der Grad eines Polynoms beliebig groß sein kann, können wir durch endlich
viele Monome nicht alle % (reellen) 
Polynome beschreiben. Deshalb ist der Vektorraum der Polynomfunktionen nicht endlich erzeugt.}
\lang{en}{If we consider the space of all polynomials, we can not build $X^{n+1}$ as a linear combination of $1, x, x^2,...,x^n$.
Since the degree of a polynomial can be arbitrarily large, we can not describe all polynomials with only a finite number of monomials.}

\tab{$M(m,n;\K)$}
\lang{de}{Die Matrizen}
\lang{en}{The matrizes}
\[ \begin{pmatrix} 1 & 0 & \ldots & 0 \\ 0 & 0 & \ldots & 0\\ \vdots & \vdots & \vdots & \vdots \\ 0 & 0 & \ldots & 0 \end{pmatrix},
\begin{pmatrix} 0 & 1 & \ldots & 0 \\ 0 & 0 & \ldots & 0\\ \vdots & \vdots & \vdots & \vdots \\ 0 & 0 & \ldots & 0 \end{pmatrix}, \ldots,
\begin{pmatrix} 0 & 0 & \ldots & 1 \\ 0 & 0 & \ldots & 0\\ \vdots & \vdots & \vdots & \vdots \\ 0 & 0 & \ldots & 0 \end{pmatrix},
\begin{pmatrix} 0 & 0 & \ldots & 0 \\ 1 & 0 & \ldots & 0\\ \vdots & \vdots & \vdots & \vdots \\ 0 & 0 & \ldots & 0 \end{pmatrix},
\begin{pmatrix} 0 & 0 & \ldots & 0 \\ 0 & 1 & \ldots & 0\\ \vdots & \vdots & \vdots & \vdots \\ 0 & 0 & \ldots & 0 \end{pmatrix}, \ldots,
\begin{pmatrix} 0 & 0 & \ldots & 0 \\ 0 & 0 & \ldots & 0\\ \vdots & \vdots & \vdots & \vdots \\ 0 & 0 & \ldots & 1 \end{pmatrix}
\]
\lang{de}{
bilden ein Erzeugendensystem von $M(m,n;\K)$, da jede Matrix $\begin{pmatrix} a_{11}& a_{12} & \ldots & a_{1n}\\ a_{21}&a_{22}&\ldots & a_{2n} \\ \vdots & \vdots & & \vdots \\ a_{m1} & a_{m2} & \ldots & a_{mn} \end{pmatrix}$ geschrieben 
werden kann als
\[\begin{pmatrix} a_{11}& a_{12} & \ldots & a_{1n}\\ a_{21}&a_{22}&\ldots & a_{2n} \\ \vdots & \vdots & & \vdots \\ a_{m1} & a_{m2} & \ldots & a_{mn} \end{pmatrix}
= a_{11} \cdot \begin{pmatrix} 1 & 0 & \ldots & 0 \\ 0 & 0 & \ldots & 0\\ \vdots & \vdots & \vdots & \vdots \\ 0 & 0 & \ldots & 0 \end{pmatrix}
 +\ldots + 
 a_{mn} \cdot \begin{pmatrix} 0 & 0 & \ldots & 0 \\ 0 & 0 & \ldots & 0\\ \vdots & \vdots & \vdots & \vdots \\ 0 & 0 & \ldots & 1 \end{pmatrix}
\]}
\lang{en}{
are a generating set of $M(m,n;\K)$, since every matrix $\begin{pmatrix} a_{11}& a_{12} & \ldots & a_{1n}\\ a_{21}&a_{22}&\ldots & a_{2n} \\ \vdots & \vdots & & \vdots \\ a_{m1} & a_{m2} & \ldots & a_{mn} \end{pmatrix}$
can be written as
\[\begin{pmatrix} a_{11}& a_{12} & \ldots & a_{1n}\\ a_{21}&a_{22}&\ldots & a_{2n} \\ \vdots & \vdots & & \vdots \\ a_{m1} & a_{m2} & \ldots & a_{mn} \end{pmatrix}
= a_{11} \cdot \begin{pmatrix} 1 & 0 & \ldots & 0 \\ 0 & 0 & \ldots & 0\\ \vdots & \vdots & \vdots & \vdots \\ 0 & 0 & \ldots & 0 \end{pmatrix}
 +\ldots + 
 a_{mn} \cdot \begin{pmatrix} 0 & 0 & \ldots & 0 \\ 0 & 0 & \ldots & 0\\ \vdots & \vdots & \vdots & \vdots \\ 0 & 0 & \ldots & 1 \end{pmatrix}
\]}
\end{tabs*}
\end{example}



\section{\lang{de}{Lineare Unabhängigkeit} \lang{en}{Linear indepence}}

\lang{de}{Wir verallgemeinern nun den aus dem \ref[basen][ersten Teil des Kurses]{def_lin_unabh}
 bekannten Begriff der linearen Abhängigkeit bzw. Unabhängigkeit.}
 \lang{en}{Now we will generalise the concept of linear (in-)dependence, which we already know from \ref[basen][Part 1 of the course]{def_lin_unabh}.}

\begin{definition}[\lang{de}{Lineare Abhängigkeit von Vektoren} \lang{en}{Linear dependence of vectors}]\label{def:lineareAbhaengigkeit}
\lang{de}{Ein Vektor ${v}$ eines Vektorraums $\vectorspace{V}$ hei"st von Vektoren ${w}_1,\ldots,{w}_k \in \vectorspace{V}$ \emph{linear abh"angig}, wenn er als Linearkombination
der Vektoren ${w}_1,\ldots,{w}_k$ geschrieben werden kann. 
Andernfalls hei"st ${v}$ \emph{linear unabh"angig von} ${w}_1,\ldots,{w}_k$.}
\lang{en}{A vector ${v}$ in a vector space $\vectorspace{V}$ is called \emph{linearly dependend} of the vectors ${w}_1,\ldots,{w}_k \in \vectorspace{V}$,
if $v$ can be written as a linear combination of the vectors ${w}_1,\ldots,{w}_k$.
Otherwise we call ${v}$ \emph{linearly independent}.}

\lang{de}{Für eine Menge von Vektoren $\{ {v}_1;\ldots;{v}_l\}\subset \vectorspace{V}$ gilt:  \,
${v}_1,\ldots,{v}_l$ heißen \emph{linear abhängig}, wenn mindestens einer der Vektoren dieser Menge
von den anderen linear abh"angig ist, und sie heißen \emph{linear unabh"angig}, wenn keiner der Vektoren von
den anderen linear abh"angig ist.}
\lang{en}{
For a set of vectors $\{ {v}_1;\ldots;{v}_l\}\subset \vectorspace{V}$ holds: \,
${v}_1,\ldots,{v}_l$ are called \emph{linearly dependent}, if a least one of the vectors in the set is linearly dependent from 
the others. They are called \emph{linearly independent}, if none of the vectors is linearly dependent from the others.}
\end{definition} 
%
%%% Video K.M. - Teil 2 von Video 10933, neu: 11271
% 
\lang{de}{Das folgende Video vertieft die Definition der Linearen (Un-)Abhängigkeit anhand einiger Beispiele.

\floatright{\href{https://api.stream24.net/vod/getVideo.php?id=10962-2-11271&mode=iframe&speed=true}
{\image[75]{00_video_button_schwarz-blau}}}}\\\\
\\

\lang{de}{Um herauszufinden, ob mehrere Vektoren ${v}_1,\ldots,{v}_l$ linear unabh"angig sind, müsste man also untersuchen,
ob ${v}_1$ als Linearkombination von ${v}_2,\ldots,{v}_l$ geschrieben werden kann, oder ob
${v}_2$ als Linearkombination von ${v}_1,{v}_3, \ldots,{v}_l$ geschrieben werden kann, etc.}
\lang{en}{If we want to know, if several vectors ${v}_1,\ldots,{v}_l$ are linearly independent, we need to check if ${v}_1$
can be displayed as a linear combination of ${v}_2,\ldots,{v}_l$, ${v}_2$ as a linear combinations of ${v}_1,{v}_3, \ldots,{v}_l$ and so on.}

         
\lang{de}{Einfacher geht es mit Hilfe des folgenden Satzes:}
\lang{en}{But it is easier with the help of the following theorem:}

\begin{theorem}[\lang{de}{Lineare Unabhängigkeit von Vektoren} \lang{en}{Linear independence of vectors}] \label{Lin_unabh_Basis}
\lang{de}{Vektoren ${v}_1,\ldots,{v}_l$ eines Vektorraums $\vectorspace{V}$ sind genau dann linear unabh"angig, wenn die Gleichung 
\[ r_1{v}_1+\ldots +r_l{v}_l=0 \]
nur f"ur $r_1=\ldots=r_l=0$ erf"ullt ist.}
\lang{en}{Vectors ${v}_1,\ldots,{v}_l$ in a vector space $\vectorspace{V}$ are linearly independent if and only if the
equations 
\[ r_1{v}_1+\ldots +r_l{v}_l=0 \]
is only fulfilled for $r_1=\ldots=r_l=0$. }
\end{theorem}

\begin{example}
\begin{tabs*}
\tab{\lang{de}{Standardvektoren} \lang{en}{Standard vectors}}
\lang{de}{
Die Standardvektoren
\[ e_1=\begin{pmatrix} 1\\ 0\\ \vdots \\ 0\end{pmatrix},\quad e_2=\begin{pmatrix} 0\\ 1\\ \vdots \\ 0\end{pmatrix}
,\quad \ldots, \quad e_n=\begin{pmatrix} 0\\ \vdots \\ 0\\ 1\end{pmatrix} \]
sind linear unabhängig. Es ist nämlich
\[ r_1 \cdot e_1+r_2 \cdot e_2+\ldots +r_n \cdot e_n=\begin{pmatrix} r_1\\ r_2\\ \vdots \\ r_n\end{pmatrix} \]
nur dann der Nullvektor, wenn $r_1=r_2= \ldots =r_n=0$ gilt.}
\lang{en}{
The standard vectors
\[ e_1=\begin{pmatrix} 1\\ 0\\ \vdots \\ 0\end{pmatrix},\quad e_2=\begin{pmatrix} 0\\ 1\\ \vdots \\ 0\end{pmatrix}
,\quad \ldots, \quad e_n=\begin{pmatrix} 0\\ \vdots \\ 0\\ 1\end{pmatrix} \]
are linearly indepdendent.
\[ r_1 \cdot e_1+r_2 \cdot e_2+\ldots +r_n \cdot e_n=\begin{pmatrix} r_1\\ r_2\\ \vdots \\ r_n\end{pmatrix} \]
is equal to the zero-vector if and only if it holds $r_1=r_2= \ldots =r_n=0$.}
%
%  \[ \vec{e}_1=\begin{pmatrix} 1\\ 0\\ \vdots \\ 0\end{pmatrix},\quad \vec{e}_2=\begin{pmatrix} 0\\ 1\\ \vdots \\ 0\end{pmatrix}
%  ,\quad \ldots, \quad \vec{e}_n=\begin{pmatrix} 0\\ \vdots \\ 0\\ 1\end{pmatrix} \]
%  sind linear unabhängig. Es ist nämlich
%  \[ r_1\vec{e}_1+r_2\vec{e}_2+\ldots +r_n \vec{e}_n=\begin{pmatrix} r_1\\ r_2\\ \vdots \\ r_n\end{pmatrix} \]
%  nur dann der Nullvektor, wenn $r_1=r_2= \ldots =r_n=0$ gilt.
%
\tab{\lang{de}{Polynomfunktionen} \lang{en}{Polynomial functions}}
\lang{de}{Im Vektorraum der Polynomfunktionen sind die Monome $x^j$ jeweils linear unabhängig, denn
ein Polynom vom Grad $n$ ist eindeutig durch die Werte an $n+1$ (verschiedenen) Stellen bestimmt. (Diese Tatsache nennt sich Identitätssatz für Polynome.)
Der Wert des konstanten Nullpolynoms ist an allen Stellen (also sogar an unendlich vielen) gleich $0$.
Damit lässt sich das Nullpolynom nur in folgender Form darstellen:
\[
0 = 0 \cdot x^0 + 0 \cdot x^1 + 0 \cdot x^2 + ... + 0 \cdot x^n.
\]}
\lang{en}{The monomials $x^j$ in the vector space of polynomial functions are each linearly independent,
because a polynomial with degree $n$ is uniquely defined by the values at $n+1$ different spots.
(We call this identity theorem for polynomials.)
The value of the constant zero-polynomial is equal to $0$ at all spots (which means at infinitely many).
Therefore the zero-polynomial can only be displayed in the following form:
\[
0 = 0 \cdot x^0 + 0 \cdot x^1 + 0 \cdot x^2 + ... + 0 \cdot x^n.
\]}
\end{tabs*}
\end{example}

\section{\lang{de}{Basis} \lang{en}{Basis}}

\lang{de}{Für den $\R^n$ hatten wir schon den Begriff der \ref[basen][Basis]{def_basis}
kennengelernt.}
\lang{en}{For $\R^n$ we already learned the notion of the \ref[basen][basis]{def_basis}.}

\begin{definition}[\lang{de}{Basis} \lang{en}{Basis}]\label{def:basis}
\lang{de}{
Eine Menge von Vektoren $\{ {v}_1;\ldots;{v}_l \}$ eines Vektorraums $\vectorspace{V}$ hei"st \emph{Basis von $\vectorspace{V}$},
wenn sich jedes Element des Vektorraums eindeutig (d.h. auf genau eine Weise) als Linearkombination der Elemente ${v}_1,\ldots,{v}_l$ schreiben lässt.}
\lang{en}{
A set of vectors $\{ {v}_1;\ldots;{v}_l \}$ of the vector space $\vectorspace{V}$ is called \emph{basis of $\vectorspace{V}$},
if every element of the vector space can be displayed uniquely (i.e. in exactly one way) as a linear combination of the elements ${v}_1,\ldots,{v}_l$.}
\end{definition}

\begin{remark}
\begin{enumerate}
\item \lang{de}{Die Darstellung eines Vektors $v \in \vectorspace{V}$ als Linearkombination der Vektoren einer Basis von $\vectorspace{V}$
ist auf \textbf{genau} eine Weise möglich, wohingegen es bei einem Erzeugendensystem nur \textbf{mindestens} eine Möglichkeit geben muss.}
\lang{en}{The representation of a vector $v \in \vectorspace{V}$ as a linear combination of vectors of a basis of $\vectorspace{V}$
is \emph{unique}, while for the linear combination of vectors of a generating set is \emph{at least} one option.}

\item \lang{de}{Jede Basis ist also auch ein Erzeugendensystem.} \lang{en}{Each basis is also a generating set.}
\item \lang{de}{Ist $\{ {v}_1;\ldots;{v}_l \}$ eine Basis von $\vectorspace{V}$, so lässt sich auch der Nullvektor auf genau eine
Weise als Linearkombination darstellen. \\ Da $0\cdot {v}_1+\ldots+ 0\cdot {v}_l= 0$ eine solche ist, ist sie
auch die einzige.\\ Dies bedeutet (gemäß Theorem \ref{Lin_unabh_Basis}), dass die Vektoren einer Basis linear unabhängig sind.}
\lang{en}{If $\{ {v}_1;\ldots;{v}_l \}$ is a basis of $\vectorspace{V}$, there also is exactly one possible linear combination
for the zero-vector.\\ Since $0\cdot {v}_1+\ldots+ 0\cdot {v}_l= 0$ is such a linear combination, it is the only one.}
\end{enumerate}
\end{remark}

\lang{de}{Zur vorigen Bemerkung gilt auch die Umkehrung. Genauer:}
\lang{en}{The reverse of the previous remark also applies:}

\begin{theorem}
\lang{de}{Eine Menge von Vektoren $\{ {v}_1;\ldots;{v}_l \}$ ist genau dann eine Basis von $\vectorspace{V}$,
wenn sie ein Erzeugendensystem ist und die Vektoren $\{ {v}_1;\ldots;{v}_l \}$ linear unabhängig sind.}
\lang{en}{A set of vectors $\{ {v}_1;\ldots;{v}_l \}$ is a basis of $\vectorspace{V}$ if and only if it is
a generating set and the vectors $\{ {v}_1;\ldots;{v}_l \}$ are linearly independent.}
\end{theorem}

%%% Video K.M.
% 
\lang{de}{
Diese Zusammenhänge werden in dem folgenden Video noch einmal erläutert 
und anhand von Beispielen veranschaulicht.

\floatright{\href{https://api.stream24.net/vod/getVideo.php?id=10962-2-10934&mode=iframe&speed=true}
{\image[75]{00_video_button_schwarz-blau}}}}\\
\\

\lang{de}{
Nun haben wir oben gelernt, dass nicht jeder Vektorraum endlich erzeugt ist. Nicht endlich erzeugte Vektorräume können also keine Basis aus endlich vielen Vektoren haben.
Für solche unendlich dimensionalen Vektorräume lässt sich aber auch der Begriff der Basis definieren. Dies wollen wir hier allerdings nicht weiter vertiefen
und beschränken uns im Folgenden meist auf endlich erzeugte Vektorräume. Für diese können wir den folgenden Satz formulieren.}
\lang{en}{
We already know, that not every vector space is finitely generated. So not finitely generated vector spaces cannot have a basis consisting of finitely many vectors.
But we can still define the concept of basis for infinitely generated vector spaces. We do not want to discuss this here and
focus only on finitely generated vector spaces. For those we have the following theorem.}

\begin{theorem}
\lang{de}{Jede Basis eines endlich erzeugten Vektorraums $\vectorspace{V}$ hat gleich viele Elemente.}
\lang{en}{Every basis of a finitely generated vector space $\vectorspace{V}$ has the same number of elements.}
\end{theorem}

\lang{de}{Dieser Satz macht die Entscheidung, ob eine gegebene Menge von Vektoren eine Basis ist, leichter.
Damit wissen wir, dass $n$ linear unabhängige Vektoren im $\R^n$ stets eine Basis des $\R^n$ bilden.}
\lang{en}{This theorem helps to decide if a given set of vectors is a basis. Therefore we know, that $n$ linearly independent
vectors of $\R^n$ is always a basis of $\R^n$.}

\begin{definition}[\lang{de}{Dimension eines Vektorraums} \lang{en}{Dimension of a vector space}]\label{def:dimension}
\lang{de}{Die Anzahl der Elemente einer Basis eines Vektorraums (und damit jeder Basis eines Vektorraums) 
nennt man die \emph{Dimension} des Vektorraumes.  Man schreibt $\text{dim}(V)$.}
\lang{en}{The number of elements of a basis of a vector space (and therefore every basis of a vector space)
is called the \emph{dimension} of the vector space. We denote $\text{dim}(V)$.}
\end{definition}


%%% Video K.M.
%
\lang{de}{
Auch hierzu finden Sie Erläuterungen und Beispiele in einem Video:

\floatright{\href{https://api.stream24.net/vod/getVideo.php?id=10962-2-10935&mode=iframe&speed=true}
{\image[75]{00_video_button_schwarz-blau}}}}\\
\\

\lang{de}{
Besitzt $\vectorspace{V}$ kein endliches Erzeugendensystem, so sagen wir $V$ ist unendlich
dimensional und schreiben $\text{dim}(V) = \infty.$}
\lang{en}{
If $\vectorspace{V}$ does not have a finite generating system, we say, that $V$ is infinite-dimensional and write $\text{dim}(V) = \infty.$}

\begin{theorem}[\lang{de}{Basisergänzungssatz} \lang{en}{Basisergänzungssatz - english?!}]\label{thm:basisergaenzungssatz}
\lang{de}{
Sind die Vektoren der Menge $\{v_1 ; . . . ; v_m \} \subset \vectorspace{V}$ linear unabhängig und ist $V$ ein endlich erzeugter Vektorraum 
mit \, $\text{dim}(V) = n $ \, ( $ \geq m$), 
so l\"asst sich $\{v_1 ; . . . ; v_m\}$ durch Hinzunahme geeigneter Elemente aus $V$ zu einer Basis erweitern.}
\lang{en}{Let the set of vectors $\{v_1 ; . . . ; v_m \} \subset \vectorspace{V}$ be linearly independent and $V$ a finitely generated
vector space with \, $\text{dim}(V) = n $ \, ( $ \geq m$). $\{v_1 ; . . . ; v_m\}$ can be expanded to a basis by adjoining suitable elements of $V$.}
\end{theorem}

%%% Video K.M.
%
\lang{de}{
In folgendem Video wird der Basisergänzungssatz erläutert und um einige interessante
Folgerungen ergänzt.

\floatright{\href{https://api.stream24.net/vod/getVideo.php?id=10962-2-10936&mode=iframe&speed=true}
{\image[75]{00_video_button_schwarz-blau}}}}\\
\\

\lang{de}{
Insbesondere gilt, dass f\"ur einen endlich erzeugten Vektorraum auch immer eine Basis existiert.}
\lang{en}{
It especially holds, that it always exists a basis for a finitely generated vector space.}

\begin{tabs*}
\tab{\lang{de}{$\R^n$} \lang{en}{$\R^n$}}
\lang{de}{
Die Standardvektoren \[ e_1=\begin{pmatrix} 1\\ 0\\ \vdots \\ 0\end{pmatrix},\quad e_2=\begin{pmatrix} 0\\ 1\\ \vdots \\ 0\end{pmatrix}
,\quad \ldots, \quad e_n=\begin{pmatrix} 0\\ \vdots \\ 0\\ 1\end{pmatrix} \]
erzeugen den $\R^n$ und sind linear unabhängig, d.h. sie bilden eine Basis von $\R^n$.}
\lang{en}{
The standard vectors \[ e_1=\begin{pmatrix} 1\\ 0\\ \vdots \\ 0\end{pmatrix},\quad e_2=\begin{pmatrix} 0\\ 1\\ \vdots \\ 0\end{pmatrix}
,\quad \ldots, \quad e_n=\begin{pmatrix} 0\\ \vdots \\ 0\\ 1\end{pmatrix} \]
generate $\R^n$ and are linearly independent, i.e. they are a basis of $\R^n$.}

\tab{\lang{de}{Polynomfunktionen} \lang{en}{Polynomnial functions}}
\lang{de}{
Für den endlich erzeugten Vektorraum der Polynomfunktionen mit Grad $\leq n$ ist beispielsweise
eine Basis durch $\{x \mapsto 1;x\mapsto x;x \mapsto x^2;...;x \mapsto x^n\}$ gegeben, denn die Funktionen erzeugen den Raum und sind linear
unabhängig.}
\lang{en}{
For the finitely generated vector space of polynomnial functions with degree $\leq n$, for example,
$\{x \mapsto 1;x\mapsto x;x \mapsto x^2;...;x \mapsto x^n\}$ is a basis, because the functions generate the space and
are linearly independent.}

\lang{de}{Achtung: Um auch die konstanten Funktionen durch Linearkombinationen erzeugen zu können,
besitzt eine Basis der Polynomfunktionen mit Grad $\leq n$ immer $n+1$ Elemente. In manchen Kontexten
werden auch die Polynomfunktionen mit Grad $<n$ betrachtet. Dann sind es $n$ Basiselemente (also z.B. $\{x\mapsto 1;x\mapsto x;...;x \mapsto x^{n-1}\}$).}
\lang{en}{Warning: Since we also want to generate the constant functions through linear combinations,
the basis of the polynomial functions with degree $\leq n$ always consists of $n+1$ elements.
In some contexts, there may be discussed polynomial functions with degree $<n$. Then the basis
consists of $n$ elements (e.g. $\{x\mapsto 1;x\mapsto x;...;x \mapsto x^{n-1}\}$).}
\end{tabs*}

\begin{remark}[\lang{de}{Koordinatenvektor / Koeffizientenvektor} \lang{en}{Coordinate vector / coeffizient vector}]\label{rem:koordinatenvektor}\label{rem:koeffizientenvektor}
 \lang{de}{Ist $B:=\{ {v}_1;\ldots;{v}_l \}$ eine Basis eines Vektorraums $\vectorspace{V}$, so lässt sich jedes Element $v \in V$ also schreiben als \[
                                                                                                                                         v = \sum_{i=1}^l \alpha_i v_i\]
        mit passenden $\alpha_i \in \K$.                                                                                                                                 
  Der Vektor $\begin{pmatrix}
               \alpha_1 \\ \vdots \\ \alpha_l
              \end{pmatrix}$ nennt sich \emph{Koordinatenvektor} (oder auch 
              \emph{Koeffizientenvektor}) von $v$ bezüglich der Basis $B.\,$
  Wir schreiben hierfür ${\,}_B v,\,$ alternativ wird auch die Schreibweise $v_B\,$ verwendet.}

 \lang{en}{If $B:=\{ {v}_1;\ldots;{v}_l \}$ is a basis of a vector space $\vectorspace{V}$, 
 we can write every element $v \in V$ as \[
                                                                                                                                        
    v = \sum_{i=1}^l \alpha_i v_i\]
        with suitable $\alpha_i \in \K$.                                                                                                                                 
  The vector $\begin{pmatrix}
               \alpha_1 \\ \vdots \\ \alpha_l
              \end{pmatrix}$ is called \emph{coordinate vector} (or
              \emph{coeffizient vector}) of $v$ referring to the basis $B.\,$
  We denote this as ${\,}_B v,\,$ alternatively we may use the notation $v_B\,$.}
  
%
%%% Video K.M.
%
\lang{de}{\floatright{\href{https://api.stream24.net/vod/getVideo.php?id=10962-2-10828&mode=iframe&speed=true}
{\image[75]{00_video_button_schwarz-blau}}}}\\
\\                                                                                                                                    
\end{remark}

\begin{example} \label{koord_vec}
\begin{tabs}
\tab{\lang{de}{$\R^n$} \lang{en}{$\R^n$}}
\lang{de}{
Bezüglich der Standardbasis des $\R^n$ ist der Vektor
$\begin{pmatrix}a_1 \\ \vdots \\ a_n\end{pmatrix}$ stets schon gleich seinem
Koordinatenvektor.}
\lang{en}{
Regarding the standard basis of $\R^n$ the vector
$\begin{pmatrix}a_1 \\ \vdots \\ a_n\end{pmatrix}$ is always equal to its coordinate vector.}

\tab{\lang{de}{$\R^2$} \lang{en}{$\R^2$}}
\lang{de}{
Der Koordinatenvektor des Vektors $\begin{pmatrix} -2 \\ 1\end{pmatrix}$ bezüglich der Standardbasis 
$\{\begin{pmatrix}1 \\ 0 \end{pmatrix}; \begin{pmatrix}0 \\1 \end{pmatrix}\}$ des $\R^2$ ist 
gegeben durch $\begin{pmatrix} -2 \\ 1\end{pmatrix}$ selbst.}
\lang{en}{
The coordinate vector of the vector $\begin{pmatrix} -2 \\ 1\end{pmatrix}$ regarding the standard basis 
$\{\begin{pmatrix}1 \\ 0 \end{pmatrix}; \begin{pmatrix}0 \\1 \end{pmatrix}\}$ des $\R^2$ ist 
is $\begin{pmatrix} -2 \\ 1\end{pmatrix}$ selbst.}

\\

\lang{de}{Bezüglich der Basis $C:=\{\begin{pmatrix}1 \\ 0 \end{pmatrix}; \begin{pmatrix}1 \\1 \end{pmatrix}\}$ des $\R^2$ schreibt man den Vektor 
$\begin{pmatrix} -2 \\ 1\end{pmatrix}$ als

 \[
 \begin{pmatrix} -2 \\ 1\end{pmatrix} = \textcolor{#CC6600}{-3} \cdot \begin{pmatrix} 1 \\ 0 \end{pmatrix}+ \textcolor{#CC6600}{1} \cdot \begin{pmatrix}1 \\ 1\end{pmatrix}
 \]
 Deshalb ist der Koordinatenvektor von $\begin{pmatrix} -2 \\ 1\end{pmatrix}$ bezüglich $C$ gegeben durch $\textcolor{#CC6600}{\begin{pmatrix}-3 \\ 1 \end{pmatrix}}$.}


\lang{en}{Regarding the basis $C:=\{\begin{pmatrix}1 \\ 0 \end{pmatrix}; \begin{pmatrix}1 \\1 \end{pmatrix}\}$ of $\R^2$ we write the vector 
$\begin{pmatrix} -2 \\ 1\end{pmatrix}$ as

 \[
 \begin{pmatrix} -2 \\ 1\end{pmatrix} = \textcolor{#CC6600}{-3} \cdot \begin{pmatrix} 1 \\ 0 \end{pmatrix}+ \textcolor{#CC6600}{1} \cdot \begin{pmatrix}1 \\ 1\end{pmatrix}
 \]
 Therefore the coordinate vector of $\begin{pmatrix} -2 \\ 1\end{pmatrix}$ with respect to $C$ is $\textcolor{#CC6600}{\begin{pmatrix}-3 \\ 1 \end{pmatrix}}$.}

\tab{\lang{de}{Polynomfunktionen} \lang{en}{Polynomial functions}}
\lang{de}{
Bezüglich der Standardbasis für den Raum der Polynomfunktionen mit Grad $\leq n$, welche
wie oben gesehen durch $\{x \mapsto 1;x\mapsto x;x \mapsto x^2;...;x \mapsto x^n\}$ gegeben ist,
ist der Koordinatenvektor für die Polynomfunktion $f(x)= a_n x^n + ... + a_1 x +a_0$ gegeben durch
$\begin{pmatrix}a_0 \\ a_1 \\ a_2 \\ \vdots \\ a_n \end{pmatrix}$.}
\lang{de}{
With respect to the standard basis of the vector space of polynomial functions with degree $\leq n$, which is
$\{x \mapsto 1;x\mapsto x;x \mapsto x^2;...;x \mapsto x^n\}$ (like we have seen above),
the coordinate vector for the polynomial function $f(x)= a_n x^n + ... + a_1 x +a_0$ is given by
$\begin{pmatrix}a_0 \\ a_1 \\ a_2 \\ \vdots \\ a_n \end{pmatrix}$.}
\end{tabs}
\end{example}

\end{visualizationwrapper}



\end{content}