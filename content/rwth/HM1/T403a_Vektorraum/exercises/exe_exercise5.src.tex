\documentclass{mumie.element.exercise}
%$Id$
\begin{metainfo}
  \name{
    \lang{de}{Ü05: Orthogonalisierung}
    \lang{en}{Ex05: Orthogonalisation}
  }
  \begin{description} 
 This work is licensed under the Creative Commons License Attribution 4.0 International (CC-BY 4.0)   
 https://creativecommons.org/licenses/by/4.0/legalcode 

    \lang{de}{Hier die Beschreibung}
    \lang{en}{}
  \end{description}
  \begin{components}
  \end{components}
  \begin{links}
    \link{generic_article}{content/rwth/HM1/T403a_Vektorraum/g_art_content_10c_Orthogonalbasen.meta.xml}{10c_Orthogonalbasen}
  \end{links}
  \creategeneric
\end{metainfo}
\begin{content}
\begin{block}[annotation]
	Im Ticket-System: \href{https://team.mumie.net/issues/14366}{Ticket 14366}
\end{block}
\begin{block}[annotation]
	Im Ticket-System: \href{http://team.mumie.net/issues/}{Ticket }
\end{block}

\title{
\lang{de}{Ü05: Gram-Schmidt Orthogonalisierung}
\lang{en}{Ex05: Gram-Schmidt process for orthogonalisation}
}

 
\lang{de}{
Eine Basis des Untervektorraums $\vectorspace{V}$ von $\R^4$ sei gegeben durch 
$\left\{ \begin{pmatrix} 1 \\ 2\\ 0 \\ 3 \end{pmatrix};
\begin{pmatrix} 4 \\ 0 \\ 5 \\ 8\end{pmatrix}; 
\begin{pmatrix} 8 \\ 1\\ 5 \\ 6 \end{pmatrix} \right\}.$ 

Bestimmen Sie mit Hilfe des Orthogonalisierungsverfahren von Gram-Schmidt (unter Verwendung des Standard-Skalarprodukts im $\R^4$)
eine Orthonormalbasis dieses Vektorraums.
}
\lang{en}{
A basis of the vector subspace $\vectorspace{V}$ of $\R^4$ is given by 
$\left\{ \begin{pmatrix} 1 \\ 2\\ 0 \\ 3 \end{pmatrix};
\begin{pmatrix} 4 \\ 0 \\ 5 \\ 8\end{pmatrix}; 
\begin{pmatrix} 8 \\ 1\\ 5 \\ 6 \end{pmatrix} \right\}.$ 

Determine with the help of the Gram-Schmidt process for orthogonalisation (with the use of the standard scalar product of $\R^4$)
a orthonormal basis of this vector space.
}

\begin{tabs*}[\initialtab{0}\class{exercise}]
  \tab{
  \lang{de}{Antwort}
  \lang{en}{Answer}
  }
\[
   \left\{\frac{1}{\sqrt{14}}\begin{pmatrix} 1 \\ 2\\ 0 \\ 3 \end{pmatrix}; \frac{1}{7} \begin{pmatrix}2 \\ -4 \\ 5 \\ 2\end{pmatrix}; \frac{1}{\sqrt{21}} \begin{pmatrix} 4 \\ 1 \\ 0 \\ -2 \end{pmatrix}   \right\}
    \]

  \tab{
  \lang{de}{Lösung}
  \lang{en}{Solution}}
  
  \begin{incremental}[\initialsteps{1}]
    \step 
%    Als Teilmenge von $\R^4$ ist $\vectorspace{V}$ offenbar ein Untervektorraum von $\R^4$. Somit können wir für
%    das GramSchmidt-Verfahren das Standard-Skalarprodukt des $\R^4$ verwenden.

\lang{de}{
    Wir beginnen mit $u'_1 = \begin{pmatrix} 1 \\ 2\\ 0 \\ 3 \end{pmatrix}$. \\
    Der normierte Vektor ist dann gegeben durch
    $u_1 =\frac{1}{ \sqrt{\langle u'_1,u'_1 \rangle}} u'_1 = \frac{1}{\sqrt{14}}\begin{pmatrix} 1 \\ 2\\ 0 \\ 3 \end{pmatrix}$.}
\lang{en}{
    We begin with mit $u'_1 = \begin{pmatrix} 1 \\ 2\\ 0 \\ 3 \end{pmatrix}$. \\
    Then the normalised vector is
    $u_1 =\frac{1}{ \sqrt{\langle u'_1,u'_1 \rangle}} u'_1 = \frac{1}{\sqrt{14}}\begin{pmatrix} 1 \\ 2\\ 0 \\ 3 \end{pmatrix}$.}
    
    \step \lang{de}{Nun bestimmen wir nach \ref[10c_Orthogonalbasen][Gram Schmidt]{gram_schmidt} \,    
     aus $v_2 := \begin{pmatrix} 4 \\ 0 \\ 5 \\ 8\end{pmatrix}$ und  $u_1$ den zweiten Vektor der ONB:
    \[
    u'_2 = v_2 - \langle v_2,u_1 \rangle u_1
         = \begin{pmatrix} 4 \\ 0 \\ 5 \\ 8\end{pmatrix} - \left[ \begin{pmatrix} 4 \\ 0 \\ 5 \\ 8\end{pmatrix} \bullet \frac{1}{\sqrt{14}}\begin{pmatrix} 1 \\ 2\\ 0 \\ 3 \end{pmatrix}\right] \cdot \frac{1}{\sqrt{14}}\begin{pmatrix} 1 \\ 2\\ 0 \\ 3 \end{pmatrix} = \begin{pmatrix}2 \\ -4 \\ 5 \\ 2\end{pmatrix}.
    \]
    Wir normieren den Vektor $u'_2$
    \[
    u_2 =\frac{1}{ \sqrt{\langle u'_2,u'_2 \rangle}} u'_2 = \frac{1}{7}\cdot \begin{pmatrix}2 \\ -4 \\ 5 \\ 2\end{pmatrix}.
    \]}
    \lang{en}{Now we determine according to the \ref[10c_Orthogonalbasen][Gram-Schmidt process]{gram_schmidt} \,    
     the second vector of the orthonormal basis starting with $v_2 := \begin{pmatrix} 4 \\ 0 \\ 5 \\ 8\end{pmatrix}$ and  $u_1$:
    \[
    u'_2 = v_2 - \langle v_2,u_1 \rangle u_1
         = \begin{pmatrix} 4 \\ 0 \\ 5 \\ 8\end{pmatrix} - \left[ \begin{pmatrix} 4 \\ 0 \\ 5 \\ 8\end{pmatrix} \bullet \frac{1}{\sqrt{14}}\begin{pmatrix} 1 \\ 2\\ 0 \\ 3 \end{pmatrix}\right] \cdot \frac{1}{\sqrt{14}}\begin{pmatrix} 1 \\ 2\\ 0 \\ 3 \end{pmatrix} = \begin{pmatrix}2 \\ -4 \\ 5 \\ 2\end{pmatrix}.
    \]
    We normalise the vector $u'_2$
    \[
    u_2 =\frac{1}{ \sqrt{\langle u'_2,u'_2 \rangle}} u'_2 = \frac{1}{7}\cdot \begin{pmatrix}2 \\ -4 \\ 5 \\ 2\end{pmatrix}.
    \]}
    
    \step \lang{de}{Wir berechnen nun nach \ref[10c_Orthogonalbasen][Gram Schmidt]{gram_schmidt} \, 
    den dritten Vektor der ONB aus $v_3 := \begin{pmatrix} 8 \\ 1\\ 5 \\ 6 \end{pmatrix},$ $u_1$ und $u_2:$
    \begin{align*}
    u'_3 &= v_3 - \langle v_3,u_1 \rangle u_1 - \langle v_3,u_2 \rangle u_2    \\
    & \\
    &= \begin{pmatrix} 8 \\ 1\\ 5 \\ 6 \end{pmatrix} - \left[ \begin{pmatrix} 8 \\ 1\\ 5 \\ 6 \end{pmatrix}\bullet \frac{1}{\sqrt{14}}\begin{pmatrix} 1 \\ 2\\ 0 \\ 3 \end{pmatrix} \right] \cdot \frac{1}{\sqrt{14}}\begin{pmatrix} 1 \\ 2\\ 0 \\ 3 \end{pmatrix} - \left[ \begin{pmatrix} 8 \\ 1\\ 5 \\ 6 \end{pmatrix}\bullet \frac{1}{7}\cdot \begin{pmatrix}2 \\ -4 \\ 5 \\ 2\end{pmatrix} \right] \cdot \frac{1}{7}\cdot \begin{pmatrix}2 \\ -4 \\ 5 \\ 2\end{pmatrix}
    = \begin{pmatrix} 4 \\ 1 \\ 0 \\ -2 \end{pmatrix}.
    \end{align*}}
    \lang{en}{Now we calulate the third vector of the orthonormal basis with the help of the \ref[10c_Orthogonalbasen][Gram-Schmidt process]{gram_schmidt} \, 
     and the vectors $v_3 := \begin{pmatrix} 8 \\ 1\\ 5 \\ 6 \end{pmatrix},$ $u_1$ and $u_2:$
    \begin{align*}
    u'_3 &= v_3 - \langle v_3,u_1 \rangle u_1 - \langle v_3,u_2 \rangle u_2    \\
    & \\
    &= \begin{pmatrix} 8 \\ 1\\ 5 \\ 6 \end{pmatrix} - \left[ \begin{pmatrix} 8 \\ 1\\ 5 \\ 6 \end{pmatrix}\bullet \frac{1}{\sqrt{14}}\begin{pmatrix} 1 \\ 2\\ 0 \\ 3 \end{pmatrix} \right] \cdot \frac{1}{\sqrt{14}}\begin{pmatrix} 1 \\ 2\\ 0 \\ 3 \end{pmatrix} - \left[ \begin{pmatrix} 8 \\ 1\\ 5 \\ 6 \end{pmatrix}\bullet \frac{1}{7}\cdot \begin{pmatrix}2 \\ -4 \\ 5 \\ 2\end{pmatrix} \right] \cdot \frac{1}{7}\cdot \begin{pmatrix}2 \\ -4 \\ 5 \\ 2\end{pmatrix}
    = \begin{pmatrix} 4 \\ 1 \\ 0 \\ -2 \end{pmatrix}.
    \end{align*}}

    \lang{de}{
    Schließlich ist nach erneuter Normierung
    \[
    u_3 =\frac{1}{ \sqrt{\langle u'_3,u'_3 \rangle}} u'_3 = \frac{1}{\sqrt{21}} \begin{pmatrix} 4 \\ 1 \\ 0 \\ -2 \end{pmatrix}.
    \]
    Eine Orthonormalbasis des Vektorraums ist somit durch $\{u_1;u_2;u_3\}$ gegeben.}
    \lang{en}{
    After the normalisation we have
    \[
    u_3 =\frac{1}{ \sqrt{\langle u'_3,u'_3 \rangle}} u'_3 = \frac{1}{\sqrt{21}} \begin{pmatrix} 4 \\ 1 \\ 0 \\ -2 \end{pmatrix}.
    \]
    So with the set $\{u_1;u_2;u_3\}$ is a orthonormalbasis of the vector space found.}
  \end{incremental}

\end{tabs*}

\end{content}