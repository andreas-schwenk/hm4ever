\documentclass{mumie.element.exercise}
%$Id$
\begin{metainfo}
  \name{
    \lang{de}{Ü01: Basen von Vektorräumen}
    \lang{en}{Ex01: Basis of vector spaces}
  }
  \begin{description} 
 This work is licensed under the Creative Commons License Attribution 4.0 International (CC-BY 4.0)   
 https://creativecommons.org/licenses/by/4.0/legalcode 

    \lang{de}{Hier die Beschreibung}
    \lang{en}{}
  \end{description}
  \begin{components}
  \end{components}
  \begin{links}
  \end{links}
  \creategeneric
\end{metainfo}
\begin{content}
\begin{block}[annotation]
	Im Ticket-System: \href{https://team.mumie.net/issues/14370}{Ticket 14370}
\end{block}
\begin{block}[annotation]
	Im Ticket-System: \href{http://team.mumie.net/issues/}{Ticket }
\end{block}

\title{
\lang{de}{Ü01: Basen von Vektorräumen}
}
\title{
\lang{en}{Ex01: Basis of vector spaces}
}

 
\lang{de}{Bilden folgende Mengen eine Basis des Raums der Polynomfunktionen mit Grad $\leq 2$?}
\lang{en}{Do the given sets form a basis of the vector space of polynomial functions with degree $\leq 2$?}
\begin{table}[\class{items}]
  \nowrap{a) $\left\{x \mapsto x^2; x\mapsto -x^2+3x \right\}$} \\
  \nowrap{b) $\left\{x \mapsto x^2+1; x \mapsto 4x^2-x+3; x \mapsto 3x^2+2x \right\}$}\\
  \nowrap{c) $\left\{x \mapsto 4x^2-x+2; x\mapsto 5x^2+3; x \mapsto -16x^2+4x-8 \right\}$}
\end{table}

\begin{tabs*}[\initialtab{0}\class{exercise}]
  \tab{
  \lang{de}{Antwort}
  }
  \tab{
  \lang{en}{Answer}
  }
\lang{de}{\begin{table}[\class{items}]

    \nowrap{a) Nein} \\
    \nowrap{b) Ja} \\
    \nowrap{c) Nein} 
  \end{table}}
\lang{en}{\begin{table}[\class{items}]

    \nowrap{a) No} \\
    \nowrap{b) Yes} \\
    \nowrap{c) No} 
  \end{table}}

  \tab{
  \lang{de}{Lösung a)}}
  \tab{
  \lang{en}{Solution a)}}
  
%  \begin{incremental}[\initialsteps{1}]
%    \step 
    \lang{de}{Da die Dimension des Vektorraums gleich $3$ ist, benötigt man mindestens
	$3$ Vektoren um jedes Polynom vom Grad $\leq 2$ erzeugen zu können. Da die 
	Menge nur aus zwei Elementen besteht, kann sie kein Erzeugendensystem
	davon bilden und ist damit auch keine Basis.}
 \lang{en}{The dimension of the vector space is equal to $3$. Therefore we need at least $3$ vectors to create
 every polynomial with degree $\leq 2$. Since the set consists only of two elements, it cannot be a generating set
 of the vector space and as a conclusion of that no basis of the vector space.} 
% \end{incremental}

  \tab{
  \lang{de}{Lösung b)}
  }
   \tab{
  \lang{en}{Solution b)}
  }
  \begin{incremental}[\initialsteps{1}]
    \step \lang{de}{Da die Menge aus $3$ Elementen besteht, müssen wir diese nur auf lineare
	Unabhängigkeit überprüfen.}
 \lang{en}{Since the set consists of $3$ elements, we only need to check if they are
 linearly independent.}
    \step \lang{en}{We name the polynomial in the given order $p,q$ and $r$. That means
    \[
    p(x)= \textcolor{#CC6600}{1}\cdot x^2 + \textcolor{#CC6600}{1} \cdot 1,
    \]
    \[ 
    q(x)= \textcolor{#0066CC}{4}\cdot x^2 + (\textcolor{#0066CC}{-1}) \cdot x + \textcolor{#0066CC}{3} \cdot 1,
    \]
    \[
    r(x)=\textcolor{#00CC00}{3} \cdot x^2 + \textcolor{#00CC00}{2} \cdot x.
    \]
    Therefore let be $a,b,c\in \R$ and we set}
    \lang{de}{Wir benennen die Polynome der angegebenen Reihe nach $p,q$ und $r$. Das bedeutet
    \[
    p(x)= \textcolor{#CC6600}{1}\cdot x^2 + \textcolor{#CC6600}{1} \cdot 1,
    \]
    \[ 
    q(x)= \textcolor{#0066CC}{4}\cdot x^2 + (\textcolor{#0066CC}{-1}) \cdot x + \textcolor{#0066CC}{3} \cdot 1,
    \]
    \[
    r(x)=\textcolor{#00CC00}{3} \cdot x^2 + \textcolor{#00CC00}{2} \cdot x.
    \]
    Dazu seien $a,b,c\in \R$ und wir setzen an }
    
	\[0 \overset{!}{=} a\cdot p(x) +b\cdot q(x)+c r(x). \]
    \step \lang{de}{Wir machen einen Koeffizientenvergleich der Monome und erhalten die drei Gleichungen (für $1$, $x$ und $x^2$)
    \[
    0 = a \cdot \textcolor{#CC6600}{1} + b \cdot \textcolor{#0066CC}{3} + c \cdot \textcolor{#00CC00}{0},
    \]
    \[
    0 = a \cdot \textcolor{#CC6600}{0} + b \cdot (\textcolor{#0066CC}{-1}) + c \cdot \textcolor{#00CC00}{2},
    \]
    \[
    0 = a \cdot \textcolor{#CC6600}{1} + b \cdot \textcolor{#0066CC}{4} + c \cdot \textcolor{#00CC00}{3},
    
    \]
    Aus der ersten Zeile ergibt sich $a=-3b$ und aus der zweiten Zeile ergibt sich
	$\frac{1}{2}b=c$. Einsetzen in die dritte Zeile liefert $ -3b+4b+ \frac{3}{2}b = 0$, was
	äquivalent dazu ist, dass $\frac{5}{2}b=0$ gilt und somit $b=0$. 
	Wenn $b=0$, dann gilt auch $a=0$ und $c=0$, also sind die Polynome
	linear unabhängig und bilden somit eine Basis. }
 \lang{en}{We compare the coefficients of the monom and receive drei equations (for $1$, $x$ and $x^2$)
    \[
    0 = a \cdot \textcolor{#CC6600}{1} + b \cdot \textcolor{#0066CC}{3} + c \cdot \textcolor{#00CC00}{0},
    \]
    \[
    0 = a \cdot \textcolor{#CC6600}{0} + b \cdot (\textcolor{#0066CC}{-1}) + c \cdot \textcolor{#00CC00}{2},
    \]
    \[
    0 = a \cdot \textcolor{#CC6600}{1} + b \cdot \textcolor{#0066CC}{4} + c \cdot \textcolor{#00CC00}{3},
    
    \]
  From the first row results $a=-3b$ and from the second row
	$\frac{1}{2}b=c$. Inserting this into the third row gives us $ -3b+4b+ \frac{3}{2}b = 0$, which is
 equivalent to $\frac{5}{2}b=0$ and therefore $b=0$. 
	If $b=0$, then it also holds $a=0$ and $c=0$. Therefore the polynomials are linearly independet and
 therefore they form a basis.}
  \end{incremental}

  \tab{
  \lang{de}{Lösung c)}
  }
  \tab{
  \lang{en}{Solution c)}
  }
%  \begin{incremental}[\initialsteps{1}]
%    \step 
    \lang{de}{Man sieht, dass das dritte Polynom genau dem $(-4)-$fachen des ersten
	Polynoms entspricht, damit sind die beiden Polynome linear abhängig.
	Damit bildet die Menge keine Basis.}
 \lang{en}{We see, that the third polynomial is equivalent to $(-4)$-times the first polynomial, so thos two
 are linearly dependent. The set does not form a basis.}
%  \end{incremental}

\end{tabs*}

\end{content}