\documentclass{mumie.element.exercise}
%$Id$
\begin{metainfo}
  \name{
    \lang{de}{Ü03: Basen/Koordinatenvektoren}
    \lang{en}{Ex03: Basis/Coordinate vectors}
  }
  \begin{description} 
 This work is licensed under the Creative Commons License Attribution 4.0 International (CC-BY 4.0)   
 https://creativecommons.org/licenses/by/4.0/legalcode 

    \lang{de}{Hier die Beschreibung}
    \lang{en}{}
  \end{description}
  \begin{components}
  \end{components}
  \begin{links}
    \link{generic_article}{content/rwth/HM1/T306_Reelle_Quadratische_Matrizen/g_art_content_15_inverse_matrix.meta.xml}{content_15_inverse_matrix}
    \link{generic_article}{content/rwth/HM1/T306_Reelle_Quadratische_Matrizen/g_art_content_17_cramersche_regel.meta.xml}{17_cramersche_regel}
    \link{generic_article}{content/rwth/HM1/T403a_Vektorraum/g_art_content_10b_lineare_abb.meta.xml}{10b_lineare_abb}
  \end{links}
  \creategeneric
\end{metainfo}
\begin{content}
\begin{block}[annotation]
	Im Ticket-System: \href{https://team.mumie.net/issues/14368}{Ticket 14368}
\end{block}

\title{
\lang{de}{Ü03: Basen/Koordinatenvektoren}
\lang{en}{Ex03: Basis/Coordinate vectors}
}

 
\lang{de}{Eine Basis des $\R^3$ sei gegeben durch 
$B := \left\{ \begin{pmatrix} 1 \\ 0 \\ 2 \end{pmatrix};
\begin{pmatrix} 3 \\ 0 \\ 5 \end{pmatrix}; 
\begin{pmatrix} 3 \\ 1\\ 7 \end{pmatrix} \right\}$.

% und $v= \begin{pmatrix}2 \\ -1 \\ 3\end{pmatrix}$ sei ein Vektor in Koordinatendarstellung bezüglich der Standardbasis des $\R^3$.

Bestimmen Sie den Koordinatenvektor von $v= \begin{pmatrix}2 \\ -1 \\ 3\end{pmatrix}\,$ bezüglich der Basis $B$.}

\lang{en}{A basis of $\R^3$ is given by 
$B := \left\{ \begin{pmatrix} 1 \\ 0 \\ 2 \end{pmatrix};
\begin{pmatrix} 3 \\ 0 \\ 5 \end{pmatrix}; 
\begin{pmatrix} 3 \\ 1\\ 7 \end{pmatrix} \right\}$.

% und $v= \begin{pmatrix}2 \\ -1 \\ 3\end{pmatrix}$ sei ein Vektor in Koordinatendarstellung bezüglich der Standardbasis des $\R^3$.

Determine the coordinate vector of $v= \begin{pmatrix}2 \\ -1 \\ 3\end{pmatrix}\,$ with respect of the basis $B$.}

\begin{tabs*}[\initialtab{0}\class{exercise}]
  \tab{
  \lang{de}{Antwort}
  \lang{en}{Answer}
  }
\[
    \begin{pmatrix}5 \\ 0 \\ -1\end{pmatrix}
    \]

  \tab{
  \lang{de}{Lösung}
  \lang{en}{Solution}}
  
  \begin{incremental}[\initialsteps{1}]
    \step % Wir nennen die oben angegebene Basis $B$. Wir müssen also ${\,}_B v$ bestimmen. \\ 
    
    \lang{de}{Gegeben ist 
    $v= \begin{pmatrix}2 \\ -1 \\ 3\end{pmatrix}= 2 \cdot \begin{pmatrix}1\\0\\0\end{pmatrix}-\begin{pmatrix}0\\1\\0\end{pmatrix}+3 \cdot \begin{pmatrix}0\\0\\1\end{pmatrix}$ 
    
    also $v={\,}_S v$ in seiner Darstellung bzgl. der Standard-Einheitsbasis des $\R^3$ und zu bestimmen ist ${\,}_B v$, der Koordinatenvektor von $v$ bzgl. der Basis $B$.}

\lang{en}{Given is 
    $v= \begin{pmatrix}2 \\ -1 \\ 3\end{pmatrix}= 2 \cdot \begin{pmatrix}1\\0\\0\end{pmatrix}-\begin{pmatrix}0\\1\\0\end{pmatrix}+3 \cdot \begin{pmatrix}0\\0\\1\end{pmatrix}$ 
    
    so $v={\,}_S v$ in its representation with respect of the standard basis of $\R^3$ und and to determine is ${\,}_B v$, the coordinate vector of $v$ with respect of the basis $B$.}

    \step
    \lang{de}{
    Im Kapitel über \ref[10b_lineare_abb][Basiswechsel und Koordinatentransformationen]{sec:basiswechsel} haben wir die 
    Basistransformation ${\,}_B T_S$ \, kennengelernt, die den Vektor ${\,}_S v\;$ von der Standardbasis $S$ des $\R^3$ 
    in seine Koordinatendarstellung ${\,}_B v$ bezüglich der Basis $B$ überführt. 
    Gemäß der \ref[10b_lineare_abb][1. Regel zum Basiswechsel]{regel_bw}
     \, gilt:  
     \begin{align}
     {\,}_B v= {\,}_B T_S \cdot {\,}_S v
     \end{align}}
     \lang{en}{
    In the chapter about \ref[10b_lineare_abb][basis transformation and coordinate transformation]{sec:basiswechsel}
    we got to know the basis transformation ${\,}_B T_S$ \, , which transform the vector ${\,}_S v\;$ from the standard basis $S$ of $\R^3$ 
    in its coordinate representation ${\,}_B v$ with respect of the basis $B$. 
    Regarding to the \ref[10b_lineare_abb][1. rule of the basis transformation]{regel_bw}
     \, holds:  
     \begin{align}
     {\,}_B v= {\,}_B T_S \cdot {\,}_S v
     \end{align}}
    
%   Aus der Vorlesung ist die Identität bekannt ${\,}_B v= {\,}_B T_S v$ mit der Standardbasis $S$ von $\R^3$.
    
    \step \lang{de}{Man liest nun die Basiswechsel-Matrix (von $B$ in die Standardbasis $S$) ab
    \[
    {\,}_S T_B= \begin{pmatrix}1 & 3 & 3 \\ 0 & 0 & 1 \\ 2 & 5 &7\end{pmatrix}.
    \]}
    \lang{en}{We can read off the basis transformation matrix (from $B$ to the standard basis $S$)
    \[
    {\,}_S T_B= \begin{pmatrix}1 & 3 & 3 \\ 0 & 0 & 1 \\ 2 & 5 &7\end{pmatrix}.
    \]}
     
%    
    \step \lang{de}{Gemäß der \ref[10b_lineare_abb][2. Regel zum Basiswechsel]{regel_bw} \, gilt \, ${\,}_B T_S = ({\,}_S T_B)^{-1}$.
     Setzt man dies in die obige Gleichung (1) ein, erhält man 
     
    \begin{align} 
    {\,}_B v =  ({\,}_S T_B)^{-1} \cdot v 
    \end{align}}

    \lang{en}{Regarding to the \ref[10b_lineare_abb][2. rule of the basis transformation]{regel_bw} \, holds \, ${\,}_B T_S = ({\,}_S T_B)^{-1}$.
     If we insert this into the above equation (1), we get 
     
    \begin{align} 
    {\,}_B v =  ({\,}_S T_B)^{-1} \cdot v 
    \end{align}}
  
    \step \lang{de}{Nun bestimmt man die Inverse von ${\,}_S T_B\;$ (siehe \ref[content_15_inverse_matrix][Berechnung der inversen Matrix]{sec:berechnung})
    % \ref[17_cramersche_regel][Cramerschen Regel]{cramersche_regel}
    \[({\,}_S T_B)^{-1} = \begin{pmatrix}-5 & -6 & 3 \\ 2 & 1 & -1 \\ 0 & 1 &0\end{pmatrix}.
    \]

    Eingesetzt in (2) erhalten wir somit gesuchten Koordinatenvektor ${\,}_B v$ \, von $v$ zur Basis $B$ 
    \[
    {\,}_B v= ({\,}_S T_B)^{-1} \cdot v = \begin{pmatrix}-5 & -6 & 3 \\ 2 & 1 & -1 \\ 0 & 1 &0\end{pmatrix} \cdot \begin{pmatrix}2 \\ -1 \\ 3\end{pmatrix}
    = \begin{pmatrix}5 \\ 0 \\ -1\end{pmatrix}.
    \]}
    \lang{en}{Now we determine the inverse matrix of ${\,}_S T_B\;$ (see \ref[content_15_inverse_matrix][Caculation of the inverse matrix]{sec:berechnung})
    % \ref[17_cramersche_regel][Cramerschen Regel]{cramersche_regel}
    \[({\,}_S T_B)^{-1} = \begin{pmatrix}-5 & -6 & 3 \\ 2 & 1 & -1 \\ 0 & 1 &0\end{pmatrix}.
    \]

    We insert this into equation (2) and get the searched coordinate vector ${\,}_B v$ \, of $v$ with respect of the basis $B$ 
    \[
    {\,}_B v= ({\,}_S T_B)^{-1} \cdot v = \begin{pmatrix}-5 & -6 & 3 \\ 2 & 1 & -1 \\ 0 & 1 &0\end{pmatrix} \cdot \begin{pmatrix}2 \\ -1 \\ 3\end{pmatrix}
    = \begin{pmatrix}5 \\ 0 \\ -1\end{pmatrix}.
    \]}

% Alternativer (vereinfachter) Lösungsweg:
\lang{de}{
    \textbf{Hinweis:} \, Alternativ kann man die Gleichung (2) auch weiter umformen, indem man beide Seiten der Gleichung von links mit der
    der Matrix ${\,}_S T_B$ multipliziert. Man erhält dann mit ${\,}_S T_B \cdot {\,}_B v =  v$  
    ein Lineares Gleichungssystem
    \[
    \begin{pmatrix}1 & 3 & 3 \\ 0 & 0 & 1 \\ 2 & 5 &7\end{pmatrix} \cdot \begin{pmatrix}x \\ y \\ z\end{pmatrix} = \begin{pmatrix}5 \\ 0 \\ -1\end{pmatrix},    
    \]
    dessen Lösung ${\,}_B v = \begin{pmatrix}5 \\0 \\ -1\end{pmatrix}$ \, ist.}
\lang{en}{
    \textbf{Note:} \, Alternatively we may continue transforming equation (2) by multiplying both sides of the
    equation with the matrix ${\,}_S T_B$ from the left side. Then we get with ${\,}_S T_B \cdot {\,}_B v =  v$  
    a system of linear equation
    \[
    \begin{pmatrix}1 & 3 & 3 \\ 0 & 0 & 1 \\ 2 & 5 &7\end{pmatrix} \cdot \begin{pmatrix}x \\ y \\ z\end{pmatrix} = \begin{pmatrix}5 \\ 0 \\ -1\end{pmatrix},    
    \]
    which has the solution ${\,}_B v = \begin{pmatrix}5 \\0 \\ -1\end{pmatrix}$ \,.}

  \end{incremental}

\end{tabs*}

\end{content}