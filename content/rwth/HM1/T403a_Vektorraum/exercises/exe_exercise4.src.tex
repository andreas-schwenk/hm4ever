\documentclass{mumie.element.exercise}
%$Id$
\begin{metainfo}
  \name{
    \lang{de}{Ü04: Lineare Abbildungen}
    \lang{en}{Ex04: Linear maps}
  }
  \begin{description} 
 This work is licensed under the Creative Commons License Attribution 4.0 International (CC-BY 4.0)   
 https://creativecommons.org/licenses/by/4.0/legalcode 

    \lang{de}{Hier die Beschreibung}
    \lang{en}{}
  \end{description}
  \begin{components}
  \end{components}
  \begin{links}
      \link{generic_article}{content/rwth/HM1/T403a_Vektorraum/g_art_content_10b_lineare_abb.meta.xml}{10b_lineare_abb}
  \end{links}
  \creategeneric
\end{metainfo}
\begin{content}
\begin{block}[annotation]
	Im Ticket-System: \href{https://team.mumie.net/issues/14367}{Ticket 14367}
\end{block}
\begin{block}[annotation]
	Im Ticket-System: \href{http://team.mumie.net/issues/}{Ticket }
\end{block}

\title{
\lang{de}{Ü04: Lineare Abbildungen}
\lang{en}{Ex04: Linear maps}
}

 
\lang{de}{Eine Basis des $\R^3$ sei gegeben durch 
$C:=\left\{ \begin{pmatrix} 1 \\ 0 \\ 2 \end{pmatrix}; 
\begin{pmatrix} 3 \\ 0 \\ 5 \end{pmatrix}; 
\begin{pmatrix} 3 \\ 1\\ 7 \end{pmatrix} \right\}$ und eine Basis des $\R^2$ sei gegeben durch
$B:=\left\{ \begin{pmatrix} 1 \\ 3\end{pmatrix}; \begin{pmatrix} 2 \\ 7 \end{pmatrix} \right\}$.
Wir betrachten die lineare Abbildung
\[
f \colon \R^3 \to \R^2, \begin{pmatrix}x \\ y \\ z \end{pmatrix} \mapsto \begin{pmatrix}2 & -1 & 0 \\ -3 & 6 & 5\end{pmatrix}\cdot \begin{pmatrix}x \\ y \\ z \end{pmatrix}.
\]
Bestimmen Sie ${\,}_B f_C$.
}

\lang{en}{A basis of $\R^3$ is given by
$C:=\left\{ \begin{pmatrix} 1 \\ 0 \\ 2 \end{pmatrix}; 
\begin{pmatrix} 3 \\ 0 \\ 5 \end{pmatrix}; 
\begin{pmatrix} 3 \\ 1\\ 7 \end{pmatrix} \right\}$ and a basis of $\R^2$ is given by
$B:=\left\{ \begin{pmatrix} 1 \\ 3\end{pmatrix}; \begin{pmatrix} 2 \\ 7 \end{pmatrix} \right\}$.
We consider the linear map
\[
f \colon \R^3 \to \R^2, \begin{pmatrix}x \\ y \\ z \end{pmatrix} \mapsto \begin{pmatrix}2 & -1 & 0 \\ -3 & 6 & 5\end{pmatrix}\cdot \begin{pmatrix}x \\ y \\ z \end{pmatrix}.
\]
Determine ${\,}_B f_C$.
}
\begin{tabs*}[\initialtab{0}\class{exercise}]
  \tab{
  \lang{de}{Antwort}
  \lang{en}{Answer}
  }
\[
    \begin{pmatrix}0 & 10 & -29 \\ 1 & -2 & 17\end{pmatrix}
    \]

  \tab{
  \lang{de}{Lösung}
  \lang{en}{Solution}}
  
  \begin{incremental}[\initialsteps{1}]
    \step \lang{de}{Wir müssen zunächst $f(v)$ für $v \in C$ bestimmen und dann davon die Koordinatendarstellung ausrechnen.}
    \lang{en}{First of all we need to determine $f(v)$ for $v \in C$ and then calculate their coordinate representation.}
    \[
%    f\left( \begin{pmatrix} 1 \\ 0 \\ 2 \end{pmatrix} \right)= \begin{pmatrix}2 \\ 7\end{pmatrix},
     f( \begin{pmatrix} 1 \\ 0 \\ 2 \end{pmatrix} )= \begin{pmatrix}2 \\ 7\end{pmatrix},
    \]
    \[
%    f\left( \begin{pmatrix} 3 \\ 0 \\ 5 \end{pmatrix} \right) = \begin{pmatrix}6 \\ 16\end{pmatrix},
     f( \begin{pmatrix} 3 \\ 0 \\ 5 \end{pmatrix} ) = \begin{pmatrix}6 \\ 16\end{pmatrix},
    \]
    \[
%    f\left( \begin{pmatrix} 3 \\ 1\\ 7 \end{pmatrix} \right) = \begin{pmatrix}5 \\ 32 \end{pmatrix}
     f( \begin{pmatrix} 3 \\ 1\\ 7 \end{pmatrix} ) = \begin{pmatrix}5 \\ 32 \end{pmatrix}
    \]
    \lang{de}{Damit ist \[{\,}_S f_C =\begin{pmatrix}2 & 6 & 5\\ 7 & 16 & 32\end{pmatrix}.\]}
    \lang{en}{With that, it is \[{\,}_S f_C =\begin{pmatrix}2 & 6 & 5\\ 7 & 16 & 32\end{pmatrix}.\]}
    
    \step \lang{de}{Man liest die Basiswechsel-Matrix
    \[
    {\,}_S T_B= \begin{pmatrix}1 & 2\\ 3 & 7\end{pmatrix}
    \]
    ab.}
    \lang{en}{We read off the basis transformation matrix
    \[
    {\,}_S T_B= \begin{pmatrix}1 & 2\\ 3 & 7\end{pmatrix}.
    \]
    }
    \step \lang{de}{Nun gilt nach den \ref[10b_lineare_abb][Regeln zum Basiswechsel]{regel_bw}:
    \[
    {\,}_B f_C= {\,}_B T_S \cdot {\,}_S f_C = ({\,}_S T_B)^{-1} \cdot {\,}_S f_C .
    \]
    Die Inverse ist
    \[
    ({\,}_S T_B)^{-1} = \begin{pmatrix}7 & -2 \\ -3 & 1 \end{pmatrix}.
    \]
    Wir erhalten das Ergebnis
    \[
    \begin{pmatrix}7 & -2 \\ -3 & 1 \end{pmatrix} \cdot \begin{pmatrix}2 & 6 & 5 \\ 7 & 16 & 32\end{pmatrix}
    = \begin{pmatrix}0 & 10 & -29 \\ 1 & -2 & 17\end{pmatrix}
    \]}
    \lang{en}{Now it holds because of the \ref[10b_lineare_abb][rules for the basis transformation]{regel_bw}:
    \[
    {\,}_B f_C= {\,}_B T_S \cdot {\,}_S f_C = ({\,}_S T_B)^{-1} \cdot {\,}_S f_C .
    \]
    The inverse matrixi is
    \[
    ({\,}_S T_B)^{-1} = \begin{pmatrix}7 & -2 \\ -3 & 1 \end{pmatrix}.
    \]
    We get the result
    \[
    \begin{pmatrix}7 & -2 \\ -3 & 1 \end{pmatrix} \cdot \begin{pmatrix}2 & 6 & 5 \\ 7 & 16 & 32\end{pmatrix}
    = \begin{pmatrix}0 & 10 & -29 \\ 1 & -2 & 17\end{pmatrix}
    \]}
  \end{incremental}

\end{tabs*}

\end{content}