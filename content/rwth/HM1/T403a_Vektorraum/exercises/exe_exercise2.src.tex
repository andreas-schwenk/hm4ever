\documentclass{mumie.element.exercise}
%$Id$
\begin{metainfo}
  \name{
    \lang{de}{Ü02: Basen/Koordinatenvektoren}
    \lang{en}{Ex02: Basis/Coordinate vectors}
  }
  \begin{description} 
 This work is licensed under the Creative Commons License Attribution 4.0 International (CC-BY 4.0)   
 https://creativecommons.org/licenses/by/4.0/legalcode 

    \lang{de}{Hier die Beschreibung}
    \lang{en}{}
  \end{description}
  \begin{components}
  \end{components}
  \begin{links}
  \end{links}
  \creategeneric
\end{metainfo}
\begin{content}
\begin{block}[annotation]
	Im Ticket-System: \href{https://team.mumie.net/issues/14369}{Ticket 14369}
\end{block}
\begin{block}[annotation]
	Im Ticket-System: \href{http://team.mumie.net/issues/}{Ticket }
\end{block}

\title{
\lang{de}{Ü02: Basen/Koordinatenvektoren}
}
\title{
\lang{en}{Ex02: Basis/Coordinate vectors}
}

 
\lang{de}{Gegeben sei der Vektorraum der Polynome mit der Basis $\{x \mapsto 1; x \mapsto x+1; x \mapsto 2x^2-x \}$.\\
Bestimmen Sie den Koordinatenvektor des Polynoms $p(x)=6x^2-7x+16$ bezüglich der angegebenen Basis.}
\lang{en}{Given is the vector space of polynomials with the basis $\{x \mapsto 1; x \mapsto x+1; x \mapsto 2x^2-x \}$.\\
Determine the coordinate vector of the polynom $p(x)=6x^2-7x+16$ with respect of the given basis.}

\begin{tabs*}[\initialtab{0}\class{exercise}]
  \tab{
  \lang{de}{Antwort}
  \lang{en}{Answer}
  }
\[
    \begin{pmatrix}20 \\ -4 \\ 3\end{pmatrix}.
    \]

  \tab{
  \lang{de}{Lösung}\lang{en}{Solution}}
  
  \begin{incremental}[\initialsteps{1}]
    \step \lang{de}{Wir suchen also eine Lösung für $a,b,c \in \R$ der Gleichung
    \[
     p(x) = a \cdot 1 + b \cdot (x+1) + c \cdot (2x^2-x).
    \]}
    \lang{en}{We are looking for a solution for $a,b,c \in \R$ to the equation
    \[
     p(x) = a \cdot 1 + b \cdot (x+1) + c \cdot (2x^2-x).
    \]}
    \step \lang{de}{Wir führen einen Koeffizientenvergleich durch     
    \begin{align*}
    \textcolor{#CC6600}{6} \cdot x^2 + \textcolor{#0066CC}{(-7)}\cdot x + \textcolor{#00CC00}{16}        
              &= a \cdot 1 + b \cdot (x+1) + c \cdot (2x^2-x)  \\
              &= \textcolor{#CC6600}{2c} \cdot x^2 +  \textcolor{#0066CC}{(b-c)} \cdot x + \textcolor{#00CC00}{(a+b)},
    \end{align*}
    der uns die folgenden drei Gleichungen liefert:
    \begin{align*}
       6 &= 2c \\
      -7 &= b - c \\
      16 &= a + b .
    \end{align*}}
    \lang{en}{We perform a coefficient comparision 
    \begin{align*}
    \textcolor{#CC6600}{6} \cdot x^2 + \textcolor{#0066CC}{(-7)}\cdot x + \textcolor{#00CC00}{16}        
              &= a \cdot 1 + b \cdot (x+1) + c \cdot (2x^2-x)  \\
              &= \textcolor{#CC6600}{2c} \cdot x^2 +  \textcolor{#0066CC}{(b-c)} \cdot x + \textcolor{#00CC00}{(a+b)},
    \end{align*}
    which provides the following three equations:
    \begin{align*}
       6 &= 2c \\
      -7 &= b - c \\
      16 &= a + b .
    \end{align*}}
    \step \lang{de}{Aus der ersten Gleichung folgt sofort $c = 3$. Die zweite Gleichung wird damit zu $-7 = b -3$, was äquivalent zu $b=-4$ ist.
    $b$ eingesetzt in die dritte Gleichung liefert $16 = a - 4 \Leftrightarrow a = 20$.}
       \lang{en}{From the first equation results $c = 3$. With that the second equation transforms to $-7 = b -3$, which is equivalent to $b=-4$.
    We insert $b$ in the third equation and get $16 = a - 4 \Leftrightarrow a = 20$.}
    
    \step \lang{de}{Der Koordinatenvektor ist somit gegeben durch
    \[
    \begin{pmatrix}20 \\ -4 \\ 3\end{pmatrix}.
    \]}
    \lang{en}{So the coordinate vector is given by
    \[
    \begin{pmatrix}20 \\ -4 \\ 3\end{pmatrix}.
    \]}
  \end{incremental}

\end{tabs*}

\end{content}