\documentclass{mumie.element.exercise}
%$Id$
\begin{metainfo}
  \name{
   \lang{de}{Ü06: Basen v. Untervektorräumen}
   \lang{en}{Ex06: Basis of vector subspaces}
 }
 \begin{description} 
 This work is licensed under the Creative Commons License Attribution 4.0 International (CC-BY 4.0)   
 https://creativecommons.org/licenses/by/4.0/legalcode 

   \lang{de}{Aufgabe 6}
   \lang{en}{Description here}
 \end{description}
 \begin{components}
 \end{components}
 \begin{links}
\link{generic_article}{content/rwth/HM1/T402_Lineare_Gleichungssysteme/g_art_content_05_gaussverfahren.meta.xml}{gaussverfahren}
\link{generic_article}{content/rwth/HM1/T403a_Vektorraum/g_art_content_10a_vektorraum.meta.xml}{vektorraum}
\end{links}
 \creategeneric
 \end{metainfo}
 \begin{content}
\begin{block}[annotation]
	Im Ticket-System: \href{https://team.mumie.net/issues/15909}{Ticket 15909}
\end{block}
   \title{
    \lang{de}{Ü06: Basen von Untervektorräumen}
    \lang{en}{Ex06: Basis of vector subspaces}
   }
   \lang{de}{Gegeben sei die Matrix $A=\begin{pmatrix}3 & 4 & 1 & 0\\ 0 & 1 & 2 & 1 \\ -3 & -2 & 3 & 2
   \end{pmatrix}$.
   Bestimmen Sie eine Basis der Lösungsmenge
   $\mathbb{L}=\{ x \in \R^4 |\, A \cdot x = 0\}$.}
    \lang{en}{Given is a matrix $A=\begin{pmatrix}3 & 4 & 1 & 0\\ 0 & 1 & 2 & 1 \\ -3 & -2 & 3 & 2
   \end{pmatrix}$.
   Determine a basis of the solution set
   $\mathbb{L}=\{ x \in \R^4 |\, A \cdot x = 0\}$.}

 \begin{tabs*}[\initialtab{0}\class{exercise}]
   \tab{
   \lang{de}{Antwort}
   \lang{en}{Answer}
  }
\lang{de}{
Zum Beispiel $\left\{\begin{pmatrix}\frac{7}{3}\\-2\\ 1 \\ 0\end{pmatrix};
\begin{pmatrix}\frac{4}{3}\\-1\\ 0 \\ 1\end{pmatrix}\right\}$.}
\lang{en}{
For example $\left\{\begin{pmatrix}\frac{7}{3}\\-2\\ 1 \\ 0\end{pmatrix};
\begin{pmatrix}\frac{4}{3}\\-1\\ 0 \\ 1\end{pmatrix}\right\}$.}

 \tab{
   \lang{de}{Lösung}
   \lang{en}{Solution}
} \begin{incremental}[\initialsteps{1}]

    \step 
     \lang{de}{Wie in der Vorlesung in den \ref[vektorraum][Beispielen zu Vektorräumen]{ex:LGS_UnterVR} erwähnt, bildet die Lösungsmenge $\mathbb{L}$ 
     eines homogenen LGS $Ax=0 \,$ mit $3$ Gleichungen und $4$ Variablen einen Untervektorraum des $\R^4$.       
     
%     Wir gehen wie im Vorlesungstext angegeben vor.
     
%     Zunächst führen wir den Gauß-Algorithmus aus, um die Lösungsmenge von $A\cdot x=0$ zu bestimmen. Es ist also ein homogenes LGS zu lösen.

     Wir bestimmen also zunächst die Lösungsmenge von $A\cdot x=0$ durch Anwendung des \ref[gaussverfahren][Gauß-Algorithmus.]{def:Gauß-Verfahren}
     Dabei können wir die rechte Seite einfach weglassen, da diese dem Nullvektor entspricht.
      
%     Die rechte Seite ist der Nullvektor. Dieser kann auch weggelassen werden.
     }

         \lang{en}{Like we have seen in the lecture during the \ref[vektorraum][examples for vector spaces]{ex:LGS_UnterVR}, the solution set $\mathbb{L}$ 
     of a homogeneous linear system $Ax=0 \,$ with $3$ equations and $4$ variables is a vector subspace of $\R^4$.       
     
%     Wir gehen wie im Vorlesungstext angegeben vor.
     
%     Zunächst führen wir den Gauß-Algorithmus aus, um die Lösungsmenge von $A\cdot x=0$ zu bestimmen. Es ist also ein homogenes LGS zu lösen.

     First of all we determine the solut set of $A\cdot x=0$ with the help of \ref[gaussverfahren][Gaussian elimination]{def:Gauß-Verfahren}.
     We may omit the right side, because it is equal to the zero-vector.
      
%     Die rechte Seite ist der Nullvektor. Dieser kann auch weggelassen werden.
     }
     
       \step \begin{align*}
       &\begin{pmatrix}3 & 4 & 1 & 0\\ 0 & 1 & 2 & 1 \\ -3 & -2 & 3 & 2
   \end{pmatrix} \\
   \rightsquigarrow & \begin{pmatrix} 3 & 4 & 1 & 0 \\ 0 & 1 & 2 & 1 \\ 0 & 2 & 4 & 2\end{pmatrix} \\
   \rightsquigarrow & \begin{pmatrix} 3 & 4 & 1 & 0 \\ 0 & 1 & 2 & 1 \\ 0 & 0 & 0 & 0\end{pmatrix}\\
   \rightsquigarrow & \begin{pmatrix} 3 & 0 & -7 & -4 \\ 0 & 1 & 2 & 1 \\ 0 & 0 & 0 & 0\end{pmatrix}\\
   \rightsquigarrow & \begin{pmatrix} 1 & 0 & -\frac{7}{3}& -\frac{4}{3}
\\ 0 & 1& 2 & 1\\ 0& 0& 0 & 0\end{pmatrix}       \end{align*}

\step \lang{de}{Die letzte Matrix ist in Stufenform. Wir erkennnen, dass der Rang der Matrix $A$ zwei ist. Damit sind zwei Variablen von $x \in \R^4$ frei wählbar.
Wir setzten $x_3=s$ und $x_4=t$. Eine andere Wahl ist auch möglich und liefert eine andere Basis, die aber mit einer passenden
Basiswechselmatrix in diese überführt werden kann.}
\lang{en}{The last matrix is in row echelon form. We see, that the rank of the matrix is $2$.
So two of the variables $x \in \R^4$ are free.
We set $x_3=s$ and $x_4=t$. You may also choose different values for the variables, which results
in a different basis. But with a basis transformation matrix wecan
transform the basis in the basis given here.}

\step \lang{de}{Aus der ersten Zeile der Matrix in Stufenform erhalten wir die Gleichung
\[
x_1 -\frac{7}{3}s-\frac{4}{3}t= 0,
\]
also
\[
x_1 = \frac{7}{3}s+\frac{4}{3}t.
\]
Aus der zweiten Zeile der Matrix in Stufenform erhalten wir
\[ x_2+2s+t=0,\]
also \[x_2=-2s-t.\]}
\lang{en}{The following equation results from the first row of the matrix in row echelon form:
\[
x_1 -\frac{7}{3}s-\frac{4}{3}t= 0,
\]
so
\[
x_1 = \frac{7}{3}s+\frac{4}{3}t.
\]
With the second row of the matrix we get
\[ x_2+2s+t=0,\]
so \[x_2=-2s-t.\]}

\step \lang{de}{Somit können wir die Lösungsmenge in der folgenden Form schreiben
\begin{align*}
\mathbb{L}&=\left\{ \begin{pmatrix}\frac{7}{3}s+\frac{4}{3}t \\ -2s-t \\ s\\t\end{pmatrix}
|\, s,t\in \R \right\}\\
&= \left\{ s\cdot \begin{pmatrix}\frac{7}{3}\\-2\\1 \\ 0\end{pmatrix}+t\cdot \begin{pmatrix}\frac{4}{3}\\-1\\0\\1\end{pmatrix}|\, s,t\in \R\right\}.
\end{align*}
Die Lösungsmenge ist also die Menge aller Linearkombinationen aus den Vektoren $\begin{pmatrix}\frac{7}{3}\\-2\\1 \\ 0\end{pmatrix}$ 
und $\begin{pmatrix}\frac{4}{3}\\-1\\0\\1\end{pmatrix}$. Folglich ist eine Basis von $\mathbb{L}$ gegeben durch
\[
\left\{ \begin{pmatrix}\frac{7}{3}\\-2\\1 \\ 0\end{pmatrix}; \begin{pmatrix}\frac{4}{3}\\-1\\0\\1\end{pmatrix} \right\}.
\]}
\lang{en}{So we can write the solution set in the following form
\begin{align*}
\mathbb{L}&=\left\{ \begin{pmatrix}\frac{7}{3}s+\frac{4}{3}t \\ -2s-t \\ s\\t\end{pmatrix}
|\, s,t\in \R \right\}\\
&= \left\{ s\cdot \begin{pmatrix}\frac{7}{3}\\-2\\1 \\ 0\end{pmatrix}+t\cdot \begin{pmatrix}\frac{4}{3}\\-1\\0\\1\end{pmatrix}|\, s,t\in \R\right\}.
\end{align*}
So the solution set consists of all possible linear combinations of the vectors $\begin{pmatrix}\frac{7}{3}\\-2\\1 \\ 0\end{pmatrix}$ 
and $\begin{pmatrix}\frac{4}{3}\\-1\\0\\1\end{pmatrix}$. Therefore a basis of $\mathbb{L}$ is given by
\[
\left\{ \begin{pmatrix}\frac{7}{3}\\-2\\1 \\ 0\end{pmatrix}; \begin{pmatrix}\frac{4}{3}\\-1\\0\\1\end{pmatrix} \right\}.
\]}
 \end{incremental}

 \end{tabs*}
 \end{content}
