
%$Id:  $
\documentclass{mumie.article}
%$Id$
\begin{metainfo}
  \name{
    \lang{de}{Überblick: Skalarprodukt und Vektorprodukt}
    \lang{en}{overview: }
  }
  \begin{description} 
 This work is licensed under the Creative Commons License Attribution 4.0 International (CC-BY 4.0)   
 https://creativecommons.org/licenses/by/4.0/legalcode 

    \lang{de}{Beschreibung}
    \lang{en}{}
  \end{description}
  \begin{components}
  \end{components}
  \begin{links}
\link{generic_article}{content/rwth/HM1/T109_Skalar-_und_Vektorprodukt/g_art_content_34_vektorprodukt.meta.xml}{content_34_vektorprodukt}
\link{generic_article}{content/rwth/HM1/T109_Skalar-_und_Vektorprodukt/g_art_content_33_winkel.meta.xml}{content_33_winkel}
\link{generic_article}{content/rwth/HM1/T109_Skalar-_und_Vektorprodukt/g_art_content_32_laenge_norm.meta.xml}{content_32_laenge_norm}
\link{generic_article}{content/rwth/HM1/T109_Skalar-_und_Vektorprodukt/g_art_content_31_skalarprodukt.meta.xml}{content_31_skalarprodukt}
\end{links}
  \creategeneric
\end{metainfo}
\begin{content}
\begin{block}[annotation]
	Im Ticket-System: \href{https://team.mumie.net/issues/30140}{Ticket 30140}
\end{block}



\begin{block}[annotation]
Im Entstehen: Überblicksseite für Kapitel Skalarprodukt und Vektorprodukt
\end{block}

\usepackage{mumie.ombplus}
\ombchapter{1}
\title{\lang{de}{Überblick: Skalarprodukt und Vektorprodukt}
       \lang{en}{Overview: Scalar product and cross product}}




\begin{block}[info-box]
\lang{de}{\strong{Inhalt}}
\lang{en}{\strong{Contents}}


\lang{de}{
    \begin{enumerate}%[arabic chapter-overview]
   \item[9.1] \link{content_31_skalarprodukt}{Skalarprodukt}
   \item[9.2] \link{content_32_laenge_norm}{Länge von Vektoren}
   \item[9.3] \link{content_33_winkel}{Winkel zwischen Vektoren}
   \item[9.4] \link{content_34_vektorprodukt}{Vektorprodukt}
     \end{enumerate}
}
\lang{en}{
    \begin{enumerate}%[arabic chapter-overview]
   \item[9.1] \link{content_31_skalarprodukt}{Scalar product}
   \item[9.2] \link{content_32_laenge_norm}{Length of a vector}
   \item[9.3] \link{content_33_winkel}{Angles between vectors}
   \item[9.4] \link{content_34_vektorprodukt}{Vector product}
     \end{enumerate}
} %lang

\end{block}

\begin{zusammenfassung}
\lang{de}{
Wir erweitern unser Wissen über Vektoren aus dem letzten Kapitel nun um zwei neue Operationen.
\\\\
Das Skalarprodukt zweier Vektoren erweist sich als wichtiges Hilfsmittel, um zum Beispiel Abstände von Punkten zu bestimmen oder Winkel zwischen Geraden zu berechnen. %Wir entdecken den bereits bekannten Satz von Pythagoras wieder und betrachten die Eigenschaften der Norm.
\\\\
Als zweite neue Operation lernen wir das Vektorprodukt kennen. Nun sind wir in der Lage im dreidimensionalen Raum einen Vektor zu finden, der zu zwei gegebenen Vektoren senkrecht steht.
\\\\
Wir schließen das Kapitel mit Rechenregeln und einer geometrischen Anwendung ab.
}
\lang{en}{
We further our understanding of vectors from the previous chapter by introducing two new operations.
\\\\
The scalar product of two vectors is a helpful tool for finding the distance between points and for 
finding the angle between lines.
\\\\
The second new operation that we introduce is the cross product. Using this, we can find a vector 
that is orthogonal to two given vectors in three-dimensional space.
\\\\
We conclude the chapter with some rules and properties of the operations, and a geometric application.
}
\end{zusammenfassung}

\begin{block}[info]\strong{\lang{de}{Lernziele}\lang{en}{Learning Goals}}

\begin{itemize}[square]
\item \lang{de}{Sie berechnen das Skalarprodukt zu gegebenen Vektoren.}
      \lang{en}{Being able to calculate the scalar product of given vectors.}
\item \lang{de}{Sie bestimmen die Länge von Vektoren.}
      \lang{en}{Being able to find the length/magnitude of a vector.}
\item \lang{de}{Sie berechnen den Abstand von Punkten im euklidischen Raum.}
      \lang{en}{Being able to calculate the distance between points in Euclidean space.}
\item \lang{de}{Sie entscheiden, ob gegebene Vektoren orthogonal zueinander stehen.}
      \lang{en}{Being able to determine if given vectors are orthogonal.}
\item \lang{de}{Sie bestimmen den Winkel zwischen Vektoren.}
      \lang{en}{Being able to find the angle between vectors.}
\item \lang{de}{Sie wenden das Vektorprodukt in geometrischen Aufgabenstellungen an.}
      \lang{en}{Being able to apply the cross product in a geometric context.}
\end{itemize}
\end{block}

\end{content}
