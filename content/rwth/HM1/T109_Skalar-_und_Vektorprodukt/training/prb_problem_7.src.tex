\documentclass{mumie.problem.gwtmathlet}
%$Id$
\begin{metainfo}
  \name{
    \lang{de}{A08: Vektorprodukt}
    \lang{en}{mc yes-no}
  }
  \begin{description} 
 This work is licensed under the Creative Commons License Attribution 4.0 International (CC-BY 4.0)   
 https://creativecommons.org/licenses/by/4.0/legalcode 

    \lang{de}{Beschreibung}
    \lang{en}{description}
  \end{description}
  \corrector{system/problem/GenericCorrector.meta.xml}
  \begin{components}
    \component{generic_image}{content/rwth/HM1/images/g_img_1_9_Vektorprodukt_1.meta.xml}{1_9_Vektorprodukt_1}
    \component{js_lib}{system/problem/GenericMathlet.meta.xml}{mathlet}
  \end{components}
  \begin{links}
  \end{links}
  \creategeneric
\end{metainfo}
\begin{content}

\usepackage{mumie.genericproblem}
\lang{de}{
	\title{A08: Vektorprodukt}
}

\begin{block}[annotation]
  Im Ticket-System: \href{http://team.mumie.net/issues/9717}{Ticket 9717}
\end{block}
\begin{problem}

	\begin{question}
		\lang{de}{\text{
			Berechnen Sie
  			$
   			\vec{x} = \begin{pmatrix} x_1 \\ x_2 \\ x_3 \end{pmatrix} = \begin{pmatrix} \var{z_1}\\ \var{z_2}\\ \var{z_3} \end{pmatrix} \times \begin{pmatrix}\var{u_1}\\ \var{u_2}\\ \var{u_3} \end{pmatrix}.
  			$
			}
			}
              \explanation[edited]{Verwenden Sie zur Berechnung des Vektorprodukts zweier Vektoren $\vec{v}$ und $\vec{w}$ das folgende Schema:\\
              \image{1_9_Vektorprodukt_1}}
              
		\type{input.number}
      	\field{rational} 
 
      \begin{variables}
            
            \randint{z_1}{-10}{10}
            \randint{z_2}{-10}{10}
            \randint{z_3}{-10}{10}
            
            \randint{u_1}{-10}{10}
            \randint{u_2}{-10}{10}
            \randint{u_3}{-10}{10}
            
            \function[calculate]{x1}{z_2*u_3-u_2*z_3}
            \function[calculate]{x2}{z_3*u_1-u_3*z_1}
            \function[calculate]{x3}{z_1*u_2-u_1*z_2}
      \end{variables}
      \begin{answer}
            \text{$x_1=$}
            \solution{x1}
      \end{answer}

      \begin{answer}
            \text{$x_2=$}
            \solution{x2}
      \end{answer}
             
       \begin{answer}
       \text{$x_3=$}
       \solution{x3}
      \end{answer}
\end{question}

	\begin{question}
		\lang{de}{\text{
			Für welche Werte von $a, b, c \in \R$ ~ gilt:\\\\
  			$
   			\begin{pmatrix} a\\ \var{z_2}\\ b \end{pmatrix} \times \begin{pmatrix}\var{u_1}\\ c\\ \var{u_3} \end{pmatrix} = \begin{pmatrix} \var{x_1} \\ \var{x_2} \\ \var{x_3} \end{pmatrix}
            $ ? \\
			}
			}
              \explanation[edited]{Bestimmen Sie das Vektorprodukt der beiden linken Vektoren in Abhängigkeit der drei Parameter $a,b$ und $c$. 
              Setzen Sie den sich ergebenden Vektor gleich dem Vektor der rechten Seite. Lösen Sie anschließend das entstandene Gleichungssystem.}
         
		\type{input.number}
      	\field{rational} 
 
      \begin{variables}
            
            \randint{z_1}{-5}{5}
            \randint{z_2}{-5}{5}
            \randint{z_3}{-5}{5}
            
            \randint{u_1}{-5}{5}
            \randint[Z]{u_2}{-5}{5}
            \randint{u_3}{-5}{5}
            
            \randadjustIf{z_1,z_3,u_1,u_3}{z_3*u_1-u_3*z_1=0}
            
            \function[calculate]{x1}{z_2*u_3-u_2*z_3}
            \function[calculate]{x2}{z_3*u_1-u_3*z_1}
            \function[calculate]{x3}{z_1*u_2-u_1*z_2}

      \end{variables}
      
      \begin{answer}
            \text{$a=$}
            \solution{z_1}
      \end{answer}

      \begin{answer}
            \text{$b=$}
            \solution{z_3}
      \end{answer}

      \begin{answer}
            \text{$c=$}
            \solution{u_2}
      \end{answer}

\end{question}

\end{problem}

\embedmathlet{mathlet}



\end{content}