\documentclass{mumie.problem.gwtmathlet}
%$Id$
\begin{metainfo}
  \name{
    \lang{de}{A01: Skalarprodukt}
    \lang{en}{problem_1}
  }
  \begin{description} 
 This work is licensed under the Creative Commons License Attribution 4.0 International (CC-BY 4.0)   
 https://creativecommons.org/licenses/by/4.0/legalcode 

    \lang{de}{Vergleichen von Zahlen und einfache Addition}
    \lang{en}{}
  \end{description}
  \corrector{system/problem/GenericCorrector.meta.xml}
  \begin{components}
    \component{js_lib}{system/problem/GenericMathlet.meta.xml}{mathlet}
  \end{components}
  \begin{links}
  \end{links}
  \creategeneric
\end{metainfo}
\begin{content}
\usepackage{mumie.ombplus}
\usepackage{mumie.genericproblem}


\lang{de}{
	\title{A01: Skalarprodukt}
}
\lang{en}{
	\title{Problem 1}
}

\begin{block}[annotation]
  Im Ticket-System: \href{http://team.mumie.net/issues/9711}{Ticket 9711}
\end{block}

\begin{problem}
 
    \begin{question}
      \lang{de}{\text{Bestimmen Sie folgende Skalarprodukte.\\
		 a)
		$
 		\begin{pmatrix} \var{x_1} \\ \var{x_2} \end{pmatrix} \bullet \begin{pmatrix}\var{y_1}\\ \var{y_2} \end{pmatrix} = $\ansref
		\\
		  b)
		$
		 \left(\begin{pmatrix} \var{x_1}\\ \var{x_2} \end{pmatrix}+\begin{pmatrix}\var{y_1} \\ \var{y_2} \end{pmatrix}\right) \bullet \left(\begin{pmatrix}\var{z_1}\\ \var{z_2} \end{pmatrix}-\begin{pmatrix}\var{u_1}\\ \var{u_2} \end{pmatrix}\right) =$ \ansref
		}
		}
            
      \type{input.number}
      \field{rational} 
 
      \begin{variables}
            \randint[Z]{x_1}{-10}{10}
            \randint[Z]{x_2}{-10}{10}
            \randint[Z]{y_1}{-10}{10}
            \randint[Z]{y_2}{-10}{10}
            \randint{u_1}{-10}{10}
            \randint{u_2}{-10}{10}
            \randint{z_1}{-10}{10}
            \randint{z_2}{-10}{10}
            \function{ea}{x_1*y_1+x_2*y_2}%Für explanation
            \function[calculate]{aa}{x_1*y_1+x_2*y_2}
            \function[calculate]{bb}{(x_1+y_1)*(z_1-u_1)+(x_2+y_2)*(z_2-u_2)}
      \end{variables}
      \begin{answer}
            \text{a)}
            \solution{aa}
            \explanation[edited]{Das Skalarprodukt zweier Vektoren berechnen Sie, indem Sie den ersten Eintrag des ersten Vektors mit dem ersten Eintrag
            des zweiten Vektors multiplizieren und derart für alle weiteren Einträge fortfahren. Dann bilden Sie die Summe aus allen Produkten. Das Skalarprodukt in a) ist gegeben durch $\var{ea}$.}
      \end{answer}

      \begin{answer}
            \text{b)}
            \solution{bb}
            \explanation[edited]{Addieren bzw. subtrahieren Sie zunächst die beiden Vektoren in den Klammern. Von den resultierenden Vektoren bilden Sie das Skalarprodukt.}
      \end{answer}
             
\end{question}



\end{problem}

\embedmathlet{mathlet}



\end{content}