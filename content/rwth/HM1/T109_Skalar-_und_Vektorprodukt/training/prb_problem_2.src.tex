\documentclass{mumie.problem.gwtmathlet}
%$Id$
\begin{metainfo}
  \name{
    \lang{de}{A02: Skalarprodukt}
    \lang{en}{input numbers}
  }
  \begin{description} 
 This work is licensed under the Creative Commons License Attribution 4.0 International (CC-BY 4.0)   
 https://creativecommons.org/licenses/by/4.0/legalcode 

    \lang{de}{Die Beschreibung}
    \lang{en}{}
  \end{description}
  \corrector{system/problem/GenericCorrector.meta.xml}
  \begin{components}
    \component{js_lib}{system/problem/GenericMathlet.meta.xml}{mathlet}
  \end{components}
  \begin{links}
  \end{links}
  \creategeneric
\end{metainfo}
\begin{content}

\usepackage{mumie.genericproblem}
\lang{de}{
	\title{A02: Skalarprodukt}
}

\begin{block}[annotation]
  Im Ticket-System: \href{http://team.mumie.net/issues/9712}{Ticket 9712}
\end{block}
\begin{problem}

	\begin{question}
		\lang{de}{
			\text{Bestimmen Sie folgende Skalarprodukte. \\
			a)
			$\begin{pmatrix}\var{x_1}\\ \var{x_2}\\ \var{x_3} \end{pmatrix} \bullet \begin{pmatrix}\var{y_1}\\ \var{y_2}\\ \var{y_3} \end{pmatrix}=$\ansref \\ 
			b) 
			$\left(\var{r} \cdot\begin{pmatrix} \var{z_1}\\ \var{z_2}\\ \var{z_3} \end{pmatrix}\right) \bullet \begin{pmatrix} \var{x_1}\\ \var{x_2}\\ \var{x_3} \end{pmatrix} =$\ansref
		}
		}
         \explanation{Das Skalarprodukt zweier Vektoren berechnen Sie, indem Sie den ersten Eintrag des ersten Vektors mit dem ersten Eintrag
      des zweiten Vektors multiplizieren und derart für alle weiteren Einträge fortfahren. Dann bilden Sie die Summe aus allen Produkten.}
      
		\type{input.number}
      \field{rational} 
 
      \begin{variables}
            \randint{x_1}{-5}{5}
            \randint{x_2}{-5}{5}
            \randint{x_3}{-5}{5}
            \randint{y_1}{-5}{5}
            \randint{y_2}{-5}{5}
            \randint{y_3}{-5}{5}
            \randint{z_1}{-5}{5}
            \randint{z_2}{-5}{5}
            \randint{z_3}{-5}{5}
       		\randint[Z]{r}{-5}{5}
            \function[calculate]{aa}{x_1*y_1+x_2*y_2+x_3*y_3}
            \function[calculate]{bb}{r*(x_1*z_1+x_2*z_2+x_3*z_3)}
      \end{variables}
      \begin{answer}
            \text{a)}
            \solution{aa}
            \explanation[edited]{Ihre Antwort in a) ist falsch.}
      \end{answer}

      \begin{answer}
            \text{b)}
            \solution{bb}
            \explanation[edited]{Ihre Antwort in b) ist falsch.}
      \end{answer}
             
\end{question}



\end{problem}

\embedmathlet{mathlet}



\end{content}