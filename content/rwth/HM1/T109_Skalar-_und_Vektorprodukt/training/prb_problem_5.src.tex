\documentclass{mumie.problem.gwtmathlet}
%$Id$
\begin{metainfo}
  \name{
    \lang{de}{A05: Abstand}
    \lang{en}{mc yes-no}
  }
  \begin{description} 
 This work is licensed under the Creative Commons License Attribution 4.0 International (CC-BY 4.0)   
 https://creativecommons.org/licenses/by/4.0/legalcode 

    \lang{de}{Beschreibung}
    \lang{en}{description}
  \end{description}
  \corrector{system/problem/GenericCorrector.meta.xml}
  \begin{components}
    \component{js_lib}{system/problem/GenericMathlet.meta.xml}{mathlet}
  \end{components}
  \begin{links}
  \end{links}
  \creategeneric
\end{metainfo}
\begin{content}

\usepackage{mumie.genericproblem}
\lang{de}{
	\title{A05: Abstand}
}

\begin{block}[annotation]
  Im Ticket-System: \href{http://team.mumie.net/issues/9715}{Ticket 9715}
\end{block}

\textit{Hinweis: Benutzen Sie sqrt() um die Wurzel einzugeben.}

\begin{problem}



	\begin{question}
		\lang{de}{\text{
			~\\Bestimmen Sie den Abstand der Punkte $P_1=(\var{x_1};\var{x_2})$ und $P_2=(\var{y_1};\var{y_2})$. Es ist $d(P_1,P_2)=$\ansref \\
			Bestimmen Sie weiterhin den Abstand der Punkte $P_3=(\var{z_1};\var{z_2};\var{z_3})$ und $P_4=(\var{u_1};\var{u_2};\var{u_3})$. Es ist $d(P_3,P_4)=$\ansref \\
		
			}
			}
            
         \explanation{Den Abstand zweier Punkte können Sie bestimmen, indem Sie die Länge des Verbindungsvektors von $P_1$ nach $P_2$ bestimmen.}
         \explanation[edited]{Die Länge des Verbindungsvektors berechnen Sie, indem Sie jeden Eintrag des Verbindungsvektors 
         quadrieren, alle Quadrate aufsummieren und aus der Summe die Wurzel ziehen.}
            
		\type{input.function}
      	\field{real} 
 
      \begin{variables}
            \randint{x_1}{-10}{10}
            \randint{x_2}{-10}{10}
            
            \randint{y_1}{-10}{10}
            \randint{y_2}{-10}{10}
            
            \randint{z_1}{-10}{10}
            \randint{z_2}{-10}{10}
            \randint{z_3}{-10}{10}
            
            \randint{u_1}{-10}{10}
            \randint{u_2}{-10}{10}
            \randint{u_3}{-10}{10}
            
            \function[calculate]{a1}{(y_1-x_1)^2+(y_2-x_2)^2}
            \function[calculate]{b1}{(u_1-z_1)^2+(u_2-z_2)^2+(u_3-z_3)^2}
            
            \function{aa}{sqrt(a1)}
            \function{bb}{sqrt(b1)}
      \end{variables}
      \begin{answer}
            \text{1.) $d(P_1,P_2)=$}
            \solution{aa}
            \inputAsFunction{x}{ax} 
            \checkFuncForZero{ax-aa}{-10}{10}{100}
      \end{answer}

      \begin{answer}
            \text{2.) $d(P_3,P_4)=$}
            \solution{bb}
            \inputAsFunction{y}{by} 
            \checkFuncForZero{by-bb}{-10}{10}{100}
      \end{answer}
             
\end{question}

\begin{question} 
	\lang{de}{
		\text{
      Bestimmen Sie den Parameter $x>0$ so, dass die beiden Punkte $A=(5;\var{b};0)$ und $B=(2;3;x)$ zueinander 
        einen Abstand von $\var{d}$ Einheiten haben.\\\\
        
        Es muss gelten: $x=$\ansref.
                        
        }}
         \explanation{Den Abstand zweier Punkte können Sie bestimmen, indem Sie die Länge des Verbindungsvektors von $P_1$ nach $P_2$ 
         bestimmen. Hier berechnen Sie diese Länge in Abhängigkeit von $x>0$ und setzen die Gleichung gleich dem gesuchten Abstand. 
         Anschließend lösen Sie die Wurzelgleichung durch Quadrieren und Umformen nach $x$ auf.}
         %\explanation[edited]{Die Länge des Verbindungsvektors berechnen Sie, indem sie jeden Eintrag des Verbindungsvektors 
         %quadrieren, alle Quadrate aufsummieren und aus der Summe die Wurzel ziehen.}
            
 
	\begin{variables}
      
       \randint{b}{2}{5}
       \randint{d}{5}{8}     
       \function[normalize]{loes}{sqrt(d^2-b^2+6*b-18)}  
      
    \end{variables}
	\type{input.function}
 	\field{rational}
    
    \begin{answer}
    	\solution{loes}
         \checkAsFunction{x}{-10}{10}{10}
    \end{answer}
   
  \end{question}

\end{problem}

\embedmathlet{mathlet}



\end{content}