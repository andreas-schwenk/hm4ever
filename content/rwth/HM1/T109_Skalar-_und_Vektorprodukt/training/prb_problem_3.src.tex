\documentclass{mumie.problem.gwtmathlet}
%$Id$
\begin{metainfo}
  \name{
    \lang{de}{A03: Länge}
    \lang{en}{mc yes-no}
  }
  \begin{description} 
 This work is licensed under the Creative Commons License Attribution 4.0 International (CC-BY 4.0)   
 https://creativecommons.org/licenses/by/4.0/legalcode 

    \lang{de}{Beschreibung}
    \lang{en}{description}
  \end{description}
  \corrector{system/problem/GenericCorrector.meta.xml}
  \begin{components}
    \component{js_lib}{system/problem/GenericMathlet.meta.xml}{mathlet}
  \end{components}
  \begin{links}
  \end{links}
  \creategeneric
\end{metainfo}
\begin{content}

\usepackage{mumie.genericproblem}
\lang{de}{
	\title{A03: Länge}
}

\begin{block}[annotation]
  Im Ticket-System: \href{http://team.mumie.net/issues/9713}{Ticket 9713}
\end{block}

\textit{Hinweis: Benutzen Sie sqrt() um die Wurzel einzugeben.}

\begin{problem}

	\begin{question}
		\lang{de}{
			\text{Bestimmen Sie die Länge folgender Vektoren. \\
			a)
			$
			  \left\|\begin{pmatrix}\var{x_1}\\ \var{x_2} \end{pmatrix}\right\|=$\ansref \\
 			b)
 			$
  			 \left\|\begin{pmatrix}\var{y_1}\\ \var{y_2}\\ \var{y_3} \end{pmatrix}\right\|=$\ansref 
		}
		}
         \explanation{Die Länge eines Vektors berechnen Sie, indem Sie jeden Eintrag des Vektors 
         quadrieren, alle Quadrate aufsummieren und aus der Summe die Wurzel ziehen.}
      
		\type{input.function}
      \field{real} 
 
      \begin{variables}
            \randint[Z]{x_1}{-10}{10}
            \randint{x_2}{-10}{10}
            \randint{y_1}{-10}{10}
            \randint{y_2}{-10}{10}
            \randint{y_3}{-10}{10}
            
            \function[calculate]{a1}{x_1^2+x_2^2}
            \function[calculate]{b1}{y_1^2+y_2^2+y_3^2}
            
            \function{aa}{sqrt(a1)}
            \function{bb}{sqrt(b1)}
      \end{variables}
      \begin{answer}
            \text{a)}
            \solution{aa}
            \inputAsFunction{x}{ax} 
            \checkFuncForZero{ax-aa}{-10}{10}{100}
            \explanation[edited]{Ihre Antwort in a) ist falsch.}
      \end{answer}

      \begin{answer}
            \text{b)}
            \solution{bb}
            \inputAsFunction{y}{by} 
            \checkFuncForZero{by-bb}{-10}{10}{100}
            \explanation[edited]{Ihre Antwort in b) ist falsch.}
      \end{answer}
             
\end{question}



\end{problem}

\embedmathlet{mathlet}



\end{content}