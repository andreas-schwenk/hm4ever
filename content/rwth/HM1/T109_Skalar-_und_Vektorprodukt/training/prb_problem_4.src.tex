\documentclass{mumie.problem.gwtmathlet}
%$Id$
\begin{metainfo}
  \name{
    \lang{de}{A04: Normalisierte Länge}
    \lang{en}{mc yes-no}
  }
  \begin{description} 
 This work is licensed under the Creative Commons License Attribution 4.0 International (CC-BY 4.0)   
 https://creativecommons.org/licenses/by/4.0/legalcode 

    \lang{de}{Beschreibung}
    \lang{en}{description}
  \end{description}
  \corrector{system/problem/GenericCorrector.meta.xml}
  \begin{components}
    \component{js_lib}{system/problem/GenericMathlet.meta.xml}{mathlet}
  \end{components}
  \begin{links}
  \end{links}
  \creategeneric
\end{metainfo}
\begin{content}

\usepackage{mumie.genericproblem}
\lang{de}{
	\title{A04: Normalisierte Länge}
}

\begin{block}[annotation]
  Im Ticket-System: \href{http://team.mumie.net/issues/9714}{Ticket 9714}
\end{block}

\textit{Hinweis: Benutzen Sie \texttt{sqrt()} um die Wurzel einzugeben.}

\begin{problem}

	\begin{question}
		\lang{de}{\text{
			Bestimmen Sie $x>0$ so, dass der Vektor \\
			$
   			\begin{pmatrix}\var{y}\\ x \end{pmatrix}
   			$  \\
  			Länge 1 besitzt. 
			}
			}
            
         \explanation[edited]{Bestimmen Sie die Länge des Vektors in Abhängigkeit von $x>0$. Die Länge eines Vektors berechnen Sie, indem Sie jeden Eintrag des Vektors 
         quadrieren, alle Quadrate aufsummieren und aus der Summe die Wurzel ziehen. Setzen Sie Ihr von $x$ abhängiges Ergebnis gleich $1$ und lösen Sie die Gleichung nach $x>0$ auf.\\
         Beispiel: Damit der Vektor $\begin{pmatrix} 0.4\\x\end{pmatrix}$ für $x>0$ die Länge $1$ hat, muss folgendes gelten: $\sqrt{0.4^2+x^2}=1$. Quadrieren der Gleichung, Umformen nach $x^2$ und anschließendes Wurzelziehen liefern: 
         $x=\sqrt{1-0.4^2}=\sqrt{0.84}$.}
         
		\type{input.function}
      	\field{rational} 
 
      \begin{variables}
            \randint{ya}{15}{30}
            \randint{yb}{1}{14}
            \function[normalize]{y}{yb/ya}
            
            \randint{y_1}{-10}{10}
            \randint{y_2}{-10}{10}
            \randint{y_3}{-10}{10}
            
            \function[calculate]{a1}{1-y^2}
            \function[calculate]{b1}{y_1^2+y_2^2+y_3^2}
            
            \function{aa}{sqrt(a1)}
            \function{bb}{sqrt(b1)}
      \end{variables}
      \begin{answer}
            \text{$x=$}
            \solution{aa}
            \inputAsFunction{x}{ax} 
            \checkFuncForZero{ax-aa}{-10}{10}{100}
      \end{answer}
             
\end{question}

\begin{question}
		\lang{de}{\text{
			Bestimmen Sie $x>0$ so, dass der Vektor \\
			$
   			\begin{pmatrix}\frac{\var{y_1}}{x} \\ \frac{\var{y_2}}{x} \\ \frac{\var{y_3}}{x} \end{pmatrix}
   			$ \\
   			Länge 1 besitzt.  
			}
			}
            
         \explanation[edited]{Bestimmen Sie die Länge des Vektors in Abhängigkeit von $x>0$. Die Länge eines Vektors berechnen Sie, indem Sie jeden Eintrag des Vektors 
         quadrieren, alle Quadrate aufsummieren und aus der Summe die Wurzel ziehen. Setzen Sie Ihr von $x$ abhängiges Ergebnis gleich $1$ und lösen Sie die Gleichung nach $x>0$ auf.\\
         Beispiel: Damit der Vektor $\begin{pmatrix} 0.4\\x\end{pmatrix}$ für $x>0$ die Länge $1$ hat, muss folgendes gelten: $\sqrt{0.4^2+x^2}=1$. Quadrieren der Gleichung, Umformen nach $x^2$ und anschließendes Wurzelziehen liefern: 
         $x=\sqrt{1-0.4^2}=\sqrt{0.84}$.}
         
		\type{input.function}
      	\field{rational} 
 
      \begin{variables}
            \randint{ya}{15}{30}
            \randint{yb}{1}{14}
            \function[normalize]{y}{yb/ya}
            
            \randint{y_1}{-10}{10}
            \randint{y_2}{-10}{10}
            \randint{y_3}{-10}{10}
            
            \function[calculate]{a1}{1-y^2}
            \function[calculate]{b1}{y_1^2+y_2^2+y_3^2}
            
            \function{aa}{sqrt(a1)}
            \function{bb}{sqrt(b1)}
      \end{variables}
  
      \begin{answer}
            \text{$x=$}
            \solution{bb}
            \inputAsFunction{y}{by} 
            \checkFuncForZero{by-bb}{-10}{10}{100}
      \end{answer}
             
\end{question}


\end{problem}

\embedmathlet{mathlet}



\end{content}