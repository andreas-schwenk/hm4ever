\documentclass{mumie.problem.gwtmathlet}
%$Id$
\begin{metainfo}
  \name{
    \lang{de}{A07: Textaufgabe}
    \lang{en}{mc yes-no}
  }
  \begin{description} 
 This work is licensed under the Creative Commons License Attribution 4.0 International (CC-BY 4.0)   
 https://creativecommons.org/licenses/by/4.0/legalcode 

    \lang{de}{Beschreibung}
    \lang{en}{description}
  \end{description}
  \corrector{system/problem/GenericCorrector.meta.xml}
  \begin{components}
    \component{generic_image}{content/rwth/HM1/images/g_tkz_T109_Problem07.meta.xml}{T109_Problem07}
    \component{generic_image}{content/rwth/HM1/images/g_img_1_9_Dreieck.meta.xml}{1_9_Dreieck}
    \component{js_lib}{system/problem/GenericMathlet.meta.xml}{mathlet}
  \end{components}
  \begin{links}
  \end{links}
  \creategeneric
\end{metainfo}
\begin{content}
\begin{block}[annotation]
	Im Ticket-System: \href{https://team.mumie.net/issues/21557}{Ticket 21557}
\end{block}

\usepackage{mumie.genericproblem}
\lang{de}{
	\title{A07: Textaufgabe}
}

\textit{Hinweis: Benutzen Sie \texttt{sqrt()} um die Wurzel einzugeben. Benutzen Sie zur Berechnung der Winkel einen Taschenrechner und runden Sie auf zwei Nachkommastellen. 
}

\begin{problem}

	   \begin{question} 
	\lang{de}{
		\text{
   Gegeben sei das folgende Dreieck $\overline{ABC}$ mit den Eckpunkten $A=(\var{a1};\var{a2};\var{a3})$, 
        $B=(\var{b1};\var{b2};\var{b3})$ und $C=(\var{c1};\var{c2};\var{c3})$.
        \begin{figure}
        \image{T109_Problem07}
        \end{figure}
        1. Bestimmen Sie die Seitenlängen $a$, $b$ sowie $c$.\\\\
        Die Seitenlängen lauten:\\\\
        $a=$\ansref Einheiten\\\\
        $b=$\ansref Einheiten\\\\
        $c=$\ansref Einheiten\\\\
        
        2. Berechnen Sie die Winkel $\alpha$, $\beta$ und $\gamma$.\\\\
        Die Winkel betragen:\\\\
        $\alpha=$\ansref Grad\\\\ 
        $\beta=$\ansref Grad\\\\
        $\gamma=$\ansref Grad\\\\
        
        3. Das Dreieck $\overline{ABC}$ wird nun verändert: Der Punkt $A$ bleibt an gleicher Stelle. Die Seitenlängen $b$ und $c$ werden
        auf das $\var{fache}$-fache verlängert, halten die Richtung aber bei. Bestimmen Sie den Winkel $\alpha_{neu}$, der
        sich am Punkt $A$ des veränderten Dreiecks ergibt.\\\\
        
        Es gilt: $\alpha_{neu}=$\ansref Grad
         
        }}
 
	\begin{variables}
       \randint{a1}{4}{6}
       \function[normalize]{a2}{a1}
       \randint{a3}{1}{3}
       \function[normalize]{b1}{a1+4}
        \function[normalize]{b2}{a1+3}
       \randint{b3}{8}{9}
       \function[normalize]{c1}{a1+2}
        \function[normalize]{c2}{a1+3}
       \randint{c3}{6}{7}
       \randint{fache}{6}{12}
       \number{r}{1}
    
       \function[normalize]{avektor1}{c1-b1}
       \function[normalize]{avektor2}{c2-b2}
       \function[normalize]{avektor3}{c3-b3}
     
       \function[normalize]{bvektor1}{a1-c1}
       \function[normalize]{bvektor2}{a2-c2}
       \function[normalize]{bvektor3}{a3-c3}
       
       \function[normalize]{cvektor1}{b1-a1}
       \function[normalize]{cvektor2}{b2-a2}
       \function[normalize]{cvektor3}{b3-a3}
       
       \function[normalize]{alaenge}{sqrt(avektor1^2+avektor2^2+avektor3^2)}  
       \function[normalize]{blaenge}{sqrt(bvektor1^2+bvektor2^2+bvektor3^2)}
       \function[normalize]{claenge}{sqrt(cvektor1^2+cvektor2^2+cvektor3^2)}
       
              
       \function[normalize]{skalarproduktalpha}{-bvektor1*cvektor1-bvektor2*cvektor2-bvektor3*cvektor3}  
       \function[normalize]{skalarproduktbeta}{-cvektor1*avektor1-cvektor2*avektor2-cvektor3*avektor3}  
       \function[normalize]{skalarproduktgamma}{-avektor1*bvektor1-avektor2*bvektor2-avektor3*bvektor3}  
     
       \function[normalize]{alpha1}{skalarproduktalpha/(claenge*blaenge)}  
       \function[normalize]{alpha}{arccos(alpha1)*360/(2*pi)}
       \function[normalize]{beta1}{skalarproduktbeta/(alaenge*claenge)}  
       \function[normalize]{beta}{arccos(beta1)*360/(2*pi)}
       \function[normalize]{gamma1}{skalarproduktgamma/(alaenge*blaenge)}  
       \function[normalize]{gamma}{arccos(gamma1)*360/(2*pi)}
       
       \function[normalize]{alphaneu}{1*r}
      
    \end{variables}
	\type{input.number}
 	\field{real}
    
    \begin{answer}
         \type{input.function}
    	\solution{alaenge}
        \checkAsFunction{x}{-1}{1}{10}
          \explanation[edited]{Um die Seitenlängen bestimmen zu können, müssen Sie die 
               Verbindungsvektoren zweier Eckpunkte des Dreiecks berechnen und deren Länge bestimmen.}
    \end{answer}
    
     \begin{answer}
          \type{input.function}
    	\solution{blaenge}
           \checkAsFunction{x}{-1}{1}{10}
    \end{answer}
    
     \begin{answer}
          \type{input.function}
    	\solution{claenge}
           \checkAsFunction{x}{-1}{1}{10}
    \end{answer}
    
     \begin{answer}
    	\solution{alpha}
         \explanation[edited]{Um die Winkel des Dreiecks bestimmen zu können, müssen Sie die 
               beiden Verbindungsvektoren der jeweiligen Ecke zu den beiden anderen Ecken berechnen und dann den Winkel zwischen diesen 
               Vektoren ermitteln.}
    \end{answer}
    
     \begin{answer}
    	\solution{beta}
    \end{answer}
    
     \begin{answer}
    	\solution{gamma}
    \end{answer}
    
     \begin{answer}
    	\solution{alpha}
    \end{answer}
   
  \end{question}
    
	
\end{problem}

\embedmathlet{mathlet}

\end{content}