%$Id:  $
\documentclass{mumie.article}
%$Id$
\begin{metainfo}
  \name{
    \lang{de}{Vektorprodukt}
    \lang{en}{Cross product}
  }
  \begin{description} 
 This work is licensed under the Creative Commons License Attribution 4.0 International (CC-BY 4.0)   
 https://creativecommons.org/licenses/by/4.0/legalcode 

    \lang{de}{Beschreibung}
    \lang{en}{Description}
  \end{description}
  \begin{components}
    \component{generic_image}{content/rwth/HM1/images/g_tkz_T109_RightHand.meta.xml}{T109_RightHand}
    \component{generic_image}{content/rwth/HM1/images/g_tkz_T109_TripleProduct.meta.xml}{T109_TripleProduct}
    \component{generic_image}{content/rwth/HM1/images/g_tkz_T109_AreaParalellogram.meta.xml}{T109_AreaParalellogram}
    \component{generic_image}{content/rwth/HM1/images/g_tkz_T109_CrossProduct_C.meta.xml}{T109_CrossProduct_C}
    \component{generic_image}{content/rwth/HM1/images/g_tkz_T109_CrossProduct_B.meta.xml}{T109_CrossProduct_B}
    \component{generic_image}{content/rwth/HM1/images/g_tkz_T109_CrossProduct_A.meta.xml}{T109_CrossProduct_A}
    \component{generic_image}{content/rwth/HM1/images/g_img_00_Videobutton_schwarz.meta.xml}{00_Videobutton_schwarz}
    \component{js_lib}{system/media/mathlets/GWTGenericVisualization.meta.xml}{mathlet1}
  \end{components}
  \begin{links}
\link{generic_article}{content/rwth/HM1/T108_Vektorrechnung/g_art_content_29_linearkombination.meta.xml}{content_29_linearkombination}
\link{generic_article}{content/rwth/HM1/T109_Skalar-_und_Vektorprodukt/g_art_content_33_winkel.meta.xml}{content_33_winkel}
\end{links}
  \creategeneric
\end{metainfo}
\begin{content}
\usepackage{mumie.ombplus}
\ombchapter{9}
\ombarticle{4}
\usepackage{mumie.genericvisualization}

\begin{visualizationwrapper}

\title{\lang{de}{Vektorprodukt}\lang{en}{Cross product}}
 
\begin{block}[annotation]
  übungsinhalt
  
\end{block}
\begin{block}[annotation]
  Im Ticket-System: \href{http://team.mumie.net/issues/9051}{Ticket 9051}\\
\end{block}

\begin{block}[info-box]
\tableofcontents
\end{block}

\section{\lang{de}{Definition des Vektorprodukts}\lang{en}{Definition of the cross product}}

\lang{de}{
Oft benötigt man zu zwei Vektoren $\vec{v}$ und $\vec{w}$ im dreidimensionalen Raum $\R^3$
einen Vektor, der zu beiden senkrecht steht. Einen solchen kann man mit dem Vektorprodukt berechnen.
}
\lang{en}{
Given two vectors $\vec{v}$ and $\vec{w}$ in three-dimensional space $\R^3$, we are often asked to 
find a third vector that is orthogonal to both. This can be calculated using the cross product.
}

\begin{definition} \label{def:Vektorprodukt}
\lang{de}{
Seien $\vec{v}=\left( \begin{smallmatrix} v_1 \\ v_2  \\ v_3\end{smallmatrix}\right)$ 
und $\vec{w}=\left( \begin{smallmatrix} w_1 \\ w_2  \\ w_3\end{smallmatrix}\right) \in \R^3$ 
zwei beliebige Vektoren, dann ist das 
\emph{Vektorprodukt} von $\vec{v}$ und $\vec{w}$ (oder \emph{Kreuzprodukt}), geschrieben als
$\vec{v} \times \vec{w}$, der Vektor
}
\lang{en}{
Let $\vec{v}=\left( \begin{smallmatrix} v_1 \\ v_2  \\ v_3\end{smallmatrix}\right)$ 
and $\vec{w}=\left( \begin{smallmatrix} w_1 \\ w_2  \\ w_3\end{smallmatrix}\right) \in \R^3$ 
be two vectors. Then the \emph{cross product} of $\vec{v}$ and $\vec{w}$ (or \emph{vector product}), 
denoted by $\vec{v} \times \vec{w}$, is the vector
}
\[
\vec{v} \times \vec{w} = \begin{pmatrix} v_2 w_3 - v_3 w_2 \\  v_3 w_1 - v_1 w_3  \\ 
v_1 w_2 - v_2 w_1 \end{pmatrix}. %\neq \begin{pmatrix} 0 \\ 0 \\ 0 \end{pmatrix}.
\]
\lang{de}{
\floatright{\href{https://www.hm-kompakt.de/video?watch=720}{\image[75]{00_Videobutton_schwarz}}}\\\\
}
\lang{en}{}
\end{definition}

\lang{de}{
Anschaulich ist das Vektorprodukt $\vec{v}\times\vec{w}$ derjenige Vektor, der sowohl rechtwinklig 
zum Vektor $\vec{v}$ als auch zum Vektor $\vec{w}$ steht. Die folgende Abbildung zeigt dies im 
$\mathbb{R}^3$.
}
\lang{en}{
It is easily verified that the cross product $\vec{v}\times\vec{w}$ is orthogonal to both $\vec{v}$ 
and $\vec{w}$. The following image demonstrates this in $\mathbb{R}^3$.
}
\begin{center}
\image{T109_CrossProduct_A}
\end{center}

%alte Bemerkung
%\begin{remark}
%Das folgende Bild dient als Merkhilfe für das Vektorprodukt

%\image[400]{vektorprodukt2}

%Für den ersten Eintrag des Vektorprodukts streicht man die erste Zeile und bildet
%die Produkte über Kreuz. Das Produkt zur gestrichelten Linie wird von dem Produkt zur durchgezogenen
%Linie abgezogen.\\
%Für die anderen Zeilen verfährt man entsprechend. Die durchgezogene Linie ist dabei diejenige, die
%links unter der gestrichenen Zeile beginnt (bzw. ganz links oben, wenn die letzte Zeile
%gestrichen wurde).
%\end{remark}

%neue bemerkung
\begin{remark}
\lang{de}{Das folgende Bild dient als Merkhilfe für das Vektorprodukt.}
\lang{en}{The following image helps to remember the definition of the cross product.}
\begin{center}
\image{T109_CrossProduct_B}
\end{center}
\lang{de}{
Für den ersten Eintrag des Vektorprodukts streicht man die erste Zeile und bildet die Produkte über 
Kreuz, wobei das eine Produkt vom anderen Produkt, wie in der Abbildung dargestellt, subtrahiert 
wird. Für die anderen Zeilen verfährt man entsprechend.
}
\lang{en}{
The first component of the cross product is calculated by ignoring the first components of $\vec{v}$ 
and $\vec{w}$, and combining remaining components as shown by the 'cross', where one product is 
subtracted from the other product. The other components of the cross product are calculated 
analogously.
}
\end{remark}

\begin{example}
\[ \begin{pmatrix} 1 \\ 0 \\ 2 \end{pmatrix} \times \begin{pmatrix} 2\\ 2\\ 4 \end{pmatrix}
= \begin{pmatrix} 0\cdot 4-2\cdot 2 \\ 2\cdot 2- 1\cdot 4 \\ 1\cdot 2-0\cdot 2 \end{pmatrix}
= \begin{pmatrix} -4 \\ 0 \\ 2 \end{pmatrix} \]
\end{example}

\begin{block}[warning]
\lang{de}{Das Vektorprodukt ist nur für Vektoren im $\R^3$ definiert.}
\lang{en}{The cross product has only been defined for vectors in $\R^3$.}
\end{block}

\begin{quickcheck}
		\field{rational}
		\type{input.number}
		\begin{variables}
			\randint{v1}{-5}{5}
			\randint{v2}{-5}{5}
			\randint{v3}{-5}{5}
			\randint{w1}{-5}{5}
			\randint{w2}{-5}{5}
			\randint{w3}{-5}{5}
			\function[calculate]{t11}{v2*w3}
			\function[calculate]{t12}{v3*w2}
			\function[calculate]{t21}{v3*w1}
			\function[calculate]{t22}{v1*w3}
			\function[calculate]{t31}{v1*w2}
			\function[calculate]{t32}{v2*w1}			
			\function{t1}{t11-t12}
			\function{t2}{t21-t22}
			\function{t3}{t31-t32}
			\function[calculate]{s1}{v2*w3-v3*w2}
			\function[calculate]{s2}{v3*w1-v1*w3}
			\function[calculate]{s3}{v1*w2-v2*w1}
		\end{variables}
		
			\text{\lang{de}{Das Vektorprodukt der Vektoren}
            \lang{en}{The cross product of the vectors}
			$\vec{v}=\begin{pmatrix} \var{v1}\\ \var{v2} \\ \var{v3}
			\end{pmatrix}$ und $\vec{w}=\begin{pmatrix} \var{w1}\\ \var{w2} \\ \var{w3}
			\end{pmatrix}$
			\lang{de}{ist}\lang{en}{is}
			\begin{table}[\class{no-padding}]
			\rowspan[l][m]{3} $\vec{v}\times \vec{w}=\begin{pmatrix} \var{v1}\\ \var{v2} \\ \var{v3}
			\end{pmatrix} \times \begin{pmatrix} \var{w1}\\ \var{w2} \\ \var{w3}
			\end{pmatrix} =
			\left(\begin{matrix} \\ \\ \\ \\ \end{matrix}\right.$ &  
			\ansref & \rowspan[l][m]{3} $\left.\begin{matrix} \\ \\ \\ \\  \end{matrix}\right)$. & \\ 
			\ansref & \\ 
			\ansref & 
			\end{table}			
			}
		
		\begin{answer}
			\solution{s1}
		\end{answer}
		\begin{answer}
			\solution{s2}
		\end{answer}
		\begin{answer}
			\solution{s3}
		\end{answer}
		\explanation{\lang{de}{Die Berechnung folgt nach obiger Regel, also}
                 \lang{en}{By the definition of the cross product,}\\
		$ \begin{pmatrix} \var{v1}\\ \var{v2} \\ \var{v3} \end{pmatrix} 
			\times \begin{pmatrix} \var{w1}\\ \var{w2} \\ \var{w3} \end{pmatrix} 
			=\begin{pmatrix}(\var{v2})\cdot (\var{w3})-(\var{v3})\cdot (\var{w2})\\ 
			(\var{v3})\cdot (\var{w1})-(\var{v1})\cdot (\var{w3})\\
			(\var{v1})\cdot (\var{w2})-(\var{v2})\cdot (\var{w1}) \end{pmatrix}
			=\begin{pmatrix}\var{t1}\\ \var{t2} \\ \var{t3} \end{pmatrix}
			=\begin{pmatrix}\var{s1}\\ \var{s2} \\ \var{s3} \end{pmatrix}. $
		}
	\end{quickcheck}


\section{\lang{de}{Eigenschaften des Vektorprodukts}
         \lang{en}{Properties of the cross product}}\label{sec:eigenschaften}

\begin{theorem}\label{thm:eigenschaften}
\lang{de}{
Seien $\vec{v}$ und $\vec{w}$ Vektoren im $\R^3$ und $\vec{n}=\vec{v} \times \vec{w}$ ihr
Vektorprodukt. Dann gelten:
}
\lang{en}{
Let $\vec{v}$ andd $\vec{w}$ be vectors in $\R^3$ and $\vec{n}=\vec{v} \times \vec{w}$ their cross 
product.
}
\begin{enumerate}
\item \lang{de}{
      $\vec{v} \times \vec{w}= \vec{0}$ $\Leftrightarrow$ $\vec{v}$ und $\vec{w}$ sind 
      \ref[content_29_linearkombination][linear abhängig]{def:linearcomb}.\\
      Insbesondere ist stets $\vec{v} \times \vec{v}=\vec{0}$.
      }
      \lang{en}{
      $\vec{v} \times \vec{w}= \vec{0}$ $\Leftrightarrow$ $\vec{v}$ and $\vec{w}$ are 
      \ref[content_29_linearkombination][linearly dependent]{def:linearcomb}.\\
      In particular, we always have $\vec{v} \times \vec{v}=\vec{0}$.
      }
\item \lang{de}{
      $\vec{v} \times \vec{w}$ ist stets \ref[content_33_winkel][orthogonal]{def:orthogonal} zu 
      $\vec{v}$ und zu $\vec{w}$.
      }
      \lang{en}{
      $\vec{v} \times \vec{w}$ is always \ref[content_33_winkel][orthogonal]{def:orthogonal} to 
      $\vec{v}$ and to $\vec{w}$.
      }
\item \lang{de}{Sei $\varphi$ der Winkel zwischen den Vektoren $\vec{v}$ und $\vec{w}$, dann gilt}
      \lang{en}{Let $\varphi$ be the angle between the vectors $\vec{v}$ and $\vec{w}$. Then}
\[  \left\| \vec{n}\right\| =\left\| \vec{v}\right\| \cdot \left\| \vec{w}\right\|\cdot \sin(\varphi). \]
\end{enumerate}
\lang{de}{
\floatright{\href{https://www.hm-kompakt.de/video?watch=723}{\image[75]{00_Videobutton_schwarz}}
\href{https://www.hm-kompakt.de/video?watch=724}{\image[75]{00_Videobutton_schwarz}}}\\\\
}
\lang{en}{}
\end{theorem}


\lang{de}{
Die in dem Satz angegebenen Eigenschaften bestimmen den Vektor $\vec{n}=\vec{v} \times \vec{w}$
schon fast. Lediglich der Gegenvektor $-\vec{n}$ besitzt auch diese Eigenschaften.
Die letzte Eigenschaft, mit der man den Vektor $\vec{n}$ dann eindeutig identifizieren kann, ist 
die folgende:
}
\lang{en}{
The properties of the cross product given in the above theorem are almost enough to uniquely 
determine the vector $\vec{n}=\vec{v} \times \vec{w}$. Only the vector $-\vec{n}$ shares these 
properties. The final property required to uniquely determine the vector $\vec{n}$ is the following:
}

\begin{rule}
\lang{de}{
Die drei Vektoren $\vec{v}$, $\vec{w}$ und $\vec{n}=\vec{v} \times \vec{w}$ bilden ein 
sogenanntes \emph{Rechtsystem}, d.h. dass $\vec{v}$, $\vec{w}$ und $\vec{n}$ in
Richtungen wie Daumen, Zeige- und Mittelfinger der rechten Hand zeigen, wenn Daumen und Zeigefinger 
gestreckt und der Mittelfinger geknickt ist.
}
\lang{en}{
The three vectors $\vec{v}$, $\vec{w}$ and $\vec{n}=\vec{v} \times \vec{w}$ form a so-called 
\emph{right-handed orthogonal system}, that is, $\vec{v}$, $\vec{w}$ and $\vec{n}$ point in the 
directions of the thumb, index and middle finger of a right hand with its thumb and index finger 
extended and its middle finger bent as in the image below.
}
\begin{center}
\image{T109_RightHand}
\end{center}
\end{rule}


\begin{example}
\lang{de}{Für $\vec{e}_1=\begin{pmatrix}1\\ 0\\ 0 \end{pmatrix}$ und 
$\vec{e}_2=\begin{pmatrix}0\\ 1\\ 0 \end{pmatrix}$ ist}
\lang{en}{For $\vec{e}_1=\begin{pmatrix}1\\ 0\\ 0 \end{pmatrix}$ and 
$\vec{e}_2=\begin{pmatrix}0\\ 1\\ 0 \end{pmatrix}$ we have}
\[ \vec{e}_1\times \vec{e}_2 
=\begin{pmatrix}0\cdot 0-0\cdot 1 \\ 0\cdot 0-1\cdot 0\\ 1\cdot 1-0\cdot 0 \end{pmatrix}
=\begin{pmatrix}0\\ 0\\ 1 \end{pmatrix}=\vec{e}_3. \]
\begin{center}
\image{T109_CrossProduct_C}
\end{center}
\end{example}

\begin{quickcheck}
		\field{rational}
		\type{input.function}
		\begin{variables}
			\randint{nv}{1}{7}
			\randint{nw}{1}{7}
			\function[calculate]{s}{nv*nw}

% 			\function[calculate]{nac2}{(a1-c1)^2+(a2-c2)^2}
% 			\function[calculate]{nbc2}{(c1-b1)^2+(c2-b2)^2}
% 			\function{ab}{sqrt(nab2)}
% 			\function{ac}{sqrt(nac2)}
% 			\function{bc}{sqrt(nbc2)}
		\end{variables}
		
			\text{\lang{de}{
      Seien $\vec{v}$ und $\vec{w}$ zwei Vektoren der Längen
			$\left\|\vec{v}\right\|=\var{nv}$ und $\left\|\vec{w}\right\|=\var{nw}$.
			Welche Länge kann das Vektorprodukt $\vec{v}\times \vec{w}$ der beiden Vektoren
			höchstens haben?\\
			Die maximal mögliche Länge ist \ansref.
      }
      \lang{en}{
      Let $\vec{v}$ and $\vec{w}$ be two vectors of lengths
			$\left\|\vec{v}\right\|=\var{nv}$ and $\left\|\vec{w}\right\|=\var{nw}$. 
      What is the maximum possible length of the cross product $\vec{v}\times \vec{w}$?\\
      The maximum possible length is \ansref.
      }}
  	
		\begin{answer}
			\solution{s}
		\end{answer}
		\explanation{\lang{de}{Für die Länge des Vektorprodukts gilt die Gleichung}
                 \lang{en}{We have an equality concerning the length of the cross product,}\\
		$ \left\| \vec{v}\times \vec{w}\right\|=\left\| \vec{v}\right\| \cdot 
\left\| \vec{w}\right\|\cdot \sin(\varphi), $\\
		\lang{de}{
    wobei $\varphi$ der Winkel zwischen $\vec{v}$ und $\vec{w}$ ist.\\
		Da für alle $\varphi$ bereits $\sin(\varphi)\leq 1$ ist, ist also 
    }
    \lang{en}{
    where $\varphi$ is the angle between $\vec{v}$ and $\vec{w}$.\\
    As $\sin(\varphi)\leq 1$ for all $\varphi$,
    }\\
		$ \left\| \vec{v}\times \vec{w}\right\|\leq \left\| \vec{v}\right\| \cdot \left\| \vec{w}\right\|
		=\var{nv}\cdot \var{nw}=\var{s}. $\\
    \lang{de}{
		Sind $\vec{v}$ und $\vec{w}$ zueinander senkrecht, so ist $\varphi=\pi/2$ und $	\sin(\varphi)=1$. 
    In diesem	Fall gilt also sogar Gleichheit.
    }
    \lang{en}{
    If $\vec{v}$ and $\vec{w}$ are orthogonal, then $\varphi=\pi/2$ and $	\sin(\varphi)=1$. In this 
    case we even have equality.
    }}
	\end{quickcheck}



\section{\lang{de}{Rechenregeln für das Vektorprodukt}
         \lang{en}{Rules for the cross product}}\label{sec:rechenregeln}

\begin{block}[warning]
\lang{de}{
Die für die Multiplikation in den reellen Zahlen bekannte Assoziativität und Kommutativität ist für 
das Vektorprodukt nicht erfüllt, d.h. im Allgemeinen gelten für Vektoren $\vec{v}$, $\vec{w}$ und 
$\vec{u}$ im $\R^3$
}
\lang{en}{
The associativity and commutativity that we know from multiplication in the real numbers are not 
properties of the cross product, that is, for most vectors $\vec{v}$, $\vec{w}$ and 
$\vec{u}$ in $\R^3$ we have
}
\[ (\vec{v}\times \vec{w})\times \vec{u} \neq \vec{v}\times (\vec{w}\times \vec{u}) \]
\lang{de}{und}\lang{en}{and}
\[ \vec{w}\times \vec{v} \neq \vec{v}\times \vec{w} \quad !\]
\end{block}

\begin{example}
\begin{tabs*}
\tab{$\vec{w}\times \vec{v} \neq \vec{v}\times \vec{w}$}
\lang{de}{Für}\lang{en}{For} 
$\vec{v}=\begin{pmatrix}1\\ 0\\ 0 \end{pmatrix}$
\lang{de}{und}\lang{en}{and} 
$\vec{w}=\begin{pmatrix}1\\ 1\\ 0 \end{pmatrix}$ 
\lang{de}{ist}\lang{en}{we have}
\[ \vec{w}\times \vec{v} = 
\begin{pmatrix}1\\ 1\\ 0 \end{pmatrix}\times \begin{pmatrix}1\\ 0\\ 0 \end{pmatrix} = 
\begin{pmatrix}0\\ 0\\ -1 \end{pmatrix},\]
\lang{de}{aber}\lang{en}{but}
\[ \vec{v}\times \vec{w}=\begin{pmatrix}1\\ 0\\ 0 \end{pmatrix}\times \begin{pmatrix}1\\ 1\\ 0 \end{pmatrix}
=\begin{pmatrix}0\\ 0\\ 1 \end{pmatrix}. \]
\tab{$(\vec{v}\times \vec{w})\times \vec{u} \neq \vec{v}\times (\vec{w}\times \vec{u})$}
\lang{de}{Für}\lang{en}{For} 
$\vec{v}=\begin{pmatrix}1\\ 0\\ 0 \end{pmatrix}$, 
$\vec{w}=\begin{pmatrix}1\\ 1\\ 0 \end{pmatrix}$ 
\lang{de}{und}\lang{en}{and} 
$\vec{u}=\begin{pmatrix}1\\ 1\\ 1 \end{pmatrix}$ 
\lang{de}{ist}\lang{en}{we have}
\[ (\vec{v}\times \vec{w})\times \vec{u} = 
\left( \begin{pmatrix}1\\ 0\\ 0 \end{pmatrix} \times 
\begin{pmatrix}1\\ 1\\ 0 \end{pmatrix}\right) \times \begin{pmatrix}1\\ 1\\ 1 \end{pmatrix} = 
\begin{pmatrix}0\\ 0\\ 1 \end{pmatrix}\times \begin{pmatrix}1\\ 1\\ 1 \end{pmatrix} = 
\begin{pmatrix}-1\\ 1\\ 0 \end{pmatrix}, \]
\lang{de}{aber}\lang{en}{or} 
\[ \vec{v}\times (\vec{w}\times \vec{u}) = 
\begin{pmatrix}1\\ 0\\ 0 \end{pmatrix}\times \left( \begin{pmatrix}1\\ 1\\ 0 \end{pmatrix} \times 
\begin{pmatrix}1\\ 1\\ 1 \end{pmatrix}\right) =
\begin{pmatrix}1\\ 0\\ 0 \end{pmatrix}\times \begin{pmatrix}1\\ -1\\ 0 \end{pmatrix} = 
\begin{pmatrix}0\\ 0\\ -1 \end{pmatrix}. \]
\end{tabs*}
\end{example}

\lang{de}{
Im vorigen Beispiel gilt für die gegebenen Vektoren $\vec{v}$ und $\vec{w}$ die Gleichung
$\vec{w}\times \vec{v}=- (\vec{v}\times \vec{w})$. Dass dies kein Zufall ist, sondern generell gilt, 
kann man allgemein nachrechnen. Es lässt sich auch mit Hilfe der 
\lref{thm:eigenschaften}{Eigenschaften des Vektorprodukts} anschaulich erklären:
}
\lang{en}{
For the vectors $\vec{v}$ and $\vec{w}$ given in the previous example we have 
$\vec{w}\times \vec{v}=- (\vec{v}\times \vec{w})$. It can be shown that this is not a coincidence, 
and holds generally. It can also be visually explained using the 
\lref{thm:eigenschaften}{properties of the cross product}:
}
%\begin{tabs*}[\initialtab{0}]
%\tab{Erklärung}
\begin{proof*}[\lang{de}{Erklärung zu}
               \lang{en}{Explanation for} $\vec{w}\times \vec{v}=- (\vec{v}\times \vec{w})$]
\begin{showhide}
\lang{de}{
$\vec{n_1}=\vec{v}\times \vec{w}$ und $\vec{n_2}=\vec{w}\times \vec{v}$ sind beides Vektoren, die 
sowohl auf $\vec{v}$ als auch auf $\vec{w}$ senkrecht stehen und die gleiche Länge, nämlich 
$\left\| \vec{v}\right\| \cdot \left\| \vec{w}\right\|\cdot \sin(\varphi)$, haben. Außerdem bilden 
$\vec{v}$, $\vec{w}$, $\vec{n_1}$ und $\vec{w}$, $\vec{v}$, $\vec{n_2}$ jeweils ein Rechtssystem.\\
Wenn man im ersten System $\vec{v}$ und $\vec{w}$ vertauscht, erhält man also ein Linkssystem 
$\vec{w}$, $\vec{v}$, $\vec{n_1}$. Ersetzt man $\vec{n_1}$ durch seinen Gegenvektor $-\vec{n_1}$ 
erhält man wieder ein Rechtssystem $\vec{w}$, $\vec{v}$, $-\vec{n_1}$.
\\\\
$-\vec{n_1}$ und $\vec{n_2}$ erfüllen also beide die oben genannten charakterisierenden Eigenschaften 
des Vektorprodukts, weshalb sie gleich sind.
}
\lang{en}{
$\vec{n_1}=\vec{v}\times \vec{w}$ and $\vec{n_2}=\vec{w}\times \vec{v}$ are vectors that are 
orthogonal to both $\vec{v}$ and $\vec{w}$, and have the same length, namely
$\left\| \vec{v}\right\| \cdot \left\| \vec{w}\right\|\cdot \sin(\varphi)$. Furthermore, 
$\vec{v}$, $\vec{w}$, $\vec{n_1}$ and $\vec{w}$, $\vec{v}$, $\vec{n_2}$ form a right-handed 
orthogonal system.\\
If we swap $\vec{v}$ and $\vec{w}$ in the prior, we obtain a left-handed orthogonal system
$\vec{w}$, $\vec{v}$, $\vec{n_1}$. Replacing $\vec{n_1}$ with its negative vector $-\vec{n_1}$ 
gives us a right-handed orthogonal system again, $\vec{w}$, $\vec{v}$, $-\vec{n_1}$.
\\\\
$-\vec{n_1}$ and $\vec{n_2}$ both fulfil the unique characteristics of a cross product, so they must 
be equal.
}
\end{showhide}
\end{proof*}
%\end{tabs*}

\lang{de}{Das Vektorprodukt erfüllt noch weitere Rechenregeln.}
\lang{en}{The cross product also satisfies other rules.}

\begin{theorem}
\lang{de}{
Für Vektoren $\vec{v}$, $\vec{w}$ und $\vec{u}$ im $\R^3$ und eine reelle Zahl $r\in \R$ gelten:
}
\lang{en}{
For vectors $\vec{v}$, $\vec{w}$ and $\vec{u}$ in $\R^3$ and any real number $r\in \R$ we have:
}
\begin{enumerate}
\item $\vec{w}\times \vec{v}=- (\vec{v}\times \vec{w})=-\vec{v}\times \vec{w}$,
\item $(r\vec{v})\times \vec{w}=r(\vec{v}\times \vec{w})$ 
\lang{de}{und}\lang{en}{and} 
$\vec{v}\times (r\vec{w})=r(\vec{v}\times \vec{w})$,
\item $\vec{v}\times (\vec{w}+\vec{u})=\vec{v}\times \vec{w}+ \vec{v}\times \vec{u}$, 
\lang{de}{sowie}\lang{en}{and}
$(\vec{v}+\vec{w})\times \vec{u}=\vec{v}\times \vec{u}+ \vec{w}\times \vec{u}$.
\end{enumerate}
\lang{de}{
Wobei auch hier wieder \glqq{}Punkt vor Strich\grqq{} zu rechnen ist.\\
\floatright{\href{https://www.hm-kompakt.de/video?watch=725}{\image[75]{00_Videobutton_schwarz}}}\\\\
}
\lang{en}{
Where the cross product (like in the order of operations of normal multiplication) is calculated 
before the addition.
}
\end{theorem}

\begin{example}
\lang{de}{
Um das Vektorprodukt der Vektoren $\vec{v}=\begin{pmatrix} 3/7 \\ 4/5 \\ 1/2 \end{pmatrix}$ und 
$\vec{w}=\begin{pmatrix}3/7 \\ 4/5 \\ 5/2 \end{pmatrix}$ zu berechnen, kann man die dritte und zweite 
Regel verwenden, um die Rechnung zu vereinfachen. Hier ist nämlich 
}
\lang{en}{
If we want to find the cross product of the vectors 
$\vec{v}=\begin{pmatrix} 3/7 \\ 4/5 \\ 1/2 \end{pmatrix}$ and 
$\vec{w}=\begin{pmatrix}3/7 \\ 4/5 \\ 5/2 \end{pmatrix}$, 
we can apply the third and the second rules to simplify the calculation, as 
}
\[ \vec{w}=\begin{pmatrix}3/7 \\ 4/5 \\ 5/2 \end{pmatrix}=\begin{pmatrix} 3/7 \\ 4/5 \\ 1/2 \end{pmatrix}
+\begin{pmatrix} 0 \\ 0 \\ 2 \end{pmatrix}=\vec{v}+2\cdot \begin{pmatrix} 0 \\ 0 \\ 1 \end{pmatrix}. \]
\lang{de}{Damit gilt:}\lang{en}{Hence}
\begin{eqnarray*}
\vec{v}\times \vec{w} &=& \vec{v}\times \vec{v}
\,\, +\,\, 2\cdot \begin{pmatrix} 3/7 \\ 4/5 \\ 1/2 \end{pmatrix}\times\begin{pmatrix} 0 \\ 0 \\ 1 \end{pmatrix} \quad \text{\lang{de}{nach 2. und 3.}\lang{en}{by 2. and 3.}}\\
&=& \vec{o} \,\, +\,\, 2\cdot \begin{pmatrix} 4/5 \\ -3/7 \\ 0 \end{pmatrix}
\,\, =\,\, \begin{pmatrix} 8/5 \\ -6/7 \\ 0 \end{pmatrix}.
\end{eqnarray*}


\end{example}

\begin{quickcheck}
		\field{rational}
		\type{input.number}
		\begin{variables}
			\randint{k}{1}{3}	% Zufallsvariable zum Vertauschen:
			\function[calculate]{d1}{(k-2)*(k-3)/2}  % "Dirac"-funktionen
			\function[calculate]{d2}{-(k-1)*(k-3)}
			\function[calculate]{d3}{(k-1)*(k-2)/2}		

			\randint{r}{1}{3}		
			\randrat[Z]{a1}{-5}{5}{4}{11}
			\randrat[Z]{a2}{-5}{5}{4}{11}
			\randrat[Z]{a3}{-5}{5}{4}{11}
			\function[calculate]{v1}{a1}
			\function[calculate]{v2}{a2}
			\function[calculate]{v3}{a3}
			\function[calculate]{w1}{a1+r*d1}
			\function[calculate]{w2}{a2+r*d2}
			\function[calculate]{w3}{a3+r*d3}

			\function[calculate]{s1}{v2*w3-v3*w2}
			\function[calculate]{s2}{v3*w1-v1*w3}
			\function[calculate]{s3}{v1*w2-v2*w1}
		\end{variables}
		
			\text{\lang{de}{Bestimmen Sie das Vektorprodukt von}
            \lang{en}{Determine the cross product of}
			$\vec{v}=\begin{pmatrix} \var{v1}\\ \var{v2} \\ \var{v3} \end{pmatrix}$ 
      \lang{de}{und}\lang{en}{and} 
      $\vec{w}=\begin{pmatrix} \var{w1}\\ \var{w2} \\ \var{w3} \end{pmatrix}$ 
      \lang{de}{möglichst geschickt wie im vorangegangenen Beispiel.}
      \lang{en}{by simplifying like in the previous example.}\\
			
			\begin{table}[\class{no-padding}]
			\rowspan[l][m]{3} $\vec{v}\times \vec{w}=\begin{pmatrix} \var{v1}\\ \var{v2} \\ \var{v3}
			\end{pmatrix} \times \begin{pmatrix} \var{w1}\\ \var{w2} \\ \var{w3}
			\end{pmatrix} =
			\left(\begin{matrix} \\ \\ \\ \\ \end{matrix}\right.$ &  
			\ansref & \rowspan[l][m]{3} $\left.\begin{matrix} \\ \\ \\ \\  \end{matrix}\right)$. & \\ 
			\ansref & \\ 
			\ansref & 
			\end{table}			
			}
		
		\begin{answer}
			\solution{s1}
		\end{answer}
		\begin{answer}
			\solution{s2}
		\end{answer}
		\begin{answer}
			\solution{s3}
		\end{answer}
		\explanation{\lang{de}{Hier gilt}\lang{en}{We have} 
      $\vec{w}=\vec{v}+\var{r}\cdot \begin{pmatrix} \var{d1}\\ \var{d2} \\ \var{d3} \end{pmatrix}$.\\ 
			\lang{de}{Also ist unter Berücksichtigung der Rechenregeln und von }
      \lang{en}{Hence, using the rules introduced above and }
      $\vec{v}\times \vec{v}=\vec{0}$:\\
			$\vec{v}\times \vec{w}=\vec{v}\times \left( \vec{v}
				+ \var{r}\cdot \begin{pmatrix} \var{d1}\\ \var{d2} \\ \var{d3} \end{pmatrix}\right)
			= \var{r}\cdot  \vec{v}\times \begin{pmatrix} \var{d1}\\ \var{d2} \\ \var{d3} \end{pmatrix}
			=\begin{pmatrix}\var{s1}\\ \var{s2} \\ \var{s3} \end{pmatrix}. $
		}
	\end{quickcheck}

\end{visualizationwrapper}

\section{\lang{de}{Anwendungen des Vektorprodukts}\lang{en}{Applications of the cross product}}\label{sec:anwendung_geometrie}

\begin{remark}\label{rem:parallelogram}
\lang{de}{
Die Länge des Vektorprodukts hat auch eine geometrische Bedeutung. Für zwei
Vektoren $\vec{v}$ und $\vec{w}$ im $\R^3$ mit eingeschlossenem Winkel $\varphi$ ist
der Wert
}
\lang{en}{
The length of a cross product of two vectors has a geometric interpretation. For two vectors 
$\vec{v}$ and $\vec{w}$ in $\R^3$ with an angle of $\varphi$ between them, the value
}
\[ \left\| \vec{v}\times \vec{w}\right\|=\left\| \vec{v}\right\| \cdot 
\left\| \vec{w}\right\|\cdot \sin(\varphi) \]
\lang{de}{
genau die Fläche des von $\vec{v}$ und $\vec{w}$ aufgespannten Parallelogramms, da die Grundseite
in der Skizze $\left\| \vec{v}\right\|$ beträgt und die Höhe 
$\sin(\varphi)\cdot \left\| \vec{w}\right\|$.
}
\lang{en}{
is exactly the area of the parallelogram spanned by (with edges) $\vec{v}$ and $\vec{w}$, as its 
base edge in the sketch has length $\left\| \vec{v}\right\|$, and the parallelogram has height 
$\sin(\varphi)\cdot \left\| \vec{w}\right\|$.
}


\begin{center}
\image{T109_AreaParalellogram}
\end{center}

\end{remark}

\begin{example}
\lang{de}{
Um den Flächeninhalt des Parallelogramms zu bestimmen, das durch die Vektoren 
$\vec{a}=\begin{pmatrix} 4 \\ 2 \end{pmatrix}$ und $\vec{b}=\begin{pmatrix} 2 \\ 3 \end{pmatrix}$ 
aufgespannt wird, kann die Situation ins Dreidimensionale eingebettet werden und das Vektorprodukt zu 
Hilfe genommen werden.
\\\\
Beim Einbetten ins Dreidimensionale ergänzt man eine dritte Koordinate in den beiden Vektoren mit 
Wert $0$: $\vec{a}=\begin{pmatrix} 4 \\ 2 \\ 0 \end{pmatrix}$ und 
$\vec{b}=\begin{pmatrix} 2 \\ 3 \\ 0 \end{pmatrix}$. Den Flächeninhalt des durch die beiden Vektoren 
aufgespannten Parallelogramms erhält man nun als Länge des Vektorprodukts $\vec{a}\times\vec{b}$:
}
\lang{en}{
To determine the area of the parallelogram spanned by the vectors 
$\vec{a}=\begin{pmatrix} 4 \\ 2 \end{pmatrix}$ and $\vec{b}=\begin{pmatrix} 2 \\ 3 \end{pmatrix}$, 
we can embed the problem into three dimensions so that we may use the cross product.
\\\\
To embed it into three dimensions, we add a third component $0$: 
$\vec{a}=\begin{pmatrix} 4 \\ 2 \\ 0 \end{pmatrix}$ and 
$\vec{b}=\begin{pmatrix} 2 \\ 3 \\ 0 \end{pmatrix}$. The area of the parallelogram spanned by these 
is now the length of the cross product $\vec{a}\times\vec{b}$:
}
\\\\
\[\left\| \vec{a}\times\vec{b} \right\|=\left\| \begin{pmatrix} 4 \\ 2 \\ 0 \end{pmatrix} \times \begin{pmatrix} 2 \\ 3 \\ 0 \end{pmatrix} 
\right\|=\left\| \begin{pmatrix} 0 \\ 0 \\ 8 \end{pmatrix} \right\|=\sqrt{64}=8\]
\end{example}

\begin{remark}\label{rem:spatprodukt}
\lang{de}{
Das sogenannte Spatprodukt ist das Skalarprodukt aus dem Vektorprodukt zweier Vektoren $\vec{v}$ und 
$\vec{w}$ und einem dritten Vektor $\vec{u}$. Der Betrag des Spatprodukts entspricht dem Volumen des 
durch die drei Vektoren $\vec{v},\vec{w}$ und $\vec{u}$ aufgespannten Spats.
}
\lang{en}{
The so-called \emph{scalar triple product} is the scalar product of the cross product of $\vec{v}$ 
and $\vec{w}$ with a third vector $\vec{u}$. The absolute value of this corresponds to the volume of 
the parallelopiped (three-dimensional parallelogram) spanned by the vectors.
}
\begin{center}
\image{T109_TripleProduct}
\end{center}
\lang{de}{
Damit ergibt sich das Volumen $V(\vec{v},\vec{w},\vec{u})$ des durch die drei Vektoren 
$\vec{v},\vec{w},\vec{u}$ aufgespannten Spats zu:
}
\lang{en}{
Hence the volume $V(\vec{v},\vec{w},\vec{u})$ of the parallelopiped spanned by the vectors 
$\vec{v},\vec{w},\vec{u}$ is:
}\\
\[V(\vec{v},\vec{w},\vec{u})=\left|(\vec{v}\times\vec{w})\bullet\vec{u}\right|\]
\end{remark}

\begin{example}
\lang{de}{
Um das Volumen $V$ des durch die drei Vektoren $\vec{a}=\begin{pmatrix} 4 \\ 2 \\ 0 \end{pmatrix}$, 
$\vec{b}=\begin{pmatrix} 1 \\ 1 \\ -1 \end{pmatrix}$ und 
$\vec{c}=\begin{pmatrix} 0 \\ 2 \\ 3 \end{pmatrix}$ aufgespannten Spats zu bestimmen, können Skalar- 
und Vektorprodukt zu Hilfe genommen werden:
}
\lang{en}{
To determine the volume $V$ of the parallelopiped spanned by the vectors 
$\vec{a}=\begin{pmatrix} 4 \\ 2 \\ 0 \end{pmatrix}$, 
$\vec{b}=\begin{pmatrix} 1 \\ 1 \\ -1 \end{pmatrix}$ and 
$\vec{c}=\begin{pmatrix} 0 \\ 2 \\ 3 \end{pmatrix}$, we use both the scalar product and the cross 
product:
}\\
\[V=\left|(\vec{a}\times\vec{b})\bullet\vec{c}\right|=\left|\begin{pmatrix} -2 \\ 4 \\ 2 \end{pmatrix}\bullet\begin{pmatrix} 0 \\ 2 \\ 3 \end{pmatrix}\right|
=\left| (-2)\cdot 0+4\cdot 2+2\cdot 3 \right|=14\]
\end{example}

\begin{quickcheck}
		\field{rational}
		\type{input.number}
		\begin{variables}
			\randint{a}{2}{9}
            \number{AB1}{5}
            \number{AB2}{-1*AB1}
            \number{AB3}{0}
            \function[normalize]{AC1}{a+1}
            \function[normalize]{AC2}{a-3}
            \number{AC3}{0}
            
            \number{Vektorprodukt1}{0}
            \number{Vektorprodukt2}{0}
            \function[normalize]{Vektorprodukt3}{10*a-10}
            
            \function[normalize]{flaeche}{Vektorprodukt3/2}            
            
		\end{variables}
		
			\text{\lang{de}{
            Bestimmen Sie den Flächeninhalt des Dreiecks mit den Eckpunkten $A=(-1;3)$, $B=(4;-2)$ 
            und $C=(\var{a};\var{a})$.
            \\\\
            Der Flächeninhalt des Dreiecks beträgt \ansref.
            }
            \lang{en}{
            Determine the area of the triangle with vertices $A=(-1;3)$, $B=(4;-2)$ and 
            $C=(\var{a};\var{a})$.
            \\\\
            The area of the triangle is \ansref.
            }
			}
		
		\begin{answer}
			\solution{flaeche}
		\end{answer}
		
		\explanation{\lang{de}{
        Um die Fläche des Dreiecks zu bestimmen, können Sie die Verbindungsvektoren der Eckpunkte 
        nehmen und sie ins Dreidimensionale einbetten. Dann verwenden Sie das Vektorprodukt, um die 
        Fläche des von den Verbindungsvektoren aufgespannten Parallelogramms zu bestimmen. Die 
        Dreiecksfläche ergibt sich dann als die Hälfte der Parallelogrammfläche. Also ist die Fläche 
        gleich
        }
        \lang{en}{
        To determine the area of the triangle, two of the vectors representing the edges of the 
        triangle (those between the vertices) can be embedded into three dimensions, and their cross 
        product taken to find the area of the parallelogram spanned by them. The area of the triangle 
        is simply half of this. Hence the area of the triangle is
        }\\
        $
        \frac{1}{2}\cdot \left| \begin{pmatrix}5\\-5\\ 0\end{pmatrix} \times 
        \begin{pmatrix}\var{AC1} \\ \var{AC2} \\ \var{AC3}\end{pmatrix}\right|.
        $
    		}
	\end{quickcheck}

\end{content}