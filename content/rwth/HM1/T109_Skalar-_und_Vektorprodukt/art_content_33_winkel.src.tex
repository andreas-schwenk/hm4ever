%$Id:  $
\documentclass{mumie.article}
%$Id$
\begin{metainfo}
  \name{
    \lang{de}{Winkel zwischen Vektoren}
    \lang{en}{Angles between vectors}
  }
  \begin{description} 
 This work is licensed under the Creative Commons License Attribution 4.0 International (CC-BY 4.0)   
 https://creativecommons.org/licenses/by/4.0/legalcode 

    \lang{de}{Beschreibung}
    \lang{en}{Description}
  \end{description}
  \begin{components}
    \component{generic_image}{content/rwth/HM1/images/g_tkz_T109_VectorProjection.meta.xml}{T109_VectorProjection}
    \component{generic_image}{content/rwth/HM1/images/g_tkz_T109_Angle.meta.xml}{T109_Angle}
    \component{generic_image}{content/rwth/HM1/images/g_tkz_T109_Square.meta.xml}{T109_Square}
    \component{generic_image}{content/rwth/HM1/images/g_tkz_T109_OrthogonalVectors.meta.xml}{T109_OrthogonalVectors}
    \component{generic_image}{content/rwth/HM1/images/g_img_00_Videobutton_schwarz.meta.xml}{00_Videobutton_schwarz}
    \component{js_lib}{system/media/mathlets/GWTGenericVisualization.meta.xml}{mathlet1}
  \end{components}
  \begin{links}
    \link{generic_article}{content/rwth/HM1/T105_Trigonometrische_Funktionen/g_art_content_17_trigonometrie_im_dreieck.meta.xml}{dreiecks-trig}
    \link{generic_article}{content/rwth/HM1/T110_Geraden,_Ebenen/g_art_content_38_abstaende.meta.xml}{abstaende}
    \link{generic_article}{content/rwth/HM1/T109_Skalar-_und_Vektorprodukt/g_art_content_31_skalarprodukt.meta.xml}{skalarprodukt}
  \end{links}
  \creategeneric
\end{metainfo}
\begin{content}
\usepackage{mumie.ombplus}
\ombchapter{9}
\ombarticle{3}
\usepackage{mumie.genericvisualization}

\begin{visualizationwrapper}

\title{\lang{de}{Winkel zwischen Vektoren}\lang{en}{Angles between vectors}}
 
\begin{block}[annotation]
  übungsinhalt
  
\end{block}
\begin{block}[annotation]
  Im Ticket-System: \href{http://team.mumie.net/issues/9050}{Ticket 9050}\\
\end{block}

\begin{block}[info-box]
\tableofcontents
\end{block}

\lang{de}{
In Abschnitt \ref[skalarprodukt][Skalarprodukt]{skalarprodukt_winkel} haben wir bereits eine 
geometrische Interpretation im Hinblick auf einen Zusammenhang mit Winkeln zwischen zwei Vektoren und 
dem Skalarprodukt hergestellt. Dieser soll hier vertieft und erklärt werden.
}
\lang{en}{
In \ref[skalarprodukt][the section where scalar products were introduced]{skalarprodukt_winkel}, 
we already gave a geometric interpretation relating the angle between two vectors and their scalar 
product. This will be elaborated on in this section.
}

\section{\lang{de}{Orthogonalität}\lang{en}{Orthogonality}}

\begin{definition}\label{def:orthogonal}
\lang{de}{
Zwei Vektoren $\vec{v}, \vec{w}\in \R^n$ sind \emph{senkrecht} (auch \emph{orthogonal} genannt) 
zueinander, wenn $\vec{v}\bullet \vec{w}=0$ gilt.\\
\floatright{\href{https://www.hm-kompakt.de/video?watch=719}{\image[75]{00_Videobutton_schwarz}}}\\\\
}
\lang{en}{
Two vectors $\vec{v}, \vec{w}\in \R^n$ are called \emph{orthogonal} if $\vec{v}\bullet \vec{w}=0$.\\
}
\end{definition}

\begin{remark}\label{rem:orthogonal_herleitung}
\begin{enumerate}
\item \lang{de}{
      Diese Definition bedeutet mit der Anschauung im $\mathbb{R}^2$ und $\mathbb{R}^3$, dass zwei 
      Vektoren $\vec{v}$ und $\vec{w}$ einen $90^\circ$-Winkel bilden, wenn das Skalarprodukt 
      $\vec{v}\bullet\vec{w}=0$ ist.
      }
      \lang{en}{
      This definition means that in the spaces $\mathbb{R}^2$ and $\mathbb{R}^3$, two vectors 
      $\vec{v}$ and $\vec{w}$ have a $90^\circ$ degree angle between them if 
      $\vec{v}\bullet\vec{w}=0$.
      }
\item \lang{de}{
      Wie man mit Hilfe des \ref[dreiecks-trig][Satzes des Pythagoras]{satz-des-pythagoras} sehen 
      kann, stimmt die Definition mit der Anschauung im $\mathbb{R}^2$ und $\mathbb{R}^3$ überein:
      }
      \lang{en}{
      As can be seen using the \ref[dreiecks-trig][Pythagorean theorem]{satz-des-pythagoras}, 
      the definition agrees with the graphical representation of vectors as arrows:
      }

      \begin{center}
      \image{T109_OrthogonalVectors}
      \end{center}

      \lang{de}{
      Der Satz des Pythagoras sagt f"ur die Vektoren in der Skizze aus, dass $\vec{v}$ und $\vec{w}$ 
      anschaulich genau dann senkrecht sind, wenn 
      $\left\| \vec{v}\right\|^2+\left\|\vec{w}\right\|^2=\left\| \vec{v}+ \vec{w}\right\|^2$ gilt. 
      \\\\
      Die linke Seite der Gleichung ist: $ \left\| \vec{v}\right\|^2+\left\|\vec{w}\right\|^2 = 
      \vec{v}\bullet \vec{v} + \vec{w}\bullet \vec{w} $
      \\\\
      Die rechte Seite ist mit Hilfe der 
      \ref[skalarprodukt][Distributivit"at und Kommutativit"at]{sec:regeln}:
      }
      \lang{en}{
      There is a variation of the Pythagorean theorem which states that for the vectors in the above 
      graphic, $\vec{v}$ and $\vec{w}$ are at a right-angle to each other (orthogonal) if and only if 
      $\left\| \vec{v}\right\|^2+\left\|\vec{w}\right\|^2=\left\| \vec{v}+ \vec{w}\right\|^2$. 
      \\\\
      The left side of the equation is: $ \left\| \vec{v}\right\|^2+\left\|\vec{w}\right\|^2 = 
      \vec{v}\bullet \vec{v} + \vec{w}\bullet \vec{w} $
      \\\\
      The right side of the equation is, by the 
      \ref[skalarprodukt][distributivity and commutativity]{sec:regeln} of the scalar product: 
      }
      \begin{eqnarray*} 
      \left\| \vec{v}+ \vec{w}\right\|^2 &=& ( \vec{v}+ \vec{w})\bullet  (\vec{v}+ \vec{w}) = 
      \vec{v}\bullet \vec{v} + \vec{w}\bullet \vec{v} + \vec{v}\bullet \vec{w} + 
        \vec{w}\bullet \vec{w} \\ &=& 
      \vec{v}\bullet \vec{v} +  \vec{w}\bullet \vec{w} + 2\cdot \vec{v}\bullet \vec{w} 
      \end{eqnarray*}
      \lang{de}{Also ist die Gleichheit genau dann gegeben, wenn $\vec{v}\bullet \vec{w}=0$.}
      \lang{en}{Hence equality occurs if and only if $\vec{v}\bullet \vec{w}=0$.}
\end{enumerate}
\end{remark}

\begin{example}
\lang{de}{
Die Vektoren $\vec{v}= \begin{pmatrix} 2\\ 1\\ 4\end{pmatrix}$ und 
$\vec{w}= \begin{pmatrix} 1\\ -6\\ 1\end{pmatrix}$ sind zueinander orthogonal/senkrecht, denn
}
\lang{en}{
The vectors $\vec{v}= \begin{pmatrix} 2\\ 1\\ 4\end{pmatrix}$ and 
$\vec{w}= \begin{pmatrix} 1\\ -6\\ 1\end{pmatrix}$ are orthogonal, as
}
\[ \vec{v}\bullet \vec{w}= \begin{pmatrix} 2\\ 1\\ 4\end{pmatrix}\bullet \begin{pmatrix} 1\\ -6\\ 1\end{pmatrix}=
2\cdot 1+1\cdot (-6)+4\cdot 1=0. \]
\lang{de}{
Der Vektor $\vec{z}= \begin{pmatrix} 2\\ 0\\ -1 \end{pmatrix}$ ist zu $\vec{v}$ auch senkrecht, jedoch nicht zu $\vec{w}$, denn
}
\lang{en}{
The vector $\vec{z}= \begin{pmatrix} 2\\ 0\\ -1 \end{pmatrix}$ is also orthogonal to $\vec{v}$, but 
not to $\vec{w}$, as
}
\[ \vec{z}\bullet \vec{v}= \begin{pmatrix} 2\\ 0\\ -1 \end{pmatrix}\bullet \begin{pmatrix} 2\\ 1\\ 4\end{pmatrix}=0 \]
\lang{de}{und}\lang{en}{and}
\[ \vec{z}\bullet \vec{w}= \begin{pmatrix} 2\\ 0\\ -1 \end{pmatrix}\bullet \begin{pmatrix} 1\\ -6\\ 1\end{pmatrix}=1\neq 0. \]
\end{example}

\begin{example}
\lang{de}{
Im $\R^2$ ist das Viereck $ABCD$ mit den Eckpunkten $A=(2; 3)$, $B=(4; 6)$,  $C=(1; 8)$ und 
$D=(-1; 5)$ gegeben. Ist dieses Viereck ein Rechteck oder sogar ein Quadrat?
}
\lang{en}{
Consider the quadrilateral $ABCD$ in $\R^2$ with vertices $A=(2; 3)$, $B=(4; 6)$,  $C=(1; 8)$ and 
$D=(-1; 5)$. Is this quadrilateral a rectangle, or even a square?
}

\begin{center}
\image{T109_Square}
\end{center}

\begin{tabs*}[\initialtab{0}]
\tab{\lang{de}{Definition Rechteck}\lang{en}{Definition of a rectangle}}
\lang{de}{
Ein ebenes Viereck ist ein \emph{Rechteck}, wenn eine der folgenden "aquivalenten Bedingungen 
erf"ullt ist:
}
\lang{en}{
A quadrilateral is a \emph{rectangle} if one of the following conditions is met:
}
\begin{enumerate}
\item \lang{de}{Alle Innenwinkel sind rechte Winkel.}
      \lang{en}{All interior angles are right angles.}
\item \lang{de}{Die Diagonalen sind gleich lang und halbieren sich gegenseitig.}
      \lang{en}{The diagonals are the same lengths and intersect each other at their centres.}
\item \lang{de}{Das Viereck ist ein Parallelogramm und besitzt einen rechten Winkel.}
      \lang{en}{The quadrilateral is a parallelogram and one of its angles is a right angle.}
\item \lang{de}{Das Viereck ist ein Parallelogramm und die Diagonalen sind gleich lang.}
      \lang{en}{The quadrilateral is a parallelogram and its diagonals are the same length.}
\end{enumerate}
\lang{de}{Bemerkung: Alle Bedingungen implizieren sogar, dass das Viereck eben ist.}
\lang{en}{} %All quadrilaterals are on a plane in English.

\tab{\lang{de}{Definition Quadrat}\lang{en}{Definition of a square}}
\lang{de}{Ein \emph{Quadrat} ist ein Rechteck, dessen Seiten alle gleich lang sind.}
\lang{en}{A \emph{square} is a rectangle whose edges are all the same length.}
\end{tabs*}

\lang{de}{
Um zu untersuchen, ob das Viereck ein Rechteck ist, ist am einfachsten zu untersuchen, ob das Viereck 
ein Parallelogramm ist und einen rechten Winkel besitzt. 
Dazu muss nur getestet werden, ob $\overrightarrow{AB}=\overrightarrow{DC}$ und 
$\overrightarrow{AB}\bullet \overrightarrow{AD}=0$ gilt.
}
\lang{en}{
The simplest method to determine if the quadrilateral $ABCD$ is a rectangle is to see if it is a 
parallelogram and if it contains a right angle. 
That is, we check if $\overrightarrow{AB}=\overrightarrow{DC}$ and if 
$\overrightarrow{AB}\bullet \overrightarrow{AD}=0$.
}

\[ \overrightarrow{AB}=\left( \begin{smallmatrix} 4-2 \\ 6-3 \end{smallmatrix} \right) = 
   \left( \begin{smallmatrix} 2 \\ 3 \end{smallmatrix} \right), \quad \overrightarrow{DC} = 
   \left( \begin{smallmatrix} 1-(-1) \\ 8-5 \end{smallmatrix} \right) = 
   \left( \begin{smallmatrix} 2 \\ 3 \end{smallmatrix} \right)\quad 
   \text{\lang{de}{und}\lang{en}{and}} 
   \quad  \overrightarrow{AD}=\left( \begin{smallmatrix} -1-2 \\ 5-3 \end{smallmatrix} \right) = 
   \left( \begin{smallmatrix} -3 \\ 2 \end{smallmatrix} \right). \]

\lang{de}{
Also gilt $\overrightarrow{AB}=\overrightarrow{DC}$ und 
$\overrightarrow{AB}\bullet \overrightarrow{AD} = 
\left( \begin{smallmatrix} 2 \\ 3 \end{smallmatrix} \right)\bullet 
\left( \begin{smallmatrix} -3 \\ 2 \end{smallmatrix} \right) = 0$, d.h. das Viereck ist ein Rechteck.
\\\\
Zuletzt m"ussen wir noch testen, ob 
$\left\| \overrightarrow{AB}\right\|=\left\|\overrightarrow{AD}\right\|$ gilt:
}
\lang{en}{
Hence $\overrightarrow{AB}=\overrightarrow{DC}$ and 
$\overrightarrow{AB}\bullet \overrightarrow{AD} = 
\left( \begin{smallmatrix} 2 \\ 3 \end{smallmatrix} \right)\bullet 
\left( \begin{smallmatrix} -3 \\ 2 \end{smallmatrix} \right) = 0$, 
so the quadrilateral is a rectangle.
\\\\
Finally we check if 
$\left\| \overrightarrow{AB}\right\|=\left\|\overrightarrow{AD}\right\|$ holds:
}
\[ \left\| \overrightarrow{AB}\right\|=\sqrt{2^2+3^2} = \sqrt{13}\quad 
\text{\lang{de}{und}\lang{en}{and}}
\quad \left\| \overrightarrow{AD}\right\| = \sqrt{(-3)^2+2^2}=\sqrt{13}. \]
\lang{de}{Also ist das Viereck sogar ein Quadrat.}
\lang{en}{Hence the quadrilateral is in fact a square.}
\end{example}

\begin{quickcheck}
		\field{rational}
		\type{input.number}
		\begin{variables}
			
			\randint{k}{0}{3}  % Zufallsvariable für verschiedene Fälle
			\function[calculate]{d0}{-(k-1)*(k-2)*(k-3)/6}  % "Dirac"-funktionen
			\function[calculate]{d1}{k*(k-2)*(k-3)/2}
			\function[calculate]{d2}{-k*(k-1)*(k-3)/2}
			\function[calculate]{d3}{k*(k-1)*(k-2)/6}
			
			\randint[Z]{u}{-2}{3}
			\randint[Z]{t}{-2}{3}
			\randint{v1}{1}{4}  
			\randint{v2}{-3}{1}

			\function[calculate]{fr1}{u*v1}  	
			\function[calculate]{fr2}{u*v2}

			\function[calculate]{fq1}{t*v2}  
			\function[calculate]{fq2}{-t*v1} 

			\function[calculate]{r1}{d0*(fr1-2*fr2)+d1*fr1+d2*fr1+d3*(fr1+fq1)}
			\function[calculate]{r2}{d0*(fr2+2*fr1)+d1*fr2+d2*fr2+d3*(fr2+fq2)}
			\function[calculate]{q1}{d0*(fr1+2*fr2)+d1*fq1+d2*(fr1+fq1)+d3*fq1}
			\function[calculate]{q2}{d0*(fr2-2*fr1)+d1*fq2+d2*(fr2+fq2)+d3*fq2}

			\randint{a1}{-6}{1}
			\randint{a2}{0}{3}
			\function[calculate]{b1}{a1+r1}
			\function[calculate]{b2}{a2+r2}
			\function[calculate]{c1}{a1+q1}
			\function[calculate]{c2}{a2+q2}
			
			\function[calculate]{s1}{q1-r1}
			\function[calculate]{s2}{q2-r2}
			
			\function[calculate]{pr1}{r1*q1+r2*q2}
			\function[calculate]{pr2}{r1*s1+r2*s2}
			\function[calculate]{pr3}{q1*s1+q2*s2}
			
		\end{variables}
		
			\text{\lang{de}{
      Welche zwei Seiten des Dreiecks $ABC$ mit
			$A=(\var{a1};\var{a2})$, $B=(\var{b1};\var{b2})$ und $C=(\var{c1};\var{c2})$
			stehen senkrecht aufeinander?
      }
      \lang{en}{
      Which two edges of the triangle $ABC$ with 
      $A=(\var{a1};\var{a2})$, $B=(\var{b1};\var{b2})$ and $C=(\var{c1};\var{c2})$ 
      are orthogonal?
      }\\
			\begin{table}
			\nowrap{0) \lang{de}{keine der Seiten}\lang{en}{None of the edges}} & & 
        \nowrap{ 1) \lang{de}{die Seiten $AB$ und $AC$}\lang{en}{The sides $AB$ and $AC$}} \\ 
			\nowrap{2) \lang{de}{die Seiten $AB$ und $BC$}\lang{en}{The sides $AB$ and $BC$}} & & 
        \nowrap{ 3) \lang{de}{die Seiten $AC$ und $BC$.}\lang{en}{The sides $AC$ and $BC$}}
			\end{table} 
      \\\\
      \lang{de}{Antwort: \ansref}
      \lang{en}{Answer: \ansref}
			}
		
		\begin{answer}
			\solution{k}
		\end{answer}
		\explanation{\lang{de}{
		Die Verbindungsvektoren sind 
    $\overrightarrow{AB}=\begin{pmatrix} \var{r1}\\ \var{r2}\end{pmatrix}$,
		$\overrightarrow{AC}=\begin{pmatrix} \var{q1}\\ \var{q2}\end{pmatrix}$ und
		$\overrightarrow{BC}=\begin{pmatrix} \var{s1}\\ \var{s2}\end{pmatrix}$.\\
		Um herauszufinden, welche Seiten zueinander senkrecht sind, muss man also nachprüfen, welche 
    dieser Vektoren zueinander senkrecht sind:
    }
    \lang{en}{
    The vectors representing the edges are 
    $\overrightarrow{AB}=\begin{pmatrix} \var{r1}\\ \var{r2}\end{pmatrix}$,
		$\overrightarrow{AC}=\begin{pmatrix} \var{q1}\\ \var{q2}\end{pmatrix}$ and
		$\overrightarrow{BC}=\begin{pmatrix} \var{s1}\\ \var{s2}\end{pmatrix}$.\\
    To find out which edges are orthogonal, we check which of these vectors are orthogonal:
    }
		\begin{eqnarray*}
		\overrightarrow{AB}\bullet \overrightarrow{AC}&=& \var{r1}\cdot \var{q1}+\var{r2}\cdot \var{q2}=\var{pr1}, \\
		\overrightarrow{AB}\bullet \overrightarrow{BC}&=& \var{r1}\cdot \var{s1}+\var{r2}\cdot \var{s2}=\var{pr2}, \\
		\overrightarrow{AC}\bullet \overrightarrow{BC}&=& \var{q1}\cdot \var{s1}+\var{q2}\cdot \var{s2}=\var{pr3}. \\
		\end{eqnarray*}
		\lang{de}{Dies zeigt, dass Antwort $\var{k}$ richtig ist.}
    \lang{en}{This shows that answer $\var{k}$ is correct.}
 		}
	\end{quickcheck}
	

\section{\lang{de}{Winkel}\lang{en}{Angles}}

\lang{de}{
Mit Hilfe des \ref[skalarprodukt][Skalarprodukts]{def:skalarprodukt} lassen sich auch Winkel zwischen 
den Vektoren definieren. Wie auch die Länge und die Orthogonalität von Vektoren stimmen für den 
$\R^2$ und den $\R^3$ die so definierten Winkel mit der Anschauung überein. Der Nachweis der 
Übereinstimmung benötigt jedoch den Kosinussatz für allgemeine Dreiecke und wird daher hier nicht 
aufgeführt.
}
\lang{en}{
We can also define the angle between vectors using the 
\ref[skalarprodukt][scalar product]{def:skalarprodukt}. Much like the length and orthogonality of 
vectors in $\R^2$ and $\R^3$, the upcoming definition for the angle corresponds with the angle in 
the visual representation of the vectors. Proving this agreement between representations requires 
the cosine rule for general triangles, so we do not give the proof here.
}

\begin{definition}\label{def:winkel}
\lang{de}{
Der Winkel $\varphi$ zwischen zwei Vektoren $\vec{v}$ und $\vec{w}$ im $\R^n$ ist der eindeutige 
Winkel zwischen $0^\circ$ und $180^\circ$ (bzw. die Zahl zwischen $0$ und $\pi$), f"ur welchen
}
\lang{en}{
The angle $\varphi$ between two vectors $\vec{v}$ and $\vec{w}$ in $\R^n$ is the unique angle between 
$0^\circ$ and $180^\circ$ (or the number between $0$ and $\pi$ in radians) for which
}
\[ \cos(\varphi)= \frac{\vec{v}\bullet \vec{w}}{\left\|\vec{v}\right\|\cdot \left\|\vec{w}\right\|} \]
\lang{de}{gilt.}
\lang{en}{holds.}

\begin{center}
\image{T109_Angle}
\end{center}

\lang{de}{
\floatright{\href{https://www.hm-kompakt.de/video?watch=717}{\image[75]{00_Videobutton_schwarz}}}\\\\
}
\lang{en}{}
\end{definition}


\begin{example}
\lang{de}{
F"ur die Vektoren $\vec{v}=\left( \begin{smallmatrix} 1 \\ 0 \\ 2 \end{smallmatrix} \right)$ und $\vec{w}= \left( \begin{smallmatrix} 1 \\ 1 \\ 2 \end{smallmatrix} \right)$
berechnen wir den Winkel $\varphi$ als
}
\lang{en}{
The angle $\varphi$ between the vectors 
$\vec{v}=\left( \begin{smallmatrix} 1 \\ 0 \\ 2 \end{smallmatrix} \right)$ and 
$\vec{w}= \left( \begin{smallmatrix} 1 \\ 1 \\ 2 \end{smallmatrix} \right)$ is
}
\[ \cos(\varphi)=\frac{\left( \begin{smallmatrix} 1 \\ 0 \\ 2 \end{smallmatrix} \right)
\bullet \left( \begin{smallmatrix} 1 \\ 1 \\ 2 \end{smallmatrix} \right)}{\left\|\left( 
\begin{smallmatrix} 1 \\ 0 \\ 2 \end{smallmatrix} \right) \right\|\cdot \left\| \left( 
\begin{smallmatrix} 1 \\ 1 \\ 2 \end{smallmatrix} \right) \right\| }
= \frac{1+0+4}{\sqrt{1^2+0^2+2^2}\cdot \sqrt{1^2+1^2+2^2}}=\frac{5}{\sqrt{30}}
 \]
\lang{de}{Daraus ergibt sich mit dem Taschenrechner $\varphi\approx 24^\circ$.}
\lang{en}{Using a calculator, we get $\varphi\approx 24^\circ$.}
\end{example}


% \begin{remark}
% Eine anschauliche Interpretation des Skalarprodukts zweier Vektoren ergibt sich im Zusammenhang mit
% der Projektion.

% \image[300]{projektion}

% Lässt man die Pfeile für die Vektoren $\vec{v}$ und  $\vec{w}$ wie in der Skizze im Ursprung beginnen, und bezeichnet
% die senkrechte Projektion des Endpunktes zu $\vec{w}$ auf die $\vec{v}$-Richtung mit $P$ und die  
% senkrechte Projektion des Endpunktes zu $\vec{v}$ auf die $\vec{w}$-Richtung mit $Q$, so ist

% \[ \left|\vec{v}\bullet \vec{w}\right| = \left\| \vec{v}\right\| \cdot \left\| \overrightarrow{OP}\right\|, \]
% sowie
% \[ \left|\vec{v}\bullet \vec{w}\right| = \left\| \vec{w}\right\| \cdot \left\| \overrightarrow{OQ}\right\|. \]

% Genauer sind sogar
% \[ \overrightarrow{OP}=\frac{\vec{v}\bullet \vec{w}}{\left\|\vec{v}\right\|^2}\cdot  \vec{v}
% \quad \text{und}\quad  \overrightarrow{OQ}=\frac{\vec{v}\bullet \vec{w}}{\left\|\vec{w}\right\|^2}\cdot  \vec{w}, \]
% d.h. das Vorzeichen des Skalarprodukts zeigt an, ob $\overrightarrow{OP}$ bzw. $\overrightarrow{OQ}$ 
% in die gleiche Richtung zeigt wie $\vec{v}$ bzw. $\vec{w}$ oder in die entgegengesetzte.

% Dies wird bei der Berechnung von \link{abstaende}{Abständen} wieder auftauchen.
% \end{remark}

\begin{remark}
\lang{de}{
Das Skalarprodukt gibt den projezierten Beitrag des einen Vektors in Richtung des anderen Vektors
an. Wir leiten dies anhand der folgenden Skizze her:
}
\lang{en}{
The scalar product gives the projective length of one vector in the direction of the other. The 
meaning of this is shown in the following graphic.
}

\begin{center}
\image{T109_VectorProjection}
\end{center}

\lang{de}{
Lässt man die Pfeile für die Vektoren $\vec{v}$ und $\vec{w}$ wie in der Skizze im Ursprung 
beginnen, und bezeichnet die senkrechte Projektion des Endpunktes zu $\vec{w}$ auf die 
$\vec{v}$-Richtung mit $P$ und die senkrechte Projektion des Endpunktes zu $\vec{v}$ auf die 
$\vec{w}$-Richtung mit $Q$, so haben wir zwei rechtwinklige Dreiecke. Man kann zeigen (siehe unten), 
dass
}
\lang{en}{
Considering the arrows for the vectors $\vec{v}$ and $\vec{w}$, joined at their start-point, we can 
sketch the projection of the vector $\vec{w}$ in the direction of $\vec{v}$, that is, the component 
of the vector $\vec{w}$ going in the direction of $\vec{v}$. This gives us the a right-angled 
triangle with an edge $OP$ along $\vec{v}$. Likewise, projecting the vector $\vec{v}$ in the 
direction of $\vec{w}$ gives us a right-angled triangle with an edge $OQ$ along $\vec{w}$. We can 
show that:
}
\[ \overrightarrow{OP} = 
\frac{\vec{v}\bullet \vec{w}}{\left\|\vec{v}\right\|^2}\cdot  \vec{v}
\quad \text{\lang{de}{und}\lang{en}{and}} \quad 
\overrightarrow{OQ} = 
\frac{\vec{v}\bullet \vec{w}}{\left\|\vec{w}\right\|^2}\cdot \vec{w}, \]
\lang{de}{
d.h. das Vorzeichen des Skalarprodukts zeigt an, ob $\overrightarrow{OP}$ bzw. $\overrightarrow{OQ}$ 
in die gleiche Richtung zeigt wie $\vec{v}$ bzw. $\vec{w}$ oder in die entgegengesetzte.
\\\\
Dies wird bei der Berechnung von \link{abstaende}{Abständen} wieder auftauchen.
}
\lang{en}{
that is, the sign of the scalar product indicates if $\overrightarrow{OP}$ or $\overrightarrow{OQ}$ 
points in the same direction as $\vec{v}$ or $\vec{w}$ respectively, or in the opposite direction.
\\\\
This will reappear when we calculate \link{abstaende}{distances between objects in space}.
}


\begin{tabs*}[\initialtab{0}]
\tab{\lang{de}{Herleitung}\lang{en}{Explanation}}
\lang{de}{Die \ref[dreiecks-trig][Trigonometrie im Dreieck]{def-trigFkt} liefert uns}
\lang{en}{The \ref[dreiecks-trig][definition of cosine]{def-trigFkt} immediately gives us}
\[
\left|{\cos(\varphi)}\right| = \frac{\| \overrightarrow{OP} \|}{\| \vec{w}\|} 
\text{\lang{de}{ bzw. }\lang{en}{ and }} 
\left|\cos(\varphi)\right| = \frac{\| \overrightarrow{OQ} \|}{\| \vec{v}\|}.
\]
\lang{de}{
Der Betrag kommt ins Spiel, da wir in den rechtwinkligen Dreiecken nur Winkel $0<\varphi<90^\circ$ 
messen können. Andererseits kennen wir die Formel für Winkel zwischen Vektoren aus Definition 
\ref{def:winkel}:
}
\lang{en}{
respectively. The absolute value is used here because only angles $0<\varphi<90^\circ$ can be 
measured on the interior of a right-angled triangle. We also know the formula for angles between 
vectors from definition \ref{def:winkel}:
}
\[ \cos(\varphi)= \frac{\vec{v}\bullet \vec{w}}{\left\|\vec{v}\right\|\cdot \left\|\vec{w}\right\|} \]
\lang{de}{Setze nun diese Gleichung oben ein und löse auf. Wir erhalten}
\lang{en}{Substituting this equation into the earlier two equations gives}
\[ \left|\vec{v}\bullet \vec{w}\right| = 
\left\| \vec{v}\right\| \cdot \left\| \overrightarrow{OP}\right\|, \]
\lang{de}{sowie}\lang{en}{and}
\[ \left|\vec{v}\bullet \vec{w}\right| = 
\left\| \vec{w}\right\| \cdot \left\| \overrightarrow{OQ}\right\|. \]
\lang{de}{
Das Vorzeichen des Skalarprodukt gibt nun an, ob das rechtwinklige Dreieck in Richtung $\vec{v}$ bzw. 
$\vec{w}$ oder in die entgegengesetzte Richtung gerichtet ist. Also je nachdem, ob $\varphi<90^\circ$ 
oder $\varphi>90^\circ$.
}
\lang{en}{
Furthermore, the sign of the scalar product reveals whether the right-angled triangle 'points' in the 
direction of $\vec{v}$ or $\vec{w}$, or in the opposite direction. That is, it reveals 
whether $\varphi<90^\circ$ or $\varphi>90^\circ$.
}
\end{tabs*}
\end{remark}

\begin{quickcheck}
		\field{real}
		\type{input.number}
		\displayprecision{1}
  		\correctorprecision{1}
  		\begin{variables}
			
			\randint{r1}{2}{5}  
			\randint{r2}{-3}{1}
			\randint{q1}{1}{4}  
			\randint{q2}{2}{5}


			\randint{a1}{-3}{1}
			\randint{a2}{0}{3}
			\function[calculate]{b1}{a1+r1}
			\function[calculate]{b2}{a2+r2}
			\function[calculate]{c1}{a1+q1}
			\function[calculate]{c2}{a2+q2}
			
			\function[calculate]{pr}{r1*q1+r2*q2}
			\function[calculate]{nr}{r1^2+r2^2}
			\function[calculate]{nq}{q1^2+q2^2}
			
			\function{cs}{pr/(sqrt(nr)*sqrt(nq))}
			
			\function[calculate]{al}{arccos(cs)*180/pi}
			
		\end{variables}
		
			\text{\lang{de}{
      Bestimmen Sie mit Hilfe des Taschenrechners im Dreieck $ABC$ den Innenwinkel $\alpha$ bei $A$, 
			wobei $A=(\var{a1};\var{a2})$, $B=(\var{b1};\var{b2})$ und $C=(\var{c1};\var{c2})$.
      \\\\
			Der Winkel ist $\alpha\approx $\ansref$\phantom{I}^\circ$ (auf eine Stelle nach dem Komma 
      gerundet).
      }
      \lang{en}{
      Using a calculator, determine the interior angle $\alpha$ of the triangle $ABC$ at $A$, where 
      $A=(\var{a1};\var{a2})$, $B=(\var{b1};\var{b2})$ and $C=(\var{c1};\var{c2})$.
      \\\\
      The angle is $\alpha\approx $\ansref$\phantom{I}^\circ$ (rounded to one decimal place after 
      the comma).
      }}
		
		\begin{answer}
			\solution{al}
		\end{answer}
		\explanation{\lang{de}{Es ist der Winkel zwischen den Verbindungsvektoren }
                 \lang{en}{We calculate the angle between the vectors }
		$\overrightarrow{AB}=\begin{pmatrix} \var{r1}\\ \var{r2}\end{pmatrix}$ 
    \lang{de}{und}\lang{en}{and}
		$\overrightarrow{AC}=\begin{pmatrix} \var{q1}\\ \var{q2}\end{pmatrix}$ 
    \lang{de}{zu berechnen.\\Mit obiger Formel gilt:}
    \lang{en}{.\\The above formula gives:}\\
		$ \cos(\alpha)=\frac{\overrightarrow{AB}\bullet \overrightarrow{AC}}{\left\| 
		\overrightarrow{AB}\right\|\cdot \left\| \overrightarrow{AC}\right\|}
		=\frac{\var{pr}}{\sqrt{\var{nr}}\cdot \sqrt{\var{nq}}}. $\\
		\lang{de}{Mit dem Taschenrechner erhält man damit im Gradmaß:}
    \lang{en}{Using a calculator we find the angle}\\
		$  \alpha\approx\var{al}^\circ . $
 		}
	\end{quickcheck}

\lang{de}{
Mit der folgenden Grafik können Sie sich die Situation durch Wahl der Punkte veranschaulichen.
}
\lang{en}{
The following interactive plot can be used to visualise the situation.
}

	\begin{genericGWTVisualization}[550][1000]{mathlet1}
		\begin{variables}
			\point[editable]{A}{rational}{0,0}
			\point[editable]{B}{rational}{2,0}
			\point[editable]{C}{rational}{1,3}
			\segment{ab}{rational}{var(A),var(B)}
			\segment{ac}{rational}{var(A),var(C)}
			\segment{bc}{rational}{var(B),var(C)}
			\number{p1}{real}{(var(B)[y]-var(A)[y])/sqrt((var(B)[x]-var(A)[x])^2+(var(B)[y]-var(A)[y])^2)}
			\number{p2}{real}{(var(C)[y]-var(A)[y])/sqrt((var(C)[x]-var(A)[x])^2+(var(C)[y]-var(A)[y])^2)}
			\angle{win}{real}{var(A), 0.7, arcsin(var(p1)), arcsin(var(p2))}
			%\number{al}{real}{(arcsin(var(p2))-arcsin(var(p1)))*180/pi}
			
		\end{variables}


		\color{A}{#0066CC}
		\label{A}{$\textcolor{#0066CC}{A}$}
		\color{B}{#0066CC}
		\label{B}{$\textcolor{#0066CC}{B}$}
		\color{C}{#0066CC}
		\label{C}{$\textcolor{#0066CC}{C}$}
		\label{win}{$\phantom{xx}\alpha$}

		\begin{canvas}
			\updateOnDrag[false]
			\plotSize{300}
			\plotLeft{-4}
			\plotRight{4}
			\plot[coordinateSystem]{ab,ac,bc,A,B,C,win}
		\end{canvas}
		\lang{de}{\text{
    $A=(\var{A}[x];\var{A}[y])$, $B=(\var{B}[x];\var{B}[y])$ und $C=(\var{C}[x];\var{C}[y])$.
    }}
    \lang{en}{\text{
    $A=(\var{A}[x];\var{A}[y])$, $B=(\var{B}[x];\var{B}[y])$ and $C=(\var{C}[x];\var{C}[y])$.
    }}
	    	\end{genericGWTVisualization}

\end{visualizationwrapper}


\end{content}