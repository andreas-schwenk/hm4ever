\documentclass{mumie.element.exercise}
%$Id$
\begin{metainfo}
  \name{
    \lang{de}{Ü03: Skalarprodukt}
    \lang{en}{Exercise 3}
  }
  \begin{description} 
 This work is licensed under the Creative Commons License Attribution 4.0 International (CC-BY 4.0)   
 https://creativecommons.org/licenses/by/4.0/legalcode 

    \lang{de}{Rechnen mit Matritzen}
    \lang{en}{}
  \end{description}
  \begin{components}
  \end{components}
  \begin{links}
  \end{links}
  \creategeneric
\end{metainfo}
\begin{content}
\usepackage{mumie.ombplus}

\title{
  \lang{de}{Ü03: Skalarprodukt}
}


\begin{block}[annotation]
  Im Ticket-System: \href{http://team.mumie.net/issues/9703}{Ticket 9703}
\end{block}



\lang{de}{ 
Bestimmen Sie $x>0$ so, dass die folgenden Gleichungen stimmen.
\begin{enumerate}[a)]
 \item a)
\[
  \begin{pmatrix}1\\ x \end{pmatrix}\bullet \begin{pmatrix}-4\\ x \end{pmatrix}  =0
\]
 \item b)
\[
   \begin{pmatrix}-2\\ x \end{pmatrix}\bullet \begin{pmatrix}-3\\ -x \end{pmatrix}  =x
\]
\item c)
\[
  \left(\begin{pmatrix}1 \\ 2 \end{pmatrix}+ x\begin{pmatrix}3\\ 7 \end{pmatrix}\right) \bullet \begin{pmatrix}-1 \\ 5 \end{pmatrix}  =17
\]
\item d)
\[
  \left(\begin{pmatrix}1 \\ 3 \end{pmatrix}+ x\begin{pmatrix}2\\ -4 \end{pmatrix}\right) \bullet \begin{pmatrix}0\\ x \end{pmatrix}  =0
\]
\item e)
\[
  \left( \begin{pmatrix}2\\ 4\\ 5 \end{pmatrix}+ \begin{pmatrix}x\\ -2\\ -x \end{pmatrix}\right) \bullet \begin{pmatrix}1\\ -1\\ -1 \end{pmatrix}  =5-3x
\]
\end{enumerate}}

\begin{tabs*}[\initialtab{0}\class{exercise}]
  \tab{
  \lang{de}{Antworten}
  \lang{en}{Answers}
  }


\begin{table}[\class{items}]
a) $x=2$\\
b) $x=2$\\
c) $x=\frac{1}{4}$\\
d) $x=\frac{3}{4}$\\
e) $x=2$
\end{table}
  \tab{
  \lang{de}{Lösung a)}}



  \begin{incremental}[\initialsteps{1}]
  \step
  
  \lang{de}{Es gilt
\[
   \begin{pmatrix}1\\ x \end{pmatrix}\bullet \begin{pmatrix}-4\\ x \end{pmatrix} =-4+x^2.
\]}
\step

\lang{de}{Wir lösen also
  \[
   -4+x^2=0 \iff x=2 \quad \text{oder}\quad x=-2
  \]
Da wir nach $x>0$ suchen, erhalten wir $x=2$.
  }
  \end{incremental}

 \tab{\lang{de}{Lösung b)}}
 \begin{incremental}[\initialsteps{1}]
  \step
 
  \lang{de}{Es gilt
\[
  \begin{pmatrix}-2\\ x \end{pmatrix} \bullet \begin{pmatrix}-3\\ -x \end{pmatrix} =6-x^2.
\]}
\step
\lang{de}{
Wir lösen also
  \[
   6-x^2=x \iff x^2+x-6=0.
  \]
Die $pq$-Formel ergibt
  \[
   x_{1,2}=-\frac{1}{2} \pm \sqrt{\frac{1}{4}+6}=-\frac{1}{2}\pm \sqrt{\frac{1}{4}+\frac{24}{4}}=-\frac{1}{2} \pm \frac{5}{2},
  \]
also $x_1=-3$ und $x_2=2$. Da wir nach $x>0$ suchen, erhalten wir $x=2$ als Lösung.}
\end{incremental}
	
	 \tab{\lang{de}{Lösung c)}}
	 \begin{incremental}[\initialsteps{1}]
  \step
	 
  \lang{de}{Es gilt
\[
   \left( \begin{pmatrix}1\\ 2 \end{pmatrix}+ x\begin{pmatrix}3\\ 7 \end{pmatrix}\right) \bullet \begin{pmatrix}-1\\ 5 \end{pmatrix}  
   =  \begin{pmatrix}1+3x\\ 2+7x \end{pmatrix}\bullet \begin{pmatrix}-1\\ 5 \end{pmatrix}  =-1-3x+10+35x=9+32x.
\]}
\step
\lang{de}{
Wir lösen also
  \[
   9+32x=17 \iff 32x=8 \iff x=\frac{1}{4}
  \]}
	\end{incremental}
	
	 \tab{\lang{de}{Lösung d)}}
	 \begin{incremental}[\initialsteps{1}]
  \step
	 
  \lang{de}{\[
   \left( \begin{pmatrix}1\\ 3 \end{pmatrix}+ x\begin{pmatrix}2\\ -4 \end{pmatrix}\right) \bullet \begin{pmatrix}0\\ x \end{pmatrix} 
   = \begin{pmatrix}1+2x\\ 3-4x \end{pmatrix}\bullet \begin{pmatrix}0\\ x \end{pmatrix}  =(3-4x)x.
\]}
\step
\lang{de}{
Wir lösen also
  \[
   (3-4x)x=0 \iff x=0 \quad \text{oder}\quad 3-4x=0 \iff x=0 \quad \text{oder}\quad x=\frac{3}{4}.
  \]
Da wir nach $x>0$ suchen, erhalten wir $x=\frac{3}{4}$.}
\end{incremental}

\tab{\lang{de}{Lösung e)}}
\begin{incremental}[\initialsteps{1}]
  \step

  \lang{de}{Es gilt
\[
  \left( \begin{pmatrix}2\\ 4\\ 5 \end{pmatrix}+ \begin{pmatrix}x\\ -2\\ -x \end{pmatrix}\right) \bullet 
  \begin{pmatrix}1\\ -1\\ -1 \end{pmatrix}  
  =   \begin{pmatrix}2+x\\ 2\\ 5-x \end{pmatrix}\bullet \begin{pmatrix}1\\ -1\\ -1 \end{pmatrix} =2+x-2-5+x=-5+2x.
\]}
\step
\lang{de}{
Wir lösen also
  \[
   -5+2x=5-3x\iff 5x=10\iff x=2.
  \]}
  \end{incremental}

\end{tabs*}
\end{content}