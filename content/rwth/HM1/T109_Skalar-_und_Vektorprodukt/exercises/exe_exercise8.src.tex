\documentclass{mumie.element.exercise}
%$Id$
\begin{metainfo}
  \name{
    \lang{de}{Ü08: Winkel}
    \lang{en}{}
  }
  \begin{description} 
 This work is licensed under the Creative Commons License Attribution 4.0 International (CC-BY 4.0)   
 https://creativecommons.org/licenses/by/4.0/legalcode 

    \lang{de}{Hier die Beschreibung}
    \lang{en}{}
  \end{description}
  \begin{components}
  \end{components}
  \begin{links}
  \end{links}
  \creategeneric
\end{metainfo}
\begin{content}
\usepackage{mumie.ombplus}

\title{
  \lang{de}{Ü08: Winkel}
}


\begin{block}[annotation]
  Im Ticket-System: \href{http://team.mumie.net/issues/9708}{Ticket 9708}
\end{block}
\lang{de}{ 
\begin{enumerate}
%
 \item Bestimmen Sie den Winkel $0^\circ\leq \alpha \leq 180^\circ$ zwischen den folgenden Vektoren. 
      Benutzen Sie dabei einen Taschenrechner und runden Sie auf 2 Nachkommastellen genau.  
      \begin{table}[\class{items}]
        \nowrap{a) $\; \begin{pmatrix}1\\ 1 \end{pmatrix} \quad \text{und} \quad \begin{pmatrix}1\\ -2 \end{pmatrix}$}
        & \nowrap{b) $\; \begin{pmatrix}1\\ 3 \end{pmatrix} \quad \text{und} \quad \begin{pmatrix}0\\ 1 \end{pmatrix}$}\\
         \nowrap{c) $\; \begin{pmatrix}4\\ 3\\ -1 \end{pmatrix} \quad \text{und} \quad \begin{pmatrix}0\\ 3\\ 1 \end{pmatrix}$} 
        & \\
     \end{table}
%
% Video
%
 \item Berechnen Sie (wo nötig unter Benutzung eines Taschenrechners) den Winkel, 
       den $\vec{a}$ und $\vec{b}$ einschließen, zu
      \begin{table}[\class{items}]
        \nowrap{a) $\; \vec{a}=\begin{pmatrix}1\\2\end{pmatrix},  \quad \vec{b}=\begin{pmatrix}3\\1\end{pmatrix}$}
        & \nowrap{b) $\; \vec{a}=\begin{pmatrix}3\\-4\\1\end{pmatrix}, \quad \vec{b}=\begin{pmatrix}2\\1\\-2\end{pmatrix}$}\\
         \nowrap{c) $\;  \vec{a}=\begin{pmatrix}1\\0\\2\end{pmatrix},  \quad \vec{b}=\begin{pmatrix}3\\1\\-3\end{pmatrix}$} 
        & \nowrap{d) $\; \vec{a}=\begin{pmatrix}2\\-4\\6\end{pmatrix}, \quad \vec{b}=\begin{pmatrix}-1\\2\\-3\end{pmatrix}$}\\
         \nowrap{e) $\; \vec{a}=\begin{pmatrix}1\\2\\-3\\1\end{pmatrix}, \quad \vec{b}=\begin{pmatrix}2\\0\\1\\-2\end{pmatrix}$}
        & \\        
     \end{table}
    Zeichnen Sie in a) die Situation und messen Sie den berechneten Winkel nach.\\
    Versuchen Sie, sich die Vektoren und Winkel bei b), c) und d) vorzustellen.
%    
\end{enumerate}}

\begin{tabs*}[\initialtab{0}\class{exercise}]
  \tab{
  \lang{de}{Antworten}
  \lang{en}{Answers}
  }
\begin{enumerate}
  \item
    \begin{table}[\class{items}]
      \nowrap{a) $\; \alpha=108,43^\circ $}
    & \nowrap{b) $\; \alpha=18,43^\circ $}\\
      \nowrap{c) $\; \alpha=60,26^\circ $}
    & \\
    \end{table}
%
% Video
%    
  \item
    \begin{table}[\class{items}]
      \nowrap{a) $\; \phi=45^\circ $}
    & \nowrap{b) $\; \phi=90^\circ $}\\
      \nowrap{c) $\; \phi \approx 107,9^\circ $}
    & \nowrap{d) $\; \phi=180^\circ $}\\
      \nowrap{e) $\; \phi \approx 105^\circ $}      
    & \\
    \end{table}
%    
 \end{enumerate}   
    
  \tab{
  \lang{de}{Lösung 1 a)}}
  
  
  \lang{de}{ Es gilt
  \[
   \cos(\alpha)=\frac{\begin{pmatrix}1\\ 1 \end{pmatrix}\bullet \begin{pmatrix}1\\ -2 \end{pmatrix} }{\norm{\begin{pmatrix}1\\1 \end{pmatrix}}\cdot\norm{\begin{pmatrix}1\\-2 \end{pmatrix}}}=\frac{1-2}{\sqrt{1^2+1^2}\cdot\sqrt{1^2+(-2)^2}}=\frac{-1}{\sqrt{2}\cdot\sqrt{5}}=\frac{-1}{\sqrt{10}}.
  \]
 Der Taschenrechner ergibt $\alpha=\arccos(\frac{-1}{\sqrt{10}})\approx 108,43^\circ$.}


 \tab{\lang{de}{Lösung 1 b)}}

 
  \lang{de}{Es gilt
  \[
   \cos(\alpha)=\frac{\begin{pmatrix}1\\ 3 \end{pmatrix}\bullet \begin{pmatrix}0\\ 1 \end{pmatrix}}{\norm{\begin{pmatrix}1\\ 3 \end{pmatrix}}\cdot\norm{\begin{pmatrix}0\\1 \end{pmatrix}}}=\frac{0+3}{\sqrt{1^2+3^2}\cdot\sqrt{0^2+1^2}}=\frac{3}{\sqrt{10}\cdot\sqrt{1}}=\frac{3}{\sqrt{10}}.
  \]
 Der Taschenrechner ergibt $\alpha=\arccos(\frac{3}{\sqrt{10}})\approx 18,43^\circ$.}
  
   \tab{\lang{de}{Lösung 1 c)}}


  \lang{de}{Es gilt
  \[
   \cos(\alpha)=\frac{\begin{pmatrix}4\\ 3\\ -1 \end{pmatrix}\bullet \begin{pmatrix}0\\3 \\ 1 \end{pmatrix}}{\norm{\begin{pmatrix}4\\ 3\\ -1 \end{pmatrix}}\cdot\norm{\begin{pmatrix}0\\3\\ 1 \end{pmatrix}}}=\frac{0+9-1}{\sqrt{4^2+3^2+(-1)^2}\cdot\sqrt{0^2+3^2+1^2}}=\frac{8}{\sqrt{26}\cdot\sqrt{10}}=\frac{8}{\sqrt{4 \cdot 13 \cdot 5}}=\frac{4}{\sqrt{65}}.
  \]
 Der Taschenrechner ergibt $\alpha=\arccos(\frac{4}{\sqrt{65}})\approx 60,26^\circ$.}
  
  \tab{\lang{de}{Lösungsvideo 2}}
  \youtubevideo[500][300]{2g_HZ5ryK58}\\
  
\end{tabs*}
\end{content}