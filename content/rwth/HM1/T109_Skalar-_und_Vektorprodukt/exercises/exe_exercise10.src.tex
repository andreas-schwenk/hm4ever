\documentclass{mumie.element.exercise}
%$Id$
\begin{metainfo}
  \name{
    \lang{de}{Ü10: Textaufgaben}
    \lang{en}{}
  }
  \begin{description} 
 This work is licensed under the Creative Commons License Attribution 4.0 International (CC-BY 4.0)   
 https://creativecommons.org/licenses/by/4.0/legalcode 

    \lang{de}{Hier die Beschreibung}
    \lang{en}{}
  \end{description}
  \begin{components}
  \end{components}
  \begin{links}
  \end{links}
  \creategeneric
\end{metainfo}
\begin{content}
\usepackage{mumie.ombplus}

\title{
  \lang{de}{Ü10: Textaufgaben}
}


\begin{block}[annotation]
  Im Ticket-System: \href{http://team.mumie.net/issues/9710}{Ticket 9710}
\end{block}

\begin{enumerate}[alph]
%
% Video-Aufgabe
%
    \item Berechnen Sie den Flächeninhalt des Parallelogramms, das durch $\vec{a}=\begin{pmatrix} 4 \\ 2 \end{pmatrix}$
          und $\vec{b}=\begin{pmatrix} 2 \\ 3 \end{pmatrix}$ aufgespannt wird,
        \begin{enumerate}[arabic] 
            \item durch die Formel "`Seite mal Höhe"', indem Sie mit dem Winkel zwischen 
                  $\vec{a}$ und $\vec{b}$ die Höhe berechnen,
             \item indem Sie die Situation ins Dreidimensionale übertragen und das Vektorprodukt
                   zu Hilfe nehmen. 
        \end{enumerate}
    \item Bestimmen Sie den Flächeninhalt des Dreieck mit den Eckpunkten
          \[A=\begin{pmatrix} -1 \\ 2 \end{pmatrix}, \quad B=\begin{pmatrix} 5 \\-1 \end{pmatrix}
           \quad \text{und} \quad C=\begin{pmatrix} 2 \\ 3 \end{pmatrix}\]
          (Tipp: Durch Verdopplung eines Dreiecks kann man ein Parallelogramm erhalten.)      
%
    \item Bestimmen Sie den Flächeninhalt des durch folgende Vektoren aufgespannten Parallelogramms
      \[
       \begin{pmatrix}1\\ 5\\ 3 \end{pmatrix} \quad \text{und} \quad \begin{pmatrix}0\\ 4 \\ 6 \end{pmatrix}.
      \]
\end{enumerate}

\begin{tabs*}[\initialtab{0}\class{exercise}]
  \tab{\lang{de}{Antworten }}
    \begin{enumerate}[alph] 
        \item Der Flächeninhalt des Parallelogramms ist $8$.
        \item Der Flächeninhalt des Dreiecks ist $7,5$.
        \item Der Flächeninhalt des Parallelogramms ist $\sqrt{376}$.
    \end{enumerate}
  
  \tab{\lang{de}{Lösungsvideo a), b) }}
  
  \youtubevideo[500][300]{qCu-ydpplPI}\\
 
  \tab{\lang{de}{Lösung c) }}
    
  \lang{de}{ Es gilt 
  \[
   \begin{pmatrix} 1 \\ 5 \\ 3 \end{pmatrix} \times \begin{pmatrix} 0 \\ 4 \\ 6 \end{pmatrix} = \begin{pmatrix}5\cdot 6-4\cdot 3 \\ 3\cdot 0-6\cdot 1 \\ 1\cdot 4-0\cdot 5 \end{pmatrix}=\begin{pmatrix}18\\ -6\\ 4 \end{pmatrix}.
  \]
   Der Flächeninhalt berechnet sich somit durch
  \[
   \norm{ \begin{pmatrix}18\\ -6\\ 4 \end{pmatrix} }=\sqrt{18^2+(-6)^2+4^2}=\sqrt{324+36+16}=\sqrt{376}.
  \]
}
 
\end{tabs*}
\end{content}