\documentclass{mumie.element.exercise}
%$Id$
\begin{metainfo}
  \name{
    \lang{de}{Ü06: Abstand}
    \lang{en}{}
  }
  \begin{description} 
 This work is licensed under the Creative Commons License Attribution 4.0 International (CC-BY 4.0)   
 https://creativecommons.org/licenses/by/4.0/legalcode 

    \lang{de}{Hier die Beschreibung}
    \lang{en}{}
  \end{description}
  \begin{components}
  \end{components}
  \begin{links}
  \end{links}
  \creategeneric
\end{metainfo}
\begin{content}
\usepackage{mumie.ombplus}

\title{
  \lang{de}{Ü06: Abstand}
}


\begin{block}[annotation]
  Im Ticket-System: \href{http://team.mumie.net/issues/9706}{Ticket 9706}
\end{block}



\lang{de}{ 
\begin{enumerate}[alph]
  \item Welchen Abstand haben die Punkte $P_1=(1;3)$ und $P_2=(4;-1)$ im $\R^2$?
  \item Welchen Abstand haben die Punkte $Q_1=(1;1;-1)$ und $Q_2=(0;0;1)$ im $\R^3$?
  \item Welchen Abstand haben die Punkte $R_1=(1;2;3;4)$ und $R_2=(2;1;2;1)$ im $\R^4$?  
  \item Bestimmen Sie den Abstand der Punkte $P_1=(1;7)$ und $P_2=(4;3)$.
  \item Bestimmen Sie den Abstand der Punkte $P_3=(3;-5;0)$ und $P_4=(-4;2;1)$.
\end{enumerate}}

\begin{tabs*}[\initialtab{0}\class{exercise}]
  \tab{
  \lang{de}{Antworten}
  \lang{en}{Answers}
  }
\begin{enumerate}[alph]
  \item $5$
  \item $\sqrt{6}$
  \item $\sqrt{12}$  
  \item $5$
  \item $\sqrt{99}$
\end{enumerate}

\tab{\lang{de}{Lösungsvideo a) - c)}}
  \youtubevideo[500][300]{v0aA_FrDeVg}\\

  \tab{
  \lang{de}{Lösung d)}}  
  
  \lang{de}{ Es gilt
  \[
   d(P_1,P_2)=\norm{\begin{pmatrix}4-1\\ 3-7 \end{pmatrix}}=\norm{\begin{pmatrix}3\\ -4 \end{pmatrix}}=\sqrt{3^2+(-4)^2}=\sqrt{9+16}=\sqrt{25}=5.
  \]}


 \tab{\lang{de}{Lösung e)}}

 
  \lang{de}{Es gilt
  \[
   d(P_3,P_4)=\norm{\begin{pmatrix}-4-3\\ 2-(-5) \\ 1-0 \end{pmatrix}}=\norm{\begin{pmatrix}-7\\ 7\\ 1 \end{pmatrix}}=\sqrt{(-7)^2+7^2+1^2}=\sqrt{49+49+1}=\sqrt{99}.
  \]}
	
\end{tabs*}
\end{content}