\documentclass{mumie.element.exercise}
%$Id$
\begin{metainfo}
  \name{
    \lang{de}{Ü07: Orthogonalität}
    \lang{en}{}
  }
  \begin{description} 
 This work is licensed under the Creative Commons License Attribution 4.0 International (CC-BY 4.0)   
 https://creativecommons.org/licenses/by/4.0/legalcode 

    \lang{de}{Hier die Beschreibung}
    \lang{en}{}
  \end{description}
  \begin{components}
  \end{components}
  \begin{links}
  \end{links}
  \creategeneric
\end{metainfo}
\begin{content}
\usepackage{mumie.ombplus}

\title{
  \lang{de}{Ü07: Orthogonalität}
}


\begin{block}[annotation]
  Im Ticket-System: \href{http://team.mumie.net/issues/9707}{Ticket 9707}
\end{block}



\lang{de}{ 
Prüfen Sie, ob folgende Vektoren orthogonal zueinander stehen.
\begin{enumerate}[a)]
 \item a)
  \[
   \begin{pmatrix}2\\3 \end{pmatrix} \quad \text{und} \quad \begin{pmatrix}-4\\5 \end{pmatrix}
  \]
 \item b)
  \[
   \begin{pmatrix}1\\2 \end{pmatrix} \quad \text{und} \quad \begin{pmatrix}6\\ -3 \end{pmatrix}
  \]
 \item c)
  \[
   \begin{pmatrix}1\\ 7\\ 5 \end{pmatrix} \quad \text{und} \quad \begin{pmatrix}-2\\ 6\\ 4 \end{pmatrix}
  \]
 \item d)
  \[
   \begin{pmatrix}1\\ 0\\ -2 \end{pmatrix} \quad \text{und} \quad \begin{pmatrix}0\\ 1\\ 0 \end{pmatrix}
  \]
\end{enumerate}}

\begin{tabs*}[\initialtab{0}\class{exercise}]
  \tab{
  \lang{de}{Antworten}
  \lang{en}{Answers}
  }


\begin{table}[\class{items}]
a) nein\\
b) ja\\
c) nein\\
d) ja
\end{table}
  \tab{
  \lang{de}{Lösung a)}}
  
  
  \lang{de}{ Die Vektoren stehen nicht orthogonal zueinander, da
  \[
    \begin{pmatrix}2\\ 3 \end{pmatrix}\bullet \begin{pmatrix}-4\\5 \end{pmatrix}  =-8+15=7\neq 0.
  \]}


 \tab{\lang{de}{Lösung b)}}

 
  \lang{de}{Die Vektoren stehen orthogonal zueinander, da
  \[
    \begin{pmatrix}1\\2 \end{pmatrix} \bullet \begin{pmatrix}6\\ -3 \end{pmatrix} =6-6= 0.
  \]}
  
  \tab{\lang{de}{Lösung c)}}

 
  \lang{de}{Die Vektoren stehen nicht orthogonal zueinander, da
  \[
    \begin{pmatrix}1\\ 7\\ 5 \end{pmatrix}\bullet \begin{pmatrix}-2\\ 6\\ 4 \end{pmatrix}  =-2+42+20=60\neq 0.
  \]}
  
  \tab{\lang{de}{Lösung d)}}

 
  \lang{de}{Die Vektoren stehen orthogonal zueinander, da
  \[
     \begin{pmatrix}1\\ 0\\ -2 \end{pmatrix}\bullet \begin{pmatrix}0\\ 1\\ 0 \end{pmatrix}   =0+0+0=0.
  \]}


	
\end{tabs*}
\end{content}