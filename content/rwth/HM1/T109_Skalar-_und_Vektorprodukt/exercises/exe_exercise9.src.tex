\documentclass{mumie.element.exercise}
%$Id$
\begin{metainfo}
  \name{
    \lang{de}{Ü09: Orthogonalität}
    \lang{en}{}
  }
  \begin{description} 
 This work is licensed under the Creative Commons License Attribution 4.0 International (CC-BY 4.0)   
 https://creativecommons.org/licenses/by/4.0/legalcode 

    \lang{de}{Hier die Beschreibung}
    \lang{en}{}
  \end{description}
  \begin{components}
  \end{components}
  \begin{links}
  \end{links}
  \creategeneric
\end{metainfo}
\begin{content}
\usepackage{mumie.ombplus}

\title{
  \lang{de}{Ü09: Orthogonalität}
}


\begin{block}[annotation]
  Im Ticket-System: \href{http://team.mumie.net/issues/9709}{Ticket 9709}
\end{block}

%
% Videoaufgaben
%
1. Geben Sie orthogonale Vektoren zu

    a) $\; \begin{pmatrix} 3 \\ 2  \end{pmatrix}, \quad$
    b) $\; \begin{pmatrix} 2 \\ -1 \end{pmatrix}, \quad$
    c) $\; \begin{pmatrix} 1 \\ 0 \\ 2  \end{pmatrix}, \quad$
    d) $\; \begin{pmatrix} 3 \\ 1 \\ -2 \end{pmatrix}, \quad$
    e) $\; \begin{pmatrix} 4 \\ 1 \\ 0 \\ 2 \end{pmatrix}$
    an.
    
%

2. Bestimmen Sie einen zu den gegebenen Vektoren orthogonalen Vektor der Länge 1, wobei der erste Eintrag des Vektors positiv zu wählen ist.
  \[
   \begin{pmatrix}-2\\ 6\\ 0 \end{pmatrix} \quad \text{und} \quad \begin{pmatrix}1\\ 4\\ 5 \end{pmatrix}
  \]
3. Bestimmen Sie einen zu den gegebenen Vektoren orthogonalen Vektor der Länge 4, wobei der erste Eintrag des Vektors positiv zu wählen ist.
  \[
   \begin{pmatrix}-1\\ 0\\ -2 \end{pmatrix} \quad \text{und} \quad \begin{pmatrix}0\\ 1\\ 1 \end{pmatrix}
  \]


\begin{tabs*}[\initialtab{0}\class{exercise}]
  \tab{
  \lang{de}{Antworten}
  \lang{en}{Answers}
  }


\begin{table}[\class{items}]
1.\\
\begin{enumerate}[alph]
    \item \lang{de}{Jede Lösung hat die Form 
          $\quad \lambda \cdot \begin{pmatrix} -2\\3 \end{pmatrix}, \quad \lambda \in \R$}
    \item \lang{de}{Jede Lösung hat die Form 
          $\quad \lambda \cdot \begin{pmatrix} 1\\2 \end{pmatrix}, \quad \lambda \in \R$}
    \item \lang{de}{Jede Lösung hat die Form 
          $\quad \lambda_1 \cdot \begin{pmatrix} 0\\1\\0 \end{pmatrix}+\lambda_2 \cdot \begin{pmatrix} -2\\0\\1 \end{pmatrix}, \quad \lambda_1, \lambda_2 \in \R$}
    \item \lang{de}{Jede Lösung hat die Form 
          $\quad \lambda_1 \cdot \begin{pmatrix} -1\\1\\-1 \end{pmatrix}+\lambda_2 \cdot \begin{pmatrix} 0\\2\\1 \end{pmatrix}, \quad \lambda_1, \lambda_2 \in \R$}
    \item \lang{de}{Jede Lösung hat die Form 
          $\quad \lambda_1 \cdot \begin{pmatrix} 0\\0\\1\\0 \end{pmatrix}+\lambda_2 \cdot \begin{pmatrix} 1\\-4\\0\\0 \end{pmatrix}+\lambda_3 \cdot \begin{pmatrix} 1\\0\\0\\-2 \end{pmatrix}, \quad \lambda_1, \lambda_2, \lambda_3 \in \R$}

\end{enumerate}
~\\
2.\\ $\frac{1}{\sqrt{1196}}\begin{pmatrix}30\\ 10\\ -14 \end{pmatrix}$\\\\
3.\\ $\frac{4}{\sqrt{6}}\begin{pmatrix}2\\ 1\\ -1 \end{pmatrix}$
\end{table}
	
  \tab{\lang{de}{Lösungsvideo 1 a) - e) }}
  \youtubevideo[500][300]{1S1GYyG8EVA}\\


  \tab{
  \lang{de}{Lösung 2}}
  \begin{incremental}[\initialsteps{1}]
  \step
  
  \lang{de}{Wir bestimmen zunächst mithilfe des Vektorprodukts einen zu den beiden Vektoren orthogonal stehenden Vektor:
  \[
   \begin{pmatrix}-2\\ 6\\ 0 \end{pmatrix} \times \begin{pmatrix}1\\ 4\\ 5 \end{pmatrix}=\begin{pmatrix}6\cdot 5-4\cdot 0\\ 0\cdot 1-5\cdot (-2)\\ (-2)\cdot 4-1\cdot 6 \end{pmatrix}=\begin{pmatrix}30\\10\\-14 \end{pmatrix}.
  \]
 }
\step

\lang{de}{Die Länge des Vektors beträgt
  \[
   \sqrt{30^2+10^2+(-14)^2}=\sqrt{900+100+196}=\sqrt{1196}.
  \]
 Also ist der gesuchte Vektor
  \[
   \frac{1}{\sqrt{1196}}\begin{pmatrix}30\\ 10\\ -14 \end{pmatrix}.
  \]
  }
  \end{incremental}

 \tab{\lang{de}{Lösung 3}}
 \begin{incremental}[\initialsteps{1}]
  \step
 
  \lang{de}{Wir bestimmen zunächst mithilfe des Vektorprodukts einen zu den beiden Vektoren orthogonal stehenden Vektor:
  \[
   \begin{pmatrix}-1\\ 0\\ -2 \end{pmatrix} \times \begin{pmatrix}0\\ 1\\ 1 \end{pmatrix}=\begin{pmatrix}0\cdot 1-1\cdot (-2)\\ -2\cdot 0-1\cdot (-1)\\ -1\cdot 1-0\cdot 0 \end{pmatrix}=\begin{pmatrix}2\\1\\-1 \end{pmatrix}.
  \]}
\step
\lang{de}{
Die Länge des Vektors beträgt
  \[
   \sqrt{2^2+1^2+(-1)^2}=\sqrt{6}.
  \]
  Der Vektor mit Länge $1$ ist demnach
   \[
   \frac{1}{\sqrt{6}}\begin{pmatrix}2\\ 1\\ -1 \end{pmatrix}.
  \]
 Also ist der gesuchte Vektor mit Länge $4$
  \[
   \frac{4}{\sqrt{6}}\begin{pmatrix}2\\ 1\\ -1 \end{pmatrix}.
  \]}
\end{incremental}
    
	
\end{tabs*}
\end{content}