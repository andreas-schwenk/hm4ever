\documentclass{mumie.element.exercise}
%$Id$
\begin{metainfo}
  \name{
    \lang{de}{Ü05: Länge}
    \lang{en}{}
  }
  \begin{description} 
 This work is licensed under the Creative Commons License Attribution 4.0 International (CC-BY 4.0)   
 https://creativecommons.org/licenses/by/4.0/legalcode 

    \lang{de}{Hier die Beschreibung}
    \lang{en}{}
  \end{description}
  \begin{components}
  \end{components}
  \begin{links}
  \end{links}
  \creategeneric
\end{metainfo}
\begin{content}
\usepackage{mumie.ombplus}

\title{
  \lang{de}{Ü05: Länge}
}


\begin{block}[annotation]
  Im Ticket-System: \href{http://team.mumie.net/issues/9705}{Ticket 9705}
\end{block}



\lang{de}{ 
\begin{enumerate}[a)]
 \item a) Bestimmen Sie $x>0$ so, dass der Vektor 
  \[
   \begin{pmatrix}\frac{1}{\sqrt{2}}\\ x \end{pmatrix}
  \]
  Länge 1 besitzt.
 \item b) Bestimmen Sie $x>0$ so, dass der Vektor 
  \[
   \begin{pmatrix}\frac{1}{x}\\ \frac{-3}{x}\\ \frac{7}{x} \end{pmatrix}
  \]
  Länge 1 besitzt.
 \item c) Bestimmen Sie $x>0$ so, dass der Vektor 
  \[
   \begin{pmatrix}1\\ 5\\ \sqrt{x} \end{pmatrix}
  \]
  Länge 6 besitzt.
\end{enumerate}}

\begin{tabs*}[\initialtab{0}\class{exercise}]
  \tab{
  \lang{de}{Antworten}
  \lang{en}{Answers}
  }


\begin{table}[\class{items}]
a) $x=\frac{1}{\sqrt{2}}$\\
b) $x=\sqrt{59}$\\
c) $x=10$
\end{table}
  \tab{
  \lang{de}{Lösung a)}}
  \begin{incremental}[\initialsteps{1}]
  \step
  
  \lang{de}{Es gilt
  \[
   \norm{\begin{pmatrix}\frac{1}{\sqrt{2}}\\ x \end{pmatrix}}=\sqrt{\left(\frac{1}{\sqrt{2}}\right)^2+x^2}=\sqrt{\frac{1}{2}+x^2}.
  \]}
\step

\lang{de}{Da wir nach $x>0$ suchen, ist die Lösung also
  \[
   \frac{1}{2}+x^2=1 \iff x^2=\frac{1}{2} \iff x=\frac{1}{\sqrt{2}}
  \]
  }
  \end{incremental}

 \tab{\lang{de}{Lösung b)}}
 \begin{incremental}[\initialsteps{1}]
  \step
 
  \lang{de}{Es gilt mit $x>0$
  \[
   \norm{\begin{pmatrix}\frac{1}{x}\\ \frac{-3}{x}\\ \frac{7}{x} \end{pmatrix}}=\sqrt{\left(\frac{1}{x}\right)^2+\left(\frac{-3}{x}\right)^2+\left(\frac{7}{x}\right)^2}=\sqrt{\frac{1}{x^2}+\frac{9}{x^2}+\frac{49}{x^2}}=\sqrt{\frac{59}{x^2}}=\frac{\sqrt{59}}{|x|}=\frac{\sqrt{59}}{x}.
  \]}
\step
\lang{de}{
Also
  \[
   \frac{\sqrt{59}}{x}=1 \iff x=\sqrt{59}.
  \]}
\end{incremental}
	
	 \tab{\lang{de}{Lösung c)}}
	 \begin{incremental}[\initialsteps{1}]
  \step
	 
  \lang{de}{Es gilt mit $x>0$
  \[
   \norm{\begin{pmatrix}1\\ 5\\ \sqrt{x} \end{pmatrix}}=\sqrt{1^2+5^2+\left(\sqrt {x}\right)^2}=\sqrt{1+25+x}=\sqrt{26+x}.
  \]}
\step
\lang{de}{
 Also
  \[
  \sqrt{26+x}=6 \Longrightarrow 26+x=36 \iff x=10.
  \]
  Einsetzen zeigt tatsächlich, dass $x=10$ eine Lösung ist.}
	\end{incremental}
	
	 
\end{tabs*}
\end{content}