%$Id:  $
\documentclass{mumie.article}
%$Id$
\begin{metainfo}
  \name{
    \lang{de}{Länge von Vektoren}
    \lang{en}{Length of a vector}
  }
  \begin{description} 
 This work is licensed under the Creative Commons License Attribution 4.0 International (CC-BY 4.0)   
 https://creativecommons.org/licenses/by/4.0/legalcode 

    \lang{de}{Beschreibung}
    \lang{en}{Description}
  \end{description}
  \begin{components}
    \component{generic_image}{content/rwth/HM1/images/g_tkz_T109_Distance3D.meta.xml}{T109_Distance3D}
    \component{generic_image}{content/rwth/HM1/images/g_tkz_T109_Parallelogram.meta.xml}{T109_Parallelogram}
    \component{generic_image}{content/rwth/HM1/images/g_tkz_T109_VectorSum.meta.xml}{T109_VectorSum}
    \component{generic_image}{content/rwth/HM1/images/g_tkz_T109_Length3D.meta.xml}{T109_Length3D}
    \component{generic_image}{content/rwth/HM1/images/g_img_00_Videobutton_schwarz.meta.xml}{00_Videobutton_schwarz}
    \component{js_lib}{system/media/mathlets/GWTGenericVisualization.meta.xml}{mathlet1}
  \end{components}
  \begin{links}
    \link{generic_article}{content/rwth/HM1/T108_Vektorrechnung/g_art_content_27_vektoren.meta.xml}{vektor-ansch}
    \link{generic_article}{content/rwth/HM1/T105_Trigonometrische_Funktionen/g_art_content_17_trigonometrie_im_dreieck.meta.xml}{dreiecks-trig}
    \link{generic_article}{content/rwth/HM1/T109_Skalar-_und_Vektorprodukt/g_art_content_33_winkel.meta.xml}{winkel}
    \link{generic_article}{content/rwth/HM1/T109_Skalar-_und_Vektorprodukt/g_art_content_31_skalarprodukt.meta.xml}{skalarprodukt}
    \link{generic_article}{content/rwth/HM1/T110_Geraden,_Ebenen/g_art_content_36_normalenformen.meta.xml}{normalenformen}
  \end{links}
  \creategeneric
\end{metainfo}
\begin{content}
\usepackage{mumie.ombplus}
\ombchapter{9}
\ombarticle{2}
\usepackage{mumie.genericvisualization}

\begin{visualizationwrapper}

\title{\lang{de}{Länge von Vektoren, Abstände von Punkten}\lang{en}{Length of a vector}}
 
\begin{block}[annotation]
  übungsinhalt: Norm/Länge eines Vektors; Norm bei Skalierung, Dreiecksungleichung und Cauchy-Schwarz;
  Abstände von Punkten\\
  TODO: Beispiel zu Eigenschaften der Norm; quickchecks
  
\end{block}
\begin{block}[annotation]
  Im Ticket-System: \href{http://team.mumie.net/issues/9049}{Ticket 9049}\\
\end{block}

\begin{block}[info-box]
\tableofcontents
\end{block}


\section{\lang{de}{Länge eines Vektors}\lang{en}{Length of a vector}}\label{sec:euklidische_norm}

\lang{de}{
Im Abschnitt \link{vektor-ansch}{Vektoren im Anschauungsraum} wurden anschaulich Vektoren als Objekte 
mit Länge und Richtung im $\R^n$ definiert und erst später mit Spaltenvektoren identifiziert.
\\\\
Für mathematische Zwecke ist es praktischer mit Spaltenvektoren zu starten (bzw. allgemein mit 
Elementen in einem Vektorraum). Mit Hilfe des Skalarprodukts kann man dann die Länge eines Vektors 
definieren.
}
\lang{en}{
\link{vektor-ansch}{In the previous section} we graphically defined vectors as objects with length 
(magnitude) and direction in $\R^n$, and then identified these with their representation as column 
vectors.
\\\\
It is more practical to consider column vectors for mathematical purposes (or more generally, 
elements of a vector space). Using the scalar product we can define the length of a vector.
}

\begin{definition}\label{def:euklidische_norm}
\lang{de}{
Es sei $\vec{v}=\begin{pmatrix}v_1\\ v_2\\ \vdots\\ v_n \end{pmatrix} \in \R^n$. Die \emph{L{\"a}nge}
von $\vec{v}$ (auch \emph{Norm} oder \emph{Betrag} genannt) ist die nicht-negative reelle Zahl 
}
\lang{en}{
Let $\vec{v}=\begin{pmatrix}v_1\\ v_2\\ \vdots\\ v_n \end{pmatrix} \in \R^n$. The \emph{length} of 
$\vec{v}$ (also called its \emph{magnitude} or its \emph{norm}) is the non-negative real number
}
\[
 \left\| \vec{v}\right\| \, := \, \sqrt{v \bullet v}\, = \, \sqrt{\sum_{i=1}^{n} v_i^2} \, = \, \sqrt{v_1^2 + v_2^2 + \cdots + v_n^2} 
. 
\]
\lang{de}{
\floatright{\href{https://www.hm-kompakt.de/video?watch=714}{\image[75]{00_Videobutton_schwarz}}}\\\\
}
\lang{en}{}
\end{definition}

\begin{example}
\lang{de}{
F"ur $\vec{v}=\left( \begin{smallmatrix} 1 \\ 3\\ 2 \end{smallmatrix}\right)$ ist zum Beispiel
}
\lang{en}{
For example, for $\vec{v}=\left( \begin{smallmatrix} 1 \\ 3\\ 2 \end{smallmatrix}\right)$ we have
}
\[ \left\| \vec{v}\right\|=
\left\| \left( \begin{smallmatrix} 1 \\ 3\\2 \end{smallmatrix}\right) \right\| 
= \sqrt{1^2 + 3^2+2^2} = \sqrt{1 + 9 + 4} = \sqrt{14}. \] 
\end{example}

\begin{remark}
\lang{de}{
Mit Hilfe des \ref[dreiecks-trig][Satzes des Pythagoras]{satz-des-pythagoras} sieht man ein, dass 
diese Definition der Länge im $\R^2$ und $\R^3$ mit der anschaulichen Länge übereinstimmt. 
Für den Ortsvektor $\vec{p}=\begin{pmatrix}p_1\\ p_2\\ p_3 \end{pmatrix}$ des Punktes 
$P=(p_1;p_2;p_3)$ ist nämlich die anschauliche Länge genau der Abstand von $P$ zum Ursprung, welcher 
sich berechnet durch
}
\lang{en}{
\ref[dreiecks-trig][The Pythagorean theorem]{satz-des-pythagoras} makes it clear that the above 
definition of the length of a vector in $\R^2$ and $\R^3$ corresponds to the length of the 
arrow representing the vector in space. 
The length of the position vector $\vec{p}=\begin{pmatrix}p_1\\ p_2\\ p_3 \end{pmatrix}$ of the point 
$P=(p_1;p_2;p_3)$ is precisely the distance of $P$ from the origin, which is calculated by
}
\[ d(P,O)=\sqrt{\left(\sqrt{p_1^2+p_2^2}\right)^2+p_3^2 }=\sqrt{p_1^2+p_2^2+p_3^2 }.\]

\begin{center}
\image{T109_Length3D}
\end{center}

\lang{de}{Also ist auch anschaulich}
\lang{en}{Correspondingly,}
\[ \left\| \vec{p}\right\|= \sqrt{p_1^2+p_2^2+p_3^2 }=\sqrt{\vec{p}\bullet \vec{p}}.\]
\end{remark}

\begin{remark}
\lang{de}{
Oft werden für die Länge eines Vektors statt $\|\vec{v}\|$ auch einfache Betragsstriche $|\vec{v}|$ 
verwendet.
}
\lang{en}{
Often the length of a vector $\|\vec{v}\|$ is also written using the absolute value notation 
$|\vec{v}|$.
}
\end{remark}

\begin{quickcheck}
		\field{real}
		\type{input.function}
		\begin{variables}
			\randint{v1}{-5}{5}
			\randint{v2}{-5}{5}
			\randint{v3}{-5}{5}
% 			\randint{w1}{-5}{5}
% 			\randint{w2}{-5}{5}
% 			\randint{w3}{-5}{5}
			\function[calculate]{nv2}{(v1)^2+(v2)^2+(v3)^2}
			\function{s}{sqrt(nv2)}
			\function{slong}{(v1)^2+(v2)^2+(v3)^2}
		\end{variables}
		
			\text{\lang{de}{
      Bestimmen Sie die Länge des Vektors 
      $\vec{v}=\begin{pmatrix} \var{v1}\\ \var{v2} \\ \var{v3} \end{pmatrix}$.\\
			Die Länge ist $\left\|\vec{v}\right\|=$\ansref.\\
			(Verwenden Sie \emph{sqrt(..)} für die Wurzel einer Zahl.)
      }
      \lang{en}{
      Determine the length of the vectors 
      $\vec{v}=\begin{pmatrix} \var{v1}\\ \var{v2} \\ \var{v3} \end{pmatrix}$.\\
      The length is $\left\|\vec{v}\right\|=$\ansref.\\
      (Use \emph{sqrt(..)} for the root of a number.)
      }}
			
		
		\begin{answer}
			\solution{s}
			\checkAsFunction{x}{-2}{2}{20}
		\end{answer}
		\explanation{\lang{de}{Die Länge berechnet sich durch }
                 \lang{en}{The length is calculated by }
		$\left\| \begin{pmatrix} \var{v1}\\ \var{v2} \\ \var{v3}
			\end{pmatrix} \right\| = \sqrt{\var{slong}} =\var{s}$.
		}
	\end{quickcheck}



\section{\lang{de}{Eigenschaften der Norm}\lang{en}{Properties of the norm}}

\begin{theorem}[\lang{de}{Eigenschaften der Norm}
                \lang{en}{Properties of the norm}]\label{thm:eigenschaften-norm}
\lang{de}{Für Vektoren $\vec{v}$ und $\vec{w}$ im $\R^n$ und eine reelle Zahl $r$ gelten:}
\lang{en}{For vectors $\vec{v}$ and $\vec{w}$ in $\R^n$ and a real number $r$ we have:}
\begin{enumerate}
\item $\left\| r\cdot \vec{v}\right\| = \left|r\right| \cdot \left\|\vec{v}\right\|$,
\item $\left\| \vec{v}+\vec{w}\right\| \leq \left\| \vec{v}\right\| + \left\| \vec{w}\right\|$ (\emph{\lang{de}{Dreiecksungleichung}\lang{en}{Triangle inequality}}),
\item $\left| \vec{v}\bullet \vec{w}\right| \leq \left\| \vec{v}\right\| \cdot \left\| \vec{w}\right\|$ 
(\emph{\lang{de}{Cauchy-Schwarz-Ungleichung}\lang{en}{Cauchy-Schwartz inequality}}).
\end{enumerate}
\lang{de}{
\floatright{\href{https://www.hm-kompakt.de/video?watch=715}{\image[75]{00_Videobutton_schwarz}}}\\\\
}
\lang{en}{}
\end{theorem}

\begin{remark}
\begin{enumerate}
\item \lang{de}{
Die erste Eigenschaft, entspricht genau der Anschauung, dass die Multiplikation eines Vektors mit 
einer reellen Zahl $r$ die Länge um $|r|$ skaliert.
}
\lang{en}{
The first property states that multiplication of a vector by a real number $r$ corresponds with 
the scaling of the length of a vector's arrow by $|r|$
}
\item \lang{de}{
Die Bedeutung der Dreiecksungleichung wird klar, wenn man sich die Definition der Summe 
$\vec{v}+\vec{w}$ vor Augen führt:
}
\lang{en}{
The meaning of the triangle inequality is clear upon seeing the sum $\vec{v}+\vec{w}$ represented 
graphically:
}

\begin{center}
\image{T109_VectorSum}
\end{center}

\lang{de}{\textit{Merkhilfe:} Der direkte Weg ist immer der Kürzeste.}
\lang{en}{\textit{Just remember: the direct path is always the shortest.}}
\item \lang{de}{
Die Cauchy-Schwarz-Ungleichung wird im Abschnitt \link{winkel}{Winkel} wichtig werden.
}
\lang{en}{
The Cauchy-Schwartz inequality becomes relevant in the \link{winkel}{next section}, which is about 
angles.
}
\end{enumerate}
\end{remark}

\begin{example}
\lang{de}{
Zu dem Vektor $\vec{v}=\left( \begin{smallmatrix} 6 \\ 3\\ 2 \end{smallmatrix}\right)$ ist ein Vektor 
gesucht, der in die gleiche Richtung zeigt, aber die Länge $1$ hat.
\\\\
Man sucht also einen Vektor der Form $r\cdot \vec{v}$ mit $r>0$ der Länge $1$. Mit obiger Regel 
bedeutet das, dass $\left\| r\cdot \vec{v}\right\| = \left|r\right| \cdot \left\|\vec{v}\right\|=1$ 
sein soll, d.h. (wegen $r>0$) dass $r=\frac{1}{\left\|\vec{v}\right\|}$.\\
Wegen
}
\lang{en}{
Consider the vector $\vec{v}=\left( \begin{smallmatrix} 6 \\ 3\\ 2 \end{smallmatrix}\right)$ and 
suppose we want to find a vector that points in the same direction, but has length $1$.
\\\\
Hence we want a vector of the form $r\cdot \vec{v}$ with $r>0$ that has length $1$. Using the first 
property from above, 
$\left\| r\cdot \vec{v}\right\| = \left|r\right| \cdot \left\|\vec{v}\right\|=1$. 
As $r>0$, we can rearrange, $r=\frac{1}{\left\|\vec{v}\right\|}$.\\
Hence, since
}
\[ \left\|\vec{v}\right\|=\sqrt{6^2+3^2+2^2}=\sqrt{49}=7, \]
\lang{de}{ist also $r=\frac{1}{7}$ und der gesuchte Vektor ist}
\lang{en}{we have $r=\frac{1}{7}$ and the desired vector is}
\[ \frac{1}{7}\cdot \begin{pmatrix} 6 \\ 3\\ 2 \end{pmatrix}=\begin{pmatrix} 6/7 \\ 3/7 \\ 2/7 \end{pmatrix}. \]
\end{example}

\begin{remark}
\lang{de}{
Das Bestimmen eines Vektors der Länge $1$ wie im vorherigen Beispiel wird bei den 
\link{normalenformen}{Normalenformen} von Geraden im $\R^2$ oder von Ebenen im $\R^3$ benötigt, wo man
einen sogenannten \emph{Einheitsnormalenvektor} bestimmen muss.
}
\lang{en}{
The above strategy for finding a vector of length $1$ is used, for example, for finding the 
\emph{unit normal vector} needed to find the \link{normalenformen}{normal form of a line or a plane}.
}
\end{remark}

\begin{quickcheck}
		\field{rational}
		\type{input.function}
		\begin{variables}
			\randint{v1}{-6}{6}
			\randint{v2}{-6}{6}
			\randint[Z]{v3}{-2}{2}
			\randint{l}{2}{5}
			\function[calculate]{nv2}{(v1)^2+(v2)^2+(v3)^2}
			\function{s}{sqrt(nv2)}
			\function[calculate]{q1}{(v1*l)}
			\function[calculate]{q2}{(v2*l)}
			\function[calculate]{q3}{(v3*l)}
			\function{s1}{q1/s}
			\function{s2}{q2/s}
			\function{s3}{q3/s}
		\end{variables}
		
			\text{\lang{de}{
      Bestimmen Sie den Vektor $\vec{w}$, der die gleiche Richtung
			wie der Vektor $\vec{v}=\begin{pmatrix} \var{v1}\\ \var{v2} \\ \var{v3}
			\end{pmatrix}$ hat und $\left\| \vec{w} \right\|=\var{l}$ erfüllt.
			\begin{table}[\class{no-padding}]
			\rowspan[l][m]{3} Der Vektor ist
      }
      \lang{en}{
      Determine the vector $\vec{w}$ that points in the same direction as the vector 
      $\vec{v}=\begin{pmatrix} \var{v1}\\ \var{v2} \\ \var{v3} \end{pmatrix}$ and satisfies 
      $\left\| \vec{w} \right\|=\var{l}$. \begin{table}[\class{no-padding}]\rowspan[l][m]{3} 
      The vector is
      }
      $\vec{w}=\left(\begin{matrix} \\ \\ \\ \\ \end{matrix}\right.$ &  
			\ansref & \rowspan[l][m]{3} $\left.\begin{matrix} \\ \\ \\ \\ \end{matrix}\right)$. & \\ 
			\ansref & \\ 
			\ansref & 
			\end{table}\\
			\lang{de}{(Verwenden Sie \emph{sqrt(..)} für die Wurzel einer Zahl.)}
      \lang{en}{(Use \emph{sqrt(..)} for the root of a number.)}
			}
		
		\begin{answer}
			\solution{s1}
			\checkAsFunction{x}{-2}{2}{20}
		\end{answer}
		\begin{answer}
			\solution{s2}
			\checkAsFunction{x}{-2}{2}{20}
		\end{answer}
		\begin{answer}
			\solution{s3}
			\checkAsFunction{x}{-2}{2}{20}
		\end{answer}
		\explanation{\lang{de}{
    Um einen Vektor mit der gleichen Richtung und Länge $\var{l}$ zu bekommen, muss man	den 
    ursprünglichen Vektor $\vec{v}$ mit 
    $\frac{\var{l}}{\left\|\vec{v}\right\|}=\frac{\var{l}}{\var{s}}$ skalieren.
    }
    \lang{en}{
    To get a vector pointing in the same direction, with length $\var{l}$, we scale the original 
    vector $\vec{v}$ by $\frac{\var{l}}{\left\|\vec{v}\right\|}=\frac{\var{l}}{\var{s}}$.
    }}
	\end{quickcheck}

\section{\lang{de}{Abstände von Punkten}\lang{en}{Distances between points}}

\lang{de}{Der Abstand zwischen Punkten lässt sich bequem auf die Länge von Vektoren zurückführen.}
\lang{en}{The distance between two points can be easily determined by finding the length of a vector.}

\begin{theorem}\label{thm:abst_pkte}
\lang{de}{
Der \emph{Abstand} $d(Q,R)$ von zwei Punkten $Q$ und $R$ im $\R^n$ ist gleich der 
L"ange des Verbindungsvektors $\overrightarrow{QR}$.\\
\floatright{\href{https://www.hm-kompakt.de/video?watch=716}{\image[75]{00_Videobutton_schwarz}}}\\\\
}
\lang{en}{
The \emph{distance} $d(Q,R)$ between two points $Q$ and $R$ in $\R^n$ is equal to the length of the 
vector between them, $\overrightarrow{QR}$.\\
}
\end{theorem}

%\begin{tabs*}[\initialtab{0}]
%\tab{Erläuterung für den Fall $n=3$}
%\begin{proof*}[Erklärung]
\begin{proof*}[\lang{de}{Erläuterung für den Fall $n=3$}
               \lang{en}{Explanation for the case $n=3$}]
\begin{showhide}
\lang{de}{
Zur Erläuterung betrachten wir den Abstand $d(Q,R)$ von zwei Punkten $Q=(q_1;q_2;q_3)$ und 
$R=(r_1;r_2;r_3)$ in $\mathbb{R}^3$.\\
Dazu denken wir uns die Punkte $Q=(q_1;q_2;q_3)$ und $R=(r_1;r_2;r_3)$ als Eckpunkte eines 
achsenparallelen Quaders im kartesischen Koordinatensystem. Der Abstand der beiden Punkte entspricht 
dann der Raumdiagonale.
}
\lang{en}{
Suppose we want to find the distance $d(Q,R)$ between two points $Q=(q_1;q_2;q_3)$ and 
$R=(r_1;r_2;r_3)$ in $\mathbb{R}^3$.\\
Consider the points $Q=(q_1;q_2;q_3)$ and $R=(r_1;r_2;r_3)$ as two corner points of a cuboid parallel 
to the axes of the $3$-dimensional Cartesian coordinates, so that the distance between them is the 
length of the diagonal of the cuboid.
}\\\\
\begin{center}
\image{T109_Distance3D}
\end{center}
\lang{de}{
Die Kantenlängen $a_1,a_2,a_3$ des Quaders entsprechen dem Betrag der Koordinatendifferenzen:
}
\lang{en}{
The edge lengths $a_1,a_2,a_3$ of the cuboid correspond to the differences of each coordinate of the 
two points:
}
\begin{align*}
a_1&=q_1-p_1\\
a_2&=q_2-p_2\\
a_3&=q_3-p_3
\end{align*}
\lang{de}{
Innerhalb des Quaders sind die Dreiecke $QAB$ sowie $QBR$ rechtwinklig, so dass man zur Berechnung 
der Länge der Diagonlen $d$ sowie der Raumdiagonalen $\vec{QR}$ (eben des gesuchten Abstand von $Q$ 
und $R$) den \ref[dreiecks-trig][Satz des Pythagoras]{satz-des-pythagoras} anwenden kann:
}
\lang{en}{The triangles $QAB$ and $QBR$ within the cuboid are right-angled, so to calculate the 
length of the diagonal $d$ and the length of the cuboid's diagonal $\vec{QR}$ (which is the distance 
between $Q$ and $R$ that we want), we simply apply the 
\ref[dreiecks-trig][Pythagorean theorem]{satz-des-pythagoras}:
}
\begin{align*}
\left\|\overrightarrow{QR}\right\|^2&=d^2+a_3^2\\
d^2&=a_1^2+a_2^2
\end{align*}
\lang{de}{
Nun setzt man die letzte Gleichung in die davor ein und ersetzt die Koordinatendifferenzen $a_i$:
}
\lang{en}{
Now we substitute the second equation into the first and substitute in the differences in coordinates 
for $a_i$:
}
\begin{align*}
\left\|\overrightarrow{QR}\right\|^2&=d^2+a_3^2\\
&=a_1^2+a_2^2+a_3^2\\
&=(q_1-p_1)^2+(q_2-p_2)^2+(q_3-p_3)^2
\end{align*}
\lang{de}{
Durch Wurzelziehen ergibt sich für den Abstand der beiden Punkte $Q$ und $R$ in $\mathbb{R}^3$:
}
\lang{en}{
Taking the square root yields the distance between the two points $Q$ and $R$ in $\mathbb{R}^3$:
}
\[d(Q,R)=\left\|\overrightarrow{QR}\right\|=\sqrt{(q_1-p_1)^2+(q_2-p_2)^2+(q_3-p_3)^2}\]

\lang{de}{
Analog kann man sich den Sachverhalt für den Abstand zweier Punkte in $\mathbb{R}^n$ für $n\neq 3$ 
herleiten.
}
\lang{en}{
An analogous method may be used to find the distance between two points in $\mathbb{R}^n$ for any 
$n\neq 3$.
}
\end{showhide}
\end{proof*}
%\end{tabs*}

\begin{example}
\lang{de}{Der Abstand der Punkte $Q=(\frac{1}{2};1)$ und $R=(\frac{3}{2};3)$ ist}
\lang{en}{The distance between the points $Q=(\frac{1}{2};1)$ and $R=(\frac{3}{2};3)$ is}
\[ d(Q,R)=\left\|\overrightarrow{QR}\right\|=\left\| \begin{pmatrix}\frac{3}{2}- \frac{1}{2}\\ 3-1 \end{pmatrix} \right\|
= \sqrt{1^2+2^2}=\sqrt{5}. \]
\end{example}


\begin{example}
\lang{de}{
Im $\R^2$ ist das Viereck $ABCD$ mit den Eckpunkten $A=(0; 0)$, $B=(6; 1)$,  $C=(10; 6)$ und 
$D=(4; 5)$ gegeben. Ist dieses Viereck ein Parallelogramm oder sogar eine Raute?
}
\lang{en}{
Consider the quadrilateral $ABCD$ in $\R^2$ with corners at $A=(0; 0)$, $B=(6; 1)$,  $C=(10; 6)$ and 
$D=(4; 5)$. Is this quadrilateral a parallelogram or a diamond?
}

\begin{center}
\image{T109_Parallelogram}
\end{center}

\begin{tabs*}[\initialtab{0}]
\tab{\lang{de}{Definition Parallelogramm}\lang{en}{Definition of a parallelogram}}
\lang{de}{
Ein ebenes Viereck ist ein \emph{Parallelogramm}, wenn eine der folgenden "aquivalenten Bedingungen erf"ullt ist:
}
\lang{en}{
A quadrilateral (a shape on a plane with four corners/vertices and four straight sides/edges) is a 
\emph{parallelogram} if one of the following equivalent conditions is met:
}

\begin{enumerate}
\item \lang{de}{Gegen"uberliegende Seiten sind parallel.}
      \lang{en}{Opposite edges are parallel to each other.}
\item \lang{de}{
      Die Diagonalen des Vierecks halbieren sich gegenseitig (d.h. ihr Schnittpunkt ist Mittelpunkt 
      beider Diagonalen).
      }
      \lang{en}{
      The diagonals of the quadrilateral intersect each other at their centres.
      }
\item \lang{de}{Zwei Seiten sind parallel und gleich lang.}
      \lang{en}{Two of the quadrilateral's edges are parallel to each other and the same length.}
\item \lang{de}{Gegen"uberliegende Seiten sind gleich lang.}
      \lang{en}{Opposite edges are the same length as each other.}
\item \lang{de}{Die Winkel an gegen"uberliegenden Ecken sind gleich gro"s.}
      \lang{en}{The interior angles at opposite vertices are the same.}
\item \lang{de}{Die Winkel an benachbarten Ecken erg"anzen sich zu $180^\circ$.}
      \lang{en}{The angles at any two vertices that share an edge add up to $180^\circ$.}
\end{enumerate}

\lang{de}{
\textbf{Bemerkung}: Abgesehen von 4. und 5. implizieren die Bedingungen sogar, dass das Viereck eben 
ist.
}
\lang{en}{} %A quadrilateral is always assumed to be on a plane in English

\tab{\lang{de}{Definition Raute}\lang{en}{Definition of a diamond}}
\lang{de}{
Eine \emph{Raute} (oder \emph{Rhombus}) ist ein Parallelogramm, dessen Seiten alle gleich lang sind.
}
\lang{en}{
A \emph{diamond} (or \emph{rhombus}) is a parallelogram whose edges are all the same length.
}
\end{tabs*}

\lang{de}{
Um zu untersuchen, ob das Viereck ein Parallelogramm ist, ist am einfachsten zu untersuchen, 
ob zwei Seiten parallel und gleich lang sind. Dies bedeutet n"amlich, dass die Vektoren 
$\overrightarrow{AB}$ und $\overrightarrow{DC}$ gleich sein m"ussten (bzw. die  Vektoren 
$\overrightarrow{AD}$ und $\overrightarrow{BC}$). 
Wir haben
}
\lang{en}{
The simplest method to determine whether $ABCD$ is a parallelogram is to see if two opposite 
edges are parallel and have the same length. That is, if the vectors $\overrightarrow{AB}$ and 
$\overrightarrow{DC}$ are the same (or the vectors $\overrightarrow{AD}$ and $\overrightarrow{BC}$). 
We have
}
\[ \overrightarrow{AB}=\left( \begin{smallmatrix} 6-0 \\ 1-0 \end{smallmatrix} \right)
=\left( \begin{smallmatrix} 6 \\ 1 \end{smallmatrix} \right) \quad
\text{und}\quad  \overrightarrow{DC}=\left( \begin{smallmatrix} 10-4 \\ 6-5 \end{smallmatrix} \right)
=\left( \begin{smallmatrix} 6 \\ 1 \end{smallmatrix} \right). \]
\lang{de}{
Also ist das Viereck $ABCD$ ein Parallelogramm.
\\\\
Um zu sehen, ob es eine Raute ist, muss man noch testen, ob $d(A,B)=d(A,D)$ ist, d.h. ob 
$\left\|\overrightarrow{AB}\right\|=\left\|\overrightarrow{AD}\right\|$ gilt.
}
\lang{en}{
Hence the quadrilateral $ABCD$ is a parallelogram.
\\\\
To determine if it is a diamond, we also check if $d(A,B)=d(A,D)$, that is, if 
$\left\|\overrightarrow{AB}\right\|=\left\|\overrightarrow{AD}\right\|$:
}
\[ \left\|\overrightarrow{AB}\right\|= 
     \sqrt{6^2+1^2}=\sqrt{37} 
       \quad \text{\lang{de}{und}\lang{en}{and}} \quad 
         \left\|\overrightarrow{AD}\right\| = 
           \left\|\left( \begin{smallmatrix} 4 \\ 5 \end{smallmatrix} \right) \right\| = 
             \sqrt{4^2+5^2}=\sqrt{41}. \]
\lang{de}{Die Seiten sind verschieden lang, also handelt es sich um keine Raute.}
\lang{en}{These edges have different lengths, so $ABCD$ is not a diamond.}

\end{example}

\begin{quickcheck}
		\field{rational}
		\type{input.function}
		\begin{variables}
			\randint{a1}{-6}{1}
			\randint{a2}{0}{3}
			\randint{b1}{3}{6}
			\randint{b2}{-3}{1}
			\randint{c1}{0}{3}
			\randint{c2}{3}{6}
			\randint[Z]{v3}{-2}{2}
			\function[calculate]{nab2}{(a1-b1)^2+(a2-b2)^2}
			\function[calculate]{nac2}{(a1-c1)^2+(a2-c2)^2}
			\function[calculate]{nbc2}{(c1-b1)^2+(c2-b2)^2}
			\function{ab}{sqrt(nab2)}
			\function{ac}{sqrt(nac2)}
			\function{bc}{sqrt(nbc2)}
		\end{variables}
		
			\text{\lang{de}{
      Bestimmen Sie die Seitenlängen des Dreiecks $ABC$ mit
			$A=(\var{a1};\var{a2})$, $B=(\var{b1};\var{b2})$ und $C=(\var{c1};\var{c2})$.\\
			Die Seitenlängen sind
			$\overline{AB}=$\ansref, $\overline{AC}=$\ansref und $\overline{BC}=$\ansref.\\
			(Verwenden Sie \emph{sqrt(..)} für die Wurzel einer Zahl.)
      }
      \lang{en}{
      Determine the edge lengths of the triangle $ABC$ with 
      $A=(\var{a1};\var{a2})$, $B=(\var{b1};\var{b2})$ and $C=(\var{c1};\var{c2})$.\\
      The edge lengths are 
      $\overline{AB}=$\ansref, $\overline{AC}=$\ansref and $\overline{BC}=$\ansref.\\
      (Use \emph{sqrt(..)} for the root of a number.)
      }}
		
		\begin{answer}
			\solution{ab}
			\checkAsFunction{x}{-2}{2}{20}
		\end{answer}
		\begin{answer}
			\solution{ac}
			\checkAsFunction{x}{-2}{2}{20}
		\end{answer}
		\begin{answer}
			\solution{bc}
			\checkAsFunction{x}{-2}{2}{20}
		\end{answer}
		\explanation{\lang{de}{
    Die Seitenlängen sind genau die Abstände der entsprechenden Eckpunkte, also gleich der
		Länge der Verbindungsvektoren. Also ist 
    }
    \lang{en}{
    The edge lengths are precisely the distances between the connected vertices, so they are equal 
    to the lengths of the vectors connecting the vertices.
    }
    $\overline{AB}=\left\|\overrightarrow{AB}\right\|=\var{ab}$,
		$\overline{AC}=\left\|\overrightarrow{AC}\right\|=\var{ac}$ \lang{de}{und}\lang{en}{and} 
		$\overline{BC}=\left\|\overrightarrow{BC}\right\|=\var{bc}$.
		}
	\end{quickcheck}


\end{visualizationwrapper}


\end{content}