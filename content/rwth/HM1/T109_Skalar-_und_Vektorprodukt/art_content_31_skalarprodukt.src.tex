%$Id:  $
\documentclass{mumie.article}
%$Id$
\begin{metainfo}
  \name{
    \lang{de}{Skalarprodukt}
    \lang{en}{Scalar product}
  }
  \begin{description} 
 This work is licensed under the Creative Commons License Attribution 4.0 International (CC-BY 4.0)   
 https://creativecommons.org/licenses/by/4.0/legalcode 

    \lang{de}{Beschreibung}
    \lang{en}{Description}
  \end{description}
  \begin{components}
    \component{generic_image}{content/rwth/HM1/images/g_tkz_T109_ScalarProduct.meta.xml}{T109_ScalarProduct}
    \component{generic_image}{content/rwth/HM1/images/g_img_00_video_button_schwarz-blau.meta.xml}{00_video_button_schwarz-blau}
    \component{generic_image}{content/rwth/HM1/images/g_tkz_T109_Robot.meta.xml}{T109_Robot}
    \component{generic_image}{content/rwth/HM1/images/g_img_00_Videobutton_schwarz.meta.xml}{00_Videobutton_schwarz}
    \component{js_lib}{system/media/mathlets/GWTGenericVisualization.meta.xml}{mathlet1}
  \end{components}
  \begin{links}
    \link{generic_article}{content/rwth/HM1/T109_Skalar-_und_Vektorprodukt/g_art_content_31_skalarprodukt.meta.xml}{content_31_skalarprodukt}
    \link{generic_article}{content/rwth/HM1/T109_Skalar-_und_Vektorprodukt/g_art_content_33_winkel.meta.xml}{content_33_winkel}
    \link{generic_article}{content/rwth/HM1/T110_Geraden,_Ebenen/g_art_content_38_abstaende.meta.xml}{abstaende}
  \end{links}
  \creategeneric
\end{metainfo}
\begin{content}
\usepackage{mumie.ombplus}
\ombchapter{9}
\ombarticle{1}
\usepackage{mumie.genericvisualization}

\begin{visualizationwrapper}

\title{\lang{de}{Skalarprodukt}\lang{en}{Scalar product}}
 
%\newcommand{\sproduct}[2]{\langle #1, #2 \rangle}
\newcommand{\sproduct}[2]{#1 \bullet #2}
 
\begin{block}[annotation]
  übungsinhalt
  
\end{block}
\begin{block}[annotation]
  Im Ticket-System: \href{http://team.mumie.net/issues/9048}{Ticket 9048}\\
\end{block}

\begin{block}[info-box]
\tableofcontents
\end{block}


\lang{de}{
Zur Motivation können Sie das folgende Beispiel anschauen, dessen
Sachverhalt Sie nach diesem Kapitel lösen werden können:
}
\lang{en}{
As motivation we have the following example, which we will have the tools to complete by the end of 
this chapter:
}
\begin{example}
  \begin{tabs*}[\initialtab{0}]
    \tab{\lang{de}{Roboter gestützte Lackierung}\lang{en}{Robot-assisted painting}}
      \begin{incremental}[\initialsteps{1}]
  \step
  \lang{de}{
  Ein Unternehmen ist für die Lackierung großflächiger Bauteile zuständig. 
  Zur Optimierung Ihrer Arbeitsweise und -abläufe will das 
  Unternehmen einen modernen Roboter erwerben, %der die Lackierung großflächiger Bauteile übernehmen soll. 
  der diese Arbeiten übernehmen soll. 
  Dieser Roboter wird in eine Wand am Punkt $A=(0;0;0,5)$ bzgl. eines intern gewählten 
  Koordinatensystems (in Metern) integriert und kann von dort aus $1$ Meter hoch sowie $3$ 
  Meter in die Tiefe fahren.
  }
  \lang{en}{
  A company is responsible for painting large building components. To optimise the process, the 
  company decides that a robot is to be used for this. This robot is to be connected to a wall at a 
  point which it stores as coordinates $A=(0;0;0.5)$, that is, $0.5$ meters above the bottom of the 
  wall. Its mount, along with the whole arm, can move $1$ meter upwards, and $3$ meters horizontally 
  along the horizontal axis. Suppose that the surface of the wall itself is the plane spanned by 
  ('generated by' in a sense) the vectors 
  $\vec{a}=\begin{pmatrix}0\\ 1 \\ 0\end{pmatrix}$ and 
  $\vec{b}=\begin{pmatrix} 0 \\ 0 \\ 1 \end{pmatrix}$.
  }\\
  %Dadurch erreicht er jeden Punkt, der durch die beiden Richtungen aufgespannt wird.\\
  \begin{center}
  \image{T109_Robot}
  \end{center}
  \lang{de}{
  Um die Bauteile lackieren zu können, muss der Roboter seinen Arm ausfahren. Er kann damit vom Punkt 
  $A$ aus den Arm bis zum Punkt $D=(1;0;0,5)$ ausfahren, um Bauteile zu lackieren. Er kann den Arm 
  ferner verlängern, um bis zu $0,3$ Meter nach oben zu schwenken, d. h. der Arm kann vom Punkt $D$ 
  aus bis zum Punkt $C=(1;0;0,8)$ schwenken, um diesen Bereich zu lackieren.\\
  \\\\
  Der zuständige Projektleiter des Unternehmens fragt sich nun:
  }
  \lang{en}{
  To be able to paint the components, the robot must extend its arm. Suppose that it initially 
  extends from its wall mount at point $A$ to point $D=(1;0;0,5)$. It can also pivot to reach up to 
  $0.3$ meters higher, for example it could move from point $D$ to point $C=(1;0;0.8)$, in order to 
  paint this region.
  \\\\
  The project leader responsible for the robot asks themselves:
  }
  \begin{itemize}
  \item \lang{de}{
        Fährt der Arm des Roboters zunächst orthogonal (also in einem $90^\circ$-Winkel) zur Wand aus?
        }
        \lang{en}{
        Does the arm of the robot initially extend orthogonally (at a $90^\circ$ angle) to the wall?
        }
  \item \lang{de}{Wie lang wird der Arm des Roboters maximal ausgefahren?}
        \lang{en}{How long is the arm of the robot when it is fully extended?}
  \item \lang{de}{Wie groß ist der Schwenkwinkel des Arms?}
        \lang{en}{How large of an angle can the arm of the robot swing along?}
  \item \lang{de}{Wie groß ist der maximale Bereich, den der Roboter lackieren kann?}
        \lang{en}{What is the largest region that the robot can paint?}
  \end{itemize}\\
  \lang{de}{Diese Fragen lassen sich mit Hilfe des Skalarproduktes beantworten.}
  \lang{en}{These questions can be answered with help from the scalar product.}
  
  \step
  \lang{de}{
  Der Arm des Roboters fährt erst senkrecht zur Wand aus. Mathematisch lässt sich dies so begründen:
  Die Wandebene wird durch die beiden Vektoren $\vec{a}=\begin{pmatrix}0\\ 1 \\ 0\end{pmatrix}$ und 
  $\vec{b}=\begin{pmatrix} 0 \\ 0 \\ 1 \end{pmatrix}$ aufgespannt. Der Verbindungsvektor vom Punkt 
  $A$ zum Punkt $D$ lautet $\vec{d}=\begin{pmatrix} 1 \\ 0 \\ 0 \end{pmatrix}$. Das Skalarprodukt ist
  }
  \lang{en}{
  The arm of the robot initially extends orthogonally to the wall. Mathematically this can be 
  justified as follows: the surface of the wall is spanned by the vectors 
  $\vec{a}=\begin{pmatrix}0\\ 1 \\ 0\end{pmatrix}$ and 
  $\vec{b}=\begin{pmatrix} 0 \\ 0 \\ 1 \end{pmatrix}$. 
  The vector from point $A$ to point $D$ is $\vec{d}=\begin{pmatrix} 1 \\ 0 \\ 0 \end{pmatrix}$. 
  The scalar products between this vector and the vectors along the surface of the wall are 
  }
  \[\vec{a}\bullet\vec{d}=\begin{pmatrix} 0 \\ 1 \\ 0 \end{pmatrix}\bullet\begin{pmatrix} 1 \\ 0 \\ 0 \end{pmatrix}=0\cdot 1+1\cdot 0+ 0\cdot 0=0\]
  \lang{de}{und}
  \lang{en}{and}
  \[\vec{b}\bullet\vec{d}=\begin{pmatrix} 0 \\ 0 \\ 1 \end{pmatrix}\bullet\begin{pmatrix} 1 \\ 0 \\ 0 \end{pmatrix}=0\cdot 1+0\cdot 0+ 1\cdot 0=0.\]
  \lang{de}{
  Damit steht der Verbindungsvektor orthogonal zu den beiden Vektoren, die die Wandebene aufspannen; 
  also steht er senkrecht zur Wand.
  }
  \lang{en}{
  Hence the vector from point $A$ to point $D$ is orthogonal to the two vectors that span the wall's 
  surface, so the robot arm extends orthogonally from the wall.
  }

  \step \lang{de}{
  Die maximale Länge des ausgefahrenen Roboterarms ergibt sich bei maximalem Schwenkwinkel.
  Die maximale Länge ist somit gleich dem Abstand des Punktes $A$ zum
  Punkt $C$. Dies lässt sich mit dem Satz von Pythagoras berechnen, oder mit Hilfe des Skalarprodukts
  }
  \lang{en}{
  Supposing the robot arm is at its maximal length when it is extended to the point $C$, the maximal 
  length is the length of the vector from $A$ to $C$. We can calculate this using either the 
  Pythagorean theorem or the scalar product
  }
  \[\left\|\vec{c}\right\|=\sqrt{\vec{c}\bullet\vec{c}} = 
  \sqrt{1^2+0^2+0\lang{de}{,}\lang{en}{.}3^2}\approx1\lang{de}{,}\lang{en}{.}044~[m]\]

  \step \lang{de}{
  Um den Schwenkwinkel zu bestimmen, werden der Verbindungsvektor 
  $\vec{d}=\begin{pmatrix} 1 \\ 0 \\ 0 \end{pmatrix}$ der Punkte $A$ und $D$ sowie der 
  Verbidungsvektor $\vec{c}$ der Punkte $A$ und $C$ benötigt: 
  $\vec{c}=\begin{pmatrix} 1 \\ 0 \\ 0,3 \end{pmatrix}$.
  Der Schwenkwinkel des Roboterarms lässt sich mit der folgenden Formel, in der das Skalarprodukt 
  wieder involviert ist, berechnen:
  }
  \lang{en}{
  To determine the angle that the robot arm can swing, we find the vector 
  $\vec{d}=\begin{pmatrix} 1 \\ 0 \\ 0 \end{pmatrix}$ from the point $A$ to the point $D$, and the 
  vector $\vec{c}=\begin{pmatrix} 1 \\ 0 \\ 0,3 \end{pmatrix}$ from the point $A$ to the point $C$. 
  The angle between these vectors can be calculated using the scalar product:
  }
  \[\arccos\left(\frac{\vec{d}\bullet\vec{c}}
  {\left\|\vec{d}\right\|\cdot\left\|\vec{c}\right\|}\right) = 
  \arccos\left(\frac{1}{1\cdot \sqrt{1\lang{de}{,}\lang{en}{.}09}}\right) \approx
  16\lang{de}{,}\lang{en}{.}7^\circ\]
  \lang{de}{Letztlich ergibt sich als maximal lackierbarer Bereich eine Fläche von}
  \lang{en}{Finally we calculate the maximal region that can be painted,}
 \[1\lang{de}{,}\lang{en}{.}3m\cdot 3m = 3\lang{de}{,}\lang{en}{.}9 m^2,\]
 \lang{de}{
 da der Arm $3$ Meter in die Tiefe fahren kann und ein Meter hinauf fahren kann. Zu dieser Höhe kommt 
 noch eine Höhe von $0,3$ Metern, da der Roboterarm zusätzlich $0,3$ Meter hinaufschwenken kann. 
 Diese einfache Rechnung setzt allerdings voraus, dass die Richtungsvektoren jeweils orthogonal 
 aufeinander stehen.
 }
 \lang{en}{
 as the arm can move $3$ meters horizontally along the wall and $1$ meter upwards. In addition to 
 this, it can swing another $0.3$ meters upwards, giving a rectangular area of $3$ meters by $1.3$ 
 meters.
 }
\end{incremental}
    \end{tabs*}
\end{example}

\lang{de}{
Um später mit geometrischen Objekten rechnen zu k"onnen, m"ussen wir in der Lage sein, auch 
Abst"ande von Punkten und Winkel zwischen Geraden etc. berechnen zu k"onnen.
\\\\
Wichtigstes Hilfsmittel dazu ist das sogenannte \textit{Skalarprodukt} zweier Vektoren.
}
\lang{en}{
In order to be able to work with geometric objects later, we must be able to calculate the distances 
between points and angles between lines, etc.
\\\\
An important tool for this is the so-called \textit{scalar product}.
}

% \begin{remark}
% Wir gehen im Folgenden von der Anschauung im $\R^2$ und $\R^3$ aus und erhalten mittels des Skalarprodukts 
% Formeln f"ur L"angen und Winkel. In der Mathematik werden jedoch normalerweise diese Formeln als Definition 
% verwendet, weil man diese auch f"ur abstrakte Vektorr"aume verwenden kann.
% \end{remark}

\section{\lang{de}{Definition des Skalarprodukts}
         \lang{en}{Definition of the scalar product}} \label{def:skalarprodukt}

\begin{definition}
\lang{de}{
Es seien $\vec{v}$, $\vec{w}$ Vektoren im $\R^n$. Das \emph{(Standard-)Skalarprodukt}
$\sproduct{\vec{v}}{\vec{w}}$ von $\vec{v}$ und $\vec{w}$ ist definiert als die reelle Zahl
}
\lang{en}{
Let $\vec{v}$ and $\vec{w}$ be vectors in $\R^n$. The \emph{scalar product} $\sproduct{\vec{v}}{\vec{w}}$ of $\vec{v}$ and $\vec{w}$ 
is defined as the real number
}
\begin{align*}
\sproduct{\vec{v}}{\vec{w}}  &=  
\sproduct{\begin{pmatrix}
v_1 \\ v_2 \\ \vdots \\ v_n
\end{pmatrix}}{\begin{pmatrix}
w_1 \\ w_2 \\ \vdots \\ w_n
\end{pmatrix}} \\
&:= 
\sum_{i=1}^n v_i \cdot w_i 
=  v_1 \cdot w_1 + v_2 \cdot w_2 + \cdots + v_n \cdot w_n %\\
%&= v^T\cdot w \quad \text{("ubliche Matrixmultiplikation)}.
\end{align*}
\lang{de}{
\floatright{\href{https://www.hm-kompakt.de/video?watch=712}{\image[75]{00_Videobutton_schwarz}}}\\\\
}
\lang{en}{
The scalar product is also called the \emph{dot product} or \emph{inner product}.
}
\end{definition}

\begin{remark}\label{skalarprodukt_winkel}
\lang{de}{
Das (Standard-)Skalarprodukt ist eine mathematische Verknüpfung, die zwei Vektoren eine (reelle) Zahl zuordnet.
\\\\
Wir haben bereits einen Zusammenhang zwischen Winkel, Längen und Skalarprodukt angekündigt.
Als Ausblick auf das gesamte Kapitel und als erste geometrische Interpretation sollen die vier in der 
folgenden Abbildung dargestellten Szenarien einmal betrachtet werden:
}
\lang{en}{
The scalar product maps two vectors onto a single (real) number.
\\\\
We have already hinted at a relationship between angles, lengths and the scalar product. As a first 
geometric interpretation we consider the scenarios considered in the following graphics.
}\\\\
\begin{center}
\image{T109_ScalarProduct}
\end{center}
\begin{enumerate}
\item \lang{de}{
Die beiden Vektoren $\vec{a}$ und $\vec{b}$ sind gleichgerichtet (und parallel). Das Skalarprodukt 
ist dann eine positive reelle Zahl. Für den entgegengesetzt gerichteten Vektor ($\vec{-b}$) ist das 
Skalarprodukt mit $\vec{a}$ eine negative reelle Zahl.\\
Dies kann durch eine Rechnung begründet werden.
}
\lang{en}{
The two vectors $\vec{a}$ and $\vec{b}$ are parallel and pointing in the same direction. Their scalar 
product is a positive real number. For the vector pointing to the opposite direction ($\vec{-b}$), 
the scalar product with $\vec{a}$ is a negative real number.\\
This can be explained by a simple calculation.
}
\begin{tabs*}[\initialtab{0}]
\tab{\lang{de}{Begründung}\lang{en}{Explanation}}
\lang{de}{
Sind die beiden Vektoren parallel (oder sogar gleich), dann ist $\vec{b}=c \cdot \vec{a}$ mit $c>0$ 
(wir nehmen natürlich auch $\vec{a} \neq \vec{0}$ an), also ein Vielfaches vom anderen Vektor. Dann 
gilt
}
\lang{en}{
If the two vectors are parallel (or even equal), then $\vec{b}=c \cdot \vec{a}$ for some $c>0$ 
(assuming of course that $\vec{a} \neq \vec{0}$). Then
}
\[
\vec{a} \bullet \vec{b} = \vec{a} \bullet (c \cdot \vec{a}) = \sum_{i=1}^n c \cdot a_i^2 = c \cdot \sum_{i=1}^n a_i^2 > 0,
\]
\lang{de}{
da Quadrate immer nicht-negativ sind. Die Zahl ist positiv, weil mindestens ein Summand positiv ist 
($\vec{a} \neq 0$). In dieser Rechnung haben wir nun schon einige Eigenschaften gesehen, die wir 
gleich auch noch einmal aufgreifen werden.
}
\lang{en}{
as squares are always non-negative. The number is positive because at least one summand is positive 
($\vec{a} \neq 0$). If the vectors pointed in opposite directions, $c$ would be negative, as would 
the scalar product.
}
\end{tabs*}

\item \lang{de}{
Die Vektoren $\vec{a}$ und $\vec{c}$ bilden einen Winkel $0^\circ<\phi<90^\circ$, d. h., sie bilden 
einen spitzen Winkel. Das Skalarprodukt wird sich als eine positive reelle Zahl herausstellen.
}
\lang{en}{
The vectors $\vec{a}$ and $\vec{c}$ have an acute angle $0^\circ<\phi<90^\circ$ between them. Their 
scalar product is also a positive real number.
}\\\\
\item \lang{de}{
Die Vektoren $\vec{a}$ und $\vec{d}$ bilden einen rechten Winkel ($90^\circ$-Winkel), d. h., sie 
stehen orthogonal zueinander. Das Skalarprodukt ist dann Null. Die Herleitung erfolgt in 
\ref[content_33_winkel][in diesem Abschnitt]{rem:orthogonal_herleitung}.
}
\lang{en}{
The vectors $\vec{a}$ and $\vec{d}$ have a right angle ($90^\circ$) between them, i.e. they are 
orthogonal. Their scalar product is zero, a fact which is derived in a 
\ref[content_33_winkel][later section]{rem:orthogonal_herleitung}.
}\\\\
\item \lang{de}{
Die Vektoren $\vec{a}$ und $\vec{e}$ bilden einen Winkel $90^\circ<\phi<180^\circ$, d. h., sie bilden 
einen stumpfen Winkel. Das Skalarprodukt wird sich als eine negative reelle Zahl herausstellen.
}
\lang{en}{
The vectors $\vec{a}$ and $\vec{e}$ have an obtuse angle $90^\circ<\phi<180^\circ$ between them. 
Their scalar product is a negative real number.
}\\\\
\end{enumerate}

\lang{de}{
Diese Zusammenhänge zwischen dem Skalarprodukt und dem Winkel zweier Vektoren werden im Abschnitt 
\ref[content_33_winkel][Winkel zwischen Vektoren]{noAnchorFound} vertieft.
}
\lang{en}{
This relationship between the scalar product and the angle between two vectors is explored further 
in a \ref[content_33_winkel][later section]{noAnchorFound}.
}
\end{remark}

\begin{example}
\lang{de}{Sind zum Beispiel}
\lang{en}{Let}
$\vec{v}=\left( \begin{smallmatrix} 1 \\ 2 \\ 1 \end{smallmatrix}\right)$ 
\lang{de}{und}\lang{en}{and} 
$\vec{w}=\left( \begin{smallmatrix} 3 \\ -2 \\ 1 \end{smallmatrix}\right)$\lang{de}{,}\lang{en}{.} 
\lang{de}{so ist das Skalarprodukt von $\vec{v}$ und $\vec{w}$ gegeben durch}
\lang{en}{Then the scalar product of $\vec{v}$ and $\vec{w}$ is}
\[
\sproduct{\vec{v}}{\vec{w}} =\sproduct{\left( \begin{smallmatrix} 1 \\ 2 \\ 1 \end{smallmatrix}\right)}{
\left( \begin{smallmatrix} 3 \\ -2 \\ 1 \end{smallmatrix}\right)}= 1 \cdot 3 + 2 \cdot (-2) + 1 \cdot 1= 3-4 +1=0.
\]
\lang{de}{Die beiden Vektoren $\vec{v}$ und $\vec{w}$ stehen orthogonal zueinander.}
\lang{en}{The two vectors $\vec{v}$ and $\vec{w}$ are orthogonal (at a right angle to each other).}
\end{example}

\begin{block}[warning]
\lang{de}{
Das Skalarprodukt zweier Vektoren $\vec{v}$, $\vec{w}\in \R^n$ kann $0$ sein, auch wenn beide 
Vektoren ungleich dem Nullvektor sind, wie im Beispiel gesehen.
}
\lang{en}{
The scalar product of two vectors $\vec{v}$, $\vec{w}\in \R^n$ can be $0$ even if neither vector is 
the zero vector.
}
\end{block}

\begin{block}[warning]
\lang{de}{
Das Skalarprodukt sollte nicht mit der Skalarmultiplikation verwechselt werden. Das Skalarprodukt
ist ein \glqq{}Produkt\grqq{} zweier Vektoren, dessen Ergebnis ein Skalar (also eine reelle Zahl) ist.\\
Bei der Skalar-Multiplikation wird hingegen ein Vektor mit einem Skalar multipliziert und das Ergebnis ist ein Vektor.
}
\lang{en}{
The scalar product should not be confused with scalar multiplication. The scalar product is a 
'product' between two vectors, whose result is a scalar (a real number in this case).\\
Scalar multiplication on the other hand is between a vector and a scalar, with a vector as a result.
}\\\\
\begin{tabs*}
\tab{\lang{de}{Beispiel}\lang{en}{Example}}
\lang{de}{
Seien die beiden Vektoren $\vec{v}=\begin{pmatrix} 1\\-1\\5\end{pmatrix}$ sowie 
$\vec{w}=\begin{pmatrix} 2\\-3\\1\end{pmatrix}$ und die reelle Zahl $\lambda=3$ gegeben.\\
Dann ist das Ergebnis der Skalar-Multiplikation aus Vektor $\vec{v}$ und Skalar $\lambda$ wieder ein 
Vektor:
}
\lang{en}{
Consider the vectors $\vec{v}=\begin{pmatrix} 1\\-1\\5\end{pmatrix}$ and 
$\vec{w}=\begin{pmatrix} 2\\-3\\1\end{pmatrix}$, and the real number $\lambda=3$.\\
Then the result of scalar multiplication between the vector $\vec{v}$ and the scalar $\lambda$ is a 
vector:
}
$\lambda\cdot \vec{v}=3\cdot\begin{pmatrix} 1\\-1\\5\end{pmatrix} = 
\begin{pmatrix} 3\cdot 1\\3\cdot (-1)\\3\cdot 5\end{pmatrix} = 
\begin{pmatrix} 3\\-3\\15\end{pmatrix}$.\\
\lang{de}{
Dem gegenüber ergibt sich bei dem Skalarprodukt der beiden Vektoren $\vec{v}$ und $\vec{w}$ 
eine reelle Zahl:
}
\lang{en}{
On the other hand, the scalar product of the two vectors $\vec{v}$ and $\vec{w}$ is a real number:
}
$\sproduct{\vec{v}}{\vec{w}} = 
\sproduct{\begin{pmatrix} 1\\-1\\5\end{pmatrix}}{\begin{pmatrix} 2\\-3\\1\end{pmatrix}} = 
1\cdot 2+(-1)\cdot (-3)+5\cdot 1 = 10$.
\end{tabs*}
\end{block}

\begin{remark}
\begin{enumerate}
\item \lang{de}{
Das Skalarprodukt ist nur für Vektoren aus demselben Vektorraum definiert!
\\\\
Im unserem Falle bedeutet dies, dass wir nur das Skalarprodukt zweier Vektoren $\vec{v}$ und 
$\vec{w}$ ermitteln können, die gleich viele Einträge bzw. Zeilen besitzen:
\\\\
Seien die Vektoren $\vec{v}=\begin{pmatrix} 2\\3\\4\end{pmatrix}$, 
$\vec{w_1}=\begin{pmatrix} 1\\-2\\3\end{pmatrix}$ sowie 
$\vec{w_2}=\begin{pmatrix} 1\\-1\end{pmatrix}$ gegeben.
\\\\
Dann ist die Verknüpfung $\sproduct{\vec{v}}{\vec{w_2}}$ nicht definiert, da die Vektoren 
unterschiedlich viele Einträge besitzen. Wohl aber kann das Skalarprodukt der Vektoren $\vec{v}$ und 
$\vec{w_1}$ bestimmt werden: $\sproduct{\vec{v}}{\vec{w_1}}=2\cdot 1+3\cdot (-2)+4\cdot 3=8$.
}
\lang{en}{
The scalar product is only defined for vectors from the same vector space!
\\\\
In our case that means that the scalar product of two vectors $\vec{v}$ and $\vec{w}$ can only be 
taken if they have the same amount of components:
\\\\
Consider the vectors $\vec{v}=\begin{pmatrix} 2\\3\\4\end{pmatrix}$, 
$\vec{w_1}=\begin{pmatrix} 1\\-2\\3\end{pmatrix}$ and 
$\vec{w_2}=\begin{pmatrix} 1\\-1\end{pmatrix}$.
\\\\
The product $\sproduct{\vec{v}}{\vec{w_2}}$ is not defined, as the two vectors do not have the same 
number of components. However, the scalar product of the vectors $\vec{v}$ and $\vec{w_1}$ can be 
found: $\sproduct{\vec{v}}{\vec{w_1}}=2\cdot 1+3\cdot (-2)+4\cdot 3=8$.
}

\item \lang{de}{
Wir verwenden hier $\sproduct{\vec{v}}{\vec{w}}$ für das Skalarprodukt, um es von anderen Produkten, 
wie der Skalarmultiplikation und später auch der Matrizen-Multiplikation, auch optisch zu 
unterscheiden. In der Literatur wird das Skalarprodukt zweier Vektoren $\vec{v}$, $\vec{w}$ oft auch 
mit $\vec{v}\cdot \vec{w}$ oder $\langle \vec{v},\vec{w}\rangle$, seltener auch $(\vec{v},\vec{w})$, 
bezeichnet.
}
\lang{en}{
We use the notation $\sproduct{\vec{v}}{\vec{w}}$ for the scalar product to distinguish it from other 
products such as scalar multiplication and later matrix multiplication. In the literature, the 
scalar product of two vectors $\vec{v}$, $\vec{w}$ is also written as $\vec{v}\cdot \vec{w}$ or 
$\langle \vec{v},\vec{w}\rangle$, and more rarely as $(\vec{v},\vec{w})$.
}

\end{enumerate}
\end{remark}

\begin{quickcheck}
		\field{rational}
		\type{input.number}
		\begin{variables}
			\randint{v1}{-5}{5}
			\randint{v2}{-5}{5}
			\randint{v3}{-5}{5}
			\randint{w1}{-5}{5}
			\randint{w2}{-5}{5}
			\randint{w3}{-5}{5}
			\function[calculate]{s}{v1*w1+v2*w2+v3*w3}
			\function{slong}{v1*w1+v2*w2+v3*w3}
		\end{variables}
		
			\text{\lang{de}{
      Bestimmen Sie das Skalarprodukt der Vektoren }
      \lang{en}{Determine the scalar product of the vectors }
      $\vec{v}=\begin{pmatrix} \var{v1}\\ \var{v2} \\ \var{v3}\end{pmatrix}$ 
      \lang{de}{und}\lang{en}{and}
			$\vec{w}=\begin{pmatrix} \var{w1} \\ \var{w2} \\ \var{w3}\end{pmatrix}$.\\
			\lang{de}{Das Skalarprodukt ist $\sproduct{\vec{v}}{\vec{w}}=$\ansref.}
      \lang{de}{The scalar product is $\sproduct{\vec{v}}{\vec{w}}=$\ansref.}}		
			
		
		\begin{answer}
			\solution{s}
		\end{answer}
		\explanation{\lang{de}{Das Skalarprodukt berechnet sich durch }
                 \lang{en}{The scalar product is calculated by }
		$\sproduct{\begin{pmatrix} \var{v1}\\ \var{v2} \\ \var{v3}
			\end{pmatrix}}{\begin{pmatrix} \var{w1} \\ \var{w2} \\ \var{w3}\end{pmatrix}} =
			\var{slong}=\var{s}$.
		}
	\end{quickcheck}
    
    \begin{quickcheck}
		\field{rational}
		\type{input.number}
		\begin{variables}
			\randint{v1}{-5}{5}
			\randint[Z]{v2}{-5}{5}
			\randint{v3}{-5}{5}
			\randint{w1}{-5}{5}
			%\randint{w2}{-5}{5}
			\randint{w3}{-5}{5}
			\function[calculate]{s}{(-v1*w1-v3*w3)/v2}
			\function{slong}{v1*w1+v2*a+v3*w3}
		\end{variables}
		
			\text{\lang{de}{Bestimmen Sie den Parameter $a$ so, dass das Skalarprodukt der Vektoren }
            \lang{en}{Determine the parameter $a$ such that the scalar product of the vectors }
      $\vec{v}=\begin{pmatrix} \var{v1}\\ \var{v2} \\ \var{v3}
			\end{pmatrix}$ \lang{de}{und}\lang{en}{and}
			$\vec{w}=\begin{pmatrix} \var{w1} \\ a \\ \var{w3}\end{pmatrix}$
			\lang{de}{Null ist.\\ Es muss gelten:}
      \lang{en}{is zero.\\ It must be:}
      $a=$\ansref.}			
			
		
		\begin{answer}
			\solution{s}
		\end{answer}
		\explanation{\lang{de}{Bestimmen Sie das Skalarprodukt }
                 \lang{en}{Determine the scalar product }
      $\sproduct{\begin{pmatrix} \var{v1}\\ \var{v2} \\ \var{v3}
			\end{pmatrix}}{\begin{pmatrix} \var{w1} \\ a \\ \var{w3}\end{pmatrix}}$, 
      \lang{de}{setzen Sie den Term gleich $0$ und formen Sie nach $a$ um.}
      \lang{en}{set it to be equal to $0$ and rearrange for $a$.}}
	\end{quickcheck}



\section{\lang{de}{Regeln für das Skalarprodukt}
         \lang{en}{Rules for the scalar product}}\label{sec:regeln}

\begin{rule}
\lang{de}{F"ur das Skalarprodukt gelten folgende Regeln:}
\lang{en}{The scalar product obeys the following rules:}
\begin{enumerate}
\item \lang{de}{
      Für $\vec{v}\in \R^n$ gilt
      \[ \sproduct{\vec{v}}{\vec{v}} > 0,\quad \text{falls} \vec{v}\neq \vec{0}, \]
      und $\sproduct{\vec{0}}{\vec{0}}=0$.
      }
      \lang{en}{
      For $\vec{v}\in \R^n$ we have
      \[ \sproduct{\vec{v}}{\vec{v}} > 0,\quad \text{if} \vec{v}\neq \vec{0}, \]
      and $\sproduct{\vec{0}}{\vec{0}}=0$.
      }
\item \lang{de}{
      Seien $\vec{v}$, $\vec{w}\in \R^n$. Dann gilt
      \[
      \sproduct{\vec{v}}{\vec{w}} = \sproduct{\vec{w}}{\vec{v}}.
      \]
      Man darf also die Vektoren $\vec{v}$ und $\vec{w}$ vertauschen. Es gilt also das 
      Kommutativgesetz.
      }
      \lang{en}{
      Let $\vec{v}$, $\vec{w}\in \R^n$. Then
      \[
      \sproduct{\vec{v}}{\vec{w}} = \sproduct{\vec{w}}{\vec{v}}.
      \]
      We can thus swap the vectors $\vec{v}$ and $\vec{w}$, i.e. the scalar product is commutative.
      }
\item \lang{de}{
      Seien $r$, $s \in \R$ und $\vec{u}$, $\vec{v}$, $\vec{w} \in \R^n$. Dann gelten
      \[
      \sproduct{\vec{v}}{\left(r \cdot \vec{w} + s \cdot \vec{u} \right)} = r \cdot (\sproduct{\vec{v}}{\vec{w}})
      + s \cdot (\sproduct{\vec{v}}{\vec{u}}),
      \]
      sowie
      \[
      \sproduct{\left(r \cdot \vec{w} + s \cdot \vec{u} \right)}{\vec{v}} = r \cdot (\sproduct{\vec{w}}{\vec{v}})
      + s \cdot (\sproduct{\vec{u}}{\vec{v}}).
      \]
      Es gilt also eine Art Distributivgesetz.
      }
      \lang{en}{
      Let $r$, $s \in \R$ and $\vec{u}$, $\vec{v}$, $\vec{w} \in \R^n$. Then
      \[
      \sproduct{\vec{v}}{\left(r \cdot \vec{w} + s \cdot \vec{u} \right)} = r \cdot (\sproduct{\vec{v}}{\vec{w}})
      + s \cdot (\sproduct{\vec{v}}{\vec{u}}),
      \]
      and
      \[
      \sproduct{\left(r \cdot \vec{w} + s \cdot \vec{u} \right)}{\vec{v}} = r \cdot (\sproduct{\vec{w}}{\vec{v}})
      + s \cdot (\sproduct{\vec{u}}{\vec{v}}).
      \]
      Hence there is a distributivity law for the scalar product over vector addition.
      }
\end{enumerate}
\lang{de}{
\floatright{\href{https://www.hm-kompakt.de/video?watch=713}{\image[75]{00_Videobutton_schwarz}}}\\\\
}
\lang{en}{}
\end{rule}

\begin{block}[warning]
\lang{de}{
Während man bei der mathematischen Verknüpfung des Skalarprodukts eine Art Kommutativ- sowie 
Distributivgestz hat, gilt keinerlei Art eines Assoziativgesetzes. Denn wie man leicht nachrechnen 
kann, gilt:
}
\lang{en}{
Although the scalar product is commutative and has some sort of distributive law, it has no 
associativity law. This is easily shown via an example:
}\\\\
$[\sproduct{\begin{pmatrix} 3\\4\end{pmatrix}}{\begin{pmatrix} -1\\2\end{pmatrix}}]\cdot\begin{pmatrix} 2\\1\end{pmatrix}\neq
\begin{pmatrix} 3\\4\end{pmatrix}\cdot[\sproduct{\begin{pmatrix} -1\\2\end{pmatrix}}{\begin{pmatrix} 2\\1\end{pmatrix}]}$
\end{block}

\begin{remark}
\lang{de}{
Die \ref[content_31_skalarprodukt][1. Regel]{sec:regeln} haben wir in der frühen 
\ref[content_31_skalarprodukt][Bemerkung]{skalarprodukt_winkel} praktisch schon einmal gesehen: Die 
Vektoren sind hier identisch (und somit parallel). Das Skalarprodukt $\sproduct{\vec{w}}{\vec{w}}$ 
ist dann eine reelle positive Zahl.
\\\\
Die \ref[content_31_skalarprodukt][2. Regel]{sec:regeln} ist im Hinblick auf unsere geometrische 
Interpretation auch nicht überraschend: Die Lage eines Vektors $\vec{v}$ zu einem Vektor $\vec{w}$ 
(bzw. der Innenwinkel zwischen beiden Vektoren) in der Bemerkung \ref{skalarprodukt_winkel} hat einen 
Zusammenhang mit dem Skalarprodukt. An der Lage der Vektoren zueinander ändert sich nichts, wenn die 
Lage des Vektors $\vec{v}$ zu $\vec{w}$ betrachtet oder die Lage des Vektors $\vec{w}$ zu $\vec{v}$ 
betrachtet wird. Das Skalarprodukt $\sproduct{\vec{v}}{\vec{w}}$ ist gleich 
$\sproduct{\vec{w}}{\vec{v}}$.
}
\lang{en}{
We have already seen \ref[content_31_skalarprodukt][rule 1]{sec:regeln} at play in a previous 
\ref[content_31_skalarprodukt][remark]{skalarprodukt_winkel}: it holds for the scalar product of a 
vector with itself, and since every vector is parallel to itself, the scalar product is a positive 
real number by that remark.
\\\\
\ref[content_31_skalarprodukt][Rule 2]{sec:regeln} also makes sense given our geometric 
interpretation: the direction of a vector $\vec{v}$ relative to a vector $\vec{w}$ (or the angle 
between them) in remark \ref{skalarprodukt_winkel} is related to the scalar product. The angle 
between the vectors $\vec{w}$ and $\vec{v}$ is of course the same as the angle between 
$\vec{v}$ to $\vec{w}$.
}
\end{remark}

\begin{example}
\lang{de}{Gesucht ist die reelle Zahl $r$ so, dass}
\lang{en}{Suppose we want to find the real number $r$ such that}
\[ \sproduct{\left( \begin{pmatrix} 2 \\ 0 \end{pmatrix}+r\cdot  \begin{pmatrix} 1 \\ 2 \end{pmatrix}\right)}{
\begin{pmatrix} 1 \\ 2 \end{pmatrix}} =0.\]
\lang{de}{Mit den Rechenregeln ist}
\lang{en}{By the rules covered in this section,}
\begin{eqnarray*}
 \sproduct{\left( \begin{pmatrix} 2 \\ 0 \end{pmatrix} + 
   r\cdot  \begin{pmatrix} 1 \\ 2 \end{pmatrix}\right)}{
\begin{pmatrix} 1 \\ 2 \end{pmatrix}} 
&=& \sproduct{\begin{pmatrix} 2 \\ 0 \end{pmatrix}}{\begin{pmatrix} 1 \\ 2 \end{pmatrix}}
+r\cdot \sproduct{\begin{pmatrix} 1 \\ 2 \end{pmatrix}}{\begin{pmatrix} 1 \\ 2 \end{pmatrix}} \\
&=& (2\cdot 1+0\cdot 2)+ r\cdot (1^2+2^2)=2+5r.
\end{eqnarray*}
\lang{de}{
Die Gleichung ist also für $r=-\frac{2}{5}$ erfüllt.
\\\\
Anmerkung: Derartige Rechnungen werden bei der Bestimmung von \link{abstaende}{Lotfußpunkten} öfter auftauchen. 
}
\lang{en}{
The equation is satisfied by $r=-\frac{2}{5}$.
\\\\
Note: such calculations is performed when \link{abstaende}{finding the orthogonal distance from a 
point to a line}.
}
\end{example}


\begin{quickcheck}
		\field{rational}
		\type{input.number}
		\begin{variables}
			\randint{v}{1}{5}
			\randint{w}{1}{5}
			\function[calculate]{v2}{v^2}
			\function[calculate]{w2}{w^2}
			\function[calculate]{r}{v/w}
			\function[calculate]{r2}{r^2}
			\function[calculate]{s1}{-r}
			\function[calculate]{s2}{r}
		\end{variables}
		
			\text{\lang{de}{
       Sei $\vec{v}$ ein Vektor mit $\sproduct{\vec{v}}{\vec{v}}=\var{v2}$ und $\vec{w}$ ein Vektor 
       mit $\sproduct{\vec{w}}{\vec{w}}=\var{w2}$. Vereinfachen Sie mit Hilfe der obigen Regeln den 
       Ausdruck $\sproduct{(\vec{v}+r\cdot \vec{w})}{(\vec{v}-r\cdot \vec{w})}$ (für $r\in \R$) und 
       bestimmen Sie alle $r$, für welche der Ausdruck gleich $0$ ist.\\
			 Der Ausdruck ist $0$ für $r=$\ansref und $r=$\ansref. (Nennen Sie die negative Lösung zuerst).
       }
       \lang{en}{
       Let $\vec{v}$ be a vector with $\sproduct{\vec{v}}{\vec{v}}=\var{v2}$ and $\vec{w}$ a vector 
       with $\sproduct{\vec{w}}{\vec{w}}=\var{w2}$. Simplify the expression 
       $\sproduct{(\vec{v}+r\cdot \vec{w})}{(\vec{v}-r\cdot \vec{w})}$ (for $r\in \R$) using the 
       rules introduced above, and determine the $r$ for which the expression is equal to $0$.\\
       The expression is equal to $0$ for $r=$\ansref and $r=$\ansref. (Negative solution first).
       }}			
			
		
		\begin{answer}
			\solution{s1}
		\end{answer}
		\begin{answer}
			\solution{s2}
		\end{answer}
		\explanation{\lang{de}{Das Skalarprodukt vereinfacht sich zu }
                 \lang{en}{The scalar product simplifies to }
		\begin{eqnarray*}
		\sproduct{(\vec{v}+r\cdot \vec{w})}{
			(\vec{v}-r\cdot \vec{w})} &=& \sproduct{\vec{v}}{(\vec{v}-r\cdot \vec{w})}
			+r\cdot \sproduct{\vec{w}}{(\vec{v}-r\cdot \vec{w})} \\
			&=& \sproduct{\vec{v}}{\vec{v}}- r\cdot \sproduct{\vec{v}}{\vec{w}}+r\cdot \sproduct{\vec{w}}{\vec{v}}
			-r^2\cdot \sproduct{\vec{w}}{\vec{w}} \\
			&=& \var{v2}-r^2\cdot \var{w2}
			\end{eqnarray*}
		\lang{de}{
    Dieser Ausdruck ist also genau dann gleich $0$, wenn $r^2=\var{r2}$,
		also für $r=\var{s1}$ und $r=\var{s2}$.
    }
    \lang{en}{
    This expression is equal to $0$ if and only if $r^2=\var{r2}$, so for $r=\var{s1}$ and 
    $r=\var{s2}$.
    }}
	\end{quickcheck}
\end{visualizationwrapper}

\lang{de}{
Die Definition des Skalarprodukts sowie dessen Eigenschaften werden im folgenden Video noch einmal näher erläutert:\\
 \floatright{\href{https://api.stream24.net/vod/getVideo.php?id=10962-2-10791&mode=iframe&speed=true}
{\image[75]{00_video_button_schwarz-blau}}}\\\\
}
\lang{en}{}

\end{content}