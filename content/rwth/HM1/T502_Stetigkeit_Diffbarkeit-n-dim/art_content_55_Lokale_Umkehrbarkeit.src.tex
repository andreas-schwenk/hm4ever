%$Id:  $
\documentclass{mumie.article}
%$Id$
\begin{metainfo}
  \name{
    \lang{en}{...}
    \lang{de}{Lokale Umkehrbarkeit von Funktionen}
   }
  \begin{description} 
 This work is licensed under the Creative Commons License Attribution 4.0 International (CC-BY 4.0)   
 https://creativecommons.org/licenses/by/4.0/legalcode 

    \lang{en}{...}
    \lang{de}{...}
  \end{description}
  \begin{components}
\component{generic_image}{content/rwth/HM1/images/g_img_00_video_button_schwarz-blau.meta.xml}{00_video_button_schwarz-blau}
  \end{components}
  \begin{links}
\link{generic_article}{content/rwth/HM1/T306_Reelle_Quadratische_Matrizen/g_art_content_15_inverse_matrix.meta.xml}{content_15_inverse_matrix}
\link{generic_article}{content/rwth/HM1/T502_Stetigkeit_Diffbarkeit-n-dim/g_art_content_54_Differentiation.meta.xml}{content_54_Differentiation}
\end{links}
  \creategeneric
\end{metainfo}

\begin{content}
\begin{block}[annotation]
	Im Ticket-System: \href{https://team.mumie.net/issues/21547}{Ticket 21547}
\end{block}
\begin{block}[annotation]
ToDo: Bessere Motivationen und Anschauungen einpflegen.
\end{block}
\usepackage{mumie.ombplus}
\ombchapter{2}
\ombarticle{3}
\usepackage{mumie.genericvisualization}
\lang{de}{\title{Lokale Umkehrbarkeit von Funktionen}}
\begin{block}[info-box]
\tableofcontents
\end{block}

\section{Lokale Umkehrbarkeit von Funktionen}
Zu entscheiden, ob eine Funktion in mehreren Veränderlichen eine Umkehrfunktion besitzt und welche 
Eigenschaften diese dann hat, ist im allgemeinen viel schwieriger zu beantworten als im eindimensionalen Fall.
Wir beschränken uns hier auf einige Bemerkungen und ein positives Resultat im Falle stetiger
Differenzierbarkeit. Auf Beweise verzichten wir an vielen Stellen.
\begin{remark}
Ist $f:U\to V$ eine stetige bijektive Abbildung zwischen zwei offenen (nicht leeren) Mengen
$U\subset \R^n$ und $V\subset \R^m$, dann ist $m=n$.

Das entspricht unserer Vorstellung davon, dass eine offene Kugel im $\R^n$ etwas quantitativ Anderes ist als
eine im $\R^m$ für $m\neq n$. Zum Beispiel lässt sich eine Kreisscheibe im $\R^2$ nicht
stetig bijektiv auf eine Vollkugel im $\R^3$ abbilden.
\end{remark}
Wenn wir bereits viel über eine umkehrbare Funktion wissen, 
berechnet sich die totale Ableitung der Umkehrfunktion ganz ähnlich zum eindimensionalen Fall.
\begin{rule}[Totale Ableitung der Umkehrfunktion]\label{tot_Abl_Umkehrf}
Es sei $U\subset \R^n$ offen und $f:U\to\R^n$ sei injektiv und stetig differenzierbar 
mit offenem Bild $V:=f(U)\subset\R^n$.
Weiter sei die Umkehrabbildung $g:=f^{-1}:V\to U$ total differenzierbar.
Dann gilt für alle $y\in V$
\[\text{D}f(g(y))\cdot \text{D}g(y)=E_n\in M(n,n;\R).\]
Insbesondere ist die Jacobi-Matrix $\text{D}f(g(y))$ invertierbar mit
\[\big(\text{D}f(g(y))\big)^{-1}=\text{D}g(y).\]
\end{rule}
\begin{proof*}
Das ist eine direkte Folge der \ref[content_54_Differentiation][Kettenregel]{thm:kettenregel_mehrdim},
die wir nach Voraussetzung anwenden können.
\begin{incremental}[\initialsteps{0}]
\step
Die Verkettung $f\circ g=\text{id}_V:V\to V$ ist die Identität $y\mapsto y$. Insbesondere ist $f\circ g$
die lineare Abbildung, die durch die Einheitsmatrix beschrieben wird, $E_n\cdot y=y$. 
Deshalb gilt (siehe \ref[content_54_Differentiation][Beispiel]{ex:lin_Abb})
\[\text{D}(f\circ g)(y)=E_n\]
für alle $y\in V$. 
Nach der Kettenregel ist aber auch $\text{D}(f\circ g)(y)=\text{D}f(g(y))\cdot \text{D}g(y)$, und damit folgt die Behauptung.
\end{incremental}
\end{proof*}
%%
\begin{example}
Die Funktion $f:\R_{>0}\times\R_{>0}\to\R_{>0}\times\R_{>0}$,
$\begin{pmatrix}x\\ y\end{pmatrix} \mapsto \begin{pmatrix}y^2\\ x^2\end{pmatrix}$ 
ist bijektiv und stetig differenzierbar mit 
$\text{D}f(\begin{pmatrix}x\\ y\end{pmatrix}) = \begin{pmatrix}0&2y\\2x& 0\end{pmatrix}$.
Ihre Umkehrabbildung $g:\begin{pmatrix}x\\ y\end{pmatrix} \mapsto \begin{pmatrix}y^{\frac{1}{2}}\\ x^{\frac{1}{2}}\end{pmatrix}$
ist ebenfalls total differenzierbar mit 
$\text{D}g(\begin{pmatrix}x\\ y\end{pmatrix} )= \begin{pmatrix}0&\frac{1}{2} y^{-\frac{1}{2}}\\\frac{1}{2}x^{-\frac{1}{2}}&0\end{pmatrix}$.
Offensichtlich gilt also
\[\text{D}g(\begin{pmatrix}a\\ b\end{pmatrix} )=\text{D}f(\begin{pmatrix}b^{\frac{1}{2}}\\ a^{\frac{1}{2}}\end{pmatrix} )^{-1}
=\begin{pmatrix}0&2a^{\frac{1}{2}}\\2b^{\frac{1}{2}}& 0\end{pmatrix}^{-1}
=\begin{pmatrix}0&\frac{1}{2}b^{-\frac{1}{2}}\\\frac{1}{2}a^{-\frac{1}{2}}& 0\end{pmatrix}.
\]
\end{example}
%Video
\floatright{\href{https://api.stream24.net/vod/getVideo.php?id=10962-2-10925&mode=iframe&speed=true}{\image[75]{00_video_button_schwarz-blau}}}\\
\\

%%
\begin{quickcheck}
\text{Es sei $f:\R^2\to\R^2$, $x\mapsto \begin{pmatrix}2&0\\0&3\end{pmatrix}\cdot x$.
Dann gilt für die Ableitung $\text{D}g(a)$ der Umkehrfunktion $g=f^{-1}$ im Punkt $a=e_1$:}
\begin{choices}{unique}
    \begin{choice}
      \text{$\begin{pmatrix}\frac{1}{2}&0\\0&\frac{1}{3}\end{pmatrix}$}
      \solution{true}
    \end{choice}
    \begin{choice}
      \text{$\begin{pmatrix}\frac{1}{2}&0\\0&\frac{1}{3}\end{pmatrix}\cdot a=\begin{pmatrix}\frac{1}{2}\\0\end{pmatrix}$}
      \solution{false}
    \end{choice}
    \begin{choice}
      \text{Die Funktion $f$ ist nicht umkehrbar, also exisiert $\text{D}g(a)$ gar nicht.}
      \solution{false}
    \end{choice}
    \begin{choice}
      \text{$\frac{1}{2}\cdot \begin{pmatrix}2&0\\0&3\end{pmatrix} $}
      \solution{false}
    \end{choice}   
  \end{choices}
  \explanation{Die Abbildung $f:x\mapsto Ax$ ist linear und bijektiv ($\det A=6\neq 0$), also ist $\text{D}f(x)=A$ in \emph{jedem} Punkt $x\in\R^2$.
  Somit ist $\text{D}g(y)=A^{-1}$ in \emph{jedem} Punkt $y\in\R^2$.}
\end{quickcheck}
%%
Wir kommen nun zu unserer Hauptaussage über die Existenz von lokalen Umkehrfunktionen.
\begin{theorem}[Lokale Umkehrbarkeit]\label{thm:lokale-umkehrbarkeit}\label{thrm:lokale_umkehrbarkeit}
Es sei $M\subset\R^n$ offen, und es sei $f:M\to\R^n$ stetig differenzierbar.
Weiter sei die Jacobi-Matrix $\text{D}f(a)\in M(n,n;\R)$ in einem Punkt $a\in M$ invertierbar.
Dann existieren offene Teilmengen $U\subset M$, mit $a\in U$, und $V\subset f(M)$, mit  $b=f(a)\in V$, 
so dass für die Einschränkung 
\[f\mid_U:U\to V\]
gilt
\begin{itemize}
\item
$f\mid_U$ ist bijektiv,
\item
die Jacobi-Matrix $\text{D}f(x)$ ist invertierbar für alle $x\in U$, 
\item
die Umkehrabbildung $g:=(f\mid_U)^{-1}:V\to U$ ist stetig differenzierbar mit
\[\text{D}g(y)=\big(\text{D}f(g(y))\big)^{-1} \quad \text{bzw. }\quad \text{D}g(f(x))=\big(\text{D}f(x)\big)^{-1}.\]
\end{itemize}
\end{theorem}
Der Satz hat seine bekannte Entsprechung im Eindimensionalen: 
Ist $f:\R\to\R$ eine stetig differenzierbaren Funktion und gilt $f'(a)\neq 0$ für ein $a\in \R$,
dann folgt aus der Stetigkeit von $f'$, dass $f'(x)\neq 0$ für alle $x$ in einer kleinen Umgebung $U=U_{\epsilon}(a)$.
Auf $U$ ist $f$ also streng monoton und natürlich stetig, somit ist $f\mid_U:U\to f(U)$ umkehrbar, die Umkehrfunktion $f^{-1}$ ist wiederum differenzierbar mit
bekannter Ableitung $(f^{-1})'(y)=\frac{1}{f'(f^{-1}(y))}$.
\\
Im Falle mehrerer Veränderlicher müssen wir analog die Invertierbarkeit der Jacobi-Matrix fordern, 
um die Funktion lokal umkehren zu können.
\begin{block}[warning]%[important]
Satz \ref{thm:lokale-umkehrbarkeit} liefert die lokale Umkehrbarkeit einer \glqq hinreichend glatten\grqq{} Funktion.
Es ist aber nicht davon auszugehen, dass die Funktion eine globale Umkehrfunktion besitzt!
\end{block}

%%
\begin{example}
Es sei $f:\R^2\to\R^2$, 
$\begin{pmatrix}x\\ y\end{pmatrix}\mapsto \begin{pmatrix}x\cos y\\ x\sin y\end{pmatrix}$.
Offensichtlich ist $f$ stetig differenzierbar mit Jacobi-Matrix
\[\text{D}f((x,y)^T)=\begin{pmatrix}\cos y&-x\sin y\\ \sin y&x\cos y\end{pmatrix}.\]
Deren Determinante ist
\[\det\big(\text{D}f((x,y)^T)\big)=
x(\cos^2 y+\sin^2y)=x.\]
Die Jacobi-Matrix ist also in allen $\begin{pmatrix}x\\y\end{pmatrix}$ mit $x\neq 0$ invertierbar.
Wählen wir  ein beliebiges $ \begin{pmatrix}a\\ b\end{pmatrix}\in\R^2$ mit $a\neq 0$, so gibt es nach Satz \ref{thm:lokale-umkehrbarkeit}
eine offene Menge $U$, die $ \begin{pmatrix}a\\ b\end{pmatrix}$ enthält, und auf der $f$ umkehrbar ist mit differenzierbarer
Umkehrfunktion $g:f(U)\to U$ und Jacobi-Matrix
\[\text{D}g(f(\begin{pmatrix}a\\ b\end{pmatrix}))=\big(\text{D}f(\begin{pmatrix}a\\ b\end{pmatrix})\big)^{-1}
=\frac{1}{a}\begin{pmatrix}a\cos b&a\sin b\\-\sin b&\cos b\end{pmatrix}.\]
Dabei haben wir die \ref[content_15_inverse_matrix][Formel]{rule:inverse-2x2}
für die Inverse einer invertierbaren $(2\times 2)$-Matrix benutzt.
\\
Die Funktion $f$ besitzt aber \emph{keine globale Umkehrfunktion}, denn sie ist nicht injektiv.
So ist $\begin{pmatrix}x\cos y\\ x\sin y\end{pmatrix}=\begin{pmatrix}x\cos (y+2\pi k)\\ x\sin (y+2\pi k)\end{pmatrix}$
für alle $k\in\Z$.
\end{example}
%Video
\floatright{\href{https://api.stream24.net/vod/getVideo.php?id=10962-2-10926&mode=iframe&speed=true}{\image[75]{00_video_button_schwarz-blau}}}\\
\\

%%
\begin{quickcheck}
\text{Es sei $f:\R^2\to\R^2$, $\begin{pmatrix}x\\y\end{pmatrix}\mapsto \begin{pmatrix}\cos x\\\sin y\end{pmatrix}$.
Markieren Sie alle wahren Aussagen.}
\begin{choices}{multiple}
    \begin{choice}
      \text{Die Funktion $f$ ist umkehrbar mit $f^{-1}((x,y)^T)=\begin{pmatrix}\arccos x\\\arcsin y\end{pmatrix}$.}
      \solution{false}
    \end{choice}
    \begin{choice}
      \text{Die Funktion $f$ ist umkehrbar auf der Menge $\{(x,y)^T\mid x\neq 2\pi k\text{ und } y\neq \pi+2\pi l \text{ für } k,l\in\Z\}$,
      denn dort ist die Jabobi-Matrix $\text{D}f((x,y)^T)=\begin{pmatrix}-\sin x&0\\0&\cos y\end{pmatrix}$ invertierbar.}
      \solution{false}
    \end{choice}
    \begin{choice}
    \text{Die Funktion $f$ ist lokal umkehrbar in jedem Punkt der Menge $\{(x,y)^T\mid x\neq \pi k\text{ und } y\neq \frac{\pi}{2}+\pi l \text{ für } k,l\in\Z\}$.}
      \solution{true}
     \end{choice}
    \begin{choice}
      \text{Die Funktion $f$ ist nicht global umkehrbar.}
      \solution{true}
    \end{choice}
    \end{choices}
\end{quickcheck}
%%
Der folgende Satz über implizite Funktionen ist ein häufig genutztes Instrument der 
mehrdimensionalen Analysis, um die Lösbarkeit anspruchsvoller Gleichungen zu sichern.
Nur die theoretische Lösbarkeit sicherzustellen scheint für Anwender 
auf den ersten Blick wenig attraktiv. 
Sie ist aber essentiell, um zum Beispiel den Erfolg einer numerischen Näherung zu garantieren.
\begin{theorem}[Satz über implizite Funktionen]\label{thm:implizite_fkten}
Es sei $M\subset \R^{m+n} $ eine offene Teilmenge, und es sei 
\[f=\:\begin{pmatrix}f_1\\\vdots\\ f_n\end{pmatrix}\::M\to\R^n\]
eine stetig differenzierbare Funktion.
Mit $x\in\R^m$ und $y\in\R^n$ schreiben wir $z=\begin{pmatrix}x\\y\end{pmatrix}\in\R^{m+n}$. 
Wir setzen für die ersten $m$ partiellen Ableitungen von $f$
\[\text{D}_xf(z):=\big(\text{D}_{1}f(z),\ldots,\text{D}_{m}f(z)\big)\in M(n,m;\R),\]
und für die letzten $n$ partiellen Ableitungen von $f$
\[\text{D}_yf(z):=\big(\text{D}_{m+1}f(z),\ldots,\text{D}_{m+n}f(z)\big)\in M(n,n;\R).\]
Es seien $a\in\R^m$ und $b\in\R^n$ gegeben mit  $c=\begin{pmatrix}a\\b\end{pmatrix}\in M$, weiter sei
\[f(c)=0\quad\text{  und }\quad \text{D}_yf(c) \text{ invertierbar}.\]
Dann
lässt sich die Gleichung
\[f(\begin{pmatrix}x\\y\end{pmatrix})=0\]
in einer Umgebung von $c$ eindeutig nach $y$ auflösen. Das heißt, es gibt eine Umgebung $U\subset\R^m$ von $a$ und
eine Umgebung $V\subset\R^n$ von $b$ sowie eine stetig differenzierbare Funktion $g:U\to V$ (eine sog. Auflösung) mit den folgenden
Eigenschaften.
\begin{enumerate}
\item[(i)]
$U\times V:=\{\begin{pmatrix}x\\y\end{pmatrix}\mid x\in U, y\in V\}\subset M$.
\item[(ii)]
$f(\begin{pmatrix}x\\g(x)\end{pmatrix})=0$ für alle $x\in U$.
\item[(iii)]
Aus $\begin{pmatrix}x\\y\end{pmatrix}\in U\times V$ mit $f(\begin{pmatrix}x\\y\end{pmatrix})=0$ folgt $y=g(x)$.
\item[(iv)]
$\text{D}g(x)=-\big(\text{D}_yf(\begin{pmatrix}x\\g(x)\end{pmatrix})\big)^{-1}\big(\text{D}_xf(\begin{pmatrix}x\\g(x)\end{pmatrix})\big)$
für alle $x\in U$.
\end{enumerate}
Insbesondere gilt $\text{D}g(a)=-\big(\text{D}_yf(c)\big)^{-1}\big(\text{D}_xf(c)\big)$.
\end{theorem}
%%
\begin{remark}
\begin{itemize}
\item
Der Satz über implizite Funktionen trägt seinen Namen zurecht.
Denn er stellt nur die Existenz der Auflösung $g$ sicher.
Diese Funktion $g$ ist eine nur implizit definierte Funktion, denn eine explizite Abbildungsvorschrift der Form $g(x)=\ldots$ liefert der Satz nicht. 
Er gibt auch keine Auskunft darüber, wie groß die offenen Mengen $U$ und $V$ sind.
\item
Der Satz über implizite Funktionen  lässt sich sofort auf Ausgangsgleichungen $f(c)=r$ 
für $r\in \R^n$ beliebig verallgemeinern, indem man im Satz $f$ durch $\tilde{f}=f-r$ ersetzt.
Weil $\text{D}\tilde{f}(x)=\text{D}f(x)$ für alle $x\in M$ gilt, übertragen sich alle Aussagen direkt.
Die Voraussetzung, dass $\text{D}_yf(c)$ invertierbar ist, ist hingegen wesentlich.
\end{itemize}
\end{remark}
%Video
\floatright{\href{https://api.stream24.net/vod/getVideo.php?id=10962-2-10927&mode=iframe&speed=true}{\image[75]{00_video_button_schwarz-blau}}}\\
\\

\begin{example}
\begin{tabs*}[\initialtab{0}]
\tab{Beispiel a)}
Wir betrachten $f:\R^2\to\R$, $\begin{pmatrix} x\\ y\end{pmatrix}\mapsto x^2+xy+y^2-1$.
Als Polynom ist $f$ stetig differenzierbar mit $\text{D}_xf=2x+y$ und $\text{D}_yf=x+2y$.
Wir wollen den Satz über implizite Funktionen anwenden und wählen uns deshalb ein 
$c=\begin{pmatrix} a\\ b\end{pmatrix}$ mit $\text{D}_yf(c)=a+2b\neq 0$, zum Beispiel $c=\begin{pmatrix}0\\1\end{pmatrix}$. 
Für dieses $c$ ist auch $f(c)=0$, somit gibt es auf einer Umgebung $U$ von $a=0$  eine Auflösung, also eine Funktion
$g:U\to\R$, sodass
\[f(\begin{pmatrix} x\\ y\end{pmatrix})=0\quad\Leftrightarrow\quad y=g(x)\]
in einer Umgebung von $c$. Die Funktion $g$ hat die Ableitung
\[g'(x)=-\frac{\text{D}_xf(\begin{pmatrix} x\\ g(x)\end{pmatrix})}{\text{D}_yf(\begin{pmatrix} x\\ g(x)\end{pmatrix})}
=-\frac{2x+g(x)}{x+2g(x)},\]
also speziell $g'(0)=-\frac{1}{2}$.
\\
In diesem speziellen Beispiel kann man eine spezielle Auflösung auch direkt finden, in dem man die 
quadratische Gleichung 
\[f(\begin{pmatrix} x\\ y\end{pmatrix})=y^2+xy+(x^2-1)=0,\]
in $y$ löst, also $x\mapsto \frac{1}{2}\big(-x\pm\sqrt{x^2-4(x^2-1)}\big)$. 
Weil dies eine in der Nähe von $a=1$ differenzierbare Funktion definieren soll,
müssen wir uns für ein Vorzeichen entscheiden. Und weil $g(0)=1$ gelten muss, kommt nur $g(x)=\frac{1}{2}\big(-x+\sqrt{x^2-4(x^2-1)}\big)$
in Frage.
\\
Die Verwendung des Satzes spart hier trotz der explizit angebbaren Auflösung immer noch Rechenarbeit, 
zum Beispiel bei der Berechnung $g'(0)$. Zur Bestätigung können Sie jetzt gerne $g'(0)$ aus der
expliziten Formel für $g$ bestimmen.
\tab{Beispiel b)}
Wir betrachten die stetig differenzierbare Funktion
\[f:\R^3\to\R,\quad x\mapsto e^{x_1}\sin x_2+e^{x_2}\sin x_3+e^{x_3}\sin x_1\]
mit totaler Ableitung
\[\text{D}f(x)=\begin{pmatrix} e^{x_1}\sin x_2+e^{x_3}\cos x_1,&
e^{x_1}\cos x_2+e^{x_2}\sin x_3,& e^{x_2}\cos x_3+e^{x_3}\sin x_1\end{pmatrix}.\]
Für $c=0$ ist $f(c)=0$ und $\text{D}f(c)=\begin{pmatrix} 1&1&1\end{pmatrix}$.
Die partielle Ableitung $\text{D}_{x_3}f(c)=1$ nach $x_3$ ist invertierbar in $c=0$.
Also gibt es nach dem Satz über implizite Funktionen eine Umgebung von $c=0$ und dort eine Auflösung
$g$, also eine differenzierbare Funktion in $\begin{pmatrix}x_1\\x_2\end{pmatrix}$ so,
dass $f(x)=0$ genau dann erfüllt ist, wenn $x_3=g(x_1,x_2)$. Die totale Ableitung von $g$ ist
\[\text{D}g((x_1,x_2)^T)=
%-\text{D}_3f(\begin{pmatrix} x_1\\x_2\\g((x_1,x_2)^T)\end{pmatrix})^{-1}\cdot
%\begin{pmatrix}\text{D}_2f(\begin{pmatrix} x_1\\x_2\\g((x_1,x_2)^T)\end{pmatrix}),
%&\text{D}_2f(\begin{pmatrix} x_1\\x_2\\g((x_1,x_2)^T)\end{pmatrix})\end{pmatrix},\]
%korrigiert:
-\text{D}_{x_3}f(\begin{pmatrix} x_1\\x_2\\g((x_1,x_2)^T)\end{pmatrix})^{-1}\cdot
\begin{pmatrix}\text{D}_{x_1}f(\begin{pmatrix} x_1\\x_2\\g((x_1,x_2)^T)\end{pmatrix}),
&\text{D}_{x_2}f(\begin{pmatrix} x_1\\x_2\\g((x_1,x_2)^T)\end{pmatrix})\end{pmatrix},\]
also speziell $\text{D}g(0)=(-1)\cdot\begin{pmatrix}1&1\end{pmatrix}=\begin{pmatrix}-1&-1\end{pmatrix}$.

\tab{Beispiel c)}
%Video
Im folgenden Video finden Sie ein weiteres explizites Beispiel für den
Satz über implizite Funktion.

\center{\href{https://api.stream24.net/vod/getVideo.php?id=10962-2-10928&mode=iframe&speed=true}{\image[75]{00_video_button_schwarz-blau}}}\\
\\

\end{tabs*}
\end{example}

\end{content}

