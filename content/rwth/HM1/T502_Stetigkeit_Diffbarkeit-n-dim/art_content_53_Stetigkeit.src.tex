
\documentclass{mumie.article}
%$Id$
\begin{metainfo}
  \name{
    \lang{en}{...}
    \lang{de}{Stetigkeit von Funktionen mehrerer Veränderlicher}
   }
  \begin{description} 
 This work is licensed under the Creative Commons License Attribution 4.0 International (CC-BY 4.0)   
 https://creativecommons.org/licenses/by/4.0/legalcode 

    \lang{en}{...}
    \lang{de}{...}
  \end{description}
  \begin{components}
    \component{generic_image}{content/rwth/HM1/images/g_img_00_video_button_schwarz-blau.meta.xml}{00_video_button_schwarz-blau}
  \end{components}
  \begin{links}
\link{generic_article}{content/rwth/HM1/T502_Stetigkeit_Diffbarkeit-n-dim/g_art_content_53_Stetigkeit.meta.xml}{content_53_Stetigkeit}
\link{generic_article}{content/rwth/HM1/T501_Orientierung_im_n-dim_Raum/g_art_content_52_Abstaende.meta.xml}{content_52_Abstaende}
\link{generic_article}{content/rwth/HM1/T204_Abbildungen_und_Funktionen/g_art_content_10_abbildungen_verkettung.meta.xml}{content_10_abbildungen_verkettung}
\link{generic_article}{content/rwth/HM1/T210_Stetigkeit/g_art_content_30_elem_funktionen.meta.xml}{content_30_elem_funktionen}
\link{generic_article}{content/rwth/HM1/T210_Stetigkeit/g_art_content_31_grenzwerte_von_funktionen.meta.xml}{content_31_grenzwerte_von_funktionen}
\link{generic_article}{content/rwth/HM1/T210_Stetigkeit/g_art_content_29_stetigkeit_definitionen.meta.xml}{content_29_stetigkeit_definitionen}
\end{links}
  \creategeneric
\end{metainfo}


\begin{content}
\begin{block}[annotation]
	Im Ticket-System: \href{https://team.mumie.net/issues/21464}{Ticket 21464}
\end{block}
\usepackage{mumie.ombplus}
\ombchapter{2}
\ombarticle{1}
\usepackage{mumie.genericvisualization}
\lang{de}{\title{Stetigkeit}}
\begin{block}[info-box]
\tableofcontents
\end{block}


\section{Stetigkeit}
Mit dem durch eine Norm auf dem $\R^n$ gewonnenen Abstandsbegriff können wir nun die Idee der Stetigkeit von reellen Funktionen
übertragen auf den $\R^n$.
Viele Aussagen und Eigenschaften aus dem eindimensionalen Fall übertragen sich in den mehrdimensionalen Fall und helfen, 
die mehrdimensionalen  Aussagen zu begründen.
\begin{definition}[Stetigkeit]\label{def:stetigkeit_n-dim}
Sei $M\subset\R^n$ eine nicht leere Teilmenge. Eine Funktion $f:M\to\R^m$ heißt \notion{stetig im Punkt $a\in M$},
wenn es zu jedem $\epsilon>0$ ein $\delta>0$ gibt, so dass für alle $x\in M$ mit $\Vert x-a\Vert<\delta$
gilt \[\Vert f(x)-f(a)\Vert<\epsilon.\]
Anders gesagt: $f$ ist stetig in $a$, wenn es zu jedem $\epsilon>0$ ein $\delta>0$ gibt, so dass 
\[f( U_\delta(a)\cap M)\subset U_\epsilon(f(a)).\]

Die Funktion $f$ heißt \notion{stetig} (auf ganz $M$), wenn $f$ in jedem Punkt $a\in M$ stetig ist. 
\end{definition}
In dieser Definition spielt es keine Rolle, welche Normen benutzt werden. 
Zwar haben die offenen Kugeln $U_\delta(a)$ bzw.
$U_\epsilon(f(a))$ in verschiedenen Normen unterschiedliche Gestalt. Aber weil alle  Normen äquivalent sind, gilt die Aussage für \emph{alle} $\epsilon >0$
bezüglich der einen Norm genau dann, wenn sie für \emph{alle } $\epsilon>0$ in einer anderen gilt. Und finden wir eine
$\delta$-Umgebung bezüglich der einen Norm, dann auch bezüglich einer anderen (mit möglicherweise kleinerem $\delta$).
\begin{example}[Koordinatenprojektionen]
Die Abbildung 
\[\text{pr}_j:\R^n\to \R,\quad x=\begin{pmatrix}x_1\\\vdots\\ x_n\end{pmatrix}\:\mapsto x_j,\]
die einen Vektor $x$ auf seine $j$-te Koordinate $x_j$ projiziert, ist stetig für $j=1,\ldots,n$.
\begin{incremental}{0}
\step
Denn seien $a\in\R^n$ und $\epsilon>0$ beliebig.  Wir zeigen, dass die Funktion $\text{pr}_j$ stetig ist, weil
die obige Definition speziell für die Wahl $\delta=\epsilon$ erfüllt ist.
Für alle $x\in\R^n$ mit $\Vert x-a\Vert_2<\delta=\epsilon$
ist dann nämlich
\[
{\vert \text{pr}_j(x)-\text{pr}_j(a)\vert}={\vert x_j-a_j\vert} =\sqrt{(x_j-a_j)^2}\leq
\sqrt{\sum_{i=1}^n (x_i-a_i)^2}={\Vert x-a\Vert}<\epsilon.
\]
\end{incremental}
\end{example}
%%
\begin{theorem}\label{thm:äquivalent-zu-stetig}
Es sei $M\subset\R^n$ und $f:M\to\R^m$ eine Funktion.
\begin{enumerate}
\item[(a)]\notion{Komponentenweise Stetigkeit}\\
Bezeichnet man
\[f(x)=\begin{pmatrix}f_1(x)\\\vdots\\ f_m(x)\end{pmatrix},\]
wo $f_i=\text{pr}_i\circ f:M\to\R$ die $i$-te Koordinatenfunktion ist,
so ist $f$ stetig in $a\in M$ genau dann, wenn jede Koordinatenfunktion $f_i$, $i=1\ldots,m$, stetig ist in $a$.
\item[(b)]\notion{Folgenstetigkeit}\\
Die Funktion $f$ ist stetig in $a\in M$ genau dann, wenn für jede Folge $x^{(k)}$ in $M$ mit $\lim_{k\to\infty}x^{(k)}=a$ gilt
\[\lim_{k\to\infty} f(x^{(k)})=f(\lim_{k\to\infty} x^{(k)})=f(a).\]
\end{enumerate}
\end{theorem}
%%
\begin{proof*}
\begin{incremental}{0}
\step
Zu (a): Nehmen wir zunächst an, dass alle Komponentenfunktionen $f_j$ stetig sind in $a$. 
Dann gibt es zu jedem $\epsilon>0$ Zahlen $\delta_j>0$ so, dass für $x\in M$ aus $\Vert x-a\Vert_2<\delta_j$
folgt $\vert f_j(x)-f_j(a)\vert<\frac{\epsilon}{\sqrt{m}}$. Wir setzen $\delta=\min_{j=1,\ldots,m}\delta_j$
und erhalten für alle $x\in M$ mit $\Vert x-a\Vert_2<\delta$
\[{\Vert f(x)-f(a)\Vert_2}=\sqrt{\sum_{j=1}^m {\vert f_j(x)-f_j(a)\vert^2}}<\sqrt{m\cdot\frac{\epsilon^2}{\sqrt{m}^2}}=\epsilon.\]
Also ist $f$ stetig in $a$.
\step
Nehmen wir umgekehrt an, dass $f$ stetig in $a$ ist. Weil für $j=1,\ldots,m$
\[{\vert f_j(x)-f_j(a)\vert}\leq\sqrt{\sum_{j=1}^m {\vert f_j(x)-f_j(a)\vert^2}}={\Vert f(x)-f(a)\Vert_2}, \]
folgt das $\epsilon$-$\delta$-Kriterium für jede Komponentenfunktion $f_j$ sofort aus dem für $f$ selbst.
\step
Zu (b):
Die Äquivalenz des $\epsilon$-$\delta$-Kriteriums mit der Folgenstetigkeit beweist man wie 
im \ref[content_29_stetigkeit_definitionen][analogen Resultat]{thm:stetigkeit_folgenkrit_aequiv_epsilon-delta} in einer Dimension.
\end{incremental}
\end{proof*}

%Video
\floatright{\href{https://api.stream24.net/vod/getVideo.php?id=10962-2-10914&mode=iframe&speed=true}{\image[75]{00_video_button_schwarz-blau}}}\\
\\
%%
Wir wenden Satz \ref{thm:äquivalent-zu-stetig} an, um Aussagen über die Stetigkeit zusammengesetzter Funktionen zu machen.
\begin{theorem}\label{thm:zusammensetzung_stetiger_funktionen}
Es sei $M\subset \R^n$ eine Teilmenge, und es seien
$f,g:M\to\R^m$ sowie $\phi:M\to\R$ Funktionen. Dann gilt:
\begin{itemize}
\item[(i)] Sind $f$ und $g$ stetig in $a\in M$, dann ist auch die Linearkombination $\alpha f+\beta g:M\to\R^m$ stetig in $a$ für alle Konstanten $\alpha,\beta\in\R$.
\item[(ii)]
Sind $f$ und $\phi$ stetig in $a$, dann ist auch das Produkt $\phi\cdot f$ stetig in $a$.
\item[(iii)] Ist $\phi(a)\neq 0$, und sind $f$ und $\phi$ stetig in $a$, dann ist auch der Quotient $\frac{1}{\phi}\cdot f$ stetig in $a$.
\item[(iv)] Ist weiter $h:M'\to\R^p$ eine Funktion mit $f(M)\subset M'$, und ist $f$ stetig in $a$ sowie $h$ stetig in $f(a)$, 
dann ist auch die Komposition $h\circ f:M\to\R^p$ stetig in $a$. 
\end{itemize}
\end{theorem}
\begin{proof*}
Alle Aussagen  sind direkte Folgerungen aus der Folgenstetigkeit und den Grenzwertsätzen, siehe auch die entsprechenden $1$-dimensionalen Aussagen
\ref[content_31_grenzwerte_von_funktionen][für Summen und Produkte]{rule:summen_und_prod_stetiger_fkt} bzw. 
\ref[content_30_elem_funktionen][für die Komposition]{thm:komposition_stetiger_fkt}.
\end{proof*}
%%
\begin{example}\label{ex:wichtige_stetige_Funktionen}
\begin{tabs*}[\initialtab{0}]
\tab{Polynomfunktionen}
Eine Polynomfunktion [kurz auch: ein \notion{Polynom}] $p:\R^n\to\R$ ist eine endliche Linearkombination von \notion{Monomen}, also von Funktionen der Form $\R^n\to\R$, 
\[x=(x_1,\ldots,x_n)^T\mapsto x_1^{\nu_1}\cdot\ldots\cdot x_n^{\nu_n}=\text{pr}_1(x)^{\nu_1}\cdot\ldots\cdot\text{pr}_n(x)^{\nu_n}\]
mit Exponenten $\nu_1,\ldots,\nu_n\in\N_0$. 
Ein Polynom $p$ vom \notion{Grad} $d$ lässt sich also schreiben in der Form
\[p(x)=\sum_{(\nu_1,\ldots,\nu_n)\in\N_0^n,\\\nu_1+\cdots\nu_n\leq d}a_{\nu_1,\ldots,\nu_n}x_1^{\nu_1}\cdots x_n^{\nu_n},\]
mit Koeffizienten $a_{\nu_1,\ldots,\nu_n}\in\R$, wobei
mindestens einer der Koeffizienten $a_{\nu_1,\ldots,\nu_n}$ mit $\nu_1+\cdots+\nu_n=d$ von Null verschieden ist.
% Die Existenz einer Zahl $d$ folgt dabei aus der Endlichkeit der Linearkombination, das heißt es gibt nur endlich viele Summanden.
% Die kleinstmögliche Zahl $d$ heißt der \notion{Grad} des Polynoms $p$. Das heißt für die obige Summe, es gibt ein Tupel $(\mu_1,\ldots,\mu_n)\in\N_0^n$ mit $\mu_1+\cdots+\mu_n=d$ und 
% $a_{\mu_1,\ldots,\mu_n}\neq 0$, aber alle $a_{\nu_1,\ldots,\nu_n}=0$ für $\nu_1+\cdots+\nu_n>d$.
\\
Polynome sind stetige Funktionen. Weil die Koordinatenprojektionen $\text{pr}_j$, $j=1,\ldots,n$, stetig sind, ist das eine Folgerung aus
Satz \ref{thm:zusammensetzung_stetiger_funktionen} (i) und (ii).
\tab{Rationale Funktionen}
Eine Rationale Funktion $f:M\to\R$ ist eine Funktion, die gegeben ist durch $x\mapsto f(x)=\frac{p(x)}{q(x)}$, 
wobei $p$ und $q$ Polynomfunktionen sind. Dabei ist die größtmögliche Definitionsmenge $M=\{x\in\R^n\mid q(x)\neq 0\}$ 
diejenige, die genau die Nullstellen der Nennerfunktion $q$ auslässt.
\\
Rationale Funktionen sind stetig. Das folgt aus Satz \ref{thm:zusammensetzung_stetiger_funktionen}, weil Polynomfunktionen stetig sind.
\tab{Lineare Abbildungen}
Es sei $A=(a_{ij})_{ij}\in M(m,n;\R)$ eine Matrix. Die zugehörige lineare Abbildung $\phi:\R^n\to\R^m$, $x\mapsto \phi(x)=A\cdot x$, ist stetig.
Denn jede der $i$-ten Koordinatenfunktionen 
\[\phi_i(x)=(A\cdot x)_i=a_{i1}x_1+\ldots+a_{in}x_n\]
ist eine Polynomfunktion, also stetig.
Weil $\phi$ genau dann stetig ist, wenn \ref[content_53_Stetigkeit][seine Koordinatenfunktionen das sind]{thm:äquivalent-zu-stetig}, ist $\phi$ stetig.
%Um das zu sehen, 
% setzt man
% \[{\Vert A\Vert}:=\sqrt{\sum_{i=1,\ldots,m,\\j=1,\ldots,n}a_{ij}^2},\]
% die sogenannte Operatornorm von $A$.
% Es gilt dann
%\[\Vert A\cdot x\Vert_2\leq \Vert A\Vert\cdot \Vert x\Vert_2.\]
%(Das ist eine Variante der sogenannten Cauchy-Schwarz-Ungleichung, die in der Linearen Algebra bewiesen wird. Wir nehmen sie hier als gegeben an.)
%Für alle $x,y\in\R^n$ gilt nun unter Verwendung der Linearität
%\[\Vert A\cdot x-A\cdot y\Vert_2= \Vert A\cdot (x-y)\Vert_2\leq \Vert A\Vert\cdot \Vert x-y\Vert_2.\]
%Es sei nun $\epsilon>0$, und dazu $\delta:=\frac{\epsilon}{\Vert A\Vert+1}$.
%Dann folgt aus $\Vert  x- y\Vert_2<\delta$, dass 
%$\Vert A\cdot x-A\cdot y\Vert_2<\epsilon$.
%(Wir haben in der Wahl von $\delta$ durch $\Vert A\Vert+1$ geteilt, 
%um den Fall $\Vert A\Vert=0$, also $A=0$, mitbehandeln zu können. Behandelten wir diesen Fall separat, so könnten wir einfach durch $\vert\!\vert A\vert\!\vert$ teilen.)
\end{tabs*}
\end{example}
%Video
\floatright{\href{https://api.stream24.net/vod/getVideo.php?id=10962-2-10916&mode=iframe&speed=true}{\image[75]{00_video_button_schwarz-blau}}}\\
\\
%
%
\begin{quickcheck}
\text{Welche der angegebenen Funktionen sind stetig auf ganz $\R^2$?}
\begin{choices}{multiple}
    \begin{choice}
      \text{$f_1:x\mapsto \begin{pmatrix}{x_1}{x_2^2+1}\\x_1-\sin x_2\end{pmatrix}\in\R^2$}
      \solution{true}
    \end{choice}
    \begin{choice}
      \text{$f_2:x \mapsto \begin{pmatrix}\cos x_2,&5\vert\!\vert x\vert\!\vert_2\end{pmatrix}\cdot x \in\R$}
      \solution{true}
    \end{choice}
    \begin{choice}
      \text{$f_3:x\mapsto \frac{1}{1-\sin^2(x_1x_2)}\in\R$}
      \solution{false}
      \end{choice} 
     \begin{choice}
     \text{$f_4:x\mapsto \frac{2x_1x_2}{x_1^2+x_2^2}\in\R$ falls $x\neq (0,0)$ und $(0,0)\mapsto 0$}
      \solution{false}
     \end{choice}
    \begin{choice}
     \text{$f_5:x\mapsto \frac{2x_1^2x_2}{x_1^2+x_2^2}\in\R$ falls $x\neq (0,0)$ und $(0,0)\mapsto 0$}
      \solution{true}
     \end{choice}
  \end{choices}
  \explanation{In den ersten beiden Funktionen $f_1$, $f_2$ sind  alle Komponenten stetige Funktionen auf $\R^2$, weil
  sie sich als Summe, Produkt bzw. Verkettung von Funktionen in einer Variable darstellen lassen, die wir
  bereits als stetige Funktionen kennen.
\\
Bei der dritten Funktion $f_3$ greift dasselbe Argument, allerdings nur dort, wo die Funktion auch definiert ist.
Der Definitionsbereich ist hier $\{(x_1,x_2)^T\in\R^2|\: x_1x_2\neq \pi k\text{ für alle } k\in\Z\}$.
\\
Ebenso sind die letzten beiden Funktionen stetig in $x\neq (0,0)$. 
Die Funktion $f_4$ ist nicht stetig in $x=(0,0)$, denn es gilt $\lim_{t\to 0}f(t,t)=\lim_{t\to 0}1=1$, 
während $\lim_{t\to 0}f(0,t)=\lim_{t\to 0}0=0$.
Dagegen ist $f_5$ auch stetig in Null, denn es gilt $f_5(x)=x_1f_4(x)$, und $f_4$ ist beschränkt.

}
\end{quickcheck}
%
Die folgende Charakterisierung von Stetigkeit ist oft sehr praktisch. (Nebenbei: Diese Charakterisierung bietet die Möglichkeit, 
den Stetigkeitsbegriff auf sehr viel allgemeinere Räume als den $\R^n$ zu übertragen.)
\begin{theorem}[Charakterisierung von Stetigkeit]\label{thm:stetig_top_charakterisierung}
Es sei $M\subset\R^n$ eine nichtleere offene Teilmenge. Eine Funktion $f:M\to\R^m$ ist genau dann stetig, 
wenn für jede offene Menge $U\subset\R^m$ das Urbild $f^{-1}(U)\subset\R^n$ offen ist.
\end{theorem}
%%
\begin{proof*}
\begin{incremental}{0}
\step
Nehmen wir zunächst an, dass $f$ stetig ist. Sei $U\subset\R^m$ eine offene Teilmenge. 
Wir müssen zeigen, dass $f^{-1}(U)$ offen ist. 
\\
Dazu sei $a\in M$ beliebig mit $f(a)\in U$. 
Weil $U$ offen ist, existiert ein $\epsilon>0$ so, dass $U_\epsilon(f(a))$ ganz in $U$ enthalten ist, $U_\epsilon(f(a))\subset U$. 
Weil $f$ stetig ist, existiert zu diesem $\epsilon$ ein $\delta>0$ so, dass $f(U_\delta(a)\cap M)\subset U_\epsilon(f(a))$. 
Diese Menge $U_\delta(a)\cap M$ ist eine offene Umgebung von $a$, die ganz in $f^{-1}(U)$ enthalten ist. 
Weil $a\in f^{-1}(U)$ beliebig war, ist $f^{-1}(U)$ offen.
\step
Nehmen wir andererseits an, dass alle Urbilder  von offenen Mengen unter $f$ wieder offene Mengen sind. 
\\
Sei $a\in M$ beliebig, und ebenfalls $\epsilon>0$ beliebig.
Weil $U_\epsilon(f(a))\subset \R^m$ eine offene Menge ist, ist $f^{-1}(U_\epsilon(f(a)))$ ebenfalls offen. 
Insbesondere gibt es also zu dem Punkt $a\in f^{-1}(U_\epsilon(f(a)))$ eine kleine Kugel $U_\delta(a)$, die ganz in dieser Urbildmenge
enthalten ist. Es gilt also $f(U_\delta(a))\subset U_\epsilon(f(a))$. 
Somit ist $f$ stetig in $a$, und weil dies für jedes $a\in M$ gilt, ist $f$ stetig auf $M$.
\end{incremental}
\end{proof*}
%%
\notion{Beachte:} In Satz \ref{thm:stetig_top_charakterisierung} ist es unerheblich, 
ob die offene Teilmenge $U\subset\R^m$ des Zielbereichs
ganz, teilweise oder gar nicht im Bild $f(M)$ der Funktion enthalten ist. 
Ist zum Beispiel $f(M)\cap U=\emptyset$, dann ist $f^{-1}(U)=\emptyset$ per Definition des 
\ref[content_10_abbildungen_verkettung][Urbilds]{def:urbild}, und die leere Menge ist offen. 



Nun wenden wir Satz \ref{thm:stetig_top_charakterisierung} an:
\begin{remark}
Ist $f:\R^n\to\R$ stetig, dann sind die folgenden Teilmengen des $\R^n$ offen
\begin{align*}
M_1&=\{x\in\R^n\mid f(x)>0\},\\
M_2&=\{x\in\R^n\mid f(x)<0\},\\
M_3&=\{x\in\R^n\mid f(x)\neq 0\}.
\end{align*}
Denn es ist $(0;\infty)$ eine offene Teilmenge von $\R$. Also ist nach Satz \ref{thm:stetig_top_charakterisierung} das Urbild $M_1=f^{-1}((0;\infty))$ offen.
Ebenso ist $M_2=f^{-1}((-\infty;0))$ offen, sowie $M_3=M_1\cup M_2$ als Vereinigung offener Mengen.
\end{remark}
\begin{example}
Die Menge $M=\{x\in\R^3\mid x_1+x_2+x_3<1\}$ ist offen.
Denn für die stetige Funktion $f:\R^3\to\R, x\mapsto x_1+x_2+x_3-1$ gilt $M=\{x\in\R^3\mid f(x)<0\}=f^{-1}((-\infty;0))$ 
ist das Urbild einer offenen Menge.
\end{example}
%Video
\floatright{\href{https://api.stream24.net/vod/getVideo.php?id=10962-2-10919&mode=iframe&speed=true}{\image[75]{00_video_button_schwarz-blau}}}\\
\\	
%%
\begin{quickcheck}
\text{Markiere die offenen Mengen:}
\begin{choices}{multiple}
    \begin{choice}
      \text{$M_1=\{x\in\R^3\mid x_1^2+x_2^2> 3\}$}
      \solution{true}
    \end{choice}
    \begin{choice}
      \text{$M_2=\{x\in\R^2\mid x_1^2+x_2^2\geq 3\}$}
      \solution{false}
    \end{choice}
    \begin{choice}
      \text{$M_3=\{x\in\R^2\mid x_1^2+x_2^2\geq -1\}$}
      \solution{true}
    \end{choice}   
    \begin{choice}
      \text{$M_4=U_2(0)\cup \{x\in\R^3\mid x_1^2+x_3^2> 1\} \subset \R^3$}
      \solution{true}
    \end{choice}
  \end{choices}
  \explanation{$M_1$ ist offen, weil Urbild des offenen Intervalls $(3;\infty)$ unter der Abbildung $f:x\mapsto x_1+x_2+x_3-1$.\\
    $M_2$ ist abgeschlossen, denn  die Randpunkte $\{x\in\R^2\mid x_1^2+x_2^2= 3\}$ gehören zu $M_2$.\\
    $M_3$ ist offen, denn $x_1^2+x_2^2\geq 0$ ist immer erfüllt, also ist $M_3=\R^2$, und damit natürlich offen.
    (Man kann auch so argumentieren: $M_3$ ist das Urbild der offenen Menge $\R$ unter der Abbildung $f:x\mapsto x_1^2+x_2^2$.)\\
    $M_4$ ist als Vereinigung offener Mengen offen.}
\end{quickcheck}
%%
\end{content}

