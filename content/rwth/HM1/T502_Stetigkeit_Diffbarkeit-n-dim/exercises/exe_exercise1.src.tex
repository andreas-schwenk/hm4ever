\documentclass{mumie.element.exercise}
%$Id$
\begin{metainfo}
  \name{
    \lang{en}{Exercise 1}
    \lang{de}{Ü01: Stetigkeit}
  }
\begin{description} 
 This work is licensed under the Creative Commons License Attribution 4.0 International (CC-BY 4.0)   
 https://creativecommons.org/licenses/by/4.0/legalcode 

    \lang{en}{...}
    \lang{de}{...}
    \lang{zh}{...}
    \lang{fr}{...}
  \end{description}
  
  \begin{components}
  \end{components}
  \begin{links}
\link{generic_article}{content/rwth/HM1/T109_Skalar-_und_Vektorprodukt/g_art_content_32_laenge_norm.meta.xml}{content_32_laenge_norm}
\link{generic_article}{content/rwth/HM1/T202_Reelle_Zahlen_axiomatisch/g_art_content_05_anordnungsaxiome.meta.xml}{content_05_anordnungsaxiome}
\link{generic_article}{content/rwth/HM1/T301_Differenzierbarkeit/g_art_content_03_hoehere_ableitungen.meta.xml}{content_03_hoehere_ableitungen}
\link{generic_article}{content/rwth/HM1/T502_Stetigkeit_Diffbarkeit-n-dim/g_art_content_53_Stetigkeit.meta.xml}{content_53_Stetigkeit}
\end{links}
  \creategeneric
\end{metainfo}

\begin{content}
\begin{block}[annotation]
	Im Ticket-System: \href{https://team.mumie.net/issues/22634}{Ticket 22634}
\end{block}
  \title{
    \lang{en}{Exercise 1}
    \lang{de}{Ü01: Stetigkeit}
  }
  
Gegeben sei die Funktion
\[
f:\R^2 \to \R^2, x \mapsto \begin{pmatrix}\sin(x_1)\sin(x_2)\\ x_2 \end{pmatrix}
\]
Zeigen Sie nur mit Hilfe der Definition, dass $f$ stetig ist.
  
  \begin{tabs*}[\initialtab{0}\class{exercise}]
    \tab{
      Lösung
    }
    \begin{incremental}[\initialsteps{1}]
      \step Wir zeigen, dass $f$ an jeder Stelle stetig ist und zwar nur mit Hilfe dieser \ref[content_53_Stetigkeit][Definition]{def:stetigkeit_n-dim}. Da die Funktion $\sin$ auf $\R$ stetig differenzierbar ist,
      erhalten wir aus dem \ref[content_03_hoehere_ableitungen][Mittelwertsatz]{thm:mittelwertsatz}
      für beliebige $x_{1},x_{2}$ die Abschätzung
\[|\sin(x_{1})-\sin(x_{2})|\leq \sup_{\zeta\in\R}|\cos(\zeta)|\cdot|x_{1}-x_{2}|\leq |x_{1}-x_{2}|\,.\]

\step Daraus folgt mit Hilfe der \ref[content_05_anordnungsaxiome][Dreiecksungleichung]{rule:anordnungsregeln}
für $x_{1},x_{2},y_{1},y_{2}\in\R$
\begin{align*}
&|\sin(x_{1})\sin(x_{2})-\sin(y_{1})\sin(y_{2})|&\\
&=|\sin(x_1)\sin(x_2) - \sin(y_1)\sin(x_2) + \sin(y_1)\sin(x_2) - \sin(y_1)\sin(y_2)|& \\
&\leq |\sin(x_{1})-\sin(y_{1})||\sin(x_{2})|+|\sin(x_2)-\sin(y_{2})||\sin(y_{1})|&\\
&\leq |x_{1}-y_{1}|+|x_{2}-y_{2}|&\,.   
\end{align*}

\step
Die Definition der \ref[content_32_laenge_norm][euklidischen Norm]{def:euklidische_norm} liefert nun die Abschätzung
\begin{align*}
 &\Vert f(x)-f(y)\Vert^{2}&& = \Vert \begin{pmatrix} \sin(x_1)\sin(x_2) - \sin(y_1)\sin(y_2)\\ x_2 - y_2\end{pmatrix}\Vert^2 &\\
%QS falsch (?):
%QS &&& = |x_{1}-y_{1}|+|x_{2}-y_{2}|^2 + |x_2-y_2|^2 & \\
%QS &&& \leq (|x_{1}-x_{2}|+|y_{1}-y_{2}|)^{2}+|y_{1}-y_{2}|^{2}&\\
%QS
%QS neu:
 &&&    = (\sin(x_1)\sin(x_2) - \sin(y_1)\sin(y_2))^{2} + (x_2-y_2)^2 & \\
 &&&    = | \sin(x_1)\sin(x_2) - \sin(y_1)\sin(y_2)|^{2} + |x_2-y_2|^2 & \\ 
 &&& \leq ( |x_{1}-y_{1}|+|x_{2}-y_{2}|)^{2} + |x_2-y_2|^2 & \\
 &&& \leq 2(|x_{1}-y_{1}|)^{2}+2(|x_{2}-y_{2}|)^{2} + |x_2-y_2|^2 & \\
%QS 

 &&& \leq 3(|x_{1}-y_1|^{2}+|x_{2}-y_{2}|^{2})& \\
 &&&   =  3 \Vert \begin{pmatrix} x_1 - y_1 \\ x_2 - y_2 \end{pmatrix}\Vert^2 &\\
 &&&   =  3 \Vert x-y \Vert^{2}.&  
\end{align*}
Hierbei haben wir die Ungleichung $(a+b)^{2}\leq 2a^{2}+2b^{2}$ verwendet, die für alle reellen Zahlen $a,b\in\R$ gültig ist, wie man mit Hilfe der binomischen Formeln nachprüft.
\step Nun sei $a\in\R^{2}$ beliebig und $\varepsilon >0$. Dann wählen wir $\delta:=\frac{\varepsilon}{\sqrt{3}}$. Es gilt nun für alle $x\in\R^{2}$  mit $\Vert x-a\Vert <\delta$
\[\Vert f(x)-f(a)\Vert \leq \sqrt{3}\Vert x-a \Vert <\sqrt{3}\delta=\varepsilon\,.\]
Daher ist $f$ stetig in $a$. 
       
    \end{incremental}
  \end{tabs*}
\end{content}

