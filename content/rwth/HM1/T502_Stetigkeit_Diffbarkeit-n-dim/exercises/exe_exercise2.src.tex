\documentclass{mumie.element.exercise}
%$Id$
\begin{metainfo}
  \name{
    \lang{en}{Exercise 2}
    \lang{de}{Ü02: Ableitung}
  }
  \begin{description} 
 This work is licensed under the Creative Commons License Attribution 4.0 International (CC-BY 4.0)   
 https://creativecommons.org/licenses/by/4.0/legalcode 

    \lang{en}{...}
    \lang{de}{...}
  \end{description}
  \begin{components}
  \end{components}
  \begin{links}
\link{generic_article}{content/rwth/HM1/T502_Stetigkeit_Diffbarkeit-n-dim/g_art_content_54_Differentiation.meta.xml}{content_54_Differentiation}
\link{generic_article}{content/rwth/HM1/T502_Stetigkeit_Diffbarkeit-n-dim/g_art_content_53_Stetigkeit.meta.xml}{content_53_Stetigkeit}
\end{links}
  \creategeneric
\end{metainfo}

\begin{content}
\begin{block}[annotation]
	Im Ticket-System: \href{https://team.mumie.net/issues/22635}{Ticket 22635}
\end{block}
  \title{
    \lang{en}{Exercise 2}
    \lang{de}{Ü02: Ableitung}
  }
  
  Es sei $f:\R^2 \to \R, x \mapsto \begin{cases}\displaystyle{\frac{(x_1x_2)^2}{x_1^4+x_2^2}}, & x \neq (0,0)^T \\ 0, & x=(0,0)^T.\end{cases}$.
 
 a) Zeigen Sie, $f$ ist stetig.
  
 b) Berechnen Sie an der Stelle $a=(0,0)^T$ alle Richtungsableitungen.

 c) Ist $f$ total differenzierbar?
  \begin{tabs*}[\initialtab{0}\class{exercise}]
    \tab{
      Lösung a)
    }
    \begin{incremental}[\initialsteps{1}]
      \step 
    In allen Punkten $a\neq (0,0)^T$ ist $f$ als rationale Funktion stetig.
    
    \step Wir müssen aber noch $a=(0,0)^T$ betrachten.   \\
    
        Für $x=(x_1,x_2)^T\in \R^2$ mit $x_2\neq 0$ gilt
        \begin{align*}
        \displaystyle 
%QS        | f(x)| =
        \abs{f(x)} = \frac{(x_1x_2)^2}{x_1^4+x_2^2} \leq \frac{(x_1x_2)^2}{x_2^2} = x_1^2
        \end{align*}
        und für alle $x=(x_1,0)^T \in \R^2$ ist %QS bereits 
        $f(x)=0$.       
    
   \step Für jede Nullfolge $x^{(k)}=\left(x_1^{(k)},x_2^{(k)}\right)^T \to (0,0)^T$ in $\R^2$ folgt damit nun 
%QS     wegen $x_1^{(k)} \to 0$ 
%QS     \[ \lim_{k\to \infty} |f(x^{(k)})| \leq \lim_{k\to \infty} x_1^{(k)} = 0, \]
        \[ \lim_{k\to \infty} \abs{f(x^{(k)})} \leq \lim_{k\to \infty} \left( x_1^{(k)}\right)^2 = 0, \]
        was gleichbedeutend ist mit $\; \lim_{k\to \infty} f(x^{(k)})=0. \;$
%QS 
        Nach dem \ref[content_53_Stetigkeit][Folgenkriterium]{thm:äquivalent-zu-stetig} 
        ist damit die Stetigkeit von $f$ gezeigt.
         
    \end{incremental}
    
    \tab{Lösung b)}
    Sei $a=0$ und $v\neq 0$ eine beliebige Richtung.
    Wir nutzen die Definition der \ref[content_54_Differentiation][Richtungsableitung]{def:richtungsableitung} 
    und erhalten
    \begin{align*}
        D_v f(0) &= \lim_{h \to 0} \frac{f(0+ hv)}{h} \\
        &=  \lim_{h \to 0} \frac{(hv_1)^2(hv_2)^2}{((hv_1)^4+(hv_2)^2)\cdot h} \\
        &=  \lim_{h \to 0} \frac{h^4 (v_1v_2)^2}{h^5(v_1)^4+h^3(v_2)^2} \\
        &=  \lim_{h \to 0} \frac{h^4 (v_1v_2)^2}{h^3 (h^2 (v_1)^4+(v_2)^2)} \\
        &=  \lim_{h \to 0} \frac{h (v_1v_2)^2}{h^2 (v_1)^4+(v_2)^2}.
    \end{align*}
    Zur Bestimmung des Grenzwertes brauchen wir eine Fallunterscheidung. Ist $v_2=0$, dann ist der Zähler, 
    und damit der ganze Bruch, konstant $0$ und es gilt $\; D_v f(0)=0.$\\
    
    Im Fall von $v_2\neq 0$ können wir die Grenzwertregeln anwenden und erhalten 
    \[D_v f(0)=\frac{\lim_{h \to 0} \; \left(h (v_1v_2)^2\right)}{\lim_{h \to 0} \left(h^2 (v_1)^4+(v_2)^2\right)}= \frac{0}{(v_2)^2}=0.\]

    Zusammengefasst ist also $D_v f(0)=0$ für alle $v\neq 0$.
    
    \tab{Lösung c)}
     \begin{incremental}[\initialsteps{1}]
      \step 
    Als rationale Funktion ist $f$ bereits in allen Punkten $a\neq 0$ stetig partiell differenzierbar und damit auch total differenzierbar.
    Wir müssen noch den Punkt $a=0$ prüfen.
    
\step %QS Wir zeigen, dass ein $L\in M(1,2;\R)$ derart existiert, dass
    Zum Nachweis der \ref[content_54_Differentiation][totalen Differenzierbarkeit]{def:total-diffbar}  
    in $a=(0,0)^T$ suchen wir eine Matrix $L\in M(1,2;\R)$, so dass gilt
    \begin{equation}
    \lim_{x\to a}\frac{f(x)-f(a)-L\cdot(x-a)}{\vert\!\vert x-a\vert\!\vert}=0.
    \end{equation}

    \step Wir zeigen nun, dass wir $L=(0, \, 0)$ wählen können:
    \begin{align*}
    \displaystyle   
        \frac{f(x)-f(0)-L\cdot x}{\Vert x \Vert} = \frac{f(x)}{\Vert x \Vert} \;
        &= \frac{f(x)}{\sqrt{x_1^2+x_2^2}} \\
        &= \frac{x_1^2 x_2^2}{(x_1^4+x_2^2)\sqrt{x_1^2+x_2^2}} \\
%QS        &\leq \begin{cases} \frac{x_1^2 x_2^2}{x_2^2 \cdot |x_1|}, & x_1x_2\neq 0 \\ 0, & x_1x_2=0\end{cases}\\
%QS        &\leq \Vert x \Vert.  
        &= \frac{x_1^2 x_2^2}{(x_1^4+x_2^2)(x_1^2+x_2^2)} \sqrt{x_1^2+x_2^2} \\
        &\leq \begin{cases} 
                \sqrt{x_1^2+x_2^2}  , & \text{falls } x_1x_2\neq 0 \\ 
                0, & \text{falls } x_1x_2=0
           \end{cases}\\              
        &\leq \Vert x \Vert.              
    \end{align*}

\step Da $x \mapsto \Vert x\Vert$ stetig ist, folgt
\begin{equation}
\lim_{x\to a}\frac{f(x)-f(a)-L\cdot(x-a)}{\vert\!\vert x-a\vert\!\vert}=0.
\end{equation}

\end{incremental}
  \end{tabs*}
\end{content}

