\documentclass{mumie.element.exercise}
%$Id$
\begin{metainfo}
  \name{
    \lang{de}{Ü08: lokale Extrema}
  }
  \begin{description} 
 This work is licensed under the Creative Commons License Attribution 4.0 International (CC-BY 4.0)   
 https://creativecommons.org/licenses/by/4.0/legalcode 

    \lang{de}{Eigenschaften von Geraden}
  \end{description}
  \begin{components}
    \component{generic_image}{content/rwth/HM1/images/g_img_T102_exercise2.meta.xml}{laufwege}
  \end{components}
  \begin{links}
  \end{links}
  \creategeneric
\end{metainfo}
%
\begin{content}
\begin{block}[annotation]
	Im Ticket-System: \href{https://team.mumie.net/issues/28517}{Ticket 28517}
\end{block}


%
 \begin{block}[annotation]
%
      Lokale Extremstellen - Textaufgaben mit Video
%
\end{block}

  \title{
    \lang{de}{Ü08: lokale Extrema}
  }
%
% Aufgabenstellung
%
% Aufgabe 8

Ziel ist die Bestimmung des an den Punkt $\vec{q}=\begin{pmatrix} 2 \\ 3 \\ 3 
\end{pmatrix}$ nächstgelegenen Punktes auf der Ebene
\[E=\lbrace \begin{pmatrix} 2 \\ -2 \\ 1 \end{pmatrix} + \lambda \begin{pmatrix} 
1 \\ 0 \\ 1 \end{pmatrix} + \mu \begin{pmatrix} 0 \\ -1 \\ 2 \end{pmatrix} 
\text{ } | \text{ } \lambda,\mu\in\R\rbrace.\]

Bestimmen Sie dazu den Abstand $d(\lambda,\mu)$ eines beliebigen mit den Parametern 
$\lambda$ und $\mu$ festgelegten Punktes der Ebene $E$ zu $\vec{q}$ und suche Sie eine 
Minimalstelle dieser Funktion.


%%%%%%%%%%%%%%%%%%%%%%%%%%%%%%%%%%%%%%%%%%%%%%%%%%%%%%%%%%%%%%%%%%%%%%%%%%%%%%%%%%%%%%%%%%%%%%%%%%%%%%%%%%%%%%%%%%%%%%%
%
% Lösung
%

\begin{tabs*}[\initialtab{0}\class{exercise}]
%%%%%%%%%%%%%%%%%%%%%%%%%%%%%%%%%%%%%%%%%%%%%%%%%%%%%%%%%%%%%%%%%%%%%%%%%%%%%%%%%%%%%%%%%%%%%%%%%%%%%%%%%%%%%%%%%%%%%%%
\tab{\lang{de}{Antwort}}
  $\mu=-1$ und $\lambda=2$.
  
%
\tab{\lang{de}{Lösungsvideo}}
%https://youtu.be/CcPJwm7QN0U
  \youtubevideo[500][300]{CcPJwm7QN0U}\\

%%%%%%%%%%%%%%%%%%%%%%%%%%%%%%%%%%%%%%%%%%%%%%%%%%%%%%%%%%%%%%%%%%%%%%%%%%%%%%%%%%%%%%%%%%%%%%%%%%%%%%%%%%%%%%%%%%%%%%%
\end{tabs*}
\end{content}