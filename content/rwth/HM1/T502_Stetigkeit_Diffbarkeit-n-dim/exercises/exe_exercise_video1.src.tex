\documentclass{mumie.element.exercise}
%$Id$
\begin{metainfo}
  \name{
    \lang{de}{Ü03: Jacobi-Matrix}
  }
  \begin{description} 
 This work is licensed under the Creative Commons License Attribution 4.0 International (CC-BY 4.0)   
 https://creativecommons.org/licenses/by/4.0/legalcode 

    \lang{de}{Eigenschaften von Geraden}
  \end{description}
  \begin{components}
    \component{generic_image}{content/rwth/HM1/images/g_img_T102_exercise2.meta.xml}{laufwege}
  \end{components}
  \begin{links}
  \end{links}
  \creategeneric
\end{metainfo}
%
\begin{content}
\begin{block}[annotation]
	Im Ticket-System: \href{https://team.mumie.net/issues/28523}{Ticket 28523}
\end{block}
\begin{block}[annotation]
Copy of : content/playground/Shafie_Shokrani/exe_E3.src.tex
\end{block}

\begin{block}[annotation]
	Im Ticket-System: \href{https://team.mumie.net/issues/ }{Ticket  }
\end{block}

%
 \begin{block}[annotation]
%
      Jacobi-Matrix und lineare Näherung - Aufgabe mit Video
%
\end{block}

  \title{
    \lang{de}{Ü03: Jacobi-Matrix}
  }
%
% Aufgabenstellung
%
%Aufgabe 3

Berechnen Sie die Jacobi-Matrix zu

\begin{enumerate}[alph]
    \item $f:\R^4 \rightarrow\R^3$, $f(x_1 ,x_2 ,x_3 , x_4 )= \begin{pmatrix} x_1 x_2 e^{x_3} 
    \\ x_2 x_3 x_4 \\ x_4 \end{pmatrix}$, 

    \item  $g:\R^2 \rightarrow\R^4$, $g(x,y)= \begin{pmatrix} 1 \\ x \\ x \\ xy \end{pmatrix}$.
\end{enumerate}
%%%%%%%%%%%%%%%%%%%%%%%%%%%%%%%%%%%%%%%%%%%%%%%%%%%%%%%%%%%%%%%%%%%%%%%%%%%%%%%%%%%%%%%%%%%%%%%%%%%%%%%%%%%%%%%%%%%%%%%
%
% Lösung
%

\begin{tabs*}[\initialtab{0}\class{exercise}]
%%%%%%%%%%%%%%%%%%%%%%%%%%%%%%%%%%%%%%%%%%%%%%%%%%%%%%%%%%%%%%%%%%%%%%%%%%%%%%%%%%%%%%%%%%%%%%%%%%%%%%%%%%%%%%%%%%%%%%%
\tab{\lang{de}{Antworten}}
  \begin{enumerate}[alph]
      \item $f'(x_1 ,x_2 ,x_3 , x_4 )= \begin{pmatrix} x_2 e^{x_3} && x_1 e^{x_3} && x_1 x_2 e^{x_3} && 0 
      \\ 0 &&  x_3 x_4 && x_2 x_4 && x_2 x_3 \\ 0 && 0 && 0 && 1 \end{pmatrix}$.
      
      \item $g'(x,y)= \begin{pmatrix} 0 && 0 \\ 1 && 0 \\ 1 && 0 \\ y && x \end{pmatrix}$.
      
  \end{enumerate}
%
\tab{\lang{de}{Lösungsvideo}}
%https://youtu.be/FapyEIgFo6I
  \youtubevideo[500][300]{FapyEIgFo6I}\\

%%%%%%%%%%%%%%%%%%%%%%%%%%%%%%%%%%%%%%%%%%%%%%%%%%%%%%%%%%%%%%%%%%%%%%%%%%%%%%%%%%%%%%%%%%%%%%%%%%%%%%%%%%%%%%%%%%%%%%%
\end{tabs*}
\end{content}