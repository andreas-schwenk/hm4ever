\documentclass{mumie.element.exercise}
%$Id$
\begin{metainfo}
  \name{
    \lang{en}{Exercise 11}
    \lang{de}{Ü11: Gradient}
  }
  \begin{description} 
 This work is licensed under the Creative Commons License Attribution 4.0 International (CC-BY 4.0)   
 https://creativecommons.org/licenses/by/4.0/legalcode 

    \lang{en}{Exercise 11}
    \lang{de}{Ü11: Gradient}
  \end{description}
  \begin{components}
  \end{components}
  \begin{links}
\link{generic_article}{content/rwth/HM1/T109_Skalar-_und_Vektorprodukt/g_art_content_33_winkel.meta.xml}{content_33_winkel}
\link{generic_article}{content/rwth/HM1/T502_Stetigkeit_Diffbarkeit-n-dim/g_art_content_54_Differentiation.meta.xml}{content_54_Differentiation}
\end{links}
  \creategeneric
\end{metainfo}

\begin{content}
\begin{block}[annotation]
	Im Ticket-System: \href{https://team.mumie.net/issues/22638}{Ticket 22638}
\end{block}
  \title{
    \lang{en}{Exercise 11}
    \lang{de}{Ü11: Gradient}
  }

Es sei $f:(-1,1)\times (-1,1)\to \R$ definiert durch
\[
f(x,y):=y \cdot \arcsin(xy).
\]
  \begin{enumerate}[alph]
    
    \item  Berechnen Sie den Gradienten sowie die Richtungsableitung von $f$ in Richtung $v=(2,1)^T$ im Punkt $(0,\frac{1}{2})^T$.

    \item  Berechnen Sie unter allen normierten Vektoren $w\in \R^2$ (d.h. $\Vert w \Vert_2=1$), diejenige Richtung, sodass $D_w f(0,\frac{1}{2})$ maximal wird. Warum widerspricht dies nicht a)?

    \item  Begründen Sie, dass der Gradient $\nabla f$ immer in diejenige Richtung zeigt, in der $D_v f$ maximal (unter allen normierten Vektoren $v$) wird, vorausgesetzt $\nabla f \neq 0$.
  \end{enumerate}
  \\
    
  \begin{tabs*}[\initialtab{0}\class{exercise}]
    \tab{Lösung a)}
    
    \begin{incremental}[\initialsteps{1}]
      \step Es sei $(x_0,y_0)^T \in (-1,1)\times (-1,1)$. \\
        Für die reelle Funktion $f_1:(-1,1) \rightarrow \R, \ x \mapsto y_0 \cdot \text{arcsin}(xy_0)$ gilt mit der Kettenregel
        \begin{align*}
            f_1'(x_0) = \frac{y_0^2}{\sqrt{1-(x_0y_0)^2}}
        \end{align*}
        und für $f_2: (-1,1) \rightarrow \R, \ y \mapsto y \cdot \text{arcsin}(x_0y)$ gilt entsprechend
        \begin{align*}
            f_2'(y_0) = \text{arcsin}(x_0y_0) + \frac{x_0y_0}{\sqrt{1-(x_0y_0)^2}}.
        \end{align*}
%QS	\step Nach Definition ist somit 
%QS
    \step Hieraus ergeben sich \ref[content_54_Differentiation][definitionsgemäß]{def:part_abl} die partiellen Ableitungen\\
    
        $\frac{\partial f}{\partial x}(x,y) = \frac{y^2}{\sqrt{1-x^2y^2}}$ sowie $\frac{\partial f}{\partial y}(x,y) = \text{arcsin}(xy) + \frac{xy}{\sqrt{1-x^2y^2}}$ für alle $x,y \in (-1,1)$. \\ \\
	\step Im Punkt $(0,\frac{1}{2})^T$ haben wir
      \begin{align*}
          \frac{\partial f}{\partial x}(0,\frac{1}{2}) &= \frac{(\frac{1}{2})^2}{\sqrt{1-0}} =  \frac{1}{4},\\
          \frac{\partial f}{\partial y}(0,\frac{1}{2}) &= \text{arcsin}(0) + 0 \cdot \frac{1}{2} \cdot \frac{1}{\sqrt{1-0}} = 0.
      \end{align*}
    
    \step Der Gradient ist somit
      \[
      \nabla f (0,\frac{1}{2}) = \begin{pmatrix}\frac{1}{4} \\ 0 \end{pmatrix}.
      \]
	\step Die Richtungsableitung in Richtung $v=(2,1)^T$ ist dann \ref[content_54_Differentiation][bekanntermaßen]{rem:richtg_abl_mv_produkt}
    gegeben durch 
      \begin{align*}
          D_v f(0,\frac{1}{2}) = \nabla f(0,\frac{1}{2}) \bullet v = \frac{1}{4}\cdot 2 + 0 \cdot 1 = \frac{1}{2}.
      \end{align*}
    \end{incremental}
    
    \tab{Lösung b)}
    
    \begin{incremental}[\initialsteps{1}]
      \step Aus a) wissen wir
        \[
        \nabla f (0,\frac{1}{2}) = \begin{pmatrix}\frac{1}{4} \\ 0 \end{pmatrix}.
        \]
        In a) haben wir außerdem
        \begin{align*}
            D_w f(0,\frac{1}{2}) = \nabla f(0,\frac{1}{2}) \bullet w = \frac{1}{4}\cdot w_1 + 0 \cdot w_2 = \frac{1}{4} w_1
        \end{align*}
        ausgenutzt.

        Damit $D_w f(0,\frac{1}{2})$ maximal wird, muss $w_1$ so groß wie möglich gewählt werden und da wir uns auf 
        normierte Vektoren $\Vert w\Vert_2 = 1$ beschränken, muss $w_1=1$ und folglich $w_2=0$ sein.

        Maximal wird $D_w f(0,\frac{1}{2})$ also für $w=(1,0)^T$ mit $D_w f(0,\frac{1}{2})=\frac{1}{4}$.
    
      \step In a) hatten wir für $v=(1,2)^T$ einen größeren Wert 
        $D_v f(0,\frac{1}{2})=\frac{1}{2}>\frac{1}{4}=D_w f(0,\frac{1}{2})$ erhalten. \\
        
       Dies ist jedoch kein Widerspruch dazu, dass $D_w f(0,\frac{1}{2})$ maximal ist, da $\Vert v \Vert_2 = \sqrt{5} \neq 1.$
       Würden wir $v$ auf eine Einheitslänge normieren, d.h. $v'=\frac{1}{\sqrt{5}}v$ würden wir $D_{v'}f(0,\frac{1}{2})=\frac{1}{2\sqrt{5}}<\frac{1}{4}$ erhalten.
    \end{incremental}
    
    \tab{Lösung c)}
    In a) und b) hatten wir schon ausgenutzt, dass $D_v f(x)=\nabla f(x) \bullet v$ ist. 
    Aus der \ref[content_33_winkel][Winkel-Formel zum Skalarprodukt]{def:winkel} erhalten wir hiermit
    \[
    D_v f(x)=\nabla f(x) \bullet v = \cos(\varphi)\cdot \Vert \nabla f(x) \Vert_2 \Vert v \Vert_2,
    \]
    wobei $\varphi$ der Winkel zwischen dem Gradienten $\nabla f(x)$ und dem Richtungsvektor $v$ ist.

    Wenn wir uns wie in b) auf normierte Richtungsvektoren $v$ beschränken, d.h. $\Vert v \Vert_2 =1$, 
    wird der Wert der Richtungsableitung $D_v f(x)$ maximal, wenn $cos(\varphi)=1$ (also maximal) ist,
    d.h. der Winkel $\varphi$ zwischen Gradient $\nabla f(x)$ und Richtungsvektor $v$ gleich Null ist.
    
    Unter der Voraussetzung, dass $v$ normiert ist, wird also die Richtungsableitung $D_v f(x)$ maximal, 
    wenn $\nabla f(x) (\neq 0)$ in dieselbe Richtung zeigt wie $v$.
    
    Wenn $\nabla f(x)=0$ ist, können wir natürlich zunächst keine Aussage über die Richtung des steilsten 
    Anstiegs treffen. In diesem Fall gibt es nämliche mehrere Fälle (Minimum, Maximum, Sattelpunkt usw.), 
    auf die wir hier nicht detaillierter eingehen wollen.
  \end{tabs*}
\end{content}

