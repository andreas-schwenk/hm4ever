\documentclass{mumie.element.exercise}
%$Id$
\begin{metainfo}
  \name{
    \lang{de}{Ü07: lokale Extrema}
  }
  \begin{description} 
 This work is licensed under the Creative Commons License Attribution 4.0 International (CC-BY 4.0)   
 https://creativecommons.org/licenses/by/4.0/legalcode 

    \lang{de}{Eigenschaften von Geraden}
  \end{description}
  \begin{components}
    \component{generic_image}{content/rwth/HM1/images/g_tkz_T502_Exercise07.meta.xml}{T502_Exercise07}
  \end{components}
  \begin{links}
  \end{links}
  \creategeneric
\end{metainfo}
%
\begin{content}
\begin{block}[annotation]
	Im Ticket-System: \href{https://team.mumie.net/issues/28518}{Ticket 28518}
\end{block}

 \begin{block}[annotation]
%
      Lokale Extremstellen - Textaufgaben mit Video
%
\end{block}

  \title{
    \lang{de}{Ü07: lokale Extrema}
  }
%
% Aufgabenstellung
%
% Aufgabe 7

Berechnen Sie eine Ausgleichsgerade zu den drei Punkten $(-2,0), (0,1)$ und 
$(1,2)$, d.h. eine Gerade, so dass die Summe der Quadrate der markierten Abstände 
(in $y$-Richtung) minimal ist.
\begin{center}
\image{T502_Exercise07}
\end{center}
%%%%%%%%%%%%%%%%%%%%%%%%%%%%%%%%%%%%%%%%%%%%%%%%%%%%%%%%%%%%%%%%%%%%%%%%%%%%%%%%%%%%%%%%%%%%%%%%%%%%%%%%%%%%%%%%%%%%%%%
%
% Lösung
%

\begin{tabs*}[\initialtab{0}\class{exercise}]
%%%%%%%%%%%%%%%%%%%%%%%%%%%%%%%%%%%%%%%%%%%%%%%%%%%%%%%%%%%%%%%%%%%%%%%%%%%%%%%%%%%%%%%%%%%%%%%%%%%%%%%%%%%%%%%%%%%%%%%
\tab{\lang{de}{Antwort}}
$\frac{9}{14}x+\frac{17}{14}$.

%
\tab{\lang{de}{Lösungsvideo}}
%https://youtu.be/011fqV2Ci84
  \youtubevideo[500][300]{011fqV2Ci84}\\

%%%%%%%%%%%%%%%%%%%%%%%%%%%%%%%%%%%%%%%%%%%%%%%%%%%%%%%%%%%%%%%%%%%%%%%%%%%%%%%%%%%%%%%%%%%%%%%%%%%%%%%%%%%%%%%%%%%%%%%
\end{tabs*}
\end{content}