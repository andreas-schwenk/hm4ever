\documentclass{mumie.element.exercise}
%$Id$
\begin{metainfo}
  \name{
    \lang{en}{Exercise 12}
    \lang{de}{Ü12: Extremstellen}
  }
  \begin{description} 
 This work is licensed under the Creative Commons License Attribution 4.0 International (CC-BY 4.0)   
 https://creativecommons.org/licenses/by/4.0/legalcode 

    \lang{en}{Exercise 12}
    \lang{de}{Übung 12}
  \end{description}
  \begin{components}
  \end{components}
  \begin{links}
\link{generic_article}{content/rwth/HM1/T502_Stetigkeit_Diffbarkeit-n-dim/g_art_content_54_Differentiation.meta.xml}{content_54_Differentiation}
\link{generic_article}{content/rwth/HM1/T106_Differentialrechnung/g_art_content_22_extremstellen.meta.xml}{content_22_extremstellen}
\end{links}
  \creategeneric
\end{metainfo}

\begin{content}
\begin{block}[annotation]
	Im Ticket-System: \href{https://team.mumie.net/issues/22639}{Ticket 22639}
\end{block}
  \title{
    \lang{en}{Exercise 12}
    \lang{de}{Ü12: Extremstellen}
  }
  Gegeben sei $f:\R^2\to \R, x \mapsto 10 (x_1-x_2^2)^2+(x_2-1)^2$.

  \begin{enumerate}[alph]   
    \item Bestimmen Sie alle Kandidaten für lokale Extremstellen von $f$.
    \item Bestimmen Sie die globale Minimalstelle von $f$.    
  \end{enumerate}

%  Wir möchten nun die globale Minimalstelle der Funktion $f$ bestimmen. Ähnlich zum 
%  \ref[content_22_extremstellen][eindimensionalen Fall]{thm:notw_bedg_lok_extremum} 
%  gilt das folgende notwendige Kriterium:\\
  
%  Besitzt eine stetig differenzierbare Funktion $f:\R^2\to \R$ in $a\in \R^2$ ein lokales Extremum (Minimum oder Maximum),
%  dann gilt $\nabla f (a)=0$.
%  \begin{enumerate}[alph]   
%    \item Bestimmen Sie alle Kandidaten für ein Extremum, d.h. bestimmen Sie alle $a\in \R^2$ mit $\nabla f(a)=0$.
%    \item Bestimmen Sie nun die globale Minimalstelle.    
%  \end{enumerate}
  \\
  
  \begin{tabs*}[\initialtab{0}\class{exercise}]
    \tab{Lösung a)}
    \begin{incremental}[\initialsteps{1}]
      \step
        $f$ ist als Polynomfunktion auf $\R^2$ stetig partiell differenzierbar mit
        \[
%QS        \nabla f(x) = \begin{pmatrix}20 (x_1-x_2^2) \\ 20 (x_1-x_2^2) \cdot 2x_2 + 2(x_2-1)\end{pmatrix}= \begin{pmatrix} 20x_1 -20 x_2^2 \\ 40 x_1 x_2 -40x_2^3-2x_2-2\end{pmatrix}
        \nabla f(x) = \begin{pmatrix}20 (x_1-x_2^2) \\ 20 (x_1-x_2^2) \cdot 2x_2 + 2(x_2-1)\end{pmatrix}= \begin{pmatrix} 20x_1 -20 x_2^2 \\ 40 x_1 x_2 -40x_2^3+2x_2-2\end{pmatrix}
        \]
%neu Beginn
        Das \ref[content_54_Differentiation][notwendige Kriterium ]{thm:notwendig_Bed_lok_Extremum} für das Vorliegen einer 
        lokalen Extremstelle von $f$ ist, dass der Gradient von $\nabla f$
        an dieser Stelle gleich $0$ ist.

%neu Ende
      \step
        Wir suchen daher alle Lösungen des Gleichungssystems $\nabla f(x)=0$: 
%QS        Es muss
        \begin{align*}
%QS        20 x_1 -20 x_2^2 &= 0 \\ 40x_1 x_2 -40x_2^3-2x_2-2 &= 0
        20 x_1 -20 x_2^2 &= 0 \\ 40x_1 x_2 -40x_2^3+2x_2-2 &= 0
        \end{align*}
%QS        gelöst werden.
        Aus der ersten Gleichung ergibt sich $x_1 = x_2^2$. Setzen wir dies in die zweite Gleichung ein, erhalten wir
        \[
%QS        40x_2^3 -40x_2^3-2x_2-2 = 0 \Leftrightarrow 2 x_2 = 2 \Leftrightarrow x_2=1
        40x_2^3 -40x_2^3+2x_2-2 = 0 \Leftrightarrow 2 x_2 = 2 \Leftrightarrow x_2=1
        \]
        Damit ist $x_1=1$ und der einzige Kandidat für eine lokale Extremstelle ist $a=(1,1)^T$.
    \end{incremental}

    \tab{Lösung b)}
    
    \begin{incremental}[\initialsteps{1}]
      \step In a) haben wir festgestellt, dass $a=(1,1)^T$ der einzige Kandidat für eine lokale Extremstelle von $f$ ist.
        Wir prüfen nun, ob $a$ die globale Minimalstelle von $f$ ist.

%FS: Nun 10 \cdot ... (vorher kein Mal-Punkt)
      \step Es ist $\; f((1,1)^T) = 10 \cdot (1-1^2)+(1-1)^2 = 0.\,$ Nun zeigen wir, dass $f(x)\geq 0$ für alle $x\in \R^2$ ist:
%FS: Nun mit underbrace-Klammern
    \begin{align*}
      f(x) = 10 \underbrace{(x_1-x_2^2)^2}_{\geq 0}+\underbrace{(x_2-1)^2}_{\geq 0} \geq 0 + 0 = 0.
      \end{align*}
      Somit ist $a=(1,1)^T$ eine globale Minimalstelle.
    \end{incremental}
  \end{tabs*}
\end{content}

