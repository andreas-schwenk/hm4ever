\documentclass{mumie.element.exercise}
%$Id$
\begin{metainfo}
  \name{
    \lang{de}{Ü05: lokale Extrema}
  }
  \begin{description} 
 This work is licensed under the Creative Commons License Attribution 4.0 International (CC-BY 4.0)   
 https://creativecommons.org/licenses/by/4.0/legalcode 

    \lang{de}{Eigenschaften von Geraden}
  \end{description}
  \begin{components}
    \component{generic_image}{content/rwth/HM1/images/g_img_T102_exercise2.meta.xml}{laufwege}
  \end{components}
  \begin{links}
  \end{links}
  \creategeneric
\end{metainfo}
%
\begin{content}
\begin{block}[annotation]
	Im Ticket-System: \href{https://team.mumie.net/issues/28520}{Ticket 28520}
\end{block}

%
 \begin{block}[annotation]
%
      Lokale Extremstellen - Textaufgaben mit Video
%
\end{block}

  \title{
    \lang{de}{Ü05: lokale Extrema}
  }
%
% Aufgabenstellung
%
% Aufgabe 5

Bestimmen Sie die Punkte, an denen die Gradienten der Funktionen

\begin{enumerate}[alph]
    \item $f(x,y)=x^4 +2y^2 -4xy$, 

    \item $f(x_1 ,x_2 ,x_3)= 2x^{2}_{1}- 2x_{1}x_{2} +x^{2}_{2}+x^{2}_{3}-2x_{1}-4x_{3}$.
\end{enumerate}
Null sind.



%%%%%%%%%%%%%%%%%%%%%%%%%%%%%%%%%%%%%%%%%%%%%%%%%%%%%%%%%%%%%%%%%%%%%%%%%%%%%%%%%%%%%%%%%%%%%%%%%%%%%%%%%%%%%%%%%%%%%%%
%
% Lösung
%

\begin{tabs*}[\initialtab{0}\class{exercise}]
%%%%%%%%%%%%%%%%%%%%%%%%%%%%%%%%%%%%%%%%%%%%%%%%%%%%%%%%%%%%%%%%%%%%%%%%%%%%%%%%%%%%%%%%%%%%%%%%%%%%%%%%%%%%%%%%%%%%%%%
\tab{\lang{de}{Antworten}}
  \begin{enumerate}[alph]
      \item $(0,0),(-1,1)$ und $(1,1)$.
      
      \item $(1,1,2)$.
      
  \end{enumerate}
%
\tab{\lang{de}{Lösungsvideo}}
%https://youtu.be/rmoEg1-Qg3M
  \youtubevideo[500][300]{rmoEg1-Qg3M}\\

%%%%%%%%%%%%%%%%%%%%%%%%%%%%%%%%%%%%%%%%%%%%%%%%%%%%%%%%%%%%%%%%%%%%%%%%%%%%%%%%%%%%%%%%%%%%%%%%%%%%%%%%%%%%%%%%%%%%%%%
\end{tabs*}
\end{content}