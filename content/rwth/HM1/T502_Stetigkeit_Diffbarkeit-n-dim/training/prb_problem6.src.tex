
\documentclass{mumie.problem.gwtmathlet}
%$Id$
\begin{metainfo}
  \name{
    \lang{de}{A06: Jacobi-Matrix}
    \lang{en}{Exercise 6}
  }
  \begin{description} 
 This work is licensed under the Creative Commons License Attribution 4.0 International (CC-BY 4.0)   
 https://creativecommons.org/licenses/by/4.0/legalcode 

    \lang{de}{}
    \lang{en}{}
  \end{description}
  \corrector{system/problem/GenericCorrector.meta.xml}
  \begin{components}
    \component{js_lib}{system/problem/GenericMathlet.meta.xml}{mathlet}
  \end{components}
  \begin{links}
  \end{links}
  \creategeneric
  \begin{taxonomy}
        \difficulty{0}
        \usage{}
        \objectives{apply,remember}
        \topic{analysis1/differentiation/function_several_real_variables/jacobi_matrix}
  \end{taxonomy}
\end{metainfo}
\begin{content}
\begin{block}[annotation]
	Im Ticket-System: \href{https://team.mumie.net/issues/22596}{Ticket 22596}
\end{block}


\usepackage{mumie.genericproblem}

\lang{de}{\title{A06: Jacobi-Matrix}}

\begin{problem}
	\begin{variables}

		\drawFromSet{a}{-4,-2,2,4,6,8} 
		\drawFromSet{b}{-3,3,5}  
		\number{c}{7}  
		\drawFromSet{d}{-4,-2,2,4,6,8}  
        %\randint[Z]{ax}{-4}{8}
        %\randint[Z]{aw}{-4}{8}
        \function[normalize]{av}{(a/2)*x^2+b*y-c}
        \function[normalize]{aw}{c*x+(d/2)*y^2-2*a}
        
%QS Vorschlag Gegenbeispiel für Globale Umkehrfunktion
%
        \function[calculate]{xn}{b/a}
        \function[calculate]{yn}{c/d}
%        
        \matrix[calculate]{aa}{a &  b \\ c & d}
        \matrix[calculate]{bb}{d/(a*d-b*c) & -b/(a*d-b*c) \\ -c/(a*d-b*c) &  a/(a*d-b*c)}
        
        \matrix{cc}{ a*x & b \\ c & d*y}


%QS Vorschlag 
%
        \function[calculate]{g1}{(a/2)+b-c}
        \function[calculate]{g2}{c+(d/2)-2*a}
        
        \matrix[calculate]{gg}{g1 \\ g2}  %gg sollte immer ungleich (1,1) sein.
        
%ursprüngl.        \matrix[calculate]{gg}{(a/2)+b-c \\ c+(d/2)-2*a}  %gg sollte immer ungleich (1,1) sein.        
%        

	\end{variables}
        
	\begin{question}

		\type{input.generic}
        \field{real}
		%\correctorprecision{2}
		%\displayprecision{2}
		\lang{de}{
	    \text{Bestimmen Sie die Jacobi-Matrix zur Funktion $f:\mathbb{R}^2\rightarrow
        \mathbb{R}^2, \, (x,y)^T \mapsto \begin{pmatrix} \var{av} \\\var{aw}\end{pmatrix}$ und werten Sie diese an der Stelle $(1,1)$ aus.
        Bestimmen Sie ferner die Inverse der Jacobi-Matrix, ausgewertet an derselben Stelle.\\\\
        }}
        \begin{answer}
            \type{input.matrix}
	    	\text{Die Jacobi-Matrix zu $f$ lautet: $\quad$ $Df(x,y)=$ } 
            \format{2}{2}
            \checkAsFunction{x,y}{-10}{10}{100}
            \solution{cc}
%QS Vorschlag 
%
            \explanation{Die Jacobi-Matrix $Df(x,y)$ enthält in der ersten Spalte die partiellen Ableitungen
                        von $f$ nach $x$ und in der zweiten Spalte die partiellen Ableitungen von $f$ nach $y$. 
                        Die jeweils andere Variable wird dabei als feste Konstante betrachtet.}
%
% ursprünglich           \explanation[edited]{Die Jacobi-Matrix $Df(x,y)$ haben Sie nicht korrekt aufgestellt. 
%                                 In der ersten Spalte müssen Sie $f$ nach $x$ ableiten, in der 
%                                 zweiten Spalte nach $y$. Die jeweils andere Variable ist dann 
%                                 eine feste Konstante.}
		\end{answer} 
        
        \begin{answer}
            \type{input.matrix}
	    	\text{Die Jacobi-Matrix ausgewertet bei $(1,1)$ lautet: $\quad$ $Df(1,1)=$ } 
            \format{2}{2}
            \checkAsFunction{x,y}{-10}{10}{100}
%QS Vorschlag 
% 
            \explanation{Setzen Sie in die zuvor bestimmte Jacobi-Matrix $Df(x,y)$ den Punkt $(1,1)$ ein.
                        Überprüfen Sie eventuell noch einmal Ihre Rechnung.}
%
% ursprünglich           \explanation[edited]{Setzen Sie in $Df(x,y)$ den Punkt $(1,1)$ ein.}
%
            \solution{aa}
		\end{answer} 
      
        \begin{answer}
            \type{input.matrix}
	    	\text{Die Inverse zur Jacobi-Matrix ausgewertet bei $(1,1)$ lautet: $\quad$ $(Df(1,1))^{-1}=$ } 
            \format{2}{2}
            %\checkAsFunction{x,y}{-10}{10}{100}
            \solution{bb}
%QS Vorschlag 
% 
%            \explanation{Bestimmen Sie die Inverse der $(2\times 2)-$Matrix $Df(1,1)$.
%                         Überprüfen Sie eventuell noch einmal Ihre Rechnung.}
%
           \explanation[edited]{Ihre Inverse von $Df(1,1)$ sollten Sie noch einmal überprüfen.}
%           
        \end{answer}
        
%QS Vorschlag 2: Da die Untersuchungen zur lokalen Umkehrbarkeit im Rahmen des MC-Teils stattfinden, 
%                passt diese Teilaufgabe besser am Ende.
%        
%        \begin{answer}
%            \type{input.matrix}
%            \format{2}{1}
%
%QS Vorschlag 1: 
% 
%            \text{Da die Jacobi-Matrix $Df(1,1)$ in $(1,1)$ invertierbar ist, ist die Funktion $f$ 
%                  in einer Umgebung von $(1,1)$ lokal umkehrbar. Sei $g$ die lokale Umkehrfunktion. 
%                  Dann gilt für die Jacobi-Matrix von $g:\; Dg(a)=(Df(1,1))^{-1}$. Hierbei ist $a=$}
%	        \solution{gg}
%            \explanation{Nach dem Satz über die lokale Umkehrbarkeit gilt $Dg(f(1,1))=(Df(1,1))^{-1}$,
%                         also ist $a=f(1,1)$.}
%%
%% ursprünglich                      
%%            \text{Die Funktion $f$ ist in einer Umgebung von $(1,1)$ lokal invertierbar. Sei $g$ die lokale Umkehrfunktion. 
%%                  Es gilt $Dg(a)=(Df(1,1))^{-1}$. Hierbei ist $a=$}
%%	        \solution{gg}
%%           \explanation[edited]{Es ist $a=f(1,1)$.}
%%
%        \end{answer}
\end{question}

\begin{question}
    \type{mc.multiple}
    \text{Welche Aussagen sind zutreffend?}
    \begin{choice}
        \text{$f$ ist stetig differenzierbar.}
        \solution{true}
    \end{choice}
    \begin{choice}
        \text{$f$ besitzt eine globale Umkehrfunktion $f^{-1}$.}
        \solution{false}
    \end{choice}
    \begin{choice}
    %
%QS Vorschlag
        \text{Sei $g$ die lokale Umkehrfunktion von $f$. Dann lässt sich über $Dg(1,1)$ zunächst keine Aussage treffen.}
%                
% ursprüngl.
%        \text{Über $Dg(1,1)$ lässt sich zunächst keine Aussage treffen.}
        \solution{true}
    \end{choice}
    \begin{choice}
        \text{Die lokale Umkehrbarkeit von $f$ in $(1,1)$ folgt aus der Existenz der Matrix $(Df(1,1))^{-1}$.}
        \solution{true}
    \end{choice}
    \begin{choice}
        \text{Für die lokale Umkehrbarkeit von $f$ in $(1,1)$ reicht es, wenn alle Werte von $Df(1,1)$ von Null verschieden sind.}
        \solution{false}
    \end{choice}
%
%QS Vorschlag (diese letzte Aussage ersetzt die letzte Frage in Teil (a))
%    
    \begin{choice}
        \text{Sei $g$ die lokale Umkehrfunktion von $f$. Dann gilt $\; Dg(\var{g1}, \var{g2})=(Df(1,1))^{-1}$.}
        \solution{true}
    \end{choice}
%    
    \explanation[equalChoice(0?????)]{$f$ ist in jeder Komponente ein Polynom und damit stetig (partiell) differenzierbar.}
%
%QS Vorschlag
    \explanation[equalChoice(?1????)]{$f$ ist im Punkt $(\frac{\var{b}}{\var{a}}, \frac{\var{c}}{\var{d}})$ nicht (lokal) umkehrbar,
                da $\, \det(Df(\frac{\var{b}}{\var{a}}, \frac{\var{c}}{\var{d}}))=0$ und die Jacobi-Matrix somit in diesem Punkt 
                nicht invertierbar ist.}
    \explanation[equalChoice(??0???)]{Aus den bisherigen Berechnungen kann nur eine Aussage über $Dg(f(1,1))$ getroffen werden, jedoch nicht 
                über $Dg(1,1)$. Es ist noch nicht einmal klar, dass $g$ im Punkt $(1,1)$ überhaupt definiert ist!}
    \explanation[equalChoice(??0???) OR equalChoice(???0??) OR equalChoice(????1?) OR equalChoice(?????0)]{Schauen Sie sich den Satz über 
                die lokale Umkehrbarkeit von Funktionen noch einmal genau an.} 
    \explanation[equalChoice(???0??)]{Prüfen Sie, unter welchen Voraussetzungen auf die Existenz einer lokalen Umkehrfunktion geschlossen werden kann.}
    \explanation[equalChoice(????1?)]{Für die lokale Umkehrbarkeit von $f$ in $(1,1)$ wird der volle Rang der Matix $Df(1,1)$ benötigt. Der 
                volle Rang einer Matrix ist jedoch nicht grundsätzlich gegeben, wenn alle Werte dieser Matrix von Null verschieden sind.
                Ein einfaches Beispiel hierfür ist eine Matrix mit zwei identischen Zeilen (oder Spalten).}
    \explanation[equalChoice(?????0)]{Nach dem Satz über die lokale Umkehrbarkeit gilt $Dg(f(1,1))=(Df(1,1))^{-1}$. Berechnen Sie also zur Prüfung der 
                letzten Aussage $f(1,1).$}
%                
% ursprüngl.
%    \explanation[equalChoice(?1???)]{Für eine globale Invertierbarkeit von $f$ müsste das Gleichungssystem $f(x,y)=(u,v)$ eindeutig lösbar sein.}
%    \explanation[equalChoice(??0??)]{Mit dem bisher Ausgerechnetem kann nur eine Aussage über $Dg(f(1,1))$ getroffen werden. Es ist noch nicht einmal klar, dass $g$ im Punkt $(1,1)$ überhaupt definiert ist!}
%    \explanation[equalChoice(???0?)]{Schauen Sie sich noch einmal genau an, für welche Funktionen auf die Existenz einer lokalen Umkehrfunktion geschlossen werden kann.}
%    \explanation[equalChoice(????1)]{Für die lokale Umkehrbarkeit benötigen Sie vollen Rang von $Df(1,1)$. Wenn alle Werte von $Df(1,1)$ von Null verschieden sind, kann der Rang aber auch $1$ sein.}
\end{question}

\end{problem}

\embedmathlet{mathlet}

\end{content}