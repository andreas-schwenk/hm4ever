\documentclass{mumie.problem.gwtmathlet}
%$Id$
\begin{metainfo}
  \name{
    \lang{en}{Exercise 5}
    \lang{de}{A05: Gradient}
  }
  \begin{description} 
 This work is licensed under the Creative Commons License Attribution 4.0 International (CC-BY 4.0)   
 https://creativecommons.org/licenses/by/4.0/legalcode 

    \lang{en}{Exercise 5}
    \lang{de}{A05: Gradient}
  \end{description}
  \corrector{system/problem/GenericCorrector.meta.xml}
  \begin{components}
    \component{js_lib}{system/problem/GenericMathlet.meta.xml}{gwtmathlet}
  \end{components}
  \begin{links}
  \end{links}
  \creategeneric
\end{metainfo}
\begin{content}

\title{A05: Gradient}
\begin{block}[annotation]
	Im Ticket-System: \href{https://team.mumie.net/issues/21862}{Ticket 21862}
\end{block}


\usepackage{mumie.genericproblem}

\begin{problem}
    \begin{question}
        \type{input.generic}
        \begin{variables}
            \randint[Z]{a}{-5}{5}
            \randint{b}{-4}{4}
            \randint{c}{-4}{4}
            \randint{d}{-4}{4}
            \randint{aa}{-4}{4}
            \randint{bb}{-4}{4}
            \function[normalize]{f}{a*x^3+b*x*y+(c*x*y+d*z)^2+aa*z^3+bb*x*y*z}
            \derivative[normalize]{g1}{f}{x}
            \derivative[normalize]{g2}{f}{y}
            \derivative[normalize]{g3}{f}{z}
            
            \randint{p1}{-4}{4}
            \randint{p2}{-4}{4}
            \randint[Z]{p3}{-3}{3}
            \substitute{hg11}{g1}{x}{p1}
            \substitute{hg12}{hg11}{y}{p2}
            \substitute{h1}{hg12}{z}{p3}
            \substitute{hg21}{g2}{x}{p1}
            \substitute{hg22}{hg21}{y}{p2}
            \substitute{h2}{hg22}{z}{p3}
            \substitute{hg31}{g3}{x}{p1}
            \substitute{hg32}{hg31}{y}{p2}
            \substitute{h3}{hg32}{z}{p3}
            \pmatrix{gr}{g1 \\ g2 \\ g3}
            \pmatrix[calculate]{grp}{h1 \\ h2 \\ h3}
            \randint[Z]{v1}{-3}{3}
            \randint{v2}{-3}{3}
            \randint{v3}{-3}{3}
            \function[calculate]{grv}{h1*v1+h2*v2+h3*v3}
            
            \pmatrix[calculate]{v}{v1 \\ v2 \\ v3}

        \end{variables}
        
        \text{Sei $f:\R^3\to \R, \, (x,y,z)^T \mapsto \var{f}.\;$ }
        \begin{answer}
            \type{input.matrix}
%            \text{Bestimmen Sie $\nabla f(x,y,z)$:}
            \text{Bestimmen Sie den Gradienten von $f: \quad \nabla f(x,y,z)=$}
            \solution{gr}
            \checkAsFunction{x,y,z}{-10}{10}{100}
            \explanation{Der Gradient ist ein Spaltenvektor. In der ersten Zeile wird $f$ nach $x$ abgeleitet, in der zweiten nach $y$ usw.}
        \end{answer}
\\
\\
        \begin{answer}
            \type{input.matrix}
            %\field{rational}
            \text{Berechnen Sie $\nabla f$ an der Stelle $a=(\var{p1},\var{p2},\var{p3})^T: \quad \nabla f(a)=$}
            \solution{grp}
            \explanation{Setzen Sie in die vorherige Antwort die Stelle $(x,y,z)=(\var{p1},\var{p2},\var{p3})$ ein.}
        \end{answer}
\\
\\        
        \begin{answer}
            \type{input.number}
            %\field{rational}
%            \text{Sei $v=(\var{v1},\var{v2},\var{v3})^T$. Bestimmen Sie die Richtungsableitung $D_v f(a)=$}
            \text{Bestimmen Sie die Richtungsableitung von $f$ im Punkt $a$ in Richtung $v=(\var{v1},\var{v2},\var{v3})^T: \quad D_v f(a)=$}
            \solution{grv}
%            \explanation[edited]{$f$ ist total differenzierbar. Damit ist D_v f(a)= \nabla f(a)\bullet v$.}
%
%QS-Vorschlag:
%
            \explanation{$f$ ist als Polynom total differenzierbar. Daher exisitert insbesondere in $a$
                         die Richtungsableitung von $f$ in Richtung $v \; (\neq 0 )\;$ und es gilt das Matrix-Vektor-Produkt
                         $\; D_v f(a)= Df (a)^T \cdot v$ bzw. das äquivalente Skalarprodukt $D_v f(a)=\nabla f(a)\bullet v %=\var{grp} \bullet \var{v}
                         $.}
%    
        \end{answer}
    \end{question}
\end{problem}




\embedmathlet{gwtmathlet}

\end{content}




