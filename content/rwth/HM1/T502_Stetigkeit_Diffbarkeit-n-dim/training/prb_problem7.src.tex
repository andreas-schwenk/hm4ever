
\documentclass{mumie.problem.gwtmathlet}
%$Id$
\begin{metainfo}
  \name{
    \lang{en}{Exercise 7}
    \lang{de}{A07: Textaufgabe}
  }
  \begin{description} 
 This work is licensed under the Creative Commons License Attribution 4.0 International (CC-BY 4.0)   
 https://creativecommons.org/licenses/by/4.0/legalcode 

    \lang{en}{...}
    \lang{de}{...}
  \end{description}
  \corrector{system/problem/GenericCorrector.meta.xml}
  \begin{components}
    \component{js_lib}{system/problem/GenericMathlet.meta.xml}{gwtmathlet}
  \end{components}
  \begin{links}
  \end{links}
  \creategeneric
\end{metainfo}
\begin{content}
\title{A07: Textaufgabe}
\begin{block}[annotation]
	Im Ticket-System: \href{https://team.mumie.net/issues/22598}{Ticket 22598}
\end{block}


\usepackage{mumie.genericproblem}

\embedmathlet{gwtmathlet}

\begin{problem}
    \begin{question}
    
        \begin{variables}
            \function{f}{x^2*e^(-y)+y^4*e^(-x)}
            %explizite Formeln
            \function[normalize]{f1}{2*x*e^(-y)-y^4*e^(-x)}
            \function[normalize]{f2}{-x^2*e^(-y)+4*y^3*e^(-x)}
            \function[normalize]{fr}{cos(w)*(2*r*cos(w)*e^(-r*sin(w))-(r*sin(w))^4*e^(-r*cos(w)))+sin(w)*(-(r*cos(w))^2*e^(-r*sin(w))+4*(r*sin(w))^3*e^(-r*cos(w)))}
            
            \function{px}{r*cos(w)}
            \function{py}{r*sin(w)}
            %MUMIE berechnet die Lösung
            \derivative[normalize]{g1}{f}{x}
            \derivative[normalize]{g2}{f}{y}
            \substitute[normalize]{hh1}{g1}{x}{px}
            \substitute{h1}{hh1}{y}{py}
            \substitute{hh2}{g2}{x}{px}
            \substitute{h2}{hh2}{y}{py}
            \function[normalize]{gr}{cos(w)*h1+sin(w)*h2}
            
            \number{three}{3}
            \number{four}{4}
            \substitute{hl1}{f1}{x}{three}
            \substitute{l1}{hl1}{y}{four}
            \substitute{hl2}{f2}{x}{three}            
            \substitute{l2}{hl2}{y}{four}
            
            \matrix[calculate]{m}{l1 & l2}
            
            \function[calculate]{ll}{1/5*(l1*3+l2*4)}
            
            \number{s}{5}
            
       \end{variables}
        \text{Ein Unternehmen ist für die Lackierung großflächiger Bauteile zuständig. 
        Sie haben dazu einen Roboter erworben, der einen sowohl ausfahrbar als auch schwenkbaren Arm besitzt, mit dem er lackiert.
        Der Roboter steht nun im Ursprung $(0,0)$. Der Roboterarm kann so das Gebiet $U_{10}(0)\subset \R^2$ erreichen.
        Der Hersteller der Maschine gibt die Leistung der Maschine in Abhängigkeit der $(x,y)$-Koordinaten des Roboterarms an.
        Die Leistung soll durch die Funktion $f:U_{10}(0) \to \R, (x,y)\mapsto \var{f}$
        beschrieben werden können.
        Nun möchte der Unternehmer wissen, ob die Roboterleistung steigt, wenn er den Arm weiter ausfährt (und den Winkel belässt).}
        \type{input.generic}
        \field{real}  
    \begin{answer}
        \type{input.function}
        \text{Bestimmen Sie zunächst die partiellen Ableitungen: $\frac{\partial f}{\partial x}(x,y) = $}
        \solution{g1}
        \inputAsFunction{x,y}{k1}
        \checkFuncForZero{g1-k1}{-10}{10}{100}
%QS
%  [edited] eingefügt
%       
        \explanation[edited]{$\frac{\partial f}{\partial x}(x,y)$ ist nicht richtig. Leiten Sie $f$ nach $x$ ab.}
    \end{answer}
    \begin{answer}
        \type{input.function}
        \text{$\frac{\partial f}{\partial y}(x,y) = $}
        \solution{g2}
        \inputAsFunction{x,y}{k2}
        \checkFuncForZero{g2-k2}{-10}{10}{100}
%QS
%  [edited] eingefügt
%
        \explanation[edited]{$\frac{\partial f}{\partial y}(x,y)$ ist nicht richtig. Leiten Sie $f$ an beliebiger Stelle nach $y$ ab.}
    \end{answer}
    \begin{answer}
        \type{input.function}
        \text{Wir wählen nun die Polarkoordinaten, $x=r \cos(w), y= r \sin(w)$.
          ($w$ bezeichnet hier den Winkel. Normalerweise nutzt man griechische Buchstaben, wie $\varphi$ oder $\theta$, 
          hierfür. Dies ist aber für Ihre Eingabe ungeeignet.) Dann setzen wir $\tilde{f}(r,w):=f(r\cos(w),r\sin(w))$.
          Bestimmen Sie die radiale Ableitung $\frac{\partial \tilde{f}}{\partial r}$ im Punkt $(r,w)$.\\ $\frac{\partial \tilde{f}}{\partial r}(r,w)=$}
        \solution{gr}
        \inputAsFunction{r,w}{k3}
        \checkFuncForZero{gr-k3}{-10}{10}{100}
%QS
%  Änderungsvorschlag
        \explanation{Es ist $\tilde{f}(r,w)=f\circ g(x,y)$ mit $g:\R^2 \to \R^2, (r,w)^T=(r \cos(w),r\sin(w))^T$.
            Nutzen Sie die Kettenregel, d.h. $D\tilde{f} = Df\circ g \cdot Dg$. Sie brauchen nur den ersten Eintrag 
            dieser Matrix berechnen!}
%ursprüngl.
%        \explanation{Nutzen Sie die Kettenregel, d.h. $D\tilde{f} = Df\circ g \cdot Dg$. Die innere Funktion ist $g:\R^2 \to \R^2, (r,w)^T=(r \cos(w),r\sin(w))^T$. Sie brauchen nur den ersten Eintrag dieser Matrix berechnen!}
    \end{answer}
    \begin{answer}
        \type{input.matrix}
        \text{Nun befindet sich der Roboterarm aktuell am Punkt $x=3$, $y=4$. Bestimmen Sie zunächst einmal $Df(3,4)=$}
        \solution{m}
        \explanation{Achten Sie auf das Format! Das Differential ist hier ein Zeilenvektor. Setzen Sie $x=3, y=4$ in $\frac{\partial f}{\partial x}(x,y)$ und $\frac{\partial f}{\partial y}(x,y)$ ein.}
    \end{answer}
    \begin{answer}
        \type{input.function}
        \text{Die radiale Ableitung von $f$ an der Stelle $a\in \R^2$ kann auch als Richtungsableitung in Richtung $v=\frac{1}{\Vert a \Vert_2}\cdot a$ interpretiert werden.
%QS
%  Änderungsvorschlag
        Berechnen Sie an der Stelle $a=(3,4)^T \;$ die Richtungsableitung $D_v f(3,4)=$}
%ursprüngl.
%        Berechnen Sie $D_v f(3,4)=$}
        \solution{ll}
        \inputAsFunction{}{k4}
        \checkFuncForZero{ll-k4}{-10}{10}{100}
%QS
%  Korrektur in Formel und ohne [edited]
%
        \explanation{Nutzen Sie die Formel $D_v f(a)= Df(a)\cdot v$.}
%ursprüngl.
%        \explanation[edited]{Nutzen Sie $D_v f(a)= Df(a)\cdot a$ aus. Es ist $a=(3,4)^T$.}
        \end{answer}
        
        \begin{answer}
            %QS
%QS  Änderungsvorschlag
        \text{Nun betrachten wir die Situation in Polarkoordinaten.
              Für die aktuelle Position $a=(3,4)^T$ des Roboterarms ist $r=$}
%ursprüngl.
%        \text{Nun betrachten wir dies in Polarkoordinaten. Für die aktuelle Position des Roboterarms ist $r=$}
            \type{input.number}
            \solution{s}
            %QS
%QS  Änderungsvorschlag
        \explanation{Es ist $r=\Vert a \Vert_2$.}
%ursprüngl.
%        \explanation{Es ist $r=\Vert a \Vert_2$ für $a=(3,4)^T$.}
        \end{answer}
        \begin{answer}
            \type{input.function}
%QS  Änderungsvorschlag            
        \text{Bestimmen Sie nun die radiale Ableitung $\frac{\partial \tilde{f}}{\partial r}$ an der aktuelle Position des Roboterarms. 
              (\textit{Hinweis: Die explizite Bestimmung des Winkels $w$ ist hierzu nicht erforderlich, da bekannt ist, dass $\, 3=r \cos(w)$ und $4=r\sin(w)$.})        
%ursprüngl.
%        \text{Setzen Sie in $\frac{\partial \tilde{f}}{\partial r}$ nun die aktuelle Position des Roboterarms ein. 
%            \textit{Hinweis: Sie brauchen den Winkel $w$ nicht explizit berechnen. Sie haben $r$ bestimmt und wissen $3=r \cos(w)$ und $4=r\sin(w)$.}\\
              An der aktuellen Roboterarm-Position ist $\frac{\partial \tilde{f}}{\partial r}=$}
            \solution{ll}
            \inputAsFunction{}{k5}
        \checkFuncForZero{ll-k5}{-10}{10}{100}
%QS  Änderungsvorschlag (ohne [edited]!)
        \explanation{Verwenden Sie das bereits berechnete Ergebnis von $\frac{\partial \tilde{f}}{\partial r}(r,w)$.
        Wie im Hinweis ist $3=r \cos(w)$ und $4=r\sin(w)$ und Sie haben $r$ vorher berechnet. Dann ist also $\cos(w) = \frac{3}{r}$ und $\sin(w) = \frac{4}{r}$. Setzen Sie nun diese Werte $r$, $\sin(w)$ und $\cos(w)$ in die vorher berechnete Formel $\frac{\partial \tilde{f}}{\partial r}(r,w)$ ein.}
        
        \end{answer}
        \begin{answer}
            \text{Abschließend beantworten Sie bitte die Eingangsfrage: Die Roboterleistung steigt, wenn ...}
            \type{mc.multiple}
            \begin{choice}
                \text{... der Arm weiter ausgefahren wird.}
                \solution{compute}
                \iscorrect{ll}{>}{0}
            \end{choice}
            \begin{choice}
                \text{... der Arm etwas eingefahren wird.}
                \solution{compute}
                \iscorrect{ll}{<}{0}
            \end{choice}
            \begin{choice}
                \text{Über den Sachverhalt kann anhand der voherigen Rechnungen keine Aussage getroffen werden.}
                \solution{compute}
                \iscorrect{ll}{=}{0}
            \end{choice}
            %FS: NEUE explanation.
            \explanation{Die radiale Ableitung gibt die Änderungsrate der Roboterleistung an, wenn der Arm etwas weiter ausgefahren wird. Ist sie positiv, dann steigt die Leistung; ist sie negativ, dann fällt die Leistung ab.}
        \end{answer}
    \end{question}
\end{problem}
\end{content}