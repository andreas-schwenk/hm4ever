\documentclass{mumie.problem.gwtmathlet}
%$Id$
\begin{metainfo}
  \name{
    \lang{en}{Problem 8}
    \lang{de}{A08: lokale Umkehrbarkeit}
  }
  \begin{description} 
 This work is licensed under the Creative Commons License Attribution 4.0 International (CC-BY 4.0)   
 https://creativecommons.org/licenses/by/4.0/legalcode 

    \lang{en}{Problem 8}
    \lang{de}{Aufgabe 8}
  \end{description}
  \corrector{system/problem/GenericCorrector.meta.xml}
  \begin{components}
    \component{js_lib}{system/problem/GenericMathlet.meta.xml}{gwtmathlet}
  \end{components}
  \begin{links}
  \end{links}
  \creategeneric
\end{metainfo}
\begin{content}
\title{A08: lokale Umkehrbarkeit}
\begin{block}[annotation]
	Im Ticket-System: \href{https://team.mumie.net/issues/22603}{Ticket 22603}
\end{block}
\usepackage{mumie.genericproblem}

\begin{problem}
    \begin{question}
        \begin{variables}
            \randint[Z]{a}{-3}{3}
            \randint{b}{0}{3}
            \function[normalize]{f}{sin(x*z)+a*y*z^b+x*ln(1+z)}
            \randint{c}{-2}{2}
            \randint[Z]{d}{-2}{2}
            \function[calculate]{fcd}{a*d*dirac(b)}
            \function[calculate]{m}{2*c+ dirac(b-1)*a*d}
            \randadjustIf{a,b,c,d}{m = 0}
            \matrix{mm}{m}
%QS neu (für die Explanation zur Vermeidung des doppelten "Minuszeichens" bei negativem\var{fcd})            
            \function[normalize]{ff}{sin(x*z)+a*y*z^b+x*ln(1+z)-fcd}
%QS Korrektur:            
            \number{px}{0}                     
            \function[calculate]{py}{-a/m*dirac(b)}
%ursprünglich  
%            \function[calculate]{px}{c/m}
%            \function[calculate]{py}{a/m*dirac(b)}
%            
        \end{variables}
        \text{Die Gleichung $\var{f}=\var{fcd}$ lässt sich lokal um die Lösung $(x_0,y_0,z_0)=(\var{c},\var{d},0)$ eindeutig nach $z=\varphi(x,y)$ auflösen. Zur Begründung der Auflösbarkeit der Gleichung nach $z$ wird nach dem Satz über implizite Funktionen die Invertierbarkeit einer gewissen Jacobi-Matrix $J$ benötigt.}
        \type{input.generic}
        \begin{answer}
            \type{input.matrix}
            \text{Geben Sie $J$ an:}
            \solution{mm}
%QS Änderung:            
            \explanation{Zunächst setzen wir $f(x,y,z):=\var{ff}.\;$ Die gesuchte Matrix $J$ entspricht dann der 
                    $1\times 1-$Matrix $D_zf=\frac{\partial f}{\partial z}$ an der Stelle $(\var{c},\var{d},0),$
                    ist also hier eine reelle Zahl.}
%QS ursprünglich:
%            \explanation{Zunächst setzen wir $f(x,y,z)=\var{f}-\var{fcd}$. $J$ ist in diesem Fall eine $1\times 1$-Matrix,
%                         d.h. eine reelle Zahl. Es ist gegeben durch $\frac{\partial f}{\partial z}(x_0,y_0,z_0)$. }
        \end{answer}
        \begin{answer}
            \type{input.number}
            \text{Bestimmen Sie $\frac{\partial \varphi}{\partial x}(\var{c},\var{d})=$}
            \solution{px}
%QS Änderung:             
            \explanation{Nach dem Satz über die impliziten Funktionen gilt 
                    $D{\varphi}(x,y)=-(D_zf(x,y,z))^{-1} \cdot D_{(x,y)}f(x,y,z)$ und $D_zf(\var{c},\var{d},0)$ 
                    ist die reelle Zahl $J.\;$ Folglich ist
                    $\frac{\partial \varphi}{\partial x}(\var{c},\var{d})=\frac{-1}{J} \cdot \frac{\partial f}{\partial x}(\var{c},\var{d},0)$.}
%QS ursprünglich:            
%            \explanation{Nach dem Satz über die impliziten Funktionen ist $\frac{\partial \varphi}{\partial x}(\var{c},\var{d})=\frac{1}{J} \cdot \frac{\partial f}{\partial x}(\var{c},\var{d},0)$}            
        \end{answer}
        \begin{answer}
            \type{input.number}
            \text{Bestimmen Sie $\frac{\partial \varphi}{\partial y}(\var{c},\var{d})=$}
            \solution{py}
%QS neu:             
            \explanation{Nach dem Satz über die impliziten Funktionen gilt 
                    $D{\varphi}(x,y)=-(D_zf(x,y,z))^{-1} \cdot D_{(x,y)}f(x,y,z)$ und $D_zf(\var{c},\var{d},0)$ 
                    ist die reelle Zahl $J.\;$ Folglich ist
                    $\frac{\partial \varphi}{\partial y}(\var{c},\var{d})=\frac{-1}{J} \cdot \frac{\partial f}{\partial y}(\var{c},\var{d},0)$.}            
        \end{answer}
    \end{question}
    
    \begin{question}
        \begin{variables}
            \randint[Z]{a}{-2}{2}
            \randint{b}{1}{3}
            \randint{c}{2}{3}
            \function[normalize]{f}{e^(x*z)+x+a*y^b}
            \function[normalize]{g}{x*y*z+z^c}
            \randint[Z]{u}{-1}{2}
            \randint[Z]{v}{-1}{2}
            %x_0=0, y_0=u, z_0=v
            \function[calculate]{fuv}{1+a*u^b}
            \function[calculate]{guv}{v^c}
            %nach (x,y) auflösen
            \function[calculate]{m11}{v+1}
            \function[calculate]{m21}{u*v}
            \function[calculate]{m12}{a*b*u^(b-1)}
            \number{m22}{0}
            %\randadjustIf{a,b,c,u,v}{v*u=0}
            
            \matrix{mm}{m11 & m12 \\ m21 & m22}                     % = J Jacobimatrix in (0,u,v)
            
            \number{m13}{0}                                         % part. Abl. v. f nach z in (0,u,v)
            \function[calculate]{m23}{c*v^(c-1)}                    % part. Abl. v. g nach z in (0,u,v)
%QS neu:
            \function[calculate]{detmm}{m11*m22-m12*m21}
            \function[calculate]{n11}{m22/detmm}
            \function[calculate]{n12}{-m12/detmm}
            \function[calculate]{n21}{-m21/detmm}
            \function[calculate]{n22}{m11/detmm}
            \matrix{nn}{n11 & n12 \\ n21 & n22}                     % = J^(-1) inverse Jacobimatrix            


            \function[calculate]{p1}{-m23*n12}                      % = -J^(-1) * (df/dz, dg/dz)^T 1. Komponente
            \function[calculate]{p2}{-m23*n22}                      % = -J^(-1) * (df/dz, dg/dz)^T 2. Komponente 

%
%QS ursprünglich:            
%            \function[calculate]{p1}{-c*v^(c-1)*m12/(a*b*u^b*v)}                    
%            \function[calculate]{p2}{c*v^(c-1)*m22/(a*b*u^b*v)}   <------ falsch (?)    
                                                                      
%QS Ergänzung:
            \function[normalize]{f0}{f-fuv}
            \function[normalize]{g0}{g-guv}
            \pmatrix{ff}{f0 \\ g0}

        \end{variables}

        \text{Das Gleichungssystem $\begin{cases}\var{f}=\var{fuv}, & \\ \var{g}=\var{guv} & \end{cases}$ lässt sich lokal
              um die Lösung $(x_0,y_0,z_0)=(0,\var{u},\var{v})$ eindeutig nach $(x,y)$ auflösen mit den Funktionen 
              $x=\varphi(z)$ und $y=\psi(z)$. Zur Begründung der Auflösbarkeit der Gleichung nach $(x,y)$ wird nach dem 
              Satz über implizite Funktionen die Invertierbarkeit einer gewissen Jacobi-Matrix $J$ benötigt.
%QS Ergänzung:
              Hierzu betrachtet man die Funktionen 
              $f(x,y,z):= \var{ff}$ mit  $f(0,\var{u},\var{v})= 0$ und $g(z):=(\varphi(z),\psi(z))$.\\
\\
              }
              
% \debug[a,b,c,u,v,mm,nn,m13,m23,p1,p2]   


        \type{input.generic}
        \begin{answer}
            \type{input.matrix}
            \text{Geben Sie $J$ an:}
            \solution{mm}
%QS neu:            
            \explanation{Die gesuchte Matrix entspricht der Matrix $D_{(x,y)}f$ an der Stelle $(0,\var{u},\var{v})$. Sie hat das Format 
                         $2 \times 2$. In der ersten Spalte wird $f$ nach $x$ abgeleitet und in der 2. Spalte nach $y$.}
%QS ursprünglich 
%            \explanation{Die gesuchte Matrix hat das Format $2 \times 2$. Die beiden gegebenen Gleichungen müssen nach $x$ und $y$ abgeleitet werden.}
        \end{answer}
        \begin{answer}
            \type{input.number}
            \text{Bestimmen Sie $\varphi'(\var{v})=$}
            \solution{p1}
%QS neu:                       
            \explanation{Nach dem Satz über implizite Funktionen gilt $D_g(z)=-(D_{(x,y)}f(x,y,z))^{-1}\cdot D_zf(x,y,z)$ 
                         und es ist $D_{(x,y)}f(0,\var{u},\var{v})$ ist $J$. Folglich ist $\varphi'(\var{v})=\frac{\partial g}{\partial x}(\var{v})$ 
                         die erste Komponente von $\; -J^{-1}\cdot D_zf(0,\var{u},\var{v})$. }    
%QS ursprünglich 
%           \explanation{Nach dem Satz über die impliziten Funktionen ist $\varphi'(\var{v})=J^{-1} \cdot \frac{\partial f}{\partial z}(x_0,y_0,z_0)$, 
%                        wobei $f= \var{f}-\var{fuv}$.}
%
        \end{answer}
        \begin{answer}
            \type{input.number}
            \text{Bestimmen Sie $\psi'(\var{v})=$}
            \solution{p2}
%QS neu:                       
            \explanation{Nach dem Satz über implizite Funktionen gilt $D_g(z)=-(D_{(x,y)}f(x,y,z))^{-1}\cdot D_zf(x,y,z)$ 
                         und es ist $D_{(x,y)}f(0,\var{u},\var{v})$ ist $J$. Folglich ist $\psi'(\var{v})=\frac{\partial g}{\partial y}(\var{v})$ die 
                         zweite Komponente von $\; -J^{-1}\cdot D_zf(0,\var{u},\var{v})$. }                
        \end{answer}
        
    \end{question}
\end{problem}


\embedmathlet{gwtmathlet}

\end{content}
