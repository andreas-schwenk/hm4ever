\documentclass{mumie.problem.gwtmathlet}
%$Id$
\begin{metainfo}
  \name{
    \lang{en}{Exercise 1}
    \lang{de}{A01: Stetigkeit}
  }
  \begin{description} 
 This work is licensed under the Creative Commons License Attribution 4.0 International (CC-BY 4.0)   
 https://creativecommons.org/licenses/by/4.0/legalcode 

    \lang{en}{Exercise 1}
    \lang{de}{A01: Stetigkeit}  \end{description}
  \corrector{system/problem/GenericCorrector.meta.xml}
  \begin{components}
    \component{js_lib}{system/problem/GenericMathlet.meta.xml}{gwtmathlet}
  \end{components}
  \begin{links}
  \end{links}
  \creategeneric
\end{metainfo}
\begin{content}
\begin{block}[annotation]
	Im Ticket-System: \href{https://team.mumie.net/issues/22643}{Ticket 22643}
\end{block}
\usepackage{mumie.genericproblem}

 \title{
    \lang{en}{}
    \lang{de}{A01: Stetigkeit}
  }
%Stetigkeit 

\begin{problem}

  \begin{question}   %Q1
     \type{mc.multiple}    

      \begin{variables}
        \randint{a}{2}{4}
        \randint{c}{-1}{1}          
        \function[normalize]{g}{a*x^2}  
        \function[normalize]{h}{x^2+y^2}
      \end{variables}

      \text{ Untersuchen Sie die Funktion $\; f:\R^2\to \R, \; \begin{pmatrix}x \\ y\end{pmatrix} \;
            \mapsto \begin{cases} \frac{\var{g}}{\sqrt{\var{h}}}, & (x,y)\neq(0,0) \\ \var{c}, & (x,y)=(0,0) 
            \end{cases}$\\ 
            \\
            
             auf Stetigkeit und geben Sie an, welche der folgenden Aussagen wahr sind.
             \\
             
            }


    %C1          
          \begin{choice}
              \text{$f$ ist stetig auf $\R^2 \setminus \{(0,0)\}$.}
              \solution{true}
          \end{choice}
    %C2   
          \begin{choice}
              \text{$f$ ist stetig in $(0,0)$, denn für jedes $\epsilon >0$ ist mit 
                    $\delta = \frac{1}{\var{a}} \epsilon\;(>0)\;$ das $\epsilon -\delta -$Kriterium erfüllt.
                   }
             \solution{compute}
             \iscorrect{c}{=}{0}                   
          \end{choice}
    %C3             
          \begin{choice}
              \text{$f$ ist stetig in $(0,0)$, da für die Nullfolge $(\frac{1}{n},\frac{1}{n})_{n \in \N}$ 
                    gilt, dass $\lim_{n\to\infty} f(\frac{1}{n},\frac{1}{n})=f(0,0)=\var{c} \;$ ist.
                   }
              \solution{false}
          \end{choice}
    %C4    
           \begin{choice}
              \text{$f$ ist stetig auf dem ganzen Definitionsbereich $\R^2$.}
             \solution{compute}
             \iscorrect{c}{=}{0}
          \end{choice}
    %C5    
           \begin{choice}
              \text{$f$ ist nicht stetig in $(0,0)$.}
             \solution{compute}
             \iscorrect{c}{!=}{0}
          \end{choice}
          
    %expl 
          \explanation[equalChoice(0????)]{$f$ ist auf $\R^2 \setminus {(0,0)}$ stetig als Verküpfung stetiger Funktionen.} 
          \explanation[equalChoice(???11) OR equalChoice(?1??1) OR equalChoice(??1?1)]
                      {Ihre Auswahl bzgl. der Stetigkeit in $(0,0)$ enthält widersprüchliche Aussagen.}          
          \explanation[equalChoice(?0000)]{Überprüfen Sie die Stetigkeit von $f$ in $(0,0)$ mittels $\epsilon -\delta -$ oder dem Folgekriterium.} 
                      
          \explanation[c != 0 AND [equalChoice(?1???) OR equalChoice(??1??) OR equalChoice(???1?) OR equalChoice(????0)]]           
                      {Für die Nullfolge $(\frac{1}{n},\frac{1}{n})_{n \in \N} \;$ 
                       ist $\lim_{n\to\infty} f(\frac{1}{n},\frac{1}{n})=0 \neq \var{c} = f(0,0).$
                       Daher ist $f$ nicht stetig in $(0,0)$.
                      }
                             
          \explanation[c = 0 AND equalChoice(?0001)]
                      {$f$ ist stetig in $(0,0)$, denn für beliebige Nullfolgen $x^{(n)}, y^{(n)}$ gilt
                      $\abs{f(x^{(n)},y^{(n)})}=\abs{\frac{\var{a}(x^{(n)})^2}{\sqrt{(x^{(n)})^2+(y^{(n)})^2}}} 
                      \leq \abs{\frac{\var{a}(x^{(n)})^2+\var{a}(y^{(n)})^2}{\sqrt{(x^{(n)})^2+(y^{(n)})^2}}}
                      = \var{a} \cdot \sqrt{(x^{(n)})^2+(y^{(n)})^2} \quad \xrightarrow{n\to\infty} \; 0 = f(0,0)$\\
                      \\
                      
                      und für $\epsilon >0$ und $(x,y)^T \in \R^2$ mit 
                      $\abs{\abs{(x,y)^T-(0,0)^T}}=\sqrt{x^2+y^2} < \frac{1}{\var{a}} \epsilon\; (:=\delta)\;$
                      gilt  entsprechend
                     $\abs{\abs{f(x,y)-f(0,0)}}=\abs{\frac{\var{a}x^2}{\sqrt{x^2+y^2}}-0} 
                      \leq \abs{\frac{\var{a} x^2+\var{a} y^2}{\sqrt{x^2+y^2}}}
                      = \var{a} \cdot \sqrt{x^2+y^2} < \var{a} \cdot \frac{1}{\var{a}} \epsilon = \epsilon.$
                      }
                    
          \explanation[c = 0 AND equalChoice(??1?0)]
                      {Der Nachweis der Stetigkeit von $f$ in $(0,0)$ mittels Folgenkriterium muss für beliebige Nullfolgen
                       $(x^{(n)}, y^{(n)})_{n \in \N}$ erfolgen. $\lim_{n\to\infty} f(\frac{1}{n},\frac{1}{n})=0 = f(0,0)$
                       reicht als Begründung nicht aus. 
                      }
 
          \explanation[c = 0 AND [equalChoice(???0?) OR equalChoice(?0???)] AND NOT equalChoice(????1)]
                      {Ihre Auswahl ist noch unvollständig.}
 
  \end{question} %Q1

%%%%%%%%%%%%%%%%%%%%%%%%%%%%%%%%%%%%%%%%%%%%%%%%%%%%%%%%%%%%%%%%%%%%%%%%%%%%%%%%%%%%%%%%%%%%%%%%%%%%%%%%%%%%%%%%%%%

  \begin{question}   %Q2
     \type{input.number}    
     \field{rational}

      \begin{variables}
        \randint[Z]{a}{2}{4}
        \randint[Z]{b}{-1}{1}
        \function[normalize]{c}{b/a}              
          
        \function[normalize]{g}{b*sin(x^2+y^2)} 
        \function[normalize]{gn}{b*sin((x^{(n)})^2+(y^{(n)})^2)}        
        \function[normalize]{h}{x^2+y^2}
      \end{variables}

      \text{Die Funktion $\; g:\R^2\to \R, \; \begin{pmatrix}x \\ y\end{pmatrix} \;
%            \mapsto \begin{cases} \frac{\var{b}\sin(\var{g})}{\var{a}\cdot (\var{h})}, & (x,y)\neq(0,0) \\ c, & (x,y)=(0,0) 
            \mapsto \begin{cases} \frac{\var{g}}{\var{a}\cdot (\var{h})}, & (x,y)\neq(0,0) \\ c, & (x,y)=(0,0) 
            \end{cases}$\\ 
            \\
            
             ist auf $\R^2 \setminus \{(0,0)\}$ stetig als Komposition stetiger Funktionen. Die Stetigkeit von $g$
             in $(0,0)$ hängt von der Wahl des Parameters $c \in \R$ ab.     
            }
            
%         \explanation{$g$ ist nur stetig in $(0,0)$, wenn $g(0,0)=c=\frac{\var{b}}{\var{a}}$, denn für jede
%                       beliebige Nullfolge $(x^{(n)}, y^{(n)})_{n \in \N} \;$ gilt\\
%                       \\

%                      $ \lim_{n\to\infty} g(x^{(n)}, y^{(n)})
%                        =\lim_{n\to\infty} \frac{\var{b}\cdot \sin((x^{(n)})^2+(y^{(n)})^2)}{\var{a}\cdot (x^{(n)})^2+(y^{(n)})^2)} 
%                        =\var{c} \cdot \lim_{n\to\infty} \frac{\sin((x^{(n)})^2+(y^{(n)})^2)}{(x^{(n)})^2+(y^{(n)})^2)}
%                        =\var{c}.
%                      $\\
%                       \\

%                     Im letzten Schritt wird verwendet, dass $\lim_{r \to 0} \frac{\sin(r^2)}{r^2}=1\;$ ist, was man 
%                     mithilfe der Regel von de L'Hospital nach zuvoriger Umwandlung der kartesischen Koordinaten
%                     $(x,y)$ in Polarkoordinaten $(r \cos(\phi), r \sin(\phi))$ zeigen kann.
%                     }
          
%FS: Vorschlag für besseres Feedback, das nicht sofort die Lösung vorgibt:
\explanation{Damit $g$ in stetig in $(0,0)$ ist, muss für jede beliebige Nullfolge $(x^{(n)}, y^{(n)})_{n \in \N} \;$ gelten: \\
$\lim_{n\to\infty} g(x^{(n)}, y^{(n)})= g(0,0)=c$.\\
\\
Verwenden Sie bei der Grenzwertbetrachtung $\lim_{r \to 0} \frac{\sin(r^2)}{r^2}=1\;$, was man 
mithilfe der Regel von de L'Hospital nach zuvoriger Umwandlung der kartesischen Koordinaten
$(x,y)$ in Polarkoordinaten $(r \cos(\phi), r \sin(\phi))$ zeigen kann.}

          \begin{answer}
              \text{$g$ ist stetig in $(0,0)$ für $c=$}
              \solution{c}
         \end{answer}
 
  \end{question} %Q2

\end{problem}
\embedmathlet{gwtmathlet}

\end{content}
