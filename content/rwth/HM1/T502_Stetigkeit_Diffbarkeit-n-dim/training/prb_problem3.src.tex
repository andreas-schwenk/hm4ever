\documentclass{mumie.problem.gwtmathlet}
%$Id$
\begin{metainfo}
  \name{
    \lang{en}{Exercise 3}
    \lang{de}{A03: Diff'barkeit}
  }
  \begin{description} 
 This work is licensed under the Creative Commons License Attribution 4.0 International (CC-BY 4.0)   
 https://creativecommons.org/licenses/by/4.0/legalcode 

    \lang{en}{Exercise 3}
    \lang{de}{A03: Diff'barkeit}
  \end{description}
  \corrector{system/problem/GenericCorrector.meta.xml}
  \begin{components}
    \component{js_lib}{system/problem/GenericMathlet.meta.xml}{gwtmathlet}
  \end{components}
  \begin{links}
  \end{links}
  \creategeneric
\end{metainfo}
\begin{content}

\title{A03: Diff'barkeit}

\begin{block}[annotation]
	Im Ticket-System: \href{https://team.mumie.net/issues/22594}{Ticket 22594}
\end{block}

\usepackage{mumie.genericproblem}

\embedmathlet{gwtmathlet}
\begin{problem}
    \begin{question}
        \type{mc.multiple}
        \text{Welche der folgenden Aussagen sind korrekt?}
        \permutechoices{1}{5}
        \begin{choice} %Q1
            \text{Für jede total differenzierbare Funktion sind alle partiellen Ableitungen stetig.}
            \solution{false}
        \end{choice}
        \begin{choice} %Q2
            \text{Für jede total differenzierbare Funktion existieren alle partiellen Ableitungen.}
            \solution{true}
        \end{choice}
        \begin{choice} %Q3
            \text{Existieren alle partiellen Ableitungen einer Funktion $f:\R^n\to \R$ in allen Punkten 
                  $x\in \R^n$, dann ist $f$ stetig.}
            \solution{false}
        \end{choice}
        \begin{choice} %Q4
            \text{Ist $f:\R^n\to \R$ stetig und alle partiellen Ableitungen existieren, 
                  dann ist $f$ total differenzierbar.}
            \solution{false}
        \end{choice}
        \begin{choice} %Q5
            \text{Sei $f:\R^3\to \R$ in $a\in \R^3$ total differenzierbar. Dann existiert eine Richtung $v\in \R^3\backslash\{0\}$, sodass die Richtungsableitung $D_{v}f(a)=0$ erfüllt.}
            \solution{true}
        \end{choice}
        \begin{choice} %Q6
            \text{Alle vorherigen Aussagen sind falsch.}
            \solution{false}
        \end{choice}
%        \explanation[equalChoice(1?????)]{Es gilt nur die umgekehrte Implikation: Sind alle partiellen Ableitungen stetig, dann ist $f$ total differenzierbar.}
%        \explanation[equalChoice(?0????)]{Für total differenzierbare Funktionen existieren alle Richtungsableitungen.}
%        \explanation[equalChoice(??1???)]{Es gibt unstetige Funktionen, die partiell differenzierbar sind (z.B. Beispielaufgabe 3c).}
%        \explanation[equalChoice(???1??)]{Es gilt nur die umgekehrte Implikation: Ist $f$ total differenzierbar, dann ist $f$ stetig und alle partiellen Ableitungen existieren.}

%QS:
        \explanation[equalChoice(00????)]{Für total differenzierbare Funktionen $f:M (\subset\R^n)\to \R^m$ existieren alle partiellen Ableitungen.}
        \explanation[equalChoice(1?????)]{Aus der totalen Differenzierbarkeit einer Funktion $f:M (\subset\R^n)\to \R^m$
                                          folgt die Existenz aller partiellen Ableitungen von $f$,
                                          jedoch nicht deren Stetigkeit.}
                                          
        \explanation[equalChoice(??1???)]{Es gibt unstetige Funktionen $f:\R^n\to \R$, die auf dem gesamten Definitionsbereich 
                                          $\R^n$ partiell differenzierbar sind. Ein Besipiel hierfür ist die Funktion  
                                          $f: \R^2\to \R, x \mapsto \begin{cases}\frac{x_1^4+x_2^4}{(x_1^2+x_2^2)^2}, & x\neq (0,0)^T, \\ 1,& x=(0,0)^T.\end{cases}$.}
%                                          
        \explanation[equalChoice(???1??)]{Es gibt stetige Funktionen $f:\R^n\to \R$, die auf dem gesamten Definitionsbereich
                                          $\R^n$ partiell differenzierbar, aber nicht in allen Punkten total differenzierbar sind.
                                          Ein Besipiel hierfür ist die Funktion 
                                          $f: \R^2\to \R, x \mapsto \begin{cases}\frac{x_1 x_2^2}{x_1^2+x_2^2}, & x\neq (0,0)^T, \\ 0,& x=(0,0)^T.\end{cases}$.
                                          \\ Umgekehrt ist aber jede total differenzierbare Funktion stetig
                                          und es existieren alle partiellen Ableitungen.}         
%       
       \explanation[equalChoice(????0?)]{Da $f:\R^3\to \R$ in $a\in \R^3$ total differenzierbar ist, existieren in $a$ 
                                          alle Richtungsableitungen $D_v f(a)$ mit $v\in \R^3\backslash\{0\}$ und es ist
                                          $D_v f(a)= \nabla f(a) \bullet v.\;$ Da $\nabla f(a)$ ein Vektor
                                          im dreidimensionalen Raum ist, existiert immer ein Vektor $v\in \R^3\backslash\{0\}$, 
                                          der orthogonal/senkrecht auf $\nabla f(a)$ steht.}
    \end{question}
\end{problem}
\end{content}





