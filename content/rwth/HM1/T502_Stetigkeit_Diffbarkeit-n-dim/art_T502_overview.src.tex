%$Id:  $
\documentclass{mumie.article}
%$Id$
\begin{metainfo}
  \name{
    \lang{de}{Überblick: Stetigkeit und Differenzierbarkeit im $\R^n$}
    \lang{en}{overview: }
  }
  \begin{description} 
 This work is licensed under the Creative Commons License Attribution 4.0 International (CC-BY 4.0)   
 https://creativecommons.org/licenses/by/4.0/legalcode 

    \lang{de}{Beschreibung}
    \lang{en}{}
  \end{description}
  \begin{components}
  \end{components}
  \begin{links}
\link{generic_article}{content/rwth/HM1/T502_Stetigkeit_Diffbarkeit-n-dim/g_art_content_55_Lokale_Umkehrbarkeit.meta.xml}{content_55_Lokale_Umkehrbarkeit}
\link{generic_article}{content/rwth/HM1/T502_Stetigkeit_Diffbarkeit-n-dim/g_art_content_54_Differentiation.meta.xml}{content_54_Differentiation}
\link{generic_article}{content/rwth/HM1/T502_Stetigkeit_Diffbarkeit-n-dim/g_art_content_53_Stetigkeit.meta.xml}{content_53_Stetigkeit}

  \end{links}
  \creategeneric
\end{metainfo}
\begin{content}
\begin{block}[annotation]
	Im Ticket-System: \href{https://team.mumie.net/issues/30097}{Ticket 30097}
\end{block}
\begin{block}[annotation]
Copy of : /home/mumie/checkin/content/rwth/HM1/T501_Orientierung_im_n-dim_Raum/art_T501_overview.src.tex
\end{block}

\begin{block}[annotation]
Im Entstehen: Überblicksseite für Kapitel Stetigkeit und Differenzierbarkeit im $\R^n$.
\end{block}

\usepackage{mumie.ombplus}
\ombchapter{1}
\lang{de}{\title{Überblick: Stetigkeit und Differenzierbarkeit im $\R^n$}}
\lang{en}{\title{}}



\begin{block}[info-box]
\lang{de}{\strong{Inhalt}}
\lang{en}{\strong{Contents}}


\lang{de}{
    \begin{enumerate}%[arabic chapter-overview]
   \item[2.1] \link{content_53_Stetigkeit}{Stetigkeit von Funktionen mehrerer Veränderlicher}
   \item[2.2] \link{content_54_Differentiation}{Mehrdimensionale Differentiation}
   \item[2.3] \link{content_55_Lokale_Umkehrbarkeit}{Lokale Umkehrbarkeit von Funktionen}
  \end{enumerate}
} %lang

\end{block}

\begin{zusammenfassung}

\lang{de}{Stetigkeit und Differenzierbarkeit haben Sie als Eigenschaften reeller, eindimensionaler Funktionen kennengelernt.
Sie lassen sich auf Funktionen in mehreren Variablen übertragen. Hier zahlen sich die abstrakten Konzepte aus, 
die wir dafür in einer Dimension eingeführt haben.

Während Konvergenz und Stetigkeit sich direkt von einer in mehrere Dimensionen übertragen lassen, ist das bei der Differenzierbarkeit schwieriger.
Hier müssen wir zwei Eigenschaften der eindimensionalen Ableitung in unterschiedliche Konzepte übertragen werden:\\
Die Beschreibung der \emph{Änderung} einer Funktion, wie es bei der Tangentensteigung der Fall ist, muss nun in jeder Richtung erfasst werden.
Das führt zum Begriff der Richtungsableitung und der partiellen Ableitung.\\
Die Suche nach einer \emph{linearen Approximation} einer Funktion, wie Sie sie zum Beispiel bei der Taylor-Approximation kennengelernt haben,
führt zum Begriff der \emph{totalen Ableitung}.\\
Anschließend ist es wichtig zu wissen, wie diese Konzepte zusammenhängen und was das für die Praxis bedeutet.


Ausblicke auf die Mächtigkeit der kennengelernten Werkzeuge geben der Satz über die lokale Umkehrbarkeit von Funktionen und der Satz über implizite Funktionen.
Sie werden diese Sätze im weiteren Verlauf Ihres Studiums für Anwendungen benötigen.
}


\end{zusammenfassung}

\begin{block}[info]\lang{de}{\strong{Lernziele}}
\lang{en}{\strong{Learning Goals}} 
\begin{itemize}[square]
\item \lang{de}{Sie kennen den Stetigkeitsbegiff im $\R^n$ und wichtige stetige Funktionen.}
\item \lang{de}{Sie überprüfen Stetigkeit in einfachen Situationen.}
\item \lang{de}{Sie berechnen Richtungsableitungen und partielle Ableitungen und interpretieren diese sinnvoll.}
\item \lang{de}{Sie kennen die Begriffe partiell differenzierbar, stetig partiell differenzierbar, total differenzierbar, stetig differenzierbar und wissen um deren Unterschiede.}
\item \lang{de}{Sie stellen Jacobi-Matrizen und Gradienten auf.}
\item \lang{de}{Sie kennen die Ableitungsregeln für Summen, Produkte und Verkettungen von Funktionen und wenden diese an.}
\item \lang{de}{Sie kennen ein notwendiges Kriterium für lokale Extremstellen.}
\item \lang{de}{Sie wissen, unter welchen Bedingungen sich mehrdimensionale Funktionen lokal umkehren lassen und wie sich die Ableitung der Umkehrfunktion dann verhält.}
\item \lang{de}{Sie kennen den Satz über implizite Funktionen.}
\end{itemize} 

\end{block}




\end{content}
