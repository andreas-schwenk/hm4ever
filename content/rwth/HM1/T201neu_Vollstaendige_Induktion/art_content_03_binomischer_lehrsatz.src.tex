%$Id:  $
\documentclass{mumie.article}
%$Id$
\begin{metainfo}
  \name{
    \lang{de}{Binomialkoeffizienten}
    \lang{en}{Binomial Coefficients}
  }
  \begin{description} 
 This work is licensed under the Creative Commons License Attribution 4.0 International (CC-BY 4.0)   
 https://creativecommons.org/licenses/by/4.0/legalcode 

    \lang{de}{Beschreibung}
    \lang{en}{Description}
  \end{description}
  \begin{components}
    \component{generic_image}{content/rwth/HM1/images/g_img_00_video_button_schwarz-blau.meta.xml}{00_video_button_schwarz-blau}
    \component{generic_image}{content/rwth/HM1/images/g_tkz_T201_PascalsTriangle_C.meta.xml}{T201_PascalsTriangle_C}
    \component{generic_image}{content/rwth/HM1/images/g_tkz_T201_PascalsTriangle_D.meta.xml}{T201_PascalsTriangle_D}
    \component{generic_image}{content/rwth/HM1/images/g_tkz_T201_PascalsTriangle_B.meta.xml}{T201_PascalsTriangle_B}
    \component{generic_image}{content/rwth/HM1/images/g_tkz_T201_PascalsTriangle_A.meta.xml}{T201_PascalsTriangle_A}
  \end{components}
  \begin{links}
    \link{generic_article}{content/rwth/HM1/T101neu_Elementare_Rechengrundlagen/g_art_content_01_zahlenmengen.meta.xml}{zahlenmengen}
    \link{generic_article}{content/rwth/HM1/T201neu_Vollstaendige_Induktion/g_art_content_02_vollstaendige_induktion.meta.xml}{vollst-ind}
  \end{links}
  \creategeneric
\end{metainfo}


\begin{content}
\begin{block}[annotation]
	Im Ticket-System: \href{https://team.mumie.net/issues/22235}{Ticket 22235}
     
    Nachfolger von: \href{http://team.mumie.net/issues/9637}{Ticket 9637}\\
\end{block}
\usepackage{mumie.ombplus}
\ombchapter{1}
\ombarticle{3}

\lang{de}{\title{Binomischer Lehrsatz}}
\lang{en}{\title{Binomial Theorem}}


\begin{block}[info-box]
\tableofcontents
\end{block}

\section{Motivation}
\lang{de}{Das Ziel dieses Abschnitts ist es, die binomische Formel $(a+b)^2=a^2+2ab+b^2$ zu verallgemeinern. 
Hier wollen wir also für zwei reelle Zahlen $a$ und $b$ und eine natürliche Zahl $n$ Potenzen der Form 
$(a + b)^n$ in eine Summe von Produkten auflösen. Betrachten wir die folgenden 
Multiplikationen von $(a+b)$ mit sich selbst:}
\lang{en}{The goal of this section is to generalise the binomial formula $(a+b)^2=a^2+2ab+b^2$. For two real 
numbers $a$ and $b$ and a natural number $n$, we wish to expand a power of the form $(a + b)^n$ into a sum of products.
Consider the following multiplications of $(a+b)$ with itself:}

 \begin{align*}
    (a+b)^2 & = & (a+b)(a+b)=a^2+2ab+b^2\\
    (a+b)^3 & = & (a+b)(a+b)^2=(a+b)(a^2+2ab+b^2)=a^3+3a^2b+3ab^2+b^3\\
    (a+b)^4 & = & (a+b)(a+b)^3=(a+b)(a^3+3a^2b+3ab^2+b^3)=a^4+4a^3b+6a^2b^2+4ab^3+b^4\\
          &\vdots&\\
    (a+b)^n & = & (a+b)(a+b)^{n-1} =\text{?}
  \end{align*}


  \lang{de}{Wie man sieht, kommt in diesem iterativ beschriebenen Prozess bei Multiplikation des Binoms $(a+b)$ mit der 
  zuvor berechneten Potenz $(a+b)^{n-1}$ in der Summenformel zu jedem Summanden je einmal $a$ und einmal $b$ als
  weiterer Faktor hinzu. Darum besteht die berechnete n-fache Multiplikation von $(a+b)$ mit sich selbst aus 
  einer Summe von Termen der Form $a^{n}b^{0}, a^{n-1}b^{1},...,a^{1}b^{n-1},a^{0}b^{n}$. Lassen Sie uns für 
  die Anzahl des Vorkommens des Terms $a^{k}b^{n-k}$ in der berechneten 
  Summenformel von $(a+b)^n$ den Ausdruck $\binom{n}{k}$ verwenden. Diese Koeffizienten nennt man die 
  \notion{Binomialkoeffizienten}. Für $n=3$ wird beispielsweise der Binomialkoeffizient von $a^{1}b^{2}$ mit 
  $\binom{3}{1}$ gezeigt und ist, wie der obigen Formel für $(a+b)^3$ zu entnehmen ist, gleich $3$.
  Die Frage ist nun, wie man Binomialkoeffizienten allgemein berechnen kann. Diese Fragestellung gehört zur 
  Kombinatorik, einer mathematischen Disziplin, die sich unter anderem mit dem Abzählen von Objekten beschäftigt.
  }
  \lang{en}{As one can see in this iterative process, with the multiplication of the binomial $(a+b)$ with the previously calculated 
  power $(a+b)^{n-1}$, an extra factor of both $a$ and $b$ is introduced to each summand. Thus multiplying $(a+b)$ with itself n times yields 
  a sum of terms of the form $a^{n}b^{0}, a^{n-1}b^{1},...,a^{1}b^{n-1},a^{0}b^{n}$. Let us denote the coefficient of the term $a^{k}b^{n-k}$ by the expression $\binom{n}{k}$.
   These coefficients are called the \notion{binomial coefficients}. For $n=3$, for example, the binomial coefficient of $a^{1}b^{2}$ is denoted 
  $\binom{3}{1}$ and, as one can see from the above formula for $(a+b)^3$, is equal to $3$. Now one may ask how to calculate general binomial coefficients. 
  The answer lies in combinatorics, a mathematical discipline which concerns itself with the counting of objects, among other things,}
  
\begin{remark}  \label{rem:pascal_dreieck}
\lang{de}{Das Problem der Berechnung von Binomialkoeffizienten kann mit einem äquivalenten kombinatorischen Problem verglichen werden, 
das anschaulicher ist, nämlich dem Problem der Konstruktion eines Pascalschen Dreiecks.}
\lang{en}{The problem of calculating binomial coefficients can be compared to an equivalent combinatorial problem, namely the construction of Pascal's triangle.}

\begin{center}
\image{T201_PascalsTriangle_A}
\end{center}
\end{remark}

\begin{proof*}[\lang{de}{Erklärung}\lang{en}{Explanation}]
\begin{showhide}
\lang{de}{Betrachten Sie dazu das Dreieck in der Abbildung. 
Die Frage, die es zu beantworten gibt, ist: Wie viele Wege führen von dem einzelnen Punkt an der Spitze des Dreiecks zu einem beliebigen 
Punkt im Dreieck. Ein Weg zu einem Zielpunkt startet von dem einzelnen Punkt an der Spitze und geht immer nach links- oder rechtsunten vom 
aktuellen Punkt, bis er am Zielpunkt ankommt.}
\lang{en}{In order to do this, consider the triangle in the diagram. The question to be answered is: how many ways lead from the single point
at the top of the triangle to an arbitrary point in the triangle. A 'way' to a target point begins from the single point at the top, and always 
goes down and either to the left or to the right at each step, until it reaches the target point.}

\begin{center}
\image{T201_PascalsTriangle_B}
\end{center}

\lang{de}{Wenn man an die Position jedes Punktes die Anzahl der Wege schreibt, die zu ihm führen, erhält man das sogenannte 
Pascalsche Dreieck. Dabei gilt der Punkt an der Spitze des Dreiecks als der Weg zu sich selbst.}
\lang{en}{If one counts at the position of each point how many different ways exist to reach it, one gets Pascal's triangle. 
  In doing so, we take the point at the top of the triangle to have a single 'way' to itself.}

\begin{center}
\image{T201_PascalsTriangle_C}
\end{center}

\lang{de}{Das Problem der Anzahl der Wege vom Punkt an der Spitze des Dreiecks zu einem Punkt im Dreieck ist äquivalent zum Problem der 
Berechnung der Binomialkoeffizienten, denn:}
\lang{en}{The problem of counting the ways from the point at the top of the triangle to a point in the triangle is equivalent to the
problem of calculating binomial coefficients, because:}
\begin{itemize}

 
 \item \lang{de}{Bei der Berechnung der Binomialkoeffizienten müssen wir bei jeder Multiplikation des Binoms $(a+b)$ mit 
    der Summenformel aus $(a+b)^{n-1}$ (gemäß dem obigen Beispiel) entscheiden, ob wir $a$ oder $b$ anwenden, 
    d.h. ob wir den einzelnen Summanden mit dem Faktor $a$ oder dem Faktor $b$ multiplizieren. }
    \lang{en}{When calculating binomial coefficients, we need to decide every time we multiply $(a+b)$ with the summation formula for $(a+b)^{n-1}$ (as in the example above)
    whether we multiply each term by $a$ or by $b$.}

 \item \lang{de}{Beim Zählen der Wege im Dreieck müssen wir bei jeder Bewegung entscheiden, ob wir nach links- oder nach 
    rechtsunten gehen.}
    \lang{en}{When counting the ways in the triangle, we need to decide with each downward move whether we move to the left or to the right.}

\end{itemize} 
\end{showhide}
\end{proof*}

\lang{de}{Der Ausdruck $\binom{n}{k}$ gibt also die Anzahl der Wege vom Punkt an der Spitze des Dreiecks zu dem $k$-ten Punkt von links in der 
$n$-ten Ebene des Pascalschen Dreiecks, wobei $n,k\in\Nzero=\{0;1;2;\ldots\}$. Das Pascalsche Dreieck sieht dann folgendermaßen aus:}
\lang{en}{Therefore the expression $\binom{n}{k}$ gives the number of ways from the point at the top of the triangle to the $k$th point in the $n$th row of Pascal's triangle, 
where $n,k\in\Nzero=\{0;1;2;\ldots\}$. Pascal's triangle then looks as follows:}

\begin{center}
\image{T201_PascalsTriangle_D}
\end{center}

\begin{remark}

  \lang{de}{Betrachten wir die Anzahl der Wege von der Spitze zu einem bestimmten Punkt im Pascalschen Dreieck, ist  
  festzustzellen, dass dieser Punkt ausschließlich über die maximal zwei darüberliegende Punkte erreicht 
  werden kann. Damit ist die Anzahl verschiedener Wege von der Spitze zu diesem Punkt gleich der Summe der 
  jeweils verschiedenen Wege von der Dreiecks-Spitze zu den darüberliegenden Punkten.
  Aus dieser Überlegung und unter Berücksichtigung der Bemerkung \ref{rem:pascal_dreieck} 
  kann man für $n,k\in\Nzero=\{0;1;2;\ldots\}$ ableiten:}
  \lang{en}{If we look at the number of ways to reach a certain point in Pascal's triangle from the top, 
  it is clear that this point can only be reached from the at most two points above it. Thus the number of different ways to reach this point 
  is the sum of the number of ways to reach the points above it. Considering this and remark \ref{rem:pascal_dreieck}, one can deduce that for $n,k\in\Nzero=\{0;1;2;\ldots\}$:}


\begin{itemize}

 \item $\binom{0}{0}=1$,
 \item $\binom{n}{k}=0$ \lang{de}{für}\lang{en}{for} $k>n$,
 \item $\binom{n}{k}=\binom{n-1}{k-1}+\binom{n-1}{k}$.

\end{itemize}
\end{remark}

\lang{de}{Im Folgenden werden sowohl die genaue Definition von $\binom{n}{k}$ als auch die oben genannten Eigenschaften als Ergebnisse seiner 
Definition vorgestellt.}
\lang{en}{In the following section we will establish the precise definition of $\binom{n}{k}$, as well as proving the above properties as consequences of this definition.}

\section{\lang{de}{Fakultät und Binomialkoeffizienten}\lang{en}{Factorials and binomial coefficients}}


\lang{de}{In der Mathematik und vor allem in der Kombinatorik und der Wahrscheinlichkeitstheorie kommt bei Rechnungen 
oft das Produkt der ersten $n$ natürlichen Zahlen vor,
also z.B. $1\cdot 2\cdot 3\cdot 4$. Deshalb gibt es für solche Produkte eine Kurzschreibweise, die \emph{Fakultät}.}
\lang{en}{In mathematics, and especially in combinatorics and probability theory, the producct of the first $n$ natural numbers often comes up, for example $1\cdot 2\cdot 3\cdot 4$.
 Because of this, such products have been given a shorthand notation- the factorial.}

\begin{definition}[\lang{de}{Fakultät}\lang{en}{Factorial}]\label{def:fakultaet}
\lang{de}{F"ur $n\in\Nzero$ definieren wir}
  \lang{en}{For $n\in\Nzero=\{0,1,2,\ldots\}$ we define}

    \[ n! \coloneq \begin{cases}1\cdot2\cdot3\cdot\ldots\cdot(n-1)\cdot n= \prod_{k=1}^n\,k \quad \phantom{m} &  \text{\lang{de}{f"ur}\lang{en}{for}} \;\;n\geq 1,\\
    1 & \text{\lang{de}{f"ur}\lang{en}{for}}\;\; n=0.
    \end{cases} \]
\end{definition}

\begin{example}
 \begin{enumerate}
       \item $3!=1\cdot2\cdot3=6\, , \quad 2!=1\cdot 2=2$.
       \item $7!= \prod_{k=1}^7\,k=1\cdot2\cdot3\cdot 4\cdot 5 \cdot 6 \cdot 7=5040.$ 
    \end{enumerate}
 \end{example}

 \begin{remark}\label{rem:fakultaet}
    \begin{enumerate}
       \item \lang{de}{F"ur alle $n\in\Nzero$ gilt die Gleichung}
       \lang{en}{For all $n\in\Nzero$ we have}
    \[ (n+1)! = n!\cdot (n+1) \; .
    \]
	\lang{de}{Dadurch erhält man eine sogenannte \emph{rekursive Definition} der Fakultät $(n+1)!$, \\
    d.h. eine Definition, die sich auf die Definition für kleinere Werte bezieht.}
    \lang{en}{From this, one obtains a so-called \emph{recursive definition} of the factorial $(n+1)!$, i.e. a definition which relates to the definition for smaller values.}
    \item
    \lang{de}{Die Konvention $0!=1$ ist sehr zweckm"a"sig. Mit ihr gelten die Rekursion sowie die Formel mit dem
    Produktzeichen ab $n=0$.}
    \lang{en}{The convention $0!=1$ is very convenient. Using it, the recursion and the formula with the product sign hold starting 
    from $n=0$.}
  
    \end{enumerate}
 \end{remark}

\begin{rule}\label{rule:fakultaet-kombinatorisch}
\lang{de}{Die Zahl $\, n! \, $ ist genau die Anzahl an Möglichkeiten, $n$ verschiedene Objekte in eine Reihenfolge zu bringen.}
\lang{en}{The number $\, n! \, $ is exactly the number of ways one can order $n$ different objects.}
\end{rule}

\begin{proof*}[Erklärung]
\lang{de}{Will man $n$ verschiedene Objekte reihen, so hat man $n$ Möglichkeiten, welches Objekt man an den Anfang stellt. Für die zweite Stelle
hat man dann stets die Auswahl aus den $(n-1)$ verbleibenden Objekten. Für die dritte Stelle sind es dann noch $(n-2)$ Möglichkeiten und so weiter,
bis man für die vorletzte Stelle noch aus $2$ Objekten wählen kann, und für die letzte Stelle noch $1$ Objekt bleibt.

Die Gesamtanzahl an Möglichkeiten ist also}
\lang{en}{If one wants to put $n$ different objects in order, then one has $n$ possible choices for which will be the first object. 
For the second position, one then chooses from the remaining $(n-1)$ objects. For the third position there remain $(n-2)$ possibilities and so on, 
until for the penultimate position one has $2$ objects left to choose from, and the final position is then determined, as only $1$ object remains.

The total number of ways to order them is then}
\[ n\cdot (n-1)\cdot (n-2)\cdots 2\cdot 1= n! \,.\]
\end{proof*}




 \begin{definition}[\lang{de}{Binomialkoeffizient}\lang{en}{Binomial coefficients}]\label{def:Binomcoeff}
  \lang{de}{Sei $n,k\in\Nzero$ und $n\geq k$. Die Zahl}
  \lang{en}{Let $n,k\in\Nzero$ and $n\geq k$. The number}
\[  \binom{n}{k}\coloneq\frac{n!}{k!(n-k)!} \]
  \lang{de}{hei"st \notion{\emph{Binomialkoeffizient}} und wird "`$n$ "uber $k$"' ausgesprochen. }
  \lang{en}{is called a \notion{binomial coefficient}, and is read as '$n$ choose $k$'.}
%  \lang{de}{Er kommt in der Mathematik an vielen Stellen vor, besonders in der Kombinatorik und 
%    Wahrscheinlichkeitstheorie. }    
  \end{definition}

\lang{de}{Der Binomialkoeffizient kommt in der Mathematik an vielen Stellen vor, besonders in der Kombinatorik und Wahrscheinlichkeitstheorie.}
 \lang{en}{Binomial coefficients appear in many places of mathematics, particularly in probability theory and combinatorics.}

\begin{example} \label{ex:spezialfaelle}
\begin{enumerate}
\item 
\[ \binom{3}{2}=\frac{3!}{2!(3-2)!}=\frac{6}{2\cdot 1}=3,\quad  \binom{4}{2}=\frac{4!}{2!(4-2)!}=\frac{24}{2\cdot 2}=6,\quad 
\binom{5}{3}=\frac{5!}{3!(5-3)!}=\frac{120}{6\cdot 2}=10, \] 
\item \lang{de}{für alle $n\in\Nzero$ ist}\lang{en}{for all $n\in\Nzero$ we have} \[ \binom{n}{0}=\frac{n!}{0!\cdot n!}=1\quad \text{und} \quad \binom{n}{n}=\frac{n!}{n!\cdot 0!}=1, \]
\item {für alle $n\in\Nzero$ ist}\lang{en}{for all $n\in\Nzero$ we have}  \[\binom{n}{1}=\frac{n!}{1!\cdot (n-1)!}=n\quad \text{und} \quad \binom{n}{n-1}=\frac{n!}{(n-1)!\cdot 1!}=n.\]
\end{enumerate}
\end{example}

\begin{theorem}
\lang{de}{Für $n,k\in\Nzero$ mit $n\geq k\;$ gibt der Binomialkoeffizient $\binom{n}{k}$ die Anzahl der Teilmengen mit 
$k$ Elementen an, die aus einer 
    $n-$elementigen Menge ausgew"ahlt werden k"onnen.
    
    Insbesondere ist der Binomialkoeffizient stets eine ganze Zahl.}
\lang{en}{For $n,k\in\Nzero$ with $n\geq k\;$, the binomial coefficient $\binom{n}{k}$ gives the number of $k$-element subsets of a $n$-element set.

In particular, binomial coefficients are always whole numbers.}
\end{theorem}

\begin{proof*}[\lang{de}{Erklärung}\lang{en}{Explanation}]
\lang{de}{Es gibt stets genau eine Teilmenge mit $0$ Elementen, nämlich die leere Menge. Für beliebiges $n\in \Nzero$ und 
$k=0$ ist daher die Anzahl der Teilmengen mit $0$ Elementen stets $1$. 
Ebenso ist $\binom{n}{0}=\frac{n!}{0!\cdot (n-0)!}=\frac{n!}{n!}=1$.

Nehmen wir also nun an, dass $k>0$ ist. Wir versuchen nun die $k-$elementige Teilmengen $\{a_1;a_2;\ldots; a_k\}$  einer $n$-elementigen Menge $M$ zu zählen. Dafür muss man
die Anzahl der Möglichkeiten zählen: Für $a_1$ hat man $n$ Möglichkeiten (jedes beliebige Element von $M$), für $a_2$ noch $n-1$ Möglichkeiten 
(jedes beliebige Element von $M$ außer $a_1$), für $a_3$ noch $n-2$ Möglichkeiten, \ldots und für $a_k$ noch $n-k+1$ Möglichkeiten.

Dies ergibt ingesamt $n\cdot (n-1)\cdots (n-k+1)$ Möglichkeiten. Jedoch wurden hier Teilmengen mehrfach gezählt, da die Reihenfolge der Elemente in der Teilmenge
keine Rolle spielt. Da die Anzahl der Möglichkeiten, die $k$ Elemente zu reihen, genau $k!$ ist, 
wurde also jede Teilmenge ebenso oft gezählt, weshalb die tatsächliche Anzahl an $k$-elementigen Teilmengen genau}
\lang{en}{There is always exactly one subset with $0$ elements, namely the empty set. For any $n\in \Nzero$ and $k=0$ the number of $0$-element subsets is therefore $1$. Indeed, we have  
$\binom{n}{0}=\frac{n!}{0!\cdot (n-0)!}=\frac{n!}{n!}=1$.

Let us now suppose that $k>0$. We will count the $k$-element subsets $\{a_1;a_2;\ldots; a_k\}$ of an $n$-element set $M$. To do this, we must count the number of possible choices: For $a_1$ 
there are $n$ choices (any element of $M$ may be chosen), for $a_2$ there are $n-1$ choices (any element of $M$ aside from $a_1$), for $a_3$ there are $n-2$ choices, ... and for $a_k$ there are $n-k+1$ choices.

This gives a total of $n\cdot (n-1)\cdots (n-k+1)$ possible choices. However, we counted some subsets more than once, since the order of the elements in a set does not matter.
Since the number of ways to order $k$ elements is exactly $k!$, each subset was counted $k!$ times. Thus the actual count of $k$-element subsets is}

\[ \frac{n\cdot (n-1)\cdots (n-k+1)}{k!}=\frac{n\cdot (n-1)\cdots (n-k+1)\cdot (n-k)!}{k!(n-k)!}=\frac{n!}{k!(n-k)!}=\binom{n}{k} \lang{en}{.}\]
\lang{de}{ist.}
\end{proof*}



\section{\lang{de}{Binomialidentitäten und der binomische Lehrsatz}\lang{en}{Binomial identities and the binomial theorem}}

\lang{de}{Die Binomialkoeffizienten erfüllen verschiedene Gleichungen. Einige davon werden hier aufgeführt.}
\lang{en}{The binomial coefficients satisfy several equations. Some of them will be presented here.}

\begin{theorem}\label{thm:identitaeten}
\begin{enumerate}
\item \lang{de}{Für $n,k\in\Nzero$ mit $k\leq n$ gilt $\binom{n}{k}=\binom{n}{n-k}$.}\lang{en}{For $n,k\in\Nzero$ with $k\leq n$ we have $\binom{n}{k}=\binom{n}{n-k}$.}
\item \lang{de}{Für $n,k\in\Nzero$ mit $k\leq n-1$ gilt: $\binom{n}{k}+\binom{n}{k+1} = \binom{n+1}{k+1}$.}\lang{en}{For $n,k\in\Nzero$ with $k\leq n-1$ we have: $\binom{n}{k}+\binom{n}{k+1} = \binom{n+1}{k+1}$.}
\end{enumerate}
\end{theorem}

\begin{proof*}[\lang{de}{Erklärung}\lang{en}{Explanation}]
\lang{de}{Die erste Gleichung sieht man durch Verwendung von $k=n-(n-k)$ direkt aus der Definition:}
\lang{en}{The first equality can be seen from applying $k=n-(n-k)$ directly to the definition:}
\[ \binom{n}{k}=\frac{n!}{k!(n-k)!}=\frac{n!}{(n-k)!k!}=\binom{n}{n-k}. \]
\lang{de}{Für die zweite Gleichung berechnet man:}
\lang{en}{For the second equality we calculate:}
\begin{eqnarray*}
\binom{n}{k}+\binom{n}{k+1} &=& \frac{n!}{k!(n-k)!}+\frac{n!}{(k+1)!(n-(k+1))!} \\
&=& \frac{n!}{k!(n-k-1)!\cdot (n-k)}+\frac{n!}{(k+1)\cdot k!(n-k-1)!} \\
&=& \frac{(k+1)\cdot n! }{(k+1)\cdot k!(n-k-1)!\cdot (n-k)}+\frac{n!\cdot (n-k)}{(k+1)\cdot k!(n-k-1)!\cdot (n-k)} \\
&=& \frac{n!\cdot \big((k+1)+(n-k)\big)}{(k+1)!(n-k)!}=\frac{(n+1)!}{(k+1)!\big((n+1)-(k+1)\big)!}\\
&=& \binom{n+1}{k+1}
\end{eqnarray*}
\end{proof*}


\begin{theorem}[\lang{de}{Binomischer Lehrsatz}\lang{en}{Binomial theorem}]\label{thm:binom}
\lang{de}{F"ur alle  $n\in\Nzero$ und alle reellen Zahlen $a,\, b$, gilt die
folgende Gleichung:}
\lang{en}{For all $n\in\Nzero$ and for all real numbers $a,\, b$, the following
equation holds:}

\[
\quad(a+b)^n=\sum^n_{k=0}\,\binom{n}{k}\; a^{n-k}\,b^k.
\]
\lang{de}{Hierbei werden $a^0$, $b^0$ und $(a+b)^0$ stets gleich $1$ gesetzt, selbst wenn $a$, $b$ oder $a+b$ gleich $0$ sind.}
\lang{en}{Here we always set $a^0$, $b^0$ and $(a+b)^0$ to be equal to $1$, even when $a$, $b$ or $a+b$ are equal to $0$.}
\end{theorem}

\begin{proof*}[\lang{de}{Erklärung}\lang{en}{Explanation}]
\lang{de}{Mit Hilfe obiger Gleichungen lässt sich diese Aussage gut mit der Beweismethode durch \link{vollst-ind}{vollständige Induktion} zeigen.
Hierfür ist zu zeigen, dass für alle $n\in\Nzero$ die Aussage }
\lang{en}{With the help of the above equations, this proposition can be shown using \link{vollst-ind}{mathematical induction}. We must show that for all $n\in\Nzero$, the equality }
\[  A(n):\quad(a+b)^n=\sum^n_{k=0}\binom{n}{k}\; a^{n-k}\,b^k \]
\lang{de}{gilt.}
\lang{en}{holds.}
 \begin{enumerate}
    \item[\textbf{(\lang{de}{IA}\lang{en}{BC})}] $n=0$: \lang{de}{Die linke Seite ist $(a+b)^0$ also per Definition $=1$. Die rechte Seite ist}
    \lang{en}{The left side is $(a+b)^0$ so by definition $=1$. The right side is}
    \[ \sum^0_{k=0}\,\binom{0}{k}\; a^{0-k}\,b^k= \binom{0}{0}\; a^{0}\,b^0=1\cdot 1\cdot 1=1. \]
    \item[\textbf{(IS)}] $n\rightarrow n+1$: \lang{de}{F"ur ein festes $n\in\Nzero$ setzen wir voraus:}
    \lang{en}{For a fixed $n\in\Nzero$ we suppose that:}
    \begin{center}\textbf{(\lang{de}{IV}\lang{en}{IH})} \lang{de}{Es gelte} $A(n):\quad(a+b)^n=\sum^n_{k=0}\,\binom{n}{k}\; a^{n-k}\,b^k$ \lang{en}{ holds},\end{center}
    \lang{de}{und zeigen, dass f"ur dieses $n$ die Aussage }
    \lang{en}{and we show that for this $n$ the proposition }
    \[ A(n+1): \quad(a+b)^{n+1}=\sum^{n+1}_{k=0}\binom{n+1}{k}\; a^{n+1-k}\,b^k\]
    \lang{de}{ gilt:}
    \lang{en}{ holds:}
    \begin{align*}
      (a+b)^{n+1} &= (a+b)^n\cdot (a+b) &\\
      &= \left( \sum^n_{k=0}\,\binom{n}{k}\; a^{n-k}\,b^k \right)\cdot (a+b)  \;\;\; &\vert \text{nach (IV)}\\
      &= \sum^n_{k=0}\,\binom{n}{k}\; a^{n+1-k}\,b^k \,+\, \sum^{n}_{k=0}\,\binom{n}{k}\; a^{n-k}\,b^{k+1} & \\
      &= \sum^n_{k=0}\,\binom{n}{k}\; a^{n+1-k}\,b^k \,+\, \sum^{n+1}_{j=1}\,\binom{n}{j-1}\; a^{n-(j-1)}\,b^{j} \;\;\; &\vert \text{mittels } j=k+1\\
	  &=  \binom{n}{0}\; a^{n+1}\,b^0 + \sum^n_{i=1}\,\left( \binom{n}{i}+ \binom{n}{i-1} \right) \; a^{n+1-i}\,b^i \,+\, \binom{n}{n+1-1}\;\; 
	  a^{0}\,b^{n+1}  \;\;\; &\vert \text{durch Umsortieren}\\
	  &= \binom{n+1}{0}\;\; a^{n+1}\,b^0+ \sum^n_{i=1}\,\binom{n+1}{i}\;\; a^{n+1-i}\,b^i \,+\, \binom{n+1}{n+1}\;\; 
	  a^{0}\,b^{n+1}  \;\;\; &\vert \text{wegen Bsp.} \ref{ex:spezialfaelle} \text{ u. Satz} \ref{thm:identitaeten}  \\
	  &= \sum^{n+1}_{i=0}\,\binom{n+1}{i}\;\; a^{n+1-i}\,b^i&
    \end{align*}
  \end{enumerate}
\end{proof*}

%Video
\lang{de}{Das folgende Video enthält einen weiteren Beweis für den binomischen Lehrsatz.

\floatright{\href{https://api.stream24.net/vod/getVideo.php?id=10962-2-10953&mode=iframe&speed=true}{\image[75]{00_video_button_schwarz-blau}}}\\
}


\begin{example}
\lang{de}{Für kleine $n$ erhält man durch Berechnen der Binomialkoeffizienten dann folgende Formeln}
\lang{en}{For a small $n$ calculating the binomial coefficients gives the following formulas}
\[  \begin{mtable}
    (a+b)^0 & = & 1\\
    (a+b)^1 & = & 1a+1b\\
    (a+b)^2 & = & 1a^2+2ab+1b^2\\
    (a+b)^3 & = & 1a^3+3a^2b+3ab^2+1b^3\\
    (a+b)^4 & = & 1a^4+4a^3b+6a^2b^2+4ab^3+1b^4
  \end{mtable}\]
\lang{de}{Für $n=2$ ist dies natürlich die in der Schule gelernte binomische Formel.}
\lang{en}{For $n=2$ this is of course the binomial formula learned in school.}
\end{example}

\begin{rule}
\lang{de}{Setzt man in obigem Satz $a=b=1$ bzw. $a=-1$, $b=1$ ein, erhält man noch zwei wichtige Identitäten für Binomialkoeffizienten:}
\lang{en}{Setting $a=b=1$ in the above statement, or respectively $a=-1$, $b=1$, one obtains two more important identities for binomial coefficients:}
\begin{enumerate}
\item \lang{de}{Für alle $n\in \Nzero$ ist $ \sum^n_{k=0}\binom{n}{k}=2^n $.}
\lang{en}{For all $n\in \Nzero$ we have $ \sum^n_{k=0}\binom{n}{k}=2^n $.}
\item \lang{de}{Für alle $n\in \N$ ist $ \sum^n_{k=0}\binom{n}{k}\;(-1)^k=0$.}
\lang{en}{For all $n\in \N$ we have $ \sum^n_{k=0}\binom{n}{k}\;(-1)^k=0$.}

\end{enumerate}
\end{rule}
\end{content}
