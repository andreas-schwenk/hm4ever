\documentclass{mumie.element.exercise}
%$Id$
\begin{metainfo}
  \name{
    \lang{de}{Ü03: Fakultät}
    \lang{en}{Exercise 3}
  }
  \begin{description} 
 This work is licensed under the Creative Commons License Attribution 4.0 International (CC-BY 4.0)   
 https://creativecommons.org/licenses/by/4.0/legalcode 

    \lang{de}{Darstellung von Zahlen als Zweierpotenzen}
    \lang{en}{}
  \end{description}
  \begin{components}
  \end{components}
  \begin{links}
  \end{links}
  \creategeneric
\end{metainfo}
\begin{content}
\begin{block}[annotation]
	Im Ticket-System: \href{https://team.mumie.net/issues/22135}{Ticket 22135}
\end{block}
\begin{block}[annotation]
Copy of \href{http://team.mumie.net/issues/9761}{Ticket 9761}: content/rwth/HM1/T201_Vollstaendige_Induktion_wichtige_Ungleichungen/exercises/exe_exercise3.src.tex
\end{block}

\usepackage{mumie.ombplus}

\title{
  \lang{de}{Übung 3}
  \lang{en}{Exercise 3}
}





\lang{de}{ Vereinfachen Sie die folgenden Ausdrücke so weit wie möglich. Hierbei sei $n\in \N$
und es gelte $n>2$. \\


a) $(n-1)!\cdot n \cdot (n+1) \qquad $ b) $\displaystyle \frac{n!}{(n-2)!} \qquad $ c) $\displaystyle \frac{6!\cdot 7 \cdot 8}{4!}$. 

}

\begin{tabs*}[\initialtab{0}\class{exercise}]
\tab{\lang{de}{Antwort}}
  
 
    \lang{de}{a) $(n+1)! \qquad $ b) $(n-1)\cdot n  \qquad $c) $\,5\cdot6\cdot7\cdot8=1680$.}

    
  \tab{\lang{de}{Lösung a)}}
  

    \lang{de}{Nach Definition der Fakultät gilt
    \[ (n-1)! = 1\cdot 2 \cdots (n-2)\cdot (n-1). \]
    Damit erhalten wir durch Einsetzen in den zu vereinfachenden Ausdruck
    \[ (n-1)!\cdot n \cdot (n+1) =  1\cdot 2 \cdots (n-2)\cdot (n-1) \cdot n \cdot (n+1), \]
    was genau der Definition von $(n+1)!$ entspricht.}

  
  \tab{\lang{de}{Lösung b)}}
  

    \lang{de}{Wir schreiben gemäß der Definition der Fakultät zunächst wieder Zähler und Nenner des Bruches aus und erhalten
\[ \frac{n!}{(n-2)!} = \frac{1\cdot 2\cdots (n-2)\cdot (n-1) \cdot n}{1\cdot 2 \cdots (n-2)}. \]
Nun sind die ersten $(n-2)$ Faktoren in Zähler und Nenner identisch und können gekürzt werden. Damit ergibt sich 
\[ \frac{n!}{(n-2)!} = (n-1)\cdot n. \]
Hinweis: Beim Rechnen mit Fakultäten sollte man stets die Definition im Hinterkopf haben und bei Rechnungen stets versuchen durch Kürzen die auftretenden Ausdrücke zu vereinfachen, damit man unnötige Multiplikationen vermeidet.}

  
  \tab{\lang{de}{Lösung c)}}
  

    \lang{de}{Es ist $6!= 1\cdot 2 \cdot 3 \cdot 4 \cdot 5 \cdot 6$ und analog $ 4!= 1\cdot 2 \cdot 3 \cdot 4$. Damit gilt mit Kürzen der auftretenden gleichen Faktoren schließlich
    \[ \frac{6! \cdot 7 \cdot 8}{4!}  
      = \frac{1\cdot 2 \cdot 3 \cdot 4 \cdot 5 \cdot 6\cdot 7 \cdot 8}{1\cdot 2 \cdot 3 \cdot 4} 
      = 5\cdot 6 \cdot 7 \cdot 8 = 1680. \]}

  

\end{tabs*}
\end{content}