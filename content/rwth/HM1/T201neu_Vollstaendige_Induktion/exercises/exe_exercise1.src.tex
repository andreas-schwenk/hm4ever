\documentclass{mumie.element.exercise}
%$Id$
\begin{metainfo}
  \name{
    \lang{de}{Ü01: Beweisverfahren}
    \lang{en}{E01: Proof}
  }
  \begin{description} 
 This work is licensed under the Creative Commons License Attribution 4.0 International (CC-BY 4.0)   
 https://creativecommons.org/licenses/by/4.0/legalcode 

    \lang{de}{Darstellung von Zahlen als Zweierpotenzen}
    \lang{en}{}
  \end{description}
  \begin{components}
  \end{components}
  \begin{links}
    \link{generic_article}{content/rwth/HM1/T201neu_Vollstaendige_Induktion/g_art_content_01_indirekter_widerspruchsbeweis.meta.xml}{link-thm:equiv-to-implication}
  \end{links}
  \creategeneric
\end{metainfo}
\begin{content}
\begin{block}[annotation]
	Im Ticket-System: \href{https://team.mumie.net/issues/22130}{Ticket 22130}
\end{block}
\usepackage{mumie.ombplus}

\title{
  \lang{de}{Ü01: Beweis}
  \lang{en}{E01: Proof}
}






  \lang{de}{F"ur eine reelle Zahl $x$ sollen folgende Aussagen betrachtet werden.}
  
  \begin{table}[\class{items}]
    \nowrap{a) $x^3+x-3>0 \Rightarrow x>0$}\\
    \nowrap{b) $x\in \N \wedge x^2-2x-3=0 \Rightarrow x=3$ }
  \end{table}
  
  \lang{de}{Formulieren Sie die Aussagen so um, wie man sie für einen indirekten Beweis 
  beziehungsweise für einen Widerspruchsbeweis benötigen würde.}
  
  \begin{tabs*}[\initialtab{0}\class{exercise}]
    \tab{
      \lang{de}{Antwort}
      } 
    
    \lang{de}{Für indirekten Beweis:}
    
    \begin{table}[\class{items}]
      \nowrap{a) $x\leq 0 \Rightarrow\, x^3+x-3\leq 0$}\\
      \nowrap{b) $x\neq 3 \Rightarrow\, x\notin \N \vee x^2-2x-3\neq 0$}
    \end{table}
    
    \lang{de}{Für Widerspruchsbeweis:}
    
    \begin{table}[\class{items}]
      \nowrap{a) $\neg(x^3+x-3>0 \wedge x\leq 0)$}\\
      \nowrap{b) $\neg(x\in \N \wedge x^2-2x-3=0 \wedge x\neq 3)$} 
    \end{table}

  \tab{\lang{de}{Lösung a)}}

       
      \lang{de}{Nach  \ref[link-thm:equiv-to-implication][Satz über die äquivalenten Implikationen]{thm:equiv-to-implication} muss für einen indirekten Beweis die Implikation $A\Rightarrow B$ durch die Implikation 
      $\neg B\Rightarrow \neg A$ ersetzt werden, für einen Widerspruchsbeweis durch die Implikation $\neg(A\wedge \neg B)$.}
      
 
      \lang{de}{Sei nun $A:x^3+x-3>0$ und $B:x>0$, dann ist $\neg A:x^3+x-3\leq 0$ und $\neg B:x\leq 0$. 
      Im indirekten Beweis wäre also zu zeigen
      \begin{center} $\underbrace{x\leq 0}_{\neg B} \Rightarrow\, \underbrace{x^3+x-3\leq 0}_{\neg A}$. \end{center} 
      Im Widerspruchsbeweis muß gezeigt werden
      \begin{center} $\neg(\underbrace{x^3+x-3>0}_{A} \wedge \underbrace{x\leq 0}_{\neg B}).$ \end{center}
      }
      
  \tab{\lang{de}{Lösung b)}}
      \lang{de}{Nach  \ref[link-thm:equiv-to-implication][Satz über die äquivalenten Implikationen]{thm:equiv-to-implication} muss für einen indirekten Beweis die Implikation $A\Rightarrow B$ durch die Implikation 
      $\neg B\Rightarrow \neg A$ ersetzt werden, für einen Widerspruchsbeweis durch die Implikation $\neg(A\wedge \neg B)$.}
  
      \lang{de}{Sei $A:x\in \N \wedge x^2-2x-3=0$ und $B:x\notin \N \vee x^2-2x-3\neq 0$, dann ist 
      $\neg A:x\notin \N \vee x^2-2x-3\neq 0\,$ und $\neg B:x\neq 3$. 
      Im indirekten Beweis wäre also zu zeigen
      \begin{center} $\underbrace{x\neq 3}_{\neg B} \Rightarrow\, \underbrace{x\notin \N \vee x^2-2x-3\neq 0\,}_{\neg A}$. \end{center} 
      Im Widerspruchsbeweis muß gezeigt werden
      \begin{center} $\neg(\underbrace{x\in \N \wedge x^2-2x-3=0 }_{A} \wedge \underbrace{x\neq 3}_{\neg B}).$ \end{center}
      }
  \end{tabs*}
\end{content}