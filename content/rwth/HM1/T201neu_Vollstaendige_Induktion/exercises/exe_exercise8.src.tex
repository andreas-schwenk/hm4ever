\documentclass{mumie.element.exercise}
%$Id$
\begin{metainfo}
  \name{
    \lang{de}{Ü08: Identitäten von Binomialkoeffizienten}
    \lang{en}{Exercise 8}
  }
  \begin{description} 
 This work is licensed under the Creative Commons License Attribution 4.0 International (CC-BY 4.0)   
 https://creativecommons.org/licenses/by/4.0/legalcode 

    \lang{de}{}
    \lang{en}{}
  \end{description}
  \begin{components}
  \end{components}
  \begin{links}
  \end{links}
  \creategeneric
\end{metainfo}
\begin{content}
\begin{block}[annotation]
	Im Ticket-System: \href{https://team.mumie.net/issues/22131}{Ticket 22131}
\end{block}
\begin{block}[annotation]
Copy of \href{http://team.mumie.net/issues/9766}{Ticket 9766}: content/rwth/HM1/T201_Vollstaendige_Induktion_wichtige_Ungleichungen/exercises/exe_exercise8.src.tex
\end{block}

\usepackage{mumie.ombplus}

\title{
  \lang{de}{Ü08: Identitäten von Binomialkoeffizienten}
  \lang{en}{Exercise 8}
}

\lang{de}{ Beweisen Sie:
\begin{enumerate}[alph]
\item Für alle $k, n \in \N_0$ mit $k \leq n$ gilt $\quad (n + 1) \binom{n}{k} = (n - k + 1) \binom{n + 1}{k}$.
\item Für alle $k, n \in \N$ mit $k \leq n$ gilt $\quad \displaystyle \frac{\binom{n}{k}}{\binom{n}{k-1}} = \frac{n - k + 1}{k}$.
\end{enumerate} }

\begin{tabs*}[\initialtab{0}\class{exercise}]
  \tab{\lang{de}{Lösung a)}}
  

    \lang{de}{Wir setzen die Definition des Binomialkoeffizienten ein und rechnen durch geschicktes Zusammenfassen die Identität nach. Im Einzelnen bedeutet das
\[ (n+1) \binom{n}{k} = (n+1) \frac{n!}{k!(n-k)!} = \frac{(n+1)n!}{k!(n-k)!} = \frac{(n+1)!}{k!(n-k)!} \quad (\ast). \]
An dieser Stelle bietet es sich an von der rechten Seite der Gleichung aus zunächst umzuformen. 
\[ (n-k+1) \binom{n+1}{k} = (n-k+1) \frac{(n+1)!}{k!(n+1-k)!} = \frac{(n-k+1)(n+1)!}{k!(n-k)!(n-k+1)} 
= \frac{(n+1)!}{k!(n-k)!} ,
\]
was genau $(\ast)$ entspricht. Damit ist die Identität bewiesen.}

  
  \tab{\lang{de}{Lösung b)}}
  

    \lang{de}{Analog zu Aufgabenteil a) setzen wir die Definitionen ein und vereinfachen den Ausdruck, z.B. durch Kürzen so weit wie möglich.
Wir erhalten 
\begin{align*} 
\displaystyle \frac{\binom{n}{k}}{\binom{n}{k-1}} = \frac{\frac{n!}{k!(n-k)!} }{\frac{n!}{(k-1)!(n-(k-1))!}} &= \frac{\cancel{n!}}{k!(n-k)!} \cdot \frac{(k-1)!(n-k+1)!}{\cancel{n!}} \\
&= \frac{\cancel{(k-1)!} \cancel{(n-k)!} (n-k+1)  }{k\cdot \cancel{(k-1)!} \cancel{(n-k)!} } = \frac{n-k+1}{k}.
\end{align*}
}


\end{tabs*}
\end{content}