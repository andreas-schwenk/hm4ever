\documentclass{mumie.element.exercise}
%$Id$
\begin{metainfo}
  \name{
    \lang{de}{Ü04: Induktion}
    \lang{en}{Exercise 4}
  }
  \begin{description} 
 This work is licensed under the Creative Commons License Attribution 4.0 International (CC-BY 4.0)   
 https://creativecommons.org/licenses/by/4.0/legalcode 

    \lang{de}{Darstellung von Zahlen als Zweierpotenzen}
    \lang{en}{}
  \end{description}
  \begin{components}
  \end{components}
  \begin{links}
  \end{links}
  \creategeneric
\end{metainfo}
\begin{content}
\begin{block}[annotation]
	Im Ticket-System: \href{https://team.mumie.net/issues/22134}{Ticket 22134}
\end{block}
\begin{block}[annotation]
Copy of \href{http://team.mumie.net/issues/9762}{Ticket 9762}: content/rwth/HM1/T201_Vollstaendige_Induktion_wichtige_Ungleichungen/exercises/exe_exercise4.src.tex
\end{block}

\usepackage{mumie.ombplus}

\title{
  \lang{de}{Ü04: Induktion}
  \lang{en}{Exercise 4}
}






\lang{de}{ Man beweise für $n\in \N$ die Aussage
\[ A(n) : \ \sum_{i=1}^n i^3 = \left( \frac{n(n+1)}{2} \right)^2. \]
 }

 \begin{tabs*}[\initialtab{0}\class{exercise}]
  \tab{\lang{de}{Lösung}}
  

    \lang{de}{
      \begin{enumerate}
        \item[(IA)] $n=1:$ Die Summe besteht nur aus einem Summanden, es ist nämlich $\sum_{i=1}^1 i^3 = 1^3=1$. \\
                    Für die rechte Seite gilt 
                    \[ \left( \frac{1(1+1)}{2} \right)^2 = \left( \frac{2}{2} \right)^2 = 1^2=1. \]
                    Damit ist $A(1): 1=1$ wahr. \\
        \item[(IV)] Es gelte $A(n)$ für ein beliebiges aber festes $n\in\N$. \\
        \item[(IS)] $n\to n+1:$ 

                  Wir müssen $A(n+1)$ zeigen, unter der Voraussetzung, dass die Induktionsvoraussetzung $A(n)$ gilt. 
                  Zu zeigen ist also
                  \[ \sum_{i=1}^{n+1} i^3 = \left( \frac{(n+1)(n+2)}{2} \right)^2. \]
                  Wir beginnen mit der Umformung der linken Seite der Gleichung: \\
                  \begin{align*}
                   \sum_{i=1}^{n+1} i^3 \; &= \sum_{i=1}^n i^3 + (n+1)^3 \quad  & \vert \text{ Herausziehen des letzten Summanden } \\
                   &= \left( \frac{n(n+1)}{2} \right)^2 + (n+1)^3 \quad   & \vert \text{ nach Induktionsvoraussetzung } \\
                   &=  \frac{(n(n+1))^2}{4} + \frac{4(n+1)^3}{4}   &  \\
                   &= \frac{n^2(n+1)^2 + 4(n+1)^3}{4}              &  \\
                   &= \frac{(n+1)^2(n^2+4(n+1))}{4} \quad          & \vert \text{ durch Ausklammern von } (n+1)^2 \\
                   &= \frac{(n+1)^2(n+2)^2}{4} \quad               & \vert \text{ nach 1. Binomischer Formel } \\
                   &= \left( \frac{(n+1)(n+2)}{2} \right)^2 \quad  & \vert \text{ Anwendung der Potenzgesetze } 
                  \end{align*}

      \end{enumerate}            

      Das Ergebnis entspricht der rechten Seite der Gleichung, was zu zeigen war.
      }


\end{tabs*}
\end{content}