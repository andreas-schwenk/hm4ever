\documentclass{mumie.element.exercise}
%$Id$
\begin{metainfo}
  \name{
    \lang{de}{Ü07: Binomialkoeffizienten}
    \lang{en}{Exercise 7}
  }
  \begin{description} 
 This work is licensed under the Creative Commons License Attribution 4.0 International (CC-BY 4.0)   
 https://creativecommons.org/licenses/by/4.0/legalcode 

    \lang{de}{}
    \lang{en}{}
  \end{description}
  \begin{components}
  \end{components}
  \begin{links}
  \end{links}
  \creategeneric
\end{metainfo}
\begin{content}
\begin{block}[annotation]
	Im Ticket-System: \href{https://team.mumie.net/issues/22129}{Ticket 22129}
\end{block}
\begin{block}[annotation]
Copy of \href{http://team.mumie.net/issues/9765}{Ticket 9765}: content/rwth/HM1/T201_Vollstaendige_Induktion_wichtige_Ungleichungen/exercises/exe_exercise7.src.tex
\end{block}

\usepackage{mumie.ombplus}

\title{
  \lang{de}{Ü07: Binomialkoeffizienten}
  \lang{en}{Exercise 7}
}


\lang{de}{ Geben Sie die Werte der folgenden Binomialkoeffizienten an. Berechnen Sie die Lösung möglichst geschickt. 
\begin{center}
 a) $\binom{3}{1}$, \qquad b) $\binom{5}{0}$, \qquad c) $\binom{6}{3}$, \qquad d) $\binom{50}{49}$,  \qquad e) $\binom{0}{0}.$
\end{center} }

\begin{tabs*}[\initialtab{0}\class{exercise}]
  \tab{\lang{de}{Lösung a)}}
  
  \begin{incremental}[\initialsteps{1}]
    \step 
    \lang{de}{Wir kürzen wieder alle gleichen Faktoren in Zähler und Nenner und erhalten
\[ \binom{3}{1} = \frac{3!}{1!(3-1)!} = \frac{1\cdot 2 \cdot 3}{1\cdot 2} = 3. \]
}
  \end{incremental}
  
  \tab{\lang{de}{Lösung b)}}
  
  \begin{incremental}[\initialsteps{1}]
    \step 
    \lang{de}{Nach Definition ergibt sich direkt
\[ \binom{5}{0} =1 . \]}
  \end{incremental}
  
  \tab{\lang{de}{Lösung c)}}
  
  \begin{incremental}[\initialsteps{1}]
    \step 
    \lang{de}{Wir kürzen geschickt und erhalten
\[ \binom{6}{3} = \frac{6!}{3!(6-3)!} = \frac{6!}{3!\cdot3!} = \frac{4\cdot5\cdot\cancel{6}}{1\cdot\cancel{2\cdot3}} =20. 
\]}
  \end{incremental}
  
  \tab{\lang{de}{Lösung d)}}
  
  \begin{incremental}[\initialsteps{1}]
    \step 
    \lang{de}{Es gilt 
\[ \binom{50}{49} = \frac{50!}{49!(50-49)!} =\frac{50!}{49!} =50. \]
}
  \end{incremental}
  
  
    \tab{\lang{de}{Lösung e)}}
  
  \begin{incremental}[\initialsteps{1}]
    \step 
    \lang{de}{
  Mit der Definition ergibt sich direkt
\[ \binom{0}{0} =1 . \]
}
  \end{incremental}

\end{tabs*}


\end{content}