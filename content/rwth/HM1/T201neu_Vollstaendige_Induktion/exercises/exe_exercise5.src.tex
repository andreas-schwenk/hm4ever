\documentclass{mumie.element.exercise}
%$Id$
\begin{metainfo}
  \name{
    \lang{de}{Ü05: Induktion}
    \lang{en}{Exercise 5}
  }
  \begin{description} 
 This work is licensed under the Creative Commons License Attribution 4.0 International (CC-BY 4.0)   
 https://creativecommons.org/licenses/by/4.0/legalcode 

    \lang{de}{Darstellung von Zahlen als Zweierpotenzen}
    \lang{en}{}
  \end{description}
  \begin{components}
  \end{components}
  \begin{links}
  \end{links}
  \creategeneric
\end{metainfo}
\begin{content}
\begin{block}[annotation]
	Im Ticket-System: \href{https://team.mumie.net/issues/22139}{Ticket 22139}
\end{block}
\begin{block}[annotation]
Copy of \href{http://team.mumie.net/issues/9763}{Ticket 9763}: content/rwth/HM1/T201_Vollstaendige_Induktion_wichtige_Ungleichungen/exercises/exe_exercise5.src.tex
\end{block}

\usepackage{mumie.ombplus}

\title{
  \lang{de}{Ü05: Induktion}
  \lang{en}{Exercise 5}
}




\lang{de}{ Man beweise für $n\in \N$, $n\ge2$ die Aussage 
\[A(n) : \prod_{i=2}^n \left( 1- \frac{1}{i^2} \right) = \frac{n+1}{2n}. \] }

\begin{tabs*}[\initialtab{0}\class{exercise}]
  \tab{\lang{de}{Lösung}}
  
  \begin{incremental}[\initialsteps{1}]
    \step 
    \lang{de}{\textbf{(IA)} $n=2:$ Das Produkt besteht nur aus einem einzigen Faktor, es ist $\prod_{i=2}^2 \left( 1-\frac{1}{i^2} \right) = 1- \frac{1}{2^2} =1- \frac{1}{4} =\frac{3}{4}$. Auf der
rechten Seite ergibt sich für $n=2$ auch
\[ \frac{2+1}{2\cdot 2} = \frac{3}{4}, \]
damit ist $A(2): \frac{3}{4} = \frac{3}{4} $ wahr. \\
\textbf{(IV)} Es gelte $A(n)$ für ein beliebiges aber festes $n\in\N$. \\
\textbf{(IS)} $n\to n+1:$ Wir müssen $A(n+1)$, also
\[ \prod_{i=2}^{n+1} \left( 1- \frac{1}{i^2} \right) = \frac{n+2}{2(n+1)} \]
unter der Voraussetzung, dass $A(n)$ gilt, zeigen. Es gilt \\
\begin{align*}
 \prod_{i=2}^{n+1} \left( 1- \frac{1}{i^2} \right) &= \prod_{i=2}^n \left( 1- \frac{1}{i^2}\right)\cdot \left( 1-\frac{1}{(n+1)^2} \right) \quad \text{ letzten Faktor herausziehen } \\
 &= \frac{n+1}{2n}\cdot \left( 1- \frac{1}{(n+1)^2} \right) \quad \text{ nach Induktionsvoraussetzung } \\
 &= \frac{n+1}{2n} \cdot \left( \frac{(n+1)^2}{(n+1)^2} - \frac{1}{(n+1)^2} \right) \\
 &= \frac{n+1}{2n} \cdot \left( \frac{(n+1)^2-1}{(n+1)^2} \right) \\
 &= \frac{n+1}{2n} \cdot \left( \frac{n^2+2n}{(n+1)^2} \right) \\
 &= \frac{n^2+2n}{2n(n+1)} \quad \text{ nach Kürzen mit } (n+1) \\
 &= \frac{n(n+2)}{2n(n+1)} = \frac{n+2}{2(n+1)} \quad \text{ nach Kürzen mit } n .
\end{align*}
Damit folgt die Behauptung mit dem Prinzip der vollständigen Induktion. }
  \end{incremental}

\end{tabs*}
\end{content}