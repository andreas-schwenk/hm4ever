\documentclass{mumie.element.exercise}
%$Id$
\begin{metainfo}
  \name{
    \lang{de}{Ü02: Beweisverfahren}
    \lang{en}{E02: Proof}
  }
  \begin{description} 
 This work is licensed under the Creative Commons License Attribution 4.0 International (CC-BY 4.0)   
 https://creativecommons.org/licenses/by/4.0/legalcode 

    \lang{de}{Darstellung von Zahlen als Zweierpotenzen}
    \lang{en}{}
  \end{description}
  \begin{components}
  \end{components}
  \begin{links}
  \end{links}
  \creategeneric
\end{metainfo}
\begin{content}
\begin{block}[annotation]
	Im Ticket-System: \href{https://team.mumie.net/issues/22128}{Ticket 22128}
\end{block}
\usepackage{mumie.ombplus}

\title{
  \lang{de}{Ü02: Beweis}
  \lang{en}{E02: Proof}
}


  


  \lang{de}{Beweisen Sie, dass für jede beliebige reelle Zahl $x$ die Aussage
  \[   x>4 \Rightarrow x^2-4x> 0 \]
  gilt. Finden Sie einen direkten Beweis, einen indirekten und einen Widerspruchsbeweis.

  \textbf{Anmerkung:} Natürlich gibt es keine eindeutige Vorgehensweise für einen Beweis. Die unten
  angegebenen sind daher nur Beispielbeweise.}
   
  \begin{tabs*}[\initialtab{0}\class{exercise}]
    \tab{
      \lang{de}{Direkter Beweis}} 
    
    \begin{incremental}[\initialsteps{1}]
      \step 
      \lang{de}{Hier ist also die Implikation $x>4 \Rightarrow x^2-4x> 0$ direkt zu zeigen, indem man sie
      in "`kleinere"' Implikationen zerlegt, von denen man weiß, dass sie wahr sind.}
      
      \step 
      \lang{de}{Angenommen $x>4$, dann folgt insbesondere $x>0$. Also können beide Seiten der Ungleichung 
      mit x multipliziert werden, ohne dass sich das Ungleichheitszeichen umdreht. Daher folgt: 
      $x\cdot x>4\cdot x$, d.h. $x^2>4x$.}
      
      \step
      \lang{de}{Subtraktion von $4x$ auf beiden Seiten ergibt:}
      
      \[ x^2-4x>0.\]

      \lang{de}{\textbf{Anmerkung:} Formal geschrieben hat man also folgende Implikationskette benutzt:}
      
      \[ x>4\Rightarrow x>4\wedge x>0\Rightarrow x^2>4x \Rightarrow x^2-4x>0 \]  
    \end{incremental}

    \tab{
      \lang{de}{Indirekter Beweis}
      }
    
    \begin{incremental}[\initialsteps{1}]
      \step 
      \lang{de}{Für einen indirekten Beweis muss man also die Implikation $\neg (x^2-4x>0) \Rightarrow \neg (x>4)$
      zeigen. Die Negation von $x^2-4x>0$ ist 
      $x^2-4x\leq 0$ und die Negation von $x>4$ ist $x\leq 4$. Zu zeigen ist also}
      \[ x^2-4x\leq 0 \Rightarrow x\leq 4. \]

      \step 
      \lang{de}{Angenommen es gilt $x^2-4x\leq 0$. Wegen $x^2-4x=x\cdot (x-4)$ bedeutet dies:} 
      \[ x>4 \wedge x(x-4)\leq 0.\]
      
      \step 
      \lang{de}{Ein Produkt ist aber nur dann $\leq 0$, wenn ein Faktor $0$ ist, oder der eine Faktor positiv 
      und der andere
      Faktor negativ ist. Aus der Bedingung $x\cdot (x-4)\leq 0$ folgt also}
      \[ (x=0 \vee x-4=0) \vee (x>0 \wedge x-4<0) \vee (x<0 \wedge x-4>0). \]

      \step 
      \lang{de}{Die Bedingung $x<0 \wedge x-4>0$ ist aber unerfüllbar, d.h. immer falsch, also folgt}
      \[ (x=0 \vee x=4) \vee (x>0 \wedge x<4). \]

      \step 
      \lang{de}{Sowohl $x=0$, als auch $x=4$, als auch $0<x<4$ implizieren $x\leq 4$. Daher folgt $x\leq 4$, was
      gerade die Negation von $x>4$ ist.  

      \textbf{Anmerkung:} Formal geschrieben hat man also folgende Äquivalenz-/Implikationskette benutzt:}
      \begin{eqnarray*} 
        \neg (x^2-4x>0) &\Leftrightarrow& x^2-4x\leq 0\Leftrightarrow x\cdot (x-4)\leq 0\\ 
          &\Rightarrow& (x=0 \vee x-4=0) \vee (x>0 \wedge x-4<0) \vee (x<0 \wedge x-4>0)\\
          &\Leftrightarrow&  (x=0 \vee x-4=0) \vee (x>0 \wedge x-4<0) \vee \text{F}\\
          &\Leftrightarrow&  (x=0 \vee x=4) \vee (x>0 \wedge x<4)\\
          &\Rightarrow& x\leq 4\Leftrightarrow \neg (x>4)
      \end{eqnarray*}
      
      \end{incremental}
      \tab{
      \lang{de}{Widerspruchsbeweis}
      }

    \begin{incremental}[\initialsteps{1}]
      \step 
      \lang{de}{Für den Widerspruchsbeweis muss man die Annahme $x>4 \wedge \neg (x^2-4x>0)$ zu einem Widerspruch,
      also einer (offensichtlich) falschen Aussage führen.}
      
      \step 
      \lang{de}{Angenommen $x>4 \wedge x^2-4x\leq 0$. Wegen $x^2-4x=x\cdot (x-4)$ bedeutet dies:} 
      \[ x>4 \wedge x(x-4)\leq 0\]
      
      \step 
      \lang{de}{Ein Produkt ist aber $\leq 0$, wenn ein Faktor $0$ ist, oder der eine Faktor positiv und der andere
      Faktor negativ ist. Wegen $x>4$ ist $(x-4)$ positiv. Also folgt aus $x>4$ und $x(x-4)\leq 0$,
      dass $x\leq 0$ ist.}
      
      \step 
      \lang{de}{$x\leq 0$ steht aber im Widerspruch zur Annahme $x>4$ (denn diese beiden Bedingungen können nicht
      gleichzeitig erfüllt sein). Oder anders ausgedrückt:
      Die Bedingung $x>4 \wedge x\leq 0$ ist für jede Zahl $x$ eine falsche Aussage, also ist
      $x>4 \wedge x\leq 0$ ein Widerspruch.

      \textbf{Anmerkung:} Formal geschrieben hat man also folgende Äquivalenz-/Implikationskette benutzt:}
      \begin{eqnarray*} 
        x>4 \wedge \neg (x^2-4x>0) &\Leftrightarrow& x>4 \wedge x^2-4x\leq 0\\ 
          &\Leftrightarrow& x>4 \wedge x\cdot (x-4)\leq 0\\
          &\Rightarrow& x>4 \wedge x\leq 0\\
          &\Leftrightarrow& \text{F}
      \end{eqnarray*}
    \end{incremental}
  \end{tabs*}
\end{content}