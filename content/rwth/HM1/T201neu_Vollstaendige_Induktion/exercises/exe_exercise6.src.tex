\documentclass{mumie.element.exercise}
%$Id$
\begin{metainfo}
  \name{
    \lang{de}{Ü06: Induktion}
    \lang{en}{Exercise 6}
  }
  \begin{description} 
 This work is licensed under the Creative Commons License Attribution 4.0 International (CC-BY 4.0)   
 https://creativecommons.org/licenses/by/4.0/legalcode 

    \lang{de}{Darstellung von Zahlen als Zweierpotenzen}
    \lang{en}{}
  \end{description}
  \begin{components}
  \end{components}
  \begin{links}
  \end{links}
  \creategeneric
\end{metainfo}
\begin{content}
\begin{block}[annotation]
	Im Ticket-System: \href{https://team.mumie.net/issues/22137}{Ticket 22137}
\end{block}
\begin{block}[annotation]
Copy of \href{http://team.mumie.net/issues/9764}{Ticket 9764}: content/rwth/HM1/T201_Vollstaendige_Induktion_wichtige_Ungleichungen/exercises/exe_exercise6.src.tex
\end{block}

\usepackage{mumie.ombplus}

\title{
  \lang{de}{Ü06: Induktion}
  \lang{en}{Exercise 6}
}






\lang{de}{ Beweisen Sie für $n\in\N$, $n\ge 4$ die Aussage
\[ A(n): n \cdot \sqrt{n} > n + \sqrt{n}. \] }

\begin{tabs*}[\initialtab{0}\class{exercise}]
  \tab{\lang{de}{Lösung}}
  

    \lang{de}{
      \begin{enumerate}
        \item[\textbf{(IA)}] $n=4:$ Es ist $\; 4\cdot \sqrt{4} = 4\cdot 2 = 8 > 6 =4+2 = 4+\sqrt{4} .\;$ Damit ist $A(4)$ wahr. \\
        \item[\textbf{(IV)}] Es gelte $A(n): n \cdot \sqrt{n} > n + \sqrt{n} \;$ für ein beliebiges aber festes $n\in\N$. \\
        \item[\textbf{(IS)}] $n\to n+1:$ Wir müssen zeigen, dass unter dieser Voraussetzung auch $A(n+1)$ gilt, also 
                  \[ (n+1) \cdot \sqrt{n+1} > (n+1) + \sqrt{n+1}. \]
                  Dazu
                  \begin{align*}
                   (n+1) \cdot \sqrt{n+1} \; &= n\sqrt{n+1} + \sqrt{n+1}  & \\
                   &> n\sqrt{n} + \sqrt{n+1} \quad  & \vert \text{ da die }\sqrt{\;}\text{-Funktion monoton ist} \\
                   &> n+\sqrt{n} + \sqrt{n+1} \quad  & \vert \text{ nach Induktionsvoraussetzung }\\
                   &> n+1 + \sqrt{n+1} \quad         &  \vert \text{ da }\sqrt{n}>1 \;  \text{ für } \; n\ge4.
                  \end{align*}
                  Damit folgt die Behauptung mit dem Prinzip der vollständigen Induktion.
      \end{enumerate} 
      }


\end{tabs*}
\end{content}