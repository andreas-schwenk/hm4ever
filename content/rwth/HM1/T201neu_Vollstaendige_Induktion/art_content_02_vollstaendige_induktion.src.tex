%$Id:  $
\documentclass{mumie.article}
%$Id$
\begin{metainfo}
  \name{
    \lang{de}{Vollständige Induktion}
    \lang{en}{Mathematical Induction}
  }
  \begin{description} 
 This work is licensed under the Creative Commons License Attribution 4.0 International (CC-BY 4.0)   
 https://creativecommons.org/licenses/by/4.0/legalcode 

    \lang{de}{Beschreibung}
    \lang{en}{Description}
  \end{description}
  \begin{components}
    \component{generic_image}{content/rwth/HM1/images/g_img_00_video_button_schwarz-blau.meta.xml}{00_video_button_schwarz-blau}
  \end{components}
  \begin{links}
    \link{generic_article}{content/rwth/HM1/T101neu_Elementare_Rechengrundlagen/g_art_content_02_rechengrundlagen_terme.meta.xml}{terme-umformungen}
    \link{generic_article}{content/rwth/HM1/T201neu_Vollstaendige_Induktion/g_art_content_03_binomischer_lehrsatz.meta.xml}{fakultaet}

  \end{links}
  \creategeneric
\end{metainfo}


\begin{content}
\begin{block}[annotation]
	Im Ticket-System: \href{https://team.mumie.net/issues/22234}{Ticket 22234}
    
    Nachfolger von: \href{http://team.mumie.net/issues/9635}{Ticket 9635}\\
\end{block}

\usepackage{mumie.ombplus}
\ombchapter{1}
\ombarticle{2}

\lang{de}{\title{Beweise durch vollständige Induktion}}
\lang{en}{\title{Proof using mathematical induction}}


\begin{block}[info-box]
\tableofcontents
\end{block}


\section{\lang{de}{Das Induktionsprinzip}\lang{en}{The principle of induction}}\label{sec:induktionsprinzip}

\lang{de}{
Im Folgenden bezeichne $A(n)$ eine Aussage, die von einer nat"urlichen Zahl $n$ abh"angt.\\
Die \emph{Beweismethode der vollständigen Induktion} ist eine Methode, um Aussagen der Form
\[ \text{für alle } n\in \N: A(n) \]
zu beweisen, wenn ein direkter Beweis nur schwer (oder gar nicht) möglich ist.
Die Methode besteht aus zwei Teilen:
}
\lang{en}{
In the following let $A(n)$ denote a proposition which depends on the natural number $n$. Mathematical induction is a method of proof
used to prove statements of the form
\[ \text{for all } n\in \N: A(n) \]
when a direct proof is too difficult (or even impossible). The method consists of two parts:
}


\begin{enumerate}
    \item[\textbf{(\lang{de}{IA}\lang{en}{BC})}] \lang{de}{Dem \emph{Induktionsanfang} (oder \emph{Induktionsverankerung}), in dem gezeigt wird, dass $A(1)$ wahr ist.}
    \lang{en}{The \emph{base case} (or \emph{anchor step}), in which it is shown that $A(1)$ is true.}
  \item[\textbf{(IS)}] \lang{de}{Dem \emph{Induktionsschritt} (oder \emph{Induktionsschluss}), in dem gezeigt wird, dass für alle $n\in \N$
    die Implikation $A(n) \Rightarrow A(n+1)$ wahr ist.}
    \lang{en}{The \emph{induction step} (or \emph{inductive step}), in which it is shown that for all $n\in \N$ the implication $A(n) \Rightarrow A(n+1)$ is true.}
\end{enumerate}

\lang{de}{
Im Induktionsschritt heißt $A(n)$ die \emph{Induktionsvoraussetzung} \textbf{(IV)}, da sie die Voraussetzung der Implikation ist.
}
\lang{en}{
In the induction step $A(n)$ is called the inductive hypothesis, since it is the hypothesis of the implication.
}

\begin{remark}
  \lang{de}{
  Im \emph{Induktionsschritt} muss man also nicht zeigen, dass $A(n)$ wahr ist, sondern man zeigt, dass für alle natürlichen Zahlen
  $n$ die Implikation $A(n)
    \Rightarrow A(n+1)$ wahr ist. Man setzt also voraus, dass die Aussage $A(n)$ für eine beliebig gewählte natürliche Zahl $n$ wahr 
    ist, und zeigt dann, dass für genau dieses $n$ folglich auch die Aussage $A(n+1)$ wahr ist.\\
    'Beliebig' hei"st dabei nicht, dass man sich ein $n$ w"ahlen darf, sondern dass an die nat"urliche Zahl, f"ur die $n$ stehen
    soll, keine Einschr"ankungen gemacht wird.\\
    Wenn also die Wahrheit des \emph{Induktionsanfangs} $A(1)$ ebenfalls gezeigt wurde, kann man sagen, dass durch Anwendung des 
    \emph{Induktionsschritt} rekursiv gilt:
    }
    \lang{en}{
    In the induction step one must therefore not show that $A(n)$ is true, but rather that for all natural numbers $n$, the implication $A(n)
    \Rightarrow A(n+1)$ is true. Thus one assumes that $A(n)$ is true for some arbitrarily chosen $n$, and shows that for this $n$, the proposition $A(n+1)$ is true.\\
    'Arbitrary' here means not that one can specifically choose $n$, but rather that there should be no restrictions on the natural number that $n$ represents.\\
    When the truth of the \emph{base case} has also been shown, then one can say that by the recursive application of the \emph{induction step}, the following holds:
    }
    \[ A(1)\Rightarrow A(2)\Rightarrow A(3)\ldots. \]
\end{remark}

\lang{de}{
Das folgenden Video bietet Ihnen auch einen einführenden Blick auf das 
Induktionsverfahren:
 
%Video
\floatright{\href{https://api.stream24.net/vod/getVideo.php?id=10962-2-10952&mode=iframe&speed=true}{\image[75]{00_video_button_schwarz-blau}}}\\
\\ 
}

\begin{example}
  \lang{de}{
  Es soll gezeigt werden, dass für jede natürliche Zahl $n$ die Summe der ersten $n$ ungeraden Zahlen genau die Zahl
    $n^2$ gibt.
  Zu zeigen ist also für alle $n\in \N: A(n)$ wobei}
  \lang{en}{
  We will prove that for every natural number $n$, the sum of the first $n$ odd numbers is exactly $n^2$. This amounts to showing for all $n\in \N: A(n)$ where
  }
  \[ A(n):\quad 1+3+\ldots + (2n-1) =n^2, \]
  \lang{de}{oder mit Summenzeichen:}\lang{en}{or with summation notation:}
  \[ A(n):\quad \sum_{k=1}^n 2k-1=n^2. \]
  \lang{de}{Beweis:}\lang{de}{Proof:}

  \begin{enumerate}
    \item[\textbf{(\lang{de}{IA}\lang{en}{BC})}] $n=1: \;$  \lang{de}{Die Summe besteht nur aus einer Zahl, nämlich der $1$ und $n^2=1$, \\
        also ist die Aussage $A(1): \sum_{k=1}^1 2k-1=2-1=1$ wahr.}
        \lang{en}{The sum consists only of one number, namely $1$. Since $n^2=1$, the proposition $A(1): \sum_{k=1}^1 2k-1=2-1=1$ is true.}
    \item[\textbf{(IS)}] $n\rightarrow n+1: \;$ \\
        \lang{de}{Sei eine beliebige Zahl $n\in\N$ fest gewählt und gelte für diese}
        \lang{en}{Let an arbitrary number $n\in\N$ be held fixed, and suppose}
         \begin{center}

           \textbf{(\lang{de}{IV}\lang{en}{IH})}$\quad A(n): \quad \sum_{k=1}^n 2k-1=1+3+\ldots + (2n-1) =n^2, \qquad$ 
         \end{center}
        \lang{de}{dann ist zu zeigen, dass für gleiches $n$ auch }
        \lang{en}{is true. Then we must show that for the same $n$, }
        \begin{center}

          $\, A(n+1): \quad \sum_{k=1}^{n+1} 2k-1=1+3+\ldots + (2(n+1)-1) =(n+1)^2 \;$
        \end{center}
        \lang{de}{gilt.}
        \lang{en}{is true.}
        
        \begin{align*}

            \underbrace{1+3+\ldots + (2(n+1)-1)}_{= \sum_{k=1}^{n+1} 2k-1} \, &= \underbrace{1+3+\ldots +(2n-1)}_{=\sum_{k=1}^n 2k-1=\,n^2} + (2(n+1)-1) &\\
          &= n^2 + (2(n+1)-1)   & \vert \text{\lang{de}{(IV) eingesetzt}\lang{en}{using (IH)}} \\
          &= n^2 + 2n+2-1       & \\
          & =n^2+2n+1           &\vert \ref[terme-umformungen][\text{\lang{de}{1. bin. Formel}\lang{en}{1. bin. Identity}}]{rule:binomische_formeln} \\
          &= (n+1)^2            &
        \end{align*}
        
  \end{enumerate}

  \lang{de}{Im Induktionsschritt $n\rightarrow n+1$ wurde also gezeigt: Wenn (für eine beliebige, feste natürliche Zahl $n$) die 
  Aussage $A(n)$ wahr ist, dann ist auch die Aussage $A(n+1)$ wahr.}
  \lang{en}{In the induction step $n\rightarrow n+1$ we showed: If (for an arbitrary, fixed natural number $n$) the proposition $A(n)$ is true, then the proposition $A(n+1)$ is also true.}
\end{example}

\section{\lang{de}{Alternative Induktionsanfänge}\lang{en}{Alternative base cases}}

\begin{remark}
\lang{de}{Man kann die Beweismethode der vollständigen Induktion auch anwenden, wenn die Aussagen $A(n)$ nur für alle natürlichen Zahlen $n$
zu beweisen ist, die größer oder gleich einer festen Zahl $n_0$ sind. In diesem Fall muss man im Induktionsanfang dann die Aussage
$A(n_0)$ zeigen. Der Induktionsschritt ändert sich nicht.\\
Ebenso lässt sich die Methode anwenden, wenn $n_0$ eine beliebige ganze Zahl ist, und die Aussagen $A(n)$ für alle ganzen Zahlen $n\geq n_0$
zu beweisen sind.\\
Als häufigster Fall neben $n_0=1$ ist hier der Fall $n_0=0$.}
\lang{en}{One can also apply mathematical induction if the proposition $A(n)$ is only to be proved for all natural numbers $n$ which are greater or equal to some fixed number $n_0$.
In this case one has to show that the proposition $A(n_0)$ is true in the base step. The induction step is not different.\\
Likewise the method of proof can be used if $n_0$ is an arbitrary integer, and the propositions $A(n)$ are to be proved for all integers $n\geq n_0$.\\
The most common case besides $n_0=1$ is the case $n_0=0$.}
\end{remark}

\begin{example}\label{ex:vollst-ind}

\begin{tabs*}[\initialtab{0}]
\tab{$\sum_{k=0}^n k=\frac{n(n+1)}{2}$ für $n\geq 0$}
\lang{de}{Zu zeigen ist, dass für alle $n\in \N_0$ die Gleichung}
\lang{en}{We will show that for all $n\in \N_0$, the equation}
\[ A(n):\quad \sum_{k=0}^n k=\frac{n(n+1)}{2} \]
\lang{de}{gilt.\\ In diesem Fall ist der Induktionsanfang also für $n=0$ durchzuführen.}
\lang{en}{holds.\\ Here, the base case is to be checked for $n=0$.}
\begin{enumerate}
    \item[\textbf{(\lang{de}{IA}\lang{en}{BC})}] $n=0$: \lang{de}{Die linke Seite ist: $\sum_{k=0}^0 k = 0$ und die rechte Seite ist $\frac{0\cdot (0+1)}{2}=0$. Also
    ist die Gleichung für $n=0$ erfüllt.}
    \lang{en}{The left side is: $\sum_{k=0}^0 k = 0$ and the right side is $\frac{0\cdot (0+1)}{2}=0$. So the equation holds for $n=0$.}
    \item[\textbf{(IS)}] $n\rightarrow n+1$: \lang{de}{Man hat also die Induktionsvoraussetzung f"ur ein festes $n\in\N$:}
    \lang{en}{Now for a fixed $n\in\N$, we have the inductive hypothesis:}
    \begin{center}\textbf{(\lang{de}{IV}\lang{en}{IH})} \quad \lang{de}{Es gelte $\quad \sum_{k=0}^n k=\frac{n(n+1)}{2}\quad $}\lang{en}{$\quad \sum_{k=0}^n k=\frac{n(n+1)}{2}\quad $ holds}\end{center}
    \lang{de}{und muss f"ur dieses $n$ die Aussage $A(n+1):\sum_{k=0}^{n+1} k=\frac{(n+1)(n+1+1)}{2} $ zeigen:}
    \lang{en}{and using this, we must show that for this $n$, the proposition $A(n+1):\sum_{k=0}^{n+1} k=\frac{(n+1)(n+1+1)}{2} $ is true:}
    \begin{align*}
      \sum_{k=0}^{n+1} k &= \underbrace{(\sum_{k=0}^{n} k )}_{=\,\frac{n(n+1)}{2}}+ (n+1)  &\\
      &= \frac{n(n+1)}{2}+ (n+1) & \vert \text{\lang{de}{nach (IV)}\lang{en}{using (IH)}} \\
      &= (n+1)\cdot \frac{n}{2}+(n+1)\cdot 1 =(n+1)\cdot (\frac{n+2}{2}) &\\
      &= \frac{(n+1)(n+2)}{2}. &
    \end{align*}
 \end{enumerate}    


\tab{$2^n \leq n!$ für $n\geq 4$} 
\lang{de}{Zu zeigen ist, dass für alle $n\geq 4$ die Ungleichung}
\lang{en}{We will show that for all $n\geq 4$ the inequality}
\[ 2^n \leq n! \]
\lang{de}{gilt. (Für die Defintion von $n!$ siehe \ref[fakultaet][here]{def:fakultaet}.)\\ 
In diesem Fall ist also $n_0=4$ und für den Induktionsanfang ist die Ungleichung $2^4\leq 4!$ zu zeigen.}
\lang{en}{holds. (For the definition of $n!$, see \ref[fakultaet][here]{def:fakultaet}.)\\
In this example $n_0=4$, and for the base case we must show the inequality $2^4\leq 4!$.}
\begin{enumerate}
    \item[\textbf{(\lang{de}{IA}\lang{en}{BC})}] $n=4$: \lang{de}{Es sind $2^4=16$ und $4!=1\cdot 2\cdot 3\cdot 4=24$. Also ist  $2^4\leq 4!$.}
    \lang{en}{Since $2^4=16$ and $4!=1\cdot 2\cdot 3\cdot 4=24$, we have $2^4\leq 4!$.}
    \item[\textbf{(IS)}] $n\rightarrow n+1$: \lang{de}{Man hat also f"ur ein festes $n\geq 4$ die Induktionsvoraussetzung:}
    \lang{en}{For a fixed $n\geq 4$ we have the induction hypothesis:}
    \begin{center}\textbf{(\lang{de}{IV}\lang{en}{IH})} \quad \lang{de}{Es gelte} $\quad 2^n\leq n!\quad $\lang{en}{ holds}\end{center}
    \lang{de}{und muss f"ur dieses $n$ die Aussage $2^{n+1}\leq (n+1)! $ zeigen:}
    \lang{en}{and using this, we must show that for this $n$, the proposition $2^{n+1}\leq (n+1)! $ is true:}
    \begin{align*}
    2^{n+1} &= \underbrace{2^n}_{\leq n!}\cdot 2 &\\
    &\leq n! \cdot 2     & \quad\vert \text{\lang{de}{nach (IV)}\lang{en}{using (IH)}} \\
    &\leq n! \cdot (n+1) & \quad\vert \text{\lang{de}{da}\lang{en}{since} }2\leq n+1 \\
    &= (n+1)!            & \quad\vert \text{\lang{de}{nach}\lang{en}{by} } \ref[fakultaet][\text{\lang{de}{dieser Bemerkung.}\lang{en}{this remark.}}]{rem:fakultaet}
      \end{align*}
 \end{enumerate}
 %Video
%\lang{de}{
 \tab{\lang{de}{$2^n \geq n^2$ für $n\geq 4$}} 
\lang{de}{Im folgendem Video wird per Induktion bewiesen, dass für alle $n\geq 4$ die Ungleichung
 $2^n \geq n^2$ gilt.
\floatright{\href{https://api.stream24.net/vod/getVideo.php?id=10962-2-10954&mode=iframe&speed=true}{\image[75]{00_video_button_schwarz-blau}}}\\
\\
}
\end{tabs*}

\end{example}

\section{\lang{de}{Anwendungsbeispiele: Beweis wichtiger Formeln}\lang{en}{Examples of application: Proofs of important formulas}}\label{sec:wichtige-beispiele}

\lang{de}{In diesem Paragraphen werden einige wichtige Gleichungen und Ungleichungen vorgestellt.}
\lang{en}{In this section we will introduce some important equations and inequalities.}

\begin{rule}[\lang{de}{Geometrische Summenformel}\lang{en}{Geometric partial sum formula}]\label{rule:geom-summe}
\lang{de}{Für jede beliebige reelle Zahl $q\ne 1$ und $n\in \Nzero$ gilt:}
\lang{en}{For any arbitrary real number $q\ne 1$ and $n\in \Nzero$ we have:}
 \[  \sum_{k=0}^n q^k=\frac{q^{n+1}-1}{q-1} \]
\lang{de}{Hierbei wird $q^0=1$ gesetzt, und daher $\sum_{k=0}^n q^k=1+q+q^2+\cdots+q^n$, selbst wenn $q=0$ ist.
 Diese Formel ist die sogenannte \emph{geometrische Summenformel}.}
\lang{en}{Here we set $q^0=1$, so that $\sum_{k=0}^n q^k=1+q+q^2+\cdots+q^n$, even when $q=0$.
This formula is called the \emph{partial sum formula of the geometric series}.}
\end{rule} 

\begin{proof*}[\lang{de}{Begründung}\lang{en}{Proof}]
     \begin{showhide}
 \lang{de}{Zum Beweis durch vollständige Induktion betrachten wir zunächst den Induktionsanfang mit $n=0$.}
 \lang{en}{For a proof by mathematical induction, we begin with a base case of $n=0$.}
 \begin{enumerate}
    \item[\textbf{(\lang{de}{IA}\lang{en}{BC})}] $n=0$: \lang{de}{Die linke Seite ist $\sum_{k=0}^0 q^k = q^0=1$ und die rechte Seite ist $\frac{q^1-1}{q-1}=1$. Also
    ist die Gleichung für $n=0$ erfüllt.}
    \lang{en}{The left side is $\sum_{k=0}^0 q^k = q^0=1$ and the right side is $\frac{q^1-1}{q-1}=1$. So the equation is satisfied for $n=0$.}
    \item[\textbf{(IS)}] $n\rightarrow n+1$: \lang{de}{Man hat also die Induktionsvoraussetzung f"ur ein festes $n\in\N$:}
    \lang{en}{Now for a fixed $n\in\N$, we have the inductive hypothesis:}
    \begin{center}\textbf{(\lang{de}{IV}\lang{en}{IH})} \quad \lang{de}{Es gelte} $\quad \sum_{k=0}^n q^k=\frac{q^{n+1}-1}{q-1}\quad$ \lang{en}{ holds}\end{center}
    \lang{de}{und muss f"ur dieses $n$ die Aussage $A(n+1):\sum_{k=0}^{n+1} q^k=\frac{q^{n+2}-1}{q-1} $ zeigen:}
    \lang{en}{and using this, we must show that for this $n$, the proposition $A(n+1):\sum_{k=0}^{n+1} q^k=\frac{q^{n+2}-1}{q-1} $ is true:}
    \begin{align*}
      \sum_{k=0}^{n+1} q^k &= (\sum_{k=0}^{n} q^k )+ q^{n+1}  \\
      &= \frac{q^{n+1}-1}{q-1}+ q^{n+1} \quad \text{\lang{de}{nach (IV)}\lang{en}{using (IH)}} \\
      &= \frac{q^{n+1}-1}{q-1}+ \frac{q^{n+2}-q^{n+1}}{q-1} \quad \text{\lang{de}{(Erweitern des zweiten Summanden)}\lang{en}{(Expanding the second summand)}}\\
      &= \frac{q^{n+2}-1}{q-1}.
    \end{align*}
 \end{enumerate}
    \end{showhide}
\end{proof*}

%Video
\lang{de}{
Die Anwendung der vollständigen Induktion zur Begründung der geometrischen Summenformel wird hier auch vorgestellt:

\floatright{\href{https://api.stream24.net/vod/getVideo.php?id=10962-2-10957&mode=iframe&speed=true}{\image[75]{00_video_button_schwarz-blau}}}\\
\\
}


\begin{rule}[\lang{de}{Bernoullische Ungleichung}\lang{en}{Bernoulli's inequality}]\label{rule:bernoulli-ungleichung}
\lang{de}{Für alle $a\in \R$ mit $a\geq -1$ und $n\in \Nzero$ gilt:}
\lang{en}{For all $a\in \R$ with $a\geq -1$ and $n\in \Nzero$:}
\[ (1+a)^n\geq 1+na. \]
\lang{de}{Hierbei wird $(1+a)^0=1$ gesetzt, selbst wenn $a=-1$ ist.}
\lang{en}{Here we set $(1+a)^0=1$, even when $a=-1$.}

\lang{de}{Diese Ungleichung nennt man die \emph{Bernoullische Ungleichung}.}
\lang{en}{This inequality is called \emph{Bernoulli's inequality}.}
\end{rule} 

\begin{proof*}[\lang{de}{Begründung}\lang{en}{Proof}]
     \begin{showhide}
\lang{de}{ Zu zeigen ist für jede beliebige reelle Zahl $a\geq -1$, dass für alle $n\in \N_0$ die Ungleichung}
\lang{en}{ We will show that for any arbitrary real number $a\geq -1$, and for all $n\in N_0$, the inequality}
 \[ A(n):\quad (1+a)^n\geq 1+na \]
\lang{de}{ gilt. 
 Zum Beweis durch vollst"andige Induktion betrachten wir also zun"achst den Induktionsanfang mit $n=0$.}
\lang{en}{ holds.
For a proof by mathematical induction, we begin with a base case of $n=0$.}
 \begin{enumerate}
    \item[\textbf{(\lang{de}{IA}\lang{en}{BC})}] $n=0$: \lang{de}{Die linke Seite ist: $(1+a)^0=1$ und die rechte Seite ist $1+0\cdot a=1$. Also
    ist die Ungleichung für $n=0$ erfüllt.}
    \lang{en}{The left side is: $(1+a)^0=1$ and the right side is $1+0\cdot a=1$. So the inequality holds for $n=0$.}
    \item[\textbf{(IS)}] $n\rightarrow n+1$: \lang{de}{Man hat also die Induktionsvoraussetzung f"ur ein festes $n\in\N$:}
    \lang{en}{Now for a fixed $n\in\N$, we have the indctive hypothesis:}
    \begin{center}\textbf{(\lang{de}{IV}\lang{en}{IH})} \quad \lang{de}{Es gelte} $\quad (1+a)^n\geq 1+na\quad $\lang{en}{ holds}\end{center}
    \lang{de}{und muss f"ur dieses $n$ die Aussage $A(n+1): (1+a)^{n+1}\geq 1+(n+1)a$  zeigen:}
    \lang{en}{and using this, we must show that for this $n$, the proposition $A(n+1): (1+a)^{n+1}\geq 1+(n+1)a$  is true::}
    \begin{align*}
     (1+a)^{n+1} &= (1+a)^n\cdot (1+a) & \\
     &\geq (1+na)\cdot (1+a)  \quad & \,\,\,\text{\lang{de}{wegen (IV) und }\lang{en}{using (IH) and }}1+a\geq 0 \\
     &= 1\cdot 1+ 1\cdot a+ na\cdot 1 +na^2 & \,\,\,\text{\lang{de}{ausmultipliziert}\lang{en}{expanding}} \\
     &\geq 1+(n+1)a & \,\,\,\text{\lang{de}{wegen }\lang{en}{because }}na^2\geq 0.
    \end{align*}
 \end{enumerate}
    \end{showhide}
\end{proof*}

%video
\lang{de}{
Eine Begründung der \emph{Bernoullischen Ungleichung} finden Sie auch in diesem Video:
\floatright{\href{https://api.stream24.net/vod/getVideo.php?id=10962-2-10956&mode=iframe&speed=true}{\image[75]{00_video_button_schwarz-blau}}}\\
\\
}

\begin{rule}[\lang{de}{Ungleichung zwischen arithmetischem und geometrischem Mittel}\lang{en}{Inequality of arithmetic and geometric means}]\label{rule:means_ineq}
\lang{de}{Für alle $m\in \N$ und nicht-negativen Zahlen $x_1,\ldots, x_m\in \R$ gilt:}
\lang{en}{For all $m\in N$ and non-negative $x_1,\ldots, x_m\in \R$, the following inequality holds:}
\[    \sqrt[m]{x_1 \cdot x_2\cdot \ldots \cdot x_m}  \leq \frac{x_1+x_2+\ldots + x_m}{m}.  \]
\lang{de}{Gleichheit gilt genau dann, wenn $x_1=x_2=\ldots =x_m$ gilt.}
\lang{en}{With equality holding if and only if $x_1=x_2=\ldots =x_m$.}

\lang{de}{Diese Ungleichung nennt man die Ungleichung zwischen dem \emph{geometrischen Mittel} 
$\,\sqrt[m]{x_1 \cdot x_2\cdot \ldots \cdot x_m}\,$ und dem \emph{arithmetischen
Mittel} $\, \frac{x_1+x_2+\ldots + x_m}{m}$.}
\lang{en}{This is an inequality between the \emph{geometric mean} $\,\sqrt[m]{x_1 \cdot x_2\cdot \ldots \cdot x_m}\,$
 and the \emph{arithmetic mean} $\, \frac{x_1+x_2+\ldots + x_m}{m}$. It is often abbreviated and called the \emph{AM-GM inequality}.}
\end{rule}

\begin{proof*}[\lang{de}{Erklärung}\lang{en}{Explanation}]
     \begin{showhide}
\lang{de}{Der Beweis der allgemeinen Formel benötigt eine vollständige Induktion für die Aussage:}
\lang{en}{The proof for the general formula needs mathematical induction on the proposition:}
\[ A(n): \sqrt[m]{x_1 \cdot x_2\cdot \ldots \cdot x_m} \leq \frac{x_1+x_2+\ldots + x_m}{m} \quad \text{für alle }m\leq 
2n\text{ und }x_1,\ldots, x_m\in \R_+. \]
\lang{de}{Wir zeigen hier \textit{nur}, den \textbf{Induktionsanfang}, d.h. den Beweis der Aussage }
\lang{en}{Here we will \textit{only} show the \textbf{base case}, i.e. the proof of the proposition }
\[ A(1): \sqrt[m]{x_1 \cdot x_2\cdot \ldots \cdot x_m} \leq \frac{x_1+x_2+\ldots + x_m}{m} \quad \text{für alle }m\leq 
2\text{ und }x_1,\ldots, x_m\in \R_+. \]

\lang{de}{Für den \textbf{Induktionschritt} ($n\to n+1$) würde man zunächst die Ungleichung für $m=2(n+1)$ aus den Ungleichungen für $m=n+1$ und $m=2$ folgern 
und anschließend die Ungleichung für $m=2n+1$ aus der für $m=2(n+1)$.

Im \textbf{Induktionsanfang} sind also die Ungleichungen für $m=1$ und $m=2$ zu zeigen:

Für $m=1$ lautet die Ungleichung $\sqrt[1]{x_1}\leq \frac{x_1}{1}$, welche offensichtlich richtig ist.

Für $m=2$ lautet die Ungleichung $\sqrt[2]{x_1x_2}\leq \frac{x_1+x_2}{2}$. Da beide Seiten nicht-negativ sind, ist dies
äquivalent zu }
\lang{en}{For the \textbf{induction step} ($n\to n+1$) one would first derive the inequality for $m=2(n+1)$ from the inequality for $m=n+1$ and $m=2$, 
and then the inequality for $m=2n+1$ from the one for $m=2(n+1)$.

In the \textbf{base case} we therefore need to show the inequalities for $m=1$ and $m=2$:

For $m=1$ the inequality reads $\sqrt[1]{x_1}\leq \frac{x_1}{1}$, which is obviously true.

For $m=2$ the inequality reads $\sqrt[2]{x_1x_2}\leq \frac{x_1+x_2}{2}$. Since both sides are non-negative, this is equivalent to }


\begin{align*}
				 & x_1x_2  &\leq & \left( \frac{x_1+x_2}{2}\right)^2 \\
\Leftrightarrow\quad & x_1x_2  &\leq & \frac{x_1^2+2x_1x_2+x_2^2}{4} \\
 \Leftrightarrow\quad & 4x_1x_2 &\leq & x_1^2+2x_1x_2+x_2^2 \\
  \Leftrightarrow\quad & 0& \leq &x_1^2-2x_1x_2+x_2^2 \\
  \Leftrightarrow\quad & 0& \leq & (x_1-x_2)^2
\end{align*}
\lang{de}{ Da Quadrate immer $\geq 0$ sind, ist die letzte Zeile richtig, und da nur Äquivalenzumformungen gemacht wurden, auch die erste Ungleichung.\\
 Außerdem ist die Gleichheit in der ersten Ungleichung genau dann gegeben, wenn sie in der letzten gegeben ist, d.h. wenn $x_1-x_2=0$, d.h. $x_1=x_2$.}
\lang{en}{ Since squares are always $\geq 0$, the last line is true, and since all the lines are equivalent, so is the first inequality. Furthermore, 
equality holds in the first inequality if and only if it holds in the last inequality, i.e. when $x_1-x_2=0$, i.e. $x_1=x_2$.}
    \end{showhide}
\end{proof*}

%Video
\lang{de}{
In dem Video wird die vorhergehende Regel erklärt.
\floatright{\href{https://api.stream24.net/vod/getVideo.php?id=10962-2-10958&mode=iframe&speed=true}{\image[75]{00_video_button_schwarz-blau}}}\\
}

\end{content}