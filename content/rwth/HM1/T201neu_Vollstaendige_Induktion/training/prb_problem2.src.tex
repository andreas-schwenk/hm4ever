\documentclass{mumie.problem.gwtmathlet}
%$Id$
\begin{metainfo}
  \name{
    \lang{de}{A02: Fakultät}
    \lang{en}{input numbers}
  }
  \begin{description} 
 This work is licensed under the Creative Commons License Attribution 4.0 International (CC-BY 4.0)   
 https://creativecommons.org/licenses/by/4.0/legalcode 

    \lang{de}{Die Beschreibung}
    \lang{en}{}
  \end{description}
  \corrector{system/problem/GenericCorrector.meta.xml}
  \begin{components}
    \component{js_lib}{system/problem/GenericMathlet.meta.xml}{mathlet}
  \end{components}
  \begin{links}
  \end{links}
  \creategeneric
\end{metainfo}
\begin{content}
\begin{block}[annotation]
	Im Ticket-System: \href{https://team.mumie.net/issues/22136}{Ticket 22136}
\end{block}
\begin{block}[annotation]
Copy of \href{http://team.mumie.net/issues/9768}{Ticket 9768}: content/rwth/HM1/T201_Vollstaendige_Induktion_wichtige_Ungleichungen/training/prb_problem2.src.tex
\end{block}


\usepackage{mumie.genericproblem}
\usepackage{mumie.ombplus}


\lang{de}{
	\title{A02: Fakultät}
}

\lang{en}{
	\title{Problem 2}
}



\begin{problem}

		\begin{variables}
			
			\randint{n}{0}{6}
			\randint{k}{5}{15}
			\randint{j}{1}{3}
			\randint{l}{10}{15}


			\function{fa}{(n)!}
			\function[calculate]{sa}{(n)!}


			\function[calculate]{tmp2}{(k-j)}

			\function{fb}{((k)!)/((tmp2)!)}
			\function[calculate]{sb}{((k)!)/((k-j)!)}

			\function[calculate]{tmp}{(l-j)}

			\function{fc}{(((l)!)-((tmp)!))/((tmp)!)}
			\function[calculate]{sc}{(((l)!)-((l-j)!))/((l-j)!)}
			
		\end{variables}
		

\randomquestionpool{1}{3}
	\begin{question}
		\text{
			Berechnen sie den folgenden Zahlenwert!
		}
		
		\type{input.number} 
 
        \field{rational}
		
		\begin{answer}
			\text{$\var{fa} =$}
			\solution{sa}
            \explanation{Tipp: Verwenden Sie die Definition von Fakultät.}
		\end{answer}
		
		
	\end{question}
		
		\begin{question}
		\text{
			Berechnen sie den folgenden Zahlenwert!
		}
		
		\type{input.number} 

        \field{rational}
		
		\begin{answer}
			\text{$\var{fb} =$}
			\solution{sb}
            \explanation{Tipp: Beachten Sie, dass für natürliche Zahlen $k,n$\\
            $(k+n)! = k! \cdot (k+1)\cdot (k+1)\cdot \ldots \cdot (k+n)$\\
            gilt und nutzen Sie dies zum Kürzen des Bruchs.}
		\end{answer}
		
		
	\end{question}
	
		\begin{question}
		\text{
			Berechnen sie den folgenden Zahlenwert!
		}
		
		\type{input.number} 
 
        \field{rational}
		
		\begin{answer}
			\text{$\var{fc} =$}
			\solution{sc}
            \explanation{Tipp: Beachten Sie, dass für natürliche Zahlen $k,n$\\
            $(k+n)! = k! \cdot (k+1)\cdot (k+1)\cdot \ldots \cdot (k+n)$\\
            gilt und nutzen Sie dies zum Kürzen des Bruchs.}
		\end{answer}
		
		
	\end{question}	
		
\end{problem}

\embedmathlet{mathlet}


\end{content}
