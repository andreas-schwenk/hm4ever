\documentclass{mumie.problem.gwtmathlet}
%$Id$
\begin{metainfo}
  \name{
    \lang{de}{A04: Summen}
    \lang{en}{input numbers}
  }
  \begin{description} 
 This work is licensed under the Creative Commons License Attribution 4.0 International (CC-BY 4.0)   
 https://creativecommons.org/licenses/by/4.0/legalcode 

    \lang{de}{Die Beschreibung}
    \lang{en}{}
  \end{description}
  \corrector{system/problem/GenericCorrector.meta.xml}
  \begin{components}
    \component{js_lib}{system/problem/GenericMathlet.meta.xml}{mathlet}
  \end{components}
  \begin{links}
  \end{links}
  \creategeneric
\end{metainfo}
\begin{content}
\begin{block}[annotation]
	Im Ticket-System: \href{https://team.mumie.net/issues/22127}{Ticket 22127}
\end{block}
\begin{block}[annotation]
Copy of \href{http://team.mumie.net/issues/9770}{Ticket 9770}: content/rwth/HM1/T201_Vollstaendige_Induktion_wichtige_Ungleichungen/training/prb_problem4.src.tex
\end{block}


\usepackage{mumie.genericproblem}
\usepackage{mumie.ombplus}


\lang{de}{
	\title{A04: Binomialkoeffizienten}
}

\lang{en}{
	\title{Problem 4}
}



\begin{problem}
\randomquestionpool{1}{4}
	\begin{question}
	\text{Berechnen Sie den folgenden Zahlenwert.}
	
	\begin{variables}
		\randint{q}{-6}{6}
		\randadjustIf{q}{q=0}
		\randint{n}{0}{2}
		\function[calculate]{f}{(q^(n+1)-1)/(q-1)}
		\function{qk}{q^k}
	\end{variables}
	
	\type{input.number} 
        \field{rational}
		
		\begin{answer}
			\text{ $\sum_{k=0}^\var{n} \var{qk} =$}
			\solution{f}
            \explanation{Tipp: Verwenden Sie die Geometrische Summenformel.}
		\end{answer}
	
	\end{question}
	
	\begin{question}
	\text{Berechnen Sie den folgenden Zahlenwert.}
	
	\begin{variables}
		\randint{qh}{-6}{6}
		\randadjustIf{qh}{qh=0 OR qh=2}
		\function[calculate]{q}{qh/2}
		\randint{n}{3}{4}
		\function[calculate]{f}{(q^(n+1)-1)/(q-1)}
		\function{qk}{q^k}
	\end{variables}
	
	\type{input.number} 
        \field{rational}
		
		\begin{answer}
			\text{ $\sum_{k=0}^\var{n} \var{qk} =$}
			\solution{f}
            \explanation{Tipp: Verwenden Sie die Geometrische Summenformel.}
		\end{answer}
	
	\end{question}
	
		\begin{question}
	\text{Berechnen Sie den folgenden Zahlenwert.}
	
	\begin{variables}
		\randint{qh}{-4}{4}
		\randadjustIf{qh}{qh=0 OR qh=2}
		\function[calculate]{q}{qh/2}
		\randint{n}{5}{9}
		\function[calculate]{f}{(q^(n+1)-1)/(q-1)}
		\function{qk}{q^k}		
	\end{variables}
	
	\type{input.number} 
        \field{rational}
		
		\begin{answer}
			\text{ $\sum_{k=0}^\var{n} \var{qk} =$}
			\solution{f}
            \explanation{Tipp: Verwenden Sie die Geometrische Summenformel.}
		\end{answer}
	
	\end{question}
	
		
			\begin{question}
	\text{Berechnen Sie den folgenden Zahlenwert.}
	
	\begin{variables}
		\randint{n}{0}{9}
		\function[calculate]{f}{2^n}
	\end{variables}
	
	\type{input.number} 
        \field{rational}
		
		\begin{answer}
			\text{ $ \sum_{k=0}^\var{n} \binom{\var{n}}{k} =$ }
			\solution{f}
            \explanation{Tipp: Verwenden Sie die Regel über binomische Koeffizienten.}
		\end{answer}
	
	\end{question}
	
\end{problem}

\embedmathlet{mathlet}


\end{content}