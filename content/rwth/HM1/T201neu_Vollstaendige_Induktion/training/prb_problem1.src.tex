\documentclass{mumie.problem.gwtmathlet}
%$Id$
\begin{metainfo}
  \name{
    \lang{de}{A01: Summen}
    \lang{en}{input numbers}
  }
  \begin{description} 
 This work is licensed under the Creative Commons License Attribution 4.0 International (CC-BY 4.0)   
 https://creativecommons.org/licenses/by/4.0/legalcode 

    \lang{de}{Die Beschreibung}
    \lang{en}{}
  \end{description}
  \corrector{system/problem/GenericCorrector.meta.xml}
  \begin{components}
    \component{js_lib}{system/problem/GenericMathlet.meta.xml}{mathlet}
  \end{components}
  \begin{links}
  \end{links}
  \creategeneric
\end{metainfo}
\begin{content}
\begin{block}[annotation]
	Im Ticket-System: \href{https://team.mumie.net/issues/22133}{Ticket 22133}
\end{block}

\usepackage{mumie.genericproblem}
\usepackage{mumie.ombplus}


\lang{de}{
	\title{A01: Binomialkoeffizintien}
}


\begin{problem}

	\begin{question}
	
		\text{
			Berechnen Sie die folgenden Zahlenwerte und vereinfachen Sie Ihr Ergebnis 
			so weit wie möglich. (Fassen sie also vollständig zusammen!)
			}
			
			
  			
		\type{input.number} 
        \field{rational}
		
		\begin{variables}
			
			\randint{ma}{4}{10}
			\randint{na}{1}{4}
			\randint{za}{1}{5}
			
			\randint{nb}{2}{4}
			\randint{mb}{4}{6}
			\randint{zb}{2}{4}
            

					
			\function[calculate]{fa}{ma*(ma*(ma+1) - na*(na-1))/2 + ma*za*(ma-na+1)}
			\function[calculate]{fb}{(zb^(mb+1)-zb^(nb))/(zb-1)}
			
		\end{variables}
		
		
		
		\begin{answer}
			\text{$\sum_{j=\var{na}}^\var{ma}  \var{ma} \cdot (j+\var{za}) = $}
			\solution{fa}
            \explanation{Beachten Sie, dass nach Indexverschiebung die Anwendung der 
                        geometrischen Summenformel möglich ist.
            \begin{align*}
                \sum_{j=a}^{b} q\cdot (j+p) &= (q\cdot\sum_{j=a}^{b}j) + (q\cdot p\cdot \sum_{j=a}^{b} 1)  &\\
                &= q\cdot(\sum_{j=0}^{b}j - \sum_{i=0}^{a-1}i) + q\cdot p\cdot \sum_{j=a}^{b} 1 & \\ % \vert \text{Indexverschiebung nach Satz 1.2.26, Kap. 1.1.}\\
                &= q\cdot(\frac{b\cdot (b+1)}{2} - \frac{(a-1)\cdot a}{2}) + q\cdot p\cdot(b-a+1),&
           \end{align*}}
		\end{answer}
		
		
		\begin{answer}
			\text{$\sum_{l=\var{nb}}^\var{mb} \var{zb}^l = $}
			\solution{fb}
            \explanation{Beachten Sie, dass nach Indexverschiebung die Anwendung der 
                        geometrischen Summenformel möglich ist.
            \begin{align*}
                \sum_{l=a}^{b} q^l &= \sum_{l=0}^{b} q^l - \sum_{k=0}^{a-1} q^k  & \\
                &= \frac{q^{b+1}-1}{q-1}-\frac{q^{a}-1}{q-1} &\\
                &= \frac{q^{b+1}-q^{a}}{q-1}.&
           \end{align*}}
		\end{answer}
		
		
		
	\end{question}

\end{problem}

\embedmathlet{mathlet}


\end{content}