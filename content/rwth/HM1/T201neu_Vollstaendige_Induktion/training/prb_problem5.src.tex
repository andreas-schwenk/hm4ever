\documentclass{mumie.problem.gwtmathlet}
%$Id$
\begin{metainfo}
  \name{
    \lang{de}{A05: Binomialkoeffizienten}
    \lang{en}{input numbers}
  }
  \begin{description} 
 This work is licensed under the Creative Commons License Attribution 4.0 International (CC-BY 4.0)   
 https://creativecommons.org/licenses/by/4.0/legalcode 

    \lang{de}{Die Beschreibung}
    \lang{en}{}
  \end{description}
  \corrector{system/problem/GenericCorrector.meta.xml}
  \begin{components}
    \component{js_lib}{system/problem/GenericMathlet.meta.xml}{mathlet}
  \end{components}
  \begin{links}
  \end{links}
  \creategeneric
\end{metainfo}
\begin{content}

\usepackage{mumie.genericproblem}
\usepackage{mumie.ombplus}


\lang{de}{
	\title{A05: Binomialkoeffizienten}
}

\lang{en}{
	\title{Problem 5}
}

\begin{block}[annotation]
  Im Ticket-System: \href{http://team.mumie.net/issues/9771}{Ticket 9771}
\end{block}

\begin{problem}

\randomquestionpool{1}{7}
	\begin{question}
	
		\text{Bestimmen Sie den Zahlenwert:}
		\type{input.number} 
        \precision{3}
        \field{rational}
		
		\begin{variables}
			
			\randint{n}{2}{10}
			\randint{l}{2}{20}
			
			\function[calculate]{x}{0.5*n^2+0.5*n} 
			
			\function[calculate]{a}{n+1}
			
		\end{variables}
	
		\begin{answer}
			\text{$\binom{\var{a}}{2}=$}
			\solution{x}
            \explanation{Tipp: Verwenden Sie Definition von Binomialkoeffizienten.}  
		\end{answer}
	
	
	\end{question}
	
	\begin{question}
	
		\text{Bestimmen Sie den Zahlenwert:}
		\type{input.number} 
        \precision{3}
        \field{rational}
		
		\begin{variables}
			
			\randint{n}{2}{10}
			\randint{l}{2}{20}
			
			\function[calculate]{x}{l} 
			
			\function[calculate]{a}{l-1}
			
		\end{variables}
	
		\begin{answer}
			\text{$\binom{\var{l}}{\var{a}}=$}
			\solution{x}
            \explanation{Tipp: Verwenden Sie Definition des Binomialkoeffizienten.}  
		\end{answer}
	
	
	\end{question}
	
	\begin{question}
	
		\text{Bestimmen Sie den Zahlenwert:}
		\type{input.number} 
        \precision{3}
        \field{rational}
		
		\begin{variables}
			
			\randint{n}{2}{10}
			\randint{l}{2}{20}
			
			\function[calculate]{x}{1} 
			
			
		\end{variables}
	
		\begin{answer}
			\text{$\binom{\var{l}}{\var{l}}=$}
			\solution{x}
            \explanation{Tipp: Verwenden Sie Definition des Binomialkoeffizienten.}  
		\end{answer}
	
	
	\end{question}
	
	\begin{question}
	
		\text{Bestimmen Sie den Zahlenwert:}
		\type{input.number} 
        \precision{3}
        \field{rational}
		
		\begin{variables}
			
			\randint{n}{2}{10}
			\randint{l}{2}{20}
			
			\function[calculate]{x}{1} 
			
		\end{variables}
	
		\begin{answer}
			\text{$\binom{\var{l}}{0}=$}
			\solution{x}
            \explanation{Tipp: Verwenden Sie Definition des Binomialkoeffizienten.}  
		\end{answer}
	
	
	\end{question}
	
	\begin{question}
	
		\text{Bestimmen Sie den Zahlenwert:}
		\type{input.number} 
        \precision{3}
        \field{rational}
		
		\begin{variables}
			
			\randint{n}{2}{10}
			\randint{l}{2}{20}
			
			\function[calculate]{x}{l} 
			
		\end{variables}
	
		\begin{answer}
			\text{$\binom{\var{l}}{1}=$}
			\solution{x}
            \explanation{Tipp: Verwenden Sie Definition des Binomialkoeffizienten.}  
		\end{answer}
	
	
	\end{question}
	
	\begin{question}
	
		\text{Bestimmen Sie den Zahlenwert:}
		\type{input.number} 
        \precision{3}
        \field{rational}
		
		\begin{variables}
			
			\randint{n}{2}{10}
			\randint{l}{2}{20}
			
			\function[calculate]{x}{(1/6)*n^3+(1/2)*n^2+(1/3)*n} 
			
			\function[calculate]{a}{n+2}
			
		\end{variables}
	
		\begin{answer}
			\text{$\binom{\var{a}}{3}=$}
			\solution{x}
            \explanation{Tipp: Verwenden Sie Definition des Binomialkoeffizienten.}  
		\end{answer}
	
	
	\end{question}
	
	\begin{question}
	
		\text{Bestimmen Sie den Zahlenwert:}
		\type{input.number} 
        \precision{3}
        \field{rational}
		
		\begin{variables}
			
			\randint{n}{2}{10}
			\randint{l}{2}{20}
			
			\function[calculate]{x}{(1/24)*(n+3)*(n+2)*(n+1)*n} 
			
			\function[calculate]{a}{n+3}
			
		\end{variables}
	
		\begin{answer}
			\text{$\binom{\var{a}}{4}=$}
			\solution{x}
            \explanation{Tipp: Verwenden Sie Definition des Binomialkoeffizienten.}  
		\end{answer}
	
	
	\end{question}
	
		
\end{problem}

\embedmathlet{mathlet}


\end{content}