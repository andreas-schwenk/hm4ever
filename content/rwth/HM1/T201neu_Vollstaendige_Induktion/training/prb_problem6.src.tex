\documentclass{mumie.problem.gwtmathlet}
%$Id$
\begin{metainfo}
  \name{
    \lang{de}{A06: Fakultät}
    \lang{en}{Factoriel}
  }
  \begin{description} 
 This work is licensed under the Creative Commons License Attribution 4.0 International (CC-BY 4.0)   
 https://creativecommons.org/licenses/by/4.0/legalcode 

    \lang{de}{Die Beschreibung}
    \lang{en}{}
  \end{description}
  \corrector{system/problem/GenericCorrector.meta.xml}
  \begin{components}
    \component{js_lib}{system/problem/GenericMathlet.meta.xml}{mathlet}
  \end{components}
  \begin{links}
  \end{links}
  \creategeneric
\end{metainfo}
\begin{content}
\begin{block}[annotation]
	Im Ticket-System: \href{https://team.mumie.net/issues/22132}{Ticket 22132}
\end{block}
\begin{block}[annotation]
Copy of \href{http://team.mumie.net/issues/9772}{Ticket 9772}: content/rwth/HM1/T201_Vollstaendige_Induktion_wichtige_Ungleichungen/training/prb_problem6.src.tex
\end{block}


\usepackage{mumie.genericproblem}
\usepackage{mumie.ombplus}


\lang{de}{
	\title{A06: Fakultät}
}

\lang{en}{
	\title{Problem 6}
}


\begin{problem}

\randomquestionpool{1}{6}
	\begin{question}
	
		\text{Fassen Sie den folgenden Ausdruck vollständig zusammen, sodass Sie als Ergebnis einen Ausdruck der Form  $x!$ erhalten:\\
        $(\var{n}+1)\cdot \var{n}! =$\ansref $!$}
		\type{input.number} 
        \field{rational}
		
		\begin{variables}
			\number{a}{0}
			\randint{n}{2}{50}
			
			\function[calculate]{nh1}{n+1}
			
		\end{variables}
	
		\begin{answer}
			\solution{nh1}
            \explanation{Tipp: Verwenden Sie die Definition der Fakultät.}
		\end{answer}
		
	\end{question}
	
	\begin{question}
	
	\text{Fassen Sie den folgenden Ausdruck vollständig zusammen, sodass Sie als Ergebnis einen Ausdruck der Form  $x!$ erhalten:\\
        $\var{nh2}\cdot \var{n}!\cdot \var{nh1} =$\ansref $!$}
		\type{input.number} 
        \field{rational}
		
		\begin{variables}
			\number{a}{0}
			
			\randint{n}{2}{50}
			\function[calculate]{nh1}{n+1}
			\function[calculate]{nh2}{n+2}
			
		\end{variables}
	
		\begin{answer}
			\solution{nh2}
            \explanation{Tipp: Verwenden Sie die Definition der Fakultät.}
		\end{answer}
	
	
	\end{question}
	
	\begin{question}
	
	\text{Fassen Sie den folgenden Ausdruck vollständig zusammen, sodass Sie als Ergebnis einen Ausdruck der Form  $x!$ erhalten:\\
        $\frac{\var{nh1}!}{\var{nh1}}=$\ansref $!$}
		\type{input.number} 
        \field{rational}
		
		\begin{variables}
			\number{a}{0}
			
			\randint{n}{2}{50}
			
			\function[calculate]{nh1}{n+1}
			
		\end{variables}
	
		\begin{answer}
			\solution{n}
            \explanation{Tipp: Verwenden Sie die Definition der Fakultät.}
		\end{answer}
	
	
	\end{question}
	
		\begin{question}
	
	\text{Fassen Sie den folgenden Ausdruck vollständig zusammen, sodass Sie als Ergebnis einen Ausdruck der Form  $x!$ erhalten:\\
        $\frac{\var{n}!}{\var{n}\cdot \var{nl1}}=$\ansref $!$}
		\type{input.number} 
        \field{rational}
		
		\begin{variables}
			
			\randint{n}{2}{50}
			\number{a}{0}
			
			\function[calculate]{nl1}{n-1}
			\function[calculate]{nl2}{n-2}
			

			
		\end{variables}
	
		\begin{answer}
			\solution{nl2}
            \explanation{Tipp: Verwenden Sie die Definition der Fakultät.}
		\end{answer}
	
	
	\end{question}
	

	\begin{question}
	
	\text{Fassen Sie den folgenden Ausdruck vollständig zusammen, sodass Sie als Ergebnis einen Ausdruck der Form  $x!$ erhalten:\\
        $(\var{k}-1)!\cdot \var{k}\cdot (\var{k}+1)=$\ansref $!$}
		\type{input.function} 
        \field{rational}
		
		\begin{variables}			
    			\number{a}{0}
		
			\function{k}{l}

			\function{k1}{(k+1)}
			
			
		\end{variables}
	
		\begin{answer}
			\solution{k1}
			\checkAsFunction{l}{-10}{10}{100}
            \explanation{Tipp: Verwenden Sie die Definition der Fakultät.}
		\end{answer}
	
	
	\end{question}

\begin{question}
	
	\text{Fassen Sie den folgenden Ausdruck vollständig zusammen, sodass Sie als Ergebnis einen Ausdruck der Form  $x!$ erhalten:\\
        $\frac{\var{k}!}{\var{k}\cdot (\var{k}-1)}=$\ansref $!$}
		\type{input.function} 
        \field{rational}
		
		\begin{variables}			
    			\number{a}{0}
		
			\function{k}{l}

			\function{k2}{(k-2)}
			
			
		\end{variables}
	
		\begin{answer}
			\solution{k2}
			\checkAsFunction{l}{-10}{10}{100}
            \explanation{Tipp: Verwenden Sie die Definition der Fakultät.}
		\end{answer}
	
	
	\end{question}
		
\end{problem}

\embedmathlet{mathlet}


\end{content}