\documentclass{mumie.problem.gwtmathlet}
%$Id$
\begin{metainfo}
  \name{
    \lang{de}{A03: Induktion, Binomialkoeffizienten}
    \lang{en}{input numbers}
  }
  \begin{description} 
 This work is licensed under the Creative Commons License Attribution 4.0 International (CC-BY 4.0)   
 https://creativecommons.org/licenses/by/4.0/legalcode 

    \lang{de}{Die Beschreibung}
    \lang{en}{}
  \end{description}
  \corrector{system/problem/GenericCorrector.meta.xml}
  \begin{components}
    \component{js_lib}{system/problem/GenericMathlet.meta.xml}{mathlet}
  \end{components}
  \begin{links}
  \end{links}
  \creategeneric
\end{metainfo}
\begin{content}
\begin{block}[annotation]
	Im Ticket-System: \href{https://team.mumie.net/issues/22138}{Ticket 22138}
\end{block}
\begin{block}[annotation]
Copy of \href{http://team.mumie.net/issues/9769}{Ticket 9769}: content/rwth/HM1/T201_Vollstaendige_Induktion_wichtige_Ungleichungen/training/prb_problem3.src.tex
\end{block}


\usepackage{mumie.genericproblem}
\usepackage{mumie.ombplus}


\lang{de}{
	\title{A03: Induktion, Binomialkoeffizienten}
}

\lang{en}{
	\title{Problem 3}
}

\begin{problem}
\randomquestionpool{1}{2}
\randomquestionpool{3}{5}
\randomquestionpool{6}{9}
\begin{question}
  \text{Kreuzen Sie alle Aussagen an, die für eine nicht weiter bestimmte beliebige reelle Folge $(a_j)_{j\ge 1}$ gelten.}
      \type{mc.multiple}

  \permutechoices{1}{4}
    \begin{choice}
      \text{Es ist $\sum_{j=1}^{n+1} a_j = \sum_{j=1}^n a_j + a_n  $.}
      \solution{false} \end{choice}
    \begin{choice}
      \text{Es ist $\sum_{j=1}^{n+1} a_j = \sum_{j=1}^n a_j +(n+1) $.}
      \solution{false} \end{choice}
    \begin{choice}
      \text{Es ist $\sum_{j=1}^{n+1} a_j = \sum_{j=1}^n a_j + a_{n+1} $.}
      \solution{true} \end{choice}
    \begin{choice}
      \text{Es ist $\sum_{j=1}^{n+1} a_j = \sum_{j=1}^n a_j \cdot a_{n+1} $.}
      \solution{false} \end{choice}
  
  \end{question}
  
  
  
  
  
\begin{question}
   \text{Kreuzen Sie alle wahren Aussagen an.}
      \type{mc.multiple}
       \explanation{Tipp: Verwenden Sie bekannte Regeln wie die Geometrische Summenformel oder die Bernoullische Ungleichung}      
  \permutechoices{1}{5}
    \begin{choice}
      \text{$\sum_{k=0}^n q^k = \frac{q^{n+1}-1}{q-1}$ für alle $q\in \R, n\in\N_0$. 
}      
      \solution{false} 

      \end{choice}
    \begin{choice}
      \text{$\sum_{k=0}^n q^k = \frac{q-1}{q^{n+1}-1}$ für alle $q\in\R\setminus\{1\},n\in\N_0$.
}
      \solution{false} 

      \end{choice}
    \begin{choice}
      \text{$\sum_{k=0}^n q^k = \frac{q^{n+1}-1}{q-1} $ für alle $q\in\R\setminus\{1\},n\in\N_0$.}
      \solution{true} \end{choice}
    \begin{choice}
      \text{$ (1+a)^n \le 1+na$ für alle $a\in\R, a\ge-1, n\in\N_0$}
      \solution{false} 

      \end{choice}
    \begin{choice}
      \text{$  (1+a)^n \ge 1+na$ für alle $a\in\R, a\ge-1, n\in\N_0$}
      \solution{true} \end{choice}  
  
  \end{question}
  
%-- zweiter Fragenpool ---  
  
  
\begin{question}
     \text{Entscheiden Sie, welche Aussage(n) für alle $n\in\N$ korrekt ist/sind. (Hinweis: Induktion)}
      \type{mc.multiple}
       \explanation{Tipp: Verwenden Sie die Induktion. Setzen Sie zunächst $n=1$ für den Induktionsanfang 
      in die Gleichung ein und schauen Sie, ob die Gleichung gilt.}      
  \permutechoices{1}{3}
    \begin{choice}
      \text{ $\sum_{i=1}^n i^2 = \frac{n(n+1)^2}{6}.$
}      
      \solution{false} 

      \end{choice}
    \begin{choice}
      \text{$\sum_{i=1}^n i^2 = \frac{1}{3}n^3+\frac{1}{2}n^2+\frac{1}{6}n.$
}
      \solution{true}
      \end{choice}
    \begin{choice}
      \text{$\sum_{i=1}^n i^2 = \frac{1}{3}n^2 + \frac{1}{3}n$.}
      \solution{false} 

      \end{choice}
  
  \end{question}
  
  
\begin{question}
       \text{Entscheiden Sie, welche Aussage(n) für alle $n\in\N$ korrekt ist/sind. (Hinweis: Induktion)}
      \type{mc.multiple}
      \explanation{Tipp: Verwenden Sie die Induktion. Setzen Sie zunächst $n=1$ für den Induktionsanfang 
      in die Gleichung ein und schauen Sie, ob die Gleichung gilt.}
      
    \permutechoices{1}{3}
    \begin{choice}
      \text{$\sum_{j=1}^n (j\cdot j!) = (n+1)! - 1 .$
}      
      \solution{true} \end{choice}
    \begin{choice}
      \text{$\sum_{j=1}^n (j \cdot j!)= n! - 1. $
}
      \solution{false} 

      \end{choice}
    \begin{choice}
      \text{$\sum_{j=1}^n (j \cdot j!) = (n+2)! - 5$.
      }

      \solution{false} 

      \end{choice}
  
  \end{question}
  
\begin{question}
   \text{ Entscheiden Sie welche Aussage(n) für alle $n\in\N$ korrekt ist/sind. (Hinweis: Induktion)}
      \type{mc.multiple}
      \explanation{Tipp: Verwenden Sie die Induktion. Setzen Sie zunächst $n=1$ für den Induktionsanfang 
      in die Gleichung ein und schauen Sie, ob die Gleichung gilt.}
  
  \permutechoices{1}{5}
    \begin{choice}
      \text{$\sum_{k=1}^n k^3 = \left( \frac{n(n+1)}{2} \right)^2.$
}      
      \solution{true} \end{choice}
    \begin{choice}
      \text{$\sum_{j=1}^n j^3 = \left( \frac{n(n+1)}{2} \right)^2.$
}
      \solution{true} \end{choice}
    \begin{choice}
      \text{ $\sum_{k=1}^n k^3 =  \frac{(n+1)^2}{2}.$
}

      \solution{false}

      \end{choice}
    \begin{choice}
      \text{$ \sum_{j=1}^n j^3 = \frac{(n+1)^2}{2}.$
}

      \solution{false} 

      \end{choice}
    \begin{choice}
      \text{$\sum_{j=1}^n j^3 = \left( \sum_{k=1}^n k \right) ^2. $
}
      \solution{true} \end{choice}  
  
  \end{question}
  
%-- dritter Fragenpool

\begin{question}
         \text{Entscheiden Sie, welche der folgenden Aussagen man \textbf{nicht} mit Induktion beweisen könnte.)}
      \type{mc.multiple}
    \explanation{Tipp: Anwendung der Induktion ist nur für natürliche Zahlen möglich, nicht für reelle.}
  
  \permutechoices{1}{3}
    \begin{choice}
      \text{$\sum_{i=1}^n i = \frac{n(n+1)}{2}$ für alle $n\in\N$.
}      
      \solution{false} \end{choice}
    \begin{choice}
      \text{$\sum_{x=0}^n x = \frac{n(n+1)}{2}$ für alle $n\in\R$.
}
      \solution{true} \end{choice}
    \begin{choice}
      \text{$\sum_{j=0}^n j = \frac{n(n+1)}{2}$ für alle $n\in\N$.
      }
      \solution{false} \end{choice}
    \begin{choice}
      \text{$2^n \le n!$ für alle $n\in\R$. 	
      }
      \solution{true} \end{choice}  
  
  \end{question}
  
\begin{question}
    \text{Die Aussage $A(n)$ soll für $n\in\N$ mittels Induktion bewiesen werden. Entscheiden Sie welche Formulierungen für die Induktionsvoraussetzung in dem Induktionsbeweis korrekt sind und kreuzen Sie diese an.
}
      \type{mc.multiple}
  \permutechoices{1}{3}
    \begin{choice}
      \text{(IV) Für alle $n\in\N$ gelte $A(n)$.}      
      \solution{false} \end{choice}
    \begin{choice}
      \text{(IV) Es gelte $A(n)$ für ein $n\in\N$.}
      \solution{true} \end{choice}
    \begin{choice}
      \text{(IV) Sei $n\in\N$ so, dass $A(n)$ gilt.}
      \solution{true} \end{choice}
  
  \end{question}
  
  
  
\begin{question}
   \text{Kreuzen Sie alle wahren Aussagen an. }
      \type{mc.multiple}
  \permutechoices{1}{5}
    \begin{choice}
      \text{Das Induktionsprinzip ist eine Beweismethode um allgemeine Aussagen, die von einer natürlichen Zahl abhängig sind, zu beweisen.
}     
      \solution{true} \end{choice}
    \begin{choice}
      \text{Bei einer vollständigen Induktion reicht es aus, den Induktionsanfang zu zeigen.
}
      \solution{false} \end{choice}
    \begin{choice}
      \text{Der Induktionsanfang kann bei einer Induktion vernachlässigt werden.
}
      \solution{false} \end{choice}
    \begin{choice}
      \text{Zu einem korrekten Induktionsbeweis gehört immer eine (korrekt) formulierte Induktionsvoraussetzung.
}
      \solution{true} \end{choice}
    \begin{choice}
      \text{Die Beweismethode der vollständigen Induktion besteht aus Induktionsanfang, Induktionsrechnung und Induktionseinsetzen.}
      \solution{false} \end{choice}  
  
  \end{question}
  
  
  
\begin{question}
   \text{Kreuzen Sie alle wahren Aussagen an. }
      \type{mc.multiple}
  \permutechoices{1}{5}
    \begin{choice}
      \text{Man kann lediglich Gleichungen mit vollständiger Induktion beweisen.
}     
      \solution{false} \end{choice}
    \begin{choice}
      \text{Summen- und Produktformeln sind typische Beispiele für Aussagen, die man mit dem Induktionsprinzip beweisen kann.
}
      \solution{true} \end{choice}
    \begin{choice}
      \text{Der Induktionsanfang wird immer für $n=1$ durchgeführt.
}
      \solution{false} \end{choice}
    \begin{choice}
      \text{ Die Induktionsmethode lässt sich auch anwenden wenn $n_0$ eine negative ganze Zahl ist und $A(n)$ eine Aussage ist, die für alle $n\ge n_0$ zu beweisen ist.
}
      \solution{true} \end{choice}
    \begin{choice}
      \text{Die Bernoullische Ungleichung kann man mit dem Prinzip der vollständigen Induktion beweisen.
}
      \solution{true} \end{choice}  
  
  \end{question}
  
  
  
\end{problem}
\embedmathlet{mathlet}


\end{content}