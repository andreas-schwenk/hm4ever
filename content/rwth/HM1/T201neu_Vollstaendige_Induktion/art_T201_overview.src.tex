%$Id:  $
\documentclass{mumie.article}
%$Id$
\begin{metainfo}
  \name{
    \lang{de}{Überblick: Beweisverfahren und mathematische Ausdrücke}
    \lang{en}{Overview: Methods of proof and mathematical expressions}
  }
  \begin{description} 
 This work is licensed under the Creative Commons License Attribution 4.0 International (CC-BY 4.0)   
 https://creativecommons.org/licenses/by/4.0/legalcode 

    \lang{de}{Beschreibung}
    \lang{en}{Description}
  \end{description}
  \begin{components}
  \end{components}
  \begin{links}
\link{generic_article}{content/rwth/HM1/T201neu_Vollstaendige_Induktion/g_art_content_03_binomischer_lehrsatz.meta.xml}{content_03_binomischer_lehrsatz}
\link{generic_article}{content/rwth/HM1/T201neu_Vollstaendige_Induktion/g_art_content_02_vollstaendige_induktion.meta.xml}{content_02_vollstaendige_induktion}
\link{generic_article}{content/rwth/HM1/T201neu_Vollstaendige_Induktion/g_art_content_01_indirekter_widerspruchsbeweis.meta.xml}{content_01_indirekter_widerspruchsbeweis}
\end{links}
  \creategeneric
\end{metainfo}
\begin{content}
\begin{block}[annotation]
	Im Ticket-System: \href{https://team.mumie.net/issues/30136}{Ticket 30136}
\end{block}
\begin{block}[annotation]
Copy of : /home/mumie/checkin/content/rwth/HM1/T202_Reelle_Zahlen_axiomatisch/art_T202_overview.src.tex
\end{block}



\begin{block}[annotation]
Im Entstehen: Überblicksseite für Kapitel Beweisverfahren und mathematische Ausdrücke
\end{block}

\usepackage{mumie.ombplus}
\ombchapter{1}
\lang{de}{\title{Überblick: Beweisverfahren und mathematische Ausdrücke}}
\lang{en}{\title{Overview: Methods of proof and mathematical expressions}}



\begin{block}[info-box]
\lang{de}{\strong{Inhalt}}
\lang{en}{\strong{Contents}}


\lang{de}{
    \begin{enumerate}%[arabic chapter-overview]
   \item[1.1] \link{content_01_indirekter_widerspruchsbeweis}{Indirekter Beweis und Widerspruchsbeweis}
   \item[1.2] \link{content_02_vollstaendige_induktion}{Vollständige Induktion}
   \item[1.3] \link{content_03_binomischer_lehrsatz}{Binomialkoeffizienten}
    \end{enumerate}
} %lang

\end{block}

\begin{zusammenfassung}

\lang{de}{In diesem Kapitel wird grundlegendes mathematisches Handwerkszeug bereitgestellt.
\\
Direkte und indirekte Schlussfolgerungen werden Ihnen in diesem Kurs und in weiterführenden 
Mathematik-Veranstaltungen immer wieder begegnen, nicht nur in formal als Beweise deklarierten Begründungen,
sondern in vielen Erläuterungen und Herleitungen. Wir erklären sie hier an einfachen Beispielen.

Das Beweisprinzip der vollständigen Induktion ist so formal bestechend wie nützlich. Sie werden immer wieder selbst
induktive Schlüsse durchführen und sollten das Prinzip grundlegend verstanden haben. 
Wir beweisen damit wichtige mathematische Sätze wie die geometrische Summenformel und die Bernoullische Ungleichung.

Außerdem wenden wir es an, um mit Hilfe der Binomialkoeffizienten den binomischen Lehrsatz zu beweisen, 
eine oft benutzte Verallgemeinerung der Ihnen geläufigen binomischen Formeln.}

\lang{en}{In this chapter we will establish basic mathematical tools.
\\
Direct and indirect arguments will be encountered repeatedly in this course and further mathematical reading, 
not only in formal proofs but also in explanations and derivations. We will explain them here using simple examples.

The principle of mathematical induction is as interesting as it is useful. You will keep using inductive reasoning, 
and should have a basic understanding of the principle. Using it, we will prove some important mathematical theorems such as 
the partial geometric sum formula and Bernoulli's inequality.

Furthermore, with the help of binomial coefficients, we will use it to prove the binomial theorem. 
This is a commonly used generalisation of the familiar binomial formulas.}

\end{zusammenfassung}

\begin{block}[info]\lang{de}{\strong{Lernziele}}
\lang{en}{\strong{Learning Goals}} 
\begin{itemize}[square]
\item \lang{de}{Ihnen sind direkte und indirekte Beweiseschlüsse bekannt. Sie erkennen diese Vorgehensweisen in gegebenen Situationen.}
      \lang{en}{Being familiar with direct and indirect proof methods. Being able to recognise them in given situations.}
\item \lang{de}{In einfachen Beispielen formulieren Sie direkte und indirekte Schlüsse korrekt.}
      \lang{en}{Being able to correctly formulate direct and indirect proofs in easy examples.}
\item \lang{de}{Die Bestandteile Induktionsanfang, Induktionsannahme und Induktionsschluss des Prinzips der
                vollständigen Induktion sind Ihnen bekannt. Sie erklären ihre Bedeutung für das Induktionsprinzip.}
      \lang{en}{Being familiar with the base case, induction hypothesis and induction step as components of mathematical induction. 
      Being able to explain their role in an inductive proof.}
\item \lang{de}{Mit dem Prinzip der vollständigen Induktion beweisen Sie formal korrekt vorgegebene Aussagen.}
      \lang{en}{Being able to correctly formally prove given propositions using the principle of mathematical induction.}
\item \lang{de}{Die geometrische Summenformel, der binomische Lehrsatz und die Bernoullische Ungleichungen gehören zu Ihrem
                mathematischen Grundwissen. Sie erkennen sie in Anwendungsbeispielen und benutzen sie selbständig.}
      \lang{en}{Being familiar with the geometric partial sum formula, the binomial theorem, and Bernoulli's inequality. 
      Recognising them when applied in examples, and being able to use them independently.}
\end{itemize}
\end{block}




\end{content}
