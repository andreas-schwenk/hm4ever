%$Id: art_content.src.tex 158928 2020-11-27 16:35:18Z mmwebclient $
\documentclass{mumie.article}
%$Id: art_content.src.tex 158928 2020-11-27 16:35:18Z mmwebclient $
\begin{metainfo}
  \name{
    \lang{de}{Formen mathematischer Beweisführung}
    \lang{en}{Methods of mathematical proof}
    }
  \begin{description} 
 This work is licensed under the Creative Commons License Attribution 4.0 International (CC-BY 4.0)   
 https://creativecommons.org/licenses/by/4.0/legalcode 


  \end{description}
  \begin{components}
    \component{generic_image}{content/rwth/HM1/images/g_img_00_video_button_schwarz-blau.meta.xml}{00_video_button_schwarz-blau}
  \end{components}
  \begin{links}
    
    \link{generic_article}{content/rwth/HM1/T101neu_Elementare_Rechengrundlagen/g_art_content_01_zahlenmengen.meta.xml}{link-ec-roots}
    \link{generic_article}{content/rwth/HM1/T101neu_Elementare_Rechengrundlagen/g_art_content_02_rechengrundlagen_terme.meta.xml}{link-rechengrundlagen}
    \link{generic_article}{content/rwth/HM1/T101neu_Elementare_Rechengrundlagen/g_art_content_04_aussagen_aequivalenzumformungen.meta.xml}{link-Inferenz}

  \end{links}
  \creategeneric
\end{metainfo}

% Besipiel für Beweis:
% api.stream24.net/vod/getVideo.php?id=10962-2-10748&mode=iframe
% Beweisarten:
% api.stream24.net/vod/getVideo.php?id=10962-2-10939&mode=iframe

%
% 

\begin{content}
\begin{block}[annotation]
	Im Ticket-System: \href{https://team.mumie.net/issues/22233}{Ticket 22233}
Copy of \href{http://team.mumie.net/issues/7777}{Ticket 7777}: content/TU9/modul_logic/mathematical_proofs/art_content.src.tex
 bis auf den Teil "Beweis durch vollständige Induktion".
\end{block}

  
  \usepackage{mumie.ombplus}
  \ombchapter{1}
  \ombarticle{1}

  \lang{de}{\title{Formen mathematischer Beweisführung}}
  \lang{en}{\title{Methods of mathematical proof}}


  \begin{block}[info-box]
    \tableofcontents
  \end{block}

  
  
  \lang{de}{
    Im Abschnitt über 
    \ref[link-Inferenz][Folgerung/Implikation und Äquivalenz ]{sec:folgerung}
    wurden die Implikation und die Äquivalenz schon vorgestellt. Im folgenden Abschnitt werden wir mathematische Sätze betrachten, 
    die oft die Form einer Implikation, d. h.
    \[ \text{Voraussetzung} \Rightarrow \text{Behauptung}, \]
    annehmen, und besprechen, wie man sie bewiesen kann. Der Beweis eines solchen Satzes besteht darin, zu zeigen, dass diese Implikation wahr ist. 
    Es gibt mehrere Möglichkeiten, einen solchen Beweis zu führen. Die grundlegenden Hauptmethoden sollen hier vorgestellt werden.
  }
  \lang{en}{
    In the section about \ref[link-Inferenz][Implication and Equivalence ]{sec:folgerung}, we introduced implications and equivalence.
    In the following section we will consider mathematical statements which often have the form of an implication, i.e.
    \[ \text{Hypothesis} \Rightarrow \text{Conclusion}, \]
    and we will discuss how one can prove them. The proof of such a statement consists in showing that this implication is true.
    There are various ways of forming such a proof. The main methods will be introduced here.
  }
  
  \section{\lang{de}{Direkter Beweis}\lang{en}{Direct Proof}}\label{sec:direkterbeweis}  
  
  \lang{de}{Beim \emph{direkten Beweis} wird die zu beweisende Implikation in eine Kette von Implikationen aufgeteilt, z.B.
  \[ \text{Voraussetzung} \Rightarrow A_1 \Rightarrow A_2 \Rightarrow A_3 \Rightarrow \text{Behauptung} \] 
  mit gewissen Aussagen $A_1$, $A_2$ und $A_3$, wobei von jeder einzelnen dieser Implikationen der Wahrheitswert W bereits 
  bekannt oder unmittelbar einsichtig ist.}
  \lang{en}{In a \emph{direct proof}, the implication to be proved is split into a chain of implications, e.g.
  \[ \text{Hypothesis} \Rightarrow A_1 \Rightarrow A_2 \Rightarrow A_3 \Rightarrow \text{Conclusion} \] 
  with certain propositions $A_1$, $A_2$ und $A_3$, for which the truth T of each implication is known or immediately obvious.
  }
   
  \begin{example}
    \lang{de}{  Für $n\in \N$ gilt:
    \[ n \text{ ungerade} \Rightarrow n^2-1 \text{ ist durch }4\text{ teilbar} \]
    Ein Beweis dieser Aussage geht zum Beispiel so:}
    \lang{en}{  For $n\in \N$ it is true that:
    \[ n \text{ odd} \Rightarrow n^2-1 \text{ is divisible by }4 \]
    An example of a proof of this statement is:
    }


    \begin{eqnarray*}
      n \text{ \lang{de}{ungerade} } 
        &\Rightarrow & \text{\lang{de}{Es gibt ein }\lang{en}{There is a }}k\in \N\text{\lang{de}{ mit }\lang{en}{ with }}n=2k-1 \\
        &\Rightarrow & \text{\lang{de}{Es gibt ein }\lang{en}{There is a }} k\in \N\text{\lang{de}{ mit }\lang{en}{ with }} n^2-1=(2k-1)^2-1=(4k^2-4k+1)-1=4k^2-4k=4\cdot (k^2-k) \\
        &\Rightarrow & \text{\lang{de}{Es gibt ein }\lang{en}{There is a }} l\in \N\text{\lang{de}{ mit }\lang{en}{ with }} n^2-1=4l \quad (\text{\lang{de}{nämlich}\lang{en}{ namely } }l=k^2-k)\\
        &\Rightarrow & n^2-1 \text{\lang{de}{ ist durch $4$ teilbar}\lang{en}{ is divisible by $4$}} 
    \end{eqnarray*}

    
    \lang{de}{(Bei der Umformung in der zweiten Zeile wurde übrigens die 
    \ref[link-rechengrundlagen][zweite Binomische Formel ]{rule:binomische_formeln}
    $(a-b)^2=a^2-2ab+b^2$ benutzt.)
    Da jede einzelne Implikation in der Kette wahr ist, ist auch die Implikation des Satzes wahr.}
    \lang{en}{For the rearrangement in the second line we used the 
    \ref[link-rechengrundlagen][second binomial identity ]{rule:binomische_formeln}
    $(a-b)^2=a^2-2ab+b^2$.)
    Since every implication in the chain is true, the implication of the original statement is also true.}
    
   \end{example}

\begin{proof*}[\lang{de}{Begründung für das Verfahren}\lang{en}{Justification for the procedure}]
     \begin{showhide}
    
    \lang{de}{Um festzustellen, dass die gesamte Implikation wahr ist, 
    wenn die einzelnen Implikationen wahr sind, vergleicht man am
    besten die Wahrheitstafeln von}
    \lang{en}{To confirm that the entire implication is true when the individual implications are true,
    we can compare the truth table of}
       
    \[ (A\Rightarrow A_1) \wedge (A_1\Rightarrow A_2) \wedge \ldots \wedge (A_n \Rightarrow B) \]
    
    \lang{de}{ und von }
    \lang{en}{ to that of  }
    
    \[ A \Rightarrow B \]
    
    \lang{de}{ Hier beispielhaft der Fall eines Zwischenschritts:}
    \lang{en}{ For example, in the case of a single intermediate implication step:}
    
    \begin{table}[\align{c}\cellaligns{ccccccccc}] % {c|c|c||c|c|c||c}
      \head
      $A$ & $A_1$ & $B$ && $A\Rightarrow A_1$ & $A_1 \Rightarrow B$ & $(A\Rightarrow A_1) \wedge (A_1\Rightarrow B)$ 
      && $A \Rightarrow B$ 
      \body
      \lang{de}{W} & \lang{de}{W}\lang{en}{T}\lang{zh}{T}\lang{fr}{$\text{V}$} & \lang{de}{W}\lang{en}{T}\lang{zh}{T}\lang{fr}{$\text{V}$} && \lang{de}{W}\lang{en}{T}\lang{zh}{T}\lang{fr}{$\text{V}$} & \lang{de}{W}\lang{en}{T}\lang{zh}{T}\lang{fr}{$\text{V}$} & \lang{de}{W}\lang{en}{T}\lang{zh}{T}\lang{fr}{$\text{V}$} && \lang{de}{W}\lang{en}{T}\lang{zh}{T}\lang{fr}{$\text{V}$} \\
      \lang{de}{W} & \lang{de}{W}\lang{en}{T}\lang{zh}{T}\lang{fr}{$\text{V}$} & \lang{de}{F}\lang{en}{F}\lang{zh}{F}\lang{fr}{$\text{F}$} && \lang{de}{W}\lang{en}{T}\lang{zh}{T}\lang{fr}{$\text{V}$} & \lang{de}{F}\lang{en}{F}\lang{zh}{F}\lang{fr}{$\text{F}$} & \lang{de}{F}\lang{en}{F}\lang{zh}{F}\lang{fr}{$\text{F}$} && \lang{de}{F}\lang{en}{F}\lang{zh}{F}\lang{fr}{$\text{F}$} \\
      \lang{de}{W} & \lang{de}{F}\lang{en}{F}\lang{zh}{F}\lang{fr}{$\text{F}$} & \lang{de}{W}\lang{en}{T}\lang{zh}{T}\lang{fr}{$\text{V}$} && \lang{de}{F}\lang{en}{F}\lang{zh}{F}\lang{fr}{$\text{F}$} & \lang{de}{W}\lang{en}{T}\lang{zh}{T}\lang{fr}{$\text{V}$} & \lang{de}{F}\lang{en}{F}\lang{zh}{F}\lang{fr}{$\text{F}$} && \lang{de}{W}\lang{en}{T}\lang{zh}{T}\lang{fr}{$\text{V}$} \\
      \lang{de}{W} & \lang{de}{F}\lang{en}{F}\lang{zh}{F}\lang{fr}{$\text{F}$} & \lang{de}{F}\lang{en}{F}\lang{zh}{F}\lang{fr}{$\text{F}$} && \lang{de}{F}\lang{en}{F}\lang{zh}{F}\lang{fr}{$\text{F}$} & \lang{de}{W}\lang{en}{T}\lang{zh}{T}\lang{fr}{$\text{V}$} & \lang{de}{F}\lang{en}{F}\lang{zh}{F}\lang{fr}{$\text{F}$} && \lang{de}{F}\lang{en}{F}\lang{zh}{F}\lang{fr}{$\text{F}$} \\
      \lang{de}{F} & \lang{de}{W}\lang{en}{T}\lang{zh}{T}\lang{fr}{$\text{V}$} & \lang{de}{W}\lang{en}{T}\lang{zh}{T}\lang{fr}{$\text{V}$} && \lang{de}{W}\lang{en}{T}\lang{zh}{T}\lang{fr}{$\text{V}$} & \lang{de}{W}\lang{en}{T}\lang{zh}{T}\lang{fr}{$\text{V}$} & \lang{de}{W}\lang{en}{T}\lang{zh}{T}\lang{fr}{$\text{V}$} && \lang{de}{W}\lang{en}{T}\lang{zh}{T}\lang{fr}{$\text{V}$} \\
      \lang{de}{F} & \lang{de}{W}\lang{en}{T}\lang{zh}{T}\lang{fr}{$\text{V}$} & \lang{de}{F}\lang{en}{F}\lang{zh}{F}\lang{fr}{$\text{F}$} && \lang{de}{W}\lang{en}{T}\lang{zh}{T}\lang{fr}{$\text{V}$} & \lang{de}{F}\lang{en}{F}\lang{zh}{F}\lang{fr}{$\text{F}$} & \lang{de}{F}\lang{en}{F}\lang{zh}{F}\lang{fr}{$\text{F}$} && \lang{de}{W}\lang{en}{T}\lang{zh}{T}\lang{fr}{$\text{V}$} \\
      \lang{de}{F} & \lang{de}{F}\lang{en}{F}\lang{zh}{F}\lang{fr}{$\text{F}$} & \lang{de}{W}\lang{en}{T}\lang{zh}{T}\lang{fr}{$\text{V}$} && \lang{de}{W}\lang{en}{T}\lang{zh}{T}\lang{fr}{$\text{V}$} & \lang{de}{W}\lang{en}{T}\lang{zh}{T}\lang{fr}{$\text{V}$} & \lang{de}{W}\lang{en}{T}\lang{zh}{T}\lang{fr}{$\text{V}$} && \lang{de}{W}\lang{en}{T}\lang{zh}{T}\lang{fr}{$\text{V}$} \\
      \lang{de}{F} & \lang{de}{F}\lang{en}{F}\lang{zh}{F}\lang{fr}{$\text{F}$} & \lang{de}{F}\lang{en}{F}\lang{zh}{F}\lang{fr}{$\text{F}$} && \lang{de}{W}\lang{en}{T}\lang{zh}{T}\lang{fr}{$\text{V}$} & \lang{de}{W}\lang{en}{T}\lang{zh}{T}\lang{fr}{$\text{V}$} & \lang{de}{W}\lang{en}{T}\lang{zh}{T}\lang{fr}{$\text{V}$} && \lang{de}{W}\lang{en}{T}\lang{zh}{T}\lang{fr}{$\text{V}$}
    \end{table}

    \lang{de}{Wenn in der Tabelle die Aussageform $(A\Rightarrow A_1) \wedge (A_1\Rightarrow B)$ den Wahrheitswert W hat, hat auch
    $A \Rightarrow B$ den Wahrheitswert W, also folgt die Richtigkeit der gesamten Implikation aus der Richtigkeit der
    einzelnen Implikationen.}
    \lang{en}{When in the table the proposition of the form $(A\Rightarrow A_1) \wedge (A_1\Rightarrow B)$ has the truth value T,
    $A \Rightarrow B$ also has the truth value T. Thus the correctness of the whole implication follows from the correctness of the individual implications.}
    
    \end{showhide}
\end{proof*}

  \section{\lang{de}{Indirekter Beweis und Widerspruchsbeweis}\lang{en}{Indirect Proof and Proof by Contradiction}}\label{sec:indirekterbeweis}
  
  \lang{de}{Manchmal ist es sehr schwer oder gar unmöglich, einen direkten Beweis zu führen. Dann ersetzt man die eigentlich zu
  zeigende Implikation durch eine dazu äquivalente Aussage.}
  \lang{en}{Sometimes it is very difficult or even impossible to do a direct proof. One can then replace the implication to be proved with an equivalent proposition.}
    
  \begin{theorem} \label{thm:equiv-to-implication}
    \lang{de}{Für beliebige Aussagen $A$ und $B$ sind die folgenden Aussagen zueinander äquivalent:}
    \lang{en}{For any propositions $A$ and $B$, the following propositions are equivalent to each other:}
       
    \begin{enumerate}
      \item $A\Rightarrow B$ (\lang{de}{entspricht dem \notion{\emph{direkten Beweis}}}\lang{en}{corresponding to \notion{\emph{direct proof}}})
      \item $\neg B \Rightarrow \neg A$ (\lang{de}{entspricht dem \notion{\emph{indirekter Beweis}}}\lang{en}{corresponding to \notion{\emph{indirect proof}}})
      \item $\neg(A\wedge \neg B)$ (\lang{de}{entspricht dem \notion{\emph{Widerspruchsbeweis}}}\lang{en}{corresponding to \notion{\emph{proof by contradiction}}})
    \end{enumerate}
  \end{theorem}
  
\begin{proof*}[\lang{de}{Beweis des Satzes}\lang{en}{Proof of the theorem}]
\begin{showhide}
        
    \lang{de}{Um zu sehen, dass die Implikationen "aquivalent sind, betrachtet man die zugeh"origen Wahrheitstafeln:}
    \lang{en}{To see that the implications are equivalent, consider the relevant truth tables:}
   
    \begin{table}[\align{c}\cellaligns{cccc}] % {c|c|c||c|c|c||c}
      \head
      $A$ & $B$ && $A\Rightarrow B$
      \body
      \lang{de}{W}\lang{en}{T}\lang{zh}{T}\lang{fr}{$\text{V}$} & \lang{de}{W}\lang{en}{T}\lang{zh}{T}\lang{fr}{$\text{V}$} && \lang{de}{W}\lang{en}{T}\lang{zh}{T}\lang{fr}{$\text{V}$} \\
      \lang{de}{W}\lang{en}{T}\lang{zh}{T}\lang{fr}{$\text{V}$} & \lang{de}{F}\lang{en}{F}\lang{zh}{F}\lang{fr}{$\text{F}$} && \lang{de}{F}\lang{en}{F}\lang{zh}{F}\lang{fr}{$\text{F}$} \\
      \lang{de}{F}\lang{en}{F}\lang{zh}{F}\lang{fr}{$\text{F}$} & \lang{de}{W}\lang{en}{T}\lang{zh}{T}\lang{fr}{$\text{V}$} && \lang{de}{W}\lang{en}{T}\lang{zh}{T}\lang{fr}{$\text{V}$} \\
      \lang{de}{F}\lang{en}{F}\lang{zh}{F}\lang{fr}{$\text{F}$} & \lang{de}{F}\lang{en}{F}\lang{zh}{F}\lang{fr}{$\text{F}$} && \lang{de}{W}\lang{en}{T}\lang{zh}{T}\lang{fr}{$\text{V}$}
    \end{table}
    
    \begin{table}[\align{c}\cellaligns{cccccc}] % {c|c|c||c|c|c||c}
      \head
      $A$ & $B$ &&  $\neg B$ & $\neg A$ & $\neg B \Rightarrow \neg A$ 
      \body
      \lang{de}{W}\lang{en}{T}\lang{zh}{T}\lang{fr}{$\text{V}$} & \lang{de}{W}\lang{en}{T}\lang{zh}{T}\lang{fr}{$\text{V}$} && \lang{de}{F}\lang{en}{F}\lang{zh}{F}\lang{fr}{$\text{F}$} & \lang{de}{F}\lang{en}{F}\lang{zh}{F}\lang{fr}{$\text{F}$} & \lang{de}{W}\lang{en}{T}\lang{zh}{T}\lang{fr}{$\text{V}$}\\
      \lang{de}{W}\lang{en}{T}\lang{zh}{T}\lang{fr}{$\text{V}$} & \lang{de}{F}\lang{en}{F}\lang{zh}{F}\lang{fr}{$\text{F}$} && \lang{de}{W}\lang{en}{T}\lang{zh}{T}\lang{fr}{$\text{V}$} & \lang{de}{F}\lang{en}{F}\lang{zh}{F}\lang{fr}{$\text{F}$} & \lang{de}{F}\lang{en}{F}\lang{zh}{F}\lang{fr}{$\text{F}$}\\
      \lang{de}{F}\lang{en}{F}\lang{zh}{F}\lang{fr}{$\text{F}$} & \lang{de}{W}\lang{en}{T}\lang{zh}{T}\lang{fr}{$\text{V}$} && \lang{de}{F}\lang{en}{F}\lang{zh}{F}\lang{fr}{$\text{F}$} & \lang{de}{W}\lang{en}{T}\lang{zh}{T}\lang{fr}{$\text{V}$} & \lang{de}{W}\lang{en}{T}\lang{zh}{T}\lang{fr}{$\text{V}$}\\
      \lang{de}{F}\lang{en}{F}\lang{zh}{F}\lang{fr}{$\text{F}$} & \lang{de}{F}\lang{en}{F}\lang{zh}{F}\lang{fr}{$\text{F}$} && \lang{de}{W}\lang{en}{T}\lang{zh}{T}\lang{fr}{$\text{V}$} & \lang{de}{W}\lang{en}{T}\lang{zh}{T}\lang{fr}{$\text{V}$} & \lang{de}{W}\lang{en}{T}\lang{zh}{T}\lang{fr}{$\text{V}$}
    \end{table}
    
    \begin{table}[\align{c}\cellaligns{cccccc}] % {c|c|c||c|c|c||c}
      \head
      $A$ & $B$ && $\neg B$ & $A\wedge \neg B$ & $\neg(A\wedge \neg B)$
      \body
      \lang{de}{W}\lang{en}{T}\lang{zh}{T}\lang{fr}{$\text{V}$} & \lang{de}{W}\lang{en}{T}\lang{zh}{T}\lang{fr}{$\text{V}$} && \lang{de}{F}\lang{en}{F}\lang{zh}{F}\lang{fr}{$\text{F}$} & \lang{de}{F}\lang{en}{F}\lang{zh}{F}\lang{fr}{$\text{F}$} & \lang{de}{W}\lang{en}{T}\lang{zh}{T}\lang{fr}{$\text{V}$} \\
      \lang{de}{W}\lang{en}{T}\lang{zh}{T}\lang{fr}{$\text{V}$} & \lang{de}{F}\lang{en}{F}\lang{zh}{F}\lang{fr}{$\text{F}$} && \lang{de}{W}\lang{en}{T}\lang{zh}{T}\lang{fr}{$\text{V}$} & \lang{de}{W}\lang{en}{T}\lang{zh}{T}\lang{fr}{$\text{V}$} & \lang{de}{F}\lang{en}{F}\lang{zh}{F}\lang{fr}{$\text{F}$} \\
      \lang{de}{F}\lang{en}{F}\lang{zh}{F}\lang{fr}{$\text{F}$} & \lang{de}{W}\lang{en}{T}\lang{zh}{T}\lang{fr}{$\text{V}$} && \lang{de}{F}\lang{en}{F}\lang{zh}{F}\lang{fr}{$\text{F}$} & \lang{de}{F}\lang{en}{F}\lang{zh}{F}\lang{fr}{$\text{F}$} & \lang{de}{W}\lang{en}{T}\lang{zh}{T}\lang{fr}{$\text{V}$} \\
      \lang{de}{F}\lang{en}{F}\lang{zh}{F}\lang{fr}{$\text{F}$} & \lang{de}{F}\lang{en}{F}\lang{zh}{F}\lang{fr}{$\text{F}$} && \lang{de}{W}\lang{en}{T}\lang{zh}{T}\lang{fr}{$\text{V}$} & \lang{de}{F}\lang{en}{F}\lang{zh}{F}\lang{fr}{$\text{F}$} & \lang{de}{W}\lang{en}{T}\lang{zh}{T}\lang{fr}{$\text{V}$}
    \end{table}
    
    \lang{de}{Dass jeweils die letzten Spalten der drei Wahrheitstafeln identisch sind, zeigt, dass für jeden Wahrheitswert von $A$ 
    und $B$ alle drei Implikationen denselben Wahrheitswert haben. Also sind sie "aquivalent.}
    \lang{en}{In each of the three truth tables the final columns are identical. This shows that for each truth value of $A$ and $B$ all three implications have the same truth value. Thus they are equivalent.}
    \end{showhide}
\end{proof*}

  \lang{de}{Beim sogenannten \emph{indirekten Beweis} (oder \emph{Beweis durch Kontraposition}) wird nun statt der Implikation
  $\text{Voraussetzung} \Rightarrow \text{Behauptung}$ die dazu äquivalente Implikation 
  \[ \text{Negation der Behauptung} \Rightarrow \text{Negation der Voraussetzung} \]
  gezeigt. Dies macht man dann üblicherweise in mehreren Schritten, wie es beim direkten Beweis erklärt wurde.}
  \lang{en}{For so-called \emph{indirect proof} (or \emph{proof by contraposition}, instead of the implication 
  $\text{Hypothesis} \Rightarrow \text{Conclusion}$, the equivalent implication
  \[ \text{Negation of the hypothesis} \Rightarrow \text{Negation of the conclusion} \]
  is shown. This is usually done in multiple steps, as was explained for direct proofs.
  }


\begin{example}
    \lang{de}{Für $n,m\in \N$ gilt: $\quad nm \text{ ungerade } \Rightarrow n \text{ ungerade und } m \text{ ungerade}$.\\
    Dies kann man durch einen indirekten Beweis zeigen, indem man also für $n,m\in \N$ die Aussage:
    \begin{center}\textit{Wenn nicht $n$ und $m$ beide ungerade sind, ist $nm$ nicht ungerade.}
    \end{center}

    zeigt:}
    \lang{en}{For $n,m\in \N$ it is true that: $\quad nm \text{ odd } \Rightarrow n \text{ odd and } m \text{ odd}$.\\
    One can show this with an indirect proof, by showing the proposition:
    \begin{center}\textit{Wenn nicht $n$ und $m$ beide ungerade sind, ist $nm$ nicht ungerade.}
    \end{center}

    to be true for $n,m\in \N$:}
   
    \lang{de}{  $n$ und $m$ nicht beide ungerade\\ $\Rightarrow$ $n$ gerade oder $m$ gerade \\
    $\Rightarrow$ es gibt $k\in \N$ mit $n=2k$ oder es gibt $l\in \N$ mit $m=2l$\\
    $\Rightarrow$ es gibt $k\in \N$ mit $nm=2km$ oder es gibt $l\in \N$ mit $nm=2nl$ \\
    $\Rightarrow$ es gibt $j\in \N$ mit $nm=2j$ (nämlich $j=km$ oder $j=ln$) \\
    $\Rightarrow$ $nm$ gerade, d.h. nicht ungerade.}
    \lang{en}{  $n$ and $m$ not both odd\\ $\Rightarrow$ $n$ even or $m$ even \\
    $\Rightarrow$ there is a $k\in \N$ with $n=2k$ or there is an $l\in \N$ with $m=2l$\\
    $\Rightarrow$ there is a $k\in \N$ with $nm=2km$ or there is an $l\in \N$ mit $nm=2nl$ \\
    $\Rightarrow$ there is a $j\in \N$ with $nm=2j$ (namely $j=km$ or $j=ln$) \\
    $\Rightarrow$ $nm$ even, i.e. not odd.}
  \end{example}


%Video

\lang{de}{
\begin{example}
Im folgenden Video finden Sie ein anderes Beispiel für direkten und indirekten Beweis:

\floatright{\href{https://api.stream24.net/vod/getVideo.php?id=10962-2-10748&mode=iframe&speed=true}{\image[75]{00_video_button_schwarz-blau}}}\\
\\

\end{example}
}

  \lang{de}{Eine dritte Möglichkeit des Beweises der Implikation $\text{Voraussetzung} \Rightarrow \text{Behauptung}$ ergibt sich
  aus der dritten Implikation in Satz \ref{thm:equiv-to-implication}. Man zeigt dabei, die Implikation
  \[ \text{Voraussetzung und Negation der Behauptung} \Rightarrow \text{F}, \] 
  d.h. aus der Annahme, dass die Voraussetzung und die Negation der Behauptung beide wahr sind, leitet man eine Aussage
  her, die bekanntermaßen falsch ist (d.h. den Wahrheitswert F hat), also einen Widerspruch.
  Diese Form des Beweises heißt daher auch \emph{Beweis durch Widerspruch} oder \emph{Widerspruchsbeweis}.}
  \lang{en}{A third method to prove the implication $\text{Hypothesis} \Rightarrow \text{Conclusion}$ arises from the third implication in Theorem \ref{thm:equiv-to-implication}.
  For this, one shows the implication
  \[ \text{Hypothesis and negation of the conclusion} \Rightarrow \text{F}, \]
  i.e. from the assumption that both the hypothesis and the negation of the conclusion are true, we derive a statement which 
  is known to be false (i.e. has the truth value F), thus a contradiction. This type of proof is called \emph{proof by contradiction}.
  }
  
  \begin{example}
    \lang{de}{Es soll die Implikation 
    \[ x^2=2 \Rightarrow x\not\in \Q \]
    gezeigt werden. (Oder anders ausgedrückt: $\sqrt{2}$ ist keine rationale Zahl.)

    Hier führt man den Beweis am besten als Widerspruchsbeweis. Man zeigt also, dass die Aussage
    $x^2=2 \wedge x\in \Q$ eine (offensichtlich) falsche Aussage impliziert. D.h. man nimmt an, dass eine rationale
    Zahl $x$ die Gleichung $x^2=2$ erfüllt, und leitet daraus eine falsche Aussage her.

    Der ausführliche Beweis kann im \ref[link-ec-roots][Beweis, dass $\sqrt{2}$ nicht rational ist]{proof:sqrt2}, nachgelesen werden.}
    \lang{en}{We wish to prove the implication:
    \[ x^2=2 \Rightarrow x\not\in \Q \]
    (or stated differently: $\sqrt{2}$ is not a rational number.)
    
    Here the best approach is a proof by contradiction. To do this, one shows that the proposition  $x^2=2 \wedge x\in \Q$ implies a (clearly) false proposition.
    That is, one assumes that a rational number $x$ satisfies the equation $x^2=2$, and derives a false statement from that assumption.
    The detailed proof can be read in \ref[link-ec-roots][Proof that $\sqrt{2}$ is not rational]{proof:sqrt2}.
    }
  \end{example}

%Video
\lang{de}{
Neben einer Zusammenfassung der verschiedenen Arten von Beweisen enthält das folgende Video ein Beispiel für einen indirekten Beweis und ein 
Beispiel für einen Widerspruchsbeweis:

\floatright{\href{https://api.stream24.net/vod/getVideo.php?id=10962-2-10939&mode=iframe&speed=true}{\image[75]{00_video_button_schwarz-blau}}}\\
}

\end{content}