\documentclass{mumie.problem.gwtmathlet}
%$Id$
\begin{metainfo}
  \name{
    \lang{en}{...}
    \lang{de}{A04: unterschiedliche Kostenarten}
  }
  \begin{description} 
 This work is licensed under the Creative Commons License Attribution 4.0 International (CC-BY 4.0)   
 https://creativecommons.org/licenses/by/4.0/legalcode 

    \lang{en}{...}
    \lang{de}{...}
  \end{description}
  \corrector{system/problem/GenericCorrector.meta.xml}
  \begin{components}
    \component{js_lib}{system/problem/GenericMathlet.meta.xml}{gwtmathlet}
  \end{components}
  \begin{links}
  \end{links}
  \creategeneric
\end{metainfo}
\begin{content}
\lang{de}{\title{A04: Unterschiedliche Kostenarten}}
\lang{en}{\title{A04: Different types of costs}}
\begin{block}[annotation]
	Im Ticket-System: \href{https://team.mumie.net/issues/24088}{Ticket 24088}
\end{block}
\usepackage{mumie.genericproblem}

     \begin{problem}
            \begin{variables}
                \drawFromSet{a}{2,3,4}
                \drawFromSet{b}{7,8,9}
                \drawFromSet{c}{7,8,9}
                \drawFromSet{f}{4,5,6}
                \drawFromSet{d}{10,11,12}
                \drawFromSet{h}{2,3,4}
                \function{gk}{a*x^2-b*x+c}
                \function{gu}{-h*x+d}
                \function{sumk}{a*x^3/3-b*x^2/2+c*x+f}
                \function{sumu}{-h*x^2/2+d*x}
                \function{avk}{a*x^2/3-b*x/2+c+f/x}
                \function[normalize,expand]{win}{-sumk+sumu}
            \end{variables}
          \begin{question}
               \type{input.function} 
               \lang{de}{\text{Gegeben seien 
               \begin{table}%[\class{layout}]
               eine Grenzkostenfunktion: &  &$K'(x)=\var{gk}$\\ 
               fixe Kosten: &  &$K_f=\var{f}$ \\
               eine Grenzerlösfunktion: &  &$E'(x)=\var{gu}$
        
               \end{table}
               
               
               Berechnen Sie 
               \begin{itemize}
               \item
               die Gesamtkostenfunktion: $K_{total}(x)=$\ansref und 
               \item
               die Durchschnittskostenfunktion: $k_{total}(x)=$\ansref.
               \end{itemize}
               }}
               \lang{en}{\text{Given
               \begin{table}%[\class{layout}]
               a marginal cost function: & &$K'(x)=\var{gk}$\\ 
               a fixed cost: & &$K_f=\var{f}$\\
               a marginal revenue function: & &$E'(x)=\var{gu}$
              \end{table}
               Calculate 
               \begin{itemize}
               \item
               the total cost function: $K_{total}(x)=$\ansref and 
               \item
               the average cost function: $k_{total}(x)=$\ansref.
               \end{itemize}
               }}
               
               \explanation{$K_{ges}=\int_0^xK'(r)dr+K_f$ 
               \lang{de}{\text{und}}
                \lang{en}{\text{and}}
               $k_{ges}(x)=\frac{K_{ges}(x)}{x}$.
               }
               \begin{answer}
                    \solution{sumk}
                    \checkAsFunction{x}{-10}{10}{100}
               \end{answer}
               \begin{answer}
                    \solution{avk}
                    \checkAsFunction{x}{-10}{10}{100}
               \end{answer}
          \end{question}
          
           \begin{question}
               \type{input.function} 
               \lang{de}{\text{Berechnen Sie nun
               \begin{itemize}
               \item
               die Erlösfunktion: $E_{total}(x)=$\ansref, sowie
               \item
               die Gewinnfunktion: $G(x)=$\ansref.
               \end{itemize}
               }}
              \lang{en}{\text{Now calculate
               \begin{itemize}
               \item
               the revenue function: $E_{ges}(x)=$\ansref, as well as
               \item
               the profit function: $G(x)=$\ansref.
               \end{itemize}
               }}
               \explanation{$E_{ges}=\int_0^xE'(r)dr$ 
               \lang{de}{\text{und}}
                \lang{en}{\text{and}}
                $G(x)=E_{ges}(x)-K_{ges}(x)$.}
               \begin{answer}
                    \solution{sumu}
                     \checkAsFunction{x}{-10}{10}{100}
               \end{answer}
              \begin{answer}
                    \solution{win}
                     \checkAsFunction{x}{-10}{10}{100}
               \end{answer}   
          \end{question}  
     \end{problem}

\embedmathlet{gwtmathlet}

\end{content}
