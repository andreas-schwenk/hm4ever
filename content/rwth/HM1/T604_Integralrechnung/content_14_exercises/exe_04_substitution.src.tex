\documentclass{mumie.element.exercise}
%$Id$
\begin{metainfo}
  \name{
    \lang{de}{Ü04: Substitution}
    \lang{en}{Exercise 4: substitution}
  }
  \begin{description} 
 This work is licensed under the Creative Commons License Attribution 4.0 International (CC-BY 4.0)   
 https://creativecommons.org/licenses/by/4.0/legalcode 

    \lang{de}{}
    \lang{en}{}
  \end{description}
  \begin{components}
  \end{components}
  \begin{links}
  \end{links}
  \creategeneric
\end{metainfo}
\begin{content}
\title{\lang{de}{Ü04: Substitution}
    \lang{en}{Exercise 4: substitution}}
\begin{block}[annotation]
	Im Ticket-System: \href{https://team.mumie.net/issues/24049}{Ticket 24049}
\end{block}

\usepackage{mumie.ombplus}



\begin{block}[annotation]
Kopie: hm4mint/T305_Integrationsetechniken/exercise 3

Im Ticket-System: \href{http://team.mumie.net/issues/10921}{Ticket 10921}
\end{block}

%######################################################FRAGE_TEXT
\lang{de}{Bestimmen Sie mit Hilfe der Substitutionsregel die folgenden Integrale:}
\lang{en}{Use the substitution rule to calculate the following integrals:}

\[ a)\quad \int_1^e \frac{\ln(x)}{x}\; dx, \qquad\qquad b)\quad \int_0^2 \frac{x}{2}e^{-x^2}\; dx, \qquad\qquad c)\quad \int_{0}^{1} 3x^3\sin(x^2)\; dx.  \]


%##################################################ANTWORTEN_TEXT
\begin{tabs*}[\initialtab{0}\class{exercise}]

\tab{\lang{de}{Antworten}  \lang{en}{Answers}}

\begin{enumerate}[a)]
\item a) $\ \int_1^e \frac{\ln(x)}{x}\; dx= \frac{1}{2}$.
\item b) $\ \int_0^2 \frac{x}{2}e^{-x^2}\; dx = \frac{1}{4}- \frac{1}{4e^4}$.
\item \lang{de}{c) $\ \int_{0}^{1} 3x^3\sin(x^2)\; dx= \frac{3}{2}(\sin(1)- \cos(1))
\approx 0,4518$.}
\lang{en}{c) $\ \int_{0}^{1} 3x^3\sin(x^2)\; dx= \frac{3}{2}(\sin(1)- \cos(1))
\approx 0.4518$.}
\end{enumerate}
 
  %++++++++++++++++++++++++++++++++++++++++++START_TAB_X
  \tab{\lang{de}{    Lösung a)    }  \lang{en}{Solution a)}}
  \begin{incremental}[\initialsteps{1}]
  
  	 %----------------------------------START_STEP_X
    \step 
    \lang{de}{   Um die Substitutionsregel anwenden zu können, müssen wir den Integranden in der
Form $c\cdot g'(x)\cdot f(g(x))$ schreiben, wobei $c$ eine reelle Zahl ist, die wir dann vor
das Integral ziehen können, und $g$ und $f$ geeignete Funktionen sind.

Wählt man in diesem Fall $g(x)=\ln(x)$, so ist $g'(x)=\frac{1}{x}$ und der Integrand ist}
\lang{en}{To use the substitution rule, we need to write the integrand in the form
$c\cdot g'(x)\cdot f(g(x))$, where $c$ is a real number that can be pulled through
the integral sign, and where $g$ and $f$ are appropriate functions.}

\[  \frac{\ln(x)}{x}=g'(x)\cdot g(x). \]
\step 
\lang{de}{Wir können also $f(y)=y$ wählen und erhalten:}
\lang{en}{We can choose $f(y)=y$ and find:}
\begin{eqnarray*}
 \int_1^e \frac{\ln(x)}{x}\; dx &=& \int_1^e g'(x)\cdot f(g(x))\; dx\\
 &=& \int_{g(1)}^{g(e)} f(y)\; dy =  \int_{g(1)}^{g(e)} y\; dy \\
 &=& \Big[ \frac{y^2}{2} \Big] _{g(1)}^{g(e)} = \Big[  \frac{g(x)^2}{2} \Big]_1^e\\
 &=&  \Big[  \frac{\ln(x)^2}{2} \Big]_1^e = \frac{\ln(e)^2}{2}- \frac{\ln(1)^2}{2}=\frac{1}{2}.
\end{eqnarray*}    
  	 %------------------------------------END_STEP_X
 
  \end{incremental}
  %++++++++++++++++++++++++++++++++++++++++++++END_TAB_X


  %++++++++++++++++++++++++++++++++++++++++++START_TAB_X
  \tab{\lang{de}{    Lösung b)    }  \lang{en}{Solution b)}}
  \begin{incremental}[\initialsteps{1}]
  
  	 %----------------------------------START_STEP_X
    \step 
    \lang{de}{  Um die Substitutionsregel anwenden zu können, müssen wir den Integranden in der
Form $c\cdot g'(x)\cdot f(g(x))$ schreiben, wobei $c$ eine reelle Zahl ist, die wir dann vor
das Integral ziehen können, und $g$ und $f$ geeignete Funktionen sind.

Da der Problemterm $e^{-x^2}$ ist, ist der erste Versuch, $g(x)=-x^2$ zu setzen. Dann ist
$g'(x)=-2x$ und daher}

\lang{en}{To use the substitution rule, we need to write the integrand in the
form $c\cdot g'(x)\cdot f(g(x))$, where $c$ is a real number that can be pulled through the integral
sign, and where $g$ and $f$ are appropriate functions.
Since $e^{-x^2}$ is the problematic term, our first attempt will be to set $g(x)=-x^2$.
Then $g'(x)=-2x$ and therefore
}
\[   \frac{x}{2}e^{-x^2} =\frac{g'(x)}{-4}\cdot e^{g(x)}. \]
\step 
\lang{de}{Mit $f(y)=e^y$ erhalten wir also die gesuchte Form. Damit ist:}
\lang{en}{So $f(y)=e^y$ yields the desired form. Therefore,}
\begin{eqnarray*}
\int_0^2 \frac{x}{2}e^{-x^2}\; dx &=& \int_0^2 \frac{g'(x)}{-4}\cdot f(g(x))\; dx
=-\frac{1}{4}\cdot  \int_0^2 g'(x)\cdot f(g(x))\; dx\\
&=& -\frac{1}{4}\cdot \int_{g(0)}^{g(2)} f(y)\; dy =   -\frac{1}{4}\cdot \int_{g(0)}^{g(2)} e^y\; dy \\
&=&   -\frac{1}{4}\cdot \Big[ e^y \Big]_{g(0)}^{g(2)} =  \left[  -\frac{e^y}{4} \right]_{g(0)}^{g(2)} \\
&=& \left[  -\frac{e^{-x^2}}{4} \right]_0^2 \\
&=&  -\frac{e^{-4}}{4} +  \frac{e^{0}}{4} = \frac{1}{4}- \frac{1}{4e^4}.
\end{eqnarray*}
   
  	 %------------------------------------END_STEP_X
 
  \end{incremental}
  %++++++++++++++++++++++++++++++++++++++++++++END_TAB_X


  %++++++++++++++++++++++++++++++++++++++++++START_TAB_X
  \tab{\lang{de}{    Lösung c)    } \lang{en}{Solution c)}}
  \begin{incremental}[\initialsteps{1}]
  
  	 %----------------------------------START_STEP_X
    \step 
    \lang{de}{   
Um die Substitutionsregel anwenden zu können, müssen wir den Integranden in der
Form $c\cdot g'(x)\cdot f(g(x))$ schreiben, wobei $c$ eine reelle Zahl ist, die wir dann vor
das Integral ziehen können, und $g$ und $f$ geeignete Funktionen sind.

Da der Problemterm $\sin(x^2)$ ist, ist der erste Versuch, $g(x)=x^2$ zu setzen. Dann ist
$g'(x)=2x$ und daher }
\lang{en}{To use the substitution rule, we need to write the integrand in the
form $c\cdot g'(x)\cdot f(g(x))$, where $c$ is a real number that can be pulled through the integral
sign, and where $g$ and $f$ are appropriate functions.
Since $\sin(x^2)$ is the problematic term, our first attempt will be to set $g(x)=x^2$.
Then $g'(x)=2x$ and therefore
}
\[ 3x^3\sin(x^2)=g'(x)\cdot \frac{3}{2}x^2\sin(g(x))=\frac{3}{2}\cdot g'(x)\cdot  g(x)\sin(g(x)).\]

\step 
\lang{de}{Setzt man also $f(y)=y\sin(y)$, so erhalten wir die gesuchte Form. Damit ist:}
\lang{en}{So if we set $f(y)=y\sin(y)$, then we get the desired form. Therefore,}
\begin{eqnarray*}
\int_{0}^{1} 3x^3\sin(x^2)\; dx &=& \int_{0}^{1} \frac{3}{2} \cdot \textcolor{#CC6600}{2x} \cdot \textcolor{#0066CC}{x^2 \sin(x^2)} \; dx \\
&=& \int_{0}^{1} \frac{3}{2}\cdot \textcolor{#CC6600}{g'(x)}\cdot \textcolor{#0066CC}{f(g(x))}\; dx\\
&=& \int_{g(0)}^{g(1)} \frac{3}{2}\cdot f(y)\; dy =  \frac{3}{2}\cdot \int_{g(0)}^{g(1)}  y\sin(y) \; dy\, .
\end{eqnarray*}
\step 
\lang{de}{Nun verbleibt, eine Stammfunktion für $y\sin(y)$ zu finden bzw. dieses Integral zu berechnen. Dies erreicht man am besten mittels partieller Integration, wie im vorigen Abschnitt erklärt}
\lang{en}{We must still find an antiderivative for $y\sin(y)$, i.e. to compute the above integral. The best way to do this is via integration by parts, as explained in the previous section.}
\lang{de}{mit $u(y)=y$ und $v'(y)=\sin(y)$ (und $u'(y)=1$, $v(y)=-\cos(y)$):}
\lang{en}{Take $u(y)=y$ and $v'(y)=\sin(y)$ (and $u'(y)=1$, $v(y)=-\cos(y)$):}
\begin{eqnarray*}
 \int_ {\alpha}^{\beta}  y\sin(y) \; dy\, &=& \left[ y\cdot (-\cos(y)) \right]_ {\alpha}^{\beta}
 -  \int_ {\alpha}^{\beta} 1\cdot (-\cos(y)) \; dy \\
 &=&  \left[ -y\cos(y) \right]_ {\alpha}^{\beta} + \left[ \sin(y) \right]_ {\alpha}^{\beta} 
 =  \left[ -y\cos(y)+ \sin(y) \right]_ {\alpha}^{\beta}.
\end{eqnarray*}
\lang{de}{D.\,h. $-y\cos(y)+\sin(y)$ ist eine Stammfunktion für $y\sin(y)$.}
\lang{en}{In other words, $-y\cos(y)+\sin(y)$ is an antiderivative of $y\sin(y)$.}

\step 
\lang{de}{Damit ist schließlich:
\begin{eqnarray*}
\int_{0}^{1} 3x^3\sin(x^2)\; dx &=&  \frac{3}{2}\cdot \int_{g(0)}^{g(1)}  y\sin(y) \; dy \\
&=&  \frac{3}{2}\cdot   \left[ -y\cos(y)+ \sin(y) \right]_{g(0)}^{g(1)} 
=  \frac{3}{2}\cdot    \left[ -x^2\cos(x^2)+\sin(x^2) \right]_{0}^1 \\
&=&   \frac{3}{2}\cdot \Big( -1\cdot \cos(1)+\sin(1) -(0+0) \Big) = \frac{3}{2}(\sin(1)- \cos(1))\\
&\approx& 0,4518.
\end{eqnarray*}
}
\lang{en}{Altogether,
\begin{eqnarray*}
\int_{0}^{1} 3x^3\sin(x^2)\; dx &=&  \frac{3}{2}\cdot \int_{g(0)}^{g(1)}  y\sin(y) \; dy \\
&=&  \frac{3}{2}\cdot   \left[ -y\cos(y)+ \sin(y) \right]_{g(0)}^{g(1)} 
=  \frac{3}{2}\cdot    \left[ -x^2\cos(x^2)+\sin(x^2) \right]_{0}^1 \\
&=&   \frac{3}{2}\cdot \Big( -1\cdot \cos(1)+\sin(1) -(0+0) \Big) = \frac{3}{2}(\sin(1)- \cos(1))\\
&\approx& 0.4518.
\end{eqnarray*}
}
  	 %------------------------------------END_STEP_X
 
  \end{incremental}
  %++++++++++++++++++++++++++++++++++++++++++++END_TAB_X


%#############################################################ENDE

    \tab{\lang{de}{Video: ähnliche Übungsaufgabe} \lang{en}{Video: similar exercise}}
  \youtubevideo[500][300]{t97KM2yTnwQ}\\

\end{tabs*}
\end{content}