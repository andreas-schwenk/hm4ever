\documentclass{mumie.element.exercise}
%$Id$
\begin{metainfo}
  \name{
    \lang{en}{Exercise 5: consumer and producer surplus}
    \lang{de}{Ü05: Konsumenten- und Produzentenrente}
  }
  \begin{description} 
 This work is licensed under the Creative Commons License Attribution 4.0 International (CC-BY 4.0)   
 https://creativecommons.org/licenses/by/4.0/legalcode 

    \lang{en}{...}
    \lang{de}{...}
  \end{description}
  \begin{components}
  \component{generic_image}{content/rwth/HM1/images/g_tkz_T604_14_Exercise05.meta.xml}{T604_14_Exercise05}
  \end{components}
  \begin{links}
  \end{links}
  \creategeneric
\end{metainfo}
\begin{content}
\title{\lang{en}{Exercise 5: consumer and producer surplus}
    \lang{de}{Ü05: Konsumenten- und Produzentenrente}}
\begin{block}[annotation]
	Im Ticket-System: \href{https://team.mumie.net/issues/24089}{Ticket 24089}
\end{block}

\lang{de}{Berechnen Sie die Konsumenten- und die Produzentenrente für die 
Angebotsfunktion $p_A(x)=0,5\cdot x+1$ und die Nachfragefunktion $p_N(x)=-0,5\cdot x+9$.}
\lang{en}{Compute the consumer and producer surplus for the
supply function $p_A(x)=0.5 \cdot x+1$ and the demand function $p_N(x)=-0.5\cdot x+9$.}

\begin{center}
\image{T604_14_Exercise05}
\end{center}
  \begin{tabs*}[\initialtab{0}\class{exercise}]
    \tab{
      \lang{en}{Solution}
      \lang{de}{Lösung}
      \lang{zh}{...}
      \lang{fr}{...}
    }
    \begin{incremental}[\initialsteps{1}]
      \step
        \lang{en}{The producer surplus is 16. \\
        The consumer surplus is also 16.\\
        The total welfare is therefore 32.
        
        We deliberately did not include units here. The units could be one thousand \euro, for example.
        }
        \lang{de}{Die Produzentenrente beträgt 16.\\
        Die Konsumentenrente beträgt ebenfalls 16.\\
        
        Die Wohlfahrt beträgt also 32.
        
        Auf Einheiten wurde hier bewusst verzichtet, es könnten beispielsweise Tausende \euro sein.
        }
        \lang{zh}{...}
        \lang{fr}{...}
      
    \end{incremental}
    \tab{
      \lang{en}{Explanation}
      \lang{de}{Erklärung}
      \lang{zh}{...}
      \lang{fr}{...}
    }
    \begin{incremental}[\initialsteps{1}]
      \step
        \lang{en}{Consider the graph. x denotes the amount and $p(x)$ the price.
        The demand function is monotonically decreasing: the higher the price, the lower the demand among the consumers.
        The opposite holds for the supply function, which is monotonically increasing: the higher the price,
        the more willing the supplier is to produce the goods. There is a large profit to be made.
        Ideally, the market price stabilizes itself where supply and demand meet: at the equilibrium quantity $x^{\ast}$ and
        the equilibrium price $p^{\ast}$.

        \\The demand function indicates how much the buyer is willing to pay for the product; the supply function indicates
        the lowest price the supplier must charge in order to produce. \\
        In the blue triangle, the supplier receives more than he has to charge, i.e. he is satisfied and is welfare increases.
        This is called the \notion{producer surplus} $PR$.
        For the buyer, this holds for the red triangle: he is satisfied, because he would buy the product at an even higher price.
        This is called the \notion{consumer surplus} $KR$.

        The \notion{total welfare} refers to the sum of the consumer and the producer surpluses,
        because this is the amount by which the welfare of society increases.

        }
        \lang{de}{Wir betrachten den Graphen: x bezeichnet die Menge und $p(x)$ den Preis.
        Die Nachfragefunktion ist monoton fallend: je größer der Preis, desto geringer ist die Nachfrage der
        Konsumenten. Genau umgekehrt verhält sich die Angebotsfunktion, sie ist monoton steigend: je größer der Preis,
        desto williger ist der Produzent, die Ware herzustellen. Ein hoher Gewinn winkt. Idealerweise stellt sich der Marktpreis ein, 
        wo sich Angebot und Nachfrage treffen: bei der Gleichgewichtsmenge $x^{\ast}$ und dem Gleichgewichtspreis $p^{\ast}$.
        
        \\Die Nachfragefunktion gibt an, wieviel der Käufer bereit ist für das Produkt zu zahlen, die Angebotsfunktion gibt an, welchen Preis 
        der Produzent mindestens verlangen muss um zu produzieren. \\
        Im blauen Dreieck bekommt der Produzent mehr als er mindestens verlangen muss, d.\,h. er ist zufrieden, 
        sein Wohlstand wächst. Dies nennt man die \notion{Produzentenrente} $PR$.
        Ebenso der Käufer im roten Dreieck: auch er ist zufrieden, denn er würde sogar zu einem höheren Preis noch kaufen. Dies nennt man die
        \notion{Konsumentenrente} $KR$.\\
        
        Die \notion{ökonomische Wohlfahrt} bezeichnet die Summe aus der Konsumenten- und der Produzentenrente, denn das ist der Betrag,
        um den der Wohlstand in der Gesellschaft steigt.
        }
        \lang{zh}{...}
        \lang{fr}{...}
      
    \end{incremental}
   \tab{
      \lang{en}{Calculation}
      \lang{de}{Rechnung}
      \lang{zh}{...}
      \lang{fr}{...}
    }
    \begin{incremental}[\initialsteps{1}]
      \step
        \lang{en}{First we find the intersection of $p_N(x)$ and $p_A(x)$:
        \begin{eqnarray*}
            -0.5x+9&=&0.5x+1\\
           \Leftrightarrow \ x&=&8
        \end{eqnarray*}
        The equilibrium quantity is $x^{\ast}=8$ and the equilibrium price is $p^{\ast}=-0.5 \cdot 8 + 9=5$.
        Now we calculate the consumer surplus by calculating the red triangle:
        $F_{\Delta}=\frac{g\cdot h}{2}$.
        So $KR=\frac{8\cdot (9-5)}{2}=16$. The producer surplus is the same: $PR=16$.

        The total welfare is the the sum of the consumer and the producer surpluses,
        because that is the amount by which the welfare of society increases: Welfare=$32$.
}
        \lang{de}{Zuerst berechnen wir den Schnittpunkt von $p_N(x)$ und $p_A(x)$:
        \begin{eqnarray*}
            -0,5x+9&=&0,5x+1\\
           \Leftrightarrow \ x&=&8
        \end{eqnarray*}
        Die Gleichgewichtsmenge ist $x^{\ast}=8$ und der Gleichgewichtspreis $p^{\ast}=-0,5\cdot 8+9=5$.
        
        Nun berechnen wir die Konsumentenrente, indem wir das rote Dreieck berechnen: 
        $F_{\Delta}=\frac{g\cdot h}{2}$.
        
        Also ist $KR=\frac{8\cdot (9-5)}{2}=16$. Die Produzentenrente hat denselben Betrag: $PR=16$
        
        Die ökonomische Wohlfahrt bezeichnet die Summe aus der Konsumenten- und der Produzentenrente, denn das ist der Betrag,
        um den der Wohlstand in der Gesellschaft steigt: Wohlfahrt$=32$.
        }
        \lang{zh}{...}
        \lang{fr}{...}
      
    \end{incremental} 
  \end{tabs*}



\end{content}

