\documentclass{mumie.element.exercise}
%$Id$
\begin{metainfo}
  \name{
    \lang{de}{Ü02: partielle Integration II}
    \lang{en}{Exercise 2: integration by parts II}
  }
  \begin{description} 
 This work is licensed under the Creative Commons License Attribution 4.0 International (CC-BY 4.0)   
 https://creativecommons.org/licenses/by/4.0/legalcode 

    \lang{de}{Hier die Beschreibung}
    \lang{en}{}
  \end{description}
  \begin{components}
  \end{components}
  \begin{links}
    \link{generic_article}{content/rwth/HM1/T209_Potenzreihen/g_art_content_28_exponentialreihe.meta.xml}{link1}
  \end{links}
  \creategeneric
\end{metainfo}
\begin{content}
\title{\lang{de}{Ü02: partielle Integration II}
    \lang{en}{Exercise 2: integration by parts II}}
\begin{block}[annotation]
	Im Ticket-System: \href{https://team.mumie.net/issues/24047}{Ticket 24047}
\end{block}
\begin{block}[annotation]
	Kopie: hm4mint/T305_Integrationstechniken/exercise 4
    
    Im Ticket-System: \href{http://team.mumie.net/issues/10922}{Ticket 10922}
\end{block}



\lang{de}{Bestimmen Sie mit Hilfe von partieller Integration eine Stammfunktion von \\
a) $\ f(x)=\sin(x)\cos(x)$ und \\
b) $\ g(x)=\exp(x)\sin(x)$.}
\lang{en}{Use integration by parts to find an antiderivative of \\
a) $\ f(x)=\sin(x)\cos(x)$ and \\
b) $\ g(x)=\exp(x)\sin(x)$.}

\begin{tabs*}[\initialtab{0}\class{exercise}]

 \tab{\lang{de}{Antwort}
 \lang{en}{Answer}}

\lang{de}{
a) Mögliche Stammfunktionen von $f$ sind $F(x)=\frac{1}{2}\sin(x)^2+C$ oder $F(x)=-\frac{1}{2}\cos(x)^2+C$ mit $C \in \R$.\\
b) Eine mögliche Stammfunktion von $g$ ist $G(x)=\frac{1}{2}\exp(x)\cdot(\sin(x)-\cos(x))$.
}

\lang{en}{
a) Possible antiderivatives of $f$ are $F(x)=\frac{1}{2}\sin(x)^2+C$ or $F(x)=-\frac{1}{2}\cos(x)^2+C$ with $C \in \R$.\\
b) One possible antiderivtive of $g$ is $G(x)=\frac{1}{2}\exp(x)\cdot(\sin(x)-\cos(x))$.
}

\tab{\lang{de}{L"osung zu a)}
\lang{en}{Solution a)}
}

 \begin{incremental}[\initialsteps{1}]
  \step \lang{de}{Wir führen eine partielle Integration aus. Wir wählen $u(x)=\sin(x)$ und $v'(x)=\cos(x)$. Wir werden gleich
  noch die Rollen vertauschen und sehen, dass eine Vertauschung der beiden Rollen von $u$ und $v'$ auch zum Ziel führt.}
  \lang{en}{We will integrate by parts. We choose $u(x)=\sin(x)$ and $v'(x) = \cos(x)$.
  We will change the roles of $u$ and $v$ below and see that the other choice of $u$ and $v'$ would also work.}
  
  \step \lang{de}{Also sei $u(x)=\sin(x)$ und $v'(x)=\cos(x)$. Dann sind $u'(x)=\cos(x)$ und $v(x)=\sin(x)$.
  Damit gilt für $a<b$ nach partieller Integration:}
  \lang{en}{So let $u(x)=\sin(x)$ and $v'(x)=\cos(x)$. Then $u'(x)=cos(x)$ and $v(x)=\sin(x)$.
  For $a<b$, integration by parts yields:}
  \[
  \int_a^b \textcolor{#0066CC}{\sin(x)\cos(x)} \, dx = \int_a^b u(x)v'(x)\, dx = [\sin(x)\sin(x)]_a^b - \int_a^b \cos(x)\sin(x)\, dx
  = [\sin(x)^2]_a^b - \int_a^b \textcolor{#0066CC}{\sin(x)\cos(x)}\, dx.
\]
 
\step \lang{de}{Wir haben nun eine Gleichung, die wir auflösen können, um eine Stammfunktion zu erhalten:}
\lang{en}{Now we have an equation that we can solve to find an antiderivative:}
\[
2 \int_a^b \textcolor{#0066CC}{\sin(x)\cos(x)}\, dx = [\sin(x)^2]_a^b.
\]
\step \lang{de}{Eine Stammfunktion ist also durch $F_1(x)=\frac{1}{2}\sin(x)^2$ gegeben.}
\lang{en}{Therefore, $F_1(x) = \frac{1}{2}\sin(x)^2$ is an antiderivative.}

\step \lang{de}{Nun vergleichen wir dieses Ergebnis mit dem, das wir erhalten, wenn wir $u(x)=\cos(x)$ und $v'(x) = \sin(x)$ gewählt hätten.
Dann ist $u'(x) = -\sin(x)$ und $v(x)= -\cos(x)$.}
\lang{en}{Now we will compare this to the result of taking $u(x)=\cos(x)$ and $v'(x)=\sin(x)$.
Then $u'(x) = -\sin(x)$ and $v(x)= -\cos(x)$.}

\step \lang{de}{Damit gilt für $a<b$} \lang{en}{So, for $a<b$,}
\[
\int_a^b \textcolor{#0066CC}{\sin(x)\cos(x)} \, dx = \int_a^b u(x)v'(x)\, dx = [\cos(x)(-\cos(x))]_a^b - \int_a^b (-\sin(x))(-\cos(x))\, dx
  = [-\cos(x)^2]_a^b - \int_a^b \textcolor{#0066CC}{\sin(x)\cos(x)}\, dx.
\]
\step \lang{de}{Wenn wir hier die Gleichung auflösen, dann erhalten wir eine mögliche Stammfunktion}
\lang{en}{When we solve this equation, we find the antiderivative}
\[
F_2(x)=-\frac{1}{2}\cos(x)^2.
\]
\step \lang{de}{Wir haben nun zwei verschiedene Stammfunktionen von $f$ bestimmt. Wie passt das zusammen? Haben wir einen Fehler gemacht?
Wir wissen bereits, dass Stammfunktionen eindeutig bis auf Addition einer Konstanten sind.
Das heißt, es müsste ein $C \in \R$ geben, sodass}
\lang{en}{Now we have found two different antiderivatives of $f$. How is this possible? Have we made a mistake?
We know that antiderivatives are unique up to addition of a constant.
That means that there must be some $C \in \R$ such that}
\[
C= F_1(x)-F_2(x)=\frac{1}{2} \sin(x)^2 - \left(-\frac{1}{2}\cos(x)^2\right)= \frac{1}{2}(\sin(x)^2+\cos(x)^2).
\]
\lang{de}{Allgemein gilt in der Mathematik $\sin(x)^2+\cos(x)^2=1$ für alle $x\in \R$. Damit ist $C=\frac{1}{2}$ und beide Ergebnisse sind kein Widerspruch.}
\lang{en}{In general, $\sin(x)^2+\cos(x)^2=1$ for $x \in \R$. Therefore, $C=\frac{1}{2}$ and the two results do not lead to a contradiction.}
\end{incremental}

\tab{\lang{de}{Lösung zu b)} \lang{en}{Solution b)}}
\begin{incremental}[\initialsteps{1}]
\step \lang{de}{Wir führen eine partielle Integration durch mit $u(x)=\sin(x)$ und $v'(x)=\exp(x)$. (Sie können die beiden Wahlen aber auch vertauschen. Dies führt zum gleichen Ergebnis.)
Dann ist $u'(x)=\cos(x)$ und $v(x)=\exp(x)$.}
\lang{en}{We integrate by parts with $u(x)=\sin(x)$ and $v'(x)=\exp(x)$.
(You could also swap the roles of $u$ and $v$. That will lead to the same result.)
Then $u'(x)=\cos(x)$ and $v(x)=\exp(x)$.}
\step \lang{de}{Damit gilt für $a<b$} \lang{en}{So, for $a<b$,}
\[
\int_a^b \textcolor{#0066CC}{\exp(x)\sin(x)}\, dx = [\exp(x)\sin(x)]_a^b - \int_a^b \textcolor{#CC6600}{\cos(x)\exp(x)}\, dx.
\]
\step \lang{de}{Das ursprüngliche Integral erkennen wir hier noch nicht wieder. Wir führen eine erneute partielle Integration durch.
Nun setzen wir $u(x)=\cos(x)$ und $v'(x)=\exp(x)$. Wichtig ist hier, dass wir wieder $v'(x)=\exp(x)$ setzen, ansonsten würden wir wieder zum Anfangspunkt zurückkehren und so gesehen unsere partielle Integration rückgängig machen.
Wir haben dann $u'(x)=-\sin(x)$ und $v(x)=\exp(x)$. Und für $a<b$ haben wir dann}
\lang{en}{We have not yet recovered the original integral. We will integrate by parts again.
Now we set $u(x)=\cos(x)$ and $v'(x)=\exp(x)$. It is important that we take $v'(x)=\exp(x)$ again, as we would otherwise return to the beginning, effectively undoing our first integration by parts.
Then we have $u'(x)=-\sin(x)$ and $v(x)=\exp(x)$. Then, for $a<b$, we have
}
\[
\int_a^b \textcolor{#CC6600}{\cos(x)\exp(x)}\, dx = [\cos(x)\exp(x)]_a^b-\int_a^b (-\sin(x))\exp(x)\, dx = [\cos(x)\exp(x)]_a^b + \int_a^b \textcolor{#0066CC}{\exp(x)\sin(x)}\, dx.
\]

\step \lang{de}{Setzen wir dies zusammen, gilt} \lang{en}{Putting this together, we have}
\[
\int_a^b \textcolor{#0066CC}{\exp(x)\sin(x)}\, dx = [\exp(x)\sin(x) - \exp(x)\cos(x)]_a^b - \int_a^b \textcolor{#0066CC}{\exp(x)\sin(x)}\, dx.
\]
\lang{de}{Die Gleichung ergibt} \lang{en}{This equation yields}
\[
\int_a^b \textcolor{#0066CC}{\exp(x)\sin(x)}\, dx = \frac{1}{2}[\exp(x)\sin(x)-\exp(x)\cos(x)]_a^b.
\]
\lang{de}{Damit ist $\frac{1}{2}\exp(x)\cdot(\sin(x)-\cos(x))$ eine gesuchte Stammfunktion.}
\lang{en}{So $\frac{1}{2}\exp(x)\cdot(\sin(x)-\cos(x))$ is an antiderivative.}
\end{incremental}
\end{tabs*}


\end{content}