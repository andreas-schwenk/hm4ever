\documentclass{mumie.element.exercise}
%$Id$
\begin{metainfo}
  \name{
    \lang{en}{Exercise 2: improper integrals}
    \lang{de}{Ü02: uneigentliche Integrale}
  }
  \begin{description} 
 This work is licensed under the Creative Commons License Attribution 4.0 International (CC-BY 4.0)   
 https://creativecommons.org/licenses/by/4.0/legalcode 

    \lang{en}{...}
    \lang{de}{...}
  \end{description}
  \begin{components}
  \end{components}
  \begin{links}
  \link{generic_article}{content/rwth/HM1/T303_Approximationen/g_art_content_06_de_l_hospital.meta.xml}{link1}
  \end{links}
  \creategeneric
\end{metainfo}
\begin{content}
\title{\lang{en}{Exercise 2: improper integrals}
    \lang{de}{Ü02: uneigentliche Integrale}}
\begin{block}[annotation]
	Im Ticket-System: \href{https://team.mumie.net/issues/24093}{Ticket 24093}
\end{block}
\begin{block}[annotation]
	Kopie: hm4mint/T304_Integrierbarkeit/exercise 10
    
    Im Ticket-System: \href{http://team.mumie.net/issues/10642}{Ticket 10642}
\end{block}


\usepackage{mumie.ombplus}




%######################################################FRAGE_TEXT
\lang{de}{ Welche der folgenden uneigentlichen Integrale existieren? Berechnen Sie gegebenenfalls
das Integral.}
\lang{en}{Which of the following improper integrals exist?
Compute the integral, if possible.}

\begin{enumerate}[a)]
\item a) $\ \int_1^\infty \frac{x^2+1}{x} \; dx$
\item b) $\ \int_0^1 \frac{(x-1)(e^x-1)}{x^2}\; dx$ 
\item c) $\ \int_1^2 \frac{x^2-4x+4}{x-2}\; dx$
\item d) $\ \int_0^{\infty} \frac{(2x+1)e^{-x}}{\sqrt{x}^3}\, dx$
\end{enumerate}
\lang{de}{\textit{Hinweis zu b)}: Eine Stammfunktion zu 
$f:(0;\infty)\to \R, \ x\mapsto \frac{(x-1)(e^x-1)}{x^2}$ ist 
$F(x)=\frac{e^x-1}{x}-\ln(x)$\\
\textit{Hinweis zu d):} Eine Stammfunktion zu 
$g:(0;\infty)\to \R, \ x\mapsto\frac{(2x+1)e^{-x}}{\sqrt{x}^3}$ ist
$G(x)=\frac{-2e^{-x}}{\sqrt{x}}$.}
\lang{en}{\textit{Hint for b)}: $F(x)=\frac{e^x-1}{x}-\ln(x)$ is an antiderivative of
$f:(0;\infty)\to \R, \ x\mapsto \frac{(x-1)(e^x-1)}{x^2}$.\\
\textit{Hint for d):} $G(x)=\frac{-2e^{-x}}{\sqrt{x}}$ is an antiderivative of
$g:(0;\infty)\to \R, \ x\mapsto\frac{(2x+1)e^{-x}}{\sqrt{x}^3}$.}


%##################################################ANTWORTEN_TEXT
\begin{tabs*}[\initialtab{0}\class{exercise}]

	%++++++++++++++++++++++++++++++++++++++++++START_TAB_X
  \tab{\lang{de}{    Antworten    } \lang{en}{Answers}}
    \lang{de}{   Die uneigentlichen Integrale a), b) und d) existieren nicht.
Für das  uneigentliche Integral in c) gilt
$ \int_1^2 \frac{x^2-4x+4}{x-2}\; dx =-\frac{1}{2}$.    }
\lang{en}{The improper integrals a), b) and d) do not exist.
The improper integral in c) is
$ \int_1^2 \frac{x^2-4x+4}{x-2}\; dx =-\frac{1}{2}$.
}
  %++++++++++++++++++++++++++++++++++++++++++++END_TAB_X

  %++++++++++++++++++++++++++++++++++++++++++START_TAB_X
  \tab{\lang{de}{    Lösung a)    } \lang{en}{Solution a)}}
%   \begin{incremental}[\initialsteps{1}]
  
%   	 %----------------------------------START_STEP_X
%     \step 
% Um eine Stammfunktion von $f(x)=\frac{x^2+1}{x}$ zu bestimmen, zerlegen wir zunächst
% den Bruch und erhalten $f(x)=x+\frac{1}{x}$. Damit ist eine Stammfunktion
% im Bereich der positiven reellen Zahlen gegeben durch $F(x)=\frac{x^2}{2}+\ln(x)$.

% \step
% Das uneigentliche Integral ist dann gleich dem folgenden Grenzwert (sofern dieser existiert):
% \[ \lim_{b\to \infty} \int_1^b \frac{x^2+1}{x} \; dx
% =  \lim_{b\to \infty} \left[ \frac{x^2}{2}+\ln(x) \right]_1^b 
% =  \lim_{b\to \infty}  \frac{b^2}{2}+\ln(b)- ( \frac{1}{2}+\ln(1) ). 
% \]
% \step
% Jedoch ist $ \lim_{b\to \infty}  \frac{b^2}{2}+\ln(b)=\infty$, weshalb der Grenzwert nicht
% existiert und daher das uneigentliche Integral auch nicht existiert.

% \step
% Dass das uneigentliche Integral nicht existiert, hätte man hier auch grafisch sehen können, 
% da nämlich die Funktionswerte von $f(x)=x+\frac{1}{x}$ gegen unendlich divergieren. Dann werden bei Vergrößern der oberen Grenze der bestimmten 
% Integrale immer größere Flächen hinzugenommen, weshalb
% der Grenzwert der Integrale nicht existieren kann.    }
%   	 %------------------------------------END_STEP_X
 
%   \end{incremental}
  %++++++++++++++++++++++++++++++++++++++++++++END_TAB_X
\lang{de}{Es ist $f(x)=\frac{x^2+1}{x}=x+\frac{1}{x}$ für alle $x\geq 1$.
Es gilt $\lim_{x\to \infty} f(x) = \infty$. Damit kann das uneigentliche Integral nicht existieren.}

\lang{en}{We have $f(x)=\frac{x^2+1}{x}=x+\frac{1}{x}$ for all $x\geq 1$.
Since $\lim_{x\to \infty} f(x) = \infty$, the improper integral cannot exist.}

  %++++++++++++++++++++++++++++++++++++++++++START_TAB_X
  \tab{\lang{de}{    Lösung b)    } \lang{en}{Solution b)}}
  \begin{incremental}[\initialsteps{1}]
  
  	 %----------------------------------START_STEP_X
    \step 
\lang{de}{Die Funktion $f(x)=\frac{(x-1)(e^x-1)}{x^2}$ ist bei $0$ nicht definiert. Daher ist das
Integral $\int_0^1 \frac{(x-1)(e^x-1)}{x^2}\; dx$ ein uneigentliches Integral, sofern es überhaupt existiert, und gegeben durch
\[ \int_0^1 \frac{(x-1)(e^x-1)}{x^2}\; dx =\lim_{a\searrow 0} \int_a^1 \frac{(x-1)(e^x-1)}{x^2}\; dx. \]}
\lang{en}{The function $f(x)=\frac{(x-1)(e^x-1)}{x^2}$ is not defined in $0$. This makes
$\int_0^1 \frac{(x-1)(e^x-1)}{x^2}\; dx$ an improper integral, assuming it exists at all, given by
\[ \int_0^1 \frac{(x-1)(e^x-1)}{x^2}\; dx =\lim_{a\searrow 0} \int_a^1 \frac{(x-1)(e^x-1)}{x^2}\; dx. \]}

\step
\lang{de}{Da nach dem Hinweis $F(x)=\frac{e^x-1}{x}-\ln(x)$ eine Stammfunktion zu $f(x)$ ist, gilt für jedes $a>0$:
\[ \int_a^1 \frac{(x-1)(e^x-1)}{x^2}\; dx=\left[ \frac{e^x-1}{x}-\ln(x) \right]_a^1
= \left( \frac{e^1-1}{1}-\ln(1) \right)- \left( \frac{e^a-1}{a}-\ln(a) \right)
= (e-1) - \frac{e^a-1}{a}+\ln(a).\]

Zu bestimmen ist also das Grenzverhalten von $(e-1) - \frac{e^a-1}{a}+\ln(a)$ für $a\searrow 0$.}

\lang{en}{According to the hint, $F(x)=\frac{e^x-1}{x}-\ln(x)$ is an antiderivative of $f(x)$.
Therefore, for every $a > 0$,
\[ \int_a^1 \frac{(x-1)(e^x-1)}{x^2}\; dx=\left[ \frac{e^x-1}{x}-\ln(x) \right]_a^1
= \left( \frac{e^1-1}{1}-\ln(1) \right)- \left( \frac{e^a-1}{a}-\ln(a) \right)
= (e-1) - \frac{e^a-1}{a}+\ln(a).\]
}

\step
\lang{de}{Mit der Regel von de l'Hospital
erhält man
\[ \lim_{a\searrow 0} \frac{e^a-1}{a} =\lim_{a\searrow 0} \frac{e^a}{1}=e^0=1.\]
Andererseits ist $ \lim_{a\searrow 0} \ln(a)=-\infty$ und damit
divergiert der Ausdruck $\left( (e-1) - \frac{e^a-1}{a}+\ln(a)\right)$ gegen $-\infty$ für
$a\searrow 0$.\\ }

\lang{en}{L'Hôpital's rule yields
\[ \lim_{a\searrow 0} \frac{e^a-1}{a} =\lim_{a\searrow 0} \frac{e^a}{1}=e^0=1.\]
On the other hand, $\lim_{a\searrow 0}\, \ln(a)=-\infty$, so the expression
$\left( (e-1) - \frac{e^a-1}{a}+\ln(a)\right)$ diverges to $-\infty$ as $a\searrow 0$.
}

\step
\lang{de}{Der Grenzwert existiert also nicht, weshalb das uneigentliche Integral nicht existiert.}
\lang{en}{So the limit does not exist, which 
means that the improper integral does not exist.}
  	 %------------------------------------END_STEP_X
 
  \end{incremental}
  %++++++++++++++++++++++++++++++++++++++++++++END_TAB_X


  %++++++++++++++++++++++++++++++++++++++++++START_TAB_X
  \tab{\lang{de}{    Lösung c)    } \lang{en}{Solution c)}}
  \begin{incremental}[\initialsteps{1}]
  
  	 %----------------------------------START_STEP_X
    \step 
\lang{de}{Beim uneigentlichen Integral $\int_1^2 \frac{x^2-4x+4}{x-2}\; dx$ ist der Integrand an der
oberen Grenze $2$ nicht definiert.\\ }
\lang{en}{In the improper integral $\int_1^2 \frac{x^2-4x+4}{x-2}\; dx$,
the integrand is not defined at the upper limit $2$.\\ }
\step
\lang{de}{Für $x$ gegen $2$ konvergiert aber sowohl Zähler als auch Nenner gegen $0$. Das bedeutet insbesondere, dass sowohl das Zählerpolynom als auch das Nennerpolynom bei
$x=2$ eine Nullstelle hat und wir durch Polynomdivision jeweils einen Linearfaktor $x-2$
abspalten können.

Hier könnte man auch direkt sehen, dass nach der 2. Binomischen Formel 
$x^2-4x+4=(x-2)^2$ gilt.}

\lang{en}{However, as $x$ tends to $2$, both the numerator and the denominator tend to $0$. In particular, this means that the polynomials in the numerator and denominator have
zeros in $x=2$, and we can remove a linear factor $x-2$ from each of them using polynomial division.

One can also see directly that $x^2-4x+4=(x-2)^2$ using the second binomial formula.
}

\step
\lang{de}{Für alle $x\neq 2$ ist damit
\[ \frac{x^2-4x+4}{x-2}=\frac{(x-2)^2}{x-2}=x-2.\]
Insbesondere ist der Integrand auf dem Intervall $[1;2)$ beschränkt und das uneigentliche
Integral existiert.
Zur Berechnung des Integrals können wir als Fortsetzung von $h_1(x)=\frac{x^2-4x+4}{x-2}$ die Funktion $h_2:[1;2]\to \R$ mit $h_2(x)=x-2$ wählen und das bestimmte Integral
$\int_1^2  h_2(x)\;dx$ berechnen:}

\lang{en}{Therefore, for all $x\neq 2$,
\[ \frac{x^2-4x+4}{x-2}=\frac{(x-2)^2}{x-2}=x-2.\]
In particular, the integrand is bounded on the interval $[1;2)$, and the improper integral exists.
To compute the integral, we can extend $h_1(x)=\frac{x^2-4x+4}{x-2}$ to the function
$h_2:[1;2]\to \R$ with $h_2(x)=x-2$ and compute the definite integral $\int_1^2  h_2(x)\;dx$:
}

\[ \int_1^2 \frac{x^2-4x+4}{x-2}\; dx=\int_1^2  (x-2)\;dx
= \left[ \frac{x^2}{2}-2x \right]_1^2=\left(  \frac{2^2}{2}-2\cdot 2\right)
-\left(  \frac{1^2}{2}-2\right) =- \frac{1}{2}.\]
  	 %------------------------------------END_STEP_X
 
  \end{incremental}
  %++++++++++++++++++++++++++++++++++++++++++++END_TAB_X
  %++++++++++++++++++++++++++++++++++++++++++START_TAB_X
  \tab{\lang{de}{    Lösung d)    } \lang{en}{Solution d)}}
  \begin{incremental}[\initialsteps{1}]
  
  	 %----------------------------------START_STEP_X
    \step 
\lang{de}{Bei diesem uneigentlichen Integral ist sowohl das Intervall unendlich als auch der Integrand
an der unteren Grenze nicht definiert. Das Integrationsintervall ist also in zwei Teilintervalle 
aufzuteilen, z.\,B. in $(0;1]$ und $[1;\infty)$, und die uneigentlichen Integrale
$\int_0^1 \frac{(2x+1)e^{-x}}{\sqrt{x}^3}\; dx$ und $\int_1^{\infty} \frac{(2x+1)e^{-x}}{\sqrt{x}^3}\, dx$ zu untersuchen.}

\lang{en}{In this improper integral, the interval is infinite and the integrand
is undefined at the lower bound. Therefore, we split the interval of integration
into two subintervals - for example, $(0;1]$ and $[1;\infty)$ - and we investigate the integrals
$\int_0^1 \frac{(2x+1)e^{-x}}{\sqrt{x}^3}\; dx$ and $\int_1^{\infty} \frac{(2x+1)e^{-x}}{\sqrt{x}^3}\, dx$.
}

\step
\lang{de}{Nach dem Hinweis ist $G(x)=\frac{-2e^{-x}}{\sqrt{x}}$ eine Stammfunktion für den
Integranden und daher}
\lang{en}{By the hint, $G(x)=\frac{-2e^{-x}}{\sqrt{x}}$ is an antiderivative
of the integrand. So}
\[ \int_0^1 \frac{(2x+1)e^{-x}}{\sqrt{x}^3}\; dx=\lim_{a\searrow 0} \int_a^1 \frac{(2x+1)e^{-x}}{\sqrt{x}^3}\; dx
=\lim_{a\searrow 0}  \left[ \frac{-2e^{-x}}{\sqrt{x}}\right]_a^1
=  \frac{-2e^{-1}}{\sqrt{1}} - \lim_{a\searrow 0} \frac{-2e^{-a}}{\sqrt{a}}.\]
\step
\lang{de}{Der Grenzwert $ \lim_{a\searrow 0} \frac{-2e^{-a}}{\sqrt{a}}$ existiert jedoch nicht,
da der Ausdruck für $a\searrow 0$ gegen $-\infty$ divergiert.

Also existiert das uneigentliche Integral $\int_0^1 \frac{(2x+1)e^{-x}}{\sqrt{x}^3}\; dx$
nicht.

Damit existiert aber auch das gesuchte Integral nicht, völlig unabhängig davon, ob
das uneigentliche Integral $\int_1^{\infty} \frac{(2x+1)e^{-x}}{\sqrt{x}^3}\, dx$
existiert und welchen Wert es hätte.}

\lang{en}{However, the limit $ \lim_{a\searrow 0} \frac{-2e^{-a}}{\sqrt{a}}$ does not exist, 
because this expression diverges to $-\infty$ as $a\searrow 0$.
So the improper integral $\int_0^1 \frac{(2x+1)e^{-x}}{\sqrt{x}^3}\; dx$
does not exist.

But this means that the original integral does not exist, regardless of whether
the improper integral $\int_1^{\infty} \frac{(2x+1)e^{-x}}{\sqrt{x}^3}\, dx$ exists
and what its value might be.
}
  	 %------------------------------------END_STEP_X
 
  \end{incremental}
  %++++++++++++++++++++++++++++++++++++++++++++END_TAB_X


%#############################################################ENDE

      \tab{\lang{de}{Video: ähnliche Übungsaufgabe} \lang{en}{Video: similar exercise}}
  \youtubevideo[500][300]{2VAB17XsKF0}\\

\end{tabs*}

\end{content}

