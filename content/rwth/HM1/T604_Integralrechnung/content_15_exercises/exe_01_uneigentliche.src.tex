\documentclass{mumie.element.exercise}
%$Id$
\begin{metainfo}
  \name{
    \lang{de}{Ü01: Integrierbarkeit}
    \lang{en}{Exercise 1: Integrability}
  }
  \begin{description} 
 This work is licensed under the Creative Commons License Attribution 4.0 International (CC-BY 4.0)   
 https://creativecommons.org/licenses/by/4.0/legalcode 

    \lang{de}{}
    \lang{en}{}
  \end{description}
  \begin{components}
  \end{components}
  \begin{links}
    \link{generic_article}{content/rwth/HM1/T304_Integrierbarkeit/g_art_content_09_integrierbare_funktionen.meta.xml}{link1}
  \end{links}
  \creategeneric
\end{metainfo}
\begin{content}
\title{
  \lang{de}{Ü01: Integrierbarkeit}
    \lang{en}{Exercise 1: Integrability}
}

\begin{block}[annotation]
	Im Ticket-System: \href{https://team.mumie.net/issues/24051}{Ticket 24051}
\end{block}
\begin{block}[annotation]
	Kopie: hm4mint/T304_Integrierbarkeit/exercise 9
    
    Im Ticket-System: \href{http://team.mumie.net/issues/10584}{Ticket 10584}
\end{block}

\usepackage{mumie.ombplus}



\lang{de}{Berechnen Sie die folgenden Integrale mit Hilfe des Hauptsatzes der
Differential- und Integralrechnung, sofern die bestimmten Integrale existieren.}
\lang{en}{Calculate the following integrals using the fundamental theorem of
calculus, if the definite integrals exist.}

\begin{enumerate}

\item[a)] $\big\int_{-2}^{-1} \frac{1+x}{x^2}\, dx$

\item[b)] $\big\int_{-2}^1 (\frac{3}{x^3} - 2x^2)\, dx$

\item[c)] $\big\int_{\pi}^{0} (2 \sin(x) - 3e^x)\, dx$

\end{enumerate}

\begin{tabs*}[\initialtab{0}\class{exercise}]

 \tab{\lang{de}{Antworten}
 \lang{en}{Answers}
 }

 \begin{enumerate}
\item[a)] $\big\int_{-2}^{-1} \frac{1+x}{x^2}\, dx = \frac{1}{2} - \ln(2)$.
\item[b)]
    \lang{de}{Das bestimmte Integral existiert nicht.}
    \lang{en}{The definite integral does not exist.}
\item[c)] $\big\int_{\pi}^{0} (2 \sin(x) - 3e^x)\, dx = 3 e^{\pi} - 7$.
\end{enumerate}



\tab{\lang{de}{L"osung zu a)}
\lang{en}{Solution for a)}
}

 \begin{incremental}[\initialsteps{1}]
  \step
    \lang{de}{Es gilt $f(x)=\frac{1+x}{x^2} = \frac{1}{x^2} + \frac{1}{x}$. Diese Funktion ist stetig und daher
  auch integrierbar auf dem Intervall $[-2;-1]$.}
    \lang{en}{$f(x)=\frac{1+x}{x^2} = \frac{1}{x^2} + \frac{1}{x}$. $f(x)$ is
    continuous and thus integrable over the interval $[-2,-1]$.}
  \step
    \lang{de}{Wir bestimmen eine Stammfunktion und wenden dann den Hauptsatz der Differential- und Integralrechnung an.}
    \lang{en}{We find an antiderivative and then use the fundamental theorem
    of calculus.}
  \step
    \lang{de}{Eine Stammfunktion ist $F(x)=- \frac{1}{x} + {\ln|x|} + C$.}
    \lang{en}{One antiderivative is $F(x)=- \frac{1}{x} + {\ln|x|}$.}
  \step
    \lang{de}{Für das bestimmte Integral folgt nach dem Hauptsatz}
    \lang{en}{Thus, using the fundamental theorem of calculus, the definite integral is:}
  \[ \int_{-2}^{-1} \frac{1+x}{x^2}\, dx = \Big[ - \frac{1}{x} + {\ln |x|} \Big]_{-2}^{-1} =
    (1 + \ln(1)) - \big( \frac{1}{2} + \ln(2) \big) = \frac{1}{2} - \ln(2) . \]
  \end{incremental}

 \tab{\lang{de}{L"osung zu b)}
 \lang{en}{Solution for b)}
 }
  \lang{de}{Das bestimmte Integral existiert nicht, da wegen $0 \in [-2;1]$ eine Polstelle von
  $\frac{3}{x^3}$ im Integrationsintervall liegt, über die nicht hinweg integriert werden darf.}
  \lang{en}{The definite integral does not exist, since $0 \in [-2, 1]$ is a pole
  of $\frac{3}{x^3}$, and we cannot integrate over a pole.}

  \tab{\lang{de}{L"osung zu c)}
  \lang{en}{Solution for c)}
  }
  \begin{incremental}[\initialsteps{1}]
  \step
    \lang{de}{$f(x)=2 \sin(x) - 3e^x$  ist stetig und daher integrierbar auf dem
  Intervall $[0;\pi]$.}
  \lang{en}{$f(x)=2 \sin(x) - 3e^x$  is continuous and thus integrable over the
  interval $[0, \pi]$.}
  \step
    \lang{de}{Wir bestimmen eine Stammfunktion und wenden dann den Hauptsatz der Differential- und Integralrechnung an.}
    \lang{en}{We find an antiderivative and use the fundamental theorem of calculus.}
  \step
    \lang{de}{Eine Stammfunktion zu $f(x)$ ist $F(x)=-2\cos(x) - 3e^x$.}
    \lang{en}{An antiderivative of $f(x)$ is $F(x)=-2\cos(x) - 3e^x$.}
  \step
    \lang{de}{In diesem Beispiel ist $\pi$ die untere und $0$ die obere Grenze.
  Der Hauptsatz kann trotzdem ohne Änderung verwendet werden.}
  \lang{en}{In this example $\pi$ is the lower limit and $0$ the upper limit of
  integration. The fundamental theorem of calculus can still be used unmodified.}
  \step
    \lang{de}{Es folgt:}
    \lang{en}{Thus:}
  \[ \int_{\pi}^0 (2 \sin(x) - 3e^x)\, dx = \big[ -2 \cos(x) - 3e^x \big]_{\pi}^0 =
  (-2-3) - (2 - 3e^{\pi}) = 3 e^{\pi} -7 . \]
  \end{incremental}
  
%      \tab{\lang{de}{Video: ähnliche Übungsaufgabe}}
%  \youtubevideo[500][300]{Kg9Y7UMTFvM}\\
  
\end{tabs*}
\end{content}

