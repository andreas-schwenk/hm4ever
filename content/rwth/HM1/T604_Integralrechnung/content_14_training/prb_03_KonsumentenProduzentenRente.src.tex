\documentclass{mumie.problem.gwtmathlet}
%$Id$
\begin{metainfo}
  \name{
    \lang{en}{...}
    \lang{de}{A03: Konsumenten- und Produzentenrente}
  }
  \begin{description} 
 This work is licensed under the Creative Commons License Attribution 4.0 International (CC-BY 4.0)   
 https://creativecommons.org/licenses/by/4.0/legalcode 

    \lang{en}{...}
    \lang{de}{...}
  \end{description}
  \corrector{system/problem/GenericCorrector.meta.xml}
  \begin{components}
    \component{js_lib}{system/problem/GenericMathlet.meta.xml}{gwtmathlet}
  \end{components}
  \begin{links}
  \end{links}
  \creategeneric
\end{metainfo}
\begin{content}
\lang{de}{\title{A03: Konsumenten- und Produzentenrente}}
\lang{en}{\title{A03: Consumer and producer surplus}}
\begin{block}[annotation]
	Im Ticket-System: \href{https://team.mumie.net/issues/24086}{Ticket 24086}
\end{block}
\usepackage{mumie.genericproblem}


\begin{problem}

%\randomquestionpool{1}{2}


     \begin{variables}
        \function{fa}{135+5*x^2}
        \function{fn}{360-4*x^2}
        \function[expand,normalize]{solx}{5}
        \function[expand,normalize]{solp}{260}
        \function[calculate]{pr}{1300-(135*5+5/3*5^3)}
        \function[calculate]{kr}{360*5-4/3*5^3-1300}
     \end{variables}
          \begin{question}
          \lang{de}{\text{Es seien die  Angebotsfunktion: $p_A(x)=135+5x^2$ und die  Nachfragefunktion $p_N(x)=360-4x^2$ gegeben.
           Bestimmen Sie den (positiven) Gleichgewichtspunkt und den Marktpreis.}}
         \lang{en}{\text{Suppose the supply function $p_A(x)=135+5x^2$ and the demand function $p_N(x)=360-4x^2$ are given.
           Determine the (positive) equilibrium point and the market price}}.
          \type{input.number}
               \begin{answer}
                 \text{$x^{\ast}=$ }
                 \solution{solx}
               \end{answer} 
               \begin{answer}
                 \text{$p^{\ast}=$}   
                 \solution{solp}  
               \end{answer}     
           \lang{de}{\explanation[edited]{Die Gleichung $p_A(x)=p_N(x)$ ist zu lösen.}}
            \lang{en}{\explanation[edited]{The equation $p_A(x)=p_N(x)$ is to be solved.}}
          \end{question} 
          
           \begin{question}
            \lang{de}{\text{Berechnen Sie nun die Konsumenten- und Produzentenrente.}}
            \lang{en}{\text{Now calculate the consumer and producer surplus.}}
            \correctorprecision{2}
            %\explanation{...}
                \begin{answer}
                    \type{input.number}
                    \text{$KR=$}
                    \solution{kr}
                    %\checkAsFunction[1e-2]{x}{-10}{10}{10}
                    \explanation{$KR=\int_0^5p_N(x)dx-5\cdot 260$
                    %=360\cdot 5-\frac{4}{3}5^3-1300=333\frac{1}{3}$
                    }
                \end{answer}
                \begin{answer}
                    \type{input.number}
                    \text{$PR=$}
                    \solution{pr}
                    %\checkAsFunction[1e-2]{x}{-10}{10}{10}
                    \explanation{$PR=1300-\int_0^5p_A(x)dx$
                    %=1300-(135\cdot 5 +\frac{5}{3}5^3)=416\frac{2}{3}$
                    }
                \end{answer}
                
           \end{question}          
     \end{problem}

\embedmathlet{gwtmathlet}

\end{content}
