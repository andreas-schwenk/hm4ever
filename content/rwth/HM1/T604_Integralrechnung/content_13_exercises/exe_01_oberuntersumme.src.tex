\documentclass{mumie.element.exercise}
%$Id$
\begin{metainfo}
  \name{
    \lang{de}{Ü01: Ober- und Untersummen}
    \lang{en}{Exercise 1: Upper and lower sums}
  }
  \begin{description} 
 This work is licensed under the Creative Commons License Attribution 4.0 International (CC-BY 4.0)   
 https://creativecommons.org/licenses/by/4.0/legalcode 

    \lang{de}{}
    \lang{en}{}
  \end{description}
  \begin{components}
  \end{components}
  \begin{links}
  \end{links}
  \creategeneric
\end{metainfo}
\begin{content}
\title{\lang{de}{Ü01: Ober- und Untersummen}
    \lang{en}{Exercise 1: Upper and lower sums}}

\begin{block}[annotation]
	Im Ticket-System: \href{https://team.mumie.net/issues/23239}{Ticket 23239}
\end{block}
\usepackage{mumie.ombplus}


\begin{block}[annotation]
Kopie: hm4mint/T304_Integrierbarkeit/exercise 4
Im Ticket-System: \href{http://team.mumie.net/issues/10579}{Ticket 10579}
\end{block}

%######################################################FRAGE_TEXT
\lang{de}{ Wir betrachten die Funktion $f(x)=\frac{1}{x}$ auf dem Intervall $[1;5]$.\\
\begin{enumerate}[a)]
\item a) Geben Sie zu jedem $n\in \N$ die Ober- und Untersumme von $f$ bezüglich der
äquidistanten Zerlegung $Z_n$ in $n$ Teilintervalle an.
\item b) Berechnen Sie die Differenzen $O(Z_n)-U(Z_n)$.
Wieso folgt aus dieser Berechnung, dass $f$ auf $[1;5]$ integrierbar ist?
\item c) Für welche $n$ ist mit Sicherheit $|\int_1^5 f(x)\, dx - O(Z_n)|\leq 0,01$?
\end{enumerate}

\textit{Hinweis:} Nutzen Sie aus, dass die Funktion monoton fallend ist.
 }
\lang{en}{Consider the function $f(x) = \frac{1}{x}$ on the interval $[1;5]$.\\
\begin{enumerate}[a)]
\item a) For each $n \in \N$, find the upper and lower sums of $f$ with
respect to an equidistant partition $Z_n$ into $n$ subintervals.
\item b) Compute the differences $O(Z_n)-U(Z_n)$.
Why does this computation imply that $f$ is integrable on $[1;5]$?
\item c) For which $n$ can we be sure that $|\int_1^5 f(x)\, dx - O(Z_n)|\leq 0.01$?
\end{enumerate}
\textit{Hint:} Use the fact that $f$ is monotonically decreasing.
}

%##################################################ANTWORTEN_TEXT
\begin{tabs*}[\initialtab{0}\class{exercise}]

\tab{\lang{de}{    Antworten    }  \lang{en}{Answers}}
    \lang{de}{ \begin{enumerate}[a)]
\item a) Obersumme und Untersumme sind
\[ O(Z_n)= \sum_{l=0}^{n-1}  \frac{4}{n+4l}, \quad U(Z_n)= \sum_{k=1}^{n}  \frac{4}{n+4k}. \]


\item b) Die Differenz beträgt
\[  O(Z_n)-U(Z_n) = \frac{16}{5n}.\]
Für Begründung der Integrierbarkeit siehe Lösung b).

\item c) Für $n\geq 320$ ist die Bedingung erfüllt.
\end{enumerate} }

\lang{en}{ \begin{enumerate}[a)]
\item a) The upper and lower sums are
\[ O(Z_n)= \sum_{l=0}^{n-1}  \frac{4}{n+4l}, \quad U(Z_n)= \sum_{k=1}^{n}  \frac{4}{n+4k}. \]


\item b) The difference is
\[  O(Z_n)-U(Z_n) = \frac{16}{5n}.\]
The proof of integrability is given in solution b).

\item c) The condition is satisfied for $n\geq 320$.
\end{enumerate} }


  %++++++++++++++++++++++++++++++++++++++++++START_TAB_X
  \tab{\lang{de}{    Lösung a)    }  \lang{en}{Solution a)}}
  \begin{incremental}[\initialsteps{1}]
  
  	 %----------------------------------START_STEP_X
    \step  
\lang{de}{Da das Intervall $[1;5]$ die Länge $4$ hat, sind die Längen der Teilintervalle bei der
äquidistanten Zerlegung $Z_n$ jeweils $\frac{4}{n}$, und die Teilungsstellen sind
\[  x_k=1+\frac{4k}{n}=\frac{n+4k}{n}\quad \text{für }k=0,\ldots, n. \] }
\lang{en}{Since the interval $[1;5]$ has length $4$, the lengths of the subintervals in the
equidistant partition $Z_n$ are all $\frac{4}{n}$, and the endpoints of the
subintervals are 
\[  x_k=1+\frac{4k}{n}=\frac{n+4k}{n}\quad \text{for }k=0,\ldots, n. \] }


\step
\lang{de}{Da die Funktion monoton fallend ist, wird das Maximum auf den Teilintervallen stets
am linken Rand angenommen und das Minimum stets am rechten Rand.}
\lang{en}{The function is monotonically decreasing, so its maximum on any subinterval
occurs in the left endpoint and the minimum in the right endpoint. }
\step
\lang{de}{Als Obersumme zur Zerlegung $Z_n$ erhält man daher}
\lang{en}{The upper sum for the partition $Z_n$ is therefore}
\[ O(Z_n)= \sum_{k=1}^n f(x_{k-1}) \cdot \frac{4}{n}
=  \frac{4}{n} \cdot\sum_{k=1}^n  \frac{n}{n+4(k-1)}
= \frac{4}{n} \cdot\sum_{l=0}^{n-1}  \frac{n}{n+4l}=\sum_{l=0}^{n-1}  \frac{4}{n+4l},
\]
\lang{de}{wobei wir im vorletzten Schritt den Summationsindex $k$ durch den Summationsindex $l$ mittels $l=k-1$ ersetzt haben.}
\lang{en}{where we replaced the summation index $k$ by $l=k-1$ in the second-to-last step.}
\step
\lang{de}{Für die Untersumme zur Zerlegung $Z_n$ erhält man entsprechend}
\lang{en}{Similarly, the lower sum for the partition $Z_n$ is}
\[ U(Z_n)=\sum_{k=1}^n f(x_{k}) \cdot \frac{4}{n}=
 \frac{4}{n} \cdot\sum_{k=1}^{n}  \frac{n}{n+4k}= \sum_{k=1}^{n}  \frac{4}{n+4k}. \]
  	 %------------------------------------END_STEP_X
 
  \end{incremental}
  %++++++++++++++++++++++++++++++++++++++++++++END_TAB_X

  %++++++++++++++++++++++++++++++++++++++++++START_TAB_X
  \tab{\lang{de}{    Lösung b)    } \lang{en}{Solution b)}}
  \begin{incremental}[\initialsteps{1}]
  
  	 %----------------------------------START_STEP_X
    \step  
\lang{de}{Aus Teil a) haben wir die Formeln $O(Z_n) =\sum_{l=0}^{n-1}  \frac{4}{n+4l}$ und
$U(Z_n)= \sum_{k=1}^{n}  \frac{4}{n+4k}$.}
\lang{en}{Part a) yields the formulas $O(Z_n) =\sum_{l=0}^{n-1}  \frac{4}{n+4l}$
and $U(Z_n)= \sum_{k=1}^{n}  \frac{4}{n+4k}$.}
\step
\lang{de}{Da sich die beiden Summen lediglich um den Summanden mit $l=0$ bzw. mit $k=n$ unterscheiden, erhält man direkt}
\lang{en}{The two sums differ only by the summand with $l=0$, i.e. $k=n$, so we immediately obtain}
\[  O(Z_n)-U(Z_n) = \sum_{l=0}^{n-1}  \frac{4}{n+4l} - \sum_{k=1}^{n}  \frac{4}{n+4k} = \frac{4}{n}- \frac{4}{n+4n}=\frac{4}{n}-\frac{4}{5n}=\frac{16}{5n}.\]

\step
\lang{de}{Nach der Definition von Integrierbarkeit benötigt man zu jedem $\epsilon>0$ eine Zerlegung $Z$, für welche $O(Z)-U(Z)<\epsilon$ gilt. Für gegebenes $\epsilon>0$ muss man hier lediglich ein $n$ wählen, das so groß ist, dass $ \frac{16}{5n}<\epsilon$ ist.
Dies ist möglich, da die Folge $( \frac{16}{5n})_{n\in \N}$ eine Nullfolge ist. Explizit kann
man eine natürliche Zahl $n>\frac{16}{5\epsilon}$ wählen.

Also ist $f$ auf dem Intervall $[1;5]$ integrierbar.}
\lang{en}{According to the definition of integrability, for every $\epsilon>0$ we need a partition $Z$ for which $O(Z)-U(Z)<\epsilon$. In our case, for a fixed $\epsilon>0$, we
only need to choose $n$ large enough such that $ \frac{16}{5n}<\epsilon$.
This is possible because the sequence $(\frac{16}{5n})_{n\in \N}$ tends to zero.
More explicitly, we can choose any natural number $n>\frac{16}{5\epsilon}$.
}
  	 %------------------------------------END_STEP_X
 
  \end{incremental}
  %++++++++++++++++++++++++++++++++++++++++++++END_TAB_X

  %++++++++++++++++++++++++++++++++++++++++++START_TAB_X
  \tab{\lang{de}{    Lösung c)    } \lang{en}{Solution c)}}
  \begin{incremental}[\initialsteps{1}]
  
  	 %----------------------------------START_STEP_X
\step 
\lang{de}{Der Integralwert liegt stets zwischen Ober- und Untersumme. Ist also deren Abstand höchstens $0,01$, so ist auch der Abstand der Obersumme zum Integral  höchstens $0,01$.}
\lang{en}{The value of the integral always lies between the upper and lower sums.
If their difference is at most $0.01$, then the difference between
the upper sum and the integral is also at most $0.01$.}

\step
\lang{de}{Wir haben also mit Sicherheit $|\int_1^5 f(x)\, dx - O(Z_n)|\leq 0,01$, wenn
$\frac{16}{5n}\leq 0,01=\frac{1}{100}$, d.\,h. wenn $n\geq 320$.}
\lang{en}{Therefore, $|\int_1^5 f(x)\, dx - O(Z_n)|\leq 0.01$ certainly holds whenever
$\frac{16}{5n}\leq 0.01=\frac{1}{100}$, i.e. $n\geq 320$.}
  	 %------------------------------------END_STEP_X
 
  \end{incremental}
  %++++++++++++++++++++++++++++++++++++++++++++END_TAB_X


%#############################################################ENDE
\end{tabs*}
\end{content}