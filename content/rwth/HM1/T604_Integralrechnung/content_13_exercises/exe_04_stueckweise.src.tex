\documentclass{mumie.element.exercise}
%$Id$
\begin{metainfo}
  \name{
    \lang{de}{Ü04: Integral von abschnittsweise definierten Funktionen}
    \lang{en}{Exercise 4: integrals of piecewise functions}
  }
  \begin{description} 
 This work is licensed under the Creative Commons License Attribution 4.0 International (CC-BY 4.0)   
 https://creativecommons.org/licenses/by/4.0/legalcode 

    \lang{de}{Hier die Beschreibung}
    \lang{en}{}
  \end{description}
  \begin{components}
    \component{generic_image}{content/rwth/HM1/images/g_tkz_T604_13_Exercise04.meta.xml}{T604_13_Exercise04}
  \end{components}
  \begin{links}
    \link{generic_element}{content/rwth/HM1/T107_Integralrechnung/exercises/g_exe_exercise_1.meta.xml}{link1}
  \end{links}
  \creategeneric
\end{metainfo}
\begin{content}
\title{\lang{de}{Ü04: Integral von abschnittsweise definierten Funktionen}
    \lang{en}{Exercise 4: integrals of piecewise functions}}
\begin{block}[annotation]
	Im Ticket-System: \href{https://team.mumie.net/issues/24043}{Ticket 24043}
\end{block}


\begin{block}[annotation]
Kopie: hm4mint/T107_Integralrechnung/exercise 2
Im Ticket-System: \href{http://team.mumie.net/issues/9602}{Ticket 9602}
\end{block}

\lang{de}{Gegeben sei die folgende abschnittsweise definierte Funktion $f(x)$: }
\lang{en}{Let $f(x)$ be the following piecewise function:}
\[ f(x) = \begin{cases} x+1, & x \in    \lang{de}{[0;3)} \lang{en}{[0, 3),} \\
                        4, & x \in      \lang{de}{[3;5)} \lang{en}{[3, 5),} \\
                        -4x+24, & x \in \lang{de}{[5;6]} \lang{en}{[5, 6].}
          \end{cases} \]
\begin{center}
\image{T604_13_Exercise04}
\end{center}


\lang{de}{Bestimmen Sie den Wert des Integrals $\int_0^6 f(x) \, dx$.}
\lang{en}{Determine the value of the integral  $\int_0^6 f(x) \, dx$.} \\
\emph{\lang{de}{Hinweis: } \lang{en}{Hint:}}
\lang{de}{Nutzen Sie die Additivität des Integrals!}
\lang{en}{Use the additive property of integrals!}

\begin{tabs*}[\initialtab{0}\class{exercise}]
  \tab{\lang{de}{Antwort}
  \lang{en}{Answer}
  }

$\int_0^6 f(x) \, dx = \lang{de}{17,5} \lang{en}{17.5}$.

\tab{
  \lang{de}{L"osung}
  \lang{en}{Solution}
}

 \begin{incremental}[\initialsteps{1}]
    \step
    \lang{de}{Wegen der Additivität des Integrals folgt:
    $\int_0^6 f(x) \, dx = \int_0^3 (x+1) \, dx + \int_3^5 4 \, dx +\int_5^6 (-4x+24) \, dx$.}
    \lang{en}{By the additive property of integrals:
    $\int_0^6 f(x) \, dx = \int_0^3 (x+1) \, dx + \int_3^5 4 \, dx +\int_5^6 (-4x+24) \, dx$.}
    \step
    \lang{de}{Nun können die drei Integrale geometrisch bestimmt werden. Wegen $f(x) \geq 0$ sind alle
    Flächen positiv orientiert.}
    \lang{en}{Now we can determine the three integrals geometrically. Since
    $f(x) \geq 0$, all areas are positively oriented. We do the calculation as
    in \link{link1}{Exercise 1}.}
    \step
    \lang{de}{Das erste Integral als Flächeninhalt eines Trapezes
    (Summe einer Rechtecks- und Dreiecksfläche):
    $\int_0^3 (x+1) \, dx = 3 \cdot 1 + \frac{1}{2} \cdot ((4-1) \cdot 3) = 7,5$,}
    \lang{en}{We calculate the first integral as the area of a trapezium (sum of a rectangular
    and triangular area):
    $\int_0^3 (x+1) \, dx = 3 \cdot 1 + \frac{1}{2} \cdot ((4-1) \cdot 3) = 7.5$,}
    \step
    \lang{de}{das zweite Integral $\int_3^5 4 \, dx = 8$ als Flächeninhalt eines Rechtecks,}
    \lang{en}{the second integral $\int_3^5 4 \, dx = 8$ as the area of a rectangle,}
    \step
    \lang{de}{das dritte Integral $\int_5^6 (-4x+24) \, dx = 2$ als Flächeninhalt eines Dreiecks: $\frac{1}{2} \cdot (1 \cdot 4)$.}
    \lang{en}{the third integral $\int_5^6 (-4x+24) \, dx = 2$ as the area of a triangle: $\frac{1}{2} \cdot (1 \cdot 4)$.}
    \step
    \lang{de}{Die Summe dieser drei Flächeninhalte ergibt den Wert}
    \lang{en}{The sum of these three areas gives us the value of the integral}
    $\int_0^6 f(x) \, dx = \lang{de}{17,5} \lang{en}{17.5}$.
    \step
    \lang{de}{Der orientierte Flächeninhalt lässt sich auch anhand der Grafik ablesen. Jedes Kästchen hat
    die Fläche $1$. }
    \lang{en}{The signed area can also be read off the graphic. Every box has an
    area of $1$.}
 \end{incremental}


 \end{tabs*}




\end{content}