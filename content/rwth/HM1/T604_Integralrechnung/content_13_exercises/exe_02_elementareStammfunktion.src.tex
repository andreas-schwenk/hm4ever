\documentclass{mumie.element.exercise}
%$Id$
\begin{metainfo}
  \name{
    \lang{de}{Ü02: elementare Stammfunktionen}
    \lang{en}{Exercise 2: elementary antiderivatives}
  }
  \begin{description} 
 This work is licensed under the Creative Commons License Attribution 4.0 International (CC-BY 4.0)   
 https://creativecommons.org/licenses/by/4.0/legalcode 

    \lang{de}{}
    \lang{en}{}
  \end{description}
  \begin{components}
  \end{components}
  \begin{links}
  \end{links}
  \creategeneric
\end{metainfo}
\begin{content}
\title{
\lang{de}{Ü02: elementare Stammfunktionen}
    \lang{en}{Exercise 2: elementary antiderivatives}
}

\begin{block}[annotation]
	Im Ticket-System: \href{https://team.mumie.net/issues/24041}{Ticket 24041}
\end{block}



\begin{block}[annotation]
Kopie: hm4mint/T304_Integrierbarkeit/exercise 7

Im Ticket-System: \href{http://team.mumie.net/issues/10582}{Ticket 10582}
\end{block}

\begin{enumerate}

\item[a)]
  \lang{de}{Bestimmen Sie eine Stammfunktion $F(x)$ von $f(x)=x^{-2} - \cos(x) -3 e^{x}$
auf dem Intervall $(0;\infty)$.}
\lang{en}{Determine an antiderivative $F(x)$ of $f(x)=x^{-2} - \cos(x) -3 e^{x}$
on the interval $(0,\infty)$.}

\item[b)]
  \lang{de}{Welche der folgenden Funktionen $F_1(x)$, $F_2(x)$, $F_3(x)$
ist eine Stammfunktion von $f(x)=e^{-x}\big(\cos(x)-\sin(x)\big)$ auf $\mathbb{R}$ ?}
\lang{en}{Which of the following functions $F_1(x)$, $F_2(x)$, $F_3(x)$
is an antiderivative of $f(x)=e^{-x}\big(\cos(x)-\sin(x)\big)$ on $\mathbb{R}$ ?}
\begin{table}[\class{items}]
\nowrap{$F_1(x) = e^{-x}\big(\sin(x)+\cos(x)\big)$, \lang{de}{oder} \lang{en}{or}     }\\
\nowrap{$F_2(x) = -e^{-x}\big(\sin(x)+\cos(x)\big)$, \lang{de}{oder} \lang{en}{or}    }\\
\nowrap{$F_3(x) = e^{-x}\sin(x)$ ?}\\
\end{table}

\end{enumerate}

\begin{tabs*}[\initialtab{0}\class{exercise}]
  \tab{\lang{de}{Antworten}
  \lang{en}{Answers}
   }


\begin{table}[\class{items}]
a) $\ F(x)=-x^{-1} - \sin(x) - 3 e^{x}$.\\
b) $\ F_3(x) = e^{-x}\sin(x)$. \\

\end{table}

\tab{\lang{de}{L"osung zu a)}
\lang{en}{Solution for a)}
}

 \begin{incremental}[\initialsteps{1}]
    \step
    \lang{de}{Die Summanden werden einzeln integriert und dann die Summenregel angewendet.}
    \lang{en}{The terms are integrated individually and then the sum rule is used.}
    \step
    \lang{de}{Eine Stammfunktion zu $x^{-2}$ ist $\frac{1}{-2+1} x^{-2+1} = - x^{-1}$.}
    \lang{en}{An antiderivative of $x^{-2}$ is $\frac{1}{-2+1} x^{-2+1} = - x^{-1}$.}
    \step
    \lang{de}{Eine Stammfunktion zu $\cos(x)$ ist $\sin(x)$.}
    \lang{en}{An antiderivative of $\cos(x)$ is $\sin(x)$.}
    \step
    \lang{de}{Eine Stammfunktion zu $e^{x}$ ist $e^x$.}
    \lang{en}{An antiderivative of $e^{x}$ is $e^x$.}
    \step
    \lang{de}{Nach der Summen- und Faktorregel ist dann $F(x)=-x^{-1} - \sin(x) - 3e^x$
    eine Stammfunktion zu $f(x)=x^{-2} - \cos(x) -3 e^{x}$ auf dem Intervall $(0;\infty)$.}
    \lang{en}{By the sum and constant rules, $F(x)=-x^{-1} - \sin(x) - 3e^x$
    is an antiderivative of $f(x)=x^{-2} - \cos(x) -3 e^{x}$ on the interval $(0,\infty).$}

 \end{incremental}

\tab{\lang{de}{L"osung zu b)}
\lang{en}{Solution for b)}
}

\begin{incremental}[\initialsteps{1}]
    \step
    \lang{de}{$f(x)=e^{-x}\big(\cos(x)-\sin(x)\big)$ ist ein \emph{Produkt} von zwei Funktionen. }
    \lang{en}{$f(x)=e^{-x}\big(\cos(x)-\sin(x)\big)$ is the \emph{product} of two functions.}

   % Die Summen- und
   % Faktorregel kann daher nicht angewendet werden, die getrennte Integration der beiden Funktionen
   % $e^{-x}$ und $\cos(x)-\sin(x)$ führt nicht zum richtigen Ergebnis!
    \step
    \lang{de}{Man prüft die angegebenen drei Funktionen durch Ableiten. Wenn die Ableitung gleich
    $f(x)$ ist, so handelt es sich um eine Stammfunktion.}
    \lang{en}{We check the three given functions by differentiation. If the
    derivative is equal to $f(x)$, then we have found an antiderivative.}
    \step
    \lang{de}{Nach Produktregel der Differentiation gilt:}
    \lang{en}{By the product rule of differentiation:}
    \[ F_1'(x) = \Big(e^{-x}\big(\sin(x)+\cos(x)\big)\Big)' = -e^{-x}\big(\sin(x)+\cos(x)\big) + e^{-x}\big(\cos(x)-\sin(x)\big) =
    e^{-x}\big(-2\sin(x)\big). \]
    \step
    \lang{de}{Es gilt $F_2(x) = -F_1(x)$. Also folgt:}
    \lang{en}{Observe that $F_2(x) = -F_1(x)$. Thus:}
    \[ F_2'(x) = - F_1'(x) = 2 e^{-x} \sin(x). \]
    \step
    \lang{de}{$F_3(x)$ ist die gesuchte Stammfunktion, denn nach Produktregel der Differentiation gilt:}
    \lang{en}{$F_3(x)$ is the needed antiderivative, since by the product rule
    of differentiation:}
    \[ F_3'(x) = \big(e^{-x}\sin(x)\big)' = -e^{-x} \sin(x) + e^{-x} \cos(x) = e^{-x}\big(\cos(x)-\sin(x)\big)=f(x). \]


\end{incremental}



 \end{tabs*}



\end{content}