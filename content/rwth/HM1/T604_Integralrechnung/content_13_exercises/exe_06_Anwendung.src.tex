\documentclass{mumie.element.exercise}
%$Id$
\begin{metainfo}
  \name{
    \lang{de}{Ü06: Kosten- und Erlösfunktion eines Unternehmens}
    \lang{en}{Exercise 6: cost and revenue functions of a business}
  }
  \begin{description} 
 This work is licensed under the Creative Commons License Attribution 4.0 International (CC-BY 4.0)   
 https://creativecommons.org/licenses/by/4.0/legalcode 

    \lang{de}{Übung 10}
    \lang{en}{Exercise 10}
  \end{description}
  \begin{components}
  \end{components}
  \begin{links}
  \end{links}
  \creategeneric
\end{metainfo}
\begin{content}
\title{\lang{de}{Ü06: Kosten- und Erlösfunktion eines Unternehmens}
    \lang{en}{Exercise 6: cost and revenue functions of a business}}
\begin{block}[annotation]
	Im Ticket-System: \href{https://team.mumie.net/issues/24045}{Ticket 24045}
\end{block}

\begin{block}[annotation]
Kopie: hm4mint/T107_Integralrechnung/exercise 10
Im Ticket-System: \href{https://team.mumie.net/issues/18757}{Ticket 18757}
\end{block}


\lang{de}{Die Kosten und Erlöse eines Unternehmens in Abhängigkeit von der Produktionsmenge $x$ seien
            gegeben durch Funktionen $K$ und $E$. Weiterhin sei bekannt, dass die Grenzkosten durch die 
            Funktion $f(x)=3x^2-24x+60$ beschrieben werden und die Gesamtkosten für $x=10$ genau $3000$ betragen.
            Die Ableitung der Erlösfunktion sei durch $E'(x)=18x+132$ gegeben.
            \\
            (Sie können davon ausgehen, dass bei einem Verkauf von $0$ keine Erlöse erwirtschaftet werden.)}
\lang{en}{The costs and revenues of a company depending on the output $x$ are given by functions $K$ and $E$, respectively. Furthermore,
it is known that the marginal cost function is described by the function $f(x)=3x^2-24x+60$ and that the costs for an output of $x=10$ are $3000$.
The derivative of the revenue function is $E'(x)=18x+132$.
\\
(You can assume that an output of $0$ generates no revenue.)}


\begin{enumerate}

\item[a)]
  \lang{de}{Ermitteln Sie die Erlösfunktion.}
\lang{en}{Determine the revenue function.}

\item[b)]
  \lang{de}{Ermitteln Sie die Kostenfunktion.}
  \lang{en}{Determine the cost function.}

\end{enumerate}

\begin{tabs*}[\initialtab{0}\class{exercise}]

 \tab{  \lang{de}{Antworten}
        \lang{en}{Answers}
 }
 \begin{enumerate}
 
  \item[a)] $E(x)=9x^2+132x$
  \item[b)] $K(x)=x^3-12x^2+60x+2600$
 \end{enumerate}

\tab{   
    \lang{de}{L"osung zu a)}
    \lang{en}{Solution for a)}
    }

 \begin{incremental}[\initialsteps{1}]
  \step
    \lang{de}{Die Erlösfunktion ist gegeben durch eine Stammfunktion von $E'$, also}
    \lang{en}{The revenue function is given by an antiderivative of $E'$, so}
    \[E(x)=9x^2+132x+C\]
    \lang{de}{für ein gewisses $C\in \mathbb{R}$.}
    \lang{en}{for some $C\in \mathbb{R}$.}
  \step
    \lang{de}{Da $E(0)=0$ bekannt ist, folgt }
    \lang{en}{Since we know $E(0)=0$, it follows}
  \[ C=0. \]
    \lang{de}{Damit gilt}
    \lang{en}{Hence}
    \[E(x)=9x^2+132x.\]
 \end{incremental}

   \tab{
   \lang{de}{L"osung zu b)}
   \lang{en}{Solution for b)}
   }
   
   \begin{incremental}[\initialsteps{1}]
  \step
    \lang{de}{Die Kostenfunktion ist gegeben durch eine Stammfunktion der Grenzkosten, also}
    \lang{en}{The cost function is given by an antiderivative of the marginal cost function, so}
    \[K(x)=x^3-12x^2+60x+C\]
    \lang{de}{für ein gewisses $C\in \mathbb{R}$.}
    \lang{en}{for some $C\in \mathbb{R}$.}
  \step
    \lang{de}{Weiterhin bekannt ist }
    \lang{en}{Futhermore, we know}
    \[K(10)=10^3-12\cdot 10^2+60\cdot 10+C =400 + C = 3000.\]
  \step
    \lang{de}{Daraus folgt $C=2600$ und wir erhalten}
    \lang{en}{It follows that $C=2600$ and we get}
  \[ K(x)=x^3-12x^2+60x+2600.  \]

  \end{incremental}
\end{tabs*}

\end{content}

