%$Id:  $
\documentclass{mumie.article}
%$Id$
\begin{metainfo}
  \name{
    \lang{de}{Der Integralbegriff}
    \lang{en}{}
  }
  \begin{description} 
 This work is licensed under the Creative Commons License Attribution 4.0 International (CC-BY 4.0)   
 https://creativecommons.org/licenses/by/4.0/legalcode 

    \lang{de}{Beschreibung}
    \lang{en}{}
  \end{description}
  \begin{components}
    \component{generic_image}{content/rwth/HM1/images/g_tkz_T604_Integral_B.meta.xml}{T604_Integral_B}
    \component{generic_image}{content/rwth/HM1/images/g_tkz_T604_Integral_A.meta.xml}{T604_Integral_A}
    \component{generic_image}{content/rwth/HM1/images/g_tkz_T604_Integral_C.meta.xml}{T604_Integral_C}
  \end{components}
  \begin{links}
    \link{generic_article}{content/rwth/HM1/T304_Integrierbarkeit/g_art_content_07_ober_und_untersumme.meta.xml}{content_07_ober_und_untersumme}
    \link{generic_article}{content/rwth/HM1/T304_Integrierbarkeit/g_art_content_10_uneigentliches_integral.meta.xml}{content_10_uneigentliches_integral}
    \link{generic_article}{content/rwth/HM1/T107_Integralrechnung/g_art_content_25_stammfunktion.meta.xml}{link2}
  \end{links}
  \creategeneric
\end{metainfo}
\begin{content}
\usepackage{mumie.ombplus}
\ombchapter{4}
\ombarticle{1}

\lang{de}{\title{Der Integralbegriff}}
\lang{en}{\title{The integral}}


\begin{block}[annotation]
  Im Ticket-System: \href{https://team.mumie.net/issues/22702}{Ticket 22702}
\end{block}

\begin{block}[info-box]
  \tableofcontents
\end{block}

\lang{de}{Nach der Differentialrechnung widmen wir uns nun der Integralrechnung. Integrale können als Umkehrung der Ableitung angesehen 
werden. Ähnlich wie die Ableitung als Steigung eines Funktionsgraphen interpretiert werden kann, besitzen auch Integrale 
eine geometrische Anschauung, nämlich als Flächeninhalt.}

\lang{en}{
Now that we are familiar with differential calculus, we will begin to study integral calculus.
Integration can be viewed as the inverse of differentiation.
Similarly to how the derivative can be interpreted as the slope of a function's graph,
the integral also has a geometric interpretation - it measures area.
}

\section{\lang{de}{Das bestimmte Integral als Flächeninhalt mit Vorzeichen}
\lang{en}{The definite integral as a signed area}
}
\lang{de}{Im Folgenden nehmen wir stets an, dass $a$ und $b$ reelle Zahlen sind und $a<b$ gilt.
Das bestimmte Integral $\big\int_a^b f(x)\, dx$ einer Funktion $f(x)$
gibt den \emph{Flächeninhalt mit Vorzeichen} (orientierter Flächeninhalt) an, den der Graph der Funktion mit der
$x$-Achse zwischen den Integrationsgrenzen $a$ und $b$ einschließt. Die Flächenteile oberhalb der $x$-Achse
werden dabei \emph{positiv} und die Flächenteile unterhalb der $x$-Achse \emph{negativ}
gewertet.}
\lang{en}{
Here and in what follows, let $a$ and $b$ be real numbers with $a<b$.
The definite integral $\big\int_a^b f(x)\, dx$ of a function $f(x)$
describes the \emph{signed area} (or oriented area) enclosed by the graph of the function
and the $x$-axis between the bounds $a$ and $b$. Here, the parts of the area above the $x$-axis
are counted \emph{positively} and those below the $x$-axis are counted \emph{negatively}.
}

\lang{de}{Das Integral $\big\int_a^b f(x)\, dx$ ist in der folgenden Abbildung beispielsweise die Summe von drei orientierten
Flächeninhalten. Der erste und dritte Flächeninhalt oberhalb der $x$-Achse zählen positiv, der zweite Flächeninhalt 
unterhalb der $x$-Achse zählt negativ.}
\lang{en}{
For example, in the figure below, the integral $\big\int_a^b f(x)\, dx$ is the sum of three
areas. The first and the third areas above the $x$-axis are counted positively and the second
area under the $x$-axis is counted negatively.
}
\begin{center}
\image{T604_Integral_C}
\end{center}

\lang{de}{Im Allgemeinen ist es ohne Weiteres nicht möglich, diese Flächeninhalte zu bestimmen. Unser Ansatz ist daher, die Fläche näherungsweise
zu bestimmen. Dazu zerlegen wir das Intervall $[a;b]$ auf beliebige Art und Weise und zeichnen Rechtecke so ein, dass die gesuchte Fläche 
durch die Flächen der Rechtecke eingeschlossen ist. Das führt uns zur \emph{Ober- und Untersumme}.
}
\lang{en}{
In general, there is no immediate way to determine these areas.
Our approach will therefore be to approximate them. We will partition the interval $[a,b]$
in an arbitrary way and then draw in rectangles according to the partition such that the
area in question is approximated by that of the union of the rectangles.
This leads to the notion of \emph{upper and lower Riemann sums}.
}


\begin{example}\label{zweizerl}
%\emph {Beispiel:} 

\lang{de}{
Wir betrachten die Funktion $f(x)=x^2+1$ auf dem Intervall $[a;b] = [0;2]$ und versuchen, die 
Fläche zwischen $x$-Achse und Funktionsgraph anzunähern. In der Abbildung 
links wird das Intervall $[0;2]$ in zwei 
Teilintervalle, in der Abbildung rechts in 
zehn Teilintervalle aufgeteilt. Die Teilintervalle sind hier alle gleich breit (man spricht dann von
einer \emph{äquidistanten Zerlegung} des Intervalls). 

Wenn wir allgemein ein Intervall $[a;b]$ in $n$ gleich breite 
Teilintervalle aufteilen wollen, so hätte übrigens jedes Teilintervall die Breite $\frac{b-a}{n}$.
}
\lang{en}{
Consider the function $f(x)=x^2+1$ on the interval $[a,b] = [0, 2]$.
We will try to approximate the area between its graph and the $x$-axis.
In the left image, the interval $[0, 2]$ has been partitioned into two subintervals;
in the right image, there are ten subintervals. The subintervals here all have the same length.
(This is sometimes called an \emph{equipartition} of the interval.)

In general, when we divide an interval $[a, b]$ into $n$ subintervals of equal length,
each subinterval will have length $\frac{b-a}{n}$.
}

\begin{center}
\image{T604_Integral_A}\hspace{1cm}\image{T604_Integral_B}
\end{center}

\lang{de}{Die \emph{Untersumme} ist die dunkel gefärbte Fläche. Sie besteht aus lauter Rechtecken, die genau so hoch sind wie der kleinste 
Funktionswert auf dem Teilintervall. Die \emph{Obersumme} ist die hell gefärbte plus die dunkel gefärbte Fläche. Die Obersumme besteht aus Rechtecken, 
die genau so hoch sind wie der größte Funktionswert auf dem Teilintervall.
}
\lang{en}{
The \emph{lower Riemann sum} is the area of the darkly shaded region.
That region is made up of rectangles whose height is the smallest value that the
function attains on the corresponding interval. The \emph{upper Riemann sum} is the area of both the
lightly and darkly shaded regions. This region consists of rectangles whose
height is the largest value that the function attains on the corresponding interval.
}

\lang{de}{Der Flächeninhalt unter dem Graphen liegt stets zwischen der Untersumme und der Obersumme. 
Je mehr Teilintervalle wir haben, desto genauer wird die Annäherung an die gesuchte Fläche. }
\lang{en}{
The area under the curve will always lie between the lower and upper Riemann sums.
As we increase the number of intervals, we obtain a better and better
approximation to the area that we are looking for.
}


\lang{de}{
Beim linken Graph wird 
das Intervall $[0;2]$ 
in die zwei Teilintervalle $[0;1]$ und $[1;2]$ zerlegt. Wir berechnen die Untersumme $U(Z_2)$ und die
Obersumme $O(Z_2)$ wie folgt: Die Intervalllänge ist $1$, der kleinste Funktionswert liegt am linken Rand, der größte am rechten Rand, also
\[ U(Z_2)= f(0)\cdot 1 + f(1)\cdot 1 = 1+2 = 3 \, \, \, \text{ und } \, \, \,
O(Z_2)=f(1) \cdot 1 + f(2)\cdot 1 = 2+5 = 7 . \]
}
\lang{en}{
In the graph on the left, the interval $[0, 2]$ has been split into the subintervals
$[0, 1]$ and $[0, 2]$. We can compute the lower and upper Riemann sums $U(Z_2)$ and $O(Z_2)$
as follows. The length of each subinterval is $1$, and the smallest and greatest values of the function
occur at the left and right endpoints, respectively. So
\[ U(Z_2)= f(0)\cdot 1 + f(1)\cdot 1 = 1+2 = 3 \, \, \, \text{ and } \, \, \,
O(Z_2)=f(1) \cdot 1 + f(2)\cdot 1 = 2+5 = 7 . \]
}

\lang{de}{Bei der Zerlegung in zehn gleich große Teilintervalle ist die Intervalllänge jeweils $\frac{2}{10}=0,2$. 
Für Unter- und Obersumme erhalten wir daher
\begin{eqnarray*}
U(Z_{10}) &=& f(0)\cdot 0,2+ f(0,2)\cdot 0,2+\ldots + f(1,8)\cdot 0,2=(1+1,04+1,16+\ldots + 4,24)\cdot 0,2=4,28\, , \\
O(Z_{10}) &=& f(0,2)\cdot 0,2+ f(0,4)\cdot 0,2+\ldots + f(2)\cdot 0,2=(1,04+1,16+\ldots + 4,24+5)\cdot 0,2=5,08\, . \\
\end{eqnarray*} 
Diese Werte sind schon deutlich genauer und sagen uns, dass die Fläche zwischen Graph und $x$-Achse zwischen 4,28 und 5,08
Flächeneinheiten groß ist.}
\lang{en}{
For the partition into ten equal subintervals, the length of each subinterval is $\frac{2}{10}=0.2$.
We find that the upper and lower Riemann sums, respectively, are
\begin{eqnarray*}
U(Z_{10}) &=& f(0)\cdot 0.2+ f(0.2)\cdot 0.2+\ldots + f(1.8)\cdot 0,2=(1+1.04+1.16+\ldots + 4.24)\cdot 0.2=4.28\, , \\
O(Z_{10}) &=& f(0.2)\cdot 0.2+ f(0.4)\cdot 0.2+\ldots + f(2)\cdot 0,2=(1.04+1.16+\ldots + 4.24+5)\cdot 0.2=5.08\, . \\
\end{eqnarray*} 
The values obtained in this way are considerably more precise
and they tell us that the area enclosed between the graph and the $x$-axis lies between
$4.28$ and $5.08$ units of area.
}
\end{example}

\begin{block}[warning]
\lang{de}{
Im Beispiel haben wir eine streng monoton 
steigende Funktion betrachtet. In diesem Fall können Ober- und Untersumme leicht berechnet werden, weil 
kleinster und größter Funktionswert immer am Rand der Teilintervalle angenommen werden. 

Bei komplizierteren Funktionen wissen wir im Vorfeld 
gar nicht, wo der kleinste und größte Funktionswert angenommen werden. 
}
\lang{en}{
In the example above, the function was strictly
monotonically increasing. This made it easy to compute
the upper and lower Riemann sums because the smallest
and greatest values of the function were always attained
at the endpoints of the subintervals in the partition.

When the functions are more complicated, it is not at all clear in advance
where the largest and smallest values of the function occur.
}

\end{block}

\begin{quickcheckcontainer}
\randomquickcheckpool{1}{2}
\begin{quickcheck}
		\field{rational}
		\type{input.number}
		\begin{variables}
			\randint{a}{-3}{1}
            \randint{b}{2}{8}
            \function[calculate]{l}{(b-a)/4}
			\function[calculate]{sol1}{a+l}
            \function[calculate]{sol2}{a+2*l}
            \function[calculate]{sol3}{a+3*l}
		\end{variables}

    \lang{de}{
		\text{Das Intervall $[\var{a};\var{b}]$ soll in vier äquidistante Teilintervalle zerlegt werden.
        Die Zerlegungsstellen sind gegeben durch: \\
        $\var{a} <$ \ansref $<$ \ansref $<$ \ansref $< \var{b}$}
		}
    \lang{en}{
    \text{Suppose we partition the interval [\var{a},\var{b}] into four subintervals
    of equal length. The points of the partition are: \\
    $\var{a} <$ \ansref $<$ \ansref $<$ \ansref $< \var{b}$}
    }
		\begin{answer}
			\solution{sol1}
		\end{answer}
        \begin{answer}
			\solution{sol2}
		\end{answer}
        \begin{answer}
			\solution{sol3}
		\end{answer}
    \lang{de}{
		\explanation{Die Länge jedes Teilintervalls ist $\var{l}$.}
    }
    \lang{en}{
    \explanation{The length of each subinterval is $\var{l}$.}
    }
	\end{quickcheck}
	
	\begin{quickcheck}
		\field{rational}
		\type{input.number}
		\begin{variables}
			\randint{a}{-3}{1}
            \randint{b}{2}{8}
            \function[calculate]{l}{(b-a)/5}
			\function[calculate]{sol1}{a+l}
            \function[calculate]{sol2}{a+2*l}
            \function[calculate]{sol3}{a+3*l}
            \function[calculate]{sol4}{a+4*l}
		\end{variables}
		\lang{de}{
		\text{Das Intervall $[\var{a};\var{b}]$ soll in fünf äquidistante Teilintervalle zerlegt werden.
        Die Zerlegungsstellen sind gegeben durch: \\
        $\var{a} <$ \ansref $<$ \ansref $<$ \ansref $<$ \ansref $< \var{b}$}
        }
    \lang{en}{
    \text{Suppose we partition the interval [\var{a},\var{b}] into five subintervals
    of equal length. The points of the partition are: \\
    $\var{a} <$ \ansref $<$ \ansref $<$ \ansref $<$ \ansref $< \var{b}$}
    }
		
		\begin{answer}
			\solution{sol1}
		\end{answer}
        \begin{answer}
			\solution{sol2}
		\end{answer}
        \begin{answer}
			\solution{sol3}
		\end{answer}
        \begin{answer}
			\solution{sol4}
		\end{answer}
		\lang{de}{
		\explanation{Die Länge jedes Teilintervalls ist $\var{l}$.}
    }
    \lang{en}{
    \explanation{The length of each subinterval is $\var{l}$.}
    }
	\end{quickcheck}
\end{quickcheckcontainer}


\lang{de}{
Wenn Ober- und Untersumme einander für immer feinere Zerlegungen des Intervalls 
immer weiter annähern (und damit auch der gesuchten Fläche), werden wir die 
Funktion als \emph{integrierbar} bezeichnen.
}
\lang{en}{
If the upper and lower Riemann sums approach each other (and therefore the area in question as well)
as the partitions of the interval become increasingly fine
then we call the function \emph{integrable}.
}

\begin{definition}
\label{def:integral}
%\emph{Definition:} 
\lang{de}{
Es sei $f(x)$ eine auf dem Intervall $[a;b]$ beschränkte Funktion (z.\,B. eine stetige Funktion).
}
\lang{en}{
Let $f(x)$ be a function that is bounded on the interval $[a, b]$.
(For example, $f$ can be any continuous function.)
}
\lang{de}{
Wir nennen $f$ \notion{integrierbar} auf dem Intervall $[a;b]$, falls sich Untersumme und Obersumme
durch Verfeinerung der Zerlegung beliebig genau annähern, d.\,h. falls es zu jeder
vorgegebenen Zahl $\epsilon>0$ (und sei sie noch so klein) eine Zerlegung $Z$ des
Intervalls $[a;b]$ gibt, sodass
\[ 0 \leq O(Z) - U(Z) \leq \epsilon \]
gilt.
}
\lang{en}{
We call $f$ \notion{integrable} on the interval $[a, b]$ if the lower and upper Riemann sums
can be made arbitrarily close by choosing a fine enough partition;
that is, for any given (arbitrarily small) number $\epsilon > 0$, there exists a
partition $Z$ of the interval $[a, b]$ for which
\[ 0 \leq O(Z) - U(Z) \leq \epsilon.\]
}

\lang{de}{
Ist $f$ integrierbar auf $[a;b]$, so gibt es eine eindeutige reelle Zahl, die bei allen
Zerlegungen des Intervalls in endlich viele Teilintervalle zwischen den Werten von Ober- und Untersumme liegt. 
Wir bezeichnen diese Zahl als \notion{bestimmtes Integral von $f$ von $a$ nach $b$} und  
schreiben dafür $\big\int_a^b f(x)\, dx$. Diese Zahl können wir als orientierten Flächeninhalt interpretieren.
}
\lang{en}{
If $f$ is integrable on $[a, b]$, then there is a unique real number that lies between
the upper and lower Riemann sums attached to any partition of the interval into finitely
many subintervals. We call this number the \notion{definite integral of $f$ from $a$ to $b$}
and denote it $\big\int_a^b f(x)\, dx$.
This number can be interpreted as the signed area.
}
\end{definition}

\lang{de}{
Die obere Integralgrenze muss nicht zwangsläufig größer als die untere Grenze sein: 
}
\lang{en}{
The upper bound of the integral does not necessarily have to be greater
than the lower bound:
}
\begin{definition} 
%\emph{Definition:} 
\lang{de}{Wenn $f(x)$ auf dem Intervall $[a;b]$ mit $a<b$ integrierbar ist, dann definiert man 
auf folgende Weise das Integral mit vertauschten Grenzen:
\[ \int_b^a f(x)\, dx = - \int_a^b f(x)\, dx \, . \]
Außerdem setzt man $\big\int_a^a f(x)\, dx = 0$.
}
\lang{en}{
If $f(x)$ is integrable on the interval $[a, b]$ where $a < b$, then the integral
with swapped bounds is defined as follows:
\[ \int_b^a f(x)\, dx = - \int_a^b f(x)\, dx \, . \]
Also, we define $\big\int_a^a f(x)\, dx = 0$.
}
\end{definition}
%Beispiel: 

  



\begin{block}[warning]
\lang{de}{Über Definitionslücken darf nicht hinweg integriert werden und auch 
Integrale über unbeschränkte Intervalle wie z.\,B. $[0;\infty)$ 
oder Integrale unbeschränkter Funktionen sind 
zunächst nicht zugelassen. Wir werden solche Integrale im Kapitel über 
\emph{uneigentliche Integrale} untersuchen. 
}
\lang{en}{
It is not valid to integrate a function across a gap where it is not defined.
Also, integrals over unbounded intervals such as $[0, \infty)$ are not allowed
for now. We will investigate integrals of this type in the chapter on
\emph{improper integrals}.
}
\end{block}


\lang{de}{Der Integrationsbereich kann auch geteilt oder aus Teilen zusammengesetzt werden:}

\lang{en}{
The range of integration can also be split into pieces or put back together:
}

\begin{rule}
\lang{de}{Für reelle Zahlen $a<b<c$ und eine auf dem Intervall $[a;c]$ integrierbare Funktion $f(x)$ gilt:}
\lang{en}{
For real numbers $a<b<c$ and a function $f$ that is integrable on the interval $[a, c]$, 
}

\[ \int_a^b f(x)\, dx + \int_b^c f(x)\, dx = \int_a^c f(x)\, dx \,  . \]
\end{rule}

\section{
\lang{de}{Stammfunktionen}
\lang{en}{Antiderivatives}
}

\lang{de}{
Die Integration hat eine enge Beziehung zur Ableitung einer Funktion und kann als ihre 
Umkehrung angesehen werden, wie wir beim \emph{Hauptsatz der Differential- und Integralrechnung} sehen werden.

Zur Vorbereitung des Satzes brauchen wir noch einen neuen Begriff, nämlich den der \emph{Stammfunktion}.
}

\lang{en}{
Integration is closely related to differentiation and can be viewed as its inverse operation,
as we will see in the \emph{fundamental theorem of calculus}.

To state the theorem, we will need to discuss a new concept, \emph{antiderivatives}.
}

\begin{definition}\label{def:stammfkt}
\begin{itemize}
\item \lang{de}{Eine auf einem Intervall $I$ differenzierbare Funktion $F$ heißt \notion{Stammfunktion} von $f$, wenn
\[ F'(x) = f(x) \]
für alle $x \in I$ erfüllt ist. }
\lang{en}{
A function $F$ that is differentiable on an interval $I$ is called an
\notion{antiderivative}, or \notion{primitive function} of $f$ if 
\[ F'(x) = f(x) \]
for all $x \in I$. 
}
\item \lang{de}{Die Menge aller Stammfunktionen von $f$ bezeichnet man als \notion{unbestimmtes Integral} und schreibt dafür
\[ \int f(x)\, dx = F(x) + C , \ C \in \mathbb{R},  \]
wobei $F(x)$ eine beliebige Stammfunktion von $f(x)$ ist.}
\lang{en}{The set of all antiderivatives of $f$ is called the \notion{indefinite integral},
written \[ \int f(x)\, dx = F(x) + C , \ C \in \mathbb{R},  \]
where $F(x)$ denotes an arbitrary antiderivative of $f(x)$.
}
\end{itemize}
\end{definition}
\lang{de}{
Der Unterschied zwischen einem bestimmten und einem unbestimmten Integral besteht dem Aussehen nach nur darin, ob 
unter und über dem Integralzeichen Zahlen stehen oder nicht. Mathematisch unterscheiden sich bestimmtes und 
unbestimmtes Ingeral hingegen sehr stark: Das bestimmte Integral ist eine Zahl (ein Flächeninhalt), während das 
unbestimmte Integral die Menge der Stammfunktionen bezeichnet.
}
\lang{en}{
The only difference in notation between definite and indefinite integrals is the
presence or lack of numbers above and below the integral sign.
However, definite and indefinite integrals are very different mathematically:
a definite integral is a number (describing an area), whereas an indefinite integral
describes a set of antiderivatives.
}

\begin{example}
%\emph{Beispiel:} 
\lang{de}{Es sei $f(x)=x^2+1$. Dann ist $F(x)=\frac{1}{3}x^3+x$ eine Stammfunktion
von $f(x)$ auf $\mathbb{R}$, denn es gilt $F'(x)=x^2+1=f(x)$.}
\lang{en}{Let $f(x)=x^2+1$. Then $F(x) = \frac{1}{3}x^3 + x$ is an antiderivative
of $f(x)$ on $\mathbb{R}$ because $F'(x) = x^2+1 = f(x)$.}
\end{example} 
\lang{de}{Wie bestimmt man eine Stammfunktion
$F(x)$, wenn nur $f(x)$ gegeben ist? Durch Umkehrung der Ableitung, d.\,h., man
 sucht eine Funktion, deren Ableitung $f(x)$ ist. \\
 }
 \lang{en}{
How do we determine an antiderivative $F(x)$ if only $f(x)$ is given?
By differentiating in reverse, i.e. we look for a function whose derivative is $f(x)$. \\
 }
 
 
 
\begin{block}[warning]
\lang{de}{Stammfunktionen sind (anders als Ableitungen) nicht eindeutig! 
Falls $F(x)$ eine Stammfunktion von $f(x)$ ist, dann ist auch $F(x)+1$ eine
Stammfunktion, denn beim Ableiten fällt diese 
Konstante weg. 

Umgekehrt unterscheiden sich zwei Stammfunktionen zur gleichen 
Funktion immer nur um Addition einer Konstante.}
\lang{en}{
Antiderivatives (unlike derivatives) are not unique!
If $F(x)$ is an antiderivative of $f(x)$ then $F(x)+1$ is also an antiderivative,
because the constant disappears upon differentiating.

Conversely, two antiderivatives of the same function always differ only by a constant.
}
\end{block}  


\lang{de}{Wie für die Differentiation gilt bei der Bildung der Stammfunktion auch die Summen- und Faktorregel: 
}
\lang{en}{
Similarly to derivatives, antiderivatives respect addition and scalar multiplication:
}

\begin{rule}
\lang{de}{
Sind $F(x)$ und $G(x)$ Stammfunktionen der Funktionen $f(x)$ bzw. $g(x)$, und sind $\alpha$
und $\beta$ reelle Zahlen, so ist eine Stammfunktion der Funktion $\alpha f(x)+\beta g(x)$ gegeben durch
\[  \alpha F(x)+\beta G(x). \]}
\lang{en}{
If $F(x)$ and $G(x)$ are antiderivatives of the functions $f(x)$ and $g(x)$, and if
$\alpha$ and $\beta$ are real numbers, then an antiderivative of the function $\alpha f(x)+\beta g(x)$
is given by 
\[  \alpha F(x)+\beta G(x). \]
}
\end{rule}
\lang{de}{
Der Beweis erfolgt durch Ableiten der Funktion $\alpha F(x)+\beta G(x)$ mit Hilfe der 
Summen- und Faktorregel für die Ableitung.
}
\lang{en}{
This is proved by differentiating the function $\alpha F(x)+\beta G(x)$ using the addition
and scalar multiplication rules for derivatives.
}
\begin{block}[warning]
\lang{de}{
Die anderen Ableitungsregeln wie Produktregel, Quotientenregel und Kettenregel lassen sich nicht direkt auf die Integration übertragen, da zum Beispiel die Ableitung eines Produkts $u(x)\cdot v(x)$ gegeben ist durch
\[  (u(x)v(x))'= u'(x)v(x)+u(x)v'(x), \]
was jedoch kein Produkt, sondern die Summe zweier Produkte ist. 

In diesen Fällen werden kompliziertere Integrationsmethoden gebraucht, 
die wir im nächsten Kapitel behandeln werden.}
\lang{en}{
The other rules of differentiation, such as the product rule, quotient rule and chain rule,
cannot be transformed directly into rules for antiderivatives. For example, the derivative
of a product $u(x)\cdot v(x)$ is
\[  (u(x)v(x))'= u'(x)v(x)+u(x)v'(x); \]
however, this is no longer a product, but rather the sum of two products.

For these cases, we will need more complicated methods of integration
that will be described in the next chapter.
}
\end{block}

\begin{quickcheckcontainer}
\randomquickcheckpool{1}{1}
\begin{quickcheck}
		\field{real}
		\type{input.function}
		\begin{variables}
			\randint{v}{0}{1}
			\randint[Z]{a}{-5}{5}
			\randint[Z]{b}{1}{4}
			\randint{c}{-4}{4}
		    \function[normalize]{ff}{a*x^3+v*b*sin(x)+(1-v)*b*cos(x)+c*exp(x)}
			\function[normalize]{sol}{3*a*x^2+v*b*cos(x)-(1-v)*b*sin(x)+c*exp(x)}
		\end{variables}


         \lang{de}{\text{Zu welcher Funktion ist $F(x)=\var{ff}$ eine Stammfunktion?\\ Antwort: Zu der Funktion
			$f(x)=$\ansref.}}
         \lang{en}{\text{
          For which function is $F(x)=\var{ff}$ an antiderivative?\\
          Answer: $f(x)=$\ansref.
         }}

		\begin{answer}
			\solution{sol}
			\checkAsFunction{x}{-5}{5}{50}
		\end{answer}
		\lang{de}{\explanation{ $F(x)$ ist nach Definition eine Stammfunktion von $f(x)$, wenn $f(x)$ die 
		Ableitungsfunktion von $F(x)$ ist. Also ist $f(x)=F'(x)=\var{sol}$.}  } 
    \lang{en}{\explanation{
    By definition, $F(x)$ is an antiderivative of $f(x)$ if $f(x)$ is the derivative of $F(x)$.
    Therefore, $f(x)=F'(x)=\var{sol}$.
    }}
	\end{quickcheck}
\end{quickcheckcontainer}
 

\lang{de}{Stammfunktionen einiger grundlegender Funktionen lassen sich
einfach angeben. Der Nachweis erfolgt durch Ableitung.}
\lang{en}{
It is not difficult to find antiderivatives of some basic functions.
These can be verified by differentiating.
}
\lang{de}{
\begin{table}
Funktion $f(x)$ & Stammfunktion $F(x)$ & Bedingung\\
$x^k$ mit $k\in \mathbb{Z}$ und $k \neq -1$ & $\frac{1}{k+1} x^{k+1} $ & $x \neq 0$, falls $k<0$ \\
$x^{-1}$ & $\ln(|x|)$ & $x \neq 0$ \\
$e^x$ & $e^x $ & $x \in \mathbb{R}$\\
$\sin(x)$ & $-\cos(x) $ & $x \in \mathbb{R}$\\
$\cos(x)$ & $\sin(x)$ &$x \in \mathbb{R}$
\end{table}}
\lang{en}{
\begin{table}
Function $f(x)$ & Antiderivative $F(x)$ & Conditions \\
$x^k$ with $k\in \mathbb{Z}$ and $k \neq -1$ & $\frac{1}{k+1} x^{k+1} $ & $x \neq 0$ if $k<0$ \\
$x^{-1}$ & $\ln(|x|)$ & $x \neq 0$ \\
$e^x$ & $e^x $ & $x \in \mathbb{R}$\\
$\sin(x)$ & $-\cos(x) $ & $x \in \mathbb{R}$\\
$\cos(x)$ & $\sin(x)$ &$x \in \mathbb{R}$
\end{table}
}

\begin{example}
%\emph{Beispiel:} 
\lang{de}{
In einem Unternehmen seien die Grenzkosten $K'(x)$ für die Produktion von $x$ Einheiten
eines Guts gegeben durch die Funktionsgleichung $K'(x) = 0,4 \cdot x^3 -4x + 5$.

Wenn wir hieraus die Kostenfunktion $K(x)$ berechnen wollen, müssen wir eine Stammfunktion der 
Grenzkostenfunktion finden. Mit anderen Worten: Wir müssen Ableiten rückgängig machen, um von 
$K'(x)$ zu $K(x)$ zu kommen. 

Indem wir uns die einzelnen Summanden in der Gleichung von $K'(x)$ separat anschauen, erhalten wir 
als Stammfunktion $K_0 (x) = 0,1 \cdot x^4 - 2 x^2 + 5x$. Dies wird jedoch nicht die gesuchte Kostenfunktion 
$K(x)$ sein. Es gilt nämlich $K_0(0) = 0$, also betragen die Kosten 0 Euro, wenn keine Einheit des Guts
produziert wird. In der Praxis ist das nahezu ausgeschlossen, da auch dann, wenn Unternehmen nichts
produzieren können, Fixkosten anfallen.

Wie wir oben schon gesehen haben, sind Stammfunktionen nicht eindeutig, sondern unterscheiden sich 
dadurch, dass man eine Konstante addiert. Diese Konstante sind hier die Fixkosten. 

Die Kostenfunktion ist daher gegeben durch 
\[
K(x) = 0,1 \cdot x^4 - 2 x^2 + 5x + K_{fix}.
\]
}

\lang{en}{
Suppose the marginal costs $K'(x)$ for a company to produce $x$ units of a good
are described by the function $K'(x)=0.4 \cdot x^3 - 4x + 5$.

To determine the cost function $K(x)$ using this, we must find an antiderivative of the
marginal cost function. In other words: we must undo the derivative to go from $K'(x)$ to $K(x)$.

By looking at the summands in the definition of $K'(x)$ separately, we find the antiderivative
$K_0(x) = 0.1 \cdot x^4 - 2x^2 + 5x$. This will not be the cost function that we are looking for, however.
Since $K_0(0)$, the costs of producing no units at all would be $0$ euros.
This is almost impossible in practice, since fixed costs apply even when companies are unable to produce anything.

As we have already seen, antiderivatives are not unique; instead, they depend on the addition of a constant.
Here, the constant represents fixed costs. So the cost function is
\[
K(x) = 0.1 \cdot x^4 - 2 x^2 + 5x + K_{fix}.
\]
}

\end{example}

\begin{quickcheckcontainer}
\randomquickcheckpool{1}{1}
\begin{quickcheck}
	\field{real}
		\type{input.function}
		\begin{variables}
			\randint{v}{0}{1}
			\randint[Z]{a}{-5}{5}
			\randint[Z]{b}{1}{4}
			\randint{c}{-4}{4}
			\randint[Z]{d}{1}{4}
			\function[calculate]{y0}{(1-v)*b+c}
		    \function[normalize]{f0}{d*x+(a/4)*x^4+v*b*sin(x)+(1-v)*b*cos(x)+c*exp(x)}
		    \function[normalize]{ff}{f0-y0}
			\function[normalize]{f}{d+a*x^3+v*b*cos(x)-(1-v)*b*sin(x)+c*exp(x)}
		\end{variables}
		
		\lang{de}{
        \text{Bestimmen Sie diejenige Stammfunktion $F(x)$ von $f(x)=\var{f}$, für welche $F(0)=0$ gilt.\\
		$F(x)=$\ansref.} }
    \lang{en}{
        \text{Determine the antiderivative $F(x)$ of $f(x)=\var{f}$, for which $F(0)=0$.\\
        $F(x)=$\ansref.}
    }
		\begin{answer}
			\solution{ff}
			\checkAsFunction{x}{-5}{5}{50}
		\end{answer}
		\lang{de}{\explanation{
        Mit der Summen-
		und der Faktorregel, sowie den Formeln für Stammfunktionen von Potenzen, Sinus, Kosinus und 
		Exponentialfunktion erhält man zunächst als Stammfunktion $G(x)=\var{f0}$.\\
		Also ist $F(x)=G(x)-C$, wobei die Konstante $C$ so gewählt werden muss, dass $F(0)=0$ gilt.\\
		Also $C=G(0)=\var{y0}$, und daher $F(x)=\var{ff}$.}
		}
    \lang{en}{\explanation{
        Using the rules for antiderivatives of sums and scalar multiples,
        as well as the formulas for the antiderivatives of powers, sine, cosine and exp,
        we first obtain the antiderivative $G(x)=\var{f0}$. \\
        Then $F(x)=G(x)-C$, where the constant $C$ must be chosen such that $F(0)=0$. \\
        Therefore, $C=G(0)=\var{y0}$ and $F(x)=\var{ff}$.
    }}
	\end{quickcheck}
\end{quickcheckcontainer}


\section{
\lang{de}{Der Hauptsatz der Differential- und Integralrechnung}
\lang{en}{Fundamental theorem of calculus}
}\label{sec:hauptsatz}
\lang{de}{
Das folgende Theorem wird uns in die Lage versetzen, Integrale und Flächeninhalte exakt auszurechnen. 
}
\lang{en}{
The theorem below will enable us to compute integrals and areas exactly.
}
\begin{theorem}[\lang{de}{Hauptsatz}
\lang{en}{Fundamental theorem of calculus}]
\lang{de}{Ist $f(x)$ auf dem Intervall $[a;b]$ stetig, dann  
existiert eine Stammfunktion $F(x)$ von $f(x)$ (d.\,h. $F'(x)=f(x)$) und 
für jede beliebige Stammfunktion $F(x)$ gilt
\[ \int_a^b f(x)\, dx =  F(b) - F(a) .\]
}
\lang{en}{
Suppose $f(x)$ is continuous on the interval $[a, b]$.
Then there exists an antiderivative $F(x)$ of $f(x)$ (i.e. $F'(x)=f(x)$),
and for any antiderivative $F(x)$,
\[ \int_a^b f(x)\, dx =  F(b) - F(a) .\]
}
\end{theorem}
\lang{de}{
Der Hauptsatz der Differential- und Integralrechnung beantwortet damit die Frage, wie man die Fläche zwischen einem
Funktionsgraphen und der $x$-Achse berechnen kann: Wir müssen dazu eine Stammfunktion finden und dann die 
Grenzen einsetzen und subtrahieren. Wir werden auch häufig die Schreibweise
\[
\Big[\, F(x)\, \Big]_a^b = F(b) - F(a)
\]
benutzen, um Rechnungen übersichtlicher zu gestalten.
}
\lang{en}{
The fundamental theorem of calculus answers the question of how to find the area enclosed
between the graph of a function and the $x$-axis: we have to find an antiderivative, 
substitute the bounds into it and then subtract. We will often use the notation
\[
\Big[\, F(x)\, \Big]_a^b = F(b) - F(a)
\]
to make calculuations easier to follow.
}

\begin{example}
\lang{de}{Die Funktion $f(x)=\frac{2}{x^2} - 6 x^2$ ist auf dem Intervall $[1;2]$ stetig, 
$F(x)=- \frac{2}{x} - 2x^3 + 20$ ist eine Stammfunktion und mit Hilfe des Hauptsatzes können wir das 
Integral 
\[ \int_1^2 \Big(\frac{2}{x^2} - 6 x^2 \Big)\, dx =
\Big[- \frac{2}{x} - 2x^3 + 20 \Big]_1^2 = 3 - 16 = -13 \]
berechnen.
}
\lang{en}{
The function $f(x)=\frac{2}{x^2} - 6x^2$ is continuous on the interval $[1, 2]$.
$F(x)=-\frac{2}{x} - 2x^3 + 20$ is an antiderivative. Using the fundamental theorem,
we can evaluate the integral:
\[ \int_1^2 \Big(\frac{2}{x^2} - 6 x^2 \Big)\, dx =
\Big[- \frac{2}{x} - 2x^3 + 20 \Big]_1^2 = 3 - 16 = -13 .\]
}
\end{example}

\begin{quickcheckcontainer}
\randomquickcheckpool{1}{1}
\begin{quickcheck}
		\field{rational}
		\type{input.number}
		\begin{variables}
			\randint[Z]{a}{-2}{-1}
			\randint[Z]{b}{1}{2}
			\randint[Z]{c3}{-4}{4}
			%\randint[Z]{c1}{-4}{4}
			\number{c1}{0}		% mit c1=0 wird es einfacher.
			\randint[Z]{c0}{-4}{4}
		    \function[normalize]{f}{c3*x^3+c1*x-c0}
			\function[normalize]{ff}{(c3/4)*x^4+(c1/2)*x^2-c0*x}
			\function[calculate]{ffa}{(c3/4)*a^4+(c1/2)*a^2-c0*a}
			\function[calculate]{ffb}{(c3/4)*b^4+(c1/2)*b^2-c0*b}
			\function[calculate]{sol}{ffb-ffa}
		\end{variables}
        
            \lang{de}{\text{Bestimmen Sie den Wert des folgenden Integrals:\\ 
			$\int_{\var{a}}^{\var{b}} (\var{f})dx=$\ansref.}}
           \lang{en}{\text{Find the value of the following integral:\\
           $\int_{\var{a}}^{\var{b}} (\var{f})dx=$\ansref.}}
			

		\begin{answer}
			\solution{sol}
		\end{answer}
		\lang{de}{\explanation{ Eine Stammfunktion von $f(x)=\var{f}$ ist $F(x)=\var{ff}$.\\
		Damit ist $\int_{\var{a}}^{\var{b}} (\var{f})dx=F(\var{b})-F(\var{a})=\var{ffb}-(\var{ffa})=\var{sol}$.}
		}
    \lang{en}{
    \explanation{$F(x)=\var{ff}$ is an antiderivative of $f(x)=\var{f}$.\\
    Therefore, $\int_{\var{a}}^{\var{b}} (\var{f})dx=F(\var{b})-F(\var{a})=\var{ffb}-(\var{ffa})=\var{sol}$.}
		}
	\end{quickcheck}
\end{quickcheckcontainer}

\end{content}