%$Id:  $
\documentclass{mumie.article}
%$Id$
\begin{metainfo}
  \name{
    \lang{de}{Uneigentliche Integrale}
    \lang{en}{}
  }
  \begin{description} 
 This work is licensed under the Creative Commons License Attribution 4.0 International (CC-BY 4.0)   
 https://creativecommons.org/licenses/by/4.0/legalcode 

    \lang{de}{Beschreibung}
    \lang{en}{}
  \end{description}
  \begin{components}
    \component{generic_image}{content/rwth/HM1/images/g_tkz_T604_Area_D.meta.xml}{T604_Area_D}
    \component{generic_image}{content/rwth/HM1/images/g_img_00_Videobutton_schwarz.meta.xml}{00_Videobutton_schwarz}
    \component{js_lib}{system/media/mathlets/GWTGenericVisualization.meta.xml}{mathlet1}
  \end{components}
  \begin{links}
    \link{generic_article}{content/rwth/HM1/T301_Differenzierbarkeit/g_art_content_02_ableitungsregeln.meta.xml}{content_02_ableitungsregeln}
    \link{generic_article}{content/rwth/HM1/T304_Integrierbarkeit/g_art_content_08_integral_eigenschaften.meta.xml}{int-barkeit}
    \link{generic_article}{content/rwth/HM1/T304_Integrierbarkeit/g_art_content_09_integrierbare_funktionen.meta.xml}{stammfunktion}
    \link{generic_article}{content/rwth/HM1/T305_Integrationstechniken/g_art_content_12_substitutionsregel.meta.xml}{substitution}
  \end{links}
  \creategeneric
\end{metainfo}
\begin{content}
\usepackage{mumie.ombplus}
\usepackage{mumie.genericvisualization}
\ombchapter{4}
\ombarticle{3}

\begin{visualizationwrapper}
\lang{de}{\title{Uneigentliche Integrale}}
\lang{en}{\title{Improper integrals}}
 

\begin{block}[annotation]

\end{block}

\begin{block}[annotation]
	Im Ticket-System: \href{https://team.mumie.net/issues/22701}{Ticket 22701}
\end{block}

\begin{block}[info-box]
\tableofcontents
\end{block}


\section{
\lang{de}{Uneigentliche Integrale}
\lang{en}{Improper integrals}
}
\lang{de}{
Wir haben in den vorigen Kapiteln Integrale $\int_a^b f(x)\, dx$ kennengelernt und mit Hilfe von Stammfunktionen
berechnet. 
Eine wichtige Voraussetzung war, dass  das Intervall $[a;b]$ ein endliches Intervall ist und dass $f$ auf dem ganzen Intervall $[a;b]$ definiert und beschränkt ist.
In diesem Abschnitt wollen wir den Begriff etwas erweitern und auch Integrale über unendlichen Intervallen definieren sowie Integrale, bei denen die Funktion $f$ unbeschränkt oder an einer Integralgrenze gar nicht definiert ist.
Da diese keine Integrale im eigentlichen Sinn sind, heißen sie \emph{uneigentliche Integrale}.
}
\lang{en}{
In the previous chapters, we introduced integrals
$\int_a^b f(x) \, dx$ and learned how to calculate them
by means of antiderivatives.
Up to now, we have always required the interval $[a, b]$ to be
finite and $f$ to be defined and bounded on the entire interval $[a, b]$.
In this chapter, we will relax those conditions and define both
integrals over infinite intervals and integrals of functions $f$
that may be unbounded or even undefined at the bounds of integration.
Since these are not true integrals in the strict sense,
we call them \emph{improper integrals}.
}


\begin{definition}
 
\lang{de}{Das \notion{uneigentliche Integral} der Funktion $f$ \"uber dem Intervall $[a;\infty)$ ist definiert als Grenzwert}
\lang{en}{
The \notion{improper integral} of the function $f$ over the integral $[a, \infty)$
is defined as the limit
}
  \\
\[ \int_a^\infty f\left(x\right) \; dx \coloneq
 \lim\limits_{d \to  \infty} \int_a^d f\left(x\right) \; dx,
 \]
 \lang{de}{
 sofern der Grenzwert und die Integrale über dem Intervall $[a; d]$ existieren. 
 (Andernfalls sagen wir: Das uneigentliche Integral $\int_a^\infty f\left(x\right) \; dx$ existiert nicht.)
 }
 \lang{en}{
  if both the integrals over the intervals $[a, d]$ and the limit exist.
  (Otherwise, we say that the improper integral $\int_a^\infty f\left(x\right) \; dx$ does not exist.)
 }

\lang{de}{
Auf die gleiche Weise definieren wir das uneigentliche Integral
  \[ \int_{-\infty}^b f(x)\, dx :=\lim_{a\to -\infty}  \int_{a}^b f(x)\, dx, \]
  sofern Integrale und Grenzwert auf der rechten Seite existieren, sowie 
\[  \int_{-\infty}^\infty f(x)\, dx:= \int_{-\infty}^0 f(x)\, dx +   \int_{0}^\infty f(x)\, dx. \]
Statt das Integral an der Stelle $0$ aufzuteilen, kann auch jede andere reelle Zahl gewählt werden.
}
\lang{en}{
Similarly, we define the improper integral
\[ \int_{-\infty}^b f(x)\, dx :=\lim_{a\to -\infty}  \int_{a}^b f(x)\, dx, \]
if the integrals and limit on the right-hand side exist, as well as
\[  \int_{-\infty}^\infty f(x)\, dx:= \int_{-\infty}^0 f(x)\, dx +   \int_{0}^\infty f(x)\, dx. \]
The integral can also be split at any other real number,
not only at $0$.
}


 %\floatright{\href{https://www.hm-kompakt.de/video?watch=609}{\image[75]{00_Videobutton_schwarz}}}\\\\
\end{definition}

\lang{de}{
Die folgende Bemerkung gibt eine Bedingung, die für die Existenz eines uneigentlichen Integrals
wie in der Definition erfüllt sein muss. 
}
\lang{en}{
The remark below gives a necessary criterion for an 
improper integrals in the sense of the above definition to exist.
}

\begin{remark}
\lang{de}{
Damit für eine stetige Funktion $f:[a;\infty)\to\R$ das uneigentliche Integral $\int_a^\infty f(x)\, dx$ existieren kann,
muss notwendigerweise $\lim_{x\to\infty}f(x) = 0$ gelten.
Ansonsten kann der Flächeninhalt unter dem Graphen beliebig groß werden.
In den nächsten Beispielen betrachten wir uneigentliche Integrale von Funktionen mit $\lim_{x\to\infty}f(x) = 0$.
Wir werden sehen, dass die Bedingung für die Existenz nicht ausreicht. Dies ist vergleichbar mit der Reihenkonvergenz.
}
\lang{en}{
For the improper integral $\int_a^{\infty} f(x) \, dx$
of a continuous function $f : [a, \infty) \to\R$ to exist,
we must have $\lim_{x\to\infty}f(x) = 0$.
Otherwise, the area under the graph will become
infinitely large. In the examples below, we will
consider the improper integrals of functions with
$\lim_{x\to\infty} f(x) = 0$ and we will see that
this condition is not sufficient. The situation
is comparable to the convergence of series.
}
\end{remark}

\begin{example}\label{ex1}
\begin{tabs*}
\tab{$\int_1^\infty \frac{1}{x}\,dx$}
\lang{de}{
Für die Funktion $f(x)=\frac{1}{x}$ existiert das uneigentliche Integral
$ \int_1^\infty \frac{1}{x}\, dx$ nicht, d.\,h. }
\lang{en}{
For the function $f(x)=\frac{1}{x}$, the improper integral
$ \int_1^\infty \frac{1}{x}\, dx$ does not exist; that is,
}
\[
  \lim\limits_{b\rightarrow \infty}\;\int_1^b \frac{1}{x}\, dx \quad \text{\lang{de}{existiert nicht}\lang{en}{does not exist}}.
 \]

 \lang{de}{
 Da die Funktion $f(x)=\frac{1}{x}$ für positive reelle Zahlen nämlich genau die Ableitung des Logarithmus ist, ist $\ln(x)$ eine Stammfunktion für $f$. Damit gilt:
 }
 \lang{en}{
  Since $f(x)=\frac{1}{x}$ is the derivative of the logarithm
  for positive real numbers, $\ln(x)$ is an antiderivative of $f$. Therefore,
 }
 \begin{eqnarray*}
  \lim\limits_{b\rightarrow \infty}\;\int_1^b \frac{1}{x}\, dx
  &=&\lim\limits_{b\rightarrow \infty} \;\left[ \ln(x)\right]_1^b\\
  &=&\lim\limits_{b\rightarrow \infty} \;\ln(b) =\infty \, .
  \end{eqnarray*}
  \lang{de}{Die bestimmten Integrale wachsen also unbeschränkt für $b \rightarrow \infty$, weshalb das uneigentliche Integral
  $ \int_1^\infty \frac{1}{x}\, dx$ nicht existiert.
  }
  \lang{en}{
  The definite integrals grow infinitely large as $b \to \infty$, so
  the improper integral $ \int_1^\infty \frac{1}{x}\, dx$ does not exist.
  }
\tab{$\int_1^\infty \frac{1}{x^2}\,dx$}
\lang{de}{Wir betrachten die Funktion $f:(0;\infty)\to \R$ mit $f(x)=\frac{1}{x^2}=x^{-2}$. Eine Stammfunktion für $f$ ist die Funktion
\[ F(x)=-{x^{-1}}, \]
wie man durch Ableiten von $F(x)$ sieht. Für $b>1$ ist damit
\begin{eqnarray*}
\int_1^b \;\frac{1}{x^2}\, dx &=&  \left[ -x^{-1}\right]_1^b\\
&=& -b^{-1} -   (-1). 
\end{eqnarray*} 
}
\lang{en}{
Consider the function $f:(0,\infty)\to\R$, $f(x)=\frac{1}{x^2}=x^{-2}$.
An antiderivative of $f$ is given by the function
\[ F(x)=-{x^{-1}}, \]
as one can see by differentiating $F(x)$. Therefore, for $b>1$,
\begin{eqnarray*}
\int_1^b \;\frac{1}{x^2}\, dx &=&  \left[ -x^{-1}\right]_1^b\\
&=& -b^{-1} -   (-1). 
\end{eqnarray*} 
}

\lang{de}{
Es folgt
}
\lang{en}{
It follows that
}
\begin{eqnarray*}
\int_1^\infty \;\frac{1}{x^2}\, dx &=& 
\lim_{b\to \infty} \int_1^b \;\frac{1}{x^2}\, dx \\
&=& \lim_{b\to \infty}  -\frac{1}{b} -   (-1) \\
&=&1.
\end{eqnarray*}
\lang{de}{
Das uneigentliche Integral $\int_1^\infty \;\frac{1}{x^2}\, dx$ existiert in diesem Fall also.
}
\lang{en}{
So in this case, the improper integral $\int_1^\infty \;\frac{1}{x^2}\, dx$ exists.
}
\end{tabs*}
\end{example}
\begin{quickcheck}
    \begin{variables}
        \randint{a}{-1}{1}
        \randint[Z]{b}{-3}{3}
        \function[normalize]{f}{b*x^(a/2)}
    \end{variables}
    \lang{de}{
    \text{Existiert das uneigentliche Integral $\int_1^\infty \var{f} \, dx$?}
    }
    \lang{en}{
    \text{Does the improper integral $\int_1^\infty \var{f} \, dx$ exist?}
    }
    \lang{de}{
    \explanation{Finden Sie eine Stammfunktion und berechnen Sie den Grenzwert.}
    }
    \lang{en}{
    \explanation{Find an antiderivative and compute the limit.}
    }
    \begin{choices}{unique}
        \begin{choice}
            \lang{de}{
            \text{Ja.}
            }
            \lang{en}{
            \text{Yes.}
            }
            \solution{false}
        \end{choice}
        \begin{choice}
            \lang{de}{
            \text{Nein.}
            }
            \lang{en}{
            \text{No.}
            }
            \solution{true}
        \end{choice}
    \end{choices}
\end{quickcheck}


\lang{de}{
Ein weiterer Typ von uneigentlichen Integralen hat zwar endliche Integralgrenzen, aber 
dafür ist die Funktion an einer der Grenzen nicht beschränkt oder sogar nicht definiert. 
In diesem Fall nähern wir uns der problematischen Integralgrenze von oben bzw. unten an. 
}
\lang{en}{
Another kind of improper integrals are those for which
the bounds of integration are finite, but where the function is unbounded
or even undefined at one or both of the bounds. In this case,
we approach the problematic bound from above (or below).
}

\begin{definition}
\lang{de}{
Es sei $f$  eine Funktion, die an $a \in \R$ nicht definiert ist oder deren Funktionswerte für $x \searrow a$ unbeschränkt wachsen. 
Weiter sei $b > a$.
}
\lang{en}{
Let $f$ be a function that is either undefined in $a \in \R$
or that is unbounded as $x \searrow a$. Let $b > a$.
}
\lang{de}{
Das \notion{uneigentliche Integral} der Funktion $f$ \"uber dem Intervall $( a;b]$ ist definiert als Grenzwert
\[ \int_a^b f\left(x\right)\; dx =
 \lim_{c \searrow a} \int_c^b f\left(x\right) \; dx, \]
  sofern der Grenzwert und die Integrale auf der rechten Seite existieren. (Andernfalls sagen wir:
 Das uneigentliche Integral $\int_a^b f\left(x\right) \; dx$ existiert nicht.)
}
\lang{en}{
The \notion{improper integral} of $f$ over the interval $(a,b]$ is defined
as the limit
\[ \int_a^b f\left(x\right)\; dx =
 \lim_{c \searrow a} \int_c^b f\left(x\right) \; dx, \]
  if the integrals on the right-hand side and their limit exist.
  (Otherwise, we say that the improper integral $\int_a^b f\left(x\right) \; dx$ does not exist.)
}
\lang{de}{
Ist $f$ an $b$ nicht definiert oder unbeschränkt für $x \nearrow b$, so definieren wir entsprechend  
  \[ \int_a^b f(x)\, dx :=\lim_{d\nearrow b}  \int_{a}^d f(x)\, dx, \]
  sofern Grenzwert und Integrale auf der rechten Seite existieren.
}
\lang{en}{
Similarly, if $f$ is undefined at $b$ or unbounded as $x\nearrow b$,
then we define
\[ \int_a^b f(x)\, dx :=\lim_{d\nearrow b}  \int_{a}^d f(x)\, dx, \]
  if the integrals and limit on the right-hand side exist.
}
\lang{de}{
Ist $f$ sowohl an $a$ als auch $b$ nicht definiert (oder unbeschränkt), definieren wir 
\[  \int_{a}^b f(x)\, dx= \int_{a}^c f(x)\, dx +   \int_{c}^b f(x)\, dx, \]
mit einer beliebigen Wahl von $c\in (a;b)$.
}
\lang{en}{
If $f$ is neither defined (or bounded) at $a$ nor at $b$,
then we define
\[  \int_{a}^b f(x)\, dx= \int_{a}^c f(x)\, dx +   \int_{c}^b f(x)\, dx, \]
for an arbitrary choice of $c\in (a,b)$.
}
\end{definition}

\begin{example}\label{ex2}
\begin{tabs*}
\tab{$\int_0^1 \frac{1}{\sqrt{x}}\, dx$}
\lang{de}{
Für die Funktion $f(x)=\frac{1}{\sqrt{x}}$ existiert das uneigentliche Integral
$ \int_0^1 \frac{1}{\sqrt{x}}\, dx$ und es gilt
\[  \int_0^1 \frac{1}{\sqrt{x}}\, dx =2 .\]
}
\lang{en}{
For the function $f(x)=\frac{1}{\sqrt{x}}$, the improper
integral $ \int_0^1 \frac{1}{\sqrt{x}}\, dx$ exists and is given by
\[  \int_0^1 \frac{1}{\sqrt{x}}\, dx =2 .\]
}

\lang{de}{
Eine Stammfunktion für $f(x)=\frac{1}{\sqrt{x}}=x^{-\frac{1}{2}}$ ist nämlich $F(x)=2x^{\frac{1}{2}}=2\sqrt{x}$. Damit folgt für $0<c<1$
\begin{eqnarray*}
 \int_c^1 \frac{1}{\sqrt{x}}\, dx &=& \left[ 2\sqrt{x}\right]_c^1 \\
 &=&  2\sqrt{1}-2\sqrt{c}=2-2\sqrt{c}. 
\end{eqnarray*}
Also ist
\[  \int_0^1 \frac{1}{\sqrt{x}}\, dx=\lim_{c\searrow 0} \int_c^1 \frac{1}{\sqrt{x}}\, dx
=\lim_{c\searrow 0} 2-2\sqrt{c} =2.\]
}
\lang{en}{
Namely, as an antiderivative of
$f(x)=\frac{1}{\sqrt{x}}=x^{-\frac{1}{2}}$, we can take
$F(x)=2x^{\frac{1}{2}}=2\sqrt{x}$.
Therefore, for $0 < c < 1$,
\begin{eqnarray*}
 \int_c^1 \frac{1}{\sqrt{x}}\, dx &=& \left[ 2\sqrt{x}\right]_c^1 \\
 &=&  2\sqrt{1}-2\sqrt{c}=2-2\sqrt{c}. 
\end{eqnarray*}
So \[  \int_0^1 \frac{1}{\sqrt{x}}\, dx=\lim_{c\searrow 0} \int_c^1 \frac{1}{\sqrt{x}}\, dx
=\lim_{c\searrow 0} 2-2\sqrt{c} =2.\]
}

\tab{$\int_0^1 \frac{1}{x}\, dx$}
\lang{de}{
Für die Funktion $f(x)=\frac{1}{x}$ existiert das uneigentliche Integral
$ \int_0^1 \frac{1}{x}\, dx$ nicht, d.\,h. 
\[
  \lim\limits_{c\searrow 0}\;\int_c^1 \frac{1}{x}\, dx \quad \text{\lang{de}{existiert nicht}\lang{en}{does not exist}}.
 \]
 Da die Funktion $f(x)=\frac{1}{x}$ für positive reelle Zahlen nämlich genau die Ableitung des Logarithmus ist, ist $\ln(x)$ eine Stammfunktion für $f$. Damit gilt:
 \begin{eqnarray*}
  \lim\limits_{c\searrow 0}\;\int_c^1 \frac{1}{x}\, dx
  &=&\lim\limits_{c\searrow 0} \;\left[ \ln(x)\right]_c^1\\
  &=&\lim\limits_{c\searrow 0} \; 0-\ln(c) =+\infty \, .
  \end{eqnarray*}
}
\lang{en}{
For the function $f(x)=\frac{1}{x}$, the improper integral
$ \int_0^1 \frac{1}{x}\, dx$ does not exist; that is,
\[
  \lim\limits_{c\searrow 0}\;\int_c^1 \frac{1}{x}\, dx \quad \text{\lang{de}{existiert nicht}\lang{en}{does not exist}}.
 \]
 Since $f(x)=\frac{1}{x}$ is the derivative of the logarithm for
 positive real numbers, $\ln(x)$ is an antiderivative of $f$. So
 \begin{eqnarray*}
  \lim\limits_{c\searrow 0}\;\int_c^1 \frac{1}{x}\, dx
  &=&\lim\limits_{c\searrow 0} \;\left[ \ln(x)\right]_c^1\\
  &=&\lim\limits_{c\searrow 0} \; 0-\ln(c) =+\infty \, .
  \end{eqnarray*}
}
  \lang{de}{Die bestimmten Integrale divergieren also gegen $+\infty$ im Grenzwert $c\searrow 0$, weshalb das uneigentliche Integral
  $ \int_0^1 \frac{1}{x}\, dx$ nicht existiert.
  }
  \lang{en}{
  The definite integrals diverge to $+\infty$ in the limit $c\searrow 0$,
  such that the improper integral $ \int_0^1 \frac{1}{x}\, dx$ does not exist.
  }
  
 \tab{$\int_0^1 \frac{1}{x^2}\, dx$}
\lang{de}{
Wir betrachten die Funktion $f:(0;\infty)\to \R$ mit $f(x)=\frac{1}{x^2}=x^{-2}$. Eine Stammfunktion für $f$ ist die Funktion
\[ F(x)=-{x^{-1}}, \]
wie man durch Ableiten von $F(x)$ sieht. Für $0<c<1$ ist damit
\begin{eqnarray*}
\int_c^1 \;\frac{1}{x^2}\, dx &=&  \left[ -x^{-1}\right]_c^1\\
&=& -1 -   (-{c^{-1}}). 
\end{eqnarray*} 
}
\lang{en}{
Consider the function $f:(0,\infty)\to \R$, $f(x)=\frac{1}{x^2}=x^{-2}$.
We can take the function \[ F(x)=-{x^{-1}}\]
as an antiderivative of $f$, as one can see by differentiating $F$.
Hence, for $0 < c < 1$,
\begin{eqnarray*}
\int_c^1 \;\frac{1}{x^2}\, dx &=&  \left[ -x^{-1}\right]_c^1\\
&=& -1 -   (-{c^{-1}}). 
\end{eqnarray*} 
}
\lang{de}{
Es folgt
\begin{eqnarray*}
\int_0^1 \;\frac{1}{x^2}\, dx &=& 
\lim_{c\searrow 0} \int_c^1 \;\frac{1}{x^2}\, dx \\
&=& \lim_{c\searrow 0}  -1 + \frac{1}{c} =+\infty.
\end{eqnarray*} 
In diesem Fall existiert das uneigentliche  Integral $\int_0^1 \;\frac{1}{x^2}\, dx$
also nicht.
}
\lang{en}{
It follows that
\begin{eqnarray*}
\int_0^1 \;\frac{1}{x^2}\, dx &=& 
\lim_{c\searrow 0} \int_c^1 \;\frac{1}{x^2}\, dx \\
&=& \lim_{c\searrow 0}  -1 + \frac{1}{c} =+\infty.
\end{eqnarray*} 
Therefore, the improper integral $\int_0^1 \;\frac{1}{x^2}\, dx$ does not
exist in this case.
}
\end{tabs*}
\end{example}

\lang{de}{
Bei uneigentlichen Integralen mit festen Zahlen als Grenzen, an denen die Funktion nicht
definiert ist, könnte man auf die Idee kommen, einfach einen Funktionswert an den Grenzen
vorzugeben. Die folgende Bemerkung zeigt, dass dies möglich ist, aber das Problem der 
Berechnung des Integrals nicht verändert.
}
\lang{en}{
Suppose we want to compute an improper integral
with finite bounds of integration at which the function
is undefined. We might come up with the idea of simply
defining arbitrary values of the function at the bounds.
The following remark shows that that is possible, but 
does not lessen the problem of computing the integral.
}

\begin{remark}
\lang{de}{
Ist die Funktion $f:(a;b]\to \R$ beschränkt, so kann man einen beliebigen Wert $w\in \R$ wählen und eine Funktion $g:[a;b]\to \R$ definieren mittels
\[  g(x)=\left\{ \begin{mtable} w, \quad & x=a, \\ f(x), & x \neq a. \end{mtable} \right. \] 
Das uneigentliche Integral $\int_a^b f(x)\, dx$ existiert genau dann, wenn das (eigentliche) Integral $\int_a^b g(x)\, dx$ existiert, und im Falle der Existenz sind sie auch beide gleich.
}
\lang{en}{
If the fuunction $f:(a,b]\to \R$ is bounded, then we can choose an
arbitrary value $w\in \R$ and define a function $g:[a,b]\to\R$ by setting
\[  g(x)=\left\{ \begin{mtable} w, \quad & x=a, \\ f(x), & x \neq a. \end{mtable} \right. \] 
The improper integral $\int_a^b f(x)\, dx$ exists if and only if
the (proper) integral $\int_a^b g(x)\, dx$ existiert, and in that case
they are equal.
}
\lang{de}{
Das bedeutet, dass der Wert des Integrals nicht von einem einzelnen Funktionswert abhängt.
}
\lang{en}{
This means that the value of the integral does not depend on the value
of the function at any single point.
}
\end{remark}


\section{
\lang{de}{Anwendung der Integralrechnung}
\lang{en}{Applications of integrals}
}

\lang{de}{
Wenn die momentane Änderungsrate einer Funktion 
(d.\,h. ihre Ableitung) gegeben ist, so liefert die Integration der Ableitung die ursprünglichen Funktionswerte, 
oder genauer gesagt: Das Integral von $f'(x)$ in den Grenzen von $a$ bis $b$ 
ergibt die gesamte Änderung $f(b)-f(a)$ der Funktionswerte.
}
\lang{en}{
If the instantaneous rate of change (i.e. the derivative)
of a function is given, then integrating the derivative
recovers the original function.
More precisely: the integral of $f'(x)$ between the bounds $a$ and $b$
yields the overall change $f(b)-f(a)$ of the function.
}


\begin{rule}
\lang{de}{Wenn die momentane Änderungsrate $f'(x)$ einer Funktion $f(x)$ gegeben ist, so bietet die Integration
von $f'(x)$ eine 
Möglichkeit der Bestimmung oder Rekonstruktion der Funktion. Sofern ein Anfangswert $c = f(a)$ gegeben ist, lässt sich 
jeder Wert $f(b)$ ausrechnen:
\[ f(b) = c +  \int_a^b f'(x)\, dx . \]
}
\lang{en}{
Given the instantaneous rate of change $f'(x)$ of a function,
the function itself can be determined or reconstructed
by integrating $f(x)$. Once an initial value $c = f(a)$ is given,
any value $f(b)$ can be calculated with
\[ f(b) = c +  \int_a^b f'(x)\, dx . \]
}
\end{rule}

\lang{de}{
Dabei ist das Integral eigentlich nichts anderes als ein stetiges Aufsummieren von 
Werten, ähnlich wie bei einer Reihe. 
}
\lang{en}{
Here, the integral is essentially a continuous sum of function values,
similar to a series.
}

\begin{example}[\lang{de}{Anwendungen}
\lang{en}{Applications}]

%Beispiele aus den Naturwissenschaften
\begin{tabs*}[\initialtab{0}] %\class{exercise}
 % \tab{Arbeit, Kraft}
 % Das Integral über die Kraft $F$ längs eines Weges ergibt die verrichtete Arbeit $W$
 % \[  W = \int_{s_0}^{s_1} F(s)\, ds \]


\tab{\lang{de}{Volumen}\lang{en}{Volume}}
	\lang{de}{Die Zulaufrate eines Wasserbehälters sei $f(t)$ (Volumen pro Zeiteinheit, d.\,h. die Ableitung des Volumens nach der Zeit). 
	Das Integral über $f(t)$ ergibt das Volumen $V$ (die Wassermenge im Behälter) zum Zeitpunkt $T$, sofern der Behälter bei
	$t=0$ leer war (Anfangswert gleich $0$):}
 \lang{en}{
  Suppose a container of water is filled at the rate $f(t)$ (Unit volume per unit of time; i.e. the derivative
  of volume with respect to time).
  Integrating $f(t)$ yields the volume $V$ (the amount of water in the container) at time $T$,
  assuming that the container is empty at time $t=0$ (initial value $0$):
 }
	\[ V = \int_0^T f(t)\, dt . \]
%begin-cosh
\tab{\lang{de}{Zahlungsstrom}
\lang{en}{Cash flow}}
	\lang{de}{
    In manchen Modellen wird angenommen, dass ein Zahlungsstrom stetig bzw. kontinuierlich fließt. 
    Dies ist etwa sinnvoll, wenn sich der Zahlungsstrom zu sehr vielen Zeitpunkten ändert.
    
    Es sei $f(t) = t^2 - 4t + 1000$ eine Funktion, die den Zahlungsstrom eines Unternehmens an einem
    Werktag beschreibt mit $t \in [9; 19]$. Der Gesamtumsatz zwischen 14 Uhr und 18 Uhr kann dann als 
    Integral berechnet werden:
    \[
    U = \int_{14}^{18} t^2 - 4t + 1000 \, dt = \left[ \frac{1}{3}t^3 - 2t^2 + 1000t \right]_{14}^{18} =  4773,33.
    \]
	}
 \lang{en}{
  Many models assume that a cash flow is continuous.
  For example, this is useful if the cash flow changes at many points in time.

  Suppose the function $f(t) = t^2 - 4t + 1000$, $t \in [9, 19]$ describes the cash flow of a company
  on a workday. The total revenue between 2:00 PM ($t = 14$) and 6:00 PM ($t = 18$)
  can be computed as an integral:
  \[
    U = \int_{14}^{18} t^2 - 4t + 1000 \, dt = \left[ \frac{1}{3}t^3 - 2t^2 + 1000t \right]_{14}^{18} =  4,773.33.
    \]
 }
%ende-cosh
\end{tabs*}
\end{example}

\lang{de}{
Eine wichtige geometrische Anwendung besteht darin, die Fläche zwischen zwei Funktionsgraphen 
rechnerisch zu ermitteln. 
}
\lang{en}{
An important geometric application is computing the area
between the graphs of two functions.
}
  
  \begin{rule}[\lang{de}{Fläche zwischen zwei Graphen} \lang{en}{Area between Two Graphs}]
  %\emph{Satz:} 
  \lang{de}{Die Fläche zwischen den Graphen von zwei stetigen Funktionen $f(x)$
   und $g(x)$ im Intervall $[a;b]$ beträgt
   \[ \int_a^b \,{\big|\, f(x) - g(x) \, \big|} \, dx \]
  und kann auf die folgende Weise berechnet werden: }
  \lang{en}{The area contained between the graphs of two continuous functions
  $f(x)$ and $g(x)$ in the interval $[a, b]$ is
  \[ \int_a^b \,{\big|\, f(x) - g(x) \, \big|} \, dx \]
  and it can be computed as follows.}
  
  \begin{enumerate}
  \item \lang{de}{Bestimme die Nullstellen von $f(x)-g(x)$ im Intervall $[a;b]$. Dies sind die Schnittstellen.}
  \lang{en}{Determine the zeros of $f(x)-g(x)$ in the interval $[a,b]$. These are the points of intersection.}
  \item \lang{de}{Integriere $f(x)-g(x)$ abschnittsweise zwischen dem linken Rand $a$, 
  den Nullstellen von $f(x)-g(x)$ im Intervall $(a;b)$ (sofern vorhanden)
   und dem rechten Rand $b$.}
   \lang{en}{
    Compute the separate integrals of $f(x)-g(x)$ over the intervals between
    the left endpoint $a$, the zeros of $f(x)-g(x)$ in the interval $(a, b)$ (if any exist),
    and the right endpoint $b$.
   }
   \item \lang{de}{Addiere die Beträge der einzelnen Integrale.}
   \lang{en}{Add the absolute values of the integrals together.}
   \end{enumerate}
\end{rule}

\lang{de}{Bei der beschriebenen abschnittsweisen  Integration darf der Betrag  \emph{nach} 
der Integration angewendet werden, so dass keine gesonderte Überlegung zum Vorzeichen
von $f(x)-g(x)$ erforderlich ist.}
\lang{en}{
When the integrals are computed separately as described above,
the absolute value may be taken \emph{after} integrating,
such that the sign of $f(x)-g(x)$ does not need to be considered.
}

 \begin{example}
 %\emph{ Beispiel:} 
 \lang{de}{Es seien $f(x)=x^3-x^2$ und $g(x)=2x$. 
Gesucht ist die Fläche zwischen den Graphen von
$f(x)$ und $g(x)$. Wir bestimmen zunächst die Nullstellen von 
\[ f(x)-g(x)=x^3-x^2-2x=x(x^2-x-2)=x(x+1)(x-2) . \]
Die Nullstellen (Schnittstellen von $f(x)$ und $g(x)$) sind daher $x=-1$, $x=0$ und $x=2$. 
Die Fläche zwischen den Graphen von  $f(x)$ und $g(x)$ 
im Intervall $[-1;2]$ ist dann die Summe der Beträge von zwei Integralen:
\[ \int_{-1}^2 {\big|\, x^3-x^2-2x\, \big|}\, dx = 
\big| \int_{-1}^0 (x^3-x^2-2x)\, dx \big| + \big| \int_0^2 (x^3-x^2-2x)\,
dx \big| \, . \]



Eine Stammfunktion von $f(x)-g(x)=x^3-x^2-2x$ ist $\frac{1}{4} x^4 - \frac{1}{3} x^3 - x^2$.
Die gesuchte Fläche beträgt dann}
\lang{en}{
Let $f(x)=x^3-x^2$ and $g(x) = 2x$. We want to find the area
enclosed by the graphs of $f(x)$ and $g(x)$. First, we find the zeros of
\[ f(x)-g(x)=x^3-x^2-2x=x(x^2-x-2)=x(x+1)(x-2) . \]
The zeros (where $f(x)$ and $g(x)$ intersect) are therefore $x=-1$, $x=0$ and $x=2$.
The area between the graphs of $f(x)$ and $g(x)$ in the interval $[-1, 2]$
is the sum of the absolute values of two integrals:
\[ \int_{-1}^2 {\big|\, x^3-x^2-2x\, \big|}\, dx = 
\big| \int_{-1}^0 (x^3-x^2-2x)\, dx \big| + \big| \int_0^2 (x^3-x^2-2x)\,
dx \big| \, . \]
For $f(x)-g(x)=x^3-x^2-2x$, we can take the antiderivative
$\frac{1}{4} x^4 - \frac{1}{3} x^3 - x^2$.
The area is then
}

\[ \big| \Big[\frac{1}{4} x^4 - \frac{1}{3} x^3 - x^2 \Big]_{-1}^0 \big| + 
\big| \Big[\frac{1}{4} x^4 - \frac{1}{3} x^3 - x^2 \Big]_{0}^2 \big| = 
\big| \frac{5}{12} \big| + \big| - \frac{8}{3} \big| = \frac{37}{12} .\]
\begin{center}
\image{T604_Area_D}
\end{center}
\end{example}

    \begin{quickcheck}
	\field{rational}
		\type{input.number}
		\begin{variables}
			%\randint[Z]{c}{-2}{2}
			\number{c}{1}   % einfacher zu rechnen mit c=1
			\randint{ns1}{-2}{0}
			\randint{ns2}{1}{2}
			\function[expand,normalize]{d}{c*(x-ns1)*(x-ns2)}
			\randint[Z]{a}{-2}{2}
			\randint[Z]{b}{-2}{2}
			\function{g}{a*x+b}
			\function[normalize,sort]{f}{d+g}
			\function[normalize]{dd}{(c/3)*x^3-(c/2)*(ns1+ns2)*x^2+c*ns1*ns2*x}
			\function[calculate]{F1}{(c/3)*ns1^3-(c/2)*(ns1+ns2)*ns1^2+c*ns1*ns2*ns1}
			\function[calculate]{F2}{(c/3)*ns2^3-(c/2)*(ns1+ns2)*ns2^2+c*ns1*ns2*ns2}
			% Stammfunktion ist: c/3*x^3-c/2*(ns1+ns2)*x^2+c*ns1*ns2*x
			\function[calculate]{F}{|F2-F1|} % int_ns1^ns2 |d(x)| 

		\end{variables}
		
		\lang{de}{\text{Bestimmen Sie den Inhalt der Fläche, die die Graphen der Funktion $f(x)=\var{f}$ und
		der Funktion $g(x)=\var{g}$ einschließen.\\
		Die Schnittstellen der Graphen liegen bei \ansref (kleinere Stelle) und \ansref (größere Stelle).\\
		Der Flächeninhalt ist dann \ansref.}}
    \lang{en}{
    \text{Determine the area of the region enclosed by the graphs of the functions
    $f(x)=\var{f}$ and $g(x)=\var{g}$. \\
    The graphs intersect in the points \ansref (smaller) and \ansref (greater). \\
    The area is \ansref.}
    }
        

		\begin{answer}
			\solution{ns1}
		\end{answer}
		\begin{answer}
			\solution{ns2}
		\end{answer}
		\begin{answer}
			\solution{F}
		\end{answer}

		\explanation{\lang{de}{Die Schnittstellen sind die Nullstellen der Differenzfunktion $(\var{f})-(\var{g})=\var{d}$.
		Diese sind $\var{ns1}$ und $\var{ns2}$ (Lösen der quadratischen Gleichung $\var{d}=0$).\\
		Der Flächeninhalt ergibt sich dann als 
		$\int_{\var{ns1}}^{\var{ns2}} |\var{d}| dx=| [ \var{dd} ]_{\var{ns1}}^{\var{ns2}}| =|\var{F2}-\var{F1}|=\var{F}$.
		} 
    \lang{en}{The points of intersection are the zeros of the difference function $(\var{f})-(\var{g})=\var{d}$.
    These are $\var{ns1}$ and $\var{ns2}$ (solutions of the quadratic equation $\var{d}=0$).\\
    The area is given by
    $\int_{\var{ns1}}^{\var{ns2}} |\var{d}| dx=| [ \var{dd} ]_{\var{ns1}}^{\var{ns2}}| =|\var{F2}-\var{F1}|=\var{F}$.
    }
        } 
	\end{quickcheck}

\end{visualizationwrapper}
\end{content}