%$Id:  $
\documentclass{mumie.article}
%$Id$
\begin{metainfo}
  \name{
    \lang{de}{Struktur der Lösungsmenge}
    \lang{en}{The solution set of a linear system}
  }
  \begin{description} 
 This work is licensed under the Creative Commons License Attribution 4.0 International (CC-BY 4.0)   
 https://creativecommons.org/licenses/by/4.0/legalcode 

    \lang{de}{Beschreibung}
    \lang{en}{Description}
  \end{description}
  \begin{components}
    \component{js_lib}{system/media/mathlets/GWTGenericVisualization.meta.xml}{mathlet1}
    \component{generic_image}{content/rwth/HM1/images/g_img_00_video_button_schwarz-blau.meta.xml}{00_video_button_schwarz-blau}
  \end{components}
  \begin{links}
  \link{generic_article}{content/rwth/HM1/T108_Vektorrechnung/g_art_content_30_basen_eigenschaften.meta.xml}{content_30_basen_eigenschaften}
  
    \link{generic_article}{content/rwth/HM1/T101neu_Elementare_Rechengrundlagen/g_art_content_05_loesen_gleichungen_und_lgs.meta.xml}{lgs}
    \link{generic_article}{content/rwth/HM1/T111neu_Matrizen/g_art_content_39b_matrizen.meta.xml}{matrizen}
    \link{generic_article}{content/rwth/HM1/T108_Vektorrechnung/g_art_content_29_linearkombination.meta.xml}{lin-komb}
    \link{generic_article}{content/rwth/HM1/T112neu_Lineare_Gleichungssysteme/g_art_content_41_gauss_verfahren.meta.xml}{gauss-verfahren}
  \end{links}
  \creategeneric
\end{metainfo}
\begin{content}
\begin{block}[annotation]
	Im Ticket-System: \href{https://team.mumie.net/issues/21349}{Ticket 21349}
\end{block}
\begin{block}[annotation]
Copy of \href{http://team.mumie.net/issues/9057}{Ticket 9057}: content/rwth/HM1/T111_



,_lineare_Gleichungssysteme/art_content_40_lineare_gleichungssysteme.src.tex
\end{block}

\usepackage{mumie.ombplus}
\ombchapter{12}
\ombarticle{1}
\usepackage{mumie.genericvisualization}

\begin{visualizationwrapper}

\title{\lang{de}{Struktur der Lösungsmenge linearer Gleichungssysteme}
       \lang{en}{The solution set of a linear system}}
 
\begin{block}[annotation]
  übungsinhalt
  
\end{block}
% \begin{block}[annotation]
%   Im Ticket-System: \href{http://team.mumie.net/issues/9057}{Ticket 9057}\\
% \end{block}

\begin{block}[info-box]
\tableofcontents
\end{block}
\lang{de}{
Am Anfang dieses Kurses haben wir \link{lgs}{Lineare Gleichungssysteme} mit zwei Unbekannten 
kennengelernt. Dabei lag der Fokus vor allem auf den verschiedenen Lösungsverfahren.
\\\\
In diesem Kapitel werden lineare Gleichungssysteme beliebiger Größe im Zusammenhang mit Matrizen
betrachtet. Anschließend wird auf die allgemeine Lösungsstruktur eingegangen.
%Diese Lösungsstruktur wird im nachfolgenden Abschnitt explizit angewandt.
\\\\
Mit der Einführung des Gauß-Verfahrens wird im nächsten Unterkapitel die Grundlage gelegt,
die Lösung eines linearen Gleichungssystems systematisch zu bestimmen.
\\\\
Als Letztes betrachten wir den Rang einer Matrix.
Dieser kann ebenfalls mit dem Gauß-Verfahren berechnet werden.\\
%und Auskunft darüber gibt, wieviele Lösungen das zugehörige LGS besitzt.
Auch im Alltag begegnen uns lineare Gleichungssysteme. Als Einstieg finden Sie dazu einige Beispiele 
in dem folgenden Video.\\
\floatright{\href{https://api.stream24.net/vod/getVideo.php?id=10962-2-10839&mode=iframe&speed=true}{\image[75]{00_video_button_schwarz-blau}}}\\\\
}
\lang{en}{
At the beginning of this course we introduced \link{lgs}{systems of linear equations} in two 
variables. In that chapter we focused on the methods by which these \emph{'linear systems'} can be 
solved.
\\\\
In this chapter we will consider systems of linear equations in arbitrarily many variables, and 
their link with matrices as introduced in the previous chapter. This will allow us to consider the 
structure of the solution set of the system.
\\\\
In the next section, we will use an algorithm called \emph{Gaussian elimination} to systematically 
calculate the solution of a linear system.
\\\\
Finally we will introduce the concept of the \emph{rank} of a matrix. This can also be calculated 
using Gaussian elimination.
}
\begin{definition}[\lang{de}{LGS}\lang{en}{Linear system}]\label{def:lgs}
\lang{de}{
Ein System von linearen Gleichungen mit mehr als einer Variablen wird 
\notion{lineares Gleichungssystem (LGS)} genannt.
}
\lang{en}{
A system of linear equations with more than one variable is called a \notion{linear system}.
}
%\lang{en}{A system of linear equations with more than one variable is called a \notion{linear system of equations}.\\
%In the following we'll use the terms "variable" und "{unknown}" interchangeably.}
\end{definition}

\lang{de}{
Die Begriffe \textit{Variable} und \textit{Unbekannte} werden in diesem Kapitel synonym verwendet.
}
\lang{en}{
The terms \textit{variable} and \textit{unknown} are synonymous in this chapter.
}

\section{\lang{de}{Matrix-Darstellung linearer Gleichungssysteme}
         \lang{en}{Matrix representation of linear systems}}\label{sec:matrix}

\lang{de}{
Für die Bestimmung der Lösungsstruktur sowie für die systematische Bestimmung der Lösung legen wir 
in der folgenden Definition eine kanonische Form (Standardform) für lineare Gleichungssysteme fest.
}
\lang{en}{
With the aim of determining the structure of the solution set and being able to systematically 
calculate this, in the following definition we provide a canonical (standard) form for a linear 
system.
}

\begin{definition}[\lang{de}{Koeffizientenmatrix}
                   \lang{en}{Coefficient matrix}]\label{def:koeffizientenmatrix}

\lang{de}{
Ein lineares Gleichungssystem (kurz LGS) mit $m$ Gleichungen und $n$ Unbekannten $x_1,\ldots, x_n$ 
kann durch Umformungen der einzelnen Gleichungen stets in die folgende Form gebracht werden:
}
\lang{en}{
Any linear system with $m$ equations and in $n$ unknowns $x_1,\ldots, x_n$ can, by manipulating each 
individual equation, be written in the following form:
}
\[
\begin{mtable}[\cellaligns{ccccccccc}]
 a_{11} x_1 & + & a_{12} x_2 & + & \cdots & + & a_{1n} x_n & = & b_1 \\
 a_{21} x_1 & + & a_{22} x_2 & + & \cdots & + & a_{2n} x_n & = & b_2 \\
 \vdots     &   & \vdots     &   &        &   & \vdots     &   & \vdots \\
 a_{m1} x_1 & + & a_{m2} x_2 & + & \cdots & + & a_{mn} x_n & = & b_m
\end{mtable}
\]
\lang{de}{
Die Einträge $a_{ij}$ und $b_i$ mit $1 \leq i \leq m$ und $1 \leq j \leq n$ sind reellwertig.\\
%gebracht werden, mit reellen Zahlen $a_{ij}$ für $1 \leq i \leq m$ und $1 \leq j \leq n$, sowie reellen Zahlen $b_1, \ldots, b_m$.
%Sei
%\[ 
%\begin{mtable}[\cellaligns{ccccccccc}]
% a_{11} x_1 & + & a_{12} x_2 & + & \cdots & + & a_{1n} x_n & = & b_1 \\
% a_{21} x_1 & + & a_{22} x_2 & + & \cdots & + & a_{2n} x_n & = & b_2 \\
% \vdots     &   & \vdots     &   &        &   & \vdots     &   & \vdots \\
% a_{m1} x_1 & + & a_{m2} x_2 & + & \cdots & + & a_{mn} x_n & = & b_m
%\end{mtable}
%\]
%ein lineares Gleichungssystem. 
Wir definieren nun die \notion{Koeffizientenmatrix} $A$ und die 
\notion{erweiterte Koeffizientenmatrix} $(A \ \mid \ b)$ wie folgt:
%werden die zum linearen Gleichungssystem geh"origen 
}
\lang{en}{
The entries $a_{ij}$ and $b_i$ with $1 \leq i \leq m$ and $1 \leq j \leq n$ are real-valued.\\
We now define the \notion{coefficient matrix} $A$ and the \notion{augmented matrix} 
$(A \ \mid \ b)$ as follows:
}
\begin{equation*}
A=\underbrace{\left( \begin{smallmatrix}
a_{11} & a_{12} & \cdots & a_{1n} \\
a_{21} & a_{22} & \cdots & a_{2n} \\
\vdots & \vdots & \ddots & \vdots \\
a_{m1} & a_{m2} & \cdots & a_{mn}
\end{smallmatrix} \right)}_{ \in M(m,n;\R)}
\end{equation*}
\begin{equation*}
%\quad \text{ und } \quad 
(A ~\mid~ b) = \underbrace{\left(  \begin{smallmatrix}
                                      a_{11} & a_{12} & \cdots & a_{1n} & | & b_1\\
                                      a_{21} & a_{22} & \cdots & a_{2n} & | &b_2\\
                                      \vdots & \vdots & \ddots & \vdots & \mid &\vdots\\
                                      a_{m1} & a_{m2} & \cdots & a_{mn} &| & b_m
                                      \end{smallmatrix} \right) }_{
\in M(m,n+1;\R)}
\end{equation*}
%\notion{Koeffizientenmatrix} und \notion{erweiterte Koeffizientenmatrix} genannt. 
\lang{de}{
Die Trennstriche bei der erweiterten Koeffizientenmatrix dienen nur der besseren Übersicht und 
dürfen auch weggelassen werden
}
\lang{en}{
The vertical dividing line in the augmented matrix is useful notation, but may also be omitted.
}\\
\end{definition}

\lang{de}{
Mit der \ref[matrizen][Matrix-Vektor-Multiplikation]{sec:produkt} kann das obige lineare 
Gleichungssystem dann kompakt geschrieben werden als
}
\lang{en}{
Using \ref[matrizen][matrix-vector multiplication]{sec:produkt} we can write the above linear 
system much more compactly as
}
\[ A\cdot x = b \qquad \text{mit}\quad b=\left(  \begin{smallmatrix}  b_1 \\    \vdots\\ b_m  \end{smallmatrix} \right) \quad \text{und}\quad x=\left(  \begin{smallmatrix}  x_1 \\    \vdots\\ x_n  \end{smallmatrix} \right). \]
\lang{de}{Diese Schreibweise werden wir im Folgenden stets benutzen.}
\lang{en}{This way of writing a linear system will be used henceforth.}

\begin{example}
\begin{tabs*}
\tab{\lang{de}{1. Beispiel}\lang{en}{Example 1}}
\lang{de}{
Im Abschnitt \ref[lgs][Lineare Gleichungssysteme]{ex:nicht-eindeutig-loesbar} haben wir als Beispiel 
das folgende lineare Gleichungssystem betrachtet:
}
\lang{en}{
We used the following linear system in an example from the initial section on 
\ref[lgs][linear systems]{ex:nicht-eindeutig-loesbar}:
}
\[ \begin{mtable}[\cellaligns{crcrcr}]
(I)&\qquad 2 \cdot  x & - & 4 \cdot  y & = & 10 \\
(II)&-3 \cdot  x & + & 6\cdot y & =  & -15
\end{mtable} \]
\lang{de}{
Es hat als Koeffizientenmatrix $A=\begin{pmatrix} 2 & -4 \\ -3 & 6\end{pmatrix}$ und als rechte 
Seite $b=\begin{pmatrix} 10\\ -15\end{pmatrix}$ sowie die erweiterte Koeffizientenmatrix
}
\lang{en}{
It has the coefficient matrix $A=\begin{pmatrix} 2 & -4 \\ -3 & 6\end{pmatrix}$ and the vector 
$b=\begin{pmatrix} 10\\ -15\end{pmatrix}$ on the right-hand side, hence the corresponding augmented 
matrix is
}
\[ (A \mid b )= \begin{pmatrix} 2 & -4& | & 10 \\ -3 & 6& | & -15\end{pmatrix}  .\]
\lang{de}{In Matrixschreibweise lautet das LGS daher}
\lang{en}{Expressed using the matrix, the linear system is}
\[   \begin{pmatrix} 2 & -4 \\ -3 & 6\end{pmatrix}\cdot \begin{pmatrix} x\\ y\end{pmatrix}= \begin{pmatrix} 10\\ -15\end{pmatrix}.\]
\tab{\lang{de}{2. Beispiel}\lang{en}{Example 2}}
\lang{de}{Das lineare Gleichungssystem}
\lang{en}{The linear system}
\[ \begin{mtable}[\cellaligns{ccrcrcrcr}]
(I)&&x_{1}&+&2x_{2}&+&x_{3}&=&4\\
(II)&&x_{1}&-&x_{2}&+&\frac{3}{2}x_{3}&=&-7\\
(III)&\qquad-&4x_{1}&+&2x_{2}&&&=&-2
\end{mtable} \]
\lang{de}{hat als Koeffizientenmatrix }
\lang{en}{has the coefficient matrix }
$ A=\begin{pmatrix} 1 & 2 & 1 \\ 1 & -1 & \frac{3}{2} \\ -4 & 2 & 0\end{pmatrix} $
\lang{de}{
und als rechte Seite $b=\begin{pmatrix} 4 \\ -7 \\ -2 \end{pmatrix} $ sowie die erweiterte 
Koeffizientenmatrix
}
\lang{en}{
and the vector $b=\begin{pmatrix} 4 \\ -7 \\ -2 \end{pmatrix} $ on the right-hand side, hence the 
corresponding augmented matrix is
}
\[ (A \mid b )= \begin{pmatrix} 1 & 2 & 1 & | & 4 \\ 1 & -1 & \frac{3}{2}& | & -7\\ -4 & 2 & 0& | &-2\end{pmatrix}.\]
\lang{de}{In Matrixschreibweise lautet das LGS daher}
\lang{en}{Expressed using the matrix, the linear system is}
\[ \begin{pmatrix} 1 & 2 & 1 \\ 1 & -1 & \frac{3}{2} \\ -4 & 2 & 0\end{pmatrix}\cdot \begin{pmatrix}x_1 \\x_2\\ x_3\end{pmatrix}
= \begin{pmatrix} 4 \\ -7 \\ -2 \end{pmatrix}. \]
\end{tabs*}
\end{example}
\lang{de}{
Um eine Struktur der Lösungsmenge anzugeben, unterscheiden wir zwischen homogenen und inhomogenen 
linearen Gleichungssystemen:
}
\lang{en}{
In order to determine the structure of the solution set, we must distinguish between homogeneous and 
inhomogeneous linear systems \emph{(note that this is \textbf{not} the same word as 
\textbf{homogenous})}:
}\\
\begin{definition}[\lang{de}{Homogenes/Inhomogenes LGS}
                   \lang{en}{Homogeneous/inhomogeneous linear system}]\label{def:homog_lgs}
\lang{de}{
Ein lineares Gleichungssystem $Ax=b$ heißt \notion{homogenes} LGS, wenn $b$ gleich dem Nullvektor 
ist, also $b_1= \cdots = b_m =0$ gilt, anderenfalls spricht man von einem \notion{inhomogenen} LGS.
\\\\
Ist $Ax=b$ ein inhomogenes LGS, so nennt man das lineare Gleichungssystem $Ax=0$ das 
\notion{zugehörige homogene LGS}.
}
\lang{en}{
A linear system $Ax=b$ is called a \notion{homogeneous} linear system if $b$ is the zero vector, so 
$b_1= \cdots = b_m =0$, otherwise it is called an \notion{inhomogeneous} linear system.
\\\\
If $Ax=b$ is an inhomogeneous linear system, then the linear system $Ax=0$ is called the 
\notion{associated homogeneous linear system}.
}
\end{definition}

\begin{example}
\begin{tabs*}
\tab{\lang{de}{1. Beispiel}\lang{en}{Example 1}}
\lang{de}{Das obige LGS}
\lang{en}{The above linear system}
\[\begin{pmatrix} 2 & -4 \\ -3 & 6\end{pmatrix}\cdot \begin{pmatrix} x\\ y\end{pmatrix} = 
  \begin{pmatrix} 10\\ -15\end{pmatrix}\]
\lang{de}{
ist inhomogen, da die rechte Seite $b=\begin{pmatrix} 10\\ -15\end{pmatrix}$ nicht der Nullvektor 
ist. Das zugehörige homogene LGS lautet
}
\lang{en}{
is inhomogeneous, as the right-hand side $b=\begin{pmatrix} 10\\ -15\end{pmatrix}$ is not equal to 
the zero vector. The associated homogeneous linear system is
}
\[\begin{pmatrix} 2 & -4 \\ -3 & 6\end{pmatrix}\cdot \begin{pmatrix} x\\ y\end{pmatrix} = 
  \begin{pmatrix} 0\\ 0\end{pmatrix}.\]
\tab{\lang{de}{2. Beispiel}\lang{en}{Example 2}}
\lang{de}{Das obige LGS}
\lang{en}{The above linear system}
\[\begin{pmatrix} 1 & 2 & 1 \\ 1 & -1 & \frac{3}{2} \\ -4 & 2 & 0\end{pmatrix}\cdot 
  \begin{pmatrix}x_1 \\x_2\\ x_3\end{pmatrix} = 
  \begin{pmatrix} 4 \\ -7 \\ -2 \end{pmatrix} \]
\lang{de}{
ist inhomogen, da die rechte Seite $b=\begin{pmatrix} 4 \\ -7 \\ -2 \end{pmatrix}$ nicht der 
Nullvektor ist. Das zugehörige homogene LGS lautet
}
\lang{en}{
is inhomogeneous, as the right-hand side $b=\begin{pmatrix} 4 \\ -7 \\ -2 \end{pmatrix}$ is not 
equal to the zero vector. The associated homogeneous linear system is
}
\[ \begin{pmatrix} 1 & 2 & 1 \\ 1 & -1 & \frac{3}{2} \\ -4 & 2 & 0\end{pmatrix}\cdot \begin{pmatrix}x_1 \\x_2\\ x_3\end{pmatrix}
= \begin{pmatrix} 0 \\ 0 \\ 0 \end{pmatrix}. \]
\end{tabs*}
\end{example}


\begin{quickcheck}
		\type{input.interval}
        \field{rational}
        \precision{3}
      \field{real}
      \begin{variables}
           \randint[Z]{a}{2}{6}
           \randint[Z]{b}{2}{8}
           \randint[Z]{c}{2}{8}
           \randint[Z]{d}{2}{8}
           \end{variables}
      \text{\lang{de}{Das folgende lineare Gleichungssystem}
            \lang{en}{The linear system}
            \[\begin{mtable}[\cellaligns{crcrcr}]
              (I)&\qquad \var{a} \cdot  x & - & \var{b}  & = & -\var{b} \\
              (II)&   &  & \var{c}\cdot y & =  &-\var{d} \cdot  x 
              \end{mtable} \]
\lang{de}{ist}
\lang{en}{is}
        } 
    \begin{choices}{unique}

        \begin{choice}
            \text{\lang{de}{homogen.}\lang{en}{homogeneous.}}
			\solution{true}
		\end{choice}
                    
        \begin{choice}
            \text{\lang{de}{inhomogen.}\lang{en}{inhomogeneous.}}
			\solution{false}
		\end{choice}
    \end{choices}{unique}
 \explanation{\lang{de}{
 Das lineare Gleichungssystem ist homogen, da es sich in der folgenden Form darstellen lässt:
 }
 \lang{en}{
 The linear system is homogeneous, as it can be written in the following form:
 }\\
 $ \begin{mtable}[\cellaligns{crcrcr}]
(I)&\qquad \var{a} \cdot  x &  &   & = & 0 \\
(II)&  \var{d} \cdot  x & + & \var{c}\cdot y & =&0
\end{mtable} $
 }
	\end{quickcheck}

\lang{de}{Im Folgenden betrachten wir die Struktur der Lösungsmenge genauer.}
\lang{en}{We now examine the structure of the solution set in more detail.}\\
\begin{definition}[\lang{de}{Lösungsmenge}\lang{en}{Solution set}]\label{def:loesmenge_lgs}
\lang{de}{
Eine \notion{L\"osung} eines linearen Gleichungssystems mit $n$ Unbekannten ist ein $n$-Tupel 
$(x_1; x_2; \ldots; x_n)$, wobei $x_1,..., x_n$ alle Gleichungen des Systems erf\"ullen müssen.\\
In Matrixschreibweise ist der {Spaltenvektor} 
$x=\left(\begin{smallmatrix} x_1\\ \vdots \\ x_n\end{smallmatrix}\right)$ damit eine Lösung von 
$A\cdot x=b$.
\\\\
Die \notion{Lösungsmenge} eines reellen LGS enthält alle Lösungen $x\in \R^n$, also alle Vektoren 
$x$, die $A \cdot x = b$ erfüllen:
}
\lang{en}{
A \notion{solution} of a linear system in $n$ variables is an $n$-tuple $(x_1; x_2; \ldots; x_n)$, 
where $x_1,..., x_n$ satisfy all equations in the linear system.\\
If the linear system is written in matrix form, the (column vector) 
$x=\left(\begin{smallmatrix} x_1\\ \vdots \\ x_n\end{smallmatrix}\right)$ is a solution of 
$A\cdot x=b$.
\\\\
The \notion{solution set} of a linear system in $\R$ contains all solutions $x\in \R^n$, that is, 
all vectors $x$ that satisfy $A \cdot x = b$:
}
\[\mathbb{L}=\{\left(\begin{smallmatrix} x_1\\ \vdots \\ x_n\end{smallmatrix}\right)\in \R^n ~|~ \left(\begin{smallmatrix} x_1\\ \vdots \\ x_n\end{smallmatrix}\right) 
\text{\lang{de}{ ist L"osung des LGS}\lang{en}{ is a solution of the linear system}}\}\]
\end{definition}

\lang{de}{
Im folgenden Video wird ein lineares Gleichungssystem in Matrixschreibweise dargestellt und anschließend mit den bereits \link{lgs}{bekannten Methoden} gelöst. \\
\floatright{\href{https://api.stream24.net/vod/getVideo.php?id=10962-2-10840&mode=iframe&speed=true}{\image[75]{00_video_button_schwarz-blau}}}\\\\
}
\lang{en}{}

\section{\lang{de}{Die L"osungsmenge homogener linearer Gleichungssysteme}
         \lang{en}{The solution set of a homogeneous linear system}}\label{sec:homogen}

\lang{de}{
Die Lösungsmenge eines homogenen linearen Gleichungssystems $Ax=0$ kann als Kombination ihrer 
\link{content_30_basen_eigenschaften}{linear unabhängigen Lösungsvektoren} $x$ dargestellt werden. 
Hierfür gelten folgende Regeln.
}
\lang{en}{
The solution set $\mathbb{L}$ of a homogeneous linear system $Ax=0$ can be written as a linear 
combination of \link{content_30_basen_eigenschaften}{linearly independent solution vectors} $x$.
}

\begin{rule}
\lang{de}{Für  ein homogenes LGS $Ax=0$ mit $m$ Gleichungen und $n$ Variablen gilt:}
\lang{en}{Given a homogeneous linear system $Ax=0$ with $m$ equations and in $n$ variables:}
\begin{enumerate}
\item \lang{de}{
      Es besitzt stets die Lösung $x_1=\ldots=x_n=0$. Diese wird die \notion{triviale Lösung} 
      genannt.
      }
      \lang{en}{
      The vector $x_1=\ldots=x_n=0$ is always a solution of the system, called the 
      \notion{trivial solution}.
      }
\item \lang{de}{
      Entweder ist $\mathbb{L}=\{ 0\}$ (d.\,h. es existiert nur die triviale Lösung) oder 
      es gibt %eine natürliche Zahl
      $k\leq n$ linear unabhängige Lösungen% $v_i \in \R^n$ ($1 \leq i < k$)
      }
      \lang{en}{
      Either $\mathbb{L}=\{ 0\}$ (so we only have the trivial solution) or there exist $k\leq n$ 
      linearly independent solutions
      }
      \[v_1=\begin{pmatrix}v_{11}\\ \vdots \\ v_{1n} \end{pmatrix},
        \quad v_2=\begin{pmatrix}v_{21}\\ \vdots \\ v_{2n} \end{pmatrix},
        \quad \ldots \quad v_k=\begin{pmatrix}v_{k1}\\ \vdots \\ v_{kn}\end{pmatrix}, \]
      \lang{de}{so dass}
      \lang{en}{such that}
      \[\mathbb{L}= \left\{ r_1v_1+r_2v_2+\cdots + r_kv_k \mid r_1,r_2,\ldots, r_k\in \R \right\}.\]
%Die Lösungsmenge bildet damit einen \ref[lin-komb][Untervektorraum]{supp:allg-vektorraum} des $\mathbb{R}^n$.
\item \lang{de}{
      Jedes homogene LGS mit mehr Variablen als Gleichungen hat mindestens eine nichttriviale 
      L"osung.
      }
      \lang{en}{
      Every homogeneous linear system with more variables than equations has at least one 
      non-trivial solution.
      }
\end{enumerate}
\lang{de}{
\floatright{\href{https://api.stream24.net/vod/getVideo.php?id=10962-2-11012&mode=iframe&speed=true}{\image[75]{00_video_button_schwarz-blau}}}\\
}
\lang{en}{}
\end{rule}

\begin{proof*}[\lang{de}{Beweis Regel}\lang{en}{Proof of rules}]
\begin{showhide}
\begin{enumerate}
\item \lang{de}{
      Bei einem homogenen LGS entspricht $b$ dem Nullvektor. Wählt man nun für $x$ ebenfalls den 
      Nullvektor, so ist die Gleichung
      }
      \lang{en}{
      In a homogeneous linear system, $b$ is the zero vector. If we set $x$ to also be the zero 
      vector, the equation
      }
      \[Ax=A\cdot \begin{pmatrix} 0 \\ 0 \\ \vdots \\ 0 \end{pmatrix} = 
        \begin{pmatrix} 0 \\ 0 \\ \vdots \\ 0 \end{pmatrix}=b\]
      \lang{de}{stets erfüllt.}
      \lang{en}{is clearly satisfied.}
\item \lang{de}{
      Sind $v_1,\ldots, v_k$ $(k \leq n)$ Lösungen des LGS $Ax=0$, dann sieht man aufgrund der 
      \ref[matrizen][Rechenregeln für die Matrix-Vektor-Multiplikation]{sec:rechenregeln}, dass für 
      beliebige reelle Zahlen $r_1,\ldots, r_k$ die Gleichung
      }
      \lang{en}{
      If $v_1,\ldots, v_k$ $(k \leq n)$ are solutions of the linear system $Ax=0$, then by the 
      \ref[matrizen][distributivity of matrix-vector multiplication]{sec:rechenregeln}, we also have
      }
      \[ A(r_1v_1+\ldots +r_k v_k)=r_1\cdot Av_1+\ldots + r_k\cdot Av_k=0+\ldots +0=0 \]
      \lang{de}{
      erfüllt ist. Damit ist jede Linearkombination der $v_1,\ldots, v_k$ auch eine Lösung.
      \\\\
      Um alle Lösungen darzustellen, müssen genügend linear unabhängige Vektoren gefunden werden, so 
      dass alle Lösungen als Linearkombination dieser Vektoren geschrieben werden können. Praktisch 
      wird dies mit dem \link{gauss-verfahren}{Gauß-Verfahren} durchgeführt.
      }
      \lang{en}{
      for any real numbers $r_1,\ldots, r_k$. Hence any linear combination of the solutions 
      $v_1,\ldots, v_k$ is also a solution.
      \\\\
      To be able to obtain all solutions as such linear combinations, we must first find 
      sufficiently many linearly independent solutions. This is achieved efficiently by 
      \link{gauss-verfahren}{Gaussian elimination.}
      }
      
\item \lang{de}{
      Hat ein homogenes LGS $m$ Gleichungen und $n$ Variablen mit $m<n$, so lässt sich das LGS 
      höchstens nach $m$ Variablen auflösen. Die übrigen Variablen lassen sich frei wählen. Damit 
      besitzt ein solches LGS mindestens eine nichttriviale Lösung. \\
      Im nächsten Kapitel \link{gauss-verfahren}{Gauß-Verfahren} werden wir diese Variablen genauer 
      kennenlernen.
      }
      \lang{en}{
      A homogeneous linear system with $m$ equations and in $n$ variables where $m<n$ can be solved 
      for at most $m$ variables. The remaining variables may be freely chosen, and so such a linear 
      system has at least one non-trivial solution (obtained by setting one of the remaining 
      variables to be any non-zero real number).\\
      \link{gauss-verfahren}{In the next section} we will study these variables further.
      }

\end{enumerate}
\end{showhide}
\end{proof*}

\begin{example}\label{ex:lsgsmenge-homogen}
\begin{tabs*}
\tab{\lang{de}{1. Beispiel}\lang{en}{Example 1}}
\lang{de}{Das homogene LGS}
\lang{en}{The homogeneous linear system}
\[\begin{pmatrix} 2 & -4 \\ -3 & 6\end{pmatrix}\cdot \begin{pmatrix} x\\ y\end{pmatrix} = 
  \begin{pmatrix} 0\\ 0\end{pmatrix}\]
\lang{de}{hat als Lösungsmenge}
\lang{en}{has the solution set}
\[\mathbb{L}=\left\{ r\cdot \begin{pmatrix} 2 \\ 1 \end{pmatrix} \, \big| \, r\in \R \right\}.\]
\lang{de}{
Hier ist also $k=1$ und $v_1=\begin{pmatrix} 2 \\ 1 \end{pmatrix}$ eine Lösung, so dass die 
Lösungsmenge des LGS geschrieben werden kann als 
}
\lang{en}{
Here we have $k=1$ linearly independent solution $v_1=\begin{pmatrix} 2 \\ 1 \end{pmatrix}$, so the 
solution set of the linear system may be written as 
}
$\mathbb{L}= \left\{ r_1v_1 \, \big| \, r_1\in \R \right\}$.
\tab{\lang{de}{2. Beispiel}\lang{en}{Example 2}}
\lang{de}{Für das homogene LGS}
\lang{en}{The homogeneous linear system}
\[ \begin{pmatrix} 1 & 2 & 1 \\ 1 & -1 & \frac{3}{2} \\ -4 & 2 & 0\end{pmatrix}\cdot \begin{pmatrix}x_1 \\x_2\\ x_3\end{pmatrix}
= \begin{pmatrix} 0 \\ 0 \\ 0 \end{pmatrix} \]
\lang{de}{
kann durch Rechnung (siehe Abschnitt \link{gauss-verfahren}{Gauß-Verfahren}) gezeigt werden, dass 
die Lösungsmenge nur aus der trivialen Lösung besteht, also $\mathbb{L}=\{ 0\}$.
%es nur die triviale Lösung als Lösung besitzt. 
%Damit ist die Lösungsmenge $\mathbb{L}=\{ 0\}$.
}
\lang{en}{
can be shown (using \link{gauss-verfahren}{Gaussian elimination}) to have only the trivial solution 
in its solution set, so $\mathbb{L}=\{ 0\}$.
}

\end{tabs*}
\end{example}


\section{\lang{de}{Die L"osungsmenge inhomogener linearer Gleichungssysteme}
         \lang{en}{The solution set of an inhomogeneous linear system}}\label{sec:inhomogen}

\lang{de}{
Die Lösungsmenge eines inhomogenen LGS kann mit Hilfe der Lösungsmenge des zugehörigen homogenen 
LGS dargestellt werden. Für die Beziehung zwischen diesen beiden Lösungsmengen gilt:
}
\lang{en}{
The solution set of an inhomogeneous linear system can be determined with the help of the associated 
homogeneous linear system. The following rules show how they are related:
}

\begin{rule}
\lang{de}{
Es sei $Ax=b$ ein inhomogenes LGS mit $m$ Gleichungen und $n$ Variablen und $\mathbb{L}_0$ die
Lösungsmenge des zugehörigen homogenen LGS $Ax=0$. 
\\\\
Dann gilt für die Lösungsmenge $\mathbb{L}$ des inhomogenen LGS:
}
\lang{en}{
Let $Ax=b$ be an inhomogeneous linear system with $m$ equations and in $n$ variables. Let 
$\mathbb{L}_0$ be the solution set of the associated homogeneous linear system $Ax=0$.
\\\\
If $\mathbb{L}$ is the solution set of the inhomogeneous linear system, then:
}
\begin{itemize}
    \item \lang{de}{Entweder ist die Lösungsmenge leer, d.\,h.~$\mathbb{L}=\emptyset$, oder es gilt}
          \lang{en}{Either the solution set is empty, i.e.~$\mathbb{L}=\emptyset$, or}
    \item $ \mathbb{L}= v+\mathbb{L}_0=\left\{ v+ w \mid w\in \mathbb{L}_0 \right\}$, 
          \lang{de}{wobei $v$ eine beliebige Lösung des inhomogenen LGS $Ax=b$ ist.}
          \lang{en}{where $v$ is an arbitrary solution of the inhomogeneous linear system $Ax=b$.}
\end{itemize}
\lang{de}{
\floatright{\href{https://api.stream24.net/vod/getVideo.php?id=10962-2-11013&mode=iframe&speed=true}{\image[75]{00_video_button_schwarz-blau}}}\\
}
\lang{en}{}
\end{rule}


\begin{proof*}[\lang{de}{Beweis Regel}\lang{en}{Proof of rule}]
\begin{showhide}
\lang{de}{
Diese Aussage erhält man wieder aus den 
\ref[matrizen][Rechenregeln für die Matrix-Vektor-Multiplikation]{sec:rechenregeln}.
Wenn nämlich die Lösungsmenge $\mathbb{L}$ nicht leer ist und $v\in \mathbb{L}$ eine beliebige 
Lösung ist, dann gilt $Av=b$.\\
Für jede Lösung $w\in \mathbb{L}_0$ des zugehörigen homogenen LGS $Ax=0$ gilt dann
}
\lang{en}{
This statement also follows from the 
\ref[matrizen][distributivity of matrix-vector multiplication]{sec:rechenregeln}. 
If the solution set $\mathbb{L}$ of the inhomogeneous linear system is not empty and 
$v\in \mathbb{L}$ is a solution, then by definition $Av=b$.\\
Thus for every solution $w\in \mathbb{L}_0$ of the associated homogeneous linear system $Ax=0$, we 
have
}
\[  A(v+w)=Av+Aw=b+0=b. \]
\lang{de}{
Also ist $v+w\in \mathbb{L}$.
\\\\
Ist andererseits $v'\in \mathbb{L}$ eine weitere Lösung des LGS $Ax=b$, so gilt
}
\lang{en}{
Hence $v+w\in \mathbb{L}$.
\\\\
Conversely, if $v'\in \mathbb{L}$ is a solution of $Ax=b$, we have
}
\[  A(v'-v)=Av'-Av=b-b=0. \]
\lang{de}{
Die Differenz $w=v'-v$ ist also eine Lösung des zugehörigen homogenen LGS $Ax=0$. Damit gilt 
$v'=v+(v'-v)=v+w$ und folglich lässt sich jede Lösung $v'$ des inhomogenen Systems als Summe einer 
festen inhomogenen Lösung $v$ und einer homogenen Lösung $w$ schreiben.
}
\lang{en}{
The difference $w=v'-v$ is a solution of the associated homogeneous linear system $Ax=0$. Hence 
$v'=v+(v'-v)=v+w$ can be expressed as a sum of the fixed inhomogeneous solution $v$ and a 
homogeneous solution $w$. That is to say, every inhomogeneous solution is obtained this way.
}
\end{showhide}
\end{proof*}


\begin{example}
\begin{tabs*}
\tab{\lang{de}{1. Beispiel}\lang{en}{Example 1}}
\lang{de}{Das inhomogene LGS}
\lang{en}{The inhomogeneous linear system}
\[ \begin{mtable}[\cellaligns{crcrcr}]
(I)&\qquad 2 \cdot  x & - & 4 \cdot  y & = & 10 \\
(II)&-3 \cdot  x & + & 6\cdot y & =  & -15
\end{mtable} \]
\lang{de}{bzw. in Matrixform}
\lang{en}{expressed in matrix form as}
\[   \begin{pmatrix} 2 & -4 \\ -3 & 6\end{pmatrix}\cdot \begin{pmatrix} x\\ y\end{pmatrix}= \begin{pmatrix} 10\\ -15\end{pmatrix}\]
\lang{de}{hat die Lösungsmenge}
\lang{en}{has the solution set}
\[ \mathbb{L}=\left\{ \begin{pmatrix}5\\ 0 \end{pmatrix}+r\cdot \begin{pmatrix} 2 \\ 1 \end{pmatrix} \, \big| \, r\in \R \right\} \]
\lang{de}{
(vgl. Abschnitt \ref[lgs][Lineare Gleichungssysteme]{ex:nicht-eindeutig-loesbar}).
Hierbei ist $v=\begin{pmatrix}5\\ 0 \end{pmatrix}$ eine spezielle Lösung und die Lösungsmenge des 
zugehörigen homogenen LGS lautet
}
\lang{en}{
(see \ref[lgs][the section where linear systems were first introduced]{ex:nicht-eindeutig-loesbar}). 
Here $v=\begin{pmatrix}5\\ 0 \end{pmatrix}$ is a fixed solution, and the solution set of the 
associated homogeneous linear system is
}
\[ \mathbb{L}_0=\left\{ r\cdot \begin{pmatrix} 2 \\ 1 \end{pmatrix} \, \big| \, r\in \R \right\} \]
\lang{de}{(siehe \lref{ex:lsgsmenge-homogen}{oben}).}
\lang{en}{(see the \lref{ex:lsgsmenge-homogen}{previous example}).}
\tab{\lang{de}{2. Beispiel}\lang{en}{Example 2}}
\lang{de}{Das LGS}
\lang{en}{The linear system}
\[\begin{pmatrix} 1 & 2 & 1 \\ 1 & -1 & \frac{3}{2} \\ -4 & 2 & 0\end{pmatrix}\cdot 
  \begin{pmatrix}x_1 \\x_2\\ x_3\end{pmatrix} = \begin{pmatrix} 4 \\ -7 \\ -2 \end{pmatrix} \]
\lang{de}{
besitzt die Lösung $\begin{pmatrix} 2 \\ 3 \\ -4 \end{pmatrix}$, wie man leicht nachrechnet.
Das zugehörige homogene LGS
}
\lang{en}{
has a solution $\begin{pmatrix} 2 \\ 3 \\ -4 \end{pmatrix}$, as is easily verified. The associated 
homogeneous linear system
}
\[ \begin{pmatrix} 1 & 2 & 1 \\ 1 & -1 & \frac{3}{2} \\ -4 & 2 & 0\end{pmatrix}\cdot 
  \begin{pmatrix}x_1 \\x_2\\ x_3\end{pmatrix} = \begin{pmatrix} 0 \\ 0 \\ 0 \end{pmatrix} \]
\lang{de}{
besitzt nur die triviale Lösung als Lösung (siehe \lref{ex:lsgsmenge-homogen}{oben}). Also ist die 
Lösungsmenge des inhomogenen LGS
}
\lang{en}{
only has the trivial solution in its solution set (see the 
\lref{ex:lsgsmenge-homogen}{previous example}). Hence the solution set of the inhomogeneous linear 
system is
}
\[ \mathbb{L}=\{ \begin{pmatrix} 2 \\ 3 \\ -4 \end{pmatrix} \}. \]
\end{tabs*}
\end{example}

\begin{quickcheck}

    \begin{variables}
    
      \randint[Z]{a}{2}{2}
      \randint[Z]{b}{6}{9}
      \randint[Z]{c}{2}{5}
      
    
      \function[calculate]{a3}{a*3}
      \function[calculate]{b3}{b*3}
      \function[calculate]{c2}{c*2}
       \function[calculate]{c4}{c*4}
      \function[calculate]{x}{(c2-b*2)/a}
     
    \end{variables}
       \text{\lang{de}{Betrachte das folgende Gleichungssystem:}
             \lang{en}{Consider the linear system}
             \[\begin{mtable}[\cellaligns{crcrcr}]
               (I)& \var{a} \cdot  x & + & \var{b} \cdot y & = & \var{c2} \\
               (II)& \var{a3} \cdot  x & + & \var{b3} \cdot y & = & \var{c4} \\
               \end{mtable}\]
             \lang{de}{Was gilt für die dazugehörige Lösungsmenge?}
             \lang{en}{What is its solution set?}
        } 
    \begin{choices}{unique}

      \begin{choice}
        \text{$\mathbb{L}=\left\{ r\cdot \begin{pmatrix} \var{x} \\ 2 \end{pmatrix} \mid\, r\in \R \right\}$}
        \solution{false}
      \end{choice}

      \begin{choice}
        \text{\text{$\mathbb{L}=\left\{\begin{pmatrix} \var{x} \\ 2 \end{pmatrix}\right\}$}}
        \solution{false}
        \end{choice}
      \begin{choice}
        \text{\text{$\mathbb{L}=\emptyset$}}
        \solution{true}
      \end{choice}
\end{choices}{unique}
\explanation{\lang{de}{
Die Lösungsmenge entspricht der leeren Menge, da durch $\text{(II)}-3\cdot\text{(I)}$ die Gleichung 
$0=-\var{c2}$ entsteht. Somit besitzt das LGS keine Lösung.
}
\lang{en}{
The empty set is the solution set, as $\text{(II)}-3\cdot\text{(I)}$ yields the equation 
$0=-\var{c2}$. Hence the linear system has no solutions.
}}
\end{quickcheck}


\end{visualizationwrapper}


\end{content}