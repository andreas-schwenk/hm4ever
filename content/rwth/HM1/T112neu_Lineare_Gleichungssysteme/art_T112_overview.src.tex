%$Id:  $
\documentclass{mumie.article}
%$Id$
\begin{metainfo}
  \name{
    \lang{de}{Überblick: Lineare Gleichungssysteme}
    \lang{en}{Overview: Linear systems}
  }
  \begin{description} 
 This work is licensed under the Creative Commons License Attribution 4.0 International (CC-BY 4.0)   
 https://creativecommons.org/licenses/by/4.0/legalcode 

    \lang{de}{Beschreibung}
    \lang{en}{Description}
  \end{description}
  \begin{components}
  \end{components}
  \begin{links}
\link{generic_article}{content/rwth/HM1/T112neu_Lineare_Gleichungssysteme/g_art_content_45_matrixrang.meta.xml}{content_45_matrixrang}
\link{generic_article}{content/rwth/HM1/T112neu_Lineare_Gleichungssysteme/g_art_content_41_gauss_verfahren.meta.xml}{content_41_gauss_verfahren}
\link{generic_article}{content/rwth/HM1/T112neu_Lineare_Gleichungssysteme/g_art_content_40_lineare_gleichungssysteme.meta.xml}{content_40_lineare_gleichungssysteme}
\end{links}
  \creategeneric
\end{metainfo}
\begin{content}
\begin{block}[annotation]
	Im Ticket-System: \href{https://team.mumie.net/issues/30137}{Ticket 30137}
\end{block}


\begin{block}[annotation]
Im Entstehen: Überblicksseite für Kapitel Lineare Gleichungssysteme
\end{block}

\usepackage{mumie.ombplus}
\ombchapter{1}
\lang{de}{\title{Überblick: Lineare Gleichungssysteme}}
\lang{en}{\title{Overview: Linear systems}}



\begin{block}[info-box]
\lang{de}{\strong{Inhalt}}
\lang{en}{\strong{Contents}}


\lang{de}{
    \begin{enumerate}%[arabic chapter-overview]
   \item[12.1] \link{content_40_lineare_gleichungssysteme}{Struktur der Lösungsmengen linearer Gleichungssysteme}
   \item[12.2] \link{content_41_gauss_verfahren}{Gauß-Verfahren}
   \item[12.3] \link{content_45_matrixrang}{Rang einer Matrix}
    \end{enumerate}
}
\lang{en}{
    \begin{enumerate}%[arabic chapter-overview]
   \item[12.1] \link{content_40_lineare_gleichungssysteme}{The solution set of a linear system}
   \item[12.2] \link{content_41_gauss_verfahren}{Gaussian elimination}
   \item[12.3] \link{content_45_matrixrang}{Rank of a matrix}
    \end{enumerate}
} %lang

\end{block}

\begin{zusammenfassung}
\lang{de}{
Mehrere linearer Gleichungen mit einer oder mehreren Unbekannten bezeichnet man als Lineares Gleichungssystem.
\\\\
Wir interessieren uns für eine systematische Ermittlung der Lösungsmenge und lernen als zentrales Werkzeug für die Berechnung das Gauß-Verfahren kennen.
\\\\
Die Struktur der Lösungsmenge analysieren wir im Detail. Gibt es eine eindeutige Lösung, unendlich viele Lösungen oder sogar gar keine Lösung? In welchem Bezug dazu steht der Rang einer Matrix?
}
\lang{en}{
A collection of linear equations in one or more variables is called a system of linear equations, or 
a linear system.
\\\\
We look at how the solution set of a linear system may be systematically determined, and we introduce 
Gaussian elimination as an algorithm for doing so.
\\\\
We examine the structure of the solution set: is there a single, unique solution? An infinite amount 
of solutions? No solutions? How does this relate to the rank of a matrix?
}
\end{zusammenfassung}

\begin{block}[info]\lang{de}{\strong{Lernziele}}
\lang{en}{\strong{Learning Goals}} 
\begin{itemize}[square]
\item \lang{de}{
      Sie erkennen, dass eine Menge von linearen Gleichungen systematisch gelöst werden kann.
      }
      \lang{en}{
      Understanding that a system of linear equations can be solved systematically.
      }
\item \lang{de}{
      Sie trainieren das Gauß-Verfahren zur Bestimmung der Lösungsmenge eines linearen 
      Gleichungssystems.
      }
      \lang{en}{
      Being able to use Gaussian elimination to determine the solution set of a linear system.
      }
\item \lang{de}{
      Sie untersuchen die Struktur der Lösung eines linearen Gleichungssystems und geben diese als 
      Menge an.
      }
      \lang{en}{
      Understanding the structure of a solution to a linear system, and being able to represent it in 
      terms of a solution set.
      }
\item \lang{de}{Sie bestimmen den Rang einer Matrix.}
      \lang{en}{Being able to determine the rank of a matrix.}
\end{itemize}
\end{block}


\end{content}
