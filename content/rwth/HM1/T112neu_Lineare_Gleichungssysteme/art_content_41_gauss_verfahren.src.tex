%$Id:  $
\documentclass{mumie.article}
%$Id$
\begin{metainfo}
  \name{
    \lang{de}{Gauß-Verfahren}
    \lang{en}{Gaussian elimination}
  }
  \begin{description} 
 This work is licensed under the Creative Commons License Attribution 4.0 International (CC-BY 4.0)   
 https://creativecommons.org/licenses/by/4.0/legalcode 

    \lang{de}{Beschreibung}
    \lang{en}{Description}
  \end{description}
  \begin{components}
    \component{js_lib}{system/media/mathlets/GWTGenericVisualization.meta.xml}{mathlet1}
    \component{generic_image}{content/rwth/HM1/images/g_img_00_video_button_schwarz-blau.meta.xml}{00_video_button_schwarz-blau}
  \end{components}
  \begin{links}
      \link{generic_article}{content/rwth/HM1/T101neu_Elementare_Rechengrundlagen/g_art_content_05_loesen_gleichungen_und_lgs.meta.xml}{content_05_loesen_gleichungen_und_lgs}
      \link{generic_article}{content/rwth/HM1/T112neu_Lineare_Gleichungssysteme/g_art_content_40_lineare_gleichungssysteme.meta.xml}{LGS}
  \end{links}
  \creategeneric
\end{metainfo}
\begin{content}
\begin{block}[annotation]
	Im Ticket-System: \href{https://team.mumie.net/issues/21350}{Ticket 21350}
\end{block}
\begin{block}[annotation]
Copy of \href{http://team.mumie.net/issues/9058}{Ticket 9058}: content/rwth/HM1/T111_Matrizen,_lineare_Gleichungssysteme/art_content_41_gauss_verfahren.src.tex
\end{block}

\usepackage{mumie.ombplus}
\ombchapter{12}
\ombarticle{2}
\usepackage{mumie.genericvisualization}

\begin{visualizationwrapper}

\title{\lang{de}{Gauß-Verfahren}\lang{en}{Gaussian elimination}}
 
\begin{block}[annotation]
  übungsinhalt
  
\end{block}
% \begin{block}[annotation]
%   Im Ticket-System: \href{http://team.mumie.net/issues/9058}{Ticket 9058}\\
% \end{block}

\begin{block}[info-box]
\tableofcontents
\end{block}


\lang{de}{
Das \notion{Gauß-Verfahren} ist eine Methode, um lineare Gleichungssysteme (LGS) systematisch zu 
lösen. Es stellt eine Verallgemeinerung des 
\link{content_05_loesen_gleichungen_und_lgs}{Additionsverfahrens} dar, und kann unabh\"angig von der 
Gr\"o{\ss}e des Systems eingesetzt werden.
\\\\
Im Folgenden betrachten wir stets lineare Gleichungssysteme der Form
}
\lang{en}{
\textit{Gaussian elimination} is an algorithm for solving systems of linear equation which 
generalises the \link{content_05_loesen_gleichungen_und_lgs}{addition ('elimination') method} to work 
for any number of equations and unknowns.
\\\\
In this section we always consider linear systems of the form
}

\begin{equation*} 
\begin{mtable}[\cellaligns{ccccccccc}]
 a_{11} x_1 & + & a_{12} x_2 & + & \cdots & + & a_{1n} x_n & = & b_1 \\
 a_{21} x_1 & + & a_{22} x_2 & + & \cdots & + & a_{2n} x_n & = & b_2 \\
 \vdots     &   & \vdots     &   &        &   & \vdots     &   & \vdots \\
 a_{m1} x_1 & + & a_{m2} x_2 & + & \cdots & + & a_{mn} x_n & = & b_m
\end{mtable} \label{eq:lgs1}
\end{equation*}
\lang{de}{
mit Elementen $a_{ij}\in \mathbb{R}$ für $1 \leq i \leq m$ und $1 \leq j \leq n$ und 
$b_1, \ldots, b_m\in \mathbb{R}$, sowie die zugehörige erweiterte Koeffizientenmatrix
}
\lang{en}{
with elements $a_{ij}\in \mathbb{R}$ for $1 \leq i \leq m$, $1 \leq j \leq n$ and 
$b_1, \ldots, b_m\in \mathbb{R}$, and the corresponding augmented matrices
}
\begin{equation*}
(A \ \mid \ b) = \underbrace{\left(  \begin{smallmatrix}
                                      a_{11} & a_{12} & \cdots & a_{1n} & | & b_1\\
                                      a_{21} & a_{22} & \cdots & a_{2n} & | &b_2\\
                                      \vdots & \vdots & \ddots & \vdots & \mid &\vdots\\
                                      a_{m1} & a_{m2} & \cdots & a_{mn} &| & b_m
                                      \end{smallmatrix} \right) }_{
\in M(m,n+1;\mathbb{R})}.
\end{equation*}


\lang{de}{
Ziel des Gauß-Verfahrens ist es, ein LGS schrittweise in eine \textbf{Stufenform} zu bringen,
aus der die Lösungsmenge bestimmt werden kann.
}
\lang{en}{
The aim of Gaussian elimination is to manipulate the linear system step-by-step into \textbf{reduced 
row echelon form}, from which the solutions can be read.
}

\section{\lang{de}{Die Stufenform}\lang{en}{Row echelon form}}

\begin{definition}[\lang{de}{Stufenform}
                   \lang{en}{Row echelon form}]\label{def:stufenformen}\label{stufenform}
\lang{de}{
Ein LGS liegt in \notion{Stufenform} vor, wenn für jede Gleichung die
%(lexikographisch) 
erste auftretende Variable
in allen folgenden Gleichungen nicht mehr auftritt (oder die Gleichung gar keine Variablen enthält).
\\\\
Betrachtet man die Stufenform der dazugehörigen erweiterten Koeffizientenmatrix, so hat diese die Form
}
\lang{en}{
A linear system is in \textbf{row echelon form} if the variable in the leading term of each equation 
has coefficient zero in all the following equations (or the equation has no variables).
\\\\
The augmented matrix corresponding to a linear system in row echelon form has the following form:
}
\[ \begin{pmatrix}
    a_{1} & \star & \cdots & \star & \star&| & b_1\\ 
     0 & a_{2} & \cdots & \star & \star&| & b_2\\ 
    \vdots & \, & \ddots & \, & \vdots&| & \vdots\\ 
 0 & 0 & \cdots & a_{k} & \star&| & b_k\\ 
   0 & 0 & \cdots & 0 & 0&| & \vdots \\ 
  0 & 0 & \cdots & 0 & 0&| & b_{m}\\ 
    \end{pmatrix},\]
\lang{de}{
wobei der Eintrag $a_{i}$ jeweils dem ersten Koeffizienten ungleich Null der i-ten Zeile entspricht. 
Diesen Koeffizienten nennen wir \notion{Stufenelement} (oder Stufeneintrag). Der "$\star$" steht 
jeweils für einen beliebigen Eintrag. 
\\\\
Das LGS liegt in \notion{reduzierter Stufenform} vor, wenn die ersten auftretenden Variablen sogar in 
keiner anderen Gleichung auftreten (also auch nicht in den darüber liegenden) und die Stufenelemente 
zusätzlich den Wert 1 besitzen.
\\\\
Die entsprechende reduzierte Stufenform der erweiterten Koeffizientenmatrix hat damit die Form
}
\lang{en}{
where the element $a_{i}$ is the first non-zero coefficient of each row. We call this coefficient the 
\notion{leading coefficient} of the row/equation. The '$\star$' here may be any element.
\\\\
The linear system is said to be in \notion{reduced row echelon form} if the variable in the leading 
term of each equation has coefficient zero in all other equations (not just the following equations), 
and the leading coefficients of each row is $1$. That is, if an row in the matrix has a leading 
coefficient, it must be $1$, and every other element of that column must be zero.
\\\\
The augmented matrix corresponding to a linear system in reduced row echelon form has the following 
form:
}
 
 \[ \begin{pmatrix}
    1 & 0 & \cdots & 0 & \star&| & b_1\\ 
     0 & 1 & \cdots & 0 & \star&| & b_2\\ 
    \vdots & \, & \ddots & \, & \vdots &| & \vdots\\ 
 0 & 0 & \cdots & 1 & \star&| & b_k\\ 
   0 & 0 & \cdots & 0 & 0&| & \vdots \\ 
  0 & 0 & \cdots & 0 & 0&| & b_{m}\\ 
  \end{pmatrix},\]
\lang{de}{
wobei zu beachten ist, dass zusätzlich nur die Einträge über einem Stufenelement Null sein müssen.
}
\lang{en}{
and it it important to note that the elements below \textbf{and} above each leading coefficient must 
be zero.
}

\end{definition}
\lang{de}{
In der Literatur spricht man oft präziser von der Zeilenstufenform. Das Stufenelement wird auch 
\emph{Pivotelement} genannt.
}
\lang{en}{
In the literature, the leading coefficient of a linear system in row echelon form is also called the 
\emph{pivot element}.
}
\begin{remark}
\lang{de}{
Falls eine Koeffizientenmatrix in Stufenform eine Nullzeile besitzt, so ist die entsprechende 
Gleichung des LGS $0=b_i$. In dieser Gleichung treten also keine Variablen mehr auf. \\
}
\lang{en}{
If a coefficient matrix in row echelon form contains a row of zeros, the corresponding equation in 
the linear system is $0=b_i$, i.e. the equation contains no variables. \\
}
 

\end{remark}

\begin{example}
\begin{tabs*}
\tab{\lang{de}{1. Beispiel}\lang{en}{Example 1}}
\lang{de}{Das LGS}
\lang{en}{The linear system}
\[ \begin{mtable}[\cellaligns{ccrcrcrcr}]
(I)&\qquad-&x&+&2y&+&3z&=&5\\
(II)&&&&y&+&4z&=&11\\
(III)&&&&&-&2z&=&-6
\end{mtable} \]
\lang{de}{
liegt in Stufenform vor. Die erste Variable der ersten Gleichung (die Variable $x$) kommt in den 
anderen nicht vor und die erste Variable in der zweiten Gleichung (die Variable $y$) kommt in der 
dritten Gleichung nicht vor.
\\\\
Es ist aber keine reduzierte Stufenform, da zum einen die Variable $y$ in der ersten Gleichung 
vorkommt, zum anderen aber auch die Variable $z$ in den anderen Gleichungen auftritt. Außerdem haben 
die Stufenelemente nicht alle den Wert 1:
}
\lang{en}{
is in row echelon form. The first variable of the first equation (the variable $x$) does not appear 
in the other equations, and the first variable of the second equation (the variable $y$) does not 
appear in the third equation.
\\\\
However, it is not in reduced row echelon form, firstly because the variable $y$ appears in the first 
equation, and secondly because the variable $z$ appears in both other equations. Furthermore, the 
leading coefficients are not all $1$:
}
\[ \begin{mtable}[\cellaligns{ccrcrcrcr}]
(I)&\qquad\textcolor{#CC6600}{-}&\textcolor{#CC6600}{1}x&+&\textcolor{#CC6600}{2}y&+&\textcolor{#CC6600}{3}z&=&5\\
(II)&&&&y&+&\textcolor{#CC6600}{4}z&=&11\\
(III)&&&&&\textcolor{#CC6600}{-}&\textcolor{#CC6600}{2}z&=&-6
\end{mtable} \]
\tab{\lang{de}{2. Beispiel}\lang{en}{Example 2}}
\lang{de}{Das LGS}
\lang{en}{The linear system}
\[ \begin{mtable}[\cellaligns{crcrcrcr}]
(I)&\qquad 2 \cdot  x & - & 4 \cdot  y & - & 6 \cdot  z & = & 10 \\
(II)&-3 \cdot  x & + & 6\cdot y & + & 9 \cdot  z & =  & -15
\end{mtable} \]
\lang{de}{
ist nicht in Stufenform, da alle Variablen der ersten Gleichung auch in der zweiten Gleichung auftreten.
\\\\
Anders verhält es sich beim LGS
}
\lang{en}{
is not in row echelon form, as all of the variables in the first equation also appear in the second 
equation.
\\\\
On the other hand, the linear system
}
\[ \begin{mtable}[\cellaligns{crcrcr}]
(I)&\qquad  x & - & 4 \cdot  y & = & 10 \\
(II)& &  & 0 & =  & 0,
\end{mtable} \]
\lang{de}{
bei dem die erste Variable $x$ in der zweiten Gleichung nicht mehr auftritt und das zugehörige Stufenelement den Wert 1 besitzt. Dieses LGS ist damit in reduzierter Stufenform.
}
\lang{en}{
is in reduced row echelon form, as the first variable $x$ does not appear in the second equation, 
and it has coefficient $1$ in the first equation.
}
\tab{\lang{de}{3. Beispiel}\lang{en}{Example 3}}
\lang{de}{Das LGS}
\lang{en}{The linear system}
\[ \begin{mtable}[\cellaligns{crcrcrcrcr}]
(I)&\qquad x_1 & + & 2 \cdot  x_2 &+ & x_3 & - & x_4 &= & 7 \\
(II)& 		  &  &           x_2 &- & x_3 &   &     &=  & 1 \\
(III)& 		  &  &               &  &     &   & x_4 &= & -1
\end{mtable} \]   
\lang{de}{
ist in Stufenform. Die Variable $x_1$ kommt nur in der ersten Gleichung vor. Die Variable $x_2$ kommt 
in der zweiten Gleichung, nicht aber in der dritten vor.
\\\\
Die Stufenform ist jedoch keine reduzierte Stufenform, obwohl die Stufenelemente 1 entsprechen. Zum 
einen tritt die Variable $x_2$ in der ersten Gleichung auf und zum anderen tritt die Variable $x_4$, 
welche die erste Variable der dritten Gleichung ist, auch in der ersten Gleichung auf:\\
}
\lang{en}{
is in row echelon form. The variable $x_1$ only appears in the first equation. The variable $x_2$ 
appears in the second equation, but not the third.
\\\\
However, it is not in reduced row echelon form, even thought the leading coefficients are all $1$. 
Firstly, the variable $x_2$ appears in the first equation, and secondly, the variable $x_4$, which is 
the leading variable of the third equation, appears in the first equation:\\
}
\[ \begin{mtable}[\cellaligns{crcrcrcrcr}]
(I)&\qquad x_1 & + & \textcolor{#CC6600}{2} \cdot  x_2 &+ & x_3 & \textcolor{#CC6600}{-} & \textcolor{#CC6600}{1}x_4 &= & 7 \\
(II)& 		  &  &           x_2 &- & x_3 &   &     &=  & 1 \\
(III)& 		  &  &               &  &     &   & x_4 &= & -1
\end{mtable} \]  
\lang{de}{
Dass die Variable $x_3$ in zwei Gleichungen auftritt, spielt keine Rolle, da sie in keiner der 
Gleichungen die erste auftretende Variable ist.
}
\lang{en}{
The fact that the variable $x_3$ appears in two equations is not relevant here, as it is not the 
leading variable in any of the equations.
}
\end{tabs*}
\end{example}

\begin{quickcheck}
		\type{input.interval}
        \field{rational}
        \precision{3}
      \field{real}
      \begin{variables}
           \randint[Z]{a}{2}{6}
           \randint[Z]{b}{2}{8}
           \randint[Z]{c}{2}{8}
           \randint[Z]{d}{2}{8}
           \end{variables}
      \text{\lang{de}{Für das lineare Gleichungssystem}
            \lang{en}{The linear system} 
\[ \begin{mtable}[\cellaligns{crcrcrcr}]
(I)&\var{a} \cdot  x & - & \var{c} \cdot y       &   &                &= & 0 \\
(II)&                &   & \var{b}\cdot y & + & \var{d} \cdot z&=  & \var{a} 
\end{mtable} \]
             \lang{de}{gilt: }\lang{en}{is: }} 
    \begin{choices}{unique}

        \begin{choice}
            \text{\lang{de}{Es liegt in Stufenform vor, aber nicht in reduzierter Stufenform.}
                  \lang{en}{In row echelon form, but not reduced row echelon form.}}
			\solution{true}
		\end{choice}
                    
        \begin{choice}
            \text{\lang{de}{Es liegt in reduzierter Stufenform vor.}
                  \lang{en}{In reduced row echelon form.}}
			\solution{false}
		\end{choice}
                \begin{choice}
            \text{\lang{de}{Es liegt nicht in Stufenform vor.}
                  \lang{en}{Not in row echelon form.}}
			\solution{false}
		\end{choice}
      
    \end{choices}{unique}
    \explanation{\lang{de}{
    Das LGS liegt in Stufenform vor, da die Variable $x$ nur in der ersten Gleichung  auftritt.\\ 
    Es liegt aber nicht in reduzierter Stufenform vor, da die Variable $y$ (als erste Variable der 
    zweiten Gleichung) ebenfalls in der ersten Gleichung auftaucht. Zusätzlich sind die 
    Stufenelemente von 1 verschieden.
    }
    \lang{en}{
    The linear system is in row echelon form, as the variable $x$ only appears in the first 
    equation.\\
    However, it is not in reduced row echelon form, as the variable $y$ (the leading variable of the 
    second equation) appears in the first equation. Furthermore, the leading coefficients are not 
    equal to $1$.
    }}	
	\end{quickcheck}
    
    
%Im Allgemeinen sieht ein LGS in Stufenform dann folgendermaßen aus mit gewissen Zahlen $k$ ($1\leq k\leq m$) und
%$1\leq i_1<i_2<\ldots <i_k\leq n$, sowie $c_i\in \R\setminus \{0\}$ und $\tilde{a}_{ij}, \tilde{b}_i\in \R$:
%\begin{equation}\label{eq:lgs2}
% \begin{mtable}[\cellaligns{rrrrrrrrrrrrrrl}]
% c_1 x_{i_1} & + & \cdots & + & \tilde{a}_{1i_2}x_{i_2} & + & \cdots & &\cdots &&\cdots && \cdots & = & \tilde{b}_1 \\
%			 &   	&  &  &   c_2 x_{i_2} & + & \cdots &+ & \tilde{a}_{2i_3}x_{i_3} & + &\cdots&&  \cdots & = & \tilde{b}_2 \\
% 			 &   	&  			 &  &                          &    &            &   &   c_3 x_{i_3} & + &\cdots&&  \cdots & = & \tilde{b}_3 \\
%    &    &          &   &                   &      &      &    &   &   & \vdots   &&   &     & \vdots \\
% 			 &   &               &   &        &    & &           &        &   &  c_k x_{i_k} & + & \cdots & = & \tilde{b}_k \\
% 			 &&&&&&&&&&&& 0 & = & \tilde{b}_{k+1} \\
% 			 &&&&&&&&&&&& \vdots & & \vdots \\
% 			 &&&&&&&&&&&& 0 & = & \tilde{b}_{m} 
%\end{mtable} 
%\end{equation}
%In der reduzierten Stufenform sind zusätzlich die Koeffizienten über den $c$'s gleich Null.
\lang{de}{Anhand der Stufenform lässt sich erkennen, ob ein LGS lösbar ist.}
\lang{en}{From the row echelon form of a linear system we can tell whether it has solutions.}
%und sie lässt sich damit explizit durch Rückwärtseinsetzen berechnen. 
\begin{theorem}[\lang{de}{Lösbarkeit}\lang{en}{Solvability}]\label{rule:loesung-parametrisiert}
\lang{de}{
Ein LGS ist genau dann lösbar, wenn für die Stufenform (\ref{stufenform}) seiner erweiterten 
Koeffizientenmatrix entweder
}
\lang{en}{
A linear system is solvable (has at least one solution) if and only if the row echelon form of its augmented matrix has either
}
%Ein LGS und ihre dazugehörige erweiterte Koeffizientenmatrix ist genau 
%dann l"osbar, wenn für ihre Stufenform (\ref{stufenform}) entweder 
\begin{itemize}
\item $k=m$ \lang{de}{oder}\lang{en}{or}
\item${b}_{k+1}=\ldots = {b}_{m} =0$
\end{itemize}
\lang{de}{
gilt.\\
Lösbarkeit bedeutet also, dass nach Umformung in Stufenform keine Zeile der Form $(0 \cdots 0 | b_i)$ 
mit $b_i \neq 0$ auftritt.
}
\lang{en}{
Hence solvability of a linear system corresponds to there not existing any rows of the form 
$(0 \cdots 0 | b_i)$ with $b_i \neq 0$ in the row echelon form of the system.
}
\end{theorem}

\lang{de}{
Ist ein LGS nicht lösbar, so gilt für seine Lösungsmenge $\mathbb{L}=\emptyset$.\\
Für die Bestimmung der Lösungsmenge eines lösbaren LGS gehen wir wie folgt vor.
}
\lang{en}{
If a linear system is not solvable, then its solution set is $\mathbb{L}=\emptyset$.\\
We determine the solution set of a solvable linear system as follows.
}

\begin{rule}[\lang{de}{Bestimmung der Lösungsmenge}\lang{en}{Determining the solution set}]
\lang{de}{
Um die Lösungsmenge zu berechnen, stellt man die erweiterte Koeffizientenmatrix wieder als lineares 
Gleichungssystem dar.\\
Unbekannte, die als Koeffizienten einen Stufeneintrag besitzen, werden \notion{abhängige Variablen} 
genannt und folglich sind alle Unbekannten ohne einen Stufeneintrag als Koeffizienten 
\notion{unabhängige} oder \notion{freie Variablen}.\\
Für letztere werden Parameter gewählt und anschließend wird das Gleichungssystem nach den abhängigen 
Variablen durch \notion{Rückwärtseinsetzen} aufgelöst.
\\\\
Ist das LGS in reduzierter Stufenform, so enthält jede Gleichung nur genau eine abhängige Variable 
und kann gegebenenfalls direkt in Abhängigkeit der Parameter abgelesen werden oder die Gleichung 
entspricht auf der linken Seite Null. Besitzt ein lösbares LGS keine freien Variablen, so besteht die 
Lösungsmenge aus genau einem eindeutigen Lösungsvektor.
}
\lang{en}{
To determine the solution set of a linear system in row echelon form, we once more consider the 
system as a system of linear equations, rather than an augmented matrix.\\
The leading variable of a row is called a \notion{dependent variable}, and naturally any variables 
which do not appear in a leading term are called \notion{independent variables}, or \notion{free 
variables}.\\
We choose parameters to represent these free variables and solve the system of equations for the 
dependent variables in a process called \notion{back substitution}.
\\\\
If the linear system is in reduced row echelon form, each non-zero equation contains precisely one 
dependent variable, and is easily rearranged to give the solution for each of these dependent 
variables in terms of the parameters, if there are any. If a solvable linear system has no free 
variables, then the solution set consists of a single solution vector.
}

\end{rule}

\lang{de}{
Rückwärtseinsetzen bedeutet, dass man bei der letzten Gleichung beginnend alle Gleichungen von unten 
nach oben nach der ersten auftretenden Variable l\"ost, indem man alle bereits in Parametern 
ausgedrückten oder durch Werte berechneten Variablen in die darüber liegenden Gleichungen einsetzt.\\
Dadurch erhält man für alle Variablen Werte oder Ausdrücke in den Parametern.
%Dadurch erhält man für alle Variablen Werte oder Ausdrücke in den Parametern.
}
\lang{en}{
Back substitution is the process wherein we rearrange the last equation for its leading variable, 
then the equation above it, and so on, until all dependent variables have been expressed in terms of 
the parameters. This is possible because in each step, we have already solved for all dependent 
variables in the equations below it, so each equation comes with at most one new dependent variable 
to rearrange for.\\
This method yields expressions (either constant or in terms of parameters) for all variables.
}
\begin{remark}
\lang{de}{
Werden die freien bzw. abhängigen Variablen anders gewählt, so ändert sich die Lösungsmenge nicht. 
Sie erhält dadurch aber eine andere Darstellung.
}
\lang{en}{
If different parameters are chosen for the free (independent) variables, the solution set does not 
change, although its representation may change.
}
\end{remark}


\begin{example} \label{gauss1}
\begin{tabs*}
\tab{\lang{de}{1. Beispiel}\lang{en}{Example 1}}
\lang{de}{Im LGS}
\lang{en}{Every variable in the linear system}
\begin{displaymath}
\begin{mtable}[\cellaligns{ccrcrcrcr}]
(I)&\qquad-&x&+&2y&+&3z&=&5\\
(II)&&&&y&+&4z&=&11\\
(III)&&&&&-&2z&=&-6
\end{mtable}
\end{displaymath}
\lang{de}{
ist jede Variable die erste auftretende Variable einer Gleichung. Es gibt daher keine freien 
Variablen, und für alle Variablen können Werte berechnet werden.\\
Zun\"achst wird die dritte Gleichung nach der Variable $z$ aufgel\"ost.
}
\lang{en}{
is the leading variable of one of the equations. Hence there are no free variables, and we can find a 
constant value for every variable.\\
Firstly, we rearrange the third equation for the variable $z$.
}
\begin{displaymath}
(III) \qquad z=3
\end{displaymath}
\lang{de}{
Diesen Wert setzt man in die zweite Gleichung ein, um die Variable $y$ zu bestimmen.
}
\lang{en}{
This value is then substituted into the second equation in order to determine the value of $y$.
}
\begin{displaymath}
(II) \qquad y+4\cdot3=11 \quad \Leftrightarrow \quad y=-1
\end{displaymath}
\lang{de}{
Anschlie{\ss}end werden beide Werte in die erste Gleichung eingesetzt, um daraus $x$ zu berechnen.
}
\lang{en}{
Next, both values are substituted into the first equation, giving us the value of $x$.
}
\begin{displaymath}
(I) \qquad -x+2\cdot(-1)+3\cdot3=5 \quad \Leftrightarrow \quad x=2
\end{displaymath}
\lang{de}{Das LGS ist eindeutig l\"osbar und besitzt die L\"osungsmenge}
\lang{en}{The linear system is uniquely solvable and has the solution set:}
\begin{displaymath}
\mathbb{L}=\left\{\begin{pmatrix}2\\ -1\\ 3\end{pmatrix} \right\}.
\end{displaymath}
\tab{\lang{de}{2. Beispiel}\lang{en}{Example 2}}
\lang{de}{Das LGS}
\lang{en}{The linear system}
\[ \begin{mtable}[\cellaligns{crcrcr}]
(I)&\qquad x & - & 4 \cdot  y & = & 10 \\
(II)& &  & 0 & =  & 0
\end{mtable} \]
\lang{de}{
ist lösbar, da es zwar eine Gleichung ohne Variablen gibt, diese aber $0=0$ entspricht, d.\,h. stets 
erfüllt ist. Diese Gleichung kann nun weggelassen werden. In der verbliebenen Gleichung (I) ist $x$ 
die erste Variable, also eine abhängige Variable, und $y$ ist eine freie Variable. Wir führen daher 
einen reellen Parameter $r$ für $y$ ein, setzen also $y=r$. Nun setzt man $y=r$ in die Gleichung (I) 
ein und löst sie nach $x$ auf und erhält
}
\lang{en}{
is solvable, as although one of the equations has no variables, $0=0$, this is always satisfied. 
This equation may hence be ignored. The leading variable of the remaining equation (I) is $x$, so 
$x$ is a dependent variable, and $y$ is a free variable. We introduce a parameter $r$ for $y$, that 
is, we set $y=r$. Substituting this into equation (I) and rearranging for $x$ yields
}
\[ x=10+4r.\]
\lang{de}{Die Lösungsmenge ist daher}
\lang{en}{Thus the solution set is}
\[ \mathbb{L}= \left\{ \begin{pmatrix}10+4r\\ r \end{pmatrix} \,\big| \, r\in \R \right\}
= \left\{ \begin{pmatrix}10\\ 0 \end{pmatrix}+r\cdot \begin{pmatrix} 4 \\ 1 \end{pmatrix} \, \big| \, r\in \R \right\}. \] 
\tab{\lang{de}{3. Beispiel}\lang{en}{Example 3}}
\lang{de}{Im LGS}
\lang{en}{The linear system}
\[ \begin{mtable}[\cellaligns{crcrcrcrcr}]
(I)&\qquad x_1 & + & 2 \cdot  x_2 &+ & x_3 & - & x_4 &= & 7 \\
(II)& 		  &  &           x_2 &- & x_3 &   &     &=  & 1 \\
(III)& 		  &  &               &  &     &   & x_4 &= & -1
\end{mtable} \]
\lang{de}{
sind die Variablen $x_1$, $x_2$ und $x_4$ die ersten auftretenden Variablen einer der Gleichungen. 
Dies sind also die abhängigen Variablen und die Variable $x_3$ ist eine freie Variable. Wir setzen 
also $x_3=t$ mit einem reellen Parameter $t$, und rechnen die anderen Variablen von unten nach oben 
aus. Die dritte Gleichung ist
}
\lang{en}{
has the variables $x_1$, $x_2$ and $x_4$ in the leading terms of each equation. These are therefore 
the dependent variables, making $x_3$ a free variable. We hence set $x_3=t$ where $t$ is a real 
parameter, and solve for the other variables, starting from the last equation. The third equation is
}
\[   x_4=-1.  \]
%Dies setzen wir in (II) ein:
\lang{de}{Für die zweite Gleichung gilt:}
\lang{en}{We rearrange the second equation,}
\[  x_2-t=1 \Leftrightarrow x_2=1+t. \]
\lang{de}{
Nun setzen wir die Ausdrücke für $x_2$, $x_3$ und $x_4$ in die erste Gleichung ein und lösen nach 
$x_1$ auf:
}
\lang{en}{
Finally, we substitute the expressions for $x_2$, $x_3$ and $x_4$ into the first equation and 
rearrange for $x_1$, giving us
}
\[ x_1+2\cdot (1+t)+t- (-1)=7  \Leftrightarrow x_1=7-2\cdot (1+t)-t-1=4-3t. \]
\lang{de}{Die Lösungsmenge ist daher:}
\lang{en}{Thus the solution set is}
\[ \mathbb{L}
= \left\{ \begin{pmatrix} 4-3t \\ 1+t \\t \\ -1\end{pmatrix} \, \big| \, t\in \R \right\}
= \left\{ \begin{pmatrix} 4 \\ 1 \\ 0 \\ -1\end{pmatrix}+ t\cdot \begin{pmatrix} -3\\ 1\\ 1\\ 0 \end{pmatrix} \, \big| \, t\in \R \right\}. \]
\end{tabs*}
\end{example}





\section{\lang{de}{Elementare Umformungen und das Gau{\ss}-Verfahren}
         \lang{en}{Elementary row operations and Gaussian elimination}} \label{ezu}
\lang{de}{
Mit dem Gauß-Verfahren lassen sich lineare Gleichungssysteme systematisch in die (reduzierte) 
Stufenform umwandeln. Im Anschluss lässt sich die Lösungsmenge, wie im vorherigen Abschnitt gesehen, 
ablesen.
}
\lang{en}{
Gaussian elimination is an algorithm for systematically changing any linear system into row echelon 
form. Having done this, the solution set can be easily determined as is shown in the previous part.
}

\begin{definition}[\lang{de}{Elementare Zeilenumformungenmformungen}\lang{en}{Elementary row operations}]\label{def:elementare_Zeilenumformungen}
\lang{de}{
Folgende sogenannte \notion{elementare Zeilenumformungen} eines LGS ver\"andern die L\"osungsmenge des LGS 
nicht.
}
\lang{en}{
The following are called \textit{elementary row operations} of a linear system and are useful because 
they do not modify the solution set in any way.
}
\begin{itemize}
\item \lang{de}{Vertauschen zweier Gleichungen.}
      \lang{en}{Exchanging/swapping two equations.}
\item \lang{de}{Multiplikation einer Gleichung mit einer Konstanten $c\neq0$.}
      \lang{en}{Multiplication of an equation by a number $c\neq0$.}
\item \lang{de}{
      Addition (Subtraktion) eines $c$-fachen einer Gleichung zu (von) einer anderen.
      }
      \lang{en}{
      Addition (or subtraction) of a multiple of one row to (from) another.
      }
\end{itemize}
\end{definition}
\lang{de}{
Die letzte elementare Zeilenumformung wird als $c\cdot (L)+(K)$ notiert. Hier beschreibt $c\cdot(L)$ das 
$c$-fache der $l$-ten Gleichung, welches zu der $k$-ten Gleichung addiert wird. Diese Umformung 
betrifft die $k$-te Gleichung, die $l$-te Gleichung selbst bleibt dabei unverändert.
}
\lang{en}{
The last of these row operations is often denoted as $c\cdot (L)+(K)$. Here $c\cdot(L)$ means the 
$l$th equation multiplied by the real number $c$, which is then added to the $k$th equation $(K)$. 
This operation modifies the $k$th equation, leaving the $l$th equation unchanged.
}
\begin{rule}[\lang{de}{Gauß-Verfahren LGS}\lang{en}{Gaussian elimination}]\label{rule:gauss}

\lang{de}{
Man l\"ost ein LGS, indem man es zun\"achst durch elementare Umformungen auf die Stufenform (oder 
sogar reduzierte Stufenform) bringt und anschlie{\ss}end die L\"osungsmenge, wie oben beschrieben, 
bestimmt.
\\\\
Genauer geht man folgendermaßen vor, wobei beispielsweise $(K)$ der $k$-ten Gleichung entspricht:
}
\lang{en}{
We solve a general linear system by firstly applying row operations to it until it is expressed in 
row echelon (or even reduced row echelon) form. We then find the solution set as is described in the previous part of this section.
\\\\
Denoting the $k$th equation as $(K)$, Gaussian elimination is performed as follows:
}
\begin{enumerate}
\item \lang{de}{
      Vertausche die Gleichungen des Gleichungssystems so, dass die erste Variable (normalerweise 
      $x_1$) in der ersten Gleichungen vorkommt (falls $x_1$ "uberhaupt nicht vorkommt, ist das 
      $x_2$ bzw. $x_3$ etc.). Wir bezeichnen den Index dieser Variablen mit $j$.
      }
      \lang{en}{
      Swap the equations of the linear system so that the first variable (usually $x_1$) appears in 
      the first equation (if the variable $x_1$ does not appear in any equation, we consider $x_2$ 
      or else $x_3$, etc.). We denote the index of this variable as $j$.
      }

%\item Teile die erste Gleichung durch den Koeffizienten der ersten Variablen.
\item \lang{de}{
      Addiere jeweils geeignete Vielfache der ersten Gleichung $(K)$ zu den darunter liegenden 
      Gleichungen $(L)$, so dass die erste Variable in den anderen Gleichungen eliminiert wird:\\
      Für alle $l$ mit  $k<l$ forme die Gleichung $(L)$ zu $-\frac{a_{lj}}{a_{kj}}(K)+(L)$ um.
      }
      \lang{en}{
      Add a multiple of the first equation $(K)$ to each of the equations $(L)$ below it, choosing 
      each multiple so that the first variable is eliminated from each of the $(L)$:\\
      for all $l$ with $l>k$ replace the equation $(L)$ with the equation 
      $-\frac{a_{lj}}{a_{kj}}(K)+(L)$.
      }
\item \lang{de}{
      Die erste Gleichung bleibt ab jetzt unverändert. Wiederhole die Schritte mit dem übrigen 
      Gleichungssystem, das heißt ab der zweiten Gleichung. Dann ab der dritten usw.,  bis nur noch 
      eine Gleichung übrig bleibt oder die linken Seiten der restlichen Gleichungen alle gleich $0$ 
      sind.
      }
      \lang{en}{
      The first equation remains unchanged from now on. Repeat the above steps on the linear system 
      obtained by ignoring the first equation, that is, beginning with the second equation. Then 
      we consider the system beginning with the third equation, etc. until only one equation 
      remains, or the left-hand side of every remaining equation is equal to $0$.
      }
\end{enumerate}
\lang{de}{
Das LGS ist damit in Stufenform.
\\\\
Um eine reduzierte Stufenform zu erhalten, verfährt man nach dieser Umformung weiter:
}
\lang{en}{
The linear system is then in row echelon form.
\\\\
To obtain the reduced row echelon form, we continue from here:
}
\begin{enumerate}
\item[4.] \lang{de}{
      Bringe den Koeffizienten der erstauftretenden Variable in jeder Zeile auf den Wert 1:\\
      Teile dazu jede Gleichung, bei der die linke Seite nicht komplett $0$ ist, durch den ersten 
      Koeffizienten:\\
      Für alle entsprechenden $l$ mit $l \geq 1$ forme die Gleichung $(L)$ zu $\frac{1}{a_{lj}}(L)$ 
      um.
      }
      \lang{en}{
      Make the leading coefficient of each equation equal to $1$ by dividing each equation whose 
      left-hand side is not zero by its leading coefficient:\\
      for all $l$ with $l \geq 1$ we replace equation $(L)$ by the equation $\frac{1}{a_{lj}}(L)$.
      }
\item[5.] \lang{de}{
      Eliminiere alle dazugehörigen Variablen in den darüber liegenden Gleichungen:\\
      Addiere geeignete Vielfache der in 5. genannten Gleichung $(L)$ zu den darüber liegenden 
      Gleichungen $(K)$.\\
      Für alle $k$ mit $1\leq k < l$ forme die Zeile $(K)$ zu $-a_{kj}(L)+(K)$ um.
      }
      \lang{en}{
      Eliminate all terms containing a non-leading dependent variable in the above equations by 
      adding appropriate multiples of equation $(L)$ to every equation $(K)$ above it, starting from 
      the final equation:\\
      for all $k$ with $1\leq k < l$ we replace the equation $(K)$ with $-a_{kj}(L)+(K)$.
      }
\end{enumerate}
\lang{de}{
\floatright{\href{https://api.stream24.net/vod/getVideo.php?id=10962-2-10841&mode=iframe&speed=true}{\image[75]{00_video_button_schwarz-blau}}}\\
}
\lang{en}{}
\end{rule}


\begin{remark}
\lang{de}{
Beim handschriftlichen Lösen eines LGS wird der Algorithmus oft nicht streng umgesetzt. Um Brüche zu 
vermeiden, werden im ersten Schritt zum Beispiel die Gleichungen so getauscht, dass der Koeffizient 
$a_{kj}=1$ ist. Außerdem kann man in Schritt 2 beispielsweise anstelle der Umformung 
$(L)-\frac{a_{lj}}{a_{kj}}(K)$ auch $-a_{kj}(L) + a_{lj}(K)$ verwenden. Sowohl die Lösungsmenge als 
auch die reduzierte Stufenform bleiben hierbei unverändert.
}
\lang{en}{
When solving a linear system by hand, we often modify the algorithm slightly for convenience. For 
example, in order to avoid fractions where possible, in the first step we may try to  get a 
coefficient $a_{kj}=1$ in front of the first variable. In the second step we may, with the same aim, 
replace $(L)$ with $-a_{kj}(L) + a_{lj}(K)$ instead of with $(L)-\frac{a_{lj}}{a_{kj}}(K)$. Both the 
solution set and the reduced row echelon form remain unaffected by these modifications.
}
\end{remark}

\begin{example}\ref{ex:gauss}
\begin{tabs*}
\tab{\lang{de}{1. Beispiel}\lang{en}{Example 1}}
\begin{displaymath}
\begin{mtable}[\cellaligns{ccrcrcrcrl}]
(I)&\qquad-&x&+&2y&+&3z&=&5&\\
(II)&\qquad&\textcolor{#CC6600}{x}&-&y&+&z&=&6&\qquad|\,\,+\text{(I)}\\
(III)&\qquad&\textcolor{#CC6600}{2x}&-&3y&-&4z&=&-5&\qquad|\,\,+2\cdot\text{(I)}
\end{mtable}
\end{displaymath}
\lang{de}{
Wir wenden die elementaren Zeilenumformungen an, um die Stufenform zu erzeugen. Die Variable $x$ kann in 
(II) und (III) eliminiert werden, indem man die erste Gleichung zur zweiten und das Doppelte der 
ersten Gleichung zur dritten addiert.
}
\lang{en}{
We will use elementary row operations to bring the system into row echelon form. The variable $x$ 
can be removed from equation $(II)$ by adding the first equation to $(II)$, and from equation 
$(III)$ by adding double the first equation to $(III)$.
}
\begin{displaymath}
\begin{mtable}[\cellaligns{ccrcrcrcrl}]
(I)&\qquad-&x&+&2y&+&3z&=&\phantom{-}5&\phantom{\qquad|\,\,+2\cdot\text{(I)}}\\
(II)&&&&y&+&4z&=&11&\\
(III)&&\phantom{2x}&&\textcolor{#CC6600}{y}&+&2z&=&5&\qquad|\,\,-\text{(II)}
\end{mtable}
\end{displaymath}
\lang{de}{
Schlie{\ss}lich erhalten wir eine Stufenform, indem man die zweite Gleichung von der dritten abzieht.
}
\lang{en}{
Lastly, we arrive in row echelon form by subtracting the second equation from the third.
}
\begin{displaymath}
\begin{mtable}[\cellaligns{ccrcrcrcrl}]
(I)&\qquad-&x&+&2y&+&3z&=&5&\phantom{\qquad|\,\,+2\cdot\text{(I)}}\\
(II)&&&&y&+&4z&=&11&\\
(III)&&\phantom{2x}&&&-&2z&=&-6&
\end{mtable}
\end{displaymath}
\lang{de}{
Das LGS kann nun wie unter Beispiel \ref{gauss1} beschrieben gel\"ost werden.
\\\\
Alternativ können wir noch weitere elementare Zeilenumformungen machen, um das LGS auf reduzierte 
Stufenform zu bringen:
}
\lang{en}{
The linear system can be solved just as in example \ref{gauss1}.
\\\\
Alternatively we can manipulate the system further, in order to bring it into reduced row echelon 
form.
}

\[\begin{mtable}[\cellaligns{ccrcrcrcrl}]
(I)&\qquad\textcolor{#CC6600}{-}&x&+&2y&+&3z&=&5&\qquad|\,\,\cdot (-1)\\
(II)&&&&y&+&4z&=&11&\qquad\\
(III)&&\phantom{2x}&&&\textcolor{#CC6600}{-}&\textcolor{#CC6600}{2}z&=&-6& \qquad|\,\,\cdot (-\frac{1}{2})  
\end{mtable}\]
\lang{de}{
Durch das Dividieren der entsprechenden Zeilen durch ihre Stufenelemente erhält man für diese 
jeweils den Wert $1$.
}
\lang{en}{
By dividing the each row by its pivot element, we obtain rows with pivot element $1$.
}
 \[ \begin{mtable}[\cellaligns{ccrcrcrcrl}]
(I)&&x&-&2y&\textcolor{#CC6600}{-}&\textcolor{#CC6600}{3z}&=&-5&\qquad|\,\,+3\text{(III)}\\
(II)&&&&y&\textcolor{#CC6600}{+}&\textcolor{#CC6600}{4z}&=&11&\qquad|\,\,-4\text{(III)}\\
(III)&&&&&&z&=&3&  
\end{mtable} \]
\lang{de}{
Anschließend wird die Variable $z$ durch ein entsprechendes Vielfaches der dritten Gleichung in der 
ersten und zweiten Gleichung eliminiert.
}
\lang{en}{
Finally we eliminate the variable $z$ in the first and second equations by adding an appropriate 
multiple of the third equation to each of them.
}
 \[ \begin{mtable}[\cellaligns{ccrcrcrcrl}]
(I)&&x&\textcolor{#CC6600}{-}&\textcolor{#CC6600}{2y}&&&=&4&\qquad|\,\,+2\text{(II)}\\
(II)&&&&y&&&=&-1&\qquad\\
(III)&&&&&&z&=&3&  
\end{mtable} \]
\lang{de}{
Um die reduzierte Stufenform zu erhalten, muss schließlich nur noch die Variable $y$ in der ersten 
Gleichung eliminiert werden.
}
\lang{en}{
To obtain the reduced row echelon form, we still need to eliminate the variable $y$ in the first 
equation.
}
 \[ \begin{mtable}[\cellaligns{ccrcrcrcrl}]
(I)&&x&&&&&=&2&\qquad\\
(II)&&&&y&&&=&-1&\qquad\\
(III)&&&&&&z&=&3&  
\end{mtable} \]
\lang{de}{Daraus lässt sich direkt die Lösung ablesen:}
\lang{en}{From this we can immediately read the solution:}
\[ \mathbb{L}=\left\{\begin{pmatrix}2\\ -1\\ 3\end{pmatrix} \right\}. \]
\tab{\lang{de}{2. Beispiel}\lang{en}{Example 2}}
\[ \begin{mtable}[\cellaligns{crcrcrl}]
(I)&\qquad 2 \cdot  x & - & 4 \cdot  y & = & 10&  \\
(II)&\textcolor{#CC6600}{-3 \cdot x} & + & 6\cdot y & =  & -15&\qquad|\,\, +\frac{3}{2}\cdot\text{(I)}
\end{mtable} \]
\lang{de}{
Addiert man das $\frac{3}{2}$-fache der ersten Gleichung auf die zweite Gleichung, so ist man schon 
bei der Stufenform angelangt.
}
\lang{en}{
Adding a $\frac{3}{2}$-times the first equation to the second equation already brings us to row 
echelon form.
}
\[ \begin{mtable}[\cellaligns{crcrcrl}]
(I)&\qquad 2 \cdot  x & - & 4 \cdot  y & = & 10& \\
(II)& &  & 0 & =  & 0.&\phantom{\qquad|\,\, +\frac{3}{2}\cdot\text{(I)}}
\end{mtable} \]
\lang{de}{
Anschließend löst man das LGS wie in Beispiel \ref{gauss1}.
Alternativ erhält man die reduzierte Stufenform, indem man die erste Gleichung durch $2$ teilt.
}
\lang{en}{
Finally we can solve the linear system like in example \ref{gauss1}.
Alternatively we can obtain the reduced row echelon form by dividing the first equation by $2$.
}
\tab{\lang{de}{3. Beispiel}\lang{en}{Example 3}}
\lang{de}{Im LGS}
\lang{en}{In the linear system}
\[  \begin{mtable}[\cellaligns{crcrcrcrcrl}]
(I)&\qquad x_1 &+& 2x_2 &+ &\phantom{3}x_3 &-  &x_4 & = & 7 & \phantom{\qquad|\,\, -2\cdot \text{(II)}}\\
(II)&  \textcolor{#CC6600}{x_1} &+ &3x_2 &  &    & -  & x_4 & = & 8&\qquad|\,\, -\text{(I)} \\
(III)& \textcolor{#CC6600}{-x_1}  &  &      &  -   &3x_3 &+   &5x_4  & = & -9& \qquad|\,\, +\text{(I)}
\end{mtable} \]
\lang{de}{eliminieren wir zunächst die Variable $x_1$ aus der zweiten und dritten Gleichung.}
\lang{en}{we eliminate first the variably $x_1$ from the second and third equations.}
% \begin{center}
%  ______________________________________________________________
%  \end{center}
\[  \begin{mtable}[\cellaligns{crcrcrcrcrl}]
(I)&\qquad \phantom{+}x_1 &+& 2x_2 &+ &\phantom{3}x_3 &-  &x_4 & = & 7 & \\
(II)&   & &x_2 & - &  x_3  &   & & = & 1 & \\
(III)&  &  &   \textcolor{#CC6600}{2x_2}   &  -   &2x_3 &+   &4x_4  & = & -2& \qquad|\,\, -2\cdot \text{(II)}
\end{mtable} \]
\lang{de}{Anschließend eliminieren wir $x_2$ aus der dritten Gleichung}
\lang{en}{Finally we eliminate $x_2$ from the third equation}
% \begin{center}
%  ______________________________________________________________
%  \end{center}
\[  \begin{mtable}[\cellaligns{crcrcrcrcrl}]
(I)&\qquad \phantom{+}x_1 &+& 2x_2 &+ &\phantom{3}x_3 &-  &x_4 & = & 7 & \phantom{\qquad|\,\, -2\cdot \text{(II)}} \\
(II)&  & &x_2 & - &  x_3  &   & & = & 1 & \\
(III)&  &  &    &     & &   &4x_4  & = & -4& 
\end{mtable} \]
\lang{de}{
und sind bei einer Stufenform angelangt. 
\\\\
Um zur reduzierten Stufenform zu gelangen, müssen wir noch die Variable $x_4$, die ja die erste 
Variable in der dritten Gleichung ist, aus der ersten Gleichung eliminieren, ebenso die Variable 
$x_2$. Dafür ist es praktisch, zunächst die dritte Gleichung durch $4$ zu teilen:
}
\lang{en}{
and arrive at row echelon form.
\\\\
To get to the reduced row echelon form, we still need to eliminate the variable $x_4$, which is the 
first variable to appear in the third equation, from the first equation. Likewise, we need to 
eliminate the variable $x_2$ from the first equation. In order to achieve this, it makes sense to 
divide the third equation by $4$:
}
\[  \begin{mtable}[\cellaligns{crcrcrcrcrl}]
(I)&\qquad \phantom{+}x_1 &+& 2x_2 &+ &\phantom{3}x_3 &\textcolor{#CC6600}{-}  &\textcolor{#CC6600}{x_4} & = & 7 & \qquad|\,\,+ \text{(III)} \\
(II)&   & &x_2 & - &  x_3  &   & & = & 1 & \\
(III)&  &  &    &     & &   &\phantom{4}x_4  & = & -1& \phantom{\qquad|\,\, -2\cdot \text{(II)}}
\end{mtable} \]
\lang{de}{Dann die dritte Gleichung zur ersten addieren}
\lang{en}{Then we add the third equation to the first}
\[  \begin{mtable}[\cellaligns{crcrcrcrcrl}]
(I)&\qquad \phantom{+}x_1 &+& \textcolor{#CC6600}{2x_2} &+ &\phantom{3}x_3 &\phantom{-}  & & = & 6 & \qquad|\,\,-2\cdot \text{(II)} \\
(II)&   & &x_2 & - &  x_3  &   & & = & 1 & \\
(III)&  &  &    &     & &   &\phantom{4}x_4  & = & -1& \phantom{\qquad|\,\, -2\cdot \text{(II)}}
\end{mtable} \]
\lang{de}{und zuletzt das Doppelte der zweiten Gleichung von der ersten subtrahieren:}
\lang{en}{and finally subtract twice the second equation from the first:}
\[  \begin{mtable}[\cellaligns{crcrcrcrcrl}]
(I)&\qquad \phantom{+}x_1 &\phantom{+}& \phantom{2x_2} &+ &3x_3 &\phantom{-}  & & = & 4 &  \\
(II)&   & &x_2 & - &  x_3  &   & & = & 1 & \\
(III)&  &  &    &     & &   &\phantom{4}x_4  & = & -1& \phantom{\qquad|\,\, -2\cdot \text{(II)}}
\end{mtable} \]
\lang{de}{
Durch Setzen von $x_3=t$ und Auflösen der ersten beiden Gleichungen nach $x_1$ bzw. $x_2$ erhält man 
direkt als Lösungsmenge
}
\lang{en}{
By setting $x_3=t$ and solving the first two equations for $x_1$ and $x_2$ respectively, we obtain 
the solution set
}
\[ \mathbb{L}
= \left\{ \begin{pmatrix} 4-3t \\ 1+t \\t \\ -1\end{pmatrix} \, \big| \, t\in \R \right\}
= \left\{ \begin{pmatrix} 4 \\ 1 \\ 0 \\ -1\end{pmatrix}+ t\cdot \begin{pmatrix} -3\\ 1\\ 1\\ 0 \end{pmatrix} \, \big| \, t\in \R \right\} \]
\lang{de}{(vgl. Beispiel \ref{gauss1}).}
\lang{en}{(as we did in example \ref{gauss1}).}
\end{tabs*}
\end{example}


\section{\lang{de}{Gauß-Verfahren mit Matrizen}
         \lang{en}{Gaussian elimination in a matrix}}\label{sec:gauss-mit-matrizen}

\lang{de}{
Für das Lösen eines linearen Gleichungssystems $Ax=b$ mit dem Gau"s-Verfahren kann man auch 
einfacher die entsprechenden Umformungen auf die Zeilen der 
\ref[LGS][erweiterten Koeffizientenmatrix]{def:koeffizientenmatrix} $(A\, |\, b)$ anwenden und aus 
der resultierenden Matrix wieder das umgeformte Gleichungssystem in reduzierter Stufenform bilden. 
Um das Verfahren auf die Matrizen zu übertragen, ist lediglich zu beachten, dass das 
Eliminieren/Nicht-Auftreten von Variablen bedeutet, dass der entsprechende Koeffizient in der Matrix 
$0$ wird/ist.
}
\lang{en}{
In order to solve a linear system $Ax=b$ using Gaussian elimination, we may simply apply the 
appropriate row operations to the rows of the 
\ref[LGS][augmented matrix]{def:koeffizientenmatrix} $(A\, |\, b)$ associated with the linear 
system, and read the resulting matrix as a system of linear equations once more, this time in 
reduced row echelon form. In applying the algorithm to a matrix it is clear that a variable not 
appearing in a row corresponds to the corresponding matrix entry being $0$.
}

\begin{definition} \label{def:Gauß-Verfahren}
\lang{de}{
Für das Gauß-Verfahren können folgende sogenannte \notion{elementare Umformungen} auf die Zeilen 
einer Matrix angewendet werden:
}
\lang{en}{
So-called \notion{elementary row operations} may be applied to the rows of a matrix when performing 
Gaussian elimination:
}
\begin{itemize}
\item \lang{de}{Vertauschen zweier Zeilen.}
      \lang{en}{Exchanging/swapping two rows.}
\item \lang{de}{Multiplikation einer Zeile mit einer Konstanten $c\neq 0$.}
      \lang{en}{Multiplication of a row by a number $c\neq0$.}
\item \lang{de}{Addition (Subtraktion) eines $c$-fachen einer Zeile zu (von) einer anderen.}
      \lang{en}{Addition (or subtraction) of one multiple of a row to (from) another.}
\end{itemize}
\end{definition}

\lang{de}{Bevor wir das Gauß-Verfahren betrachten, zunächst einige \textit{Vorbemerkungen}:}
\lang{en}{Before considering the Gaussian elimination algorithm, we make some \textit{remarks}:}
\begin{itemize}
    \item \lang{de}{Zu Beginn betrachten wir die gesamte erweiterte Koeffizientenmatrix.}
          \lang{en}{We initially consider the entire augmented matrix that represents the system.}
    \item \lang{de}{
          Die Schritte des Verfahrens werden wiederholt auf einer Teilmatrix durchgeführt. Wenn im 
          Folgenden z.\,B. von der ersten Zeile die Rede ist, dann ist hiermit immer die erste Zeile 
          der aktuell betrachteten Teilmatrix gemeint.
          }
          \lang{en}{
          The steps of the algorithm are repeated on a sub-matrix. For example, when we refer to the 
          first row in the following description, we mean the first row of the sub-matrix currently 
          being considered.
          }
    \item \lang{de}{
          Wir schreiben $(K)$ für alle Einträge der $k$-ten Zeile, also 
          $a_{k1}, a_{k2}, \ldots, a_{kn}$.
          \\\\
          Handschriftlich adressiert man die Zeilen meist mit römischen Zahlen: (I), (II), $\ldots$
          }
          \lang{en}{
          We take $(K)$ to mean the elements of the $k$th row, that is, 
          $a_{k1}, a_{k2}, \ldots, a_{kn}$.
          \\\\
          Roman numerals $(I)$, $(II)$, $\ldots$ are usually used to represent the rows when doing 
          the algorithm by hand.
          }
\end{itemize}
\begin{rule}[\lang{de}{Gauß-Verfahren Matrizen}\lang{en}{Gaussian elimination with matrices}] 
\begin{enumerate}
\item[1.] \lang{de}{
          Suche die erste Spalte, die auch Einträge ungleich $0$ enthält. Den Index dieser Spalte 
          bezeichnen wir nun mit $j$.
          }
          \lang{en}{
          Find the first column containing a non-zero entry. We denote the index of this column by 
          $j$.
          }
\item[2.] \lang{de}{
          Vertausche die Zeilen der Matrix ggf. so, dass in dieser Spalte der erste Koeffizient 
          $a_{kj}$ nicht gleich $0$ ist.\\
          Dieser Koeffizient entspricht dem Stufenelement dieser Zeile.
          }
          \lang{en}{
          Swap the rows of the matrix such that the first entry $a_{kj}$ of this column is 
          non-zero.\\
          This entry is now the pivot element of the first row $(K)$.
          }
%\item Teile die erste Zeile durch den Koeffizienten der ersten Variablen.
\item[3.] \lang{de}{
          Addiere jeweils geeignete Vielfache der ersten Zeile $(K)$ zu den darunter liegenden 
          Zeilen $(L)$, so dass alle anderen Koeffizienten dieser Spalte gleich $0$ werden:\\
          Für alle $l$ mit  $k<l$ forme die Zeile $(L)$ zu $-\frac{a_{lj}}{a_{kj}}(K)+(L)$ um.
          }
          \lang{en}{
          Add an appropriate multiple of the first row $(K)$ to each other row $(L)$ beneath it, 
          such that every other coefficient of this column becomes equal to $0$:\\
          for all $l$ with $k<l$ replace the row $(L)$ with $-\frac{a_{lj}}{a_{kj}}(K)+(L)$.
          }
\item[4.] \lang{de}{
          Die erste Zeile der aktuellen Teilmatrix wird nun nicht mehr verändert. Wir führen das 
          Gauß-Verfahren ab Schritt 1 für eine neue Teilmatrix fort: Diese besteht aus der zweiten 
          bis zur letzten Zeile der aktuellen Teilmatrix.
          \\\\
          Das Verfahren wird beendet, wenn nur noch eine Zeile übrig ist oder die linken Seiten der 
          restlichen Zeilen alle gleich 0 sind.
          }
          \lang{en}{
          The first row of the current sub-matrix in consideration will now remain unchanged. We 
          repeat the above steps on the new sub-matrix formed by only considering the remaining 
          rows, starting from step 1. That is, we repeat the above steps for the matrix comprising 
          the second row to the last row of the previous matrix.
          \\\\
          The algorithm ends when only one row remains, or the left ('coefficient') sides of the 
          remaining rows only contain entries equal to $0$.
          }
\end{enumerate}
\lang{de}{
Die erweiterte Koeffizientenmatrix liegt nun in Stufenform vor.
\\\\
Um eine reduzierte Stufenform zu erhalten, verfährt man nach dieser Umformung weiter:
}
\lang{en}{
The augmented matrix is now in row echelon form.
\\\\
To obtain a reduced row echelon form, we continue as follows:
}
\begin{enumerate}
\item[5.] \lang{de}{
          Bringe jedes Stufenelement auf den Wert 1:\\
          Teile dazu jede Zeile, bei der die linke Seite nicht komplett $0$ ist, durch ihren 
          Stufeneintrag:\\
          Für alle entsprechenden $l$ mit $l\geq 1$ forme die Zeile $(L)$ zu $\frac{1}{a_{lj}}(L)$ 
          um.
          }
          \lang{en}{
          Change every pivot element to $1$.\\
          Divide each row whose left side has non-zero entries by its pivot element:\\
          for all $l$ with $l\geq 1$ replace the row $(L)$ with $\frac{1}{a_{lj}}(L)$.
          }
\item[6.] \lang{de}{
          Bringe alle Koeffizienten über einem Stufenelement auf den Wert 0.\\
          Addiere dazu geeignete Vielfache der in 5. genannten Zeilen $(L)$ zu den darüber liegenden 
          Zeilen $(K)$.\\
          Für alle $k$ mit $1\leq k < l$ forme die Zeile $(K)$ zu $-a_{kj}(L)+(K)$ um.
          }
          \lang{en}{
          Change all entries above a pivot element to $0$.\\
          Do this by adding an appropriate multiple of each row $(L)$ from 5. to each row $(K)$ 
          above it:\\
          for all $k$ with $1\leq k < l$ replace the row $(K)$ with $-a_{kj}(L)+(K)$.
          }
\end{enumerate}
\lang{de}{
\floatright{\href{https://api.stream24.net/vod/getVideo.php?id=10962-2-10842&mode=iframe&speed=true}{\image[75]{00_video_button_schwarz-blau}}}\\
}
\lang{en}{}
\end{rule}

\begin{remark}
\lang{de}{
Auch hier werden beim handschriftlichen Lösen oft Zeilen im zweiten Schritt derart getauscht, dass 
die numerische Rechnung einfacher wird. Ebenso werden häufig in Schritt 3 die beiden zu addierenden 
Zeilen geeignet multipliziert, um beispielsweise Brüche zu vermeiden.
}
\lang{en}{
Here too we often simplify calculation of the algorithm by hand, especially in the second step, 
where rows may be swapped in a more convenient way, and in the third step, where a multiple of row 
$(L)$ may be considered in order to avoid fractions.
}
\end{remark}

\begin{example}
\lang{de}{Für das LGS}
\lang{en}{The linear system}
\[  \begin{mtable}[\cellaligns{crcrcrcrcrl}]
(I)&\qquad x_1 &+& 2x_2 &+ &\phantom{3}x_3 &-  &x_4 & = & 7 & \phantom{\qquad|\,\, -2\cdot \text{(II)}}\\
(II)&  x_1  &+ &3x_2 &  &    & -  & x_4 & = & 8& \\
(III)& -x_1  &  &      &  -   &3x_3 &+   &5x_4  & = & -9& 
\end{mtable} \]
\lang{de}{erhalten wir die folgende erweiterte Koeffizientenmatrix:}
\lang{en}{has the following coefficient matrix:}
\[
\begin{pmatrix}
    1 & 2 & 1 & -1 & | & 7\\ 
    \textcolor{#CC6600}{1} & 3 & 0 & -1  &| & 8\\
    \textcolor{#CC6600}{-1} & 0 & -3 & 5 &| &  -9
 %    \rowops \add[-1]01 \add[+1]02
   \end{pmatrix}
   ~
\begin{matrix}
    \phantom{}\\
    |-(I) + (II)\\
    |(I) + (III)
\end{matrix}\]
\lang{de}{
Die erste Spalte enthält Koeffizienten ungleich $0$, sogar in  der ersten Zeile, weshalb wir keine 
Zeilen tauschen müssen. Subtrahieren der ersten Zeile von der zweiten und Addieren der ersten Zeile 
auf die dritte liefert
}
\lang{en}{
The first column contains non-zero entries, even in the first row, so no rows need to be swapped. 
We subtract the first row from the second and add the first row to the third, which yields
}
\[    \rightsquigarrow~~
\begin{pmatrix}
    1 & 2 & 1 & -1 &| & 7\\ 
    0 & 1 & -1& 0  &| & 1\\
    0 & 2 & -2 & 4 & | &  -2
 %    \rowops \add[-2]12
\end{pmatrix}.\]
\lang{de}{
Damit ist der erste Durchlauf fertig und man fährt genauso mit der Matrix aus der zweiten und 
dritten Zeile fort, schreibt aber weiterhin die erste Zeile mit:
}
\lang{en}{
The first iteration of Gaussian elimination is hence complete, and we continue on the matrix 
made of the second and third rows of the current matrix. We still write the first row, but leave it 
unchanged:
}

\[    \rightsquigarrow~~
\begin{pmatrix}
    \textcolor{gray}{1} & \textcolor{gray}{2} & \textcolor{gray}{1} & \textcolor{gray}{-1} &\textcolor{gray}{|} & \textcolor{gray}{7}\\ 
    0 & 1 & -1& 0  &| & 1\\
    0 & \textcolor{#CC6600}{2} & -2 & 4 & | &  -2
 %    \rowops \add[-2]12
   \end{pmatrix}
   ~
\begin{matrix}
    \phantom{}\\
    \phantom{}\\
    |(-2)(II) + (III)
\end{matrix}\]

\lang{de}{
Betrachtet man also nur die kleinere Matrix, enthält die erste Spalte nur $0$, aber die zweite 
nicht. Auch hier sind keine Zeilen zu tauschen. Dann wird das Doppelte der zweiten Zeile von der 
dritten subtrahiert
}
\lang{en}{
If we now consider only the smaller matrix, then the first column only contains $0$, but the second 
does not. Again we do not need to swap any rows. We subtract twice the second row from the third row,
}
\[     \rightsquigarrow~~
\begin{pmatrix}
    \textcolor{gray}{1} & \textcolor{gray}{2} & \textcolor{gray}{1} & \textcolor{gray}{-1} &\textcolor{gray}{|} & \textcolor{gray}{7}\\ 
    0 & 1 & -1  & 0&| & 1\\
    0 & 0 & 0 &4 &| &   -4
% \rowops \mult[\frac{1}{4}]2   
   \end{pmatrix}, \]
\lang{de}{
wonach auch der zweite Durchlauf fertig ist und man die Stufenform erreicht hat.
\\\\
Die Stufeneinträge sind im Folgenden blau markiert:
}
\lang{en}{
at which point we have completed the second iteration of the algorithm and attained row echelon form.
\\\\
If we mark the pivot elements in blue, we get:
}
\[    \rightsquigarrow~~
\begin{pmatrix}
    \textcolor{#0066CC}{1} & {2} & {1} &{-1} &{|} & {7}\\ 
    0 & \textcolor{#0066CC}{1} & -1  & 0&| & 1\\
    0 & 0 & 0 &\textcolor{#0066CC}{4} &| &   -4
% \rowops \mult[\frac{1}{4}]2   
   \end{pmatrix}
   ~
\begin{matrix}
    \phantom{}\\
    \phantom{}\\
    |\cdot \frac{1}{4} (III)
\end{matrix}\]

\lang{de}{Für die reduzierte Stufenform teilen wir also die dritte Zeile durch $4$}
\lang{en}{In order to obtain the reduced row echelon form, we divide the third row by $4$,}
\[   
   \rightsquigarrow~~\begin{pmatrix}
    \textcolor{#0066CC}{1} & 2 & 1 & \textcolor{#CC6600}{-1} &| & 7\\ 
    0 & \textcolor{#0066CC}{1} & -1  & 0&| & 1\\
    0 & 0 & 0 &\textcolor{#0066CC}{1} &| &   -1
   \end{pmatrix}
   ~
\begin{matrix}
    |(III)+(I)\\
    \phantom{}\\
    \phantom{}
\end{matrix}\]
\lang{de}{
und addieren die dritte Zeile auf die erste Zeile, um die rote $1$ über dem letzten Stufeneintrag zu 
eliminieren
}
\lang{en}{
and add the third row to the first row, to eliminate the red $1$ above the final pivot point,
}
\[     \rightsquigarrow~~
\begin{pmatrix}
    \textcolor{#0066CC}{1} & \textcolor{#CC6600}{2} & 1 & 0 &| & 6\\ 
    0 & \textcolor{#0066CC}{1} & -1  & 0&| & 1\\
    0 & 0 & 0 &\textcolor{#0066CC}{1} &| &   -1
%     \rowops \add[-2]10
   \end{pmatrix}
   ~
\begin{matrix}
    |(-2)(II)+(I)\\
    \phantom{}\\
    \phantom{}
\end{matrix}\]
\lang{de}{
und subtrahieren schließlich das Doppelte der zweiten Zeile von der ersten, um auch die rote $2$ zu 
eliminieren:
}
\lang{en}{
Finally, we subtract twice the second row from the first row, in order to also eliminate the red $2$.
}
\[     \rightsquigarrow~~
\begin{pmatrix}
    \textcolor{#0066CC}{1} & 0 & 3 & 0 &| & 4\\ 
    0 & \textcolor{#0066CC}{1} & -1  & 0&| & 1\\
    0 & 0 & 0 &\textcolor{#0066CC}{1} &| &   -1
   \end{pmatrix}
\]
\lang{de}{Als Gleichungssystem ist dies dann}
\lang{en}{This augmented matrix represents the linear system}
\[  \begin{mtable}[\cellaligns{crcrcrcrcrl}]
(I)&\qquad \phantom{+}x_1 &\phantom{+}& \phantom{2x_2} &+ &3x_3 &\phantom{-}  & & = & 4 &  \\
(II)&   & &x_2 & - &  x_3  &   & & = & 1 & \\
(III)&  &  &    &     & &   &\phantom{4}x_4  & = & -1&
\end{mtable} \]
\lang{de}{
und daraus lässt sich die Lösungsmenge direkt ablesen, indem man $x_3=t$ mit $t \in \R$ setzt:
}
\lang{en}{
and from this we can immediately read the solution set, by setting $x_3=t$ with $t \in \R$.
}
\[ \mathbb{L}
= \left\{ \begin{pmatrix} 4-3t \\ 1+t \\t \\ -1\end{pmatrix} \, \big| \, t\in \R \right\}
= \left\{ \begin{pmatrix} 4 \\ 1 \\ 0 \\ -1\end{pmatrix}+ t\cdot \begin{pmatrix} -3\\ 1\\ 1\\ 0 \end{pmatrix} \, \big| \, t\in \R \right\} \]
\lang{de}{(vgl. Beispiel \ref{gauss1}).}
\lang{en}{(see example\ref{gauss1}).}
\end{example}
\end{visualizationwrapper}


\end{content}