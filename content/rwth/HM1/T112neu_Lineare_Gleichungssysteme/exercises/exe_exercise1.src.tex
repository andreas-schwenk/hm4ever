\documentclass{mumie.element.exercise}
%$Id$
\begin{metainfo}
  \name{
    \lang{de}{Ü01: Lösung LGS}
    \lang{en}{Exercise 1}
  }
  \begin{description} 
 This work is licensed under the Creative Commons License Attribution 4.0 International (CC-BY 4.0)   
 https://creativecommons.org/licenses/by/4.0/legalcode 

    \lang{de}{Hier die Beschreibung}
    \lang{en}{}
  \end{description}
  \begin{components}
  \end{components}
  \begin{links}
  \end{links}
  \creategeneric
\end{metainfo}
\begin{content}
\begin{block}[annotation]
	Im Ticket-System: \href{https://team.mumie.net/issues/21352}{Ticket 21352}
\end{block}
\begin{block}[annotation]
Copy of \href{http://team.mumie.net/issues/9570}{Ticket 9570}: content/rwth/HM1/T111_Matrizen,_lineare_Gleichungssysteme/exercises/exe_exercise1.src.tex
\end{block}

\title{
  \lang{de}{Ü01: Lösung LGS}
  \lang{en}{Exercise 1}
}

\begin{block}[annotation]
	Matritzen, Lineare Gleichungen: Übungen
\end{block}

\lang{de}{Prüfen Sie, ob das jeweils angegebene Zahlenpaar eine Lösung des linearen Gleichungssystems
\begin{displaymath} 
\begin{mtable}[\cellaligns{ccrcrcr}]
(I)&\qquad&3x&+&4y&=&8\\
(II)&\qquad&-2x&+&3y&=&-11
\end{mtable}
\end{displaymath}   
ist.}
\lang{en}{Check to see which of the following pairs of numbers are a solution to the linear system:
\begin{displaymath} 
\begin{mtable}[\cellaligns{ccrcrcr}]
(I)&\qquad&3x&+&4y&=&8\\
(II)&\qquad&-2x&+&3y&=&-11
\end{mtable}
\end{displaymath}   
}
\begin{table}[\class{items}]
  \nowrap{(a) $ \ (4\lang{de}{;}\lang{en}{,} -1)$} & \nowrap{(b) $ \ (-4\lang{de}{;}\lang{en}{,} 5)$}\\
  \nowrap{(c) $ \ (1\lang{de}{;}\lang{en}{,} -3)$} & \nowrap{(d) $ \ (2\lang{de}{;}\lang{en}{,} 2)$}
\end{table}

\begin{tabs*}[\initialtab{0}\class{exercise}]
  \tab{
  \lang{de}{Antwort}
  \lang{en}{Answer}
  }
  \begin{table}[\class{items}]
    (a) Ja. & (b) Nein.\\
    (c) Nein. & (d) Nein.
  \end{table}

  \tab{
  \lang{de}{Lösung}
  \lang{en}{Solution}
  }
  \begin{incremental}[\initialsteps{1}]
    \step \lang{de}{Ein Zahlenpaar ist genau dann eine Lösung des LGS, wenn es beide Gleichungen löst. }
    \lang{en}{In order to check whether a pair of numbers is a solution to the linear system, we need to substitute the values for
    $x$ and $y$ into both equations and see whether or not both equations are fulfilled (if both equations claim true statements).}
     
    \step \lang{de}{Setzt man beispielsweise das Zahlenpaar aus (a) in die erste Gleichung ein, so erhält man 
    \[3\cdot 4+4\cdot(-1)=12-4=8.\]
    Es erfüllt also diese Gleichung.\\
    Ebenso erfüllt das Zahlenpaar aus (b) die erste Gleichung.
    
    Beim Einsetzen des Zahlenpaars aus (c) in die erste Gleichung kommt man auf 
    \[3\cdot 1+4\cdot(-3)=3-12=-9\neq 8.\]
    Es erfüllt also die erste Gleichung nicht.\\
    Auch das Zahlenpaar aus (d) erfüllt die erste Gleichung nicht, somit scheiden die Zahlenpaare aus (c) und (d) als mögliche Lösung aus.}
    
    \lang{en}{For example, if we take the pair of numbers from (a) and substitute it into the first equation, we get $3\cdot 4+4\cdot(-1)=12-4=8$.
    This pair satisfies this equation.\\
    The pair from (b) also satisfies the first equation.
    
    By inserting the pair from (c) into the first equation, we get $3\cdot 1+4\cdot(-3)=3-12=-9$, hence it doesn't satisfy the first equation.\\
    The pair from (d) doesn't satisfy the first equation, hence we can exclude the pairs from (c) and (d) as being possible solutions.}
    
    \step \lang{de}{Setzt man das Zahlenpaar aus (a) nun in die zweite Gleichung ein, so erhält man 
    \[-2\cdot 4+3\cdot(-1)=-8-3=-11.\]
    Es erfüllt folglich auch diese Gleichung und ist somit eine Lösung des linearen Gleichungssystems.
    
    Mit dem Einsetzen des Zahlenpaars aus (b) in die zweite Gleichung ergibt sich
    \[-2\cdot(-4)+3\cdot 5=8+15=23\neq -11 ,\] 
    es erfüllt die zweite Gleichung
    also nicht und ist somit auch keine Lösung des linearen Gleichungssystems.
    
    (Das Zahlenpaar aus (c) würde zwar die zweite Gleichung erfüllen, ist aber schon als Lösung ausgeschlossen worden.\\
    Das Zahlenpaar aus (d) erfüllt auch die zweite Gleichung nicht.)}
 
    
    \lang{en}{If we now substitute the pair from (a) into the second equation, we get $-2\cdot 4+3\cdot(-1)=-8-3=-11$. It fulfills the second
    equation and is a solution to the linear system. 
    
    After substituting the pair from (b) into the second equation we get $-2\cdot(-4)+3\cdot 5=8+15=23$, which doesn't satisfy the second equation, and hence
    isn't a solution to the linear system.
    
    [The pair from (c) would have satisfied the second equation, but we've already excluded it as a solution.\\
    The pair from (d) doesn't fulfill the second equation.]}
    
    \step \lang{de}{Alternativ lässt sich die Lösung mit einem der bekannten Verfahren berechnen. Anschließend vergleicht man diese Lösung mit den angebenen Zahlenpaaren.}
    
  \end{incremental}
\end{tabs*}
\end{content}