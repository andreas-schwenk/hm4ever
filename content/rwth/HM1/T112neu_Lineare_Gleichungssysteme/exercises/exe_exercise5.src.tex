\documentclass{mumie.element.exercise}
%$Id$
\begin{metainfo}
  \name{
    \lang{de}{Ü05: Parabelgleichung}
    \lang{en}{Exercise 5}
  }
  \begin{description} 
 This work is licensed under the Creative Commons License Attribution 4.0 International (CC-BY 4.0)   
 https://creativecommons.org/licenses/by/4.0/legalcode 

    \lang{de}{Hier die Beschreibung}
    \lang{en}{}
  \end{description}
  \begin{components}
  \end{components}
  \begin{links}
  \end{links}
  \creategeneric
\end{metainfo}
\begin{content}
\begin{block}[annotation]
	Im Ticket-System: \href{https://team.mumie.net/issues/21356}{Ticket 21356}
\end{block}
\begin{block}[annotation]
Copy of \href{http://team.mumie.net/issues/9577}{Ticket 9577}: content/rwth/HM1/T111_Matrizen,_lineare_Gleichungssysteme/exercises/exe_exercise8.src.tex
\end{block}

\title{
  \lang{de}{Ü05: Parabelgleichung}
  \lang{en}{Exercise 5}
}

\begin{block}[annotation]
	Matritzen, Lineare Gleichungen: Übungen
\end{block}

\lang{de}{
\begin{itemize}
\item[(a)]
Durch die Punkte $P=(-1; -4)$, $Q=(1; 6)$ und $R=(3; 0)$ verläuft genau eine Parabel. Stellen Sie ein lineares Gleichungssystem
        für die Koeffizienten $a$, $b$ und $c$ der Parabelgleichung $y=ax^2+bx+c$ auf und bestimmen Sie dessen Lösungsmenge und damit die Gleichung der
        Parabel, die durch die Punkte $P$, $Q$ und $R$ verläuft.
\item[(b)]
Für welches Polynom $p$ dritten Grades gilt
$p(-1)=-3$, $p(0)=3$, $p(1)=1$ und $p(2)=3$?
\end{itemize}
}
\lang{en}{Exactly one parabola goes through the points $P=(-1, -4)$, $Q=(1, 6)$, and $R=(3, 0)$. Create a linear system for the coefficients $a$, $b$, and $c$ of the parabola
        $y=ax^2+bx+c$. Determine the solution set of the system and hence the equation of the parabola that goes through the points $P$, $Q$, and $R$.
}

\begin{tabs*}[\initialtab{0}\class{exercise}]
  \tab{
  \lang{de}{Antwort}
  \lang{en}{Answer}
  }
  \lang{de}{
\begin{table}[\class{items}]
(a)\\
 Das lineare Gleichungssystem für die Koeffizienten lautet
\begin{displaymath}
\begin{mtable}[\cellaligns{ccrcrcrcr}]
(I)&\qquad&a&-&b&+&c&=&-4\\
(II)&&a&+&b&+&c&=&6\\
(III)&&9a&+&3b&+&c&=&0
\end{mtable}
\end{displaymath}
und dessen Lösungsmenge ist
\begin{displaymath}
\mathbb{L}=\{(-2; 5; 3)\},
\end{displaymath}
d.\,h. die eindeutige Lösung des linearen Gleichungssystems für $a$, $b$ und $c$ ist $a=-2$, $b=5$ und $c=3$.\\
Die Gleichung der Parabel, die durch die Punkte $P$, $Q$ und $R$ verläuft, lautet somit $y=-2x^2+5x+3$.\\
\\
(b) 
\[p(x)=2x^3-4x^2+3\]

\end{table}}

  \lang{en}{
\begin{table}[\class{items}]
The linear system for the coefficients is
\begin{displaymath}
\begin{mtable}[\cellaligns{ccrcrcrcr}]
(I)&\qquad&a&-&b&+&c&=&-4\\
(II)&&a&+&b&+&c&=&6\\
(III)&&9a&+&3b&+&c&=&0
\end{mtable}
\end{displaymath}
and its solution set is
\begin{displaymath}
\mathbb{L}=\{(-2, 5, 3)\}.
\end{displaymath}
The unique solution of the linear system for $a$, $b$, and $c$ is $a=-2$, $b=5$ and $c=3$.\\
The equation of the parabola that goes through the points $P$, $Q$, and $R$ is $y=-2x^2+5x+3$.
\end{table}}

\tab{
  \lang{de}{Lösung (a)}
  \lang{en}{Solution}
  }
  
  \begin{incremental}[\initialsteps{1}]
    \step 
    \lang{de}{Da die drei Punkte $P$, $Q$ und $R$ auf einer Parabel mit der Gleichung $y=ax^2+bx+c$ liegen sollen, müssen ihre Koordinaten diese Gleichung
    jeweils erfüllen.}
    \lang{en}{Since the three points $P$, $Q$, and $R$ lie on the parabola with equation $y=ax^2+bx+c$, the coordinates of the points need to satisfy this equation.}
     
    \step \lang{de}{Setzt man $P=(-1;-4)$ in die Parabelgleichung ein, also $x=-1$ und $y=-4$, so folgt $-4=a-b+c$. Analog folgt durch Einsetzen von $Q=(1;6)$, dass $6=a+b+c$, und durch Einsetzen von $R=(3;0)$, dass $0=9a+3b+c$.
    
    Dies führt auf das folgende lineare Gleichungssystem für die Koeffizienten $a$, $b$ und $c$: }
    \lang{en}{If we substitute $P=(-1,-4)$ (i.e. $x=-1$ and $y=-4$) into the parabola, we get $-4=a-b+c$. Analogously we can substitute in $Q=(1,6)$ to get $6=a+b+c$ and $R=(3,0)$ to get $0=9a+3b+c$.
    
    This leads to the the following linear system for the coefficients $a$, $b$, and $c$:}
    \begin{displaymath}
\begin{mtable}[\cellaligns{ccrcrcrcrc}]
(I)&\qquad&a&-&b&+&c&=&-4&\phantom{.}\\
(II)&&a&+&b&+&c&=&6&|-(I)\\
(III)&&9a&+&3b&+&c&=&0& |-(I)
\end{mtable}
\end{displaymath}
    
    \step \lang{de}{Die Lösungsmenge dieses linearen Gleichungssystems erhält man nun durch Anwendung des Gauß-Verfahrens.}
    \lang{en}{The solution set of the linear system is obtained via Gaussian Elimination.}
    \step \lang{de}{Multiplizieren Sie beispielsweise die erste Gleichung auf beiden Seiten mit $-1$ und addieren Sie sie dann zur zweiten bzw. dritten Gleichung, so dass in diesen $c$ nicht mehr vorkommt.}
    \lang{en}{Multiply both sides of the first equation by $-1$ and add the result to the second and third equation so that $c$ does not appear in either the second or third equation anymore.}
    \step \lang{de}{Das lineare Gleichungssystem sieht dann wie folgt aus:}
    \lang{en}{The linear system then looks as follows:}
    \begin{displaymath}
\begin{mtable}[\cellaligns{ccrcrcrcr}]
(I)&\qquad&a&-&b&+&c&=&-4\phantom{.}\\
(II)&&&&2b&&&=&10\phantom{.}\\
(III)&&8a&+&4b&&&=&4\lang{de}{}\lang{en}{\phantom{.}}
\end{mtable}
\end{displaymath}

    \step \lang{de}{Auch wenn das lineare Gleichungssystem noch nicht in Stufenform ist, lässt sich seine Lösungsmenge bereits jetzt ermitteln.\\
 Aus der zweiten Gleichung folgt nämlich sofort $b=5$. Einsetzen von $b$ in die dritte Gleichung ergibt dann $8a+20=4$ und somit $a=-2$. Zuletzt lässt sich $c$ durch 
    Einsetzen von $a$ und $b$ in die erste Gleichung bestimmen. Es ergibt sich $-2-5+c=-4$, woraus $c=3$ folgt.}
    
    %Anmerkung:\\
    %Um das lineare Gleichungssystem in Stufenform zu bringen, hätte man entweder noch in der dritten Gleichung $b$ eliminieren müssen oder aber auch einfach
    %die zweite und dritte Gleichung vertauschen können.}
    %\lang{en}{Even though the linear system is not in row echelon form, the solution set can already be determined.\\
    %From the second equation we get immediately that $b=5$. Substituting $b$ into the third equation results in $8a+20=4$ and hence $a=-2$. Lastly, $c$ can be determined by
    %substituting $a$ and $b$ into the first equation. This results in $-2-5+c=-4$ from which we get that $c=3$.
    
    %Note:\\
    %In order to put the linear system into row echelon form, we could have either eliminated $b$ from the third equation or swapped the second and third equation.}
    \step\lang{de}{Die Lösungsmenge des linearen Gleichungssystems ist somit 
    \[\mathbb{L}=\{(-2; 5; 3)\}\]
    und die Gleichung der Parabel, die durch die Punkte $P$, $Q$ und $R$ verläuft,
    lautet 
    \[y=-2x^2+5x+3.\]} 
    \lang{en}{The solution set of the linear system is $\mathbb{L}=\{(-2, 5, 3)\}$ and the equation of the parabola that goes through the points $P$, $Q$, and $R$ is $y=-2x^2+5x+3$.} 
  \end{incremental}
   \tab{\lang{de}{Lösungsvideo (b)}}
  \lang{de}{\youtubevideo[500][300]{9WDeeUdiiFc}}
\tab{
  \lang{de}{Hinweis}
  \lang{en}{Note}
  }

   \lang{de}{Beachten Sie, dass es auch andere Möglichkeiten gibt, das lineare Gleichungssystem mit dem Gauß-Verfahren umzuformen. Der aufgeführte Weg ist nur eine Variante. Außerdem ist die Stufenform nicht eindeutig, die Antworten beeinflusst dies aber nicht.} 
   \lang{en}{Note that there are other possibilities for reducing the linear system into row echelon form using Gaussian Elimination. The steps performed here are only one variant and the row echelon form of a linear system is not unique, but the answers aren't influenced by this at all.}

  
 \end{tabs*}
\end{content}