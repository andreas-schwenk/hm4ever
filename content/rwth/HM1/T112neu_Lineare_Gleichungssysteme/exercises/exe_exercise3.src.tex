\documentclass{mumie.element.exercise}
%$Id$
\begin{metainfo}
  \name{
    \lang{de}{Ü03: Gauß-Verfahren}
    \lang{en}{Exercise 3}
  }
  \begin{description} 
 This work is licensed under the Creative Commons License Attribution 4.0 International (CC-BY 4.0)   
 https://creativecommons.org/licenses/by/4.0/legalcode 

    \lang{de}{Hier die Beschreibung}
    \lang{en}{}
  \end{description}
  \begin{components}
  \end{components}
  \begin{links}
  \end{links}
  \creategeneric
\end{metainfo}
\begin{content}
\begin{block}[annotation]
	Im Ticket-System: \href{https://team.mumie.net/issues/21354}{Ticket 21354}
\end{block}
\begin{block}[annotation]
Copy of \href{http://team.mumie.net/issues/9575}{Ticket 9575}: content/rwth/HM1/T111_Matrizen,_lineare_Gleichungssysteme/exercises/exe_exercise6.src.tex
\end{block}

\title{
  \lang{de}{Ü03: Gauß-Verfahren}
  \lang{en}{Exercise 3}
}

\begin{block}[annotation]
	Matritzen, Lineare Gleichungen: Übungen
\end{block}

\lang{de}{Bestimmen Sie die Lösungsmenge der folgenden linearen Gleichungssysteme. Verwenden Sie dazu das Gauß-Verfahren.}
\lang{en}{Determine the solution set of each of the following linear systems. Use Gaussian Elimination.}
\begin{table}[\class{items}]
\nowrap{(a) \begin{displaymath}
\begin{mtable}[\cellaligns{ccrcrcrcr}]
(I)&&x_1&-&x_2&-&x_3&=&0\\
(II)&&x_1&+&x_2&+&3x_3&=&4\\
(III)&&2x_1&&&+&x_3&=&3
\end{mtable}
\end{displaymath}
}\\
\nowrap{(b) \begin{displaymath}
\begin{mtable}[\cellaligns{ccrcrcrcr}]
(I)&&2x_1&+&4x_2&+&2x_3&=&2\\
(II)&&x_1&+&2x_2&+&2x_3&=&2\\
(III)&&&&3x_2&-&2x_3&=&4
\end{mtable}
\end{displaymath}
}\\
\nowrap{(c) \begin{displaymath}
\begin{mtable}[\cellaligns{ccrcrcrcrcrcr}]
(I)&&x_1&+&x_2&-&x_3&+&x_4&-&x_5&=&1\\
(II)&-&x_1&+&x_2&-&3x_3&&&+&3x_5&=&2\\
(III)&-&2x_1&&&+&x_3&+&5x_4&+&4x_5&=&1\\
(VI)&&&&x_2&-&2x_3&+&x_4&-&x_5&=&-1\\
(V) &&2x_1&&&+&2x_3&+&x_4&-&2x_5&=&1\\
\end{mtable}
\end{displaymath}
}\\
  \nowrap{(d) \begin{displaymath}
\begin{mtable}[\cellaligns{ccrcrcrcr}]
(I)&\qquad&x&+&4y&+&6z&=&1\\
(II)&&2x&+&3y&+&7z&=&1\\
(III)&&3x&+&2y&+&8z&=&2
\end{mtable}
\end{displaymath}
}\\
  \nowrap{(e) \begin{displaymath}
\begin{mtable}[\cellaligns{ccrcrcrcr}]
(I)&\qquad&3x&-&2y&+&5z&=&8\\
(II)&&6x&+&5y&-&2z&=&-5\\
(III)&&9x&-&3y&-&z&=&-31
\end{mtable}
\end{displaymath}
}\\
  \nowrap{(f) \begin{displaymath}
\begin{mtable}[\cellaligns{ccrcrcrcr}]
(I)&\qquad&x&-&2y&+&4z&=&-6\\
(II)&&2x&-&2y&+&3z&=&0\\
(III)&&3x&-&4y&+&7z&=&-6
\end{mtable}
\end{displaymath}
}\\
\nowrap{(g) \begin{displaymath}
\begin{mtable}[\cellaligns{ccrcrcrcr}]
(I)&\qquad&x&+&2y&-&5z&=&3\\
(II)&&&&&&0&=&0\\
(III)&&&&&&0&=&0
\end{mtable}
\end{displaymath}
}
\end{table}

\begin{tabs*}[\initialtab{0}\class{exercise}]
  \tab{
  \lang{de}{Antwort}
  \lang{en}{Answer}
  }
\begin{table}[\class{items}]
    \nowrap{(a) $\mathbb{L}=\{(1;0;1)\}$}\\
    \nowrap{(b) $\mathbb{L}=\{(-4;2;1)\}$}\\
    \nowrap{(c) $\mathbb{L}=\{(0;5;2;-1;1)\}$}\\
    \nowrap{(d) $\mathbb{L}=\emptyset$}\\
    \nowrap{(e) $\mathbb{L}=\{(-2\lang{de}{;}\lang{en}{,}  3\lang{de}{;}\lang{en}{,}  4)\}$}\\
    \nowrap{(f) $\mathbb{L}=\{(6+t\lang{de}{;}\lang{en}{,} 6+\frac{5}{2}t\lang{de}{;}\lang{en}{,} t) \,|\, t\in\R\}$ \lang{de}{(Andere Darstellungen der Lösungsmenge sind möglich!)}\lang{en}{(Other representations of the 
    solution set are possible!)}}\\
    \nowrap{(g) $\mathbb{L}=\{(3-2s+5t\lang{de}{;}\lang{en}{,} s\lang{de}{;}\lang{en}{,} t) \,|\, s, t\in\R\}$ \lang{de}{(Andere Darstellungen der Lösungsmenge sind möglich!)}\lang{en}{(Other representations of the 
    solution set are possible!)}}
\end{table}
     \tab{\lang{de}{Lösungsvideo (a)-(c)}}
     
  \youtubevideo[500][300]{fXnWcT-qf6Q}\\
  \tab{
  \lang{de}{Lösung (d)}
  \lang{en}{Solution (d)}
  }
  
  \begin{incremental}[\initialsteps{1}]
    \step 
    \lang{de}{Um die Lösungsmenge des angegebenen linearen Gleichungssystems zu bestimmen, bringen Sie es zunächst mit dem Gauß-Verfahren auf Stufenform.}
    \lang{en}{In order to determine the solution set of a given linear system we need to first bring it into row echelon form using Gaussian Elimination.}
     
    \step \lang{de}{Multiplizieren Sie beispielsweise die erste Gleichung auf beiden Seiten mit $-2$ (bzw. $-3$) und addieren Sie sie dann zur zweiten  (bzw. dritten) Gleichung, so dass in diesen $x$ nicht mehr vorkommt.}
    \lang{en}{Multiply both sides of the first equation by $-2$ and add it to the second equation; multiply both sides of the first equation by $-3$ and add it to the third. In equations two and three, the variable $x$ won't appear anymore.}
    \step \lang{de}{Das lineare Gleichungssystem sieht dann wie folgt aus:}
    \lang{en}{The linear system then looks as follows:}
    \begin{displaymath}
\begin{mtable}[\cellaligns{ccrcrcrcrc}]
(I)&\qquad&x&+&4y&+&6z&=&1&\phantom{.}\\
(II)&&&-&5y&-&5z&=&-1&\phantom{.}\\
(III)&&&-&10y&-&10z&=&-1&|-2(II)+(III)\phantom{.}
\end{mtable}
\end{displaymath}
    
    \step \lang{de}{Im nächsten Schritt bietet es sich an, die zweite Gleichung mit $-2$ zu multiplizieren und dann zur dritten Gleichung zu addieren, damit dort schließlich $y$ nicht mehr vorkommt.\\
    Das lineare Gleichungssystem sieht dann wie folgt aus:}
    \lang{en}{In the next step it's easiest to multiply the second equation by $-2$ and then add it to the third equation so that the variable $y$ doesn't appear anymore.\\
    The linear system then looks as follows:}
    \begin{displaymath}
\begin{mtable}[\cellaligns{ccrcrcrcr}]
(I)&\qquad&x&+&4y&+&6z&=&1\phantom{.}\\
(II)&&&-&5y&-&5z&=&-1\phantom{.}\\
(III)&&&&&&0&=&1\lang{de}{}\lang{en}{\phantom{.}}
\end{mtable}
\end{displaymath}
    
    \step \lang{de}{Das lineare Gleichungssystem ist jetzt in Stufenform. Da die dritte Gleichung eine falsche Aussage enthält, ist das lineare Gleichungssystem nicht lösbar und somit }
    
   \lang{en}{The linear system is now in row echelon form. Because the third equation contains a false statement, the linear system is unsolvable and hence}  
 \[ \mathbb{L}=\emptyset.\]
 \end{incremental}

  \tab{
  \lang{de}{Lösung (e)}
  \lang{en}{Solution (e)}
  }
  \begin{incremental}[\initialsteps{1}]
    \step 
    \lang{de}{Um die Lösungsmenge des angegebenen linearen Gleichungssystems zu bestimmen, bringen Sie es zunächst mit dem Gauß-Verfahren auf Stufenform.}
    \lang{en}{In order to determine the solution set of a given linear system we need to first bring it into row echelon form using Gaussian Elimination.}
     
    \step \lang{de}{Multiplizieren Sie beispielsweise die erste Gleichung auf beiden Seiten mit $-2$ (bzw. $-3$) und addieren Sie sie dann zur zweiten (bzw. dritten) Gleichung, so dass in diesen $x$ nicht mehr vorkommt.}
    \lang{en}{Multiply both sides of the first equation by $-2$ and add it to the second equation; multiply both sides of the first equation by $-3$ and add it to the third equation. Equations two and three now don't have
    the variable $x$ in them anymore.}
    \step \lang{de}{Das lineare Gleichungssystem sieht dann wie folgt aus:}
    \lang{en}{The linear system now looks as follows:}
    \begin{displaymath}
\begin{mtable}[\cellaligns{ccrcrcrcrc}]
(I)&\qquad&3x&-&2y&+&5z&=&8&\phantom{.}\\
(II)&&&&9y&-&12z&=&-21&\phantom{.}\\
(III)&&&&3y&-&16z&=&-55&|(-\frac{1}{3}) \cdot (II)+(III)\phantom{.}
\end{mtable}
\end{displaymath}
    
    \step \lang{de}{Im nächsten Schritt bietet es sich an, die zweite Gleichung mit $-\frac{1}{3}$ zu multiplizieren und dann zur dritten Gleichung zu addieren, damit dort schließlich $y$ nicht mehr vorkommt.\\
    Das lineare Gleichungssystem sieht dann wie folgt aus:}
    \lang{en}{In the next step it's easiest to multiply the second equation by $-\frac{1}{3}$ and add it to the third so that the variable $y$ doesn't appear in the third equation anymore.\\
    The linear system then looks as follows:}
    \begin{displaymath}
\begin{mtable}[\cellaligns{ccrcrcrcr}]
(I)&\qquad&3x&-&2y&+&5z&=&8\phantom{.}\\
(II)&&&&9y&-&12z&=&-21\phantom{.}\\
(III)&&&&&-&12z&=&-48\lang{de}{}\lang{en}{\phantom{.}}
\end{mtable}
\end{displaymath}
    
    \step \lang{de}{Das lineare Gleichungssystem ist jetzt in Stufenform und kann durch Rückwärtseinsetzen gelöst werden. 
    
    Aus der dritten Gleichung erhält man $z=4$. 
    
    Setzt man dies in die zweite Gleichung ein, so ergibt sich $y=3$. 
    
    Schließlich erhält man $x$ durch Einsetzen von $y$ und $z$ in die erste Gleichung und damit $x=-2$. 
    
    Das lineare Gleichungssystem ist also eindeutig lösbar mit }
    \lang{en}{The linear system is now in row echelon form and we can use back substitution in order to solve it. From the third equation we get $z=4$. Substituting this into the second equation gives us
    $y=3$. As our last step, inserting the known values for $y$ and $z$ into the first equation we can solve for $x$; the result is $x=-2$. The linear system is uniquely solvable with}
 \[\mathbb{L}=\{(-2; 3; 4)\}.\]
 \end{incremental}

  \tab{
  \lang{de}{Lösung (f)}
  \lang{en}{Solution (f)}
  }
  \begin{incremental}[\initialsteps{1}]
    \step 
    \lang{de}{In diesem Beispiel bestimmen wir die Lösungsmenge alternativ, indem das Gauß-Verfahren auf die erweiterte Koeffizientenmatrix angewandt wird. Diese lautet:
    }
    \[
(A \mid b) ~=~
\begin{pmatrix}
    1 & -2 & 4 &|& -6 \\
    2 & -2 & 3 &|& 0 \\
    3 & -4 & 7 &|& -6
\end{pmatrix}
\]
    
    
    \lang{de}{Um nun die Lösungsmenge des angegebenen linearen Gleichungssystems zu bestimmen, bringen wir es zunächst mit dem Gauß-Verfahren auf Stufenform.}
    %\lang{en}{In order to determine the solution set of a given linear system we need to first bring it into row echelon form using Gaussian Elimination.}
     
   \step \lang{de}{Der aktuelle \textit{Stufeneintrag} ist $a_{11}$ und wird im Folgenden rot markiert.
   Um die Einträge $a_{21}$ und $a_{31}$ (blau) zu eliminieren, werden die folgenden Zeilenoperationen
   durchgeführt:}
   \[
   \rightsquigarrow~~
\begin{pmatrix}
    \textcolor{red}{1} & -2 & 4 &|& -6 \\
    \textcolor{blue}{2} & -2 & 3 &|& 0 \\
    \textcolor{blue}{3} & -4 & 7 &|& -6
\end{pmatrix}
~
\begin{matrix}
    \\
    |(-2) \cdot (I) + (II)\\
    |(-3) \cdot (I) + (III)
\end{matrix}
\]
   
    \lang{de}{Man erhält:}
\[
\rightsquigarrow~~
\begin{pmatrix}
    1 & -2 & 4 &|& -6 \\
    0 & \textcolor{blue}{2} & \textcolor{blue}{-5} &|& \textcolor{blue}{12} \\
    0 & \textcolor{blue}{2} & \textcolor{blue}{-5} &|& \textcolor{blue}{12} \\
\end{pmatrix}
\]

    \step \lang{de}{Der neue \textit{Stufeneintrag} ist $a_{22}$. Man eliminiert
    nun $a_{32}$:}
\[
\rightsquigarrow~~
\begin{pmatrix}
    1 & -2 & 4 &|& -6 \\
    0 & \textcolor{red}{2} & -5 &|& 12 \\
    0 &  \textcolor{blue}{2} & -5 &|& 12
\end{pmatrix}
~
\begin{matrix}
    \\
    \\
    |-(II) + (III)
\end{matrix}
\]

    \lang{de}{Man erhält:}
\[
\rightsquigarrow~~
\begin{pmatrix}
    1 & -2 & 4 &|& -6 \\
    0 &  2 & -5 &|& 12 \\
    0 & 0 & 0 &|& 0
\end{pmatrix}
\]
      
    \lang{de}{Nun liegt die Stufenform vor. 
    Man wählt $z$ als freie Variable
    und drückt diese durch den Parameter $t\in \R$ aus.
    Durch Rückwärtseinsetzen bestimmt 
    man zunächst $y$ und dann $x$:
    %und setzt die freie Variable $z$ 
    %einem Parameter $t\in \R$. 

\[
\begin{mtable}[\cellaligns{lrl}]
   (III) \, \,&z&=t\\
   (II) &2y-5t &= 12 ~\Leftrightarrow~ y = \frac{1}{2}(12+5t)=6+\frac{5}{2}t \\
   (I) &x-2(6+\frac{5}{2}t)+4t&=-6 ~\Leftrightarrow~ x =6+t
\end{mtable}
\]}
\step \lang{de}{Damit lautet die Lösungsmenge}
\[\mathbb{L}=\{(6+t\lang{de}{;}\lang{en}{,} 6+\frac{5}{2}t\lang{de}{;}\lang{en}{,} t) ~|~ t\in\R\}.\]
   \end{incremental}
  \tab{
  \lang{de}{Lösung (g)}
  \lang{en}{Solution (g)}}
  \begin{incremental}[\initialsteps{1}]
    \step 
    \lang{de}{Dieses lineare Gleichungssystem ist bereits in Stufenform. An ihr erkennt man, dass zwei Variablen frei wählbar sind. Setzen Sie beispielsweise $y=s\in\R$ und $z=t\in\R$.}
    \lang{en}{This system is already in row echelon form, and we can recognize right away that two variables can be chosen freely. Let, for example, $y=s\in\R$ and $z=t\in\R$.}
     
    \step \lang{de}{Durch Einsetzen von $y=s$ und $z=t$ in die erste Gleichung erhalten Sie dann $x=3-2s+5t$ und somit}
	\lang{en}{By substituting $y=s$ and $z=t$ into the first equation we get $x=3-2s+5t$ and hence}
\[ \mathbb{L}=\{(3-2s+5t; s;t) ~|~ s, t\in\R\}.\]
  \end{incremental} 

\tab{
  \lang{de}{Hinweis}
  \lang{en}{Note}
  }
  
  \lang{de}{Beachten Sie, dass es auch andere Möglichkeiten gibt, die linearen Gleichungssysteme mit dem Gauß-Verfahren auf Stufenform zu bringen. Die aufgeführten Wege sind nur eine Variante. Außerdem ist die Stufenform nicht eindeutig. 
  Die Lösungsmenge der linearen Gleichungssysteme in (c) und (d) haben darüber hinaus verschiedene Darstellungsmöglichkeiten. Abgesehen davon sind die Lösungsmengen in der jeweiligen Teilaufgabe jedoch immer gleich.} 

  \lang{en}{Note that there are other possibilities for reducing the linear system to row echelon form using Gaussian Elimination. The steps performed here are only one variant. Also, the row echelon form of a system is not unique, but the answers aren't influenced by this at all.
  The solution sets of the linear systems in (c) and (d) have different possibilities for their representation, however the different representations for the solution set of c) are equivalent; all different representations for d) are also equivalent.}

\end{tabs*}
\end{content}