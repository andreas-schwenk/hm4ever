\documentclass{mumie.element.exercise}
%$Id$
\begin{metainfo}
  \name{
    \lang{de}{Ü02: Matrixschreibweise}
    \lang{en}{Exercise 2}
  }
  \begin{description} 
 This work is licensed under the Creative Commons License Attribution 4.0 International (CC-BY 4.0)   
 https://creativecommons.org/licenses/by/4.0/legalcode 

    \lang{de}{Hier die Beschreibung}
    \lang{en}{}
  \end{description}
  \begin{components}
  \end{components}
  \begin{links}
  \end{links}
  \creategeneric
\end{metainfo}
\begin{content}
\begin{block}[annotation]
	Im Ticket-System: \href{https://team.mumie.net/issues/21355}{Ticket 21355}
\end{block}
\begin{block}[annotation]
Copy of \href{http://team.mumie.net/issues/9573}{Ticket 9573}: content/rwth/HM1/T111_Matrizen,_lineare_Gleichungssysteme/exercises/exe_exercise4.src.tex
\end{block}

\title{
  \lang{de}{Ü02: Matrixschreibweise}
  \lang{en}{Exercise 2}
}

\begin{block}[annotation]
	Matritzen, Lineare Gleichungen: Übungen
\end{block}

\lang{de}{Geben Sie die folgenden linearen Gleichungssysteme in der Matrixschreibweise
$A\cdot x =b$ an.}
(a)
\[\begin{matrix}
   3x_{1}&+&10x_{2}&-&x_{3}&=&12\\
    x_{1}&+&3x_{2}& +&4x_{3}&=&0
  \end{matrix}\]
(b)
\[\begin{matrix}
   x_{1}&&&-&5x_{3}&+&x_4&=&0\\
    2x_{1}&+&3x_{2}&&& -&x_{4}&=&2\\
    &&4x_{2}&&& +&3x_{4}&=&-1\\
  \end{matrix}\]


\begin{tabs*}[\initialtab{0}\class{exercise}]
  \tab{
  \lang{de}{Lösung (a)}
  }
  \lang{de}{Das lineare Gleichungssystem kann wie folgt in Matrixschreibweise $A \cdot x=b$ angegeben werden:
\[\begin{pmatrix}
     3&10&-1\\1&3&4
    \end{pmatrix} \cdot 
    \begin{pmatrix}x_1\\x_2 \\ x_3\end{pmatrix}= \begin{pmatrix}
                                12\\0
                               \end{pmatrix},\]
  
also 
\[A=\begin{pmatrix}
     3&10&-1\\1&3&4
    \end{pmatrix}\quad,\quad b=\begin{pmatrix}
                                12\\0
                               \end{pmatrix}\,.\]}
  
    \tab{\lang{de}{Lösungsvideo (b)}}
  \youtubevideo[500][300]{QJKQ6Rf-Pto}\\
  
  \end{tabs*}
\end{content}