\documentclass{mumie.element.exercise}
%$Id$
\begin{metainfo}
  \name{
    \lang{de}{Ü06: Rang}
    \lang{en}{Exercise 6}
  }
  \begin{description} 
 This work is licensed under the Creative Commons License Attribution 4.0 International (CC-BY 4.0)   
 https://creativecommons.org/licenses/by/4.0/legalcode 

    \lang{de}{Rechnen mit Matritzen}
    \lang{en}{}
  \end{description}
  \begin{components}
  \end{components}
  \begin{links}
  \end{links}
  \creategeneric
\end{metainfo}
\begin{content}
\begin{block}[annotation]
	Im Ticket-System: \href{https://team.mumie.net/issues/21351}{Ticket 21351}
\end{block}
\begin{block}[annotation]
Copy of \href{http://team.mumie.net/issues/18630}{Ticket 18630}: content/rwth/HM1/T112_Rechnen_mit_Matrizen/exercises/exe_exercise5.src.tex
\end{block}

\usepackage{mumie.ombplus}

\title{
  \lang{de}{Ü06: Rang}
  \lang{en}{Exercise 6}
}

\begin{block}[annotation]
  Rechnen mit Matritzen
     
\end{block}




\lang{de}{
    Welchen Rang haben die folgenden Matrizen über $\mathbb{R}$? 
    Welche der Matrizen haben \textit{vollen} Rang?
}
\begin{enumerate}

    \item[(a)]
        \[
            A = 
            \begin{pmatrix}
                2 & 0 & 4 \\ 
                0 & 1 & 1 \\
                1 & 0 & 2 \\
           \end{pmatrix}
        \]
        
    \item[(b)]
        \[
            B = 
            \begin{pmatrix}
                2 & 3 & 7 \\ 
                0 & 2 & 1 \\
                0 & 0 & -2 \\
           \end{pmatrix}
        \]

    \item[(c)]
        \[
            C = 
            \begin{pmatrix}
                1 & 2 & 4 \\ 
                0 & 0 & 1 \\
           \end{pmatrix}
        \]
        
    \item[(d)] 
        \[
            D = 
            \begin{pmatrix}
                1 & 3 & 0 \\ 
                0 & 1 & 2 \\
                -1 & -2 & c \\
           \end{pmatrix}
        \]     
        mit $c \in \mathbb{R}$
\end{enumerate}


\begin{tabs*}[\initialtab{0}\class{exercise}]
  \tab{
  \lang{de}{Antwort}
  }
\begin{itemize}
\item[(a)] Rang(A) = 2, kein voller Rang
\item[(b)] Rang(B) = 3, voller Rang
\item[(c)] Rang(C) = 2, voller Rang
\item[(d)] Rang(D) = 2 für c=2, kein voller Rang und\\
 Rang(D)=3 für $c \neq 2$, voller Rang
\end{itemize}
    \tab{\lang{de}{Lösung (a)}}
    \lang{de}{
        Wir bestimmen zunächst den Zeilenrang der $(3 \times 3)$-Matrix $A$.
        Mit dem Gauß-Verfahren bringt man die Matrix
        durch elementare Umformungen der Zeilen auf Stufenform.
        Durch Multiplikation der letzten Zeile mit dem Faktor 2 erhalten wir:
        \[
            \begin{pmatrix}
                2 & 0 & 4 \\ 
                0 & 1 & 1 \\
                2 & 0 & 4 \\
           \end{pmatrix}
              ~
\begin{matrix}
    \\
    \\
    |-(I)+(III)
\end{matrix}\]
        Subtraktion der ersten Zeile von der letzten Zeile ergibt:
        \[\rightsquigarrow~~
            \begin{pmatrix}
                2 & 0 & 4 \\ 
                0 & 1 & 1 \\
                0 & 0 & 0 \\
           \end{pmatrix}
        \]
        Der Zeilenrang von $A$ ergibt sich nun durch die Anzahl an Zeilen,
        die ungleich der Nullzeile sind. Also ist der Zeilenrang hier
        gleich 2. 
        
        Alternativ hätte man auch den Spaltenrang von $A$ bestimmen können.
        Der Spaltenrang ist definiert durch den Zeilenrang von $A^T$.
        \[
            A^T = 
            \begin{pmatrix}
                2 & 0 & 1 \\ 
                0 & 1 & 0 \\
                4 & 1 & 2 \\
           \end{pmatrix}
              ~
\begin{matrix}
    \\
    \\
    |-2(I)-(II)+(III)
\end{matrix}\]
        Die dritte Zeile entspricht dem 2-fachen der ersten Zeile plus
        der zweiten Zeile. Man erhält:
        \[\rightsquigarrow~~
            \begin{pmatrix}
                2 & 0 & 1 \\ 
                0 & 1 & 0 \\
                0 & 0 & 0 \\
           \end{pmatrix}
        \]
        Der Spaltenrang ist somit ebenfalls 2.
        %\end{remark}
        
        Zeilenrang und Spaltenrang einer Matrix sind immer gleich.
        Matrix $A$ hat also Rang 2.
        Eine $(3 \times 3)$-Matrix hat vollen Rang, wenn ihr Rang 3 ist.
        Dies ist hier nicht der Fall.
    }
    
    \tab{\lang{de}{Lösung (b)}}
    \lang{de}{
        Die $(3 \times 3)$-Matrix $B$ liegt bereits in Stufenform vor.
        Der Zeilenrang von $A$ ergibt sich durch die Anzahl an Zeilen,
        die ungleich der Nullzeile sind.
        Also ist der Zeilenrang und damit auch der Rang 3.
        Die Matrix $B$ hat also vollen Rang.
    }

    \tab{\lang{de}{Lösung (c)}}
    \lang{de}{
        Die $(2 \times 3)$-Matrix $C$ liegt bereits in Stufenform vor.
        Der Zeilenrang von $A$ ergibt sich durch die Anzahl an Zeilen,
        die ungleich der Nullzeile sind. 
        Also ist der Zeilenrang und damit auch
        der Rang 2.
        
        Eine $(m \times n)$-Matrix hat vollen Rang,
        wenn der Rang von $A$ gleich der kleineren der beiden
        Zahlen $m$ und $n$ ist.
        Wir bestimmen $\min\{m,n\}=\min\{2, 3\}=2$ und stellen fest,
        dass dies mit dem Rang übereinstimmt.
        Matrix $C$ hat also auch vollen Rang.
    }

 \tab{\lang{de}{Lösungsvideo (d)}}
  \youtubevideo[500][300]{lBT5RV9MS2w}\\


\end{tabs*}







% ALTE AUFGABEN:

%\lang{de}{ Welche der folgenden Aussagen sind für Matrizen $A, B, C \in \mathbb{M}(n,n;\mathbb{R})$ wahr?
%\begin{enumerate}
% \item[a)]
%\[
%    (A \cdot B) \cdot C = A \cdot (B \cdot C)
%\]
% \item[b)]
%\[
%    A \cdot (B + C) = AB + AC
%\]
% \item[c)] 
%\[
%    A \cdot B \cdot C = A \cdot C \cdot B
%\]
%\end{enumerate}}
%
%\begin{tabs*}[\initialtab{0}\class{exercise}]
%  \tab{\lang{de}{Lösung a)}}
%  \lang{de}{
%wahr (Assoziativregel).
%\begin{example}
%\[
%A = \begin{pmatrix}
%  0 & -2 \\ 
%  3 & -5
% \end{pmatrix}
%,~
%B = \begin{pmatrix}
%  1 & 3 \\ 
%  0 & 2
% \end{pmatrix}
%,~
%C = \begin{pmatrix}
%  1 & -1 \\ 
%  3 & 2
% \end{pmatrix}
%\]
%
%\[
%(A \cdot B) \cdot C = 
%\left(
%\begin{pmatrix}
%  0 & -2 \\ 
%  3 & -5
%\end{pmatrix}
%\cdot
%\begin{pmatrix}
%  1 & 3 \\ 
%  0 & 2
%\end{pmatrix}
%\right)
%\cdot
%\begin{pmatrix}
%  1 & -1 \\ 
%  3 & 2
%\end{pmatrix}
%=
%\begin{pmatrix}
%  0 & -4 \\ 
%  3 & -1
%\end{pmatrix}
%\cdot
%\begin{pmatrix}
%  1 & -1 \\ 
%  3 & 2
%\end{pmatrix}
%=
%\begin{pmatrix}
%  -12 & -8 \\ 
%  0 & -5
%\end{pmatrix}
%\]
%
%\[
%A \cdot (B \cdot C) = 
%\begin{pmatrix}
%  0 & -2 \\ 
%  3 & -5
%\end{pmatrix}
%\cdot
%\left(
%\begin{pmatrix}
%  1 & 3 \\ 
%  0 & 2
%\end{pmatrix}
%\cdot
%\begin{pmatrix}
%  1 & -1 \\ 
%  3 & 2
%\end{pmatrix}
%\right)
%=
%\begin{pmatrix}
%  0 & -2 \\ 
%  3 & -5
%\end{pmatrix}
%\cdot
%\begin{pmatrix}
%  10 & 5 \\ 
%  6 & 4
%\end{pmatrix}
%=
%\begin{pmatrix}
%  -12 & -8 \\ 
%  0 & -5
%\end{pmatrix}
%\]
%\end{example}
%
%}
%
% \tab{\lang{de}{Lösung b)}}
%  \lang{de}{
%wahr (Distributivregel).
%
%\begin{example}
%\[
%A = \begin{pmatrix}
%  0 & -2 \\ 
%  3 & -5
% \end{pmatrix}
%,~
%B = \begin{pmatrix}
%  1 & 3 \\ 
%  0 & 2
% \end{pmatrix}
%,~
%C = \begin{pmatrix}
%  1 & -1 \\ 
%  3 & 2
% \end{pmatrix}
%\]
%
%\[
%A \cdot (B + C) = 
%\begin{pmatrix}
%  0 & -2 \\ 
%  3 & -5
%\end{pmatrix}
%\cdot
%\left(
%\begin{pmatrix}
%  1 & 3 \\ 
%  0 & 2
%\end{pmatrix}
%+
%\begin{pmatrix}
%  1 & -1 \\ 
%  3 & 2
%\end{pmatrix}
%\right)
%=
%\begin{pmatrix}
%  0 & -2 \\ 
%  3 & -5
%\end{pmatrix}
%\cdot
%\begin{pmatrix}
%  2 & 2 \\ 
%  3 & 4
%\end{pmatrix}
%=
%\begin{pmatrix}
%  -6 & -8 \\ 
%  -9 & -14
%\end{pmatrix}
%\]
%
%\[
%AB + AC = 
%\begin{pmatrix}
%  0 & -2 \\ 
%  3 & -5
%\end{pmatrix}
%\cdot
%\begin{pmatrix}
%  1 & 3 \\ 
%  0 & 2
%\end{pmatrix}
%+
%\begin{pmatrix}
%  0 & -2 \\ 
%  3 & -5
%\end{pmatrix}
%\cdot
%\begin{pmatrix}
%  1 & -1 \\ 
%  3 & 2
%\end{pmatrix}
%=
%\begin{pmatrix}
%  0 & -4 \\ 
%  3 & -1
%\end{pmatrix}
%\cdot
%\begin{pmatrix}
%  -6 & -4 \\ 
%  -12 & -13
%\end{pmatrix}
%=
%\begin{pmatrix}
%  -6 & -8 \\ 
%  -9 & -14
%\end{pmatrix}
%\]
%\end{example}
%
%}
%
% \tab{\lang{de}{Lösung c)}}
%  \lang{de}{
%falsch: die Matrizenmultiplikation ist \textbf{nicht} kommutativ!
%
%\begin{example}
%\[
%A = \begin{pmatrix}
%  0 & -2 \\ 
%  3 & -5
% \end{pmatrix}
%,~
%B = \begin{pmatrix}
%  1 & 3 \\ 
%  0 & 2
% \end{pmatrix}
%,~
%C = \begin{pmatrix}
%  1 & -1 \\ 
%  3 & 2
% \end{pmatrix}
%\]
%
%\[
%A \cdot B \cdot C = 
%\begin{pmatrix}
%  0 & -2 \\ 
%  3 & -5
%\end{pmatrix}
%\cdot
%\begin{pmatrix}
%  1 & 3 \\ 
%  0 & 2
%\end{pmatrix}
%\cdot
%\begin{pmatrix}
%  1 & -1 \\ 
%  3 & 2
%\end{pmatrix}
%=
%\begin{pmatrix}
%  -12 & -8 \\ 
%  0 & -5
%\end{pmatrix}
%\]
%
%\[
%A \cdot C \cdot B = 
%\begin{pmatrix}
%  0 & -2 \\ 
%  3 & -5
%\end{pmatrix}
%\cdot
%\begin{pmatrix}
%  1 & -1 \\ 
%  3 & 2
%\end{pmatrix}
%\cdot
%\begin{pmatrix}
%  1 & 3 \\ 
%  0 & 2
%\end{pmatrix}
%=
%\begin{pmatrix}
%  -6 & -26 \\ 
%  -12 & -62
%\end{pmatrix}
%\]
%\end{example}
%
%}
%\end{tabs*}


\end{content}