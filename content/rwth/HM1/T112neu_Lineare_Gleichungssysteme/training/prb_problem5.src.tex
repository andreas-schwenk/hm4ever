\documentclass{mumie.problem.gwtmathlet}
%$Id$
\begin{metainfo}
  \name{
    \lang{de}{A05: Rang}
    \lang{en}{mc yes-no}
  }
  \begin{description} 
 This work is licensed under the Creative Commons License Attribution 4.0 International (CC-BY 4.0)   
 https://creativecommons.org/licenses/by/4.0/legalcode 

    \lang{de}{Beschreibung}
    \lang{en}{description}
  \end{description}
  \corrector{system/problem/GenericCorrector.meta.xml}
  \begin{components}
    \component{js_lib}{system/problem/GenericMathlet.meta.xml}{gwtmathlet}
  \end{components}
  \begin{links}
  \end{links}
  \creategeneric
\end{metainfo}
\begin{content}
\begin{block}[annotation]
	Im Ticket-System: \href{https://team.mumie.net/issues/21357}{Ticket 21357}
\end{block}
\begin{block}[annotation]
Copy of \href{http://team.mumie.net/issues/9673}{Ticket 9673}: content/rwth/HM1/T112_Rechnen_mit_Matrizen/training/prb_problem6.src.tex
\end{block}

\usepackage{mumie.genericproblem}

\lang{de}{
	\title{A05: Rang}
}



\begin{problem}
% 	





     \begin{question} 
     
     
     \begin{variables}
		\randint[Z]{a}{-9}{9}
		\randint[Z]{b}{-9}{9}
		\randint[Z]{c}{-9}{9}
		\randint[Z]{d}{-9}{9}
		\randint[Z]{ee}{-9}{9}
		\randint[Z]{f}{-9}{9}
		\randint[Z]{k}{-9}{9}
        \randint[Z]{l}{-9}{9}
		
			%\matrix[calculate]{m1}{
  			%a & a & k*a \\
     %0 & a & k*a \\
     %0 & a & k*a 
     % 	}
      		\matrix[calculate]{m2}{
  			 a & -l*a & -(l+1)*a \\
             a & -l*a & -(l+1)*a \\
             0 & (l+1)*a & (l+1)*a 
      	}
     % 		\matrix[calculate]{m3}{
  		%	a & b & c \\
     %0 & 0 & c \\
     %0 & b & 0 
     % 	}
      	\matrix[calculate]{m4}{
  			a & b & c \\
            d & ee & f \\
            0 & 0 & 0
      	}
      	\matrix[calculate]{m5}{
  			a & 0 & 0 & 0\\
            0 & b & b & 0 \\
            0 & 2*b & 3*b & 4*b \\
            0 & c & c & 5*c
      	}
        \matrix[calculate]{m6}{
  			a & b & c \\
            a*2 & b*4 & c*5
      	}
	\end{variables}
     
     
     \lang{de}{ 
      	\text{Entscheiden Sie, ob folgende Matrizen vollen Rang besitzen.}
      }
    	\permutechoices{1}{3}
    	\type{mc.yesno}
		%\begin{choice}
  		%	\text{$\var{m1}$}
  		%	\solution{false}
		%\end{choice}
		\begin{choice}
  			\text{$\var{m2}$}
  			\solution{false}
		\end{choice}
		\begin{choice}
  			\text{$\var{m4}$}
  			\solution{false}
		\end{choice}
		%\begin{choice}
  		%	\text{$\var{m3}$}
  		%	\solution{true}
		%\end{choice}
		\begin{choice}
  			\text{$\var{m5}$}
  			\solution{true}
		\end{choice}
                
        \explanation{
         Bringen Sie die jeweilige Matrix zunächst mit Hilfe des Gauß-Verfahrens in Stufenform.
         Der Rang enspricht dann der Anzahl an Zeilen, die ungleich der Nullzeile sind.
         Eine $(n \times n)$-Matrix $A$ hat dann vollen Rang, wenn der Rang von $A$ gleich $n$ ist.
        }
        
    \end{question}
    
    
    
    
\begin{question} 
   \begin{variables}
      \randint[Z]{a}{-9}{9}
      \randint[Z]{b}{-9}{9}
      \randint[Z]{c}{-9}{9}
      \randint[Z]{d}{-9}{9}
      \randint[Z]{ee}{-9}{9}
      \randint[Z]{f}{-9}{9}
      \randint[Z]{k}{-9}{9}
      \matrix[calculate]{m6}{
          a & b & c \\
          a*2 & b*3 & c*4
      }
    \end{variables}
    \lang{de}{ 
        \text{Entscheiden Sie, ob die folgende Matrix vollen Rang besitzt.}
    }
    %\permutechoices{1}{3}
    \type{mc.yesno}
    \begin{choice}
        \text{$\var{m6}$}
        \solution{true}
    \end{choice}      
    \explanation{
     Bringen Sie die Matrix zunächst mit Hilfe des Gauß-Verfahrens in Stufenform.
     Der Rang enspricht dann der Anzahl an Zeilen, die ungleich der Nullzeile sind.
     Eine $(m \times n)$-Matrix $A$ hat dann vollen Rang, wenn der Rang von $A$ gleich $\min\{m,n\}$ ist.
     Der Rang ist hier 2 und entspricht $\min\{2, 3\}=2$.
    }        
\end{question}    
    
    
    
    
\begin{question}
  \begin{variables}
    \randint[Z]{a}{-9}{9}
    \randint[Z]{b}{-9}{9}
    \randint[Z]{c}{-9}{9}
    \randint[Z]{d}{-9}{9}
    \randint[Z]{ee}{-9}{9}
    \randint[Z]{f}{-9}{9}
    \randint[Z]{k}{-9}{9}
    \matrix[calculate]{r1}{
      a & a & k*a \\
      0 & a & k*a \\
      0 & a & k*a 
    }
    \number{r1s}{2}
  \end{variables}
  \type{input.number}
  \displayprecision{3}
  \correctorprecision{2}
  \field{integer}
  \lang{de}{
      \text{Geben Sie den Rang für die folgende Matrix an:\\
      $A=\var{r1}$}
  }
  \begin{answer}
      \text{Rang(A)=}\solution{r1s}
      \explanation{
         Bringen Sie die Matrix zunächst mit Hilfe des Gauß-Verfahrens in Stufenform.
         Sie stellen fest, dass die dritte Zeile in dieser Aufgabe dann eine Nullzeile ist.
         Der Rang enspricht dann der Anzahl an Zeilen, die ungleich der Nullzeile sind,
         also hier 2.
      }
  \end{answer}
\end{question} 


\begin{question}
  \begin{variables}
    \randint[Z]{a}{-9}{9}
    \randint[Z]{b}{-9}{9}
    \randint[Z]{c}{-9}{9}
    \randint[Z]{d}{-9}{9}
    \randint[Z]{ee}{-9}{9}
    \randint[Z]{f}{-9}{9}
    \randint[Z]{k}{-9}{9}
    \matrix[calculate]{r2}{
      a & b & c \\
      0 & 0 & c \\
      0 & b & 0 
    }    
    \number{r2s}{3}
  \end{variables}
  \type{input.number}
  \displayprecision{3}
  \correctorprecision{2}
  \field{integer}
  \lang{de}{
      \text{Geben Sie den Rang für die folgende Matrix an:\\
      $B=\var{r2}$}
  }
  \begin{answer}
      \text{Rang(B)=}\solution{r2s}
      \explanation{
        Bringen Sie die Matrix zunächst mit Hilfe des Gauß-Verfahrens in Stufenform.
        Am einfachsten tauscht man hier die zweite und die dritte Zeile.
        Der Rang enspricht dann der Anzahl an Zeilen, die ungleich der Nullzeile sind,
        also hier 3.
      }
  \end{answer}
\end{question} 

    
    
\end{problem}

\embedmathlet{gwtmathlet}

\end{content}
