\documentclass{mumie.problem.gwtmathlet}
%$Id$
\begin{metainfo}
  \name{
    \lang{de}{A01: LGS (2x2)}
    \lang{en}{}
  }
  \begin{description} 
 This work is licensed under the Creative Commons License Attribution 4.0 International (CC-BY 4.0)   
 https://creativecommons.org/licenses/by/4.0/legalcode 

    \lang{de}{Beschreibung}
    \lang{en}{}
  \end{description}
  \corrector{system/problem/GenericCorrector.meta.xml}
  \begin{components}
    \component{js_lib}{system/problem/GenericMathlet.meta.xml}{mathlet}
  \end{components}
  \begin{links}
  \end{links}
  \creategeneric
\end{metainfo}
\begin{content}
\begin{block}[annotation]
	Im Ticket-System: \href{https://team.mumie.net/issues/21358}{Ticket 21358}
\end{block}
\begin{block}[annotation]
Copy of \href{http://team.mumie.net/issues/9579}{Ticket 9579}: content/rwth/HM1/T111_Matrizen,_lineare_Gleichungssysteme/training/prb_problem2.src.tex
\end{block}

\usepackage{mumie.genericproblem}

\lang{de}{
	\title{A01: LGS (2x2)}
}
\lang{en}{
	\title{Problem 1}
}



\begin{problem}

	\randomquestionpool{1}{2}
		
	\begin{question}
		\lang{de}{\text{Geben Sie die Lösung des folgenden linearen Gleichungssystems an.\\
        \begin{align*}
        \var{a}x&+\var{b}y& \ = \ &\var{c}\\
        x&\var{d1}y& \ = \ &\var{f}
        \end{align*}
        }}
        \lang{en}{\text{Find the solution of the following linear system.\\
        \begin{align*}
        \var{a}x&+\var{b}y& \ = \ &\var{c}\\
        x&\var{d1}y& \ = \ &\var{f}
        \end{align*}
        }}
        \lang{de}{\explanation{Am einfachsten multiplizieren Sie die zweite Gleichung mit $\var{z}$ und addieren dann die beiden Gleichungen. In der resultierenden Gleichung kommt schließlich $x$ nicht mehr vor.}}
        \lang{en}{\explanation{The simplest way is to multiply the second equation by $\var{z}$ and add both equations together. The resulting equation won't have the variable $x$ in it.}}
        \type{input.number}
       
        \begin{variables}
            \randint[Z]{a}{-10}{10}
            \randint[Z]{b}{0}{10}
            \randint[Z]{d1}{-10}{0}
            \randadjustIf{a,b,d1}{a*d1-b=0 OR a*a=1 OR b*b=1 OR d1*d1=1 OR abs(b)=abs(d1)}
            
            \randint{s}{-5}{5}
            \randint{t}{-5}{5}
            
            \function[calculate]{c}{a*s+b*t} 
            \function[calculate]{f}{s+d1*t}
            
            \function[calculate]{z}{-a}
        \end{variables}

        \begin{answer}
              \text{$x=$}
              \solution{s}
        \end{answer}
        
        \begin{answer}
        	  \text{$y=$}
              \solution{t} 
        \end{answer}
        
    \end{question}

	\begin{question}
		\lang{de}{\text{Geben Sie die Lösung des folgenden linearen Gleichungssystems an.\\
        \begin{align*}
        \var{a}x&\var{b}y& \ = \ &\var{c}\\
        \var{d}x&+\var{d1}y& \ = \ &\var{f}
        \end{align*}
        }}
        \lang{en}{\text{Find the solution of the following linear system.\\
        \begin{align*}
        \var{a}x&\var{b}y& \ = \ &\var{c}\\
        \var{d}x&+\var{d1}y& \ = \ &\var{f}
        \end{align*}
        }}
        \lang{de}{\explanation{Am einfachsten multiplizieren Sie die erste Gleichung mit $\var{q}$ und addieren dann die beiden Gleichungen. In der resultierenden Gleichung kommt schließlich $y$ nicht mehr vor.}}
        \lang{en}{\explanation{The simplest way is to multiply the first equation by $\var{q}$ and add both equations together. The resulting equation won't have the variable $y$ in it.}}
        \type{input.number}
       
        \begin{variables}
            \randint[Z]{a}{-10}{10}
            \randint[Z]{b}{-5}{0}
            \randint[Z]{d}{-10}{10}
            \randint[Z]{q}{2}{3}
            \randadjustIf{a,b,d,q}{-a*q*b-b*d=0 OR a*a=1 OR d*d=1 OR b*b=1 OR abs(a)=abs(d)}
            
            \function[calculate]{d1}{-q*b}
            
            \randint{s}{-5}{5}
            \randint{t}{-5}{5}
            
            \function[calculate]{c}{a*s+b*t} 
            \function[calculate]{f}{d*s+d1*t}
        \end{variables}

        \begin{answer}
              \text{$x=$}
              \solution{s}
        \end{answer}
        
        \begin{answer}
        	  \text{$y=$}
              \solution{t} 
        \end{answer}
        
    \end{question}
    
\end{problem}

\embedmathlet{mathlet}

\end{content}