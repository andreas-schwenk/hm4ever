\documentclass{mumie.problem.gwtmathlet}
%$Id$
\begin{metainfo}
  \name{
    \lang{de}{A04: Parabelgleichung}
    \lang{en}{}
  }
  \begin{description} 
 This work is licensed under the Creative Commons License Attribution 4.0 International (CC-BY 4.0)   
 https://creativecommons.org/licenses/by/4.0/legalcode 

    \lang{de}{Beschreibung}
    \lang{en}{}
  \end{description}
  \corrector{system/problem/GenericCorrector.meta.xml}
  \begin{components}
    \component{js_lib}{system/problem/GenericMathlet.meta.xml}{mathlet}
  \end{components}
  \begin{links}
  \end{links}
  \creategeneric
\end{metainfo}
\begin{content}
\usepackage{mumie.genericproblem}

\lang{de}{
	\title{A04: Parabelgleichung}
}
\lang{en}{
	\title{Problem 4}
}


\begin{block}[annotation]
	Im Ticket-System: \href{https://team.mumie.net/issues/21359}{Ticket 21359}
\end{block}
\begin{block}[annotation]
Copy of \href{http://team.mumie.net/issues/9584}{Ticket 9584}: content/rwth/HM1/T111_Matrizen,_lineare_Gleichungssysteme/training/prb_problem7.src.tex
\end{block}

\begin{problem}

	\begin{question} 
        \lang{de}{\text{Durch die Punkte $P=(\var{p1}; \var{p2})$ und $Q=(\var{q1}; \var{q2})$ gehen unendlich viele Parabeln, die wir nun ermitteln wollen. Stellen Sie ein lineares Gleichungssystem
        für die Koeffizienten $a$, $b$ und $c$ der Parabelgleichung $y=ax^2+bx+c$ auf und bestimmen Sie dessen Lösungsmenge.\\
        Setzen Sie dabei bitte die von Ihnen gewählte frei wählbare Variable gleich $t$, damit Ihre Lösung vom System richtig korrigiert werden kann. Verwenden Sie bitte außerdem Brüche und keine gerundeten Werte.
        }}
        \lang{en}{\text{Infinitely many parabolas go through the points $P=(\var{p1}, \var{p2})$ and $Q=(\var{q1}, \var{q2})$. Create a linear
        system of equations for the coefficients $a$, $b$ and $c$ of the parabola with equation $y=ax^2+bx+c$ and determine its solution set.\\
        Set the parameter you introduce for one of the variables equal to $t$ so that the system can check your answer properly. When inputting your answer use fractions instead of rounded values.
        }}
        \lang{de}{\explanation{Das lineare Gleichungssystem erhalten Sie durch Einsetzen der $x$- und $y$-Koordinaten der gegebenen Punkte in die Parabelgleichung.
        Bringen Sie dieses lineare Gleichungssystem dann mit dem Gauß-Verfahren auf Stufenform und lösen Sie es anschließend durch Rückwärtseinsetzen.}}
        \lang{en}{\explanation{You can get the linear system by substituting the $x$- and $y$-coordinates of the given points into the parabola equation.
        Reduce this linear system to row echelon form with Gaussian Elimination and solve it with back substitution then.}}
        
        \type{input.function}
        \field{rational}
        
        \begin{variables}
            \randint[Z]{p1}{-5}{5}
            \randint{p2}{-5}{5}
            \randint[Z]{q1}{-5}{5}
            \randint{q2}{-5}{5}
            \randadjustIf{p1,q1}{p1=q1}
                        
            \function[calculate]{a}{p1*p1}
            \function[calculate]{b}{p1}
            \function[calculate]{c}{1}
            \function[calculate]{d}{q1*q1}
            \function[calculate]{d1}{q1}
            \function[calculate]{f}{1}
            
            \function[calculate]{j}{p2}
            \function[calculate]{k}{q2}
                                    
            \function[calculate]{g}{0}
            \function[calculate]{h}{0}
            \function[calculate]{i1}{0}
                     
            \function[calculate]{l}{0}
                      
            \function[normalize]{lsg1}{(j*d1-k*b)/(a*d1-b*d)+t*(f*b-c*d1)/(a*d1-b*d)}
            \function[normalize]{lsg2}{(a*k-d*j)/(a*d1-b*d)+t*(d*c-a*f)/(a*d1-b*d)}
            \function{lsg3}{t}     
        \end{variables}
        
        \begin{answer}
              \text{$a=$}
              \solution{lsg1}
              \inputAsFunction{t}{u}
        \end{answer}
        
        \begin{answer}
        	  \text{$b=$}
              \solution{lsg2} 
              \inputAsFunction{t}{v}
              
        \end{answer}
        
        \begin{answer}
        	  \text{$c=$}
              \solution{lsg3} 
              \inputAsFunction{t}{w}
              
              \checkFuncForZero{abs(a*u[x]+b*v[x]+c*w[x]-j)+abs(d*u[x]+d1*v[x]+f*w[x]-k)+abs(g*u[x]+h*v[x]+i1*w[x]-l)+abs((x-u[x])*(x-v[x])*(x-w[x]))}{-10}{10}{100} 
        \end{answer}
        
	\end{question}
	
\end{problem}

\embedmathlet{mathlet}

\end{content}
