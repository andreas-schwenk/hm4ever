%$Id:  $
\documentclass{mumie.article}
%$Id$
\begin{metainfo}
  \name{
    \lang{de}{Rang einer Matrix}
    \lang{en}{Rank of a matrix}
  }
  \begin{description} 
 This work is licensed under the Creative Commons License Attribution 4.0 International (CC-BY 4.0)   
 https://creativecommons.org/licenses/by/4.0/legalcode 

    \lang{de}{Beschreibung}
    \lang{en}{Description}
  \end{description}
  \begin{components}
    \component{generic_image}{content/rwth/HM1/images/g_img_00_Videobutton_schwarz.meta.xml}{00_Videobutton_schwarz}
    \component{generic_image}{content/rwth/HM1/images/g_img_00_video_button_schwarz-blau.meta.xml}{00_video_button_schwarz-blau}
    \component{js_lib}{system/media/mathlets/GWTGenericVisualization.meta.xml}{mathlet1}
  \end{components}
  \begin{links}
    \link{generic_article}{content/rwth/HM1/T108_Vektorrechnung/g_art_content_30_basen_eigenschaften.meta.xml}{content_30_basen_eigenschaften}
    \link{generic_article}{content/rwth/HM1/T403_Quadratische_Matrizen,_Determinanten/g_art_content_07_quadratische_matrizen.meta.xml}{content_07_quadratische_matrizen}
    \link{generic_article}{content/rwth/HM1/T109_Skalar-_und_Vektorprodukt/g_art_content_31_skalarprodukt.meta.xml}{skalarprodukt}
    \link{generic_article}{content/rwth/HM1/T111neu_Matrizen/g_art_content_43_matrizenmultiplikation.meta.xml}{matrix-mult}
    \link{generic_article}{content/rwth/HM1/T112neu_Lineare_Gleichungssysteme/g_art_content_41_gauss_verfahren.meta.xml}{gauss-verfahren}
     \link{generic_article}{content/rwth/HM1/T112neu_Lineare_Gleichungssysteme/g_art_content_41_gauss_verfahren.meta.xml}{gauss1}
  \end{links}
  \creategeneric
\end{metainfo}
\begin{content}
\begin{block}[annotation]
	Im Ticket-System: \href{https://team.mumie.net/issues/21348}{Ticket 21348}
\end{block}
\begin{block}[annotation]
Copy of \href{https://team.mumie.net/issues/18626}{Ticket 18626}: content/rwth/HM1/T112_Rechnen_mit_Matrizen/art_content_45_matrixrang.src.tex
\end{block}

\usepackage{mumie.ombplus}
\ombchapter{12}
\ombarticle{3}
\usepackage{mumie.genericvisualization}

\begin{visualizationwrapper}

\title{\lang{de}{Rang einer Matrix}\lang{en}{Rank of a matrix}}
 
%\begin{block}[annotation]
%  übungsinhalt
%  
%\end{block}
% \begin{block}[annotation]
%   Im Ticket-System: \href{http://team.mumie.net/issues/18625}{Ticket 18625}\\
% \end{block}

\begin{block}[annotation]
  Die Inhalte dieses Kapitels standen zuvor in \href{http://team.mumie.net/issues/9061}{Ticket 9061}.
  Die Ausgliederung erhöht die Übersichtlichkeit.
\end{block}

\begin{block}[info-box]
\tableofcontents
\end{block}


\lang{de}{
Die 
\link{content_30_basen_eigenschaften}{lineare Unabhängigkeit}
haben wir bereits im Kapitel Vektorrechnung kennengelernt.
\\\\
In diesem Kapitel führen wir zunächst den \textbf{Zeilenrang} einer Matrix ein,
der die maximale Anzahl linear unabhängiger Zeilenvektoren angibt.
Zur numerischen Bestimmung nutzen wir das im 
\link{gauss-verfahren}{letzten Unterkapitel}
kennengelernte Gauß-Verfahren. Anschließend bringen wir den Rang der Koeffizientenmatrix eines linearen Gleichungssystems mit dessen Lösbarkeit in Verbindung.
}
\lang{en}{
The concept of \link{content_30_basen_eigenschaften}{linear independence} was introduced in the 
the chapter on vectors.
\\\\
In this chapter we introduce the \textbf{row rank} of a matrix, which gives the maximum number of 
linearly independent row vectors in the matrix. To numerically determine the rank, we employ the 
\link{gauss-verfahren}{Gaussian elimination} algorithm from the previous section. Finally, we look 
at the relationship between the rank of a coefficient matrix and the solvability of a linear system.
}



\section{\lang{de}{Zeilenrang und Spaltenrang}
         \lang{en}{Row rank and column rank}}\label{sec:zeilenrang-spaltenrang}

\begin{definition}[\lang{de}{Zeilenrang/Spaltenrang}\lang{en}{Row/column rank}]
\lang{de}{
Für eine $(m\times n)$-Matrix $A$ über $\R$ ist der \notion{Zeilenrang von $A$} die maximale Anzahl 
an linear unabhängigen Zeilenvektoren.\\
Analog ist der \notion{Spaltenrang von $A$} definiert als die maximale Anzahl an linear unabhängigen 
Spaltenvektoren.
}
\lang{en}{
Given an $(m\times n)$-matrix $A$ over $\R$, the \notion{row rank of $A$} is the maximum number of 
linearly independent row vectors contained in the matrix.\\
Analogously, the \notion{column rank} of $A$ is defined as the maximum number of linearly independent 
column vectors contained in the matrix.\\
Note that \textit{'contained'} here means \textit{'corresponding to a full row/column of the matrix'.}
}
\end{definition}
\lang{de}{
Mit dem \ref[gauss-verfahren][Gauß-Verfahren]{sec:gauss-mit-matrizen} kann man beliebige Matrizen 
durch elementare Zeilenumformungen in Stufenform bringen.
}
\lang{en}{
Using \ref[gauss-verfahren][Gaussian elimination]{sec:gauss-mit-matrizen} we can change any matrix 
into row echelon form using only elementary row operations.
}
\begin{remark}
\lang{de}{
Der Zeilenrang von $A$ entspricht der Anzahl der Zeilen, die ungleich der Nullzeile sind, nachdem $A$ 
durch das Gauß-Verfahren auf Stufenform gebracht worden ist.\\
Der Spaltenrang von $A$ ist folglich der Zeilenrang von $A^T$.
}
\lang{en}{
The row rank of $A$ is equal to the number of non-zero rows in $A$, after $A$ has been changed into 
row echelon form by Gaussian elimination.\\
The column rank of $A$ is the row rank of $A^T$.
}
\end{remark}

\begin{example}
\lang{de}{Wir betrachten die $(4\times 5)$-Matrix}
\lang{en}{We consider the $(4\times 5)$-matrix}
\[A=
\begin{pmatrix}
    1 & 2 & 1 & -1 & 7\\ 
    1 & 3 & 0 & -1 & 8\\
    -1 & 0 & -3 & 5 & 9 \\
    1 & 1 & 2 & -1 & 6
   \end{pmatrix}. \]
\lang{de}{Durch elementare Zeilenumformungen erhalten wir}
\lang{en}{Elementary row operations give us}
\[  \rightsquigarrow~~
  \begin{pmatrix}
    1 & 2 & 1 & -1 & 7\\ 
    0 & 1 & -1& 0  & 1\\
    0 & 2 & -2 & 4 & 16 \\
	0 & -1& 1 &  0 & -1    
   \end{pmatrix}
\rightsquigarrow~~
   \begin{pmatrix}
     1 & 2 & 1 & -1 & 7\\ 
     0 & 1 & -1& 0  & 1\\
   	 0 & 0 & 0 &4 &  14 \\
	 0 & 0 & 0 & 0 & 0
   \end{pmatrix}.   
\]
\lang{de}{
Die Stufenform besitzt also drei Zeilen ungleich $0$, weshalb der Zeilenrang der Matrix $A$ gleich $3$ ist.
\\\\
Für den Spaltenrang von $A$ bringen wir
}
\lang{en}{
The row echelon form of $A$ has three non-zero rows, so the row rank of the matrix $A$ is $3$.
\\\\
To find the column rank, we change
}
\[A^T= \begin{pmatrix} 1&1&-1&1 \\ 2&3&0&1\\ 1&0&-3&2\\ -1&-1&5&-1\\ 7&8&9&6   \end{pmatrix}\]
\lang{de}{auf Stufenform:}
\lang{en}{into row echelon form:}
\[  \rightsquigarrow~~ \begin{pmatrix} 1&1&-1&1 \\ 0&1&2&-1\\ 0&-1&-2&1\\ 0&0&4&0\\ 0&1&16&-1 \end{pmatrix}
 \rightsquigarrow~~ \begin{pmatrix}1&1&-1&1 \\ 0&1&2&-1\\ 0&0&0&0\\ 0&0&4&0\\ 0&0&14&0\end{pmatrix}
 \rightsquigarrow~~ \begin{pmatrix}1&1&-1&1 \\ 0&1&2&-1\\ 0&0&14&0\\ 0&0&4&0\\0&0&0&0\end{pmatrix}
 \rightsquigarrow~~\begin{pmatrix}1&1&-1&1 \\ 0&1&2&-1\\ 0&0&14&0\\ 0&0&0&0\\0&0&0&0\end{pmatrix}.\]
\lang{de}{
Auch hier bleiben drei Zeilen ungleich der Nullzeile. Der Spaltenrang der Matrix $A$ ist also auch gleich $3$.
}
\lang{en}{
Here too we have three non-zero rows. The column rank of the matrix $A$ is hence also $3$.
}
\end{example}



\begin{theorem}
\lang{de}{Für jede $(m\times n)$-Matrix $A$ über $\R$ ist der Zeilenrang gleich dem Spaltenrang.}
\lang{en}{The row and column ranks of every $(m\times n)$-matrix $A$ over $\R$ are equal.}
\end{theorem}


\section{\lang{de}{Rang}\lang{en}{Rank}}\label{sec:rang}

\begin{definition}[\lang{de}{Rang}\lang{en}{Rank}]
\lang{de}{
Da der Zeilenrang und der Spaltenrang einer Matrix $A$ immer gleich sind, wird diese Zahl auch kurz 
\notion{Rang von $A$} genannt und mit Rang($A$) bezeichnet.
\\\\
Man sagt, dass eine $(m\times n)$-Matrix $A$ \notion{vollen Rang} hat, wenn der Rang von $A$ gleich 
der kleineren der beiden Zahlen $m$ und $n$ ist, d.\,h. wenn
Rang($A$)$=\min\{m,n\}$ gilt.\\
\floatright{\href{https://www.hm-kompakt.de/video?watch=816}{\image[75]{00_Videobutton_schwarz}}
\href{https://www.hm-kompakt.de/video?watch=817}{\image[75]{00_Videobutton_schwarz}}}\\\\
}
\lang{en}{
As the row rank and column rank of a matrix $A$ are always equal, we instead simply use the 
terminology \notion{rank of $A$} when referring to them, and denote it by rank{$A$}.
\\\\
We say that an $(m\times n)$-matrix $A$ has \notion{full rank} if the rank of $A$ is equal to the 
smaller of the two numbers $m$ and $n$, that is, if rank($A$)$=\min\{m,n\}$.
}
\end{definition}

\begin{example}
\lang{de}{Die $(4\times 5)$-Matrix}
\lang{en}{The $(4\times 5)$-matrix}
\[A=
\begin{pmatrix}
    1 & 2 & 1 & -1 & 7\\ 
    1 & 3 & 0 & -1 & 8\\
    -1 & 0 & -3 & 5 & 9 \\
    1 & 1 & 2 & -1 & 6
   \end{pmatrix} \]
\lang{de}{
aus obigem Beispiel hat Rang $3$. Der Rang ist also sowohl kleiner als die Zeilenzahl, als auch 
kleiner als die Spaltenzahl. Daher hat $A$ keinen vollen Rang.
}
\lang{en}{
from the above example has rank $3$. Its rank is less than both the number of rows and the number of 
columns of the matrix. Hence $A$ does not have full rank.
}
\end{example}

\begin{remark}\label{rem:sing_reg}
\lang{de}{
Eine quadratische $(n\times n)$-Matrix $A$ hat vollen Rang, wenn der Rang von $A$ gleich $n$ ist.\\
Diese Matrizen werden \notion{regulär} genannt. Wir nennen quadratische Matrizen \notion{singulär}, wenn sie keinen vollen Rang besitzen.\\
}
\lang{en}{
A square $(n\times n)$-matrix $A$ has full rank if it has rank equal to $n$.\\
Such a matrix is called \notion{regular}. We call a square matrix \notion{singular} if it does not 
have full rank.
}
\end{remark}
\lang{de}{
Quadratische Matrizen werden ausführlich im \link{content_07_quadratische_matrizen}{Kursteil 3b} behandelt.
}
\lang{en}{
Square matrices are covered in detail in \link{content_07_quadratische_matrizen}{chapter 3b}.
}
 

\begin{quickcheck}
    \type{input.number}
      \precision{3}
      \field{real}
      \begin{variables}
           \randint[Z]{m}{2}{4}
           \randint[Z]{n}{5}{6}
           \randint[Z]{k}{5}{6}
           \randint[Z]{l}{2}{4}
\function[calculate]{k4}{4*k}
\function[calculate]{l4}{4*l}
       \end{variables}
      \text{\lang{de}{
      Die Matrix 
  $A=\begin{pmatrix} \var{m} & \var{n} & 1\\ 0 &\var{k}&\var{l} \\ 0 & \var{k4} & c \end{pmatrix}$
      hat für $c=$\ansref keinen vollen Rang.\\
      \explanation{
      $A$ ist eine $(3 \times 3)$-Matrix. 
      Für $c=\var{l4}$ entspricht die dritte Zeile dem Vierfachen
      der zweiten Zeile. 
      Damit existiert nach Zeilenumformung eine Nullzeile. 
      Der Rang ist dadurch $Rang(A)=2$ und nicht voll.
      }}
      \lang{en}{
      The matrix 
  $A=\begin{pmatrix} \var{m} & \var{n} & 1\\ 0 &\var{k}&\var{l} \\ 0 & \var{k4} & c \end{pmatrix}$
      does not have full rank if $c=$\ansref.\\
      \explanation{
      $A$ is a $(3 \times 3)$-matrix. 
      If $c=\var{l4}$, the third row is equal to four times the second row. 
      In this case, Gaussian elimination will produce a row of $0$ entries. 
      The rank would then be $Rang(A)=2$, which is not full rank.
      }}}
  
    \begin{answer}
            \solution{l4}
      \end{answer}
\end{quickcheck}

\section{\lang{de}{Rang und Lösbarkeit eines LGS}\lang{en}{Rank and solvability of a linear system}}

\lang{de}{
Mit Hilfe des Rangs einer Koeffizientenmatrix lassen sich Aussagen über die Lösbarkeit des 
dazugehörigen linearen Gleichungssystems treffen.
}
\lang{en}{
Given the rank of a coefficient matrix, we may make statements about the solvability of the 
associated linear system.
}
\begin{theorem}\label{thm:rang_loesg_lgs}
\lang{de}{Ein lineares Gleichungssystem mit $m$ Gleichungen und $n$ Variablen besitzt ...}
\lang{en}{A linear system with $m$ equations and $n$ variables has ...}
\begin{itemize}
\item[1.] \lang{de}{
          \notion{keine Lösung}, wenn der Rang der Koeffizientenmatrix nicht dem Rang der erweiterten 
          Koeffizientenmatrix entspricht, also wenn
          \[\text{Rang}(A)\neq \text{Rang}(A|b).\]
          }
          \lang{en}{
          \notion{no solution} if the rank of its coefficient matrix is not equal to the rank of the 
          augmented matrix, i.e. if
          \[\text{rank}(A)\neq \text{rank}(A|b).\]
          }
\item[2.] \lang{de}{
          \notion{genau eine Lösung}, wenn der Rang von Koeffizientenmatrix und von erweiterter 
          Koeffizientenmatrix $n$ ist, also wenn
          \[\text{Rang}(A)=\text{Rang}(A|b)\,\,\, \text{und}\,\,\, \text{Rang}(A)=n. \]
          }
          \lang{en}{
          \notion{exactly one solution} if the ranks of its coefficient matrix and of the augmented 
          matrix are both equal to $n$, i.e. if
          \[\text{rank}(A)=\text{rank}(A|b)\,\,\, \text{and}\,\,\, \text{rank}(A)=n. \]
          }
\item[3.] \lang{de}{
          \notion{unendlich viele Lösungen}, wenn die Ränge von Koeffizientenmatrix und erweiterter 
          Koeffizientenmatrix zwar gleich, aber kleiner als $n$ sind, d.\,h.
          \[\text{Rang}(A)=\text{Rang}(A|b)\,\,\, \text{und}\,\,\, \text{Rang}(A)<n.\]
          }
          \lang{en}{
          \notion{infinitely many solutions} if the ranks of the coefficient matrix and of the 
          augmented matrix are equal to each other, but less than $n$, i.e. if
          \[\text{rank}(A)=\text{rank}(A|b)\,\,\, \text{and}\,\,\, \text{rank}(A)<n.\]
          }
\end{itemize}
\end{theorem}

\begin{proof*}[\lang{de}{Beweis Theorem}\lang{en}{Proof of theorem}]
\begin{showhide}
\begin{itemize}
\item[1.] \lang{de}{
          Wenn der Rang der Koeffizientenmatrix nicht dem Rang der erweiterten Koeffizientenmatrix 
          entspricht, so hat diese die reduzierte Stufenform
          }
          \lang{en}{
          If the rank of the coefficient matrix is not equal to the rank of the augmented matrix, 
          then the latter in reduced row echelon form is
          }
 \[ \begin{pmatrix}
    1 & 0 & \cdots & 0 & \star&| & b_1\\ 
     0 & 1 & \cdots & 0 & \star&| & b_2\\ 
    \vdots & \, & \ddots & \, & \vdots&| & \vdots\\ 
 0 & 0 & \cdots & 1 & \star&| & b_k\\ 
   0 & 0 & \cdots & 0 & 0&| & \vdots \\ 
  0 & 0 & \cdots & 0 & 0&| & b_{m}\\ 
  \end{pmatrix}\]
          \lang{de}{
          mit mindestens einem $b_i \neq 0$ für $k<i\leq m$, was im Widerspruch zu der Nullzeile der 
          linken Seite steht. Damit besitzt das dazugehörige LGS keine Lösung.\\
          Für die Lösbarkeit eines LGS muss der Rang der Koeffizientenmatrix folglich immer gleich 
          dem Rang der erweiterten Koeffizientenmatrix sein.
          }
          \lang{en}{
          with at least one $b_i \neq 0$ for $k<i\leq m$ with a row of $0$ entries on its left. This 
          corresponds to the equation $0=b_i \neq 0$, a contradiction, so the linear system has no 
          solutions.\\
          To be solvable, the coefficient matrix of a linear system must have the same rank as the 
          augmented matrix.
          }
\item[2.] \lang{de}{
          Wenn der Rang der Koeffizientenmatrix $n$ ist, so ist in ihrer reduzierten Stufenform 
          dieselbe Anzahl an Spalten und Zeilen ungleich Null. Die reduzierte Stufenform der 
          erweiterten Koeffizientenmatrix ist
          }
          \lang{en}{
          If the rank of the coefficient matrix is $n$, then in its reduced row echelon form, there 
          are $n$ non-zero rows and $n$ non-zero columns. The reduced row echelon form of the 
          augmented matrix is
          }
 \[ \begin{pmatrix}
    1 & 0 & \cdots & 0 & 0&| & b_1\\ 
     0 & 1 & \cdots & 0 & 0&| & b_2\\ 
    \vdots & \, & \ddots & \, & \vdots&| & \vdots\\ 
 0 & 0 & \cdots & 1 & 0&| & b_{n-1}\\ 
  0 & 0 & \cdots & 0 & 1&| & b_n\\
    0 & 0 & \cdots & 0 & 0&| & 0\\ 
     \vdots & \, & \cdots & & \vdots&| & \vdots\\  
  0 & 0 & \cdots & 0 & 0&| & 0\\ 
  \end{pmatrix}.\]
          \lang{de}{
          Das heißt, es existieren $n$ Stufenelemente (Einsen) zu $n$ Variablen und damit ist keine 
          von ihnen eine freie Variable.\\
          Eine eindeutige Lösung lässt sich hier direkt ablesen.
          }
          \lang{en}{
          This means that there exist $n$ pivot elements/entries and $n$ variables, hence there are 
          no free variables.\\
          A unique solution can be directly read from the reduced row echelon form in this case.
          }

\item[3.] \lang{de}{
          Wenn der Rang der (erweiterten) Koeffizientenmatrix $k$ kleiner als $n$ ist, so existieren 
          im Gegensatz zu 2. freie Variablen.
          }
          \lang{en}{
          If the ranks of the coefficient matrix and the augmented matrix are both $k$ and less than 
          $n$, then unlike in case 2., there exist free variables.
          }
 \[ \begin{pmatrix}
    1 & 0 & \cdots & 0 & \star&| & b_1\\ 
     0 & 1 & \cdots & 0 & \star&| & b_2\\ 
    \vdots & \, & \ddots & & \vdots&| & \vdots\\ 
 0 & 0 & \cdots & 1 & \star&| & b_k\\ 
    0 & 0 & \cdots & 0 & 0&| & 0\\ 
     \vdots & \, & \cdots & \, & \vdots&| & \vdots\\  
  0 & 0 & \cdots & 0 & 0&| & 0\\ 
  \end{pmatrix}.\]
          \lang{de}{
          Da freie Variablen als reellwertige Parameter gesetzt werden können, existieren unendlich 
          viele Lösungen.
          }
          \lang{en}{
          As free variables can be replaced with real-valued parameters, there exist infinitely many 
          solutions, found by varying these parameters.
          }
\end{itemize}
\end{showhide}
\end{proof*}

\begin{example}
\begin{tabs*}
\tab{\lang{de}{1. Beispiel}\lang{en}{Example 1}}
\lang{de}{Für das lineare Gleichungssystem}
\lang{en}{The linear system}
\begin{displaymath}
\begin{mtable}[\cellaligns{ccrcrcrcr}]
\text{(I)}&\qquad-&x&+&2y&+&3z&=&5\\
\text{(II)}&&&&y&+&4z&=&11\\
\text{(III)}&&&&&-&2z&=&-6
\end{mtable}
\end{displaymath}
\lang{de}{mit erweiterter Koeffizientenmatrix}
\lang{en}{has the augmented matrix}
\[ \begin{pmatrix} -1 & 2 & 3 & | & 5 \\0 & 1 & 4 & | & 11 \\0 & 0 & -2 & | & -6 \end{pmatrix} \]
\lang{de}{
gilt $\text{Rang}(A)=\text{Rang}(A|b)=3$, was auch der Anzahl der dazugehörigen Variablen entspricht. 
Das heißt, das LGS besitzt genau eine Lösung:
}
\lang{en}{
with $\text{rank}(A)=\text{rank}(A|b)=3$, which is equal to the number of variables in the system. 
Therefore the linear system has a unique solution:
}

\[\mathbb{L}=\left\{\begin{pmatrix}2\\ -1\\ 3\end{pmatrix} \right\}\]
\lang{de}{(siehe \link{gauss1}{letztes Kapitel}).}
\lang{en}{(see the \link{gauss1}{previous chapter}).}
\\

\tab{\lang{de}{2. Beispiel}\lang{en}{Example 2}}
\lang{de}{Für das lineare Gleichungssystem}
\lang{en}{The linear system}
\[ \begin{mtable}[\cellaligns{crcrcr}]
(I)&\qquad x & - & 4 \cdot  y & = & 10 \\
(II)& &  & 0 & =  & 0
\end{mtable} \]
\lang{de}{mit erweiterter Koeffizientenmatrix}
\lang{en}{has the augmented matrix}
\[ \begin{pmatrix} 1 & -4 & | & 10 \\0 &0 & | & 0 \end{pmatrix} \]
\lang{de}{
gilt $\text{Rang}(A)=\text{Rang}(A|b)=1$. Das LGS besitzt aber 2 Variablen und hat damit unendlich 
viele Lösungen:
}
\lang{en}{
with $\text{rank}(A)=\text{rank}(A|b)=1$. The linear system has two variables, so it has infinitely 
many solutions:
}

\[ \mathbb{L}= \left\{ \begin{pmatrix}10\\ 0 \end{pmatrix}+r\cdot \begin{pmatrix} 4 \\ 1 \end{pmatrix} \, \big| \, r\in \R \right\} \] 
\lang{de}{(siehe \link{gauss1}{letztes Kapitel}).}
\lang{en}{(see the \link{gauss1}{previous chapter}).}

\tab{\lang{de}{3. Beispiel}\lang{en}{Example 3}}
\lang{de}{Für das lineare Gleichungssystem}
\lang{en}{The linear system}
\begin{displaymath}
\begin{mtable}[\cellaligns{ccrcrcrcr}]
\text{(I)}&\qquad-&x&+&2y&+&3z&=&5\\
\text{(II)}&&&&y&+&4z&=&11\\
\text{(III)}&&&&&&0&=&-6
\end{mtable}
\end{displaymath}
\lang{de}{mit erweiterter Koeffizientenmatrix}
\lang{en}{has the augmented matrix}
\[ \begin{pmatrix} -1 & 2 & 3 & | & 5 \\0 & 1 & 4 & | & 11 \\0 & 0 & 0 & | & -6 \end{pmatrix} \]
\lang{de}{
gilt $\text{Rang}(A)=2$ und $\text{Rang}(A|b)=3$.\\
Das heißt, das LGS ist nicht lösbar und
}
\lang{en}{
with $\text{rank}(A)=2$ and $\text{rank}(A|b)=3$.\\
Therefore the linear system has no solutions:
}
\[\mathbb{L}=\emptyset.\]

\end{tabs*}
\end{example}

\end{visualizationwrapper}


\end{content}
