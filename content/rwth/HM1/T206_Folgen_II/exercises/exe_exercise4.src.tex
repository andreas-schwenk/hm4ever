\documentclass{mumie.element.exercise}
%$Id$
\begin{metainfo}
  \name{
    \lang{de}{Ü04: Folgenkonvergenz (komplexwertig)}
    \lang{en}{Exercise 4}
  }
  \begin{description} 
 This work is licensed under the Creative Commons License Attribution 4.0 International (CC-BY 4.0)   
 https://creativecommons.org/licenses/by/4.0/legalcode 

    \lang{de}{}
    \lang{en}{}
  \end{description}
  \begin{components}
  \end{components}
  \begin{links}
  \end{links}
  \creategeneric
\end{metainfo}
\begin{content}
\usepackage{mumie.ombplus}

\title{
  \lang{de}{Ü04: Folgenkonvergenz (komplexwertig)}
}


\begin{block}[annotation]
  Im Ticket-System: \href{http://team.mumie.net/issues/9848}{Ticket 9848}
\end{block}


\lang{de}{ Sei $(a_n)_{n\in\N}$ eine konvergente komplexe Folge mit Grenzwert $a\in \C $. Untersuchen Sie folgendene Folgen auf Konvergenz.
\begin{enumerate}[alph]
 \item $(b_n)_{n\in\N}$ mit \[b_n=\left(2+\frac{1}{n^2}\right)\overline{a_n}+i\left(1+\frac{1}{n}\right)^n\]
 \item $(c_n)_{n\in\N}$ mit \[c_n=(-1)^na_n\]
 \item $(d_n)_{n\in\N}$ mit \[d_n=a_{n+1}-a_n\] 
 \item $(e_n)_{n\in\N}$ mit \[e_n=a_{n+1}-\overline{a_n}\] 
\end{enumerate}
 }

\begin{tabs*}[\initialtab{0}\class{exercise}]
  \tab{
  \lang{de}{Antworten}
  \lang{en}{Answers}
  }


\begin{table}[\class{items}]
a) $b_n$ ist konvergent.\\
b) $c_n$ ist im allgemeinen (unbestimmt) divergent, aber konvergent gegen $0$, wenn $a=0$.\\
c) $d_n$ ist konvergent mit Grenzwert $0$.\\
d) $e_n$ ist konvergent mit Grenzwert $2\Im(a)$.
\end{table}

  \tab{\lang{de}{    Lösung a)    }}
  %#################################
  
  
  \begin{incremental}[\initialsteps{1}]
  
  
    \step 
    \lang{de}{   Da
  \begin{align*}
   &\lim_{n\to\infty}(1+\frac{1}{n})^n=e, \\
   &\lim_{n\to\infty}\frac{1}{n^2}=0, \\
   &\lim_{n\to\infty}\overline{a_n}=\overline{a},
  \end{align*}
 gilt mit den Grenzwertsätzen
  \[
   \lim_{n\to\infty}b_n=(2+0)\overline{a}+ie=2\overline{a}+ie.
  \]
  Die Folge $b_n$ ist also konvergent.
     }
     %#########################
     
  
  \end{incremental}
  


  \tab{\lang{de}{    Lösung b)    }}
  %#################################
  
  
  \begin{incremental}[\initialsteps{1}]
  
  
    \step 
    \lang{de}{  Falls $a=0$ gilt, so ist auch $c_n$ konvergent mit Grenzwert $0$. Für $a\neq 0$ betrachten wir
     die beiden Teilfolgen $c_{2n}=(-1)^{2n}a_n=a_n$ und $c_{2n-1}=(-1)^{2n-1}a_n=-a_n$. Es gilt
  \begin{align*}
   &\lim_{n\to\infty}c_{2n}=\lim_{n\to\infty}a_n=a\\
   &\lim_{n\to\infty}c_{2n-1}=\lim_{n\to\infty}-a_n=-a.
  \end{align*}
 Also gibt es Teilfolgen mit unterschiedlichen Grenzwerten und $(c_n)_{n\in\N}$ ist (unbestimmt) divergent.
    }
     %#########################
     
  
  \end{incremental}
  
  

  \tab{\lang{de}{    Lösung c)    }}
  %#################################
  
  
  \begin{incremental}[\initialsteps{1}]
  
  
    \step 
    \lang{de}{   Mit
  \[
   \lim_{n\to\infty}a_n=a
  \]
 gilt auch
  \[
   \lim_{n\to\infty}a_{n+1}=a.
  \]
 Damit folgt
  \[
   \lim_{n\to\infty}d_n=a-a=0
  \] und die Konvergenz der Folge.
    }
     %#########################
     
  
  \end{incremental}
  
  

  \tab{\lang{de}{    Lösung d)    }}
  %#################################
  
  
  \begin{incremental}[\initialsteps{1}]
  
  
    \step 
    \lang{de}{    Mit
  \[
   \lim_{n\to\infty}a_n=a
  \]
 gilt auch
  \[
   \lim_{n\to\infty}a_{n+1}=a.
  \]
 Da 
  \[
   \lim_{n\to\infty}\overline{a_n}=\overline{a}
  \]
 folgt
  \[
   \lim_{n\to\infty}e_n=a-\overline{a}=2\text{Im}(a).
  \]
  Die Folge $e_n$ ist also konvergent.

    }
     %#########################
     
  
  \end{incremental}
\end{tabs*}
\end{content}