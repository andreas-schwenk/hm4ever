\documentclass{mumie.element.exercise}
%$Id$
\begin{metainfo}
  \name{
    \lang{de}{Ü01: Bestimmte Divergenz}
    \lang{en}{Exercise 1}
  }
  \begin{description} 
 This work is licensed under the Creative Commons License Attribution 4.0 International (CC-BY 4.0)   
 https://creativecommons.org/licenses/by/4.0/legalcode 

    \lang{de}{}
    \lang{en}{}
  \end{description}
  \begin{components}
  \end{components}
  \begin{links}
\link{generic_article}{content/rwth/HM1/T205_Konvergenz_von_Folgen/g_art_content_15_monotone_konvergenz.meta.xml}{content_15_monotone_konvergenz}
\end{links}
  \creategeneric
\end{metainfo}
\begin{content}
\usepackage{mumie.ombplus}

\title{
  \lang{de}{Ü01: Bestimmte Divergenz}
}


\begin{block}[annotation]
  Im Ticket-System: \href{http://team.mumie.net/issues/9845}{Ticket 9845}
\end{block}



\lang{de}{ Zeigen Sie, dass folgende Folgen bestimmt gegen unendlich divergieren.
\begin{enumerate}[alph]
 \item $(a_n)_{n\in\N}$ mit \[a_n=n\left(1+\frac{1}{n}\right)^n\]
 \item $(b_n)_{n\in\N}$ mit \[b_n=\frac{n^3+7n^2+1}{n+1}\]
 \item $(c_n)_{n\in\N}$ mit \[c_n=\frac{n^2}{q^n},\] wobei $0<q<1$.
\end{enumerate}
 }

\begin{tabs*}[\initialtab{0}\class{exercise}]


  \tab{\lang{de}{    Lösung a)    }}
  %#################################
  
  
  \begin{incremental}[\initialsteps{1}]
  
  
    \step 
    \lang{de}{   Wie im \ref[content_15_monotone_konvergenz][letzten Kapitel]{thm:ezahl} behandelt, gilt 
  \[
   \lim_{n\to\infty}\left(1+\frac{1}{n}\right)^n=e.
  \]
 Außerdem gilt $\lim_{n\to\infty}n=\infty$. Mit den Rechenregeln zur bestimmten Divergenz folgt also auch $\lim_{n\to\infty}a_n=\infty$. Die Folge $a_n$ ist also bestimmt divergent.
     }
     %#########################
     
  
  \end{incremental}
  
    \tab{\lang{de}{    Lösung b)    }}
  %#################################
  
  
  \begin{incremental}[\initialsteps{1}]
  
  
    \step 
    \lang{de}{   Es gilt
  \[
   \lim_{n\to\infty}b_n=\lim_{n\to\infty}\frac{n^3+7n^2+1}{n+1}=\lim_{n\to\infty}\frac{n^2+7n+\frac{1}{n}}{1+\frac{1}{n}}.
  \]
 Mit den Rechenregeln zur bestimmten Divergenz folgt also $\lim_{n\to\infty}b_n=\infty$.
    }
     %#########################
     
  
  \end{incremental}
    
    \tab{\lang{de}{    Lösung c)    }}
  %#################################
  
  
  \begin{incremental}[\initialsteps{1}]
  
  
    \step 
    \lang{de}{   Wie im \ref[content_15_monotone_konvergenz][letzten Kapitel]{ex:beispiele} behandelt, gilt für alle $0<q<1$
  \[
   \lim_{n\to\infty}q^n=0,
  \]
  also mit $q^n>0$ für alle $n\in\N$
  \[
   \lim_{n\to\infty}\frac{1}{q^n}=\infty.
  \]
 Außerdem gilt $\lim_{n\to\infty}n^2=\infty$. Mit den Rechenregeln zur bestimmten Divergenz folgt also auch $\lim_{n\to\infty}a_n=\infty$.
    }
     %#########################
     
  
  \end{incremental}

\end{tabs*}
\end{content}