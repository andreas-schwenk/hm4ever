\documentclass{mumie.element.exercise}
%$Id$
\begin{metainfo}
  \name{
    \lang{de}{Ü02: Grenzwertberechnung}
    \lang{en}{Exercise 2}
  }
  \begin{description} 
 This work is licensed under the Creative Commons License Attribution 4.0 International (CC-BY 4.0)   
 https://creativecommons.org/licenses/by/4.0/legalcode 

    \lang{de}{}
    \lang{en}{}
  \end{description}
  \begin{components}
  \end{components}
  \begin{links}
  \end{links}
  \creategeneric
\end{metainfo}
\begin{content}
\usepackage{mumie.ombplus}

\title{
  \lang{de}{Ü02: Grenzwertberechnung}
}


\begin{block}[annotation]
  Im Ticket-System: \href{http://team.mumie.net/issues/9846}{Ticket 9846}
\end{block}



1. Untersuchen Sie folgende Folgen auf Konvergenz, bestimmte Divergenz und unbestimmte Divergenz.
\begin{enumerate}[alph]
 \item $(a_n)_{n\in\N}$ mit \[a_n=(-1)^n\frac{2n^2-1}{n^3+n^2+1}\]
 \item $(b_n)_{n\in\N}$ mit \[b_n=(-1)^n\frac{n^2+2n+1}{n+1}\]
 \item $(c_n)_{n\in\N}$ mit 
  \[
   c_n=\begin{cases}
        -5,&1\leq n\leq 100\\
        \frac{1}{n},&100< n\leq 10^6\\
        -n^2,&n> 10^6
       \end{cases}
  \]
\end{enumerate}

\begin{tabs*}[\initialtab{0}\class{exercise}]

 \tab{\lang{de}{Antworten}}
 1.\\
 a) $a_n$ ist konvergent gegen $0$.\\
 b) $b_n$ ist divergent.\\
 c) $c_n$ ist bestimmt divergent gegen $-\infty$.\\\\
 

  \tab{\lang{de}{    Lösung 1 a)    }}
  %#################################
  
  
  \begin{incremental}[\initialsteps{1}]
  
  
    \step 
    \lang{de}{  Es gilt
  \[
   \lim_{n\to\infty}\abs{\frac{2n^2-1}{n^3+n^2+1}}=\lim_{n\to\infty}\abs{\frac{\frac{2}{n}-\frac{1}{n^3}}{1+\frac{1}{n}+\frac{1}{n^3}}}=0.
  \]
 Also gilt $\lim_{n\to\infty}\abs{a_n}=\lim_{n\to\infty}a_n=0$. Die Folge ist also konvergent mit Grenzwert $0$.
     }
     %#########################
     
  
  \end{incremental}
  


  \tab{\lang{de}{    Lösung  1 b)    }}
  %#################################
  
  
  \begin{incremental}[\initialsteps{1}]
  
  
    \step 
    \lang{de}{   Es gilt
  \[
   \frac{n^2+2n+1}{n+1}=\frac{(n+1)^2}{n+1}=n+1,
  \]
 also
  \[
   b_n=(-1)^n(n+1).
  \]
 Genauso wie im Vorlesungsteil für $(-1)^nn$ gezeigt, kann man auch hier auf unbestimmte Divergenz schließen. 
    }
     %#########################
     
  
  \end{incremental}
  


  \tab{\lang{de}{    Lösung  1 c)    }}
  %#################################
  
  
  \begin{incremental}[\initialsteps{1}]
  
  
    \step 
    \lang{de}{   Da es bei Konvergenz nicht auf das Verhalten der ersten (endlich vielen) Folgenglieder ankommt, können wir alle Folgenglieder für $n\leq 10^6$ vernachlässigen und betrachten nur die Folgenglieder für $n> 10^6$. D.h. 
  \[
   c_n=-n^2.
  \]
 Es gilt damit $\lim_{n\to\infty}c_n=-\infty$. Die Folge $(c_n)_{n\in\mathbb{N}}$ ist demnach bestimmt divergent gegen $-\infty$.
    }
     %#########################
     
  
  \end{incremental}  
  
\end{tabs*}


2. Geben Sie den Grenzwert der folgenden Folgen in $\mathbb{R}\subset\{\pm\infty\}$ an.\\

 \begin{table}[\class{items}]
       \nowrap{a) $\big{(}\frac{2n-1}{4n+3}\big{)}_{n\in\mathbb{N}}$} 
     & \nowrap{b) $\big{(}\frac{3n}{n^2-3}\big{)}_{n\in\mathbb{N}}$} 
     & \nowrap{c) $\big{(}\frac{2n^2+3}{2-n^2}\big{)}_{n\in\mathbb{N}}$} \\
       \nowrap{d) $\big{(}\frac{n^2}{n+1}\big{)}_{n\in\mathbb{N}}$}
     & \nowrap{e) $\big{(}\frac{(n+2)^2}{2n^2+1}\big{)}_{n\in\mathbb{N}}$}
     & \nowrap{f) $\big{(}\frac{n^3-3n^2+1}{1-n^2}\big{)}_{n\in\mathbb{N}}$} \\
       \nowrap{g) $\big{(}\frac{4n+2}{(3n-1)^2}\big{)}_{n\in\mathbb{N}}$}
     & \nowrap{h) $\big{(}\frac{3n^2}{(2n+1)^2}\big{)}_{n\in\mathbb{N}}$}
     & \nowrap{i) $\big{(}\frac{n(4n-1)^2}{(2n+1)^3}\big{)}_{n\in\mathbb{N}}$} \\
    \end{table}
    
    

3. Geben Sie den Grenzwert der folgenden Folgen in $\mathbb{R}\subset\{\pm\infty\}$ an.\\

 \begin{table}[\class{items}]
       \nowrap{a) $\big{(}\frac{n^4}{2^n}\big{)}_{n\in\mathbb{N}}$} 
     & \nowrap{b) $\big{(}\frac{3^n}{n^3}\big{)}_{n\in\mathbb{N}}$} 
     & \nowrap{c) $\big{(}n^4\cdot (\frac{1}{3})^n\big{)}_{n\in\mathbb{N}}$} \\
       \nowrap{d) $\big{(}\frac{1}{\sqrt{n}+1}\big{)}_{n\in\mathbb{N}}$}
     & \nowrap{e) $\big{(}\frac{\sqrt{n}}{\sqrt[3]{n}+1}\big{)}_{n\in\mathbb{N}}$}
     & \nowrap{f) $\big{(}\frac{n^2}{\sqrt{2n^4+n}}\big{)}_{n\in\mathbb{N}}$} \\
       \nowrap{g) $\big{(}\sqrt[3]{n}\big{)}_{n\in\mathbb{N}}$}
     & \nowrap{h) $\big{(}\frac{n^2+n\cdot 2^n}{3^n}\big{)}_{n\in\mathbb{N}}$}
     & \nowrap{i) $\big{(}\frac{1-2^n}{n^3+1}\big{)}_{n\in\mathbb{N}}$} \\
    \end{table}
    
\begin{tabs*}[\initialtab{0}\class{exercise}]
  \tab{
  \lang{de}{Antworten}
  \lang{en}{Answers}
  }


\begin{table}[\class{items}]

2.\\
a) Folge konvergiert gegen $\frac{1}{2}$.\\
b) Folge konvergiert gegen $0$.\\
c) Folge konvergiert gegen $-2$.\\
d) Folge divergiert bestimmt gegen $\infty$.\\
e) Folge konvergiert gegen $\frac{1}{2}$.\\
f) Folge divergiert bestimmt gegen $-\infty$.\\
g) Folge konvergiert gegen $0$.\\
h) Folge konvergiert gegen $\frac{3}{4}$.\\
i) Folge konvergiert gegen $2$.\\\\

3.\\
a) Folge konvergiert gegen $0$.\\
b) Folge divergiert bestimmt gegen $\infty$.\\
c) Folge konvergiert gegen $0$.\\
d) Folge konvergiert gegen $0$.\\
e) Folge divergiert bestimmt gegen $\infty$.\\
f) Folge konvergiert gegen $\frac{1}{\sqrt{2}}$.\\
g) Folge divergiert bestimmt gegen $\infty$.\\
h) Folge konvergiert gegen $0$.\\
i) Folge divergiert bestimmt gegen $-\infty$.\\\\

\end{table}
 
    \tab{\lang{de}{Lösungsvideo 2 a) - i)}}	
    \youtubevideo[500][300]{f8KXBEqj1pg}
    
      \tab{\lang{de}{Lösungsvideo 3 a) - i)}}	

    \youtubevideo[500][300]{Om-E0R3AV2M}
    
\end{tabs*}
\end{content}