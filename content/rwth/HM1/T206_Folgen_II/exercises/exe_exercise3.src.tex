\documentclass{mumie.element.exercise}
%$Id$
\begin{metainfo}
  \name{
    \lang{de}{Ü03: Grenzwertberechnung (komplexwertig)}
    \lang{en}{Exercise 3}
  }
  \begin{description} 
 This work is licensed under the Creative Commons License Attribution 4.0 International (CC-BY 4.0)   
 https://creativecommons.org/licenses/by/4.0/legalcode 

    \lang{de}{}
    \lang{en}{}
  \end{description}
  \begin{components}
  \end{components}
  \begin{links}
  \end{links}
  \creategeneric
\end{metainfo}
\begin{content}
\usepackage{mumie.ombplus}

\title{
  \lang{de}{Ü03: Grenzwertberechnung (komplexwertig)}
}


\begin{block}[annotation]
  Im Ticket-System: \href{http://team.mumie.net/issues/9847}{Ticket 9847}
\end{block}



\lang{de}{ Untersuchen Sie folgende Folgen auf Konvergenz und geben Sie gegebenenfalls den Grenzwert an..
\begin{enumerate}[alph]
 \item $(a_n)_{n\in\N}$ mit \[a_n=1+\frac{i}{n}\]
 \item $(b_n)_{n\in\N}$ mit \[b_n=\frac{i}{n^2(1+2i)}\]
 \item $(c_n)_{n\in\N}$ mit \[c_n=\frac{n^3i}{n^3+2in^2}\]
 \item $(d_n)_{n\in\N}$ mit \[d_n=\frac{1+q^n}{1+p^n},\] für $\vert p\vert>1$ und $\vert q\vert<1$.
\end{enumerate} }

\begin{tabs*}[\initialtab{0}\class{exercise}]

  \tab{\lang{de}{    Antworten    }}
  
  a) $a_n$ konvergiert gegen $1$.\\
  b) $b_n$ konvergiert gegen $0$.\\
  c) $c_n$ konvergiert gegen $i$.\\
  d) $d_n$ konvergiert gegen $0$.


  \tab{\lang{de}{    Lösung a)    }}
  %#################################
  
  
  \begin{incremental}[\initialsteps{1}]
  
  
    \step 
    \lang{de}{   Es gilt nach Grenzwertsätzen: Da
  \[
   \lim_{n\to\infty}\frac{i}{n}=0
  \]
 gilt
  \[
   \lim_{n\to\infty}a_n=1.
  \]
  Die Folge $(a_n)_{n\in\mathbb{N}}$ ist demnach konvergent gegen $1$.
    }
     %#########################
     
  
  \end{incremental}
  


  \tab{\lang{de}{    Lösung b)    }}
  %#################################
  
  
  \begin{incremental}[\initialsteps{1}]
  
  
    \step 
    \lang{de}{   Es gilt nach Grenzwertsätzen
  \[
   \lim_{n\to\infty}\frac{i}{n^2(1+2i)}=\lim_{n\to\infty}\frac{1}{n^2}\cdot\lim_{n\to\infty}\frac{i}{(1+2i)}=0\cdot \frac{i}{(1+2i)}=0.
  \]
    }
     %#########################
     
  
  \end{incremental}
  
  

  \tab{\lang{de}{    Lösung c)    }}
  %#################################
  
  
  \begin{incremental}[\initialsteps{1}]
  
  
    \step 
    \lang{de}{   Es gilt nach Grenzwertsätzen
  \[
   \lim_{n\to\infty}\frac{n^3i}{n^3+2in^2}=\lim_{n\to\infty}\frac{i}{1+2i\frac{1}{n}}=\frac{i}{1+2i\cdot 0}=i.
  \]
    }
     %#########################
     
  
  \end{incremental}
  
  

  \tab{\lang{de}{    Lösung d)    }}
  %#################################
  
  
  \begin{incremental}[\initialsteps{1}]
  
  
    \step 
    \lang{de}{   Es gilt nach Grenzwertsätzen
  \[
   \frac{1+q^n}{1+p^n}=\frac{\frac{1}{p^n}+(\frac{q}{p})^n}{\frac{1}{p^n}+1}.
  \]
 Da $\vert\frac{q}{p}\vert<\vert\frac{1}{p}\vert<1$ folgt
  \[
   \lim_{n\to\infty}d_n=\frac{0+0}{0+1}=0.
  \]
    }
     %#########################
     
  
  \end{incremental}
\end{tabs*}
\end{content}