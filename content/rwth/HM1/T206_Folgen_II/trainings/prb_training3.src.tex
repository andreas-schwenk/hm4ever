\documentclass{mumie.problem.gwtmathlet}
%$Id$
\begin{metainfo}
  \name{
    \lang{de}{A03: Konvergenzuntersuchung (komplexwertig)}
    \lang{en}{}
  }
  \begin{description} 
 This work is licensed under the Creative Commons License Attribution 4.0 International (CC-BY 4.0)   
 https://creativecommons.org/licenses/by/4.0/legalcode 

    \lang{de}{}
    \lang{en}{}
  \end{description}
  \corrector{system/problem/GenericCorrector.meta.xml}
  \begin{components}
    \component{js_lib}{system/problem/GenericMathlet.meta.xml}{gwtmathlet}
  \end{components}
  \begin{links}
  \end{links}
  \creategeneric
\end{metainfo}
\begin{content}
\usepackage{mumie.genericproblem}

\lang{de}{
	\title{A03: Konvergenzuntersuchung (komplexwertig)}	
}


\begin{block}[annotation]
  Im Ticket-System: \href{http://team.mumie.net/issues/9851}{Ticket 9851}
\end{block}

\begin{problem}
	
	\randomquestionpool{1}{5}
	
    %%=======================================================================================0
    %%=================================AAAAAAAAAAAAAAAAAAA===================================0
    %%=======================================================================================0
    
	\begin{question}
	
		\begin{variables}
			
			\randint{a}{1}{9}
			\randint{b}{1}{9}
			
			\randint{ee}{2}{9}
			
			\function[normalize]{aa}{(a/ee^n)+(b*(z/n))}
			
			
			
		\end{variables}
	
		
      		\text{
				Entscheiden Sie, für welche $z \in \mathbb{C}$ die Folge $a_n = \var{aa}$ ($n\in \N$) konvergiert.
			}
    	\explanation{Gehen Sie genau so vor wie bei reellen Folgen und betrachten Sie $z$ als Parameter.}
    	\type{mc.unique} 
    	
    	\begin{choice}
    	 	\text{ Genau für $z=0$  }
  		 	\solution{false}
		\end{choice}
    	\begin{choice}
    	 	\text{ Für alle $z \in \mathbb{C}$  }
  		 	\solution{true}
		\end{choice}
    	\begin{choice}
    	 	\text{ Genau für $z\in \mathbb{C} \backslash \{ 0 \} $ }
  		 	\solution{false}
		\end{choice}
   	\begin{choice}
   	 	\text{ Es gibt kein solches $z\in \mathbb{C}$ }
 		 	\solution{false}
		\end{choice}
		
    \end{question}
    
    %%=======================================================================================0
    %%================================BBBBBBBBBBBBBBBBBBBBBBBB===============================0
    %%=======================================================================================0
    
    \begin{question}
	
		\begin{variables}
			
			\randint{a}{1}{9}
			\randint{b}{1}{9}
			
			\randint{ee}{2}{9}
			
			
			
			\function[normalize]{bb}{(a/ee^n)+(b*(z*n))}
			
			
			
		\end{variables}
	
		\text{
				Entscheiden Sie, für welche $z \in \mathbb{C}$ die Folge $a_n = \var{bb}$ ($n\in \N$) konvergiert.
			}
            \explanation{Gehen Sie genau so vor wie bei reellen Folgen und betrachten Sie $z$ als Parameter.}
    	\type{mc.unique} 
    	
    	\begin{choice}
    	 	\text{ Genau für $z=0$  }
  		 	\solution{true}
		\end{choice}
    	\begin{choice}
    	 	\text{ Für alle $z \in \mathbb{C}$  }
  		 	\solution{false}
		\end{choice}
    	\begin{choice}
    	 	\text{ Genau für $z\in \mathbb{C} \backslash \{ 0 \} $ }
  		 	\solution{false}
		\end{choice}
   	\begin{choice}
   	 	\text{ Es gibt kein solches $z\in \mathbb{C}$ }
 		 	\solution{false}
		\end{choice}
		
    \end{question}
    
    %%=======================================================================================0
    %%==============================CCCCCCCCCCCCCCCCCCCCCCCCCC===============================0
    %%=======================================================================================0
    \begin{question}
	
		\begin{variables}
			
			\randint{a}{1}{9}
			\randint{b}{1}{9}
			
			\randint{ee}{2}{9}
			
			
			
			\function[normalize]{cc}{(a*ee^n)+(b*(z/n))}
			
			
			
		\end{variables}
	
		\text{
				Entscheiden Sie, für welche $z \in \mathbb{C}$ die Folge $a_n = \var{cc}$ ($n\in \N$) konvergiert.
			}
            \explanation{Gehen Sie genau so vor wie bei reellen Folgen und betrachten Sie $z$ als Parameter.}
    	\type{mc.unique} 
    	
    	\begin{choice}
    	 	\text{ Genau für $z=0$  }
  		 	\solution{false}
		\end{choice}
    	\begin{choice}
    	 	\text{ Für alle $z \in \mathbb{C}$  }
  		 	\solution{false}
		\end{choice}
    	\begin{choice}
    	 	\text{ Genau für $z\in \mathbb{C} \backslash \{ 0 \} $ }
  		 	\solution{false}
		\end{choice}
   	\begin{choice}
   	 	\text{ Es gibt kein solches $z\in \mathbb{C}$ }
 		 	\solution{true}
		\end{choice}
		
    \end{question}
    
    %%=======================================================================================0
    %%==============================DDDDDDDDDDDDDDDDDDDDDDDD=================================0
    %%=======================================================================================0
    \begin{question}
	
		\begin{variables}
			
			\randint{a}{1}{9}
			\randint{b}{1}{9}
			\randint{c}{1}{9}
			\randint{d}{1}{9}
			
			
			\function[normalize]{dd}{z*(((a*n^3)+(b*i*n))/((c*n^3)+(d*n^2)))}
			
			
			
		\end{variables}
	
		\text{
				Entscheiden Sie, für welche $z \in \mathbb{C}$ die Folge $a_n = \var{dd}$ ($n\in \N$) konvergiert.
			}
            \explanation{Gehen Sie genau so vor wie bei reellen Folgen und betrachten Sie $z$ als Parameter.}
    	\type{mc.unique} 
    	
    	\begin{choice}
    	 	\text{ Genau für $z=0$  }
  		 	\solution{false}
		\end{choice}
    	\begin{choice}
    	 	\text{ Für alle $z \in \mathbb{C}$  }
  		 	\solution{true}
		\end{choice}
    	\begin{choice}
    	 	\text{ Genau für $z\in \mathbb{C} \backslash \{ 0 \} $ }
  		 	\solution{false}
		\end{choice}
   	\begin{choice}
   	 	\text{ Es gibt kein solches $z\in \mathbb{C}$ }
 		 	\solution{false}
		\end{choice}
		
    \end{question}
    
    %%=======================================================================================0
    %%===============================EEEEEEEEEEEEEEEEEEEEE===================================0
    %%=======================================================================================0
    \begin{question}
	
		\begin{variables}
			
			\randint{a}{1}{9}
			\randint{b}{1}{9}
			\randint{c}{1}{9}
			\randint{d}{1}{9}
			
			
			\function[normalize]{ff}{z*(((a*i*n^5)+(b*n^2))/((c*n^2)+(d*i*n^2)))}
			
		\end{variables}
	
		\text{
				Entscheiden Sie, für welche $z \in \mathbb{C}$ die Folge $a_n = \var{ff}$ ($n\in \N$) konvergiert.
			}
            \explanation{Gehen Sie genau so vor wie bei reellen Folgen und betrachten Sie $z$ als Parameter.}
    	\type{mc.unique} 
    	
    	\begin{choice}
    	 	\text{ Genau für $z=0$  }
  		 	\solution{true}
		\end{choice}
    	\begin{choice}
    	 	\text{ Für alle $z \in \mathbb{C}$  }
  		 	\solution{false}
		\end{choice}
    	\begin{choice}
    	 	\text{ Genau für $z\in \mathbb{C} \backslash \{ 0 \} $ }
  		 	\solution{false}
		\end{choice}
   	\begin{choice}
   	 	\text{ Es gibt kein solches $z\in \mathbb{C}$ }
 		 	\solution{false}
		\end{choice}
		
    \end{question}
    
    
\end{problem}

\embedmathlet{gwtmathlet}
\end{content}