%$Id:  $
\documentclass{mumie.article}
%$Id$
\begin{metainfo}
  \name{
    \lang{de}{Komplexe Folgen}
    \lang{en}{}
  }
  \begin{description} 
 This work is licensed under the Creative Commons License Attribution 4.0 International (CC-BY 4.0)   
 https://creativecommons.org/licenses/by/4.0/legalcode 

    \lang{de}{Beschreibung}
    \lang{en}{}
  \end{description}
  \begin{components}
  \end{components}
  \begin{links}
    \link{generic_article}{content/rwth/HM1/T203_komplexe_Zahlen/g_art_content_08aneu_komplexeZahlen_intro.meta.xml}{content_08aneu_komplexeZahlen_intro}
    \link{generic_article}{content/rwth/HM1/T203_komplexe_Zahlen/g_art_content_08bneu_komplexeZahlen_geom.meta.xml}{content_08bneu_komplexeZahlen_geom}
    \link{generic_article}{content/rwth/HM1/T206_Folgen_II/g_art_content_20_komplexe_folgen.meta.xml}{content_20_komplexe_folgen}
    \link{generic_article}{content/rwth/HM1/T205_Konvergenz_von_Folgen/g_art_content_15_monotone_konvergenz.meta.xml}{monot-konv}
    \link{generic_article}{content/rwth/HM1/T201neu_Vollstaendige_Induktion/g_art_content_02_vollstaendige_induktion.meta.xml}{vollst-ind}
  \end{links}
  \creategeneric
\end{metainfo}
\begin{content}
\usepackage{mumie.ombplus}
\ombchapter{6}
\ombarticle{2}

\lang{de}{\title{Komplexe Folgen}}
 
\begin{block}[annotation]
  Definition komplexer Folgen und deren Konvergenz; übertragung der reellen Konvergenz
  
\end{block}
\begin{block}[annotation]
  Im Ticket-System: \href{http://team.mumie.net/issues/9677}{Ticket 9677}\\
\end{block}

\begin{block}[info-box]
\tableofcontents
\end{block}


In diesem Abschnitt werden komplexe Folgen behandelt und die Konzepte für reelle Folgen auf
komplexe Folgen übertragen.

\section{Komplexe Folgen}

\begin{definition}
Eine \notion{komplexe} (oder komplexwertige) \notion{Folge} ist eine Funktion $a:\N\to \C$.\\
Statt $a:\N\to \C,n\mapsto a(n)$ bezeichnet man eine solche Folge üblicherweise mit 
\[ (a_n)_{n \in \N}, \] 
wobei $a_n:=a(n)$ der Funktionswert an der Stelle $n$ ist. Man nennt $a_n$ auch \notion{n-tes
Folgeglied} der Folge $(a_n)_{n \in \N}$.
\end{definition}

\begin{example}
\begin{enumerate}
\item Da wir die reellen Zahlen als Teilmenge der komplexen Zahlen auffassen, ist jede reelle Folge auch eine komplexe Folge.
\item Geometrische Folgen $(uq^n)_{n \in \N}$ mit $u,q\in \C\setminus\{0\}$ sind komplexe Folgen, z.B.
die Folge $(i^n)_{n \in \N}$, d.h. $(i, -1, -i, 1, i, -1, \ldots )$.
\item Die rekursiv definierte Folge  $(a_n)_{n \in \N}$ mit $a_1=i$ und
\[  a_{n+1}=1+\frac{a_n}{2}\quad \text{ für alle }n\in \N \]
ist eine komplexe Folge.
\end{enumerate}
\end{example}

Konvergenz und Beschränktheit komplexer Folgen definiert man wie für reelle Folgen, wobei hier der 
\ref[content_08bneu_komplexeZahlen_geom][Betrag komplexer Zahlen]{betragkz}
 ins Spiel kommt.

\begin{definition}\label{def:konvergent-komplexe-folge}
Eine Folge komplexer Zahlen $(a_n)_{n\in \N}$ heißt \notion{konvergent}, wenn eine Zahl $a \in \C$ existiert, so dass
die folgende Eigenschaft erfüllt ist:

 F"ur jedes $\epsilon > 0$ gibt es eine Zahl  \nowrap{$N_{\epsilon} \in \N $,}
  so dass \[ | a_n - a | < \epsilon \text{ f"ur alle } n \geq N_{\epsilon}. \]

Eine solche Zahl $a$ (sofern sie existiert) nennt man \notion{Limes} oder \notion{Grenzwert} der 
Folge  $(a_n)_{n\in \N}$ und schreibt
\[  \lim_{n\to\infty} a_n =a. \]
Man sagt auch: Die Folge $(a_n)_{n\in \N}$ konvergiert gegen $a$.

Eine komplexe Folge, die gegen $0$ konvergiert, nennt man \notion{Nullfolge}.

Eine komplexe Folge $(a_n)_{n\in \N}$ heißt \notion{ beschränkt}, wenn ihre Wertemenge $\{ a_n | n\in \N\}$ beschränkt ist, d.h. wenn es eine reelle Zahl $c>0$ gibt, so dass
\[ |a_n|\leq c   \text{ f"ur alle } n \in \N. \]
\end{definition}

Da die Menge der komplexen Zahlen eine Ebene bilden, sind die Begriffe \emph{monoton}~ und 
\emph{nach oben beschränkt}~ bzw. \emph{nach unten beschränkt}~ nicht sinnvoll. Wohl jedoch kann man 
von der Monotonie der Folge der Beträge $(\abs{a_n})_{n\in\N}$ sprechen, was man im 
\ref[content_20_komplexe_folgen][Beispiel]{ex:1} weiter unten sehen kann.

\begin{remark}
Eine komplexe Folge $(a_n)_{n\in \N}$ konvergiert also genau dann gegen $a \in \C$, wenn die
reelle Folge $( | a_n - a |)_{n\in \N}$ eine Nullfolge ist.
\end{remark}

\begin{example}\label{ex:1}
\begin{enumerate}
\item Für $|q|\leq 1$ ist die geometrische Folge $(uq^n)_{n \in \N}$ beschränkt, denn für alle $n \in \N$ gilt:
\[   |uq^n|=|u|\cdot |q|^n\leq |u|. \]
Für $|q|<1$ konvergiert diese Folge sogar gegen $0$, denn die reelle Folge $ (|u|\cdot |q|^n)_{n \in \N}$ ist eine Nullfolge (vgl. \ref[monot-konv][Abschnitt Monotone Konvergenz]{ex:beispiele}).
\item Betrachten wir die Folge $(a_n)_{n \in \N}$ mit $a_1=i$ und
\[  a_{n+1}=1+\frac{a_n}{2}\quad \forall n\in \N. \]
Wir zeigen mit vollständiger Induktion, dass die Folge beschränkt ist und genauer, dass $|a_n|\leq 2$ für alle $n\in \N$ gilt.

IA: Für $a_1=i$ ist $|a_1|=|i|=1\leq 2$.

IS: $n\to n+1$: Sei $|a_n|\leq 2$ für ein $n\in \N$. Dann gilt:
\[  |a_{n+1}|=|1+\frac{a_n}{2}|\leq |1|+|\frac{a_n}{2}|=1+\frac{|a_n|}{2}\leq 1+\frac{2}{2}=2.\]
Beim ersten $\leq$ haben wir dabei die Dreiecksungleichung verwendet und beim zweiten $\leq$ die Induktionsvoraussetzung $|a_n|\leq 2$.

Die Folge ist sogar konvergent mit Grenzwert $2$.
\begin{incremental}[\initialsteps{0}]
\step Für alle $n\in \N$ gilt nämlich
\[  |a_{n+1}-2|= |1+\frac{a_n}{2}-2|= |\frac{a_n-2}{2}|=\frac{|a_n-2|}{2}. \]
Die reelle Folge $(b_n)_{n\in \N}=(|a_n-2|)_{n\in \N}$ erfüllt also $b_1=|a_1-2|=|i-2|=\sqrt{5}$ und
$b_{n+1}=\frac{1}{2}b_n$. Sie ist also die geometrische Folge $(2\sqrt{5}(\frac{1}{2})^n)_{n\in \N}$ und 
konvergiert wegen $0<\frac{1}{2}<1$ gegen $0$. Also ist $2$ der Grenzwert der Folge $(a_n)_{n \in \N}$.
\end{incremental}
\end{enumerate}
\end{example}

\section{Konvergenzkriterium und Grenzwertregeln}\label{sec:konvergenzkriterium}


Ein wichtiges Kriterium für komplexe Folgen erhält man durch Betrachtung der 
\ref[content_08aneu_komplexeZahlen_intro][Real- und Imaginärteile]{def:real-imaginaer-teil}:

\begin{theorem}
Sei $(a_n)_{n\in \N}$ eine komplexe Folge und $a \in \C$.
\begin{enumerate}
\item Die Folge $(a_n)_{n\in \N}$ ist genau dann beschränkt, wenn die reellen Folgen
$(\Re(a_n))_{n\in \N}$ und $(\Im(a_n))_{n\in \N}$ beschränkt sind.
\item Die Folge $(a_n)_{n\in \N}$ konvergiert genau dann, wenn die reellen Folgen
$(\Re(a_n))_{n\in \N}$ und $(\Im(a_n))_{n\in \N}$ konvergieren. Ist $a=\lim_{n\to \infty}a_n$,
so gilt
\[ \lim_{n\to \infty} \Re(a_n) =\Re(a) \quad \text{und} \quad \lim_{n\to \infty} \Im(a_n) =\Im(a). \]
\item Konvergente komplexe Folgen sind beschränkt.
\item Ist die Folge $(a_n)_{n\in \N}$ konvergent mit Grenzwert $a \in \C$, so konvergiert auch jede
Teilfolge gegen $a$.
\end{enumerate}
\end{theorem}

\begin{example}
\begin{enumerate}
\item Die Folge $(i^n)_{n \in \N}$ ist beschränkt, denn $|i^n|=|i|^n=1^n=1$ für alle $n\in \N$.
Die Folge der Realteile ist $(b_n)_{n\in \N}$ mit
\[   b_n=\begin{cases} 0 & \text{falls }n \text{ ungerade}\\
(-1)^{n/2}& \text{falls }n \text{ gerade} \end{cases}, \]
denn für $n$ gerade, d.h. $n=2k$, ist $i^n=i^{2k}=(-1)^k=(-1)^{n/2}$ und für $n$ ungerade, d.h. $n=2k-1$, ist $i^n=i\cdot i^{2k-2}=i\cdot (-1)^{k-1}$.

Entsprechend ist die Folge  $(\Im(i^n))_{n\in \N}$ gegeben durch
\[   \Im(i^n)=\begin{cases} (-1)^{(n-1)/2} & \text{falls }n \text{ ungerade}\\
0& \text{falls }n \text{ gerade} \end{cases}. \]

Beide Folgen haben Werte im Intervall $[-1, 1]$, weshalb sie beschränkt sind.

Weder die Folge $(\Re(i^n))_{n\in \N}$ noch die Folge $(\Im(i^n))_{n\in \N}$ sind konvergent, da sie die divergente Folge $((-1)^k)_{k\in \N}$ als Teilfolge haben. Insbesondere ist auch die Folge
$(i^n)_{n \in \N}$ nicht konvergent.
\item Für die Folge $(a_n)_{n \in \N}$ mit $a_1=i$ und
\[  a_{n+1}=1+\frac{a_n}{2}\quad \forall n\in \N \]
ist die Folge der Realteile $(x_n)_{n \in \N}=(\Re(a_n))_{n \in \N}$ gegeben durch
$x_1=\Re(a_1)=\Re(i)=0$ und
\[ x_{n+1}=\Re(a_{n+1})=\Re (1+\frac{a_n}{2})=1+\frac{\Re(a_n)}{2}=1+\frac{x_n}{2}.\]
Induktiv erhält man damit für $x_n$ die \ref[vollst-ind][geometrische Summe]{rule:geom-summe}
\[ x_n=\sum_{k=0}^{n-2} (\frac{1}{2})^k =\frac{(\frac{1}{2})^{n-1}-1}{\frac{1}{2}-1}
=\frac{(\frac{1}{2})^{n-1}-1}{-\frac{1}{2}} =2\left(1-(\frac{1}{2})^{n-1}\right) .\]
Damit gilt:
\[ \lim_{n\to \infty} \Re(a_n)= \lim_{n\to \infty} x_n= \lim_{n\to \infty}2\left(1-(\frac{1}{2})^{n-1}\right) =2\cdot 1=2.\]

Die Folge der Imaginärteile $(y_n)_{n \in \N}=(\Im(a_n))_{n \in \N}$ ist gegeben durch
$y_1=\Im(a_1)=\Im(i)=1$ und 
\[ y_{n+1}=\Im(a_{n+1})=\Im (1+\frac{a_n}{2})=\frac{\Im(a_n)}{2}=\frac{y_n}{2}. \]
Dies ist also die geometrische Folge $(y_n)_{n \in \N}=((\frac{1}{2})^{n-1})_{n \in \N}$ und konvergiert gegen $0$.

Die Folge $a_n$ ist also konvergent mit Grenzwert
\[  \lim_{n\to \infty} a_n=  \lim_{n\to \infty}x_n+ i\cdot  \lim_{n\to \infty}y_n=2+i\cdot 0=2,\]
was die Rechnung in obigem Beispiel bestätigt. 
\end{enumerate}
\end{example}

\begin{quickcheckcontainer}
\randomquickcheckpool{1}{1}
\begin{quickcheck}
\type{input.function}
		\field{rational}
            
       	\begin{variables}
			\randint{x2}{2}{9}	
            \randint{a}{2}{7} 
            \randint{b}{2}{7} 
            \function{f}{-b*i}
		\end{variables}
	
		\lang{de}{
			\text{Gegeben sei die komplexe Folge $(a_n)_{n\in\N}$ mit $a_n=\frac{\var{x2}}{n}+\frac{\var{a}+\var{b}n}{in}$.\\\\
            
            Gegen welchen Grenzwert konvergiert die Folge $(a_n)_{n\in\N}$?\\\\
            
            $\lim\limits_{n\to\infty}{a_n}=$\ansref\\\\
                      
        }
       		}
		
    	\begin{answer}
		\solution{f}
        \checkAsFunction{f}{-10}{10}{100}
		\end{answer}
        
        \explanation{Betrachten Sie gemäß obiger Regel die Folge der Real- und Imaginärteile zunächst getrennt voneinander.} 
	\end{quickcheck}
\end{quickcheckcontainer}

\begin{remark}
In den meisten Fällen ist es nicht möglich, eine rekursiv definierte Folge so einfach in Real- und Imaginärteil zu zerlegen wie
im vorigen Beispiel. Auch für andere explizite Beispiele ist die Zerlegung in Real- und Imaginärteil oft nicht einfach.

Der vorige Satz dient jedoch oft für theoretische Konvergenz- und Grenzwertüberlegungen wie den folgenden Grenzwertregeln.
\end{remark}


Die Grenzwertregeln für reelle Folgen kann man auf komplexe Folgen übertragen:

\begin{rule}[Grenzwertregeln]
Seien  $(a_n)_{n\in \N}$, $(b_n)_{n\in \N}$ konvergente komplexe Folgen mit
$\lim_{n\to\infty} a_n =a$ und $\lim_{n\to\infty} b_n =b$. Dann gilt für alle $\alpha, \beta \in \C$:
\begin{enumerate}
\item Die Folge $( \alpha a_n + \beta b_n )_{n\in \N}$ konvergiert und \[ \lim_{n\to\infty} ( \alpha a_n + \beta b_n )
= \alpha \cdot a + \beta \cdot b. \]
\item Die Folge $( a_n \cdot b_n )_{n\in \N}$ konvergiert und \[\lim_{n\to\infty} ( a_n \cdot b_n ) = a \cdot b. \]
\item Falls $b_n \neq 0$ für alle $n \in \N$ und $b \neq 0$, so konvergiert die Folge $( a_n /b_n )_{n\in \N}$ und
\[  \lim_{n\to\infty} ( \frac{a_n}{b_n} )=\frac{a}{b}.\]
\end{enumerate}
Außerdem gilt:
\[  \lim_{n\to\infty} \bar{a_n}=\bar{a} \quad \text{und} \quad \lim_{n\to \infty} |a_n|=|a|. \]
\end{rule}

\begin{remark}
Bestimmte Divergenz (gegen $\pm\infty$) macht bei komplexen Folgen keinen Sinn.
\end{remark}

\end{content}