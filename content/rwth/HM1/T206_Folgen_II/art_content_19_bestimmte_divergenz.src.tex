%$Id:  $
\documentclass{mumie.article}
%$Id$
\begin{metainfo}
  \name{
    \lang{de}{Bestimmte Divergenz}
    \lang{en}{}
  }
  \begin{description} 
 This work is licensed under the Creative Commons License Attribution 4.0 International (CC-BY 4.0)   
 https://creativecommons.org/licenses/by/4.0/legalcode 

    \lang{de}{Beschreibung}
    \lang{en}{}
  \end{description}
  \begin{components}
  \component{generic_image}{content/rwth/HM1/images/g_img_00_video_button_schwarz-blau.meta.xml}{00_video_button_schwarz-blau}
\component{generic_image}{content/rwth/HM1/images/g_img_00_Videobutton_schwarz.meta.xml}{00_Videobutton_schwarz}
\end{components}
  \begin{links}
\link{generic_article}{content/rwth/HM1/T205_Konvergenz_von_Folgen/g_art_content_14_konvergenz.meta.xml}{content_14_konvergenz}
\end{links}
  \creategeneric
  \begin{taxonomy}
        \difficulty{0}
        \usage{}
        \objectives{apply,writing_math}
        \topic{analysis1}
  \end{taxonomy}
\end{metainfo}
\begin{content}
\usepackage{mumie.ombplus}
\ombchapter{6}
\ombarticle{1}

\lang{de}{\title{Bestimmte Divergenz}}
 
\begin{block}[annotation]
  Bestimmte Divergenz; Beispiele; Rechenregeln
  
\end{block}
\begin{block}[annotation]
  Im Ticket-System: \href{http://team.mumie.net/issues/9676}{Ticket 9676}\\
\end{block}

\begin{block}[info-box]
\tableofcontents
\end{block}

In den vergangenen Abschnitten hatten wir uns vor allem konvergente Folgen angeschaut und Folgen auf Konvergenz
untersucht. Im folgenden Abschnitt betrachten wir divergente Folgen, charakterisieren diese und untersuchen sie 
insbesondere auf \notion{bestimmte Divergenz}.

\section{Bestimmte Divergenz}
\label{def:bestimmteDivergenz}

\begin{definition}
Eine reelle Folge $(a_n)_{n\in \N}$ heißt \notion{bestimmt divergent} gegen 
$\; +\infty$ (oder kurz $\infty$), geschrieben
\[  \lim_{n\to \infty} a_n =+\infty, \]
wenn es für jedes $M\in \R$ ein $N_M\in \N$ gibt, so dass
\[    a_n>M  \quad \text{für alle }n\geq N_M. \]
Man sagt auch, dass die Folge  $(a_n)_{n\in \N}$ \notion{gegen $\infty$ strebt}.

Eine reelle Folge $(a_n)_{n\in \N}$ heißt \notion{bestimmt divergent} gegen 
$\; -\infty$, geschrieben
\[  \lim_{n\to \infty} a_n =-\infty, \]
wenn es für jedes $M\in \R$ ein $N_M\in \N$ gibt, so dass
\[    a_n<M  \quad \text{für alle }n\geq N_M. \]
Man sagt auch, dass die Folge  $(a_n)_{n\in \N}$ \notion{gegen $-\infty$ strebt}.\\
\floatright{\href{https://www.hm-kompakt.de/video?watch=308}{\image[75]{00_Videobutton_schwarz}}}\\\\
\end{definition}

\begin{remark}
\begin{enumerate}
\item 
Anschaulich gesprochen bedeutet bestimmte Divergenz gegen $+\infty$, dass für jede noch so große Schranke $M$ ab einer bestimmten Stelle alle Folgenglieder größer als diese Schranke sind.

Entsprechend bedeutet bestimmte Divergenz gegen $-\infty$, dass für jede noch so kleine Schranke $M$ (also betragsmäßig große, aber negative Zahl $M$) ab einer bestimmten Stelle alle Folgenglieder kleiner als diese Schranke sind.
\item Bestimmt divergente Folgen sind also nicht beschränkt und somit nicht konvergent. 
Bestimmt divergente Folgen sind erst recht divergent.
\end{enumerate}
\end{remark}

\begin{rule}
Eine Folge $(a_n)_{n\in \N}$ ist genau dann bestimmt divergent gegen $-\infty$, wenn die Folge $(-a_n)_{n\in \N}$ bestimmt divergent gegen $\infty$ ist.
\end{rule}

%\begin{quickcheckcontainer}
%\randomquickcheckpool{1}{1}
%\begin{quickcheck}
		%
     	%	\begin{variables}
		%	\randint{a}{1}{9}
		%	\randint{b}{1}{9}
		%	\randint{c}{1}{9}
		%	\randint{d}{1}{9}
		%	\randint{ee}{1}{9}
		%	\randint{f}{1}{9}
    %		\randint{g}{1}{9}
		%	\randint{h}{1}{9}
	%		\randint{j}{1}{9}
%			
	%		\function{aa}{a}
%			\function{ab}{b}
%			\function{ac}{c}
%
%			\function{ba}{(-1)^n/n}
	%		\function{bb}{((-1)^(2n+d))/n}
%
	%		\function{ca}{ee*1}
%			\function{cb}{((1+1/n)^n) * ((i*n^2+h)/((-(f*n)) - g))}
%
	%		\function{da}{((-1)^n)/n}
	%		\function{db}{((-1)^(2*j+n))/(n^2)}
	%		
	%	\end{variables}
	%
	%	\lang{de}{ 
 %     		\text{
%Untersuchen Sie die Folge $(a_n)_{n\in\N}$ auf Konvergenz, bestimmte Divergenz und unbestimmte Divergenz:\\
     % 							
	%			Die Folge ist gegeben durch \\
	%			$ a_n = \begin{cases} \var{ca},& \text{ falls } 1\leq n\leq 10, \\
	%			\var{cb},& \text{ falls } 10< n\end{cases}$
	%		}
    %	}
    % \explanation{Eine Folge hat unendlich viele Folgenglieder, so dass endlich viele Ausreißer
      %  - insbesondere die ersten endlich vielen Folgenglieder - auf die Konvergenz einer Folge keinen Einfluss haben. 
     %   Betrachten Sie außerdem die beiden Faktoren getrennt voneinander.}
    %	
    %	\type{mc.unique} 
    %	
    %	\begin{choice}
    %	 	\text{ Konvergenz }
  	%	 	\solution{false}
	%	\end{choice}
    %	\begin{choice}
    %	 	\text{ Bestimmte Divergenz gegen $\infty$ }
  	%	 	\solution{false}
	%	\end{choice}
    %	\begin{choice}
    %	 	\text{ Bestimmte Divergenz gegen $-\infty$ }
  	%	 	\solution{true}
	%	\end{choice}
    %	\begin{choice}
    %	 	\text{ Unbestimmte Divergenz }
  	%	 	\solution{false}
	%	\end{choice}
 %                    
%		\end{quickcheck}
%\end{quickcheckcontainer}

\begin{example}
\begin{enumerate}
\item Die Folge $(a_n)_{n\in \N}=(n)_{n\in \N}$ ist bestimmt divergent gegen $+\infty$.

Für jedes $M\in \R$ kann man nämlich als $N_M$ die kleinste natürliche Zahl wählen, die größer als $M$ ist, dann gilt für alle $n\geq N_M$
\[   a_n=n\geq N_M>M. \]
\item Die Folge $((-1)^nn)_{n\in \N}$ ist zwar divergent, aber nicht bestimmt divergent. 

Für beliebiges $M\geq 0$ ist nämlich $(-1)^nn=-n<0\leq M$ für alle ungeraden
$n$, weshalb es kein $N_M$ gibt, wie man es für die bestimmte Divergenz gegen $+\infty$ benötigt.

Für beliebiges $M<0$ ist andererseits $(-1)^nn=n>0>M$ für alle geraden
$n$, weshalb es hier kein $N_M$ gibt, wie man es für die bestimmte Divergenz gegen $-\infty$ benötigt.
\end{enumerate}
\end{example}

Für monotone Folgen tritt unbestimmte Divergenz nicht auf:

\begin{rule}
Jede monotone Folge ist konvergent oder bestimmt divergent. Genauer:
\begin{itemize}
\item Sie ist konvergent, wenn sie beschränkt ist.
\item Sie ist bestimmt divergent gegen $\infty$, wenn sie nach oben unbeschränkt ist. In dem Fall muss sie monoton wachsend sein.
\item Sie ist bestimmt divergent gegen $-\infty$, wenn sie nach unten unbeschränkt ist. In dem Fall muss sie monoton fallend sein.
\end{itemize}
\end{rule}



\section{Rechenregeln mit bestimmter Divergenz}\label{sec:rechenregeln}

\begin{rule}[\lang{de}{Kehrwerte von Folgen, die\\ gegen $\pm\infty$ oder $0$ streben}
\lang{en}{reciprocal of sequences\\ tending to $\pm\infty$ or $0$}
]
\begin{enumerate}
\item \lang{de}{Sei $(x_n)$ eine Folge mit $x_n\neq 0$, die gegen $\infty$ oder $-\infty$ strebt, d.h. }
\lang{en}{Let $(x_n)_{n\in \N}$ be a sequence with $x_n\neq 0$ which tends to $\infty$ or $-\infty$, i.e.,}
\[
\lim_{n\to\infty} x_n = \infty\qquad\text{
\lang{de}{oder}
\lang{en}{or}
}\qquad
\lim_{n\to\infty} x_n = -\infty.
\]
\lang{de}{Dann ist die Folge der Kehrwerte eine Nullfolge:}
\lang{en}{Then the sequence of reciprocal elements converges to zero:}
\[
\lim_{n\to\infty} \,\frac{1}{x_n}\, = 0. 
\]


\item 
\lang{de}{Der umgekehrte Schluss ist nur m"oglich, wenn wir das Vorzeichen der Folgenglieder kennen:}
\lang{en}{The inverse conclusion is only possible if we know the sign of the elements of the sequence:
}

\[
\lim_{n\to\infty} y_n = 0 , \quad y_n >0 \text{ für alle } n\in\N \quad \Longrightarrow\quad \lim_{n\to\infty} \,\frac{1}{y_n}\, = +\infty,
\]
\[
\lim_{n\to\infty} y_n = 0, \quad y_n <0 \text{ für alle } n\in\N \quad\Longrightarrow\quad \lim_{n\to\infty} \,\frac{1}{y_n}\, = -\infty.
\]
\lang{de}{Nat"urlich gen"ugt es, dass die Bedingungen $x_n\neq 0$, $y_n>0$ oder $y_n<0$ f"ur alle \nowrap{$n\geq n_0$,} \nowrap{$n_0$ geeignet gew"ahlt,}
erf"ullt sind.}
\lang{en}{Clearly, it is sufficient that the conditions $x_n\neq 0$, $y_n>0$ or $y_n<0$ are satisfied for all
\nowrap{$n\geq n_0$} for suitably chosen $n_0$.}
\end{enumerate}
\end{rule}

\begin{proof*}[Erklärung]
Intuitiv ist klar, wenn etwas immer größer wird, werden die Kehrwerte immer kleiner, tendieren also gegen $0$. Formal sieht man es folgendermaßen:

Für eine gegen $\infty$ bestimmt divergente Folge $(x_n)_{n\in \N}$ gibt es für jedes $\epsilon>0$ und $M=\frac{1}{\epsilon}>0$ eine natürliche Zahl $N_\epsilon\in \N$ mit 
\[ x_n> M \quad  \text{für alle }n\geq N_\epsilon. \]
Damit ist aber für alle $n\geq N_\epsilon$ auch
\[  0<\frac{1}{x_n}<\frac{1}{M}=\epsilon. \]
Also ist die Folge $\left(\frac{1}{x_n}\right)_{n\in \N}$ eine Nullfolge.

Für Folgen, die gegen $-\infty$ steben, überlegt man es sich mit negativen Vorzeichen genauso, oder indem man zum Negativen der Folge übergeht:
\[  \lim_{n\to \infty} x_n= -\infty \Rightarrow \lim_{n\to \infty} -x_n= \infty 
\Rightarrow \lim_{n\to \infty}  - \frac{1}{x_n}=0 \Rightarrow \lim_{n\to \infty}   \frac{1}{x_n}=0 .\]

Ist umgekehrt eine Nullfolge $(y_n)_{n\in \N}$ gegeben, so gibt es also für jedes
$\epsilon>0$ ein $N_\epsilon\in \N$ mit 
\[   |y_n|<\epsilon \quad  \text{für alle }n\geq N_\epsilon. \]
Das heißt für $M>0$ und $\epsilon=\frac{1}{M}$ gibt es ein $N_M\in \N$ mit 
\[ |\frac{1}{y_n}| >\frac{1}{\epsilon}=M \quad  \text{für alle }n\geq N_\epsilon. \]
Sind nun alle $y_n>0$, so ist diese Ungleichung gleichbedeutend mit
$\frac{1}{y_n}>M$, weshalb die Folge $(\frac{1}{y_n})_{n\in \N}$ gegen $\infty$ strebt.\\
Sind alle $y_n<0$, so ist diese Ungleichung gleichbedeutend mit$-\frac{1}{y_n}>M$, d.h. $\frac{1}{y_n}<-M$, weshalb die Folge $(\frac{1}{y_n})_{n\in \N}$ dann gegen $-\infty$ strebt.
\end{proof*}

\begin{block}[warning]
Ist $(y_n)_{n\in \N}$ eine Nullfolge, bei welcher unendlich viele Folgenglieder positiv und unendlich viele Folgenglieder negativ sind (und keines gleich $0$), 
so strebt die Folge $(\frac{1}{y_n})_{n\in \N}$ weder gegen $+\infty$ noch gegen
$-\infty$. Die Folge $(\frac{1}{y_n})_{n\in \N}$ ist zwar divergent, aber nicht bestimmt divergent. 
\end{block}

\begin{example}
Für $q>1$ ist die geometrische Folge $(q^n)_{n\in \N}$ bestimmt divergent gegen $+\infty$, d.h.
\[ \lim_{n\to \infty} q^n =\infty. \]

Für $q>1$ ist nämlich $0<\frac{1}{q}<1$. Wie im vorigen Abschnitt gezeigt ist die Folge $\left(\left(\frac{1}{q}\right)^n\right)_{n\in \N}$ eine Nullfolge
und alle Folgenglieder sind positiv. Also gilt
\[ \lim_{n\to \infty} q^n =\infty. \]

Für $q<-1$ ist die geometrische Folge $(q^n)_{n\in\N}$ entsprechend obiger Warnung zwar divergent, aber nicht bestimmt 
divergent.\\
Zwar ist die Folge $(\frac{1}{q^n})_{n\in\N}=((\frac{1}{q})^n)_{n\in\N}$ eine Nullfolge, doch hat sie unendlich viele Folgenglieder die 
positiv sowie negativ sind. Die Folge divergiert nicht bestimmt, denn $(q^{2k})_{k\in\N}$ und $(q^{2k+1})_{k\in\N}$ sind Teilfolgen von $(q^n)_{n\in\N}$. Es gelten:
\[\lim\limits_{n\to\infty}q^{2k}=\lim\limits_{n\to\infty}(q^2)^k\lim\limits_{n\to\infty}(q^k)^2=\infty\]
\[\lim\limits_{n\to\infty}q^{2k+1}=\lim\limits_{n\to\infty}q^{2k}\cdot q=-\infty\]
Somit divergiert $(q^n)_{n\in\N}$ für $q<-1$, ist aber weder bestimmt divergent gegen $+\infty$ noch gegen $-\infty$.
\end{example}


Einige Grenzwertregeln konvergenter Folgen können auch auf bestimmt divergente Folgen übertragen werden.

\begin{rule}
Sei $(a_n)_{n\in \N}$ eine Folge, die gegen $\infty$ strebt, und
$(b_n)_{n\in \N}$ eine konvergente Folge mit Grenzwert $b$ dann gelten:
\begin{enumerate}
\item Die Folge $(a_n+b_n)_{n\in \N}$ ist bestimmt divergent gegen $\infty$.
\item Ist $b>0$, so ist die Folge $(a_nb_n)_{n\in \N}$ bestimmt divergent gegen $\infty$.
\item Ist $b<0$, so  ist die Folge $(a_nb_n)_{n\in \N}$ bestimmt divergent gegen
$-\infty$.
\item Für $b=0$ lässt sich über die  Folge $(a_nb_n)_{n\in \N}$ keine allgemeine Konvergenzaussage machen.
\item Die Folge $(\frac{b_n}{a_n})_{n\geq n_0}$ ist eine Nullfolge.
\end{enumerate}
Für Folgen $(a_n)_{n\in \N}$, die bestimmt divergent gegen $-\infty$ sind, gelten entsprechend diese Regeln mit umgekehrten Vorzeichen.
\end{rule}


\begin{remark}
Salopp formuliert hat man also auch hier die Grenzwertregeln, die man für konvergente Folgen kennt, wobei man
\begin{eqnarray*}
	b+ \infty &=& \infty +b= \infty \quad \text{für alle }b\in \R, \\
	b\cdot \infty &=& \infty \quad \text{für }b>0, \\
	b\cdot \infty &=& -\infty \quad \text{für }b<0, \\
		\frac{b}{\infty} &=& 0 	 \quad \text{für alle }b\in \R.
\end{eqnarray*}
rechnet.

Zu beachten ist aber, dass "`$0\cdot \infty$"' nicht definiert ist. Als Produkt einer
Nullfolge und einer gegen $\infty$ strebenden Folge kann nämlich jede beliebige Folge auftreten.
\begin{incremental}[\initialsteps{0}]
\step Um dies einzusehen, starten wir mit einer beliebigen Folge $(c_n)_{n\in \N}$. Dann wählen wir die Folge
$(a_n)_{n\in \N}$ mit $a_n=n\cdot \max\{1,|c_n|\}$ für alle $n\in \N$, sowie
die Folge $(b_n)_{n\in \N}$ mit $b_n=\frac{c_n}{a_n}$ für alle $n\in \N$.

Da $a_n\geq n$ für alle $n\in \N$ gilt, ist $(a_n)_{n\in \N}$ bestimmt divergent gegen $\infty$.\\
Die Folge $(b_n)_{n\in \N}$ hingegen ist eine Nullfolge, denn
\[  |b_n|=|\frac{c_n}{a_n}|\leq \frac{|c_n|}{n|c_n|}=\frac{1}{n} \]
für alle $n\in \N$.

Das Produkt der beiden Folgen, ist aber $(a_nb_n)_{n\in \N}=(c_n)_{n\in \N}$.
\end{incremental}
\end{remark}

\begin{quickcheckcontainer}
\randomquickcheckpool{1}{1}
\begin{quickcheck}
\type{input.number}
		\field{rational}
            
       	\begin{variables}
			\randint{x2}{2}{4}	
            \randint{a}{2}{4}  
            \number{loes0}{-infty}
            \number{loes1}{infty}
		\end{variables}
	
		\lang{de}{
			\text{Gegeben seien die beiden Folgen $(a_n)_{n\in\N}$ mit $a_n=-\frac{\var{x2}+\var{a} n}{n}$ und $(b_n)_{n\in\N}$ 
            mit $b_n=\frac{\var{x2}-n^\var{a}}{n+\var{x2}}$.\\\\
            
            Bestimmen Sie die folgenden Limiten:\\\\
            
            $\lim\limits_{n\to\infty}{a_n+b_n}=$\ansref\\\\
            
             $\lim\limits_{n\to\infty}{a_n\cdot b_n}=$\ansref\\\\
        }
       		}
		
    	\begin{answer}
		\solution{loes0}
		\end{answer}
            
		\begin{answer}
		\solution{loes1}
		\end{answer}
        
        \explanation{Wenden Sie die obigen Regeln auf die beiden Folgen $(a_n)_{n\in\N}$ und $(b_n)_{n\in\N}$ an.} 
	\end{quickcheck}
\end{quickcheckcontainer}

\begin{example}
\lang{de}{Betrachtet man die Folge $(a_n)_{n\in \N}$ mit 
\[ a_n=\frac{n^2+2n-1}{2n-1}=\frac{n+2-\frac{1}{n}}{2-\frac{1}{n}} \]
für alle $n\in \N$. Dann erhält man
\[ \lim_{n\to \infty} a_n =\lim_{n\to \infty}  \frac{n+2-\frac{1}{n}}{2-\frac{1}{n}}
= \frac{\infty +2 - 0}{2 -0}=\frac{\infty}{2} = \frac{1}{2}\cdot \infty=\infty.\]
Die Folge ist also bestimmt divergent gegen $\infty$.

Würde man direkt mit dem ursprünglichen Bruch rechnen, erhielte man
\[ \lim_{n\to \infty} a_n =\lim_{n\to \infty} \frac{n^2+2n-1}{2n-1}= 
\frac{\infty+\infty -1}{\infty -1},\]
was keinen Sinn ergibt.
Auch mit den folgenden Regeln lässt sich dies nicht bestimmen, weshalb man die vorigen Umformungen machen sollte.}
\end{example}

Ähnliche Beispiele werden in diesem Video mitsamt der Wiederholung der Definition \lref{def:bestimmteDivergenz}{bestimmter Divergenz} erläutert:\\\\
\floatright{\href{https://api.stream24.net/vod/getVideo.php?id=10962-2-11268&mode=iframe&speed=true}
{\image[75]{00_video_button_schwarz-blau}}}\\\\

\begin{quickcheckcontainer}
\randomquickcheckpool{1}{1}
\begin{quickcheck}
\type{input.number}
		\field{rational}
            
       	\begin{variables}
			\randint{x2}{2}{9}	
            \randint{a}{2}{7} 
            \randint{b}{2}{7} 
            \number{loes1}{-infty}
		\end{variables}
	
		\lang{de}{
			\text{Gegeben sei die Folge $(a_n)_{n\in\N}$ mit 
            $a_n=-\frac{n^3+\var{x2}n^2-\var{a}n+\var{b}}{\var{x2}n^2-\var{b}}$.\\\\
            
            Bestimmen Sie den folgenden Genzwert:\\\\
            
            $\lim\limits_{n\to\infty}{a_n}=$\ansref\\\\
                      
        }
       		}
		
    	\begin{answer}
		\solution{loes1}
		\end{answer}
        
        \explanation{Verfahren Sie analog wie im vorausgehenden Beispiel.} 
	\end{quickcheck}
\end{quickcheckcontainer}

\begin{rule}
Seien $(a_n)_{n\in \N}$ und $(b_n)_{n\in \N}$ bestimmt divergente Folgen, dann
gelten:
\begin{enumerate}
\item Ist $\lim_{n\to \infty} a_n=\lim_{n\to \infty} b_n=\infty$, dann ist auch
\[ \lim_{n\to \infty} (a_n+b_n )=\infty.\]
\item Ist $\lim_{n\to \infty} a_n=\lim_{n\to \infty} b_n=-\infty$, dann ist auch
\[ \lim_{n\to \infty} (a_n+b_n )=-\infty.\]
\end{enumerate}
In den anderen Fällen lässt sich über die  Folge $(a_n+b_n)_{n\in \N}$ keine allgemeine Konvergenzaussage machen.

Für die Folge $(a_nb_n)_{n\in \N}$ gilt:
\begin{enumerate}
\item Ist $\lim_{n\to \infty} a_n=\lim_{n\to \infty} b_n=\infty$ oder
$\lim_{n\to \infty} a_n=\lim_{n\to \infty} b_n=-\infty$, dann ist
\[ \lim_{n\to \infty} (a_n\cdot b_n )=\infty.\]
\item Ist $\lim_{n\to \infty} a_n=\infty$ und $\lim_{n\to \infty} b_n=-\infty$, oder
ist $\lim_{n\to \infty} a_n=-\infty$ und $\lim_{n\to \infty} b_n=\infty$, dann ist
\[ \lim_{n\to \infty} (a_n\cdot b_n )=-\infty.\]
\end{enumerate}
\end{rule}


\begin{remark}
Salopp formuliert kann man also auch mit den "`Grenzwerten"' $\infty$ und $-\infty$ rechnen, wobei man
\begin{eqnarray*}
	\infty+ \infty &=& \infty \\
	\infty\cdot \infty &=& \infty
\end{eqnarray*}
rechnet und wegen $-\infty=(-1)\cdot \infty$ auch
\begin{eqnarray*}
	-\infty+ (-\infty) &=& -\infty \\
	\infty\cdot -\infty &=& -\infty\cdot \infty = -\infty \\
	-\infty\cdot -\infty &=& \infty
\end{eqnarray*}

Zu beachten ist aber , dass "'$\infty -\infty$"' nicht definiert ist. Auch ein Quotient
"`$\frac{\infty}{\infty}$"' ist nicht definiert. Als Differenz und Quotient zweier gegen $\infty$ strebenden Folgen kann nämlich jede beliebige Folge auftreten.
\end{remark}


\begin{tabs*}[\initialtab{0}]
  
\tab{\lang{de}{Beispiel}}
\lang{de}{In dem obigen Beispiel $(a_n)_{n\in \N}$ mit 
\[ a_n=\frac{n^2+2n-1}{2n-1} \]
für alle $n\in \N$ gilt für die Folge der Zähler
\[ \lim_{n\to \infty}  (n^2+2n-1) = \infty+\infty -1 =\infty. \]
Für die Folge der Nenner gilt
\[ \lim_{n\to \infty} (2n-1) = 2\cdot \infty -1 =\infty. \]
Für die Folge $(a_n)_{n\in \N}$ hätte man damit
\[  \lim_{n\to \infty} a_n =\frac{\infty}{\infty}, \]
was jedoch nicht definiert ist. Auf diese Weise lässt sich also der Grenzwert bzw.
die bestimmte Divergenz nicht berechnen.

Die Lösung in obigem Beispiel war, den Bruch $a_n$ mit $\frac{1}{n}$ zu erweitern, damit die Folge der neuen Nenner  konvergiert.}
\end{tabs*}

\begin{rule}
Sei $n\in\mathbb{N}$ sowie $a, Q\in\mathbb{R}$. Dann gilt für $Q>1$:
\[\lim_{n\to\infty} \frac{Q^n}{n^a}=\infty\]
\end{rule}

\begin{proof*}[Erklärung]
\begin{incremental}[\initialsteps{0}]
\step
Mit der Argumentation des Beweises zum \ref[content_14_konvergenz][Theorem]{thm:exp-schlaegt-pol} im letzten Kapitel erhalten
wir für $Q>1$, dass die Folge $(n^a(\frac{1}{Q})^n)_{n\in\N}$ eine Nullfolge ist. Zudem bemerken wir, dass in der Argumentation
gar keine Betragsstriche gebraucht werden: Die Folge ist eine monoton fallende positive Nullfolge, und demnach die Folge ihrer 
Kehrwerte bestimmt divergent gegen $+\infty$.
\end{incremental}
\end{proof*}

\begin{example}
\begin{enumerate}
\item Die Folge $(a_n)_{n\in\mathbb{N}}=\frac{3^n}{n^9}$ divergiert gemäß Regel bestimmt gegen unendlich. Entsprechend divergiert
$(b_n)_{n\in\mathbb{N}}=\frac{-3^n}{n^9}$ bestimmt gegen minus unendlich.
\item Es gilt: $\lim\limits_{n\to\infty}\frac{5^n}{3^n+n^7}=\infty$, da die Basis $3$ im Nenner kleiner als die Basis $5$ im Zähler ist.
\item Es gilt: $\lim\limits_{n\to\infty}\frac{3^n\cdot n^9}{n^2}=\lim\limits_{n\to\infty}\frac{3^n}{n^{-7}}=\infty$
\end{enumerate}
\end{example}

\end{content}