{"zxx":"%This work is licensed under the Creative Commons License Attribution 4.0 International (CC-BY 4.0)\n%https://creativecommons.org/licenses/by/4.0/legalcode\n\\documentclass[rgb]{standalone}\n\\usepackage{tkz-euclide}\n\\usepackage{amsmath}\n\\definecolor{myorange}{hsb}{0.0833, 1, 0.8}\n\\definecolor{mygreen}{hsb}{0.3333, 1, 0.8}\n\\definecolor{myblue}{hsb}{0.5833, 1, 0.8}\n\\definecolor{mymagenta}{hsb}{0.8333, 1, 0.8}\n\\begin{document}\n\t\\begin{tikzpicture}[scale=1.5, font=\\Large]\n\t    \\draw[ultra thick, -latex] (7.5,10.25) -- (7.5,9.25);\n\t    \\draw[ultra thick, -latex] (7.5,8.25) -- (7.5,7.25);\n\t    \\draw[ultra thick, -latex] (7.5,6.25) -- (7.5,5.25);\n\t    \\draw[ultra thick, -latex] (7.5,4.25) -- (7.5,3.25);\n\t    \\draw[ultra thick, -latex] (7.5,4.25) -- (1.25,3.25);\n\t    \\draw[ultra thick, -latex] (7.5,1.25) -- (0,0.25);\n\t    \\draw[ultra thick, -latex] (7.5,1.25) -- (5,0.25);\n\t    \\draw[ultra thick, -latex] (7.5,1.25) -- (10,0.25);\n\t    \\draw[ultra thick, -latex] (7.5,1.25) -- (15,0.25);\n\t    \\draw[thick, rounded corners, fill=myblue, fill opacity=0.1] (3,10.25) rectangle ++(9,1);\n\t    \\node[anchor=center] at (7.5,10.75) {Stammfunktion einer rationalen Funktion $R(x)$ bestimmen};\n\t    \\draw[thick, rounded corners, fill, fill opacity=0.1] (2.25,8.25) rectangle ++(10.5,1);\n\t    \\node[anchor=center] at (7.5,9) {Polynomdivision $R(x)=p(x)+r(x)$,};\n\t    \\node[anchor=center] at (7.5,8.5) {wobei $p$ ein Polynom und $r$ eine echt gebrochen rationale Funktion sind};\n\t    \\draw[thick, rounded corners, fill, fill opacity=0.1] (4.5,6.25) rectangle ++(6,1);\n\t    \\node[anchor=center] at (7.5,6.75) {Nenner von $r$ vollst\\\"andig faktorisieren};\n\t    \\draw[thick, rounded corners, fill, fill opacity=0.1] (5.5,4.25) rectangle ++(4,1);\n\t    \\node[anchor=center] at (7.5,4.75) {Partialbruchzerlegung};\n\t    \\draw[thick, rounded corners, fill, fill opacity=0.1] (4,1.25) rectangle ++(7,2);\n\t    \\draw[thick, rounded corners, fill, fill opacity=0.1] (-1,1.25) rectangle ++(4.5,2);\n\t    \\node[anchor=center] at (1.25,3) {Summanden vom Typ};\n\t    \\node[anchor=center] at (1.25,2.25) {$\\dfrac{1}{(x-a)^k}$};\n\t    \\node[anchor=center] at (1.25,1.5) {Regel 3.1 anwenden};\n\t    \\node[anchor=center] at (7.5,3) {Summanden vom Typ};\n\t    \\node[anchor=center] at (7.5,2.25) {$\\dfrac{rx+s}{(x^2+bx+c)^n}$};\n\t    \\node[anchor=center] at (7.5,1.5) {Wie in Bemerkung 3.5 beschrieben aufspalten};\n\t\t\\draw[thick, rounded corners, fill, fill opacity=0.1] (-2.25,-1.75) rectangle ++(4.5,2);\n\t\t\\draw[thick, rounded corners, fill, fill opacity=0.1] (2.75,-1.75) rectangle ++(4.5,2);\n\t\t\\draw[thick, rounded corners, fill, fill opacity=0.1] (7.75,-1.75) rectangle ++(4.5,2);\n\t\t\\draw[thick, rounded corners, fill, fill opacity=0.1] (12.75,-1.75) rectangle ++(4.5,2);\n\t\t\\node[anchor=center] at (0,0) {Summanden vom Typ};\n\t\t\\node[anchor=center] at (0,-0.75) {$\\dfrac{2x+b}{x^2+bx+c}$};\n\t\t\\node[anchor=center] at (0,-1.5) {Theorem 3.2 Fall 1};\n\t\t\\node[anchor=center] at (5,0) {Summanden vom Typ};\n\t\t\\node[anchor=center] at (5,-0.75) {$\\dfrac{1}{x^2+bx+c}$};\n\t\t\\node[anchor=center] at (5,-1.5) {Theorem 3.2 Fall 2.1};\n\t\t\\node[anchor=center] at (10,0) {Summanden vom Typ};\n\t\t\\node[anchor=center] at (10,-0.75) {$\\dfrac{2x+b}{(x^2+bx+c)^n},\\quad n>1$};\n\t\t\\node[anchor=center] at (10,-1.5) {Bemerkung 3.5};\n\t\t\\node[anchor=center] at (15,0) {Summanden vom Typ};\n\t\t\\node[anchor=center] at (15,-0.75) {$\\dfrac{1}{(x^2+bx+c)^n},\\quad n>1$};\n\t\t\\node[anchor=center] at (15,-1.5) {Komplizierte Methoden n\\\"otig};\n\t\\end{tikzpicture}\n\\end{document}","de":"\\documentclass[rgb]{standalone}\n\\usepackage{tkz-euclide}\n\\usepackage{amsmath}\n\\definecolor{myorange}{hsb}{0.0833, 1, 0.8}\n\\definecolor{mygreen}{hsb}{0.3333, 1, 0.8}\n\\definecolor{myblue}{hsb}{0.5833, 1, 0.8}\n\\definecolor{mymagenta}{hsb}{0.8333, 1, 0.8}\n\\begin{document}\n\t\\begin{tikzpicture}[scale=1.5, font=\\Large]\n\t    \\draw[ultra thick, -latex] (7.5,10.25) -- (7.5,9.25);\n\t    \\draw[ultra thick, -latex] (7.5,8.25) -- (7.5,7.25);\n\t    \\draw[ultra thick, -latex] (7.5,6.25) -- (7.5,5.25);\n\t    \\draw[ultra thick, -latex] (7.5,4.25) -- (7.5,3.25);\n\t    \\draw[ultra thick, -latex] (7.5,4.25) -- (1.25,3.25);\n\t    \\draw[ultra thick, -latex] (7.5,1.25) -- (0,0.25);\n\t    \\draw[ultra thick, -latex] (7.5,1.25) -- (5,0.25);\n\t    \\draw[ultra thick, -latex] (7.5,1.25) -- (10,0.25);\n\t    \\draw[ultra thick, -latex] (7.5,1.25) -- (15,0.25);\n\t    \\draw[thick, rounded corners, fill=myblue, fill opacity=0.1] (3,10.25) rectangle ++(9,1);\n\t    \\node[anchor=center] at (7.5,10.75) {Stammfunktion einer rationalen Funktion $R(x)$ bestimmen};\n\t    \\draw[thick, rounded corners, fill, fill opacity=0.1] (2.25,8.25) rectangle ++(10.5,1);\n\t    \\node[anchor=center] at (7.5,9) {Polynomdivision $R(x)=p(x)+r(x)$,};\n\t    \\node[anchor=center] at (7.5,8.5) {wobei $p$ ein Polynom und $r$ eine echt gebrochen rationale Funktion sind};\n\t    \\draw[thick, rounded corners, fill, fill opacity=0.1] (4.5,6.25) rectangle ++(6,1);\n\t    \\node[anchor=center] at (7.5,6.75) {Nenner von $r$ vollst\\\"andig faktorisieren};\n\t    \\draw[thick, rounded corners, fill, fill opacity=0.1] (5.5,4.25) rectangle ++(4,1);\n\t    \\node[anchor=center] at (7.5,4.75) {Partialbruchzerlegung};\n\t    \\draw[thick, rounded corners, fill, fill opacity=0.1] (4,1.25) rectangle ++(7,2);\n\t    \\draw[thick, rounded corners, fill, fill opacity=0.1] (-1,1.25) rectangle ++(4.5,2);\n\t    \\node[anchor=center] at (1.25,3) {Summanden vom Typ};\n\t    \\node[anchor=center] at (1.25,2.25) {$\\dfrac{1}{(x-a)^k}$};\n\t    \\node[anchor=center] at (1.25,1.5) {Regel 3.1 anwenden};\n\t    \\node[anchor=center] at (7.5,3) {Summanden vom Typ};\n\t    \\node[anchor=center] at (7.5,2.25) {$\\dfrac{rx+s}{(x^2+bx+c)^n}$};\n\t    \\node[anchor=center] at (7.5,1.5) {Wie in Bemerkung 3.5 beschrieben aufspalten};\n\t\t\\draw[thick, rounded corners, fill, fill opacity=0.1] (-2.25,-1.75) rectangle ++(4.5,2);\n\t\t\\draw[thick, rounded corners, fill, fill opacity=0.1] (2.75,-1.75) rectangle ++(4.5,2);\n\t\t\\draw[thick, rounded corners, fill, fill opacity=0.1] (7.75,-1.75) rectangle ++(4.5,2);\n\t\t\\draw[thick, rounded corners, fill, fill opacity=0.1] (12.75,-1.75) rectangle ++(4.5,2);\n\t\t\\node[anchor=center] at (0,0) {Summanden vom Typ};\n\t\t\\node[anchor=center] at (0,-0.75) {$\\dfrac{2x+b}{x^2+bx+c}$};\n\t\t\\node[anchor=center] at (0,-1.5) {Theorem 3.2 Fall 1};\n\t\t\\node[anchor=center] at (5,0) {Summanden vom Typ};\n\t\t\\node[anchor=center] at (5,-0.75) {$\\dfrac{1}{x^2+bx+c}$};\n\t\t\\node[anchor=center] at (5,-1.5) {Theorem 3.2 Fall 2.1};\n\t\t\\node[anchor=center] at (10,0) {Summanden vom Typ};\n\t\t\\node[anchor=center] at (10,-0.75) {$\\dfrac{2x+b}{(x^2+bx+c)^n},\\quad n>1$};\n\t\t\\node[anchor=center] at (10,-1.5) {Bemerkung 3.5};\n\t\t\\node[anchor=center] at (15,0) {Summanden vom Typ};\n\t\t\\node[anchor=center] at (15,-0.75) {$\\dfrac{1}{(x^2+bx+c)^n},\\quad n>1$};\n\t\t\\node[anchor=center] at (15,-1.5) {Komplizierte Methoden n\\\"otig};\n\t\\end{tikzpicture}\n\\end{document}","en":"\\documentclass[rgb]{standalone}\n\\usepackage{tkz-euclide}\n\\usepackage{amsmath}\n\\definecolor{myorange}{hsb}{0.0833, 1, 0.8}\n\\definecolor{mygreen}{hsb}{0.3333, 1, 0.8}\n\\definecolor{myblue}{hsb}{0.5833, 1, 0.8}\n\\definecolor{mymagenta}{hsb}{0.8333, 1, 0.8}\n\\begin{document}\n\t\\begin{tikzpicture}[scale=1.5, font=\\Large]\n\t    \\draw[ultra thick, -latex] (7.5,10.25) -- (7.5,9.25);\n\t    \\draw[ultra thick, -latex] (7.5,8.25) -- (7.5,7.25);\n\t    \\draw[ultra thick, -latex] (7.5,6.25) -- (7.5,5.25);\n\t    \\draw[ultra thick, -latex] (7.5,4.25) -- (7.5,3.25);\n\t    \\draw[ultra thick, -latex] (7.5,4.25) -- (1.25,3.25);\n\t    \\draw[ultra thick, -latex] (7.5,1.25) -- (0,0.25);\n\t    \\draw[ultra thick, -latex] (7.5,1.25) -- (5,0.25);\n\t    \\draw[ultra thick, -latex] (7.5,1.25) -- (10,0.25);\n\t    \\draw[ultra thick, -latex] (7.5,1.25) -- (15,0.25);\n\t    \\draw[thick, rounded corners, fill=myblue, fill opacity=0.1] (3,10.25) rectangle ++(9,1);\n\t    \\node[anchor=center] at (7.5,10.75) {Finding an antiderivative of a rational function $R(x)$};\n\t    \\draw[thick, rounded corners, fill, fill opacity=0.1] (2.25,8.25) rectangle ++(10.5,1);\n\t    \\node[anchor=center] at (7.5,9) {Polynomial division $R(x)=p(x)+r(x)$,};\n\t    \\node[anchor=center] at (7.5,8.5) {where $p$ is a polynomial and $r$ is a real rational function};\n\t    \\draw[thick, rounded corners, fill, fill opacity=0.1] (4.5,6.25) rectangle ++(6,1);\n\t    \\node[anchor=center] at (7.5,6.75) {Fully factorise the denominator of $r$};\n\t    \\draw[thick, rounded corners, fill, fill opacity=0.1] (4.5,4.25) rectangle ++(6,1);\n\t    \\node[anchor=center] at (7.5,4.75) {Partial fraction decomposition};\n\t    \\draw[thick, rounded corners, fill, fill opacity=0.1] (4,1.25) rectangle ++(7,2);\n\t    \\draw[thick, rounded corners, fill, fill opacity=0.1] (-1,1.25) rectangle ++(4.5,2);\n\t    \\node[anchor=center] at (1.25,3) {Summands of the form};\n\t    \\node[anchor=center] at (1.25,2.25) {$\\dfrac{1}{(x-a)^k}$};\n\t    \\node[anchor=center] at (1.25,1.5) {Apply rule 3.1};\n\t    \\node[anchor=center] at (7.5,3) {Summands of the form};\n\t    \\node[anchor=center] at (7.5,2.25) {$\\dfrac{rx+s}{(x^2+bx+c)^n}$};\n\t    \\node[anchor=center] at (7.5,1.5) {Break up like in remark 3.5};\n\t\t\\draw[thick, rounded corners, fill, fill opacity=0.1] (-2.25,-1.75) rectangle ++(4.5,2);\n\t\t\\draw[thick, rounded corners, fill, fill opacity=0.1] (2.75,-1.75) rectangle ++(4.5,2);\n\t\t\\draw[thick, rounded corners, fill, fill opacity=0.1] (7.75,-1.75) rectangle ++(4.5,2);\n\t\t\\draw[thick, rounded corners, fill, fill opacity=0.1] (12.75,-1.75) rectangle ++(4.5,2);\n\t\t\\node[anchor=center] at (0,0) {Summands of the form};\n\t\t\\node[anchor=center] at (0,-0.75) {$\\dfrac{2x+b}{x^2+bx+c}$};\n\t\t\\node[anchor=center] at (0,-1.5) {Theorem 3.2 case 1};\n\t\t\\node[anchor=center] at (5,0) {Summands of the form};\n\t\t\\node[anchor=center] at (5,-0.75) {$\\dfrac{1}{x^2+bx+c}$};\n\t\t\\node[anchor=center] at (5,-1.5) {Theorem 3.2 case 2.1};\n\t\t\\node[anchor=center] at (10,0) {Summands of the form};\n\t\t\\node[anchor=center] at (10,-0.75) {$\\dfrac{2x+b}{(x^2+bx+c)^n},\\quad n>1$};\n\t\t\\node[anchor=center] at (10,-1.5) {Remark 3.5};\n\t\t\\node[anchor=center] at (15,0) {Summands of the type};\n\t\t\\node[anchor=center] at (15,-0.75) {$\\dfrac{1}{(x^2+bx+c)^n},\\quad n>1$};\n\t\t\\node[anchor=center] at (15,-1.5) {Complicated methods needed};\n\t\\end{tikzpicture}\n\\end{document}"}