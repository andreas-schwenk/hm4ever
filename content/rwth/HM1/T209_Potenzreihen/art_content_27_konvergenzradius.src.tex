%$Id:  $
\documentclass{mumie.article}
%$Id$
\begin{metainfo}
  \name{
    \lang{de}{Potenzreihen und Konvergenzradius}
    \lang{en}{}
  }
  \begin{description} 
 This work is licensed under the Creative Commons License Attribution 4.0 International (CC-BY 4.0)   
 https://creativecommons.org/licenses/by/4.0/legalcode 

    \lang{de}{Beschreibung}
    \lang{en}{}
  \end{description}
  \begin{components}
    \component{generic_image}{content/rwth/HM1/images/g_tkz_T209_ConversionRadius.meta.xml}{T209_ConversionRadius}
    \component{generic_image}{content/rwth/HM1/images/g_img_00_video_button_schwarz-blau.meta.xml}{00_video_button_schwarz-blau}
  \end{components}
  \begin{links}
    \link{generic_article}{content/rwth/HM1/T209_Potenzreihen/g_art_content_27_konvergenzradius.meta.xml}{content_27_konvergenzradius}
    \link{generic_article}{content/rwth/HM1/T303_Approximationen/g_art_content_04_taylor_polynom.meta.xml}{content_04_taylor_polynom}
    \link{generic_article}{content/rwth/HM1/T209_Potenzreihen/g_art_content_28_exponentialreihe.meta.xml}{content_28_exponentialreihe}
    \link{generic_article}{content/rwth/HM1/T208_Reihen/g_art_content_25_konvergenz_kriterien.meta.xml}{konv-krit}
    \link{generic_article}{content/rwth/HM1/T208_Reihen/g_art_content_24_reihen_und_konvergenz.meta.xml}{reihen-und-konv}
  \end{links}
  \creategeneric
\end{metainfo}
\begin{content}
\usepackage{mumie.ombplus}
\ombchapter{9}
\ombarticle{1}

\lang{de}{\title{Potenzreihen und deren Konvergenzradius}}
 
\begin{block}[annotation]
  
  
\end{block}
\begin{block}[annotation]
  Im Ticket-System: \href{http://team.mumie.net/issues/9684}{Ticket 9684}\\
\end{block}

\begin{block}[info-box]
\tableofcontents
\end{block}


\section{Potenzreihen}


Potenzreihen sind Reihen, deren Glieder eine besondere Form, zum Beispiel $a_n z^n$, haben,
wobei die $a_n$ komplexe Zahlen sind und $z$ eine Unbestimmte. Ihre Partialsummen sind also Polynome in $z$.
Potenzreihen definieren, wenn sie denn konvergieren, Funktionen. 
Die prominentesten Beispiele  $\exp$, $\sin$, $\cos$ betrachten wir ausführlich im 
\ref[content_28_exponentialreihe][nächsten Abschnitt]{sec:exp-reihe}.
Umgekehrt erlauben die meisten "{gutartigen}" Funktionen wenigstens in der Nähe eines gewählten Punktes 
eine Darstellung als Potenzreihe. Das ist die sogenannte 
\ref[content_04_taylor_polynom][Taylor-Reihe]{sec:polynom}, die im Kapitel Approximation besprochen wird.


\begin{definition}\label{def:Potenzreihe}
Es sei $z_0\in \C$ und $(a_n)_{n\geq 0}$ eine komplexe Folge.  Dann heißt
eine Reihe der Form
\[   \sum_{n=0}^\infty a_n(z-z_0)^n \]
eine \notion{Potenzreihe in $z$ mit Entwicklungspunkt $z_0$}. Die Zahlen $a_n$ nennt
man die \notion{Koeffizienten} der Potenzreihe.
Sind alle auftretenden Zahlen reell, so spricht man von einer \notion{reellen Potenzreihe}.

Hierbei ist stets $(z-z_0)^0=1$, selbst für $z=z_0$. Im Falle $z=z_0$ besitzt die Potenzreihe dann
lediglich einen von Null verschiedenen Summanden, nämlich $a_0$.
\end{definition}

\begin{remark}
\begin{enumerate}
\item Die Zahlen $z_0$ und $a_n$ werden bei Potenzreihen als fest gewählt angesehen, wohingegen $z$ als Variable gesehen wird. Indem man zunächst  $z_0$ und $(a_n)_{n\geq 0}$ fest wählt und dann
verschiedene Werte für $z$ einsetzt, erhält man komplexe (oder reelle) Funktionen
\[  f:D_f\to \C, \ z\mapsto \sum_{n=0}^\infty a_n(z-z_0)^n. \]
Der Definitionsbereich dieser Funktion ist dann die Menge aller $z$, für die die entsprechende Reihe konvergiert 
(siehe \lref{sec:konvergenzradius}{Konvergenzradius}).
\item Setzt man $w=z-z_0$, so ist
\[  \sum_{n=0}^\infty a_n(z-z_0)^n =\sum_{n=0}^\infty a_n w^n \]
eine Potenzreihe in $w$ mit Entwicklungspunkt $0$, und die eine Reihe konvergiert genau dann, wenn die
andere Reihe konvergiert. Wir werden daher im Folgenden nur Potenzreihen mit Entwicklungspunkt $0$
betrachten.
\end{enumerate}
\end{remark}

\begin{example}
\begin{enumerate}
\item
Die \ref[konv-krit][Exponentialreihe]{ex:exp-reihe} für beliebiges $z\in \C$ 
\[ \exp(z)=\sum_{n=0}^\infty \frac{z^n}{n!} \]
ist eine Potenzreihe in $z$ mit Entwicklungspunkt $0$.
\item Die \ref[reihen-und-konv][geometrische Reihe]{ex:geom-reihe}
\[ \sum_{k=0}^\infty q^k \]
ist eine Potenzreihe in $q$ mit Entwicklungspunkt $0$.
\end{enumerate}
\end{example}

Das folgende Abelsche Lemma ist unser Schlüssel zum Konvergenzverhalten von Potenzreihen.

\begin{theorem}[Abelsches Lemma] \label{thm:abelsches-lemma}
Es sei $(a_n)_{n\geq 0}$ eine komplexe Folge und $r>0$ so, dass die Folge $(a_n r^n)_{n\geq 0}$ beschränkt ist. Dann konvergiert die Potenzreihe
\[  \sum_{n=0}^\infty a_n z^n \]
absolut für alle $z\in \C$ mit $|z|<r$.

Etwas allgemeiner gilt sogar: Für beliebiges $j\in \Nzero$ konvergiert dann die Potenzreihe
\[  \sum_{n=0}^\infty a_n n^j z^n \]
absolut für alle $z\in \C$ mit $|z|<r$.
\end{theorem}

\begin{remark}
Es ist wichtig zu bemerken, dass die absolute Konvergenz nur für solche $z$ mit $|z|<r$ gezeigt werden kann.
Im Beweis werden wir benutzen, dass $\frac{\vert z\vert}{r}<1$. Für  $z$ mit $\vert z\vert =r$, also Punkten auf dem Kreis (d.h. der Kreislinie) selbst, kann keine allgemeine Aussage gemacht werden!
Die Menge der $z\in\C$, für die hier die absolute Konvergenz gezeigt wird, ist also eine Kreisscheibe mit Radius $r$, deren Rand, das heißt der Kreis mit Radius $r$, nicht dazugehört.

Wenn die Reihe $\sum_{n=0}^\infty a_n n^j z^n$ absolut konvergiert, dann konvergiert auch die Reihe $\sum_{n=0}^\infty a_n z^n $
absolut nach dem Majorantenkriterium. Das Abelsche Lemma liefert sozusagen eine Umkehrung für den Spezialfall aus dem Satz. 

Häufig wird als Abelsches Lemma daher auch die Aussage bezeichnet, dass die beiden Potenzreihen $\sum_{n=0}^\infty a_n n^j z^n$ und 
$\sum_{n=0}^\infty a_n z^n$ innerhalb des gleichen Radius konvergieren. 
\end{remark}

\begin{proof*}[Beweis des Abelschen Lemmas]
\begin{incremental}[\initialsteps{0}]
\step
Wir zeigen gleich die allgemeinere Aussage mit dem \ref[konv-krit][Wurzelkriterium]{thm:wurzelkriterium}.\\
Da $(a_n r^n)_{n\geq 0}$ beschränkt ist, gibt es ein $K\in \R$ mit $K\geq |a_n r^n|$ für alle $n\geq 0$.
Für $z\in \C\setminus \{0\}$ mit $|z|<r$ gilt dann für alle $n\geq 0$
\[ \sqrt[n]{|a_nn^j z^n|}=\sqrt[n]{n^j \, | \, a_n r^n \, | \, \cdot \left(\frac{|z|}{r}\right)^n}\leq \sqrt[n]{n}^j\cdot 
\sqrt[n ]{K}\cdot \frac{|z|}{r}. \]

\step
Es ist $\frac{r}{|z|}>1$ und wir wählen nun noch ein $s\in (1;\frac{r}{|z|})$.
Da $\lim_{n\to \infty} \sqrt[n]{n}=1$ und $\lim_{n\to \infty} \sqrt[n]{K}=1$ ist (s. Abschnitt
\ref[konv-krit][Konvergenzkriterien für Folgen]{sec:wichtige-beispiele}),
gibt es ein $n_0\in \N$ so, dass
$\sqrt[n]{n}^j \cdot \sqrt[n]{K}<s$ für alle $n\geq n_0$. 
Damit gilt für alle $n\geq n_0$
\[ \sqrt[n]{|a_nn^j z^n|}\leq  \sqrt[n]{n}^j \cdot\sqrt[n]{K}\cdot \frac{|z|}{r}<s \cdot \frac{|z|}{r}=:\beta. \]
Wir wählen also $\beta:=s\cdot \frac{|z|}{r}$ und für den Beweis der absoluten Konvergenz ist es hinreichend, wenn $\beta<1$ sein sollte.
Dies ist aber der Fall, da wir $s < \frac{r}{|z|}$ gewählt hatten.

Also ist die Reihe $\sum_{n=0}^\infty a_n n^j z^n$ absolut konvergent.
\end{incremental}
\end{proof*}

\begin{example}
Für die geometrische Reihe $\sum_{n=0}^\infty z^n$ ist die Folge der Koeffizienten gerade die konstante
Folge $(1)_{n\in \N}$. Diese ist beschränkt, weshalb wir den Satz mit $z=1$ anwenden können.
Die geometrische Reihe ist also für alle $z$ mit $|z|<1$ absolut konvergent, was wir auch schon
im Abschnitt \ref[konv-krit][Konvergenzkriterien für Reihen]{ex:absolut-konvergenz} direkt gesehen haben.

Genau genommen wurde für den Beweis der obigen Aussage auch schon benutzt, dass die geometrische
Reihe für $|z|<1$ absolut konvergiert.
\end{example}

Die Konvergenz von Potenzreihen wird auch im folgenden Video behandelt:
\floatright{\href{https://api.stream24.net/vod/getVideo.php?id=10962-2-10868&mode=iframe&speed=true}{\image[75]{00_video_button_schwarz-blau}}}\\

\section{Konvergenzradius}\label{sec:konvergenzradius}



In diesem Paragraphen beschäftigen wir uns genauer mit dem Bereich derjenigen $z$, für die eine
gegebene Potenzreihe $\sum_{n=0}^\infty a_n z^n$ bzw. allgemeiner eine Potenzreihe 
$\sum_{n=0}^\infty a_n(z-z_0)^n$  konvergiert.

Wir werden sehen, dass dieser Bereich in der komplexen Ebene eine Kreisscheibe mit dem Entwicklungspunkt $z_0$ als Mittelpunkt ist.
Dabei lassen wir für den Radius $R$ dieser Kreisscheibe die beiden Extreme $R=0$ und $R=\infty$ zu. 
Im ersten Fall besteht die Kreischeibe dann aus einem einzigen Punkt, dem Entwicklungspunkt der Reihe, 
während im zweiten Fall die unendlich große Kreischeibe die gesamte komplexe Ebene meint.

\begin{definition}\label{def:konvergenzradius}
Für eine Potenzreihe $\sum_{n=0}^\infty a_n(z-z_0)^n$ ist der \notion{Konvergenzradius} $R\in \R_{\geq 0}\cup \{\infty\}$ definiert durch
\[   R= \sup\{ r\in \R_{\geq 0} \,|\,  \text{Die Folge }(a_nr^n)_{n\geq 0}\text{ ist beschränkt} \}. \]
\end{definition}
Man beachte, dass die obige Menge, deren Supremum den Konvergenzradius bestimmt, immer $r=0$ enthält, also nicht leer ist.



\begin{example}\label{ex:geom-und-exp-reihe}
\begin{tabs*}[\initialtab{0}]
\tab{geometrische Reihe} Für die geometrische Reihe  $\sum_{n=0}^\infty z^n$ ist die Folge $(1\cdot r^n)_{n\geq 0}$ genau dann beschränkt, 
wenn $|r|\leq 1$ ist. Also ist ihr Konvergenzradius $R=\sup\{ r\in \R_{\geq 0} | r\leq 1\}=\sup [0;1] =1$.
\tab{Exponentialreihe} Um den Konvergenzradius für die Exponentialreihe $\sum_{n=0}^\infty \frac{z^n}{n!}$ zu berechnen,
müssen wir Folgen $(  \frac{r^n}{n!})_{n\geq 0}$ für feste reelle Zahlen $r>0$ betrachten.

Jede dieser Folgen hat positive Glieder, und für $n+1>r$ (also $n>r-1$) gilt
\[  \frac{r^{n+1}}{(n+1)!}= \frac{r^n}{n!}\cdot \frac{r}{n+1}< \frac{r^n}{n!}. \]
Die Folge ist für $n>r-1$ also monoton fallend und damit insbesondere beschränkt.
Weil für Konvergenzaussagen die ersten, endlichen vielen Glieder mit $n\leq r-1$ keine Rolle spielen,
ist der Konvergenzradius $R=\sup\{ r\in \R_{\geq 0}\}=\infty$.
\end{tabs*}
\end{example}

Der Begriff \emph{Konvergenzradius} ist gerechtfertigt durch den folgenden Satz zur Aussage über die
Konvergenz der Potenzreihe.


\begin{theorem}\label{thm:konvergenzbereich}
Es sei $\sum_{n=0}^\infty a_n (z-z_0)^n$ eine komplexe Potenzreihe mit Entwicklungspunkt $z_0$ und
$R$ ihr Konvergenzradius. Dann gelten:
\begin{enumerate}
\item Ist $R=0$, so konvergiert die Reihe nur für $z=z_0$.
\item Ist $0<R<\infty$, so ist die Reihe für $|z - z_0|<R$ absolut konvergent und für $|z - z_0|>R$ divergent.
\item Ist $R=\infty$, so ist die Reihe für alle $z\in \C$ absolut konvergent.
\end{enumerate}
\end{theorem}

\begin{remark}
Der Bereich, in dem die Potenzreihe $\sum_{n=0}^\infty a_n (z-z_0)^n$ konvergiert, 
umfasst also die Punkte im Innern der Kreisscheibe mit Mittelpunkt $z_0$  und Radius $R$. 
Eventuell gehören auch noch Punkte dazu, die auf dem Rand liegen, d.\,h. solche $z\in \C$ mit $|z-z_0|=R$.

\begin{figure}
\image{T209_ConversionRadius}
\caption{Die Menge aller Punkte $z$ mit $|z-3|<2$ ist ein Kreis um $z_0=3$ mit Radius $2$.}
\end{figure}

Der Satz macht keine Aussage über die Konvergenz auf dem Rand, und in der Tat treten auch mehrere verschiedene Fälle auf: 
Konvergenz bei keinem Randpunkt, Konvergenz bei allen Randpunkten oder auch Konvergenz bei manchen Randpunkten 
(s. \lref{ex:konvergenz-auf-rand}{Beispiele}).
\end{remark}

\begin{proof*}[Beweis des Theorems]
\begin{incremental}[\initialsteps{1}]
\step
Wie bereits oben beschrieben, kann man durch die Ersetzung $w=z-z_0$ den allgemeinen Fall auf den Fall $z_0=0$ zurückführen.

\step
Es sei also $\sum_{n=0}^\infty a_n z^n$ eine Potenzreihe mit Entwicklungspunkt $0$ und $R$ ihr Konvergenzradius.
Die Definition von $R$ als Supremum besagt, dass die Folge  $(a_nr^n)_{n\geq 0}$ beschränkt ist für alle 
$r<R$ und unbeschränkt ist für alle $r>R$.

\step
Ist nun $|z|>R$ (falls $R<\infty$), so ist also die Folge $(a_n |z|^n)_{n\geq 0}$ unbeschränkt, und daher kann auch die Folge
$(a_n z^n)_{n\geq 0}$ nicht beschränkt sein. Für eine konvergente Reihe $\sum_{n=0}^\infty a_n z^n$ müsste aber die Folge 
$(a_n z^n)_{n\geq 0}$ eine Nullfolge und insbesondere beschränkt sein.\\
Folglich konvergiert die Potenzreihe für $|z|>R$ nicht.

Ist $|z|<R$ (falls $R\neq 0$), so können wir ein $r\in \R$ mit $|z|<r<R$ wählen. Dann ist aber die Folge $(a_nr^n)_{n\geq 0}$ beschränkt 
und nach  \lref{thm:abelsches-lemma}{obigem Satz} konvergiert die Potenzreihe
für alle $w$ mit $|w|<r$, also insbesondere für $z$.

Im speziellen Fall $R=0$ ist noch zu bemerken, dass die Potenzreihe $\sum_{n=0}^\infty a_n z^n$ für $z=0$ konvergiert, 
da außer dem ersten Summanden $a_0z^0=a_0$ alle anderen gleich $0$ sind. 
\end{incremental}
\end{proof*}


Wie schon oben erwähnt, gibt es keine allgemeine Aussage über die Konvergenz auf dem Rand.

\begin{example}\label{ex:konvergenz-auf-rand}
\begin{tabs*}[\initialtab{0}]
\tab{geometrische Reihe $\sum_{n=0}^\infty z^n$}
Für die geometrische Reihe $\sum_{n=0}^\infty z^n$ hatten wir als Konvergenzradius $R=1$ berechnet.
Für $|z|=1$ bilden die Glieder keine Nullfolge, weshalb insbesondere die Reihe $\sum_{n=0}^\infty z^n$
für diese $z$ nicht konvergiert. Die geometrische Reihe konvergiert also an keinem Randpunkt.
\tab{Logarithmusreihe $\sum_{n=1}^\infty \frac{(-1)^{n+1}z^n}{n}$}
Betrachtet man die sogenannte Logarithmusreihe $\sum_{n=1}^\infty \frac{(-1)^{n+1}z^n}{n}$, so ist
die zugehörige Folge $(\frac{(-1)^{n+1}r^n}{n})_{n\geq 1}$ für $r=1$ beschränkt, d.\,h., der Konvergenzradius $R$ ist mindestens $1$.
Andererseits erhalten wir für $z=-1$ das Negative der harmonischen Reihe, welche nicht konvergiert.
Daher kann der Konvergenzradius auch nicht größer als $1$ sein. Also ist $R=1$.

Für $z=1$ erhält man jedoch die \ref[reihen-und-konv][alternierende harmonischen Reihe]{ex:alter-harm-reihe}, welche konvergiert. 


Für diese Potenzreihe gibt es also Werte auf dem Rand, für die die Reihe konvergiert (z.\,B. $z=1$), und Werte auf dem Rand, für die die Reihe nicht konvergiert (z.B. $z=-1$).

\emph{Bemerkung:} Der Begriff "`Logarithmusreihe"' kommt daher, dass sie mit der Logarithmusfunktion $\ln$ zusammenhängt. Genauer gilt für alle reellen Zahlen $z$ mit $-1<z\leq 1$:
\[   \sum_{n=1}^\infty \frac{(-1)^{n+1}z^n}{n} = \ln(1+z). \]
Den Grenzwert der alternierenden harmonischen Reihe erkennen wir  nun als $\ln(2)$.
\tab{$\sum_{n=1}^\infty \frac{z^n}{n^2}$}
Auch für diese Potenzreihe ist der Konvergenzradius $1$, denn wir haben im Abschnitt \ref[konv-krit][Konvergenzkriterien für Reihen]{ex:ableitungen-der-geom-reihe} gesehen, dass die Reihe für $|z|<1$ konvergiert und für $|z|>1$ divergiert.
Wir haben auch schon \ref[konv-krit][gesehen]{ex:weitere-reihen}, dass die Reihe
$\sum_{n=1}^\infty \frac{1}{n^2}$ konvergiert. Insbesondere ist die Reihe
$\sum_{n=1}^\infty \frac{z^n}{n^2}$ für alle $z$ mit $|z|=1$ absolut konvergent und damit auch konvergent.

In diesem Beispiel konvergiert die Potenzreihe also für alle Punkte auf dem Rand.
\end{tabs*}
\end{example}

Die Definition des Konvergenzradius wird mit Beispielen in diesem Video erläutert:
\floatright{\href{https://api.stream24.net/vod/getVideo.php?id=10962-2-10869&mode=iframe&speed=true}{\image[75]{00_video_button_schwarz-blau}}}\\
\\
\\

Wir haben den Konvergenzradius als Supremum, also als kleinste obere Schranke einer Menge, kennengelernt.
Das legt nahe, dass er sich oft als Grenzwert explizit bestimmen lässt. 
Die folgenden Sätze geben  solche Grenzwertformeln an.
%In vielen Fällen lässt sich der Konvergenzradius auch als Grenzwert berechnen.



\begin{theorem}[Cauchy-Hadamard]\label{thm:Cauchy-Hadamard}
%\textbf{Satz von Cauchy-Hadamard}\\
Es sei $\sum_{n=0}^\infty a_n (z-z_0)^n$ eine Potenzreihe. Wenn der Grenzwert
\[  \lim_{n\to \infty} {\sqrt[n]{|a_n|}} \]
existiert (oder $=\infty$ ist), dann gilt für den Konvergenzradius $R$ der Potenzreihe
\[  R=   \frac{1}{\lim_{n\to \infty}\sqrt[n]{|a_n|}}. \]
\end{theorem}

\begin{proof*}
Ist $A={\lim_{n\to \infty}\sqrt[n]{|a_n|}}$, dann gilt
\[ \lim_{n\to\infty}\sqrt[n]{|a_nz^n|}={\vert z\vert} \cdot \lim_{n\to \infty}\sqrt[n]{|a_n|}={\vert z\vert}\cdot A
%\:\begin{cases} <1 &\text{ falls } \vert z\vert <\rho\\>1&\text{ falls }\vert z\vert >\rho
%\end{cases}
.\]
Nach dem \ref[konv-krit][Wurzelkriterium]{thm:wurzelkriterium} konvergiert die Reihe für ${\vert z\vert}<\frac{1}{A}$ absolut, 
und sie divergiert für ${\vert z\vert}>\frac{1}{A}$. Also ist $R=\frac{1}{A}$ der Konvergenzradius.
(Hierbei haben wir symbolisch $\frac{1}{A}=0$ im Fall $A=\infty$ gesetzt, bzw. $\frac{1}{A}=\infty$ im Fall $A=0$.)
\end{proof*}

Mit dem Quotientenkriterium zeigt man:
\begin{theorem}[Quotientenregel]\label{thm:quot-regel}
Es sei $\sum_{n=0}^\infty a_n (z-z_0)^n$ eine Potenzreihe und es gelte $a_n \neq 0$ für alle genügend großen $n \geq n_0$. Wenn der Grenzwert
\[  q=\lim_{n\to \infty} {\vert\frac{a_{n+1}}{a_n}\vert} \]
existiert (oder $=\infty$ ist), dann gilt für den Konvergenzradius $R$ der Potenzreihe
\[  R=   \frac{1}{q}. \]
\end{theorem}

\begin{remark}
Ersetzt man im Satz von Cauchy-Hadamard den Limes durch den Limes superior, 
so zeigt man sogar die folgende Aussage:
Der Konvergenzradius $R$ der Potenzreihe ist gegeben durch 
\[ \frac{1}{R}=   {\underset{n\to \infty}{\lim \sup}\sqrt[n]{|a_n|}}.\]
Manchmal wird der Konvergenzradius durch diese Cauchy-Hadamard-Formel definiert 
und dann die Äquivalenz zu unserer Definition gezeigt.
\end{remark}

Die Theoreme zur Berechnung des Konvergenzradius werden auch im folgenden Video bewiesen und erklärt:
\floatright{\href{https://api.stream24.net/vod/getVideo.php?id=10962-2-10870&mode=iframe&speed=true}{\image[75]{00_video_button_schwarz-blau}}}\\


\begin{quickcheck}
\text{ Wählen Sie die richtige Antwort aus! \\
Der Konvergenzbereich einer komplexen Potenzreihe mit Konvergenzradius $R>0$ ist...
}
\begin{choices}{unique}
\begin{choice}
\text{ein Intervall der Breite $2R$.}
\solution{false}
\end{choice}
\begin{choice}
\text{ein Kreis mit Radius $R$.}
\solution{false}
\end{choice}
\begin{choice}
\text{eine Kreisscheibe mit Radius $R$.}
\solution{true}
\end{choice}
\begin{choice}
\text{ein Quadrat mit Seitenlänge $R$.}
\solution{false}
\end{choice}
\end{choices}
\end{quickcheck}
%%%
\begin{quickcheck}
\type{input.number}
\field{rational}
\begin{variables}
   \randint{q}{2}{9}
   \function[calculate]{R1}{1}
   \function[calculate]{R2}{q}
   \function[calculate]{R3}{1/q}
  \end{variables}
 
  \text{
    Bestimmen Sie die Konvergenzradien der folgenden Reihen.\\
    $\sum_{k=0}^\infty z^k$ hat Konvergenzradius $R_1=$\ansref,\\
    $\sum_{k=0}^\infty \frac{z^k}{\var{q}^k}$ hat Konvergenzradius $R_2=$\ansref,\\
    $\sum_{k=0}^\infty \var{q}^k{z^k}$ hat Konvergenzradius $R_3=$\ansref.
    }
    \begin{answer}
    \solution{R1}
    \end{answer}
    \begin{answer}
    \solution{R2}
    \end{answer}
    \begin{answer}
    \solution{R3}
    \end{answer}

\end{quickcheck}

Das folgende Video behandelt als Zusatz die Frage nach der Stetigkeit von Potenzreihen:
\floatright{\href{https://api.stream24.net/vod/getVideo.php?id=10962-2-10874&mode=iframe&speed=true}{\image[75]{00_video_button_schwarz-blau}}}
\end{content}