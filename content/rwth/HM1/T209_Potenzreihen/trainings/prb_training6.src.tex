\documentclass{mumie.problem.gwtmathlet}
%$Id$
\begin{metainfo}
  \name{
    \lang{en}{Problem 6}
    \lang{de}{A06: Eigenschaften}
  }
  \begin{description} 
 This work is licensed under the Creative Commons License Attribution 4.0 International (CC-BY 4.0)   
 https://creativecommons.org/licenses/by/4.0/legalcode 

    \lang{en}{...}
    \lang{de}{...}
  \end{description}
  \corrector{system/problem/GenericCorrector.meta.xml}
  \begin{components}
    \component{js_lib}{system/problem/GenericMathlet.meta.xml}{gwtmathlet}
  \end{components}
  \begin{links}
  \end{links}
  \creategeneric
  \begin{taxonomy}
        \difficulty{1}
        \usage{}
        \objectives{remember,understand}
        \topic{analysis1/series/power_series/ps_converg_diverge_crit}
  \end{taxonomy}
\end{metainfo}
\begin{content}
\begin{block}[annotation]
	Im Ticket-System: \href{https://team.mumie.net/issues/19870}{Ticket 19870}
\end{block}
\usepackage{mumie.genericproblem}
\lang{de}{\title{A06: Eigenschaften}}
\lang{en}{\title{Problem 6}}
\begin{block}[annotation]
Aufgabenpool mit vier sehr ähnlichen Aufgaben, aus dem zufällig ausgewählt wird.
\end{block}

\begin{problem}
\randomquestionpool{1}{4}
 %% Aufgabe 1
  \begin{question}
   \text{Entscheiden Sie, welche der folgenden Aussagen wahr sind.}
    \type{mc.multiple}
   	%%
	\begin{choice}
        		\text{(a)  Der Konvergenzradius der Potenzreihe $\sum_{n=0}^\infty n^nz^n$ 
                ist gleich $0$.}
		\solution{true}
    \end{choice}
	\begin{choice}
        \text{(b) Die Potenzreihen $\sin(x)$, $\cos(x)$, $\exp(x)$ konvergieren für 
       $\vert x\vert\leq 1$.  }
		\solution{true}
    \end{choice}
	\begin{choice}
     	\text{(c) Der Konvergenzradius von $\sum_{n=1}^{\infty} a_n n z^n$ ist echt kleiner als der 
        von $\sum_{n=0}^\infty a_n z^n$.}
		\solution{false}
     \end{choice}
     \begin{choice}
     \text{Keine der angegebenen Aussagen ist wahr.}
     \solution{false}
     \end{choice}
        \explanation[NOT[equalChoice(0000)] AND equalChoice(0??0)]{
        (a) ist wahr: Es ist $\lim_{n\to\infty} \sqrt[n]{n^n}=\lim_{n\to\infty}n=\infty$. 
         Nach Cauchy-Hadamard ist der Konvergenzradius also $\frac{1}{\infty}=0$.}
        \explanation[NOT[equalChoice(0000)]AND equalChoice(?0?0)]{(b) ist wahr: Die Reihen konvergieren sogar auf ganz $\C$.}
        \explanation[NOT[equalChoice(0000)] AND equalChoice(??10)]{(c) ist falsch: 
            Beide Reihen haben nach dem Abelschen Lemma denselben Konvergenzradius.}
         \explanation[equalChoice(???1)]{Zwei der Aussagen sind richtig. Prüfen Sie noch einmal mit bekannten Konvergenzkriterien.}
  \end{question}
%% Aufgabe 2
  \begin{question}
    \type{mc.multiple}
    \text{Entscheiden Sie, welche der folgenden Aussagen wahr sind.}
	%%
	\begin{choice}
        \text{(a)  Der Konvergenzradius der Potenzreihe $\sum_{n=0}^\infty n^nz^n$ ist gleich $1$.}
		\solution{false}
%         \explanation[equalChoice(1??)]{
%          (a) ist falsch: Es ist $\lim_{n\to\infty} \sqrt[n]{n^n}=\lim_{n\to\infty}n=\infty$. 
%          Nach Cauchy-Hadamard ist der Konvergenzradius also $\frac{1}{\infty}=0$.}
   	\end{choice}
	\begin{choice}
        \text{(b) Die Potenzreihen $\sin(x)$, $\cos(x)$, $\exp(x)$ haben einen gemeinsamen Konvergenzbereich. }
		\solution{true}
%         \explanation[equalChoice(?0?)]{(b) ist wahr: Die Reihen konvergieren alle auf ganz $\C$.}
     \end{choice}
	\begin{choice}
       	\text{(c) Der Konvergenzradius von $\sum_{n=1}^{\infty} a_nn^2z^n$ ist echt kleiner als der von $\sum_{n=0}^\infty a_nz^n$.}
		\solution{false}
%         \explanation[equalChoice(??1)]{ (c) ist falsch: 
%             Beide Reihen haben nach dem Abelschen Lemma denselben Konvergenzradius.}
     \end{choice}
    \begin{choice}
        \text{Keine der angegebenen Aussagen ist wahr.}
        \solution{false}
     \end{choice}
      \explanation[NOT[equalChoice(0000)] AND equalChoice(1??0)]{
         (a) ist falsch: Es ist $\lim_{n\to\infty} \sqrt[n]{n^n}=\lim_{n\to\infty}n=\infty$. 
         Nach Cauchy-Hadamard ist der Konvergenzradius also $\frac{1}{\infty}=0$.} 
      \explanation[NOT[equalChoice(0000)] AND equalChoice(?0?0)]{(b) ist wahr: Die Reihen konvergieren alle auf ganz $\C$.}
      \explanation[NOT[equalChoice(0000)] AND equalChoice(??10)]{ (c) ist falsch: 
            Beide Reihen haben nach dem Abelschen Lemma denselben Konvergenzradius.}
      \explanation[equalChoice(???1)]{Eine der Aussagen ist richtig. Prüfen Sie noch einmal mit bekannten Konvergenzkriterien.}

  \end{question}



%% Aufgabe 3
  \begin{question}
    \type{mc.multiple}
    \text{Entscheiden Sie, welche der folgenden Aussagen wahr sind.}
	%%
	\begin{choice}
        \text{(a)  Der Konvergenzradius der Potenzreihe $\sum_{n=0}^\infty n^nz^n$ ist gleich $1$.}
		\solution{false}
%         \explanation[equalChoice(1??)]{
%          (a) ist falsch: Es ist $\lim_{n\to\infty} \sqrt[n]{n^n}=\lim_{n\to\infty}n=\infty$. 
%          Nach Cauchy-Hadamard ist der Konvergenzradius also $\frac{1}{\infty}=0$.}
        	\end{choice}
	\begin{choice}
        \text{(b) Die Potenzreihen $\sin(x)$, $\cos(x)$, $\exp(x)$ haben Konvergenzradius $\infty$. }
		\solution{true}
%         \explanation[equalChoice(?0?)]{(b) ist wahr: Die Reihen konvergieren alle auf ganz $\C$.}
     \end{choice}
	\begin{choice}
       	\text{(c) Der Konvergenzradius von $\sum_{n=1}^{\infty} a_nn^2z^n$ ist gleich dem von $\sum_{n=0}^\infty a_nz^n$.}
		\solution{true}
%        \explanation[equalChoice(??0)]{(c) ist wahr: Beide Reihen haben nach dem Abelschen Lemma denselben Konvergenzradius.}
     \end{choice}
     \begin{choice}
     \text{Keine der angegebenen Aussagen ist wahr.}
     \solution{false}
     \end{choice}
    \explanation[NOT[equalChoice(0000)] AND equalChoice(1??0)]{
         (a) ist falsch: Es ist $\lim_{n\to\infty} \sqrt[n]{n^n}=\lim_{n\to\infty}n=\infty$. 
         Nach Cauchy-Hadamard ist der Konvergenzradius also $\frac{1}{\infty}=0$.}    
    \explanation[NOT[equalChoice(0000)] AND equalChoice(?0?0)]{(b) ist wahr: Die Reihen konvergieren alle auf ganz $\C$.}  
    \explanation[NOT[equalChoice(0000)] AND equalChoice(??00)]{(c) ist wahr: Beide Reihen haben nach dem Abelschen Lemma denselben Konvergenzradius.}
    \explanation[equalChoice(???1)]{Zwei der Aussagen sind richtig. Prüfen Sie noch einmal mit bekannten Konvergenzkriterien.}
  \end{question}

  
%% Aufgabe 4
  \begin{question}
    \type{mc.multiple}
    \text{Entscheiden Sie, welche der folgenden Aussagen wahr sind.}
	%%
	\begin{choice}
        		\text{(a)  Der Konvergenzradius der Potenzreihe $\sum_{n=0}^\infty n^nz^n$ ist 
                gleich $0$.}
		\solution{true}
%         \explanation[equalChoice(0??)]{
%          (a) ist korrekt: Es ist $\lim_{n\to\infty} \sqrt[n]{n^n}=\lim_{n\to\infty}n=\infty$. Nach Cauchy-Hadamard ist der Konvergenzradius also $\frac{1}{\infty}=0$.}
    \end{choice}
	\begin{choice}
        \text{(b) Die Potenzreihen $\sin(x)$, $\cos(x)$, $\exp(x)$ konvergieren für $\vert x\vert\leq 1$.  }
		\solution{true}
%         \explanation[equalChoice(?0?)]{(b) ist korrekt: Die Reihen konvergieren sogar auf ganz $\C$.}
     \end{choice}
	\begin{choice}
       	\text{(c) Der Konvergenzradius von $\sum_{n=1}^{\infty} a_nnz^n$ ist gleich dem von 
        $\sum_{n=0}^\infty a_nz^n$.}
		\solution{true}
%         \explanation[equalChoice(??0)]{ (c) ist korrekt: 
%             Beide Reihen haben nach dem Abelschen Lemma denselben Konvergenzradius.}
     \end{choice}
     \begin{choice}
     \text{Keine der angegebenen Aussagen ist wahr.}
     \solution{false}
     \end{choice}
    \explanation[NOT[equalChoice(0000)] AND  equalChoice(0??0)]{
         (a) ist korrekt: Es ist $\lim_{n\to\infty} \sqrt[n]{n^n}=\lim_{n\to\infty}n=\infty$. Nach Cauchy-Hadamard ist der Konvergenzradius also $\frac{1}{\infty}=0$.}
    \explanation[NOT[equalChoice(0000)] AND  equalChoice(?0?0)]{(b) ist korrekt: Die Reihen konvergieren sogar auf ganz $\C$.} 
    \explanation[NOT[equalChoice(0000)] AND equalChoice(??00)]{ (c) ist korrekt: 
            Beide Reihen haben nach dem Abelschen Lemma denselben Konvergenzradius.}
    \explanation[equalChoice(???1)]{Unter den Aussagen sind richtige. Prüfen Sie noch einmal mit bekannten Konvergenzkriterien.}
  \end{question}  
\end{problem}
\embedmathlet{gwtmathlet}

\end{content}
