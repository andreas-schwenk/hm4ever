\documentclass{mumie.problem.gwtmathlet}
%$Id$
\begin{metainfo}
  \name{
    \lang{en}{Problem 3}
    \lang{de}{A03: Konvergenzradius}
  }
  \begin{description} 
 This work is licensed under the Creative Commons License Attribution 4.0 International (CC-BY 4.0)   
 https://creativecommons.org/licenses/by/4.0/legalcode 

    \lang{en}{...}
    \lang{de}{...}
  \end{description}
  \corrector{system/problem/GenericCorrector.meta.xml}
  \begin{components}
    \component{js_lib}{system/problem/GenericMathlet.meta.xml}{gwtmathlet}
  \end{components}
  \begin{links}
  \end{links}
  \creategeneric
\end{metainfo}
\begin{content}
\begin{block}[annotation]
	Im Ticket-System: \href{https://team.mumie.net/issues/16438}{Ticket 16438}
\end{block}
\usepackage{mumie.genericproblem}
\lang{de}{
	\title{A03: Konvergenzradius}
}


\begin{problem}

\randomquestionpool{1}{2}
\randomquestionpool{3}{4}


\begin{question}
    \text{Berechnen Sie den Konvergenzradius der  Potenzreihe\\
    $\sum_{n=0}^\infty (\var{f}) x^n$.\\
    Der Konvergenzradius ist gegeben durch \ansref .}
    \begin{variables}
        \randint[Z]{a}{-10}{10}
        \randint{b}{-10}{10}
        \randint{c}{-10}{10}
        \function[normalize]{f}{a*n^2+b*n+c}
        \number{s}{1}
    \end{variables}
    
    \type{input.number}
    \field{rational}
    
    \begin{answer}
        %\text{Der Konvergenzradius ist gegeben durch}
        \solution{s}
        \explanation{Benutze die Quotientenregel. Auch möglich: Abelsches Lemma.}
    \end{answer}
    
\end{question}
%2
\begin{question}
    \begin{variables}
        \randint[Z]{a}{-10}{10}
        \randint{b}{-10}{10}
        \randint{c}{-10}{10}
        \function[normalize]{f}{a*n^3+b*n+c}
        \number{s}{1}
    \end{variables}
    \text{Berechnen Sie den Konvergenzradius der  Potenzreihe\\
    $\sum_{n=1}^\infty \frac{\var{f}}{n^2} x^n$.\\
    Der Konvergenzradius ist gegeben durch \ansref .}
    \type{input.number}
    \field{rational}
    
    \begin{answer}
        %\text{Der Konvergenzradius ist gegeben durch}
        \solution{s}
        \explanation{Benutze die Quotientenregel oder das Abelsche Lemma.}
    \end{answer}
\end{question}
%3
\begin{question}
    \begin{variables}
        \randint{a}{2}{6}
        \randint{b}{2}{6}
        \randint[Z]{c}{-5}{5}
        \function[calculate]{s}{(b/a)^2}
    \end{variables}
    \text{Berechnen Sie den Konvergenzradius der  Potenzreihe\\
    $\sum_{n=0}^\infty \left(\frac{\var{c}+\var{a}^n}{\var{b}^n}\right)^2 x^n$.\\
    Der Konvergenzradius ist gegeben durch \ansref .}
    \type{input.number}
    \field{rational}
    
    \begin{answer}
        %\text{Der Konvergenzradius ist gegeben durch}
        \solution{s}
        \explanation{Benutze die Quotientenregel. Siehe Beispielaufgabe 2.}
    \end{answer}
\end{question}
%4
\begin{question}
    \begin{variables}
        \randint{a}{2}{6}
        \randint[Z]{c}{-5}{5}
        \randint[Z]{aa}{1}{5}
        \randint[Z]{bb}{1}{5}
        \randint[Z]{v}{-1}{1}
        \function[normalize]{f}{v*aa*n^2+v*bb}
        \function[calculate]{s}{1/a}
    \end{variables}
    \text{Berechnen Sie den Konvergenzradius der  Potenzreihe\\
    $\sum_{n=0}^\infty \frac{\var{c}+\var{a}^n}{\var{f}} x^n$.\\
    Der Konvergenzradius ist gegeben durch \ansref .}
    \type{input.number}
    \field{rational}
    
    \begin{answer}
        %\text{Der Konvergenzradius ist gegeben durch}
        \solution{s}
        \explanation{Benutze die Quotientenregel oder betrachte verwandte geometrische Reihen.}
    \end{answer}
\end{question}

\end{problem}

\embedmathlet{gwtmathlet}

\end{content}
