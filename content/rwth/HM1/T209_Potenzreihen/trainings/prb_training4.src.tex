\documentclass{mumie.problem.gwtmathlet}
%$Id$
\begin{metainfo}
  \name{
    \lang{en}{Problem 4}
    \lang{de}{A04: Koeffizienten}
  }
  \begin{description} 
 This work is licensed under the Creative Commons License Attribution 4.0 International (CC-BY 4.0)   
 https://creativecommons.org/licenses/by/4.0/legalcode 

    \lang{en}{...}
    \lang{de}{...}
  \end{description}
  \corrector{system/problem/GenericCorrector.meta.xml}
  \begin{components}
    \component{js_lib}{system/problem/GenericMathlet.meta.xml}{gwtmathlet}
  \end{components}
  \begin{links}
  \end{links}
  \creategeneric
\end{metainfo}
\begin{content}
\begin{block}[annotation]
	Im Ticket-System: \href{https://team.mumie.net/issues/18221}{Ticket 18221}
\end{block}
\lang{de}{
	\title{A04: Koeffizienten}
}
\usepackage{mumie.genericproblem}

\begin{block}[annotation]
Aufgabenauswahl aus drei verschiedenen Reihen und teils randomisierten Koeffizienten.
\end{block}


\begin{problem}
\randomquestionpool{1}{3}

%1
\begin{question}
\begin{variables}
	%\randint[Z]{c}{-9}{9}
    %\randint[Z]{a}{-9}{9}
	%\randint[Z]{k}{1}{9}
	%\function[calculate]{sol}{abs(c)}
    \function{sol0}{e}
    \function{sol1}{e}
    \function{sol2}{e/2}
    \function{sol3}{e/6}
\end{variables}
	\type{input.function}
	\field{real} 
	\precision{3}
    \lang{de}{
	    \text{
	    Bestimmen Sie die Koeffizienten der Potenzreihe
        $\sum_{n=0}^{\infty} a_n z^n=\exp(z+1)$ im Entwicklungspunkt $z_0=0$.\\
        Geben Sie Ihre Ergebnisse exakt (oder auf drei Nachkommastellen genau) an.\\
        Der $0$-te Koeffizient ist gegeben durch $a_0=$ \ansref .\\
        Der $1$-te Koeffizient ist gegeben durch $a_1=$ \ansref .\\
        Der $2$-te Koeffizient ist gegeben durch $a_2=$ \ansref .\\
        Der $3$-te Koeffizient ist gegeben durch $a_3=$ \ansref .
    }}
    
    \begin{answer}
        \solution{sol0}
        \checkAsFunction[0.0005]{x}{0}{1}{2}          
    \end{answer}  
     \begin{answer}
        \solution{sol1}
        \checkAsFunction[0.0005]{x}{0}{1}{2}          
    \end{answer}
     \begin{answer}
        \solution{sol2}
        \checkAsFunction[0.0005]{x}{0}{1}{2}          
    \end{answer}
     \begin{answer}
        \solution{sol3}
        \checkAsFunction[0.0005]{x}{0}{1}{2}          
    \end{answer}
    \explanation{Es ist $\exp(z+1)=\exp(z)\cdot\exp(1)=e\cdot\exp(z)$. 
        Mit dem $n$-ten Koeffizienten $\frac{1}{n!}$ der Potenzreihe $\exp(z)$ ist also $a_n=\frac{e}{n!}$.}
\end{question}
%
%2
\begin{question}
\begin{variables}
	%\randint[Z]{c}{-9}{9}
    %\randint[Z]{a}{-9}{9}
	%\randint[Z]{k}{1}{9}
	%\function[calculate]{sol}{abs(c)}
    \function{sol0}{e}
    \function{sol1}{e}
    \function{sol2}{e/2}
    \function{sol3}{e/6}
\end{variables}
	\type{input.function}
	\field{real} 
	\precision{3}
    \lang{de}{
	    \text{
	    Bestimmen Sie die Koeffizienten der Potenzreihe
        $\sum_{n=0}^{\infty} a_n (z-1)^n=\exp(z)$ im Entwicklungspunkt $z_0=1$.\\
        Geben Sie Ihre Ergebnisse exakt (oder auf drei Nachkommastellen genau) an.\\
        Der $0$-te Koeffizient ist gegeben durch $a_0=$ \ansref .\\
        Der $1$-te Koeffizient ist gegeben durch $a_1=$ \ansref .\\
        Der $2$-te Koeffizient ist gegeben durch $a_2=$ \ansref .\\
        Der $3$-te Koeffizient ist gegeben durch $a_3=$ \ansref .
    }}
    
    \begin{answer}
        \solution{sol0}
        \checkAsFunction[0.0005]{x}{0}{1}{2}          
    \end{answer}  
     \begin{answer}
        \solution{sol1}
        \checkAsFunction[0.0005]{x}{0}{1}{2}          
    \end{answer}
     \begin{answer}
        \solution{sol2}
        \checkAsFunction[0.0005]{x}{0}{1}{2}          
    \end{answer}
     \begin{answer}
        \solution{sol3}
        \checkAsFunction[0.0005]{x}{0}{1}{2}          
    \end{answer}
    \explanation{Es ist $\exp(z)=\exp(z-1)\cdot\exp(1)=e\cdot\exp(z-1)$. 
        Mit dem $n$-ten Koeffizienten $\frac{1}{n!}$ der Potenzreihe $\exp(z-1)$ ist also $a_n=\frac{e}{n!}$.}
\end{question}
%
%3
\begin{question}
\begin{variables}
	\randint[Z]{c}{2}{10}
    %\randint[Z]{a}{-9}{9}
	%\randint[Z]{k}{1}{9}
	%\function[calculate]{sol}{abs(c)}
    \function{sol0}{1}
    \function[calculate]{sol1}{c}
    \function[calculate]{sol2}{c^2/2}
    \function[calculate]{sol3}{c^3/6}
\end{variables}
	\type{input.function}
	\field{rational} 
	\precision{3}
    \lang{de}{
	    \text{
	    Bestimmen Sie die Koeffizienten der Potenzreihe
        $\sum_{n=0}^{\infty} a_n z^n=\exp(\var{c}z)$.\\
        Geben Sie Ihre Ergebnisse exakt (oder auf drei Nachkommastellen genau) an.\\
        Der $0$-te Koeffizient ist gegeben durch $a_0=$ \ansref .\\
        Der $1$-te Koeffizient ist gegeben durch $a_1=$ \ansref .\\
        Der $2$-te Koeffizient ist gegeben durch $a_2=$ \ansref .\\
        Der $3$-te Koeffizient ist gegeben durch $a_3=$ \ansref .
    }}
    
    \begin{answer}
        \solution{sol0}
        \checkAsFunction[0.0005]{x}{0}{1}{2}          
    \end{answer}  
     \begin{answer}
        \solution{sol1}
        \checkAsFunction[0.0005]{x}{0}{1}{2}          
    \end{answer}
     \begin{answer}
        \solution{sol2}
        \checkAsFunction[0.0005]{x}{0}{1}{2}          
    \end{answer}
     \begin{answer}
        \solution{sol3}
        \checkAsFunction[0.0005]{x}{0}{1}{2}          
    \end{answer}
    \explanation{Es ist $\exp(\var{c}z)=\sum_{n=0}^\infty \frac{(\var{c}z)^n}{n!}=\sum_{n=0}^\infty \frac{\var{c}^n}{n!}z^n$. 
       Es ist also $a_n=\frac{\var{c}^n}{n!}$.}
\end{question}
\end{problem}

\embedmathlet{gwtmathlet}

\end{content}
