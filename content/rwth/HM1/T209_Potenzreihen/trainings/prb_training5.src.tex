\documentclass{mumie.problem.gwtmathlet}
%$Id$
\begin{metainfo}
  \name{
    \lang{en}{Problem 5}
    \lang{de}{A05: Additionstheorem}
  }
  \begin{description} 
 This work is licensed under the Creative Commons License Attribution 4.0 International (CC-BY 4.0)   
 https://creativecommons.org/licenses/by/4.0/legalcode 

    \lang{en}{...}
    \lang{de}{...}
  \end{description}
  \corrector{system/problem/GenericCorrector.meta.xml}
  \begin{components}
    \component{js_lib}{system/problem/GenericMathlet.meta.xml}{gwtmathlet}
  \end{components}
  \begin{links}
  \end{links}
  \creategeneric
\end{metainfo}
\begin{content}
\begin{block}[annotation]
	Im Ticket-System: \href{https://team.mumie.net/issues/18224}{Ticket 18224}
\end{block}
\usepackage{mumie.genericproblem}
\lang{de}{\title{A05: Additionstheorem}}
\lang{en}{\title{Problem 5}}
\begin{block}[annotation]
Aufgabenpool mit vier sehr ähnlichen Aufgaben, aus dem zufällig ausgewählt wird.
\end{block}

\begin{problem}
\randomquestionpool{1}{4}
 %% Aufgabe 1
  \begin{question}
    \type{mc.multiple}
    \text{Entscheiden Sie, welche der folgenden Aussagen wahr sind.}
	%%
	\begin{choice}
        		\text{(a) Es gilt $\sin(x-y)=\sin(x)\cos(y)-\cos(x)\sin(y)$ für alle $x,y\in\R$. }
		\solution{true}
        \explanation[NOT[equalChoice(0000)] AND equalChoice(0??0)]{
        (a) ist eine Form des Additionstheorems des Sinus, also korrekt.}
        	\end{choice}
	\begin{choice}
        \text{(b)  Es gilt $\cos(2x)=\cos^2(x)-\sin^2(x)$ für alle $x\in\R$. }
		\solution{true}
       \explanation[NOT[equalChoice(0000)] AND equalChoice(?0?0)]{(b) ist das Additionstheorem des Kosinus für $x=y$, also korrekt.}
    \end{choice}
	\begin{choice}
     	\text{(c) Es gilt $\sin^2(2y)=4\cos^2(y)\sin^2(y)$ für alle $y\in\R$.}
		\solution{true}
        \explanation[NOT[equalChoice(0000)] AND equalChoice(??00)]{(c) ist korrekt: Man quadriere das Additionstheorem des Sinus für $x=y$.}
     \end{choice}
     \begin{choice}
     \text{(d) Keine der Aussagen ist wahr.}
     \explanation[equalChoice(???1)]{Nicht alle angegebenen Aussagen sind falsch. Versuchen Sie, Spezialfälle der Additionstheoreme zu erkennen.}
     \solution{false}
     \end{choice}
  \end{question}
%% Aufgabe 2
  \begin{question}
    \type{mc.multiple}
    \text{Entscheiden Sie, welche der folgenden Aussagen wahr sind.}
	%%
	\begin{choice}
        \text{(a)  Es gilt $\cos(2x)=\cos^2(x)-\sin^2(x)$ für alle $x\in\R$.}
		\solution{true}
        \explanation[NOT[equalChoice(0000)] AND equalChoice(0??0)]{
         (a) ist das Additionstheorem des Kosinus für $x=y$, also korrekt.}
        	\end{choice}
	\begin{choice}
        \text{(b) Es gilt $\sin(x-y)=\cos(x)\cos(y)-\sin(x)\sin(y)$ für alle $x,y\in\R$. }
		\solution{false}
        \explanation[NOT[equalChoice(0000)] AND equalChoice(?1?0)]{(b) ist falsch. Für die Einsetzung $x=y=0$ ist die linke Seite gleich $0$, aber die rechte Seite gleich $1$.}
     \end{choice}
	\begin{choice}
       	\text{(c) Es gilt $\cos(x-y)=\cos(x)\cos(y)+\sin(x)\sin(y)$ für alle $x,y\in\R$.}
		\solution{true}
        \explanation[NOT[equalChoice(0000)] AND equalChoice(??00)]{ (c) ist eine spezielle Form des Additionstheorems des Kosinus, also korrekt.}
     \end{choice}
     \begin{choice}
     \text{(d) Keine der Aussagen ist wahr.}
     \solution{false}
     \explanation[equalChoice(???1)]{Nicht alle angegebenen Aussagen sind falsch. Versuchen Sie, Spezialfälle der Additionstheoreme zu erkennen.}
     \end{choice}
  \end{question}



%% Aufgabe 3
  \begin{question}
    \type{mc.multiple}
    \text{Entscheiden Sie, welche der folgenden Aussagen wahr sind.}
	%%
	\begin{choice}
        \text{(a)  Es gilt $\cos(x-y)=-\cos(x)\sin(y)+\sin(x)\cos(y)$ für alle $x,y\in\R$..}
		\solution{false}
        \explanation[NOT[equalChoice(0000)] AND equalChoice(1??0)]{
         (a) ist falsch. Für $x=y=0$ hat die linke Seite den Wert $1$, die rechte aber den Wert $0$.}
        	\end{choice}
	\begin{choice}
        \text{(b) Es gilt $\cos(2x)=\cos^2(x)+\sin^2(x)$ für alle $x\in\R$. }
		\solution{false}
        \explanation[NOT[equalChoice(0000)] AND equalChoice(?1?0)]{(b) ist falsch. Die linke Seite kann negative Werte annehmen, die rechte nicht.}
     \end{choice}
	\begin{choice}
        \text{(c) Es gilt $\sin(x+y)=\cos(x)\sin(y)+\sin(x)\cos(y)$ für alle $x,y\in\R$. }
		\solution{true}
        \explanation[NOT[equalChoice(0000)] AND equalChoice(??00)]{(c) ist das Additionstheorem des Sinus, also korrekt.}
	\end{choice}
    \begin{choice}
     \text{(d) Keine der Aussagen ist wahr.}
     \solution{false}
          \explanation[equalChoice(???1)]{Nicht alle angegebenen Aussagen sind falsch. Versuchen Sie, Spezialfälle der Additionstheoreme zu erkennen.}
     \end{choice}
  \end{question}

  
%% Aufgabe 4
  \begin{question}
    \type{mc.multiple}
    \text{Entscheiden Sie, welche der folgenden Aussagen wahr sind.}
	%%
	\begin{choice}
        		\text{(a) Es gilt $\cos(x-y)=\cos(x)\cos(y)+\sin(x)\sin(y)$ für alle $x,y\in\R$. }
		\solution{true}
        \explanation[NOT[equalChoice(0000)] AND equalChoice(0??0)]{
         (a) ist das Additionstheorem des Kosinus.}
        	\end{choice}
	\begin{choice}
        \text{(b) Es gilt $\sin(-2y)=-2\sin^2(y)$ für alle $y\in\R$. }
		\solution{false}
        \explanation[NOT[equalChoice(0000)] AND equalChoice(?1?0)]{(b) ist falsch: Die linke Seite nimmt positive Werte an, die rechte nicht. }
     \end{choice}
	\begin{choice}
        \text{(c) Es gilt $\sin(x+y)=\cos(x)\cos(y)-\sin(x)\sin(y)$ für alle $x,y\in\R$. }
		\solution{false}
        \explanation[NOT[equalChoice(0000)] AND equalChoice(??10)]{(c) ist falsch: 
        Für $x=y=0$ ist die linke Seite gleich $0$, die rechte aber gleich $1$.}
	\end{choice}
    \begin{choice}
     \text{(d) Keine der Aussagen ist wahr.}
     \solution{false}
          \explanation[equalChoice(???1)]{Nicht alle angegebenen Aussagen sind falsch. Versuchen Sie, SJellen, Finnja 	[missing key: Generic.Unknown] 	[missing key: Generic.Unknown] 	[missing key: Generic.Unknown] 	Ja 	[missing key: Generic.Unknown] 	[missing key: Generic.Unknown] 	[missing key: Generic.Unknown] 	[missing key: Generic.Unknown] 	[missing key: Generic.Unknown] 	[missing key: Generic.Unknown] 	[missing key: Generic.Unknown] 	[missing key: Generic.Unknown] 	[missing key: Generic.Unknown] 	[missing key: Generic.Unknown] 	[missing key: Generic.Unknown] 	[missing key: Generic.Unknown] 	
pezialfälle der Additionstheoreme zu erkennen.}
     \end{choice}
  \end{question}  

\end{problem}
\embedmathlet{gwtmathlet}
\end{content}

  