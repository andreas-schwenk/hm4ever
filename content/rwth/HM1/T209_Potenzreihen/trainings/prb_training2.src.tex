\documentclass{mumie.problem.gwtmathlet}
%$Id$
\begin{metainfo}
  \name{
    \lang{de}{A02: Konvergenzradius}
    \lang{en}{}
  }
  \begin{description} 
 This work is licensed under the Creative Commons License Attribution 4.0 International (CC-BY 4.0)   
 https://creativecommons.org/licenses/by/4.0/legalcode 

    \lang{de}{Beschreibung}
    \lang{en}{}
  \end{description}
  \corrector{system/problem/GenericCorrector.meta.xml}
  \begin{components}
    \component{js_lib}{system/problem/GenericMathlet.meta.xml}{mathlet}
  \end{components}
  \begin{links}
\link{generic_article}{content/rwth/HM1/T209_Potenzreihen/g_art_content_27_konvergenzradius.meta.xml}{content_27_konvergenzradius}
\end{links}
  \creategeneric
\end{metainfo}
\begin{content}
\usepackage{mumie.genericproblem}

\lang{de}{
	\title{A02: Konvergenzradius}
}

\begin{block}[annotation]
	Im Ticket-System: \href{http://team.mumie.net/issues/9922}{Ticket 9922}
\end{block}

\begin{block}[annotation]
Aufgabenauswahl aus drei verschiedenen Reihen und randomisierten Koeffizienten.
\end{block}



\begin{problem}
\randomquestionpool{1}{3}


\begin{question}

\begin{variables}
	\randint[Z]{c}{-9}{9}
    \randint[Z]{a}{-9}{9}
	\randint[Z]{k}{1}{9}
	\function[calculate]{sol}{abs(c)}
	
\end{variables}
	\type{input.number}
	\field{real} 
	\precision{3}
    \lang{de}{
	    \text{
	    Bestimmen Sie den Konvergenzradius der Potenzreihe
        $\sum_{n=0}^{\infty} \frac{\var{a}\cdot n^{\var{k}}}{\var{c}^n} z^n.$\\
        Geben Sie Ihr Ergebnis als Bruch oder auf drei Nachkommastellen genau an.\\
        Der Konvergenzradius ist gegeben durch $r =$ \ansref .
    }}
    
    \begin{answer}
    %\text{Der Konvergenzradius ist gegeben durch $r =$}
        \solution{sol}        
        \explanation{Der Konvergenzradius einer Reihe der Form $\sum_{n=0}^\infty a\cdot n^k(\frac{z}{c})^n$ 
          ist (für $a\neq 0$)   nach dem Abelschen Lemma 
        derselbe wie der von $\sum_{n=0}^\infty (\frac{z}{c})^n$, also $r=\vert c\vert$ .}
     \end{answer}   
    
\end{question}
%2
\begin{question}

 \begin{variables}
	\randint[Z]{c}{-9}{9}
	\randint[Z]{a}{-9}{9}
	\randint[Z]{k}{1}{9}
	\function[calculate]{sol}{1/abs(c)}
	
 \end{variables}
	\type{input.number}
	\field{real}
    \precision{3}
    \lang{de}{
	    \text{
	    Bestimmen Sie den Konvergenzradius der Potenzreihe
        $\sum_{n=0}^{\infty} \frac{\var{c}^n}{\var{a} \cdot n^{\var{k}}} z^n.$\\
        Geben Sie Ihr Ergebnis als Bruch oder auf drei Nachkommastellen genau an.
    }}
    
    \begin{answer}
    \text{Der Konvergenzradius ist gegeben durch $r =$}
	    \solution{sol}
        \explanation{Der Konvergenzradius einer Reihe der Form $\sum_{n=0}^\infty \frac{(cz)^n}{an^k}$ 
         ist $r=\frac{1}{\vert c\vert}$, weil er nach dem Abelschen Lemma
         derselbe ist wie der von  $\sum_{n=0}^\infty (cz)^n$.}
    \end{answer}
    
\end{question}
%3
\begin{question}

 \begin{variables}
	\randint{a}{1}{2}
    \function[normalize]{aa}{(n^a)}
	\function[calculate]{sol}{1/e^(a-1)}
	
 \end{variables}
	\type{input.function}
	\field{real} 
	\precision{3}
    \lang{de}{
	    \text{
	    Bestimmen Sie den Konvergenzradius der Potenzreihe
        $\sum_{n=0}^{\infty} (1+\frac{1}{n})^{\var{aa}} z^n.$\\
        Geben Sie Ihr Ergebnis exakt oder auf drei Nachkommastellen genau an.
    }}
    
    \begin{answer}
    \text{Der Konvergenzradius ist gegeben durch $r =$}
	    \solution{sol}
        \checkAsFunction[1E-3]{x}{-1}{1}{10}
       \explanation{Wende den Satz von Cauchy-Hadamard an!}
    \end{answer}
    
\end{question}

\end{problem}


\embedmathlet{mathlet}

\end{content}