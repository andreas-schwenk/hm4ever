%$Id:  $
\documentclass{mumie.article}
%$Id$
\begin{metainfo}
  \name{
    \lang{de}{Überblick: Potenzreihen}
    \lang{en}{overview: }
  }
  \begin{description} 
 This work is licensed under the Creative Commons License Attribution 4.0 International (CC-BY 4.0)   
 https://creativecommons.org/licenses/by/4.0/legalcode 

    \lang{de}{Beschreibung}
    \lang{en}{}
  \end{description}
  \begin{components}
  \end{components}
  \begin{links}
\link{generic_article}{content/rwth/HM1/T209_Potenzreihen/g_art_content_28_exponentialreihe.meta.xml}{content_28_exponentialreihe}
\link{generic_article}{content/rwth/HM1/T209_Potenzreihen/g_art_content_27_konvergenzradius.meta.xml}{content_27_konvergenzradius}
\end{links}
  \creategeneric
\end{metainfo}
\begin{content}
\begin{block}[annotation]
	Im Ticket-System: \href{https://team.mumie.net/issues/30128}{Ticket 30128}
\end{block}


\begin{block}[annotation]
Im Entstehen: Überblicksseite für Kapitel Potenzreihen
\end{block}

\usepackage{mumie.ombplus}
\ombchapter{1}
\lang{de}{\title{Überblick: Potenzreihen}}
\lang{en}{\title{}}



\begin{block}[info-box]
\lang{de}{\strong{Inhalt}}
\lang{en}{\strong{Contents}}


\lang{de}{
    \begin{enumerate}%[arabic chapter-overview]
   \item[9.1] \link{content_27_konvergenzradius}{Potenzreihen und deren Konvergenzradius}
   \item[9.2] \link{content_28_exponentialreihe}{Exponentialreihe, Sinus und Kosinus}
   \end{enumerate}
} %lang

\end{block}

\begin{zusammenfassung}

\lang{de}{Potenzreihen entstehen, wenn man in die Glieder einer Reihe die Potenzen einer Funktionsvariablen schreibt. 
Die Partialsummen von Potenzreihen sind also Polynome. Potenzreihen stellen eine sehr wichtige Klasse von Funktionen dar.
Man studiert sie am besten gleich auf den komplexen Zahlen. \\
Ihr Konvergenzbereich ist immer eine Kreisscheibe um den Entwicklungspunkt, deren Radius als Konvergenzradius bezeichnet wird.
Für jedes Argument in dieser (offenen) Kreisscheibe konvergiert die Potenzreihe absolut, definiert dort dadurch eine Funktion, 
die besonders gute Eigenschaften hat. Einige davon (Stetigkeit, Differenzierbarkeit) werden wir in späteren Kapiteln beweisen.
\\
Über den Satz von Cauchy-Hadamard und die Quotientenregel kann der Konvergenzradius explizit bestimmt werden.

Mit der Exponential-, der Sinus- und der Kosinusreihe stellen wir die für die Anwendung wichtigsten Potenzreihen vor.
Durch die Identifikation dieser Reihen mit den bekannten Funktion werden wir wichtige Eigenschaften dieser Funktionen beweisen sowie
$\cos(x)$ und $\sin(x)$ als Real- bzw. Imaginärteil
der Exponentialreihe $\exp(ix)$ begreifen.
}


\end{zusammenfassung}

\begin{block}[info]\lang{de}{\strong{Lernziele}}
\lang{en}{\strong{Learning Goals}} 
\begin{itemize}[square]
\item \lang{de}{Sie kennen die Begriffe Entwicklungspunkt, Konvergenzkreis(scheibe) und Konvergenzradius einer Potenzreihe und deren Bedeutung. 
Sie  veranschaulichen diese geometrisch.}
\item \lang{de}{Sie berechnen  den Konvergenzradius  in Beispielen mit Hilfe der Formel von Cauchy-Hadamard und der Quotientenregel.}
\item \lang{de}{Sie erkennen wichtige Potenzreihen wie $\exp$, $\sin$ und $\cos$, deren Eigenschaften 
und können den Zusammenhang zwischen ihnen erklären.}
%\item \lang{de}{}
\end{itemize}
\end{block}




\end{content}
