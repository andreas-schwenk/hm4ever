\documentclass{mumie.element.exercise}
%$Id$
\begin{metainfo}
  \name{
    \lang{en}{Exercise 3}
    \lang{de}{Ü03: Produktreihe}
  }
  \begin{description} 
 This work is licensed under the Creative Commons License Attribution 4.0 International (CC-BY 4.0)   
 https://creativecommons.org/licenses/by/4.0/legalcode 

    \lang{en}{Exercise 3}
    \lang{de}{Ü03: Produktreihe}
  \end{description}
  \begin{components}
  \end{components}
  \begin{links}
\link{generic_article}{content/rwth/HM1/T208_Reihen/g_art_content_26_produkt_von_reihen.meta.xml}{content_26_produkt_von_reihen}
\end{links}
  \creategeneric
\end{metainfo}

\begin{content}
\begin{block}[annotation]
	Im Ticket-System: \href{https://team.mumie.net/issues/16436}{Ticket 16436}
\end{block}
  \title{
    \lang{en}{Ü03: Produktreihe}
    \lang{de}{Übung 3}
  }
  \lang{en}{}
  \lang{de}{Entwickeln Sie $f(z)=\exp(z)\cdot \frac{1}{1-z}$ in eine Potenzreihe mit Entwicklungspunkt $z_0=0$. Welchen Konvergenzradius besitzt sie?} 
  \begin{tabs*}[\initialtab{0}\class{exercise}]   
    \tab{
      \lang{en}{a}
      \lang{de}{Lösung}
    }
    \begin{incremental}[\initialsteps{1}]
      \step
Wir kennen die Potenzreihenentwicklungen beider Faktoren, sodass wir die des Produkts durch 
deren \ref[content_26_produkt_von_reihen][Cauchy-Produkt]{def:cauchy-prod} bestimmen können:
  Es gilt
\[\exp(z)= \sum_{k=0}^{\infty} \frac{z^k}{k !}\]
f\"ur alle $z\in\mathbb{C}$, und für $\vert z\vert < 1$ gilt
\[
\frac{1}{1-z} = \sum_{k=0}^\infty z^k.
\]
\step 
Damit erhalten wir
\begin{align*}
f(z) = \exp(z) \frac{1}{1-z} &=& \left(\sum_{l=0}^{\infty} \frac{z^l}{l !}\right)
\left(\sum_{k=0}^{\infty} z^k\right) \\
&=&\sum_{n=0}^{\infty} \sum_{m=0}^n \frac{z^{m}}{m !} \cdot z^{n-m} \\
&=& \sum_{n=0}^\infty \left( \sum_{m=0}^n \frac{1}{m!}\right) z^n
\end{align*}
für alle $z$ mit $\vert z \vert < 1$. 
Die Rechnung gilt nach dem \ref[content_26_produkt_von_reihen][Cauchy-Produkt-Satz]{thm:cauchy-prod} für alle $z$ mit $\vert z \vert <1$.
Für diese Werte ist die entstehende Potenzreihe in jedem Fall konvergent. Der Konvergenzradius ist also mindestens $1$.
Grundsätzlich könnte sich der Konvergenzbereich auch vergrößern. Wir betrachten den Fall, dass $z=1$ ist.
Die Reihe in $z=1$ ist
\[
\sum_{n=0}^\infty \left( \sum_{m=0}^n \frac{1}{m!}\right)
\]
und diese ist nicht konvergent, da $\left(  \sum_{m=0}^n \frac{1}{m!}\right)_{n\in \N}$ keine Nullfolge ist. Wäre der Konvergenzradius größer als $1$, müsste die Potenzreihe in $z=1$ konvergieren.
Daraus folgern wir, dass der Konvergenzradius höchstens $1$ sein kann.
Zusammen mit obiger Überlegung ergibt sich, dass der Konvergenzradius genau $1$ sein muss.
    \end{incremental}
  \end{tabs*}
\end{content}

