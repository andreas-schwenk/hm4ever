\documentclass{mumie.element.exercise}
%$Id$
\begin{metainfo}
  \name{
    \lang{en}{Ü05: Additionstheorem}
    \lang{de}{Ü05: Additionstheorem}
  }
  \begin{description} 
 This work is licensed under the Creative Commons License Attribution 4.0 International (CC-BY 4.0)   
 https://creativecommons.org/licenses/by/4.0/legalcode 

    \lang{en}{...}
    \lang{de}{...}
  \end{description}
  \begin{components}
  \end{components}
  \begin{links}
\link{generic_article}{content/rwth/HM1/T209_Potenzreihen/g_art_content_28_exponentialreihe.meta.xml}{content_28_exponentialreihe}
\end{links}
  \creategeneric
\end{metainfo}
\begin{content}
\begin{block}[annotation]
	Im Ticket-System: \href{https://team.mumie.net/issues/18209}{Ticket 18209}
\end{block}
\title{
\lang{de}{Ü05: Additionstheorem}
}
Beweisen Sie das Additionstheorem des Sinus
\[\sin(x+y)=\sin(x)\cos(y)+\cos(x)\sin(y)\:.\]

\begin{tabs*}[\initialtab{0}\class{exercise}]
\tab{Lösung}
Wir verwenden die \ref[content_28_exponentialreihe][Interpretation von Kosinus und Sinus als Real- bzw. Imaginärteil von $\exp(ix)$]{thm:sin_cos_exp} 
und benutzen die Funktionalgleichung  
$\exp(x+y)=\exp(x)\cdot\exp(y)$ der Exponentialfunktion:
\begin{eqnarray*}
&&\sin(x)\cos(y)+\cos(x)\sin(y)\\
&=&\frac{1}{4i}\big(\exp(ix)-\exp(-ix)\big)\big(\exp(iy)+\exp(-iy)\big)\\
&&
+\frac{1}{4i}\big(\exp(ix)+\exp(-ix)\big)\big(\exp(iy)-\exp(-iy)\big)\\
&=&
\frac{1}{4i}\big(\exp(i(x+y))+\exp(i(x-y))-\exp(i(y-x))-\exp(-i(x+y))\\
&&\quad +\exp(i(x+y))-\exp(i(x-y))+\exp(i(y-x))-\exp(-i(x+y))\big)\\
&=&\frac{1}{4i}\big(2\exp(i(x+y))-2\exp(-i(x+y))\big)\\
&=&\sin(x+y)\:.
\end{eqnarray*}
\end{tabs*}

\end{content}

