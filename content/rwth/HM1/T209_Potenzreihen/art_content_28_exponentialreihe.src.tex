%$Id:  $
\documentclass{mumie.article}
%$Id$
\begin{metainfo}
  \name{
    \lang{de}{Exponentialreihe, Sinus und Kosinus}
    \lang{en}{}
  }
  \begin{description} 
 This work is licensed under the Creative Commons License Attribution 4.0 International (CC-BY 4.0)   
 https://creativecommons.org/licenses/by/4.0/legalcode 

    \lang{de}{Beschreibung}
    \lang{en}{}
  \end{description}
  \begin{components}
    \component{generic_image}{content/rwth/HM1/images/g_tkz_T209_HyperbolicFunctions.meta.xml}{T209_HyperbolicFunctions}
    \component{generic_image}{content/rwth/HM1/images/g_tkz_T209_ComplexExponential.meta.xml}{T209_ComplexExponential}
    \component{generic_image}{content/rwth/HM1/images/g_tkz_T209_Exponential.meta.xml}{T209_Exponential}
    \component{generic_image}{content/rwth/HM1/images/g_img_00_video_button_schwarz-blau.meta.xml}{00_video_button_schwarz-blau}
    \component{generic_image}{content/rwth/HM1/images/g_img_00_Videobutton_schwarz.meta.xml}{00_Videobutton_schwarz}
  \end{components}
  \begin{links}
    \link{generic_article}{content/rwth/HM1/T208_Reihen/g_art_content_25_konvergenz_kriterien.meta.xml}{konv-krit}
    \link{generic_article}{content/rwth/HM1/T205_Konvergenz_von_Folgen/g_art_content_15_monotone_konvergenz.meta.xml}{monot-konv}
    \link{generic_article}{content/rwth/HM1/T208_Reihen/g_art_content_26_produkt_von_reihen.meta.xml}{cauchy-produkt}
    \link{generic_article}{content/rwth/HM1/T104_weitere_elementare_Funktionen/g_art_content_15_exponentialfunktionen.meta.xml}{exponentialfunktion}
    \link{generic_article}{content/rwth/HM1/T105_Trigonometrische_Funktionen/g_art_content_19_allgemeiner_sinus_cosinus.meta.xml}{sinus-und-kosinus}
  \end{links}
  \creategeneric
\end{metainfo}
\begin{content}
\usepackage{mumie.ombplus}
\ombchapter{9}
\ombarticle{2}

\lang{de}{\title{Exponentialreihe, Sinus und Kosinus}}
 
\begin{block}[annotation]
  
  
\end{block}
\begin{block}[annotation]
  Im Ticket-System: \href{http://team.mumie.net/issues/9685}{Ticket 9685}\\
\end{block}

\begin{block}[info-box]
\tableofcontents
\end{block}




\section{Exponentialreihe und Exponentialfunktion}\label{sec:exp-reihe}

Wir haben im Abschnitt \ref[konv-krit][Konvergenzkriterien für Reihen]{ex:exp-reihe} gesehen, dass die
 Exponentialreihe $\exp(z)=\sum_{n=0}^\infty \frac{z^n}{n!}$ für jedes $z\in \C$ absolut konvergiert.
Mit Hilfe der Potenzreihe können wir also (wie schon durch die Benennung $\exp(z)$ angedeutet) eine komplexwertige Funktion $\exp$ mit Definitionsbereich $\C$ definieren. 
 

\begin{definition}
Die komplexe Exponentialfunktion $\exp:\C\to \C$ ist definiert durch die Exponentialreihe
\[  z\:\mapsto\:\exp(z)= \sum_{n=0}^\infty \frac{z^n}{n!} \]
für alle $z\in \C$.\\
\floatright{\href{https://www.hm-kompakt.de/video?watch=328}{\image[75]{00_Videobutton_schwarz}}}\\\\
\end{definition}
Natürlich suggeriert die Benennung der Exponentialreihe als $\exp(x)$ den Zusammenhang mit der
\ref[exponentialfunktion][natürlichen Exponentialfunktion]{eulernum} $e^x$ nicht von ungefähr.
Dass  diese Funktionen für reelle Zahlen $x$ wirklich übereinstimmen, 
ist aber an dieser Stelle nicht leicht zu sehen, 
denn wir haben noch keinen Begriff von  Stetigkeit oder gar Differenzierbarkeit zur Verfügung, 
die wir erst in den nächsten Kapiteln einführen. 
Stattdessen zeigen wir hier erst einmal die Übereinstimmung der Funktionen auf den rationalen Zahlen.

%Der Zusammenhang mit der aus der Schule bekannten $e$-Funktion $e^x$ ergibt sich aus:

\begin{theorem}\label{thm:exp_auf_QQ}
Für alle rationalen Zahlen $x\in \Q$ ist
\[  \exp(x) = e^x ,\]
wobei $e$ die \ref[monot-konv][Eulersche Zahl]{sec:eulersche-zahl} bezeichnet.
\end{theorem}



\begin{proof*}
Zunächst ist einzusehen, dass $\exp(1)=\sum_{k=0}^\infty \frac{1}{k!}$ mit dem Grenzwert
\[ e= \lim_{n\to \infty} \left(1+\frac{1}{n}\right)^n \]
übereinstimmt.
\begin{incremental}[\initialsteps{0}]
\step
Hierzu sei $s_n:=\sum_{k=0}^n \frac{1}{k!}$ die $n$-te Partialsumme von $\exp(1)$ für $n\in \N$.
Dann gilt für alle $n\in \N$:
\begin{eqnarray*}
 \left(1+\frac{1}{n}\right)^n &=& \sum_{k=0}^n \binom{n}{k} (\frac{1}{n})^k 
= \sum_{k=0}^n \frac{n(n-1)\cdots (n-k+1)}{k! \cdot n^k} \\
& \leq & \sum_{k=0}^n \frac{1}{k!} = s_n.
\end{eqnarray*}
Also ist auch 
\[ e= \lim_{n\to \infty} \left(1+\frac{1}{n}\right)^n \leq\lim_{n\to \infty}s_n =\exp(1).\]
\step
Weiter ist aber für festes $m\in \N$ und $n\geq m$:
\begin{eqnarray*}
\left(1+\frac{1}{n}\right)^n &=& \sum_{k=0}^n \frac{n(n-1)\cdots (n-k+1)}{k! \cdot n^k} 
 \geq  \sum_{k=0}^m \frac{n(n-1)\cdots (n-k+1)}{k! \cdot n^k} \\
 &\geq & \sum_{k=0}^m \frac{1}{k!}\left( \frac{n-m+1}{n}\right)^k \geq  
 \sum_{k=0}^m \frac{1}{k!}\left( \frac{n-m+1}{n}\right)^m \\
 &=& \left( \frac{n-m+1}{n}\right)^m \cdot s_m =\left( 1 - \frac{m-1}{n}\right)^m \cdot s_m .
 \end{eqnarray*}
Damit ist
\[  e= \lim_{n\to \infty} \left(1+\frac{1}{n}\right)^n \geq \lim_{n\to \infty} \left( 1 - \frac{m-1}{n}\right)^m \cdot s_m =s_m \]
für alle $m\in \N$, und damit auch
\[ \exp(1)=\lim_{m\to \infty} s_m \leq e.\]

Insgesamt erhält man also $\exp(1)=e$.
\end{incremental}
Des Weiteren ist nach Definition $\exp(0)=1=e^0$.
Dass $\exp(x)$ und $e^x$ wirklich für alle rationalen Zahlen übereinstimmen, zeigt man dann schrittweise
mit Hilfe der \ref[cauchy-produkt][Funktionalgleichung für die Exponentialreihe]{ex:funktionalgleichung-exp}
%\[ \left( \sum_{n=0}^\infty \frac{z^n}{n!} \right) \left( \sum_{n=0}^\infty \frac{w^n}{n!}\right)
%= \sum_{n=0}^\infty \frac{(z+w)^n}{n!}, \text{d.h.}\]
\[ \exp(z)\cdot \exp(w)=\exp(z+w) \]
für alle $z,w\in \C$.
%die wir im Abschnitt \ref[cauchy-produkt][Produkt von Reihen]{ex:funktionalgleichung-exp} gesehen hatten.
\begin{incremental}[\initialsteps{0}]
\step Zunächst erhält man aus obiger Gleichung induktiv für alle $m\in \N$ und $z\in \C$ die Gleichung $\exp(mz)=\exp(z)^m$,
denn
\[ \exp(mz)=\exp((m-1)z+z)=\exp((m-1)z)\cdot \exp(z)=\exp(z)^{m-1}\cdot \exp(z)=\exp(z)^m. \]
Für alle $m\in \N$ gilt somit
\[ \exp(m)=\exp(m\cdot 1)=\exp(1)^m= e^m.\]
\step Für positive rationale Zahlen $\frac{p}{q}\in \Q$ ist dann
\[ \left(\exp\left(\frac{p}{q}\right)\right)^q = \exp(q\cdot \frac{p}{q})=\exp(p)=e^p. \]
Weil die Exponentialreihe nur positive Koeffizienten hat, ist $\exp(\frac{p}{q})$ positiv.
Also folgt wegen der Eindeutigkeit der positiven Wurzel
\[ \exp(\frac{p}{q}) =\sqrt[q]{e^p}=e^{\frac{p}{q}}. \]
\step Schließlich erhält man für negative rationale Zahlen $-x$ (mit $x\in \Q_+$):
\[ \exp(-x)=\frac{\exp(x+(-x))}{\exp(x)}=\frac{1}{e^x}=e^{-x}. \]
\end{incremental}
\end{proof*}


\begin{remark}
Mit der \link{exponentialfunktion}{intuitiven Einführung} von $e^x$ für reelle $x$ wie in der Schule gilt
der obige Satz sogar für alle reellen Zahlen $x$. Formal korrekt kann man also $e^x$ als Grenzwert der
Reihe $ \sum_{n=0}^\infty \frac{x^n}{n!}$ definieren und erhält den aus der Schule bekannten Graphen
für diese reelle Funktion.
% Graf der e-Funktion
\begin{center}
\image{T209_Exponential}
\end{center}
%
\end{remark}

Die (komplexe) Exponentialfunktion hat besondere Eigenschaften, die hier aufgeführt werden:

\begin{theorem}\label{thm:eigenschaften_von_exp}
Die komplexe Exponentialfunktion $\exp$ hat für alle $z,w\in \C$ die folgenden Eigenschaften:
\begin{enumerate}
\item $\exp(z+w)=\exp(z)\exp(w)$,
\item $\exp(z)\neq 0$ und $\exp(-z)=\frac{1}{\exp(z)}$,
\item für die konjugiert komplexe Zahl $\bar{z}$ gilt $\exp(\bar{z})=\overline{\exp(z)}$,
\item für alle reellen Zahlen $x\in \R$ ist $\exp(x)\in \R$ und $\exp(x)>0$,
\item für alle rein imaginären Zahlen $iy$ ($y\in \R$) ist $|\exp(iy)|=1$,
\item $|\exp(z)|=\exp(\Re(z))$.
\end{enumerate}
\end{theorem}

\begin{proof*}
Die erste Gleichung ist die \ref[cauchy-produkt][Funktionalgleichung für die Exponentialreihe]{ex:funktionalgleichung-exp}.
\begin{incremental}[\initialsteps{0}]
\step
Aus der Eigenschaft $\exp(0)=1$ folgt dann unmittelbar
\[ 1=\exp(z-z)=\exp(z)\exp(-z), \] 
was $\exp(z)\neq 0$ und $\exp(-z)=\frac{1}{\exp(z)}$ impliziert.
\step
Dadurch, dass die Exponentialreihe reelle Koeffizienten hat, gilt für die Partialsummen
\[  \sum_{n=0}^m \frac{\bar{z}^n}{n!}= \overline{ \sum_{n=0}^m \frac{z^n}{n!} } \]
und daher auch für die Grenzwerte
\[ \exp(\bar{z}) =\sum_{n=0}^\infty \frac{\bar{z}^n}{n!}=\overline{ \sum_{n=0}^\infty \frac{z^n}{n!} }
=\overline{\exp(z)}. \]
\step
Die Bedingung $\exp(x)\in \R_+$ für reelle $x\in \R$ erhält man für positive reelle $x$ direkt
aus der Definition $\exp(x)= \sum_{n=0}^\infty \frac{x^n}{n!}>0$. 
\step
Für negative $x$ ist
\[ \exp(x)=\frac{1}{\exp(-x)}>0, \]
da $-x$ positiv ist.

Für rein imaginäre Zahlen $iy$ ($y\in \R$) ist $\overline{iy}=-iy$ und damit ist
\[ |\exp(iy)|=\sqrt{\exp(iy) \overline{\exp(iy)}}=\sqrt{\exp(iy) \exp(\overline{iy})}
=\sqrt{\exp(iy) \exp(-iy)}=\sqrt{1}=1. \]

Die letzte Gleichung folgt schließlich aus $\exp(\Re(z))=| \exp(\Re(z))|$ und $|\exp(i\Im(z))| =1$:
\[ |\exp(z)|=| \exp(\Re(z)+i\Im(z))|=| \exp(\Re(z))\cdot  \exp(i\Im(z))|=
| \exp(\Re(z))|\cdot |\exp(i\Im(z))| = \exp(\Re(z)) .\]
\end{incremental}
\end{proof*}

Das folgende Video beschäftigt sich ausführlich mit der Exponentialfunktion:
\floatright{\href{https://api.stream24.net/vod/getVideo.php?id=10962-2-10871&mode=iframe&speed=true}{\image[75]{00_video_button_schwarz-blau}}}\\

\section{Sinus und Kosinus}\label{sec:sinus-kosinus}

Für die bekannten \link{sinus-und-kosinus}{Winkelfunktionen} Sinus und Kosinus erhält
man ebenso Darstellungen als Potenzreihen.
Ein Beweis für das folgende Theorem bleiben wir schuldig, 
benötigt er doch weit mehr mathematische Theorie. Das
Theorem kann auch als Definition der Kosinus- und Sinus-Reihen verstanden werden.




\begin{theorem}\label{thm:sin_cos_exp}
Für jede reelle Zahl $x$ gilt:
\[   \cos(x)=\Re(\exp(ix))=\frac{1}{2}\left(\exp(ix)+\exp(-ix)\right)\]
%\quad \text{und} 
und
%\quad 
\[\sin(x)=\Im(\exp(ix))=\frac{1}{2i}(\exp(ix)-\exp(-ix)). \]
Insbesondere hat man die Potenzreihen-Darstellungen
\[ \cos(x)= \sum_{k=0}^\infty\;(-1)^k\; \frac{x^{2k}}{(2k)!} 
    = 1 -\frac{x^2}{2!} + \frac{x^4}{4!} - \ldots \]
    und
    \[ \sin(x) = \sum_{k=0}^\infty \; (-1)^k\;\frac{x^{2k+1}}{(2k+1)!} 
    =x - \frac{x^3}{3!} + \frac{x^5}{5!}- \ldots\, ,\]
die für alle $x\in \R$ (sogar für alle $x\in\C$) absolut konvergieren.\\
\floatright{\href{https://www.hm-kompakt.de/video?watch=330}{\image[75]{00_Videobutton_schwarz}}}\\\\
\end{theorem}




\begin{remark}
\label{rem:sin_cos_exp}
\begin{enumerate}
\item
Nimmt man 
\[   \cos(x)=\Re(\exp(ix)) \quad \text{und} \quad \sin(x)=\Im(\exp(ix)) \]
als gegeben an, erhält man die Potenzreihen-Darstellungen aus der Potenzreihe für $\exp$.
Es ist nämlich
\begin{eqnarray*}
\exp(ix) &=& \sum_{n=0}^\infty \frac{(ix)^n}{n!}
= \sum_{n=0}^\infty \frac{i^n x^n}{n!}\\
&=& \sum_{n\geq 0, n\text{ gerade}} \frac{(-1)^{n/2} x^n}{n!}
+  \sum_{n\geq 1, n\text{ ungerade}} \frac{i\cdot (-1)^{(n-1)/2} x^n}{n!}\\
&=&  \sum_{k=0}^\infty  \frac{(-1)^k x^{2k}}{(2k)!}
+ i \sum_{k=0}^\infty \;\frac{(-1)^k x^{2k+1}}{(2k+1)!} \, .
\end{eqnarray*}
\item Aus der Gleichung $|\exp(ix)|=1$ für reelle $x$ sieht man, 
dass die Werte $\exp(ix)$ auf dem komplexen Einheitskreis liegen. 
Trägt man Real- und Imaginärteil an den Achsen ab, sieht man, dass 
tatsächlich $\Re(\exp(ix))=\cos(\varphi)$ und $\Im(\exp(ix))=\sin(\varphi)$ für einen Winkel $\varphi$ gelten.
Dass aber $\varphi=x$ (jedenfalls bis auf Vielfache von $2\pi$ im Bogenma"s) gilt, ist, wie erwähnt, mathematisch tiefliegender.
%
\begin{center}
\image{T209_ComplexExponential}
\end{center}
%
\item Aus der Gleichung $|\exp(ix)|=1$ erhält man den trigonometrischen Satz des Pythagoras:
\[ 1= |\exp(ix)|^2=\Re(\exp(ix))^2+ \Im(\exp(ix))^2=\cos(x)^2+\sin(x)^2 \]
für alle $x\in \R$.
\item Die Additionstheoreme
\[ \cos(x+y)=\cos(x)\cos(y)-\sin(x)\sin(y)\]
und
\[\sin(x+y)=\sin(x)\cos(y)+\cos(x)\sin(y) \]
lassen sich wiederum direkt aus der Gleichung
\[  \exp(ix+iy)=\exp(ix)\exp(iy) \]
herleiten.
\end{enumerate}
\begin{tabs*}[\initialtab{0}]
\tab{Beispiel}
Wir beweisen das Additionstheorem für den Kosinus
\[\cos(x+y)=\cos(x)\cos(y)-\sin(x)\sin(y).\]
Unter Verwendung der Definition von Sinus und Kosinus und der Funktionalgleichung der Exponentialreihe rechnen wir
\begin{eqnarray*}
&&\cos(x)\cos(y)-\sin(x)\sin(y)\\&=&
\frac{1}{4}\big(\exp(ix)+\exp(-ix)\big)\big(\exp(iy)+\exp(-iy)\big)\\
&&-\frac{1}{4i^2}\big(\exp(ix)-\exp(-ix)\big)\big(\exp(iy)-\exp(-iy)\big)\\
&=&\frac{1}{4}\big(\exp(i(x+y))+\exp(i(y-x))+\exp(i(x-y))+\exp(-i(x+y))\\
&&\quad  +\exp(i(x+y))-\exp(i(y-x))-\exp(i(x-y))+\exp(-i(x+y))\big)\\
&=&\frac{1}{4}\big(2\exp(i(x+y))+2\exp(-i(x+y))\big)\\
&=&\cos(x+y)\:.
\end{eqnarray*}
\end{tabs*}
\end{remark}
\begin{remark}\label{rem:euler_formel}
Mit dem Wissen über die trigonometrischen Funktionen, dass $\cos(\pi)=-1$ und $\sin(\pi)=0$ ist,
erhält man die fundamentale Gleichung (\textbf{Eulersche Identität})
\[   e^{i\pi}=-1\quad  \text{ bzw. }\quad e^{i\pi}+1=0\:, \]
wobei in Analogie zur reellen Exponentialfunktion $e^{i\pi}$ als $\exp(i\pi)$ zu verstehen ist.
\end{remark}

Den Zusammenhang zwischen der komplexen Exponentialfunktion und den Winkelfunktionen
erläutert das folgende Video:
\floatright{\href{https://api.stream24.net/vod/getVideo.php?id=10962-2-10872&mode=iframe&speed=true}{\image[75]{00_video_button_schwarz-blau}}}\\


\section{Sinus hyperbolicus und Kosinus hyperbolicus}\label{sec:sinh-cosh}
Es hindert uns nichts daran, die Exponentialreihe auch ohne eine imaginäre Einheit mitzuschleppen in zwei Teilreihen aufzuteilen. 
Dann gelangt man zu den folgenden Funktionen.
\begin{definition}\label{def:sinh-cosh}
Die Reihe 
\[\sinh (x)=\sum_{n=0}^\infty \frac{x^{2n+1}}{(2n+1)!}\]
heißt die \notion{Sinus hyperbolicus}-Reihe, und die Reihe
\[\cosh (x)=\sum_{n=0}^\infty \frac{x^{2n}}{(2n)!}\]
heißt die \notion{Kosinus hyperbolicus}-Reihe.\\
\floatright{\href{https://www.hm-kompakt.de/video?watch=333}{\image[75]{00_Videobutton_schwarz}}}\\\\
\end{definition}
Obwohl diese Reihen offenbar für komplexe Argumente definiert sind, 
betrachtet man sie normalerweise als reelle Funktionen und verwendet daher das Argument $x$.
\begin{theorem}\label{thm:sinh-cosh}
Die Reihen Sinus hyperbolicus und Kosinus hyperbolicus haben Konvergenzradius unendlich, sind also für jede komplexe Zahl konvergent.
Es gelten
\[\sinh(x)=\frac{1}{2}(e^x-e^{-x})=-i\sin(ix),\]
\[\cosh(x)=\frac{1}{2}(e^x+e^{-x})=\cos(ix),\]
sowie
\[e^x=\cosh(x)+\sinh(x).\]
Weiter ergibt sich
\[\cosh^2(x)-\sinh^2(x)=1.\]
\begin{center}
\image{T209_HyperbolicFunctions}
\end{center}

Durch den Graphen des Kosinus hyperbolicus wird zum Beispiel eine an zwei gleich hohen Punkten befestigte durchhängende Kette dargestellt, 
daher wird dieser Graph auch als \emph{Kettenlinie} bezeichnet.
\end{theorem}

\end{content}