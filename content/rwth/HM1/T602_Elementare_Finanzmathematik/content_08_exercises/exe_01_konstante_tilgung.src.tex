\documentclass{mumie.element.exercise}
%$Id$
\begin{metainfo}
  \name{
    \lang{en}{Exercise 1: Constant amortization}
    \lang{de}{Ü01: konstante Tilgung}
    \lang{zh}{...}
    \lang{fr}{...}
  }
  \begin{description} 
 This work is licensed under the Creative Commons License Attribution 4.0 International (CC-BY 4.0)   
 https://creativecommons.org/licenses/by/4.0/legalcode 

    \lang{en}{...}
    \lang{de}{...}
    \lang{zh}{...}
    \lang{fr}{...}
  \end{description}
  \begin{components}
  \end{components}
  \begin{links}
  \end{links}
  \creategeneric
\end{metainfo}
\begin{content}
\title{
    \lang{en}{Exercise 1: Constant amortization}
    \lang{de}{Ü01: konstante Tilgung}
    \lang{zh}{...}
    \lang{fr}{...}
}
\begin{block}[annotation]
	Im Ticket-System: \href{https://team.mumie.net/issues/22669}{Ticket 22669}
\end{block}


\lang{de}{Eine Schuld von 36.000 \euro wird zu Jahresbeginn zu 10\% ausgeliehen.
Sie soll in 3 Jahren durch nachschüssige, konstante Raten getilgt werden.
Wie lautet der Tilgungsplan?}

\lang{en}{
A loan of 36,000 \euro with an interest rate of 10 \% 
is taken out at the beginning of the year.
It should be paid off in 3 years with constant payments in arrears.
What will the amortization schedule be?
}

\begin{tabs*}[\initialtab{0}\class{exercise}]
\tab{\lang{de}{Lösung} \lang{en}{Solution}}
    \begin{incremental}[\initialsteps{1}]
        \step \lang{de}{ Die Lösung ist:
        \begin{table}
        \head
        k  &  $S_{k-1}$   &  $Z_k$   &   $T_k$  & $A_k$ & $S_k$  \\
        \body
        \textbf{1}  &   36.000   &    3.600   &  12.000 &  15.600  & 24.000  \\
        \textbf{2}  &   24.000   &    2.400   &  12.000 &  14.400  & 12.000  \\
        \textbf{3}  &   12.000   &    1.200   &  12.000 &  12.300  & 0  \\
        \textbf{4}  &   0& & & &
        \end{table} }
        \lang{en}{ The solution is:
        \begin{table}
        \head
        k  &  $S_{k-1}$   &  $Z_k$   &   $T_k$  & $A_k$ & $S_k$  \\
        \body
        \textbf{1}  &   36,000   &    3,600   &  12,000 &  15,600  & 24,000  \\
        \textbf{2}  &   24,000   &    2,400   &  12,000 &  14,400  & 12,000  \\
        \textbf{3}  &   12,000   &    1,200   &  12,000 &  12,300  & 0  \\
        \textbf{4}  &   0& & & &
        \end{table}}
        \step
        \lang{de}{\begin{align*}
        \text{Schuld  }\quad&S= 36.000 \text{ \euro}\\
        \text{Laufzeit}\quad&n= 3 \text{ Jahre}\\
        \text{Zinssatz}\quad&i= 10 \text{ \%}\\
        \text{Tilgung }\quad&T_k=T=12.000 \text{ \euro}\\
        \end{align*}
        Die Annuität ist die Summe aus Tilgung und Zinsen:$A_k=T_k +Z_k$.
        $S_{k-1}$ ist der Schuldenstand zu Beginn des k-ten Jahres.
        $S_k$ ist der Schuldenstand am Ende des k-ten Jahres.}
        \lang{en}{
        \begin{align*}
        \text{Debt  }\quad&S= 36,000 \text{ \euro}\\
        \text{Time}\quad&n= 3 \text{ years}\\
        \text{Interest rate}\quad&i= 10 \text{ \%}\\
        \text{Amortization}\quad&T_k=T=12,000 \text{ \euro}\\
        \end{align*}
        The annuity is the sum of amortization and interest: $A_k=T_k +Z_k$.
        $S_{k-1}$ is the debt remaining at the beginning of the k-th year.
        $S_k$ is the debt remaining at the end of the k-th year.}
     \end{incremental}
\end{tabs*}

\end{content}


  