\documentclass{mumie.element.exercise}
%$Id$
\begin{metainfo}
  \name{
    \lang{en}{Exercise 2: Fixed annuity}
    \lang{de}{Ü02: exakte Annuität}
    \lang{zh}{...}
    \lang{fr}{...}
  }
  \begin{description} 
 This work is licensed under the Creative Commons License Attribution 4.0 International (CC-BY 4.0)   
 https://creativecommons.org/licenses/by/4.0/legalcode 

    \lang{en}{...}
    \lang{de}{...}
    \lang{zh}{...}
    \lang{fr}{...}
  \end{description}
  \begin{components}
  \end{components}
  \begin{links}
  \end{links}
  \creategeneric
\end{metainfo}
\begin{content}

\begin{block}[annotation]
	Im Ticket-System: \href{https://team.mumie.net/issues/22681}{Ticket 22681}
\end{block}



\title{
    \lang{en}{Exercise 2: Fixed annuity}
    \lang{de}{Ü02: exakte Annuität}
    \lang{zh}{...}
    \lang{fr}{...}
}

\lang{de}{Ein Kredit von 36.000 \euro soll innerhalb von drei Jahren mittels jährlicher
Annuitätentilgung zurückgezahlt werden. Der Kreditzins beträgt $i=10$ \%.\\
Stellen Sie den Tilgungsplan auf.}
\lang{en}{A loan of 36,000 \euro will be paid off in three years
through yearly annuity payments. The interest rate on the loan
is $i=10$ \%. \\
Set up an amortization schedule.
}

\begin{tabs*}[\initialtab{0}\class{exercise}]
\tab{\lang{de}{benötigte Gleichungen}  \lang{en}{Equations}}
\lang{de}{Wir betrachten jetzt den Fall, dass die Annuität konstant ist. Wir sprechen dann auch von 
\notion{exakter Annuität}: $A_1=A_2=\ldots=A_n=A$.

Dann entspricht die Bestimmung der Annuität genau der Berechnung einer jährlichen nachschüssigen 
Rentenzahlung konstanter Höhe $A$ aus einem Rentenbarwert $S_0$ bei der Laufzeit $n$. Hierbei ist 
$S_0$ die Kreditsumme. Aus der Rentenrechnung ergibt sich daher:

\[A=S_0\frac{q^n(q-1)}{q^n-1}\]

Die Restschuld $S_k$ lässt sich dann als Differenz von zinsbedingt angewachsener 
Schuld $S_0q^{k-1}$ und den gezahlten Annuitäten $A\frac{q^k-1}{q-1}$ darstellen. Nach längerer
Rechnung ergibt sich:
\[S_k=S_0q^{k-1}-A\frac{q^k-1}{q-1}=S_0\frac{q^n-q^{k-1}}{q^n-1}\]

Mit $T_1=S_0\frac{q-1}{q^n-1}$ ergibt sich:
\[T_k=T_1q^{k-1}\text{ und }Z_k=A-T_1q^{k-1}\text{ sowie }S_k=S_0-T_1\frac{q^{k-1}-1}{q-1}.\]
}

\lang{en}{In the case considered here, the annuity payments are constant.
We refer to this as a
\notion{fixed annuity}: $A_1=A_2=\ldots=A_n=A$.

Determining the payments is equivalent to computing the constant annual payments $A$ of an annuity
in arrears with present value $S_0$ over a period of $n$ years.
Here, $S_0$ is the loan amount. Using the annuity formulas, we find

\[A=S_0\frac{q^n(q-1)}{q^n-1}\]

The remaining debt $S_k$ can be computed as the difference between the
debt with accrued interest $S_0q^{k-1}$ and the accumulated annuity
$A \frac{q^k-1}{q-1}$. After some long calculations, we find:


\[S_k=S_0q^{k-1}-A\frac{q^k-1}{q-1}=S_0\frac{q^n-q^{k-1}}{q^n-1}.\]

Using $T_1=S_0\frac{q-1}{q^n-1}$, we find:
\[T_k=T_1q^{k-1}\text{ and }Z_k=A-T_1q^{k-1}\text{ as well as }S_k=S_0-T_1\frac{q^{k-1}-1}{q-1}.\]
}


\tab{\lang{de}{Lösung} \lang{en}{Solution}}
    \begin{incremental}[\initialsteps{1}]
        \step \lang{de}{Die Lösung ist:
        \begin{table}
            \head
            k   & $S_{k-1}$  &  $Z_k$   &   $T_k$  & $A$    & $S_k$\\
            \body
            \textbf{1} & 36.000 &    3.600      &  10.876,13 &  14.476,13    & 25.123,87\\
            \textbf{2} & 25.123,87 &    2.512,39   &  11.963,75 &  14.476,13    & 13.160,12\\
            \textbf{3} & 13.160,12 &    1.316,01   &  13.160,12 &  14.476,13    & 0\\
        \end{table}}
        \lang{en}{The solution is:
        \begin{table}
            \head
            k   & $S_{k-1}$  &  $Z_k$   &   $T_k$  & $A$    & $S_k$\\
            \body
            \textbf{1} & 36,000 &    3,600      &  10,876.13 &  14,476.13    & 25,123.87\\
            \textbf{2} & 25,123.87 &    2,512.39   &  11,963.75 &  14,476.13    & 13,160.12\\
            \textbf{3} & 13,160.12 &    1,316.01   &  13,160.12 &  14,476.13    & 0\\
        \end{table}}
        \step
        \lang{de}{Mit $S_0=36.000$ \euro und $n=3$ Jahre ergibt sich die Annuität $A$ zu:
        \[A=S_0\frac{q^3(q-1)}{q^3-1}=14.476,13 \text{\euro}\]
        Mit $Z_1=3.600$ ergibt sich für die Tilgungen...
        \begin{align*}
        T_1&=A-Z_1&=10.876,13 \text{ \euro }\\
        T_2&=T_1\cdot q&=11.963,75\text{ \euro }\\
        T_3&=T_1\cdot q^2&=13.160,12\text{ \euro }
        \end{align*}
        ... und für die Restschulden:
        \begin{align*}
        S_0&=S_0-T_1\frac{q^0-1}{q-1}&=S_0-0&=36.000 \text{ \euro }\\
        S_1&=S_0-T_1\frac{q^1-1}{q-1}&=S_0-T_1&=25.123,87\text{ \euro }\\
        S_2&=S_0-T_1\frac{q^2-1}{q-1}&=S_0-2,1T_1&=13.160,12\text{ \euro }
        \end{align*}}
        \lang{en}{Since $S_0=36,000$ \euro and the time is $n=3$ years,
        the annuity $A$ is given by:
        \[A=S_0\frac{q^3(q-1)}{q^3-1}=14,476.13 \text{\euro}\]
        Since $Z_1=3,600$, the loan repayments are
        \begin{align*}
        T_1&=A-Z_1&=10,876.13 \text{ \euro }\\
        T_2&=T_1\cdot q&=11,963.75\text{ \euro }\\
        T_3&=T_1\cdot q^2&=13,160.12\text{ \euro }
        \end{align*}
        and the remaining debt is
        \begin{align*}
        S_0&=S_0-T_1\frac{q^0-1}{q-1}&=S_0-0&=36,000 \text{ \euro }\\
        S_1&=S_0-T_1\frac{q^1-1}{q-1}&=S_0-T_1&=25,123.87\text{ \euro }\\
        S_2&=S_0-T_1\frac{q^2-1}{q-1}&=S_0-2,1T_1&=13,160.12\text{ \euro }
        \end{align*}}
    \end{incremental}
\end{tabs*}

\end{content}

