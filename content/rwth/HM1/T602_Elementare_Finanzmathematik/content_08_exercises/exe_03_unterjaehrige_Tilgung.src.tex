\documentclass{mumie.element.exercise}
%$Id$
\begin{metainfo}
  \name{
    \lang{en}{Exercise 3: sub-annual amortization}
    \lang{de}{Ü03: unterjährige Tilgung}
    \lang{zh}{...}
    \lang{fr}{...}
  }
  \begin{description} 
 This work is licensed under the Creative Commons License Attribution 4.0 International (CC-BY 4.0)   
 https://creativecommons.org/licenses/by/4.0/legalcode 

    \lang{en}{...}
    \lang{de}{...}
    \lang{zh}{...}
    \lang{fr}{...}
  \end{description}
  \begin{components}
  \end{components}
  \begin{links}
  \end{links}
  \creategeneric
\end{metainfo}
\begin{content}
\title{
    \lang{en}{Exercise 3: sub-annual amortization}
    \lang{de}{Ü03: unterjährige Tilgung}
    \lang{zh}{...}
    \lang{fr}{...}
}
\begin{block}[annotation]
	Im Ticket-System: \href{https://team.mumie.net/issues/22687}{Ticket 22687}
\end{block}





\lang{de}{
Wir behandeln hier noch den Fall der \notion{sofortigen 
Tilgungsverrechnung}, d.h. innerhalb einer Zinsperiode werden mehrere Tilgungen 
geleistet, die sofort mit der noch zu zahlenden Restschuld verrechnet werden. 

Dabei ist bei einer Zinsperiode von einem Jahr:
\begin{table}
$T_{i,n}$&  die Tilgungsrate der Periode i im Jahr m, $i=1,\ldots,k$\\
$A_{i,n}=a$&  die konstante Annuität für die Periode i im Jahr m, $i=1,\ldots,k$\\
$Z_n$ & die nachschüssige Zinszahlung für Jahr m\\
$S_{i,n}$& der Schuldenstand zu Beginn der i-ten Periode im Jahr m\\
$k$& Anzahl der Perioden (Tilgungen) im Jahr m
\end{table}
}

\lang{en}{
We will consider the case of \notion{immediate settlement},
i.e. where multiple amortization payments occur within a single interest period
and are immediately deducted from the outstanding debt.

For an interest period of one year, define
\begin{table}
$T_{i,n}$&  the i-th amortization payment in the year n, $i=1,\ldots,k$\\
$A_{i,n}=a$&  the constant annuity for the i-th period of year n, $i=1,\ldots,k$\\
$Z_n$ & the interest payment (in arrears) for year n\\
$S_{i,n}$& outstanding debt at the beginning of the i-th period of year n\\
$k$& number of periods (payments) in year n
\end{table}
}

\begin{tabs*}[\initialtab{0}\class{exercise}]


\tab{\lang{de}{benötigte Gleichungen} \lang{en}{Equations}}

\lang{de}{
Dann ergibt sich im Jahr n ein Zins von:\\
\[Z_n=\sum_{i=1}^k Z_{i,n}=\sum_{i=1}^k \frac{p}{100k}S_{i,n}
=\frac{p}{100k}\sum_{i=1}^k[S_{1,n}-(i-1)a]\]
(Der Faktor $(i-1)$ ist der Tatsache geschuldet, dass es bei nachschüssigen Zahlungen auf die letzte Zahlung keine Zinsen gibt.)
Unter Verwendung der Gauss'schen Sumenformel kann $Z_n$ dann noch weiter umgeformt werden zu:\\
\[Z_n=\frac{p}{100k}\left(k\cdot S_{1,n}-\frac{a\cdot(k-1)\cdot k}{2}\right)\]
Der \notion{konforme Jahreszins} beläuft sich dann also auf:
\[Z_n=\frac{p}{100}S_{1,n}-\frac{a\cdot(k-1)\cdot p}{200}\]
also der Jahreszins auf die Kreditsumme vermindert um die einfache Verzinsung der Annuitäten $a$.
Wir können so eine \notion{konforme Ersatzannuität} berechnen mit
\[A=a\left(k+\frac{(k-1)\cdot p}{200}\right) . \]
Dann lässt sich die Restschuld folgendermaßen berechnen:
\[S_{1,n+1}=q\cdot S_{1,n}-A\]
}

\lang{en}{
In year n, the interest is:\\
\[Z_n=\sum_{i=1}^k Z_{i,n}=\sum_{i=1}^k \frac{p}{100k}S_{i,n}
=\frac{p}{100k}\sum_{i=1}^k[S_{1,n}-(i-1)a]\]
(The factor $(i-1)$ appears due to the fact that the final payment
in an annuity in arrears does not accrue interest.)
Using the Gauss summation formula, $Z_n$ can be further simplified to\\
\[Z_n=\frac{p}{100k}\left(k\cdot S_{1,n}-\frac{a\cdot(k-1)\cdot k}{2}\right)\]
The \notion{conformal annual interest rate} is therefore
\[Z_n=\frac{p}{100}S_{1,n}-\frac{a\cdot(k-1)\cdot p}{200},\]
i.e. the annual interest on the debt minus the simple interest $a$ on the annuities.
In this way, we can compute a conformal alternative annuity as
\[A=a\left(k+\frac{(k-1)\cdot p}{200}\right) . \]
The debt remaining can now be calculated as follows:
\[S_{1,n+1}=q\cdot S_{1,n}-A\]
}


\tab{\lang{de}{Aufgabe} \lang{en}{Problem}}

\lang{de}{Ein Kredit von 10.000 \euro soll bei jährlicher Verzinsung von 9\% in zwei Jahren
durch vierteljährlich-nachschüssig zu zahlende konstante Annuitäten zurückgezahlt werden.
Bei der Verzinsung sollen die unterjährig gezahlten Tilgungsbeträge berücksichtigt werden.
Geben Sie den Tilgungsplan für das erste Jahr an. }

\lang{en}{
A loan of 10,000 \euro with an annual interest rate of 9\% is to be repaid
over two years with constant quarterly annuity payments in arrears.
The sub-annual amortization payments are taken into account as the
interest is calculated. Write out an amortization schedule for the first year.
}

\tab{\lang{de}{Rechnung} \lang{en}{Calculations}}

\lang{de}{Mit $n=2$, $k=4$, $p=9$, $S_{1,1}=10.000$ und  $q=1,09$
berechnet sich $A$ zu
\[ A=a\left(4+\frac{9\cdot 3}{200}\right)=4,135\cdot a\]

Iterativ können wir nun $a$ berechnen, da nach zwei Jahren die Restschuld
0 sein muss:
\begin{eqnarray*}
S_{1,2}=&1,09\cdot 10.000-4,135\cdot a&=5215,326\\
S_{1,3}=&1,09\cdot S_{1,2}-4,135\cdot a&=0\\
=& 1,09 (1,09\cdot 10.000 -4,136\cdot a)-4,135\cdot a&=0\\
=& 1,09^2\cdot 10.000-4,135\cdot a(1+1,09)&=0\\
=&11.881&=4,135\cdot a\cdot 2,09\\
\iff&a&=1374,78
\end{eqnarray*}

Mit $A=4,135\cdot a$ ist:
\[A=5684,69 \text{ und } S_{1,2}=5215,32\]
Somit ist der Zins fürs erste Jahr:
\begin{align*}
Z_1&=&900-1374,78\cdot 0,135 \\
&=&714,41
\end{align*}

Da unterjährig einfach verzinst wird, entfällt auf ein Quartal der Zins $z=178,60$.
}

\lang{en}{With $n=2$, $k=4$, $p=9$, $S_{1,1}=10,000$ and  $q=1.09$,
we compute
\[ A=a\left(4+\frac{9\cdot 3}{200}\right)=4.135\cdot a.\]
We can now compute $a$ iteratively, since the debt remaining after two
years must be 0:
\begin{eqnarray*}
S_{1,2}=&1.09\cdot 10,000-4.135\cdot a&=5215.326\\
S_{1,3}=&1.09\cdot S_{1,2}-4.135\cdot a&=0\\
=& 1.09 (1.09\cdot 10,000 -4.136\cdot a)-4.135\cdot a&=0\\
=& 1.09^2\cdot 10,000-4.135\cdot a(1+1.09)&=0\\
=&11,881&=4.135\cdot a\cdot 2.09\\
\iff&a&=1374.78
\end{eqnarray*}

Since $A=4.135\cdot a$, we find
\[A=5684.69 \text{ and } S_{1,2}=5215.32\]
So the interest in the first year is:
\begin{align*}
Z_1&=&900-1374.78\cdot 0.135 \\
&=&714.41
\end{align*}

Since sub-annual interest is calculated as simple interest,
the interest accrued in each quarter is $z=178,60$.


}



\tab{\lang{de}{Lösung} \lang{en}{Solution}}

\lang{de}{
Damit sieht der Tilgungsplan folgendermaßen aus:

\begin{table}
\head
n&RS&a&z&t
\body
\textbf{1} & 10.000 & 1374,78 & 178,60 & 1196,17\\
\textbf{2} & 8.803,83 & 1374,78 & 178,60 & 1196,17\\
\textbf{3} & 7607,66 & 1374,78 & 178,60 & 1196,17\\
\textbf{4} & 6412,49 & 1374,78 & 178,60 & 1196,17\\
\textbf{5} & 5215,32 &        &       &
\end{table}

Der Check auf Richtigkeit ist hier, dass die Restschuld zu Beginn der 5. Periode
gleich dem berechneten Wert für $S_{1,2}=5215,326$ ist.
}

\lang{en}{
The amorization schedule is therefore:

\begin{table}
\head
n&RS&a&z&t
\body
\textbf{1} & 10,000 & 1374.78 & 178.60 & 1196.17\\
\textbf{2} & 8,803,83 & 1374.78 & 178.60 & 1196.17\\
\textbf{3} & 7607.66 & 1374.78 & 178.60 & 1196.17\\
\textbf{4} & 6412.49 & 1374.78 & 178.60 & 1196.17\\
\textbf{5} & 5215.32 &        &       &
\end{table}

As a sanity check, observe that the debt remaining at the beginning
of the 5th period equals the value we computed for $S_{1,2}=5215.326$.

}
\end{tabs*}
\end{content}

