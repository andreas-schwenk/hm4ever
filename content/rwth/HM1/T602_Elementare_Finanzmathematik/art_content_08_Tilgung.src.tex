
%$Id:  $
\documentclass{mumie.article}
%$Id$
\begin{metainfo}
  \name{
    \lang{de}{Tilgungsrechnung}
    \lang{en}{}
  }
  \begin{description} 
 This work is licensed under the Creative Commons License Attribution 4.0 International (CC-BY 4.0)   
 https://creativecommons.org/licenses/by/4.0/legalcode 

    \lang{de}{Beschreibung}
    \lang{en}{}
  \end{description}
  \begin{components}

  \end{components}
  \begin{links}
\link{generic_article}{content/rwth/HM1/T601_GrundlagenWiWi/g_art_content_03_Folgen_Reihen.meta.xml}{content_03_Folgen_Reihen}
\end{links}
  \creategeneric
\end{metainfo}
\begin{content}
\usepackage{mumie.ombplus}
\ombchapter{2}
\ombarticle{3}
\usepackage{mumie.genericvisualization}

\begin{visualizationwrapper}

\lang{de}{\title{Tilgungsrechnung}}
\lang{en}{\title{repayment calculation}}
\begin{block}[annotation]
	Im Ticket-System: \href{https://team.mumie.net/issues/22690}{Ticket 22690}

\end{block}

\begin{block}[annotation]
\end{block}

\begin{block}[info-box]
\tableofcontents
\end{block}



\section {\lang{de}{Grundbegriffe} \lang{en}{Basic terms}}

\lang{de}{Die Tilgungsrechnung befasst sich mit der Rückzahlung von Krediten und Darlehen. Die Art und Weise der Rückzahlung der Schuld wird vorzugsweise im Vorhinein
festgelegt. Neben der Tilgung des Geldbetrages fallen zusätzlich Zinsen auf das bereitgestellte Kapital an.} 
\lang{en}{The repayment calculation deals with the repayment of credits and loans. The method of repayment of the debt is preferably determined in advance.
 In addition to the repayment of the amount of money, interest also accrues on the capital provided.}\\
\begin{definition}
\lang{de}{Die \textbf{Tilgung} (oder auch \textbf{Tilgungsrate}) bezeichnet den Betrag, der zu einem regelmäßigen Zeitintervall abgetragen wird.}
\lang{en}{The \textbf{repayment} (or also \textbf{repayment rate}) denotes the amount deducted at a regular time interval.}
\\
\begin{itemize}
\item \lang{de}{$T_k$: Tilgungsrate in Periode $k$}
      \lang{en}{$T_k$: Repayment rate in period $k$}
\end{itemize}

\lang{de}{Die \textbf{Restschuld} bezeichnet die Schuld nach dem Ablauf eines bestimmten Zeitintervalls,
nachdem bereits ein Teil der Schuld getilgt wurde.}
\lang{en}{The residual debt denotes the debt after the expiry of a certain time interval, after part of the debt has already been repaid.}
\begin{itemize}
\item \lang{de}{$S_k$: Restschuld nach Ablauf von $k$ Zeitintervallen}
   \lang{en}{$S_k$: residual debt after expiry of $k$ time intervals }
\item \lang{de}{$S_0$: Anfangsschuld (entspricht dem Anfangskapital $K_0$)}
    \lang{en}{$S_0$: Initial debt (equal to initial capital $K_0$)}
\end{itemize}

\lang{de}{Eine \textbf{Annuität} bezeichnet die Zahlung des Schuldners in einem bestimmten Zeitintervall. 
Die {Annuität} ist hierbei die Summe der \textbf{Tilgungsrate} und der anfallenden \textbf{Zinsen}.} 
\lang{en}{A \textbf{annuity} denotes the payment of the debtor in a certain time interval.
The annuity is the sum of the \textbf{repayment rate} and the accrued \textbf{interest}.}
\begin{itemize}
\item \lang{de}{$A_k$: Annuität in Periode $k$}
      \lang{en}{$A_k$: Annuity in period $k$}
\end{itemize}
\lang{de}{In den meisten Fällen entsprechen die Perioden Jahren. Dem Namen nach handelt es sich bei Annuitäten 
auch um jährliche Zahlungen (lat. \emph{annus:} Jahr).}
\lang{en}{In most cases the periods correspond to years. 
As the name suggests, annuities are annual payments (lat. \emph{annus:} year).}
\end{definition}

\begin{example}
\lang{de}{Eine Bank leiht Ihnen $1.000$ € bei $10\%$ Zinsen. 
Sie vereinbaren eine jährliche Rückzahlung von $300$ €.}
\lang{en}{A bank lends you $1,000$ € at $10\%$ interest. 
You agree to repay $300$ € annually.}\\
\lang{de}{Wir erstellen einen Tilgungsplan für die ersten drei Jahre:}
\lang{en}{We create a repayment plan for the first three years:}\\
\lang{de}{\begin{table}
    \head
       Jahr & Restschuld zu Jahresbeginn & Zinsen & Tilgung & Jahreszahlung & Restschuld am Jahresende
       
    \body
       $1$ & $1.000$ € & $100$ € & $200$ € & $300$ € & $800$ €\\
       $2$ & $800$ € & $80$ € & $220$ € & $300$ € & $580$ €\\
       $3$ & $580$ € & $58$ € & $242$ € & $300$ € & $338$ €   
  \end{table}}
  \lang{en}{\begin{table}
    \head
      Year & Remaining debt at the beginning of the year & Interest & Repayment & Annual payment & Remaining debt at the end of the year
    \body
       $1$ & $1,000$ € & $100$ € & $200$ € & $300$ € & $800$ €\\
       $2$ & $800$ € & $80$ € & $220$ € & $300$ € & $580$ €\\
       $3$ & $580$ € & $58$ € & $242$ € & $300$ € & $338$ €   
  \end{table}}

\lang{de}{Wenn nach einem Jahr $300$€ gezahlt werden, reduziert sich die Restschuld auf $800$ €, denn nach einem Jahr sind $100$ € Zinsen fällig.
Die $300$ € Jahreszahlung setzen sich aus $100$ € Zinsen und $200$ € Tilgung zusammen. Der Schuldbetrag selbst reduziert sich lediglich um den Tilgungsbetrag.} 

\lang{en}{If $300$€ is paid after one year, the remaining debt is reduced to $800$ €, because $100$ € interest is due after one year.
The $300$ € annual payment is made up of $100$ € interest and $200$ € repayment. The debt amount itself is only reduced by the repayment amount.}
\end{example}

\begin{remark}
\lang{de}{Ganz allgemein ergibt sich bei einem Zinssatz $i$ und einer Gesamtlaufzeit $n$ folgender \textbf{allgemeiner Tilgungsplan}:}
\lang{en}{In general, for an interest rate $i$ and a total time $n$, the following \textbf{general repayment schedule} results:}
\\
\lang{de}{\begin{table}
    \head
       Jahr & Restschuld zu Jahresbeginn & Zinsen & Tilgung & Jahreszahlung
    \body
       $1$ & $S_0$  & $Z_1 = S_0\cdot i$  & $T_1$  & $A_1 = Z_1 + T_1$  \\
       $2$ & $S_1 =S_0 - T_1$  & $Z_2 = S_1\cdot i$  & $T_2$  & $A_2 = Z_2 + T_2$  \\
       ... & ... & ... & ... & ... \\
       $k$ & $S_{k-1} =S_0 - T_1 - ... - T_{k-1}$  & $Z_k = S_k-1\cdot i$  & $T_k$  & $A_k = Z_k + T_k$ \\
       ... & ... & ... & ... & ...\\
       $n$ & $S_{n-1} =S_0 - T_1 - ... - T_{n-1}$  & $Z_n = S_n-1\cdot i$  & $T_n$  & $A_n = Z_n + T_n$ \\
       $n+1$ & $0$ &  &  &  
       
  \end{table}}
  \lang{en}{\begin{table}
    \head
       Year & Remaining debt at the beginning of the year & Interest & Redemption & Annual paymen
    \body
       $1$ & $S_0$  & $Z_1 = S_0\cdot i$  & $T_1$  & $A_1 = Z_1 + T_1$  \\
       $2$ & $S_1 =S_0 - T_1$  & $Z_2 = S_1\cdot i$  & $T_2$  & $A_2 = Z_2 + T_2$  \\
       ... & ... & ... & ... & ... \\
       $k$ & $S_{k-1} =S_0 - T_1 - ... - T_{k-1}$  & $Z_k = S_k-1\cdot i$  & $T_k$  & $A_k = Z_k + T_k$ \\
       ... & ... & ... & ... & ...\\
       $n$ & $S_{n-1} =S_0 - T_1 - ... - T_{n-1}$  & $Z_n = S_n-1\cdot i$  & $T_n$  & $A_n = Z_n + T_n$ \\
       $n+1$ & $0$ &  &  &  
       
  \end{table}}
\end{remark}

\lang{de}{In den nachfolgenden Abschnitten werden zwei Tilgungsmodelle, die \textbf{Ratentilgung} und die \textbf{Annuitätentilgung}, behandelt.}
\lang{en}{In the following sections, two repayment models, the \textbf{instalment repayment} and the \textbf{annuity repayment}, will be discussed.}
\section{\lang{de}{Ratentilgung} \lang{en}{Repayment by instalments}}
\lang{de}{Von \textbf{Ratentilgung} wird gesprochen, wenn die \textbf{Tilgungsraten} über den gesamten Zeitraum der Rückzahlung konstant sind.
Die gezahlte Rate pro Zeitabschnitt, die \textbf{Annuität}, ist hingegen durch die Addition der \textbf{Tilgungsrate} zu den anfallenden \textbf{Zinsen}
immer verschieden.}
\lang{en}{The term \textbf{amortisation} is used when the \textbf{amortisation rates} are constant over the entire repayment period.
The instalment paid per period,on the other hand, the \textbf{annuity}, is always different due to the addition of the \textbf{amortisation instalment} to the accruing \textbf{interest}.}
\\

\begin{rule}\lang{de}{[Ratentilgung]} \lang{en}{Repayment by instalments}
\lang{de}{Die konstante \textbf{Tilgungsrate T} berechnet sich durch die Division der Anfangsschuld $S_0$ durch die Laufzeit $n$, also}
\lang{en}{The constant \textbf{repayment rate T} is calculated by dividing the initial debt $S_0$ by the maturity $n$, i.e.}
\[ 
T = \frac{S_0}{n} .
\] 

\lang{de}{Die \textbf{Restschuld} nach $k$ Zeitintervallen berechnet sich durch}
\lang{en}{The \textbf{residual debt} after $k$ time intervals is calculated by}
\[ 
S_k = S_0 - k\cdot T = S_0 \left(1-\frac{k}{n}\right).
\] 

\lang{de}{Die zu leistende \textbf{Zinszahlung} pro Zeitintervall $k$ beläuft sich auf}
\lang{en}{The \textbf{interest} payment to be made per time interval $k$ amounts to}
\[ 
Z_k = S_0 \left(1-\frac{k-1}{n}\right)\cdot i    \ \ \ \ \lang{de}{\text{mit}}
                                                          \lang{en}{\text{with}}
                                                          \ \  k=1, 2, 3, \ldots, n-1, n .
\] 

\lang{de}{Die zu zahlende Rate (Annuität) pro Zeitintervall $k$ berechnet sich als Summe von \textbf{Zins-} und \textbf{Tilgungsbetrag}}
\lang{en}{The instalment (annuity) to be paid per time interval $k$ is calculated as the sum of \textbf{interest} and \textbf{amortisation}}
\[ 
A_k = T + Z_k = S_0 \left(\frac{1}{n} +i\left(1-\frac{k-1}{n}\right) \right).
\] 
\end{rule}

\begin{proof*}\lang{de}{[Herleitung]} \lang{en}{[Derivation]}
\lang{de}{Da die Tilgungsrate konstant ist, muss lediglich $S_0$ durch die Laufzeit $n$ dividiert werden,
um die Tilgungsrate $T=\frac{S_0}{n}$ zu erhalten. Die Restschuld $S_k$ ist einfach die Anfangsschuld $S_0$ abzüglich 
der $k$ bereits gezahlten Tilgungsraten, wodurch sich die angegebene Formel ergibt.}
\lang{en}{Since the repayment rate is constant, only $S_0$ has to be divided by the term $n$,
to get the repayment rate $T=\frac{S_0}{n}$. The residual debt $S_k$ is simply the initial debt $S_0$ minus 
the $k$ repayments already paid, giving the formula given.}
\lang{de}{Für die Zinszahlung $Z_k$ müssen wir die Restschuld nach $k-1$ Zeitintervallen betrachten und mit dem Zinssatz $i$
multiplizieren, wodurch wir}
\lang{en}{For the interest payment $Z_k$, we need to look at the residual debt after $k-1$ time intervals and multiply by the interest rate $i$,
which gives us}
\[
Z_k = S_{k-1} \cdot i = S_0 \left( 1 - \frac{k-1}{n}\right) \cdot i
\]
\lang{de}{erhalten. Die Annuität ist nun lediglich die Summe aus den Formeln für $T$ und $Z_k$.}
\lang{en}{The annuity is now simply the sum of the formulas for $T$ and $Z_k$. }
\end{proof*}

\begin{example}
\lang{de}{Bei einem Kredit von $3.000$ € zu $10\%$ Zinsen ergibt sich bei einer jährlichen Tilgung von 600€ folgender Tilgungsplan:}
\lang{en}{For a loan of $3,000$ at $10\%$ interest, an annual repayment of 600€ results in the following repayment schedule:}
\\
\lang{de}{\begin{table}
    \head
       Jahr & Restschuld zu Jahresbeginn & Zinsen & Tilgung & Jahreszahlung 
    \body
       $1$ & $3.000$ € & $300$ € & $600$ € & $900$ € \\
       $2$ & $2.400$ € & $240$ € & $600$ € & $840$ € \\
       $3$ & $1.800$ € & $180$ € & $600$ € & $780$ €\\
       $4$ & $1.200$ € & $120$ € & $600$ € & $720$ € \\
       $5$ & $600$ € & $60$ € & $600$ € & $660$ € \\
       $6$ & $0$ € &  &  & 
  \end{table}}
  \lang{en}{\begin{table}
    \head
       Year & Remaining debt at the beginning of the year & Interest & Repayment & Annual payment 
    \body
       $1$ & $3,000$ € & $300$ € & $600$ € & $900$ € \\
       $2$ & $2,400$ € & $240$ € & $600$ € & $840$ € \\
       $3$ & $1,800$ € & $180$ € & $600$ € & $780$ €\\
       $4$ & $1,200$ € & $120$ € & $600$ € & $720$ € \\
       $5$ & $600$ € & $60$ € & $600$ € & $660$ € \\
       $6$ & $0$ € &  &  & 
  \end{table}}
\lang{de}{Nach fünf Jahren ist der Kredit vollständig zurückgezahlt. \\
Die Laufzeit lässt sich auch mit
\[ 
n = \frac{S_0}{T} = \frac{3.000}{600} = 5
\] berechnen.}
\lang{en}{After five years the loan is fully repaid. \\
The time can also be calculated with
\[ 
n = \frac{S_0}{T} = \frac{3.000}{600} = 5.
\]}
\end{example}

\begin{quickcheck}
		\field{rational}
		\type{input.function}
		\begin{variables}
			
		 
            \function{T}{5000}
            \function{T1}{5000}
            
            \function{S0}{50000}
            \function{S1}{45000}
            \function{S2}{40000}
            
            \function{Z1}{2500}
            \function{Z2}{2250}
            
            
            \function{A1}{7500}
            \function{A2}{7250}
           
  		\end{variables}
		
		\lang{de}{\text{%\notion{Kurztest:}\\
        Ein Darlehen über $50.000 $ € soll bei einem Zinssatz von $5$\%, bei gleichbleibender Tilgungsrate, innerhalb von 10 Jahren getilgt werden. \\
        Erstellen Sie den Tilgungsplan für die ersten zwei Jahre.\\
        
        Die Tilgungsrate beträgt: \ansref €
        
  \begin{table}
    \head
       Jahr & Restschuld zu Jahresbeginn & Zinsen & Tilgung & Jahreszahlung 
    \body
       $1$ & \ansref € & \ansref € & \ansref € & \ansref € \\
       $2$ & \ansref € & \ansref € & \ansref € & \ansref € \\
       $3$ & \ansref € &  &  &
      
  \end{table}
        
       }}  
       		\lang{en}{\text{%\notion{Kurztest:}\\
       A loan of $50,000$ € is to be repaid within 10 years at an interest rate of $5$\%, with a constant repayment rate. \\
        Erstellen Sie den Tilgungsplan für die ersten zwei Jahre.\\
        
     The repayment rate is: \ansref €     
  \begin{table}
    \head
       Year & Remaining debt at the beginning of the year & Interest & Repayment & Annual payment 
    \body
       $1$ & \ansref € & \ansref € & \ansref € & \ansref € \\
       $2$ & \ansref € & \ansref € & \ansref € & \ansref € \\
       $3$ & \ansref € &  &  &
      
  \end{table}
        
       }}  
		
		\begin{answer}
			\solution{T}
		\end{answer}
        \begin{answer}
			\solution{S0}
		\end{answer}
         \begin{answer}
			\solution{Z1}
		\end{answer}
        \begin{answer}
			\solution{T1}
		\end{answer}
        \begin{answer}
			\solution{A1}
		\end{answer}
        \begin{answer}
			\solution{S1}
		\end{answer}
         \begin{answer}
			\solution{Z2}
		\end{answer}
        \begin{answer}
			\solution{T}
		\end{answer}
        \begin{answer}
			\solution{A2}
		\end{answer}
        \begin{answer}
			\solution{S2}
		\end{answer}
         
        
        \lang{de}{\explanation{
        Die Tilgungsrate wird mit $T = \frac{S_0}{n}$ berechnet. \\
        $T = \frac{50.000 \text{€}}{10} = 5.000$ €.\\
        
        Für den Zinsbetrag wird die Restschuld zu Jahresbeginn mit dem Faktor $0,05$ multipliziert ($5\%$ Zinsen).\\
        
        Die Summe von Zinsbetrag und Tilgungsrate ergibt die Jahreszahlung. \\
        
        Zuletzt wird die Tilgungsrate von der Restschuld zu Jahresbeginn abgezogen. So erhalten wir die Restschuld des nachfolgenden Jahres.}}
        
		   \lang{en}{\explanation{
        The repayment rate is calculated as $T = \frac{S_0}{n}$. \\
        $T = \frac{50,000 \text{€}}{10} = 5,000$ €.\\
        
        For the interest amount, the remaining debt at the beginning of the year is multiplied by a factor of $0.05$ ($5\%$ interest).\\
        
        The sum of the interest amount and the repayment rate is the annual payment. \\
        
        Finally, the repayment instalment is subtracted from the remaining debt at the beginning of the year. This gives us the remaining debt for the following year.}}
\end{quickcheck}


\section {\lang{de}{Annuitätentilgung} \lang{en}{Annuity repayment}}

\lang{de}{Von \textbf{Annuitätentilgung} wird gesprochen, wenn die \textbf{Annuität}, also die Zahlung pro Zeitintervall, 
über den gesamten Zeitraum der Rückzahlung konstant ist. Hierbei ist die \textbf{Tilgungsrate} pro Zeitintervall verschieden.}
\lang{en}{The term \textbf{annuity repayment} is used when the \textbf{annuity}, i.e. the payment per time interval, 
is constant over the entire period of repayment. Here the \textbf{amortisation rate} per time interval is different.}\\
\lang{de}{Die Berechnung der konstanten \textbf{Annuität} ist hier ein bisschen aufwändiger.}
\lang{en}{The calculation of the constant \textbf{annuality} is a bit more elaborate here.}
\begin{rule}\lang{de}{[Annuitätentilgung]}
\lang{de}{Wir bezeichnen mit $n$ die Laufzeit. 
Die konstante \textbf{Annuität A} berechnet sich durch Division der Anfangsschuld $S_0$ durch einen Faktor $a_n$:}
\lang{en}{We denote by $n$ the repayment period. 
The constant \textbf{annuity A} is calculated by dividing the initial debt $S_0$ by a factor $a_n$: }\\
\[
A = \frac{S_0}{a_n},
\]

\lang{de}{wobei sich der Faktor $a_n$ durch} 
\lang{en}{where the factor $a_n$ is given by} 
\[
a_n = \frac{q^n-1}{q^n\cdot(q-1)} 
\]
\lang{de}{mit dem Zinsfaktor
\[
q = 1 + i
\]}
\lang{en}{with the interest factor
\[
q = 1 + i.
\]}

\lang{de}{berechnen lässt.}\\

\lang{de}{Wie zuvor setzt sich die Annuität aus \textbf{Tilgungszahlung} und \textbf{Zinszahlung} nach $k$ Zeitintervallen
\[ 
A = T_k + Z_k 
\] zusammen.} 
\lang{en}{As before, the annuity is composed of \textbf{repayment} and the \textbf{interest payment} after $k$ time intervals
\[ 
A = T_k + Z_k.
\]} 


\lang{de}{Die \textbf{Tilgungszahlung} zum Zeitpunkt $k$ berechnet sich durch}
\lang{en}{The \textbf{repayment} at time $k$ is calculated by}
\[ 
T_k = T_1 \cdot q^{k-1} ; \quad T_1 = A - i\cdot S_0 .
\]

\lang{de}{Die \textbf{Restschuld} zu einem bestimmten Zeitpunkt $k$ lässt sich mit}
\lang{en}{The \textbf{residual debt} at a given time $k$ can be calculated with}
\[ 
S_k = S_0 - (T_1 + T_2 + ... + T_k) = A\cdot a_{n-k} 
\]
\lang{de}{berechnen.} 

\end{rule}
\begin{proof*}\lang{de}{[Herleitung]} \lang{en}{[Derivation]}
\begin{incremental}[\initialsteps{1}]
\step
\lang{de}{Die Anfangsschuld $S_0$ wird durch die $n$-malige Zahlung der konstanten Annuität $A$ beglichen. Wenn wir 
annehmen, dass auch die gezahlten Annuitäten zum Zinssatz $i$ verzinst werden, haben wir am Ende des 
Tilgungszeitraums den Betrag}
\lang{en}{The initial debt $S_0$ is settled by paying the constant annuity $A$ $n$ times. If we 
assume that the annuities paid also bear interest at the rate $i$, then at the end of the 
redemption period we have the amount}
\[
S_0 \cdot q^n = A + A \cdot q + A \cdot q^2 + \ldots + A \cdot q^{n-1}
\]
\lang{de}{bzw.} \lang{en}{or} 
\[
S_0 \cdot q^n = A \cdot \left( 1+ q + q^2 + \ldots + q^{n-1} \right) = A \cdot \frac{q^n -1}{q -1},
\]
\lang{de}{wobei wir für die letzte Gleichheit die Formel für die \ref[content_03_Folgen_Reihen][geometrische Reihe]{ex:konvergenz-geo-reihe} genutzt haben. Die Annahme, dass
auch die gezahlten Annuitäten verzinst werden, haben wir hier nur gebraucht, um eine Gleichung zu erhalten, 
durch die wir die Unbekannte $A$ berechnen können. Durch Umformen
der Gleichung erhalten wir} 
\lang{en}{where for the last equality we have used the formula for the \ref[content_03_Folgen_Reihen][geometric series]{ex:convergent-geo-series}. The assumption that
the annuities paid are also interest-bearing is only needed here to obtain an equation
for calculating the unknown $A$. By transforming the equation we obtain} 
\[
A= S_0 \cdot \frac{q^n (q-1)}{q^n-1} = \frac{S_0}{a_n}.
\]

\step
\lang{de}{Durch Multiplikation der Gleichung mit $a_n$ erhalten wir $S_0 = A \cdot a_n$. Die gleiche Rechnung hätten wir 
auch mit der Restschuld $S_k$ statt der Anfangsschuld $S_0$ durchführen können, wobei wir dann die Laufzeit $n$
durch die Restlaufzeit $n-k$ ersetzen müssten. Auf diese Weise erhalten wir die Formel für 
die Restschuld }
\lang{en}{By multiplying the equation by $a_n$ we get $S_0 = A \cdot a_n$. We could have done the same calculation 
with the residual debt $S_k$ instead of the initial debt $S_0$, in which case we would divide the maturity $n$
by the residual term $n-k$. In this way we obtain the formula for 
the residual debt }
\[
S_k = A \cdot a_{n-k}.
\]

\step
\lang{de}{Um die Formel für die Tilgungszahlung herzuleiten, gehen wir so ähnlich wie am Anfang vor und nutzen aus, dass wir die Annuität $A$ als 
Summe $A=T_1 + Z_1$ schreiben können:}
\lang{en}{To derive the formula for the redemption payment, we proceed similarly to the beginning and exploit the fact that we can write the annuity $A$ as 
Sum $A=T_1 + Z_1$ can be written:}
\begin{align*}
S_0 \cdot q^n &= A + A \cdot q + A \cdot q^2 + \ldots + A \cdot q^{n-1} \\
          &= (T_1 + Z_1) + (T_1+Z_1) \cdot q + (T_1 + Z_1) \cdot q^2 + \ldots + (T_1 + Z_1) \cdot q^{n-1} \\
          &= T_1 + T_1 \cdot q + T_1 \cdot q^2 + \ldots + T_1 \cdot q^{n-1}  + Z_1 + Z_1 \cdot q + Z_1 \cdot q^2 + \ldots + Z_1 \cdot q^{n-1}.
\end{align*}
\lang{de}{Wenn wir nun die anfallenden Zinsen und Zinseszinsen ignorieren, können wir sehen, wie die Anfangsschuld über die Tilgungszahlungen abgetragen wird:}
\lang{en}{If we now ignore the accruing interest and compound interest, we can see how the initial debt is paid off via the redemption payments:}
\[
S_0 = T_1 + T_1 \cdot q + T_1 \cdot q^2 + \ldots + T_1 \cdot q^{n-1}.
\]
\lang{de}{Die erste Tilgungszahlung $T_1$ ist Annuität $A$ abzüglich der Zinsen $Z_1 = i \cdot S_0$, also $T_1 = A - i \cdot S_0$. Die nächsten Tilgungszahlungen sind dann $T_2 = T_1 \cdot q$, 
$T_3 = T_1 \cdot q^2$ usw.}
\lang{en}{The first redemption payment $T_1$ is annuity $A$ minus interest $Z_1 = i \cdot S_0$, so $T_1 = A - i \cdot S_0$. The next redemption payments are then $T_2 = T_1 \cdot q$, 
$T_3 = T_1 \cdot q^2$ etc.}
\end{incremental}
\end{proof*}

\begin{example}
\lang{de}{Ein Kredit von $3.000$ € zu $10\%$ Zinsen soll, bei gleichbleibend großen Zahlungen, in fünf Jahren zurückgezahlt werden.}
\lang{en}{A loan of $3,000$ € at $10\%$ interest is to be repaid in five years, in payments of a constant size.}\\
\lang{de}{Zuerst wird die Annuität berechnet: $A = \frac{S_0}{a_5}$.}
\lang{en}{First the annuity is calculated: $A = \frac{S_0}{a_5}$.}\\

\lang{de}{Es gilt 
$a_5 = \frac{q^5-1}{q^5\cdot (q-1)} = \frac{1,1^5-1}{1,1^5\cdot (1,1-1)} = 3,79079$\\
und somit ergibt sich $A = \frac{3000}{3,79079} = 791,39 $ €.}
\lang{en}{Then
$a_5 = \frac{q^5-1}{q^5\cdot (q-1)} = \frac{1.1^5-1}{1.1^5\cdot (1.1-1)} = 3.79079$\\
and thus $A = \frac{3000}{3.79079} = 791.39 $ €.}
\lang{de}{\begin{table}
    \head
       Jahr & Restschuld zu Jahresbeginn & Zinsen & Tilgung & Jahreszahlung 
    \body
       $1$ & $3.000$ € & $300$ € & $491,39$ € & $791,39$ € \\
       $2$ & $2.508,61$ € & $250,86$ € & $540,53$ € & $791,39$ € \\
       $3$ & $1.968,08$ € & $196,81$ € & $594,58$ € & $791,39$ €\\
       $4$ & $1.373,50$ € & $137,35$ € & $654,04$ € & $791,39$ € \\
       $5$ & $719,46$ € & $71,95$ € & $719,44$ € & $791,39$ € \\
       $6$ & $0,02$ € &  &  & 
  \end{table}}
  
\lang{en}{\begin{table}
    \head
      Year & Remaining debt at the beginning of the year & Interest & Repayment & Annual payment
    \body
       $1$ & $3,000$ € & $300$ € & $491.39$ € & $791.39$ € \\
       $2$ & $2,508.61$ € & $250.86$ € & $540.53$ € & $791.39$ € \\
       $3$ & $1,968.08$ € & $196.81$ € & $594.58$ € & $791.39$ €\\
       $4$ & $1,373.50$ € & $137.35$ € & $654.04$ € & $791.39$ € \\
       $5$ & $719.46$ € & $71.95$ € & $719.44$ € & $791.39$ € \\
       $6$ & $0.02$ € &  &  & 
  \end{table}}
\lang{de}{Bei einer gleichbleibenden Annuität von $791,39 $ € bleibt nach fünf Jahren eine noch zu begleichende Restschuld von $0,02 $ €.}
\lang{en}{With a constant annuity of $791.39 $ €, a remaining debt of $0.02 $ € remains to be paid after five years.}\\
\end{example}

\begin{quickcheck}
		\field{rational}
		\type{input.function}
		\begin{variables}
			
		 
            \function{n}{10}
            
           
  		\end{variables}
		
		\lang{de}{\text{%\notion{Kurztest:}\\
        Ein Kredit über $50.000$ € soll bei einem Zinssatz von $5$\%, bei gleichbleibender Annuität von $6.500$ € pro Jahr, zurückgezahlt werden. \\
        Nach wie vielen Jahren ist der Kredit abbezahlt? \\
        (Runden Sie auf ganze Jahre!)
        
        Der Kredit ist nach \ansref Jahren abbezahlt.}}   
		\lang{en}{\text{%\notion{Kurztest:}\\
        A loan of $50,000$ € is to be repaid at an interest rate of $5$\%, with a constant annuity of $6,500$ € per year. \\
        After how many years will the loan be paid off? \\
        (Round to the nearest year!)
        
        The loan will be paid off after \ansref years.}} 
		\begin{answer}
			\solution{n}
		\end{answer}
                 
        
        \lang{de}{\explanation{
        Die Annuität von $6.500$ € berechnet sich mit $A = \frac{S_0}{a_n}$.\\
        \begin{align*}
        & \quad 6.500 &\ =\  50.000 \cdot 1,05^n \cdot \frac{1,05-1}{1,05^n-1} &\text{|} :50.000\\
        &\Leftrightarrow \quad \frac{13}{100} &\ =\  1,05^n \cdot \frac{1,05-1}{1,05^n-1} &\text{|} :(1,05-1)\\
        &\Leftrightarrow \quad \frac{13}{5} &\ =\  \frac{1,05^n}{1,05^n-1} &\text{|} \text{ umformen}\\
        &\Leftrightarrow \quad \frac{13}{5} &\ =\  \frac{1}{1-\frac{1}{1,05^n}} & \text{|} \text{ Kehrwerte bilden}\\
        &\Leftrightarrow \quad \frac{5}{13} &\ =\  1-\frac{1}{1,05^n} &\text{|} +\frac{1}{1,05^n}; -\frac{5}{13}\\
        &\Leftrightarrow \quad \frac{1}{1,05^n} &\ =\  1-\frac{5}{13} = \frac{8}{13} &\text{|} \text{ Kehrwerte bilden}\\
        &\Leftrightarrow \quad 1,05^n &\ =\  \frac{8}{13} &\text{|} \text{ Logarithmus}\\
        &\Leftrightarrow \quad n &\ =\  \log_{1,05}(\frac{8}{13}) &\\
        &\Leftrightarrow \quad n &\ =\  9,95 & \\
        &\Rightarrow 10 \text{ Jahre} & &
        
        \end{align*}
       }}
                \lang{en}{\explanation{
      The annuity of $6,500$ € is calculated with  $A = \frac{S_0}{a_n}$.\\
        \begin{align*}
        & \quad 6,500 &\ =\  50,000 \cdot 1.05^n \cdot \frac{1.05-1}{1.05^n-1} &\text{|} :50,000\\
        &\Leftrightarrow \quad \frac{13}{100} &\ =\  1.05^n \cdot \frac{1.05-1}{1.05^n-1} &\text{|} :(1.05-1)\\
        &\Leftrightarrow \quad \frac{13}{5} &\ =\  \frac{1.05^n}{1.05^n-1} &\text{|} \text{ transform }\\
        &\Leftrightarrow \quad \frac{13}{5} &\ =\  \frac{1}{1-\frac{1}{1.05^n}} & \text{|} \text{ Calculate reciprocal}\\
        &\Leftrightarrow \quad \frac{5}{13} &\ =\  1-\frac{1}{1.05^n} &\text{|} +\frac{1}{1.05^n}; -\frac{5}{13}\\
        &\Leftrightarrow \quad \frac{1}{1.05^n} &\ =\  1-\frac{5}{13} = \frac{8}{13} &\text{|} \text{ Calculate reciprocal}\\
        &\Leftrightarrow \quad 1.05^n &\ =\  \frac{8}{13} &\text{|} \text{ Logarithm}\\
        &\Leftrightarrow \quad n &\ =\  \log_{1.05}(\frac{8}{13}) &\\
        &\Leftrightarrow \quad n &\ =\  9.95 & \\
        &\Rightarrow 10 \text{ Jahre} & &
        
        \end{align*}
       }}
        
		
\end{quickcheck}

\end{visualizationwrapper}

\end{content}

