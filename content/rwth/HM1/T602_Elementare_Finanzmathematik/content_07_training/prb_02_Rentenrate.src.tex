\documentclass{mumie.problem.gwtmathlet}
%$Id$
\begin{metainfo}
  \name{
    \lang{en}{...}
    \lang{de}{A02: Rentenrate}
    \lang{zh}{...}
    \lang{fr}{...}
  }
  \begin{description} 
 This work is licensed under the Creative Commons License Attribution 4.0 International (CC-BY 4.0)   
 https://creativecommons.org/licenses/by/4.0/legalcode 

    \lang{en}{...}
    \lang{de}{...}
    \lang{zh}{...}
    \lang{fr}{...}
  \end{description}
  \corrector{system/problem/GenericCorrector.meta.xml}
  \begin{components}
    \component{js_lib}{system/problem/GenericMathlet.meta.xml}{gwtmathlet}
  \end{components}
  \begin{links}
  \end{links}
  \creategeneric
\end{metainfo}
\begin{content}
\lang{de}{\title{A02: Rentenrate}}
\lang{en}{\title{A02: annuity payments}}
\begin{block}[annotation]
	Im Ticket-System: \href{https://team.mumie.net/issues/22686}{Ticket 22686}
\end{block}


\usepackage{mumie.genericproblem}

     \begin{problem}
        \begin{variables}
            \drawFromSet{r0}{15000,20000,25000,30000}
            \randint{n}{4}{6}
            \randint{p}{2}{4}
            \function[calculate,2]{q}{1+p/100}
            \function[calculate,2]{rn}{r0*(q^(n+1)-q^n)/(q^n-1)}
        \end{variables}
          \begin{question}
               \type{input.function}
               \lang{de}{\text{Ein Student erhält für sein $\var{n}$-jähriges
               Studium $\var{r0}$ \euro (Rentenbarwert) geschenkt. Welchen 
               Betrag (Rentenrate) kann er nachschüssig bei einer jährlichen Verzinsung von
               $\var{p}$ \% pro Jahr abheben, damit das Geld eben diese $\var{n}$ 
               Jahre reicht?\\
               Rentenrate r=\ansref\euro}}
               \lang{en}{\text{A student receives a gift of $\var{r0}$ \euro 
               (annuity cash value) to cover $\var{n}$ years of study.
               What amount (annuity payment) can he withdraw in arrears at an annual interest rate of
               $\var{p}$ \% per year in order for the money to last just these $\var{n}$ 
               years?\\
               Annuity payment r=\ansref\euro}}
               \begin{answer}
                 \solution{rn}
               \end{answer}
               \lang{de}{\explanation[edited]{Für den Zinsfaktor q gilt: $q=1+\frac{p}{100}$
                und somit für die Rentenrate: $r=R_0\cdot\frac{q^{n+1}-q^n}{q^n-1}.$
                %=\var{r0}\frac{\var{q}^{\var{n}+1}-\var{q}^{\var{n}}}{\var{q}^{\var{n}}-1}
                %=\var{rn}$ \euro.
                }}
                \lang{en}{\explanation[edited]{The interest factor q satisfies $q=1+\frac{p}{100}$
                and the annuity rate satisfies $r=R_0\cdot\frac{q^{n+1}-q^n}{q^n-1}.$
                %=\var{r0}\frac{\var{q}^{\var{n}+1}-\var{q}^{\var{n}}}{\var{q}^{\var{n}}-1}
                %=\var{rn}$ \euro.
                }}
          \end{question}     
     \end{problem}

\embedmathlet{gwtmathlet}

\end{content}
