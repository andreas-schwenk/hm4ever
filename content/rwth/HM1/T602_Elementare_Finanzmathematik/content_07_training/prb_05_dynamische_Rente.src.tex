\documentclass{mumie.problem.gwtmathlet}
%$Id$
\begin{metainfo}
  \name{
    \lang{en}{...}
    \lang{de}{A05: dynamische Rente}
    \lang{zh}{...}
    \lang{fr}{...}
  }
  \begin{description} 
 This work is licensed under the Creative Commons License Attribution 4.0 International (CC-BY 4.0)   
 https://creativecommons.org/licenses/by/4.0/legalcode 

    \lang{en}{...}
    \lang{de}{...}
    \lang{zh}{...}
    \lang{fr}{...}
  \end{description}
  \corrector{system/problem/GenericCorrector.meta.xml}
  \begin{components}
    \component{js_lib}{system/problem/GenericMathlet.meta.xml}{gwtmathlet}
  \end{components}
  \begin{links}
  \end{links}
  \creategeneric
\end{metainfo}
\begin{content}
\lang{de}{\title{A05: dynamische Rente}}
\lang{en}{\title{A05: variable annuity}}
\begin{block}[annotation]
	Im Ticket-System: \href{https://team.mumie.net/issues/22673}{Ticket 22673}
\end{block}


\usepackage{mumie.genericproblem}

     \begin{problem}
     \begin{variables}
        \randint{m}{3}{6}
        \randint{w}{1}{2}
        \function[calculate]{r0}{100000*m}
        \function[calculate,2]{p}{0.5*m}
        \function[calculate,2]{pw}{0.5*w}
        \function[calculate,2]{rew}{(p-pw)*r0/100}
        \function{q}{1+p/100}
        \function[calculate,3]{qw}{1+pw/100}
        \function[calculate,2]{r20}{r0*(qw-q)/((qw/q)^(20)-1)}
     \end{variables}
          \begin{question}
          \type{input.number}
         \lang{de}{\text{Herr L. hat den Rest seines Lottogewinnes von $\var{r0}$
          \euro am Jahresbeginn zu einem festen Zinssatz von $p=\var{p}$ \% angelegt. Die Rente soll 
          jährlich nachschüssig um 
          $\var{pw}$ \% steigen. Es handelt sich also um eine dynamische Rente mit dem Faktor $w=\var{qw}$. 
          Wie hoch ist der (erste) Rentenbetrag bei einer ewigen Rente,
          die er jährlich nachschüssig abheben kann?\\
          Berechnen Sie im Vergleich dazu die erste Rentenrate bei 20-jähriger Laufzeit.\\
          $r_{\text{ewig}}=$\ansref \euro\\
          $r_{20}=$\ansref \euro
          }}
          \lang{en}{\text{Mr L. invested the rest of his lottery winnings of $\var{r0}$
          \euro at the beginning of the year at a fixed interest rate of $p=\var{p}$ \%. The annuity is to 
          increase by $\var{pw}$ \% after it is paid. It is therefore a variable annuity with the factor $w=\var{qw}$. 
          What is the (first) annuity amount in the case of a perpetual annuity
          that he can withdraw annually in arrears?\\
          In comparison, calculate the first annuity instalment for a 20-year term.\\
          $r_{\text{eternal}}=$\ansref \euro\\
          $r_{20}=$\ansref \euro
          }}
               \begin{answer}
                    \solution{rew}
               \end{answer} 
               \begin{answer}
                    \solution{r20}
               \end{answer}               
          \lang{de}{\explanation{(vgl. exercises)Die jährliche Rentenrate ist der Zinsbetrag abzüglich der Steigerung:
          $r_{\text{ewig}}=R_0\cdot\frac{p-w}{100}=\var{rew}$ \euro; bei begrenzter Laufzeit gilt:
          $r_{20}=R_0\cdot\frac{w-q}{\left(\frac{w}{q}\right)^m-1}=\var{r20}$ \euro.}}
          \lang{en}{\explanation{(cf. exercises)The annual annuity rate is the interest amount minus the increase:
          $r_{\text{ewig}}=R_0\cdot\frac{p-w}{100}=\var{rew}$ \euro; for a limited term:
          $r_{20}=R_0\cdot\frac{w-q}{\left(\frac{w}{q}\right)^m-1}=\var{r20}$ \euro.}}
          \end{question}
          
     \end{problem}

\embedmathlet{gwtmathlet}

\end{content}
