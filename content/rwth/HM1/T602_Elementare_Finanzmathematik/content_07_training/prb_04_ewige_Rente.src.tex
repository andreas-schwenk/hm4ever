\documentclass{mumie.problem.gwtmathlet}
%$Id$
\begin{metainfo}
  \name{
    \lang{en}{...}
    \lang{de}{A04: ewige Rente}
    \lang{zh}{...}
    \lang{fr}{...}
  }
  \begin{description} 
 This work is licensed under the Creative Commons License Attribution 4.0 International (CC-BY 4.0)   
 https://creativecommons.org/licenses/by/4.0/legalcode 

    \lang{en}{...}
    \lang{de}{...}
    \lang{zh}{...}
    \lang{fr}{...}
  \end{description}
  \corrector{system/problem/GenericCorrector.meta.xml}
  \begin{components}
    \component{js_lib}{system/problem/GenericMathlet.meta.xml}{gwtmathlet}
  \end{components}
  \begin{links}
  \end{links}
  \creategeneric
\end{metainfo}
\begin{content}
\lang{de}{\title{A04: ewige Rente}}
\lang{en}{\title{A04: perpetuity}}
\begin{block}[annotation]
	Im Ticket-System: \href{https://team.mumie.net/issues/22682}{Ticket 22682}
\end{block}

\usepackage{mumie.genericproblem}

     \begin{problem}
     \begin{variables}
        \randint{m}{2}{4}
        \function[calculate]{r0}{100000*m}
        \function[calculate,2]{p}{0,5*m}
        \function[calculate,2]{rew}{p*r0/100}
        \function[calculate,2]{q}{1+p/100}
        \function[calculate,2]{r40}{r0*(q^(41)-q^(40))/(q^(40)-1)}
     \end{variables}
          \begin{question}
          \type{input.number}
          \lang{de}{\text{Herr L. hat den Rest seines Lottogewinnes von $\var{r0}$
          \euro am Jahresbeginn zu einem festen Zinssatz von $\var{p}$ \% angelegt. Wie hoch ist die ewige Rente,
          die er jährlich nachschüssig abheben kann?\\
          Berechnen Sie im Vergleich dazu die Rentenrate bei 40-jähriger Laufzeit.\\
          $r_{\text{ewig}}=$\ansref \\
          $r_{40}=$ \ansref
          }}
          \lang{en}{\text{Mr L. invested the rest of his lottery winnings of $\var{r0}$
          \euro at the beginning of the year at a fixed interest rate of $\var{p}$ \%. How much is the perpetuity
          that he can withdraw in arrears each year?\\
          In comparison, calculate the annuity payments for a 40-year term.\\
          $r_{\text{eterity}}=$\ansref \\
          $r_{40}=$ \ansref
          }}
               \begin{answer}
                    \solution{rew}
               \end{answer}
               \begin{answer}
                    \solution{r40}
               \end{answer}               
          \lang{de}{\explanation{Bei einer ewigen Rente entspricht die jährliche Auszahlung dem Zinsbetrag.
          %$r_{\text{ewig}}=R_0\cdot\frac{p}{100}=\var{rew}$. 
          Die 40-jährige Laufzeit hat eine höhere Rate, 
          da am Ende kein Kapital übrig bleibt. 
          %$r_{40}=R_0\frac{q^{41}-q^{40}}{q^{40}-1}=\var{r40}$\euro.
          }}
          \lang{en}{\explanation{In the case of a perpetuity, the annual payout is equal to the interest amount.
          %$r_{\text{eternal}}=R_0\cdot\frac{p}{100}=\var{rew}$. 
          The 40-year term has a higher rate, 
          as there is no capital left over at the end. 
          %$r_{40}=R_0\frac{q^{41}-q^{40}}{q^{40}-1}=\var{r40}$\euro.
          }}
          \end{question}
          
     \end{problem}

\embedmathlet{gwtmathlet}

\end{content}
