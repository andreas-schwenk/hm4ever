\documentclass{mumie.problem.gwtmathlet}
%$Id$
\begin{metainfo}
  \name{
    \lang{en}{...}
    \lang{de}{A01: Rentenbarwert}
    \lang{zh}{...}
    \lang{fr}{...}
  }
  \begin{description} 
 This work is licensed under the Creative Commons License Attribution 4.0 International (CC-BY 4.0)   
 https://creativecommons.org/licenses/by/4.0/legalcode 

    \lang{en}{...}
    \lang{de}{...}
    \lang{zh}{...}
    \lang{fr}{...}
  \end{description}
  \corrector{system/problem/GenericCorrector.meta.xml}
  \begin{components}
    \component{js_lib}{system/problem/GenericMathlet.meta.xml}{gwtmathlet}
  \end{components}
  \begin{links}
  \end{links}
  \creategeneric
\end{metainfo}
\begin{content}
\lang{de}{\title{A01: Rentenbarwert}}
\lang{en}{\title{A01: Present value of an annuity}}
\begin{block}[annotation]
	Im Ticket-System: \href{https://team.mumie.net/issues/22689}{Ticket 22689}
\end{block}


\usepackage{mumie.genericproblem}

     \begin{problem}
          \begin{variables}
               \randint{a}{2}{5}
               \function[calculate]{r}{500*a}
               \randint{n}{4}{6}
               \randint{p}{2}{4}
               \function[calculate,2]{q}{1+p/100}
               \function[calculate,2]{r0}{r*(q^n-1)/(q^(n+1)-q^n)}
          \end{variables}       
          \begin{question}
               \type{input.function}
               \lang{de}{\text{Ein Student erhält von seinem Onkel $\var{n}$ Jahre lang nachschüssig
               $\var{r}$ \euro pro Jahr, die er mit einem Zinssatz zu $\var{p}$ \% anspart. Wieviel
               ist diese Rente, Stand heute, wert?\\
               Rentenbarwert: $R_0=$\ansref\euro}}
               \lang{en}{\text {A student receives $\var{r}$ \euro per year from his uncle,
               paid in arrears, over a period of $\var{n}$ years.
               He saves it in an account that earns interest at the rate of $\var{p}$ \%. How much
               is this annuity worth as of today?\\
               Annuity cash value: $R_0=$\ansref\euro}}
               \begin{answer}
                     \solution{r0}
               \end{answer} 
               \lang{de}{\explanation[edited]{Für den nachschüssigen Rentenendwert $R_n$ ergibt sich über die 
               geometrische Reihe: $R_n=r\frac{q^n-1}{q-1}$. Um den Rentenbarwert zu berechnen, muss $R_n$ mit
               $q^{-n}$ abgezinst werden:\\
               $R_0=r\frac{q^n-1}{q-1}\cdot\frac{1}{q^n}$
               %=r\frac{q^n-1}{q^{n+1}-q^n}
               %=\var{r}\frac{\var{q}^{\var{n}}-1}{\var{q}^{\var{n}+1}-\var{q}^{\var{n}}}=\var{r0}$ \euro.
               }}
               \lang{en}{\explanation[edited]{The final value of the annuity in arrears, $R_n$, is obtained via the 
               geometric series: $R_n=r\frac{q^n-1}{q-1}$. To calculate the annuity's present value, $R_n$ must be discounted
               by the factor $q^{-n}$:\\
               $R_0=r\frac{q^n-1}{q-1}\cdot\frac{1}{q^n}$
               %=r\frac{q^n-1}{q^{n+1}-q^n}
               %=\var{r}\frac{\var{q}^{\var{n}}-1}{\var{q}^{\var{n}+1}-\var{q}^{\var{n}}}=\var{r0}$ \euro.
               }}
          \end{question}     
     \end{problem}

\embedmathlet{gwtmathlet}

\end{content}
