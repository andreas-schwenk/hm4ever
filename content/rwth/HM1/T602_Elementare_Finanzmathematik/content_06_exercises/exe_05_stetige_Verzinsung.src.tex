\documentclass{mumie.element.exercise}
%$Id$
\begin{metainfo}
  \name{
    \lang{en}{Ü05: continuous compounding}
    \lang{de}{Ü05: stetige Verzinsung}
    \lang{zh}{...}
    \lang{fr}{...}
  }
  \begin{description} 
 This work is licensed under the Creative Commons License Attribution 4.0 International (CC-BY 4.0)   
 https://creativecommons.org/licenses/by/4.0/legalcode 

    \lang{en}{...}
    \lang{de}{...}
    \lang{zh}{...}
    \lang{fr}{...}
  \end{description}
  \begin{components}
  \end{components}
  \begin{links}
  \end{links}
  \creategeneric
\end{metainfo}
\begin{content}
\title{\lang{en}{Ü05: continuous compounding}
    \lang{de}{Ü05: stetige Verzinsung}
    \lang{zh}{...}
    \lang{fr}{...}}
\begin{block}[annotation]
	Im Ticket-System: \href{https://team.mumie.net/issues/22683}{Ticket 22683}
\end{block}



\lang{de}{Berechnen Sie für ein Kapital von $K_0=500.000$ \euro das Kapital nach 10 Jahren
bei einem Zinssatz von $i=2\%$ bei stetiger Verzinsung. Vergleichen Sie dies
mit dem Endkapital bei jährlicher Verzinsung.}

\lang{en}{Given a starting capital of $K_0 = 500,000$ \euro, compute its value after 10 years
of continuously compounding interest at a rate of $i=2\%$. Compare this with the ending capital
after annually compounding interest.
}

  \begin{tabs*}[\initialtab{0}\class{exercise}]
    \tab{
      \lang{de}{Lösung} \lang{en}{Solution}
    }

\lang{de}{Wir verwenden hierzu die Formel für unterjährige Verzinsung und entwickeln sie 
entsprechend weiter, indem wir die Anzahl der Zinsperioden pro Jahr gegen $\infty$
gehen lassen. \\
Weiterhin benutzen wir den bekannten Grenzwert:

\[\lim_{n\to\infty}\left(1+\frac{1}{n}\right)^n=e\approx 2,718\ldots \text{ und setzen }\frac{1}{k}=\frac{p}{m\cdot 100}: \]}
\lang{en}{We use the formula for periodic compound interest, letting the number of
interest periods per year tend to $\infty$.
Also, we use the familiar limit:

\[\lim_{n\to\infty}\left(1+\frac{1}{n}\right)^n=e\approx 2.718\ldots \text{ and set }\frac{1}{k}=\frac{p}{m\cdot 100}: \]
}
 

\begin{eqnarray*}
K_n&=&\lim_{m\to\infty}K_0\left(1+i_m\right)^{mn}\\
&=&\lim_{m\to\infty}K_0\left(1+\frac{p}{100\cdot m}\right)^{mn}\\
&=&\lim_{m\to\infty}K_0\left[\left(1+\frac{p}{100\cdot m}\right)^m\right]^n\\
&=&\lim_{k\to\infty}K_0\left[\left(1+\frac{1}{k}\right)^{\frac{kp}{100}}\right]^n\\
&=&\lim_{k\to\infty}K_0\left[\left(1+\frac{1}{k}\right)^k\right]^{\frac{pn}{100}}\\
&=&K_0\cdot e^{\frac{pn}{100}}
\end{eqnarray*}
\lang{de}{Dann gilt:
\begin{align*}
K_{10}&=&500.000\cdot e^{\frac{2\cdot 10}{100}}\\
&=&500.000\cdot e^{0,2}\\
&=&610.701,38 \text{ \euro}
\end{align*}}
\lang{en}{Then:
\begin{align*}
K_{10}&=&500,000\cdot e^{\frac{2\cdot 10}{100}}\\
&=&500,000\cdot e^{0.2}\\
&=&610,701.38 \text{ \euro}
\end{align*}}
\lang{de}{Im Vergleich dazu gilt bei jährlicher Verzinsung:}
\lang{en}{We compare this to annually compounding interest:}
\lang{de}{\begin{align*}
K_{10}&=&500.000\left(1+\frac{2}{100}\right)^{10}\\
&=&500.000\cdot 1,02^{10}\\
&=&609.497,21 \text{ \euro}
\end{align*}
Der Unterschied beträgt über 1000 \euro!}
\lang{en}{\begin{align*}
K_{10}&=&500,000\left(1+\frac{2}{100}\right)^{10}\\
&=&500,000\cdot 1.02^{10}\\
&=&609,497.21 \text{ \euro}
\end{align*}
The difference is more than 1000 \euro!}
\end{tabs*}
\end{content}

