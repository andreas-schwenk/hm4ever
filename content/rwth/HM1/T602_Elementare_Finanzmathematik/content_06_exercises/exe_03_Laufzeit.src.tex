\documentclass{mumie.element.exercise}
%$Id$
\begin{metainfo}
  \name{
    \lang{en}{Ü03: Investment length}
    \lang{de}{Ü03: Anlagezeitraum}
    \lang{zh}{...}
    \lang{fr}{...}
  }
  \begin{description} 
 This work is licensed under the Creative Commons License Attribution 4.0 International (CC-BY 4.0)   
 https://creativecommons.org/licenses/by/4.0/legalcode 

    \lang{en}{...}
    \lang{de}{...}
    \lang{zh}{...}
    \lang{fr}{...}
  \end{description}
  \begin{components}
  \end{components}
  \begin{links}
  \end{links}
  \creategeneric
\end{metainfo}
\begin{content}
\title{\lang{en}{Ü03: Investment length}
    \lang{de}{Ü03: Anlagezeitraum}
    \lang{zh}{...}
    \lang{fr}{...}}
\begin{block}[annotation]
	Im Ticket-System: \href{https://team.mumie.net/issues/22691}{Ticket 22691}
\end{block}


\lang{de}{Ein Kapital $K_0=1000$ \euro soll bei jährlicher Verzinsung mit 6 \% mindestens auf über
1300 \euro anwachsen. \\

Wie lange muss das Geld mindestens angelegt werden, um diesen Einlösungsbetrag zu erreichen?}

\lang{en}{ A starting capital of $K_0=1000$ \euro should increase to at least 1300 \euro at an annual
interest rate of 6 \%.

For how long must the money be invested to attain this value?
}

\begin{tabs*}[\initialtab{0}\class{exercise}]
\tab{\lang{de}{Lösung} \lang{en}{Solution}}
    \begin{incremental}[\initialsteps{1}]
        \step \lang{de}{Die Lösung ist:
        \begin{align*}
        5\text{ Jahre}
        \end{align*}} \lang{en}{The solution is:
        \begin{align*}
        5\text{ years}
        \end{align*}}
        \step
        \lang{de}{Wir gehen aus von der Gleichung:
        \begin{align*}
        K_n&=&K_0\left(1+\frac{p}{100}\right)^n\\
        1300&=&1000(1,06)^n\\
        1,3&=&1,06^n
        \end{align*}}
        \lang{en}{We begin with the equation:
        \begin{align*}
        K_n&=&K_0\left(1+\frac{p}{100}\right)^n\\
        1300&=&1000(1.06)^n\\
        1.3&=&1.06^n
        \end{align*}}
        \step
        \lang{de}{Um nach dem Exponenten aufzulösen, müssen wir die Gleichung logarithmieren.}
        \lang{en}{To solve for the exponent, we have to take logarithms.}
        \step
        \lang{de}{Hier sind zwei Möglichkeiten aufgeführt:
        \begin{itemize}
        \item[a)]
        \begin{align*}
        \ln 1,3&=&\ln 1,06^n\\
        \ln 1,3&=&n\cdot\ln 1,06\\
        n&=&\frac{\ln 1,3}{\ln 1,06}=4,5
        \end{align*}
        Es geht auch z.B. $\log_{10}$. 
        \item[b)]
        \begin{align*}
        \log_{1,06}1,3&=&n\\
        n&=&4,5
        \end{align*}
        \end{itemize}
        Das Geld muss also mindestens 5 Jahre lang angelegt werden.}
        \lang{en}{Here are two possibilities:
        \begin{itemize}
        \item[a)]
        \begin{align*}
        \ln 1.3&=&\ln 1.06^n\\
        \ln 1.3&=&n\cdot\ln 1.06\\
        n&=&\frac{\ln 1.3}{\ln 1.06}=4.5
        \end{align*}
        We could also have used $\log_{10}$. 
        \item[b)]
        \begin{align*}
        \log_{1.06}1.3&=&n\\
        n&=&4.5
        \end{align*}
        \end{itemize}
        Therefore, the money must be invested for at least 5 years.}
        \end{incremental}
\tab{\lang{de}{Bemerkung} \lang{en}{Remark}}
        \[K_n=K_0\left(1+\frac{p}{100}\right)^n\]
        \lang{de}{Soll nach dem Zinssatz aufgelöst werden, gilt:}
        \lang{en}{If we solve for the interest rate, we find:}
        \[i=\sqrt[n]{\frac{K_n}{K_0}}-1\]
        \lang{de}{Soll nach dem Anfangskapital aufgelöst werden, gilt:}
        \lang{en}{If we solve for the starting capital, we find:}
        \[K_0=K_n\left(1+\frac{p}{100}\right)^{-n}\]
\end{tabs*}
\end{content}

