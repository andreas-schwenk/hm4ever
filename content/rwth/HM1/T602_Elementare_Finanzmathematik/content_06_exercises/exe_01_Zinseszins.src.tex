\documentclass{mumie.element.exercise}
%$Id$
\begin{metainfo}
  \name{
    \lang{en}{Ü01: Compound interest}
    \lang{de}{Ü01: Zinses-Zins}
    \lang{zh}{...}
    \lang{fr}{...}
  }
  \begin{description} 
 This work is licensed under the Creative Commons License Attribution 4.0 International (CC-BY 4.0)   
 https://creativecommons.org/licenses/by/4.0/legalcode 

    \lang{en}{...}
    \lang{de}{...}
    \lang{zh}{...}
    \lang{fr}{...}
  \end{description}
  \begin{components}
  \end{components}
  \begin{links}
  \end{links}
  \creategeneric
\end{metainfo}
\begin{content}
\title{
    \lang{en}{Ü01: Compound interest}
    \lang{de}{Ü01: Zinses-Zins}
    \lang{zh}{...}
    \lang{fr}{...}}
\begin{block}[annotation]
	Im Ticket-System: \href{https://team.mumie.net/issues/22695}{Ticket 22695}
\end{block}


\lang{de}{Ein Sparbrief mit 4 \% Verzinsung wird mit einer Laufzeit von $n=6$ Jahren für 
1000 \euro angeboten. Wie hoch ist der Einlösungsbetrag?}
\lang{en}{A savings bond with 4 \% interest over $n=6$ years is offered for 
1000 \euro. What is its redemption value?}

\begin{tabs*}[\initialtab{0}\class{exercise}]
\tab{
   \lang{de}{Lösung}
   \lang{en}{Solution}
    }
\lang{de}{Das Kapital $K_0=1000$ wird incl. der Zinsen jährlich mit $i=p$ \% verzinst. Die entstehende
Folge des Kapitals $K_0, K_1, K_2,\ldots, K_n$ mit 
\[K_n=K_{n-1}\cdot \left(1+\frac{p}{100}\right)=K_0\left(1+\frac{p}{100}\right)^n\] 
ist eine geometrische
Folge: \[\frac{K_n}{K_{n-1}}=\text{const}=(1+\frac{p}{100})=q\]

Dann gilt:\\
\[K_6=1000\cdot\left(1+0,04\right)^6=1265,32 \text{ \euro}\]}

\lang{en}{The principal $K_0=1000$, together with its interest, accumulates interest annually at the rate $i=p$.
The sequence $K_0,K_1,K_2,\ldots,K_n$ of principal amounts, with

\[K_n=K_{n-1}\cdot \left(1+\frac{p}{100}\right)=K_0\left(1+\frac{p}{100}\right)^n\] 
is a geometric sequence: \[\frac{K_n}{K_{n-1}}=\text{const}=(1+\frac{p}{100})=q\]

Therefore:\\
\[K_6=1000\cdot\left(1+0.04\right)^6=1265.32 \text{ \euro}\]

}

\lang{de}{$q^n$ nennt man auch \textbf{Aufzinsungsfaktor} und\\
$\frac{1}{q^n}=q^{-n}$ \textbf{Abzinsungsfaktor}.}
\lang{en}{$q^n$ is also known as the \textbf{compounding factor}
and $\frac{1}{q^n} = q^{-n}$ is also known as the \textbf{discount factor}.}
\end{tabs*}
\end{content}

