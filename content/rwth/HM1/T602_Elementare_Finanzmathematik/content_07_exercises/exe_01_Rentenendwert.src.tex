\documentclass{mumie.element.exercise}
%$Id$
\begin{metainfo}
  \name{
    \lang{en}{Exercise 1: Future value of an annuity}
    \lang{de}{Ü01: Rentenendwert}
    \lang{zh}{...}
    \lang{fr}{...}
  }
  \begin{description} 
 This work is licensed under the Creative Commons License Attribution 4.0 International (CC-BY 4.0)   
 https://creativecommons.org/licenses/by/4.0/legalcode 

    \lang{en}{...}
    \lang{de}{...}
    \lang{zh}{...}
    \lang{fr}{...}
  \end{description}
  \begin{components}
\component{generic_image}{content/rwth/HM1/images/g_tkz_T602_07_Exercise01_B.meta.xml}{T602_07_Exercise01_B}
\component{generic_image}{content/rwth/HM1/images/g_tkz_T602_07_Exercise01_A.meta.xml}{T602_07_Exercise01_A}
\component{generic_image}{content/rwth/HM1/T602_Elementare_Finanzmathematik/content_07_exercises/g_img_rente1.meta.xml}{rente11}
\component{generic_image}{content/rwth/HM1/T602_Elementare_Finanzmathematik/content_07_exercises/g_img_rente_v.meta.xml}{rente_v1}

\end{components}
  \begin{links}
  \end{links}
  \creategeneric
\end{metainfo}
\begin{content}
\title{
    \lang{en}{Exercise 1: Future value of an annuity}
    \lang{de}{Ü01: Rentenendwert}
    \lang{zh}{...}
    \lang{fr}{...}
}

\begin{block}[annotation]
	Im Ticket-System: \href{https://team.mumie.net/issues/22672}{Ticket 22672}
\end{block}




\lang{de}{Berechnen Sie folgende Rentenendwerte $R_n$:

Genau 10 Jahre lang werden 12000 \euro zu 4 \% angelegt. Wie hoch ist 
der Betrag, der zu Beginn des 11. Jahres bereitsteht?\\
\begin{itemize}
    \item[a)] bei \textbf{nachschüssiger} Zahlung der Rate zum Jahresende:
    \item[b)] bei \textbf{vorschüssiger} Zahlung der Rate zu Jahresbeginn:
\end{itemize}
}
\lang{en}{Compute the future value $R_n$ of the following annuities.

12000 \euro is invested for exactly 10 years at a rate of 4 \%.
What amount is available at the beginning of the eleventh year,\\
\begin{itemize}
    \item[a)] if the annuity is paid \textbf{in arrears}:
    \item[b)] if the annuity is paid \textbf{in advance}:
\end{itemize}
}
\begin{tabs*}[\initialtab{0}\class{exercise}]
\tab{\lang{de}{Lösung a)} \lang{en}{Solution a)}}
    \begin{incremental}[\initialsteps{1}]
        \step \lang{de}{Die Lösung ist:
        \begin{align*}
        R_{10}=144.073,29 \text{ \euro}
        \end{align*} }
        \lang{en}{The solution is:
        \begin{align*}
        R_{10}=144,073.29 \text{\euro}
        \end{align*} }
        \step
        \lang{de}{
        \[r=12000\qquad q=1+\frac{4}{100}=1,04\]

        \[R_{10}=rq^9+rq^8+rq^7+\ldots +rq^2+rq+r=r\sum_{i=0}^9q^i\]
        }
        \lang{en}{
        \[r=12000\qquad q=1+\frac{4}{100}=1.04\]

        \[R_{10}=rq^9+rq^8+rq^7+\ldots +rq^2+rq+r=r\sum_{i=0}^9q^i\]
        }
        \begin{figure}
        \image{T602_07_Exercise01_A}
        \end{figure}
        
        \step
        \lang{de}{
        Der zehnte Jahresbeitrag wird nicht mehr verzinst, da die Zahlungen nachschüssig sind.
        Diese Summe kann mit Hilfe der geometrischen Reihe ausgeführt werden:\\ 
        \[R_{10}=r\frac{q^{10}-1}{q-1}=144.073,29 \text{ \euro.}\] }
        \lang{en}{
        The contribution in year 10 no longer accrues interest, as the payments occur in arrears.
        This sum can be calculated using the geometric series:\\
        \[R_{10}=r\frac{q^{10}-1}{q-1}=144,073.29 \text{ \euro.}\] 
        }
    \end{incremental}
\tab{\lang{de}{Lösung b)} \lang{en}{Solution b)}}
    \begin{incremental}[\initialsteps{1}]
        \step \lang{de}{Die Lösung ist:
        \begin{align*}
        R_{10}=149.836,22\text{ \euro} 
        \end{align*}}
        \lang{en}{The solution is:
        \begin{align*}
        R_{10}=149,836.22\text{ \euro} 
        \end{align*}}
        \step
        \lang{de}{
        \[r=12000 \qquad q=1+\frac{4}{100}=1,04\]

        \[R_{10}=rq^10+rq^9+rq^8+\ldots +rq^3+rq^2+rq=rq\sum_{i=0}^9q^i\]
        }
        \lang{en}{
        \[r=12000 \qquad q=1+\frac{4}{100}=1.04\]

        \[R_{10}=rq^10+rq^9+rq^8+\ldots +rq^3+rq^2+rq=rq\sum_{i=0}^9q^i\]
        
        }
        \begin{figure}
        \image{T602_07_Exercise01_B}
        \end{figure}
        \step
        \lang{de}{Bei vorschüssiger Zahlung wird alles einmal mehr verzinst.
        Wir greifen wieder auf die geometrische Reihe zurück:
        \[R_{10}=rq\frac{q^{10}-1}{q-1}=149.836,22 \text{ \euro.}\] }
        \lang{en}{If the interest is accrued in advance, then everything earns interest one more time.
        We use the geometric series again:
        \[R_{10}=rq\frac{q^{10}-1}{q-1}=149,836.22 \text{ \euro.}\] 
        }
        
    \end{incremental}
\end{tabs*}
\end{content}

