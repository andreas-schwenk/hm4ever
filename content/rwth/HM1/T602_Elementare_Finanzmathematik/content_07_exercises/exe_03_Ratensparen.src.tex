\documentclass{mumie.element.exercise}
%$Id$
\begin{metainfo}
  \name{
    \lang{en}{Exercise 3: An installment savings plan}
    \lang{de}{Ü03: Ratensparvertrag}
    \lang{zh}{...}
    \lang{fr}{...}
  }
  \begin{description} 
 This work is licensed under the Creative Commons License Attribution 4.0 International (CC-BY 4.0)   
 https://creativecommons.org/licenses/by/4.0/legalcode 

    \lang{en}{...}
    \lang{de}{...}
    \lang{zh}{...}
    \lang{fr}{...}
  \end{description}
  \begin{components}
  \end{components}
  \begin{links}
  \end{links}
  \creategeneric
\end{metainfo}
\begin{content}
\title{
    \lang{en}{Exercise 3: An installment savings plan}
    \lang{de}{Ü03: Ratensparvertrag}
    \lang{zh}{...}
    \lang{fr}{...}
}
\begin{block}[annotation]
	Im Ticket-System: \href{https://team.mumie.net/issues/22693}{Ticket 22693}
\end{block}

\label{Ratensparen_re}
\lang{de}{Berechnen Sie folgende \textbf{Rentenendwerte} $R_n$:

Nun werden 10 Jahre monatlich 1000 \euro eingezahlt und zu 4 \% verzinst. Es gibt also $k=12$ Zahlungen pro Jahr. 
Die Zinsperiode beträgt 1 Jahr. Wie hoch ist der Betrag
der zu Beginn des 11. Jahres bereitsteht?
\begin{itemize}
    \item[a)] bei nachschüssiger Zahlung der Rate zum Monatsende:
    \item[b)] bei vorschüssiger Zahlung der Rate zu Monatsbeginn:
\end{itemize}    }
\lang{en}{Compute the \textbf{future values} $R_n$ of the following annuities:

For 10 years, 1000 \euro is invested monthly at an interest rate of 4 \%. 
In particular, there are $k=12$ deposits per year. 
The interest period is 1 year. What amount is available at the
beginning of the 11th year?
\begin{itemize}
    \item[a)] if the deposits occur in arrears, at the end of the month:
    \item[b)] if the deposits occur in advance, at the beginning of the month:
\end{itemize}    }

\begin{tabs*}[\initialtab{0}\class{exercise}]
\tab{\lang{de}{Lösung a)} \lang{en}{Solution a)}}
    \begin{incremental}[\initialsteps{1}]
        \step \lang{de}{Die Lösung ist:
        \begin{align*}
        R_{10}=146.714,63 \text{ \euro}
        \end{align*}}
        \lang{en}{The solution is:
        \begin{align*}
        R_{10}=146,714.63 \text{ \euro}
        \end{align*}}
        \step
        \lang{de}{
        \[r=1000\qquad q=1+\frac{4}{100}=1,04\qquad k=12\]
        }
        \lang{en}{
        \[r=1000\qquad q=1+\frac{4}{100}=1.04\qquad k=12\]
        }
        \step
        \lang{de}{Aus den m Zahlungen pro Zinsperiode berechnet man zunächst 
        entsprechende nachschüssige jährliche Zahlungen $r_e$ mit einfacher Verzinsung. Im nächsten Schritt wird daraus der
        Rentenendwert berechnet.}
        \lang{en}{First we combine the m deposits per interest period to an equivalent
        annual deposit (in arrears) $r_e$, assuming simple interest.
        Then we will use this to compute the future value of the annuity.
        
        }
        \begin{eqnarray*}
        r_e&=&r+r\left(1+\frac{p}{100\cdot k}\right)+r\left(1+\frac{2p}{100\cdot k}\right)+\ldots +r\left(1+\frac{(k-1)p}{100\cdot k}\right)\\
        &=&r\left(k+\frac{(k-1)p}{200}\right)
        \end{eqnarray*}
        \lang{de}{Die Summe wurde mit der Gauss'schen Summenformel $\sum_{i=1}^ni=\frac{n(n+1)}{2}$ ausgeführt.}
        \lang{en}{The sum was calculated using Gauss's formula $\sum_{i=1}^ni=\frac{n(n+1)}{2}$.}
        \lang{de}{\[r_e=1000\left(12+\frac{4\cdot 11}{200}\right)=12.220 \text{ \euro}\]}
        \lang{en}{\[r_e=1000\left(12+\frac{4\cdot 11}{200}\right)=12,220 \text{ \euro}\]}
        \step
        \lang{de}{Mit dieser jährlichen Ersatzzahlung $r_e$ kann nun der nachschüssige Rentenendwert 
        berechnet werden:
        \begin{eqnarray*}
        R_{10}&=&r_e\frac{q^10-1}{q-1}\\
        &=&12.220\frac{1,04^{10}-1}{1,04-1}\\
        &=&146.714,63 \text{ \euro}
        \end{eqnarray*}        }
        \lang{en}{Using the equivalent annual annuity, we can now compute the future value:
        \begin{eqnarray*}
        R_{10}&=&r_e\frac{q^10-1}{q-1}\\
        &=&12,220\frac{1.04^{10}-1}{1.04-1}\\
        &=&146,714.63 \text{ \euro}
        \end{eqnarray*}
      
        }
       
    \end{incremental}
\tab{\lang{de}{Lösung b)} \lang{en}{Solution b)}}
    \begin{incremental}[\initialsteps{1}]
        \step \lang{de}{Die Lösung ist:
        \begin{align*}
        R_0=147.194,87 \text{ \euro}
        \end{align*}}
        \lang{en}{The solution is:
        \begin{align*}
        R_0=147,194.87 \text{ \euro}
        \end{align*}}
        \step
        \lang{de}{\[r=1000\qquad q=1+\frac{4}{100}=1,04\qquad k=12\]}
        \lang{en}{\[r=1000\qquad q=1+\frac{4}{100}=1.04\qquad k=12\]}
        \step
        \lang{de}{Aus den m Zahlungen pro Zinsperiode berechnet man zunächst 
        entsprechende vorschüssige jährliche Zahlungen $r_e$ mit einfacher Verzinsung. Im nächsten Schritt wird daraus der
        Rentenendwert berechnet.
        \begin{eqnarray*}
        r_e&=&r+r\left(1+\frac{p}{100\cdot k}\right)+r\left(1+\frac{2p}{100\cdot k}\right)+\ldots +r\left(1+\frac{kp}{100\cdot k}\right)\\
        &=&r\left(k+\frac{(k+1)p}{200}\right)
        \end{eqnarray*}
        Die Summe wurde mit der Gauss'schen Summenformel $\sum_{i=1}^ni=\frac{n(n+1)}{2}$ ausgeführt.
        \[r_e=1000\left(12+\frac{4\cdot 13}{200}\right)=12.260 \text{ \euro}\] }
        \lang{en}{First we combine the m deposits per interest period to an equivalent
        annual deposit (in advance) $r_e$, assuming simple interest. Then we will use this to compute the
        future value of the annuity.
        \begin{eqnarray*}
        r_e&=&r+r\left(1+\frac{p}{100\cdot k}\right)+r\left(1+\frac{2p}{100\cdot k}\right)+\ldots +r\left(1+\frac{kp}{100\cdot k}\right)\\
        &=&r\left(k+\frac{(k+1)p}{200}\right)
        \end{eqnarray*}
        The sum was calculated using Gauss's formula $\sum_{i=1}^ni=\frac{n(n+1)}{2}$.
        \[r_e=1000\left(12+\frac{4\cdot 13}{200}\right)=12,260 \text{ \euro}\]
        }
        \step
        \lang{de}{Mit dieser jährlichen Ersatzzahlung $r_e$ kann nun der nachschüssige Rentenendwert 
        berechnet werden:
        \begin{eqnarray*}
        R_{10}&=&r_e\frac{q^10-1}{q-1}\\
        &=&12.260\frac{1,04^{10}-1}{1,04-1}\\
        &=&147.194,87 \text{ \euro}
        \end{eqnarray*}}
        \lang{en}{Using the equivalent annual annuity, we can now compute the future value:
        \begin{eqnarray*}
        R_{10}&=&r_e\frac{q^10-1}{q-1}\\
        &=&12,260\frac{1.04^{10}-1}{1.04-1}\\
        &=&147,194.87 \text{ \euro}
        \end{eqnarray*}}
    \end{incremental}
\end{tabs*}


\end{content}

