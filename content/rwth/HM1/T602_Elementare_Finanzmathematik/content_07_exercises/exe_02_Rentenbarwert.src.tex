\documentclass{mumie.element.exercise}
%$Id$
\begin{metainfo}
  \name{
    \lang{en}{Exercise 2: Present value of an annuity}
    \lang{de}{Ü02: Rentenbarwerte}
    \lang{zh}{...}
    \lang{fr}{...}
  }
  \begin{description} 
 This work is licensed under the Creative Commons License Attribution 4.0 International (CC-BY 4.0)   
 https://creativecommons.org/licenses/by/4.0/legalcode 

    \lang{en}{...}
    \lang{de}{...}
    \lang{zh}{...}
    \lang{fr}{...}
  \end{description}
  \begin{components}
  \end{components}
  \begin{links}
  \end{links}
  \creategeneric
\end{metainfo}
\begin{content}
\title{
    \lang{en}{Exercise 2: Present value of an annuity}
    \lang{de}{Ü02: Rentenbarwerte}
    \lang{zh}{...}
    \lang{fr}{...}
}
\begin{block}[annotation]
	Im Ticket-System: \href{https://team.mumie.net/issues/22680}{Ticket 22680}
\end{block}


\lang{de}{Berechnen Sie folgende \textbf{Rentenbarwerte} $R_0$:

Aus welchem zum Zeitpunkt 0 eingezahlten Betrag kann 10 Jahre lang eine 
konstante jährliche Rente von 12000 \euro gezahlt werden, wenn die Anlage zu 4 \% erfolgt?
\\
\begin{itemize}
    \item[a)] bei \textbf{nachschüssiger} Zahlung der Rente zum Jahresende:
    \item[b)] bei \textbf{vorschüssiger} Zahlung der Rente zu Jahresbeginn:
\end{itemize}    
}
\lang{en}{Compute the \textbf{present value} $R_0$ of the following annuities:

What amount must be invested at time $0$ to yield a constant annual annuity
of 12000 \euro for 10 years, assuming the investment earns interest at a rate of 4 \%?

\begin{itemize}
    \item[a)] if the annuity is paid out \textbf{in arrears}, at the end of the year:
    \item[b)] if the annuity is paid out \textbf{in advance}, at the beginning of the year:
\end{itemize}      
}

\begin{tabs*}[\initialtab{0}\class{exercise}]
\tab{\lang{de}{Lösung a)} \lang{en}{Solution a)}}
    \begin{incremental}[\initialsteps{1}]
        \step \lang{de}{Die Lösung ist:
        \begin{align*}
        R_0=97.330,69 \text{ \euro}
        \end{align*}}
        \lang{en}{The solution is:
        \begin{align*}
        R_0=97,330.69 \text{ \euro}
        \end{align*}}
        \step
        \lang{de}{\[r=12000\qquad q=1+\frac{4}{100}=1,04\]}
        \lang{en}{\[r=12000\qquad q=1+\frac{4}{100}=1.04\]}
        \step
        \lang{de}{Der nachschüssige Rentenendwert von 
        \[R_n=r\frac{q^n-1}{q-1}\]
        muss mit dem Faktor $q^n$ abgezinst werden:
        \[R_0=R_nq^n=r\frac{q^n-1}{q^{n+1}-q^n}\]
        (Der Faktor $\frac{q^n-1}{q^{n+1}-q^n}$ heißt auch 
        \textbf{nachschüssiger Rentenbarwertfaktor}.)
        Damit ergibt sich:
        \begin{eqnarray*}
        R_0&=&12000\frac{1,04^{10}-1}{1,04^{11}-1,04^{10}}\\
        &=&97.330,69 \text{ \euro}
        \end{eqnarray*}}
        \lang{en}{
        The future value \[R_n=r\frac{q^n-1}{q-1}\]
        must be discounted by the factor $q^n$:
        \[R_0=R_nq^n=r\frac{q^n-1}{q^{n+1}-q^n}\]
        (The factor $\frac{q^n-1}{q^{n+1}-q^n}$ is also called the
        \textbf{annuity factor} for an annuity in arrears.)
        So we have:
        \begin{eqnarray*}
        R_0&=&12000\frac{1.04^{10}-1}{1.04^{11}-1.04^{10}}\\
        &=&97,330.69 \text{ \euro}
        \end{eqnarray*}
        }
       
    \end{incremental}
\tab{\lang{de}{Lösung b)} \lang{en}{Solution b)}}
    \begin{incremental}[\initialsteps{1}]
        \step \lang{de}{Die Lösung ist:
        \begin{align*}
        R_0=101.223,92\text{ \euro} 
        \end{align*}}
        \lang{en}{The solution is:
        \begin{align*}
        R_0=101,223.92\text{ \euro} 
        \end{align*}}
        \step
        \lang{de}{\[r'=12000 \qquad q=1+\frac{4}{100}=1,04\]}
        \lang{en}{\[r'=12000 \qquad q=1+\frac{4}{100}=1.04\]}
        \step
        \lang{de}{Da bei vorschüssiger Zahlung, also einer Rentenzahlung zu Beginn einer Zinsperiode,
        jede Rate einmal mehr verzinst wird, ist die vorschüssige Rente gleichwertig zu einer 
        nachschüssigen mit der Rate $r=r'\cdot q=12.480 \text{ \euro}$.}
        \lang{en}{Since each payment accrues interest one additional time in 
        an annuity an advance, i.e. an annuity paid at the beginning of each
        interest period, the annuity in advance is equivalent to an annuity in arrears
        with a payment of $r=r'\cdot q=12.480 \text{ \euro}$.}
        \step
        \lang{de}{
        (Der Faktor $q\cdot \frac{q^n-1}{q^{n+1}-q^n}$ heißt auch 
        \textbf{vorschüssiger Rentenbarwertfaktor}.)
         Damit ergibt sich:
        \begin{eqnarray*}
        R_0&=&12000\cdot1,04\frac{1,04^{10}-1}{1,04^{11}-1,04^{10}}\\
        &=&12480\frac{1,04^{10}-1}{1,04^{11}-1,04^{10}}\\
        &=&101.223,92 \text{ \euro}
        \end{eqnarray*}}
        \lang{en}{
        (The factor $q\cdot \frac{q^n-1}{q^{n+1}-q^n}$ is also called the
        \textbf{annuity factor} for an annuity in advance.)
        So we have:
        \begin{eqnarray*}
        R_0&=&12000\cdot1.04\frac{1.04^{10}-1}{1.04^{11}-1.04^{10}}\\
        &=&12480\frac{1.04^{10}-1}{1.04^{11}-1.04^{10}}\\
        &=&101,223.92 \text{ \euro}
        \end{eqnarray*}}
    \end{incremental}
\end{tabs*}

\end{content}

