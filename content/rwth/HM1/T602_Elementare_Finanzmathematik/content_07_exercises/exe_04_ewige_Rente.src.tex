\documentclass{mumie.element.exercise}
%$Id$
\begin{metainfo}
  \name{
    \lang{en}{Exercise 4: Perpetual annuity}
    \lang{de}{Ü04: ewige Rente}
    \lang{zh}{...}
    \lang{fr}{...}
  }
  \begin{description} 
 This work is licensed under the Creative Commons License Attribution 4.0 International (CC-BY 4.0)   
 https://creativecommons.org/licenses/by/4.0/legalcode 

    \lang{en}{...}
    \lang{de}{...}
    \lang{zh}{...}
    \lang{fr}{...}
  \end{description}
  \begin{components}
  \end{components}
  \begin{links}
  \end{links}
  \creategeneric
\end{metainfo}
\begin{content}
\title{
    \lang{en}{Exercise 4: Perpetual annuity}
    \lang{de}{Ü04: ewige Rente}
    \lang{zh}{...}
    \lang{fr}{...}
}

\begin{block}[annotation]
	Im Ticket-System: \href{https://team.mumie.net/issues/22677}{Ticket 22677}
\end{block}




\lang{de}{Es soll die \textbf{unendliche Rente} r berechnet werden, die jährlich aus 
einem Lottogewinn von $R_0=1.000.000$ \euro gezahlt werden kann. Die Zinsperiode beträgt 
1 Jahr mit dem Zinssatz $i=0,1 \%$.

\begin{itemize}
    \item[a)] bei nachschüssiger Zahlung der Rentenrate zum Jahresende:
    \item[b)] bei vorschüssiger Zahlung der Rentenrate zu Jahresbeginn:
\end{itemize}    }
\lang{en}{We will compute the \textbf{perpetual annuity} r that can be paid out annually
from a lottery winnings of $R_0=1,000,000$ \euro. Interest is accumulated annually at
an interest rate of $i=0.1 \%$.

\begin{itemize}
    \item[a)] If the annuity is paid in arrears, at the end of the year:
    \item[b)] If the annuity is paid in advance, at the beginning of the year:
\end{itemize}    }

\begin{tabs*}[\initialtab{0}\class{exercise}]
\tab{\lang{de}{Lösung a)} \lang{en}{Solution a)}}
    \begin{incremental}[\initialsteps{1}]
\step \lang{de}{Die Lösung ist:
        \begin{align*}
        R_0=1000,00 \text{ \euro}
        \end{align*}}
      \lang{en}{The solution is:
        \begin{align*}
        R_0=1000.00 \text{ \euro}.
        \end{align*}}
\step
\lang{de}{Wir benutzen, dass zur Berechnung des Rentenbarwertes jede Rentenrate r auf den Zeitpunkt 0
abgezinst werden muss:
\[R_0=r\frac{1}{q}+r\frac{1}{q^2}+r\frac{1}{q^3}+\ldots=\frac{r}{q}\sum_{i=0}^{infty}\frac{1}{q^i}=100\frac{r}{p}\]
Da $\frac{1}{q}<1$ kann die Summe berechnet werden zu: $\frac{1}{1-\frac{1}{q}}=\frac{q}{q-1}$.
Umgekehrt gilt also:
\[r=R_0\cdot\frac{p}{100}\]
Die Rentenrate entspricht also den jährlichen Zinsen. Damit ist:

\[r=1.000.000\frac{1}{1000}=1000 \text{ \euro}\]}
\lang{en}{We use the fact that each annuity payment r is discounted
to the time 0 when computing the present value:
\[R_0=r\frac{1}{q}+r\frac{1}{q^2}+r\frac{1}{q^3}+\ldots=\frac{r}{q}\sum_{i=0}^{infty}\frac{1}{q^i}=100\frac{r}{p}\]
Since $\frac{1}{q}<1$, the series can be simplified to $\frac{1}{1-\frac{1}{q}}=\frac{q}{q-1}$.
Therefore,
\[r=R_0\cdot\frac{p}{100}\]
In other words, the annuity payment equals the annual interest. So
\[r=1,000,000 \cdot \frac{1}{1000}=1000 \text{ \euro}.\]
}
\end{incremental}

\tab{\lang{de}{Lösung b)} \lang{en}{Solution b)}}
    \begin{incremental}[\initialsteps{1}]
        \step \lang{de}{Die Lösung ist:
        \begin{align*}
        R_0=999,00 \text{ \euro}
        \end{align*}}
        \lang{en}{The solution is:
        \begin{align*}
        R_0=999.00 \text{ \euro}.
        \end{align*}}
        \step
\lang{de}{Erfolgt die Rentenausschüttung vorschüssig zu Beginn des Jahres, so fallen Lottogewinn und erste
Rentenzahlung zusammen, d.h. die erste Rate wird nicht abgezinst und es gilt:
\[R_0=r+r\frac{1}{q}+r\frac{1}{q^2}+r\frac{1}{q^3}+\ldots=r\sum_{i=0}^{infty}\frac{1}{q^i}=100\frac{rq}{p}\]
Umgekehrt gilt also:
\begin{align*}
r&=&R_0\cdot\frac{p}{100\cdot q}\\
r&=&1.000.000\frac{0,001}{1,001}=999,00 \text{ \euro}
\end{align*}}
\lang{en}{If the annuity payments occur in advance, at the beginnning of
the year, then the lottery winnings and the first annuity payment
occur simultaneously. This means that the first payment is not discounted,
so
\[R_0=r+r\frac{1}{q}+r\frac{1}{q^2}+r\frac{1}{q^3}+\ldots=r\sum_{i=0}^{infty}\frac{1}{q^i}=100\frac{rq}{p}\]
Conversely,
\begin{align*}
r&=&R_0\cdot\frac{p}{100\cdot q}\\
r&=&1,000,000 \cdot \frac{0.001}{1.001}=999.00 \text{ \euro}.
\end{align*}}

\end{incremental}
\end{tabs*}
\end{content}

