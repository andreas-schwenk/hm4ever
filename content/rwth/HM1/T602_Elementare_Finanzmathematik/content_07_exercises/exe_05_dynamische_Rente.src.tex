\documentclass{mumie.element.exercise}
%$Id$
\begin{metainfo}
  \name{
    \lang{en}{Exercise 5: Variable annuity}
    \lang{de}{Ü05: dynamische Rente}
    \lang{zh}{...}
    \lang{fr}{...}
  }
  \begin{description} 
 This work is licensed under the Creative Commons License Attribution 4.0 International (CC-BY 4.0)   
 https://creativecommons.org/licenses/by/4.0/legalcode 

    \lang{en}{...}
    \lang{de}{...}
    \lang{zh}{...}
    \lang{fr}{...}
  \end{description}
  \begin{components}
  \end{components}
  \begin{links}
  \end{links}
  \creategeneric
\end{metainfo}
\begin{content}
\title{\lang{en}{Exercise 5: Variable annuity}
    \lang{de}{Ü05: dynamische Rente}
    \lang{zh}{...}
    \lang{fr}{...}}
\begin{block}[annotation]
	Im Ticket-System: \href{https://team.mumie.net/issues/22685}{Ticket 22685}
\end{block}




\lang{de}{Berechnen Sie den Rentenendwert der folgenden dynamischen Rente:

Ein Rentennehmer erhält 10 Jahre lang eine Rentenrate, die jährlich um 1 \% steigt. 
Die erste Rentenrate ist $r=1000 \text{ \euro}$.
Die Rente wird nachschüssig ausbezahlt mit 1-jähriger Zinsperiode und einer Verzinsung
von 2 \%.}

\lang{en}{Compute the future value of the following variable payout annuity:

The benificiary of an annuity receives payments for 10 years at a rate that
increases by 1 \% annually. The first payment is $r=1000 \text{ \euro}$.
The annuity is paid in arrears with an interest period of 1 year
at an interest rate of 2 \%.
}

\begin{tabs*}[\initialtab{0}\class{exercise}]
\tab{\lang{de}{Lösung} \lang{en}{Solution}}
    \begin{incremental}[\initialsteps{1}]
        \step \lang{de}{Die Lösung ist:
        \begin{align*}
        R_{10}=11437,23 \text{ \euro}
        \end{align*}}
        \lang{en}{The solution is:
        \begin{align*}
        R_{10}=11437.23 \text{ \euro}
        \end{align*}}
        \step
\lang{de}{Damit ist der Steigerungsfaktor $w=1,01$. Der Zinsfaktor beträgt $q=1,02$.}
\lang{en}{The investment increment factor is therefore $w=1.01$.
The interest factor is $q=1.02$.}
\step
\lang{de}{Es gilt:
\begin{align*}
R_m&=&rq^{m-1}+rwq^{m-2}+\ldots+rw^nq^{m-n-1}+\ldots+rw^{m-1}\\
&=& rq^{m-1}\left(1+\frac{w}{q}+\frac{w^2}{q^2}+\ldots+\frac{w^{m-1}}{q^{m-1}}\right)\\
&=& \begin{cases}r\cdot m\cdot q^{m-1}& \text{für }w=q\\
r\frac{w^m-q^m}{w-q}& \text{für }w\neq q
\end{cases}
\end{align*}}
\lang{en}{We have:
\begin{align*}
R_m&=&rq^{m-1}+rwq^{m-2}+\ldots+rw^nq^{m-n-1}+\ldots+rw^{m-1}\\
&=& rq^{m-1}\left(1+\frac{w}{q}+\frac{w^2}{q^2}+\ldots+\frac{w^{m-1}}{q^{m-1}}\right)\\
&=& \begin{cases}r\cdot m\cdot q^{m-1}& \text{if }w=q;\\
r\frac{w^m-q^m}{w-q}& \text{if }w\neq q.
\end{cases}
\end{align*}}
\step
\lang{de}{In diesem Fall gilt also:
\[R_{10}=1000\frac{1,01^{10}-1,02^{10}}{1,01-1,02}=11437,23 \text{ \euro}\]}
\lang{en}{Therefore, in this case,
\[R_{10}=1000\frac{1.01^{10}-1.02^{10}}{1.01-1.02}=11437.23 \text{ \euro}\]}
\end{incremental}
\end{tabs*}
\end{content}

