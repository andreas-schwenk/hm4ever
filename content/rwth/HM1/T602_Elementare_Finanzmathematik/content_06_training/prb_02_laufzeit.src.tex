\documentclass{mumie.problem.gwtmathlet}
%$Id$
\begin{metainfo}
  \name{
    \lang{en}{...}
    \lang{de}{A02: Anlagezeitraum}
    \lang{zh}{...}
    \lang{fr}{...}
  }
  \begin{description} 
 This work is licensed under the Creative Commons License Attribution 4.0 International (CC-BY 4.0)   
 https://creativecommons.org/licenses/by/4.0/legalcode 

    \lang{en}{...}
    \lang{de}{...}
    \lang{zh}{...}
    \lang{fr}{...}
  \end{description}
  \corrector{system/problem/GenericCorrector.meta.xml}
  \begin{components}
    \component{js_lib}{system/problem/GenericMathlet.meta.xml}{gwtmathlet}
  \end{components}
  \begin{links}
  \end{links}
  \creategeneric
\end{metainfo}
\usepackage{mumie.genericproblem}

\begin{content}
\lang{de}{\title{A02: Anlagezeitraum}}
\lang{en}{\title{A02: Investment period}}
\begin{block}[annotation]
	Im Ticket-System: \href{https://team.mumie.net/issues/22678}{Ticket 22678}
\end{block}



     \begin{problem}
     \begin{variables}
        \randint{p}{2}{6}
        \randint{n}{3}{7}
        \drawFromSet{k0}{10000,20000,30000}
        \function[calculate,2]{kn}{k0(1+p/100)^n}
        \function[calculate,2]{basis}{1+p/100}
     \end{variables}
          \begin{question}
          \type{input.number}
          \lang{de}{\text{Ein Kapital $K_0=\var{k0}$\euro ist bei $\var{p}$\% jährlichem Zins auf 
          $K_n=\var{kn}$\euro angewachsen. Wie lange war es zinseszinslich angelegt?\\
          $n=$\ansref Jahre}}
          \lang{en}{\text{A capital $K_0=\var{k0}$\euro has grown at $\var{p}$\% annual interest to
          $K_n=\var{kn}$\euro. For how long was it invested?\\
          $n=$\ansref years}}
               \begin{answer}
                    \solution{n}
               \end{answer}
               \lang{de}{\explanation{Wenn die Zinseszinsformel nach n aufgelöst werden soll,
               muss der Logarithmus benutzt werden.}}
               \lang{en}{\explanation{Use a logarithm to solve for n
               in the compound interest formula.}}
              % $n=\log_{\var{basis}}\frac{\var{kn}}{\var{k0}}=\frac{\log_{10}\frac{\var{kn}}{\var{k0}}}{\log_{10}\var{basis}}=\var{n}$ Jahre
          \end{question}     
     \end{problem}



\embedmathlet{gwtmathlet}

\end{content}
