\documentclass{mumie.problem.gwtmathlet}
%$Id$
\begin{metainfo}
  \name{
    \lang{en}{...}
    \lang{de}{A05: Vergleich Zinsarten}
    \lang{zh}{...}
    \lang{fr}{...}
  }
  \begin{description} 
 This work is licensed under the Creative Commons License Attribution 4.0 International (CC-BY 4.0)   
 https://creativecommons.org/licenses/by/4.0/legalcode 

    \lang{en}{...}
    \lang{de}{...}
    \lang{zh}{...}
    \lang{fr}{...}
  \end{description}
  \corrector{system/problem/GenericCorrector.meta.xml}
  \begin{components}
    \component{js_lib}{system/problem/GenericMathlet.meta.xml}{gwtmathlet}
  \end{components}
  \begin{links}
  \end{links}
  \creategeneric
\end{metainfo}
\begin{content}
\lang{de}{\title{A05: Vergleich Zinsarten}}
\lang{en}{\title{A05: Comparison of interest types}}
\begin{block}[annotation]
	Im Ticket-System: \href{https://team.mumie.net/issues/22679}{Ticket 22679}
\end{block}


\usepackage{mumie.genericproblem}


\begin{problem}
     \begin{variables}
        \randint{p}{1}{5}
        \randint{n}{3}{7}
        \randint{mal}{2}{3}
        \function[calculate]{k0}{mal*10000}
        \function[calculate,2]{ke}{k0*(1+n*p/100)}
        \function[calculate,2]{k1}{k0*(1+p/100)^n}
        \function[calculate,2]{k2}{k0*(1+p/200)^(2*n)}
        \function[calculate,2]{k4}{k0*(1+p/400)^(4*n)}
        \function[calculate,2]{ks}{k0*e^(n*p/100)}
     \end{variables}
     

        \begin{question}
          \type{input.number}
          \lang{de}{\text{Ein Sparer legt $\var{k0}$\euro bei einer Bank 
          zu $\var{p}$\% Zinsen für $\var{n}$ Jahre an. 
          Welchen Betrag erhält er bei:\\
         \begin{itemize}
          \item[\textbf{(a)}]
          einfacher Verzinsung: $K_e=$\ansref \euro\\
         \item[\textbf{(b)}]
         bei unterjähriger Verzinsung:  $K_{\var{n};1}=$ \ansref \euro\\
         bei halbjährlicher Verzinsung:  $K_{\var{n};2}=$ \ansref \euro\\
         bei quartalsmäßiger Verzinsung: $K_{\var{n};4}=$ \ansref \euro \\
         \item[\textbf{(c)}]
         bei stetiger Verzinsung: $K_s=$ \ansref \euro
         \end{itemize}
       }}
       \lang{en}{\text{A customer deposits $\var{k0}$\euro with a bank 
          at $\var{p}$\% interest for $\var{n}$ years. 
          What amount will he have assuming:\\
           \begin{itemize}
          \item[\textbf{(a)}]
          simple interest: $K_e=$\ansref \euro\\
         \item[\textbf{(b)}]
         annually compounded interest: $K_{\var{n};1}=$ \ansref \euro\\
         half-yearly compounded interest: $K_{\var{n};2}=$ \ansref \euro\\
         quarterly compounded interest: $K_{\var{n};4}=$ \ansref \euro \\
         \item[\textbf{(c)}]
         continuous interest: $K_s=$ \ansref \euro
         \end{itemize}
       }}
       
       
       
       \begin{answer}
                    \solution{ke}
               \end{answer}
            \begin{answer}
                \solution{k1}
            \end{answer}
            \begin{answer}
                \solution{k2}
            \end{answer}
            \begin{answer}
                \solution{k4}
            \end{answer}
            \begin{answer}
               \solution{ks}
            \end{answer}
               \lang{de}{\explanation[edited(ans_1)]{Die anfallenden Zinsen müssen nur addiert werden.}}
               \lang{en}{\explanation[edited(ans_1)]{The accrued interest only has to be added}}
               %$K_e=\var{k0}(1+\frac{\var{n}\cdot\var{p}}{100})=\var{ke}$\euro
               \lang{de}{\explanation[edited(ans_2)]{Achten Sie darauf, wie oft verzinst wird.}}
               \lang{en}{\explanation[edited(ans_2)]{Pay attention to how often interest is paid}}
               %Es ist: 
         %\begin{eqnarray*}
         %K_{\var{n};1}&=&\var{k0}(1+\frac{\var{p}}{100})^{\var{n}}=\var{k1}\text{ \euro}\\
         %K_{\var{n};2}&=&\var{k0}(1+\frac{\var{p}}{200})^{\var{n}\cdot 2}=\var{k2}\text{ \euro}\\
         %K_{\var{n};4}&=&\var{k0}(1+\frac{\var{p}}{400})^{\var{n}\cdot 4}=\var{k4}\text{ \euro}
        %\end{eqnarray*}                     
         
                \lang{de}{\explanation[edited(ans_5)]{Hier muss ein exponentieller Term ausgewertet werden.}}
                \lang{en}{\explanation[edited(ans_5)]{An exponential term must be evaluated here.}}
                %$K_s=\var{k0}\cdot e^{\frac{\var{p}\cdot\var{n}}{100}}=\var{ks}$ \euro
        \end{question}
          
  %      \begin{question}
  %       \type{input.number}
  %       \text{\\bei unterjähriger Verzinsung:  $K_{\var{n};1}=$ \ansref \euro\\
  %       bei halbjährlicher Verzinsung:  $K_{\var{n};2}=$ \ansref \euro\\
  %       bei quartalsmäßiger Verzinsung: $K_{\var{n};4}=$ \ansref \euro
  %        }
  %       \begin{answer}
  %         \solution{k1}
  %       \end{answer}
  %       \begin{answer}
  %         \solution{k2}
  %       \end{answer}
  %       \begin{answer}
  %         \solution{k4}
  %       \end{answer}
  %       \explanation{Es ist: 
  %       \begin{eqnarray*}
  %       K_{\var{n};1}&=&\var{k0}(1+\frac{\var{p}}{100})^{\var{n}}=\var{k1}\text{ \euro}\\
  %       K_{\var{n};2}&=&\var{k0}(1+\frac{\var{p}}{200})^{\var{n}\cdot 2}=\var{k2}\text{ \euro}\\
  %       K_{\var{n};4}&=&\var{k0}(1+\frac{\var{p}}{400})^{\var{n}\cdot 4}=\var{k4}\text{ \euro}
  %       \end{eqnarray*}                     
  %       }
  %     \end{question}
  %     
  %     \begin{question}
  %         \type{input.number}
  %         \text{\\bei stetiger Verzinsung: $K_s=$\ansref \euro}
  %         \begin{answer}
  %             \solution{ks}
  %         \end{answer}
  %      \explanation[edited(ans_1)]{$K_s=\var{k0}\cdot e^{\frac{\var{p}\cdot\var{n}}{100}}=\var{ks}$ \euro}
  %      \end{question}
\end{problem}

\embedmathlet{gwtmathlet}

\end{content}