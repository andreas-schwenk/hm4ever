\documentclass{mumie.problem.gwtmathlet}
%$Id$
\begin{metainfo}
  \name{
    \lang{en}{...}
    \lang{de}{A03: exakte Annuität}
    \lang{zh}{...}
    \lang{fr}{...}
  }
  \begin{description} 
 This work is licensed under the Creative Commons License Attribution 4.0 International (CC-BY 4.0)   
 https://creativecommons.org/licenses/by/4.0/legalcode 

    \lang{en}{...}
    \lang{de}{...}
    \lang{zh}{...}
    \lang{fr}{...}
  \end{description}
  \corrector{system/problem/GenericCorrector.meta.xml}
  \begin{components}
    \component{js_lib}{system/problem/GenericMathlet.meta.xml}{gwtmathlet}
  \end{components}
  \begin{links}
  \end{links}
  \creategeneric
\end{metainfo}
\begin{content}
\title{A03: exakte Annuität}

\begin{block}[annotation]
	Im Ticket-System: \href{https://team.mumie.net/issues/22696}{Ticket 22696}
\end{block}

\usepackage{mumie.genericproblem}


     \begin{problem}
          \begin{variables}
                %\number{S}{10000}
                %\number{p}{2.25}
                \drawFromSet{S}{10000,20000,30000}
                \drawFromSet{p}{2,2.25,3,3.25,4,4.25,5,5.25}
                \function[calculate]{ijahr}{4*p}
                \function{q}{1+p/100}
                \function[calculate,2]{A}{S*q^8*(q-1)/(q^8-1)}
                \function{T1}{S*(q-1)/(q^8-1)}
                \function{T2}{T1*q}
                \function{T3}{T1*q^2}
                \function{T4}{T1*q^3}
                \function{Z1}{A-T1}
                \function{Z2}{A-T2}
                \function{Z3}{A-T3}
                \function{Z4}{A-T4}
                \function{RS1}{S}
                \function{RS2}{S-T1}
                \function{RS3}{S-T1(q+1)}
                \function{RS4}{S-T1(q^3-1)/(q-1)}
                \function{RS5}{S-T1(q^4-1)/(q-1)}
          \end{variables}
          \begin{question}
               \type{input.number}
               \text{Eine Kreditsumme von $\var{S}$ \euro soll mit einem Jahreszins
               von $i=\var{ijahr}$ \% in 2 Jahren abbezahlt werden. Die Zahlungen erfolgen vierteljährlich,
               so dass mit einem Periodenzinssatz von $i_k=\var{p}$\% zu rechnen ist. \\
               
               Erstellen Sie den Tilgungsplan für das erste Jahr.}

     \begin{table}
     \head
     k&$S_{k-1}$&$Z_k$&$T_k$&$A$&$S_k$
     \body
     \textbf{1} & \ansref & \ansref & \ansref & \ansref & \ansref\\
     \textbf{2}&\ansref&\ansref&\ansref&\ansref&\ansref\\
     \textbf{3}&\ansref&\ansref&\ansref&\ansref&\ansref\\
     \textbf{4}&\ansref&\ansref&\ansref&\ansref&\ansref
     \end{table} 
               
           %1.
               \begin{answer}
               \solution{RS1}
               \end{answer}
               
               \begin{answer}
               \solution{Z1}
               \end{answer}
               
               \begin{answer}
               \solution{T1}
               \end{answer}
               
               \begin{answer}
               \solution{A}
               \end{answer}
               
               \begin{answer}
               \solution{RS2}
               \end{answer}
               
           %2.
               \begin{answer}
               \solution{RS2}
               \end{answer}
               
               \begin{answer}
               \solution{Z2}
               \end{answer}
               
               \begin{answer}
               \solution{T2}
               \end{answer}
               
               \begin{answer}
               \solution{A}
               \end{answer}
               
               \begin{answer}
               \solution{RS3}
               \end{answer}
               
           %3.
               \begin{answer}
               \solution{RS3}
               \end{answer}
               
               \begin{answer}
               \solution{Z3}
               \end{answer}
               
               \begin{answer}
               \solution{T3}
               \end{answer}
               
               \begin{answer}
               \solution{A}
               \end{answer}
               
               \begin{answer}
               \solution{RS4}
               \end{answer}
               
           %4.
               \begin{answer}
               \solution{RS4}
               \end{answer}
               
               \begin{answer}
               \solution{Z4}
               \end{answer}
               
               \begin{answer}
               \solution{T4}
               \end{answer}
               
               \begin{answer}
               \solution{A}
               \end{answer}
               
               \begin{answer}
               \solution{RS5}
               \end{answer}
               
               \explanation[edited]{Zuallererst muss die Annuität $A=S_0\frac{q^n(q-1)}{q^n-1}$ berechnet werden. Hier ist mit
               $n=8=2\cdot4$ wegen der vierteljährlichen Betrachtung zu rechnen.
               Um Rundungsfehler zu vermeiden, bietet es sich an, im Taschenrechner die exakten Zwischenergebnisse zu speichern und bei
               der weiteren Rechnung wiederzuverwenden.
               }
               \end{question}
           
     \end{problem}

\embedmathlet{gwtmathlet}

\end{content}
