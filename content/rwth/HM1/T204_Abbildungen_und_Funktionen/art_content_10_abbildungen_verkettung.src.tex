%$Id:  $
\documentclass{mumie.article}
%$Id$
\begin{metainfo}
  \name{
    \lang{de}{Abbildungen und Verkettungen}
    \lang{en}{}
  }
  \begin{description} 
 This work is licensed under the Creative Commons License Attribution 4.0 International (CC-BY 4.0)   
 https://creativecommons.org/licenses/by/4.0/legalcode 

    \lang{de}{Beschreibung}
    \lang{en}{}
  \end{description}
  \begin{components}
    \component{generic_image}{content/rwth/HM1/images/g_tkz_T204_Map_C.meta.xml}{T204_Map_C}
    \component{generic_image}{content/rwth/HM1/images/g_tkz_T204_Map_B.meta.xml}{T204_Map_B}
    \component{generic_image}{content/rwth/HM1/images/g_tkz_T204_Map_A.meta.xml}{T204_Map_A}
    \component{generic_image}{content/rwth/HM1/images/g_tkz_T204_Graph_C.meta.xml}{T204_Graph_C}
    \component{generic_image}{content/rwth/HM1/images/g_tkz_T204_Graph_B.meta.xml}{T204_Graph_B}
    \component{generic_image}{content/rwth/HM1/images/g_tkz_T204_Graph_A.meta.xml}{T204_Graph_A}
    \component{generic_image}{content/rwth/HM1/images/g_img_00_Videobutton_schwarz.meta.xml}{00_Videobutton_schwarz}
    \component{generic_image}{content/rwth/HM1/images/g_img_00_video_button_schwarz-blau.meta.xml}{00_video_button_schwarz-blau}
  \end{components}
  \begin{links}
    \link{generic_article}{content/rwth/HM1/T204_Abbildungen_und_Funktionen/g_art_content_11_injektiv_surjektiv_bijektiv.meta.xml}{content_11_injektiv_surjektiv_bijektiv}
    \link{generic_article}{content/rwth/HM1/T204_Abbildungen_und_Funktionen/g_art_content_10_abbildungen_verkettung.meta.xml}{content_10_abbildungen_verkettung}
    \link{generic_article}{content/rwth/HM1/T102neu_Einfache_Reelle_Funktionen/g_art_content_06_funktionsbegriff_und_lineare_funktionen.meta.xml}{funktionsbegriff_und_lineare_funktionen}
% alt    \link{generic_article}{content/rwth/HM1/T102_Einfache_Funktionen,_grundlegende_Begriffe/g_art_content_04_lineare_funktionen.meta.xml}{link-reelle-funktionen}
  \end{links}
  \creategeneric
\end{metainfo}
\begin{content}
\usepackage{mumie.ombplus}
\ombchapter{4}
\ombarticle{1}

\lang{de}{\title{Abbildungen und ihre Verkettungen}}
 
\begin{block}[annotation]
  Aus OMB+-Kapitel Logik kopiert (ohne Ergänzung "injektiv etc.", aber mit allgemeinerer Komposition)
  
\end{block}
\begin{block}[annotation]
  Im Ticket-System: \href{http://team.mumie.net/issues/9652}{Ticket 9652}\\
\end{block}

\begin{block}[info-box]
\tableofcontents
\end{block}


Im Rahmen dieses Kapitels wollen wir die Grundlagen der bereits bekannten reellen Funktionen 
aus 
% \link{link-reelle-funktionen}{Teil 1, Kapitel 2.1} 
\link{funktionsbegriff_und_lineare_funktionen}{Funktionsbegriff reeller Funktionen}
auf Zuordnungen zwischen allgemeinen Mengen 
übertragen. Im Speziellen beschäftigen wir uns mit einer besonderen Form der Zuordnung, der sogenannten 
Abbildung.  




\section{Abbildungen}\label{sec:abbildungen}

\begin{definition}[Abbildung] \label{def:Abbildungen}
  Seien $M$ und $N$ Mengen. \\
  
  Eine Vorschrift $T$, die jedem $x$ aus der Menge $M$ \emph{genau ein} $y$ aus 
  der Menge $N$ zuordnet, heißt \emph{\notion{Abbildung}}$ \,$ (oder auch 
  \emph{\notion{Funktion}}, 
  \emph{\notion{Operator}} bzw. \emph{\notion{Transformation}}) von $M$ nach $N$.

  Das Element $y$, das einem Element $x$ zugeordnet wird, nennt man dann \notion{\emph{Bild}} oder 
  \emph{\notion{Funktionswert} von $x$ unter der
  Abbildung $T, \,$} und schreibt für dieses Element oft $T(x)$.

  Für $\,$ \emph{"`$T$ ist eine Abbildung von $M$ nach $N$"'} $\,$ schreibt man
  \begin{center}
    $ T:M\to N \quad $ oder ausführlicher $\quad T:M\to N,\, x\mapsto T(x).$
  \end{center}
%
%%%%%%%%%%%%%%%%%%%%%%%%%%%%%%
\end{definition}

\begin{example}\label{example:abbildungen}
\begin{enumerate}
\item Aus der Schule bekannte reelle Funktionen
(vgl. 
% \link{link-reelle-funktionen}{Abschnitt "`Lineare Funktionen"'}), 
\link{funktionsbegriff_und_lineare_funktionen}{Abschnitt "`Reelle Funktionen"'}), 
wie
\begin{enumerate}
\item $f:\R\to \R,\, x\mapsto x^2$,
\item $g:\R\setminus \{0\}\to \R\setminus \{0\},\, x\mapsto \frac{1}{x}$.
\end{enumerate}
\item Die Identitätsabbildung $\text{id}_M$ auf einer Menge $M$, die jedem Element $x$ das Element selbst zuordnet. In Formeln
\[  \text{id}_M:M\to M, x\mapsto x. \]
\item Sei $S$ die Menge der Studenten an der Universität X. Dann ist 
\[ m:S\to \N, s\mapsto \text{Matrikelnummer des Studenten }s \]
eine Abbildung von der Menge $S$ in die Menge der natürlichen Zahlen. 
\item Für $M=\{0;1;3\}$ und $N=\{1;2\}$ ist die Vorschrift $T$:
\[ 0\mapsto 1,\quad 1\mapsto 1,\quad 3\mapsto 2  \]
eine Abbildung.
\end{enumerate}
\end{example}

Nach diesen Beispielen für Abbildungen stellt sich die Frage: Welche Vorschriften sind denn keine Abbildungen?

\begin{example}\label{example:keine-abbildung}
Für $M=\{0;1;3\}$ und $N=\{1;2\}$ ist die Vorschrift $F$
\begin{center}"{}Ordne jeder Zahl $x$ in $M$ die Zahlen in $N$ zu, die größer als $x$ sind."\end{center}
keine Abbildung. Dies hat sogar zwei Gründe. Zum einen werden nach dieser Vorschrift der Zahl $0$ in $M$ zwei Zahlen
in $N$ zugeordnet, nämlich die $1$ und die $2$. Bei einer Abbildung $F:M\to N$ darf jedem Element im $M$ aber nur \emph{ein} Element
in $N$ zugeordnet werden. Zum anderen wird nach dieser Vorschrift der Zahl $3$ in $M$ keine Zahl in $N$ zugeordnet. Bei einer
Abbildung $F:M\to N$ muss aber \emph{jedem} Element in $M$ ein Element in $N$ zugeordnet werden.
\end{example}

\begin{quickcheck}
    \text{Welche der folgenden Vorschriften ist eine Abbildung?}
    \begin{choices}{multiple}
        \begin{choice}
            \text{$S:\{x\in \R\,|\, x>0\} \to \R, \; x\mapsto \pm \sqrt{x}$}
            \solution{false}
		\end{choice}
        \begin{choice}
            \text{$T:\R\to \Z, \; r\mapsto $ kleinste ganze Zahl $\geq r$}
            \solution{true}
		\end{choice}
 \end{choices}
 \explanation{Es gilt $\, S(1)=\pm 1$, d.h. dem Element $1 \in \{x\in \R\,|\, x>0\} \,$ 
        werden zwei Zahlen in $\R$ zugeordnet. Daher ist $S$ keine Abbildung. 
        $T$ hingegen ist eine wohldefinierte Abbildung, denn sie rundet auf die nächstgrößere 
        ganze Zahl. So ist z.B. $\, T(\frac{1}{2})=1$.}

\end{quickcheck}

\begin{definition}[Definitionsbereich, Zielbereich und Wertemenge]\label{def:Wertemenge}
Für eine Abbildung $T:M\to N$ heißt die Menge $M$ der \emph{\notion{Definitionsbereich}} von $T$ (auch bezeichnet mit $D_T$)
und $N$ der \emph{\notion{Zielbereich}} von $T$. Der Zielbereich sollte nicht mit der \emph{\notion{Wertemenge}} von $T$ (bezeichnet mit 
$W_T$) verwechselt werden, welcher definiert ist als
\[ W_T=\{ y\in N\,|\, \text{es gibt ein }x\in M\text{ mit }T(x)=y \}.\]
Die Wertemenge (auch bekannt als \emph{\notion{Bildmenge}}) ist also eine Teilmenge des Zielbereichs $N$. Ihre Elemente sind genau diejenigen Elemente in $N$, welche 
Funktionswerte von Elementen $x$ aus $M$ unter der Abbildung $T$ sind.
\end{definition}

\begin{remark}
Die Begriffe \emph{Zielbereich}, \emph{Wertebereich bzw. Wertemenge} und \emph{Bildbereich bzw. Bildmenge} sind
in der Literatur nicht einheitlich definiert. Wir verwenden die Begriffe gemäß der oben aufgeführten Definition \ref{def:Wertemenge}. 
\end{remark}

\begin{definition}[Urbild]\label{def:urbild}
  Ist $T:M\to N$ eine Abbildung, dann heißt für $y\in N$ die Menge
  \[ T^{-1}(y):=\{x\in M| T(x)=y\}\]
  das \textbf{Urbild} von $y$.
  Analog heißt für eine Teilmenge $U\subset N$ die Menge
  \[T^{-1}(U):=\{x\in M| T(x)\in U\}\]
  das Urbild von $U$.
\end{definition}

\begin{remark}
     Die Notation $T^{-1}$ in der Definition des Urbilds \ref{def:urbild} ist identisch mit der
     Notation für die
     \ref[content_11_injektiv_surjektiv_bijektiv][Umkehrabbildung]{sec:umkehrabbildung}.\\
     Sie kann zur Darstellung des Urbildes auch dann verwendet werden, wenn die Umkehrabbildung
     nicht existiert.
\end{remark}

\begin{example}\label{example:bereiche}
\begin{enumerate}
\item Die Abbildung $f:\R\to \R,\, x\mapsto x^2$ hat als Definitionsbereich und Zielbereich $\R$. Die Wertemenge besteht
jedoch nur aus den reellen Zahlen, die $\geq 0$ sind, d.h. $W_f=[0,\infty)$.\\
Das Urbild für die Menge $U_1=\{1\}$ ist beispielsweise $f^{-1}(U_1)=\{-1,1\}$. 
Das Urbild für die Menge $U_2=\R_{<0}$ ist beispielsweise leer.
\item Die Abbildung $g:\R\setminus \{0\}\to \R^*,\, x\mapsto \frac{1}{x}$ hat als Definitionsbereich und 
Zielbereich die Menge $\R\setminus \{0\}$. Hier ist die Wertemenge sogar gleich dem Zielbereich, weil jede Zahl $y\ne 0$
der Funktionswert einer Zahl $x$ aus der Menge $\R\setminus \{0\}$ ist, nämlich von der Zahl $x=\frac{1}{y}$, d.h. $W_g=\R\setminus \{0\}$.
\item Für die Abbildung
\[ T:S\to \N, s\mapsto \text{Matrikelnummer des Studenten }s \]
aus Beispiel \ref{example:abbildungen}, ist die Menge $S$, also die Menge der 
Studenten an der Universität X, der
Definitionsbereich und die Menge aller natürlichen Zahlen $\N$ ist der Zielbereich.
Die Wertemenge besteht jedoch nur aus den natürlichen Zahlen, die tatsächlich als Matrikelnummer vergeben sind.
\end{enumerate}
\end{example}

\begin{block}[warning]
Beim Rechnen mit reellen Funktionen wird oft nur die Funktionsvorschrift angegeben, z.B. $f(x)=x^2$ oder $g(x)=\frac{1}{x}$, 
ohne Definitionsbereich und Zielbereich explizit zu nennen. Man impliziert damit dann, dass der Zielbereich $\R$ ist, und 
der Definitionsbereich der sogenannte \emph{maximale Definitionsbereich} ist, d.h. die Teilmenge der reellen Zahlen, 
die als Wert f"ur die Variable $x$ in die Funktionsvorschrift $f(x)$ bzw. $g(x)$ eingesetzt werden d"urfen
(vgl. 
% \ref[link-reelle-funktionen][Abschnitt "`Lineare Funktionen"']{max-defbereich}).
\ref[funktionsbegriff_und_lineare_funktionen][Abschnitt "`Reelle Funktionen"']{max-defbereich}).

%Im 1. Beispiel von \ref{example:bereiche} wäre der Definitionsbereich von $f$ also 
%gleich $\R$. Der Definitionsbereich von $g$ im 2. Beispiel von \ref{example:bereiche} 
%lautet $D_g=\R\setminus \{0\}=\R^*$.
\end{block}
Eine kurze Zusammenfassung des Abbildungsbegriffs kann dem folgenden Video entnommen werden:\\
\floatright{\href{https://api.stream24.net/vod/getVideo.php?id=10962-2-10806&mode=iframe&speed=true}{\image[75]{00_video_button_schwarz-blau}}}\\


\section{Grafische Darstellung von Abbildungen}\label{sec:graphik}

Um Abbildungen anschaulich darzustellen, gibt es im Wesentlichen zwei Möglichkeiten, wobei nicht jede der beiden
Möglichkeiten immer verwendet werden kann.

Eine Möglichkeit ist es, den sogenannten \emph{Graph einer Abbildung} zu zeichnen/skizzieren, was insbesondere bei 
Abbildungen verwendet wird, deren Definitions- und Zielbereich Teilmengen der reellen Zahlen sind.

\begin{definition}[Graph]
Sei $f:M\to N$ eine Abbildung.
Der \emph{Graph der Abbildung $f$} ist die Menge aller Paare $(x;y)$ in $M\times N$, für die $y=f(x)$ gilt, also
\[ \text{Graph}(f)=\{ (x;y)\in M\times N\,|\, y=f(x) \} = \{ (x;f(x))\in M\times N\,|\, x\in M \}.  \]
\end{definition}

\begin{example}
Für die Abbildung $f_1:\R\to \R,x\mapsto x^2$ ist der Graph von $f_1$ die Menge
\[ \text{Graph}(f_1)= \{ (x; x^2)\,|\, x\in \R\} \subseteq \R^2,  \]
da ja $f_1(x)=x^2$ für jedes $x\in \R$ gilt. Die Punkte des Graphen sind also genau die Punkte auf der Normalparabel, welche
in der Zeichnung orange eingezeichnet ist. 

\begin{center}
\image{T204_Graph_A}
\end{center}

Entsprechend sind die grüne und die blaue Kurve, die Punkte, die auf dem Graphen der Funktion 
$f_2:\R\to \R, x\mapsto x^2+2$ bzw. der Funktion $f_3:\R\to \R, x\mapsto x^2-1$ liegen.
\end{example}

\begin{example}
Man betrachte die Abbildung $f:\N\to \{ -1; 1\}, n\mapsto (-1)^n$. Definitionsbereich und Zielbereich sind beides Teilmengen
von $\R$, weshalb wir auch hier wieder den Graph als Punkte in der Ebene darstellen können wie in folgender Zeichnung:

\begin{center}
\image{T204_Graph_B}
\end{center}

\end{example}

\begin{remark}
Aus der graphischen Darstellung einer Zuordnung von einer Teilmenge von $\R$ in eine Teilmenge von $\R$ kann man Folgendes visuell ablesen: %An dem Graphen im $\R^2$ einer Zuordnung von einer Teilmenge von $\R$ in eine Teilmenge von $\R$ kann man auch einiges ablesen:
\begin{enumerate}
\item Der Definitionsbereich $D$ ist die Menge aller $x\in \R$, f"ur die die senkrechte Gerade durch den Punkt $(x;0)$ den Graphen schneidet.
\item Die Zuordnung ist genau dann eine Abbildung, wenn f"ur jedes $x\in D$ die senkrechte Gerade durch den Punkt $(x;0)$, den 
Graphen in genau einem Punkt schneidet.
\end{enumerate}
\end{remark}

\begin{example}
In der folgenden Grafik beschreibt die blaue Kurve keinen Graphen einer Abbildung. Dies wird ersichtlich, wenn bspw. die senkrechte Gerade an der $x$-Koordinate $x=4$ betrachtet wird. Die Gerade schneidet die blaue Kurve an zwei Stellen, folglich liegt keine Abbildung vor.\\ %denn zum Beispiel die senkrechte Gerade durch den Punkt $(4;0)$ schneidet die Kurve in zwei Punkten.\\
Die rote Kurve beschreibt jedoch eine Abbildung, da jede senkrechte Gerade die Kurve in h"ochstens einem Punkt schneidet. Da eine
senkrechte Gerade die rote Kurve genau dann schneidet, wenn die $x$-Koordinate zwischen $-2$ und $2$ (jeweils einschließlich) liegt,
ist der Definitionsbereich die Menge $\{x\in \R\,|\, -2\leq x\leq 2 \}$, also das Intervall $[-2;2]$. 

\begin{center}
\image{T204_Graph_C}
\end{center}

\end{example}


Die einleitend erwähnte zweite Möglichkeit, Abbildungen grafisch darzustellen, wird vor allem bei Abbildungen zwischen
endlichen Mengen verwendet. Hierbei werden die zwei Mengen durch Aufzählung in einer kreisähnlichen Fläche 
%(ähnlich wie bei den \ref[link-mengen][Mengen-Diagrammen]{sec:mengen-diagramme}) 
dargestellt und die Zuordnung mittels Pfeilen zwischen den Elementen.

\begin{example}\label{example-abb-graphik}
Für $M=\{0;1;3\}$ und $N=\{1;2\}$ und die Abbildung $T$ mit
\[ 0\mapsto 1,\quad 1\mapsto 1,\quad 3\mapsto 2  \]
ist die grafische Darstellung:

\begin{center}
\image{T204_Map_A}
\end{center}
\end{example}

Anhand der grafischen Darstellung kann man direkt erkennen, ob eine Zuordnung wirklich eine Abbildung ist.
\begin{rule}
Die Zuordnung ist eine Abbildung, wenn \textbf{bei jedem Element} des Definitionsbereichs \textbf{genau ein Pfeil startet}.
\end{rule}

\begin{example}\label{example-graphik-nicht-abbildung}
Die grafische Darstellung der Zuordnung aus Beispiel \ref{example:keine-abbildung} ist die Folgende.
Es ist leicht zu erkennen, dass diese Zuordnung keine Abbildung ist. Ein Grund ist, dass bei der Zahl $0$ zwei Pfeile starten,
ein anderer Grund ist, dass bei der Zahl $3$ kein Pfeil startet. Beides ist 
% - QS-Vorschlag
gemäß Definition \ref{def:Abbildungen}
%
bei Abbildungen nicht zulässig.

\begin{figure}
\image{T204_Map_B}
\caption{Eine Zuordnung, die keine Abbildung ist.}
\end{figure}
\end{example}

\section{Komposition von Abbildungen}

\begin{definition}[Komposition]\label{kompositionv}
Sind $T:L\to M$ und $S:M\to N$ Abbildungen. \\
Dann ist die \emph{\notion{zusammengesetzte Abbildung} $S\circ T$},
gelesen "`$S$ nach $T$"' (auch \emph{\notion{Komposition der Abbildungen}} $T$ und $S$ genannt), eine Abbildung von $L$ nach 
$N$ definiert durch
\[    S\circ T: L\to N, x\mapsto S(T(x)). \]
Man bildet also zunächst ein Element in $L$ mittels $T$ auf ein Element in $M$ ab und anschließend dieses Element
mittels $S$ auf ein Element in $N$.\\

Allgemeiner definiert man die Komposition zweier Abbildungen $T:L\to M$ und $S:M'\to N$ auch dann, wenn die \emph{Werte}menge von $T$ lediglich
eine \emph{Teil}menge des Definitionsbereich von $S$ ist. In diesem Fall ist nämlich $T(x)\in M'$ für alle $x\in L$, weshalb $S(T(x))$ 
ein wohldefinierte Ausdruck
und daher
\[    S\circ T: L\to N, x\mapsto S(T(x)) \]
eine wohldefinierte Abbildung ist.
\end{definition}

\begin{remark}
Manche Autoren verwenden die Begriffe \emph{Hintereinanderausführung} oder \emph{Verkettung}. Diese Begriffe sind äquivalent zu \emph{Komposition}.
\end{remark}

% \begin{block}[warning]
% Bei der Komposition $S\circ T$ zweier Abbildungen $S$ und $T$ muss immer sicher gestellt sein, dass $T$ in den (oder einen Teil des) Definitionsbereich(es) von $S$ abbildet.
% \end{block}

\begin{example}\label{ex:kompositionen}
\begin{enumerate}
\item Seien $f:\R\to \R, x\mapsto x^2$ und $g:\R\to \R, x\mapsto x+1$, dann sind
\[   (g\circ f)(x)=g(f(x))=g(x^2)=x^2+1 \]
 und 
\[  (f\circ g)(x)=f(g(x))=f(x+1)=(x+1)^2. \]
für alle reellen Zahlen $x$.
\item Für $L=\{0; 2\}$, $M=\{0;1;3\}$ und $N=\{1;2\}$ mit den Abbildungen $T:L\to M$ gegeben durch
\[ 0\mapsto 3,\quad 2\mapsto 0 \]
und $S:M\to N$ gegeben durch
\[ 0\mapsto 1,\quad 1\mapsto 1,\quad 3\mapsto 2  \]
ist die zusammengesetzte Abbildung gegeben durch
\[ 0\mapsto S(T(0))=S(3)=2,\quad 2\mapsto S(T(2))=S(0)=1. \]
Grafisch sieht dies folgendermaßen aus:

\begin{center}
\image{T204_Map_C}
\end{center}

Die Pfeile für $S\circ T$ erhält man also, wenn ausgehend von der Menge $L$ den Pfeilen gefolgt 
wird. Hier im Beispiel wären das die Pfade \emph{von der $0$ über die $3$ zur $2$} und 
\emph{von der $2$ über die $0$ zur $1$}.
\end{enumerate}
\end{example}

\begin{block}[warning]
Sind $f$ und $g$ zwei Abbildungen, dann sind im Allgemeinen $f\circ g$ und $g\circ f$ \textbf{nicht} gleich, selbst wenn
beide Kompositionen definiert sind.\\
Dies ist in Punkt 1 des vorigen Beispiels gut zu sehen.
\end{block}




\begin{quickcheck}
    \begin{variables}
      		\randint{a}{-3}{3}
      		\randint[Z]{b}{-2}{2}
      		\randint{c}{1}{4}
      		\randint[Z]{d}{-2}{2}
      		\function[normalize]{f}{x^2+a*x+b}
      		\function[normalize]{g}{c*x+d}
      		\function[normalize]{fg}{(c*x+d)^2+a*(c*x+d)+b}
      		\function[normalize]{gf}{c*(x^2+a*x+b)+d}
      \end{variables}

      \type{input.function}
      \field{real}
      \text{Bestimmen Sie die Komposition $f\circ g$ der Abbildungen\\
      $f:\R\to \R, x\mapsto \var{f}$ und $g:\R\to \R, x\mapsto \var{g}$.\\ $(f\circ g)(x)=$\ansref}
   
      \begin{answer}
          \solution{fg}
          \checkAsFunction{x}{-10}{10}{10}
      \end{answer}
\end{quickcheck}
Die folgenden Videos erläutern den Begriff der \ref[content_10_abbildungen_verkettung][Komposition]{kompositionv} nochmals anhand von kurzen Beispielen:\\
\floatright{
    \href{https://api.stream24.net/vod/getVideo.php?id=10962-2-10808&mode=iframe&speed=true}{\image[75]{00_video_button_schwarz-blau}}
    \href{https://www.hm-kompakt.de/video?watch=155}{\image[75]{00_Videobutton_schwarz}}
  }
\end{content}