%$Id:  $
\documentclass{mumie.article}
%$Id$
\begin{metainfo}
  \name{
    \lang{de}{Reelle Funktionen und Monotonie}
    \lang{en}{}
  }
  \begin{description} 
 This work is licensed under the Creative Commons License Attribution 4.0 International (CC-BY 4.0)   
 https://creativecommons.org/licenses/by/4.0/legalcode 

    \lang{de}{Beschreibung}
    \lang{en}{}
  \end{description}
  \begin{components}
    \component{generic_image}{content/rwth/HM1/images/g_tkz_T204_Example_D.meta.xml}{T204_Example_D}
    \component{generic_image}{content/rwth/HM1/images/g_tkz_T204_Example_C.meta.xml}{T204_Example_C}
    \component{generic_image}{content/rwth/HM1/images/g_tkz_T204_Example_B.meta.xml}{T204_Example_B}
    \component{generic_image}{content/rwth/HM1/images/g_tkz_T204_Example_A.meta.xml}{T204_Example_A}
    \component{generic_image}{content/rwth/HM1/images/g_tkz_T106_SquareRootFunction.meta.xml}{T106_SquareRootFunction}
    \component{generic_image}{content/rwth/HM1/images/g_tkz_T106_Cubic.meta.xml}{T106_Cubic}
    \component{generic_image}{content/rwth/HM1/images/g_tkz_T106_LinearFunction_A.meta.xml}{T106_LinearFunction_A}
    \component{generic_image}{content/rwth/HM1/images/g_img_00_video_button_schwarz-blau.meta.xml}{00_video_button_schwarz-blau}
  \end{components}
  \begin{links}
    \link{generic_article}{content/rwth/HM1/T102neu_Einfache_Reelle_Funktionen/g_art_content_06_funktionsbegriff_und_lineare_funktionen.meta.xml}{reelle-funktionen}
% alt    \link{generic_article}{content/rwth/HM1/T102_Einfache_Funktionen,_grundlegende_Begriffe/g_art_content_04_lineare_funktionen.meta.xml}{reelle-funktionen}
    \link{generic_article}{content/rwth/HM1/T202_Reelle_Zahlen_axiomatisch/g_art_content_06_supremum_infimum.meta.xml}{beschr-mengen}
    \link{generic_article}{content/rwth/HM1/T202_Reelle_Zahlen_axiomatisch/g_art_content_05_anordnungsaxiome.meta.xml}{anordnung}
    \link{generic_article}{content/rwth/HM1/T202_Reelle_Zahlen_axiomatisch/g_art_content_07_vollstaendigkeit.meta.xml}{vollst}
    \link{generic_article}{content/rwth/HM1/T103_Polynomfunktionen/g_art_content_09_polynome.meta.xml}{polynome}
    \link{generic_article}{content/rwth/HM1/T103_Polynomfunktionen/g_art_content_10_polynomdivision.meta.xml}{poly-div}
  \end{links}
  \creategeneric
\end{metainfo}
\begin{content}
\usepackage{mumie.ombplus}
\ombchapter{4}
\ombarticle{3}

\lang{de}{\title{Reelle Funktionen und Monotonie}}
 
\begin{block}[annotation]
  Inhalt: Definition reelle Funktion (vgl. Vorkursteil, content_04) ; Verknüpfung reeller Funktionen ($+,-,\cdot,/$);
  Beschränktheit;
   Monotonie (vgl. auch diffrechnung/content_22), sowie Folgerung der Injektivität
    aus der Monotonie; (partielle) Umkehrabbildung
  
%  kopiert aus: OMB+-Kap. VI.1,Stens-Skript
\end{block}
\begin{block}[annotation]
  Im Ticket-System: \href{http://team.mumie.net/issues/9654}{Ticket 9654}\\
\end{block}

\begin{block}[info-box]
\tableofcontents
\end{block}

Im folgenden Abschnitt werden wir uns mit speziellen Abbildungen beschäftigen, 
nämlich mit den \emph{reellen Funktionen}, bei denen Definitions- und Zielbereich 
Teilmengen der reellen Zahlen sind. Diese wurden im Abschnitt 
% QS-Änderung (Vorschlag)
% \link{reelle-funktionen}{"`Reelle Funktionen"'} betrachtet und 
%
\ref[reelle-funktionen]["`Reelle Funktionen"']{sec:funktion} bereits eingeführt und 
werden hier nun weiter vertieft.

\section{\lang{de}{Reelle Funktionen}\label{funktion} }\label{sec:reelle_funk}

 \lang{de}{\begin{definition}[Reelle Funktion]
	 Eine Abbildung $f:D\to N$, für welche der Definitionsbereich $D$ und der
	 Zielbereich $N$ Teilmengen von $\R$ sind, heißt \emph{reelle Funktion}.
     \end{definition}
	 
     \begin{definition}[Nullstellen reeller Funktionen]
	 Die \emph{Nullstellen} einer reellen Funktion $f$ sind die Stellen $x\in D$ im Definitionsbereich,
		an denen die Funktion den Wert $f(x)=0$ annimmt.
	 \end{definition} }
	 
\begin{remark}	 
Da eine Vergrößerung des Zielbereichs für die meisten Belange unerheblich ist,
 werden wir fast immer annehmen, dass $N=\R$ ist.	 
\end{remark}

Die bekanntesten Beispiele für reelle Funktionen sind solche, die durch einfache Funktionsvorschriften definiert sind, 
wie 
% QS-Änderung (Vorschlag)
% \link{polynome}{Polynomfunktionen} oder \link{poly-div}{rationale Funktionen},
%
\ref[polynome][Polynomfunktionen]{sec:polynomfkt} oder 
\ref[poly-div][rationale Funktionen]{sec:rationale_fkt},
und deren Graphen sich leicht zeichnen lassen. 
Es gibt aber durchaus auch andere reelle Funktionen (siehe folgendes Beispiel \ref{ex:unstetige-funktionen}).

\begin{example}\label{ex:unstetige-funktionen}
\begin{enumerate}
\item Die durch die Gauß-Klammer gegebene Funktion
\[   [\cdot ]:\R\to \Z, x\mapsto [x]:=\max\{ n\in \Z | n\leq x\}, \]
ist eine reelle Funktion, deren Graph viele \emph{Sprünge} aufweist. 

\begin{center}
\image{T204_Example_A}
\end{center}

Die dicken Punkte an den Streckenenden sollen andeuten, dass dieser Endpunkt zum Graphen gehört,
während die anderen Endpunkte nicht zum Graphen gehören.


\item Die \emph{Dirichletsche Sprungfunktion}
\[  D:\R\to {\{0;1\}}, x \mapsto \begin{cases} 1 & \text{falls }x\in \Q \\
0 & \text{falls }x\notin \Q \end{cases}\]
ist ebenfalls eine reelle Funktion.
Ihr Graph lässt sich nicht zeichnen, da er genau aus den Punkten $(a;1)$ mit rationalen Zahlen $a$
und aus Punkten $(b;0)$ mit irrationalen Zahlen $b$ besteht. Beim Versuch, alle Punkte 
$(a;1)$ mit rationalen Zahlen $a$ zu markieren, würde man nämlich die komplette Gerade $y=1$ zeichnen,
und beim Versuch, alle Punkte 
$(b;0)$ mit irrationalen Zahlen $b$ zu markieren, würde man die komplette Gerade $y=0$ zeichnen.
\end{enumerate}
\end{example}

Reelle Funktionen können addiert, subtrahiert, multipliziert und in manchen Fällen
auch dividiert werden, indem man die Funktionswerte addiert, subtrahiert, multipliziert
 bzw. dividiert.

\begin{definition}[Verknüpfungen reeller Funktionen]\label{def:reelleFkt}
Sei $D\subseteq \R$ nicht-leer und $f,g:D\to \R$ reelle Funktionen 
und $r\in \R$.\\
Dann definiert man die reellen Funktionen
\begin{enumerate}
\item   $\; f+g:D\to \R$,\, $\; f-g:D\to \R \; $ und $\; r\cdot f:D\to \R \; $ durch
    \begin{eqnarray*}
       (f+g)(x) &:=& f(x)+g(x) \quad \text{für alle}\quad x\in D, \\
       (f-g)(x) &:=& f(x)-g(x) \quad  \text{für alle}\quad x\in D, \\
       (r\cdot f)(x) &:=& r\cdot f(x) \qquad \text{für alle}\quad x\in D,
     \end{eqnarray*}
 \item  $\; f\cdot g:D\to \R \;$ durch 
    \[  (f\cdot g)(x):=f(x) \cdot g(x) \quad \text{für alle}\quad x\in D. \]
 \item  $\;$ Falls $g(x)\neq 0$ für alle $x\in D$ ist, definiert man 
        $\frac{f}{g}:D\to \R \; $ durch
    \[  \left( \frac{f}{g} \right) (x):= \frac{f(x)}{g(x)} \quad  \text{für alle}\quad x\in D. \]
\end{enumerate}
Auf den rechten Seiten der Definitionen wird jeweils die Operation in $\R$ verwendet.
\end{definition}
Eine Zusammenfassung der bisher angesprochenen Themen kann dem folgenden Video entnommen werden:\\
\floatright{\href{https://api.stream24.net/vod/getVideo.php?id=10962-2-10889&mode=iframe&speed=true}{\image[75]{00_video_button_schwarz-blau}}}\\
\\

\begin{example}
Für $\; f:\R\to \R, \, x\mapsto 2x^3$, $\; g:\R\to \R, \, x\mapsto x^2+1\;$ 
und $\;r=\frac{1}{2} \,$ sind
  \begin{tabs*}[\initialtab{1}\class{example}]
    \tab{$f+g$}
      \begin{align*}
       f+g :\,\R\to \R,\, x\mapsto f(x)+g(x)=2 x^3+x^2+1
      \end{align*}
    \tab{$f-g$}
      \begin{align*}
       f-g :\,\R\to \R,\, x\mapsto f(x)-g(x)=2 x^3-x^2-1
      \end{align*}
    \tab{$r\cdot f$}
      \begin{align*}
       r\cdot f :\,\R\to \R,\, x\mapsto r\cdot f(x)=\frac{1}{2}\cdot 2 x^3=x^3
      \end{align*}
    \tab{$f\cdot g$}
      \begin{align*}
       f\cdot g :\,\R\to \R,\, x\mapsto f(x)\cdot g(x)=2x^3\cdot (x^2+1)=2x^5+2x^3
      \end{align*}
    \tab{$\frac{f}{g}$}    
      Da $g(x)\neq 0$ für alle $x\in \R$ gilt, ist auch $\frac{f}{g}$ auf $\R$ definiert 
      \[  \frac{f}{g}:\, \R\to \R,\, x\mapsto \frac{f(x)}{g(x)}=\frac{2x^3}{x^2+1} \]
  \end{tabs*}
\end{example}


\begin{remark}
\begin{enumerate}
\item Die so definierten Verknüpfungen auf der Menge der reellen Funktionen $D\to \R$
erfüllen die üblichen Rechenregeln, wie sich leicht nachrechnen lässt.
\item Oft definiert man auch Verknüpfungen für reelle Funktionen 
$f:D_1\to \R$ und $g:D_2\to \R$, deren Definitionsbereiche also nicht gleich sind. In diesem
Fall ist der Definitionsbereich von $f+g$, von $f-g$ und von $f\cdot g$ die Schnittmenge
$D_1\cap D_2$. Zu beachten ist jedoch, dass die Schnittmenge dann nicht leer sein sollte.

Ebenso kann man den Quotienten $\frac{f}{g}$ auch definieren, wenn die Funktionswerte
von $g$ nicht alle von Null verschieden sind. Der Definitionsbereich ist dann jedoch lediglich
die Menge $\{ x\in D_1\cap D_2 | g(x)\neq 0\}$. Auch hier ist zu beachten, dass für 
eine sinnvolle Funktion diese Menge nicht leer sein sollte.

Man erhält die gleichen Funktionen, wenn man zunächst $f$ und $g$ auf die Menge 
$D_1\cap D_2$ bzw. auf $\{ x\in D_1\cap D_2 | g(x)\neq 0\}$ einschränkt, d.h. als
Funktionen $D_1\cap D_2\to \R$ bzw. $\{ x\in D_1\cap D_2 | g(x)\neq 0\}\to \R$ auffasst,
und dann die obigen Definitionen für die eingeschränkten Funktionen verwendet.
Wir werden daher den allgemeinen Fall nicht weiter behandeln.
\end{enumerate}
\end{remark}


Eine Eigenschaft, die später noch wichtig wird, ist die Folgende:

\begin{definition}[Beschränktheit reeller Funktionen]\label{def:bounded}
%Eine reelle Funktion $f:D\to \R$ heißt \notion{nach oben beschränkt} bzw.
%\notion{nach unten beschränkt} bzw. \notion{beschränkt}, wenn die Wertemenge $W_f$ von $f$
%die bereits bekannten \link{beschr-mengen}{Eigenschaften} aufweist.
%Nach \textcolor{red}{oben beschränkt} bzw. nach \textcolor{green}{unten beschränkt} bzw. \textcolor{blue}{beschränkt} ist eine reelle Funktion dann, wenn es eine reelle Zahl $c\in \R$ gibt, so dass für alle $x\in D$
% \[  \textcolor{red}{f(x)\leq c} \quad \text{bzw.}\quad \textcolor{green}{f(x)\geq c}\quad \text{bzw.}\quad \textcolor{blue}{|f(x)|\leq c} \]
% gilt.
% 
%%%%%%%%%%%%%%% QS-Vorschlag %%%%%%%%%%%%%%%%%%%%%%%%%%%%%%%%%%%%%%%%%%%%%%%%%%%%%%%%%%%%%
%\\
%--------------- QS-Vorschlag -------------------------------------------------------------
%\\
Eine reelle Funktion $f:D\to \R\, $ heißt \emph{nach oben, nach unten} oder  
generell \emph{beschränkt}, wenn die Wertemenge $W_f$ von $f$ die entsprechende
\ref[beschr-mengen][Eigenschaft als Menge]{def:beschraenkt} aufweist.\\

Das bedeutet
\begin{description} 
 This work is licensed under the Creative Commons License Attribution 4.0 International (CC-BY 4.0)   
 https://creativecommons.org/licenses/by/4.0/legalcode 

    \item $f$ heißt \emph{\notion{nach oben beschränkt}}, wenn es eine Zahl $c \in \R$ gibt, 
        so dass $f(x) \leq c \;$ für alle $x \in D$ gilt.
    \item $f$ heißt \emph{\notion{nach unten beschränkt}}, wenn es eine Zahl $c \in \R$ gibt, 
        so dass $f(x) \geq c \;$ für alle $x \in D$ gilt.
    \item $f$ heißt \emph{\notion{beschränkt}}, wenn es eine Zahl $c \in \R$ gibt, 
        so dass $\abs{f(x)} \leq c \;$ für alle $x \in D$ gilt.        
\end{description}

%%%%%%%%%%%%% Ende QS-Vorschlag %%%%%%%%%%%%%%%%%%%%%%%%%%%%%%%%%%%%%%%%%%%%%%%%%%%%%%%%%%
\end{definition}

\begin{example}
  \begin{tabs*}[\initialtab{1}\class{example}]
      \tab{$f(x)=x^2$} Die Funktion $f:\R\to \R, x\mapsto x^2$ ist nach unten beschränkt, da $f(x)=x^2\geq 0$ für
      alle $x\in \R$. Sie ist jedoch nicht nach oben beschränkt.\\
      Diese Überlegungen werden auch vom Funktionsgraphen bestätigt. 
      \begin{center}
      \image{T204_Example_B}
      \end{center}
      \tab{$f(x)=\frac{1}{1+x^2}$} Die Funktion $f:\R\to \R, x\mapsto \frac{1}{1+x^2}$ ist beschränkt, denn für alle $x\in \R$ gilt 
      \begin{align*}
          x^2\geq 0\quad\Rightarrow\quad 1+x^2\geq 1>0 \quad\Rightarrow\quad f(x)=\frac{1}{1+x^2}>0
      \end{align*} 
      und zum anderen existiert ein Maximum bei $x_{max}=0$
      \begin{align*}
          x_{max}^2 = 0\quad\Rightarrow\quad 1+x_{max}^2= 1 \quad\Rightarrow\quad f(x_{max})=\frac{1}{1+x_{max}^2}= 1\leq 1.
      \end{align*} 
      Kombiniert man diese Erkenntnisse gilt für alle $x\in \R$: $|f(x)|\leq 1$. Diese Überlegungen werden auch vom Funktionsgraphen bestätigt. 
      \begin{center}
      \image{T204_Example_C}
      \end{center}
    \end{tabs*}
\end{example}

	 
\section{\lang{de}{Monotonie und Umkehrfunktion}\lang{en}{Monotonicity and inverse function}\label{Monotonie} }


  \lang{de}{\begin{definition}[monoton, streng monoton]\label{def:monotonie}
		 Eine reelle Funktion $f:D\to \R$ ist \notion{monoton wachsend},
		 wenn für ein beliebiges Paar $x_1, x_2\in D$, für die die Ungleichung $x_1 \leq x_2$ gilt, 
		 die Ungleichung $f(x_1)\leq f(x_2)$ für die zugehörigen Funktionswerte folgt.\\
		
		 Sie heißt \notion{streng monoton wachsend}, falls aus der strengeren Ungleichung $x_1 < x_2$ 
		 die strengere Ungleichung $f(x_1) < f(x_2)$ folgt.\\
		 
		 Eine reelle Funktion $f:D\to \R$  ist \notion{monoton fallend}, 
		 wenn für ein beliebiges Paar $x_1, x_2\in D$, für die die Ungleichung $x_1 \leq x_2$ gilt, 
		 die Ungleichung $f(x_1)\geq f(x_2)$ für die zugehörigen Funktionswerte folgt.\\
		 
		 Sie heißt \notion{streng monoton fallend}, falls die strengere Folgerung 
		 $x_1<x_2 \Rightarrow f(x_1)> f(x_2)$ gilt.
		 
		 Eine reelle Funktion heißt \notion{monoton}, wenn sie monoton wachsend oder fallend ist. 
	 \end{definition} }
	 
  \lang{en}{\begin{definition}[Monotonic, Strictly monotonic]
		 A function $f:D\to \R$ is said to be \notion{monotonically increasing} if for any given 
		 pair of arguments $x_1, x_2\in D$  such that $x_1 \leq x_2$,
		 the inequality $f(x_1)\leq f(x_2)$ is satisfied.
		 
		 A function $f:D\to \R$ is said to be \notion{strictly monotonically increasing} if, 
		 for the strict inequality $x_1 < x_2$ the strict inequality $f(x_1) < f(x_2)$ is satisfied. 
		 Note here that
		 strict is meant in an inequality sense: strictly less than vs. less than or equal to.
		 
		 A function $f:D\to \R$ is said to be \notion{monotonically decreasing} if for any given 
		 pair of arguments $x_1, x_2\in D$ such that $x_1 \leq x_2$,
		 the inequality $f(x_1)\geq f(x_2)$ is satisfied.
		 
		 A function $f:D\to \R$ is said to be \notion{strictly monotonically decreasing} if, 
		 for the strict inequality $x_1 < x_2$ the strict inequality $f(x_1) > f(x_2)$ is satisfied. 
		 Note here as well that
		 strict is meant in an inequality sense: strictly greater than vs. greater than or equal to.

		A function $f:D\to \R$ is called \notion{monotone} or \notion{monotonic} if it is either 
		monotonically increasing or monotonically decreasing.
	 \end{definition} }

\begin{example}\label{ex:monotone_funktionen}
\begin{tabs*}[\initialtab{1}\class{example}]
\tab{$f(x) = a x + b$}

	\lang{de}{Der Funktionsgraph einer linearen Funktion $\;f(x) = a x + b$ 
	    ist eine Gerade mit der Steigung $a$ und $y$-Achsenabschnitt
		$b$ (Schnittpunkt mit der $y$-Achse). Die linearen Funktionen $f(x) = ax+b$ sind \textcolor{#0066CC}{streng 
		monoton wachsend}, wenn die Steigung größer Null ist, d.h. \textcolor{#0066CC}{$a > 0$} und 
		\textcolor{#CC6600}{streng monoton fallend}, wenn die Steigung negativ ist, d.h. \textcolor{#CC6600}{$a<0$}. 
		Für $a=0$ sind sie konstant und damit sowohl monoton wachsend als auch monoton fallend. }
    
    \lang{en}{The graph of a linear function  $\;f(x)=a x + b$ is a line with slope $a$ and $y$-intercept $b$ (the intersection point with the $y$-axis).
    	Linear functions $f(x) = ax+b$ are \textcolor{#0066CC}{strictly monotonically increasing} if the slope is greater than zero \textcolor{#0066CC}{($a>0$)} and \textcolor{blue}{strictly monotonically decreasing} if the slope is negative \textcolor{blue}{($a<0$)}. 
    	If $a=0$ linear functions are constant and hence both monotonically increasing and monotonically decreasing. }
\begin{center}
\image{T106_LinearFunction_A}
\end{center}

\tab{$f(x) = a x^3$}
    \lang{de}{	Eine Funktion der Form $f(x) = a x^3$ (Polynom 3. Grades) ist \textcolor{#0066CC}{streng monoton wachsend}, wenn der Koeffizient 
		$a$ positiv ist, \textcolor{#0066CC}{$a > 0$}, und \textcolor{#CC6600}{streng monoton fallend},
		 wenn \textcolor{#CC6600}{$a < 0$} ist.}
	
	 \lang{en}{A function of the form $f(x) = ax^3$ (a third degree polynomial) is \textcolor{#0066CC}{strictly monotonically increasing} if the coefficient $a$ is positive \textcolor{#0066CC}{($a>0$)} and \textcolor{#CC6600}{strictly monotonically decreasing} if \textcolor{#CC6600}{$a<0$}.}
	 
\begin{center}
\image{T106_Cubic}
\end{center}
\tab{$f(x) = \sqrt{x}$}
    \lang{de}{Die Quadratwurzelfunktion $f(x) = \sqrt{x}$ ist \textcolor{#0066CC}{streng monoton wachsend}.}
    
     \lang{en}{The square root $f(x) = \sqrt{x}$ is \textcolor{#0066CC}{strictly monotonically increasing}.}
\begin{center}
\image{T106_SquareRootFunction}
\end{center}
\tab{$f(x)=[x]$}
Die Funktion $f(x)=[x]=\max\{ n\in \Z | n\leq x\}$ ist monoton wachsend, denn für
$x_1\leq x_2$ ist auch 
\[ [x_1]=\max\{ n\in \Z | n\leq x_1\}\leq \max\{ n\in \Z | n\leq x_2\}=[x_2]. \]
Sie ist jedoch nicht streng monoton wachsend, da zum Beispiel $0<\frac{1}{2}$, aber $[0]=0=[\frac{1}{2}]$.
\begin{center}
\image{T204_Example_A}
\end{center}
\end{tabs*}
\end{example}

\begin{quickcheckcontainer}
 
\randomquickcheckpool{1}{4}
% \begin{quickcheck}
%     \explanation{Eine Funktion $f:D\rightarrow \R$ ist monoton wachsend, wenn für ein beliebiges Paar $x_1,x_2\in D$, für die die Ungleichung $x_1\leq x_2$ gilt, die Ungleichung $f(x_1)\leq f(x_2)$ für die zugehörigen Funktionswerte folgt.}		\type{mc.unique}
% 		\begin{variables}
% 			\randint[Z]{a}{-5}{5}
% 			\randint{b}{2}{3}
% 			\function[normalize]{f}{a*x^b}
% 		\end{variables}
		
% 		\text{$f: \R_+ \to \R, x\mapsto \var{f}$ ist monoton wachsend.}
% 		\begin{choices}{unique}
%          \begin{choice}
% 			\text{Wahr}
% 			\solution[compute]{a > 0}
% 		\end{choice}
% 		\begin{choice}
% 			\text{Falsch}
% 			\solution[compute]{a < 0}
% 		\end{choice}
%         \end{choices}
% 	\end{quickcheck}	

	\begin{quickcheck}    
    \explanation{Eine Funktion $f:D\rightarrow \R$ ist monoton wachsend, wenn für ein beliebiges Paar $x_1,x_2\in D$, für die die Ungleichung $x_1\leq x_2$ gilt, die Ungleichung $f(x_1)\leq f(x_2)$ für die zugehörigen Funktionswerte folgt.}		\type{mc.unique}
		\begin{variables}
			\randint[Z]{a}{-5}{5}
			\function[normalize]{f}{a*x^2}
		\end{variables}
		
		\text{$f: \R_- \to \R, x\mapsto \var{f}$ ist monoton wachsend.}
		\begin{choices}{unique}
        \begin{choice}
			\text{Wahr}
			\solution[compute]{a < 0}
		\end{choice}
		\begin{choice}
			\text{Falsch}
			\solution[compute]{a > 0}
		\end{choice}
        \end{choices}
	\end{quickcheck}
    
	\begin{quickcheck}
    \explanation{Eine Funktion $f:D\rightarrow \R$ ist monoton wachsend, wenn für ein beliebiges Paar $x_1,x_2\in D$, für die die Ungleichung $x_1\leq x_2$ gilt, die Ungleichung $f(x_1)\leq f(x_2)$ für die zugehörigen Funktionswerte folgt.}		\type{mc.unique}
		\begin{variables}
			\randint[Z]{a}{-5}{5}
			\function[normalize]{f}{a*x^3}
		\end{variables}
		
		\text{$f: \R_- \to \R, x\mapsto \var{f}$ ist monoton wachsend.}
		\begin{choices}{unique}
        \begin{choice}
			\text{Wahr}
			\solution[compute]{a > 0}
		\end{choice}
		\begin{choice}
			\text{Falsch}
			\solution[compute]{a < 0}
		\end{choice}
        \end{choices}
	\end{quickcheck}
	
	\begin{quickcheck}
    \explanation{Eine Funktion $f:D\rightarrow \R$ ist monoton wachsend, wenn für ein beliebiges Paar $x_1,x_2\in D$, für die die Ungleichung $x_1\leq x_2$ gilt, die Ungleichung $f(x_1)\leq f(x_2)$ für die zugehörigen Funktionswerte folgt.}		\type{mc.unique}
		\begin{variables}
			\randint[Z]{a}{-5}{5}
			\randint{d}{-10}{10}
			\function[normalize]{f}{a*x^2+a*x+d}
		\end{variables}
		
		\text{$f: \R_+ \to \R, x\mapsto \var{f}$ ist monoton wachsend.}
		\begin{choices}{unique}
        \begin{choice}
			\text{Wahr}
			\solution[compute]{a > 0}
		\end{choice}
		\begin{choice}
			\text{Falsch}
			\solution[compute]{a < 0}
		\end{choice}
        \end{choices}
	\end{quickcheck}
	
	\begin{quickcheck}
        \explanation{Eine Funktion $f:D\rightarrow \R$ ist monoton fallend, wenn für ein beliebiges Paar $x_1,x_2\in D$, für die die Ungleichung $x_1\leq x_2$ gilt, die Ungleichung $f(x_1)\geq f(x_2)$ für die zugehörigen Funktionswerte folgt.}
		
		\begin{variables}
			\randint[Z]{c}{-5}{5}
			\randint{d}{-10}{10}
			\function[normalize]{f}{c*x^3+d}
		\end{variables}
		
		\text{$f: \R_+ \to \R, x\mapsto \var{f}$ ist monoton fallend.}
		\begin{choices}{unique}
        \begin{choice}
			\text{Wahr}
			\solution[compute]{c < 0}
		\end{choice}
		\begin{choice}
			\text{Falsch}
			\solution[compute]{c > 0}
		\end{choice}
        \end{choices}
	\end{quickcheck}

\end{quickcheckcontainer}

\begin{theorem}\label{thm:inversmonoton}
\begin{enumerate}
\item Ist $f:D\to \R$ eine streng monoton wachsende reelle Funktion und $W=W_f$ ihre Wertemenge,
dann ist $f$ injektiv und die partielle Umkehrfunktion $f^{-1}:W\to D$ ist ebenfalls
streng monoton wachsend.
\item Ist $f:D\to \R$ eine streng monoton fallende reelle Funktion und $W=W_f$ ihre Wertemenge,
dann ist $f$ injektiv und die partielle Umkehrfunktion $f^{-1}:W\to D$ ist ebenfalls
streng monoton fallend.
\end{enumerate}
\end{theorem}	 

\begin{proof}[Beweis]
Per Definition ist eine Funktion $f:D\to \R$ injektiv, wenn für alle $x_1,x_2\in D$ mit $x_1\neq x_2$
die Funktionswerte $f(x_1)$ und $f(x_2)$ verschieden sind. 
  \begin{incremental}
      \step Sind nun $x_1,x_2\in D$ mit $x_1\neq x_2$, dann ist entweder $x_1<x_2$ oder $x_1>x_2$. 
  Ist nun $f$ streng monoton wachsend, dann folgt aus der Definition, dass im ersten Fall 
  $f(x_1)<f(x_2)$ gilt und im zweiten Fall $f(x_1)>f(x_2)$. In beiden Fällen ist also 
  $f(x_1)\neq f(x_2)$. Daher ist $f$ injektiv.
      \step Für eine streng monoton fallende Funktion $f$ drehen sich lediglich die Vergleichszeichen $<$ und $>$
  bei den Funktionswerten um, weshalb man auch in diesem Fall die Injektivität bekommt.
      \step Um zu zeigen, dass die Umkehrfunktion $f^{-1}:W\to D$ ebenfalls streng monoton wachsend ist, falls
  $f$ streng monoton wachsend ist, seien $w,w'\in W$ mit $w<w'$, und $x=f^{-1}(w)$ und $x'=f^{-1}(w')$.
      \step Wäre $x'\leq x$, so wäre wegen der Monotonie von $f$: $f(x')\leq f(x)$. Allerdings ist
  \[ f(x')=f(f^{-1}(w'))=w'>w=f(f^{-1}(w))=f(x). \]
  Also ist $x<x'$. Da $w,w'\in W$ mit $w<w'$ beliebig waren, ist somit $f^{-1}$ streng monoton wachsend.\\
  Für eine streng monoton fallende Funktion $f$ drehen sich lediglich manche der Vergleichszeichen 
  $<$ und $>$ um, weshalb man entsprechend zeigt, dass in diesem Fall $f^{-1}$ streng monoton fallend
  ist.
  \end{incremental}
\end{proof}

\begin{example}\label{ex:n-te-wurzel}
Für $n\in \N$ betrachten wird die $n$-te Potenzfunktion 
\[ f:\R_+\to \R, x\mapsto x^n \]
als Funktion auf der Menge der nicht-negativen reellen Zahlen $\R_+$.

Nach den \ref[anordnung][Anordnungsregeln zu Potenzen]{rule:potenzen} ist für
$0\leq x_1<x_2$ auch $x_1^n< x_2^n$. Also ist die Funktion $f$ streng monoton steigend.

Die Wertemenge der Funktion $f$ ist $W_f=\R_+$, denn zum einen ist für $x\geq 0$ auch $x^n\geq 0$,
d.h. die Wertemenge ist Teilmenge von $\R_+$, und zum anderen gibt es nach dem
\ref[vollst][Satz zur Existenz $n$-ter Wurzeln]{thm:n-te-wurzel} zu jedem 
$y\in \R_+$ eine Zahl $x\in \R_+$ mit $\,x^n=y\;$ (genannt $\,\sqrt[n]{y}\,$), d.h. 
$y\in W_f$.

Die Umkehrfunktion zur $n$-ten Potenzfunktion ist daher die \emph{$n$-te Wurzel-Funktion}
\[ f^{-1}:\R_+\to \R_+, y\mapsto \sqrt[n]{y}. \]
Die zugehörigen Graphen sind in der nachfolgenden Abbildung dargestellt.\\
\begin{center}
\image{T204_Example_D}
\end{center}
Die Umkehrfunktion ist ebenfalls streng monoton wachsend.
\end{example}

Der \lref{def:monotonie}{Monotoniebegriff} und die Zusammenhänge können auch dem folgenden Video entnommen werden:\\
\floatright{\href{https://api.stream24.net/vod/getVideo.php?id=10962-2-10890&mode=iframe&speed=true}{\image[75]{00_video_button_schwarz-blau}}}\\
\\
	 
\end{content}