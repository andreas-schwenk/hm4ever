\documentclass{mumie.element.exercise}
%$Id$
\begin{metainfo}
  \name{
    \lang{de}{Ü02: Abbildungen}
    \lang{en}{Exercise 2}
  }
  \begin{description} 
 This work is licensed under the Creative Commons License Attribution 4.0 International (CC-BY 4.0)   
 https://creativecommons.org/licenses/by/4.0/legalcode 

    \lang{de}{Hier die Beschreibung}
    \lang{en}{}
  \end{description}
  \begin{components}
   \component{generic_image}{content/rwth/HM1/images/g_tkz_T204_Exercise02.meta.xml}{T204_Exercise02}
  \end{components}
  \begin{links}
  \end{links}
  \creategeneric
\end{metainfo}
\begin{content}
\title{
\lang{de}{Ü02: Abbildungen}
\lang{en}{Exercise 2}
}
\begin{block}[annotation]
  Im Ticket-System: \href{http://team.mumie.net/issues/9806}{Ticket 9806}
\end{block}

\begin{figure}
\image{T204_Exercise02}
\end{figure}
Welche Abbildungen sind in der Graphik dargestellt? Welche dieser Abbildungen lassen sich zusammensetzen?
Bestimmen Sie auch alle möglichen zusammengesetzten Abbildungen.

\begin{tabs*}[\initialtab{0}\class{exercise}]
  \tab{\lang{de}{Antworten}} 
  Die Mengen $M$ (blau), $N$ (rot), $P$ (gelb) und $R$ (gr"un) sind gegeben durch
  \[ M=\{ 0;-1;3;7\},\quad N=\{0;1;2;4\},\quad P=\{-3;2;4\}\quad \text{und}\quad  R=\{1;9\}.\]
\begin{itemize}
\item Die durchgezogenen Pfeile stellen eine Abbildung $T_1:M\to N$ dar mit
	\[ T_1(0)=1, \, T_1(-1)=4, \, T_1(7)=4, \, T_1(3)=2. \]
	\item Die gestrichelten Pfeile stellen eine Abbildung $T_2:P\to M$ dar mit
	\[ T_2(2)=-1, \, T_2(4)=3, \, T_2(-3)=0. \]
	\item Die gepunkteten Pfeile stellen eine Abbildung $T_3:P\to R$ dar mit
	\[ T_3(2)=9, \, T_3(4)=1, \, T_3(-3)=9. \]
	\item Die Strichpunkt-Pfeile stellen eine Abbildung $T_4:R\to N$ dar mit
	\[ T_4(1)=2\, \text{ und }\, T_4(9)=1. \]
\end{itemize}
Die Abbildungen $T_2:P\to M$ und $T_1:M\to N$ lassen sich zu einer Abbildung $T_1\circ T_2:P\to N$
zusammensetzen, und die Abbildungen $T_3:P\to R$ und $T_4:R\to N$ lassen sich zu einer Abbildung 
$T_4\circ T_3:P\to N$ zusammensetzen.

Hierbei sind die Zuordnungen für die Abbildung $T_1\circ T_2:P\to N$ geben als
\[ 2\mapsto 4,\,  4\mapsto 2 \text{ und }-3\mapsto 1. \]
Die Zuordnungen für die Abbildung $T_4\circ T_3:P\to N$ sind geben als
\[ 2\mapsto 1,\,  4\mapsto 2 \text{ und }-3\mapsto 1. \]


   \tab{\lang{de}{Lösung zu den Abbildungen}}
  
  \begin{incremental}[\initialsteps{1}]
  \step Definitionsbereich der Abbildungen ist stets der Bereich, in dem die Pfeile starten, und 
  Zielbereich ist der Bereich, in dem die Pfeile enden.
  \step Die durchgezogenen Pfeile stellen daher eine Zuordnung $T_1:M\to N$ dar, welche 
  in der Tat eine Abbildung ist, weil bei jedem Element in $M$ genau ein Pfeil startet, der in $N$ endet.\\
  Ebenso stellen die gestrichelten Pfeile eine Abbildung $T_2:P\to M$ dar, die
  gepunkteten Pfeile eine Abbildung $T_3:P\to R$ und die Strichpunkt-Pfeile eine 
  Abbildung $T_4:R\to N$.
  \step Um angeben zu können, auf welche Elemente aus den Zielbereichen die Elemente aus den
  Definitionsbereichen abgebildet werden, muss man lediglich den Pfeilen folgen. Man sieht daher,
  dass für die Abbildung $T_1:M\to N$ gilt:
  \[ 0\mapsto 1, \, -1\mapsto 4, \, 7\mapsto 4, \, 3\mapsto 2. \]
  Oder anders ausgedrückt:
  \[ T_1(0)=1, \, T_1(-1)=4, \, T_1(7)=4, \, T_1(3)=2. \]
  \step Für die anderen Abbildungen erhält man entsprechend:
  \[ T_2(2)=-1, \, T_2(4)=3, \, T_2(-3)=0, \]
  \[ T_3(2)=9, \, T_3(4)=1, \, T_3(-3)=9 \]
  sowie
  \[ T_4(1)=2\, \text{ und }\, T_4(9)=1. \]
  \end{incremental}

   \tab{\lang{de}{Lösung zur Zusammensetzung der Abbildungen}}
  
  \begin{incremental}[\initialsteps{1}]
  \step Zwei Abbildungen lassen sich genau dann zusammensetzen, wenn der Zielbereich der einen
  Abbildung gleich dem Definitionsbereich der anderen Abbildung ist. Oder in der Graphik 
  ausgedrückt: Wenn dort, wo die einen Pfeile enden, andere Pfeile starten.
  \step Im vorliegenden Beispiel lassen sich also die Abbildungen $T_2:P\to M$ und $T_1:M\to N$
  zusammensetzen, sowie die Abbildungen $T_3:P\to R$ und $T_4:R\to N$. Beide Zusammensetzungen
  ergeben dann Abbildungen von $P$ nach $N$.
  \step Die Abbildung $T_1\circ T_2:P\to N$ erfüllt dann:
  \[ (T_1\circ T_2)(2)=T_1(T_2(2))=T_1(-1)=4, \]
  d.h. $2$ wird auf $4$ abgebildet. Entsprechend sind
  \begin{align*}
  (T_1\circ T_2)(4) &=& T_1(T_2(4))=T_1(3)=2, \\
  (T_1\circ T_2)(-3) &=& T_1(T_2(-3))=T_1(0)=1.
  \end{align*}
  Für die Abbildung $T_4\circ T_3:P\to N$ erhält man
  \begin{align*}
  (T_4\circ T_3)(2) &=& T_4(T_3(2))=T_4(9)=1,\\
  (T_4\circ T_3)(4) &=& T_4(T_3(4))=T_4(1)=2,\\
  (T_4\circ T_3)(-3) &=& T_4(T_3(-3))=T_4(9)=1.  
  \end{align*}
  \end{incremental}


\end{tabs*}
\end{content}