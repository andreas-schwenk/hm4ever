\documentclass{mumie.element.exercise}
%$Id$
\begin{metainfo}
  \name{
    \lang{de}{Ü03: Komposition}
    \lang{en}{Exercise 3}
  }
  \begin{description} 
 This work is licensed under the Creative Commons License Attribution 4.0 International (CC-BY 4.0)   
 https://creativecommons.org/licenses/by/4.0/legalcode 

    \lang{de}{Hier die Beschreibung}
    \lang{en}{}
  \end{description}
  \begin{components}
  \end{components}
  \begin{links}
  \end{links}
  \creategeneric
\end{metainfo}
\begin{content}
\title{
\lang{de}{Ü03: Komposition}
\lang{en}{Exercise 3}
}
\begin{block}[annotation]
  Im Ticket-System: \href{http://team.mumie.net/issues/9807}{Ticket 9807}
\end{block}


Gegeben sind die vier Abbildungen
\begin{itemize}
\item $f:\N\to \Z, n\mapsto (-2)^n$,
\item $g:\Z\to \N, z\mapsto z^2+1$,
\item $h:\Z\to \Z, z\mapsto -z$,
\item $k:\N\to \N, n\mapsto 1$.
\end{itemize}
Welche dieser Abbildungen lassen sich zusammensetzen? Bestimmen
Sie alle Zusammensetzungen von zwei (evtl. auch gleichen) Abbildungen.


\begin{tabs*}[\initialtab{0}\class{exercise}]
  \tab{\lang{de}{Antworten}} 
     Die möglichen Zusammensetzungen sind
     \begin{itemize}
     \item $f\circ g:\Z\to \Z, z\mapsto (-2)^{z^2+1}$,
     \item $f\circ k:\N\to \Z, n\mapsto -2$,
     \item $g\circ f:\N\to \N, n\mapsto ((-2)^n)^2+1=(-2)^{2n}+1=4^n+1$,
     \item $g\circ g\colon \Z \to \N$, $z\mapsto (z^2+1)^2+1$,
     \item $g\circ h:\Z\to \N, z\mapsto (-z)^2+1=z^2+1$,
     \item $g\circ k: \N \to \Z$, $z\mapsto 1^2+1=2$,
     \item $h\circ f:\N\to \Z, n\mapsto -(-2)^n$,
     \item $h\circ g:\Z\to \Z, z\mapsto -(z^2+1)=-z^2-1$,
     \item $h\circ h:\Z\to \Z, z\mapsto -(-z)=z$,
     \item $h\circ k\colon \N \to \Z$, $z\mapsto -1$,
     \item $k\circ g:\Z\to \N, z\mapsto 1$,
     \item $k\circ k:\N\to \N, n\mapsto 1$.
     \end{itemize}
   \tab{\lang{de}{Lösung}}
  
Ganz allgemein lassen sich Abbildungen $T$ und $S$ zu $T\circ S$ zusammensetzen, wenn der Zielbereich von $S$ eine Teilmenge des Definitionsbereichs von $T$ ist. Also sind die möglichen Zusammensetzungen: $f\circ g$, $f\circ k$, $g\circ f$, $g\circ g$, $g\circ h$, $g\circ k$, $h\circ f$, $h\circ g$, $h\circ h$, $h\circ k$, $k\circ g$, $k\circ k$.\\
Dabei ist der Definitionsbereich der Komposition wieder der Definitionsbereich der zuerst ausgeführten Abbildung und der Zielbereich ist der Zielbereich der zuletzt ausgeführten Abbildung. \\
Die Abbildungsvorschrift bestimmt man durch Hintereinanderausführung der Abbildungen, also z.B.
  \[
   (f\circ g)(z)=f(g(z))=f(z^2+1)=(-2)^{z^2+1}.
  \]

 

\end{tabs*}
\end{content}