\documentclass{mumie.element.exercise}
%$Id$
\begin{metainfo}
  \name{
    \lang{de}{Ü06: Verknüpfungen}
    \lang{en}{Exercise 6}
  }
  \begin{description} 
 This work is licensed under the Creative Commons License Attribution 4.0 International (CC-BY 4.0)   
 https://creativecommons.org/licenses/by/4.0/legalcode 

    \lang{de}{Hier die Beschreibung}
    \lang{en}{}
  \end{description}
  \begin{components}
  \end{components}
  \begin{links}
  \end{links}
  \creategeneric
\end{metainfo}
\begin{content}
\title{
\lang{de}{Ü06: Verknüpfungen}
\lang{en}{Exercise 6}
}
\begin{block}[annotation]
  Im Ticket-System: \href{http://team.mumie.net/issues/9810}{Ticket 9810}
\end{block}

\lang{de}{Berechnen Sie für 
\begin{align*}
 f: & \R\to \R, x\mapsto x^2+x, \\
 g: & \R\setminus\{0\} \to \R, x\mapsto x^3 - \frac{1}{x}
\end{align*}
die Ausdrücke }

\begin{table}[\class{items}]
  \nowrap{a) $(f+g)(x)$,} &
  \nowrap{b) $(f-g)(x)$,} &
  \nowrap{c) $2\cdot f(x)$,} &
  \nowrap{d) $(f\cdot g)(x)$}
\end{table}
und geben Sie die Funktion an (mit Definitionsbereich und Zielbereich).

\begin{tabs*}[\initialtab{0}\class{exercise}]
  \tab{\lang{de}{Antworten}\lang{en}{Answer}} 
  
  \begin{table}[\class{items}]
  a) $f+g :\R\setminus\{0\} \to \R , x\mapsto x^2+x+x^3-\frac{1}{x}$ \\
  b) $f-g :\R\setminus\{0\} \to \R , x \mapsto x^2+x-x^3+ \frac{1}{x}$ \\
  c) $2\cdot f: \R \to \R , x\mapsto 2x^2+2x$ \\
  d) $f\cdot g: \R\setminus\{0\} \to \R , x \mapsto x^5 +x^4 -x-1 .$
  \end{table}
  
	\tab{\lang{de}{Lösung a)}}
Nach Definition ist die Summe der Funktionen dadurch bestimmt, dass man die Summe der
Funktionswerte berechnet, also ergibt sich
\[f+g :\R\setminus\{0\} \to \R , x\mapsto f(x) +g(x)= x^2+x+x^3-\frac{1}{x} .\]
Der Definitionsbereich ist der Schnitt der Definitionsbereiche von $f$ und $g$.
	
    \tab{\lang{de}{Lösung b)}}
	\[ f-g :\R\setminus\{0\} \to \R , x \mapsto f(x)- g(x) = x^2+x-x^3+ \frac{1}{x} .\]
Der Definitionsbereich ist der Schnitt der Definitionsbereiche von $f$ und $g$.
	
	\tab{\lang{de}{Lösung c)}}
	\[ 2\cdot f: \R \to \R , x\mapsto 2\cdot f(x) = 2\cdot(x^2+x)= 2x^2+2x. \]
Der Definitionsbereich ist gleich dem der Funktion $f$.

	\tab{\lang{de}{Lösung d)}}
	\[ f\cdot g: \R\setminus\{0\} \to \R , x \mapsto f(x) \cdot g(x) = (x^2+x)(x^3-\frac{1}{x})= x^5 +x^4 -x-1 . \]
Der Definitionsbereich ist der Schnitt der Definitionsbereiche von $f$ und $g$.
	
\end{tabs*}
\end{content}