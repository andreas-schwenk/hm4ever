\documentclass{mumie.element.exercise}
%$Id$
\begin{metainfo}
  \name{
    \lang{de}{Ü01: Abbildungen}
    \lang{en}{Exercise 1}
  }
  \begin{description} 
 This work is licensed under the Creative Commons License Attribution 4.0 International (CC-BY 4.0)   
 https://creativecommons.org/licenses/by/4.0/legalcode 

    \lang{de}{Hier die Beschreibung}
    \lang{en}{}
  \end{description}
  \begin{components}
  \end{components}
  \begin{links}
  \end{links}
  \creategeneric
\end{metainfo}
\begin{content}
\title{
\lang{de}{Ü01: Abbildungen}
\lang{en}{Exercise 1}
}
\begin{block}[annotation]
  Im Ticket-System: \href{http://team.mumie.net/issues/9805}{Ticket 9805}
\end{block}

\lang{de}{Entscheiden Sie, ob eine Abbildung vorliegt oder nicht.}

\begin{table}[\class{items}]
\nowrap{a)  $f_1: \N\to\N, n\mapsto n+1$,} \\
\nowrap{b)  $f_2: \R \to \R, x\to x^2+3$,} \\
\nowrap{c)  $f_3:\R\setminus\{5\} \to \R, x\mapsto 3x$,} \\
\nowrap{d)  $f_4: \N \times \N \to \N, (x,y)\mapsto x-y$,} \\
\nowrap{e)  $f_5: \R\setminus\{3\} \to \R, x\mapsto \frac{1}{x-3}$,} \\
\nowrap{f)  $f_6:\Z \to \N, n\mapsto n^3$,} \\
\nowrap{g)  $f_7: \N\to \N, n\mapsto n-1$,} \\
\nowrap{h)  $f_8: \R \to \R, x\mapsto \pm \sqrt{x}$.}
\end{table}

\begin{tabs*}[\initialtab{0}\class{exercise}]
  \tab{
  \lang{de}{Antworten}
  }
  \begin{table}[\class{items}]
    \nowrap{a) ja} &
    \nowrap{b) ja} &
    \nowrap{c) ja} & 
    \nowrap{d) nein} \\
    \nowrap{e) ja} &
    \nowrap{f) nein} &
    \nowrap{g) nein} &
    \nowrap{h) nein} 
    
  \end{table}

  \tab{
  \lang{de}{Lösung a)}}
  $f_1$ ist eine Abbildung, da jede natürliche Zahl einen eindeutigen Nachfolger hat, welcher selbst eine natürliche Zahl ist.
  \tab{
  \lang{de}{Lösung b)}}
  Für jedes $x\in\R$ ist auch $x^2+3\in \R$, damit ist auch $f_2$ eine Abbildung.

  \tab{
  \lang{de}{Lösung c)}}
  Jedem $x\in \R\setminus\{5\}$ wird ein eindeutiges Element der reellen Zahlen zugeordnet, damit ist auch $f_3$ eine Abbildung. \\
Hinweis: Der Definitionsbereich ist hier nicht maximal, die Funktion könnte problemlos auf $\R$ fortgesetzt werden.

  \tab{
  \lang{de}{Lösung d)}}
Bei $f_4$ handelt es sich nicht um eine Abbildung, da z.B.
$f_4((1,5))=-4 \notin \N$ gilt.

  \tab{
  \lang{de}{Lösung e)}}
  Da die einzige Nullstelle im Nenner der Funktion bei $x=3$ vorliegt und der Definitionsbereich $x=3$ ausschließt, handelt es sich auch bei $f_5$ wieder um eine Abbildung.

  \tab{
  \lang{de}{Lösung f)}}
  Da beispielsweise $f_6(-1)=-1\notin \N$ handelt es sich nicht um eine Abbildung. 
  
  \tab{
  \lang{de}{Lösung g)}}
	Da $n=1$ in den natürlichen Zahlen keinen Vorgänger hat, denn $0\notin\N$ wird $n=1$ kein Wert im Zielbereich zugeordnet, damit handelt es sich bei $f_7$ nicht um eine Abbildung.
	
	\tab{
  \lang{de}{Lösung h)}}
	Es liegt keine Abbildung vor, da jedem Element außer der Null immer zwei Funktionswerte zugeordnet werden.
	
	
\end{tabs*}

\end{content}