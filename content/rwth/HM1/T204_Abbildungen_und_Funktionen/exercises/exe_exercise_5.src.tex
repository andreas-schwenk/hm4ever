\documentclass{mumie.element.exercise}
%$Id$
\begin{metainfo}
  \name{
    \lang{de}{Ü05: Umkehrfunktionen}
    \lang{en}{Exercise 5}
  }
  \begin{description} 
 This work is licensed under the Creative Commons License Attribution 4.0 International (CC-BY 4.0)   
 https://creativecommons.org/licenses/by/4.0/legalcode 

    \lang{de}{Hier die Beschreibung}
    \lang{en}{}
  \end{description}
  \begin{components}
  \end{components}
  \begin{links}
  \end{links}
  \creategeneric
\end{metainfo}
\begin{content}
\title{
\lang{de}{Ü05: Umkehrfunktionen}
\lang{en}{Exercise 5}
}
\begin{block}[annotation]
  Im Ticket-System: \href{http://team.mumie.net/issues/9809}{Ticket 9809}
\end{block}

\lang{de}{Bestimmen Sie die Wertemenge der Funktionen und bilden Sie die partielle Umkehrfunktion.}

\begin{table}[\class{items}]
  \nowrap{a) $\, f:\R\to \R, \; x \mapsto 2x+1,$} \\
  \nowrap{b) $\, g:\left.\left[\frac{3}{4},\infty\right.\right) \to \R, \; x\mapsto 2x^2-3x+1,$}\\
  \nowrap{c) $\, h: \R\setminus\{1\} \to \R, \; x\mapsto 3- \frac{1}{x-1}.$}
\end{table}

\begin{tabs*}[\initialtab{0}\class{exercise}]
  \tab{\lang{de}{Antworten}\lang{en}{Answer}} 
  
  \begin{table}[\class{items}]
  a) $\, W_f =\R$, & $f^{-1}:\R\to\R, \; y\mapsto \frac{y-1}{2},$ \\
  b) $\, W_g = \left.\left[-\frac{1}{8},\infty\right. \right), \;$ & 
     $g^{-1}:\left.\left[-\frac{1}{8},\infty\right. \right) \to 
     \left.\left[\frac{3}{4},\infty\right.\right) , \; y\mapsto \sqrt{\frac{1}{2}(y+\frac{1}{8})}+\frac{3}{4}$,\\
  c) $\, W_h= \R\setminus\{3\}, \; $ & $h^{-1}:  \R\setminus\{3\} \to \R\setminus\{1\}, 
    \; y\mapsto \frac{1}{3-y}+1$.
  \end{table}
  
	\tab{\lang{de}{Lösung a)}}
    $f$ ist surjektiv, denn für jedes $y\in\R$ ist $f(x)=y \,$
    äquivalent zu $\, 2x+1=y$, was wir nach $x$ auflösen können. 
    Wir erhalten dann $\,x= \frac{y-1}{2}$, womit wir direkt die 
    Umkehrfunktion bestimmt haben. 

	\tab{\lang{de}{Lösung b)}}
	\begin{incremental}[\initialsteps{1}]
  	\step 
		Der Graph der Funktion $g$ ist eine nach oben geöffnete Parabel. Wir bestimmen die Scheitelpunktsform
\begin{align*}
2x^2-3x+1 &= 2(x^2-\frac{3}{2}x)+1 \\
&= 2(x^2-\frac{3}{2}x +(\frac{3}{4})^2-(\frac{3}{4})^2 ) +1 \\
&= 2 (( x- \frac{3}{4})^2-\frac{9}{16})+1 \\
&= 2(x-\frac{3}{4})^2 - \frac{1}{8}.
\end{align*}
Damit können wir den Scheitelpunkt $S=\left(\frac{3}{4}, - \frac{1}{8} \right)$ ablesen und erhalten, da die Parabel nach oben geöffnet ist die Wertemenge $W_g= \left.\left[-\frac{1}{8},\infty\right. \right)$.
	\step
		Um die Umkehrfunktion zu finden, setzen wir $y=g(x)$ und lösen nach x auf
\begin{align*}
&&\quad y = 2(x-\frac{3}{4})^2 - \frac{1}{8} \\
&\Leftrightarrow &\quad  y+ \frac{1}{8}= 2(x-\frac{3}{4})^2 \\
&\Leftrightarrow &\quad  \frac{1}{2} (  y+ \frac{1}{8}) = (x-\frac{3}{4})^2 \\
&\Rightarrow     &\quad  \sqrt{\frac{1}{2} (  y+ \frac{1}{8})} = x-\frac{3}{4} \\
&\Leftrightarrow &\quad  x = \sqrt{\frac{1}{2} (  y+ \frac{1}{8})} +\frac{3}{4},
\end{align*}
womit wir die Umkehrfunktion direkt erhalten haben.
		\end{incremental}
	
	\tab{\lang{de}{Lösung c)}}
    Die gegebene Funktion entspricht einer Normalhyperbel, deren Mittelpunkt im Punkt 
    $M(1,3)$ liegt. Die y-Koordinate des Mittelpunkts einer Parabel gehört nicht zum Wertebereich.
    Der Wertebereich ergibt sich somit zu $\, W_h= \R\setminus\{3\}$.\\ 
	Die Umkehrfunktion ergibt sich durch Umformen der gegebenen Funktion nach $x$
  \begin{align*}
   3-\frac{1}{x-1}=y \quad &\Leftrightarrow &\quad 3(x-1)-1=y(x-1)\\ 
                      &\Leftrightarrow &\quad 3x-3-1-yx=-y\\ 
                      &\Leftrightarrow &\quad (3-y)x=3-y+1\\ 
                      &\Leftrightarrow &\quad x=1+\frac{1}{3-y}.
  \end{align*}

	
\end{tabs*}
\end{content}