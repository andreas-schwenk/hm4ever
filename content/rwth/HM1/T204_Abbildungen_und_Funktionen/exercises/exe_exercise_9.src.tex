\documentclass{mumie.element.exercise}
%$Id$
\begin{metainfo}
  \name{
    \lang{de}{Ü09: Mächtigkeit}
    \lang{en}{Exercise 9}
  }
  \begin{description} 
 This work is licensed under the Creative Commons License Attribution 4.0 International (CC-BY 4.0)   
 https://creativecommons.org/licenses/by/4.0/legalcode 

    \lang{de}{}
    \lang{en}{}
  \end{description}
  \begin{components}
  \end{components}
  \begin{links}
  \end{links}
  \creategeneric
\end{metainfo}
\begin{content}
\begin{block}[annotation]
	Im Ticket-System: \href{https://team.mumie.net/issues/22807}{Ticket 22807}
\end{block}




\usepackage{mumie.ombplus}

\title{
  \lang{de}{Ü09: Mächtigkeit}
  \lang{en}{Exercise 9}
}






  \lang{de}{Welche der folgenden Mengenpaare haben die gleiche Mächtigkeit?}
  
  \begin{table}[\class{items}]
    \nowrap{a) $\N_{0}$, $\N$,}\\
    \nowrap{b) $\Z$, Menge der ungeraden Zahlen,}\\
    \nowrap{c) $\N$, $\R\setminus\{0;1;2\}$,}\\
    \nowrap{d) $\Q$, Menge der irrationalen Zahlen,}\\
    \nowrap{e) $\Z$, $(0;1)$.}
  \end{table}
  

  
  \begin{tabs*}[\initialtab{0}\class{exercise}]
    \tab{
      \lang{de}{Antwort}
     } 
    
     Die Mengenpaare in a) und b) haben die gleiche Mächtigkeit.
    

    \tab{
      \lang{de}{Lösung a)}}
    

      \lang{de}{$\N_{0}$ und $\N$ haben die gleiche Mächtigkeit, da die Abbildung $f:\N_{0}\rightarrow\N $, $n\mapsto n+1$ eine Bijektion ist.}
    \tab{
      \lang{de}{Lösung b)}}
      \lang{de}{$\Z$ ist abzählbar unendlich und gleichmächtig mit $\N$. Die Menge der ungeraden Zahlen 
      $U:=\{n\in\N$ $|$ Es gibt ein $ m\in\N, n=2m+1\}$ ist auch abzählbar,
       weil $g:\N\to U$, $n\mapsto 2n+1$ eine bijektive Abbildung ist. D.h. nach der Definition der Gleichmächtigkeit 
       haben die beiden Mengen die gleiche Mächtigkeit.}
    \tab{
      \lang{de}{Lösung c)}}
      \lang{de}{$\R\setminus\{0;1;2\}$ ist überabzählbar, da die folgende Abbildung bijektiv ist:
      \[h:\R\to \R\setminus\{0;1;2\}, \text{ } x\mapsto \begin{cases}
                                            x & \text{ falls } x\notin \{0;1;2;3;...\},\\
                                            x+3 & \text{ falls }x\in \{0;1;2;3;...\}. 
                                            \end{cases} \]
      $\N$ ist aber abzählbar. Daher haben die beiden Mengen verschiedene Mächtigkeiten.}
    \tab{
      \lang{de}{Lösung d)}} 
      \lang{de}{$\Q$ ist eine abzählbare Menge. Wenn die Menge der irrationalen Zahlen ($\R\setminus\Q$) abzählbar wäre, dann wäre 
      $\R$ die Vereinigung zwei abzählbarer Mengen und somit selbst abzählbar. Das steht aber im Widerspruch zur Überabzälbarkeit von $\R$.
      Daher ist die Menge der irrationalen Zahlen eine überabzählbare Menge. }
    \tab{
      \lang{de}{Lösung e)}}
      \lang{de}{$\Q$ ist eine abzählbare Menge. Das Intervall $(0;1)$ dagegen ist überabzählbar, da die folgende Abbildung bijektiv ist:
      \[f':\R\to (0;1),\text{ }x\mapsto \frac{1}{\pi}(\arctan(x)+\frac{\pi}{2}).\]}

  \end{tabs*}
\end{content}