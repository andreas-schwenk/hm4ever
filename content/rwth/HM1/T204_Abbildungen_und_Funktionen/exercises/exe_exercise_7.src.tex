\documentclass{mumie.element.exercise}
%$Id$
\begin{metainfo}
  \name{
    \lang{de}{Ü07: Eigenschaften}
    \lang{en}{Exercise 7}
  }
  \begin{description} 
 This work is licensed under the Creative Commons License Attribution 4.0 International (CC-BY 4.0)   
 https://creativecommons.org/licenses/by/4.0/legalcode 

    \lang{de}{Hier die Beschreibung}
    \lang{en}{}
  \end{description}
  \begin{components}
  \end{components}
  \begin{links}
    \link{generic_article}{content/rwth/HM1/T204_Abbildungen_und_Funktionen/g_art_content_12_reelle_funktionen_monotonie.meta.xml}{content_12_reelle_funktionen_monotonie}
    \link{generic_article}{content/rwth/HM1/T202_Reelle_Zahlen_axiomatisch/g_art_content_05_anordnungsaxiome.meta.xml}{anordnung}
  \end{links}
  \creategeneric
\end{metainfo}
\begin{content}
\title{
\lang{de}{Ü07: Eigenschaften}
\lang{en}{Exercise 7}
}
\begin{block}[annotation]
  Im Ticket-System: \href{http://team.mumie.net/issues/9811}{Ticket 9811}
\end{block}

\lang{de}{Untersuchen Sie die Funktion $f:\R_+ \to \R, f(x)= x^3+x^2$ mittels Monotonie auf Injektivität.}


\begin{tabs*}[\initialtab{0}\class{exercise}]
  \tab{\lang{de}{Lösung}} 
Seien $x_1,x_2 \in \R_+$ und es gelte $x_1<x_2$. Dann gilt nach den \ref[anordnung][Anordnungsregeln zu Potenzen]{rule:potenzen} 
auch $x_1^2 < x_2^2$ und ebenso $x_1^3<x_2^3$. Dann können wir die beiden Ungleichungen addieren und erhalten direkt 
$x_1^3+x_1^2 < x_2^3+x_2^2 $. Das bedeutet, dass die Funktion streng monoton wachsend ist 
und somit 
% Ergänzungsvorschlag QS
nach dem \ref[content_12_reelle_funktionen_monotonie][Theorem über streng monotone Funktionen]{thm:inversmonoton}
%
auch injektiv.
\end{tabs*}
\end{content}
