\documentclass{mumie.element.exercise}
%$Id$
\begin{metainfo}
  \name{
    \lang{de}{Ü08: Mächtigkeit}
    \lang{en}{Exercise 8}
  }
  \begin{description} 
 This work is licensed under the Creative Commons License Attribution 4.0 International (CC-BY 4.0)   
 https://creativecommons.org/licenses/by/4.0/legalcode 

    \lang{de}{}
    \lang{en}{}
  \end{description}
  \begin{components}
  \end{components}
  \begin{links}
  \end{links}
  \creategeneric
\end{metainfo}
\begin{content}
\begin{block}[annotation]
	Im Ticket-System: \href{https://team.mumie.net/issues/22806}{Ticket 22806}
\end{block}




\usepackage{mumie.ombplus}

\title{
  \lang{de}{Ü08: Mächtigkeit}
  \lang{en}{Exercise 8}
}






  \lang{de}{Bestimmen Sie die Mächtigkeit folgender Mengen.}
  
  \begin{table}[\class{items}]
    \nowrap{a) Die Lösungsmenge der Gleichung $(x-3)^2 (x+91)^3 (x+17)(x^2 -2)=0$},\\
    \nowrap{b) $\{x\in \N $ $|$ $ x $ ist keine Quadratzahl$ \}$ },\\
    \nowrap{c) $\{3;8;18^2;\frac{36}{3};\frac{44}{11};\frac{54}{18}\}.$}
  \end{table}
  

  
  \begin{tabs*}[\initialtab{0}\class{exercise}]
    \tab{
      \lang{de}{Antworten}} 
      a) $5$,\\
      b) $\infty$,\\
      c) $5$.    
    
    

    \tab{
      \lang{de}{Lösung a)}}
    
      \lang{de}{Die Gleichung hat die folgenden Lösungen, die sich aus der Linearfaktozerlegung 
      direkt ablesen lassen:
      \[ 3,-91,-17,\sqrt{2},-\sqrt{2}.\]}
      Die Lösungsmenge der Gleichung hat also 5 Elemente.
    
    \tab{
      \lang{de}{Lösung b)}}
      \lang{de}{$\{x\in \N $ $|$ $ x $ ist keine Quadratzahl$ \}$ ist die Menge aller natürlichen Zahlen bis auf 
      vollständige Quadrate. Dass heißt
      \[ \{x\in \N\, | \, x \text{ ist keine Quadratzahl} \}=\{2;3;5;6;7;8;10;11;...\}, \]
      die eine unendliche Menge ist.}
      
    \tab{
      \lang{de}{Lösung c)}}  
      \lang{de}{Die Menge \[\{3;8;18^2;\frac{36}{3};\frac{44}{11};\frac{54}{18}\}=\{3;8;18;12;4;3\}\]
      hat 5 Elemente, da die wiederholten Elemente nur einmal gezählt werden.}
      
    
  \end{tabs*}
\end{content}