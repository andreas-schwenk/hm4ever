\documentclass{mumie.element.exercise}
%$Id$
\begin{metainfo}
  \name{
    \lang{de}{Ü04: Eigenschaften}
    \lang{en}{Exercise 4}
  }
  \begin{description} 
 This work is licensed under the Creative Commons License Attribution 4.0 International (CC-BY 4.0)   
 https://creativecommons.org/licenses/by/4.0/legalcode 

    \lang{de}{Hier die Beschreibung}
    \lang{en}{}
  \end{description}
  \begin{components}
  \end{components}
  \begin{links}
  \end{links}
  \creategeneric
\end{metainfo}
\begin{content}
\title{
\lang{de}{Ü04: Eigenschaften}
\lang{en}{Exercise 4}
}
\begin{block}[annotation]
  Im Ticket-System: \href{http://team.mumie.net/issues/9808}{Ticket 9808}
\end{block}

\lang{de}{1) Entscheiden Sie ob die folgenden Abbildungen injektiv, surjektiv oder bijektiv sind. }
\begin{table}[\class{items}]
  \nowrap{a) $f_1: \N\to\N, n\mapsto n+1,$} \\ \nowrap{b) $f_2: \Z \to \Z, x\mapsto x+1,$} \\
  \nowrap{c) $f_3:\R \to \R, x\mapsto \frac{1}{2}x,$} \\ \nowrap{d) $f_4:  \R \to \R, x\mapsto x^2,$} \\
  \nowrap{e) $f_5:  \R \to \R_+, x\mapsto x^2,$} \\ \nowrap{f) $f_6: \R \to \R, x\mapsto x^2-3x+2.$}
\end{table}

\lang{de}{2) Sei $\N=\{1,2,3,...\}$ die Menge der natürlichen Zahlen und $M=\{2,4,6,...\}$ die Menge der geraden Zahlen. Sind die folgenden Abbildungen $f:M\mapsto\N$ injektiv, surjektiv bzw. bijektiv?}

\begin{table}[\class{items}]
  \nowrap{a) $f(m)=m$} \\
  \nowrap{b) $f(m)=\frac{m}{2}$} \\
  \nowrap{c) $f(m)=\lceil \frac{m}{4} \rceil$}, wobei $\lceil x \rceil$ Aufrundung auf die nächstgrößere natürliche Zahl $n\geq x$ bedeutet.
\end{table}

\begin{tabs*}[\initialtab{0}\class{exercise}]
  \tab{\lang{de}{Antworten}\lang{en}{Answers}} 
  
  Aufgabe 1
  \begin{table}[\class{items}]
  a) injektiv, &
  b) bijektiv, \\
  c) bijektiv, &
  d) weder injektiv noch surjektiv, \\
  e) surjektiv, &
  f) weder injektiv noch surjektiv.
  \end{table}
  
  Aufgabe 2
  \begin{table}[\class{items}]
  a) injektiv, nicht surjektiv \\
  b) injektiv und surjektiv, also bijektiv \\
  c) surjektiv, nicht injektiv \\
  \end{table}
  
	\tab{\lang{de}{Lösung 1a)}}
	\begin{incremental}[\initialsteps{1}]
  	\step 
		$f_1$ ist injektiv, denn angenommen für $n, m \in \N \,$
        sei $\, f_1(n)=f_1(m)$, also $n+1=m+1$. Dies impliziert direkt $n=m$. 
	\step
		Aber $f_1$ ist nicht surjektiv, denn $1 \notin W_{f_1}$.
\end{incremental}

	\tab{\lang{de}{Lösung 1b)}}
	\begin{incremental}[\initialsteps{1}]
  	\step 
		Analog zu a) zeigt man die Injektivität.
	\step
		Außerdem ist $f_2$ surjektiv, da für jedes $x\in \Z$ auch $x-1\in\Z$ und somit gilt $f_2(x-1)=x$. Damit ist $f_2$ sogar bijektiv.
	\end{incremental}
	
	\tab{\lang{de}{Lösung 1c)}}
	\begin{incremental}[\initialsteps{1}]
  	\step 
		Es gilt Injektivität wie in a) und b).
	\step
		Die Funktion ist surjektiv, da für jedes $x\in\R$ auch $2x\in\R$ und $f_3(2x)=x$ gilt. Damit ist die Funktion bijektiv.
	\end{incremental}
	
	\tab{\lang{de}{Lösung 1d)}}
	\begin{incremental}[\initialsteps{1}]
  	\step 
		Es ist $f_4(-2)=4=f_4(2)$, also ist $f_4$ nicht injektiv.
	\step
		Außerdem existiert beispielsweise kein $x\in\R$ mit $x^2=-3$, weshalb die Funktion nicht surjektiv ist.
	\end{incremental}
	
	\tab{\lang{de}{Lösung 1e)}}
	\begin{incremental}[\initialsteps{1}]
  	\step 
		Wie in d) ist die Funktion $f_5$ nicht injektiv.
	\step
		Allerdings ist die Funktion $f_5$ surjektiv, da für jedes $c\in \R_+$ beispielsweise $x_0=\sqrt{c}$ die Identität $f_5(x_0)=c$ erfüllt.
	\end{incremental}

	\tab{\lang{de}{Lösung 1f)}}
	\begin{incremental}[\initialsteps{1}]
  	\step 
		Es gilt $f_6(2)=0=f_6(1)$, damit ist die Funktion nicht injektiv.
	\step
		Außerdem ist $f_6$ nicht surjektiv, da beispielsweise für $c=-4$ die Gleichung $f_6(x)=c$ nicht lösbar ist. Das sieht man ein, indem man versucht die quadratische Gleichung $x^2-3x+2=-4$ beispielsweise mit der $p-q-$ Formel zu lösen.
	\end{incremental}
	
     \tab{\lang{de}{Lösungsvideo 2)}}	
        Die Lösungswege zu Aufgabe 2 sind dem folgenden Video zu entnehmen:\\
        \youtubevideo[500][300]{k5JprY261ZU}\\
\end{tabs*}
\end{content}