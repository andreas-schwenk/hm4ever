%$Id:  $
\documentclass{mumie.article}
%$Id$
\begin{metainfo}
  \name{
    \lang{de}{Überblick: Abbildungen}
    \lang{en}{overview: }
  }
  \begin{description} 
 This work is licensed under the Creative Commons License Attribution 4.0 International (CC-BY 4.0)   
 https://creativecommons.org/licenses/by/4.0/legalcode 

    \lang{de}{Beschreibung}
    \lang{en}{}
  \end{description}
  \begin{components}
  \end{components}
  \begin{links}
\link{generic_article}{content/rwth/HM1/T102neu_Einfache_Reelle_Funktionen/g_art_T102_overview.meta.xml}{T102_overview}
\link{generic_article}{content/rwth/HM1/T204_Abbildungen_und_Funktionen/g_art_content_13_unabzaehlbarkeit.meta.xml}{content_13_unabzaehlbarkeit}
\link{generic_article}{content/rwth/HM1/T204_Abbildungen_und_Funktionen/g_art_content_12_reelle_funktionen_monotonie.meta.xml}{content_12_reelle_funktionen_monotonie}
\link{generic_article}{content/rwth/HM1/T204_Abbildungen_und_Funktionen/g_art_content_11_injektiv_surjektiv_bijektiv.meta.xml}{content_11_injektiv_surjektiv_bijektiv}
\link{generic_article}{content/rwth/HM1/T204_Abbildungen_und_Funktionen/g_art_content_10_abbildungen_verkettung.meta.xml}{content_10_abbildungen_verkettung}
\end{links}
  \creategeneric
\end{metainfo}
\begin{content}
\begin{block}[annotation]
	Im Ticket-System: \href{https://team.mumie.net/issues/30133}{Ticket 30133}
\end{block}




\begin{block}[annotation]
Im Entstehen: Überblicksseite für Kapitel Abbildungen
\end{block}

\usepackage{mumie.ombplus}
\ombchapter{1}
\lang{de}{\title{Überblick: Abbildungen}}
\lang{en}{\title{}}



\begin{block}[info-box]
\lang{de}{\strong{Inhalt}}
\lang{en}{\strong{Contents}}


\lang{de}{
    \begin{enumerate}%[arabic chapter-overview]
   \item[4.1] \link{content_10_abbildungen_verkettung}{Abbildungen und ihre Verkettungen}
   \item[4.2] \link{content_11_injektiv_surjektiv_bijektiv}{Injektivität, Surjektivität und Bijektivität}
   \item[4.3] \link{content_12_reelle_funktionen_monotonie}{Reelle Funktionen und Monotonie}
   \item[4.4] \link{content_13_unabzaehlbarkeit}{Abzählbarkeit}
   \end{enumerate}
} %lang

\end{block}

\begin{zusammenfassung}

\lang{de}{Der Abbildungsbegriff wurde bereits in \link{T102_overview}{Teil 1 Kapitel 2}  eingeführt für reelle Funktionen.
Hier erweitern wir ihn ganz allgemein auf beliebige Mengen und beschäftigen uns mit möglichen Eigenschaften von Funktionen,
der Injektivität, Surjektivität und Bijektivität. Damit verstehen wir auch universell, wann eine Funktion umkehrbar ist, und welche Funktionen verknüpft werden können.

Die Auswirkungen dieser Eigenschaften für reelle Funktionen werden studiert und graphisch veranschaulicht.

Zuletzt beschäftigen wir uns mit der Mächtigkeit von Mengen und stellen fest, dass es zu viele reelle Zahlen gibt, als
dass wir sie  abzählen (durchnumerieren) könnten.}


\end{zusammenfassung}

\begin{block}[info]\lang{de}{\strong{Lernziele}}
\lang{en}{\strong{Learning Goals}} 
\begin{itemize}[square]
\item \lang{de}{Sie kennen die Begriffe Definitionsbereich, Wertebereich und Bild einer Funktion und geben diese in Beispielen an.}
\item \lang{de}{Sie verknüpfen Funktionen auf verschiedene Weisen.}
\item \lang{de}{Sie prüfen reelle Funktionen und ihre Funktionsgraphen auf Eigenschaften wie Injetivität, Surjektivität, Bijektivität, Monotonie, Nullstellen, Beschränktheit.}
\item \lang{de}{Sie skizzieren den Graphen einer umkehrbaren reellen Funktion.}
\item \lang{de}{Sie kennen den Begriff der Überabzählbarkeit und wissen, weshalb die reellen Zahlen diese Eigenschaft haben.}
\end{itemize}
\end{block}




\end{content}
