\documentclass{mumie.problem.gwtmathlet}
%$Id$
\begin{metainfo}
  \name{
    \lang{de}{A08: Eigenschaften}
    \lang{en}{Problem 8}
  }
  \begin{description} 
 This work is licensed under the Creative Commons License Attribution 4.0 International (CC-BY 4.0)   
 https://creativecommons.org/licenses/by/4.0/legalcode 

    \lang{de}{Beschreibung}
    \lang{en}{}
  \end{description}
  \corrector{system/problem/GenericCorrector.meta.xml}
  \begin{components}
    \component{js_lib}{system/problem/GenericMathlet.meta.xml}{mathlet}
  \end{components}

  \creategeneric
\end{metainfo}
\begin{content}
\begin{block}[annotation]
	Im Ticket-System: \href{https://team.mumie.net/issues/22805}{Ticket 22805}
\end{block}
\usepackage{mumie.genericproblem}
\lang{de}{\title{A08: Eigenschaften}}
\lang{en}{\title{Problem 8}}


\begin{problem}

	\begin{question}
      
      
      \begin{variables}
			\randint{a}{2}{5}
			\randint{b}{12}{16}
			\function[calculate]{c}{a+2*(b-a)}
            \function[calculate]{d}{a+3*(b-a)}
            \function[calculate]{e}{a+4*(b-a)}
            \function[calculate]{g}{2*(b-a)}
		\end{variables}
        
        \type{mc.multiple}	
        
          \text{Sei $f:A\to B, \text{ } x\mapsto \begin{cases}
                                            x & \text{ falls } x\notin C,\\
                                            x+\var{g} & \text{ falls }x\in C 
                                            \end{cases}$
            eine bijektive Abbildung, wobei $C:=\{\var{a};\var{b};\var{c};\var{d};\var{e};...\}$. Entscheiden Sie, welche Aussagen
            über $A$ und $B$ zutreffen können.}
            \explanation{Überprüfen Sie, ob tatsächlich für jedes $x\in A$ ein $y\in B$ mit $y=f(x)$ und für jedes $y\in B$ ein 
            $x\in A$ mit $y=f(x)$  existiert.}


		\begin{choice}
			\text{$A=C$ und $B=\N\setminus\{\var{a};\var{b}\}$.}
			\solution{false}
		\end{choice}
        \begin{choice}
			\text{$A=C$ und $B=\N\setminus C$.}
			\solution{false}
		\end{choice}
        \begin{choice}
			\text{$A=\N$ und $B=\N\setminus C$.}
			\solution{false}
		\end{choice}
        \begin{choice}
			\text{$A=\N$ und $B=\N\setminus\{\var{a};\var{b}\}$.}
			\solution{true}
		\end{choice}
        \begin{choice}
			\text{$A=\Q$ und $B=C$.}
			\solution{false}
		\end{choice}
        \begin{choice}
			\text{$A=\Q$ und $B=\Q\setminus\{\var{a};\var{b}\}$.}
			\solution{true}
		\end{choice}
        \begin{choice}
			\text{$A=\Q$ und $B=\Q\setminus C$.}
			\solution{false}
		\end{choice}
              \begin{choice}
			\text{$A=\R$ und $B=C$.}
			\solution{false}
		\end{choice}
        \begin{choice}
			\text{$A=\R$ und $B=\R\setminus C$.}
			\solution{false}
		\end{choice}
        \begin{choice}
			\text{$A=\R$ und $B=\R\setminus\{\var{a};\var{b}\}$.}
			\solution{true}
		\end{choice}
    \end{question}
        
    \begin{question}
        
        \type{mc.multiple}
        \text{Welche Schlussfolgerungen über $A$ und $B$ können aus der Existenz dieser bijektiven Abbildung gezogen werden?\\
        Dass $A$ und $B$..}
        
        
        
        \begin{choice}
			\text{vollständig sind.}
			\solution{false}
		\end{choice}
        \begin{choice}
			\text{beschränkt sind.}
			\solution{false}
		\end{choice}
        \begin{choice}
			\text{$A$ und $B$ beschränkt sind.}
			\solution{false}
		\end{choice}
        \begin{choice}
			\text{abzählbar sind.}
			\solution{false}
		\end{choice}
        \begin{choice}
			\text{gleiche Mächtigkeit haben.}
			\solution{true}
		\end{choice}
        \begin{choice}
			\text{überabzählbar sind.}
			\solution{false}
		\end{choice}
  

        
	\end{question}	
	
	
\end{problem}

\embedmathlet{mathlet}
\end{content}