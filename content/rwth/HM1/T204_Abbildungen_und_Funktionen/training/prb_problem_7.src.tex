\documentclass{mumie.problem.gwtmathlet}
%$Id$
\begin{metainfo}
  \name{
    \lang{de}{A07: Eigenschaften}
    \lang{en}{Problem 7}
  }
  \begin{description} 
 This work is licensed under the Creative Commons License Attribution 4.0 International (CC-BY 4.0)   
 https://creativecommons.org/licenses/by/4.0/legalcode 

    \lang{de}{Beschreibung}
    \lang{en}{}
  \end{description}
  \corrector{system/problem/GenericCorrector.meta.xml}
  \begin{components}
    \component{js_lib}{system/problem/GenericMathlet.meta.xml}{mathlet}
  \end{components}
  \begin{links}
    \link{generic_article}{content/rwth/HM1/T204_Abbildungen_und_Funktionen/g_art_content_11_injektiv_surjektiv_bijektiv.meta.xml}{content_11_injektiv_surjektiv_bijektiv}
    \link{generic_article}{content/rwth/HM1/T204_Abbildungen_und_Funktionen/g_art_content_12_reelle_funktionen_monotonie.meta.xml}{MonotonieTeil}
  \end{links}
  \creategeneric
\end{metainfo}
\begin{content}
\usepackage{mumie.genericproblem}
\lang{de}{\title{A07: Eigenschaften}}
\lang{en}{\title{Problem 7}}
\begin{block}[annotation]
  Im Ticket-System: \href{http://team.mumie.net/issues/9818}{Ticket 9818}
\end{block}

\begin{problem}
\randomquestionpool{1}{5}
	\begin{question}
      \explanation[equalChoice(1???1) OR equalChoice(?1??1) OR equalChoice(??1?1) OR equalChoice(???11)]{Ihre Antworten schließen sich gegenseitig aus.}
      \explanation[equalChoice(00000)]{Eine Antwort muss zutreffen.}
      \explanation[equalChoice(11??0)]{Die Eigenschaften 'streng monoton wachsend' und 'streng monoton fallend' können nicht gleichzeitig erfüllt sein.}
      \explanation[equalChoice(10??0) OR equalChoice(01??0) OR equalChoice(00001)]{Die gegebene Funktion ist ein Polynom. Beachten Sie die Koeffizienten der Funktion. Bei der Beantwortung der Frage sind zwingend Definitions- und Zielbereich zu berücksichtigen. Für eine erste Vermutung kann auch die Betrachtung des Funktionsgraphen helfen. }
      \explanation[equalChoice(100?0) OR equalChoice(010?0)]{Eine streng monotone Funktion ist stets injektiv.}
      \explanation[equalChoice(???10) AND a<0]{Surjektiv ist die Abbildung, wenn Wertemenge und Zielbereich übereinstimmen. Beispielsweise ist die Zahl $1$ ein Element des Zielbereichs, wird aber nicht von $f$ zugeordnet, somit kann $f$ nicht surjektiv sein.}
      \explanation[equalChoice(???10) AND a>0]{Surjektiv ist die Abbildung, wenn Wertemenge und Zielbereich übereinstimmen. Beispielsweise ist die Zahl $-1$ ein Element des Zielbereichs, wird aber nicht von $f$ zugeordnet, somit kann $f$ nicht surjektiv sein.}
      %\explanation{Verwenden sie die Definitionen für die \ref[MonotonieTeil][Monotonie]{def:monotonie} \ref[InSurBiTeil][bla]{def:in-sur-bi}}
      %\explanation{Eine Funktion $f:D\rightarrow \R$ ist monoton wachsend, wenn für ein beliebiges Paar $x_1,x_2\in D$, für die die Ungleichung $x_1\leq x_2$ gilt, die Ungleichung $f(x_1)\leq f(x_2)$ für die zugehörigen Funktionswerte folgt.}		\type{mc.unique}
		\begin{variables}
			\randint[Z]{a}{-5}{5}
			\randint{b}{2}{3}
			\function[normalize]{f}{a*x^b}
		\end{variables}
        \type{mc.multiple}	
          \text{Sei $f: \R_+ \to \R, x\mapsto \var{f}$. Entscheiden Sie, welche Aussagen zutreffen.}
        
		\begin{choice}
			\text{$f$ ist streng monoton wachsend.}
			\solution{compute}
			\iscorrect{a}{>}{0}
		\end{choice}
		\begin{choice}
			\text{$f$ ist streng monoton fallend.}
			\solution{compute}
			\iscorrect{a}{<}{0}
		\end{choice}
        \begin{choice}
            \text{$f$ ist injektiv.}
            \solution{true}
        \end{choice}
        \begin{choice}
            \text{$f$ ist surjektiv.}
            \solution{false}
        \end{choice}
        \begin{choice}
            \text{Keine der Aussagen ist wahr.}
            \solution{false}
        \end{choice}
	\end{question}	

	\begin{question}
      \explanation[equalChoice(1???1) OR equalChoice(?1??1) OR equalChoice(??1?1) OR equalChoice(???11)]{Ihre Antworten schließen sich gegenseitig aus.}
      \explanation[equalChoice(00000)]{Eine Antwort muss zutreffen.}
      \explanation[equalChoice(11??0)]{Die Eigenschaften 'streng monoton wachsend' und 'streng monoton fallend' können nicht gleichzeitig erfüllt sein.}
      \explanation[equalChoice(10??0) OR equalChoice(01??0) OR equalChoice(00001)]{Die gegebene Funktion ist ein Polynom. Beachten Sie die Koeffizienten der Funktion. Bei der Beantwortung der Frage sind zwingend Definitions- und Zielbereich zu berücksichtigen. Für eine erste Vermutung kann auch die Betrachtung des Funktionsgraphen helfen. }
      \explanation[equalChoice(100?0) OR equalChoice(010?0)]{Eine streng monotone Funktion ist stets injektiv.}
      \explanation[equalChoice(???10) AND a<0]{Surjektiv ist die Abbildung, wenn Wertemenge und Zielbereich übereinstimmen. Beispielsweise ist die Zahl $1$ ein Element des Zielbereichs, wird aber nicht von $f$ zugeordnet, somit kann $f$ nicht surjektiv sein.}
      \explanation[equalChoice(???10) AND a>0]{Surjektiv ist die Abbildung, wenn Wertemenge und Zielbereich übereinstimmen. Beispielsweise ist die Zahl $-1$ ein Element des Zielbereichs, wird aber nicht von $f$ zugeordnet, somit kann $f$ nicht surjektiv sein.}
      %\explanation{Eine Funktion $f:D\rightarrow \R$ ist monoton wachsend, wenn für ein beliebiges Paar $x_1,x_2\in D$, für die die Ungleichung $x_1\leq x_2$ gilt, die Ungleichung $f(x_1)\leq f(x_2)$ für die zugehörigen Funktionswerte folgt.}		\type{mc.unique}
		\begin{variables}
			\randint[Z]{a}{-5}{5}
			\function[normalize]{f}{a*x^2}
		\end{variables}
		\type{mc.multiple}
		\text{Sei $f: \R_- \to \R, x\mapsto \var{f}$. Entscheiden Sie, welche Aussagen zutreffen.}
		\begin{choice}
			\text{$f$ ist streng monoton wachsend.}
			\solution{compute}
			\iscorrect{a}{<}{0}
		\end{choice}
		\begin{choice}
			\text{$f$ ist streng monoton fallend.}
			\solution{compute}
			\iscorrect{a}{>}{0}
		\end{choice}
        \begin{choice}
            \text{$f$ ist injektiv.}
            \solution{true}
        \end{choice}
        \begin{choice}
            \text{$f$ ist surjektiv.}
            \solution{false}
        \end{choice}
        \begin{choice}
            \text{Keine der Aussagen ist wahr.}
            \solution{false}
        \end{choice}
	\end{question}
    
	\begin{question}
      \explanation[equalChoice(1???1) OR equalChoice(?1??1) OR equalChoice(??1?1) OR equalChoice(???11)]{Ihre Antworten schließen sich gegenseitig aus.}
      \explanation[equalChoice(00000)]{Eine Antwort muss zutreffen.}
      \explanation[equalChoice(11??0)]{Die Eigenschaften 'streng monoton wachsend' und 'streng monoton fallend' können nicht gleichzeitig erfüllt sein.}
      \explanation[equalChoice(10??0) OR equalChoice(01??0) OR equalChoice(00001)]{Die gegebene Funktion ist ein Polynom. Beachten Sie die Koeffizienten der Funktion. Bei der Beantwortung der Frage sind zwingend Definitions- und Zielbereich zu berücksichtigen. Für eine erste Vermutung kann auch die Betrachtung des Funktionsgraphen helfen. }
      \explanation[equalChoice(100?0) OR equalChoice(010?0)]{Eine streng monotone Funktion ist stets injektiv.}
      \explanation[equalChoice(???10) AND a<0]{Surjektiv ist die Abbildung, wenn Wertemenge und Zielbereich übereinstimmen. Beispielsweise ist die Zahl $-1$ ein Element des Zielbereichs, wird aber nicht von $f$ zugeordnet, somit kann $f$ nicht surjektiv sein.}
      \explanation[equalChoice(???10) AND a>0]{Surjektiv ist die Abbildung, wenn Wertemenge und Zielbereich übereinstimmen. Beispielsweise ist die Zahl $1$ ein Element des Zielbereichs, wird aber nicht von $f$ zugeordnet, somit kann $f$ nicht surjektiv sein.}
      %\explanation{Eine Funktion $f:D\rightarrow \R$ ist monoton wachsend, wenn für ein beliebiges Paar $x_1,x_2\in D$, für die die Ungleichung $x_1\leq x_2$ gilt, die Ungleichung $f(x_1)\leq f(x_2)$ für die zugehörigen Funktionswerte folgt.}		\type{mc.unique}
		\begin{variables}
			\randint[Z]{a}{-5}{5}
			\function[normalize]{f}{a*x^3}
		\end{variables}
		\type{mc.multiple}
		\text{$f: \R_- \to \R, x\mapsto \var{f}$ ist monoton wachsend.}
		\begin{choice}
			\text{$f$ ist streng monoton wachsend.}
			\solution{compute}
			\iscorrect{a}{>}{0}
		\end{choice}
		\begin{choice}
			\text{$f$ ist streng monoton fallend.}
			\solution{compute}
			\iscorrect{a}{<}{0}
		\end{choice}
        \begin{choice}
            \text{$f$ ist injektiv.}
            \solution{true}
        \end{choice}
        \begin{choice}
            \text{$f$ ist surjektiv.}
            \solution{false}
        \end{choice}
        \begin{choice}
            \text{Keine der Aussagen ist wahr.}
            \solution{false}
        \end{choice}
	\end{question}
	
	\begin{question}
      \explanation[equalChoice(1???1) OR equalChoice(?1??1) OR equalChoice(??1?1) OR equalChoice(???11)]{Ihre Antworten schließen sich gegenseitig aus.}
      \explanation[equalChoice(00000)]{Eine Antwort muss zutreffen.}
      \explanation[equalChoice(11??0)]{Die Eigenschaften 'streng monoton wachsend' und 'streng monoton fallend' können nicht gleichzeitig erfüllt sein.}
      \explanation[equalChoice(10??0) OR equalChoice(01??0) OR equalChoice(00001)]{Die gegebene Funktion ist ein Polynom. Beachten Sie die Koeffizienten der Funktion. Bei der Beantwortung der Frage sind zwingend Definitions- und Zielbereich zu berücksichtigen. Für eine erste Vermutung kann auch die Betrachtung des Funktionsgraphen helfen. }
      \explanation[equalChoice(100?0) OR equalChoice(010?0)]{Eine streng monotone Funktion ist stets injektiv.}
      \explanation[equalChoice(???10) AND a<0]{Surjektiv ist die Abbildung, wenn Wertemenge und Zielbereich übereinstimmen. Beispielsweise ist die Zahl $11$ ein Element des Zielbereichs, wird aber nicht von $f$ zugeordnet, somit kann $f$ nicht surjektiv sein.}
      \explanation[equalChoice(???10) AND a>0]{Surjektiv ist die Abbildung, wenn Wertemenge und Zielbereich übereinstimmen. Beispielsweise ist die Zahl $-11$ ein Element des Zielbereichs, wird aber nicht von $f$ zugeordnet, somit kann $f$ nicht surjektiv sein.}
      %\explanation{Eine Funktion $f:D\rightarrow \R$ ist monoton wachsend, wenn für ein beliebiges Paar $x_1,x_2\in D$, für die die Ungleichung $x_1\leq x_2$ gilt, die Ungleichung $f(x_1)\leq f(x_2)$ für die zugehörigen Funktionswerte folgt.}		\type{mc.unique}
		\begin{variables}
			\randint[Z]{a}{-5}{5}
			\randint{d}{-10}{10}
			\function[normalize]{f}{a*x^2+a*x+d}
		\end{variables}
		\type{mc.multiple}
		\text{Sei $f: \R_+ \to \R, x\mapsto \var{f}$. Entscheiden Sie, welche Aussagen zutreffen.}
        
		\begin{choice}
			\text{$f$ ist streng monoton wachsend.}
			\solution{compute}
			\iscorrect{a}{>}{0}
		\end{choice}
		\begin{choice}
			\text{$f$ ist streng monoton fallend.}
			\solution{compute}
			\iscorrect{a}{<}{0}
		\end{choice}
        \begin{choice}
            \text{$f$ ist injektiv.}
            \solution{true}
        \end{choice}
        \begin{choice}
            \text{$f$ ist surjektiv.}
            \solution{false}
        \end{choice}
        \begin{choice}
            \text{Keine der Aussagen ist wahr.}
            \solution{false}
        \end{choice}
	\end{question}
	
	\begin{question}
      \explanation[equalChoice(1???1) OR equalChoice(?1??1) OR equalChoice(??1?1) OR equalChoice(???11)]{Ihre Antworten schließen sich gegenseitig aus.}
      \explanation[equalChoice(00000)]{Eine Antwort muss zutreffen.}
      \explanation[equalChoice(11??0)]{Die Eigenschaften 'streng monoton wachsend' und 'streng monoton fallend' können nicht gleichzeitig erfüllt sein.}
      \explanation[equalChoice(10??0) OR equalChoice(01??0) OR equalChoice(00001)]{Die gegebene Funktion ist ein Polynom. Beachten Sie die Koeffizienten der Funktion. Bei der Beantwortung der Frage sind zwingend Definitions- und Zielbereich zu berücksichtigen. Für eine erste Vermutung kann auch die Betrachtung des Funktionsgraphen helfen. }
      \explanation[equalChoice(100?0) OR equalChoice(010?0)]{Eine streng monotone Funktion ist stets injektiv.}
      %\explanation{Eine Funktion $f:D\rightarrow \R$ ist monoton fallend, wenn für ein beliebiges Paar $x_1,x_2\in D$, für die die Ungleichung $x_1\leq x_2$ gilt, die Ungleichung $f(x_1)\geq f(x_2)$ für die zugehörigen Funktionswerte folgt.}
		\type{mc.multiple}
		\begin{variables}
			\randint[Z]{c}{-5}{5}
			\randint{d}{-10}{10}
			\function[normalize]{f}{c*x^3+d}
		\end{variables}
		
		\text{Sei $f: \R \to \R, x\mapsto \var{f}$. Entscheiden Sie, welche Aussagen zutreffen.}
        
		\begin{choice}
			\text{$f$ ist streng monoton wachsend.}
			\solution{compute}
			\iscorrect{c}{>}{0}
		\end{choice}
		\begin{choice}
			\text{$f$ ist streng monoton fallend.}
			\solution{compute}
			\iscorrect{c}{<}{0}
		\end{choice}
        \begin{choice}
            \text{$f$ ist injektiv.}
            \solution{true}
        \end{choice}
        \begin{choice}
            \text{$f$ ist surjektiv.}
            \solution{true}
        \end{choice}
        \begin{choice}
            \text{Keine der Aussagen ist wahr.}
            \solution{false}
        \end{choice}
	\end{question}

\end{problem}

\embedmathlet{mathlet}
\end{content}
