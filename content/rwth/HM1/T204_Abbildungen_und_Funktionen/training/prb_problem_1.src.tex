\documentclass{mumie.problem.gwtmathlet}
%$Id$
\begin{metainfo}
  \name{
    \lang{de}{A01: Abbildungen}
    \lang{en}{Problem 1}
  }
  \begin{description} 
 This work is licensed under the Creative Commons License Attribution 4.0 International (CC-BY 4.0)   
 https://creativecommons.org/licenses/by/4.0/legalcode 

    \lang{de}{Beschreibung}
    \lang{en}{}
  \end{description}
  \corrector{system/problem/GenericCorrector.meta.xml}
  \begin{components}
    \component{js_lib}{system/problem/GenericMathlet.meta.xml}{mathlet}
  \end{components}
  \begin{links}
  \end{links}
  \creategeneric
\end{metainfo}
\begin{content}
\usepackage{mumie.genericproblem}
\lang{de}{\title{A01: Abbildungen}}
\lang{en}{\title{Problem 1}}
\begin{block}[annotation]
  Im Ticket-System: \href{http://team.mumie.net/issues/9812}{Ticket 9812}
\end{block}

\begin{problem}
 \randomquestionpool{1}{4}
 \randomquestionpool{5}{6}
% Q1 
  \begin{question}
  \text{Sind die folgenden Zuordnungen Abbildungen?}
  \explanation{Eine Abbildung ist eine Zuordnung zwischen zwei Mengen, 
  die jedem Element des Definitionsbereich genau ein Element des Zielbereichs zuordnet.}
  
  \permutechoices{1}{2}
      \type{mc.yesno}
   \begin{variables}
   			\randint{a}{1}{5}
   			\randint[Z]{b}{-3}{3}
   			\randint{c}{3}{7}
   			\function[normalize]{fn}{a*n}
       \end{variables}
       
  \begin{choice}
      \text{$T:\R\to \R, x\mapsto \sqrt{x}$}
      \solution{false}
	  \end{choice}
    \begin{choice}
      \text{$T:\N\to \N, n\mapsto \var{fn}$}
      \solution{true}
	\end{choice}
	\explanation[equalChoice(1?)]{Die Wurzel einer negativen reellen Zahl ist keine reelle Zahl bzw. zunächst auch nicht definiert.}	
 \end{question}
% Q2 
 \begin{question}
  \text{Sind die folgenden Zuordnungen Abbildungen?}
  \explanation{Eine Abbildung ist eine Zuordnung zwischen zwei Mengen, die jedem Element des Definitionsbereich genau ein Element des Zielbereichs zuordnet.}
  
  \permutechoices{1}{2}
      \type{mc.yesno}
   \begin{variables}
   			\randint{a}{-3}{5}
   			\randint[Z]{b}{-3}{3}
   			\randint{c}{3}{7}
       \end{variables}	
      \begin{choice}
      \text{$T:\Z\to \Q, n\mapsto \frac{\var{b}}{n}$}
      \solution{false}
		\end{choice}
      \begin{choice}
      \text{$S:\{1;\ldots; \var{c}\} \to \{1;\ldots; \var{c}\},i\mapsto $ alle 
      Zahlen der Menge $\ne i$}
      \solution{false}
		\end{choice}
      \explanation[equalChoice(1?)]{$T(0)$ ist nicht definiert, da nicht durch $0$ geteilt werden darf.}
      \explanation[equalChoice(?1)]{$S$ bildet jeweils auf mehrere Elemente (Zahlen) ab.}
  \end{question}
% Q3  
 \begin{question}
  \text{Sind die folgenden Zuordnungen Abbildungen?}
  \explanation{Eine Abbildung ist eine Zuordnung zwischen zwei Mengen, die jedem Element des Definitionsbereich genau ein Element des Zielbereichs zuordnet.}
  
  \permutechoices{1}{2}
      \type{mc.yesno}
   \begin{variables}
   			\randint{a}{-3}{5}
   			\randint[Z]{b}{-3}{3}
   			\randint{c}{3}{7}
       \end{variables}
       
  \begin{choice}
      \text{$S:\{x\in \R\,|\, x>0\} \to \R, x\mapsto \pm \sqrt{x}$}
      \solution{false}
		\end{choice}
      \begin{choice}
      \text{$T:\R\to \Z, r\mapsto $ größte ganze Zahl $\leq r$}
      \solution{true}
		\end{choice}
	\explanation[equalChoice(1?)]{$S(1)=\pm 1$, d.h. dem Element $1$ werden zwei Zahlen zugeordnet.}
\end{question}
% Q4
 \begin{question}
  \text{Sind die folgenden Zuordnungen Abbildungen?}
  \explanation{Eine Abbildung ist eine Zuordnung zwischen zwei Mengen, die jedem Element des Definitionsbereich genau ein Element des Zielbereichs zuordnet.}
  
  \permutechoices{1}{2}
      \type{mc.yesno}
   \begin{variables}
   			\randint{a}{-3}{5}
   			\randint[Z]{b}{-3}{3}
   			\randint{c}{3}{7}
       \end{variables}	
      \begin{choice}
            \text{$T:\Q\to \Z,$ jedem Bruch wird sein Nenner zugeordnet.}
            \solution{false}
		\end{choice}
        \begin{choice}
      \text{$S:\{1;2\} \to \{1;2\},i\mapsto $ alle 
      Zahlen der Menge $\ne i$}
        \solution{true}
      \end{choice}
      \explanation[equalChoice(1?)]{Es ist zum Beispiel $\frac{1}{2}=\frac{2}{4}=\frac{-1}{-2}$. Um die Zuordnung $T(x)$ eindeutig zu machen, müsste es heißen: Jedem \textbf{vollständig gekürzten} Bruch wird sein (positiver) Nenner zugeordnet.}
      \explanation[equalChoice(?1)]{$S$ bildet jeweils auf mehrere Elemente (Zahlen) ab.}

  \end{question}
% Q5  
  \begin{question}
  \text{Bestimmen Sie die Wertemenge der Abbildung \\
  $ f:\{1;3; \var{b}; \var{c}\}\to \N, n\mapsto \var{fn} $ }
  \explanation{Die Wertemenge besteht aus den Elementen des Zielbereichs, die Funktionswerte
  sind, d.h. aus $f(1); f(3); \cdots$.}
  
    \precision{0}
  
      \type{input.finite-number-set}
   \begin{variables}
   			\randint{a}{1}{5}
   			\randint{b}{4}{6}
   			\randint{c}{7}{10}
   			\function[normalize]{fn}{a*n}
   			\function[calculate]{s1}{a}
   			\function[calculate]{s2}{3*a}
   			\function[calculate]{s3}{a*b}
   			\function[calculate]{s4}{a*c} 			
       \end{variables}
  \begin{answer}
  \text{Die Wertemenge ist $W_f=$ }
  \solution{s1, s2, s3, s4}
\end{answer}
\end{question}
% Q6 
 \begin{question}
  \text{Bestimmen Sie die Wertemenge der Abbildung \\
  $ f:\{3; \var{b}; \var{c}\}\to \N, n\mapsto \var{d}-\var{fn}  $}
  \explanation{Die Wertemenge besteht aus den Elementen des Zielbereichs, die Funktionswerte
  sind, d.h. aus $f(3); f(\var{b}); \cdots$.}
  
  \precision{0}
      \type{input.finite-number-set}
   \begin{variables}
   			\randint{a}{1}{5}
   			\randint{b}{4}{6}
   			\randint{c}{7}{10}
   			\randint{f}{4}{10}
   			\function[normalize]{fn}{a*n}
   			\function[calculate]{d}{a*c+f}
   			\function[calculate]{s2}{d-3*a}
   			\function[calculate]{s3}{d-a*b}
   			\function[calculate]{s4}{d-a*c} 			
       \end{variables}
  \begin{answer}
  \text{Die Wertemenge ist $W_f=$ }
  \solution{s2, s3, s4}
\end{answer}
\end{question}
\end{problem}
\embedmathlet{mathlet}
\end{content}