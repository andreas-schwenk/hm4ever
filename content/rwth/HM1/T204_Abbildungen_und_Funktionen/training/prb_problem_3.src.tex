\documentclass{mumie.problem.gwtmathlet}
%$Id$
\begin{metainfo}
  \name{
    \lang{de}{A03: Kompositionen}
    \lang{en}{Problem 3}
  }
  \begin{description} 
 This work is licensed under the Creative Commons License Attribution 4.0 International (CC-BY 4.0)   
 https://creativecommons.org/licenses/by/4.0/legalcode 

    \lang{de}{Beschreibung}
    \lang{en}{}
  \end{description}
  \corrector{system/problem/GenericCorrector.meta.xml}
  \begin{components}
    \component{js_lib}{system/problem/GenericMathlet.meta.xml}{mathlet}
  \end{components}
  \begin{links}
  \end{links}
  \creategeneric
\end{metainfo}
\begin{content}
\usepackage{mumie.genericproblem}
\lang{de}{\title{A03: Kompositionen}}
\lang{en}{\title{Problem 3}}
\begin{block}[annotation]
  Im Ticket-System: \href{http://team.mumie.net/issues/9814}{Ticket 9814}
\end{block}

\begin{problem}

	\randomquestionpool{1}{2}
      \begin{question}
      
      \begin{variables}
      		\randint{a}{-3}{3}
      		\randint[Z]{b}{-2}{2}
      		\randint{c}{1}{4}
      		\randint[Z]{d}{-2}{2}
      		\function[expand,normalize]{f}{x^2+a*x+b}
      		\function[expand,normalize]{g}{c*x+d}
      		\function[expand,normalize]{fg}{(c*x+d)^2+a*(c*x+d)+b}
      		\function[expand,normalize]{gf}{c*(x^2+a*x+b)+d}
      \end{variables}

      \type{input.function}
      \field{real}
      \text{Bestimmen Sie die Kompositionen $f\circ g$ und $g\circ f$ der Abbildungen\\
      $f:\R\to \R, x\mapsto \var{f}$ und $g:\R\to \R, x\mapsto \var{g}$.\\
      Multiplizieren Sie alle Terme aus.}     

      \begin{answer}
          \text{$(f\circ g)(x)=$}
          \solution{fg}
		  \allowForInput[false]{)}
          \checkAsFunction{x}{-10}{10}{10}
          \explanation{$(f\circ g)(x)=f(g(x))$. Man muss also in die Vorschrift für $f$ statt $x$ den Ausdruck für $g(x)$ einsetzen. Multiplizieren Sie anschließend aus.}
      \end{answer}
      
      \begin{answer}
		 \text{$(g\circ f)(x)=$}
          \solution{gf}
		  \allowForInput[false]{)}
          \checkAsFunction{x}{-10}{10}{10}
          \explanation{$(g\circ f)(x)=g(f(x))$. Man muss also in die Vorschrift für $g$ statt $x$ den Ausdruck für $f(x)$ einsetzen. Multiplizieren Sie anschließend aus.}
      \end{answer}
  \end{question} 


  \begin{question}
  \explanation{$(T\circ S)(n)=T(S(n))$. Man muss also in die Vorschrift für $T$ statt $x$ den Ausdruck für $S(n)$ einsetzen.}  
  \begin{variables}
  	\randint{x1}{0}{2}
  	\randint{x2}{3}{5}
  	\randint{x3}{6}{7}
  	\randint{a}{2}{4}
  	\randint{b}{1}{3}
  	\randint{m}{1}{2}
	\function[calculate]{d}{a*x3+b+m}  
	\function[calculate]{d1}{d+1}  
	\function[normalize]{sn}{a*n+b}  
	\function{ts}{d1-sn}  
	\substitute{y1}{ts}{n}{x1}  
	\substitute{y2}{ts}{n}{x2}  
	\substitute{y3}{ts}{n}{x3}  
  \end{variables}

      \type{input.number}
      \field{real}
      \text{Gegeben seien die Abbildungen\\
      $S:\{\var{x1};\var{x2};\var{x3} \}\to \{1;\ldots ; \var{d}\}, n\mapsto \var{sn}$ und\\
      $T:\{1;\ldots ; \var{d}\}\to \{1;\ldots ; \var{d}\}, x\mapsto \var{d1}-x$.\\
      Bestimmen Sie die Funktionswerte der zusammengesetzten Abbildung $T\circ S$.}      

      
      \begin{answer}
      \text{$(T\circ S)(\var{x1})=$}
      \solution{y1}
      \end{answer}
      
      \begin{answer}
      \text{$(T\circ S)(\var{x2})=$}
      \solution{y2}
      \end{answer}
      
      \begin{answer}
      \text{$(T\circ S)(\var{x3})=$}
      \solution{y3}
      \end{answer}
  \end{question} 

\end{problem}

\embedmathlet{mathlet}
\end{content}