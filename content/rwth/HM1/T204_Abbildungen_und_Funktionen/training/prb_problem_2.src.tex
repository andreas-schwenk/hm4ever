\documentclass{mumie.problem.gwtmathlet}
%$Id$
\begin{metainfo}
  \name{
    \lang{de}{A02: Abbildungen}
    \lang{en}{Problem 2}
  }
  \begin{description} 
 This work is licensed under the Creative Commons License Attribution 4.0 International (CC-BY 4.0)   
 https://creativecommons.org/licenses/by/4.0/legalcode 

    \lang{de}{Beschreibung}
    \lang{en}{}
  \end{description}
  \corrector{system/problem/GenericCorrector.meta.xml}
  \begin{components}
    \component{generic_image}{content/rwth/HM1/images/g_tkz_T204_Problem02_B.meta.xml}{T204_Problem02_B}
    \component{generic_image}{content/rwth/HM1/images/g_tkz_T204_Problem02_C.meta.xml}{T204_Problem02_C}
    \component{generic_image}{content/rwth/HM1/images/g_tkz_T204_Problem02_A.meta.xml}{T204_Problem02_A}
    \component{js_lib}{system/problem/GenericMathlet.meta.xml}{mathlet}
  \end{components}
  \begin{links}
  \end{links}
  \creategeneric
\end{metainfo}
\begin{content}
\usepackage{mumie.genericproblem}
\lang{de}{\title{A02: Abbildungen}}
\lang{en}{\title{Problem 2}}
\begin{block}[annotation]
  Im Ticket-System: \href{http://team.mumie.net/issues/9813}{Ticket 9813}
\end{block}

\begin{problem}

\randomquestionpool{1}{3}
  \begin{question}
   \text{\begin{figure}\image{T204_Problem02_A}\end{figure}
   Welche der obigen Graphiken stellt eine Abbildung dar?
    }
  \explanation{Eine Abbildung ist eine Zuordnung zwischen zwei Mengen, 
    die jedem Element des Definitionsbereich genau ein Element des Zielbereichs 
    zuordnet. Graphisch gesehen bedeutet das, dass eine Abbildung dann vorliegt, 
    wenn von allen Elementen des Definitionsbereichs genau ein Pfeil in den 
    Zielbereich führt. Beachten Sie auch unbedingt die Richtung der Pfeile.}
   	
      \type{mc.yesno}
       \field{real}
      \precision{3}
      \begin{variables}
       \end{variables}
 		\begin{choice}
      \text{erste Graphik}
      \solution{false}
		\end{choice}
 		\begin{choice}
      \text{zweite Graphik}
      \solution{true}
		\end{choice}
 		\begin{choice}
      \text{dritte Graphik}
      \solution{true}
		\end{choice}
       %\explanation[equalChoice(??0)]{Schauen Sie sich noch einmal die dritte Graphik an und nehmen Sie die Menge $N$ als Definitionsbereich und $M$ als Zielbereich.}
  \end{question} 

\begin{question}
   \text{\begin{figure}\image{T204_Problem02_C}\end{figure}
   Die Graphik stellt eine Abbildung dar. Geben Sie Definitionsbereich, 
   Zielbereich und Wertemenge der Abbildung an.
    }
  %\explanation{Der Definitionsbereich besteht aus allen Elementen des Definitionsbereichs. Der Zielbereich besteht aus allen Elementen des Zielbereichs. Die Wertemenge besteht aus allen Elementen des Zielbereichs, die tatsächlich zugeordnet worden sind.}
   	
      \type{input.finite-number-set}
       \field{real}
      \precision{3}
      \begin{variables}
      \number{m1}{-1}
      \number{m2}{0}
      \number{m3}{2}
      \number{m4}{3}
      \number{m5}{7}
      \number{n1}{0}
      \number{n2}{1}
      \number{n3}{2}
      \number{n4}{4}
      \number{n5}{9}           
       \end{variables}
 		\begin{answer}
      \text{Definitionsbereich}
      \solution{m1,m2,m3,m4,m5}
      \explanation{Der Definitionsbereich besteht aus allen Elementen der Menge $M$.}
		\end{answer}
 		\begin{answer}
      \text{Zielbereich}
      \solution{n1,n2,n3,n4,n5}
      \explanation{Der Zielbereich besteht aus allen Elementen der Menge $N$.}
		\end{answer}
 		\begin{answer}
      \text{Wertemenge}
      \solution{n4}
      \explanation{Die Wertemenge besteht aus allen Elementen des Zielbereichs, die auch tatsächlich zugeordnet werden.}
		\end{answer}

  \end{question} 
\begin{question}
   \text{\begin{figure}\image{T204_Problem02_B}\end{figure}
   Die Graphik stellt eine Abbildung dar. Geben Sie Definitionsbereich, Zielbereich und
   Wertemenge der Abbildung an.
    }
  %\explanation{Der Definitionsbereich besteht aus allen Elementen des Definitionsbereichs. Der Zielbereich besteht aus allen Elementen des Zielbereichs. Die Wertemenge besteht aus allen Elementen des Zielbereichs, die tatsächlich zugeordnet worden sind.}
   	
      \type{input.finite-number-set}
       \field{real}
      \precision{3}
      \begin{variables}
      \number{m1}{0}
      \number{m2}{3}
      \number{n2}{1}
      \number{n3}{2}
      \number{n4}{4}
       \end{variables}
 		\begin{answer}
      \text{Definitionsbereich}
      \solution{n2,n3,n4}
      \explanation{Es ist eine Abbildung mit Definitionsbereich $N$ abgebildet.}
		\end{answer}
 		\begin{answer}
      \text{Zielbereich}
      \solution{m1,m2}
      \explanation{Der Zielbereich besteht hier aus allen Elementen der Menge $M$.}
		\end{answer}
 		\begin{answer}
      \text{Wertemenge}
      \solution{m1,m2}
      \explanation{Die Wertemenge besteht aus allen Elementen des Zielbereichs, die auch tatsächlich zugeordnet werden.}
		\end{answer}

  \end{question} 
\end{problem}

\embedmathlet{mathlet}
\end{content}