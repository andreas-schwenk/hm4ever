\documentclass{mumie.problem.gwtmathlet}
%$Id$
\begin{metainfo}
  \name{
    \lang{de}{A04: Eigenschaften}
    \lang{en}{Problem 4}
  }
  \begin{description} 
 This work is licensed under the Creative Commons License Attribution 4.0 International (CC-BY 4.0)   
 https://creativecommons.org/licenses/by/4.0/legalcode 

    \lang{de}{Beschreibung}
    \lang{en}{}
  \end{description}
  \corrector{system/problem/GenericCorrector.meta.xml}
  \begin{components}
    \component{js_lib}{system/problem/GenericMathlet.meta.xml}{mathlet}
  \end{components}
  \begin{links}
  \end{links}
  \creategeneric
\end{metainfo}
\begin{content}
\usepackage{mumie.genericproblem}
\lang{de}{\title{A04: Eigenschaften}}
\lang{en}{\title{Problem 4}}
\begin{block}[annotation]
  Im Ticket-System: \href{http://team.mumie.net/issues/9815}{Ticket 9815}
\end{block}

\begin{problem}

	\randomquestionpool{1}{6}
	\randomquestionpool{7}{12}
%1. Frage, von Pool 1 
	\begin{question}
    %\explanation{Abbildungen sind Zuordnungen zwischen zwei Mengen, die jedem Element des 
    %Definitionsbereich genau ein Element des Zielbereichs zuordnet.\\ 
    %Eine Abbildung ist injektiv, wenn für verschiedene Elemente $m_1$ und $m_2$ des 
    %Definitionsbereichs stets die Bilder $f(m_1)$ und $f(m_2)$ verschieden sind.\\
    %Eine Abbildung ist surjektiv, wenn Zielbereich und Wertemenge sind identisch.\\
    %Eine Abbildung ist bijektiv, wenn die Abbildung injektiv und surjektiv ist.}
		\begin{variables}
			\randint[Z]{k}{-10}{10}
			\function[expand, normalize]{f}{n+k}
		\end{variables}
		\text{Entscheiden Sie, welche Aussagen für die Zuordnung 
        $f: \mathbb{N} \to \mathbb{Z} , n \mapsto \var{f}$ korrekt sind und kreuzen Sie wahre Aussagen an.}
		\type{mc.multiple}
		\begin{choice}
  			\lang{de}{\text{$f$ ist keine Abbildung.}}
  			\solution{false}			
		\end{choice}
		\begin{choice}
  			\lang{de}{\text{$f$ ist eine Abbildung.}}
  			\solution{true}			
		\end{choice}
		\begin{choice}
  			\lang{de}{\text{$f$ ist injektiv.}}
  			\solution{true}			
		\end{choice}
		\begin{choice}
  			\lang{de}{\text{$f$ ist surjektiv.}}
  			\solution{false}			
		\end{choice}
		\begin{choice}
  			\lang{de}{\text{$f$ ist bijektiv.}}
  			\solution{false}			
		\end{choice}
        \explanation[equalChoice(00???)]{Eine Abbildung ist eine Zuordnung zwischen zwei Mengen, die jedem Element des Definitionsbereichs genau ein Element des Zielbereichs zuordnet.}
        \explanation[equalChoice(11???)]{Eine Abbildung ist eine Zuordnung zwischen zwei Mengen, die jedem Element des Definitionsbereichs genau ein Element des Zielbereichs zuordnet.}
        \explanation[equalChoice(10???)]{Eine Abbildung ist eine Zuordnung zwischen zwei Mengen, die jedem Element des Definitionsbereichs genau ein Element des Zielbereichs zuordnet.}
        \explanation[equalChoice(??0??)]{Die Abbildung ist injektiv, weil aus $f(m_1)=f(m_2)$ bereits $m_1=m_2$ folgt.}
        \explanation[equalChoice(???1?)]{Die Zahl $-42\in \Z$, ein Element des Zielbereichs, wird nicht von $f$ zugeordnet, somit kann $f$ nicht surjektiv sein.}
        \explanation[equalChoice(????1)]{Eine Abbildung ist bijektiv, wenn die Abbildung injektiv und surjektiv ist.}
	\end{question}
	
%2. Frage, von Pool 1 
	\begin{question}
  		\begin{variables}
			\randint{l}{-10}{10}
			\function[calculate]{s}{l^2}
			\function[expand, normalize]{f}{l*n}
		\end{variables}
		\text{Entscheiden Sie, welche Aussagen für die Zuordnung $f: \mathbb{N} \to \mathbb{Z} , n \mapsto \var{f}$ korrekt sind und kreuzen Sie wahre Aussagen an.}
		\type{mc.multiple}
		\begin{choice}
  			\lang{de}{\text{$f$ ist keine Abbildung.}}
  			\solution{false}			
		\end{choice}
		\begin{choice}
  			\lang{de}{\text{$f$ ist eine Abbildung.}}
  			\solution{true}			
		\end{choice}
		\begin{choice}
  			\lang{de}{\text{$f$ ist injektiv.}}
  			\solution{compute}
			\iscorrect{s}{>}{0}	
		\end{choice}
		\begin{choice}
  			\lang{de}{\text{$f$ ist surjektiv.}}
  			\solution{false}			
		\end{choice}
		\begin{choice}
  			\lang{de}{\text{$f$ ist bijektiv.}}
  			\solution{false}			
		\end{choice}
        \explanation[equalChoice(00???)]{Eine Abbildung ist eine Zuordnung zwischen zwei Mengen, die jedem Element des Definitionsbereichs genau ein Element des Zielbereichs zuordnet.}
        \explanation[equalChoice(11???)]{Eine Abbildung ist eine Zuordnung zwischen zwei Mengen, die jedem Element des Definitionsbereichs genau ein Element des Zielbereichs zuordnet.}
        \explanation[equalChoice(10???)]{Eine Abbildung ist eine Zuordnung zwischen zwei Mengen, die jedem Element des Definitionsbereichs genau ein Element des Zielbereichs zuordnet.}
        \explanation[equalChoice(??0??) AND s > 0]{Die Abbildung ist injektiv, weil aus $f(m_1)=f(m_2)$ bereits $m_1=m_2$ folgt.}
        \explanation[equalChoice(??1??) AND s = 0]{Die Abbildung bildet alle natürlichen Zahlen auf $0$ ab. Verschiedene Elemente des Definitionsbereichs werden also nicht auf verschiedene Elemente des Zielbereichs abgebildet.}
        \explanation[equalChoice(???1?)]{Surjektiv ist die Abbildung, wenn Wertemenge und Zielbereich übereinstimmen. Beispielsweise ist die Zahl $-11\in \Z$ ein Element des Zielbereichs, wird aber nicht von $f$ zugeordnet, somit kann $f$ nicht surjektiv sein.}
        \explanation[equalChoice(????1)]{Eine Abbildung ist bijektiv, wenn die Abbildung injektiv und surjektiv ist.}

	\end{question}	
	
%3. Frage, von Pool 1 
	\begin{question}
  		\text{Entscheiden Sie, welche Aussagen für die Zuordnung $f: \mathbb{R} \to \mathbb{R} , x \mapsto x^2$ korrekt sind und kreuzen Sie wahre Aussagen an.}
		\type{mc.multiple}
		\begin{choice}
  			\lang{de}{\text{$f$ ist keine Abbildung.}}
  			\solution{false}			
		\end{choice}
		\begin{choice}
  			\lang{de}{\text{$f$ ist eine Abbildung.}}
  			\solution{true}			
		\end{choice}
		\begin{choice}
  			\lang{de}{\text{$f$ ist injektiv.}}
  			\solution{false}
		\end{choice}
		\begin{choice}
  			\lang{de}{\text{$f$ ist surjektiv.}}
  			\solution{false}			
		\end{choice}
		\begin{choice}
  			\lang{de}{\text{$f$ ist bijektiv.}}
  			\solution{false}			
		\end{choice}
        \explanation[equalChoice(00???)]{Eine Abbildung ist eine Zuordnung zwischen zwei Mengen, die jedem Element des Definitionsbereichs genau ein Element des Zielbereichs zuordnet.}
        \explanation[equalChoice(11???)]{Eine Abbildung ist eine Zuordnung zwischen zwei Mengen, die jedem Element des Definitionsbereichs genau ein Element des Zielbereichs zuordnet.}
        \explanation[equalChoice(10???)]{Eine Abbildung ist eine Zuordnung zwischen zwei Mengen, die jedem Element des Definitionsbereichs genau ein Element des Zielbereichs zuordnet.}
        \explanation[equalChoice(??1??)]{Es ist $f(-1)=1=f(1)$. Das Element $1$ des Zielbereichs wird also mindestens zweimal angesprochen, folglich kann keine injektive Abbildung vorliegen.}
        \explanation[equalChoice(???1?)]{$-1\in \R$ ist ein Element des Zielbereichs, wird aber nicht von $f$ zugeordnet, somit kann $f$ nicht surjektiv sein.}
        \explanation[equalChoice(????1)]{Eine Abbildung ist bijektiv, wenn die Abbildung injektiv und surjektiv ist.}

	\end{question}
	%4. Frage, von Pool 1 
	\begin{question}
    	\text{Entscheiden Sie, welche Aussagen für die Zuordnung $f: \mathbb{R} \to \mathbb{R}_{\leq 0} , x \mapsto x^2$ korrekt sind und kreuzen Sie wahre Aussagen an.}
		\type{mc.multiple}
		\begin{choice}
  			\lang{de}{\text{$f$ ist keine Abbildung.}}
  			\solution{true}			
		\end{choice}
		\begin{choice}
  			\lang{de}{\text{$f$ ist eine Abbildung.}}
  			\solution{false}			
		\end{choice}
		\begin{choice}
  			\lang{de}{\text{$f$ ist injektiv.}}
  			\solution{false}
		\end{choice}
		\begin{choice}
  			\lang{de}{\text{$f$ ist surjektiv.}}
  			\solution{false}			
		\end{choice}
		\begin{choice}
  			\lang{de}{\text{$f$ ist bijektiv.}}
  			\solution{false}			
		\end{choice}
        \explanation[equalChoice(00???)]{Eine Abbildung ist eine Zuordnung zwischen zwei Mengen, die jedem Element des Definitionsbereichs genau ein Element des Zielbereichs zuordnet.}
        \explanation[equalChoice(11???)]{Eine Abbildung ist eine Zuordnung zwischen zwei Mengen, die jedem Element des Definitionsbereichs genau ein Element des Zielbereichs zuordnet.}
        \explanation[equalChoice(01???)]{Schauen Sie noch einmal auf den angegebenen Zielbereich!}
    \end{question}
	%5. Frage, von Pool 1 
	\begin{question}
  		\text{Entscheiden Sie, welche Aussagen für die Zuordnung $f: \mathbb{R} \to \mathbb{R}_{\geq 0} , x \mapsto x^2$ korrekt sind und kreuzen Sie wahre Aussagen an.}
		\type{mc.multiple}
		\begin{choice}
  			\lang{de}{\text{$f$ ist keine Abbildung.}}
  			\solution{false}			
		\end{choice}
		\begin{choice}
  			\lang{de}{\text{$f$ ist eine Abbildung.}}
  			\solution{true}			
		\end{choice}
		\begin{choice}
  			\lang{de}{\text{$f$ ist injektiv.}}
  			\solution{false}
		\end{choice}
		\begin{choice}
  			\lang{de}{\text{$f$ ist surjektiv.}}
  			\solution{true}			
		\end{choice}
		\begin{choice}
  			\lang{de}{\text{$f$ ist bijektiv.}}
  			\solution{false}			
		\end{choice}
        \explanation[equalChoice(00???)]{Eine Abbildung ist eine Zuordnung zwischen zwei Mengen, die jedem Element des Definitionsbereichs genau ein Element des Zielbereichs zuordnet.}
        \explanation[equalChoice(11???)]{Eine Abbildung ist eine Zuordnung zwischen zwei Mengen, die jedem Element des Definitionsbereichs genau ein Element des Zielbereichs zuordnet.}
        \explanation[equalChoice(10???)]{Eine Abbildung ist eine Zuordnung zwischen zwei Mengen, die jedem Element des Definitionsbereichs genau ein Element des Zielbereichs zuordnet.}
        \explanation[equalChoice(??1??)]{Es ist $f(-1)=1=f(1)$. Das Element $1$ des Zielbereichs wird also mindestens zweimal angesprochen, folglich kann keine injektive Abbildung vorliegen.}
        \explanation[equalChoice(???0?)]{Jede nicht-negative Zahl ist ein Quadrat einer reellen Zahl.}
        \explanation[equalChoice(????1)]{Eine Abbildung ist bijektiv, wenn die Abbildung injektiv und surjektiv ist.}

	\end{question}
%6. Frage, von Pool 1 
	\begin{question}
    	\text{Entscheiden Sie, welche Aussagen für die Zuordnung $f: \mathbb{R} \to \mathbb{Z} , x \mapsto x^2$ korrekt sind und kreuzen Sie wahre Aussagen an.}
		\type{mc.multiple}
		\begin{choice}
  			\lang{de}{\text{$f$ ist keine Abbildung.}}
  			\solution{true}			
		\end{choice}
		\begin{choice}
  			\lang{de}{\text{$f$ ist eine Abbildung.}}
  			\solution{false}			
		\end{choice}
		\begin{choice}
  			\lang{de}{\text{$f$ ist injektiv.}}
  			\solution{false}
		\end{choice}
		\begin{choice}
  			\lang{de}{\text{$f$ ist surjektiv.}}
  			\solution{false}			
		\end{choice}
		\begin{choice}
  			\lang{de}{\text{$f$ ist bijektiv.}}
  			\solution{false}			
		\end{choice}
        \explanation[equalChoice(00???)]{Eine Abbildung ist eine Zuordnung zwischen zwei Mengen, die jedem Element des Definitionsbereichs genau ein Element des Zielbereichs zuordnet.}
        \explanation[equalChoice(11???)]{Eine Abbildung ist eine Zuordnung zwischen zwei Mengen, die jedem Element des Definitionsbereichs genau ein Element des Zielbereichs zuordnet.}
        \explanation[equalChoice(01???)]{Schauen Sie noch einmal auf den Zielbereich! Bildet $f$ die reelle Zahl $\frac{1}{2}$ auf eine ganze Zahl ab?}
	\end{question}

	%7. Frage (Pool 2)
	\begin{question}
    	\text{Entscheiden Sie, welche Aussagen für die Zuordnung $f: \mathbb{R} \to \mathbb{R} , x \mapsto x^3$ korrekt sind und kreuzen Sie wahre Aussagen an.}
		\type{mc.multiple}
		\begin{choice}
  			\lang{de}{\text{$f$ ist keine Abbildung.}}
  			\solution{false}			
		\end{choice}
		\begin{choice}
  			\lang{de}{\text{$f$ ist eine Abbildung.}}
  			\solution{true}			
		\end{choice}
		\begin{choice}
  			\lang{de}{\text{$f$ ist injektiv.}}
  			\solution{true}
		\end{choice}
		\begin{choice}
  			\lang{de}{\text{$f$ ist surjektiv.}}
  			\solution{true}			
		\end{choice}
		\begin{choice}
  			\lang{de}{\text{$f$ ist bijektiv.}}
  			\solution{true}			
		\end{choice}
        \explanation[equalChoice(00???)]{Eine Abbildung ist eine Zuordnung zwischen zwei Mengen, die jedem Element des Definitionsbereichs genau ein Element des Zielbereichs zuordnet.}
        \explanation[equalChoice(11???)]{Eine Abbildung ist eine Zuordnung zwischen zwei Mengen, die jedem Element des Definitionsbereichs genau ein Element des Zielbereichs zuordnet.}
        \explanation[equalChoice(10???)]{Eine Abbildung ist eine Zuordnung zwischen zwei Mengen, die jedem Element des Definitionsbereichs genau ein Element des Zielbereichs zuordnet.}
        \explanation[equalChoice(??0??)]{Die Abbildung ist injektiv, weil aus $f(m_1)=f(m_2)$ bereits $m_1=m_2$ folgt.}
        \explanation[equalChoice(???0?)]{Surjektiv ist die Abbildung, wenn die Wertemenge mit dem Zielbereich übereinstimmt.}
        \explanation[equalChoice(????0)]{Eine Abbildung ist bijektiv, wenn die Abbildung injektiv und surjektiv ist.}

	\end{question}
	
	%8. Frage (Pool 2)
	\begin{question}
   		\text{Entscheiden Sie, welche Aussagen für die Zuordnung $f: \mathbb{R}_- \to \mathbb{R} , x \mapsto x^3$ korrekt sind und kreuzen Sie wahre Aussagen an.
		Hierbei ist $\mathbb{R}_-:=\{x \in \mathbb{R} \vert x < 0\}$.}
		\type{mc.multiple}
		\begin{choice}
  			\lang{de}{\text{$f$ ist keine Abbildung.}}
  			\solution{false}			
		\end{choice}
		\begin{choice}
  			\lang{de}{\text{$f$ ist eine Abbildung.}}
  			\solution{true}			
		\end{choice}
		\begin{choice}
  			\lang{de}{\text{$f$ ist injektiv.}}
  			\solution{true}
		\end{choice}
		\begin{choice}
  			\lang{de}{\text{$f$ ist surjektiv.}}
  			\solution{false}			
		\end{choice}
		\begin{choice}
  			\lang{de}{\text{$f$ ist bijektiv.}}
  			\solution{false}			
		\end{choice}
        \explanation[equalChoice(00???)]{Eine Abbildung ist eine Zuordnung zwischen zwei Mengen, die jedem Element des Definitionsbereichs genau ein Element des Zielbereichs zuordnet.}
        \explanation[equalChoice(11???)]{Eine Abbildung ist eine Zuordnung zwischen zwei Mengen, die jedem Element des Definitionsbereichs genau ein Element des Zielbereichs zuordnet.}
        \explanation[equalChoice(10???)]{Eine Abbildung ist eine Zuordnung zwischen zwei Mengen, die jedem Element des Definitionsbereichs genau ein Element des Zielbereichs zuordnet.}
        \explanation[equalChoice(??0??)]{Die Abbildung ist injektiv, weil aus $f(m_1)=f(m_2)$ bereits $m_1=m_2$ folgt.}
        \explanation[equalChoice(???1?)]{Die reelle Zahl $1\in \R$, ein Element des Zielbereichs, wird von $f$ nicht zugeordnet, da $f(x)<0$ für alle Elemente $x$ des Definitionsbereichs, somit kann $f$ nicht surjektiv sein.}
        \explanation[equalChoice(????1)]{Eine Abbildung ist bijektiv, wenn die Abbildung injektiv und surjektiv ist.}

	\end{question}
%9. Frage (Pool 2)
	\begin{question}
    	\text{Entscheiden Sie, welche Aussagen für die Zuordnung $f: \mathbb{R}_+ \to \mathbb{R} , x \mapsto x^3$ korrekt sind und kreuzen Sie wahre Aussagen an.
		Hierbei ist $\mathbb{R}_+:=\{x \in \mathbb{R} \vert x > 0\}$.}
		\type{mc.multiple}
		\begin{choice}
  			\lang{de}{\text{$f$ ist keine Abbildung.}}
  			\solution{false}			
		\end{choice}
		\begin{choice}
  			\lang{de}{\text{$f$ ist eine Abbildung.}}
  			\solution{true}			
		\end{choice}
		\begin{choice}
  			\lang{de}{\text{$f$ ist injektiv.}}
  			\solution{true}
		\end{choice}
		\begin{choice}
  			\lang{de}{\text{$f$ ist surjektiv.}}
  			\solution{false}			
		\end{choice}
		\begin{choice}
  			\lang{de}{\text{$f$ ist bijektiv.}}
  			\solution{false}			
		\end{choice}
        \explanation[equalChoice(00???)]{Eine Abbildung ist eine Zuordnung zwischen zwei Mengen, die jedem Element des Definitionsbereichs genau ein Element des Zielbereichs zuordnet.}
        \explanation[equalChoice(11???)]{Eine Abbildung ist eine Zuordnung zwischen zwei Mengen, die jedem Element des Definitionsbereichs genau ein Element des Zielbereichs zuordnet.}
        \explanation[equalChoice(10???)]{Eine Abbildung ist eine Zuordnung zwischen zwei Mengen, die jedem Element des Definitionsbereichs genau ein Element des Zielbereichs zuordnet.}
        \explanation[equalChoice(??0??)]{Die Abbildung ist injektiv, weil aus $f(m_1)=f(m_2)$ bereits $m_1=m_2$ folgt.}
        \explanation[equalChoice(???1?)]{Die Zahl $-1\in \R$, ein Element des Zielbereichs, wird nicht von $f$ zugeordnet, da $f(x)>0$ für alle Elemente $x$ des Definitionsbereichs, somit kann $f$ nicht surjektiv sein.}
        \explanation[equalChoice(????1)]{Eine Abbildung ist bijektiv, wenn die Abbildung injektiv und surjektiv ist.}

	\end{question}
	%10. Frage (Pool 2)
	\begin{question}
    	\text{Entscheiden Sie, welche Aussagen für die Zuordnung $f: \mathbb{Z} \to \mathbb{R} , x \mapsto x^3$ korrekt sind und kreuzen Sie wahre Aussagen an.}
		\type{mc.multiple}
		\begin{choice}
  			\lang{de}{\text{$f$ ist keine Abbildung.}}
  			\solution{false}			
		\end{choice}
		\begin{choice}
  			\lang{de}{\text{$f$ ist eine Abbildung.}}
  			\solution{true}			
		\end{choice}
		\begin{choice}
  			\lang{de}{\text{$f$ ist injektiv.}}
  			\solution{true}
		\end{choice}
		\begin{choice}
  			\lang{de}{\text{$f$ ist surjektiv.}}
  			\solution{false}			
		\end{choice}
		\begin{choice}
  			\lang{de}{\text{$f$ ist bijektiv.}}
  			\solution{false}			
		\end{choice}
        \explanation[equalChoice(00???)]{Eine Abbildung ist eine Zuordnung zwischen zwei Mengen, die jedem Element des Definitionsbereichs genau ein Element des Zielbereichs zuordnet.}
        \explanation[equalChoice(11???)]{Eine Abbildung ist eine Zuordnung zwischen zwei Mengen, die jedem Element des Definitionsbereichs genau ein Element des Zielbereichs zuordnet.}
        \explanation[equalChoice(10???)]{Eine Abbildung ist eine Zuordnung zwischen zwei Mengen, die jedem Element des Definitionsbereichs genau ein Element des Zielbereichs zuordnet.}
        \explanation[equalChoice(??0??)]{Die Abbildung ist injektiv, weil aus $f(m_1)=f(m_2)$ bereits $m_1=m_2$ folgt.}
        \explanation[equalChoice(???1?)]{Die Zahl $2 \in \R$, ein Element des Zielbereichs, wird von $f$ nicht zugeordnet, da $\sqrt[3]{2}\not\in \Z$, somit kann $f$ nicht surjektiv sein.}
        \explanation[equalChoice(????1)]{Eine Abbildung ist bijektiv, wenn die Abbildung injektiv und surjektiv ist.}

	\end{question}
	%11.Frage (Pool 2)
	\begin{question}
    	\begin{variables}
			\randint{a}{-10}{-1}
			\randint{b}{-10}{10}
			\function[normalize]{f}{a*(x+b)^4}
		\end{variables}
		\text{Entscheiden Sie, welche Aussagen für die Zuordnung $f: \mathbb{R} \to \mathbb{R}_+ , x \mapsto \var{f}$ korrekt 
        sind und kreuzen Sie wahre Aussagen an.}
		\type{mc.multiple}
		\begin{choice}
  			\lang{de}{\text{$f$ ist keine Abbildung.}}
  			\solution{true}			
		\end{choice}
		\begin{choice}
  			\lang{de}{\text{$f$ ist eine Abbildung.}}
  			\solution{false}			
		\end{choice}
		\begin{choice}
  			\lang{de}{\text{$f$ ist injektiv.}}
  			\solution{false}			
		\end{choice}
		\begin{choice}
  			\lang{de}{\text{$f$ ist surjektiv.}}
  			\solution{false}			
		\end{choice}
		\begin{choice}
  			\lang{de}{\text{$f$ ist bijektiv.}}
  			\solution{false}			
		\end{choice}
        \explanation[equalChoice(00???)]{Eine Abbildung ist eine Zuordnung zwischen zwei Mengen, die jedem Element des Definitionsbereichs genau ein Element des Zielbereichs zuordnet.}
        \explanation[equalChoice(11???)]{Eine Abbildung ist eine Zuordnung zwischen zwei Mengen, die jedem Element des Definitionsbereichs genau ein Element des Zielbereichs zuordnet.}
        \explanation[equalChoice(01???)]{Schauen Sie sich noch einmal den Zielbereich an! Wird auch wirklich dorthin abgebildet?}
	\end{question}
	%12.Frage (Pool 2)
    \begin{question}
    	\begin{variables}
			\randint{a}{1}{10}
			\randint{b}{-10}{10}
			\function[normalize]{f}{a*(x+b)^4}
            \function[calculate]{help1}{-b-1}
            \function[calculate]{help2}{-b+1}
		\end{variables}
		\text{Entscheiden Sie, welche Aussagen für die Zuordnung $f: \mathbb{R} \to \mathbb{R}_{\geq 0} , x \mapsto \var{f}$ korrekt 
        sind und kreuzen Sie wahre Aussagen an.}
		\type{mc.multiple}
		\begin{choice}
  			\lang{de}{\text{$f$ ist keine Abbildung.}}
  			\solution{false}			
		\end{choice}
		\begin{choice}
  			\lang{de}{\text{$f$ ist eine Abbildung.}}
  			\solution{true}			
		\end{choice}
		\begin{choice}
  			\lang{de}{\text{$f$ ist injektiv.}}
  			\solution{false}			
		\end{choice}
		\begin{choice}
  			\lang{de}{\text{$f$ ist surjektiv.}}
  			\solution{true}			
		\end{choice}
		\begin{choice}
  			\lang{de}{\text{$f$ ist bijektiv.}}
  			\solution{false}			
		\end{choice}
        \explanation[equalChoice(00???)]{Eine Abbildung ist eine Zuordnung zwischen zwei Mengen, die jedem Element des Definitionsbereichs genau ein Element des Zielbereichs zuordnet.}
        \explanation[equalChoice(11???)]{Eine Abbildung ist eine Zuordnung zwischen zwei Mengen, die jedem Element des Definitionsbereichs genau ein Element des Zielbereichs zuordnet.}
        \explanation[equalChoice(10???)]{Eine Abbildung ist eine Zuordnung zwischen zwei Mengen, die jedem Element des Definitionsbereichs genau ein Element des Zielbereichs zuordnet.}
        \explanation[equalChoice(??1??)]{Es ist $f(\var{help1})=\var{a}=f(\var{help2})$. Kann $f$ dann injektiv sein?}
        \explanation[equalChoice(???0?)]{Surjektiv ist die Abbildung, falls die Wertemenge mit dem Zielbereich übereinstimmt.}
        \explanation[equalChoice(????1)]{Eine Abbildung ist bijektiv, wenn die Abbildung injektiv und surjektiv ist.}
	\end{question}
\end{problem}

\embedmathlet{mathlet}
\end{content}