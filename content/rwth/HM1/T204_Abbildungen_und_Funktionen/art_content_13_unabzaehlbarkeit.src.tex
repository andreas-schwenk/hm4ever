%$Id:  $
\documentclass{mumie.article}
%$Id$
\begin{metainfo}
  \name{
    \lang{de}{Abzählbarkeit}
    \lang{en}{}
  }
  \begin{description} 
 This work is licensed under the Creative Commons License Attribution 4.0 International (CC-BY 4.0)   
 https://creativecommons.org/licenses/by/4.0/legalcode 

    \lang{de}{Beschreibung}
    \lang{en}{}
  \end{description}
  \begin{components}
  \component{generic_image}{content/rwth/HM1/images/g_img_00_video_button_schwarz-blau.meta.xml}{00_video_button_schwarz-blau}
  \component{generic_image}{content/rwth/HM1/images/g_tkz_T204_DiagonalScheme.meta.xml}{T204_DiagonalScheme}
  \end{components}
  \begin{links}
    \link{generic_article}{content/rwth/HM1/T207_Intervall_Schachtelung/g_art_content_23_intervallschachtelung.meta.xml}{content_23_intervallschachtelung}
    \link{generic_article}{content/rwth/HM1/T204_Abbildungen_und_Funktionen/g_art_content_11_injektiv_surjektiv_bijektiv.meta.xml}{content_11_injektiv_surjektiv_bijektiv}
    \link{generic_article}{content/rwth/HM1/T202_Reelle_Zahlen_axiomatisch/g_art_content_07_vollstaendigkeit.meta.xml}{Vollständigkeit}
    \link{generic_article}{content/rwth/HM1/T201neu_Vollstaendige_Induktion/g_art_content_01_indirekter_widerspruchsbeweis.meta.xml}{indbeweis}
  \end{links}
  \creategeneric
\end{metainfo}

%
\begin{content}
\begin{block}[annotation]
	Im Ticket-System: \href{https://team.mumie.net/issues/22236}{Ticket 22236}.
     Dieser Abschnitt ist neu. Zunächst wird die Mächtigkeit und Abzählbarkeit einer Menge definiert.
     Im Anschluß daran wird die Abzählbarkeit von $\Z,\N\times\N$ und $\Q$ und zum Schluß
     die Unabzählbarkeit von $\R$ gezeigt.
\end{block}
\usepackage{mumie.ombplus}
\ombchapter{4}
\ombarticle{4}

\lang{de}{\title{Die Überabzählbarkeit von $\R$}}
 


\begin{block}[info-box]
\tableofcontents
\end{block}

Die \ref[Vollständigkeit][{Vollständigkeit}]{def:vollstaendigkeit} von $\R$ gibt Anlass zur Vermutung, dass die Anzahl der Elemente von $\R$ größer ist als die Anzahl der 
Elemente in $\N$, obwohl beide Mengen unendlich viele Elemente haben. Georg Cantor (1845-1918) war der erste Mathematiker, der dieser
Idee auf den Grund ging und mathematisch bewies, dass es mehr als eine Art von Unendlichkeit gibt. Im Folgenden stellen wir kurz 
seine Ideen zum Zählen der Elemente und zum Vergleich der Mächtigkeit von Mengen der natürlichen und reellen Zahlen vor.  


\section{Abzählbarkeit}

\begin{definition}[Mächtigkeit und endliche Menge]\label{def:maechtigkeit}
Sei $M$ eine Menge, so bezeichnet \#$M$ die Anzahl ihrer (verschiedenen) Elemente. Wenn wir sagen, 
$M$ hat $n$ Elemente mit $n\in\N$, bedeutet das, dass $M$ eine \notion{endliche Menge} ist und eine Bijektion
$f:M\rightarrow\{1;2;3;\ldots ;n\}$ existiert, andernfalls bezeichnet man $M$ als \notion{unendliche Menge} . \\

Die Mächtigkeit von $M$ wird auch mit $ord(M)$ und $\vert M\vert$ bezeichnet.

\end{definition}


\begin{example}[Mächtigkeit]
 \begin{enumerate}
 \item  \#$\emptyset=0$.
 \item  \#$\N=\infty$, 
 
 \#$\{2n\vert n\in\N\}=\infty$, 
 
 \#$\Q=\infty$, 
 
 \#$\R=\infty$. 
 
 $\N$, $\{2n\vert n\in\N\}$, $\Q$ und $\R$ sind unendliche Mengen. 
 \item  \#$\{1;0;5;5;2;1\}=4$.
 \end{enumerate}

\end{example}
Obwohl $\N$ und $\R$ beide die Mächtigkeit gleich unendlich haben, werden wir am Ende dieses Teils sehen, dass sie nicht die gleiche 
Anzahl von Elementen haben. Das heißt, sie haben nicht die gleiche Mächtigkeit und wir sprechen hier von \textit{unterschiedlichen 
Unendlichkeiten}. Dazu werden wir in der nächsten Definition sehen, wann zwei Mengen die gleiche Mächtigkeit haben.

\begin{definition}[Gleichmächtigkeit zweier Mengen]\label{def:Gleichmaechtigkeit}
Seien $X$ und $Y$ zwei Mengen.  Wir sagen, dass \#$X=$\#$Y$, wenn es eine bijektive Abbildung $f:X\rightarrow Y$ gibt
und bezeichnet in dem Fall die beiden Mengen $X$ und $Y$ als \notion{gleichmächtig.}
\end{definition}

\begin{definition}[Abzählbarkeit]\label{def:abzaehlbarkeit}

Eine Menge $X$ ist \notion{abzählbar}, wenn entweder $X$ \emph{endlich} ist oder \#$X=$\#$\N$ gilt. Wenn 
\#$X=$\#$\N$ gilt, bezeichnet man $X$ auch als \notion{abzählbar unendlich.}
Ist $X$ weder \emph{endlich} noch \emph{abzählbar unendlich}, so sagen wir, dass $X$ \notion{überabzählbar} ist.

\end{definition}

\begin{example}[Abzählbarkeit]
 \begin{enumerate}
  \item $\N$ ist abzählbar unendlich, weil $id:\N\rightarrow\N$ eine bijektive Abbildung ist, also
    nach \ref{def:maechtigkeit} \#$\N=$\#$\N$ gilt. 

  \item $\Z$ ist auch abzählbar unendlich, da die Funktion
    \[  f:\N\to \Z, \text{ } n \mapsto \begin{cases} \frac{n}{2} & \text{falls }n \text{ gerade,} \\
    \frac{1-n}{2} & \text{falls }n \text{ ungerade,} \end{cases}\]
    bijektiv ist und somit nach Definition \ref{def:maechtigkeit} \#$\Z=$\#$\N$ gilt.
 \end{enumerate}  
\end{example}


\begin{theorem}\label{thm:abzbedingungen}
Sei $X$ eine nichtleere Menge. Die folgenden Aussagen sind äquivalent:
 \begin{enumerate}
 \item[$(1)$] $X$ ist abzählbar.
 \item[$(2)$] Es existiert eine surjektive Abbildung $f:\N\rightarrow X$.
 \item[$(3)$] Es existiert eine injektive Abbildung $g:X\rightarrow\N$.
 \end{enumerate}
\end{theorem}

\begin{proof*}
\begin{showhide}
Um die Äquivalenz der drei Aussagen zu zeigen, beweisen wir, dass $(1) \Rightarrow (2)$, $(2) \Rightarrow (3)$ und $(3) \Rightarrow (1)$.
 \begin{enumerate}

 \item[$(1) \Rightarrow (2):$] Nehmen wir an, dass $X$ abzählbar ist.  Wir wollen zeigen, dass es eine surjektive Abbildung $f:\N\rightarrow X$ gibt. 
 Wir müssen hier zwei Fälle betrachten:
 
 1. $\, X$ ist endlich. Dann gibt es nach Definition \ref{def:maechtigkeit} für ein $n\in\N$ und 
    eine bijektive Abbildung $h_{1}:\{1;2;\ldots;n\}\rightarrow X$.  Man wähle nun ein beliebiges  $a\in X$ und definiere:
 \[ f:\N\rightarrow X,\text{ } k \mapsto \begin{cases} h_{1}(k) & \text{falls }1\leq k\leq n\\
    a & \text{falls }k\geq n+ 1 \end{cases}.\]
    Es gilt $f(\{1;2;\ldots;n\})=h_{1}(\{1;2;\ldots;n\})=X$, weil $h_{1}$ bijektiv ist. Hieraus folgt aber, dass $f$  eine surjektive Abbildung ist.

 2. $\, X$ ist unendlich und nach Voraussetzung $(1)$ abzählbar, daher nach Definition \ref{def:abzaehlbarkeit}
 abzählbar unendlich und es gilt #$X=$#$\N$. Nach Definition \ref{def:Gleichmaechtigkeit} existiert folglich eine 
 bijektive Abbildung $\, f:\N \rightarrow X$, die 
 \ref[content_11_injektiv_surjektiv_bijektiv][definitionsgemäß]{def:in-sur-bi} insbesondere auch eine surjektive Abbildung ist.
   
 \item[$(2) \Rightarrow (3):$]  
 Sei $f:\N\rightarrow X$ eine surjektive Funktion. Falls $f$ zudem auch injektiv ist, 
 dann ist $f$  \ref[content_11_injektiv_surjektiv_bijektiv][definitionsgemäß]{def:in-sur-bi} eine bijektive Abbildung, ebenso wie die 
 Umkehrfunktion $f^{-1}$. Insbesondere ist somit $g:=f^{-1}:X\rightarrow\N$ eine injektive Abbildung. 
 
 
 Andernfalls, falls also $f$ nicht injektiv ist, gibt es eine Menge $M\subset\N$, für 
 die gilt $f(M)=X$. Dann ist die Restriktion von $f$ auf $M$ eine Bijektion. Daraus folgt, dass die Abbildung $g:X\rightarrow M$ mit der 
 gleichen Abbildungsvorschrift wie von $f^{-1}$ eine Bijektion ist. So ist $g:X\rightarrow\N$ mit dieser Abbildungsvorschrift eine injektive Abbildung.
  
 \item[$(3) \Rightarrow (1):$]
 Sei $g:X\rightarrow\N$ eine injektive Funktion. Wir wollen zeigen, dass $X$ abzählbar ist. Falls $X$ endlich ist, gilt dies bereits per 
 Definition \ref{def:abzaehlbarkeit}. Falls $X$ eine unendliche Menge ist, ist gemäß derselben Definition zu zeigen, dass 
 #$X=$#$\N$. Wir betrachten hierzu die Menge $M:=g(X)\subseteq\N$. Da $X$  unendlich und $g$ injektiv ist, ist diese Menge $M$
 ebenfalls unendlich und als Teilmenge von $\N$ ist sie auch abzählbar. 
    \begin{showhide}
    Wir beweisen, dass $M$ abzählbar ist, indem wir die Elemente von $M$ abzählbar auflisten und somit eine bijektive Abbildung zwischen 
    $\N$ und $M$ aufzeigen.
     Sei $m_{i}$ für $i\in\N$ wie folgt definiert: 
    \[ m_{1}:= min(M), m_{2}:= min(M\setminus\{m_{1}\}), m_{3}:= min(M\setminus\{m_{1},m_{2}\}),
    \ldots.\]
     So haben wir gezeigt, dass $M:=\{m_{1};m_{2};m_{3};\ldots\}$ abzählbar ist.
    \end{showhide}
 Damit gilt nach Definition \ref{def:abzaehlbarkeit} $\;$#$M=\,$#$\N$. 
 
 Außerdem ist die Funktion $h_{2}:X\rightarrow M,$ mit $h_{2}(x) =g(x)$ für alle $x\in X$ offensichtlich bijektiv.
 Daher ist nach \ref{def:abzaehlbarkeit} $\;$#$X=\,$#$M \underset{(s.o.)}{=} $#$\N$,
 womit die Abzählbarkeit von $X$ gezeigt ist.
 
  \end{enumerate}
\end{showhide}
\end{proof*}

\section{$\Q$ ist abzählbar}

Um zu beweisen, dass die Menge der rationalen Zahlen eine abzählbar unendliche Menge ist, müssen wir vorher zeigen, dass eine abzählbare 
Vereinigung von abzählbaren Mengen abzählbar ist. Dazu ist allerdings zunächst der folgende Satz notwendig.

\begin{theorem}\label{thm:abzcartesisch}
$\N\times\N$ ist abzählbar.
\end{theorem}

Hinter dem Beweis dieses Satzes steckt als Idee das sogenannte Cantorsche Diagonalverfahren. Visualisiert man $\N\times\N$ als die Menge der 
positiv ganzzahligen Punkte in der Koordinatenebene, dann ist das „Durchlaufen“ von $\N\times\N$ gut zu veranschaulichen und der Name
Diagonalverfahren wird verständlich.
\begin{center}
\image{T204_DiagonalScheme}
\end{center}

\begin{proof*}
\begin{showhide}
Wir wollen zeigen, dass $F$, so wie es unten definiert ist, eine Bijektion darstellt:
  \[  F:\N\rightarrow\N\times\N,\text{ } n\mapsto (l,m+2-l), \]
 wobei
  \[ m\in\N_{0}\text{ und } 1\leq l\leq m+1 \text{ mit } n=\frac{m(m+1)}{2}+l. \]
 Dafür müssen wir zunächst beweisen, dass $F$ eine Funktion ist. Wenn wir beweisen können, dass zu jedem $n\in\N$ 
 genau ein Paar $(m,l) \in \N\times\N$ mit  $m\in\N_{0}$ und $1\leq l\leq m+1$ existiert, so dass
  \[ n=\frac{m(m+1)}{2}+l, \]

  erhalten wir das angestrebte Resultat als Folgerung. \\
    
  Betrachten wir hierzu die Mengen 
  $A_{k}=\{n\in\N \,\vert\, \frac{k(k+1)}{2}<n\leq\frac{(k+1)(k+2)}{2}\}$ für $k\in\N_{0}$. 
  Dann ist offenbar $\N$ die disjunkte Vereinigung aller $A_{k}$ (für $k\in\N_{0}$) und folglich liegt jede natürliche Zahl $n$ 
  (aufgrund der Disjunktheit) in genau einer dieser Mengen. Es gibt also zu jedem $n \in \N$ eine eindeutige Zahl 
  $m \in \N_{0}$, so dass $n \in A_m$ und daher 
  \[0 < n-\frac{m(m+1)}{2}\leq\frac{(m+2)(m+1)}{2}-\frac{m(m+1)}{2}=m+1. \]
  Wählen wir zudem $\,l:=n-\frac{m(m+1)}{2}\,$ können wir auf diese Weise für jedes $n\in\N$ ein eindeutiges 
  Paar $(m,l)$ mit den gewünschten Eigenschaften finden. 


  $F$ ist surjektiv, weil ein Urbild von dem beliebigen Paar der natürlichen Zahlen $(j,k)$ ist
  \[ \frac{(j+k-2)(j+k-1)}{2}+j,\]
  was wiederum eine natürliche Zahl ist.
  
  Hiermit ist nach Satz \ref{thm:abzbedingungen} bewiesen, dass $\N\times\N$ abzählbar ist. 
  Wir werden trotzdem noch zeigen, dass $F$ auch injektiv ist. Seien 
  \[ n=\frac{m(m+1)}{2}+l\text{ und } n\prime=\frac{m\prime(m\prime+1)}{2}+l\prime\text{ mit } F(n)=F(n\prime) \]
  wobei
  \[n,n\prime,m,m\prime\in\N\text{  und  }l,l\prime\in\N_{0}.\]
  Daraus folgt $(l,m+2-l)=(l\prime,m\prime+2-l\prime)$, also $l=l\prime$ und $m+2-l=m\prime+2-l\prime$. Die Kombination der 
  letzten beiden Gleichungen ergibt $l=l\prime$ und $m=m\prime$. Wir haben also $n=n\prime$ und somit haben wir gezeigt, dass $F$ 
  injektiv ist.
\end{showhide}
\end{proof*}

\begin{theorem}\label{thm:abzvereinigung}
Eine abzählbare Vereinigung abzählbarer Mengen ist abzählbar.
\end{theorem}

\begin{proof*}
\begin{showhide}
Seien $M_{1},M_{2},M_{3},\ldots$ nicht-leere abzählbare Mengen und $M=\bigcup^{\infty}_{n=1}M_{n}$. Wir wollen beweisen, dass 
$M$ eine abzählbare Menge ist. Nach Satz \ref{thm:abzbedingungen} gibt es für jedes $n\in\N$ eine surjektive Abbildung $f_{n}:\N\rightarrow 
M_{n}$. Betrachten Sie die Abbildung $F$ aus Satz \ref{thm:abzcartesisch}:
\[  F:\N\rightarrow\N\times\N,\text{ } n\mapsto (l,m+2-l) \]
mit
\[ m\in\N_{0}\text{ und } 1\leq l\leq m+1 \text{ mit } n=\frac{m(m+1)}{2}+l. \]
Sie kann kommponentenweise mit $F(n)=(F_{1}(n),F_{2}(n))$ angegeben werden, wobei $F_{1}(n)=l$ und $F_{2}(n)=m+2-l$. Wir definieren
\[ g:\N\rightarrow M\]
durch
\[g(n)=f_{F_{1}(n)}(F_{2}(n)).\]

Nach Satz \ref{thm:abzcartesisch} für jedes $n\in\N$ gibt es nur ein $F_{1}(n)$ und $F_{2}(n)$. Also ist $g$ eine Abbildung. 

Zu zeigen ist nun, dass $g$ surjektiv ist. Nach der Definition der Vereinigung gibt es für jedes $x\in M$ ein $l\in\N$, so dass $x\in M_{l}$.  
Das bedeutet aber, dass für eine natürliche Zahl $k$ gilt $f_{l}(k)=x$. Wählt man $m$ so, dass $k=m+2-l$, so gibt es nach Satz \ref{thm:abzcartesisch} 
ein eindeutiges $n$ mit $g(n)=f_{l}(m+2-l)=f_{l}(k)=x$. Das heißt, $g$ ist surjektiv.
\end{showhide}
\end{proof*}

\begin{theorem}
$\Q$ ist abzählbar.
\end{theorem}

\begin{proof*}
Man hat
\[\Q=\bigcup^{\infty}_{n=1}M_{n}, \text{ }\text{ }M_{n}:=\{\frac{x}{n}\vert x\in\Z\}.\]
Mit $\Z$ ist jedes $M_{n}$ abzählbar, also nach Satz \ref{thm:abzvereinigung} auch $\Q$.
\end{proof*}

\section{$\R$ ist überabzählbar}\label{sec:R_ueberabz}

\begin{theorem}[Überabzählbarkeit von $\R$]
$\R$ ist überabzählbar.
\end{theorem}

\begin{proof*}
Wir verwenden einen \ref[indbeweis][{indirekten Beweis}]{indirekten Beweis}, um die Überabzählbarkeit 
von $\R$ zu zeigen. \\

Nehmen wir also an, $\R$ sei abzählbar.
Das bedeutet, dass man die Elemente von $\R$ in der Form $\{x_{1};x_{2};x_{3};\ldots\}$ auflisten kann. Zu dieser Aufstellung 
der reellen Zahlen definieren wir eine Liste von Intervallen $I_{n}$, so dass
\[ x_{n}\notin I_{n}\text{ für jedes } n\in\N.\]
Wir definieren die $I_{n}$ rekursiv. Zunächst ist $I_{1}:=[x_{1}+1;x_{1}+2]$. Dann wird $I_{n+1}$  aus $I_{n}$ wie folgt konstruiert.
Wir teilen $I_{n}$ in drei gleichlange Intervalle. Dann wählen wir als $I_{n+1}$ ein geschlossenes Teilintervall, das $x_{n+1}$ nicht enthält. Für jedes 
$n\in \N$ ist also $I_{n+1}$ eine Teilmenge von $I_{n}$. Die Schnittmenge dieser abzählbar vielen geschlossenen Intervalle ist nicht 
leer, weil keins davon leer ist. Nach dem 
\ref[content_23_intervallschachtelung][Intervallschachtelungsprinzip]{thm:intervallschachtelungsprinzip}
gibt also eine reelle Zahl, die in allen Intervallen liegt.
\\
Es sei nun $a$ die reelle Zahl, die in allen Intervallen liegt, d.h.
\begin{center}
$a\in I_{n} \quad$  für alle $\,n\in\N.$
\end{center}
Da $a$ eine reelle Zahl ist, kommt sie in der obigen Auflistung $\R=\{x_{1};x_{2};x_{3};\ldots\}$ vor. Sei $k\in\N$ die Zahl, für 
die gilt $a=x_{k}$. Daraus folgt jedoch $x_{k}\in I_{k}$. Das widerspricht aber der Definition von $I_{k}$. $\R$ kann daher nicht 
abzählbar sein.
\end{proof*}

%Video

Das folgende Video enthält eine Erklärung des Inhalts dieses Teils.
\floatright{\href{https://api.stream24.net/vod/getVideo.php?id=10962-2-10822&mode=iframe&speed=true}{\image[75]{00_video_button_schwarz-blau}}}\\


\end{content}

