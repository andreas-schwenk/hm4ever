\documentclass{mumie.problem.gwtmathlet}
%$Id$
\begin{metainfo}
  \name{
    \lang{de}{A08: Höhere Ableitungen}
    \lang{en}{}
  }
  \begin{description} 
 This work is licensed under the Creative Commons License Attribution 4.0 International (CC-BY 4.0)   
 https://creativecommons.org/licenses/by/4.0/legalcode 

    \lang{de}{}
    \lang{en}{}
  \end{description}
  \corrector{system/problem/GenericCorrector.meta.xml}
  \begin{components}
    \component{js_lib}{system/problem/GenericMathlet.meta.xml}{mathlet}
  \end{components}
  \begin{links}
  \end{links}
  \creategeneric
\end{metainfo}
\begin{content}
\usepackage{mumie.genericproblem}

\lang{de}{\title{A08: Höhere Ableitungen}}

\begin{block}[annotation]
	Im Ticket-System: \href{http://team.mumie.net/issues/10289}{Ticket 10289}
\end{block}

\begin{problem}
	
	 \randomquestionpool{1}{2}
	
%######################################################QUESTION_START	
	\begin{question}	
	%+++++++++++++++++++VARIABLES++++++++++++++++++++++
		\begin{variables}
			\randint[Z]{k}{2}{6}
			\randint[Z]{l}{1}{4}
			\function[calculate]{n}{l+k}
			\randint[Z]{a}{2}{6}
			
			\function{func1}{a*x^k}
			\function{sol}{0}
			
		\end{variables}
        
            \explanation{Hierbei handelt es sich um ein Monom. Es ist für $f(x) = a \cdot x^n$ stets $f'(x) = a \cdot n \cdot x^{n-1}$. Ist der Grad der Ableitung höher als der Grad des Monoms ist sie stets Null.}
	%+++++++++++++++++++TYPE+++++++++++++++++++++++++++	
		\type{input.function} %input.text %input.cases.function %input.finite-number-set %input.interval %...http://team.mumie.net/projects/support/wiki/DifferentAnswerType

	%+++++++++++++++++++TITLE++++++++++++++++++++++++++	
	    \lang{de}{\text{Was ist die $\var{n}$-te Ableitung der Funktion $f(x)=\var{func1}$}}
	%+++++++++++++++++++ANSWERS++++++++++++++++++++++++    
	    \begin{answer}
	    	\text{Die Ableitung ist }
			\solution{sol}
			\checkAsFunction{x}{-1}{1}{100}
	    \end{answer}    
	\end{question}

%######################################################QUESTION_END

%######################################################QUESTION_START	
	\begin{question}	
	%+++++++++++++++++++VARIABLES++++++++++++++++++++++
		\begin{variables}
			\randint[Z]{n}{3}{5}
			\randint[Z]{b}{2}{4}
			\randint[Z]{a}{2}{4}
			
			\function{func1}{a*e^(b*x)}
			\function{sol}{a*b^n*e^(b*x)}
			
		\end{variables}
        
        \explanation{Hier hilft die Kettenregel weiter mit $f(x) = g(h(x))$, wo $g(x) = e^x $ und $h(x) = a \cdot x$.}
	%+++++++++++++++++++TYPE+++++++++++++++++++++++++++	
		\type{input.function} %input.text %input.cases.function %input.finite-number-set %input.interval %...http://team.mumie.net/projects/support/wiki/DifferentAnswerType

	%+++++++++++++++++++TITLE++++++++++++++++++++++++++	
	    \lang{de}{\text{Was ist die $\var{n}$-te Ableitung der Funktion $f(x)=\var{func1}$}}
	%+++++++++++++++++++ANSWERS++++++++++++++++++++++++    
	    \begin{answer}
	    	\text{Die Ableitung ist }
			\solution{sol}
			\checkAsFunction{x}{-1}{1}{100}
	    \end{answer}    
	\end{question}
%######################################################QUESTION_END

\end{problem}

\embedmathlet{mathlet}

\end{content}