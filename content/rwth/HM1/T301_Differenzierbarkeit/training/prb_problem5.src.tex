\documentclass{mumie.problem.gwtmathlet}
%$Id$
\begin{metainfo}
  \name{
    \lang{de}{A05: Monotonie}
    \lang{en}{problem_5}
  }
  \begin{description} 
 This work is licensed under the Creative Commons License Attribution 4.0 International (CC-BY 4.0)   
 https://creativecommons.org/licenses/by/4.0/legalcode 

    \lang{de}{}
    \lang{en}{}
  \end{description}
  \corrector{system/problem/GenericCorrector.meta.xml}
  \begin{components}
    \component{js_lib}{system/problem/GenericMathlet.meta.xml}{mathlet}
  \end{components}
  \begin{links}
  \end{links}
  \creategeneric
\end{metainfo}
\begin{content}
\usepackage{mumie.ombplus}
\usepackage{mumie.genericproblem}


\lang{de}{
	\title{A05: Monotonie}
}
\begin{block}[annotation]
      
\end{block}
\begin{block}[annotation]
  Im Ticket-System: \href{http://team.mumie.net/issues/10286}{Ticket 10286}
\end{block}

\begin{problem}
	
\randomquestionpool{1}{6}
	
%Frage 1 von 8	
\begin{question}
	\lang{de}{
		\text{Gegeben $f(x) =\var{f}$. Bestimmen Sie das Intervall $I$, 
		auf dem $f$ monoton wachsend ist. }
		\explanation{Da $f$ differenzierbar ist, reicht es herauszufinden, wo $f'$ verschwindet oder positive Werte annimmt.}
	}
	\lang{en}{
		\text{Given $f(x) = \var{f}$, determine the interval $I$ on which $f$ is monotonically increasing.}
		
	}
	\begin{variables}
		\randint[Z]{a}{-3}{5}
	    \randint[Z]{b}{1}{7}
    	\function[calculate]{h}{2*b}
    	\function{f}{(x-a)^(h)}
    	\derivative[normalize]{g}{f}{x}
    	\functionNormalize{g}
		\function{loes}{infinity}
		\function[calculate]{s}{h-1}
	\end{variables}
	\type{input.number}
	\begin{answer}
    	\lang{de}{\text{ Die linke Intervallgrenze ist:}}
       	\lang{en}{\text{The left interval boundary is:}}
		\solution{a}
	\end{answer}
	\begin{answer}
    	\lang{de}{\text{ Die rechte Intervallgrenze ist:}}
       	\lang{en}{\text{The right interval boundary is:}}
		\solution{loes}
	\end{answer}
\end{question}
	
%Frage 2 von 8
\begin{question}
	\lang{de}{
		\text{Gegeben $f(x) =\var{f}$ auf $(-\infinity;\var{grenz}]$. Bestimmen Sie das Intervall $I$ in $(-\infinity;\var{grenz}]$, 
		auf dem $f$ monoton wachsend ist. }
		\explanation{Da $f$ differenzierbar ist, reicht es herauszufinden, wo $f'$ verschwindet oder positive Werte annimmt.}
	}
	\lang{en}{
		\text{Given $f(x) =\var{f}$ on $(-\infinity,\var{grenz}]$, find the interval $I$ in $(-\infinity;\var{grenz}]$ on which $f$ is monotonically increasing.}
	\explanation{}
	}
	\begin{variables}
		\randint[Z]{a}{1}{5}
	   	\randint[Z]{b}{1}{7}
		\function[calculate]{h}{2*b}
		\function[calculate]{s}{h+1}
		\function[calculate,2]{grenz}{a/2}
		\function{f}{1/((x-a)^(h))}
	   	\function{loes}{-infinity}
	   	\functionNormalize{f}
	\end{variables}
	\type{input.number}
	\begin{answer}
    	\lang{de}{\text{ Die linke Intervallgrenze ist:}}
       	\lang{en}{\text{The left interval boundary is:}}
		\solution{loes}
	\end{answer}
	\begin{answer}
      	\lang{de}{\text{ Die rechte Intervallgrenze ist:}}
      	\lang{en}{\text{The right interval boundary is:}}
		\solution{grenz}
	\end{answer}
\end{question}
	
%Frage 3 von 8
\begin{question}
	\lang{de}{
		\text{Gegeben $f(x) =\var{f}$. Bestimmen Sie das Intervall $I$, 
		auf dem $f$ monoton wachsend ist. }
		\explanation{Da $f$ differenzierbar ist, reicht es herauszufinden, wo $f'$ verschwindet oder positive Werte annimmt.}
	}
	\lang{en}{
		\text{Given $f(x) =\var{f}$, find the interval $I$ on which $f$ is monotonically increasing.}
		\explanation{}
	}
	\begin{variables}
	    \randint[Z]{a}{-3}{5}
	   	\randint[Z]{b}{1}{7}
		\function[calculate]{h}{2*b+1}
		\function[calculate]{s}{2*b}
		\function{f}{(x-a)^(h)}
	   	\function{loes}{infinity}
	   	\function{loesz}{-infinity}
	\end{variables}		
	\type{input.number}
	\begin{answer}
  		\lang{de}{\text{ Die linke Intervallgrenze ist:}}
   		\lang{en}{\text{The left interval boundary is:}}
		\solution{loesz}
	\end{answer}
	\begin{answer}
 		\lang{de}{\text{ Die rechte Intervallgrenze ist:}}
   		\lang{en}{\text{The right interval boundary is:}}
		\solution{loes}
	\end{answer}
\end{question}
	
%Frage 4 von 8
\begin{question}
	\lang{de}{
		\text{Gegeben $f(x) =\var{f}$ auf $[0;2\pi]$. Bestimmen Sie das Intervall $I$ in $[0;2\pi]$,\\
		auf dem $f$ monoton wachsend ist.}
		\explanation{Da $f$ differenzierbar ist, reicht es herauszufinden, wo $f'$ verschwindet oder positive Werte annimmt.}
	}
	\lang{en}{
		\text{Given $f(x) =\var{f}$ on $[0,2\pi]$, find the interval $I$ in $[0,2\pi]$ on which $f$ is monotonically increasing.}
		\explanation{$f'(x)=-\sin(x+\pi)$}
	}
	\begin{variables}
		\function{f}{cos(x+pi)}
   		\function{loes}{0}
   		\function{loesz}{pi}
	\end{variables}
	\type{input.function}
	\begin{answer}
    	\lang{de}{\text{ Die linke Intervallgrenze ist:}}
       	\lang{en}{\text{The left interval boundary is:}}
		\solution{loes}
		\checkAsFunction{x}{-10}{10}{100}	
	\end{answer}
	\begin{answer}
    	\lang{de}{\text{ Die rechte Intervallgrenze ist:}}
       	\lang{en}{\text{The right interval boundary is:}}
		\solution{loesz}
		\checkAsFunction{x}{-10}{10}{100}	
	\end{answer}
\end{question}
	
%Frage 5 von 8
\begin{question}
	\lang{de}{
		\text{Gegeben $f(x) =\var{f}$ auf $\R$. Bestimmen Sie das Intervall $I$, 
		auf dem $f$\\ monoton fallend ist. }
		\explanation{Da $f$ differenzierbar ist, reicht es herauszufinden, wo $f'$ verschwindet oder negative Werte annimmt.}
	}
	\lang{en}{
		\text{Given $f(x) =\var{f}$ on $\R$, find the interval $I$ on which $f$ is monotonically decreasing.}
		\explanation{$f'(x)=\frac{2x}{x^2+1}-1=-\frac{(x-1)^2}{x^2+1}\leq 0$}
	}
	\begin{variables}
    	\randint[Z]{a}{-3}{5}
		\function{f}{ln(x^2+1)-x}
	   	\function{loes}{-infinity}
	   	\function{loesz}{infinity}
	\end{variables}
	\type{input.number}
	\begin{answer}
    	\lang{de}{\text{ Die linke Intervallgrenze ist:}}
       	\lang{en}{\text{The left interval boundary is:}}
		\solution{loes}
	\end{answer}
	\begin{answer}
      	\lang{de}{\text{ Die rechte Intervallgrenze ist:}}
       	\lang{en}{\text{The right interval boundary is:}}
		\solution{loesz}
	\end{answer}
\end{question}
	
%Frage 6 von 8
\begin{question}
	\lang{de}{
		\text{Gegeben $f(x) =\var{f}$ auf $\R$. Bestimmen Sie das Intervall $I$, 
		auf dem $f$\\ monoton wachsend ist. }
		\explanation{Da $f$ differenzierbar ist, reicht es herauszufinden, wo $f'$ verschwindet oder positive Werte annimmt.}
	}
	\lang{en}{
		\text{Given $f(x) =\var{f}$ on $\R$, find the interval $I$ on which $f$ is monotonically increasing.}
		\explanation{$f'(x)=-2xe^{-x^2}$}
	}
	\begin{variables}
    	\randint[Z]{a}{-3}{5}
		\function{f}{exp(-x^2)}
	   	\function{loes}{-infinity}
	   	\number{loesz}{0}
	\end{variables}
	\type{input.number}
	\begin{answer}
  		\lang{de}{\text{ Die linke Intervallgrenze ist:}}
   		\lang{en}{\text{The left interval boundary is:}}
		\solution{loes}
	\end{answer}
	\begin{answer}
  		\lang{de}{\text{ Die rechte Intervallgrenze ist:}}
   		\lang{en}{\text{The right interval boundary is:}}
		\solution{loesz}
	\end{answer}
\end{question} 
    
%Frage 7 von 8
\begin{question} 
  	\lang{de}{
  		\text{Sei $f(x)=x^3-\var{abmall}x^2+\var{abmal}x+\var{idc}$. }
		\explanation{$f'(x)=3x^2-\var{s}x+\var{abmal}$.\\ Die station\"{a}ren Stellen sind die Nullstellen von $f'$.\\
		$f'$ ist zwischen den station\"{a}ren Stellen negativ, dort f\"{a}llt $f$ also.}
	}
	\lang{en}{
		\text{Let $f(x)=x^3-\var{abmall}x^2+\var{abmal}x+\var{idc}$.}
		\explanation{$f'(x)=3x^2-\var{s}x+\var{abmal}$\\ The critical points are the roots of $f'$. $f'$ is negative between its two critical points so the function $f$ is decreasing there.} 
	}
	\type{mc.multiple}
	\begin{variables}
		\randint[Z]{a}{1}{3}
		\randint[Z]{c}{1}{10}
		\randint[Z]{b}{4}{7}
		\function[calculate]{zwida}{2*a}
		\function[calculate]{zwidb}{2*b}
		\function[calculate]{idc}{c}
		\function[calculate]{abplus}{a+b}
		\function[calculate]{abmall}{3*(a+b)}
		\function[calculate]{abmal}{12*a*b}
		\function[calculate]{zwidamin}{2*a-1}
		\function[calculate]{zwidaminn}{2*a+1}
		\function[calculate]{zwidbplus}{2*b+2}
		\function[calculate]{zwidbpluss}{2*b+1}
		\function[calculate]{hh}{c*b}
		\function[calculate]{s}{6*(a+b)}
	\end{variables}
	\permutechoices{1}{4}
	\begin{choice}
		\lang{de}{\text{$f$ hat die station\"{a}ren Stellen $\var{zwida}$ und $\var{zwidb}$.}}
		\lang{en}{\text{$f$ has critical points at $x=\var{zwida}$ and $x=\var{zwidb}$.}}
		\solution{true}
	\end{choice}
	\begin{choice}
		\lang{de}{\text{$f$ hat die station\"{a}ren Stellen $\var{zwida}$ und $\var{zwidbpluss}$.}}
		\lang{en}{\text{$f$ has critical points at $x=\var{zwida}$ and $x=\var{zwidbpluss}$.}}
		\solution{false}
	\end{choice}
	\begin{choice}
		\lang{de}{\text{$f$ hat die station\"{a}ren Stellen $\var{zwidaminn}$ und $\var{zwidbpluss}$.}}
		\lang{en}{\text{$f$ has critical points at $x=\var{zwidaminn}$ and $x=\var{zwidbpluss}$.}}
		\solution{false}
	\end{choice}
	\begin{choice}
		\lang{de}{\text{$f$ hat die station\"{a}ren Stellen $\var{zwidaminn}$ und $\var{zwidb}$.}}
		\lang{en}{\text{$f$ has critical points at $x=\var{zwidaminn}$ and $x=\var{zwidb}$.}}
		\solution{false}
	\end{choice}
	\begin{choice}
		\lang{de}{\text{$f$ ist zwischen den station\"{a}ren Stellen monoton steigend.}}
		\lang{en}{\text{$f$ is monotonically increasing between its critical points.}}
		\solution{false}
	\end{choice}
\end{question}
  
%Frage 8 von 8
 \begin{question} 
  	\lang{de}{
  		\text{Sei $f(x)=x^3-\var{abmall}x^2+\var{abmal}x+\var{idc}$.}
		\explanation{$f'(x)=3x^2-\var{s}x+\var{abmal}$.\\ Die station\"{a}ren Stellen sind die Nullstellen von $f'$.\\
		$f'$ ist zwischen den station\"{a}ren Stellen negativ, dort f\"{a}llt $f$ also.}
	}
	\lang{en}{
		\text{Let $f(x)=x^3-\var{abmall}x^2+\var{abmal}x+\var{idc}$.}
		\explanation{$f'(x)=3x^2-\var{s}x+\var{abmal}$\\
		The critical points are the roots of $f'$. $f'$ is negative between its two critical points so the function $f$ is decreasing there.}
	}
	\type{mc.multiple}
	\begin{variables}
		\randint[Z]{a}{1}{3}
		\randint[Z]{c}{1}{10}
		\randint[Z]{b}{4}{7}
		\function[calculate]{zwida}{2*a}
		\function[calculate]{zwidb}{2*b}
		\function[calculate]{idc}{c}
		\function[calculate]{abplus}{a+b}
		\function[calculate]{abmall}{3*(a+b)}
		\function[calculate]{abmal}{12*a*b}
		\function[calculate]{zwidamin}{2*a-1}
		\function[calculate]{zwidaminn}{2*a-2}
		\function[calculate]{zwidbplus}{2*b+2}
		\function[calculate]{zwidbpluss}{2*b+1}
		\function[calculate]{hh}{c*b}
		\function[calculate]{s}{6*(a+b)}
	\end{variables}
	\permutechoices{1}{4}
	\begin{choice}
		\lang{de}{\text{$f$ hat die station\"{a}ren Stellen $\var{zwida}$ und $\var{zwidb}$.}}
		\lang{en}{\text{$f$ has critical points at $x=\var{zwida}$ and $x=\var{zwidb}$.}}
		\solution{true}
	\end{choice}
	\begin{choice}
		\lang{de}{\text{$f$ hat die station\"{a}ren Stellen $\var{zwida}$ und $\var{zwidbplus}$.}}
		\lang{en}{\text{$f$ has critical points at $x=\var{zwida}$ and $x=\var{zwidbplus}$.}}
		\solution{false}
	\end{choice}
	\begin{choice}
		\lang{de}{\text{$f$ hat die station\"{a}ren Stellen $\var{zwidamin}$ und $\var{zwidbplus}$.}}
		\lang{en}{\text{$f$ has critical points at $x=\var{zwidamin}$ and $x=\var{zwidbplus}$.}}
		\solution{false}
	\end{choice}
	\begin{choice}
		\lang{de}{\text{$f$ hat die station\"{a}ren Stellen $\var{zwidamin}$ und $\var{zwidb}$.}}
		\lang{en}{\text{$f$ has critical points at $x=\var{zwidamin}$ and $x=\var{zwidb}$.}}
		\solution{false}
	\end{choice}
	\begin{choice}
		\lang{de}{\text{$f$ ist zwischen den station\"{a}ren Stellen monoton fallend.}}
		\lang{en}{\text{$f$ is monotonically decreasing between its critical points.}}
		\solution{true}
	\end{choice}
\end{question}
    
\end{problem}

\embedmathlet{mathlet}
\end{content}