\documentclass{mumie.problem.gwtmathlet}
%$Id$
\begin{metainfo}
  \name{
    \lang{de}{A04: Funktionsgraph}
    \lang{en}{problem4}
  }
  \begin{description} 
 This work is licensed under the Creative Commons License Attribution 4.0 International (CC-BY 4.0)   
 https://creativecommons.org/licenses/by/4.0/legalcode 

    \lang{de}{}
    \lang{en}{}
  \end{description}
  \corrector{system/problem/GenericCorrector.meta.xml}
  \begin{components}
    \component{js_lib}{system/problem/GenericMathlet.meta.xml}{mathlet}
  \end{components}
  \begin{links}
  \end{links}
  \creategeneric
\end{metainfo}
\begin{content}
\usepackage{mumie.ombplus}
\usepackage{mumie.genericproblem}


\lang{de}{
	\title{A04: Funktionsgraph}
}
\begin{block}[annotation]
      
\end{block}
\begin{block}[annotation]
  Im Ticket-System: \href{http://team.mumie.net/issues/10285}{Ticket 10285}
\end{block}


\begin{problem}

\randomquestionpool{1}{6}
%\randomquestionpool{7}{10}

%Frage 1 von 10
\begin{question}
	\lang{de}{
		\text{Gegeben sei $f'(x)=2xe^{x^2}$. Welcher der Graphen kann nur $f$ darstellen?\\
		Mit Ihrer Maus k\"{o}nnen Sie den Ausschnitt und die Gr\"{o}{\ss}e des Bildes ver\"{a}ndern.}
		\explanation{}
	}
	\lang{en}{
		\text{Let $f'(x)=2xe^{x^2}$.  Only one of the following graphs is a depiction of the function $f$. Which one is it?\\
		You can pan and zoom the graph using your mouse.}
		\explanation{}
	}
	\begin{variables}   
		\function{f_lin}{e^(x^2)}
		\function{f_exp}{e^(x)}
		\function{f_expab}{xe^(x^2)}
	\end{variables}	
	\type{mc.unique}
	\field{rational}
	\plotF{1}{f_lin}
	\plotFrom{1}{-1}
	\plotTo{1}{1}
	\plotColor{1}{red}
	\plotF{2}{f_exp}
	\plotFrom{2}{-1}
	\plotTo{2}{1}
	\plotColor{2}{blue}	
	\plotF{3}{f_expab}
	\plotFrom{3}{-1}
	\plotTo{3}{1}
	\plotColor{3}{green}
	\plotLeft{-1}
	\plotRight{1}
	\plotSize{400}
	\begin{choice}
	 	\lang{de}{\text{Die rote Kurve entspricht $f$.}}
		\lang{en}{\text{The red curve is the graph of $f$.}}
		\solution{true}
	\end{choice}
	\begin{choice}
	 	\lang{de}{\text{Die blaue Kurve entspricht $f$.}}
		\lang{en}{\text{The blue curve is the graph of $f$.}}
		\solution{false}
	\end{choice}
	\begin{choice}
		\lang{de}{\text{Die gr\"{u}ne Kurve entspricht $f$.}}
		\lang{en}{\text{The green curve is the graph of $f$.}}
		\solution{false}
	\end{choice}
   \explanation{Denken Sie hier an die Eigenschaften der Exponentialfunktion. Es gilt $f(x) = f'(x)$ für $f(x) = e^x$.}
    
    \end{question}

%Frage 2 von 10
\begin{question}
	\lang{de}{
		\text{Gegeben sei $f'(x)=\frac{2x}{\sqrt{x^2+1}}$. Welcher der Graphen kann nur $f$ darstellen?\\
		Mit Ihrer Maus k\"{o}nnen Sie den Ausschnitt und die Gr\"{o}{\ss}e des Bildes ver\"{a}ndern.}
		\explanation{}
	}
	\lang{en}{
		\text{Let $f'(x)=\frac{2x}{\sqrt{x^2+1}}$.  Only one of the following graphs is a depiction of the function $f$. Which one is it?\\
		You can pan and zoom the graph using your mouse.}
	}
	\begin{variables}   
		\function{f_lin}{2*sqrt(x^2+1)}
		\function{f_exp}{x*sqrt(x^2+1)}
		\function{f_expab}{2*(x-0.5)^2*sqrt(x^2+1)}
	\end{variables}
	\type{mc.unique}
	\field{rational}
	\plotF{1}{f_lin}
	\plotFrom{1}{-1}
	\plotTo{1}{1}
	\plotColor{1}{red}
	\plotF{2}{f_exp}
	\plotFrom{2}{-1}
	\plotTo{2}{1}
	\plotColor{2}{blue}	
	\plotF{3}{f_expab}
	\plotFrom{3}{-1}
	\plotTo{3}{1}
	\plotColor{3}{green}
	\plotLeft{-1}
	\plotRight{1}
	\plotSize{400}
	\begin{choice}
	 	\lang{de}{\text{Die rote Kurve entspricht $f$.}}
		\lang{en}{\text{The red curve is the graph of $f$.}}
		\solution{true}
	\end{choice}
	\begin{choice}
	 	\lang{de}{\text{Die blaue Kurve entspricht $f$.}}
		\lang{en}{\text{The blue curve is the graph of $f$.}}
		\solution{false}
	\end{choice}
	\begin{choice}
		\lang{de}{\text{Die gr\"{u}ne Kurve entspricht $f$.}}
		\lang{en}{\text{The green curve is the graph of $f$.}}
		\solution{false}
	\end{choice}
    \explanation{Denken Sie an den Zusammenhang zwischen Extrema einer Funktion und den Nullstellen der Ableitung.}
\end{question}

%Frage 3 von 10
\begin{question}
	\lang{de}{
		\text{Gegeben sei $f'(x)=\frac{4}{\sqrt{x^2+1}}$. Welcher der Graphen kann nur $f$ darstellen?\\
		Mit Ihrer Maus k\"{o}nnen Sie den Ausschnitt und die Gr\"{o}{\ss}e des Bildes ver\"{a}ndern.}
		\explanation{}
	}
	\lang{en}{
	 	\text{Let $f'(x)=\frac{4}{\sqrt{x^2+1}}$.  Only one of the following graphs is a depiction of the function $f$. Which one is it?\\
		You can pan and zoom the graph using your mouse.}
	 }
	\begin{variables}   
		\function{f_lin}{4*arsinh(x)}
		\function{f_exp}{x^3*arsinh(x)}
		\function{f_expab}{2*sqrt(x^2+1)}
	\end{variables}	
	\type{mc.unique}
	\field{rational}
	\plotF{1}{f_expab}
	\plotFrom{1}{-2}
	\plotTo{1}{2}
	\plotColor{1}{red}
	\plotF{2}{f_exp}
	\plotFrom{2}{-2}
	\plotTo{2}{2}
	\plotColor{2}{blue}	
	\plotF{3}{f_lin}
	\plotFrom{3}{-2}
	\plotTo{3}{2}
	\plotColor{3}{green}
	\plotLeft{-2}
	\plotRight{2}
	\plotSize{400}
	\begin{choice}
	 	\lang{de}{\text{Die rote Kurve entspricht $f$.}}
		\lang{en}{\text{The red curve is the graph of $f$.}}
		\solution{false}
	\end{choice}
	\begin{choice}
	 	\lang{de}{\text{Die blaue Kurve entspricht $f$.}}
		\lang{en}{\text{The blue curve is the graph of $f$.}}
		\solution{false}
	\end{choice}
	\begin{choice}
		\lang{de}{\text{Die gr\"{u}ne Kurve entspricht $f$.}}
		\lang{en}{\text{The green curve is the graph of $f$.}}
		\solution{true}
	\end{choice}
    \explanation{Denken Sie an den Zusammenhang zwischen Extrema einer Funktion und den Nullstellen der Ableitung.}
\end{question}
	
%%%%%%%%%%%%%%%%
%Frage 4 von 10
\begin{question}
	\lang{de}{
		\text{Gegeben sei der Graph von $f'(x)$. Welche der folgenden Funktionen kommt nur f\"{u}r $f$ in Frage?\\
		Mit Ihrer Maus k\"{o}nnen Sie den Ausschnitt und die Gr\"{o}{\ss}e des Bildes ver\"{a}ndern.}
		\explanation{}
	}
	\lang{en}{
	 	\text{Given the graph of $f'(x)$, which of the following functions could be the function $f$?
	 	You can pan and zoom the graph using your mouse.}
	}
	\begin{variables}   
		\function{f}{cos(x)}
	\end{variables}
	\type{mc.unique}
	\field{rational}
	\plotF{1}{f}
	\plotFrom{1}{-6}
	\plotTo{1}{6}
	\plotColor{1}{red}
	\plotLeft{-6}
	\plotRight{6}
	\plotSize{400}
	\begin{choice}
		\text{$f(x)=\sin(x)$}
		\solution{true}
	\end{choice}
	\begin{choice}	 
		\text{$f(x)=x\cdot\sin(x)$}
		\solution{false}
	\end{choice}
	\begin{choice}	
		\text{$f(x)=\cos(x)$}
		\solution{false}
	\end{choice}
    \explanation{Denken Sie hier an die trigonometrischen Funktionen und ihre Ableitungen. Auch Kenntnisse über ihrer Graphen ist von Vorteil. }
    % \explanation{Bei dem Graph der Ableitung handelt es sich um eine Sinus-Kurve, denn $ \sin(0)=1$. Damit muss $f$ eine Cosinus-Kurve sein.}
\end{question}
	
%Frage 5 von 10
\begin{question}
	\lang{de}{
		\text{Gegeben sei der Graph von $f'(x)$. Welche der folgenden Funktionen kommt nur f\"{u}r $f$ in Frage?\\
		Mit Ihrer Maus k\"{o}nnen Sie den Ausschnitt und die Gr\"{o}{\ss}e des Bildes ver\"{a}ndern.}
		
	}
	\lang{en}{
		\text{Given the graph of $f'(x)$, which of the following functions could be the function $f$?
	 	You can pan and zoom the graph using your mouse.}
	}
	\begin{variables}   		
		\function{f}{e^x}
	\end{variables}
	\type{mc.unique}
	\field{rational}
	\plotF{1}{f}
	\plotColor{1}{red}
	\plotFrom{1}{-6}
	\plotTo{1}{6}
	\plotLeft{-6}
	\plotRight{6}
	\plotSize{400}
	\begin{choice}
		\text{$f(x)=e^x$}
		\solution{true}
	\end{choice}
	\begin{choice}
		\text{$f(x)=e^{-x}$}
		\solution{false}
	\end{choice}
	\begin{choice}
		\text{$f(x)=-e^x$}
		\solution{false}
	\end{choice}
   \explanation{Genaue Kenntnisse der Exponentialfunktion und ihrer Ableitung helfen hier weiter.}
\end{question}
	
%Frage 6 von 10
\begin{question}
	\lang{de}{
		\text{Gegeben sei der Graph von $f'(x)$. Welche der folgenden Funktionen kommt nur f\"{u}r $f$ in Frage?\\
		Mit Ihrer Maus k\"{o}nnen Sie den Ausschnitt und die Gr\"{o}{\ss}e des Bildes ver\"{a}ndern.}
		\explanation{}
	}
	\lang{en}{
		\text{Given the graph of $f'(x)$, which of the following functions could be the function $f$?
	 	You can pan and zoom the graph using your mouse.}
	}
	\begin{variables}   
		\function{f}{3*x^2+2x}
	\end{variables}
	\type{mc.unique}
	\field{rational}
	\plotF{1}{f}
	\plotFrom{1}{-6}
	\plotTo{1}{6}
	\plotColor{1}{red}
	\plotLeft{-6}
	\plotRight{6}
	\plotSize{400}
	\begin{choice}
		\text{$f(x)=x^3+x^2$}
		\solution{true}
	\end{choice}
	\begin{choice}
		\text{$f(x)=-x^4+x$}
		\solution{false}
	\end{choice}
	\begin{choice}
		\text{$f(x)=x^3-2x$}
		\solution{false}
	\end{choice}
    \explanation{Denken Sie an den Zusammenhang zwischen Extrema einer Funktion und den Nullstellen der Ableitung.}
\end{question}
	
%%%%%%%%%%%%%%%%%
%Frage 7 von 10
\begin{question}
	\lang{de}{
		\text{Gegeben $f(x) =\var{f}$ auf $[0;2\pi]$. 
		Wieviele station\"{a}re Stellen besitzt $f$? }
		\explanation{Die Ableitung ist $-\var{a}\sin(\var{a}x)+\var{a}$. 
		Es muss also bestimmt werden, an wie vielen Stellen\\ in $[0;2\pi]$ die Funktion $\sin(\var{a}x)$ den Wert $1$ annimmt. }
	}
	\lang{en}{
		\text{Let $f(x) =\var{f}$ on $[0,2\pi]$. 
		How many critical points does $f$ have?}
		\explanation{The derivative is $-\var{a}\sin(\var{a}x)+\var{a}$. 
		Determine at how many points in $[0,2\pi]$ the function $\sin(\var{a}x)$ is $1$.}
	}
	\begin{variables}
		\randint[Z]{a}{2}{10}
	    \function{f}{cos(a*x)+a*x}
		\function{loesz}{a}
	\end{variables}
	\type{input.number}
	\begin{answer}
	  	\lang{de}{\text{ Anzahl:}}
	   	\lang{en}{\text{Number:}}
		\solution{loesz}		
	\end{answer}
\end{question}
		
%Frage 8 von 10		
\begin{question}
	\lang{de}{
		\text{Gegeben $f(x) =\var{f}$ auf $(0;\infinity)$. 
		Wieviele station\"{a}re Stellen besitzt $f$? }
		\explanation{Die Ableitung ist $x^{\var{b}}(\var{a}\ln(x)+1)$. 
		Es muss also bestimmt werden, wieviele Nullstellen\\ $\var{a}\ln(x)+1$ auf $(0;\infinity)$ hat. }
	}
	\lang{en}{
		\text{Let $f(x) = \var{f}$ on $(0,\infinity)$. 
		How many critical points does $f$ have?}
		\explanation{The derivative is $x^{\var{b}}(\var{a}\ln(x)+1)$. The number of roots of $\var{a}\ln(x)+1$ on $(0,\infinity)$ needs to be determined.}
	}
	\begin{variables}
    	\randint{a}{3}{10}
		\function{f}{x^a*ln(x)}
	    \function[calculate]{b}{a-1}
	   	\number{loes}{1}
	\end{variables}
	\type{input.number}
	\begin{answer}
    	\lang{de}{\text{Anzahl:}}
       	\lang{en}{\text{Number:}}
		\solution{loes}		
	\end{answer}
\end{question}
	
%Frage 9 von 10
\begin{question}
	\lang{de}{
		\text{Gegeben $f(x) =\var{f}$ auf $(0;\infinity)$. 
		Wieviele station\"{a}re Stellen besitzt $f$? }
		\explanation{Die Ableitung ist $x^{\var{b}}(\var{a}\ln(x)+1)$. 
		Es muss also bestimmt werden, wieviele Nullstellen\\ $\var{a}\ln(x)+1$ auf $(0;\infinity)$ hat. }
	}
	\lang{en}{
		\text{Let $f(x) = \var{f}$ on $(0,\infinity)$. 
		How many critical points does $f$ have?}
		\explanation{The derivative is $x^{\var{b}}(\var{a}\ln(x)+1)$. The number of roots of $\var{a}\ln(x)+1$ on $(0,\infinity)$ needs to be determined.}
	}	
	\begin{variables}
		\randint{a}{3}{10}
	    \function{f}{x^a*ln(x)}
		\function[calculate]{b}{a-1}
		\number{loes}{1}
	\end{variables}
	\type{input.number}
	\begin{answer}
    	\lang{de}{\text{Anzahl:}}
       	\lang{en}{\text{Number:}}
		\solution{loes}
	\end{answer}
\end{question}
	
%Frage 10 von 10
\begin{question}
	\lang{de}{
		\text{Gegeben $f(x) =\var{f}$ auf $\R$. 
		Wieviele station\"{a}re Stellen besitzt $f$? }
		\explanation{Die Ableitung ist $x^{\var{u}}(\var{s}+\var{t}x^{\var{t}})e^{x^{\var{t}}}$. 
		Es muss also bestimmt werden, wieviele Nullstellen\\ diese Funktion hat. }
	}
	\lang{en}{
		\text{Let $f(x) = \var{f}$ on $(0,\infinity)$. 
		How many critical points does $f$ have?}
		\explanation{The derivative is $x^{\var{u}}(\var{s}+\var{t}x^{\var{t}})e^{x^{\var{t}}}$. The number of roots of this function on $(0,\infinity)$ needs to be determined.}
	}
	\begin{variables}
		\randint{a}{3}{10}
	    \function[calculate]{s}{2*a}
	    \function[calculate]{u}{2*a-1}
	    \randint{b}{3}{10}
	    \function[calculate]{t}{2*b-1}
		\function{f}{x^s*exp(x^t)}
	    \number{loes}{2}
		\functionNormalize{f}
	\end{variables}
	\type{input.number}
	\begin{answer}
      	\lang{de}{\text{Anzahl:}}
       	\lang{en}{\text{Number:}}
		\solution{loes}
	\end{answer}
\end{question}

\end{problem}

\embedmathlet{mathlet}
\end{content}