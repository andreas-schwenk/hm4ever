%$Id:  $
\documentclass{mumie.article}
%$Id$
\begin{metainfo}
  \name{
    \lang{de}{Eigenschaften und höhere Ableitungen}
    \lang{en}{Properties and higher derivatives}
  }
  \begin{description} 
 This work is licensed under the Creative Commons License Attribution 4.0 International (CC-BY 4.0)   
 https://creativecommons.org/licenses/by/4.0/legalcode 

    \lang{de}{Beschreibung}
    \lang{en}{Description}
  \end{description}
  \begin{components}
    \component{generic_image}{content/rwth/HM1/images/g_tkz_T106_ParabolaTangents.meta.xml}{T106_ParabolaTangents}
    \component{generic_image}{content/rwth/HM1/images/g_img_00_Videobutton_schwarz.meta.xml}{00_Videobutton_schwarz}
     \component{generic_image}{content/rwth/HM1/images/g_img_00_video_button_schwarz-blau.meta.xml}{00_video_button_schwarz-blau}
  \end{components}
  \begin{links}
    \link{generic_article}{content/rwth/HM1/T106_Differentialrechnung/g_art_content_22_extremstellen.meta.xml}{extremstellen}
    \link{generic_article}{content/rwth/HM1/T207_Intervall_Schachtelung/g_art_content_23_intervallschachtelung.meta.xml}{intervallschachtelung}
    \link{generic_article}{content/rwth/HM1/T211_Eigenschaften_stetiger_Funktionen/g_art_content_33_zwischenwertsatz.meta.xml}{zwischenwertsatz}
    \link{generic_article}{content/rwth/HM1/T106_Differentialrechnung/g_art_content_23_kurvendiskussion.meta.xml}{kurvendiskussion}
    \link{generic_article}{content/rwth/HM1/T211_Eigenschaften_stetiger_Funktionen/g_art_content_33_zwischenwertsatz.meta.xml}{saetze-stetig}
  \end{links}
  \creategeneric
\end{metainfo}
\begin{content}
\usepackage{mumie.ombplus}
\ombchapter{1}
\ombarticle{3}

\title{\lang{de}{Eigenschaften differenzierbarer Funktionen und höhere Ableitungen}
       \lang{en}{Properties of differentiable functions and higher derivatives}}

\begin{block}[annotation]


\end{block}
\begin{block}[annotation]
  Im Ticket-System: \href{http://team.mumie.net/issues/10031}{Ticket 10031}\\
\end{block}

\begin{block}[info-box]
\tableofcontents
\end{block}

\section{\lang{de}{Stetigkeit}\lang{en}{Continuity}}\label{sec:stetigkeit}

\lang{de}{
Bisher hatten wir Stetigkeit und Differenzierbarkeit von Funktionen separat behandelt. Wir werden 
jedoch im folgenden sehen, dass Differenzierbarkeit eine stärkere Bedingung ist und Stetigkeit 
impliziert.
}
\lang{en}{
Until now we have handled continuity and differentiability of functions seperately. The following 
theorem states that differentiability is a stronger condition and that it implies continuity.
}

\begin{theorem}\label{thm:diffstetig}
\lang{de}{Sei $f:D\to \R$ eine Funktion.}
\lang{en}{Let $f:D\to \R$ be a function.}
\begin{enumerate}
\item \lang{de}{Ist $f$ an einer Stelle $x_0\in D$ differenzierbar, so ist $f$ in $x_0$ auch stetig.}
      \lang{en}{If $f$ is differentiable at a point $x_0\in D$, then $f$ is continuous at $x_0$.}
\item \lang{de}{Ist $f$ auf $E\subseteq D$ differenzierbar, so ist $f$ auf $E$ stetig.}
      \lang{en}{If $f$ is differentiable on $E\subseteq D$, then $f$ is continuous on $E$.}
\end{enumerate}
\lang{de}{
\floatright{\href{https://www.hm-kompakt.de/video?watch=507}{\image[75]{00_Videobutton_schwarz}}}\\\\
}
\lang{en}{}
\end{theorem}


\begin{proof*}[\lang{de}{Beweis der Stetigkeit bei Differenzierbarkeit}
               \lang{en}{Proof of differentiability implying continuity}]
\begin{incremental}[\initialsteps{0}]
\step \lang{de}{
Die zweite Aussage ist eine direkte Folgerung aus der ersten Aussage.\\
Sei nun $x_0\in D$ eine Stelle, an der $f$ differenzierbar ist, dann existiert also der Grenzwert
}
\lang{en}{
The second statement is a direct consequence of the first.\\
Let $x_0\in D$ be a point at which $f$ is differentiable, then the following limit exists:
}
\[ f'(x)= \lim_{x\to x_0} \frac{f(x)-f(x_0)}{x-x_0}. \]
\step
\lang{de}{Mit den Grenzwertregeln ist dann aber}
\lang{en}{By the limit rules we hence have}
\begin{eqnarray*}
\lim_{x\to x_0} f(x) &=& f(x_0)+  \lim_{x\to x_0} (f(x)-f(x_0)) \\ 
&=& f(x_0)+  \lim_{x\to x_0} \frac{f(x)-f(x_0)}{x-x_0} \cdot   \lim_{x\to x_0} (x-x_0) \\
&=& f(x_0) + f'(x_0)\cdot 0 =f(x_0)
\end{eqnarray*} 
\lang{de}{Also ist $f$ an der Stelle $x_0$ stetig.}
\lang{en}{Therefore $f$ is continuous at the point $x_0$.}
\end{incremental}
\end{proof*}




\section{\lang{de}{Extrema, Monotonie und Mittelwertsatz}
         \lang{en}{Extrema, monotonicity and mean value theorem}}\label{sec:monotonie}

\lang{de}{
Nun interessieren wir uns daf\"{u}r, in welchen Bereichen eine Funktion $f$ w\"{a}chst oder 
f\"{a}llt. Z.B. ist dies interessant, wenn man nach Extremstellen sucht. Die detaillierteren 
Zusammenhänge der ersten und zweiten Ableitung mit dem Verlauf der Funktion $f$ wurden schon in Teil 
1 in den Abschnitten \link{extremstellen}{Monotonie und Extremstellen} und 
\link{kurvendiskussion}{Kurvendiskussion} behandelt. Hier soll lediglich der im 
\link{extremstellen}{Abschnitt Monotonie und Extremstellen} anschauliche Zusammenhang 
mathematisch begründet werden.
}
\lang{en}{
Now we consider in which regions of the domain a function $f$ increases or decreases. For example, 
this is useful for when we search for extrema. The detailed relationship between the first and 
second derivative and the graph of the function $f$ was already discussed in the earlier sections on 
\link{extremstellen}{monotonicity and extrema} and \link{kurvendiskussion}{curves}. The relationship 
introduced in \link{extremstellen}{the former} will be mathematically justified in this section.
}

\begin{theorem}[\lang{de}{Notwendige Bedingung für lokale Extrema}
                \lang{en}{Necessary conditions for local extrema}]\label{thm:NB_extrema} 
\lang{de}{
Ist $f$ eine auf einem Intervall $I$ differenzierbare Funktion und $x_0$ im Inneren des Intervalls 
$I$ eine Stelle, an der $f$ ein lokales Maximum oder lokales Minimum hat, so gilt
}
\lang{en}{
If $f$ is a function that is differentiable on an an interval $I$ and $x_0$ is a point in $I$ at 
at which $f$ has a local maximum or a local minimum, then
}
\[ f'(x_0)=0. \]
\lang{de}{
\floatright{\href{https://www.hm-kompakt.de/video?watch=522}{\image[75]{00_Videobutton_schwarz}}}\\\\
}
\lang{en}{}
\end{theorem}


\begin{proof*}[\lang{de}{Beweis der Notwendigen Bedingung für lokale Extrema}
               \lang{en}{Proof of the necessary condition for local extrema}]
\begin{incremental}[\initialsteps{0}]
\step \lang{de}{
Hat $f$ an der Stelle $x_0$ ein lokales Maximum, so gilt in einer kleinen Umgebung 
$U_\epsilon(x_0)=(x_0-\epsilon,x_0+\epsilon)$ von $x_0$
}
\lang{en}{
If $f$ has a local maximum at the point $x_0$, so in a small neighbourhood 
$U_\epsilon(x_0)=(x_0-\epsilon,x_0+\epsilon)$ of $x_0$ we have
}
\[ f(x)\leq f(x_0) \quad \text{\lang{de}{für alle }\lang{en}{for all }}x\in  U_\epsilon(x_0).\]

\step
\lang{de}{Dann ist aber der Differenzenquotient}
\lang{en}{The the quotient whose limit we define as the derivative is}
  	\[		\frac{f(x_0+h)-f(x_0)}{h}\leq 0, \]
\lang{de}{wenn $0<h<\epsilon$ und}
\lang{en}{if $0<h<\epsilon$ and}
		\[		\frac{f(x_0+h)-f(x_0)}{h}\geq 0, \]
\lang{de}{wenn $-\epsilon<h<0$. Also gilt}
\lang{en}{if $-\epsilon<h<0$. Thus}
		\[  \lim_{h\nearrow 0} \frac{f(x_0+h)-f(x_0)}{h}\geq 0\geq  \lim_{h\searrow 0} \frac{f(x_0+h)-f(x_0)}{h}. \]
\step
\lang{de}{
Da $f$ aber differenzierbar ist, sind beide Grenzwerte gleich und daher insbesondere gleich $0$. D.h.
}
\lang{en}{
As $f$ is differentiable, the two limits are equal and in particular they are equal to $0$. Hence
}
\[  f'(x_0)= 	\lim_{h\to 0} \frac{f(x_0+h)-f(x_0)}{h}	=0. \]

\lang{de}{
Für ein lokales Minimum geht der Beweis genauso. Lediglich die Vergleichszeichen für die 
Differenzenquotienten und die einseitigen Grenzwerte drehen sich alle um.
}
\lang{en}{
The proof for a local minimum is analogous. Only the inequalities and one-sided limits are reversed.
}
\end{incremental}
\end{proof*}



\begin{quickcheck}
		\field{rational}
%		\type{mc.unique}
		\begin{variables}
			\randint[Z]{x1}{-3}{1}
			\randint[Z]{v}{-1}{1}
			\randint[Z]{d}{1}{3}
            \function[calculate]{x2}{x1+2*d}
            \function[calculate]{a}{-3/2*(x1+x2)}
            \function[calculate]{b}{3*(x1*x2)}
			\randint[Z]{c}{-3}{3}
		    \function[expand, normalize]{df}{3*(x-x1)*(x-x2)}
			\function[normalize]{f}{x^3+a*x^2+b*x+c}
            \function[calculate]{x3}{x1+v*d}
		\end{variables}
		
		\text{\lang{de}{
        Wir betrachten die Funktion $f(x)=\var{f}$, deren Ableitungsfunktion
        $f'(x)=\var{df}$ ist. Welche der folgenden Stellen ist mit Sicherheit keine Extremstelle
        von $f$?
        }
        \lang{en}{
        Consider the function $f(x)=\var{f}$, whose derivative is $f'(x)=\var{df}$. At which of the 
        following points is there definitely \textbf{not} an extremum of $f$?
        }}
%        \permutechoices{1}{3}
\begin{choices}{unique}
\begin{choice}
			\text{$x=\var{x1}$}
            \solution{false}
       	\end{choice}
		\begin{choice}
			\text{$x=\var{x2}$}
            \solution{false}
       	\end{choice}
		\begin{choice}
			\text{$x=\var{x3}$}
            \solution{true}
       	\end{choice}
\end{choices}
\explanation{\lang{de}{Verschwindet $f'(x)$ nicht, so kann in $x$ auch keine Extremstelle vorliegen.}
             \lang{en}{If $f'(x)$ does not vanish at a point $x$, then there is no extremum at $x$.}
		}
	\end{quickcheck}

\lang{de}{
Monotonie beinhaltet, dass die Ableitung an allen Stellen im Intervall ein einheitliches Vorzeichen hat.
}
\lang{en}{
Monotonicity on an interval implies that the slope has the same sign at every point in that interval.
}


\begin{theorem}\label{thm:diffmonotonie}
\lang{de}{Ist $f$ eine auf einem Intervall $I$ differenzierbare Funktion, so gelten:}
\lang{en}{If $f$ is a function that is differentiable on $I$, then we have:}
\begin{align*}
    \lang{de}{&\underline{f'(x)\;\text{ für alle }x\in I}&& \underline{\text{Monotonieverhalten von}\;f\;\text{auf}\;I}}
    \lang{en}{&\underline{f'(x)\;\text{ on}\;I}&& \underline{\text{Monotonicity of}\;f\;\text{on}\;I}}
    \\
    \lang{de}{&f'(x)\geq 0 &\Leftrightarrow &\:f\;\text{monoton wachsend}}
    \lang{en}{&f'(x)\geq 0 &\Leftrightarrow &\:f\;\text{monotonically increasing}}
    \\
    \lang{de}{&f'(x)\leq 0 &\Leftrightarrow &\:f\;\text{monoton fallend}}
    \lang{en}{&f'(x)\leq 0 &\Leftrightarrow &\:f\;\text{monotonically decreasing}}
    \\
    \lang{de}{\text{und}&&&}
    \lang{en}{\text{and}&&&}
    \\
    \lang{de}{&f'(x)> 0 &\Rightarrow &\:f\;\text{streng monoton wachsend}}
    \lang{en}{&f'(x)> 0 &\Rightarrow &\:f\;\text{strictly monotonically increasing}}
    \\
    \lang{de}{&f'(x)< 0 &\Rightarrow &\:f\;\text{streng monoton fallend}}
    \lang{en}{&f'(x)< 0 &\Rightarrow &\:f\;\text{strictly monotonically decreasing}}
    \\
\end{align*}
\lang{de}{
\floatright{\href{https://www.hm-kompakt.de/video?watch=520}{\image[75]{00_Videobutton_schwarz}}}\\~
}
\lang{en}{}
\end{theorem}


\begin{proof*}[\lang{de}{Beweis der Monotonieaussagen}\lang{en}{Proof of the monotonicity statements}]
\begin{incremental}[\initialsteps{0}]
\step
	\lang{de}{Ist $f$ auf dem Intervall $I$ monoton wachsend und $x$ aus $I$. Dann ist also}
  \lang{en}{If $f$ is monotonically increasing on the interval $I$ and $x$ is a point in $I$, then}
	\[
		f(x+h)-f(x)\geq 0\;\;\;\;\text{ \lang{de}{f\"{u}r alle}\lang{en}{for all}}\;\;\;h> 0\;\;\;\text{\lang{de}{ bzw.}\lang{en}{ or}}
	\]
	\[
		f(x+h)-f(x)\leq 0\;\;\;\text{ \lang{de}{f\"{u}r alle}\lang{en}{for all}}\;\;\;h< 0\;\;\;\;\;\;\;
	\] 
	\lang{de}{mit $x+h\in I$. Dann ist aber der Differenzenquotient in jedem der beiden F\"{a}lle}
  \lang{en}{with $x+h\in I$. In both cases the quotient whose limit is defined as the derivative is}
	\[
		\frac{f(x+h)-f(x)}{h}\geq 0.
	\] 
    
  \lang{de}{
  Dann ist aber auch der Grenzwert des Differenzenquotienten, als die Ableitung an der Stelle $x$
  }
  \lang{en}{
  But then the limit of this quotient, i.e. the derivative at the point $x$
  }
\[
f'(x) = \lim_{h\rightarrow 0} \frac{f(x+h)-f(x)}{h}
\] 	
\lang{de}{
gr\"{o}{\ss}er oder gleich $0$. Das hei{\ss}t $f'(x)\geq 0$. Entsprechend ist bei einer monoton 
fallenden Funktion der Differenzenquotient $\frac{f(x+h)-f(x)}{h}$ stets kleiner oder gleich $0$, 
weshalb in diesem Fall $f'(x)\leq 0$ für alle $x\in I$.
}
\lang{en}{
is also greater than or equal to  $0$. That is, $f'(x)\geq 0$. Similarly, if $f$ is monotonically 
decreasing, the quotient $\frac{f(x+h)-f(x)}{h}$ is always smaller than or equal to $0$, in which 
case $f'(x)\leq 0$ for all $x\in I$.
}

\step \lang{de}{
Die umgekehrte Schlussrichtung, also dass es ein $x_0\in I$ gibt mit $f'(x_0)< 0$, wenn $f$ auf $I$ 
nicht monoton wachsend ist, folgt aus dem \lref{thm:mittelwertsatz}{Mittelwertsatz}. Ist nämlich $f$ 
nicht monoton wachsend, so gibt es Stellen $a<b$ mit $f(a)>f(b)$. Nach dem Mittelwertsatz gibt es 
dann eine Stelle $\xi$ mit $a<\xi<b$ so, dass
}
\lang{en}{
The converse implication, that there exists $x_0\in I$ with $f'(x_0)< 0$ if $f$ is not monotonically 
increasing on $I$ follows from the \lref{thm:mittelwertsatz}{mean value theorem}. Indeed, if $f$ is 
not monotonically increasing, there exist points $a<b$ with $f(a)>f(b)$. By the mean value theorem 
there then exists a point $\xi$ with $a<\xi<b$ such that
}
\[  f'(\xi)= \frac{f(b)-f(a)}{b-a}<0. \]
\lang{de}{
Entsprechend gibt es für eine Funktion $f$, die nicht monoton fallend ist, Stellen $a<b$ mit 
$f(a)<f(b)$ und daher nach dem  Mittelwertsatz eine weitere Stelle $\xi$ mit $a<\xi<b$ so, dass
}
\lang{en}{
Similarly, if a function $f$ is not monotonically decreasing, there exist points $a<b$ with 
$f(a)<f(b)$ and by the mean value theorem there exists a point $\xi$ with $a<\xi<b$ such that
}
\[  f'(\xi)= \frac{f(b)-f(a)}{b-a}>0. \]


\lang{de}{
Um schließlich zu sehen, dass die Ungleichung $f'(x)>0$ für alle $x\in I$ die strenge Monotonie 
impliziert, nimmt man zunächst an, dass $f$ nicht streng monoton ist. 
%Da $f'(x)\geq 0$ aber die Monotonie impliziert, wäre $f$ also monoton wachsend, aber nicht streng monoton wachsend.
Dann gibt es aber $a,b$ in $I$ mit $a<b$ und $f(a)=f(b)$, und wieder wegen dem \lref{thm:mittelwertsatz}{Mittelwertsatz} eine Stelle $\xi$ mit $a<\xi<b$ so, dass
}
\lang{en}{
To show that the inequality $f'(x)>0$ being satisfied for all $x\in I$ implies that $f$ is strictly 
monotonically increasing, we assume towards a contradiction that $f$ is not strictly monotonically 
increasing. But then there exist points $a,b$ in $I$ with $a<b$ and $f(a)=f(b)$, and again by the 
\lref{thm:mittelwertsatz}{mean value theorem} there exists a point $\xi$ with $a<\xi<b$ such that
}
\[  f'(\xi)= \frac{f(b)-f(a)}{b-a}=0. \]
\lang{de}{Also ist $f'(x)$ nicht für alle $x\in I$ positiv.}
\lang{en}{Therefore $f'(x)$ is not strictly positive for all $x\in I$, a contradiction.}
    

\end{incremental}
\end{proof*}


%
%\begin{example}
%	\image{image1}
%	\\
%	\lang{de}{Die Funktion $f(x)=x^2$ ist monoton fallend auf $(-\infty;0]$ und monoton steigend auf $[0;\infty)$. An einer Stelle $x_0$ mit 
%	$x_0\leq 0$ hat also die Tangente eine Steigung $f'(x_0)\leq 0$, an einer Stelle $x_1$ mit 
%	$x_1\geq 0$ hat die Tangente eine positive Steigung $f'(x_1)\geq 0$.\\
%	Da die Tangente in $x_0 <0$ sogar eine negative Steigung hat, ist $f'(x) <0$ f\"{u}r $x\in (-\infty;0)$, 
%	und damit ist $f$ auf $(-\infty;0)$ streng monoton fallend. 
%	Analog gilt $f'(x)>0$ auf $(0;\infty)$ und $f$ ist dort streng monoton steigend. }
%	\lang{en}{The function $f(x)=x^2$ is monotonically decreasing on $(-\infty,0]$ and monotonically increasing on $[0,\infty)$. The slope of the tangent at any point $x_0$ with 
%	$x_0\leq 0$ is $f'(x_0)\leq 0$. The slope of the tangent at any point $x_1$ with $x_1\geq 0$ is positive: $f'(x_1)\geq 0$.\\
%	Because the tangent at $x_0<0$ has a negative slope, $f'(x) <0$ for $x\in (-\infty,0)$, and hence $f$ is strictly monotonically decreasing on $(-\infty,0)$. Analogous results
%	hold for $f'(x)>0$ on $(0,\infty)$: $f$ is strictly monotonically increasing on $(0,\infty)$.}
%\end{example}

\begin{block}[warning]
\lang{de}{
Die Funktion $f$ kann streng monoton wachsen bzw. streng monoton fallen, auch wenn an manchen Stellen 
die Ableitung gleich Null ist, wie am Beispiel $f(x)=ax^3$ (mit $a>0$ bzw. mit $a<0$) zu sehen ist.
}
\lang{en}{
The function $f$ can be strictly monotonically increasing or strictly monotonically decreasing even 
if its derivative takes the value $0$ at certain points, like for example $f(x)=ax^3$ with either 
$a>0$ or $a<0$.
}
\end{block}


\begin{example}
\begin{center}
\image{T106_ParabolaTangents}
\end{center}
	\lang{de}{
  Die Funktion $f(x)=x^2$ ist monoton fallend auf $(-\infty;0]$ und monoton steigend auf 
  $[0;\infty)$. An einer Stelle $x_0$ mit $x_0\leq 0$ hat also die Tangente eine Steigung 
  $f'(x_0)\leq 0$, an einer Stelle $x_1$ mit $x_1\geq 0$ hat die Tangente eine positive Steigung 
  $f'(x_1)\geq 0$.\\
	Da die Tangente in $x_0 <0$ sogar eine negative Steigung hat, ist $f'(x) <0$ f\"{u}r 
  $x\in (-\infty;0)$, und damit ist $f$ auf $(-\infty;0]$ streng monoton fallend. 
	Analog gilt $f'(x)>0$ auf $[0;\infty)$ und $f$ ist dort streng monoton steigend.
  }
	\lang{en}{
  The function $f(x)=x^2$ is monotonically decreasing on $(-\infty;0]$ and monotonically increasing 
  on $[0;\infty)$. The slope of the tangent at any point $x_0$ with $x_0\leq 0$ is therefore 
  $f'(x_0)\leq 0$, and the slope of the tangent at any point $x_1$ with $x_1\geq 0$ is 
  $f'(x_1)\geq 0$.\\
	Because the tangent of a point $x_0<0$ even has a strictly negative slope, $f$ is strictly 
  monotonically decreasing on $(-\infty,0)$. Analogously $f'(x)>0$ on $[0,\infty)$, implying that $f$ 
  is strictly monotonically increasing on $[0;\infty)$.
  }
\end{example}

\lang{de}{
Das Steigungsverhalten einer differenzierbaren Funktion steht also in engem Zusammenhang mit dem 
Vorzeichen ihrer Ableitung. Verschwindet die Ableitung einer Funktion $f$ an einer Stelle $x_0$, so 
ist die Tangente an den Graphen in diesem Punkt waagerecht. Die lineare Approximation steigt nicht 
und f\"{a}llt nicht. Deswegen nennt man eine solche Stelle auch eine station\"{a}re Stelle.
\\\\
Wir enden diesen Paragraphen mit dem Mittelwertsatz, der im Monotoniebeweis schon benutzt wurde, und 
welcher generell einen Zusammenhang zwischen Funktionswerten und Werten der Ableitung liefert.
}
\lang{en}{
The slope of a differentiable function is closely related to the sign of its first derivative.
If the first derivative of a function $f$ vanishes at a point $x_0$, then the tangent of the graph at 
that point is horizontal. Points on the graph of a function where the the tangent line is horizontal 
are called critical or stationary points.
\\\\
We finally see the mean value theorem, which has been applied already in the earlier proof. It gives 
a useful relationship between values of a function and values of its derivative.
}



\begin{theorem}[\lang{de}{Mittelwertsatz}\lang{en}{Mean value theorem}]\label{thm:mittelwertsatz}
  \lang{de}{
  Sei $f:I\to\R$ eine  Funktion auf einem \notion{Intervall} $I \subseteq \R$, die auf ganz $I$ 
  stetig ist und die im Inneren von $I$ differenzierbar ist. Sind $a,b \in I,\; a<b$, so gibt es eine 
  Stelle $\xi$ zwischen $a$ und $b$ so, dass
  }
  \lang{en}{
  Let $f:I\to\R$ be a function defined and continuous on an \notion{interval} $I \subseteq \R$, and 
  suppose $f$ is differentiable on the interior of $I$. If $a,b \in I,\; a<b$ then there exists a 
  point $\xi$ between $a$ and $b$ such that
  }
  \[ 
  \frac{f(b)-f(a)}{b-a} = f'(\xi ).
  \]
  
  
\end{theorem}



\begin{proof*}[\lang{de}{Beweis Mittelwertsatz}\lang{en}{Proof of the mean value theorem}]
\begin{incremental}[\initialsteps{0}]
\step \lang{de}{
Betrachten wir zunächst den Fall, dass $f(b)=f(a)$ ist. In diesem Fall muss eine Stelle $\xi$
zwischen $a$ und $b$ gefunden werden, welche $f'(\xi)=0$ erfüllt.
\\\\
Falls $f$ konstant ist, gilt $f'(\xi)=0$ für alle $\xi\in [a,b]$, weshalb wir im folgenden nur 
nicht-konstante $f$ betrachten.\\
Da $f$ auf $[a,b]$ stetig ist, nimmt die Funktion $f$ auf diesem Intervall ein Maximum und ein 
Minimum an (siehe \link{saetze-stetig}{Abschnitt "`Sätze zu stetigen Funktionen"'}). Da $f$ nicht 
konstant ist, liegt eines der beiden im Inneren des Intervalls, weshalb an dieser Stelle die 
Ableitung verschwindet. Diese Stelle kann man also als $\xi$ wählen.
}
\lang{en}{
First let us consider the case $f(b)=f(a)$. We look for a point $\xi$ between $a$ and $b$ such that 
$f'(\xi)=0$.
\\\\
If $f$ is constant, then $f'(\xi)=0$ for all $\xi\in [a,b]$, so we suppose that $f$ is not constant.\\
As the function $f$ is continuous on $[a,b]$, it takes a maximum and a minimum value on the interval 
(see \link{saetze-stetig}{the section on continuous functions}). As $f$ is not constant, one of the 
two lies on the interior of the interval, and at this point the derivative of the function vanishes. 
This point may be chosen as $\xi$.
}

\step
\lang{de}{Im Fall $f(b)\neq f(a)$ betrachten wir die Hilfsfunktion}
\lang{en}{In the case where $f(b)\neq f(a)$, we consider the function}
\[  h(x)=f(x)-\frac{f(b)-f(a)}{b-a}\cdot (x-a). \]
\lang{de}{Für diese gilt}
\lang{en}{We note that}
\[ h(a)=f(a)-\frac{f(b)-f(a)}{b-a}\cdot (a-a)=f(a) \]
\lang{de}{und}
\lang{en}{and}
\[ h(b)=f(b)-\frac{f(b)-f(a)}{b-a}\cdot (b-a)=f(a)=h(a). \]
\lang{de}{Auch ist $h$ auf $I$ differenzierbar mit}
\lang{en}{The function $h$ is also differentiable on $I$, with}
\[  h'(x)=f'(x)-\frac{f(b)-f(a)}{b-a}. \]
\lang{de}{
Nach dem bisher gezeigten gibt es also eine Stelle $\xi\in (a,b)$ mit $h'(\xi)=0$, d.h.
}
\lang{en}{
The function $h$ falls under the simpler case that we have proved already, so there exists a point 
$\xi\in (a,b)$ with $h'(\xi)=0$, that is
}
\[ f'(\xi)=\frac{f(b)-f(a)}{b-a}. \]

\end{incremental}
\end{proof*}

\lang{de}{
Auch das folgende Video behandelt den Mittelwertsatz und Monotonie.
\\\\
\floatright{\href{https://api.stream24.net/vod/getVideo.php?id=10962-2-10761&mode=iframe&speed=true}{\image[75]{00_video_button_schwarz-blau}}}\\
\\\\
}
\lang{en}{}
\section{\lang{de}{Höhere Ableitungen}\lang{en}{Higher derivatives}}\label{sec:hoehere-abl}

\lang{de}{
Für eine differenzierbare Funktion $f:D\to \R$ hatten wir die Ableitung auch wieder als Funktion 
$f':D\to \R$ aufgefasst. Ist diese wieder differenzierbar, so kann man auch sie ableiten und erhält 
eine Funktion $(f')'$. Man nennt diese dann die zweite Ableitung von $f$.
}
\lang{en}{
A differentiable function $f:D\to \R$ has a derivative function $f':D\to \R$. If this derivative is 
also differentiable, we may label its derivative  $(f')'$. This is called the second derivative of 
$f$.
}

\begin{definition} \label{def:höhere_Ableitungen}
  	\lang{de}{
  	Sei $f$ eine differenzierbare Funktion mit Ableitung $f'$. Ist $f'$ an einer Stelle $x_0$ 
    differenzierbar, d.h. existiert der Grenzwert
    }
    \lang{en}{
		Let $f$ be a differentiable function with derivative $f'$. If $f'$ is differentiable at a 
    point $x_0$, i.e. the limit
    }
		\[
			\lim_{h\rightarrow 0}\frac{f'(x_0+h)-f'(x_0)}{h},
		\]
    \lang{de}{
    so existiert also $(f')'(x_0)$. Wenn die Ableitungsfunktion $f'$ in allen Stellen des 
    Definitionsbereiches differenzierbar ist, hei{\ss}t $f$ \textit{zweimal differenzierbar}. Man 
    schreibt dann statt $(f')'$ kurz $f''$.\\
		Die \textit{zweite Ableitung von $f$ in $x_0$} ist also die Ableitung von $f'$ und
    }
    \lang{en}{
    exists, then $(f')'(x_0)$ exists too. If the first derivative $f'$ is differentiable at all 
    of the points in its domain, then the function $f$ is called \textit{twice differentiable}. 
    Instead of denoting its second derivative $(f')'$, we simply write $f''$.\\
		The \textit{second derivative of $f$ at $x_0$} is just the derivative of $f'$ at $x_0$:
  }
		\[
			f''(x_0)=\lim_{h\rightarrow 0}\frac{f'(x_0+h)-f'(x_0)}{h}.
		\] 

	
\end{definition}

\begin{example} 
	\begin{enumerate}
	\item
	    \lang{de}{Sei $f(x)=x^3+2x^2+5$. Dann ist $f'(x)=3x^2+4x$ und $f''(x)=6x+4$.}
	    \lang{en}{Let $f(x)=x^3+2x^2+5$. Then $f'(x)=3x^2+4x$ and $f''(x)=6x+4$.}
	\item 
	    \lang{de}{Sei $f(x)=\sin(x)$. Es ist $f'(x)=\cos(x)$, folglich $f''(x)=-\sin(x)$.}
	    \lang{en}{Let $f(x)=\sin(x)$. Then $f'(x)=\cos(x)$ and $f''(x)=-\sin(x)$.}
	\item 
	    \lang{de}{Sei $f(x)=\cos(x)$. Wir erhalten $f'(x)=-\sin(x)$ und $f''(x)=-\cos(x)$.}
	    \lang{en}{Let $f(x)=\cos(x)$. Then $f'(x)=-\sin(x)$ and $f''(x)=-\cos(x)$.}
	\item 
	    \lang{de}{Sei $f(x)=e^{2x}$. Nach der Kettenregel ist $f'(x)=2e^{2x}$ und $f''(x)=4e^{2x}$.}
	    \lang{en}{Let $f(x)=e^{2x}$. By the chain rule, $f'(x)=2e^{2x}$ and $f''(x)=4e^{2x}$.}
	\item 
	    \lang{de}{Sei $f(x)=\ln(x)$. Dann ist $f'(x)=\frac{1}{x}$ und $f''(x)=-\frac{1}{x^2}$.}
	    \lang{en}{Let $f(x)=\ln(x)$. Then $f'(x)=\frac{1}{x}$ and $f''(x)=-\frac{1}{x^2}$.}
	\end{enumerate}
\end{example}



\begin{definition}\label{def:higher_deriv}
\label{higher derivative}
\lang{de}{
Wie auch die zweite Ableitung werden die dritte, vierte, $\ldots$, $n$-te Ableitung von $f$ als 
Ableitungen der zweiten, dritten, $\ldots$, $(n-1)$-te Ableitung von $f$ definiert, sofern letztere 
existieren und differenzierbar sind. 
\\\\
Man sagt dann, dass $f$ \notion{dreimal, viermal,  $\ldots$, $n$-mal differenzierbar} ist. Die $n$-te 
Ableitungsfunktion von $f$ wird auch mit $f^{(n)}$ bezeichnet, um zu viele Striche zu vermeiden, aber 
auch für kleine $n$ verwendet, also z.B.
}
\lang{en}{
Much like the second derivative, we may define the third, fourth, $\ldots$, $n$th derivative of $f$ 
as the derivative of the second, third, $\ldots$, $(n-1)$th derivative of $f$ respectively, so long 
as the latter exists and is differentiable.
}
\[   f^{(1)}=f',\quad f^{(2)}=f'',\quad f^{(3)}=f'''. \]
\lang{de}{
Existiert die $n$-te Ableitungsfunktion von $f$ und ist diese stetig, so sagt man, dass $f$ 
\notion{$n$-mal stetig differenzierbar} ist. Wir setzen ausserdem $f^{(0)} = f$.
}
\lang{en}{
If the $n$th derivative of $f$ exists and is continuous, then we say that $f$ is \notion{$n$ times 
continuously differentiable}. We set $f^{(0)} = f$ by convention.
}
\end{definition}

\begin{remark}
\lang{de}{
Damit eine Funktion $n$-mal differenzierbar sein kann, muss insbesondere eine $(n-1)$-te 
Ableitungsfunktion existieren und diese muss differenzierbar sein, also insbesondere stetig. $n$-mal 
differenzierbare Funktionen sind also insbesondere $(n-1)$-mal stetig differenzierbar.
}
\lang{en}{
For a function to be $n$ times differentiable, the $(n-1)$th derivative must exist and be 
differentiable, and hence continuous. $n$ times differentiable functions are therefore in particular 
all $(n-1)$ times continuously differentiable.
}
\end{remark}



 \begin{example}%\textit{Beispiel:}\\
 \begin{tabs*}[\initialtab{0}]
\tab{$x^3 + 2x^2 + 5$}
\lang{de}{Für $f(x)=x^3+2x^2+5$ hatten wir schon $f'(x)=3x^2+4x$ und $f''(x)=6x+4$. Dann sind}
\lang{en}{We have already seen that if $f(x)=x^3+2x^2+5$ then $f'(x)=3x^2+4x$ and $f''(x)=6x+4$. Then}
\[ f^{(3)}(x)=(f'')'(x)=6\quad \text{\lang{de}{und}\lang{en}{and}}\quad f^{(4)}(x)=(f^{(3)})'(x)=0. \]
\lang{de}{Alle höheren Ableitungen $f^{(n)}(x)$ mit $n>4$ sind dann aber auch konstant $0$.}
\lang{en}{All higher derivatives $f^{(n)}(x)$ for $n>4$ are therefore constant and equal to $0$.}

\tab{\lang{de}{Polynom vom Grad $k$}\lang{en}{Polynomial of degree $k$}}
\lang{de}{
Allgemeiner ist die Ableitung eines Polynoms vom Grad $k$ ein Polynom vom Grad $k-1$, die zweite 
Ableitung vom Grad $k-2$ etc., bis schließlich die $k$-te Ableitung konstant ist. Alle höheren 
Ableitungen sind dann die Nullfunktion.
}
\lang{en}{
More generally, the derivative of a polynomial of degree $k$ is itself a polynomial of degree $k-1$, 
the second derivative is a polynomial of degree $k-2$ etc., until finally the $k$th derivative is 
constant. All higher derivatives are simply the zero function.
}

\tab{$\sin(x)$}
\lang{de}{
Für $f(x)=\sin(x)$ hatten wir $f'(x)=\cos(x)$ und  $f''(x)=-\sin(x)=-f(x)$. Damit sind
}
\lang{en}{
We have already seen that if $f(x)=\sin(x)$ then $f'(x)=\cos(x)$ and $f''(x)=-\sin(x)=-f(x)$. Then
}
\[ f^{(3)}(x)=(f'')'(x)=-f'(x)=-\cos(x) \quad \text{\lang{de}{und}\lang{en}{and}}\quad f^{(4)}(x)=(f^{(3)})'(x)=-f''(x)=f(x). \]
\lang{de}{
Die vierte Ableitung ist also die Funktion selbst, weshalb wir generell für $k\geq 0$ die Beziehung
}
\lang{en}{
The fourth derivative is therefore the function itself, so in general for $k\geq 0$ there is a 
relationship
}
\[  f^{(k+4)}=f^{(k)} \]
\lang{de}{erhalten, d.h. für alle $n\geq 0$ gilt:}
\lang{en}{so we have for any $n\geq 0$:}
\begin{eqnarray*}
f^{(4n)}(x) = \; \; \; \; \;  f^{(0)}(x) &=& \sin(x), \\
f^{(4n+1)}(x) &=& \cos(x), \\
f^{(4n+2)}(x) &=& -\sin(x), \\
f^{(4n+3)}(x) &=& -\cos(x). \\
\end{eqnarray*}

\tab{$e^{2x}$}
\lang{de}{
Für $f(x)=e^{2x}$ sind $f'(x)=2e^{2x}$ und $f''(x)=4e^{2x}$. Man sieht hier schon ein Muster für die 
höheren Ableitungen, nämlich
}
\lang{en}{
We have already seen that if $f(x)=e^{2x}$ then $f'(x)=2e^{2x}$ and $f''(x)=4e^{2x}$. Here too we see 
a pattern for the higher derivatives,
}
\[  f^{(n)}(x)=2^n\cdot e^{2x}, \]
\lang{de}{
das man leicht mit vollständiger Induktion beweisen kann:\\
Der Induktionsanfang $n=0$ ist richtig, denn $f^{(0)}(x)=f(x)=2^0e^{2x} = e^{2x}$.\\
Ist nun $f^{(n)}(x)=2^n\cdot e^{2x}$ für ein $n$, so ist
}
\lang{en}{
which can be easily proved by induction:\\
The base case $n=0$ is true, as $f^{(0)}(x)=f(x)=2^0e^{2x} = e^{2x}$.\\
Now if we assume $f^{(n)}(x)=2^n\cdot e^{2x}$ to be true, then
}
\[  f^{(n+1)}(x)=(f^{(n)})'(x)=(2^n\cdot e^{2x})'=2^n\cdot 2e^{2x}=2^{n+1}\cdot e^{2x}. \]


\end{tabs*}
\end{example}





   \begin{quickcheck}
		\field{rational}
		\type{input.function}
		\begin{variables}
			\randint[Z]{a}{-3}{5}
		    \function[normalize]{f}{(x-a)/(x+a)}
            \function[normalize]{df}{(2*a)/(x+a)^2}
            \derivative[normalize]{ddf}{df}{x}
            \derivative[normalize]{df3}{ddf}{x}
		\end{variables}
		
		\text{\lang{de}{Die höheren Ableitungen der Funktion $f(x)=\var{f}$ sind}
        \lang{en}{The higher derivatives of the function $f(x)=\var{f}$ are}\\
        $f'(x)= $\ansref, \\ $f''(x)= $\ansref, \\ $f'''(x)= $\ansref.}
		
		\begin{answer}
			\solution{df}
			\checkAsFunction{x}{-1}{1}{10}
		\end{answer}
		\begin{answer}
			\solution{ddf}
			\checkAsFunction{x}{-1}{1}{10}
		\end{answer}
		\begin{answer}
			\solution{df3}
			\checkAsFunction{x}{-1}{1}{10}
		\end{answer}
		\explanation{\lang{de}{
        Die nächst höhere Ableitung ist stets die Ableitung der vorhergehenden 
        höheren Ableitung.
        }
        \lang{en}{
        A higher derivative can be found by differentiating the previous higher derivative. 
        }}
	\end{quickcheck}
    
\lang{de}{
Das folgende Video behandelt noch einmal höhere Ableitungen.
    \floatright{\href{https://api.stream24.net/vod/getVideo.php?id=10962-2-10764&mode=iframe&speed=true}{\image[75]{00_video_button_schwarz-blau}}}\\
}
\lang{en}{}    

\end{content}