%$Id:  $
\documentclass{mumie.article}
%$Id$
\begin{metainfo}
  \name{
    \lang{de}{Überblick: Differenzierbarkeit}
    \lang{en}{Overview: Differentiability}
  }
  \begin{description} 
 This work is licensed under the Creative Commons License Attribution 4.0 International (CC-BY 4.0)   
 https://creativecommons.org/licenses/by/4.0/legalcode 

    \lang{de}{Beschreibung}
    \lang{en}{Description}
  \end{description}
  \begin{components}
  \end{components}
  \begin{links}
\link{generic_article}{content/rwth/HM1/T106_Differentialrechnung/g_art_T106_overview.meta.xml}{T106_overview}
\link{generic_article}{content/rwth/HM1/T301_Differenzierbarkeit/g_art_content_03_hoehere_ableitungen.meta.xml}{content_03_hoehere_ableitungen}
\link{generic_article}{content/rwth/HM1/T301_Differenzierbarkeit/g_art_content_02_ableitungsregeln.meta.xml}{content_02_ableitungsregeln}
\link{generic_article}{content/rwth/HM1/T301_Differenzierbarkeit/g_art_content_01_differenzenquotient.meta.xml}{content_01_differenzenquotient}
\end{links}
  \creategeneric
\end{metainfo}
\begin{content}
\begin{block}[annotation]
	Im Ticket-System: \href{https://team.mumie.net/issues/30124}{Ticket 30124}
\end{block}




\begin{block}[annotation]
Im Entstehen: Überblicksseite für Kapitel  Differenzierbarkeit
\end{block}

\usepackage{mumie.ombplus}
\ombchapter{1}
\title{\lang{de}{Überblick: Differenzierbarkeit}\lang{en}{Overview: Differentiability}}




\begin{block}[info-box]
\lang{de}{\strong{Inhalt}}
\lang{en}{\strong{Contents}}


\lang{de}{
    \begin{enumerate}%[arabic chapter-overview]
   \item[1.1] \link{content_01_differenzenquotient}{Differenzierbarkeit}
   \item[1.2] \link{content_02_ableitungsregeln}{Ableitungsregeln}
   \item[1.3] \link{content_03_hoehere_ableitungen}{Eigenschaften differenzierbarer Funktionen und höhere Ableitungen}
    \end{enumerate}
}
\lang{en}{
    \begin{enumerate}%[arabic chapter-overview]
   \item[1.1] \link{content_01_differenzenquotient}{Differentiability}
   \item[1.2] \link{content_02_ableitungsregeln}{Differentiation rules}
   \item[1.3] \link{content_03_hoehere_ableitungen}{Properties of differentiable functions and higher derivatives}
    \end{enumerate}
} %lang

\end{block}

\begin{zusammenfassung}

\lang{de}{
Bereits in \link{T106_overview}{Teil 1 Kapitel 6} haben Sie einen elementaren Begriff von 
Differentiation kennengelernt und Ableitungsregeln angewendet.
In diesem Kapitel geht es  darum, ein tieferes Verständnis der Differenzierbarkeit zu entwickeln.
Inzwischen verfügen wir mit dem Grenzwertbegriff über das nötige Werkzeug dafür. Wir können 
Funktionen nicht nur auf Differenzierbarkeit testen und Regeln anwenden, sondern diese auch begründen!
Weiter werden Funktionen untersucht, die nicht (überall) differenzierbar sind.\\
Wichtige Eigenschaften differenzierbarer Funktionen wie deren Stetigkeit und ihr Monotonieverhalten 
werden nun nicht nur postuliert sondern auch untermauert. Wir beweisen den für Anwendungen wichtigen 
Mittelwertsatz.\\
Ableitungsfunktionen können selbst wieder sehr gute Eigenschaften haben, was zu Begriffen wie stetige 
Differenzierbarkeit und mehrfache Differenzierbarkeit führt.\\
Höhere Ableitungen beliebiger Ordnung kann man oft systematisch bestimmen und durch vollständige 
Induktion beweisen.
}
\lang{en}{
Differentiation was first introduced in the \link{T106_overview}{sixth chapter} of this course, 
alongside some of the rules associated with it. 
In this chapter we deepen our understanding of differentiability. To do this, we give a more rigorous 
definition using the notion of limits, which is called differentiation from first principles. No 
longer can we only test functions for differentiability and apply the rules of differentiation, we 
now have the tools for proving them. We consider some functions that are not (everywhere) 
differentiable.\\
Important properties of differentiable functions such as their continuity and the relationship 
between their derivatives and monotonicity can now be justified rigorously. An important statement 
to this end is the mean value theorem, which we prove.\\
Derivative functions themselves can have nice properties, leading to definitions for continuous 
differentiability and higher derivatives.\\
Higher derivatives can often be systematically determined, and we give an example where such a 
formula can be proved by induction.
}


\end{zusammenfassung}

\begin{block}[info]\lang{de}{\strong{Lernziele}}\lang{en}{\strong{Learning Goals}} 
\begin{itemize}[square]
\item \lang{de}{
      Sie wenden die Ableitungsregeln für Summen, Produkte, Quotienten und Verkettungen auch in 
      schwierigeren Situationen sicher an.
      }
      \lang{en}{
      Being able to apply the sum, product and chain rules for differentiation even in more difficult 
      situations.
      }
\item \lang{de}{
      Sie kennen die Ableitungen vieler \glqq elementarer\grqq{} Funktionen und kennen Funktionen, 
      die nicht differenzierbar sind.
      }
      \lang{en}{
      Knowing the derivatives of some 'elementary' functions and knowing some functions that are 
      not differentiable.
      }
\item \lang{de}{
      Sie untersuchen Funktionen mit Hilfe des Differenzenquotienten auf Differenzierbarkeit.
      }
      \lang{en}{
      Being able to investigate the differentiability of a function using differentiation from first 
      principles.
      }
\item \lang{de}{
      Sie wissen um wichtige Eigenschaften differenzierbarer Funktionen und Folgerungen daraus wie 
      Stetigkeit, lokale Extrema, Monotonieverhalten und den Mittelwertsatz, die Sie in konkreten 
      Situationen erkennen und anwenden.
      }
      \lang{en}{
      Knowing important properties of differentiable functions and their conditions, such as 
      continuity, local extrema, monotonicity and the mean value theorem, and being able to apply 
      these in examples.
      }
\item \lang{de}{
      Sie kennen den Begriff der stetigen Differenzierbarkeit und dessen Unterschied zu Begriff der 
      Differenzierbarkeit.
      }
      \lang{en}{
      Knowing the definition of a continuously differentiable function, and how this differs from 
      the definition of a differentiable function.
      }
\item \lang{de}{
      Sie kennen den Begriff der höheren Ableitung und deren Interpretation. Sie testen Funktionen 
      auf mehrfache Differenzierbarkeit.
      }
      \lang{en}{
      Knowing the definition of the higher derivatives of a function and its interpretation. Being 
      able to test functions for the existence of a higher derivative.
      }
\item \lang{de}{Sie berechnen höhere Ableitungen und untersuchen sie auf Systematik.}
      \lang{en}{Being able to calculate higher derivatives and systematic patterns between them.}
\end{itemize}
\end{block}




\end{content}
