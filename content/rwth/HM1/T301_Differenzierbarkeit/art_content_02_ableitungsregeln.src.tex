%$Id:  $
\documentclass{mumie.article}
%$Id$
\begin{metainfo}
  \name{
    \lang{de}{Ableitungsregeln}
    \lang{en}{Differentiation rules}
  }
  \begin{description} 
 This work is licensed under the Creative Commons License Attribution 4.0 International (CC-BY 4.0)   
 https://creativecommons.org/licenses/by/4.0/legalcode 

    \lang{de}{Beschreibung}
    \lang{en}{Description}
  \end{description}
  \begin{components}
    \component{generic_image}{content/rwth/HM1/images/g_img_00_Videobutton_schwarz.meta.xml}{00_Videobutton_schwarz}
    \component{js_lib}{system/media/mathlets/GWTGenericVisualization.meta.xml}{mathlet1}
    \component{generic_image}{content/rwth/HM1/images/g_img_00_video_button_schwarz-blau.meta.xml}{00_video_button_schwarz-blau}
  \end{components}
  \begin{links}
    \link{generic_article}{content/rwth/HM1/T301_Differenzierbarkeit/g_art_content_02_ableitungsregeln.meta.xml}{content_02_ableitungsregeln}
    \link{generic_article}{content/rwth/HM1/T301_Differenzierbarkeit/g_art_content_01_differenzenquotient.meta.xml}{diffbarkeit}
    \link{generic_article}{content/rwth/HM1/T210_Stetigkeit/g_art_content_31_grenzwerte_von_funktionen.meta.xml}{grenzw-funk}
    \link{generic_article}{content/rwth/HM1/T204_Abbildungen_und_Funktionen/g_art_content_11_injektiv_surjektiv_bijektiv.meta.xml}{umkehrfkt}
  \end{links}
  \creategeneric
\end{metainfo}
\begin{content}
\usepackage{mumie.ombplus}
\ombchapter{1}
\ombarticle{2}
\usepackage{mumie.genericvisualization}

\begin{visualizationwrapper}

\title{\lang{de}{Ableitungsregeln und Ableitung der Umkehrfunktion und wichtige Formeln}
       \lang{en}{Differentiation rules}}
 
\begin{block}[annotation]
  
\end{block}
\begin{block}[annotation]
  Im Ticket-System: \href{http://team.mumie.net/issues/10032}{Ticket 10032}\\
\end{block}

\begin{block}[info-box]
\tableofcontents
\end{block}

\section{\lang{de}{Ableitungsregeln für zusammengesetzte Funktionen}
         \lang{en}{Derivative of sums and products of functions}}
% Kopie aus Teil1, content_20 + Begründungen + Quotientenregel

\lang{de}{
In diesem Paragraph beschäftigen wir uns mit den Ableitungen von Summen, Differenzen, Produkten und Quotienten differenzierbarer Funktionen.
}
\lang{en}{
In this paragraph we consider the derivatives of sums, differences, products and quotients of 
differentiable functions.
}

\begin{rule}[\lang{de}{Summenregel und Faktorregel}
             \lang{en}{Sum and constant factor rules}] \label{rule:summenregel}
\lang{de}{
Sind $f$ und $g$ zwei auf einer Teilmenge $D\subseteq \R$ differenzierbare Funktionen und $c\in \R$, so sind auch die Funktionen
$f+g$, $f-g$ und $cf$ auf $D$ differenzierbar und es gelten für alle $x\in D$:
}
\lang{en}{
Let $f$ and $g$ be two functions differentiable on a subset $D\subseteq \R$, and let $c\in \R$. Then 
the functions $f+g$, $f-g$ and $cf$ are also differentiable on $D$, and for all $x\in D$:
}
\begin{align*}
(f+g)'(x) &= f'(x) + g'(x)  &\quad & \text{\lang{de}{Summenregel}\lang{en}{Sum rule}} \\
(f-g)'(x) &= f'(x) - g'(x) && \\
(cf)'(x) &= c\cdot  f'(x)  &\quad & \text{\lang{de}{Faktorregel}\lang{en}{Constant factor rule}}
\end{align*}
\lang{de}{
Induktiv erhält man dadurch auch, dass jede endliche Summe von differenzierbaren Funktionen $f_1, f_2, \ldots , f_n$ wieder differenzierbar ist
und die Ableitungsfunktion summandenweise berechnet werden kann.\\
\floatright{\href{https://www.hm-kompakt.de/video?watch=510}{\image[75]{00_Videobutton_schwarz}}}\\\\
}
\lang{en}{
By induction we get that every finite sum of differentiable functions $f_1, f_2, \ldots , f_n$ is 
itself differentiable, and that its derivative can be calculated as the sum of the summands' 
derivatives.
}
\end{rule}

\begin{proof*}[\lang{de}{Beweis der Summen- und Faktorregel}
               \lang{en}{Proof of the sum and constant factor rules}]
\begin{incremental}[\initialsteps{0}]
\step
\lang{de}{Sind $f$ und $g$ bei $x_0$ differenzierbar, so existieren die Grenzwerte}
\lang{en}{Let $f$ and $g$ be differentiable at $x_0$, so the following limits exist:}
\[ f'(x_0)=\lim_{h\rightarrow 0} \frac{f(x_0+h)-f(x_0)}{h}\quad 
\text{\lang{de}{und}\lang{en}{and}}\quad 
g'(x_0)=\lim_{h\rightarrow 0} \frac{g(x_0+h)-g(x_0)}{h}. \]
\step
\lang{de}{
Mit den \ref[grenzw-funk][Grenzwertregeln]{sec:grenzwertregeln} erhält man dann für die Ableitung von $f+g$:
}
\lang{en}{
Using the \ref[grenzw-funk][rules for limits]{sec:grenzwertregeln} we obtain the derivative of $f+g$:
}
\begin{eqnarray*}
(f+g)'(x_0) &=& \lim_{h\rightarrow 0} \frac{(f+g)(x_0+h)-(f+g)(x_0)}{h} \\
&=& \lim_{h\rightarrow 0}  \frac{f(x_0+h)+g(x_0+h)-f(x_0)-g(x_0)}{h} \\
&=& \lim_{h\rightarrow 0} \frac{f(x_0+h)-f(x_0)}{h} + \lim_{h\rightarrow 0} \frac{g(x_0+h)-g(x_0)}{h} \\
&=& f'(x_0)+ g'(x_0).
\end{eqnarray*}

\step
\lang{de}{
Ebenfalls mit den \ref[grenzw-funk][Grenzwertregeln]{sec:grenzwertregeln} erhält man die Ableitung von $cf$:
}
\lang{en}{
Likewise by using the \ref[grenzw-funk][rules for limits]{sec:grenzwertregeln} we obtain the 
derivative of $cf$:
}
\begin{eqnarray*}
(cf)'(x_0) &=&  \lim_{h\rightarrow 0}\frac{(cf)(x_0+h)-(cf)(x_0)}{h} \\
&=& \lim_{h\rightarrow 0}  \frac{c\cdot f(x_0+h)-c\cdot f(x_0)}{h} \\
&=& c\cdot \lim_{h\rightarrow 0} \frac{f(x_0+h)-f(x_0)}{h}= c\cdot f'(x_0).
\end{eqnarray*}

\step
\lang{de}{
Das Resultat f\"ur $f-g$ kann man \"ahnlich zeigen. Man erh\"alt es aber auch wie folgt schrittweise aus der Summenregel 
und der Faktorregel, die man auf die geeigneten Funktionen anwendet. 
Deshalb wird die Regel f\"ur Differenzen in der Literatur meist weggelassen.\\
Nach der Faktorregel, angewandt auf $g$ und $c=-1$, ist n\"amlich mit $g$ auch $-g$ differenzierbar und $(-g)'(x)=-g'(x)$. Nach der Summenregel, 
angewandt auf $f$ und $-g$, ist dann $f-g=f+(-g)$ differenzierbar und
}
\lang{en}{
The result for $f-g$ can be shown similarly. We may also obrain it as follows by applying the sum and 
the constant factor rules to the appropriate functions. For this reason, this result is often not 
mentioned as its own case in the literature.\\ 
By the constant factor rule applied to $g$ and $c=-1$, $-g$ is differentiable with $(-g)'(x)=-g'(x)$. 
The sum rule applied to $f$ and $-g$ then gives us the differentiability of $f-g=f+(-g)$, with
}
\[ (f-g)'(x)= f'(x)+ (-g)'(x)=f'(x) - g'(x). \]

\end{incremental}
\end{proof*}



\begin{example}%\textit{Beispiel:}\\
\lang{de}{
Wir berechnen die Ableitung der Funktion $f(x)=x^3+2x+1$. Diese Funktion ist die Summe der 
Funktionen $x^3$ und $2x$ und der konstanten Funktion $1$. Nach der Faktorregel und dem \ref[diffbarkeit][Beispiel im vorigen Abschnitt]{sec:abl-elem-funk}
ist die Ableitung der Funktion $g(x)=2x$ gegeben durch
}
\lang{en}{
We calculate the derivative of the function $f(x)=x^3+2x+1$. This function is the sum of the 
functions $x^3$ and $2x$ and the constant function $1$. By the constant factor rule and the 
\ref[diffbarkeit][example in the previous section]{sec:abl-elem-funk}, the derivative of the function 
$g(x)=2x$ is given by
}
\[  g'(x)=2\cdot 1=2. \]
\lang{de}{Anschließend kann die Summenregel angewandt werden:}
\lang{en}{Finally we apply the sum rule:}
\[f'(x)=3x^2+2+0=3x^2+2.\]
\end{example}

\begin{tabs*}[\initialtab{0}] \label{tab:potenzreihen-ableitung}
\tab{\lang{de}{Ableitung von Potenzreihen}\lang{en}{Derivative of power series}}
\lang{de}{
Auch für Funktionen, die durch Potenzreihen gegeben sind, ist die Ableitung durch
summandenweises ableiten gegeben. Dies ist aus der obigen Regel zwar nicht direkt ersichtlich, weil 
die Regel zur summandenweisen Ableitung allgemein nur für Summen aus endlich vielen Funktionen gilt, 
soll aber hier ohne Begründung angenommen werden. Formal lautet die Aussage für Potenzreihen also:
}
\lang{en}{
Functions given by power series can also be differentiated term-by-term. This does not in fact follow 
directly from the above rule, as the rule only applies to the derivative of finite sums, but we may 
use this fact here without dealing with the proof. Formally, the statement for power series is as 
follows.
}



\begin{theorem}\label{thm:potenzreihen-ableitung}
\lang{de}{
Seien $x_0\in \R$, $f(x)= \sum_{n=0}^\infty c_n(x-x_0)^n$, $r> 0$ der Konvergenzradius von $f$ und 
$I=(x_0-r;x_0+r)$.  Dann ist $f$ auf $I$ differenzierbar und die Ableitung von $f$ ist auf 
$I$ durch summandenweises Ableiten gegeben, d.h.
}
\lang{en}{
Let $x_0\in \R$, $f(x)= \sum_{n=0}^\infty c_n(x-x_0)^n$. Let $r> 0$ be the convergence radius of $f$ 
and let $I=(x_0-r;x_0+r)$. Then $f$ is differentiable on $I$ and the derivative of $f$ is given by 
termwise differentiation on $I$, that is,
}
\[  f'(x)= \sum_{n=1}^\infty n\cdot c_n(x-x_0)^{n-1}\quad 
\text{\lang{de}{für alle}\lang{en}{for all} }x\in I.  \]
\lang{de}{$f'$ konvergiert auf $I$ als Potzenzreihe absolut.}
\lang{en}{The power series $f'$ converges absolutely on $I$.}
\end{theorem}
\end{tabs*}

\begin{quickcheckcontainer}
\randomquickcheckpool{1}{2}
\begin{quickcheck}
		\field{rational}
		\type{input.function}
		\begin{variables}
			\randint[Z]{n}{-9}{9}
			\randint[Z]{a}{2}{9}
			\randint[Z]{b}{-6}{9}
		    \function[normalize]{f}{a*sin(x)+b*x^n}
			\function[normalize]{ff}{a*cos(x)+n*b*x^(n-1)}
            \function[normalize]{g}{n*x^(n-1)}
		\end{variables}
		
		\text{\lang{de}{Die Ableitungsfunktion der Funktion $f(x)=\var{f}$ ist $f'(x)= $\ansref.}
          \lang{en}{The derivative function of $f(x)=\var{f}$ is $f'(x)= $\ansref.}}
		
		\begin{answer}
			\solution{ff}
			\checkAsFunction{x}{-1}{1}{10}
		\end{answer}
		\explanation{\lang{de}{
    Die Ableitungsfunktion von $\sin(x)$ ist $\cos(x)$ und die von $x^{\var{n}}$ 
    ist $\var{g}$. Dann muss man die Summen- und Faktorregel anwenden.
    }
		\lang{en}{
    The derivative function of $\sin(x)$ is $\cos(x)$ and the derivative of $x^{\var{n}}$ 
    is $\var{g}$. We then simply apply the sum and constant factor rules.
    }}
	\end{quickcheck}
    
   \begin{quickcheck}
		\field{rational}
		\type{input.function}
		\begin{variables}
			\randint[Z]{n}{-9}{9}
			\randint[Z]{a}{2}{9}
			\randint[Z]{b}{-6}{9}
		    \function[normalize]{f}{a*cos(x)+b*x^n}
			\function[normalize]{ff}{-a*sin(x)+n*b*x^(n-1)}
            \function[normalize]{g}{n*x^(n-1)}
		\end{variables}
		
		\text{\lang{de}{Die Ableitungsfunktion der Funktion $f(x)=\var{f}$ ist $f'(x)= $\ansref.}
          \lang{en}{The derivative function of $f(x)=\var{f}$ is $f'(x)= $\ansref.}}
		
		\begin{answer}
			\solution{ff}
			\checkAsFunction{x}{-1}{1}{10}
		\end{answer}
		\explanation{\lang{de}{
    Die Ableitungsfunktion von $\cos(x)$ ist $-\sin(x)$ und die von $x^{\var{n}}$ 
    ist $\var{g}$. Dann muss man die Summen- und Faktorregel anwenden.
    }
    \lang{en}{
    The derivative function of $\cos(x)$ is $-\sin(x)$ and the derivative of $x^{\var{n}}$ 
    is $\var{g}$. We then simply apply the sum and constant factor rules.
    }}
	\end{quickcheck}
    
    
\end{quickcheckcontainer} 

\lang{de}{
Wie Produkte und Quotienten differenzierbarer Funktionen abgeleitet werden, beschreiben die Produktregel und die Quotientenregel.
}
\lang{en}{
The derivatives of the product and quotient of differentiable functions are given by the product and 
quotient rules respectively.
}
\begin{rule}[\lang{de}{Produktregel und Quotientenregel}
             \lang{en}{Product and quotient rules}]\label{rule:produkt_quotient_regel}
\lang{de}{
Sind $f$ und $g$ zwei auf einer Teilmenge $D\subseteq \R$ differenzierbare Funktionen, so ist auch deren Produkt $f\cdot g$ auf $D$ differenzierbar
und f\"ur alle $x\in D$ gilt:
}
\lang{en}{
Let $f$ and $g$ be two functions differentiable on a subset $D\subseteq \R$. Then their product 
$f\cdot g$ auf $D$ is also differentiable on $D$, and for all $x\in D$:
}
\[(f\cdot g)'(x)=f'(x)\cdot g(x)+f(x)\cdot g'(x).\]
\lang{de}{
Des Weiteren ist auch ihr Quotient $\frac{f}{g}$ auf der Menge $\{x\in D\,|\, g(x)\ne 0\}$ differenzierbar  
und f\"ur alle $x\in D$ gilt:
}
\lang{en}{
Furthermore, their quotient $\frac{f}{g}$ is differentiable on the set $\{x\in D\,|\, g(x)\ne 0\}$ 
and for all $x\in D$:
}
\[ \left(\frac{f}{g}\right)'(x)= \frac{f'(x)\cdot g(x)-f(x)\cdot g'(x)}{g(x)^2}. \]
\lang{de}{
\floatright{\href{https://www.hm-kompakt.de/video?watch=511}{\image[75]{00_Videobutton_schwarz}}\href{https://www.hm-kompakt.de/video?watch=512}{\image[75]{00_Videobutton_schwarz}}}\\
\\
}
\lang{en}{}
\end{rule}
 
 
 
 
 
 
\begin{proof*}[\lang{de}{Beweis der Produkt- und Quotientenregel}
               \lang{en}{Proof of the product and quotient rules}]
\begin{incremental}[\initialsteps{0}]
\step \lang{de}{
F\"ur das Produkt berechnet man mit Hilfe der \ref[grenzw-funk][Grenzwertregeln]{sec:grenzwertregeln}:
}
\lang{en}{
For the product we use the \ref[grenzw-funk][rules for limits]{sec:grenzwertregeln} to calculate the 
following:
}
\begin{eqnarray*}
(f\cdot g)'(x)&=&\lim_{h\rightarrow 0}\frac{(f\cdot g)(x+h)-(f\cdot g)(x)}{h}\\
&=&\lim_{h\rightarrow 0}\frac{f(x+h)\cdot g(x+h)-f(x)\cdot g(x)}{h}\\
&=&\lim_{h\rightarrow 0}\frac{f(x+h)\cdot g(x+h)\textcolor{#CC6600}{-f(x)\cdot g(x+h)+f(x)\cdot g(x+h)}-f(x)\cdot g(x)}{h}\\
&=&\lim_{h\rightarrow 0}\frac{(f(x+h)-f(x))\cdot g(x+h)+f(x) \cdot (g(x+h) -g(x))}{h}\\
&=&\lim_{h\rightarrow 0} \bigg(\frac{f(x+h)- f(x)}{h}\cdot g(x+h)+f(x)\cdot \frac{g(x+h) -g(x)}{h}\bigg)\\
&=&
f'(x)\cdot g(x)+f(x)\cdot g'(x).
\end{eqnarray*} 

\step
\lang{de}{F\"ur die Quotientenregel berechnet man ganz \"ahnlich wie oben:}
\lang{en}{For the quotient rule we calculate, in a similar fashion to above,}
\begin{eqnarray*}
\left(\frac{f}{g}\right)(x+h) - \left(\frac{f}{g}\right)(x) &=& \frac{f(x+h)}{g(x+h)}-\frac{f(x)}{g(x)} \\
&=& \frac{f(x+h)g(x)-f(x)g(x+h)}{g(x+h)g(x)} \\
&=& \frac{f(x+h)g(x) \textcolor{#CC6600}{-f(x)g(x) +f(x)g(x)}-f(x)g(x+h)}{g(x+h)g(x)} \\
&=& \frac{(f(x+h)-f(x))\cdot g(x)-f(x) \cdot (g(x+h) -g(x))}{g(x+h)g(x)} 
\end{eqnarray*} 
\lang{de}{Daher gilt}
\lang{en}{Hence}
\begin{eqnarray*}
\left(\frac{f}{g}\right)'(x)&=&\lim_{h\rightarrow 0} \frac{1}{h} \bigg( \left(\frac{f}{g}\right)(x+h) - \left(\frac{f}{g}\right)(x)\bigg) \\
&=&\lim_{h\rightarrow 0} \frac{(f(x+h)-f(x))\cdot g(x)-f(x) \cdot (g(x+h) -g(x))}{h\cdot g(x+h)g(x)}\\ 
&=&\lim_{h\rightarrow 0}\frac{1}{g(x+h)g(x)} \frac{(f(x+h)-f(x))\cdot g(x)-f(x) \cdot (g(x+h) -g(x))}{h}\\ 
%&=&\lim_{h\rightarrow 0}\frac{f(x+h)\cdot g(x+h)-f(x)\cdot g(x+h)+f(x)\cdot g(x+h)-f(x)\cdot g(x)}{h}\\
%&=&\lim_{h\rightarrow 0}\frac{(f(x+h)-f(x))\cdot g(x+h)+f(x) \cdot (g(x+h) -g(x))}{h}\\
&=&\lim_{h\rightarrow 0}\frac{1}{g(x+h)g(x)} \cdot  \bigg(\frac{f(x+h)- f(x)}{h}\cdot g(x)-f(x)\cdot \frac{g(x+h) -g(x)}{h}\bigg)\\
&=& \frac{f'(x)\cdot g(x)-f(x)\cdot g'(x)}{g(x)^2}
\end{eqnarray*}

\end{incremental}
\end{proof*}


 \begin{example}%\textit{Beispiel:}\\
 \begin{tabs*}[\initialtab{0}]
\tab{$x^2\cdot \sin(x)$}
\lang{de}{
Wir berechnen als erstes Beispiel die Ableitung der Funktion $F(x)={x^2}\cdot \sin (x)$. Diese ist 
das Produkt $F(x)=f(x)\cdot g(x)$ der differenzierbaren Funktionen 
$f(x)={x^2}$ und $g(x)=\sin (x)$. Die Ableitung von $f(x)$ ist $2x$, die Ableitung von $g(x)$ ist $\cos (x)$. Dann ist
}
\lang{en}{
As a first example we calculate the derivative of the function $F(x)={x^2}\cdot \sin (x)$. This is 
the product $F(x)=f(x)\cdot g(x)$ of the differentiable functions $f(x)={x^2}$ and $g(x)=\sin (x)$. 
The derivative of $f(x)$ is $2x$ and the derivative of $g(x)$ is $\cos (x)$. Hence
}
\[F'(x)=f'(x)\cdot g(x)+f(x)\cdot g'(x)=2x\cdot\sin (x)+{x^2}\cdot \cos (x).\]
\tab{\lang{de}{Tangens $\tan(x)$ und Kotangens $\cot(x)$}
     \lang{en}{Tangent $\tan(x)$ and cotangent $\cot(x)$}}
\lang{de}{
Die Tangensfunktion $\tan(x)$ ist der Quotient von Sinus und Kosinus. Da Sinus und Kosinus auf ganz $\R$ differenzierbar sind, ist also der Tangens
auf seiner ganzen Definitionsmenge differenzierbar. Setzt man $f(x)=\sin(x)$ und $g(x)=\cos(x)$, so kann man
mit den Ableitungen $f'(x)=\sin'(x)=\cos(x)$ und $g'(x)=\cos'(x)=-\sin(x)$ und der Quotientenregel, 
die Ableitung des Tangens bestimmen:
}
\lang{en}{
The tangent function $\tan(x)$ is the quotient of the sine and cosine functions. As sine and cosine 
are both differentiable on all of $\R$, the tangent function is differentiable on its whole domain. 
If we set $f(x)=\sin(x)$ and $g(x)=\cos(x)$, we may use the derivatives $f'(x)=\sin'(x)=\cos(x)$ and 
$g'(x)=\cos'(x)=-\sin(x)$, together with the quotient rule, to determine the derivative of the 
tangent function:
}
\begin{eqnarray*}
\tan'(x) &=& \left( \frac{\sin}{\cos}\right)'(x) = \frac{\sin'(x)\cos(x)-\sin(x)\cos'(x)}{\cos(x)^2} \\
&=& \frac{\cos(x)\cos(x)-\sin(x)\cdot (-\sin(x))}{\cos(x)^2} = \frac{\cos(x)^2+\sin(x)^2}{\cos(x)^2}.
\end{eqnarray*}
\lang{de}{
Letzteres kann man auf zwei Arten vereinfachen. Zum einen kann man den trigonometrischen Pythagoras $\cos(x)^2+\sin(x)^2=1$ anwenden, und erh\"alt
}
\lang{en}{
Finally we may simplify this in two ways. One way is to apply the Pythagorean theorem 
$\cos(x)^2+\sin(x)^2=1$ and obtain
}
\[ \tan'(x)  =\frac{1}{\cos(x)^2}. \]
\lang{de}{Zum anderen kann man aber auch den Bruch aufteilen und erh\"alt}
\lang{en}{Otherwise we can also seperate the resulting fraction into two, then simplify to obtain}
\[  \tan'(x)  = \frac{\cos(x)^2}{\cos(x)^2} + \frac{\sin(x)^2}{\cos(x)^2} = 1+ \tan(x)^2. \]
\lang{de}{F\"ur den Kotangens rechnet man ganz entsprechend:}
\lang{en}{For the cotangent we similarly calculate:}
\begin{eqnarray*}
\cot'(x) &=& \left( \frac{\cos}{\sin}\right)'(x) = \frac{\cos'(x)\sin(x)-\cos(x)\sin'(x)}{\sin(x)^2} \\
&=& \frac{-\sin(x)\sin(x)-\cos(x) \cos(x))}{\sin(x)^2} = - \frac{\sin(x)^2+\cos(x)^2}{\sin(x)^2}.
\end{eqnarray*}
\lang{de}{Auch hier erh\"alt man durch Vereinfachen zum einen}
\lang{en}{Here too we arrive, after simplification, at the derivative}
\[ \cot'(x)= -\frac{1}{\sin(x)^2} \]
\lang{de}{und zum anderen}
\lang{en}{and at another way of writing it,}
\[ \cot'(x)= -1-\cot(x)^2. \]
\end{tabs*}
\end{example}

\begin{quickcheckcontainer}
\randomquickcheckpool{1}{1}
\begin{quickcheck}
		\field{rational}
		\type{input.function}
		\begin{variables}
			\randint[Z]{a}{-2}{2}
			\randint[Z]{b}{1}{4}
			\randint{l}{1}{4}
			\function[calculate]{l1}{l-1}
			
			\randint{k}{1}{4}	% Zufallsvariable zum Vertauschen:
			\function[calculate]{d1}{-(k-2)*(k-3)*(k-4)/6}  % "Dirac"-funktionen
			\function[calculate]{d2}{(k-1)*(k-3)*(k-4)/2}
			\function[calculate]{d3}{-(k-1)*(k-2)*(k-4)/2}
			\function[calculate]{d4}{(k-1)*(k-2)*(k-3)/6}
			
			\function[normalize]{f1}{sin(x)}
			\function[normalize]{f2}{cos(x)}
			\function[normalize]{f3}{exp(x)}
			\function[normalize]{f4}{ln(x)}

			\function[expand,normalize]{g}{d1*f1+d2*f2+d3*f3+d4*f4}						
			\function[expand,normalize]{dg}{d1*f2+d2*(-f1)+d3*f3+d4*(1/x)}

			\function[normalize]{p}{a*x^l+b}
			\function[normalize]{dp}{l*a*x^(l-1)}
		    \function[normalize]{f}{p*g}
			\function[normalize]{df}{dp*g+p*dg}
		\end{variables}

			\text{\lang{de}{
      Bestimmen Sie mit Hilfe der Formeln und der Ableitungsregeln die Ableitung der
			Funktion $f(x)=\var{f}$.\\ Die Ableitungsfunktion ist: $f'(x)=$ \ansref.
      }
      \lang{en}{
      Use the formulas and rules for differentiation to determine the derivative of the function 
      $f(x)=\var{f}$.\\ The derivative function is: $f'(x)=$ \ansref.
      }}

		\begin{answer}
			\solution{df}
			\checkAsFunction{x}{-2}{2}{20}
		\end{answer}
		\explanation{\lang{de}{
    Mit der Produktregel ist zun\"achst $f'(x)=(\var{p})'\cdot \var{g}+(\var{p})\cdot (\var{g})'$.\\
		F\"ur $(\var{p})'$ rechnet man:\\
    }
    \lang{en}{
    Firstly we apply the product rule, $f'(x)=(\var{p})'\cdot \var{g}+(\var{p})\cdot (\var{g})'$.\\
    We can then calculate 
    }
		$(\var{p})'=\var{a}\cdot (x^{\var{l}})'+ (\var{b})'=\var{a}\cdot \var{l}\cdot x^{\var{l1}}+0=\var{dp}$\\
		\lang{de}{
    und $(\var{g})'=\var{dg}$ mit obiger Formel. Also insgesamt:\\
    }
    \lang{en}{
    and we already know $(\var{g})'=\var{dg}$ from the previous section. Hence we have:\\
    }
		$f'(x)=\var{df}$.
		}
		
	\end{quickcheck}
\end{quickcheckcontainer}

\begin{remark}%\textit{Bemerkung:}\\
\lang{de}{
Im vorigen Abschnitt ist dargelegt, dass die Ableitung jeder konstanten Funktion $0$ ist. Nach der 
Produktregel ist also, falls $g(x)=c$ konstant ist:
}
\lang{en}{
In the earlier section we stated that the derivative of every constant function is $0$. According to 
the product rule, if we treat $g(x)=c$ as a constant function:
}
\[(c\cdot f)'(x)=(g\cdot f)'(x)=g'(x)\cdot f(x)+g(x)\cdot f'(x)=0 \cdot f(x)+c \cdot f'(x)=c \cdot f'(x).\]
\lang{de}{
Die Faktorregel ist also ein Spezialfall der Produktregel.
}
\lang{en}{
From this example we can see that the constant rule is simply a special case of the product rule.
}
\end{remark}

\begin{remark}
\lang{de}{
Als Spezialfall der Quotientenregel kann man auch die Ableitung von Funktionen der Form 
$\frac{1}{g(x)}$ bestimmen, wenn man die Ableitung von $g(x)$ kennt, n\"amlich:
}
\lang{en}{
We can consider derivatives of function of the form $\frac{1}{g(x)}$ as a special case of the 
quotient rule, if we know the derivative of $g(x)$. That is,
}
\[ \left( \frac{1}{g}\right)'(x)= \frac{0\cdot g(x)-1\cdot g'(x)}{g(x)^2}= -\frac{g'(x)}{g(x)^2}.\]
\lang{de}{Diese Formel werden wir unten auch mit der Kettenregel herleiten.}
\lang{en}{We will also derive these formulas using the chain rule below.}
\end{remark}

\begin{example}\label{bsp:negative-potenzen}
\lang{de}{
F\"ur $n\in \N$ betrachten wir die Potenzfunktion $f(x)=x^{-n}=\frac{1}{x^n}$. 
Mit $g(x)=x^n$ gilt nach der eben gemachten Bemerkung
}
\lang{en}{
Let $n\in \N$ and consider the power function $f(x)=x^{-n}=\frac{1}{x^n}$.
Setting $g(x)=x^n$ lets us apply the previous remark, yielding
}
\[  f'(x)=-\frac{g'(x)}{g(x)^2}=-\frac{nx^{n-1}}{x^{2n}}=-nx^{-n-1}. \]
\end{example}




\begin{quickcheck}
		\field{rational}
		\type{input.function}
		\begin{variables}
			\randint[Z]{n}{-9}{9}
			\randint[Z]{a}{-9}{9}
			\randint[Z]{b}{-6}{9}
			\randint[Z]{c}{-6}{9}
			\randint[Z]{d}{-6}{9}
		    \function[normalize]{f}{(x^2+a*x+b)/(c*x+d)} 
            \function[normalize] {f_1}{x^2+a*x+b}
            \function[normalize] {f_2}{c*x+d}
			\derivative[normalize]{df}{f}{x}
            \function[normalize]{dfz}{(2*x+a)*(c*x+d)-(x^2+a*x+b)*c}
            \function[normalize]{dfn}{(c*x+d)^2}
		\end{variables}
		
		\text{\lang{de}{
      Bestimmen Sie mit Hilfe der Quotientenregel die Ableitung der
			Funktion $f(x)=\var{f}$.\\ Die Ableitungsfunktion ist $f'(x)=$ \ansref.
      }
      \lang{en}{
      Use the quotient rule to determine the derivative of the function $f(x)=\var{f}$.\\
      The derivative function is $f'(x)=$ \ansref.
      }}
		\begin{answer}
			\solution{df}
			\checkAsFunction{x}{-1}{1}{10}
		\end{answer}
        \explanation{\lang{de}{
        Wir wenden die Quotientenregel auf die Zählerfunktion $\var{f_1}$
        und die Nennerfunktion $\var{f_2}$ an.
        }
        \lang{en}{
        We apply the quotient rule with $\var{f_1}$ as the numerator and $\var{f_2}$ as the 
        denominator.
        }}
		%\explanation{Nach der Quotientenregel ist die Ableitungsfunktion gegeben durch
        %$\frac{\var{dfz}}{\var{dfn}}$, worin man noch Terme zusammenfassen könnte.
        %}
	\end{quickcheck}

\section{\lang{de}{Ableitung der Verkettung von Funktionen}
         \lang{en}{Derivative of the composition of functions}}\label{sec:kettenregel}
% Kopie aus Teil1, content_21 + Begr\"undung


\lang{de}{Die Ableitung der Verkettung zweier Funktionen f\"uhrt auf die Kettenregel.}
\lang{en}{The derivative of the composition of two functions is described by the chain rule.}
\begin{rule}[\lang{de}{Kettenregel}\lang{en}{Chain rule}]
\label{rule:kettenregel}\\
\lang{de}{
Sind $f$ und $g$ Funktionen und $x_0\in \R$ eine Stelle, so dass $g$ bei $x_0$ differenzierbar ist 
und $f$ bei $g(x_0)$ differenzierbar ist , so ist die Verkettung $f\circ g$ bei $x_0$ differenzierbar 
und es gilt
}
\lang{en}{
Let $f$ and $g$ be functions and $x_0\in \R$ be a point such that $g$ is differentiable at $x_0$, 
and $f$ is differentiable at $g(x_0)$. Then the composition of the two functions $f\circ g$ is 
differentiable at $x_0$ and we have
}
\[(f\circ g)'(x_0)=f'(g(x_0))\cdot g'(x_0).\]
\lang{de}{
\floatright{\href{https://www.hm-kompakt.de/video?watch=513}{\image[75]{00_Videobutton_schwarz}}}\\\\
}
\lang{en}{}
\end{rule}
\lang{de}{
Man bildet also $f'$ an der Stelle $g(x_0)$ sowie $g'$ an der Stelle $x_0$ und multipliziert diese beiden Werte miteinander.
}
\lang{en}{
We evaluate $f'$ at the point $g(x_0)$ and $g'$ at the point $x_0$, then multiply these results together.
}

\begin{proof*}[\lang{de}{Beweis der Kettenregel}\lang{en}{Proof of the chain rule}]
\begin{incremental}[\initialsteps{0}]
\step \lang{de}{
Um diese Regel einzusehen, schreiben wir den Differenzenquotienten f\"ur $f\circ g$ zun\"achst etwas komplizierter:
}
\lang{en}{
Firstly we consider the quotient for $f\circ g$ whose limit we take to find its derivative, and 
express it in a slightly longer form:
}
\[ \frac{(f\circ g)(x)-(f\circ g)(x_0)}{x-x_0}= \frac{f(g(x))-f(g(x_0))}{g(x)-g(x_0)}\cdot \frac{g(x)-g(x_0)}{x-x_0}. \]
\step
\lang{de}{Nun gilt}
\lang{en}{This allows us to recognise}
\[ \lim_{x\to x_0} \frac{g(x)-g(x_0)}{x-x_0} =g'(x_0), \]
\lang{de}{
da $g$ bei $x_0$ differenzierbar ist. 
Wegen der Differenzierbarkeit von $f$ gilt
}
\lang{en}{
as $g$ is differentiable at $x_0$. 
By the differentiability of $f$ we have
}
\[  \lim_{y\to g(x_0)} \frac{f(y)-f(g(x_0))}{y-g(x_0)}= f'(g(x_0)). \]
\step \lang{de}{
Da $g$ bei $x_0$ aber auch stetig ist (vgl. Abschnitt ...), gilt
}
\lang{en}{
As $g$ is continuous at $x_0$ (it must be, to be differentiable), we have 
}
\[  \lim_{x\to x_0} g(x) = g(x_0), \]
\lang{de}{und daher insbesondere}
\lang{en}{so in particular}
\[ \lim_{x\to x_0} \frac{f(g(x))-f(g(x_0))}{g(x)-g(x_0)}= f'(g(x_0)). \]
\step \lang{de}{
Insgesamt also:
}
\lang{en}{
In general,
}
\begin{eqnarray*}
(f\circ g)'(x_0) &=&  \lim_{x\to x_0} \frac{(f\circ g)(x)-(f\circ g)(x_0)}{x-x_0} \\
&=&  \lim_{x\to x_0}  \frac{f(g(x))-f(g(x_0))}{g(x)-g(x_0)} \cdot  \lim_{x\to x_0} \frac{g(x)-g(x_0)}{x-x_0} \\
&=& f'(g(x_0))\cdot g'(x_0).
\end{eqnarray*}
\lang{de}{
\textbf{Achtung:} Diese Begr\"undung ist nicht formal korrekt, da wir einen Term $g(x)-g(x_0)$ im 
Nenner eingef\"ugt haben, welcher nur f\"ur $g(x)\neq g(x_0)$ definiert ist. Sie zeigt aber die 
Hauptidee f\"ur die Regel, weshalb wir in diesem Kurs auf einen formal korrekten Beweis verzichten.
}
\lang{en}{
\textbf{Warning:} This reasoning is not entirely formal, as we have a denominator $g(x)-g(x_0)$ which 
is only non-zero for $g(x)\neq g(x_0)$. We do not cover a formal proof in this course, 
as the above is sufficient for understanding the spirit of the rule, and certainly suffices for 
applying it.
}

\end{incremental}
\end{proof*}







\begin{example}%\textit{Beispiel:}\\
 \begin{tabs*}[\initialtab{0}]
\tab{$\sin(x^2)$}
\lang{de}{
Wir berechnen als Beispiel die Ableitung der Funktion $\sin(x^2)$. Setzen wir als \"{a}u{\ss}ere 
Funktion $f(y)=\sin y$ und als innere Funktion $g(x)=x^2$, so ist $f(g(x))=\sin(x^2)$. Wir k\"{o}nnen 
also die Kettenregel benutzen. Zun\"{a}chst ist $f'(y)=\cos y$. F\"{u}r $y=g(x)=x^2$ ergibt das:
}
\lang{en}{
As an example, we will calculate the derivative of the function $\sin(x^2)$. First we let the outer 
function be $f(y)=\sin y$ and the inner function be $g(x)=x^2$, so that $f(g(x))=\sin(x^2)$. Since 
this is the composition of two functions, we may use the chain rule here. 
We already know that $f'(y)=\cos y$. For $y=g(x)=x^2$ we get that:
}
\[f'(g(x))=\cos (g(x))=\cos(x^2).\] 
\lang{de}{Nun m\"{u}ssen wir noch die Ableitung von $g$ berechnen. Wir erhalten:}
\lang{en}{We still need to find the derivative of $g$. We get:}
\[g'(x)=2x.\]
\lang{de}{Nach der Kettenregel ist dann die Ableitung von $f(g(x))=\sin(x^2)$:}
\lang{en}{According to the chain rule, the derivative of $f(g(x))=\sin(x^2)$ is:}
\[(f\circ g)'(x)=f'(g(x))\cdot g'(x)=\cos (x^2)\cdot 2x=2x\cos (x^2).\]
\lang{de}{
Wenn man ein wenig Erfahrung beim Ableiten von Funktionen gesammelt hat, braucht man nat\"{u}rlich 
nicht mehr diese vielen kleinen Einzelschritte zu gehen! Die Rechnung kann man in etwa auf die letzte 
Zeile des Beispieles reduzieren.
}
\lang{en}{
Once we have gained enough experience differentiating functions, we no longer need to perform every 
small step. In practice, we may only write the last line of the example.
}
\tab{$\frac{1}{f(x)}$}
\lang{de}{
Kennt man die Ableitung einer Funktion $f(x)$, so kennt man auch die Ableitung von $\frac{1}{f(x)}$. 
Sei $f$ differenzierbar und $f(x)\neq 0$ und sei $h(y)=\frac{1}{y}$. Dann ist 
$\frac{1}{f(x)}=(h\circ f)(x)$ nach der Kettenregel differenzierbar und
}
\lang{en}{
If we know the derivative of a function $f(x)$, then we also know the derivative of $\frac{1}{f(x)}$. 
If $f$ is differentiable and $f(x)\neq 0$, if we let $h(y)=\frac{1}{y}$ then 
$\frac{1}{f(x)}=(h\circ f)(x)$ is differentiable by the chain rule, and
}
\[
\label{gleichung2}\left(\frac{1}{f(x)}\right)'=h'(f(x))\cdot f'(x)=-\frac{1}{(f(x))^2}\cdot f'(x)=-\frac{f'(x)}{(f(x))^2}.
\]


\tab{$x^{-n}$}
\lang{de}{
Als Spezialfall der Regel für $\frac{1}{f(x)}$ bekommt man auch die Ableitungen der Potenzfunktionen 
$F(x)=x^{-n}$ mit nat\"urlichen Zahlen $n\in \N$ (vgl. 
\lref{bsp:negative-potenzen}{obiges Beispiel}). Hierzu w\"ahlt man zum Beispiel als \"außere Funktion 
$f(y)=y^{-1}$ und als innere Funktion $g(x)=x^n$. Dann gilt wegen $f'(y)=-\frac{1}{y^2}$ und 
$g'(x)=nx^{n-1}$ f\"ur alle $x\neq 0$:
}
\lang{en}{
We may consider the derivative of the function $F(x)=x^{-n}$ for natural numbers $n\in \N$ as a 
special case of the rule for $\frac{1}{f(x)}$ (see \lref{bsp:negative-potenzen}{above example}). 
For example, if we take $f(y)=y^{-1}$ to be the outer function and $g(x)=x^n$ to be the inner 
function. Then as $f'(y)=-\frac{1}{y^2}$ and $g'(x)=nx^{n-1}$ for all $x\neq 0$, we have
}
\begin{eqnarray*}
F'(x) &=& f'(g(x))\cdot g'(x) = -\frac{1}{g(x)^2}\cdot nx^{n-1} \\
&=& -n\cdot \frac{x^{n-1}}{x^{2n}}= -n\cdot x^{-n-1}.
\end{eqnarray*}
\tab{$a^x$}
\lang{de}{
F\"ur allgemeine Exponentialfunktionen $f(x)=a^x$ lassen sich nun auch die Ableitungen bestimmen. 
Hierzu beachte man, dass die reelle Potenz $a^x$ definiert ist als $\exp(x\cdot \ln(a))$. Setzt man 
$g(x)=\exp(x)$ als \"au"sere Funktion und $h(x)=x\cdot \ln(a)$ als innere Funktion, so ist
}
\lang{en}{
We may now find the derivative of a general exponential function $f(x)=a^x$. To do this, we note that 
the real power $a^x$ is defined as $\exp(x\cdot \ln(a))$. If we set $g(x)=\exp(x)$ as the outer 
function and $h(x)=x\cdot \ln(a)$ as the inner function, then
}
\[ f'(x)=g'(h(x))\cdot h'(x)=\exp(h(x))\cdot \ln(a)=\exp(x\cdot \ln(a))\cdot \ln(a) =a^x\cdot \ln(a). \]

\end{tabs*}
\end{example}







\begin{quickcheck}
		\field{rational}
		\type{input.function}
		\begin{variables}
			\randint{k}{1}{4}	% Zufallsvariable f\"ur Auswahl:
			\function[calculate]{d1}{-(k-2)*(k-3)*(k-4)/6}  % "Dirac"-funktionen
			\function[calculate]{d2}{(k-1)*(k-3)*(k-4)/2}
			\function[calculate]{d3}{-(k-1)*(k-2)*(k-4)/2}
			\function[calculate]{d4}{(k-1)*(k-2)*(k-3)/6}

			\randint{l}{1}{2}   %Zufallsvariable f\"ur Auswahl
			\function[calculate]{l1}{-(l-2)}  % "Dirac"-funktionen
			\function[calculate]{l2}{(l-1)}

			\function{h0}{x^2+1}
			\function[normalize]{h}{l1*x+l2*h0}

			\function[normalize]{f1}{sin(h)}
			\function[normalize]{f2}{cos(h)}
			\function[normalize]{f3}{exp(h)}
			\function[normalize]{f4}{(h^3+1)}
			\function[expand,normalize]{f}{d1*f1+d2*f2+d3*f3+d4*f4}
			\function[normalize]{df}{d1*f2+d2*(-f1)+d3*f3+d4*3*h^2}  % Ableitung von f an der Stelle h

			% g=h0°f (falls l1=1) oder g=f°h0 (falls l2=1) mit f in fi
			\function[normalize]{g}{l2*f+l1*(f^2+1)}
			\function[expand,normalize]{dg}{l2*df*2*x+l1*(2*df*f)}
            \function[normalize]{ss}{l2*h0+l1*f}            
		\end{variables}

			\text{\lang{de}{
        Bestimmen Sie mit der Kettenregel die Ableitung der Funktion $h(x)=\var{g}$.\\
				Die Ableitung ist $h'(x)=$ \ansref.
        }
        \lang{en}{
        Use the chain rule to determine the derivative of the function $h(x)=\var{g}$.\\
        The derivative is $h'(x)=$ \ansref.
        }}

     \begin{answer}
          \solution{dg}
%		  \allowForInput[false]{)}
          \checkAsFunction{x}{-2}{2}{20}
      \end{answer}
      \explanation{\lang{de}{
      Die Kettenregel sagt aus, dass für $(f\circ g)(x_0) = f(g(x_0))$ die Ableitung durch 
      $f'(g(x_0))\cdot g'(x_0)$ gegeben wird.
      }
      \lang{en}{
      The chain rule tells us that the derivative of $(f\circ g)(x_0) = f(g(x_0))$ is given by the 
      function $f'(g(x_0))\cdot g'(x_0)$.
      }}
\end{quickcheck}


\begin{example}
\lang{de}{
Mit Hilfe der Ableitung des Logarithmus 
\ref[content_02_ableitungsregeln][(siehe hier)]{ex:abl-umkehrfunktion} und der Kettenregel lassen 
sich auch die Ableitungen reeller Potenzfunktionen $f(x)=x^r$ bestimmen. Hierzu beachte man, dass die 
reelle Potenz $x^r$ definiert ist als $\exp(r\cdot \ln(x))$. Damit gilt f\"ur die Ableitung
}
\lang{en}{
Using the \ref[content_02_ableitungsregeln][derivative of the logarithm]{ex:abl-umkehrfunktion} and 
the chain rule, we can also find the derivative of power functions $f(x)=x^r$ by expressing it as 
$\exp(r\cdot \ln(x))$. Using this, we have the derivative
}
\begin{eqnarray*}
f'(x) &=& \exp'(r\ln(x))\cdot r\cdot \ln'(x) \\
&=&  \exp(r\ln(x))\cdot r\cdot \frac{1}{x}\\ 
&=& x^r\cdot r\cdot x^{-1}=rx^{r-1}
\end{eqnarray*}
\end{example}



\section{\lang{de}{Ableitung von Umkehrfunktionen}\lang{en}{Derivative of an inverse}}

\lang{de}{
Mit Hilfe der Kettenregel l\"asst sich auch eine Formel f\"ur die Ableitung der 
\ref[umkehrfkt][Umkehrfunktion]{def:partielle-umkehrfunktion} finden.
}
\lang{en}{
Using the chain rule we may derive a formula for the 
\ref[umkehrfkt][inverse function]{def:partielle-umkehrfunktion}.
}

\begin{rule}\label{rule:abl-umkehrfkt}
\lang{de}{
Ist $f:D\to W$ umkehrbar und differenzierbar in $x_0$ mit $f'(x_0)\neq 0$, so ist die Umkehrfunktion 
$f^{-1}$ differenzierbar in $y_0=f(x_0)$ und es gilt
}
\lang{en}{
If $f:D\to W$ is invertible and differentiable at $x_0$ with $f'(x_0)\neq 0$, then the inverse 
function $f^{-1}$ is differentiable at $y_0=f(x_0)$ and we have
}
\[  (f^{-1})'(y_0)=\frac{1}{f'(x_0)}=\frac{1}{f'(f^{-1}(y_0))}. \]
\lang{de}{
\floatright{\href{https://www.hm-kompakt.de/video?watch=518}{\image[75]{00_Videobutton_schwarz}}}\\\\
}
\lang{en}{}
\end{rule}

\begin{proof*}[\lang{de}{Beweis der Ableitungsregel für die Umkehrfunktion}
               \lang{en}{Proof of the derivative rule for the inverse function}]
\begin{incremental}[\initialsteps{0}]
\step \lang{de}{
Nach Definition der Umkehrfunktion ist $x=(f^{-1}\circ f)(x)$ f\"ur alle $x\in D$. Unter der Annahme, 
dass die Umkehrfunktion in $y_0$ differenzierbar ist, bildet man auf beiden Seiten die Ableitungen an 
der Stelle $x_0$, und erh\"alt man mit der Kettenregel
}
\lang{en}{
By definition, the inverse function satisfies $x=(f^{-1}\circ f)(x)$ for all $x\in D$. Under the 
assumption that the inverse is differentiable at $y_0$, we differentiate both sides of this 
equation at $x_0$, and using the chain rule obtain 
}
\[ 1= (f^{-1})'(f(x_0))\cdot f'(x_0). \]
\step 
\lang{de}{Also}
\lang{en}{So}
\[ (f^{-1})'(f(x_0))=\frac{1}{f'(x_0)}=\frac{1}{f'(f^{-1}(y_0))}. \]
\lang{de}{
Den technischen Beweis, dass die Umkehrfunktion wirklich bei $y_0$ differenzierbar ist, lassen wir weg.
}
\lang{en}{
We omit here the technical proof that the inverse function is in fact differentiable at $y_0$.
}
\end{incremental}
\end{proof*}


\begin{quickcheck}
		\field{rational}
		\type{input.function}
		\begin{variables}
			\randint[Z]{a0}{1}{5}
            \number{a1}{1}
            \function[calculate]{a}{1+a0/a1}
            \function[calculate]{b}{a+1/a}
		    \function[normalize]{f}{x+1/x}            
			\derivative[normalize]{df}{f}{x}
            \function[calculate]{dgb}{a^2/(a^2-1)}
		\end{variables}
		
		\text{\lang{de}{
        Die Funktion $f:(2,\infty)\to \R$ mit $f(x)=x+\frac{1}{x}$ besitzt eine reelle
        Umkehrfunktion $g:(2,\infty)\to \R$. Bestimmen Sie die Ableitung von $g$ an der 
        Stelle $y=f(\var{a})=\var{b}$.\\ Die Ableitungsfunktion von $f$ ist $f'(x)=$ \ansref.\\
        Damit ist $g'(\var{b})=$\ansref.
        }
        \lang{en}{
        The function $f:(2,\infty)\to \R$ defined by $f(x)=x+\frac{1}{x}$ has a real inverse function 
        $g:(2,\infty)\to \R$. Determine the derivative of $g$ at the point $y=f(\var{a})=\var{b}$.\\ 
        The derivative of $f$ is $f'(x)=$ \ansref.\\
        Hence $g'(\var{b})=$\ansref.
        }}
		\begin{answer}
			\solution{df}
			\checkAsFunction{x}{-1}{1}{10}
		\end{answer}
		\begin{answer}
			\solution{dgb}
			\checkAsFunction{x}{-1}{1}{10}
		\end{answer}
		\explanation{\lang{de}{Nach der Regel für die Ableitung der Umkehrfunktion ist}
        \lang{en}{By the rule for differentiating an inverse function,}
        $g'(\var{b})=\frac{1}{f'(\var{a})}=\frac{1}{1-\frac{1}{\var{a}^2}}=\frac{\var{a}^2}{\var{a}^2-1}$.
        }
	\end{quickcheck}
    
    
\begin{remark}
\lang{de}{
Die Formel f\"ur die Ableitung der Umkehrfunktion l\"asst sich auch gut veranschaulichen, wenn man 
sich erinnert, dass man den Graphen der Umkehrfunktion $f^{-1}$ dadurch bekommt, dass man den Graphen 
der Funktion $f$ an der ersten Winkelhalbierenden spiegelt. Die Tangente an den Funktionsgraphen an 
der Stelle $x_0$ wird dann beim Spiegeln eine Tangente an den Graphen der Umkehrfunktion an der 
Stelle $y_0=f(x_0)$, und deren Steigung ist genau der Kehrbruch der Steigung der urspr\"unglichen 
Tangente.
}
\lang{en}{
The formula for the derivative of the inverse function can be justified by recalling that the graph 
of the inverse function $f^{-1}$ is the reflection of the graph of the function $f$ in the line 
$x=y$. The tangent of the function $f$ at a point $x_0$ is hence transformed into a tangent of the 
inverse function $f^{-1}$ at the point $y_0=f(x_0)$, whose gradient is the reciprocal of the 
gradient of the original tangent.
}

%BILD EINFÜGEN
	\begin{genericGWTVisualization}[550][1000]{mathlet1}
		\begin{variables}
			\randint{randomA}{1}{2}
			\function{f}{real}{exp(x)}
			\function{g}{real}{ln(x)}
			\pointOnCurve[0.1,4]{P}{real}{var(g)}{0.5}
			\number{x0}{real}{var(P)[x]}
			\number{y0}{real}{var(P)[y]}
			\number{t}{real}{1/var(x0)}
			\point{Q}{real}{var(x0)+1, var(y0)+var(t)}
			\point{R}{real}{var(x0)+1, var(y0)}
			\line{l}{real}{var(P),var(Q)}
			\segment{h}{real}{var(P),var(R)}
			\segment{v}{real}{var(Q),var(R)}
			\point{P2}{real}{var(P)[y], var(P)[x]}
			\point{Q2}{real}{var(Q)[y], var(Q)[x]}
			\point{R2}{real}{var(R)[y], var(R)[x]}
			\line{l2}{real}{var(P2),var(Q2)}
			\segment{h2}{real}{var(P2),var(R2)}
			\segment{v2}{real}{var(Q2),var(R2)}
			\function{d}{real}{x}
		\end{variables}
		\color{P}{#0066CC}
		\color{P2}{#0066CC}
		\color{l}{#0066CC}
		\color{h}{#0066CC}
		\color{v}{#0066CC}
		\color{l2}{#0066CC}
		\color{h2}{#0066CC}
		\color{v2}{#0066CC}
		\color{d}{LIGHT_GRAY}
		\label{P}{$\textcolor{#0066CC}{P}$}

		\begin{canvas}
			\plotSize{300}
			\plotLeft{-3}
			\plotRight{3}
			\plot[coordinateSystem]{f,g,d, l,h,v,l2,h2,v2, P, P2}
		\end{canvas}
		\lang{de}{\text{
    Die Umkehrfunktion erh\"alt man durch Spiegeln an der Diagonalen. Ebenso das Steigungsdreieck am 
    gespiegelten Punkt.
    }}
    \lang{en}{\text{
    The inverse function is obtained by a reflection in the diagonal, as is the triangle by which we 
    calculate the gradient of the tangent.
    }}
\end{genericGWTVisualization}
\end{remark}

\begin{example}\label{ex:abl-umkehrfunktion}
\begin{tabs*}[\initialtab{0}]
\tab{$\sqrt[n]{x}$}
\lang{de}{
Die $n$-te Wurzelfunktion $g(x)=\sqrt[n]{x}=x^{\frac{1}{n}}$ ist die Umkehrfunktion der $n$-ten 
Potenzfunktion $f:\R_+\to \R_+$ mit $f(x)=x^n$. F\"ur $x>0$ ist letztere differenzierbar mit 
$f'(x)=nx^{n-1}\neq 0$. Also ist auch die Wurzelfunktion f\"ur alle $x>0$ differenzierbar und es 
gilt:
}
\lang{en}{
The $n$th root function $g(x)=\sqrt[n]{x}=x^{\frac{1}{n}}$ is the inverse function of the $n$th 
power function $f:\R_+\to \R_+$ where $f(x)=x^n$. For $x>0$, the latter is differentiable with 
$f'(x)=nx^{n-1}\neq 0$. Therefore the $n$th root function is differentiable for all $x>0$ and we have:
}
\begin{eqnarray*}
g'(x) &=& \frac{1}{f'(g(x))}=\frac{1}{n(g(x))^{n-1}} \\
&=& \frac{1}{n\cdot \sqrt[n]{x}^{n-1}}=\frac{1}{n}\cdot x^{- \frac{n-1}{n}}\\
&=& \frac{1}{n}\cdot x^{\frac{1}{n}-1}.
\end{eqnarray*} 
\lang{de}{
Für den Punkt $x=0$ lässt sich die Regel nicht Anwenden, da dort $f'(0) = 0$ gilt. Zudem ist die 
Wurzelfunktion in $x=0$ gar nicht differenzierbar.
}
\lang{en}{
The rule may not be applied for $x=0$, as $f'(0) = 0$. Hence the $n$th root function is simply not 
differentiable at $x=0$.
}
\tab{$\ln(x)$}
\lang{de}{
Der nat\"urliche Logarithmus $g(x)=\ln(x)$ ist die Umkehrfunktion der Exponentialfunktion 
$\exp(x)=e^x$. Wegen $\exp'(x)=\exp(x)>0$ f\"ur alle $x\in \R$ ist auch der nat\"urliche Logarithmus 
auf seinem gesamten Definitionsbereich $\R_{>0}=\{ x\in \R\,|\, x>0\}$ differenzierbar und es gilt
}
\lang{en}{
The natural logarithm $g(x)=\ln(x)$ is the inverse function of the exponential function 
$\exp(x)=e^x$. Because $\exp'(x)=\exp(x)>0$ for all $x\in \R$, the natural logarithm is also 
differentiable on its entire domain $\R_{>0}=\{ x\in \R\,|\, x>0\}$ and we have
}
\[ \ln'(x) =  \frac{1}{f'(g(x))}= \frac{1}{\exp(\ln(x))}=\frac{1}{x}. \]
\tab{$\arctan(x)$}
\lang{de}{
Der Hauptzweig der Tangensfunktion $\tan:(-\frac{\pi}{2},\frac{\pi}{2})\to \R$ ist auf dem ganzen 
Definitionsbereich differenzierbar mit $\tan'(x)=1+\tan(x)^2>0$. Also ist deren Umkehrfunktion, der 
Arkustangens $\arctan$, auch auf ganz $\R$ differenzierbar und dessen Ableitung berechnet sich als
}
\lang{en}{
The \emph{principal branch} of the tangent function $\tan:(-\frac{\pi}{2},\frac{\pi}{2})\to \R$ is 
differentiable on its entire domain with $\tan'(x)=1+\tan(x)^2>0$. Therefore its inverse function 
$\arctan$ is also differntiable on its entire domain $\R$ with derivative
}
\[  \arctan'(y)=\frac{1}{\tan'(\arctan(y))} 
=\frac{1}{1+\tan(\arctan(y))^2}=\frac{1}{1+y^2}. \]
\end{tabs*}
\end{example}

\lang{de}{
Dieses abschließende Video fasst einige Ableitungsregeln zusammen.
\floatright{\href{https://api.stream24.net/vod/getVideo.php?id=10962-2-10758&mode=iframe&speed=true}{\image[75]{00_video_button_schwarz-blau}}}\\
}
\lang{en}{}

\end{visualizationwrapper}


\end{content}