\documentclass{mumie.element.exercise}
%$Id$
\begin{metainfo}
  \name{
    \lang{de}{Ü01: Tangente}
    \lang{en}{Exercise 1}
  }
  \begin{description} 
 This work is licensed under the Creative Commons License Attribution 4.0 International (CC-BY 4.0)   
 https://creativecommons.org/licenses/by/4.0/legalcode 

    \lang{de}{Hier die Beschreibung}
    \lang{en}{}
  \end{description}
  \begin{components}
  \end{components}
  \begin{links}
  \end{links}
  \creategeneric
\end{metainfo}
\begin{content}
\title{
  \lang{de}{Ü01: Tangente}
  \lang{en}{Exercise 1}
}



\begin{block}[annotation]
  Im Ticket-System: \href{http://team.mumie.net/issues/10274}{Ticket 10274}
\end{block}



\lang{de}{

Geben Sie die Geradengleichung der Tangente an die Funktionsgraphen von
\begin{table}[\class{items}]
a) $f(x) = x^2$ in $x_0 = \frac{1}{2}$.\\
b) $f(x) = \frac{1}{x}$ in $x_0 = 2$.\\
c) $f(x) = e^x$ in $x_0 = 0$.
\end{table}
an und fertigen Sie entsprechende Zeichnungen an.



\begin{table}[\class{items}]
d) Bestimmen Sie die Gleichung der Tangente für die Funktion $f(x)=e^x$ an der Stelle $x_0=0$.\\
e) Bestimmen Sie die Gleichung der Tangente für die Funktion $f(x)=x^{2}-4x+6$ an der Stelle $x_{0}=1$.
\end{table}
}


\begin{tabs*}[\initialtab{0}\class{exercise}]

\tab{
  \lang{de}{Lösungsvideo a) bis c)}
  }
\youtubevideo[500][300]{OL1UjpKyE5E}\\

  \tab{
  \lang{de}{Lösung d)}
  \lang{en}{Solution}
  }
  
  \begin{incremental}[\initialsteps{1}]
    \step \lang{de}{Die Ableitung der $e$-Funktion ist wieder die $e$-Funktion. Also ist $f'(x)=e^x$ für alle $x\in\R$. Daher erhalten wir $f'(x_{0})=e^0=1$.
Aus der Gleichung $f(x_{0})=e^0=1$ bekommen wir somit die Gleichung der Tangenten als
\[T(x)=f'(x_{0})(x-x_{0})+f(x_{0})=1\cdot (x-0)+1=x+1\,.\]}    
  \end{incremental}

\tab{
	\lang{de}{Lösung e)}
  	\lang{en}{Solution}
  	}
  
  \begin{incremental}[\initialsteps{1}]
    \step \lang{de}{Nach den Ableitungsregeln gilt $f'(x)=2x-4$ für alle $x\in\R$. Daher erhalten wir $f'(x_{0})=-2$.}
    \step \lang{de}{Aus der Gleichung $f(x_{0})=f(1)=3$ bekommen wir somit die Gleichung der Tangenten als
\[T(x)=f'(x_{0})(x-x_{0})+f(x_{0})=-2(x-1)+3=-2x+5\,.\]}
  \end{incremental}

  

\end{tabs*}

\end{content}