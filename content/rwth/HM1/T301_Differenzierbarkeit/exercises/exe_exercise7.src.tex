\documentclass{mumie.element.exercise}
%$Id$
\begin{metainfo}
  \name{
    \lang{de}{Ü07: Monotonie}
    \lang{en}{Exercise 7}
  }
  \begin{description} 
 This work is licensed under the Creative Commons License Attribution 4.0 International (CC-BY 4.0)   
 https://creativecommons.org/licenses/by/4.0/legalcode 

    \lang{de}{Hier die Beschreibung}
    \lang{en}{}
  \end{description}
  \begin{components}
  \end{components}
  \begin{links}
  \end{links}
  \creategeneric
\end{metainfo}
\begin{content}
\title{
  \lang{de}{Ü07: Monotonie}
  \lang{en}{Exercise 7}
}


\begin{block}[annotation]
  Im Ticket-System: \href{http://team.mumie.net/issues/10280}{Ticket 10280}
\end{block}

\lang{de}{Gegeben sei die Funktion $f(x)=x\ln x$ f\"{u}r $x>0$. Wo steigt $f$ monoton, wo f\"{a}llt $f$ monoton?}

\begin{tabs*}[\initialtab{0}\class{exercise}]
  \tab{
  \lang{de}{Antwort}
  }
\lang{de}{$f$ ist auf dem Intervall $(0;e^{-1})$ monoton fallend und auf dem Intervall 
$(e^{-1};\infty)$ monoton steigend.}

  \tab{
  \lang{de}{L\"{o}sung}
  }
  \begin{incremental}[\initialsteps{1}]
  \step \lang{de}{Wir bemerken, dass $f$ differenzierbar ist und wollen die Monotoniekriterien, die sich aus dem 
  Vorzeichen der Ableitung ergeben, anwenden. Wir berechnen $f'(x)$ mit der Produktregel,}
   \step \[f'(x)=\ln x+x\frac{1}{x}=\ln x+1.\]
   \step \lang{de}{Da $\ln x+1<0$ f\"{u}r $x\in (0; e^{-1})$, ist $f$ dort monoton fallend, und sogar streng monoton fallend.}
   
   \step \lang{de}{Da $\ln x+1>0$ f\"{u}r $x\in (e^{-1};\infty)$, ist $f$ dort monoton steigend, und sogar streng monoton steigend.}
  

   
  
  \end{incremental}


\end{tabs*}

\end{content}