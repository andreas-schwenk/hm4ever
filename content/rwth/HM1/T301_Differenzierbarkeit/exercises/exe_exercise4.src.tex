\documentclass{mumie.element.exercise}
%$Id$
\begin{metainfo}
  \name{
    \lang{de}{Ü04: Kettenregel}
    \lang{en}{Exercise 4}
  }
  \begin{description} 
 This work is licensed under the Creative Commons License Attribution 4.0 International (CC-BY 4.0)   
 https://creativecommons.org/licenses/by/4.0/legalcode 

    \lang{de}{Hier die Beschreibung}
    \lang{en}{}
  \end{description}
  \begin{components}
  \end{components}
  \begin{links}
  \end{links}
  \creategeneric
\end{metainfo}
\begin{content}

\title{
\lang{de}{Ü04: Kettenregel}
}
 \begin{block}[annotation]
  Im Ticket-System: \href{http://team.mumie.net/issues/10277}{Ticket 10277}
\end{block}
 
 
\lang{de}{

Für die folgenden Funktionen sollen Sie ebenfalls die Ableitungen via Kettenregel berechnen. Den Definitionsbereich 
müssen Sie nicht angeben.

\begin{table}[\class{items}]
\nowrap{a) $f(x)= \sin(3x)$} \\
\nowrap{b) $f(s)=\sin^3(s)$} \\
\nowrap{c) $f(x) = (e^x)^2$}\\
\nowrap{d) $f(x) = e^{x^2}$}\\
\nowrap{e) $f(x) = (\sin^4(x) + 1) $}\\
\nowrap{f) $f(x) = \sin(e^{c \cdot t}) $}\\
\nowrap{g) $f(z)= \sqrt[3]{\sqrt{z^2 + 1}}  $}
\end{table}

Berechnen Sie die Ableitungen der folgenden Funktionen mit Hilfe der Kettenregel. Geben Sie auch ihre Definitionsbereiche an.}

\begin{table}[\class{items}]
\nowrap{h) $f(x)=\exp(\sin(2x))$} \\
\nowrap{i) $f(x)=\ln(x^{2}+1)$} \\
\nowrap{j) $f(x)=\cos^{2}(\ln(x+1))$}
\end{table}


 


\begin{tabs*}[\initialtab{0}\class{exercise}]

\tab{\lang{de}{Lösungsvideo a) - g)}}
  \youtubevideo[500][300]{_6YAXhKWhnE}\\

  \tab{
  \lang{de}{Lösung h)}
  \lang{en}{Solution}
  }
  
  Die Funktion $f$ ist auf ganz $\R$ definiert. Mit Hilfe der Kettenregel erhalten wir als Ableitung
\[f'(x)=\exp'(\sin(2x))\cdot \sin'(2x)\cdot (2x)'=\exp(\sin(2x))\cos(2x)\cdot 2=2\exp(\sin(2x))\cos(2x)\,.\]

\tab{
\lang{de}{Lösung i)}
}
Die Funktion $f$ ist auf ganz $\R$ definiert. Mit Hilfe der Kettenregel erhalten wir als Ableitung
\[f'(x)=\ln'(x^{2}+1)\cdot (x^{2}+1)'=\frac{2x}{x^{2}+1}\,.\]

	\tab{
	\lang{de}{Lösung j)}
	}
	Die Funktion $f$ ist überall dort definert, wo $x+1$ postiv ist. Daher lautet der Definitionsbereich von $f$ genau $(-1,\infty)$.  Mit Hilfe der Kettenregel erhalten wir als Ableitung
\begin{align*}
   f'(x)&=&&2\cos(\ln(x+1))\cdot \cos'(\ln(x+1))\cdot(\ln'(x+1))\cdot (x+1)'&\\
&=&&2\cos(\ln(x+1))(-\sin(\ln(x+1)))\cdot\frac{1}{x+1}\,&\\
&=&&\frac{-2}{x+1}\cos(\ln(x+1))\sin(\ln(x+1))\,.&
  \end{align*}
  
   
  
 \end{tabs*}
  
\end{content}