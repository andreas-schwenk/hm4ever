\documentclass{mumie.element.exercise}
%$Id$
\begin{metainfo}
  \name{
    \lang{de}{Ü06: Umkehrfunktion}
    \lang{en}{Exercise 6}
  }
  \begin{description} 
 This work is licensed under the Creative Commons License Attribution 4.0 International (CC-BY 4.0)   
 https://creativecommons.org/licenses/by/4.0/legalcode 

    \lang{de}{}
    \lang{en}{}
  \end{description}
  \begin{components}
  \end{components}
  \begin{links}
  \end{links}
  \creategeneric
\end{metainfo}
\begin{content}
\usepackage{mumie.ombplus}

\title{\lang{de}{Ü06: Umkehrfunktion}}

\begin{block}[annotation]
  Im Ticket-System: \href{http://team.mumie.net/issues/10279}{Ticket 10279}
\end{block}

%######################################################FRAGE_TEXT
\lang{de}{ Gegeben ist die Funktion
\[ f:(0,\infty)\to\R,x\mapsto xe^x. \]
a) Zeigen Sie, dass die Funktion $f$ streng monoton wachsend ist.\\
b) Bestimmen Sie die Ableitung der Umkehrfunktion von $f$ an der Stelle $e$.
 }

%##################################################ANTWORTEN_TEXT
\begin{tabs*}[\initialtab{0}\class{exercise}]

  %++++++++++++++++++++++++++++++++++++++++++START_TAB_X
  \tab{\lang{de}{    Lösung a)    }}
  \begin{incremental}[\initialsteps{1}]
  
  	 %----------------------------------START_STEP_X
    \step 
    \lang{de}{ Wir bemerken, dass $f$ als Produkt differenzierbarer Funktionen differenzierbar ist.
Also können wir aus dem Vorzeichen der Ableitung $f'$ Rückschlüsse auf das Monotonieverhalten von $f$ ziehen.
    Wir bestimmen zunächst die Ableitungsfunktion von $f$ mit Hilfe der Produktregel 
$(u\cdot v)'(x)=u'(x)v(x)+u(x)v'(x)$ angewandt auf $u(x)=x$ und $v(x)=e^x$.

Dann ist
\[ f'(x)=1\cdot e^x+x\cdot e^x=(1+x)e^x. \]
Für $x>0$ gilt nun $f'(x)=(1+x)e^x>0$.
Da die Ableitungsfunktion stets positiv ist, ist $f$ streng monoton wachsend.
     }
  	 %------------------------------------END_STEP_X
 
  \end{incremental}
  %++++++++++++++++++++++++++++++++++++++++++++END_TAB_X
  
  %++++++++++++++++++++++++++++++++++++++++++START_TAB_X
  \tab{\lang{de}{    Lösung b)    }}
  \begin{incremental}[\initialsteps{1}]
  
  	 %----------------------------------START_STEP_X
    \step 
    \lang{de}{      Nach der Formel für die Ableitung der Umkehrfunktion ist
\[  (f^{-1})'(e)=\frac{1}{f'(f^{-1}(e))}. \]
 $f^{-1}(e)$ ist die Stelle $x_0$, für die $f(x_0)=e$ gilt. Wie leicht zu sehen ist,
 ist $f(1)=1\cdot e^1=e$, also $f^{-1}(e)=1$.
 
Da $f'(x)=(1+x)e^x$ ist, wie in a) berechnet wurde, gilt also
\[  (f^{-1})'(e)=\frac{1}{f'(f^{-1}(e))}=\frac{1}{f'(1)}=\frac{1}{2e}. \]
 }
  	 %------------------------------------END_STEP_X
 
  \end{incremental}
  %++++++++++++++++++++++++++++++++++++++++++++END_TAB_X


%#############################################################ENDE
\end{tabs*}
\end{content}