\documentclass{mumie.element.exercise}
%$Id$
\begin{metainfo}
  \name{
    \lang{de}{Ü03: Ableitungsregeln}
    \lang{en}{Exercise 3}
  }
  \begin{description} 
 This work is licensed under the Creative Commons License Attribution 4.0 International (CC-BY 4.0)   
 https://creativecommons.org/licenses/by/4.0/legalcode 

    \lang{de}{Hier die Beschreibung}
    \lang{en}{}
  \end{description}
  \begin{components}
  \end{components}
  \begin{links}
  \end{links}
  \creategeneric
\end{metainfo}
\begin{content}
\title{
  \lang{de}{Ü03: Ableitungsregeln}
  \lang{en}{Exercise 3}
}





\begin{block}[annotation]
  TODO: Konzeption 2
\end{block}

\begin{block}[annotation]
  Im Ticket-System: \href{http://team.mumie.net/issues/10276}{Ticket 10276}
\end{block}





\lang{de}{

Berechnen Sie die Ableitungen der folgenden Funktionen mit der Pordukt- bzw. Quotientenregel.

\begin{table}[\class{items}]
\nowrap{a) $f(x)=x \cdot e^x$} \\
\nowrap{b) $f(t) = t^2 \cdot \sin(t)$} \\
\nowrap{c) $f(a) = (a^2 -1)(1+a^2)$}\\
\nowrap{d) $f(z) = z\cdot\sqrt{z} $}\\
\nowrap{e) $ f(x) = x \cdot \ln(x) $}\\
\nowrap{f) $ f(\omega) = \cos(\omega) \cdot \tan(\omega) $}\\
\nowrap{g) $ f(x) = \frac{x^2 +2x}{3x +1} $}\\
\nowrap{h) $f(x) = \frac{1 }{x^2 + x +1} $}\\
\nowrap{i) $f(s) = \frac{s^2 + 4s +5}{s^3} $}\\
\nowrap{j) $f(x) = \frac{\sqrt[3]{x} }{x} $}\\
\nowrap{k) $f(s) = \frac{\ln(s)}{s} $  }\\
\nowrap{l) $f(w) = \frac{\tan(w)}{\sin(w)} $ }\\
\nowrap{m) $f(x)=x^{4}+\ln(x)\cdot \sqrt{x}$.} \\
\nowrap{n) $f(x)=\sin(x)\cos(x)$.}
\end{table}

}

\begin{tabs*}[\initialtab{0}\class{exercise}]

 


 \tab{
  \lang{de}{Antwort}
  \lang{en}{Answer}
  }
  \begin{table}[\class{items}]
  \nowrap{a) $f'(x)=e^{x}(1+x)$} \\
  \nowrap{b) $f'(t)=t (2 \sin(t) + t \cos(t))$} \\
  \nowrap{c) $f'(a)=4 a^3$} \\
  \nowrap{d) $f'(z)=\frac{3}{2} \sqrt{z}$} \\
  \nowrap{e) $f'(x)=\ln(x) +1$} \\
  \nowrap{f) $f'(w)=\cos(w)$} \\
  \nowrap{g) $f'(x)=\frac{3x^2 + 2x + 2}{(3x+1)^2}$} \\
  \nowrap{h) $f'(x)=\frac{-(2x +1)}{(x^2 + x +1)^2}$} \\
  \nowrap{i) $f'(s)=\frac{-s^2 -8s- 15}{s^4}$} \\
  \nowrap{j) $f'(x)= - \frac{2}{3 \sqrt[3]{x^5}}$} \\
  \nowrap{k) $f'(s)=\frac{1 - \ln(s)}{s^2}$} \\
  \nowrap{l) $f'(\omega)= \frac{\sin(\omega)}{\cos^2(\omega)}$} \\
\nowrap{m) $f'(x)=4x^{3}+\frac{\sqrt{x}}{x}+\frac{\ln(x)}{2\sqrt{x}}$} \\
\nowrap{n) $f'(x)=\cos^{2}(x)-\sin^{2}(x)$}
\end{table}

\tab{\lang{de}{Lösungsvideo a) - l)}}
   \youtubevideo[500][300]{vwilDEhSoX0}\\
  
 
  
  \tab{
  \lang{de}{Lösung m)}
  \lang{en}{Solution}
  }
  
  \begin{incremental}[\initialsteps{1}]
    \step \lang{de}{Wir verwenden wieder die Produktregel mit $u(x)=\ln(x)$ und $v(x)=\sqrt{x}$ für alle $x>0$.}
    \step \lang{de}{Aus der Linearität der Ableitung ergibt sich somit 
\[f'(x)=\frac{d}{dx}(x^{4})+u'(x)v(x)+u(x)v'(x)=4x^{3}+\frac{\sqrt{x}}{x}+\frac{\ln(x)}{2\sqrt{x}}\]
für alle $x>0$ als Ableitung.}
  \end{incremental}  

 \tab{
  \lang{de}{Lösung n)}
  \lang{en}{Solution}
  }
  
 \lang{de}{Wir setzen $u(x)=\sin(x)$ und $v(x)=\cos(x)$. Aus der Produktregel erhalten wir mit Hilfe der Ableitungsformeln für Sinus und Cosinus
\[f'(x)=u'(x)v(x)+u(x)v'(x)=\cos(x)\cos(x)+\sin(x)[-\sin(x)]=\cos^{2}(x)-\sin^{2}(x)\]
für alle $x\in\R$.} 



\end{tabs*}

\end{content}