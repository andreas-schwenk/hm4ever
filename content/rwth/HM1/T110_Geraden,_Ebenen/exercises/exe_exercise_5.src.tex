\documentclass{mumie.element.exercise}
%$Id$
\begin{metainfo}
  \name{
    \lang{de}{Ü05: Geradenschnittpunkt}
    \lang{en}{}
  }
  \begin{description} 
 This work is licensed under the Creative Commons License Attribution 4.0 International (CC-BY 4.0)   
 https://creativecommons.org/licenses/by/4.0/legalcode 

    \lang{de}{Hier die Beschreibung}
    \lang{en}{}
  \end{description}
  \begin{components}
  \end{components}
  \begin{links}
  \end{links}
  \creategeneric
\end{metainfo}
\begin{content}
\title{
  \lang{de}{Ü05: Geradenschnittpunkt}
  \lang{en}{Exercise 5}
}


\begin{block}[annotation]
	Im Ticket-System: \href{http://team.mumie.net/issues/22815}{Ticket 22815}
\end{block}

  	\lang{de}{Ermitteln Sie den Schnittpunkt $S$ der beiden Geraden}
  	\lang{en}{Find the intersection point $S$ of the lines:}
	\begin{align*}
		g:& \vec{x} = \begin{pmatrix} -3\\-1\\-3\end{pmatrix} + \lambda\cdot \begin{pmatrix}2\\1\\2\end{pmatrix},\quad\lambda\in\R, \\
		h:& \vec{x} = \begin{pmatrix} 7\\3\\-1\end{pmatrix} + \mu\cdot \begin{pmatrix}3\\1\\-1\end{pmatrix},\quad\mu\in\R.
	\end{align*}

\begin{tabs*}[\initialtab{0}\class{exercise}]
	\tab{\lang{de}{Antwort}\lang{en}{Answer}}
		  \lang{de}{Der Schnittpunkt der beiden Geraden $g$ und $h$ ist $S = (1;1;1)$.}
		  \lang{en}{The intersection point of the two lines $g$ and $h$ is $S=(1,1,1)$.}
  
  	\tab{\lang{de}{Lösung}\lang{en}{Solution}}
		\begin{incremental}[\initialsteps{1}]
			\step	
			    \lang{de}{Wir setzen die beiden Geradendarstellungen gleich, also}
			    \lang{en}{Set the equations of both lines equal to each other so that we can calculate their intersection point:}
				\[\begin{pmatrix} -3\\-1\\-3\end{pmatrix} + \lambda\cdot \begin{pmatrix}2\\1\\2\end{pmatrix} = \begin{pmatrix} 7\\3\\-1\end{pmatrix} + \mu\cdot \begin{pmatrix}3\\1\\-1\end{pmatrix}\lang{de}{,}\lang{en}{.}\]
   				\lang{de}{damit wir den Schnittpunkt ausrechnen können.}
    
			\step
    			\lang{de}{Der Ansatz liefert die drei Gleichungen}
    			\lang{en}{The above Ansatz gives us three equations:}
				\begin{align*}
					-3 +2\lambda & = 7+3\mu \\
					-1+\lambda & = 3+\mu \\
					-3+2\lambda & = -1-\mu.
				\end{align*}
			    \lang{de}{Umschreiben der drei Gleichungen liefert das lineare Gleichungssystem}
			    \lang{en}{Simplfying the equations gives us the linear system:}
				\begin{align*}
					2\lambda-3\mu & = 10 \\
					\lambda-\mu & = 4 \\
					2\lambda+\mu & = 2.
				\end{align*}
				
			\step
			    \lang{de}{Abziehen der Hälfte der ersten Gleichung von der zweiten, und der ersten Gleichung von der dritten, ergibt}
			    \lang{en}{Subtracting half of the first equation from the second, and subtracting the first equation from the third gives:}
				\begin{align*}
					2\lambda-3\mu & = 10 \\
					\frac{1}{2}\mu & = -1 \\
					4\mu & = -8.
				\end{align*}	
				
			\step
			    \lang{de}{Die letzte Gleichung liefert nun $\mu = -2$, was ebenso die zweite Gleichung erfüllt. Aus der ersten Gleichung ergibt 
			    sich damit $\lambda = 2$. Damit gibt es also einen Schnittpunkt der beiden Geraden (so wie das die Aufgabe fordert).}
			    \lang{en}{The last equation gives us that $\mu = -2$, which is what the second equation tells us as well. From
			    the first equation we get that $\lambda=2$. Because we found a solution, there is an intersection point of the lines (just as the question asks).}
   
			\step
			    \lang{de}{Einsetzen von $\lambda=2$ bzw. $\mu=-2$ in $g$ bzw. $h$ ergibt}
			    \lang{en}{Substituting $\lambda=2$ and $\mu=-2$ into $g$ and $h$ gives us}
				\[\begin{pmatrix} -3\\-1\\-3\end{pmatrix} + 2\cdot \begin{pmatrix}2\\1\\2\end{pmatrix} = \begin{pmatrix} 1\\1\\1\end{pmatrix},\]
			    \lang{de}{bzw.}
			    \lang{en}{and}
				\[\begin{pmatrix} 7\\3\\-1\end{pmatrix} -2\cdot \begin{pmatrix}3\\1\\-1\end{pmatrix} = \begin{pmatrix} 1\\1\\1\end{pmatrix}.\]
			    \lang{de}{Der Schnittpunkt $S$ besitzt also die Koordinaten $S=(1;1;1)$.}
			    \lang{en}{The intersection point is $S=(1,1,1)$.}
		\end{incremental}
\end{tabs*}
\end{content}