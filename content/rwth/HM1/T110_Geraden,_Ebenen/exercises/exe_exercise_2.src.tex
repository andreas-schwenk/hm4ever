\documentclass{mumie.element.exercise}
%$Id$
\begin{metainfo}
  \name{
    \lang{de}{Ü02: Ebenengleichung}
    \lang{en}{}
  }
  \begin{description} 
 This work is licensed under the Creative Commons License Attribution 4.0 International (CC-BY 4.0)   
 https://creativecommons.org/licenses/by/4.0/legalcode 

    \lang{de}{Hier die Beschreibung}
    \lang{en}{}
  \end{description}
  \begin{components}
  \end{components}
  \begin{links}
  \end{links}
  \creategeneric
\end{metainfo}
\begin{content}
\title{
	\lang{de}{Ü02: Ebenengleichung}
	\lang{en}{Exercise 2}
}

\begin{block}[annotation]
	Im Ticket-System: \href{http://team.mumie.net/issues/9628}{Ticket 9628}
\end{block}

\begin{enumerate}
%
% Video
%
  \item Sei $E$ eine Ebene durch die Punkte $P_1 = (2;1;0)$,  $P_2 = (0;1;3)$ und  $P_3 = (2;0;3).$
    \begin{enumerate}[alph]
      \item Geben Sie mehrere Parameterdarstellungen (mit verschiedenen Orts- bzw. 
            Richtungsvektoren) für $E$ an.
      \item Wird die  Ebene $E$ durch
           $ E_1 = \{ \begin{pmatrix} -2\\3\\0 \end{pmatrix}+ \gamma \cdot \begin{pmatrix} 2\\-2\\3 \end{pmatrix}
                    + \delta \cdot \begin{pmatrix} -4\\2\\0 \end{pmatrix} \mid \gamma, \delta \in \R \} $
           beschrieben? Was muss dazu überprüft werden?
       
    \end{enumerate}        
\begin{tabs*}[\initialtab{0}\class{exercise}]
	\tab{\lang{de}{Antwort}\lang{en}{Answer}}
      \begin{enumerate}[alph]
        \item $E = \{ \begin{pmatrix} 2\\1\\0 \end{pmatrix}+ \alpha \cdot \begin{pmatrix} -2\\0\\3 \end{pmatrix}
                    + \beta \cdot \begin{pmatrix} 0\\-1\\3 \end{pmatrix} \mid \alpha, \beta \in \R \} $

        \item $E_1$ beschreibt $E$, da der Ortsvektor von $E_1$ in $E$ liegt. Zudem sind die Richtungsvektoren von $E_1$
              Linearkombinationen der Richtungsvektorn von $E$ und nicht Vielfache voneinander.       
      \end{enumerate}
      
     \tab{\lang{de}{Lösungsvideo 1}}
          \youtubevideo[500][300]{sv_NpRa99HM}\\  
          
\end{tabs*}        


  \item
	\lang{de}{Beschreiben Sie die Ebene $E$ durch den Punkt $P = (-1;-1;4)$ mit den Richtungsvektoren $\vec{v} = \begin{pmatrix} -1\\-3\\1\end{pmatrix}$ und $\vec{w} = \begin{pmatrix} 1\\2\\3\end{pmatrix}$ in Parameterform.}
	\lang{en}{Find an equation of the plane $E$ which goes through the point $P = (-1,-1,4)$ with direction vectors $\vec{v} = \begin{pmatrix} -1\\-3\\1\end{pmatrix}$ and $\vec{w} = \begin{pmatrix} 1\\2\\3\end{pmatrix}$.}  

\begin{tabs*}[\initialtab{0}\class{exercise}]
	\tab{\lang{de}{Antwort}\lang{en}{Answer}}
  		\lang{de}{Die Ebene $E$ ist gegeben durch}
  		\lang{en}{The plane $E$ is given by}
		\[E: \vec{x} = \begin{pmatrix} -1\\-1\\4\end{pmatrix} + \lambda\cdot\begin{pmatrix} -1\\-3\\1\end{pmatrix} + \mu\cdot\begin{pmatrix}1\\2\\3\end{pmatrix},\quad \lambda,\mu\in\R.\]
	
	\tab{\lang{de}{Ansatz}\lang{en}{Ansatz}}
  		\lang{de}{Die Ebene $E$ durch den Punkt $P$ mit den Richtungsvektoren $\vec{v}$ und $\vec{w}$ besitzt in Punkt-Richtungs-Form die Darstellung}
  		\lang{en}{The plane $E$ which goes through the point $P$ with direction vectors $\vec{v}$ and $\vec{w}$ can be written in point-direction form as}
  		\[E: \vec{x} = \vec{OP} + \lambda\cdot\vec{v} + \mu\cdot\vec{w},\quad \lambda,\mu\in\R.\]
	
	\tab{\lang{de}{Ebenengleichung}\lang{en}{Equation of the Plane}}
		\begin{incremental}[\initialsteps{1}]
		\step
		    \lang{de}{Der Ortsvektor $\vec{OP}$ zum Punkt $P$ ist gegeben durch}
		    \lang{en}{The position vector $\vec{OP}$ is given by}
			\[\vec{OP} = \begin{pmatrix} -1\\-1\\4\end{pmatrix}.\]
		\step
		    \lang{de}{Die Ebene $E$ ist damit gegeben durch}
		    \lang{en}{The plane $E$ can be written}
			\[E: \vec{x} = \begin{pmatrix} -1\\-1\\4\end{pmatrix} + \lambda\cdot\begin{pmatrix} -1\\-3\\1\end{pmatrix} + \mu\cdot\begin{pmatrix}1\\2\\3\end{pmatrix},\quad \lambda,\mu\in\R.\]
		\end{incremental}
\end{tabs*}

  \item
	\lang{de}{Bestimmen Sie die Ebene durch die drei Punkte $P = (1;1;1)$, $Q = (2;3;1)$ und $R = (-1;-1;2)$ in Parameterform.}
	\lang{en}{Let $E$ be the equation of the plane through the points $P = (1,1,1)$, $Q = (2,3,1)$, and $R = (-1,-1,2)$. Find an equation for $E$.}

\begin{tabs*}[\initialtab{0}\class{exercise}]
	\tab{\lang{de}{Antwort}\lang{en}{Answer}}	
		\lang{de}{Die Ebene $E$ ist zum Beispiel gegeben durch}
		\lang{en}{$E$ can be written}
  		\[E: \vec{x} = \begin{pmatrix} 1\\1\\1\end{pmatrix} + \lambda\cdot\begin{pmatrix} 1\\2\\0\end{pmatrix} + \mu\cdot \begin{pmatrix} -2\\-2\\1\end{pmatrix},\quad \lambda,\mu\in\R.\]
	\tab{\lang{de}{Ansatz}\lang{en}{Ansatz}}
  		\lang{de}{Die Ebene $E$ durch diese drei Punkte lässt sich in Drei-Punkte-Darstellung schreiben als}
  		\lang{en}{The plane $E$ can be written in three-point-representation as}
		\[E: \vec{x} = \vec{OP} + \lambda\cdot \vec{PQ} + \mu\cdot\vec{PR},\quad \lambda,\mu\in\R.\]
  	\tab{\lang{de}{Ebenengleichung}\lang{en}{Equation of the Plane}}
		\begin{incremental}[\initialsteps{1}]
			\step
			    \lang{de}{Der Ortsvektor $\vec{OP}$ zum Punkt $P$ ist gegeben durch}
			    \lang{en}{The position vector $\vec{OP}$ is given by}
				\[\vec{OP} = \begin{pmatrix} 1\\1\\1\end{pmatrix}.\]
			\step
			    \lang{de}{Der Vektor $\vec{PQ}$ berechnet sich aus}
			    \lang{en}{The vector $\vec{PQ}$ can be calculated to be}
				\[\vec{PQ} = \vec{OQ} - \vec{OP} = \begin{pmatrix} 2\\3\\1\end{pmatrix} - \begin{pmatrix} 1\\1\\1\end{pmatrix} = \begin{pmatrix} 1\\2\\0\end{pmatrix}.\]
			\step
			    \lang{de}{Analog ergibt sich}
			    \lang{en}{Analogously,}
				\[\vec{PR} = \vec{OR} - \vec{OP} = \begin{pmatrix} -1\\-1\\2\end{pmatrix} - \begin{pmatrix} 1\\1\\1\end{pmatrix} = \begin{pmatrix} -2\\-2\\1\end{pmatrix}.\]
			\step
			    \lang{de}{Also ist $E$ gegeben durch}
			    \lang{en}{Finally, $E$ can be written}
				\[E: \vec{x} = \begin{pmatrix} 1\\1\\1\end{pmatrix} + \lambda\cdot\begin{pmatrix} 1\\2\\0\end{pmatrix} + \mu\cdot \begin{pmatrix} -2\\-2\\1\end{pmatrix},\quad \lambda,\mu\in\R.\]
		\end{incremental}       
        
\end{tabs*}
\end{enumerate}
\end{content}