\documentclass{mumie.element.exercise}
%$Id$
\begin{metainfo}
  \name{
    \lang{de}{Ü03: Normalenvektor}
    \lang{en}{}
  }
  \begin{description} 
 This work is licensed under the Creative Commons License Attribution 4.0 International (CC-BY 4.0)   
 https://creativecommons.org/licenses/by/4.0/legalcode 

    \lang{de}{Hier die Beschreibung}
    \lang{en}{}
  \end{description}
  \begin{components}
  \end{components}
  \begin{links}
  \end{links}
  \creategeneric
\end{metainfo}
\begin{content}
\title{
	\lang{de}{Ü03: Normalenvektor}
	\lang{en}{Exercise 3}
}


\begin{block}[annotation]
	Im Ticket-System: \href{http://team.mumie.net/issues/9629}{Ticket 9629}
\end{block}

	\lang{de}{Es sei $g=\{ \begin{pmatrix}3\\2\end{pmatrix} + \lambda \begin{pmatrix}\frac{1}{2}\\-1\end{pmatrix} \vert \lambda \in \R \}$. 
	Geben Sie einen Normalenvektor für $g$ an und schreiben Sie $g$ in Koordinatenform um.
	}

\begin{tabs*}[\initialtab{0}\class{exercise}]
	\tab{\lang{de}{Antwort}\lang{en}{Answer}}	
		\begin{align*}
 		\vec{n} = \begin{pmatrix}1\\\frac{1}{2}\end{pmatrix}, \quad g= \{ \begin{pmatrix}x_1\\x_2\end{pmatrix}\in\R^2 \vert x_1+\frac{1}{2}x_2 =4 \}.
		\end{align*}
	\tab{\lang{de}{Lösung}}
		\begin{incremental}[\initialsteps{1}]
		\step
  		\lang{de}{Wir suchen zunächst einen Normalenvektor, d.h. einen Vektor, der orthogonal zum 
		Richtungsvektor $\vec{v} = \begin{pmatrix}\frac{1}{2}\\-1\end{pmatrix}$ von $g$ ist. }
		\step
		\lang{de}{
		Hier eignet sich zum Beispiel der Vektor $\vec{n}=\begin{pmatrix}1\\ \frac{1}{2} \end{pmatrix}$, da 
		$\vec{v}\bullet \vec{n} = \frac{1}{2}\cdot 1 - 1\cdot \frac{1}{2}=0$ gilt. Jedes Vielfache 
		dieses Vektors ist aber auch korrekt. }
		\step
		\lang{de}{Daher erhält man dann}
		\begin{align*}
 			g&= \left\{\begin{pmatrix}x_1\\x_2\end{pmatrix}\in \R^2 \vert \left( \begin{pmatrix}x_1\\x_2\end{pmatrix}-\begin{pmatrix}3\\2\end{pmatrix} \right)\bullet \begin{pmatrix}1\\ \frac{1}{2} \end{pmatrix} = 0  \right\}, \text{ bzw.} \\
 			g&= \left\{\begin{pmatrix}x_1\\x_2\end{pmatrix}\in \R^2 \vert  \begin{pmatrix}x_1\\x_2\end{pmatrix}\bullet \begin{pmatrix}1\\ \frac{1}{2} \end{pmatrix} = \begin{pmatrix}3\\2\end{pmatrix} \bullet\begin{pmatrix}1\\ \frac{1}{2} \end{pmatrix} =4 \right\}, \text{ bzw.} \\
			g&= \left\{ \begin{pmatrix}x_1\\x_2\end{pmatrix} \in \R^2 \vert x_1 + \frac{1}{2} x_2 =4 \right\}.
 		\end{align*}
		\end{incremental}
\end{tabs*}


\end{content}