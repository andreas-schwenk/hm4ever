\documentclass{mumie.element.exercise}
%$Id$
\begin{metainfo}
  \name{
    \lang{de}{Ü04: Ebenendarstellungen}
    \lang{en}{}
  }
  \begin{description} 
 This work is licensed under the Creative Commons License Attribution 4.0 International (CC-BY 4.0)   
 https://creativecommons.org/licenses/by/4.0/legalcode 

    \lang{de}{Hier die Beschreibung}
    \lang{en}{}
  \end{description}
  \begin{components}
  \end{components}
  \begin{links}
  \end{links}
  \creategeneric
\end{metainfo}
\begin{content}
\title{
	\lang{de}{Ü04: Ebenendarstellungen}
	\lang{en}{Exercise 4}
}


\begin{block}[annotation]
	Im Ticket-System: \href{http://team.mumie.net/issues/9630}{Ticket 9630}
\end{block}

\lang{de}{
\begin{enumerate}[alph]
%
% Video
%
  \item Stellen Sie eine Ebene durch die Punkte $P_1 = \begin{pmatrix} -1\\0\\2 \end{pmatrix}$,  
        $P_2 = \begin{pmatrix} 2\\1\\1 \end{pmatrix}$ und  $P_3 = \begin{pmatrix} 1\\-1\\2 \end{pmatrix}$ \\
        
        in Parameter- und in Normalendarstellung dar. \\
        
        Prüfen Sie, ob der Punkt $Q = \begin{pmatrix} -2\\3\\1 \end{pmatrix}$ in der Ebene liegt.

      \item Stellen Sie die Ebene, die durch $P = \begin{pmatrix} 3\\1\\0 \end{pmatrix}$ führt und senkrecht zu 
            $\vec{n}=\begin{pmatrix} 1\\2\\-1 \end{pmatrix}$ ist, in Parameter- und in Normalendarstellung dar. \\      


  \item Es sei eine Ebene $E$ in $\R^3$ gegeben durch
		$\quad E = \left\{ \begin{pmatrix}1\\2\\3\end{pmatrix} + s \begin{pmatrix}0\\1\\-1\end{pmatrix} + t \begin{pmatrix}1\\2\\1\end{pmatrix} \bigg\vert \, s,t \in \R  \right\}. $\\
        
		Bestimmen Sie einen Normalenvektor $\vec{n}$ zu $E$ und bestimmen Sie weiter die
		Koordinatenform der Ebene.
\end{enumerate} 
}

\begin{tabs*}[\initialtab{0}\class{exercise}]
	\tab{\lang{de}{Antwort}\lang{en}{Answer}}	
      \begin{enumerate}[alph]
        \item Parameterdarstellung: $\;E= \left\{ \begin{pmatrix}-1\\0\\2\end{pmatrix} + \alpha \begin{pmatrix}3\\1\\-1 \end{pmatrix} + \beta \begin{pmatrix}2\\-1\\0 \end{pmatrix}  \bigg\vert \alpha, \beta \in \R \right\},$\\
        
              Normalendarstellung: $\quad E= \left\{ \begin{pmatrix}x_1\\x_2\\x_3\end{pmatrix}\in \R^3 \bigg\vert x_1+2x_2+5x_3 =9 \right\}.$\\
              
              $Q$ liegt in $E$.
              
        \item Parameterdarstellung: $\;E= \left\{ \begin{pmatrix}3\\1\\0\end{pmatrix} + \alpha \begin{pmatrix}1\\0\\1 \end{pmatrix} + \beta \begin{pmatrix}0\\1\\2 \end{pmatrix}  \bigg\vert \alpha, \beta \in \R \right\},$\\
        
              Normalendarstellung: $\quad E= \left\{ \begin{pmatrix}x_1\\x_2\\x_3\end{pmatrix}\in \R^3 \bigg\vert x_1+2x_2-x_3 =5 \right\}.$
        
        \item $\vec{n}=\begin{pmatrix}3\\-1\\-1\end{pmatrix}, \quad\, E= \left\{ \begin{pmatrix}x_1\\x_2\\x_3\end{pmatrix}\in \R^3 \bigg\vert 3x_1-x_2-x_3 =-2 \right\}.$
       \end{enumerate}
       
     \tab{\lang{de}{Lösungsvideo a) + b)}}
         \youtubevideo[500][300]{uI9Pm8CNZnU}\\       
        
	\tab{\lang{de}{Lösung c)}}
		\begin{incremental}[\initialsteps{1}]
		\step
  		\lang{de}{
		Ein Normalenvektor zu $E$ ist gegeben durch das Kreuzprodukt der beiden Richtungsvektoren,
		also gilt}
		\[ \vec{n} = \begin{pmatrix}0\\1\\-1\end{pmatrix} \times \begin{pmatrix}1\\2\\1\end{pmatrix} = \begin{pmatrix}3\\-1\\-1\end{pmatrix}. \]
		\lang{de}{Jedes Vielfache dieses Vektors ist auch korrekt.}
		\step
		\lang{de}{Dann gilt weiterhin}
		\begin{align*}
  		E&= \left\{ \begin{pmatrix}x_1\\x_2\\x_3\end{pmatrix}\in\R^3 \bigg\vert \left( \begin{pmatrix}x_1\\x_2\\x_3\end{pmatrix} - \begin{pmatrix}1\\2\\3\end{pmatrix}\right) \bullet \begin{pmatrix}3\\-1\\-1\end{pmatrix} =0\right\}, \text{ bzw.} \\
		E&= \left\{ \begin{pmatrix}x_1\\x_2\\x_3\end{pmatrix}\in \R^3 \bigg\vert \begin{pmatrix}x_1\\x_2\\x_3\end{pmatrix}\bullet\begin{pmatrix}3\\-1\\-1\end{pmatrix}= \begin{pmatrix}1\\2\\3\end{pmatrix}\bullet \begin{pmatrix}3\\-1\\-1\end{pmatrix} = -2  \right\}, \text{ bzw.} \\
		E&= \left\{ \begin{pmatrix}x_1\\x_2\\x_3\end{pmatrix}\in \R^3 \bigg\vert 3x_1-x_2-x_3 =-2 \right\}.
		\end{align*}
		\end{incremental}
        
        
\end{tabs*}


\end{content}