\documentclass{mumie.element.exercise}
%$Id$
\begin{metainfo}
  \name{
    \lang{de}{Ü06: Lagebeziehungen}
    \lang{en}{}
  }
  \begin{description} 
 This work is licensed under the Creative Commons License Attribution 4.0 International (CC-BY 4.0)   
 https://creativecommons.org/licenses/by/4.0/legalcode 

    \lang{de}{Hier die Beschreibung}
    \lang{en}{}
  \end{description}
  \begin{components}
  \end{components}
  \begin{links}
  \end{links}
  \creategeneric
\end{metainfo}
\begin{content}
\title{
  \lang{de}{Ü06: Lagebeziehungen}
  \lang{en}{Exercise 6}
}

\begin{block}[annotation]
	Im Ticket-System: \href{http://team.mumie.net/issues/22816}{Ticket 22816}
\end{block}
\begin{enumerate}
  \item
  	\lang{de}{Wie liegen die beiden Ebenen}
  	\lang{en}{How do the two planes}
	\begin{align*}
		E: & \vec{x} =  \begin{pmatrix} 2\\3\\1\end{pmatrix} + \lambda \cdot\begin{pmatrix}  1\\1\\1\end{pmatrix} +    \mu\cdot\begin{pmatrix} 1\\0\\-1\end{pmatrix},\quad \lambda,\mu\in\R \\
		F: & \vec{x} =  \begin{pmatrix} 2\\-1\\-1 \end{pmatrix} +    \rho \cdot\begin{pmatrix} -2\\-1\\0\end{pmatrix} + \sigma\cdot\begin{pmatrix}  1\\3\\5\end{pmatrix},\quad \rho,\sigma\in\R
	\end{align*}
  	\lang{de}{zueinander?}
  	\lang{en}{lie relative to each other?}
  \item Berechnen Sie die Schnittmenge von 
        $\;E= \left\{ \begin{pmatrix}3\\-1\\0\end{pmatrix} + \alpha \begin{pmatrix}-1\\2\\3 \end{pmatrix} + \beta \begin{pmatrix}0\\0\\1 \end{pmatrix}  \mid \alpha, \beta \in \R \right\}\;$
        mit der Geraden $\;g= \left\{ \begin{pmatrix}1\\-1\\-1\end{pmatrix} + \lambda \begin{pmatrix}0\\2\\1 \end{pmatrix} \mid \lambda \in \R \right\}$, 
        indem Sie
    \begin{enumerate}
      \item die Parameterdarstellung von $E$ benutzen.
      \item $E$ in Normalendarstellung darstellen und diese nutzen.
    \end{enumerate}
\end{enumerate}  

\begin{tabs*}[\initialtab{0}\class{exercise}]
	\tab{\lang{de}{Antwort}\lang{en}{Answer}}
      \begin{enumerate}
        \item
          \lang{de}{Die Ebenen $E$ und $F$ liegen parallel, sind aber nicht identisch.}
          \lang{en}{The planes $E$ and $F$ are parallel, but not identical.}
          
        \item Schnittpunkt $\vec{s}=\begin{pmatrix} 1\\3\\1\end{pmatrix}.$
        
      \end{enumerate}
  		
    \tab{\lang{de}{Lösung 1}\lang{en}{Solution}}
    	\begin{incremental}[\initialsteps{1}]
			\step	
			    \lang{de}{Wir machen den Ansatz}
			    \lang{en}{Make the Ansatz:}
				\[\begin{pmatrix} 2\\3\\1\end{pmatrix} + \lambda \cdot\begin{pmatrix}  1\\1\\1\end{pmatrix} +    \mu\cdot\begin{pmatrix} 1\\0\\-1\end{pmatrix} = \begin{pmatrix} 2\\-1\\-1 \end{pmatrix} +    \rho \cdot\begin{pmatrix} -2\\-1\\0\end{pmatrix} + \sigma\cdot\begin{pmatrix}  1\\3\\5\end{pmatrix}.\]
			\step
			    \lang{de}{Dies führt auf das Gleichungssystem}
			    \lang{en}{This leads to the system}
				\begin{align*}
					2+\lambda+\mu ~& = 2-2\rho+\sigma \\
					3+\lambda ~& = -1-\rho+3\sigma \\
					1+\lambda-\mu ~& = -1+5\sigma
				\end{align*}
			    \lang{de}{mit drei Gleichungen für die vier Unbekannten $\lambda$, $\mu$, $\rho$ und $\sigma$.
				Umgeschrieben lautet das Gleichungssystem}
				\lang{en}{with three equations in four unknowns $\lambda$, $\mu$, $\rho$ and $\sigma$.
				Simplifying things we get the system:}
				\begin{align*}
					\lambda+\mu+2\rho-\sigma ~ & = 0 \\
					\lambda +\rho - 3\sigma~ & = -4 \\
					\lambda-\mu-5\sigma ~& = -2.
				\end{align*}
			\step
			    \lang{de}{Abziehen der ersten Gleichung von der zweiten bzw. dritten Gleichung ergibt}
			    \lang{en}{Subtracting the first equation from the second, and subtracting the first equation from the third gives:}
				\begin{align*}
					\lambda+\mu+2\rho-\sigma ~ & = 0 \\
					-\mu-\rho-2\sigma~ & = -4 \\
					-2\mu-2\rho-4\sigma~ & = -2.
				\end{align*}
			\step
			    \lang{de}{Abziehen des zweifachen der mittleren Gleichung von der letzten Gleichung ergibt}
			    \lang{en}{Subtracting two times the middle equation from the last equation results in}
				\begin{align*}
					\lambda+\mu+2\rho-\sigma ~ & = 0 \\
					-\mu-\rho-2\sigma ~& = -4 \\
					0~ & = 6.
				\end{align*}
			\step
			    \lang{de}{Die letzte Gleichung liefert den Widerspruch $0=6$. Damit ist das Gleichungssystem nicht lösbar. 
				Somit haben die Ebenen $E$ und $F$ keinen gemeinsamen Punkt, liegen also parallel.}
				\lang{en}{The last equation gives us a contradiction, $0=6$, hence the linear system is not solvable.
				The planes $E$ and $F$ do not have a single point in common, hence they are parallel. }
		\end{incremental}
        
  	\tab{\lang{de}{Alternativlösung 1}\lang{en}{Alternative solution}}
		\begin{incremental}[\initialsteps{1}]
			\step	
			    \lang{de}{Wir bestimmen die Normalenvektoren der beiden Ebenen $E$ und $F$:}
			    \lang{en}{We determine the normal vectors of the two planes $E$ and $F$:}
				\[\begin{pmatrix} 1\\1\\1\end{pmatrix} \times \begin{pmatrix}  1\\0\\-1\end{pmatrix} = \begin{pmatrix} -1\\2\\-1 \end{pmatrix}\]
                \[\begin{pmatrix} -2\\-1\\0\end{pmatrix} \times \begin{pmatrix}  1\\3\\5\end{pmatrix} = \begin{pmatrix} -5\\10\\-5 \end{pmatrix}\]
             \step
                \lang{de}{Wir sehen, dass die beiden Normalenvektoren Vielfache voneinander sind, sie also parallel liegen. Daher müssen die
                beiden Ebenen $E$ und $F$ parallel oder identisch sein.}
			    \lang{en}{We see that the two normal vectors are multiples of each other, so they are parallel. Therefore, the two planes
                $E$ and $F$ must be parallel or identical.}
             \step
                \lang{de}{Falls die beiden Ebenen $E$ und $F$ mindestens einen gemeinsamen Punkt haben, so haben sie alle Punkte gemeinsam und wären identisch.
                Haben sie keinen gemeinsamen Punkt, so sind sie parallel. Wir überprüfen, ob der Punkt $\begin{pmatrix} 2\\3\\1\end{pmatrix}$ auf der anderen 
                Ebene $F$ liegt:}
			    \lang{en}{If the two planes $E$ and $F$ have at least one common point, they have all the points in common and would be identical. 
                If they do not have a common point, they are parallel. We'll check that the point $(2;3;1)$ is at
                the other plane $F$:}
                \[\begin{pmatrix} 2\\3\\1 \end{pmatrix}=\begin{pmatrix} 2\\-1\\-1 \end{pmatrix} +    \rho \cdot\begin{pmatrix} -2\\-1\\0\end{pmatrix} + \sigma\cdot\begin{pmatrix}  1\\3\\5\end{pmatrix}\]
            \step    
                \lang{de}{Dies führt zum linearen Gleichungssystem:}
			    \lang{en}{This leads to the linear equation system:}
                \begin{align*}
					0 ~& = -2\rho+\sigma \\
					4 ~& = -\rho+3\sigma \\
					2 ~& = 5\sigma
				\end{align*}
                \lang{de}{mit drei Gleichungen für die beiden Unbekannten $\rho$ und $\sigma$. Aus der dritten Zeile ergibt sich:}
				\lang{en}{with three equations in two unknowns $\rho$ and $\sigma$. From the third line it follows:}
                \[\sigma~=\frac{2}{5}\]
                \lang{de}{Durch Einsetzen und Auflösen der zweiten Zeile ergibt sich $\rho~=-\frac{14}{5}$. Mit diesen Werten für $\sigma$ und $\rho$ ist die erste
                Gleichung aber nicht erfüllt. Der Punkt $(2;3;1)$ liegt also nicht auf der Ebene $F$. Die beiden Ebenen $E$ und $F$ sind parallel.}
				\lang{en}{Inserting and dissolving the second line results in $\rho~=-\frac{14}{5}$. However, the first equation is not satisfied with these values 
                for $\sigma$ and $\rho$. Thus, the point $(2;3;1)$ is not at the plane $F$. The two planes $E$ and $F$ are parallel.}
		\end{incremental}
        
      \tab{\lang{de}{Lösungsvideo 2}}
         \youtubevideo[500][300]{c1nbWuF2J24}\\
        
\end{tabs*}
\end{content}