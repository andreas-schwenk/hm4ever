\documentclass{mumie.element.exercise}
%$Id$
\begin{metainfo}
  \name{
    \lang{de}{Ü01: Geradengleichung}
    \lang{en}{}
  }
  \begin{description} 
 This work is licensed under the Creative Commons License Attribution 4.0 International (CC-BY 4.0)   
 https://creativecommons.org/licenses/by/4.0/legalcode 

    \lang{de}{Hier die Beschreibung}
    \lang{en}{}
  \end{description}
  \begin{components}
  \end{components}
  \begin{links}
  \end{links}
  \creategeneric
\end{metainfo}
\begin{content}
\title{
	\lang{de}{Ü01: Geradengleichung}
	\lang{en}{Exercise 1}
}

\begin{block}[annotation]
	Zwei-Punkte-Darstellung einer Geraden
\end{block}
\begin{block}[annotation]
  	Im Ticket-System: \href{http://team.mumie.net/issues/9627}{Ticket 9627}
\end{block}

\begin{enumerate}[alph]
%
% Video
%
  \item Geben Sie eine Darstellung der Geraden $g$ im $\R^3$ an, die durch $P_1 = (3;1;2)$ und $P_2 = (1;1;-1)$
        verläuft. \\
        Liegt $Q = (-1;1;-4)$ auf $g$?
        
  \item Geben Sie eine Darstellung der Geraden $g$ im $\R^4$ an, die durch $P_1 = (-1;0;1;2)$ und 
        $P_2 = (2;1;0;3)$ verläuft. 
        
\begin{tabs*}[\initialtab{0}\class{exercise}]
	\tab{\lang{de}{Antwort}\lang{en}{Answer}}
      \begin{enumerate}[alph]
        \item $g: \vec{x} = \begin{pmatrix} 3\\1\\2\end{pmatrix} + \lambda\cdot\begin{pmatrix} -2\\0\\3\end{pmatrix},\quad \lambda\in\R \quad$  und $Q$ liegt auf $g$, denn\\
        
              $\begin{pmatrix} -1\\1\\4\end{pmatrix} =  \begin{pmatrix} 3\\1\\2\end{pmatrix} + 2 \cdot \begin{pmatrix} -2\\0\\3\end{pmatrix}.$ 

        \item $g: \vec{x} = \begin{pmatrix} -1\\0\\1\\2\end{pmatrix} + \lambda\cdot\begin{pmatrix} 3\\1\\-1\\1\end{pmatrix},\quad \lambda\in\R.$
	  	
      \end{enumerate}
      
     \tab{\lang{de}{Lösungsvideo a) + b)}}
        \youtubevideo[500][300]{akeXeYrRE5A}\\  
\end{tabs*}        
%
  \item
	\lang{de}{Ermitteln Sie eine Gleichung der Geraden durch den Punkt $P = (1;2;1)$ mit Richtungsvektor $\vec{v} = \begin{pmatrix} -2\\3\\-1\end{pmatrix}$. Bestimmen Sie die Gerade in Parameterform.}  
	\lang{en}{Let $g$ be the line through the point $P = (1,2,1)$ with direction vector $\vec{v} = \begin{pmatrix} -2\\3\\-1\end{pmatrix}$. Find an equation for $g$.}

\begin{tabs*}[\initialtab{0}\class{exercise}]
	\tab{\lang{de}{Antwort}\lang{en}{Answer}}
		\lang{de}{Die Gerade $g$ ist gegeben durch}
		\lang{en}{The line $g$ is given by}
		\[g: \vec{x} = \begin{pmatrix} 1\\2\\1\end{pmatrix} + \lambda\cdot\begin{pmatrix} -2\\3\\-1\end{pmatrix},\quad \lambda\in\R.\]
	
	\tab{\lang{de}{Ansatz}\lang{en}{Ansatz}}
  		\lang{de}{Die Gerade $g$ durch den Punkt $P$ mit Richtungsvektor $\vec{v}$ besitzt in Punkt-Richtungs-Form die Darstellung}
		\lang{en}{The line $g$ can be written in point-direction form as}
		\[g: \vec{x} = \vec{OP} + \lambda\cdot\vec{v},\quad \lambda\in\R.\]
	
	\tab{\lang{de}{Geradengleichung}\lang{en}{The Equation of the Line}}
		\begin{incremental}[\initialsteps{1}]
		\step
		    \lang{de}{Der Ortsvektor $\vec{OP}$ zum Punkt $P$ ist gegeben durch}
		    \lang{en}{The position vector $\vec{OP}$ is given by}
			\[\vec{OP} = \begin{pmatrix} 1\\2\\1\end{pmatrix}.\]
		\step
		    \lang{de}{Also ist die Gerade $g$ gegeben durch}
		    \lang{en}{The line $g$ can be written}
			\[g: \vec{x} = \begin{pmatrix} 1\\2\\1\end{pmatrix} + \lambda\cdot\begin{pmatrix} -2\\3\\-1\end{pmatrix},\quad \lambda\in\R.\]
		\end{incremental}
\end{tabs*}

  \item
	\lang{de}{Berechnen Sie die Gerade durch die beiden Punkte $P = (5;3;-1)$ und $Q = (2;1;0)$. Bestimmen Sie die Gerade in Parameterform.}
	\lang{en}{Calculate the line through the points $P = (5,3,-1)$ and $Q = (2,1,0)$.}

\begin{tabs*}[\initialtab{0}\class{exercise}]
	\tab{\lang{de}{Antwort}\lang{en}{Answer}}
		\lang{de}{Eine mögliche Geradengleichung von $g$ ist gegeben durch}
		\lang{en}{The line through $P$ and $Q$ is i. e. given by}
	  	\[g: \vec{x} = \begin{pmatrix} 5\\3\\-1\end{pmatrix} + \lambda\cdot\begin{pmatrix} -3\\-2\\1\end{pmatrix},\quad \lambda\in\R.\]
	
	\tab{\lang{de}{Ansatz}\lang{en}{Ansatz}}
  		\lang{de}{Die Gerade $g$ durch die Punkte $P$ und $Q$ besitzt in Zwei-Punkt-Form die Darstellung}
  		\lang{en}{Let $g$ be the line through the points $P$ and $Q$. In two-point-form, $g$ can be written}
  		\[g: \vec{x} = \vec{OP} + \lambda\cdot\vec{PQ},\quad \lambda\in\R,\]
		\lang{de}{mit dem Stützvektor $\vec{OP}$ ($=$ ist der Ursprung) und dem Richtungsvektor $\vec{PQ}$.}
		\lang{en}{where the support vector is $\vec{OP}$ and the direction vector is $\vec{PQ}$.}
	
	\tab{\lang{de}{Geradengleichung}\lang{en}{The Equation of the Line}}
		\begin{incremental}[\initialsteps{1}]
			\step
			    \lang{de}{Der Ortsvektor $\vec{OP}$ zum Punkt $P$ ist gegeben durch}
			    \lang{en}{The position vector $\vec{OP}$ to the point $P$ is given by}
				\[\vec{OP} = \begin{pmatrix} 5\\3\\-1\end{pmatrix}.\]
			\step
			    \lang{de}{Der Vektor $\vec{PQ}$ berechnet sich aus}
			    \lang{en}{The vector $\vec{PQ}$ can be calculated from}
				\[\vec{PQ} = \vec{OQ} - \vec{OP} = \begin{pmatrix} 2\\1\\0\end{pmatrix} - \begin{pmatrix} 5\\3\\-1\end{pmatrix} = \begin{pmatrix} -3\\-2\\1\end{pmatrix}.\]
			\step
			    \lang{de}{Also ist die Gerade $g$ gegeben durch}
			    \lang{en}{The line can now be written}
				\[g: \vec{x} = \begin{pmatrix} 5\\3\\-1\end{pmatrix} + \lambda\cdot\begin{pmatrix} -3\\-2\\1\end{pmatrix},\quad \lambda\in\R.\]
		\end{incremental}
        
\end{tabs*}
\end{enumerate}
\end{content}