\documentclass{mumie.element.exercise}
%$Id$
\begin{metainfo}
  \name{
    \lang{de}{Ü08: Abstand}
    \lang{en}{}
  }
  \begin{description} 
 This work is licensed under the Creative Commons License Attribution 4.0 International (CC-BY 4.0)   
 https://creativecommons.org/licenses/by/4.0/legalcode 

    \lang{de}{Hier die Beschreibung}
    \lang{en}{}
  \end{description}
  \begin{components}
  \end{components}
  \begin{links}
  \end{links}
  \creategeneric
\end{metainfo}
\begin{content}
\title{
  \lang{de}{Ü08: Abstand}
  \lang{en}{Exercise 8}
}
\begin{block}[annotation]
	Im Ticket-System: \href{http://team.mumie.net/issues/22530}{Ticket 22530}
\end{block}


  	\lang{de}{Gegeben sei der Punkt $P=\begin{pmatrix}2\\8\end{pmatrix}$ und die Gerade
	\[g= \left\{\begin{pmatrix}1\\2\end{pmatrix} + \lambda \begin{pmatrix}2\\-4\end{pmatrix} \vert \lambda \in \R \right\}.\]
	Bestimmen Sie die Lagebeziehung von $g$ und $P$ und geben Sie den Abstand an.
	}

\begin{tabs*}[\initialtab{0}\class{exercise}]
	\tab{\lang{de}{Antwort}\lang{en}{Answer}}
  		\lang{de}{\begin{enumerate}
 		\item $P$ liegt nicht auf g.
 		\item dist$(g,P)= \frac{8}{\sqrt{5}}$.
		\end{enumerate}
		}
  		
  	\tab{\lang{de}{Lösung}\lang{en}{Solution}}

		\begin{incremental}[\initialsteps{1}]
			\step	
			    \lang{de}{Wir haben gesehen, dass für die Lagebeziehung von Gerade und Punkt im $\R^2$ genau
				zwei Fälle auftreten können. Entweder liegt der Punkt auf der Geraden oder nicht. Liegt
				$P$ auf $g$ so ist der Abstand gleich Null, während er sonst mit dem Lotfußpunktverfahren
				bestimmt werden kann. Also testen wir zunächst, ob $P$ auf $g$ liegt.}
			    
			\step
			    \lang{de}{Die Gleichung
				\[ \begin{pmatrix}2\\8\end{pmatrix} = \begin{pmatrix}1\\2\end{pmatrix} + \lambda \begin{pmatrix}2\\-4\end{pmatrix} \]
				liefert die Gleichungen $2=1+2\lambda$ und $8=2-4\lambda$, d.h. $\lambda= \frac{1}{2}$ und
				$\lambda=-\frac{3}{2}$, was nicht gleichzeitig erfüllt werden kann. Also liegt $P$ nicht auf $g$.
				}
			\step
			    \lang{de}{Wir bestimmen also das Lot von $P$ auf $g$ mit der Formel
				\[ \vec{OL} = \begin{pmatrix}1\\2\end{pmatrix} + r_0 \begin{pmatrix}2\\-4\end{pmatrix}, \]
				mit
				\begin{align*}
 				r_0 &= \frac{\left( \begin{pmatrix}2\\8\end{pmatrix}- \begin{pmatrix}1\\2\end{pmatrix}\right) \bullet \begin{pmatrix}2\\-4\end{pmatrix} }{\| \begin{pmatrix}2\\-4\end{pmatrix}\|^2} \\
 				&= \frac{\begin{pmatrix}1\\6\end{pmatrix}\bullet \begin{pmatrix}2\\-4\end{pmatrix}}{2^2 + (-4)^2} \\
 				&= \frac{2+6\cdot (-4)}{4+16} = - \frac{11}{10}.
				\end{align*}
				}
			\step
			    \lang{de}{Dann ist also 
				\[ \vec{OL} = \begin{pmatrix}1\\2\end{pmatrix} - \frac{11}{10} \begin{pmatrix}2\\-4\end{pmatrix} = \begin{pmatrix}-\frac{6}{5}\\\frac{32}{5}\end{pmatrix}
				\]
				und der Abstand ergibt sich aus
				\[ \| \begin{pmatrix}-\frac{6}{5}\\\frac{32}{5}\end{pmatrix} - \begin{pmatrix}2\\8\end{pmatrix} \| = \sqrt{\frac{16^2+8^2}{5^2}}=\sqrt{\frac{64}{5}}=\frac{8}{\sqrt{5}}. \]}
			
		\end{incremental}
\end{tabs*}
\end{content}