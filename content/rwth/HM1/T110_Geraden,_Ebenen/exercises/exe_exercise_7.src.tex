\documentclass{mumie.element.exercise}
%$Id$
\begin{metainfo}
  \name{
    \lang{de}{Ü07: Lagebeziehungen}
    \lang{en}{}
  }
  \begin{description} 
 This work is licensed under the Creative Commons License Attribution 4.0 International (CC-BY 4.0)   
 https://creativecommons.org/licenses/by/4.0/legalcode 

    \lang{de}{Hier die Beschreibung}
    \lang{en}{}
  \end{description}
  \begin{components}
  \end{components}
  \begin{links}
  \end{links}
  \creategeneric
\end{metainfo}
\begin{content}
\title{
	\lang{de}{Ü07: Lagebeziehungen}
	\lang{en}{Exercise 7}
}


\begin{block}[annotation]
  	Im Ticket-System: \href{http://team.mumie.net/issues/9633}{Ticket 9633}
\end{block}

	\begin{enumerate}
		\item[a)]
		    \lang{de}{Bestimmen Sie $t\in\R$ so, dass $P = (1;t;3)$ in der Ebene}
		    \lang{en}{Find the value of the parameter $t\in\R$ so that $P = (1,t,3)$ lies in the plane: }
			\[E: \vec{x} = \begin{pmatrix} 2\\1\\3\end{pmatrix} + \lambda\cdot \begin{pmatrix} 1\\1\\-1\end{pmatrix} + \mu\cdot \begin{pmatrix} -2\\3\\1\end{pmatrix},\quad \lambda,\mu\in\R\lang{en}{.}\]
		    \lang{de}{liegt.}
		    \lang{en}{}
	    
		\item[b)]
		    \lang{de}{Berechnen Sie die Schnittgerade der beiden Ebenen}
		    \lang{en}{Calculate the intersecting line of the two planes:}
			\begin{align*}
				E_1: & \vec{x} =  \begin{pmatrix} 1\\2\\-1\end{pmatrix} + \lambda \cdot\begin{pmatrix}  1\\2\\0\end{pmatrix} +    \mu\cdot\begin{pmatrix} -1\\-2\\3\end{pmatrix},\quad \lambda,\mu\in\R \\
				E_2: & \vec{x} =  \begin{pmatrix} 3\\0\\1 \end{pmatrix} +    \rho \cdot\begin{pmatrix} -2\\1\\1\end{pmatrix} + \sigma\cdot\begin{pmatrix}  4\\ 1\\4\end{pmatrix},\quad \rho,\sigma\in\R. 
			\end{align*}
	\end{enumerate}

\begin{tabs*}[\initialtab{0}\class{exercise}]
  	\tab{\lang{de}{Antwort}\lang{en}{Answer}}
  		\lang{de}{a) Damit $P$ in der Ebene $E$ liegt, muss $t= 5$ gelten.}
  		\lang{en}{a) In order for $P$ to lie in the plane $E$, $t$ must be equal to $5$.}	
  		\lang{de}{b) Die Schnittgerade der beiden Ebenen $E_1$ und $E_2$ ist gegeben durch}
  		\lang{en}{b) The intersecting line of the planes $E_1$ and $E_2$ is:}
		\[g: \vec{x} = \begin{pmatrix} \frac{3}{5}\\\frac{6}{5}\\\frac{11}{5} \end{pmatrix} + t \cdot\begin{pmatrix} \frac{6}{5}\\\frac{12}{5}\\\frac{27}{5} \end{pmatrix},\quad t\in\R.\]

  	\tab{\lang{de}{Lösung a)}\lang{en}{Solution a)}}
		\begin{incremental}[\initialsteps{1}]
			\step	
			    \lang{de}{Wir machen den Ansatz}
			    \lang{en}{Make the Ansatz:}
				\[\begin{pmatrix} 1\\t\\3\end{pmatrix} = \begin{pmatrix} 2\\1\\3\end{pmatrix} + \lambda\cdot \begin{pmatrix} 1\\1\\-1\end{pmatrix} + \mu\cdot \begin{pmatrix} -2\\3\\1\end{pmatrix}.\]
			\step
			    \lang{de}{Dies führt auf das Gleichungssystem}
			    \lang{en}{This leads to the system}
				\begin{align*}
					1 ~& = 2 + \lambda - 2\mu \\
					t ~& = 1 + \lambda + 3\mu \\
					3 ~& = 3 - \lambda + \mu
				\end{align*}
			    \lang{de}{mit drei Gleichungen für die drei Unbekannte $\lambda$, $\mu$ und $t$.
				Umgeschrieben lautet das Gleichungssystem}
				\lang{en}{with three equations in three unknowns $\lambda$, $\mu$, and $t$.
				Simplifying the system we get:}
				\begin{align*}
					\lambda - 2\mu  ~& = -1 \\
					\lambda + 3\mu -t ~& = -1 \\
					 - \lambda + \mu ~& = 0.
				\end{align*}
			\step
			    \lang{de}{Aus der dritten Gleichung erhalten wir $\lambda = \mu$ und damit aus der ersten Gleichung $-\mu = -1$, also $\lambda = \mu = 1$. 
				Damit liefert die zweite Gleichung}
				\lang{en}{The third equations gives us that $\lambda=\mu$ and from the first equation, $-\mu=-1$, hence $\lambda = \mu = 1$.
				The second equation now gives us}
				\[1+3-t = -1,\]
			    \lang{de}{also}
			    \lang{en}{which means that the parameter is}
			    $t = 5$.	
		\end{incremental}
	
	\tab{\lang{de}{Lösung b)}\lang{en}{Solution b)}}
		\begin{incremental}[\initialsteps{1}]
			\step
			    \lang{de}{Wir machen den Ansatz}
			    \lang{en}{Make the Ansatz:}
				\[\begin{pmatrix} 1\\2\\-1\end{pmatrix} + \lambda \cdot\begin{pmatrix}  1\\2\\0\end{pmatrix} +    \mu\cdot\begin{pmatrix} -1\\-2\\3\end{pmatrix} = \begin{pmatrix} 3\\0\\1 \end{pmatrix} +    \rho \cdot\begin{pmatrix} -2\\1\\1\end{pmatrix} + \sigma\cdot\begin{pmatrix}  4\\ 1\\4\end{pmatrix}.\]
			\step
			    \lang{de}{Dieser Ansatz führt auf das Gleichungssystem}
			    \lang{en}{This Ansatz leads to the system}
				\begin{align*}
					1+\lambda - \mu & = 3-2\rho + 4\sigma \\
					2 + 2\lambda - 2\mu & = \rho + \sigma \\
					-1 + 3\mu & = 1 + \rho + 4\sigma
				\end{align*}
			    \lang{de}{mit drei Gleichungen für vier Unbekannte. Umgeschrieben lautet es}
			    \lang{en}{with three equations in four unknowns. Simplifying the system gets us to}
				\begin{align*}
					\lambda - \mu + 2\rho - 4\sigma & = 2 \\
					2\lambda - 2\mu -\rho - \sigma & = -2 \\
					3\mu -\rho - 4\sigma & = 2.
				\end{align*}
			\step
			    \lang{de}{Wir formen nun dieses Gleichungssystem schrittweise um, 
				indem wir zunächst das Zweifache der ersten Zeile von der zweiten abziehen und dann die zweite und dritte Gleichung tauschen:}
				\lang{en}{We transform this system step-by-step: taking mulyiplying the first equation by 
				two and subtracting it from the second equation; take the second equation and swap it with the third equation we get:}
				\begin{align*}
					\lambda - \mu + 2\rho - 4\sigma & = 2 \\
					3\mu -\rho - 4\sigma & = 2 \\
					 -5\rho + 7\sigma & = -6.		
				\end{align*}
			    \lang{de}{Mit $\sigma = t$ erhalten wir nun $\rho = \frac{6}{5} + \frac{7}{5}t$ aus der dritten Gleichung. Die mittlere Gleichung liefert dann}
			    \lang{en}{If we let $\sigma=t$ we get $\rho = \frac{6}{5} + \frac{7}{5}t$ from the third equation. The middle equation then gives us that}
				\[\mu = \frac{16}{15} + \frac{9}{5} t.\]
			    \lang{de}{Die erste Gleichung ergibt dann schließlich }
			    \lang{en}{The first equation finally gives us:}
				\[\lambda = \frac{2}{3} + 3t.\]
			\step
			    \lang{de}{Einsetzen dieser Werte für $\rho$ und $\sigma$ in $E_2$ ergibt die Geradengleichung der Schnittgeraden $g$:}
			    \lang{en}{Substituting the values for $\rho$ and $\sigma$ into $E_2$ results in the equation of the intersecting line $g$:}
				\[g: \vec{x} = \begin{pmatrix} 3\\0\\1 \end{pmatrix} +    \bigg(\frac{6}{5} + \frac{7}{5}t\bigg) \cdot\begin{pmatrix} -2\\1\\1\end{pmatrix} + t\cdot\begin{pmatrix}  4\\ 1\\4\end{pmatrix} = \begin{pmatrix} \frac{3}{5}\\\frac{6}{5}\\\frac{11}{5} \end{pmatrix} + t \cdot\begin{pmatrix} \frac{6}{5}\\\frac{12}{5}\\\frac{27}{5} \end{pmatrix},\quad t\in\R.\]
			\step
			    \lang{de}{Die gleiche Gerade erhalten wir natürlich, indem wir $\lambda$ und $\mu$ in $E_1$ einsetzen:}
			    \lang{en}{The same line could have been calculated by substituting $\lambda$ and $\mu$ into $E_1$:}
				\[g: \vec{x} = \begin{pmatrix} 1\\2\\-1\end{pmatrix} + \bigg(\frac{2}{3} + 3t\bigg) \cdot\begin{pmatrix}  1\\2\\0\end{pmatrix} +    \bigg(\frac{16}{15} + \frac{9}{5} t\bigg)\cdot\begin{pmatrix} -1\\-2\\3\end{pmatrix} = \begin{pmatrix} \frac{3}{5} \\\frac{6}{5}\\\frac{11}{5} \end{pmatrix} + t \cdot\begin{pmatrix} \frac{6}{5}\\\frac{12}{5}\\\frac{27}{5} \end{pmatrix},\quad t\in\R.\]
		\end{incremental}
\end{tabs*}
\end{content}