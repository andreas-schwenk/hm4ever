\documentclass{mumie.problem.gwtmathlet}
%$Id$
\begin{metainfo}
  \name{
    \lang{de}{Aufgabe 3}
    \lang{en}{input numbers}
  }
  \begin{description} 
 This work is licensed under the Creative Commons License Attribution 4.0 International (CC-BY 4.0)   
 https://creativecommons.org/licenses/by/4.0/legalcode 

    \lang{de}{Die Beschreibung}
    \lang{en}{}
  \end{description}
  \corrector{system/problem/GenericCorrector.meta.xml}
  \begin{components}
    \component{js_lib}{system/problem/GenericMathlet.meta.xml}{gwtmathlet}
  \end{components}
  \begin{links}
  \end{links}
  \creategeneric
\end{metainfo}
\begin{content}
\begin{block}[annotation]
	Im Ticket-System: \href{https://team.mumie.net/issues/22764}{Ticket 22764}
\end{block}
\usepackage{mumie.genericproblem}

\lang{de}{\title{Aufgabe 3}}
\lang{en}{\title{Problem 3}}


\begin{block}[annotation]
	Im Ticket-System: \href{http://team.mumie.net/issues/9645}{Ticket 9645}
\end{block}




\begin{problem}

	\begin{question}
		\lang{de}{\text{
			Gegeben sei die Gerade
			$ g= \left\{\begin{pmatrix}\var{a1}\\ \var{a2} \end{pmatrix}+ \lambda \begin{pmatrix}\var{b1}\\ \var{b2}\end{pmatrix} \vert \lambda \in \R \right\}. $
			Bestimmen Sie einen Normalenvektor $\vec{n}$ von $g$.
 			}
		}
	\explanation[edited]{Ein Normalenvektor einer Gerade steht senkrecht zu dieser und damit auch zu allen möglichen Richtungsvektoren der Gerade.
    Ein Normalenvektor der Geraden $g$ ist also ein zum gegebenen Richtungsvektor senkrecht stehender Vektor, so dass das Skalarprodukt des Richtungsvektors mit
    dem gesuchten Normalenvektor Null ergibt.}
		
		\type{input.function}
		\field{real}
		
		
		\begin{variables}
			\randint{a1}{-15}{15}
			\randint{a2}{-15}{15}
			
			\randadjustIf{a1,a2}{a1^2+a2^2=0}
			\randint{b1}{-9}{9}
			\randint{b2}{-9}{9}
			
			\randadjustIf{b1,b2}{b^2+b2^2=0}
			
			\function[calculate]{n1}{-b2}
			\function[calculate]{n2}{b1}
		\end{variables}
		
		\begin{answer}
			\text{$n_1 = $}
			\solution{n1}
			\inputAsFunction{x}{nx}
		\end{answer}
		
		\begin{answer}
			\text{$n_2 = $}
			\solution{n2}
			\inputAsFunction{x}{ny}
			\checkFuncForZero{n1*ny-nx*n2}{-10}{10}{100}
		\end{answer}
		
	\end{question}
	
	\begin{question}
		\lang{de}{\text{
			Gegeben sei die Ebene 
			$ E= \left\{ \begin{pmatrix}\var{a1}\\\var{a2}\\\var{a3}\end{pmatrix} + r \begin{pmatrix}\var{b1}\\\var{b2}\\\var{b3}\end{pmatrix} + s\begin{pmatrix}\var{c1}\\\var{c2}\\\var{c3}\end{pmatrix} \vert \, r,s\in \R \right\}. $
			Bestimmen Sie einen Normalenvektor $\vec{n} = \begin{pmatrix} n_1 \\ n_2 \\ n_3 \end{pmatrix}$ der Ebene und geben Sie $E$ in 
			Koordinatenform an, das heißt geben Sie Koordinaten $z_1, z_2, z_3, z_4, z_5$ und $z_6$ an, so dass \\
			$E = \left\{ x\in \R^3 \vert z_1 x_1 + z_2 x_2 +z_3 x_3 =  z_1  z_4 + z_2 z_5 + z_3 z_6   \right\}$.
			}
		}
	\explanation[edited]{Ein Normalenvektor einer Ebene steht senkrecht zu dieser. Damit steht ein Normlenvektor einer Ebene auch senkrecht
    zu den durch die Ebene aufgespannten Richtungsvektoren. Man erhält einen zu den Richtungsvektoren senkrecht stehenden Vektor, indem man
    das Vektorprodukt der beiden Richtungsvektoren bildet.}
		
		\type{input.function}
		\field{real}
		
		
		\begin{variables}
			
			\randint{a1}{-15}{15}
			\randint{a2}{-15}{15}
			\randint{a3}{-15}{15}
			\randadjustIf{a1,a2,a3}{a1^2+a2^2+a3^2=0}
			\randint{b1}{-9}{9}
			\randint{b2}{-9}{9}
			\randint{b3}{-9}{9}
			\randadjustIf{b1,b2,b3}{b1^2+b2^2+b3^2=0}
			\randint{c1}{-9}{9}
			\randint{c2}{-9}{9}
			\randint{c3}{-9}{9}
			\randadjustIf{c1,c2,c3}{c1^2+c2^2+c3^2=0}
			\randadjustIf{c1,c2,c3}{(b1*c2-c1*b2)^2+(b2*c3-c2*b3)^2+(b1*c3-b3*c1)^2=0}
			\function[calculate]{n1}{b2*c3-b3*c2}
			\function[calculate]{n2}{b3*c1-b1*c3}
			\function[calculate]{n3}{b1*c2-b2*c1}
			\function[calculate]{z1}{n1}
			\function[calculate]{z2}{n2}
			\function[calculate]{z3}{n3}
			\function[calculate]{z4}{a1}
			\function[calculate]{z5}{a2}
			\function[calculate]{z6}{a3}
		\end{variables}
		
		\begin{answer}
			\text{$n_1 = $}
			\solution{n1}
			\inputAsFunction{x}{nx}
		\end{answer}
		
		\begin{answer}
			\text{$n_2 = $}
			\solution{n2}
			\inputAsFunction{x}{ny}
		\end{answer}
		
		\begin{answer}
			\text{$n_3 = $}
			\solution{n3}
			\inputAsFunction{x}{nz}
		\end{answer}
		\begin{answer}
			\text{$z_1 = $}
			\solution{z1}
			\inputAsFunction{x}{z11}
		\end{answer}
		
		\begin{answer}
			\text{$z_2 = $}
			\solution{z2}
			\inputAsFunction{x}{z22}
		\end{answer}
		
		\begin{answer}
			\text{$z_3 = $}
			\solution{z3}
			\inputAsFunction{x}{z33}
			
		\end{answer}
		
		\begin{answer}
			\text{$z_4 = $}
			\solution{z4}
			\inputAsFunction{x}{z44}
		\end{answer}
		
		\begin{answer}
			\text{$z_5 = $}
			\solution{z5}
			\inputAsFunction{x}{z55}
			
		\end{answer}
		\begin{answer}
			\text{$z_6 = $}
			\solution{z6}
			\inputAsFunction{x}{z66}
			\checkFuncForZero{(z44*n1+z55*n2+z66*n3-(n1*z4+n2*z5+n3*z6))^2+(n1*z22-z11*n2)^2+(n2*z33-z22*n3)^2+(n1*z33-n3*z11)^2+(n1*ny-nx*n2)^2+(n2*nz-ny*n3)^2+(n1*nz-n3*nx)^2}{-10}{10}{100}
            	\explanation{Die Komponenten eines Normalenvektors entsprechen den Vorfaktoren der $x_i$ der Koordinatengleichung der Ebene. Die rechte Seite der Koordinatengleichung der Ebene
    ergibt sich durch Einsetzen eines auf der Ebene liegenden Punktes, der die Koordinatengleichung erfüllen muss.}
		\end{answer}
	\end{question}
	
\end{problem}

\embedmathlet{gwtmathlet}

\end{content}