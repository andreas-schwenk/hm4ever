\documentclass{mumie.problem.gwtmathlet}
%$Id$
\begin{metainfo}
  \name{
    \lang{de}{A05: Punktprobe}
    \lang{en}{input function}
  }
  \begin{description} 
 This work is licensed under the Creative Commons License Attribution 4.0 International (CC-BY 4.0)   
 https://creativecommons.org/licenses/by/4.0/legalcode 

    \lang{de}{Die Beschreibung}
    \lang{en}{description}
  \end{description}
  \corrector{system/problem/GenericCorrector.meta.xml}
  \begin{components}
    \component{js_lib}{system/problem/GenericMathlet.meta.xml}{gwtmathlet}
  \end{components}
  \begin{links}
  \end{links}
  \creategeneric
\end{metainfo}
\begin{content}
\usepackage{mumie.genericproblem}

\lang{de}{\title{A05: Punktprobe}}
\lang{en}{\title{Problem 5}}


\begin{block}[annotation]
	Im Ticket-System: \href{http://team.mumie.net/issues/9659}{Ticket 9659}
\end{block}

\begin{problem}

	\begin{question}
		\lang{de}{\text{
			Gegeben sei die Ebene 
			$ E= \left\{\begin{pmatrix}x_1 \\ x_2\\ x_3 \end{pmatrix} \in \R^3 \,\,\vert \,\, \var{f} = \var{d} \right\} $
			und der Punkt $P=\begin{pmatrix} \var{p_1} \\ \var{p_2} \\ \var{p_3} \end{pmatrix}$. 
		}
		}
		\explanation[edited]{Überprüfen Sie durch Einsetzen, ob der Punkt $P$ die Koordinatengleichung der Ebene $E$ erfüllt. Wenn ja, dann liegt der Punkt $P$ auf
        der Ebene $E$; ansonsten nicht.}
		
		\type{mc.yesno}
				
		
		\begin{variables}
            
			\randint{n_1}{-9}{9}
			\randint{n_2}{-9}{9}
			\randint{n_3}{-9}{9}
			\randadjustIf{n_1,n_2,n_3}{n_1^2+n_2^2+n_3^2=0}
			
            \function[normalize]{f}{n_1*x_1+n_2*x_2+n_3*x_3}
			\randint[Z]{p_1}{-10}{10}
			\randint[Z]{p_2}{-10}{10}
			\randint[Z]{p_3}{-10}{10}
			\randint[Z]{d0}{-5}{5}
            \randint{w}{0}{1}
			\function[calculate]{d}{p_1*n_1+p_2*n_2+p_3*n_3+w*d0}

		\end{variables}
		
		\begin{choice}
			\text{Liegt $P$ in der Ebene $E$?}
			\solution{compute}
			\iscorrect{w}{=}{0}
		\end{choice}
		
		
	\end{question}
	
	\begin{question}
		\lang{de}{\text{
			Gegeben sei die Gerade 
			$ g= \{ \begin{pmatrix} \var{u_1} \\ \var{u_2} \end{pmatrix}+s\begin{pmatrix} \var{v_1} \\ \var{v_2} \end{pmatrix} \vert s\in \R \} $ und der Punkt
			$P= \begin{pmatrix} \var{p_1} \\ \var{p_2} \end{pmatrix}$. \\
		}
		}
		\explanation[edited]{Überprüfen Sie durch Gleichsetzen des Punktes $P$ mit der Geradengleichung $g$, ob der Punkt $P$ auf der Geraden 
        $g$ liegt. Falls das entstehende Gleichungssystem lösbar ist, liegt der Punkt $P$ auf
        der Geraden; ansonsten nicht.}
		
		\type{mc.yesno}
		\field{rational}
		
		\begin{variables}
			
			\randint{sz}{-3}{3}
			\randint[Z]{sn}{-4}{4}
			
			\randint{u_1}{-9}{9}
			\randint{u_2}{-9}{9}
			
			\randint[Z]{v_1}{-9}{9}
			\randint[Z]{v_2}{-9}{9}

			\randint[Z]{d}{-5}{5}
            \randint{w}{0}{1}
			\function[calculate]{p_1}{u_1+sz*v_1/sn +w*d}
			\function[calculate]{p_2}{u_2+sz*v_2/sn }

		\end{variables}
		
		
		
		\begin{choice}
			\text{Liegt $P$ auf der Geraden $g$?}
			\solution{compute}
			\iscorrect{w}{=}{0}
		\end{choice}
		
		
		
	\end{question}
	
\end{problem}

\embedmathlet{gwtmathlet}




\end{content}