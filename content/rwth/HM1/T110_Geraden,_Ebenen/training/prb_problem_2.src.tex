\documentclass{mumie.problem.gwtmathlet}
%$Id$
\begin{metainfo}
  \name{
    \lang{de}{A02: Ebenengleichung}
    \lang{en}{input numbers}
  }
  \begin{description} 
 This work is licensed under the Creative Commons License Attribution 4.0 International (CC-BY 4.0)   
 https://creativecommons.org/licenses/by/4.0/legalcode 

    \lang{de}{Die Beschreibung}
    \lang{en}{}
  \end{description}
  \corrector{system/problem/GenericCorrector.meta.xml}
  \begin{components}
    \component{js_lib}{system/problem/GenericMathlet.meta.xml}{mathlet}
  \end{components}
  \begin{links}
  \end{links}
  \creategeneric
\end{metainfo}
\begin{content}
\usepackage{mumie.genericproblem}

\lang{de}{\title{A02: Ebenengleichung}}
\lang{en}{\title{Problem 2}}

\begin{block}[annotation]
	Richtungvektoren einer Ebene
\end{block}
\begin{block}[annotation]
	Im Ticket-System: \href{http://team.mumie.net/issues/9641}{Ticket 9641}
\end{block}

\begin{problem}

%Question 1 of 1
% \begin{question}
% 	\lang{de}{
%      	\text{Geben Sie die Komponenten möglicher Vektoren $\vec{v}$ und $\vec{w}$ der Ebene $E: \vec{x} = \vec{a} + \lambda\cdot\vec{v}+\mu\cdot\vec{w}$, $\lambda,\mu\in\R$ \\
%       	durch die Punkte $P = (\var{Px};\var{Py};\var{Pz})$, $Q = (\var{Qx};\var{Qy};\var{Qz})$ und $R = (\var{Rx};\var{Ry};\var{Rz})$ an.}
%     	\explanation[edited]{Der Vektor $\vec{a}$ ist der Ortsvektor eines Punktes auf der Ebene, die Vektoren $\vec{v}$ und $\vec{w}$ zwei nichtparallele Richtungsvektoren.}
%    	}
% 	\lang{en}{
% 		\text{Let $E$ be the plane $E: \vec{x} = \vec{a} + \lambda\cdot\vec{v}+\mu\cdot\vec{w}$, $\lambda,\mu\in\R$ that goes through the points $P = (\var{Px},\var{Py},\var{Pz})$, $Q = (\var{Qx},\var{Qy},\var{Qz})$, and $R = (\var{Rx},\var{Ry},\var{Rz})$. 
% 		Find possible components of the vectors $\vec{v}$ and $\vec{w}$.}
% 		\explanation[edited]{The vector $\vec{a}$ is the position vector of a point in the plane; the vectors $\vec{v}$ and $\vec{w}$ are two non-parallel direction vectors.}
% 	}
% 	\begin{variables}
% 		\randint{Px}{-5}{5}
% 		\randint{Py}{-5}{5}
% 		\randint{Pz}{-5}{5}
% 		\randint{Qx}{-5}{5}
% 		\randint{Qy}{-5}{5}
% 		\randint{Qz}{-5}{5}
% 		\randint{Rx}{-5}{5}
% 		\randint{Ry}{-5}{5}
% 		\randint{Rz}{-5}{5}
% 		\function[calculate]{ax}{Px}
% 		\function[calculate]{ay}{Py}
% 		\function[calculate]{az}{Pz}
% 		\function[calculate]{vx}{Qx-Px}
% 		\function[calculate]{vy}{Qy-Py}
% 		\function[calculate]{vz}{Qz-Pz}
% 		\function[calculate]{wx}{Rx-Px}
% 		\function[calculate]{wy}{Ry-Py}
% 		\function[calculate]{wz}{Rz-Pz}
% 		\function[calculate]{nx}{vy*wz-vz*wy}
% 		\function[calculate]{ny}{vz*wx-vx*wz}
% 		\function[calculate]{nz}{vx*wy-vy*wx}
% 		\randadjustIf{Px,Py,Pz,Qx,Qy,Qz,Rx,Ry,Rz}{((Qx-Px)*(Ry-Py)-(Qy-Py)*(Rx-Px))^2+((Qy-Py)*(Rz-Pz)-(Qz-Pz)*(Ry-Py))^2+((Qz-Pz)*(Rx-Px)-(Qx-Px)*(Rz-Pz))^2 = 0}
% 		\randadjustIf{Px,Py,Pz,Qx,Qy,Qz,Rx,Ry,Rz}{Qx=Px and Qy=Py and Qz=Pz}
% 		\randadjustIf{Px,Py,Pz,Qx,Qy,Qz,Rx,Ry,Rz}{Rx=Px and Ry=Py and Rz=Pz}
%         \function{fx}{vx}
%    	 	\function{fy}{vy}
% 		\function{fz}{vz}
%         \function{gx}{wx}
%    	 	\function{gy}{wy}
% 		\function{gz}{wz}
% 	\end{variables}
% 	\type{input.function}
% 	\field{real}
% 	\begin{answer}
% 		\text{$v_x = $}
% 		\solution{fx}
%         \inputAsFunction{x}{kx}
%     \end{answer}
%     \begin{answer}
%         \text{$v_y = $}
%         \solution{fy}
%         \inputAsFunction{x}{ky}
%     \end{answer}
%     \begin{answer}
%         \text{$v_z = $}
%         \solution{fz}
%         \inputAsFunction{x}{kz}
%     \end{answer}
% 	\begin{answer}
% 		\text{$w_x = $}
% 		\solution{gx}
%         \inputAsFunction{x}{lx}
%     \end{answer}
%     \begin{answer}
%         \text{$w_y = $}
%         \solution{gy}
%         \inputAsFunction{x}{ly}
%     \end{answer}
%     \begin{answer}
%         \text{$w_z = $}
%         \solution{gz}
%         \inputAsFunction{x}{lz}
%         \checkFuncForZero{(ny*(kx*ly-ky*lx)-nz*(kz*lx-kx*lz))^2 + (nz*(ky*lz-kz*ly)-nx*(kx*ly-ky*lx))^2 + (nx*(kz*lx-kx*lz)-ny*(ky*lz-kz*ly))^2 + 1-(sign((kx*ly-ky*lx)^2+(ky*lz-kz*ly)^2+(kz*lx-kx*lz)^2))^2}{-10}{10}{100}
%     \end{answer}
% \end{question}

\begin{question}
	\lang{de}{
     	\text{Bestimmen Sie eine Parameterform 
        $E= \left\{ \begin{pmatrix} a_1 \\ a_2 \\ a_3 \end{pmatrix}+ \alpha \begin{pmatrix} v_1 \\ v_2 \\ v_3 \end{pmatrix} + \beta \begin{pmatrix} w_1 \\ w_2 \\ w_3 \end{pmatrix} \vert \alpha,\beta \in \R \right\}$\\\\
      	der Ebene, die die Punkte $P = \begin{pmatrix} \var{Px} \\ \var{Py} \\ \var{Pz}\end{pmatrix}$, $Q = \begin{pmatrix} \var{Qx} \\ \var{Qy} \\ \var{Qz}\end{pmatrix}$ und $R = \begin{pmatrix} \var{Rx} \\ \var{Ry} \\ \var{Rz}\end{pmatrix}$ \\\\
        enthält. \\\\
        $a_1 = $ \ansref , ~~~~~ $v_1 = $ \ansref , ~~~~~ $w_1 = $ \ansref \\
        $a_2 = $ \ansref , ~~~~~ $v_2 = $ \ansref , ~~~~~ $w_2 = $ \ansref \\
        $a_3 = $ \ansref , ~~~~~ $v_3 = $ \ansref , ~~~~~ $w_3 = $ \ansref }
   	}
    \explanation[edited]{Der Vektor $\vec{a}$ ist der Ortsvektor eines Punktes auf der Ebene (z.B. $\vec{OP}$), die Vektoren $\vec{v}$ und $\vec{w}$ zwei
    nichtparallele Richtungsvektoren (z.B. $\vec{PQ}$ und $\vec{PR}$.)
   	}

    \begin{variables}
		\randint{Px}{-5}{5}
		\randint{Py}{-5}{5}
		\randint{Pz}{-5}{5}
		\randint{Qx}{-5}{5}
		\randint{Qy}{-5}{5}
		\randint{Qz}{-5}{5}
		\randint{Rx}{-5}{5}
		\randint{Ry}{-5}{5}
		\randint{Rz}{-5}{5}
		\function[calculate]{ax}{Px}
		\function[calculate]{ay}{Py}
		\function[calculate]{az}{Pz}
		\function[calculate]{vx}{Qx-Px}
		\function[calculate]{vy}{Qy-Py}
		\function[calculate]{vz}{Qz-Pz}
		\function[calculate]{wx}{Rx-Px}
		\function[calculate]{wy}{Ry-Py}
		\function[calculate]{wz}{Rz-Pz}
		\function[calculate]{nx}{vy*wz-vz*wy}
		\function[calculate]{ny}{vz*wx-vx*wz}
		\function[calculate]{nz}{vx*wy-vy*wx}
        % wenn der Normalenvektor gleich dem Nullvektor ist, spannen die 3 Punkte keine Ebene auf
		\randadjustIf{Px,Py,Pz,Qx,Qy,Qz,Rx,Ry,Rz}{nx^2+ny^2+nz^2 = 0}
%		\randadjustIf{Px,Py,Pz,Qx,Qy,Qz,Rx,Ry,Rz}{((Qx-Px)*(Ry-Py)-(Qy-Py)*(Rx-Px))^2+((Qy-Py)*(Rz-Pz)-(Qz-Pz)*(Ry-Py))^2+((Qz-Pz)*(Rx-Px)-(Qx-Px)*(Rz-Pz))^2 = 0}
		\randadjustIf{Px,Py,Pz,Qx,Qy,Qz,Rx,Ry,Rz}{Qx=Px and Qy=Py and Qz=Pz}
		\randadjustIf{Px,Py,Pz,Qx,Qy,Qz,Rx,Ry,Rz}{Rx=Px and Ry=Py and Rz=Pz}
    \end{variables}
	
    \type{input.function}
	\field{real}
	
    \begin{answer}
		\solution{ax}
        \inputAsFunction{x}{axx}
    \end{answer}
	\begin{answer}
		\solution{vx}
        \inputAsFunction{x}{kx}
    \end{answer}
	\begin{answer}
		\solution{wx}
        \inputAsFunction{x}{lx}
    \end{answer}
    \begin{answer}
        \solution{ay}
        \inputAsFunction{x}{ayy}
    \end{answer}
    \begin{answer}
        \solution{vy}
        \inputAsFunction{x}{ky}
    \end{answer}
    \begin{answer}
        \solution{wy}
        \inputAsFunction{x}{ly}
    \end{answer}
    \begin{answer}
        \solution{az}
        \inputAsFunction{x}{azz}
% Ortsvektor kann beliebiger Punkt der Ebene sein
        \checkFuncForZero{((nx*axx)+(ny*ayy)+(nz*azz))-((nx*ax)+(ny*ay)+(nz*az))}{-10}{10}{100}
    \end{answer}
    \begin{answer}
        \solution{vz}
        \inputAsFunction{x}{kz}
    \end{answer}
    \begin{answer}
        \solution{wz}
        \inputAsFunction{x}{lz}
        \checkFuncForZero{(ny*(kx*ly-ky*lx)-nz*(kz*lx-kx*lz))^2 + (nz*(ky*lz-kz*ly)-nx*(kx*ly-ky*lx))^2 + (nx*(kz*lx-kx*lz)-ny*(ky*lz-kz*ly))^2 + 1-(sign((kx*ly-ky*lx)^2+(ky*lz-kz*ly)^2+(kz*lx-kx*lz)^2))^2}{-10}{10}{100}
    \end{answer}
\end{question}

\end{problem}

\embedmathlet{mathlet}
\end{content}