\documentclass{mumie.problem.gwtmathlet}
%$Id$
\begin{metainfo}
  \name{
    \lang{de}{A03: Normalenvektor}
    \lang{en}{input numbers}
  }
  \begin{description} 
 This work is licensed under the Creative Commons License Attribution 4.0 International (CC-BY 4.0)   
 https://creativecommons.org/licenses/by/4.0/legalcode 

    \lang{de}{Die Beschreibung}
    \lang{en}{}
  \end{description}
  \corrector{system/problem/GenericCorrector.meta.xml}
  \begin{components}
    \component{js_lib}{system/problem/GenericMathlet.meta.xml}{gwtmathlet}
  \end{components}
  \begin{links}
  \end{links}
  \creategeneric
\end{metainfo}
\begin{content}
\usepackage{mumie.genericproblem}

\lang{de}{\title{A03: Normalenvektor}}
\lang{en}{\title{Problem 3}}


\begin{block}[annotation]
	Im Ticket-System: \href{http://team.mumie.net/issues/9645}{Ticket 9645}
\end{block}




\begin{problem}

	\begin{question}
		\lang{de}{\text{
			Gegeben sei die Gerade $g= \left\{\begin{pmatrix}\var{a1}\\ \var{a2} \end{pmatrix}+ \lambda \begin{pmatrix}\var{b1}\\ \var{b2}\end{pmatrix} \vert \lambda \in \R \right\}. $Bestimmen Sie einen Normalenvektor $\vec{n}$ von $g$.\\$n_1=$\ansref, \\ $n_2=$\ansref.\\Bestimmen Sie nun die Koordinatenform von $g$:\\$g = \{ \vec{x}\in \R^2 | $ \ansref $x_1 +($ \ansref $x_2) = $\ansref $\}.$
 			}
		}
	
		
		\type{input.function}
		\field{real}
		
		
		\begin{variables}
			\randint{a1}{-15}{15}
			\randint{a2}{-15}{15}
			
			\randadjustIf{a1,a2}{a1^2+a2^2=0}
			\randint{b1}{-9}{9}
			\randint{b2}{-9}{9}
			
			\randadjustIf{b1,b2}{b1^2+b2^2=0}
			
			\function[calculate]{n1}{-b2}
			\function[calculate]{n2}{b1}
            
            \function[calculate]{d}{n1*a1+n2*a2}
		\end{variables}
		
		\begin{answer}
			\text{$n_1 = $}
			\solution{n1}
			\inputAsFunction{}{nx}
		\end{answer}
		
		\begin{answer}
			\text{$n_2 = $}
			\solution{n2}
			\inputAsFunction{}{ny}
			\checkFuncForZero{(n1*ny-nx*n2)^2+dirac(nx^2+ny^2)}{-10}{10}{100}
            %Prüft, ob der Normalenvektor richtig ist. dirac(xn^2+ny^2) fängt n=(0,0) ab.
            \explanation[edited]{Ein Normalenvektor einer Gerade steht senkrecht zu dieser und damit auch zu allen möglichen Richtungsvektoren der Gerade.
    Ein Normalenvektor der Geraden $g$ ist also ein zum gegebenen Richtungsvektor senkrecht stehender Vektor, so dass das Skalarprodukt des Richtungsvektors mit
    dem gesuchten Normalenvektor Null ergibt.}
		\end{answer}
        \begin{answer}
			\text{$n_1 = $}
			\solution{n1}
			\inputAsFunction{}{ax}
		\end{answer}
		
		\begin{answer}
			\text{$n_2 = $}
			\solution{n2}
			\inputAsFunction{}{ay}
			\checkFuncForZero{(ax*ny-nx*ay)^2+dirac(ax^2+ay^2)}{-10}{10}{100}
            %Prüft, ob der Normalenvektor richtig ist. dirac(xn^2+ny^2) fängt n=(0,0) ab.
            \explanation[edited]{Für die Koeffizienten vor $x_1$ bzw. $x_2$ können die Vektoreinträge von $\vec{n}$ genommen werden.}
		\end{answer}
        
       \begin{answer}
			\text{$d = $}
			\solution{d}
			\inputAsFunction{}{dd}
			\checkFuncForZero{ax*a1+ay*a2-dd}{-10}{10}{100}
            %Berücksichtigt Folgefehler
            \explanation[edited]{In der Koordinatenform kann auf der rechten Seite der Gleichung $\vec{n}\bullet \vec{OP}$ für einen beliebigen Punkt $P$ auf $g$ gewählt werden.}
		\end{answer}
		
	\end{question}
	
	\begin{question}
		\lang{de}{\text{Gegeben sei die Ebene$ E= \left\{ \begin{pmatrix}\var{a1}\\\var{a2}\\\var{a3}\end{pmatrix} + r \begin{pmatrix}\var{b1}\\\var{b2}\\\var{b3}\end{pmatrix} + s\begin{pmatrix}\var{c1}\\\var{c2}\\\var{c3}\end{pmatrix} \vert \, r,s\in \R \right\}. $Bestimmen Sie einen Normalenvektor $\vec{n} = \begin{pmatrix} n_1 \\ n_2 \\ n_3 \end{pmatrix}$ der Ebene:\\$n_1 =$\ansref, \\ $n_2 =$\ansref, \\ $n_3=$\ansref.\\Geben Sie $E$ in Koordinatenform an: \\$E = \{ \vec{x} \in \R^3 \vert$ \ansref $x_1 + ($\ansref $x_2) +($\ansref $x_3) = $\ansref $\}$.}
		}
		
		\type{input.function}
		\field{real}
		
		
		\begin{variables}
			
			\randint{a1}{-6}{6}
			\randint{a2}{-6}{6}
			\randint{a3}{-6}{6}
			\randadjustIf{a1,a2,a3}{a1^2+a2^2+a3^2=0}
			\randint{b1}{-6}{6}
			\randint{b2}{-6}{6}
			\randint{b3}{-6}{6}
			\randadjustIf{b1,b2,b3}{b1^2+b2^2+b3^2=0}
			\randint{c1}{-6}{6}
			\randint{c2}{-6}{6}
			\randint{c3}{-6}{6}
			\randadjustIf{c1,c2,c3}{c1^2+c2^2+c3^2=0}
			\randadjustIf{c1,c2,c3}{(b1*c2-c1*b2)^2+(b2*c3-c2*b3)^2+(b1*c3-b3*c1)^2=0}
			\function[calculate]{n1}{b2*c3-b3*c2}
			\function[calculate]{n2}{b3*c1-b1*c3}
			\function[calculate]{n3}{b1*c2-b2*c1}
			\function[calculate]{z1}{n1}
			\function[calculate]{z2}{n2}
			\function[calculate]{z3}{n3}
			\function[calculate]{d}{a1*n1+a2*n2+a3*n3}
		\end{variables}
		
		\begin{answer}
			\text{$n_1 = $}
			\solution{n1}
			\inputAsFunction{}{nx}
		\end{answer}
		
		\begin{answer}
			\text{$n_2 = $}
			\solution{n2}
			\inputAsFunction{}{ny}
		\end{answer}
		
		\begin{answer}
			\text{$n_3 = $}
			\solution{n3}
			\inputAsFunction{}{nz}
            \explanation[edited]{Ein Normalenvektor einer Ebene steht senkrecht zu dieser. Damit steht ein Normalenvektor einer Ebene auch senkrecht
    zu den durch die Ebene aufgespannten Richtungsvektoren. Man erhält einen zu den Richtungsvektoren senkrecht stehenden Vektor, indem man
    z.B. das Vektorprodukt der beiden Richtungsvektoren bildet.}
            \checkFuncForZero{(n1*ny-nx*n2)^2+(n2*nz-ny*n3)^2+(n1*nz-n3*nx)^2+dirac(nx^2+ny^2+nz^2)}{-10}{10}{100}
  		\end{answer}
		\begin{answer}
			\text{$z_1 = $}
			\solution{z1}
			\inputAsFunction{}{z11}
		\end{answer}
		
		\begin{answer}
			\text{$z_2 = $}
			\solution{z2}
			\inputAsFunction{}{z22}
		\end{answer}
		
		\begin{answer}
			\text{$z_3 = $}
			\solution{z3}
			\inputAsFunction{}{z33}
			\checkFuncForZero{(nx*z22-z11*ny)^2+(ny*z33-z22*nz)^2+(nx*z33-nz*z11)^2+dirac(z11^2+z22^2+z33^2)}{-10}{10}{100}
            %Berücksichtigt Folgefehler beim Normalenvektor
            \explanation[edited]{Die Komponenten eines Normalenvektors entsprechen den Vorfaktoren der $x_i$ der Koordinatengleichung der Ebene.}
		\end{answer}
		
		\begin{answer}
			\text{$d = $}
			\solution{d}
			\inputAsFunction{}{zz}
			\checkFuncForZero{a1*z11+a2*z22+a3*z33-zz}{-10}{10}{100}
            %Berücksichtigt Folgefehler
            \explanation[edited]{Die rechte Seite der Koordinatengleichung der Ebene
    ergibt sich durch Einsetzen eines auf der Ebene liegenden Punktes, der die Koordinatengleichung erfüllen muss.}
        \end{answer}
	\end{question}
	
\end{problem}

\embedmathlet{gwtmathlet}

\end{content}