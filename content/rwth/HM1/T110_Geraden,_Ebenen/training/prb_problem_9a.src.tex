\documentclass{mumie.problem.gwtmathlet}
%$Id$
\begin{metainfo}
  \name{
    \lang{de}{A09: Abstand}
    \lang{en}{input numbers}
  }
  \begin{description} 
 This work is licensed under the Creative Commons License Attribution 4.0 International (CC-BY 4.0)   
 https://creativecommons.org/licenses/by/4.0/legalcode 

    \lang{de}{Die Beschreibung}
    \lang{en}{}
  \end{description}
  \corrector{system/problem/GenericCorrector.meta.xml}
  \begin{components}
    \component{js_lib}{system/problem/GenericMathlet.meta.xml}{mathlet}
  \end{components}
  \begin{links}
  \end{links}
  \creategeneric
\end{metainfo}
\begin{content}
\usepackage{mumie.genericproblem}

\lang{de}{\title{A09: Abstand}}
\lang{en}{\title{Problem 9}}

\begin{block}[annotation]
	Schnittpunkt zwischen zwei Geraden. 
\end{block}
\begin{block}[annotation]
	Im Ticket-System: \href{https://team.mumie.net/issues/22820}{Ticket 22820}
\end{block}



\begin{problem}

%Question 1 of 1 	
\begin{question} 
	\lang{de}{ 
    	\text{
        
        Gegeben ist die Gerade\\
        $g: \vec{x} = \begin{pmatrix} 4\\0\\\var{b}\end{pmatrix} + \lambda\cdot \begin{pmatrix} \var{r1}\\\var{r2}\\\var{r3}\end{pmatrix}$, $\lambda\in\R$\\\\\\
                
        1. Bestimmen Sie den Lotfußpunkt $L$ vom Punkt $P(\var{f};0;0)$ auf die Gerade $g$. Ermitteln Sie auch den Abstand des Punktes $P$ von der Geraden $g$.\\\\
        \textit{Hinweis: Runden Sie Ihre Ergebnisse bei der Eingabe mit dem Taschenrechner auf zwei Nachkommastellen.}\\
        Der Lotfußpunkt liegt bei $L=($\ansref$;$\ansref$;$\ansref$)$.\\ Der Abstand des Punktes $P$ von $g$ beträgt \ansref.\\\\\\
        
        2. Berechnen Sie ferner den Abstand der Geraden\\ 
        $h:  \vec{x} = \begin{pmatrix} 4\\0\\0\end{pmatrix} + \mu\cdot \begin{pmatrix} \var{c}\\\var{c}\\\var{c}\end{pmatrix}$, $\mu\in\R$\\
        von der Geraden $g$.\\\\
        Der Abstand der beiden Geraden beträgt \ansref.        
       }}
    	 
    \begin{variables}
      \randint{a}{3}{6}
      \randint{b}{2}{5}
      \randint{c}{1}{3}
      \randint{d}{3}{5}
      \function[normalize]{r1}{2*c}
      \function[normalize]{r2}{d*c}
      \function[normalize]{r3}{d*c}
      \randint{f}{3}{6}
      
      \function[normalize]{Sx}{4+2*c*l}
      \function[normalize]{Sy}{d*c*l}
      \function[normalize]{Sz}{b+d*c*l}
      
      \function[normalize]{l}{(2*c*f-8*c-b*c*d)/(4*c*c+2*d*d*c*c)}
      \function[normalize]{abstand1}{sqrt((Sx-f)^2+(Sy)^2+(Sz)^2)}
      
      \function[normalize]{abstand2}{(b*d*c*c-2*b*c*c)/(sqrt(2*(2*c*c-d*c*c)^2))}
      
	\end{variables}
	\type{input.number}
    \field{real}
    
    \begin{answer}
		\solution{Sx}
        \explanation[edited]{Man kann eine Hilfsebene (in Normalen-/Koordinatenform) durch den Punkt $P$ aufstellen, deren Normalenvektor dem Richtungsvektor der Geraden $g$
        entspricht. In die Koordinatenform-Gleichung setzt man die Gerade $g$ ein und ermittelt die Lösung $\lambda$ der Gleichung. $\lambda$ wiederum in die Gerade $g$
        eingesetzt liefert den Lotfußpunkt $L$.}
	\end{answer}
    \begin{answer}
		\solution{Sy}
	\end{answer}
    \begin{answer}
		\solution{Sz}
	\end{answer}

    \begin{answer}
		\solution{abstand1}
        \explanation[edited]{Man erhält den Abstand des Punktes $P$ von der Geraden $g$ durch den Abstand von $P$ zum ermittelten Lotfußpunkt $L$.}
	\end{answer}
    
    \begin{answer}
		\solution{abstand2}
        \explanation[edited]{Sind $g= \vec{p} + r \vec{v}, r \in \mathbb{R},$ und  $h= \vec{q} + s \vec{w},s \in \mathbb{R},$ 
zwei Geraden im $\mathbb{R}^3$, die nicht parallel sind, und $\vec{n}$ ein Vektor, der zu beiden Geraden orthogonal ist, 
so lässt sich der Abstand von $g$ und $h$ berechnen durch\\

$d(g,h)= \frac{\left| (\vec{q}-\vec{p})\bullet \vec{n} \right|}{\|\vec{n}\|}$.}
	\end{answer}

\end{question}

\end{problem}

\embedmathlet{mathlet}

\end{content}