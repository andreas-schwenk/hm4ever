%$Id:  $
\documentclass{mumie.article}
%$Id$
\begin{metainfo}
  \name{
    \lang{de}{Abstände Punkte-Geraden-Ebenen}
    \lang{en}{Distances between points, lines and planes}
  }
  \begin{description} 
 This work is licensed under the Creative Commons License Attribution 4.0 International (CC-BY 4.0)   
 https://creativecommons.org/licenses/by/4.0/legalcode 

    \lang{de}{Beschreibung}
    \lang{en}{Description}
  \end{description}
  \begin{components}
    \component{generic_image}{content/rwth/HM1/images/g_tkz_T110_DistancePointLineExample.meta.xml}{T110_DistancePointLineExample}
    \component{generic_image}{content/rwth/HM1/images/g_tkz_T110_Pyramid.meta.xml}{T110_Pyramid}
    \component{generic_image}{content/rwth/HM1/images/g_tkz_T110_DistanceSkewLines.meta.xml}{T110_DistanceSkewLines}
    \component{generic_image}{content/rwth/HM1/images/g_tkz_T110_DistancePointPlaneRule.meta.xml}{T110_DistancePointPlaneRule}
    \component{generic_image}{content/rwth/HM1/images/g_tkz_T110_DistancePointLineRule.meta.xml}{T110_DistancePointLineRule}
    \component{generic_image}{content/rwth/HM1/images/g_tkz_T110_DistancePointPlane.meta.xml}{T110_DistancePointPlane}
    \component{generic_image}{content/rwth/HM1/images/g_tkz_T110_DistancePointLine.meta.xml}{T110_DistancePointLine}
    \component{generic_image}{content/rwth/HM1/images/g_img_00_video_button_schwarz-blau.meta.xml}{00_video_button_schwarz-blau}
  \end{components}
  \begin{links}
    \link{generic_article}{content/rwth/HM1/T110_Geraden,_Ebenen/g_art_content_37_schnittpunkte.meta.xml}{lagebeziehungen}
    \link{generic_article}{content/rwth/HM1/T110_Geraden,_Ebenen/g_art_content_35_parameterformen.meta.xml}{parameterformen}
    \link{generic_article}{content/rwth/HM1/T110_Geraden,_Ebenen/g_art_content_36_normalenformen.meta.xml}{normalenformen}
  \end{links}
  \creategeneric
\end{metainfo}
\begin{content}
\usepackage{mumie.ombplus}
\ombchapter{10}
\ombarticle{4}
\usepackage{mumie.genericvisualization}

\begin{visualizationwrapper}

\title{\lang{de}{Abstände zwischen Punkten, Geraden und Ebenen}
       \lang{en}{Distances between points, lines and planes}}
 
\begin{block}[annotation]
  übungsinhalt
  
\end{block}
\begin{block}[annotation]
  Im Ticket-System: \href{http://team.mumie.net/issues/9055}{Ticket 9055}\\
\end{block}

\begin{block}[info-box]
\tableofcontents
\end{block}

\lang{de}{
Ziel dieses Abschnitts ist es, Abst"ande von Punkten zu Geraden oder Ebenen, sowie Abst"ande zwischen 
zwei Geraden, Geraden und Ebenen bzw. zwei Ebenen zu berechnen, sofern sie sich nicht schneiden. 
Solche Abstände sind stets durch Verbindungen gegeben, die orthogonal zu den Geraden und/oder Ebenen 
sind.
}
\lang{en}{
The aim of this section is to introduce methods for calculating the distance between a point and a 
line, between two lines, between a line and a point and between two planes, provided that they do 
not intersect. The minimal such distances are those that interest us, and these are given by the 
lengths of vectors that are orthogonal to the line/plane being considered.
}

\section{\lang{de}{Lot und Lotfu"spunkt}\lang{en}{Orthogonal line through a point}}\label{sec:lot}

\begin{definition}
\begin{enumerate}
\item \lang{de}{
      Seien im $\R^2$ oder $\R^3$ eine Gerade $g$ und ein Punkt $P$, der nicht auf $g$ liegt, 
      gegeben. Dann ist das \emph{Lot von $P$ auf $g$} definiert als die Gerade $l$ durch $P$, die 
      $g$ senkrecht schneidet.\\
      Der \emph{Lotfu"spunkt} (oder auch Fu"spunkt des Lotes) $L$ ist definiert als der Schnittpunkt 
      des Lotes mit der Geraden $g$.
      }
      \lang{en}{
      Consider a line $g$ and point $P$ that does not lie on $G$, in $\R^2$ or $\R^3$. Then there 
      exists a line $l$ through $P$ that is orthogonal to $g$.
      }

\begin{center}
\image{T110_DistancePointLine}
\end{center}

\item \lang{de}{
      Seien im $\R^3$ eine Ebene $E$ und ein Punkt $P$, der nicht auf $E$ liegt, gegeben. Dann ist 
      das \emph{Lot von $P$ auf $E$} definiert als die Gerade $l$ durch $P$, die $E$ senkrecht 
      schneidet, also senkrecht auf den beiden Richtungsvektoren von $E$ steht.\\
      Der \emph{Lotfu"spunkt} $L$ ist definiert als der Schnittpunkt des Lotes mit der Ebene $E$.
      }
      \lang{en}{
      Consder a plane $E$ and point $P$ that does not lie on $E$, in $\R^3$. Then there exists a 
      line $l$ through $P$ that is orthogonal to $E$, i.e. orthogonal to the direction vectors of 
      $E$.
      }

\begin{center}
\image{T110_DistancePointPlane}
\end{center}
\end{enumerate}
\end{definition}


\begin{rule}
\lang{de}{
Sei $P$ ein Punkt mit Ortsvektor $\vec{p}$ und 
$g = \left\{\vec{u} + r\vec{v} \, | \, r \in \R \right\}$ eine Gerade. Wenn $P$ nicht auf $g$ liegt, 
dann gilt f"ur den Lotfu"spunkt $L$ 
}
\lang{en}{
Let $P$ be a point with position vector $\vec{p}$ and 
$g = \left\{\vec{u} + r\vec{v} \, | \, r \in \R \right\}$ a line. If $P$ does not lie on $g$, then 
there is a line $l$ through $P$ orthogonal to $g$. Say $L$ is the othogonal intersection of this 
line with $g$, then
}
\[ \overrightarrow{OL}= \vec{u}+r_0\vec{v},\] 
\lang{de}{wobei $r_0$ die L"osung der Gleichung}
\lang{en}{where $r_0$ is the solution of the equation}
\[ \left(  \vec{u}+r \vec{v}-\vec{p}\right) \bullet\vec{v} =0 \]
\lang{de}{
ist.\\
Das hei"st:
}
\lang{en}{
in $r$.\\
Therefore, by distributivity:
}
\[ r_0= \frac{(\vec{p}-\vec{u})\bullet \vec{v}}{\|\vec{v}\|^2}.\] 

\begin{center}
\image{T110_DistancePointLineRule}
\end{center}

\lang{de}{Das Lot $l$ von $P$ auf $g$ ist dann die Gerade durch $P$ und $L$, d.h.}
\lang{en}{The line $l$ through $P$ and orthogonal to $g$ goes through $P$ and $L$, so}
\[ l=\left\{\vec{p} + s\overrightarrow{PL}\, \, | \,\, s \in \R \right\}.\] 
\end{rule}

\begin{example}\label{ex:lot-auf-gerade}
\lang{de}{
Wir betrachten den Punkt $P=(1;3)$ und die Gerade $g=\left\{  \begin{pmatrix} 0 \\ 2 \end{pmatrix} + 
r  \begin{pmatrix} 1 \\ 2 \end{pmatrix}  \, \, | \, \, r \in \R \right\}$.
\\\\
Der Punkt $P=(1;3)$ liegt nicht auf $g$, denn die Gleichung 
$\begin{pmatrix} 1 \\ 3 \end{pmatrix}=\begin{pmatrix} 0 \\ 2 \end{pmatrix} + 
r  \begin{pmatrix} 1 \\ 2 \end{pmatrix} $, führt zu den zwei Gleichungen $1=0+ 1\cdot r$ und 
$3=2+2\cdot r$, d.h. zu $1=r$ und $1=2\cdot r$, welche nicht beide gleichzeitig erfüllbar sind.
\\\\
Wir berechnen den Lotfu"spunkt:
Nach der Formel ist $\overrightarrow{OL}=  \begin{pmatrix} 0 \\ 2 \end{pmatrix}  + 
r_0  \begin{pmatrix} 1 \\ 2 \end{pmatrix} $ mit
}
\lang{en}{
Consider the point $P=(1;3)$ and the line $g=\left\{  \begin{pmatrix} 0 \\ 2 \end{pmatrix} + 
r  \begin{pmatrix} 1 \\ 2 \end{pmatrix}  \, \, | \, \, r \in \R \right\}$.
\\\\
The point $P=(1;3)$ does not lie on $g$, as the equation 
$\begin{pmatrix} 1 \\ 3 \end{pmatrix}=\begin{pmatrix} 0 \\ 2 \end{pmatrix} + 
r  \begin{pmatrix} 1 \\ 2 \end{pmatrix} $, leads to the two equations $1=0+ 1\cdot r$ and 
$3=2+2\cdot r$, that is, $1=r$ and $1=2\cdot r$. Together, these yield a contradiction.
\\\\
Hence we can find a point $L$ on $g$ such that $\overrightarrow{PL}$ is orthogonal to $g$. 
By the above rule, $\overrightarrow{OL} =  \begin{pmatrix} 0 \\ 2 \end{pmatrix}  + 
r_0  \begin{pmatrix} 1 \\ 2 \end{pmatrix}$, where
}
\[r_0 = \frac{\Big(\begin{pmatrix} 1 \\ 3 \end{pmatrix} - 
  \begin{pmatrix} 0 \\ 2 \end{pmatrix}  \Big)\bullet  
  \begin{pmatrix} 1 \\ 2 \end{pmatrix} }{\| \begin{pmatrix} 1 \\ 2 \end{pmatrix} \|^2 } = 
  \frac{(1-0)\cdot 1+(3-2)\cdot 2}{(1^2+2^2)}=\frac{3}{5}.\]

\lang{de}{Also}\lang{en}{Thus}
\[\overrightarrow{OL}=  \begin{pmatrix} 0 \\ 2 \end{pmatrix}  + 
  \frac{3}{5}  \begin{pmatrix} 1 \\ 2 \end{pmatrix} = 
  \begin{pmatrix}3/5 \\ 16/5 \end{pmatrix}. \]
\lang{de}{Die Lotgerade $l$ ist dann:}
\lang{en}{The line $l$ through $P$ and $L$ is therefore:}
\begin{eqnarray*}
 l &=& \left\{  \begin{pmatrix} 1 \\ 3 \end{pmatrix} + s \begin{pmatrix} -2/5 \\ 1/5 \end{pmatrix} \, \, | \,\, s\in \R \right\} \\
 &=& \left\{  \begin{pmatrix} 1 \\ 3 \end{pmatrix} + t \begin{pmatrix} -2 \\ 1 \end{pmatrix} \, \, | \,\, t\in \R \right\}.
 \end{eqnarray*}
 
\begin{center}
\image{T110_DistancePointLineExample}
\end{center}

\end{example}

\begin{rule}\label{rule:lot-auf-ebene}
\lang{de}{
Seien ein Punkt $P$ und eine Ebene $E=\{\vec{x} \mid \vec{x}\bullet \vec{n}= d \}$ gegeben und $P$ 
liege nicht auf $E$, dann ist das Lot $l$ von $P$ auf $E$ gegeben durch
}
\lang{en}{
Consider a plane $E = \{\vec{x} \mid \vec{x}\bullet \vec{n}= d \}$ and a point $P$ not on $E$. Then 
the line through $P$ orthogonal to $E$ is given by
}
\[ l=\left\{\vec{p} + s\vec{n} \mid s \in \R \right\}.\] 

\lang{de}{
Der Lotfu"spunkt $L$ ist der Schnittpunkt von $l$ mit $E$, d.h. 
$\overrightarrow{OL}=\vec{p} + s_0\vec{n}$, wobei $s_0$ die L"osung der Gleichung
}
\lang{en}{
The intersection of $l$ with $E$ is the point with position vector 
$\overrightarrow{OL}=\vec{p} + s_0\vec{n}$ where $s_0$ is the solution of
}
\[ \left( \vec{p} + s\vec{n}\right) \bullet \vec{n}= d \]
\lang{de}{ist. Das hei"st:}
\lang{en}{in $s$. That means}
\[ s_0= \frac{d- \vec{p}  \bullet \vec{n}}{\|\vec{n}\|^2}. \]

\begin{center}
\image{T110_DistancePointPlaneRule}
\end{center}
\end{rule}

\begin{example}\label{ex:lot-auf-ebene}
\lang{de}{Wir betrachten den Punkt $P=(3;2;1)$ und die Ebene}
\lang{en}{Consider the point $P=(3;2;1)$ and the plane}
\[E = \left\{   \begin{pmatrix}x_1 \\ x_2\\ x_3 \end{pmatrix}\in \R^3\, \mid\,  
  \begin{pmatrix}-1 \\ 2\\ 1 \end{pmatrix}\bullet 
  \begin{pmatrix}x_1 \\ x_2\\ x_3 \end{pmatrix}=3 \right\} \]
\lang{de}{
(vgl. \ref[lagebeziehungen][Abschnitt Lagebeziehungen]{ex:punkt-ebene}).
\\\\
Für die Lotgerade hat man dann als Stützvektor den Ortsvektor von $P$ und als Richtungsvektor 
den Normalenvektor $\vec{n}=\begin{pmatrix}-1 \\ 2\\ 1 \end{pmatrix}$ der Ebene $E$. Also
}
\lang{en}{
(see \ref[lagebeziehungen][previous section]{ex:punkt-ebene}).
\\\\
To find the line $l$ through $P$ and orthogonal to $E$, we start with $P$ as the position vector of 
a point on the line, and take the normal vector $\vec{n}=\begin{pmatrix}-1 \\ 2\\ 1 \end{pmatrix}$ 
of the plane to be the direction vector. So
}

\[l=\left\{ \begin{pmatrix}3\\ 2\\ 1 \end{pmatrix} + 
  s\cdot \begin{pmatrix}-1 \\ 2\\ 1 \end{pmatrix} \mid  s\in \R \right\}. \]
\lang{de}{
Für den Lotfußpunkt $L$ bestimmen wir den Schnittpunkt von $l$ mit $E$, d.h. wir setzen die 
Parameterform von $l$ in die Gleichung von $E$ ein und erhalten
}
\lang{en}{
To find the point $L$ where the line $l$ intersects the plane, we substitute the parametrised form 
of $l$ into the equation for $E$ to obtain
}
\begin{align*}
 && \hspace{1cm} \begin{pmatrix}-1 \\ 2\\ 1 \end{pmatrix}\bullet 
\Big(  \begin{pmatrix}3\\ 2\\ 1 \end{pmatrix}+s\cdot \begin{pmatrix}-1 \\ 2\\ 1 \end{pmatrix} \Big) &= 3 \\
&\Leftrightarrow & \begin{pmatrix}-1 \\ 2\\ 1 \end{pmatrix}\bullet \begin{pmatrix}3\\ 2\\ 1 \end{pmatrix}
+s\cdot \begin{pmatrix}-1 \\ 2\\ 1 \end{pmatrix}\bullet \begin{pmatrix}-1 \\ 2\\ 1 \end{pmatrix} &= 3 \\
&\Leftrightarrow & \hspace{4cm} s\cdot \left\| \begin{pmatrix}-1 \\ 2\\ 1 \end{pmatrix}\right\|^2 &=3- 
\begin{pmatrix}-1 \\ 2\\ 1 \end{pmatrix}\bullet \begin{pmatrix}3\\ 2\\ 1 \end{pmatrix} \\
&\Leftrightarrow & \hspace{7cm} s &= \frac{3- 
\begin{pmatrix}-1 \\ 2\\ 1 \end{pmatrix}\bullet \begin{pmatrix}3\\ 2\\ 1 \end{pmatrix}}{\left\| \begin{pmatrix}-1 \\ 2\\ 1 \end{pmatrix}\right\|^2}
 \\
&\Leftrightarrow & \hspace{7cm} s &= \frac{3- (-1\cdot 3+2^2+1^2)}{(-1)^2+2^2+1^2}= \frac{1}{6}
\end{align*}
\lang{de}{
(Die vorletzte Zeile ist übrigens genau die Formel in obiger Regel und die hier gemachten 
Umformungen liefern die Begründung, weshalb die Formel in der Regel gilt.)
\\\\
Der Lotfußpunkt hat also den Ortsvektor
}
\lang{en}{
(The penultimate row is in fact the formula from the above rule, and the above manipulation 
motivate why the formula from the rule works.)
\\\\
The point $L$ therefore has the position vector
}
\[ \vec{OL}= \begin{pmatrix}3\\ 2\\ 1 \end{pmatrix}+\frac{1}{6}\cdot \begin{pmatrix}-1 \\ 2\\ 1 \end{pmatrix}
=\begin{pmatrix}17/6 \\ 14/6 \\ 7/6 \end{pmatrix},\]
\lang{de}{und daher ist }
\lang{en}{and thus }
$L=(\frac{17}{6};\frac{14}{6};\frac{7}{6})$.
\end{example}


\begin{remark}
\lang{de}{
Der Lotfußpunkt und die Lotgerade lassen sich auch im Fall eines Punktes und einer Ebene in 
Parameterform ähnlich wie für den Fall eines Punktes und einer Geraden berechnen. Die Rechnung wird 
jedoch etwas komplizierter.
}
\lang{en}{
The line $l$ through a point $P$ and orthogonal to a plane $E$ expressed in parametrised form can be 
found using a similar method to the case with a point $P$ and line $g$. The calculation itself is 
however somewhat more complicated.
}
\end{remark}

\section{\lang{de}{Abst"ande von Punkten zu Geraden und Ebenen}
         \lang{en}{Distances between points and lines or planes}}

\lang{de}{
Da Abstände immer durch orthogonale Verbindungen gegeben sind, erhalten wir folgende 
Abstandsberechnungen.
}
\lang{en}{
As distances between points and lines or planes are always given by the orthogonal line between two 
graphs, we obtain the following rules.
}

%Den \emph{Abstand $d(P,g)$} eines Punkte $P$ von einer Geraden $g$  zu bestimmen, ist nun einfach:

\begin{rule}\label{rule:abst_pkt_gerade}\label{rule:abst_pkt_ebene}
\lang{de}{
Der \emph{Abstand $d(P,g)$} eines Punktes $P$ von einer Geraden $g$ ist genau der Abstand von $P$ 
zum Lotfu"spunkt $L$ von $P$ auf $g$, also
}
\lang{en}{
The \emph{distance $d(P,g)$} of a point $P$ from a line $g$ is exactly the distance from $P$ to the 
closest point on that line, which is the point $L$ where the line $l$ through $P$ and orthogonal to 
$g$ intersects $g$. So,
}
\[ d(P,g)=d(P,L). \]

\lang{de}{
Ebenso erh"alt man den \emph{Abstand $d(P,E)$} eines Punktes $P$ von einer Ebene $E$ als den Abstand 
von $P$ zum Lotfu"spunkt $L$ von $P$ auf $E$, d.h.
}
\lang{en}{
Similarly, the \emph{distance $d(P,E)$} of a point $P$ from a plane $E$ is exactly the distance from 
$P$ to the closest point on that plane, line, which is the point $L$ where the line $l$ through $P$ 
and orthogonal to $E$ intersects $E$. So,
}
\[ d(P,E)=d(P,L)=\|s_0\cdot\vec{n}\|=\frac{|d-\vec{p}\bullet \vec{n}|}{\|\vec{n}\|}, \]
\lang{de}{
wobei die Ebene $E$ die Normalenform $E=\{\vec{x} \mid \vec{x}\bullet \vec{n}= d \}$ hat und $s_0$ 
wie in Regel \ref{rule:lot-auf-ebene} gegeben ist.
\\\\
\floatright{\href{https://api.stream24.net/vod/getVideo.php?id=10962-2-10794&mode=iframe&speed=true}{\image[75]{00_video_button_schwarz-blau}}}\\
}
\lang{en}{
where the plane $E$ is given in the normal form 
$E=\{\vec{x} \mid \vec{x}\bullet \vec{n}= d \}$ and $s_0$ is given as in rule 
\ref{rule:lot-auf-ebene}.
}
\end{rule}

\begin{remark}
\begin{enumerate}
\item \lang{de}{
      In der letzten Formel für den Abstand $d(P,E)$ des Punktes zur Ebene taucht der Lotfußpunkt 
      gar nicht mehr auf. Man kann in diesem Fall also auch direkt den Abstand ausrechnen, ohne den 
      Lotfußpunkt vorher bestimmen zu müssen.
      }
      \lang{en}{
      The point $L$ does not even appear in the final formula for the distance $d(P,E)$ of a point 
      from a plane. In this case the distance can be calculated without needing to find $L$.
      }
\item \lang{de}{
      Die angegebenen Verfahren zur Bestimmung des Lotfu"spunktes funktionieren auch, wenn der Punkt 
      $P$ auf der Geraden bzw. auf der Ebene liegt. In diesem Fall erh"alt man $L=P$ und der Abstand 
      ist $0$.\\
      Die Lotgerade zur Geraden im $\R^3$ existiert in diesem Fall jedoch nicht.
      }
      \lang{en}{
      The methods given above for determining $L$ also technically work if $P$ lies on the line $g$ 
      (in $\R^2$) or the plane $E$ (in $\R^3$). They yield $L=P$ so the distance is $0$.\\
      However, if $P$ lies on a line $g$ in $\R^3$, there is no unique line through $P$ and with 
      orthogonal intersection with $g$.
      }
\end{enumerate}
\end{remark}

\begin{example}
\begin{enumerate}
\item \lang{de}{
      In Beispiel \ref{ex:lot-auf-gerade}, hatten wir den Punkt $P=(1;3)$ und die Gerade 
      $g=\left\{\begin{pmatrix} 0 \\ 2 \end{pmatrix} + 
      r  \begin{pmatrix} 1 \\ 2 \end{pmatrix} \, \, | \, \, r \in \R \right\}$. 
      Der Lotfußpunkt von $P$ auf $g$ war $L=(\frac{3}{5}; \frac{16}{5})$.
      \\\\
      Daher ist der Abstand von $P$ zu $g$:
      }
      \lang{en}{
      In example \ref{ex:lot-auf-gerade} we considered the point $P=(1;3)$ and the line 
      $g=\left\{\begin{pmatrix} 0 \\ 2 \end{pmatrix} + 
      r  \begin{pmatrix} 1 \\ 2 \end{pmatrix} \, \, | \, \, r \in \R \right\}$. 
      The point $L$ at which the line $l$ through $P$ intersects $g$ orthogonally was 
      $L=(\frac{3}{5}; \frac{16}{5})$.
      \\\\
      Hence the distance from $P$ to $G$ is:
      }
      \[d(P,g)=d(P,L)=\left\| \vec{PL}\right\| = 
        \left\| \begin{pmatrix} 3/5 -1 \\ 16/5-3 \end{pmatrix} \right\| = 
        \sqrt{(-\frac{2}{5})^2+(\frac{1}{5})^2} = \sqrt{\frac{5}{25}} = \frac{\sqrt{5}}{5}. \]
\item \lang{de}{In Beispiel \ref{ex:lot-auf-ebene}, hatten wir den Punkt $P=(3;2;1)$ und die Ebene }
      \lang{en}{In example \ref{ex:lot-auf-ebene} we considered the point $P=(3;2;1)$ and the plane }
      $E=\left\{\begin{pmatrix}x_1 \\ x_2\\ x_3 \end{pmatrix}\in \R^3\, \mid\, 
      \begin{pmatrix}-1 \\ 2\\ 1 \end{pmatrix}\bullet 
      \begin{pmatrix}x_1 \\ x_2\\ x_3 \end{pmatrix} = 3 \right\}$.


\lang{de}{Daher ist der Abstand von $P$ zu $E$ nach obiger Formel:}
\lang{en}{Hence, using the above formula, the distance from $P$ to $E$ is:}
\[d(P,E)=\frac{|3- \begin{pmatrix}3 \\ 2\\ 1 \end{pmatrix} \bullet 
  \begin{pmatrix}-1 \\ 2\\ 1 \end{pmatrix}|}{\|\begin{pmatrix}-1 \\ 2\\ 1 \end{pmatrix}\|} = 
  \frac{|3- (-1\cdot 3+2^2+1^2)|}{\sqrt{(-1)^2+2^2+1^2}} = \frac{1}{\sqrt{6}}. \]
  \end{enumerate}
  \end{example}

\lang{de}{Gerade in geometrischen Anwendungen müssen oft Abstände berechnet werden.}
\lang{en}{Geometric applications often require distances to be found.}

\begin{example}
\lang{de}{
Im $\R^3$ seien die Punkte $A=(1; 1; 0)$, $B=(2; 1; 2)$, $C=(3; 3; 4)$ und $S=(4; 0; 0)$ gegeben. 
Bestimmen Sie das Volumen der dreiseitigen Pyramide mit Spitze $S$ und Grundfl"ache Dreieck $ABC$.
}
\lang{en}{
Consider the points $A=(1; 1; 0)$, $B=(2; 1; 2)$, $C=(3; 3; 4)$ and $S=(4; 0; 0)$ in $\R^3$. 
Determine the volume of the three-sided pyramid with tip $S$ and the triangle $ABC$ as its base.
}

\begin{center}
\image{T110_Pyramid}
\end{center}

\begin{tabs*}
\tab{\lang{de}{Formel f"ur das Volumen $V$ der Pyramide}
     \lang{en}{Formula for the volume $V$ of the pyramid}}
\begin{eqnarray*}
V &=& \frac{1}{3}\cdot \text{\lang{de}{Grundfl"ache}\lang{en}{Base area}}\cdot 
\text{\lang{de}{H"ohe}\lang{en}{Height}} = \frac{1}{3}\cdot 
\text{\lang{de}{Fl"ache}\lang{en}{Area}}(\Delta ABC)\cdot d(S,E_{ABC}),
\end{eqnarray*} 
\lang{de}{wobei $E_{ABC}$ die Ebene durch die drei Punkte $A$, $B$ und $C$ ist.}
\lang{en}{where $E_{ABC}$ is the plane containing the three points $A$, $B$ and $C$.}
\tab{\lang{de}{Formel f"ur die Fl"ache $F$ der dreieckigen Grundseite}
     \lang{en}{Formula for the area $F$ of the triangular base}}
\begin{eqnarray*}
F &=& \frac{1}{2}\cdot \text{\lang{de}{Grundseite}\lang{en}{Base}}\cdot 
\text{\lang{de}{H"ohe}\lang{en}{Height}} = \frac{1}{2}\cdot d(A,B)\cdot d(C,g_{AB}),
\end{eqnarray*}
\lang{de}{wobei $g_{AB}$ die Gerade durch $A$ und $B$ ist.}
\lang{en}{where $g_{AB}$ is the line containing $A$ and $B$.}
\end{tabs*}
\lang{de}{Das Volumen der Pyramide ist}\lang{en}{The volume of the pyramid is}
\begin{eqnarray*}
V &=& \frac{1}{3}\big(\frac{1}{2}\cdot d(A,B)\cdot d(C,g_{AB})\big)\cdot d(S,E_{ABC}) \\
&=& \frac{1}{6}d(A,B)\cdot d(C,g_{AB})\cdot d(S,E_{ABC}),
\end{eqnarray*} 
\lang{de}{
wobei $g_{AB}$ die Gerade durch die Punkte $A$ und $B$ ist und $E_{ABC}$ die Ebene durch die Punkte 
$A$, $B$ und $C$.
}
\lang{en}{
where $g_{AB}$ is the line containing $A$ and $B$ and $E_{ABC}$ is the plane containing $A$, $B$ and 
$C$.
}

\begin{enumerate}
\item $d(A,B)=\|\overrightarrow{AB}\|=\left\| \begin{pmatrix} 1 \\ 0 \\ 2 \end{pmatrix}  \right\| =\sqrt{1+4}=\sqrt{5}$.
\item \lang{de}{
      Berechnung von $d(C,g_{AB})$:\\
      Die Gerade durch die Punkte $A$ und $B$ ist gegeben durch (vgl. 
      \ref[parameterformen][Abschnitt Parameterformen]{sec:geraden})
      }
      \lang{en}{
      Calculating $d(C,g_{AB})$:\\
      The line through the points $A$ and $B$ is given by (see the 
      \ref[parameterformen][section on parametrised lines and planes]{sec:geraden})
      }
      \[ g_{AB}=\left\{  \begin{pmatrix} 1 \\ 1 \\ 0 \end{pmatrix} + 
      r  \begin{pmatrix} 1 \\ 0 \\ 2 \end{pmatrix} \, | \, r \in \R \right\}. \]
      \lang{de}{
      Der Lotfu"spunkt $L_C$ von $C$ auf $g_{AB}$ ist daher 
      }
      \lang{en}{
      The line $l_C$ through $C$ and orthogonal to $g_{AB}$ intersects (orthogonally) with 
      $g_{AB}$ at the point $L_C$ given by 
      }
      $\overrightarrow{OL_C}=\overrightarrow{OA}+r_0\vec{v}$ mit $\vec{v}=\overrightarrow{AB} =
      \begin{pmatrix} 1 \\ 0 \\ 2 \end{pmatrix}$ 
      \lang{de}{und}\lang{en}{and}
      \[ r_0=\frac{\overrightarrow{AC}\bullet \vec{v}}{\|\vec{v}\|^2} =
      \frac{ \begin{pmatrix} 2 \\ 2 \\ 4 \end{pmatrix} \bullet 
      \begin{pmatrix} 1 \\ 0 \\ 2 \end{pmatrix} }{\sqrt{5}^2} = \frac{2+8}{5} = 2. \]
      \lang{de}{Also ist }\lang{en}{Thus}
      $d(C,g_{AB})=d(C,L_C)=\left\|  \begin{pmatrix} 1 \\ 1 \\ 0 \end{pmatrix} + 
      2 \begin{pmatrix} 1 \\ 0 \\ 2 \end{pmatrix} -  
      \begin{pmatrix} 3 \\ 3 \\ 4 \end{pmatrix}  \right\| = 
      \left\|  \begin{pmatrix} 0 \\ -2 \\ 0 \end{pmatrix} \right\| = 2.$
\item \lang{de}{
      Berechnung von $d(S,E_{ABC})$:\\
      Die Ebene durch die Punkte $A$, $B$ und $C$ ist gegeben durch (vgl. 
      \ref[parameterformen][Abschnitt Parameterformen]{sec:ebenen})
      }
      \lang{en}{
      Calculating $d(S,E_{ABC})$:\\
      The plane containing the points $A$, $B$ and $C$ is given by (see the 
      \ref[parameterformen][section on parametrised lines and planes]{sec:ebenen})
      }
      \[E_{ABC}=\left\{  \begin{pmatrix} 1 \\ 1 \\ 0 \end{pmatrix} + 
      r \begin{pmatrix} 1 \\ 0 \\ 2 \end{pmatrix} + 
      s  \begin{pmatrix} 2 \\ 2 \\ 4 \end{pmatrix}  \, | \, r, s \in \R \right\}. \]
      \lang{de}{
      Daraus lässt sich eine Normalenform von $E$ berechnen als (vgl. 
      \ref[normalenformen][Abschnitt Normalenformen]{sec:ebenen})
      }
      \lang{en}{
      Using this we can calculate a normal form of $E$ as (see the 
      \ref[normalenformen][section on the normal form of lines and planes]{sec:ebenen})
      }
      \[E = \left\{  \begin{pmatrix} x_1 \\ x_2 \\ x_3 \end{pmatrix} \, | \,
      \begin{pmatrix} -4 \\ 0 \\ 2 \end{pmatrix} \bullet \vec{x}    =-4 \right\}\]
      \lang{de}{Der Lotfu"spunkt $L_S$ von $S$ auf $E_{ABC}$ erfüllt daher }
      \lang{en}{
      The line $l_S$ through $S$ and orthogonal to $E_{ABC}$ intersects (orthogonally) with 
      $E_{ABC}$ at the point $L_S$ given by 
      }
      $\overrightarrow{OL_S}=\overrightarrow{OS}+t_0\vec{n}$ 
      \lang{de}{mit}\lang{en}{with} 
      $\vec{n}= \begin{pmatrix} -4 \\ 0 \\ 2 \end{pmatrix} $ 
      \lang{de}{und}\lang{en}{and} 
      \[ t_0=\frac{d-\overrightarrow{OS}\bullet \vec{n}}{\|\vec{n}\|^2} = 
      \frac{-4- \begin{pmatrix} 4 \\ 0 \\ 0 \end{pmatrix} \bullet 
      \begin{pmatrix} -4 \\ 0 \\ 2 \end{pmatrix}}
      {\left\| \begin{pmatrix} -4 \\ 0 \\ 2 \end{pmatrix}\right\|^2} = \frac{-4-(-16)}{16+0+4} = 
      \frac{12}{20}=\frac{3}{5}. \]
      \lang{de}{Also ist}\lang{en}{Thus} 
      $d(S,E_{ABC})=d(S,L_S)=\|\overrightarrow{SL_S}\ |= \|t_0\vec{n}\|=|t_0|\sqrt{20} = 
      \frac{6}{5}\sqrt{5}$.
\end{enumerate}
\lang{de}{Insgesamt also}
\lang{en}{In general we therefore have}
\[ V=\frac{1}{6}d(A,B) d(C,g_{AB}) d(S,E_{ABC})
=\frac{1}{6}\cdot \sqrt{5}\cdot 2\cdot \frac{6}{5}\sqrt{5}=2.\] 
\end{example}

\section{\lang{de}{Abst"ande paralleler Geraden bzw. Ebenen}
         \lang{en}{Distances between parallel lines or planes}}

\begin{rule}\label{rule:abst_ebene_ebene}
\lang{de}{
Wenn sich zwei Ebenen $E_1$ und $E_2$ im $\R^3$ nicht schneiden, sind sie \textbf{parallel}. 
In diesem Fall kann man ihren \emph{Abstand $d(E_1,E_2)$} berechnen, indem man einen Punkt auf einer 
der beiden Ebenen w"ahlt und dessen Abstand zur anderen Ebene berechnet.\\
Sind beide Ebenen in Normalenform gegeben $E_1=\{\vec{x} \mid \vec{x}\bullet \vec{n}= d_1 \}$ 
und $E_2=\{\vec{x} \mid \vec{x}\bullet \vec{n}= d_2 \}$ mit demselben Normalenvektor $\vec{n}$, 
so kann man den Abstand auch direkt berechnen durch
}
\lang{en}{
If two planes $E_1$ and $E_2$ in $\R^3$ do not intersect, they are \textbf{parallel}. In this case 
we can calculate the \emph{distance $d(E_1,E_2)$} between them by choosing any point on one of the 
two planes, then finding the distance between this point and the other plane.\\
If the two planes are given in normal form, $E_1=\{\vec{x} \mid \vec{x}\bullet \vec{n}= d_1 \}$ 
and $E_2=\{\vec{x} \mid \vec{x}\bullet \vec{n}= d_2 \}$, with the same normal vector $\vec{n}$, 
then the distance between them can also be directly calculated by
}
\[ d(E_1,E_2) = \frac{|d_2-d_1|}{\|\vec{n}\|}. \]  
\end{rule}

\begin{example}
\lang{de}{Wir betrachten die Ebenen}
\lang{en}{Consider the planes}
\[E=\left\{   \begin{pmatrix}x_1 \\ x_2\\ x_3 \end{pmatrix}\in \R^3\, \mid\, 
\begin{pmatrix}-1 \\ 2\\ 1 \end{pmatrix}\bullet \begin{pmatrix}x_1 \\ x_2\\ x_3 \end{pmatrix} = 
3 \right\}\]
\lang{de}{und}\lang{en}{and}
\[F=\left\{\begin{pmatrix}x_1 \\ x_2\\ x_3 \end{pmatrix}\in \R^3\, \mid\, 
\begin{pmatrix}-2 \\ 4\\ 2 \end{pmatrix}\bullet \begin{pmatrix}x_1 \\ x_2\\ x_3 \end{pmatrix} = 
4 \right\}.\]
\lang{de}{
Da die Normalenvektoren Vielfache voneinander sind, sind die Ebenen parallel.
\\\\
Um den Abstand der Ebenen zu berechnen gibt es jetzt im Wesentlichen drei Möglichkeiten
}
\lang{en}{
As the normal vectors are multiples of each other, the two planes are parallel.
\\\\
We now have three methods with which we could calculate the distance between the planes.
}
\begin{tabs*}
\tab{\lang{de}{mit Punkt auf $F$}\lang{en}{with a point on $F$}}
\lang{de}{
Wir suchen einen Punkt auf $F$, z.B. $Q=(0;1;0)$ und berechnen mit der Punkt-Ebenen-Abstandsformel
}
\lang{en}{
We look for a point on $F$, for example $Q=(0;1;0)$, and use the formula for the distance between 
a point and a plane:
}
\[d(E,F)=d(E,Q)=\frac{\left|3-\begin{pmatrix}-1 \\ 2\\ 1 \end{pmatrix}\bullet 
\begin{pmatrix}0 \\ 1\\ 0 \end{pmatrix}\right|}
{\left\Vert \begin{pmatrix}-1 \\ 2\\ 1 \end{pmatrix}\right\Vert} = 
\frac{|3-2|}{\sqrt{6}}=\frac{1}{\sqrt{6}}.\]
\tab{\lang{de}{mit Punkt auf $E$}\lang{en}{with a point on $E$}}
\lang{de}{
Wir suchen einen Punkt auf $E$, z.B. $P=(0;0;3)$ und berechnen mit der Punkt-Ebenen-Abstandsformel
}
\lang{en}{
We look for a point on $E$, for example $P=(0;0;3)$, and use the formula for the distance between 
a point and a plane:
}
\[ d(E,F)=d(P,F)=\frac{\left|4-\begin{pmatrix}-2 \\ 4\\ 2 \end{pmatrix}\bullet 
\begin{pmatrix}0 \\ 0\\ 3 \end{pmatrix}\right|}
{\left\|\begin{pmatrix}-2 \\ 4\\ 2 \end{pmatrix}\right\Vert} = 
\frac{|4-6|}{\sqrt{24}}=\frac{2}{\sqrt{24}}=\frac{1}{\sqrt{6}}.\]
\tab{\lang{de}{mit gleichen Normalenvektoren}\lang{en}{by scaling the normal vectors}}
\lang{de}{
Wir skalieren die Normalenform von $F$ (oder analog die von $E$) so, dass sie den gleichen 
Normalenvektor hat wie die Normalenform von $E$:
}
\lang{en}{
We scale the normal form of $F$ (or analogously the normal form of $E$) so that it has the same 
normal vector as in the normal form of $E$:
}
\[ \begin{pmatrix}-2 \\ 4\\ 2 \end{pmatrix}\bullet \begin{pmatrix}x_1 \\ x_2\\ x_3 \end{pmatrix}=4 \Leftrightarrow
\begin{pmatrix}-1 \\ 2\\ 1 \end{pmatrix}\bullet \begin{pmatrix}x_1 \\ x_2\\ x_3 \end{pmatrix}=2 \quad (\text{Teilen durch }2). \]
\lang{de}{Dann verwenden wir die obige Abstandsformel:}
\lang{en}{Then we apply the formula given in the above rule:}
\[ d(E,F)= \frac{\left|3-2\right|}{\left\|\begin{pmatrix}-1 \\ 2\\ 1 \end{pmatrix}\right\Vert}
= \frac{1}{\sqrt{6}}.\]
\end{tabs*}
%\begin{center}
%\image{T110_DistanceParallelPlanes}
%\end{center}
\end{example}


\begin{rule}\label{rule:abst_gerade_gerade}\label{rule:abst_gerade_ebene}
\lang{de}{
F"ur eine Gerade $g$, die zu einer Ebene $E$ \textbf{parallel} ist, erhält man deren 
\emph{Abstand $d(g,E)$}, indem man einen Punkt auf der Geraden $g$ w"ahlt und dessen Abstand zur 
Ebene $E$ berechnet.
\\\\
Auch f"ur zwei zueinander \textbf{parallele} Geraden $g$ und $h$ berechnet man deren  
\emph{Abstand $d(g,h)$}, indem man einen Punkt auf einer der beiden Geraden w"ahlt und dessen 
Abstand zur anderen Geraden berechnet.
}
\lang{en}{
We obtain the \emph{distance $d(g,E)$} between a line $g$ and a plane $E$ to which it is 
\textbf{parallel} by choosing a point on the line $g$ and calculating its distance from $E$.
\\\\
Similarly, the \emph{distance $d(g,h)$} between two \textbf{parallel} lines $g$ and $h$ is obtained 
by choosing a point on one of the lines and calculating its distance from the other line.
}
\end{rule}


\section{\lang{de}{Abstand windschiefer Geraden im $\R^3$}
         \lang{en}{Distance between skew lines in $\R^3$}}\label{def:abst_windschief_gerade}

\begin{definition}
\lang{de}{
Sind $g= \left\{ \vec{p} + r \vec{v}\mid r \in \R \right\}$ und 
$h= \left\{ \vec{q} + s \vec{w} \mid s \in \R \right\}$ zwei Geraden, die nicht parallel sind, so 
gibt es auf $g$ und $h$ jeweils einen eindeutigen Punkt $L_g$ bzw.~$L_h$ so, dass der Abstand 
$d(L_g,L_h)$ minimal ist unter allen Abst"anden von Punkten auf $g$ und Punkten auf $h$. Per 
Definition ist dann
}
\lang{en}{
Let $g= \left\{ \vec{p} + r \vec{v}\mid r \in \R \right\}$ and 
$h= \left\{ \vec{q} + s \vec{w} \mid s \in \R \right\}$ be two lines that are not parallel. Then 
there are points $L_g$ and ~$L_h$ on the lines $g$ and $h$ that minimise the distance $d(L_g,L_h)$. 
By definition we then have
}
\[d(g,h)=d(L_g,L_h).\]
\lang{de}{
$L_g$ und $L_h$ sind eindeutig dadurch charakterisiert, dass die Gerade durch $L_g$ und $L_h$ sowohl 
auf $g$ als auch auf $h$ senkrecht steht.
}
\lang{en}{
$L_g$ and $L_h$ are uniquely characterised by the following property: the line through $L_g$ and 
$L_h$ is orthogonal to both $g$ and $h$.
}
\end{definition}

\begin{rule}
\lang{de}{
Die Punkte $L_g$ und $L_h$ sind gegeben durch $\overrightarrow{OL_g}= \vec{p} + r_0\vec{v}$ und 
$\overrightarrow{OL_h}= \vec{q} + s_0\vec{w}$, wobei $ \begin{pmatrix} r_0 \\ s_0 \end{pmatrix} $ 
die eindeutige L"osung des Gleichungssystems
}
\lang{en}{
The points $L_g$ and $L_h$ are given by $\overrightarrow{OL_g}= \vec{p} + r_0\vec{v}$ and 
$\overrightarrow{OL_h}= \vec{q} + s_0\vec{w}$, where $ \begin{pmatrix} r_0 \\ s_0 \end{pmatrix} $ 
is the solution to the system of equations
}
 \[ \begin{bmatrix} ( \vec{p} + r \vec{v} -  \vec{q} - s\vec{w}) \bullet \vec{v} &=& 0 \\
 ( \vec{p} + r \vec{v} -  \vec{q} - s\vec{w}) \bullet \vec{w} &=& 0
\end{bmatrix} \quad  \Leftrightarrow \] 
\[ \begin{bmatrix} 
\left\|\vec{v}\right\|^2\cdot r & -& (\vec{v}\bullet \vec{w})\cdot s &=&  (\vec{q}-\vec{p})\bullet \vec{v} \\
  (\vec{v}\bullet \vec{w})\cdot r  & -&  \left\|\vec{w}\right\|^2\cdot s &=& (\vec{q}-\vec{p})\bullet \vec{w}\end{bmatrix} \]
\lang{de}{ist.}
\lang{en}{which is unique.}

\begin{center}
\image{T110_DistanceSkewLines}
\end{center}
\end{rule}

\begin{example}
\lang{de}{Wir betrachten die beiden Geraden }
\lang{en}{Consider the two lines }
$g=\left\{  \begin{pmatrix} 1\\ 1 \\ 0 \end{pmatrix}  
+ r \cdot  \begin{pmatrix} 1 \\ 0\\ 2 \end{pmatrix}  \, | \,\, r\in \R \right\}$ 
\lang{de}{und}\lang{en}{and} 
$h=\left\{  \begin{pmatrix} 2\\ 3 \\ 4 \end{pmatrix}  
+ s \cdot  \begin{pmatrix} 1 \\ 1\\ 2 \end{pmatrix}  \, | \,\, s\in \R \right\}$, 
\lang{de}{
welche zueinander windschief sind. % (vgl.~Bsp. im \link{lagebeziehungen}{vorigen Abschnitt}).
\\\\
Welchen Abstand haben diese zwei Geraden zueinander?
\\\\
F"ur die Lotfu"spunkte $L_g$ und $L_h$ ist also das folgende Gleichungssystem zu l"osen:
}
\lang{en}{
which are skew.
\\\\
What is the distance between these two lines?
\\\\
To find the closest points $L_g$ and $L_h$ on each line, we solve the following system of equations:
}
\[ \begin{bmatrix}  \Big(  \begin{pmatrix} 1\\ 1 \\ 0 \end{pmatrix}  
+ r \cdot  \begin{pmatrix} 1 \\ 0\\ 2 \end{pmatrix} -  \begin{pmatrix} 2\\ 3 \\ 4 \end{pmatrix}  
- s \cdot  \begin{pmatrix} 1 \\ 1\\ 2 \end{pmatrix}  \Big)\bullet
  \begin{pmatrix} 1 \\ 0\\ 2 \end{pmatrix}  =0 \\
\Big(  \begin{pmatrix} 1\\ 1 \\ 0 \end{pmatrix}  
+ r \cdot  \begin{pmatrix} 1 \\ 0\\ 2 \end{pmatrix} -  \begin{pmatrix} 2\\ 3 \\ 4 \end{pmatrix}  
- s \cdot  \begin{pmatrix} 1 \\ 1\\ 2 \end{pmatrix}  \Big)\bullet 
 \begin{pmatrix} 1 \\ 1\\ 2 \end{pmatrix}  =0 \end{bmatrix} 
\Leftrightarrow  \begin{bmatrix}
 5r &-& 5s &=& 9 \\  5r& - & 6s &=& 11
\end{bmatrix} \]
\lang{de}{
Zieht man die zweite Gleichung von der ersten ab, erhält man $s=-2$. 
Setzt man nun $s=-2$ in die erste Gleichung und löst nach $r$ auf, erhält man 
$r=\frac{1}{5}(9-10)=-\frac{1}{5}$.
Die Lotfußpunkte sind also $L_g=(\frac{4}{5}; 1; -\frac{2}{5})$ und $L_h=(0; 1; 0)$.
Der Abstand betr"agt damit:
}
\lang{en}{
Subtracting the second equation from the first gives $s=-2$. Setting $s=-2$ in the first equation 
and rearranging for $r$ yields $r=\frac{1}{5}(9-10)=-\frac{1}{5}$. 
The required points are hence $L_g=(\frac{4}{5}; 1; -\frac{2}{5})$ and $L_h=(0; 1; 0)$. 
The distance between the two lines is therefore:
}
\[ d(g,h)=d(L_g,L_h)=\sqrt{(0-\frac{4}{5})^2+(1-1)^2+(0+\frac{2}{5})^2}=  \sqrt{\frac{16}{25}+\frac{4}{25}}=\frac{2}{5}\sqrt{5}. \]
\end{example}

\lang{de}{
Wenn man die Lotfu"spunkte nicht ben"otigt, sondern nur den Abstand zwischen den windschiefen 
Geraden sucht, gibt es auch folgende Formel:
}
\lang{en}{
If we do not need to find the points on the two lines that are the closest to each other, we can 
also use the following formula:
}
\begin{rule}\label{def:abst_windschief_gerade_n}
\lang{de}{
Sind $g= \left\{ \vec{p} + r \vec{v}\mid r \in \R \right\}$ und 
$h= \left\{ \vec{q} + s \vec{w} \mid s \in \R \right\}$ zwei Geraden im $\R^3$, die nicht parallel 
sind, und $\vec{n}$ ein Vektor, der zu beiden Geraden orthogonal ist, so l"asst sich der Abstand von 
$g$ und $h$ berechnen durch
}
\lang{en}{
If $g= \left\{ \vec{p} + r \vec{v}\mid r \in \R \right\}$ and 
$h= \left\{ \vec{q} + s \vec{w} \mid s \in \R \right\}$ are two non-parallel lines in $\R^3$, and 
$\vec{n}$ is a vector orthogonal to both lines, then the distance between $g$ and $h$ is given by
}
\[ d(g,h)= \frac{\left| (\vec{q}-\vec{p})\bullet \vec{n} \right|}{\|\vec{n}\|}. \]
\end{rule}

\begin{example}
\lang{de}{Im Beispiel mit }
\lang{en}{In the example with }
$g=\left\{  \begin{pmatrix} 1\\ 1 \\ 0 \end{pmatrix}  
+ r  \begin{pmatrix} 1 \\ 0\\ 2 \end{pmatrix}  \, | \,\, r\in \R \right\}$ 
\lang{de}{und}\lang{en}{and} 
$h=\left\{  \begin{pmatrix} 2\\ 3 \\ 4 \end{pmatrix}  
+ s  \begin{pmatrix} 1 \\ 1\\ 2 \end{pmatrix}  \, | \,\, s\in \R \right\}$ 
\lang{de}{ist}\lang{en}{we have}
\[ \vec{n}= \vec{v}\times \vec{w}= \begin{pmatrix} 1 \\ 0\\ 2 \end{pmatrix}  \times 
 \begin{pmatrix} 1 \\ 1\\ 2 \end{pmatrix} 
%= \begin{pmatrix} 0\cdot 2-2\cdot 1 \\ 2\cdot 1-1\cdot 2\\ 1\cdot 1-0\cdot 1 \end{pmatrix} 
= \begin{pmatrix} -2 \\ 0\\ 1 \end{pmatrix}  \quad\text{\lang{de}{und}\lang{en}{and}}\quad \|\vec{n}\|=\sqrt{(-2)^2+1^2}=\sqrt{5} \]
\lang{de}{und daher}\lang{en}{and thus}
\[ d(g,h)= \frac{1}{\sqrt{5}}\cdot \left| \Big( \begin{pmatrix} 2\\ 3 \\ 4 \end{pmatrix} -  \begin{pmatrix} 1\\ 1 \\ 0 \end{pmatrix}  \Big)\bullet  \begin{pmatrix} -2 \\ 0\\ 1 \end{pmatrix}  \right|
= \frac{1}{\sqrt{5}} \big(1\cdot (-2)+3\cdot 0+4\cdot 1\big)=\frac{2}{\sqrt{5}}
=\frac{2}{5}\sqrt{5}. 
\]
\end{example}

\end{visualizationwrapper}


\end{content}