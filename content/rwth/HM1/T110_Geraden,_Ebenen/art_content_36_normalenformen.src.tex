%$Id:  $
\documentclass{mumie.article}
%$Id$
\begin{metainfo}
  \name{
    \lang{de}{Normalenformen von Geraden und Ebenen}
    \lang{en}{Normal form of lines and planes}
  }
  \begin{description} 
 This work is licensed under the Creative Commons License Attribution 4.0 International (CC-BY 4.0)   
 https://creativecommons.org/licenses/by/4.0/legalcode 

    \lang{de}{Beschreibung}
    \lang{en}{Description}
  \end{description}
  \begin{components}
    \component{generic_image}{content/rwth/HM1/images/g_img_00_Videobutton_schwarz.meta.xml}{00_Videobutton_schwarz}
     \component{generic_image}{content/rwth/HM1/images/g_img_00_video_button_schwarz-blau.meta.xml}{00_video_button_schwarz-blau}
    \component{generic_image}{content/rwth/HM1/images/g_tkz_T110_UmwandlungEbenen.meta.xml}{T110_UmwandlungEbenen}
    \component{generic_image}{content/rwth/HM1/images/g_tkz_T110_NormalFormPlane_B.meta.xml}{T110_NormalFormPlane_B}
    \component{generic_image}{content/rwth/HM1/images/g_tkz_T110_NormalFormPlane_A.meta.xml}{T110_NormalFormPlane_A}
    \component{generic_image}{content/rwth/HM1/images/g_tkz_T110_PointNormalFormLine.meta.xml}{T110_PointNormalFormLine}
    \component{generic_image}{content/rwth/HM1/images/g_tkz_T110_NormalFormLine.meta.xml}{T110_NormalFormLine}
  \end{components}
  \begin{links}
    \link{generic_article}{content/rwth/HM1/T109_Skalar-_und_Vektorprodukt/g_art_content_34_vektorprodukt.meta.xml}{vektorprodukt}
    \link{generic_article}{content/rwth/HM1/T101neu_Elementare_Rechengrundlagen/g_art_content_05_loesen_gleichungen_und_lgs.meta.xml}{lgs}
    \link{generic_article}{content/rwth/HM1/T109_Skalar-_und_Vektorprodukt/g_art_content_33_winkel.meta.xml}{winkel}
  \end{links}
  \creategeneric
\end{metainfo}
\begin{content}
\usepackage{mumie.ombplus}
\ombchapter{10}
\ombarticle{2}
\usepackage{mumie.genericvisualization}

\begin{visualizationwrapper}

\title{\lang{de}{Normalenformen von Geraden und Ebenen}\lang{en}{Normal form of lines and planes}}
 
\begin{block}[annotation]
  übungsinhalt
  
\end{block}
\begin{block}[annotation]
  Im Ticket-System: \href{http://team.mumie.net/issues/9053}{Ticket 9053}\\
\end{block}

\begin{block}[info-box]
\tableofcontents
\end{block}

\section{\lang{de}{Normalenform einer Geraden im $\R^2$}
         \lang{en}{Normal form of a line in $\R^2$}}\label{sec:normalform_geraden}

\begin{definition}\label{def:norm_vec_gerade}
\lang{de}{
Es sei $g$ eine Gerade im $\R^2$.\\ Ein \emph{Normalenvektor} der Geraden $g$ ist ein vom Nullvektor 
verschiedener Vektor $\vec{n}$, der senkrecht auf $g$ steht, also orthogonal zum Richtungsvektor von 
$g$ ist. Ist die L\"ange des Vektors $\vec{n}$ gleich $1$, so spricht man auch von einem 
\emph{Einheitsnormalenvektor} oder \emph{normierten Nomalenvektor} und schreibt h\"aufig 
$\vec{n}_{0}$.
}
\lang{en}{
Let $g$ be a line in $\R^2$.\\ A \emph{normal vector} of the line $g$ is a non-zero vector $\vec{n}$ 
that is orthogonal to $g$, that is, orthogonal to the direction vector of $g$. A normal vector of 
length $1$ is called a \emph{unit normal vector} or \emph{normalised normal vector}, and is often 
denoted by $\vec{n}_{0}$.
}
\end{definition}

\begin{example}
\lang{de}{
F"ur $g = \left\{\begin{pmatrix} 1 \\ 2 \end{pmatrix} + 
r \begin{pmatrix} 3/2 \\ 1/2 \end{pmatrix} \, \mid \, r \in \R \right\}$ 
ist $\vec{n}= \begin{pmatrix} -1/2 \\ 3/2 \end{pmatrix} $ ein Normalenvektor, aber auch jedes 
Vielfache von diesem Vektor, also zum Beispiel
}
\lang{en}{
The vector $\vec{n}= \begin{pmatrix} -1/2 \\ 3/2 \end{pmatrix} $ is a normal vector of the line given 
by $g = \left\{\begin{pmatrix} 1 \\ 2 \end{pmatrix} + 
r \begin{pmatrix} 3/2 \\ 1/2 \end{pmatrix} \, \mid \, r \in \R \right\}$, as is any multiple of this 
vector, i.e.
}
\[ \vec{n}_1 =  \begin{pmatrix} -1 \\ 3 \end{pmatrix}  \]

\begin{center}
\image{T110_NormalFormLine}
\end{center}
\end{example}

\begin{rule}
\lang{de}{
Einen Normalenvektor zu $g = \{  \begin{pmatrix} p_1\\ p_2 \end{pmatrix} + 
r  \begin{pmatrix} v_1 \\ v_2 \end{pmatrix}  \, \mid \, r\in \R \}$ zu finden, ist einfach.\\
Ein Vektor ist ja senkrecht zu einem anderen Vektor, wenn das Skalarprodukt der beiden gleich $0$ ist 
(vgl. Abschnitt \link{winkel}{Winkel zwischen Vektoren}).\\
Der Normalenvektor steht senkrecht auf der Geraden, somit muss das Skalarprodukt zwischen 
Normalenvektor und Richtungsvektor der Geraden gleich $0$ sein.\\
Da
}
\lang{en}{
It is easy to find a normal vector to $g = \{  \begin{pmatrix} p_1\\ p_2 \end{pmatrix} + 
r  \begin{pmatrix} v_1 \\ v_2 \end{pmatrix}  \, \mid \, r\in \R \}$.\\
A vector is orthogonal to another vector when their scalar product is equal to zero 
(see \link{winkel}{the section on angles between vectors}).\\
A normal vector is orthogonal to the line, so the scalar product of a normal vector with the 
direction vector of the line is equal to $0$.\\
As
}
\[ \begin{pmatrix} v_1 \\ v_2 \end{pmatrix}\bullet \begin{pmatrix}-v_2 \\ v_1  \end{pmatrix}= v_1\cdot (-v_2)+v_2\cdot v_1=0, \]
\lang{de}{kann man}\lang{en}{we may choose}
\[ \vec{n}=\begin{pmatrix}-v_2 \\ v_1  \end{pmatrix} \]
\lang{de}{wählen.}\lang{en}{as a normal vector.}
\end{rule}

\begin{definition}\label{def:pkt_norm_form_gerade}\label{def:koord_form_gerade}
\lang{de}{
Sei $g=\{ \vec{p}+r\vec{v} \mid r\in \R \}$ eine Gerade im $\R^2$ und $\vec{n}$ ein Normalenvektor 
zu $g$. Dann l"asst sich die Gerade $g$ auch beschreiben durch
}
\lang{en}{
Let $g=\{ \vec{p}+r\vec{v} \mid r\in \R \}$ be a line in $\R^2$ and $\vec{n}$ a normal vector of $g$. 
Then the line $g$ can be described as
}
\[ g= \{ \vec{x}\in \R^2 \mid (\vec{x}-\vec{p})\bullet \vec{n}=0 \}. \]
\lang{de}{Diese Darstellung der Geraden nennt man \emph{Punkt-Normalenform} von $g$.}
\lang{en}{This is called a \emph{point-normal form} of the line $g$.}

\begin{center}
\image{T110_PointNormalFormLine}
\end{center}

\lang{de}{Durch Umformen erh"alt man}
\lang{en}{By rearranging we obtain}
\[g= \{ \vec{x}\in \R^2 \mid \vec{x} \bullet \vec{n}=d \} \quad 
  \text{\lang{de}{mit}\lang{en}{with}} 
  \quad d= \vec{p} \bullet \vec{n},\]
\lang{de}{
welche \emph{Normalenform} genannt wird.
\\
Durch Ausschreiben des Skalarprodukts erh\"alt man die \emph{Koordinatenform} der Geraden
}
\lang{en}{
which is called the \emph{normal form} of $g$.
\\\\
By the definition of the scalar product, the \emph{coordinate form} of the line is
}
\[ g=\left\{  \begin{pmatrix}x_1 \\ x_2 \end{pmatrix}\in \R^2\, \mid\, n_1x_1+n_2x_2=d \right\}. \]

\end{definition}


\begin{example}
\lang{de}{
F"ur $g = \left\{\begin{pmatrix} 1 \\ 2 \end{pmatrix} + 
r \begin{pmatrix} 3/2 \\ 1/2 \end{pmatrix}   \, \mid \, r \in \R \right\}$ 
hatten wir einen Normalenvektor $\vec{n}= \begin{pmatrix} -1 \\ 3 \end{pmatrix} $ und daher
}
\lang{en}{
For $g = \left\{\begin{pmatrix} 1 \\ 2 \end{pmatrix} + 
r \begin{pmatrix} 3/2 \\ 1/2 \end{pmatrix}   \, \mid \, r \in \R \right\}$ 
we have a normal vector $\vec{n}= \begin{pmatrix} -1 \\ 3 \end{pmatrix} $ and hence
}
\begin{eqnarray*} g &=&  \{ \begin{pmatrix}x_1 \\ x_2 \end{pmatrix}\in \R^2 
\mid \Big(\begin{pmatrix}x_1 \\ x_2 \end{pmatrix}- \begin{pmatrix} 1 \\ 2 \end{pmatrix} \Big)\bullet 
\begin{pmatrix}-1 \\ 3 \end{pmatrix}=0 \} \\
 &=&  \{ \begin{pmatrix}x_1 \\ x_2 \end{pmatrix}\in \R^2 
\mid \begin{pmatrix}x_1 \\ x_2 \end{pmatrix} \bullet \begin{pmatrix}-1 \\ 3 \end{pmatrix}
=  \begin{pmatrix} 1 \\ 2 \end{pmatrix}\bullet \begin{pmatrix}-1 \\ 3 \end{pmatrix}= 5 \} \\
 &=&  \{ \begin{pmatrix}x_1 \\ x_2 \end{pmatrix}\in \R^2 \mid
-x_1+3x_2=5 \}.
\end{eqnarray*}
\end{example}


\begin{quickcheck}
		\field{rational}
		\type{input.function}
		\begin{variables}
			
		 
            \function{x}{5}
            \function{y}{2}
            
            \function{e}{24}
           
  		\end{variables}
		
		\text{\lang{de}{%\notion{Kurztest:}\\
          Eine Gerade geht durch den Punkt $\begin{pmatrix}4 \\ 2 \end{pmatrix}$ und besitzt den 
          Normalenvektor $\vec{n}= \begin{pmatrix}5 \\ 2 \end{pmatrix}$.\\
          Geben Sie die Gleichung der Geraden in der Koordinatenform an. \\
          (Tipp: Stellen Sie zuerst die Punkt-Normalenform auf und wandeln Sie diese anschließend in 
          die Koordinatenform um.)
          }
          \lang{en}{
          A line passes through the point $\begin{pmatrix}4 \\ 2 \end{pmatrix}$ and has a normal 
          vector  $\vec{n}= \begin{pmatrix}5 \\ 2 \end{pmatrix}$.\\
          Give the coordinate form of the equation of the line.\\
          (Tip: first find the point-normal form and then transform this into the coordinate form.)
          }\\
          $g = \{ \begin{pmatrix}x_1 \\ x_2 \end{pmatrix}\in \R^2 \mid$ 
          \ansref $x_1+$ \ansref $x_2=$\ansref $ \}$
        }
        
		
		\begin{answer}
			\solution{x}
		\end{answer}
        \begin{answer}
			\solution{y}
		\end{answer}
         \begin{answer}
			\solution{e}
		\end{answer}
         
        
        \explanation{\lang{de}{Die Punkt-Normalenform lautet:}
        \lang{en}{The point-normal form is:}\\
        $g= \{ \vec{x}\in \R^2 \mid (\vec{x}-\vec{p})\bullet \vec{n}=0 \}$.\\
        \lang{de}{Nach dem Einsetzen von Punkt und Normalenvektor erhält man:}
        \lang{en}{By substituting the points into this we obtain:}\\
        $g= \{ \vec{x}\in \R^2 \mid (\begin{pmatrix}x_1 \\ x_2 \end{pmatrix}-\begin{pmatrix}4 \\ 2 \end{pmatrix})\bullet \begin{pmatrix}5 \\ 2 \end{pmatrix}=0 \}$.\\
        \lang{de}{Die Umwandlung in die Koordinatenform entsteht durch das Lösen des Skalarproduktes:}
        \lang{en}{This is changed into the coordinate form by evaluating the scalar product:}\\
        $g   =  \{ \begin{pmatrix}x_1 \\ x_2 \end{pmatrix}\in \R^2 \mid \Big(\begin{pmatrix}x_1 \\ x_2 \end{pmatrix}- \begin{pmatrix} 4 \\ 2 \end{pmatrix} \Big)\bullet \begin{pmatrix}5 \\ 2 \end{pmatrix}=0 \} $\\
         $g    =  \{ \begin{pmatrix}x_1 \\ x_2 \end{pmatrix}\in \R^2 
\mid \begin{pmatrix}x_1 \\ x_2 \end{pmatrix} \bullet \begin{pmatrix} 5\\ 2 \end{pmatrix}
=  \begin{pmatrix} 4 \\ 2 \end{pmatrix}\bullet \begin{pmatrix}5 \\2 \end{pmatrix}= 24 \} $
        $ g   =  \{ \begin{pmatrix}x_1 \\ x_2 \end{pmatrix}\in \R^2 \mid
5x_1+2x_2=24 \}.$    
        }
		
\end{quickcheck}




\begin{rule}
\lang{de}{
Wird der Normalenvektor $\vec{n}$ durch seine L\"ange bzw. durch seinen Betrag $|\vec{n}|$ geteilt, 
so erhält man den normierten Normalenvektor $\vec{n}_0$, der die Länge $1$ besitzt.
}
\lang{en}{
If we divide a normal vector $\vec{n}$ by its length $|\vec{n}|$, we obtain the unit normal vector 
$\vec{n}_0$ which has length $1$.
}\\
\[ \vec{n}_0 = \frac{\vec{n}}{|\vec{n}|} \]
\end{rule}

\begin{definition}
\lang{de}{Die \emph{Hessesche Normalenform}}
\lang{en}{The \emph{Hessian normal form}}\label{def:hesse_nf_gerade}
\[g= \{ \vec{x}\in \R^2 \mid \vec{x} \bullet \vec{n}_0=d_0 \} \quad 
  \text{\lang{de}{mit}\lang{en}{with}} 
  \quad d_0= \vec{p} \bullet \vec{n}_0,\]
\lang{de}{
einer Geraden $g$ wird durch die Division der \emph{Normalenform} durch den Betrag des 
Normalenvektors $|\vec{n}|$ erreicht.
\\\\
Durch Ausschreiben des Skalarprodukts erh"alt man die \emph{Hessesche Normalenform in Koordinaten} 
der Geraden
}
\lang{en}{
of a line $g$ is found by dividing the \emph{normal form} by the length of the normal vector 
$|\vec{n}|$.
\\\\
Evaluating the scalar product yields the \emph{Hessian normal form in coordinates} of the line,
}
\[ g=\left\{  \begin{pmatrix}x_1 \\ x_2 \end{pmatrix}\in \R^2\, \mid\, {n_0}_1x_1+{n_0}_2x_2=d_0 \right\}. \]

\lang{de}{
\floatright{\href{https://api.stream24.net/vod/getVideo.php?id=10962-2-10795&mode=iframe&speed=true}{\image[75]{00_video_button_schwarz-blau}}}\\
}
\lang{en}{}
\end{definition}

\begin{example}
\lang{de}{
Mit $g = \{ \begin{pmatrix}x_1 \\ x_2 \end{pmatrix}\in \R^2 \mid -x_1+3x_2=5 \}$ ist die 
\emph{Normalenform} der Geraden $g$ gegeben.\\
Ihr Normalenvektor $\vec{n}= \begin{pmatrix} -1 \\ 3 \end{pmatrix}$ ist als Koeffizienten vor $x_1$ 
und $x_2$ abzulesen.
\\\\
Zur Berechnung der \emph{Hesseschen Normalenform} wird die Gleichung der \emph{Normalenform} durch den Betrag von $\vec{n}$ dividiert.
}
\lang{en}{
Consider a line $g$ with normal form 
$g = \{ \begin{pmatrix}x_1 \\ x_2 \end{pmatrix}\in \R^2 \mid -x_1+3x_2=5 \}$.\\
Its normal vector $\vec{n}= \begin{pmatrix} -1 \\ 3 \end{pmatrix}$ can be read from the coefficients 
of $x_1$ and $x_2$.
\\\\
To find the \emph{Hessian normal form} we divide the equation of the \emph{normal form} by the length 
of the normal vector $\vec{n}$.
}\\
\[ |\vec{n}| = \sqrt{(-1)^2+(3)^2} = \sqrt{10} \]
\lang{de}{Somit ist}\lang{en}{Thus}
\[g =  \{ \begin{pmatrix}x_1 \\ x_2 \end{pmatrix}\in \R^2 \mid -\frac{1}{\sqrt{10}}x_1+\frac{3}{\sqrt{10}}x_2=\frac{5}{\sqrt{10}} \}\]
\lang{de}{die Hessesche Normalenform von $g$.}
\lang{en}{is the Hessian normal form of $g$.}
\end{example}

\begin{quickcheck}
		\field{rational}
		\type{input.function}
		\begin{variables}
			
		 
            \function{n}{5}
            \function{x}{3/5}
            \function{y}{0}
            \function{z}{-4/5}
            \function{e}{2}
           
  		\end{variables}
		
		\text{\lang{de}{%\notion{Kurztest:}\\
        Bestimmen Sie die Hessesche Normalenform der Ebene $E: 3x_1-4x_3=10$. \\
        Der Betrag des Normalenvektors $|\vec{n}|= $\ansref. \\
        Somit ist die Ebene in der Hesseschen Normalenform  
        }
        \lang{en}{
        Determine the Hessian normal form of the plane $E: 3x_1-4x_3=10$. \\
        The length of the normal vector is $|\vec{n}|= $\ansref. \\
        Using this, the plane in Hessian normal form is
        }
        $E_0$: \ansref $x_1 + $\ansref $x_2 +$ \ansref $x_3 =$ \ansref.
        }
        
		
		\begin{answer}
			\solution{n}
		\end{answer}
        \begin{answer}
			\solution{x}
		\end{answer}
         \begin{answer}
			\solution{y}
		\end{answer}
         \begin{answer}
			\solution{z}
		\end{answer}
         \begin{answer}
			\solution{e}
		\end{answer}
        
        \explanation{\lang{de}{
        Der Normalenvektor wird von den Vorfaktoren von $x_1$, $x_2$ und $x_3$ definiert. \\
        $\vec{n}=\begin{pmatrix}3 \\0 \\ -4 \end{pmatrix}$.\\
        Sein Betrag berechnet sich mit $|\vec{n}|=\sqrt{3^2+(-4)^2}=\sqrt{25}=5$.\\
        Zur Umwandlung der Ebenengleichung in die Hessesche Normalenform, muss die Koordinatenform 
        durch den Betrag des Normalenvektors geteilt werden. 
        Diese Division führt zum Endergebnis: $E_0: \frac{3}{5}x_1-\frac{4}{5}x_3=2$.
        }
        \lang{en}{
        The normal vector can be read from the coefficients of $x_1$, $x_2$ and $x_3$. \\
        $\vec{n}=\begin{pmatrix}3 \\0 \\ -4 \end{pmatrix}$.\\
        Its length is $|\vec{n}|=\sqrt{3^2+(-4)^2}=\sqrt{25}=5$.\\
        To transform the equation of the plane into Hessian normal form, the coordinate form must be 
        divided by the length of the normal vector. 
        This yields $E_0: \frac{3}{5}x_1-\frac{4}{5}x_3=2$.
        }}
		
\end{quickcheck}


\begin{block}[warning]
\lang{de}{
Eine Normalenform, Hessesche Normalenform und Koordinatenform einer Geraden gibt es nur im $\R^2$. Im 
$\R^3$ (oder im $\R^n$ mit $n\geq 3$) gibt es zu Geraden keine Normalenform, da es viele verschiedene 
Richtungen gibt, die zur Geraden senkrecht sind.
}
\lang{en}{
The normal form, Hessian normal form and coordinate form of a line only exist in $\R^2$. In $\R^3$ 
(or in $\R^n$ with $n\geq 3$) there cannot be a normal form of a line, as there are multiple 
directions to which the line is orthogonal.
}
\end{block}


\section{\lang{de}{Umrechnungen der Darstellungsformen von Geraden}
         \lang{en}{Converting between representations of a line}}\label{sec:umrechng_geradenform}

\lang{de}{
Wie man aus einer Parameterform eine Normalenform und eine Koordinatenform bekommt, war ja oben schon 
zu sehen. Hier wird noch erklärt, wie man aus einer Koordinatenform der Geraden im $\R^2$ eine 
Parameterform erhält.
\\\\
Um aus der Koordinatenform 
$g=\left\{  \begin{pmatrix}x_1 \\ x_2 \end{pmatrix}\in \R^2\, \mid\, n_1x_1+n_2x_2=d \right\}$ 
einer Geraden $g$ wieder eine Parameterform zu bekommen, 
kann man zum einen die L"osungsmenge des Gleichungssystems $n_1x_1+n_2x_2=d$ (1 Gleichung, 
2 Variablen) bestimmen (vgl. Abschnitt \link{lgs}{Lineare Gleichungssysteme}).
}
\lang{en}{
We have already seen above the conversion of a parametrised line into normal form or coordinate form. 
Now we explain how to find the parametrised form from the coordinate form of a line in $\R^2$.
\\\\
To obtain the parametrised form of a line in coordinate form 
$g=\left\{  \begin{pmatrix}x_1 \\ x_2 \end{pmatrix}\in \R^2\, \mid\, n_1x_1+n_2x_2=d \right\}$,
we find the solution set of the linear system $n_1x_1+n_2x_2=d$ (1 equation, 2 variables) using 
\link{lgs}{methods from an earlier chapter}.
}

\begin{example}
\lang{de}{
Die \emph{Koordinatenform} 
$g =  \{ \begin{pmatrix}x_1 \\ x_2 \end{pmatrix}\in \R^2 \mid -x_1+3x_2=5 \}$ liefert die Gleichung:
}
\lang{en}{
The \emph{coordinate form} 
$g =  \{ \begin{pmatrix}x_1 \\ x_2 \end{pmatrix}\in \R^2 \mid -x_1+3x_2=5 \}$ uses the equation:
}

\[-x_1 + 3x_2 = 5.\]

\lang{de}{Löst man diese Gleichung nach $x_1$ auf, erhält man}
\lang{en}{Rearranging this for $x_1$ gives}\\

\[x_1 = - 5 + 3x_2.\]

\lang{de}{
Die Parameterabhängigkeit zwischen $x_1$ und $x_2$ kann durch einen Parameter $s$ veranschaulicht 
werden.\\
Setzt man $x_2 = s$, dann erhält man das lineare Gleichungssstem
}
\lang{en}{
The relationship between $x_1$ and $x_2$ can be represented via a parameter $s$.\\
Set $x_2 = s$, so we obtain the linear system
}
\begin{align*}x_2 &= s \\ x_1 &= 3s - 5\end{align*}

\lang{de}{Das Ergebnis von $x_1$ und $x_2$ nun als Vektor notiert ist}
\lang{en}{The parametrised $x_1$ and $x_2$ can be written as a vector,}

\[
\begin{pmatrix}x_1 \\ x_2 \end{pmatrix} = \begin{pmatrix}-5 + 3s \\ s \end{pmatrix} = 
\begin{pmatrix}-5 \\ 0 \end{pmatrix} + s \begin{pmatrix}3 \\ 1 \end{pmatrix},
\]
\lang{de}{also lautet die \emph{Parameterform}}
\lang{en}{so the \emph{parametrised form} of the line is}
\[g=\left\{  \begin{pmatrix} -5 \\ 0 \end{pmatrix}  + s \begin{pmatrix} 3 \\ 1 \end{pmatrix} \, \mid 
  \, s \in \R \right\}. \]
\end{example}

\lang{de}{
Alternativ kann man auch einen Punkt auf $g$ berechnen (indem man z.B. $x_1=0$ w"ahlt und 
$x_2=\frac{d}{n_2}$ bzw. $x_2=0$ w"ahlt und $x_1=\frac{d}{n_1}$, je nachdem, ob $n_1$ oder $n_2$ 
nicht $0$ ist) und einen Richtungsvektor von $g$ berechnen, z.B. 
$\vec{v}= \begin{pmatrix}n_2 \\ -n_1\end{pmatrix}$.
}
\lang{en}{
Alternatively we can also find a point on $g$ (by choosing for example $x_1=0$ and 
$x_2=\frac{d}{n_2}$, or $x_2=0$ and $x_1=\frac{d}{n_1}$ depending on which of $n_1$ or $n_2$ is 
non-zero) and a direction vector of $g$, i.e. $\vec{v}= \begin{pmatrix}n_2 \\ -n_1\end{pmatrix}$.
}

\begin{example}
\lang{de}{
F"ur $g =  \{ \begin{pmatrix}x_1 \\ x_2 \end{pmatrix}\in \R^2 \mid -x_1+3x_2=5 \}$ bestimmt man 
zuerst den Normalenvektor
}
\lang{en}{
A normal vector of $g =  \{ \begin{pmatrix}x_1 \\ x_2 \end{pmatrix}\in \R^2 \mid -x_1+3x_2=5 \}$ is
}
\[\vec{n} = \begin{pmatrix}-1 \\ 3 \end{pmatrix}.\]

\lang{de}{
Der Normalenvektor steht senkrecht auf dem Richtungsvektor der Geraden, daher wäre ein passender 
Richtungsvektor
}
\lang{en}{
The normal vector is orthogonal to the direction vector of the line, so an appropriate direction 
vector would be
}

\[ \vec{v} = \begin{pmatrix}3 \\ 1 \end{pmatrix}.\]

\lang{de}{
Zur Berechnung des Stützpunktes kann in der \emph{Koordinatengleichung} $-x_1 + 3x_2 = 5$ die Variable $x_2 = 0$ gesetzt werden, dann berechnet man $x_1$.\\
Dann erhält man $x_1 = -5$.
\\\\
Somit liegt der Stützpunkt bei $P=(-5; 0)$.
\\\\
Eine mögliche \emph{Parameterform} lautet also
}
\lang{en}{
To find a constant point on the line we can set $x_2 = 0$ in the equation $-x_1 + 3x_2 = 5$, and find 
$x_1$.\\
We obtain $x_1 = -5.$
\\\\
Hence we can choose $P=(-5; 0)$ as our constant point.
\\\\
A \emph{parametrisation} of the line is therefore
}

\[ g=\left\{  \begin{pmatrix} -5 \\ 0 \end{pmatrix}  + s  \begin{pmatrix} 3 \\ 1 \end{pmatrix}   \, \mid \, s \in \R \right\}. \]
\end{example}

\section{\lang{de}{Normalenform einer Ebene im $\R^3$}
         \lang{en}{Normal form of a plane in $\R^3$}}\label{sec:ebenen}


\begin{definition}\label{def:norm_vec_ebene}
\lang{de}{
Es sei $E$ eine Ebene im $\R^3$. Ein \emph{Normalenvektor} der Ebene $E$ ist ein vom Nullvektor 
verschiedener Vektor $\vec{n}$, der senkrecht auf $E$ steht, also orthogonal zu den beiden 
Richtungsvektoren der Ebene ist. Ist die L\"ange des Vektors $\vec{n}$ gleich $1$, so spricht man 
auch von einem \emph{Einheitsnormalenvektor} und schreibt h\"aufig $\vec{n}_{0}$.
}
\lang{en}{
Let $E$ be a plane in $\R^3$. A \emph{normal vector} of the plane $E$ is a non-zero vector $\vec{n}$ 
that is orthogonal to $E$, that is, orthogonal to both direction vectors of the plane. If the length 
of $\vec{n}$ is equal to $1$, then we call it a \emph{unit normal vector} or a \emph{normalised 
normal vector} and often denote it by $\vec{n}_{0}$.
}

\begin{center}
\image{T110_NormalFormPlane_A}
\end{center}
\end{definition} 

\begin{rule}
\lang{de}{
Einen Normalenvektor zu einer Ebene zu bestimmen, ist einfach, wenn man zwei Richtungsvektoren kennt.
Sind nämlich $\vec{v}$ und $\vec{w}$ Richtungsvektoren von $E$, so liefert das Vektorprodukt 
$\vec{v} \times \vec{w}$ einen Normalenvektor (vgl. \ref[vektorprodukt][Eigenschaften des Vektorprodukts]{sec:eigenschaften}).
}
\lang{en}{
It is easy to find a normal vector to a plane if we have two non-zero non-parallel direction vectors. 
If $\vec{v}$ and $\vec{w}$ are such direction vectors of $E$, the cross product 
$\vec{v} \times \vec{w}$ immediately yields a normal vector (see 
\ref[vektorprodukt][properties of the cross product]{sec:eigenschaften}).
}
\end{rule}

\begin{example}
\lang{de}{
Eine Ebene besitze die zwei Richtungsvektoren $\vec{v} = \begin{pmatrix} 1 \\ 0 \\ 2 \end{pmatrix}$ 
und $\vec{w} = \begin{pmatrix} 2 \\ 2 \\ 4 \end{pmatrix}$.
\\\\
Der Normalenvektor $\vec{n}$ der Ebene steht senkrecht auf $\vec{v}$ und $\vec{w}$ und wird durch das 
Vektorprodukt berechnet.
}
\lang{en}{
Suppose a plane has two direction vectors $\vec{v} = \begin{pmatrix} 1 \\ 0 \\ 2 \end{pmatrix}$ and 
$\vec{w} = \begin{pmatrix} 2 \\ 2 \\ 4 \end{pmatrix}$.
\\\\
The normal vector $\vec{n}$ of the plane is orthogonal to both $\vec{v}$ and $\vec{w}$ and is found 
via the cross product of these vectors.
}

\[\vec{n} = 
  \begin{pmatrix} 1 \\ 0 \\ 2 \end{pmatrix} \times \begin{pmatrix} 2 \\ 2 \\ 4 \end{pmatrix} = 
  \begin{pmatrix} 0 \cdot 4 - 2 \cdot 2 \\ 2 \cdot 2 - 1 \cdot 4 \\ 1 \cdot 2 - 0 \cdot 2 \end{pmatrix} = 
  \begin{pmatrix} -4 \\ 0 \\ 2 \end{pmatrix}.\]

\end{example}

\begin{definition}\label{def:norm_form_ebene}\label{def:hesse_nf_ebene}
\lang{de}{
Sei $E=\left\{ \vec{p}+r\vec{v}+s\vec{w} \mid r,s\in \R \right\}$ eine Ebene und $\vec{n}$ ein 
Normalenvektor zu $E$ (z.B. $\vec{n}=\vec{v}\times \vec{w}$). Dann l"asst sich die Ebene $E$ auch 
beschreiben durch
}
\lang{en}{
Let $E=\left\{ \vec{p}+r\vec{v}+s\vec{w} \mid r,s\in \R \right\}$ be a plane and $\vec{n}$ a normal 
vector of $E$ (e.g. $\vec{n}=\vec{v}\times \vec{w}$). Then the plane $E$ can also be described by
}
\[ E=\{ \vec{x}\in \R^3 \mid (\vec{x}- \vec{p}) \bullet \vec{n}=0 \}. \]
\lang{de}{Diese Darstellung der Ebene nennt man \emph{Punkt-Normalenform} von $E$.}
\lang{en}{This representation of a plane is called the \emph{point-normal form} of $E$.}

\begin{center}
\image{T110_NormalFormPlane_B}
\end{center}

\lang{de}{Durch Umformen erh"alt man}
\lang{en}{Rearranging gives us}
\[ E= \{ \vec{x}\in \R^3 \mid \vec{x} \bullet \vec{n}=d \} \quad 
\text{\lang{de}{mit}\lang{en}{with}} 
\quad d = \vec{p} \bullet \vec{n},\]
\lang{de}{
welche \emph{Normalenform} genannt wird, bzw. \emph{Hessesche Normalenform}, wenn der Normalenvektor 
$\vec{n}$ die L"ange $1$ hat.
\\\\
Durch Ausschreiben des Skalarprodukts erh"alt man die \emph{Koordinatenform} der Ebene
}
\lang{en}{
which is called the \emph{normal form}, or the \emph{Hessian normal form} if the normal vector 
$\vec{n}$ has length $1$.
\\\\
By evaluating the scalar product we obtain the \emph{coordinate form} of the plane,
}
\[E = \left\{   \begin{pmatrix}x_1 \\ x_2\\ x_3 \end{pmatrix}\in \R^3\, \mid\, 
  n_1x_1+n_2x_2+n_3x_3=d \right\}. \]

\lang{de}{
\floatright{\href{https://api.stream24.net/vod/getVideo.php?id=10962-2-10796&mode=iframe&speed=true}{\image[75]{00_video_button_schwarz-blau}}
\href{https://www.hm-kompakt.de/video?watch=732}{\image[75]{00_Videobutton_schwarz}}}\\\\
}
\lang{en}{}
\end{definition}

\begin{example}\label{bsp:Ebene:ParameterKoordinatenform}
\lang{de}{F"ur }\lang{en}{For }
$E=\left\{  \begin{pmatrix} 1 \\ 1 \\ 0 \end{pmatrix}  
+ r  \begin{pmatrix} 1 \\ 0 \\ 2 \end{pmatrix}  
+ s  \begin{pmatrix} 2 \\ 2 \\ 4 \end{pmatrix}  \mid r, \ s \in \R \right\}$ 
\lang{de}{ist}\lang{en}{we have a normal vector}
\[ \vec{n}=  \begin{pmatrix} 1 \\ 0 \\ 2 \end{pmatrix} \times 
 \begin{pmatrix} 2 \\ 2 \\ 4 \end{pmatrix}  
=\begin{pmatrix} -4 \\ 0 \\ 2 \end{pmatrix} \]
\lang{de}{ein Normalenvektor und daher ist}\lang{en}{and hence}
\begin{eqnarray*}
E &=& \left\{   \begin{pmatrix} x_1 \\ x_2 \\ x_3 \end{pmatrix} \in \R^3 \, \mid\,
\Big(    \begin{pmatrix} x_1 \\ x_2 \\ x_3 \end{pmatrix} -  \begin{pmatrix} 1 \\ 1 \\ 0 \end{pmatrix}  \Big)
\bullet \begin{pmatrix} -4 \\ 0 \\ 2 \end{pmatrix} =0 \right\}\\
 &=&  \left\{   \begin{pmatrix} x_1 \\ x_2 \\ x_3 \end{pmatrix} \in \R^3 \, \mid\,
  \begin{pmatrix} x_1 \\ x_2 \\ x_3 \end{pmatrix} \bullet \begin{pmatrix} -4 \\ 0 \\ 2 \end{pmatrix} 
  =  \begin{pmatrix} 1 \\ 1 \\ 0 \end{pmatrix} \bullet \begin{pmatrix} -4 \\ 0 \\ 2 \end{pmatrix} = -4 \right\}\\
 &=&  \left\{   \begin{pmatrix} x_1 \\ x_2 \\ x_3 \end{pmatrix} \in \R^3 \, \mid\,
 -4x_1+2x_3=-4 \right\}.
\end{eqnarray*}
\end{example}


\begin{block}[warning]
\lang{de}{
Eine Normalenform und eine Koordinatenform einer Ebene gibt es nur im $\R^3$. Im $\R^n$ mit $n>3$ 
gibt es zu Ebenen keine Normalenform, da es viele Richtungen gibt, die zur Ebene senkrecht sind.
}
\lang{en}{
The normal form and the coordinate form of a plane only exist in $\R^3$. In $\R^n$ with $n>3$ there 
cannot be a normal form of a plane, as there are multiple directions to which the plane is orthogonal.
}
\end{block}

\section{\lang{de}{Umrechnungen der Darstellungsformen von Ebenen}
         \lang{en}{Converting between representations of a plane}}\label{sec:umrechng_ebenenform}

\lang{de}{
Die nun bekannten Darstellungsformen der Ebenen, wie \emph{Parameterform}, \emph{Punkt-Normalenform}, 
\emph{Hessesche Normalenform} und \emph{Koordinatenform} lassen sich problemlos ineinander umformen.\\
Die folgende Abbildung soll die Umformung verdeutlichen.
}
\lang{en}{
Now that we have introduced the different ways of representing a plane in $\R^3$, i.e. the 
\emph{parametrised form}, \emph{point-normal form}, \emph{Hessian normal form} and 
\emph{coordinate form}, they can be converted between.\\
The following figure illustrates how this is done in each case.
}

\begin{center}
\image{T110_UmwandlungEbenen}
\end{center}

\lang{de}{
Wie man aus der Parameterform eine Normalenform und eine Koordinatenform bekommt, ist in Beispiel 
\ref{bsp:Ebene:ParameterKoordinatenform} zu sehen. Hier wird noch erklärt, wie man aus einer 
Koordinatenform der Ebene im $\R^3$ eine Parameterform erhält.
}
\lang{en}{
The conversion from a parametrised equation of a plane in $\R^3$ to normal form and to coordinate 
form is shown in example \ref{bsp:Ebene:ParameterKoordinatenform}. Below we explain how to convert 
from coordinate form to parametrised form.
}

\begin{rule}
\lang{de}{
Um aus der Koordinatenform 
$E=\left\{\begin{pmatrix}x_1 \\ x_2\\x_3 \end{pmatrix}\in \R^3\, \mid 
\, n_1x_1+n_2x_2+n_3x_3=d \right\}$ 
wieder eine Parameterform zu bekommen, ist eine Möglichkeit, die L"osungsmenge des Gleichungssystems 
$n_1x_1+n_2x_2+n_3x_3=d$ (1 Gleichung, 3 Variablen) zu bestimmen (vgl. Abschnitt 
\link{lgs}{Lineare Gleichungssysteme}).
\\\\
Alternativ kann man auch
}
\lang{en}{
To parametrise the coordinate equation 
$E=\left\{\begin{pmatrix}x_1 \\ x_2\\x_3 \end{pmatrix}\in \R^3\, \mid 
\, n_1x_1+n_2x_2+n_3x_3=d \right\}$ 
we can again find the solution set of the linear system $n_1x_1+n_2x_2+n_3x_3=d$ (1 equation, 3 
variables) \link{lgs}{using methods from an earlier chapter}.
\\\\
Alternatively we can
}
\begin{itemize}
\item \lang{de}{
      einen Punkt auf $E$ berechnen, indem man z.B. $x_1=x_2=0$ w"ahlt und $x_3=\frac{d}{n_3}$ 
      bzw. $x_1=x_3=0$ w"ahlt und $x_2=\frac{d}{n_2}$ bzw. $x_2=x_3=0$ w"ahlt und 
      $x_1=\frac{d}{n_1}$, je nachdem, ob $n_1$, $n_2$ oder $n_3$ nicht $0$ ist,
      }
      \lang{en}{
      find a point on $E$ (by choosing  for example $x_1=x_2=0$ and $x_3=\frac{d}{n_3}$, or 
      $x_1=x_3=0$ and $x_2=\frac{d}{n_2}$, or $x_2=x_3=0$ and $x_1=\frac{d}{n_1}$, depending on which 
      of $n_1$, $n_2$ or $n_3$ is non-zero),
      }
\item \lang{de}{und zwei Richtungsvektoren von $E$ berechnen, z.B. zwei der drei Vektoren}
      \lang{en}{and find two direction vectors of $E$, i.e. any two of the three vectors}
      \[\begin{pmatrix} n_2 \\ -n_1\\ 0\end{pmatrix},
        \quad \begin{pmatrix} n_3 \\ 0\\ -n_1\end{pmatrix}\quad 
        \text{\lang{de}{und}\lang{en}{and}} \quad \begin{pmatrix} 0 \\ n_3\\ -n_2\end{pmatrix} \]
      \lang{de}{(je nachdem welche zwei linear unabhängig sind).}
      \lang{en}{that are linearly independent.}
\end{itemize}
\end{rule}

\lang{de}{Eine Parameterform der Ebene}
\lang{en}{Suppose we want to find the parametrised form of the plane given by}
\[ E =  \left\{   \begin{pmatrix} x_1 \\ x_2 \\ x_3 \end{pmatrix} \in \R^3 \, \mid\,
 -4x_1+2x_3=-4 \right\}\] 
\lang{de}{
soll bestimmt werden.\\
Hier gibt es analog zur Berechnung der Parameterform einer Geraden zwei Möglichkeiten:
}
\lang{en}{
in coordinate form.\\
Like when parametrising a line, we have two possible strategies:
}

\begin{example}

\lang{de}{
Der Normalenvektor, den man aus der Koordinatengleichung abliest, ist 
$\vec{n}=\begin{pmatrix}-4\\ 0\\ 2 \end{pmatrix}$.
\\\\
Um einen Punkt auf $E$ zu bekommen, wählt man z.B. $x_1=x_2=0$ und erhält für $x_3$ die Gleichung 
$2x_3=-4$, also $x_3=-2$.\\
Ein Punkt auf $E$ ist also z.B. $P=(0;0;-2)$.\\
\emph{Achtung:} Hier konnte man z.B. nicht $x_1=x_3=0$ wählen, weil dies zur unlösbaren Gleichung 
$0+0=-4$ führt.
\\\\
Um zwei Richtungsvektoren zu bekommen, betrachtet man am einfachsten die oben angegebenen drei 
Vektoren
}
\lang{en}{
The normal vector $\vec{n}=\begin{pmatrix}-4\\ 0\\ 2 \end{pmatrix}$ can be read from the coefficients 
of the coordinates.
\\\\
To find a point on $E$, we choose for example $x_1=x_2=0$ and obtain for $x_3$ the equation 
$2x_3=-4$, so $x_3=-2$.\\
Hence $P=(0;0;-2)$ is a point on $E$.\\
\emph{Beware:} we cannot choose i.e. $x_1=x_3=0$ here, as this leads to the unsolvable equation 
$0+0=-4$.
\\\\
To find two direction vectors, we simply choose from the three vectors given in the above rule,
}
\[\begin{pmatrix} n_2 \\ -n_1\\ 0\end{pmatrix} = \begin{pmatrix} 0 \\ 4\\ 0\end{pmatrix},
  \quad \begin{pmatrix} n_3 \\ 0\\ -n_1\end{pmatrix}=\begin{pmatrix} 2 \\ 0\\ 4\end{pmatrix}\quad 
  \text{\lang{de}{und}\lang{en}{and}} 
  \quad \begin{pmatrix} 0 \\ n_3\\ -n_2\end{pmatrix}=\begin{pmatrix} 0 \\ 2\\ 0\end{pmatrix}. \]
\lang{de}{
Der dritte ist ein Vielfaches des ersten, aber der zweite und der erste sind linear unabhängig. Eine 
Parameterform von $E$ ist daher gegeben durch
}
\lang{en}{
The third vector is a multiple of the first, but the second and the first are linearly independent. 
$E$ can hence be parametrised as
}
\[ E=\left\{  \begin{pmatrix} 0 \\ 0 \\ -2 \end{pmatrix}  
+ r  \begin{pmatrix} 0 \\ 4 \\ 0 \end{pmatrix}  
+ s  \begin{pmatrix} 2 \\ 0 \\ 4 \end{pmatrix}  \mid r, \ s \in \R \right\}. \]
\end{example}

\lang{de}{Alternativ kann man auch hier das lineare Gleichungssystem mit zwei Unbekannten lösen:}
\lang{en}{Alternatively we can also solve the linear system with two unknowns:}\\

\begin{example}
\lang{de}{Die Gleichung der Koordinatenform war}
\lang{en}{The equation in coordinate form was}
\[
-4x_1 + 2x_3 = -4.
\]

\lang{de}{
Da in dieser Gleichung die unbekannte $x_2$ nicht auftaucht, kann sie als Variable $x_2=r$ angesehen 
werden.
\\\\
Löst man die oben stehende Gleichung nun nach $x_3$ auf, erhält man:
}
\lang{en}{
As the unknown $x_2$ does not appear in this equation, we can view it as a parameter $x_2=r$.
\\\\
Rearranging the above equation for $x_3$ gives:
}

\[
x_3 = -2 +2x_1.
\]

\lang{de}{
Die Parameterabhängigkeit zwischen $x_1$ und $x_3$ kann durch den Parameter $s$ veranschaulicht 
werden.\\
Setzt man $x_1=s$, dann erhält man das Gleichungssystem
}
\lang{en}{
The relationship between $x_1$ and $x_3$ can be represented via a parameter $s$.\\
Setting $x_1=s$ we obtain the system
}

\begin{align*}
	  x_1 &= s\\
    x_2 &= r\\
	  x_3 &= -2+2s
\end{align*}

\lang{de}{Das Ergebnis von $x_1$, $x_2$ und $x_3$ wird nun als Vektor notiert}
\lang{en}{The parametrised $x_1$, $x_2$ and $x_3$ can now be written as a vector,}

\[ \begin{pmatrix} x_1 \\ x_2 \\ x_3 \end{pmatrix} = \begin{pmatrix} s \\ r \\ -2+2s \end{pmatrix}  
= \begin{pmatrix} 0 \\ 0 \\ -2 \end{pmatrix}  + r \begin{pmatrix} 0 \\ 1 \\ 0 \end{pmatrix} 
+ s \begin{pmatrix} 1 \\ 0 \\ 2 \end{pmatrix},
\]

\lang{de}{also lautet die Parameterform}
\lang{en}{so the \emph{parametrised form} of the plane is}

\[E=\left\{  \begin{pmatrix} 0 \\ 0 \\ -2 \end{pmatrix}  + 
  r  \begin{pmatrix} 0 \\ 1 \\ 0 \end{pmatrix} + 
  s  \begin{pmatrix} 1 \\ 0 \\ 2 \end{pmatrix}  \mid r, \ s \in \R \right\}. \]

\lang{de}{
Auf dem ersten Blick sind die beiden errechneten Parameterformen nicht identisch - jedoch fällt 
schnell auf, dass die Richtungsvektoren $\begin{pmatrix} 0 \\ 1 \\ 0 \end{pmatrix}$ und 
$\begin{pmatrix} 0 \\ 4 \\ 0 \end{pmatrix}$, sowie 
$\begin{pmatrix} 1 \\ 0 \\ 2 \end{pmatrix}$ und $\begin{pmatrix} 2 \\ 0 \\ 4 \end{pmatrix}$ jeweils 
linear abhängig sind. Daher sind die Ebenen identisch.
}
\lang{en}{
At first glance, the two parametrised equations are not the same - however, it is easy to see that 
the direction vectors $\begin{pmatrix} 0 \\ 1 \\ 0 \end{pmatrix}$ and 
$\begin{pmatrix} 0 \\ 4 \\ 0 \end{pmatrix}$, and the direction vectors 
$\begin{pmatrix} 1 \\ 0 \\ 2 \end{pmatrix}$ and $\begin{pmatrix} 2 \\ 0 \\ 4 \end{pmatrix}$ are 
linearly dependent. Hence the planes described by the two parametrisations are identical.
}



\end{example}




\end{visualizationwrapper}


\end{content}