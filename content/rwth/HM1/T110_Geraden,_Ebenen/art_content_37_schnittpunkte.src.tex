%$Id:  $
\documentclass{mumie.article}
%$Id$
\begin{metainfo}
  \name{
    \lang{de}{Lagebeziehungen}
    \lang{en}{Intersections between points, lines and planes}
  }
  \begin{description} 
 This work is licensed under the Creative Commons License Attribution 4.0 International (CC-BY 4.0)   
 https://creativecommons.org/licenses/by/4.0/legalcode 

    \lang{de}{Beschreibung}
    \lang{en}{Description}
  \end{description}
  \begin{components}
    \component{generic_image}{content/rwth/HM1/images/g_tkz_T110_ParallelPlanes.meta.xml}{T110_ParallelPlanes}
    \component{generic_image}{content/rwth/HM1/images/g_tkz_T110_IdenticalPlanes.meta.xml}{T110_IdenticalPlanes}
    \component{generic_image}{content/rwth/HM1/images/g_tkz_T110_IntersectingPlanes.meta.xml}{T110_IntersectingPlanes}
    \component{generic_image}{content/rwth/HM1/images/g_tkz_T110_LineParallelToPlane.meta.xml}{T110_LineParallelToPlane}
    \component{generic_image}{content/rwth/HM1/images/g_tkz_T110_LineInPlane.meta.xml}{T110_LineInPlane}
    \component{generic_image}{content/rwth/HM1/images/g_tkz_T110_LineIntersectingPlane.meta.xml}{T110_LineIntersectingPlane}
    \component{generic_image}{content/rwth/HM1/images/g_tkz_T110_ParallelLines.meta.xml}{T110_ParallelLines}
    \component{generic_image}{content/rwth/HM1/images/g_tkz_T110_IdenticalLines.meta.xml}{T110_IdenticalLines}
    \component{generic_image}{content/rwth/HM1/images/g_tkz_T110_SkewLines.meta.xml}{T110_SkewLines}
    \component{generic_image}{content/rwth/HM1/images/g_tkz_T110_IntersectingLines.meta.xml}{T110_IntersectingLines}
    \component{generic_image}{content/rwth/HM1/images/g_img_00_video_button_schwarz-blau.meta.xml}{00_video_button_schwarz-blau}
    \component{js_lib}{system/media/mathlets/GWTGenericVisualization.meta.xml}{mathlet1}
  \end{components}
  \begin{links}
  \end{links}
  \creategeneric
\end{metainfo}
\begin{content}
\usepackage{mumie.ombplus}
\ombchapter{10}
\ombarticle{3}
\usepackage{mumie.genericvisualization}

\begin{visualizationwrapper}

\title{\lang{de}{Lagebeziehungen zwischen Punkten, Geraden und Ebenen}
       \lang{en}{Intersections between points, lines and planes}}
 
\begin{block}[annotation]
  Inhalt: OMB+-Abschnitt X.6 ergänzt durch Lagebestimmungen mit Ebenen in Normalenform
  
\end{block}
\begin{block}[annotation]
  Im Ticket-System: \href{http://team.mumie.net/issues/9054}{Ticket 9054}\\
\end{block}

\begin{block}[info-box]
\tableofcontents
\end{block}

\lang{de}{
In diesem Abschnitt beschränken wir uns auf den dreidimensionalen Raum $\R^3$ und untersuchen
die Lagebeziehungen zwischen Punkten, Geraden und Ebenen im $\R^3$.
}
\lang{en}{
In this section we restrict ourselves to considering the intersections that can occur between points, 
lines and planes in three-dimensional space $\R^3$.
}


\section{\lang{de}{Allgemeine Lagebeziehungen}\lang{en}{General relationships}}
\label{sec:6_1_Allgemeine_Lagebeziehungen}
%\lref{Allgemeine Lagebeziehungen}{Allgemeine Lagebeziehungen}}}

\lang{de}{Für die Lagebeziehungen von Punkten, Geraden und Ebenen können folgende Fälle auftreten:}
\lang{en}{The following cases can arise for the relationships between points, lines, and planes:}

%\image{6_0_Lagebeziehung_P-g-E}

\begin{table}
	&    \lang{de}{Punkt}  \lang{en}{Point}  
	&    \lang{de}{Gerade} \lang{en}{Line}   
	&    \lang{de}{Ebene}  \lang{en}{Plane}     \\
	     \lang{de}{Punkt}  \lang{en}{Point}
	&    \lang{de}{Punkte sind identisch oder\newline verschieden} \lang{en}{Points are either identical or \newline distinct}
	&    \lang{de}{Punkt liegt auf Gerade oder \newline nicht }    \lang{en}{Points either lie on a line or \newline do not lie on a line}
	&    \lang{de}{Punkt liegt in Ebene oder \newline nicht}       \lang{en}{Points either lie on a plane or \newline do not lie on a plane}     \\
	     \lang{de}{Gerade} \lang{en}{Line}
	& 
	&    \lang{de}{Geraden sind identisch,\newline liegen parallel,\newline schneiden sich in einem Punkt oder \newline liegen windschief }
		 \lang{en}{Lines are identical,\newline parallel,\newline intersect each other at a point or are \newline skew}
	&   \lang{de}{Gerade liegt in Ebene,\newline parallel zu Ebene oder \newline durchstößt Ebene }
	    \lang{en}{Lines lie on a plane,\newline are parallel to a plane, or \newline intersect a plane}      \\
	    \lang{de}{Ebene } \lang{en}{Plane}
	& 
	& 
	&   \lang{de}{Ebenen sind identisch,\newline liegen parallel oder \newline schneiden sich in einer Geraden} 
	    \lang{en}{Planes are identical,\newline parallel, or \newline intersect each other on a line}
\end{table}


\section{\lang{de}{Punkt - Punkt}\lang{en}{Point - Point}}
\label{sec:6_2_Punkt-Punkt}

\lang{de}{Zwei Punkte $P$ und $Q$ können entweder gleich oder verschieden sein.}
\lang{en}{Two points can either be the same or different.}

\begin{rule}
  \lang{de}{
  Zwei Punkte $P=(p_1;p_2;p_3)$ und $Q=(q_1;q_2;q_3)$ in $\R^3$ sind genau dann gleich, wenn sie in 
  jeder Koordinate übereinstimmen, d. h. wenn
  }
  \lang{en}{
  Two points $P=(p_1;p_2;p_3)$ and $Q=(q_1;q_2;q_3)$ in $\R^3$ are the same if and only if each of 
  their coordinates is the same. That is, if
  }
	\[p_1 = q_1,\quad p_2=q_2,\quad\text{\lang{de}{und}\lang{en}{and}}\quad p_3=q_3\]
	\lang{de}{gelten.	Ansonsten sind die Punkte verschieden.}
	\lang{en}{If two points are not the same, they are different.}
\end{rule}



\section{\lang{de}{Punkt - Gerade}\lang{en}{Point - Line}}
\label{sec:6_3_Punkt-Gerade}

\lang{de}{Ein Punkt $P$ kann entweder auf einer Geraden $g$ liegen oder nicht.}
\lang{en}{A point can either lie on a line or not lie on a line.}

\begin{rule}
  \lang{de}{Ein Punkt $P$ liegt genau dann auf der Geraden}
  \lang{en}{Consider the line $g$}
	\[g: \vec{x} = \vec{OQ} + \lambda\cdot \vec{v},\quad \lambda\in\R\]
  \lang{de}{
	durch $Q$ mit Richtungsvektor $\vec{v}$, wenn es eine reelle Zahl $\lambda\in\R$ gibt, so dass 
  $\vec{OP} = \vec{OQ} + \lambda\cdot \vec{v}$ gilt, d. h.	wenn der Vektor $\vec{OP}$ zu den 
  Vektoren $\vec{x}$ gehört, die die Gerade beschreiben.
  }
  \lang{en}{
	which goes through the point $Q$ with direction vector $\vec{v}$. A point $P$ lies on $g$ if 
  there is a real number $\lambda\in\R$ such that $\vec{OP} = \vec{OQ} + \lambda\cdot \vec{v}$. This 
  is the same as saying that the vector $\vec{OP}$ is of the same form as $\vec{x}$ (as these are the 
  vectors that describe the line $g$).
  }
\end{rule}

\begin{example}
  \lang{de}{
  Wir prüfen, ob der Punkt $P=(1;2;3)$ auf der Geraden $g$ liegt, die durch den Punkt $Q=(-1;2;0)$ 
  und den Richtungsvektor $\vec{v} = \begin{pmatrix} 2\\0\\3\end{pmatrix}$ gegeben ist, also
  }
  \lang{en}{
  Suppose that the line $g$ goes through the point $Q=(-1,2,0)$ and has a direction vector of 
  $\vec{v} = \begin{pmatrix} 2\\0\\3\end{pmatrix}$. Let us now check whether or not the point 
  $P=(1,2,3)$ lies on this line.
  }
	\[g: \vec{x} = \begin{pmatrix} -1\\2\\0\end{pmatrix} + 
    \lambda\cdot \begin{pmatrix} 2\\0\\3\end{pmatrix},\quad \lambda\in\R\lang{de}{.}\]
	
  \lang{de}{Der Ansatz}
  \lang{en}{Substituting into the equation}
	\[\vec{OP} = \vec{OQ} + \lambda\cdot \vec{v}\]
  \lang{de}{liefert}
  \lang{en}{gives us}
	\[\begin{pmatrix} 1\\2\\3\end{pmatrix} = 
    \begin{pmatrix} -1\\2\\0\end{pmatrix} + \lambda\cdot \begin{pmatrix} 2\\0\\3\end{pmatrix} = 
    \begin{pmatrix} -1+2\lambda\\2\\3\lambda\end{pmatrix}.\]
  \lang{de}{Dies ist ein lineares Gleichungssystem mit den drei Gleichungen}
  \lang{en}{This is a system of linear equations with three equations}
	\begin{align*}
		1 & = -1+2\lambda\\
		2 & = 2 \\
		3 & = 3\lambda
	\end{align*}
  \lang{de}{
  und der Unbekannten $\lambda$. Dieses Gleichungssystem besitzt genau eine Lösung $\lambda=1$. Also 
  liegt der Punkt $P$ auf der Geraden $g$.
  }
	\lang{en}{
  and a single unknown, $\lambda$. This system has exactly one solution $\lambda=1$, hence the point 
  $P$ does lie on $g$.
  }
\end{example}

\section{\lang{de}{Punkt - Ebene}\lang{en}{Point - Plane}}
\label{sec:6_4_Punkt-Ebene}

\lang{de}{Ein Punkt $P$ kann entweder in einer Ebene $E$ liegen oder nicht.}
\lang{en}{A point can either lie in a plane or not lie in a plane.}

\begin{rule}
  \lang{de}{Ein Punkt $P$ liegt genau dann in der Ebene}
  \lang{en}{A point $P$ lies on the plane}
	\[E: \vec{x} = \vec{OQ} + \lambda\cdot \vec{v} + \mu\cdot\vec{w},\quad \lambda,\mu\in\R\]
  \lang{de}{
  durch $Q$ mit Richtungsvektoren $\vec{v}$ und $\vec{w}$, wenn es zwei reelle Zahlen 
  $\lambda,\mu\in\R$ gibt, so dass $\vec{OP} = \vec{OQ} + \lambda\cdot \vec{v} + \mu\cdot\vec{w}$ 
  gilt, d. h. wenn	der Vektor $\vec{OP}$ zu den Vektoren $\vec{x}$ gehört, die die Ebene beschreiben.
  }
	\lang{en}{
  containing the point $Q$ with direction vectors $\vec{v}$ and $\vec{w}$ if and only if there exist 
  two real numbers $\lambda,\mu\in\R$ such that 
  $\vec{OP} = \vec{OQ} + \lambda\cdot \vec{v} + \mu\cdot\vec{w}$.
	This is the same as	saying that the vector $\vec{OP}$ is of the same form as $\vec{x}$ (as these 
  are the vectors that describe the plane).
  }
\end{rule}

\lang{de}{
Einfacher zu überprüfen ist es, wenn die Ebene $E$ in Normalenform oder Koordinatenform vorliegt.
}
\lang{en}{
It is easier to check this if the plane $E$ is expressed in normal form or coordinate form.
}
\begin{rule}
  \lang{de}{Ein Punkt $P=(p_1;p_2;p_3)$ liegt genau dann in der Ebene}
  \lang{en}{A point $P=(p_1;p_2;p_3)$ lies on the plane}
  \[ E= \{\vec{x}\in \R^3 \mid \vec{n} \bullet \vec{x}=d \} = 
  \left\{\begin{pmatrix}x_1 \\ x_2\\ x_3 \end{pmatrix}\in \R^3\, \mid\, 
  n_1x_1+n_2x_2+n_3x_3=d \right\}, \]
  \lang{de}{wenn}\lang{en}{if and only if}
	\[ \vec{n}\bullet \overrightarrow{OP} = d \]
	\lang{de}{bzw.}\lang{en}{or}
	\[ n_1p_1+n_2p_2+n_3p_3=d \]
	\lang{de}{gilt.}\lang{en}{holds.}
\end{rule}

\begin{example}\label{ex:punkt-ebene}
  \lang{de}{
  Wir prüfen, ob der Punkt $P=(3;2;1)$ in der Ebene $E$ liegt, die durch den Punkt $Q=(1;2;0)$ und 
  die Richtungsvektoren $\vec{v} = \begin{pmatrix} 1\\0\\1\end{pmatrix}$ und 
  $\vec{w} = \begin{pmatrix} 1\\1\\-1\end{pmatrix}$ gegeben ist, also
  }
  \lang{en}{
  Let $E$ be the plane that goes through the point $Q=(1,2,0)$ and has direction vectors 
  $\vec{v} = \begin{pmatrix} 1\\0\\1\end{pmatrix}$ and 
  $\vec{w} = \begin{pmatrix} 1\\1\\-1\end{pmatrix}$. 
  We can now check whether the point $P=(3,2,1)$ lies in $E$.
  }
	\[E: \vec{x} = \begin{pmatrix} 1\\2\\0\end{pmatrix} + 
    \lambda\cdot \begin{pmatrix} 1\\0\\1\end{pmatrix} + 
    \mu\cdot\begin{pmatrix} 1\\1\\-1\end{pmatrix} ,\quad \lambda,\mu\in\R.\]
\begin{tabs*}
\tab{\lang{de}{mit Parameterform}\lang{en}{In parameter form}}    
  \lang{de}{Der Ansatz}
  \lang{en}{Substituting into the equation}
	\[\vec{OP} = \vec{OQ} + \lambda\cdot \vec{v} + \mu\cdot\vec{w}\]
  \lang{de}{liefert}
  \lang{en}{gives us}
	\[\begin{pmatrix} 3\\2\\1\end{pmatrix} = 
    \begin{pmatrix} 1\\2\\0\end{pmatrix} + 
    \lambda\cdot \begin{pmatrix} 1\\0\\1\end{pmatrix} + 
    \mu\cdot\begin{pmatrix} 1\\1\\-1\end{pmatrix} = 
    \begin{pmatrix} 1+\lambda + \mu\\2+\mu\\\lambda-\mu\end{pmatrix}.\]
  \lang{de}{Dies ist ein lineares Gleichungssystem mit den drei Gleichungen}
  \lang{en}{This is a linear system with three equations}
	\begin{align*}
		3 & = 1+\lambda + \mu\\
		2 & = 2 + \mu\\
		1 & = \lambda - \mu
	\end{align*}
  \lang{de}{
  und den zwei Unbekannten $\lambda$ und $\mu$.	Dieses Gleichungssystem besitzt keine Lösung, weil 
  die zweite Gleichung $\mu = 0$ und damit die erste Gleichung $\lambda = 2$ ergibt. Die dritte 
  Gleichung liefert dann den Widerspruch $1=2$. Also liegt der Punkt $P$ nicht in der Ebene $E$.
  }
	\lang{en}{
  and two unknowns, $\lambda$ and $\mu$. This system has no solution, because the second equation 
  gives us $\mu=0$, which leads to $\lambda=2$ from the first equation. Under these conditions, the 
  third equation leads to the contradiction $1=2$, hence the point $P$ does not lie in $E$.
  }
\tab{\lang{de}{mit Koordinatenform}\lang{en}{In coordinate form}}
  \lang{de}{Ein Normalenvektor zu der Ebene $E$ ist}
  \lang{en}{The vector}
  \[\vec{n} = \vec{v}\times \vec{w} = 
    \begin{pmatrix} 1\\ 0\\ 1\end{pmatrix} \times \begin{pmatrix} 1\\ 1\\ -1\end{pmatrix} = 
    \begin{pmatrix} -1\\ 2\\ 1\end{pmatrix}. \]
  \lang{de}{Weiter ist}
  \lang{en}{is a normal vector of the plane $E$. Furthermore,}
  \[\vec{n}\bullet \vec{OQ} = 
    \begin{pmatrix} -1\\ 2\\ 1\end{pmatrix}\bullet \begin{pmatrix} 1\\2\\0\end{pmatrix}=3.\]
  \lang{de}{Eine Koordinatenform für $E$ ist daher}
  \lang{en}{Hence $E$ can be expressed in coordinate form as}
  \[E=\left\{\begin{pmatrix}x_1 \\ x_2\\ x_3 \end{pmatrix}\in \R^3\, \mid\,-x_1+2x_2+x_3=3 \right\}.\]
  \lang{de}{
  Nun kann getestet werden, ob der Punkt $P=(3;2;1)$ in der Ebene $E$ liegt, indem man die 
  Koordinaten von $P$ in die linke Seite der Gleichung einsetzt:
  }
  \lang{en}{
  Now we can test whether the point $P=(3;2;1)$ lies on the plane $E$ by substituting the coordinates 
  of $P$ into the left-hand side of the equation:
  }
  \[  -p_1+2p_2+p_3=-3+2\cdot 2+ 1=2\neq 3. \]
  \lang{de}{Also liegt $P$ nicht in der Ebene $E$.}
  \lang{en}{Hence $P$ does not lie on the plane $E$.}
\end{tabs*}
\end{example}


\section{\lang{de}{Gerade - Gerade}\lang{en}{Line - Line}}
\label{sec:6_5_Gerade-Gerade}

\lang{de}{
Zwei Geraden $g$ und $h$ können entweder identisch sein, oder parallel (aber nicht identisch) sein, 
oder sich in genau einem Punkt schneiden, oder windschief sein.
}
\lang{en}{
Two lines can either be identical, be parallel (but not identical), intersect each other at exactly 
one point, or be skew to each other.
}

\begin{rule}\label{rule:lage_geraden}
  \lang{de}{
  Die zwei Geraden $g: \vec{x} = \vec{OP} + \lambda\cdot \vec{v},\; \lambda\in\R$ und 
  $h: \vec{x} = \vec{OQ} + \mu\cdot \vec{w},\; \mu\in\R$ mit Stützpunkten $P$ bzw. $Q$ und 
  Richtungsvektoren $\vec{v}$ bzw. $\vec{w}$
  }
  \lang{en}{
  Let $g$ be the line $g: \vec{x} = \vec{OP} + \lambda\cdot \vec{v},\; \lambda\in\R$ that goes 
  through the point $P$ and has direction vector $\vec{v}$ and let $h$ be the line 
  $h: \vec{x} = \vec{OQ} + \mu\cdot \vec{w},\; \mu\in\R$ that goes through the point $Q$ and has 
  direction vector $\vec{w}$. These two lines are:
  }
	\begin{itemize}
		\item \lang{de}{
          sind identisch, wenn $\vec{v}$ und $\vec{w}$ parallel sind und $P$ auf der Geraden $h$ 
          liegt.
          }
      		\lang{en}{identical if $\vec{v}$ and $\vec{w}$ are parallel and $P$ lies on $h$.}
		\item \lang{de}{
          sind parallel (aber nicht identisch), wenn $\vec{v}$ und $\vec{w}$ parallel sind und $P$ 
          nicht auf der Geraden $h$ liegt.
          }
      		\lang{en}{
          parallel (but not identical) if $\vec{v}$ and $\vec{w}$ are parallel and $P$ does not lie 
          on $h$.
          }
		\item \lang{de}{
          schneiden sich in genau einem Punkt, wenn $\vec{v}$ und $\vec{w}$ nicht parallel sind und 
          es zwei reelle Zahlen $\lambda,\mu\in\R$ gibt, so dass
    			\[\vec{OP} + \lambda\cdot \vec{v} = \vec{OQ} + \mu\cdot \vec{w}\]
    			gilt. Der zu diesem Vektor gehörende Punkt ist dann der gesuchte Schnittpunkt.
          }
    			\lang{en}{
          intersect each other at exactly one point if $\vec{v}$ and $\vec{w}$ are not parallel, and 
          there exist two real numbers $\lambda,\mu\in\R$ such that
  			  \[\vec{OP} + \lambda\cdot \vec{v} = \vec{OQ} + \mu\cdot \vec{w}.\]
    			The point that belongs to this vector is the desired intersection point.
          }
		\item \lang{de}{
          liegen windschief zueinander, wenn $\vec{v}$ und $\vec{w}$ nicht parallel sind und sich die 
          Geraden in keinem Punkt schneiden, es also keine solche Zahlen $\lambda,\mu\in\R$ gibt.
          }
      		\lang{en}{
          skew to each other if $\vec{v}$ and $\vec{w}$ are not parallel and they do not intersect in 
          a point (i.e. there are no such numbers $\lambda,\mu\in\R$).
          }
	\end{itemize}
  \lang{de}{
  Die ersten drei Fälle können auch in der Ebene auftreten, windschiefe Geraden gibt es nur im 
  dreidimensionalen Raum $\R^3$ (und in höheren Dimensionen).
  }
  \lang{en}{
  The first three cases can take place even in two dimensions. Skew lines can only occur in three 
  dimensions ($\R^3$) or higher dimensions.
  }
\end{rule}

\begin{example}
\begin{tabs*}[\initialtab{0}]
	\tab{\lang{de}{Identische Geraden}\lang{en}{Identical lines}}
  	\lang{de}{In diesem Beispiel betrachten wir die beiden Geraden $g$ und $h$ gegeben durch}
    \lang{en}{Let $g$ and $h$ be lines defined as follows:}
  	\begin{align*}
  		g: \vec{x} & = \begin{pmatrix} 0\\1\\2\end{pmatrix} + 
         \lambda\cdot \begin{pmatrix} 1\\1\\1\end{pmatrix},\quad \lambda\in\R,\\
  		h: \vec{x} & = \begin{pmatrix} 3\\4\\5\end{pmatrix} + 
         \mu\cdot \begin{pmatrix} -1\\-1\\-1\end{pmatrix},\quad \mu\in\R.
  	\end{align*}
    \lang{de}{
    Da die beiden Richtungsvektoren $\begin{pmatrix} 1\\1\\1\end{pmatrix}$ und 
    $\begin{pmatrix} -1\\-1\\-1\end{pmatrix}$ parallel sind, können $g$ und $h$ entweder identisch 
    oder parallel zueinander sein. Der Ansatz
    }
  	\lang{en}{
    Because the two direction vectors $\begin{pmatrix} 1\\1\\1\end{pmatrix}$ and 
    $\begin{pmatrix} -1\\-1\\-1\end{pmatrix}$ are parallel, $g$ and $h$ are either identical or 
    parallel to each other. The equation
    }
  	\[\begin{pmatrix} 3\\4\\5\end{pmatrix} =\begin{pmatrix} 0\\1\\2\end{pmatrix} + 
      \lambda\cdot \begin{pmatrix} 1\\1\\1\end{pmatrix}\]
    \lang{de}{führt auf das Gleichungssystem}
    \lang{en}{leads to the system}
  	\begin{align*}
  		3 & =\lambda \\
  		4 & =1+\lambda \\
  		5 & =2+\lambda,
  	\end{align*}
    \lang{de}{
    welches die Lösung $\lambda=3$ besitzt. Der Stützpunkt $(3;4;5)$ der Geraden $h$ liegt somit auf 
    der Geraden $g$, woraus folgt, dass $g$ und $h$ identisch sind.
    }
    \lang{en}{
    which has solution $\lambda=3$. Therefore, the point $(3;4;5)$ on $h$ does not lie on $g$, which 
    means that $g$ and $h$ are parallel.
    }
    \begin{center}
    \image{T110_IdenticalLines}
    \end{center}    
	\tab{\lang{de}{Parallele Geraden}\lang{en}{Parallel lines}}
		\lang{de}{In diesem zweiten Beispiel untersuchen wir die Lagebeziehung der beiden Geraden}
    \lang{en}{Let $g$ and $h$ be the lines defined as follows:}
  	\begin{align*}
  		g: \vec{x} & = \begin{pmatrix} 0\\1\\1\end{pmatrix} + 
         \lambda\cdot \begin{pmatrix} 1\\1\\1\end{pmatrix},\quad \lambda\in\R,\\
  		h: \vec{x} & = \begin{pmatrix} 3\\4\\6\end{pmatrix} + 
         \mu\cdot \begin{pmatrix} -1\\-1\\-1\end{pmatrix},\quad \mu\in\R.
  	\end{align*}
    \lang{de}{
    Da die beiden Richtungsvektoren unverändert parallel sind, können $g$ und $h$ auch hier nur 
    identisch oder parallel zueinander sein. Der Ansatz
    }
  	\lang{en}{
    Because the two direction vectors are parallel, $g$ and $h$ are either identical or 
    parallel to each other.	The equation
    }
  	\[\begin{pmatrix} 3\\4\\6\end{pmatrix} = 
      \begin{pmatrix} 0\\1\\1\end{pmatrix} + \lambda\cdot \begin{pmatrix} 1\\1\\1\end{pmatrix}\]
    \lang{de}{führt auf das Gleichungssystem}
    \lang{en}{leads to the system}
  	\begin{align*}
  		3 & =\lambda \\
  		4 & =1+\lambda \\
  		6 & =1+\lambda,
  	\end{align*}
    \lang{de}{
    welches keine Lösung besitzt, da die letzten beiden Gleichungen sich widersprechen. Der 
    Stützpunkt $(3;4;6)$ der Geraden $h$ liegt somit nicht auf der Geraden $g$, woraus folgt, dass 
    $g$ und $h$ parallel sind.
    }
    \lang{en}{
    which has no solution. Therefore, the point $(3;4;6)$ on $h$ does not lie on $g$, which means 
    that $g$ and $h$ are parallel.
    }
    \begin{center}
    \image{T110_ParallelLines}
    \end{center}
	\tab{\lang{de}{Sich schneidende Geraden}\lang{en}{Intersecting lines}}
		\lang{de}{
    In diesem dritten Beispiel untersuchen wir die Lagebeziehung der beiden Geraden $g$ und $h$ 
    gegeben durch
    }
    \lang{en}{
    Let $g$ and $h$ be the lines defined as follows:
    }
  	\begin{align*}
  		g: \vec{x} & = \begin{pmatrix} 1\\0\\1\end{pmatrix} + 
         \lambda\cdot \begin{pmatrix} 1\\2\\3\end{pmatrix},\quad \lambda\in\R,\\
  		h: \vec{x} & = \begin{pmatrix} 1\\1\\3\end{pmatrix} + 
         \mu\cdot \begin{pmatrix} 1\\1\\1\end{pmatrix},\quad \mu\in\R.
  	\end{align*}
    \lang{de}{
    Da $\begin{pmatrix} 1\\2\\3\end{pmatrix}$ und $\begin{pmatrix} 1\\1\\1\end{pmatrix}$ nicht 
    parallel sind, können sich $g$ und $h$ entweder in genau einem Punkt schneiden, oder windschief 
    zueinander sein. Der Ansatz
    }
  	\lang{en}{
    Because $\begin{pmatrix} 1\\2\\3\end{pmatrix}$ and $\begin{pmatrix} 1\\1\\1\end{pmatrix}$ are not 
    parallel, $g$ and $h$ either intersect each other at exactly one point, or are skew to each 
    other. The equation
    }
  	\[\begin{pmatrix} 1\\0\\1\end{pmatrix} + \lambda\cdot \begin{pmatrix} 1\\2\\3\end{pmatrix} = 
      \begin{pmatrix} 1\\1\\3\end{pmatrix} + \mu\cdot \begin{pmatrix} 1\\1\\1\end{pmatrix}\]
    \lang{de}{führt auf das Gleichungssystem}
    \lang{en}{leads to the system}
  	\begin{align*}
  		1+\lambda & = 1+\mu \\
  		2\lambda & = 1+\mu \\
  		1+3\lambda & = 3+\mu,
  	\end{align*}
    \lang{de}{also}
    \lang{en}{which after some simplification is}
    \begin{align*}
  		\lambda - \mu & = 0 \\
  		2\lambda - \mu & = 1 \\
  		3\lambda -\mu & = 2.
  	\end{align*}
    \lang{de}{
    Dieses Gleichungssystem besitzt die Lösung $\lambda=1$, $\mu=1$. Daher schneiden sich die Geraden 
    in genau einem Punkt, der sich durch Einsetzen dieser Werte in die Geraden $g$ bzw. $h$ ergibt. 
    Mit $\lambda = 1$ in der Geradengleichung von $g$ ergibt sich der Punkt $S$ mit dem Ortsvektor
    }
  	\lang{en}{
    The solution of this system is $\lambda=1$, $\mu=1$, hence the lines intersect each other at 
    exactly one point. Substituting these values into either equation $g$ or $h$ gives the same 
    result. For example, substituting $\lambda=1$ into the equation of $g$ yields the point $S$ 
    with position vector
    }
  	\[\vec{OS} = \begin{pmatrix} 1\\0\\1\end{pmatrix} + 1\cdot \begin{pmatrix} 1\\2\\3\end{pmatrix} = 
      \begin{pmatrix}2\\2\\4\end{pmatrix}.\]
    \lang{de}{
    Derselbe Punkt ergibt sich natürlich auch durch Einsetzen von $\mu=1$ in die Geradengleichung von 
    $h$:
    }
    \lang{en}{
    The same point can be found by setting $\mu=1$ in the equation of $h$:
    }
  	\[\vec{OS} = \begin{pmatrix} 1\\1\\3\end{pmatrix} + 1\cdot \begin{pmatrix} 1\\1\\1\end{pmatrix} = 
      \begin{pmatrix}2\\2\\4\end{pmatrix}.\]
    \lang{de}{Also ist $S=(2;2;4)$ der Schnittpunkt der beiden Geraden.}
    \lang{en}{The point $S=(2,2,4)$ is the intersection point of the two lines.}
    \begin{center}
    \image{T110_IntersectingLines}
    \end{center}
	\tab{\lang{de}{Windschiefe Geraden}\lang{en}{Skew lines}}
		\lang{de}{In diesem letzten Beispiel betrachten wir die beiden Geraden}
    \lang{en}{Let $g$ and $h$ be the lines defined as follows:}
  	\begin{align*}
  		g: \vec{x} & = \begin{pmatrix} 1\\3\\1\end{pmatrix} + 
         \lambda\cdot \begin{pmatrix} 2\\-1\\0\end{pmatrix},\quad \lambda\in\R,\\
  		h: \vec{x} & = \begin{pmatrix} 0\\2\\3\end{pmatrix} + 
         \mu\cdot \begin{pmatrix} 1\\1\\0\end{pmatrix},\quad \mu\in\R.
  	\end{align*}
    \lang{de}{
    Da $\begin{pmatrix} 2\\-1\\0\end{pmatrix}$ und $\begin{pmatrix} 1\\1\\0\end{pmatrix}$ nicht 
    parallel sind, können sich $g$ und $h$ entweder nur in genau einem Punkt schneiden, oder 
    windschief zueinander sein.	Der Ansatz
    }
	  \lang{en}{
    Because $\begin{pmatrix} 2\\-1\\0\end{pmatrix}$ and $\begin{pmatrix} 1\\1\\0\end{pmatrix}$ are 
    not parallel, $g$ and $h$ either intersect each other at exactly one point, or are skew to 
    each other.	The equation}
  	\[\begin{pmatrix} 1\\3\\1\end{pmatrix} + \lambda\cdot \begin{pmatrix} 2\\-1\\0\end{pmatrix} = 
      \begin{pmatrix} 0\\2\\3\end{pmatrix} + \mu\cdot \begin{pmatrix} 1\\1\\0\end{pmatrix}\]
    \lang{de}{führt auf das Gleichungssystem}
    \lang{en}{leads to the system}
  	\begin{align*}
  		1+2\lambda & = \mu \\
  		3-\lambda & = 2+\mu \\
  		1 & = 3,
  	\end{align*}
    \lang{de}{
    welches aufgrund der dritten Gleichung offensichtlich keine Lösung besitzt. Die beiden Geraden 
    $g$ und $h$ sind daher windschief zueinander.
    }
    \lang{en}{
    which has no solution. The two lines $g$ and $h$ are therefore skew.
    }
    \begin{center}
    \image{T110_SkewLines}
    \end{center}
\end{tabs*}
\end{example}

\section{\lang{de}{Gerade - Ebene}\lang{en}{Line - Plane}}
\label{sec:6_6_Gerade-Ebene}

\lang{de}{
Eine Gerade $g$ kann entweder in einer Ebene $E$ liegen, parallel zu $E$ liegen, oder die Ebene $E$ 
in genau einem Punkt durchstoßen (also schneiden).
}
\lang{en}{
A line can either lie on a plane, be parallel to a plane, or intersect a plane at exactly one point.
}

\begin{rule}
  \lang{de}{
  Gegeben seien eine Gerade $g: \vec{x} = \vec{OP} + \lambda\cdot \vec{u},\; \lambda\in\R$ mit 
  Stützpunkt $P$ und Richtungsvektor $\vec{u}$ und eine Ebene 
  $E: \vec{x} = \vec{OQ} + \mu\cdot \vec{v} + \sigma \cdot\vec{w},\; \mu,\sigma\in\R$ mit Stützpunkt 
	$Q$ und Richtungsvektoren $\vec{v}$ und $\vec{w}$. Besitzt das Gleichungssystem gegeben durch
  }
	\lang{en}{
  Let $g$ be the line $g: \vec{x} = \vec{OP} + \lambda\cdot \vec{u},\; \lambda\in\R$ which goes 
  through the point $P$ and has direction vector $\vec{u}$,	and let $E$ be the plane 
  $E: \vec{x} = \vec{OQ} + \mu\cdot \vec{v} + \sigma \cdot\vec{w},\; \mu,\sigma\in\R$ which goes 
  through the point $Q$ and has direction vectors $\vec{v}$ and $\vec{w}$. If the system
  }
	\[\vec{OP} + \lambda\cdot \vec{u} = \vec{OQ} + \mu\cdot \vec{v} + \sigma \cdot\vec{w}\]
  \lang{de}{f\"ur die reellen Zahlen $\lambda$, $\mu$ und $\sigma$ }
  \lang{en}{with real numbers $\lambda$, $\mu$ and $\sigma$ has:}
	\begin{itemize}
		\item \lang{de}{unendlich viele Lösungen $(\lambda;\mu;\sigma)$, so liegt $g$ in der Ebene $E$,}
      		\lang{en}{infinitely many solutions $(\lambda,\mu,\sigma)$, then $g$ lies in the plane.}
		\item \lang{de}{
          genau eine Lösung $(\lambda;\mu;\sigma)$, so durchstößt $g$ die Ebene $E$ in dem durch 
      		genau diese Parameter festgelegten Punkt,
          }
      		\lang{en}{
          exactly one solution $(\lambda,\mu,\sigma)$, then $g$ intersects the plane.
          }
		\item \lang{de}{keine Lösung, so liegt $g$ parallel zur Ebene $E$.}
      		\lang{en}{no solution, then $g$ is parallel to the plane.}
	\end{itemize}
\end{rule}

\lang{de}{Auch hier ist die Bestimmung einfacher, wenn die Ebene in Normalenform vorliegt.}
\lang{en}{Here it is also easier if the plane is in normal form.}
\begin{rule}
\lang{de}{
Gegeben seien eine Gerade $g: \vec{x} = \vec{OP} + \lambda\cdot \vec{u},\; \lambda\in\R$ 
und eine Ebene $E= \{ \vec{x}\in \R^3 \mid \vec{n} \bullet \vec{x}=d \}$. Wenn die Gleichung
}
\lang{en}{
Consider the line $g: \vec{x} = \vec{OP} + \lambda\cdot \vec{u},\; \lambda\in\R$ and the plane 
$E= \{ \vec{x}\in \R^3 \mid \vec{n} \bullet \vec{x}=d \}$. If the equation
}
\[ \vec{n}\bullet \left(\vec{OP} + \lambda\cdot \vec{u}\right)=d \]
\begin{itemize}
\item \lang{de}{für alle $\lambda\in \R$ erfüllt ist, so liegt $g$ in der Ebene $E$,}
      \lang{en}{holds for all $\lambda\in \R$, then $g$ lies on the plane $E$,}
\item \lang{de}{
      für genau ein $\lambda\in \R$ erfüllt ist, so durchstößt $g$ die Ebene $E$ in dem durch 
      genau dieses $\lambda$ festgelegten Punkt,
      }
      \lang{en}{
      holds for exaclty one $\lambda\in \R$, then $g$ intersects the plane $E$ precisely once, at the 
      point given by this $\lambda$,
      }
\item \lang{de}{für kein $\lambda\in \R$ erfüllt ist, so liegt $g$ parallel zur Ebene $E$.}
      \lang{en}{does not hold for any $\lambda\in \R$, then $g$ lies parallel to the plane $E$.}
\end{itemize}
\end{rule}

\lang{de}{
\floatright{\href{https://api.stream24.net/vod/getVideo.php?id=10962-2-10797&mode=iframe&speed=true}{\image[75]{00_video_button_schwarz-blau}}}\\
}
\lang{en}{}

\begin{example}
\begin{tabs*}[\initialtab{0}]
	\tab{\lang{de}{Gerade in Ebene}\lang{en}{Line on plane}}
    \lang{de}{Wir untersuchen die Lagebeziehung der Geraden $g$ gegeben durch}
    \lang{en}{Let $g$ be the line}
  	\[g: \vec{x} = \begin{pmatrix} 2\\-1\\1\end{pmatrix} + 
         \lambda\cdot \begin{pmatrix} -1\\-1\\-1\end{pmatrix},\quad \lambda\in\R,\]
    \lang{de}{und der Ebene $E$ gegeben durch}
    \lang{en}{and $E$ be the plane}
  	\[E: \vec{x} = \begin{pmatrix} -1\\-1\\-1\end{pmatrix} + 
         \mu\cdot \begin{pmatrix} -1\\2\\0\end{pmatrix} + 
         \sigma\cdot\begin{pmatrix} 0\\3\\1\end{pmatrix},\quad \mu,\sigma\in\R.\]
    \lang{de}{Der Ansatz}
    \lang{en}{The equation}
  	\[\begin{pmatrix} 2\\-1\\1\end{pmatrix} + \lambda\cdot \begin{pmatrix} -1\\-1\\-1\end{pmatrix} = 
      \begin{pmatrix} -1\\-1\\-1\end{pmatrix} + \mu\cdot \begin{pmatrix} -1\\2\\0\end{pmatrix} + 
      \sigma\cdot\begin{pmatrix} 0\\3\\1\end{pmatrix}\]
    \lang{de}{führt auf das Gleichungssystem}
    \lang{en}{leads to the system}
  	\begin{align*}
  		2-\lambda & = -1-\mu \\
  		-1-\lambda & = -1+2\mu+3\sigma \\
  		1-\lambda & = -1+\sigma,
  	\end{align*}
    \lang{de}{also}
    \lang{en}{which after simplification is}
  	\begin{align*}
	    \lambda  & = 3 +\mu\\
  		\mu +\sigma & = -1 \\
  		\mu +\sigma & = -1.
  	\end{align*}
    \lang{de}{
    Das Gleichungssystem besteht somit aus nur zwei linear unabhängigen Gleichungen und hat unendlich 
    viele Lösungen. Daraus folgt, dass die Gerade $g$ in der Ebene $E$ liegt.
    }
  	\lang{en}{
    This system of equations has infinitely many solutions, hence the line $g$ lies on the plane $E$.
    }
\begin{center}
\image{T110_LineInPlane}
\end{center}
	\tab{\lang{de}{Gerade durchstößt Ebene}\lang{en}{Line intersecting plane}}
    \lang{de}{In diesem zweiten Beispiel betrachten wir die Gerade $g$ gegeben durch}
    \lang{en}{In this second example we consider instead the line $g$ given by}
  	\[g: \vec{x} = \begin{pmatrix} 0\\0\\-1\end{pmatrix} + 
         \lambda\cdot \begin{pmatrix} -1\\-1\\-2\end{pmatrix},\quad \lambda\in\R,\]
    \lang{de}{und weiterhin die Ebene $E$ gegeben durch}
    \lang{en}{and continue to consider the plane $E$ given by}
  	\[E: \vec{x} = \begin{pmatrix} -1\\-1\\-1\end{pmatrix} + 
         \mu\cdot \begin{pmatrix} -1\\2\\0\end{pmatrix} + 
         \sigma\cdot\begin{pmatrix} 0\\3\\1\end{pmatrix},\quad \mu,\sigma\in\R.\]
  	\lang{de}{
    Statt der Parameterform können wir auch die Normalenform der Ebene benutzen. Ein Normalenvektor 
    zu der Ebene $E$ ist
    }
    \lang{en}{
    Instead of the parametrised form of the plane, we can use the normal form. A normal vector to 
    the plane $E$ is
    }
    \[\vec{n}=\begin{pmatrix}  -1\\2\\0\end{pmatrix} \times \begin{pmatrix}  0\\3\\1\end{pmatrix} = 
      \begin{pmatrix} 2\\ 1\\ -3\end{pmatrix}. \]
    \lang{de}{Weiter ist}\lang{en}{Furthermore,}
    \[ \vec{n}\bullet \begin{pmatrix} -1\\-1\\-1\end{pmatrix} = 
    \begin{pmatrix} 2\\ 1\\ -3\end{pmatrix}\bullet \begin{pmatrix} -1\\-1\\-1\end{pmatrix}=0.\]
    \lang{de}{Eine Normalenform für $E$ ist daher}
    \lang{en}{$E$ can hence be expressed in normal form as}
    \[E=\left\{\begin{pmatrix}x_1 \\ x_2\\ x_3 \end{pmatrix}\in \R^3\, \mid\, 
      \begin{pmatrix} 2\\ 1\\ -3\end{pmatrix}\bullet \vec{x}=0 \right\}. \]
    \lang{de}{
    Nun kann getestet werden, wie die Gerade $g$ zu der Ebene $E$ liegt, indem man die Parameterform 
    von $g$ in die Gleichung von $E$ einsetzt:
    }
    \lang{en}{
    Now we can test where the line $g$ lies relative to the plane $E$ by substituting the 
    parameterised equation of $g$ into the equation of $E$:
    }
    \begin{align*}
    	& & \hspace{1.5cm}\begin{pmatrix} 2\\ 1\\ -3\end{pmatrix}\bullet 
      	\Big( \begin{pmatrix} 0\\0\\-1\end{pmatrix} + 
          \lambda\cdot \begin{pmatrix} -1\\-1\\-2\end{pmatrix} \Big) &= 0 \\
    	& \Leftrightarrow & \begin{pmatrix} 2\\ 1\\ -3\end{pmatrix}\bullet 
        \begin{pmatrix} 0\\0\\-1\end{pmatrix} + 
          \lambda \cdot \begin{pmatrix} 2\\ 1\\ -3\end{pmatrix}\bullet 
            \begin{pmatrix} -1\\-1\\-2\end{pmatrix} &=0 \\
    	& \Leftrightarrow & \hspace{6.25cm} 3 +3\lambda &=0 \\
    	& \Leftrightarrow & \hspace{7.7cm} \lambda &=-1.
    \end{align*}
    \lang{de}{
    Die Gleichung hat also genau eine Lösung $\lambda=-1$, weshalb die Gerade die Ebene schneidet. 
    Der Schnittpunkt hat den Ortsvektor
    }
    \lang{en}{
    The equation therefore has exactly one solution $\lambda=-1$, so the line intersects the plane. 
    The point of intersection has position vector
    }
    \[\begin{pmatrix} 0\\0\\-1\end{pmatrix} + (-1)\cdot \begin{pmatrix} -1\\-1\\-2\end{pmatrix} = 
      \begin{pmatrix} 1\\1\\1\end{pmatrix}, \]
    \lang{de}{d.h. der Schnittpunkt ist $S = (1;1;1)$.}
    \lang{en}{that is, the line intercepts the plane at $S = (1;1;1)$.}
    \begin{center}
    \image{T110_LineIntersectingPlane}
    \end{center}    
  \tab{\lang{de}{Gerade parallel zu Ebene}\lang{en}{Line parallel to plane}}
    \lang{de}{
    In diesem dritten Beispiel untersuchen wir die Lagebezieung der Gerade $g$ gegeben durch
    }
    \lang{en}{
    In this thrid example we study the line $g$ given by
    }
  	\[g: \vec{x} = \begin{pmatrix} -1\\2\\2\end{pmatrix} + 
      \lambda\cdot \begin{pmatrix} -1\\-1\\-1\end{pmatrix},\quad \lambda\in\R,\]
    \lang{de}{und der bereits in den vorigen Beispielen verwendeten Ebene $E$ gegeben durch}
    \lang{en}{and the plane $E$ already used in the previous examples}
  	\[E: \vec{x} = \begin{pmatrix} -1\\-1\\-1\end{pmatrix} + 
      \mu\cdot \begin{pmatrix} -1\\2\\0\end{pmatrix} + 
      \sigma\cdot\begin{pmatrix} 0\\3\\1\end{pmatrix},\quad \mu,\sigma\in\R.\]
    \lang{de}{
    Wir testen, wie die Gerade $g$ zur Ebene $E$ liegt, indem wir die Parameterform von $g$ in 
    Normalenform der Ebene einsetzen, die wir im letzten Beispiel bestimmt haben:
    }
    \lang{en}{
    We test how the line $g$ lies in relation to the plane $E$ by substituting the parametrised 
    equation of $g$ into the normal form of the plane, which we determined in the previous example:
    }
    \begin{align*}
      & \hspace{1.5cm}\begin{pmatrix} 2\\ 1\\ -3\end{pmatrix}\bullet 
        \Big( \begin{pmatrix} -1\\2\\2\end{pmatrix} + 
          \lambda\cdot \begin{pmatrix} -1\\-1\\-1\end{pmatrix} \Big) &= 0 \\
      \Leftrightarrow & \begin{pmatrix} 2\\ 1\\ -3\end{pmatrix}\bullet 
        \begin{pmatrix} -1\\2\\2\end{pmatrix} + 
          \lambda \cdot \begin{pmatrix} 2\\ 1\\ -3\end{pmatrix}\bullet 
            \begin{pmatrix} -1\\-1\\-1\end{pmatrix} &=0 \\
      \Leftrightarrow & \hspace{7.3cm}-6  &=0.
    \end{align*}
    \lang{de}{
    Die Gleichung besitzt somit keine Lösung, woraus folgt, dass die Gerade parallel zur Ebene liegt.
    }
    \lang{en}{
    This is a contradiction, so the equation has no solutions, and the line is parallel to the plane.
    }
\begin{center}
\image{T110_LineParallelToPlane}
\end{center}
\end{tabs*}
\end{example}

\section{\lang{de}{Ebene - Ebene}\lang{en}{Plane - Plane}}
\label{sec:6_7_Ebene-Ebene}

\lang{de}{
Zwei Ebenen können entweder identisch sein, oder parallel (aber nicht identisch sein), oder sich in 
einer Geraden schneiden.
}
\lang{en}{
Two planes can either be identical, be parallel (and not identical), or intersect each other at 
a line.
}

\begin{rule}
  \lang{de}{Gegeben seien die zwei Ebenen}
  \lang{en}{Let $E$ and $F$ be the planes}
	\begin{align*}
		E: \vec{x} & = \vec{OP} + \lambda\cdot \vec{u} + \mu\cdot \vec{v},\; \lambda,\mu\in\R,\\
		F: \vec{x} & = \vec{OQ} + \rho\cdot \vec{w} + \sigma \cdot\vec{y},\; \rho,\sigma\in\R\lang{en}{.}
	\end{align*}
  \lang{de}{
  mit den Stützpunkten $P$ bzw. $Q$ und den Richtungsvektoren $\vec{u}$, $\vec{v}$ bzw. $\vec{w}$, 
  $\vec{y}$. Besitzt das Gleichungssystem gegeben durch
  }
	\lang{en}{
  It is easy to see that $E$ goes through the point $P$ and has direction vectors $\vec{u}$ and 
  $\vec{v}$, whereas $F$ goes through the point $Q$ and has direction vectors $\vec{w}$ and 
  $\vec{y}$. The linear system
  }
	\[\vec{OP} + \lambda\cdot \vec{u} + \mu\cdot \vec{v} = \vec{OQ} + \rho\cdot \vec{w} + 
    \sigma \cdot\vec{y}\]
  \lang{de}{f\"ur die vier reellen Zahlen $\lambda$, $\mu$, $\rho$ und $\sigma$}
  \lang{en}{in $\lambda$, $\mu$, $\rho$, and $\sigma$ can have:}
	\begin{itemize}
		\item \lang{de}{
          unendlich viele Lösungen $(\lambda;\mu;\rho;\sigma)$, wobei die Lösungsmenge durch 
      		zwei reelle Parameter beschrieben wird, so sind die Ebenen $E$ und $F$ identisch,
          }
      		\lang{en}{
          infinitely many solutions $(\lambda,\mu,\rho,\sigma)$, where the solution set is 
      		described by varying two real parameters. In this case, the planes are identical.
          } 
		\item \lang{de}{
          unendlich viele Lösungen $(\lambda;\mu;\rho;\sigma)$, wobei die Lösungsmenge durch einen 
          reellen Parameter beschrieben wird, so besitzen die Ebenen $E$ und $F$ eine Schnittgerade, 
          die durch Einsetzen ermittelt werden kann,
          }
      		\lang{en}{
          infinitely many solutions $(\lambda,\mu,\rho,\sigma)$, where the solution set is described 
      		by varying one real parameter. In this case, the planes intersect each other at a line.
          }
		\item \lang{de}{keine Lösung, so liegen $E$ und $F$ parallel zueinander.}
      		\lang{en}{no solutions, in which case the planes are parallel to each other.}
	\end{itemize}
\end{rule}

\lang{de}{
Ist eine der beiden Ebenen in Normalenform gegeben, kann man die gegenseitige Lage einfacher folgendermaßen bestimmen:
}
\lang{en}{
If one of the two planes is given in normal form, the relationship between them is more easily found, 
by the following method:
}

\begin{rule}
  \lang{de}{Gegeben seien die zwei Ebenen}
  \lang{en}{Consider the two planes}
  \[ E: \vec{x}  = \vec{OP} + \lambda\cdot \vec{u} + \mu\cdot \vec{v},\; \lambda,\mu\in\R, \]
  \lang{de}{und}\lang{en}{and}
  \[ F= \{ \vec{x}\in \R^3 \mid \vec{n} \bullet \vec{x}=d \}. \]
  \lang{de}{Wenn die Gleichung}
  \lang{en}{If the equation}
  \[ \vec{n}\bullet \left(\vec{OP} + \lambda\cdot \vec{u} + \mu\cdot \vec{v} \right)=d \]
  \lang{de}{mit $\lambda,\mu \in \R$}
  \lang{en}{with $\lambda,\mu \in \R$}
  \begin{itemize}
    \item \lang{de}{
          für alle $\lambda,\mu \in \R$ erfüllt ist, so sind die Ebenen $E$ und $F$ identisch,
          }
          \lang{en}{
          holds for all $\lambda,\mu \in \R$, then the planes $E$ and $F$ are identical,
          }
    \item \lang{de}{
          lösbar ist, aber nicht für alle Paare $(\lambda;\mu)$, so besitzen die Ebenen $E$ und $F$ 
          eine Schnittgerade, die durch Einsetzen ermittelt werden kann,
          }
          \lang{en}{
          holds only for some pairs $(\lambda;\mu)$, then the planes $E$ and $F$ intersect at a line, 
          whose equation can be determined by substitution,
          }
    \item \lang{de}{
          für keine Wahl von $\lambda,\mu\in \R$ erfüllt ist, so liegen die Ebenen $E$ und $F$ 
          parallel zueinander.
          }
          \lang{en}{
          does not hold for any $\lambda,\mu\in \R$, then the planes $E$ and $F$ lie parallel to each 
          other.
          }
\end{itemize}
\end{rule}

\lang{de}{
\floatright{\href{https://api.stream24.net/vod/getVideo.php?id=10962-2-10797&mode=iframe&speed=true}{\image[75]{00_video_button_schwarz-blau}}}\\
}
\lang{en}{}

\begin{example}
\begin{tabs*}[\initialtab{0}]
	\tab{\lang{de}{Identische Ebenen}\lang{en}{Identical planes}}
    \lang{de}{In diesem Beispiel untersuchen wir die Lagebeziehung der beiden Ebenen}
    \lang{en}{What type of intersection do the following two planes have with each other?}
  	\begin{align*}
  		E: \vec{x} & = \begin{pmatrix} -2\\-2\\-3\end{pmatrix} + 
         \lambda\cdot \begin{pmatrix} 1\\1\\2\end{pmatrix} + 
         \mu\cdot\begin{pmatrix} 1\\0\\1\end{pmatrix},\quad \lambda,\mu\in\R,\\
  		F: \vec{x} & = \begin{pmatrix} 2\\1\\4\end{pmatrix} + 
         \rho\cdot \begin{pmatrix} -1\\-2\\-3\end{pmatrix} + 
          \sigma\cdot\begin{pmatrix} -2\\1\\-1\end{pmatrix},\quad \rho,\sigma\in\R.
  	\end{align*}
    \lang{de}{Ein Normalenvektor zu der Ebene $F$ ist}
    \lang{en}{A normal vector of the plane $F$ is}
    \[\vec{n}=\begin{pmatrix} -1\\-2\\-3\end{pmatrix} \times \begin{pmatrix} -2\\1\\-1\end{pmatrix} = 
    \begin{pmatrix} 5\\ 5\\ -5\end{pmatrix}. \]
    \lang{de}{Weiter ist}
    \lang{en}{Furthermore,}
    \[\vec{n}\bullet \begin{pmatrix} 2\\1\\4\end{pmatrix} = 
      \begin{pmatrix} 5\\ 5\\ -5\end{pmatrix}\bullet \begin{pmatrix} 2\\1\\4\end{pmatrix}=-5.\]
    \lang{de}{Eine Normalenform für $F$ ist daher}
    \lang{en}{Hence $F$ can be expressed in normal form as}
    \[F=\left\{\begin{pmatrix}x_1 \\ x_2\\ x_3 \end{pmatrix}\in \R^3\, \mid\, 
    \begin{pmatrix} 5\\ 5\\ -5\end{pmatrix}\bullet \vec{x}=-5 \right\}. \]
    \lang{de}{
    Nun kann getestet werden, wie die Ebene $E$ zu der Ebene $F$ liegt, indem man die Parameterform 
    von $E$ in die Gleichung von $F$ einsetzt: 
    }
    \lang{en}{
    Now we can test where the plane $E$ lies relative to the plane $F$ by substituting the 
    parametrised equation of $E$ into the equation of $F$:
    }
    \begin{align*}
      & \hspace{2.7cm}\begin{pmatrix} 5\\ 5\\ -5\end{pmatrix}\bullet 
        \Big( \begin{pmatrix} -2\\-2\\-3\end{pmatrix} + 
          \lambda\cdot \begin{pmatrix} 1\\1\\2\end{pmatrix} + 
            \mu\cdot\begin{pmatrix} 1\\0\\1\end{pmatrix} \Big) &= -5 \\
      \Leftrightarrow & \begin{pmatrix} 5\\ 5\\ -5\end{pmatrix}\bullet 
        \begin{pmatrix} -2\\-2\\-3\end{pmatrix} + 
          \lambda \cdot \begin{pmatrix} 5\\ 5\\ -5\end{pmatrix}\bullet 
            \begin{pmatrix} 1\\1\\2\end{pmatrix} + 
              \mu\cdot \begin{pmatrix} 5\\ 5\\ -5\end{pmatrix}\bullet 
                \begin{pmatrix} 1\\0\\1\end{pmatrix} &=5 \\
      \Leftrightarrow & \hspace{10.0cm}-5  &=-5 
    \end{align*}
    \lang{de}{
    Die Gleichung ist somit für alle $\lambda,\mu\in\mathbb{R}$ erfüllt, woraus folgt, dass die 
    Ebenen $E$ und $F$ identisch sind.
    }
    \lang{en}{
    The equation is hence satisfied by all $\lambda,\mu\in\mathbb{R}$, so the planes $E$ and $F$ are 
    identical.
    }
  	\begin{center}
    \image{T110_IdenticalPlanes}
    \end{center}
  \tab{\lang{de}{Sich schneidende Ebenen}\lang{en}{Intersecting planes}}
    \lang{de}{In diesem zweiten Beispiel betrachten wir die beiden Ebenen}
    \lang{en}{What type of intersection do the following two planes have with each other?}
  	\begin{align*}
  		E: \vec{x} & = \begin{pmatrix} -2\\-2\\-3\end{pmatrix} + 
         \lambda\cdot \begin{pmatrix} 1\\1\\2\end{pmatrix} + 
         \mu\cdot\begin{pmatrix} 1\\0\\1\end{pmatrix},\quad \lambda,\mu\in\R,\\
  		F: \vec{x} & = \begin{pmatrix} -1\\0\\3\end{pmatrix} + 
         \rho\cdot \begin{pmatrix} 2\\-1\\0\end{pmatrix} + 
         \sigma\cdot\begin{pmatrix} 0\\1\\-1\end{pmatrix},\quad \rho,\sigma\in\R.
  	\end{align*}
    \lang{de}{Der Ansatz}
    \lang{en}{The equation}
  	\[\begin{pmatrix} -2\\-2\\-3\end{pmatrix} + \lambda\cdot \begin{pmatrix} 1\\1\\2\end{pmatrix} + 
        \mu\cdot\begin{pmatrix} 1\\0\\1\end{pmatrix} = 
      \begin{pmatrix} -1\\0\\3\end{pmatrix} + \rho\cdot \begin{pmatrix} 2\\-1\\0\end{pmatrix} + 
        \sigma\cdot\begin{pmatrix} 0\\1\\-1\end{pmatrix}\]
    \lang{de}{führt auf das Gleichungssystem}
    \lang{en}{leads to the system}
  	\begin{align*}
  		-2+\lambda + \mu & = -1+2\rho \\
  		-2+\lambda & = -\rho + \sigma \\
  		-3+2\lambda+\mu & = 3-\sigma,
  	\end{align*}
    \lang{de}{also}
    \lang{en}{which after some simplification is}
  	\begin{align*}
  		\lambda + \mu  - 2\rho & = 1 \\
  		\lambda +  \rho - \sigma & = 2 \\
  		2\lambda + \mu +\sigma & = 6.
  	\end{align*}
    \lang{de}{
    Dieses Gleichungssystem mit drei Gleichungen für vier Unbekannte besitzt die parameterabhängige 
    Lösung
    }
  	\lang{en}{
    This system, with three equations in four unknowns, has the parameter-dependent solution}
  	\begin{align*}
  		\lambda & = -1+3t\\
  		\mu & = 8-7t\\
  		\rho & = 3-2t\\
  		\sigma & = t
  	\end{align*}
    \lang{de}{
    mit $t\in\R$. Hieraus folgt, dass sich die beiden Ebenen in einer Geraden schneiden. Einsetzen 
    von $\lambda$ und $\mu$ in die Ebenengleichung für $E$ bzw. von $\rho$ und $\sigma$ in die 
    Ebenengleichung für $F$ ergibt in beiden Fällen die Schnittgerade
    }
    \lang{en}{
    where $t\in\R$. Substituting $\lambda$ and $\mu$ into the equation for $E$, (or alternatively, 
    substituting $\rho$ and $\sigma$ into $F$) yields the line
    }
  	\[g: \vec{x} = \begin{pmatrix} 5\\-3\\3\end{pmatrix} + 
         t\cdot \begin{pmatrix} -4\\3\\-1\end{pmatrix},\quad t\in\R.\]
    \begin{center}
    \image{T110_IntersectingPlanes}
    \end{center}
	\tab{\lang{de}{Parallele Ebenen}\lang{en}{Parallel planes}}
    \lang{de}{In diesem letzten Beispiel untersuchen wir die Lagebeziehung der beiden Ebenen}
    \lang{en}{What type of intersection do the following two planes have with each other?}
  	\begin{align*}
  		E: \vec{x} & = \begin{pmatrix} -1\\-2\\0\end{pmatrix} + 
         \lambda\cdot \begin{pmatrix} 2\\0\\-1\end{pmatrix} + 
         \mu\cdot\begin{pmatrix} 2\\1\\-2\end{pmatrix},\quad \lambda,\mu\in\R,\\
  		F: \vec{x} & = \begin{pmatrix} -1\\0\\3\end{pmatrix} + 
         \rho\cdot \begin{pmatrix} 2\\-1\\0\end{pmatrix} + 
         \sigma\cdot\begin{pmatrix} 0\\1\\-1\end{pmatrix},\quad \rho,\sigma\in\R.
  	\end{align*}
    \lang{de}{Ein Normalenvektor zu der Ebene $F$ ist}
    \lang{en}{A normal vector to the plane $F$ is}
    \[\vec{n}=\begin{pmatrix}  2\\-1\\0\end{pmatrix} \times \begin{pmatrix} 0\\1\\-1\end{pmatrix} = 
      \begin{pmatrix} 1\\ 2\\ 2\end{pmatrix}. \]
    \lang{de}{Weiter ist}
    \lang{en}{Furthermore,}
    \[\vec{n}\bullet \begin{pmatrix} -1\\0\\3\end{pmatrix} = 
      \begin{pmatrix} 1\\ 2\\ 2\end{pmatrix}\bullet \begin{pmatrix} -1\\0\\3\end{pmatrix}=5.\]
    \lang{de}{Eine Normalenform für $F$ ist daher}
    \lang{en}{Hence the plane $F$ can be expressed in normal form as}
    \[F=\left\{\begin{pmatrix}x_1 \\ x_2\\ x_3 \end{pmatrix}\in \R^3\, \mid\, 
        \begin{pmatrix} 1\\ 2\\ 2\end{pmatrix}\bullet \vec{x}=5 \right\}. \]
    \lang{de}{
    Nun kann getestet werden, wie die Ebene $E$ zu der Ebene $F$ liegt, indem man die 
    Parameterform von $E$ in die Gleichung von $F$ einsetzt:
    }
    \lang{en}{
    Now we can test where the plane $E$ lies relative to the plane $F$ by substituting the 
    parametrised equation of $E$ into the equation of $F$:
    }
    \begin{align*}
      & \hspace{2.7cm}\begin{pmatrix} 1\\ 2\\ 2\end{pmatrix}\bullet 
        \Big( \begin{pmatrix} -1\\-2\\0\end{pmatrix} + 
          \lambda\cdot \begin{pmatrix} 2\\0\\-1\end{pmatrix} + 
            \mu\cdot\begin{pmatrix} 2\\1\\-2\end{pmatrix} \Big) &= 5 \\
      \Leftrightarrow & \begin{pmatrix} 1\\ 2\\ 2\end{pmatrix}\bullet 
        \begin{pmatrix} -1\\-2\\0\end{pmatrix} + 
          \lambda \cdot \begin{pmatrix} 1\\ 2\\ 2\end{pmatrix}\bullet 
            \begin{pmatrix} 2\\0\\-1\end{pmatrix} + 
              \mu\cdot \begin{pmatrix} 1\\ 2\\ 2\end{pmatrix}\bullet 
                \begin{pmatrix} 2\\1\\-2\end{pmatrix} &=5 \\
      \Leftrightarrow & \hspace{10.0cm}-5  &=5 
    \end{align*}
    \lang{de}{Da die Gleichung keine Lösung besitzt, liegen $E$ und $F$ parallel zueinander.}
    \lang{en}{As the equation has no solutions, $E$ and $F$ lie parallel to each other.}
\begin{center}
\image{T110_ParallelPlanes}
\end{center}
\end{tabs*}
\end{example}



\end{visualizationwrapper}


\end{content}