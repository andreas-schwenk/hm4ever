%$Id:  $
\documentclass{mumie.article}
%$Id$
\begin{metainfo}
  \name{
    \lang{de}{Parameterformen von Geraden und Ebenen}
    \lang{en}{Parametrised lines and planes}
  }
  \begin{description} 
 This work is licensed under the Creative Commons License Attribution 4.0 International (CC-BY 4.0)   
 https://creativecommons.org/licenses/by/4.0/legalcode 

    \lang{de}{Beschreibung}
    \lang{en}{Description}
  \end{description}
  \begin{components}
    \component{generic_image}{content/rwth/HM1/images/g_img_00_Videobutton_schwarz.meta.xml}{00_Videobutton_schwarz}
   \component{generic_image}{content/rwth/HM1/images/g_img_00_video_button_schwarz-blau.meta.xml}{00_video_button_schwarz-blau}
   \component{generic_image}{content/rwth/HM1/images/g_tkz_T110_PointTwoDirectionsPlane.meta.xml}{T110_PointTwoDirectionsPlane}
    \component{generic_image}{content/rwth/HM1/images/g_tkz_T110_ThreePointPlane.meta.xml}{T110_ThreePointPlane}
    \component{generic_image}{content/rwth/HM1/images/g_tkz_T110_PointDirectionLine.meta.xml}{T110_PointDirectionLine}
    \component{generic_image}{content/rwth/HM1/images/g_tkz_T110_TwoPointLine.meta.xml}{T110_TwoPointLine}
  \end{components}
  \begin{links}
    \link{generic_article}{content/rwth/HM1/T104_weitere_elementare_Funktionen/g_art_content_14_potenzregeln.meta.xml}{power-rules}
    \link{generic_article}{content/rwth/HM1/T102neu_Einfache_Reelle_Funktionen/g_art_content_06_funktionsbegriff_und_lineare_funktionen.meta.xml}{04-lineare-funktionen}
  \end{links}
  \creategeneric
\end{metainfo}
\begin{content}
\usepackage{mumie.ombplus}
\ombchapter{10}
\ombarticle{1}
\usepackage{mumie.genericvisualization}

\begin{visualizationwrapper}

\title{\lang{de}{Parameterformen von Geraden und Ebenen}\lang{en}{Parametrised lines and planes}}
 
\begin{block}[annotation]
  übungsinhalt
  
\end{block}
\begin{block}[annotation]
  Im Ticket-System: \href{http://team.mumie.net/issues/9052}{Ticket 9052}\\
\end{block}

\begin{block}[info-box]
\tableofcontents
\end{block}

\lang{de}{
Zur Motivation können Sie das folgende Beispiel anschauen, dessen
Sachverhalt Sie nach diesem Kapitel lösen werden können:
}
\lang{en}{
The following example serves as motivation for this chapter:
}
\begin{example}
  \begin{tabs*}[\initialtab{0}]
    \tab{\lang{de}{Roboter mit Teleskopgreifarm}\lang{en}{Robot with telescopic arm}}
%       \begin{incremental}[\initialsteps{1}]
%   \step
    \lang{de}{Ein Roboter kann seinen Teleskoparm auf der Ebene mit der Gleichung}
    \lang{en}{A robot can freely move its entire telescopic arm on the plane with equation}
    \\\\
    $E= \left\{ \begin{pmatrix} 1\\-2\\3\end{pmatrix} + \alpha \begin{pmatrix}1\\0\\2\end{pmatrix} + 
    \beta \begin{pmatrix}0\\1\\2\end{pmatrix} \vert \alpha,\beta\in\R\right\}$\\\\ 
    \lang{de}{
    frei bewegen. Das Koordinatensystem ist in Dezimetern ausgelegt. Allerdings kann er seinen 
    Teleskoparm nur senkrecht zur Ebene aus- und einfahren.
    \\\\
    Nach Beschäftigung mit dem folgenden Kapitel können Sie ermitteln, an welcher Position 
    $R=(r_1;r_2;r_3)$ der Teleskoparm zum Ausfahren auf der Ebene stehen muss, um einen Gegenstand an 
    der Position $P=(4;3;10)$ greifen zu können. Außerdem können Sie bestimmen, wie weit der Arm 
    ausfahren muss, um diesen Gegenstand greifen zu können.
    }
    \lang{en}{
    where the unit of the coordinate system is a decimetre (a tenth of a metre). The arm can then be 
    extended orthogonally to the plane.
    \\\\
    Upon completion of this following chapter, we will have the tools to determine which position 
    $R=(r_1;r_2;r_3)$ on the plane the telescopic arm would have to extend from in order to reach the 
    point $P=(4;3;10)$. Furthermore, we will be able to tell how far the arm must extend to reach it.
    }
    \\
%   \step
\begin{showhide}
    \lang{de}{
    Man kann die Ebene $E$ der Einfachheit wegen von der gegebenen Parameterform in die 
    Koordinatenschreibweise umformen. Dazu bestimmt man zunächst einen Normalenvektor. Dies können 
    wir mit Hilfe des Kreuzproduktes aus den Spannvektoren der Ebene $E$ berechnen:
    }
    \lang{en}{
    We can easily represent the plane, which is currently in a parametrised form, as an equation in 
    the coordinates. Firstly we determine a normal vector. This can be found using the cross product 
    of the vectors spanning the plane:
    }
    %einen Normalenvektor $\vec{n}$ der Ebene bestimmt und einen Punkt der Ebene $E$,
    %z. B. den Stützvektor, in die Koordinatenform einsetzt und nach dem Parameter $f$ umformt:
    \[\vec{n}=\begin{pmatrix} 1\\0\\2\end{pmatrix}\times\begin{pmatrix}0\\1\\2\end{pmatrix}=\begin{pmatrix}-2\\-2\\1\end{pmatrix}\]
    \lang{de}{
    Damit ergibt sich als Ansatz für die Gleichung der Ebene $E$ in Koordinatenform: 
    $-2x_1-2x_2+x_3=f$.\\
    Um den Parameter $f$ zu ermitteln, wird ein Punkt der Ebene, hier der Stützvektor, in die 
    Koordinatenform eingesetzt und nach $f$ umgeformt:
    }
    \lang{en}{
    From this we can tell that the equation of the plane $E$ in coordinate form is 
    $-2x_1-2x_2+x_3=f$.\\
    The parameter $f$ can be found by substituting a point on the plane into the equation, for 
    example the position vector $\begin{pmatrix} 1\\0\\2\end{pmatrix}$ from the parametrisation:
    }
    \[-2\cdot 1-2\cdot (-2)+3=f\Leftrightarrow f=5\]
    \lang{de}{Die Koordinatenform lautet also:}
    \lang{en}{The coordinate form is therefore:}
    \[E=\{x\in\mathbb{R}^3\vert -2x_1-2x_2+x_3=5\}\]
    \lang{de}{
    Wir können nun schon die Geradengleichung $g$ aufstellen, auf der der Teleskoparm aus- und 
    einfährt, um den Punkt $P$ greifen zu können:
    }
    \lang{en}{
    We can now already determine the equation of the line from which the telescopic arm must extend 
    to reach the point $P=(4;3;10)$:
    }
    \[g=\{\begin{pmatrix} 4\\3\\10\end{pmatrix}+\gamma\cdot\begin{pmatrix} -2\\-2\\1\end{pmatrix}\vert
    \gamma\in\mathbb{R}\}\]
    \lang{de}{
    Einsetzen der Geraden $g$ in die Koordinatenform der Ebene $E$ liefert die Position $R$, an der 
    der Roboter stehen muss, um den Gegenstand durch Ausfahren des Teleskoparms greifen zu können:
    }
    \lang{en}{
    Substituting the equation of the line $g$ into the coordinate form of the plane $E$ gives us 
    their intersection $R$, which is the point on the plane where the robot arm must be to reach $P$:
    }
    \[-2\cdot(4-2\gamma)-2\cdot(3-2\gamma)+(10+\gamma)=5\Leftrightarrow\gamma=1\]
    \lang{de}{
    Man setzt $\gamma=1$ in die Gerade $g$ ein und erhält $R=(2;1;11)$.
    \\\\
    Der Abstand des Punktes $P$ zum Punkt $R$ auf der Ebene $E$ entspricht der Länge, die der 
    Teleskoparm des Roboters ausfahren muss:
    }
    \lang{en}{
    We substitute $\gamma=1$ into the line $g$ and obtain $R=(2;1;11)$.
    }
    \[d(P,R)=\abs{\vec{PR}}=\abs{\begin{pmatrix} 2\\1\\11\end{pmatrix} -
      \begin{pmatrix}4\\3\\10\end{pmatrix}} = 
      \abs{\begin{pmatrix} -2\\-2\\1 \end{pmatrix}} = 
      \abs{\vec{n}}\]
    \lang{de}{
    Die Länge entspricht also dem Betrag des gewählten Normalenvektors: 
    $d=\sqrt{(-2)^2+(-2)^2+1^2}=3$.
    \\\\
    Der Teleskoparm des Roboters muss also am Punkt $R=(2;1;11)$ stehen. Außerdem muss der 
    Teleskoparm $3$ Dezimeter weit ausfahren.
    }
    \lang{en}{
    The length of the arm corresponds to the magnitude of the normal vector of the line, as this is 
    the vector from $P$ to $R$: $d=\sqrt{(-2)^2+(-2)^2+1^2}=3$.
    \\\\
    To summarise, the telescopic arm of the robot must extend $3$ decimetres from the point 
    $R=(2;1;11)$ on the plane in order to reach the point $P$.
    }
% \end{incremental}
\end{showhide}
    \end{tabs*}
\end{example}


\section{\lang{de}{Geraden}\lang{en}{Lines}}
\label{sec:geraden}

\lang{de}{
Zwei verschiedene Punkte $P$ und $Q$ im $n$-dimensionalen Raum $\R^n$ definieren eindeutig eine 
Gerade $g$, die durch diese beiden Punkte verläuft.
}
\lang{en}{
Two distinct points $P$ and $Q$ in $n$-dimensional space $\R^n$ uniquely determine the line $g$ 
containing both points.
}

\begin{center}
	\image{T110_TwoPointLine}
\end{center}


\begin{rule}[\lang{de}{Zwei-Punkte-Darstellung einer Geraden}
             \lang{en}{Two-point form of a line}]\label{rule:zwei_pkt_geraden}
  \lang{de}{
  Die Gerade $g$ durch zwei verschiedene Punkte $P$ und $Q$ im $\R^n$ ist gegeben als die Menge aller 
  Punkte, deren Ortsvektoren von der Form
  }
  \lang{en}{
  The line $g$ through two distinct points $P$ and $Q$ in $\R^n$ is given by the set of all points 
  whose position vectors have the form
  }
	\[\vec{x} = \vec{OP} + \lambda\cdot \bigg(\vec{OQ} - \vec{OP}\bigg)\]
  \lang{de}{
	sind, wobei $\lambda\in\R$ die reellen Zahlen durchläuft.
	Wegen $\vec{OQ} - \vec{OP} = \vec{PQ}$ lässt sich die Gerade auch schreiben als
  }
  \lang{en}{
  where $\lambda\in\R$ runs through all real numbers.
	Because $\vec{OQ} - \vec{OP} = \vec{PQ}$, the line can also be written as
  }
	\[g: \vec{x} = \vec{OP} + \lambda\cdot \vec{PQ}, \quad\lambda\in\R.\]
	\lang{de}{Oft schreibt man kürzer auch}
  \lang{en}{We often write the set of points with these position vectors simply as a set of vectors}
	\[g = \{\vec{OP} + \lambda\cdot \vec{PQ} \mid \lambda\in\R\},\]
  \lang{de}{
	wobei hier die Punkte mit den zugehörigen Ortsvektoren identifiziert werden
	(vgl. auch den Abschnitt über \link{04-lineare-funktionen}{Lineare Funktionen}). Diese Form 
	der Darstellung einer Geraden nennt man \emph{Parameterform}.
  }
  \lang{en}{
  and identify the vectors with coordinates (see \link{04-lineare-funktionen}{linear functions}). 
  This is a \emph{parametrised} representation, as $\lambda$ is varied to obtain the points on the 
  line.
  }    
   \end{rule}

\begin{quickcheck}
		\field{rational}
		\type{input.number}
		\begin{variables}
			\randint{v1}{-5}{5}
			\randint{v2}{-5}{5}
			\randint{v3}{-5}{0}
			\randint{w1}{-5}{5}
			\randint{w2}{-5}{5}
			\randint{w3}{1}{5}
			\function[calculate]{t1}{w1-v1}
			\function[calculate]{t2}{w2-v2}
			\function[calculate]{t3}{w3-v3}
		\end{variables}
		
			\text{\lang{de}{
      Bestimmen Sie mit obiger Methode eine Parameterform der Geraden $g$ 
			durch die Punkte $P=(\var{v1}; \var{v2}; \var{v3})$ und $Q=(\var{w1}; \var{w2}; \var{w3})$.
			\\\\
			Man erhält die Parameterform
      }
      \lang{en}{
      Determine with the above method the parametrised form of the line $g$ going through the points 
      $P=(\var{v1}; \var{v2}; \var{v3})$ and $Q=(\var{w1}; \var{w2}; \var{w3})$.
      \\\\
      The parametrised form is
      }
      \begin{table}[\class{no-padding}]
			\rowspan[l][m]{3} $g:\vec{x}=
			\left(\begin{matrix} \\ \\ \\ \\ \end{matrix}\right.$ &  
			\ansref & \rowspan[l][m]{3} $\left.\begin{matrix} \\ \\ \\ \\  \end{matrix}\right)
			+\lambda \cdot \left(\begin{matrix} \\ \\ \\ \\ \end{matrix}\right.$& 
			\ansref & \rowspan[l][m]{3} $\left.\begin{matrix} \\ \\ \\ \\  \end{matrix}\right), \quad \lambda\in \R  $. & \\ 
			\ansref & \ansref & \\ 
			\ansref & \ansref & 
			\end{table}			
			}
		
		\begin{answer}
			\solution{v1}
		\end{answer}
		\begin{answer}
			\solution{t1}
		\end{answer}
		\begin{answer}
			\solution{v2}
		\end{answer}
		\begin{answer}
			\solution{t2}
		\end{answer}
		\begin{answer}
			\solution{v3}
		\end{answer}
		\begin{answer}
			\solution{t3}
		\end{answer}
		\explanation{\lang{de}{
    Als Stützvektor sollte man den Ortsvektor von $P$ nehmen, also 
		$\begin{pmatrix} \var{v1}\\ \var{v2} \\ \var{v3} \end{pmatrix}$ und 
		als Richtungsvektor den Verbindungsvektor $\overrightarrow{PQ}$, also
    }
    \lang{en}{
    The position vector of $P$, $\begin{pmatrix} \var{v1}\\ \var{v2} \\ \var{v3} \end{pmatrix}$, is 
    the choice used in the above method for the constant vector, and the vector $\overrightarrow{PQ}$ 
    the parametrised direction vector,
    }\\
		$ \overrightarrow{PQ} = 
    \begin{pmatrix} \var{w1}-(\var{v1})\\ \var{w2}-(\var{v2}) \\ \var{w3}-(\var{v3}) \end{pmatrix} = 
    \begin{pmatrix}\var{t1}\\ \var{t2} \\ \var{t3} \end{pmatrix}. $
		}
	\end{quickcheck}
	


\lang{de}{
Alternativ dazu lässt sich eine Gerade auch mittels eines Punktes und einer vom Nullvektor 
verschiedenen Richtung beschreiben.
}
\lang{en}{
Alternatively, a line in space can be described by a point in space and a non-zero direction vector.
}

\begin{center}
\image{T110_PointDirectionLine}
\end{center}

\begin{rule}[\lang{de}{Punkt-Richtungs-Darstellung einer Geraden}
             \lang{en}{Point-direction form of a line}]\label{rule:pkt_richt_geraden}
  \lang{de}{
  Die Gerade $g$ durch den Punkt $P$ in Richtung des Vektors $\vec{v}\neq \vec{0}$ ist gegeben durch
  }
  \lang{en}{
  The line $g$ containing the point $P$ and extending in the direction given by the non-zero vector 
  $\vec{v}$ is given by the set of all points whose position vectors have the form
  }
	\[\vec{x} = \vec{OP} + \lambda\cdot \vec{v},\]
  \lang{de}{
	wobei $\lambda\in\R$ die reellen Zahlen durchläuft.
	Dabei heißt dann $\vec{OP}$ \emph{Stützvektor} und $\vec{v}$ \emph{Richtungsvektor} der Geraden.	
	Die Gerade $g$ ist die durch diese Gleichung beschriebene Menge, d. h.
  }
  \lang{en}{
  where $\lambda\in\R$ runs through all real numbers. In this form, the constant vector $\vec{OP}$ is 
  sometimes referred to as the \emph{support vector} and $\vec{v}$ is called the \emph{direction 
  vector} of the line. We often write the set of points with these position vectors simply as a set 
  of vectors
  }
	\[g = \{\vec{OP} + \lambda\cdot \vec{v} \mid \lambda\in\R\},\]
	\lang{de}{
  vgl. Abschnitt über \link{04-lineare-funktionen}{Lineare Funktionen}.
  \\\\
  \floatright{\href{https://www.hm-kompakt.de/video?watch=726}{\image[75]{00_Videobutton_schwarz}}}\\~
  }
  \lang{en}{
  and identify the vectors with coordinates (see \link{04-lineare-funktionen}{linear functions}).
  }
\end{rule}

\lang{de}{
Selbstverständlich lassen sich beide Darstellungen ineinander überführen. Im einen Fall ist 
$\vec{PQ}$ ein Richtungsvektor der Geraden, im anderen Fall ist $\vec{OQ} = \vec{OP} + \vec{v}$ der 
Ortsvektor eines zweiten Punktes auf der Geraden.
}
\lang{en}{
Of course one representation can be transformed into the other representation. In the first case, we 
can take $\vec{PQ}$ to be a direction vector of the line, and in the other case, 
$\vec{OQ} = \vec{OP} + \vec{v}$ can be taken to be the position vector of a second point on the line.
}

\begin{example}
	\lang{de}{
  Zu den Punkten $P = (1;1;1)$ und $Q = (2;2;3)$ im Raum und einer gegebenen Richtung 
  $\vec{v} = \begin{pmatrix} 2\\0\\-2\end{pmatrix}$ bestimmen wir zwei Geraden durch $P$. Die Gerade 
  $g_1$ durch $P$ und $Q$ ist gegeben durch
  }
  \lang{en}{
  Let $P = (1,1,1)$, $Q = (2,2,3)$ and $\vec{v} = \begin{pmatrix} 2\\0\\-2\end{pmatrix}$. From 
  these, we can determine two lines that go through $P$. The first line, $g_1$, goes through both $P$ 
  and $Q$ and is given by
  }
	\[g_1: \vec{x} = \vec{OP} + \lambda\cdot \vec{PQ} = 
    \begin{pmatrix} 1\\1\\1\end{pmatrix} + \lambda\cdot\begin{pmatrix} 1\\1\\2\end{pmatrix}, 
    \quad\lambda\in\R,\]
  \lang{de}{denn }\lang{en}{as }
  $\vec{PQ} = \vec{OQ} - \vec{OP} = 
   \begin{pmatrix} 2\\2\\3\end{pmatrix} - \begin{pmatrix} 1\\1\\1\end{pmatrix} = 
   \begin{pmatrix} 1\\1\\2\end{pmatrix}$.
  \lang{de}{Die Gerade $g_2$ durch $P$ mit der Richtung $\vec{v}$ ist gegeben durch}
  \lang{en}{The second line, $g_2$, goes through $P$ in the direction of $\vec{v}$ and is given by}
	\[g_2: \vec{x} = \vec{OP} + \lambda\cdot \vec{v} = \begin{pmatrix} 1\\1\\1\end{pmatrix} + \lambda\cdot\begin{pmatrix} 2\\0\\-2\end{pmatrix}, \quad\lambda\in\R.\]
\end{example}

\begin{quickcheck}
		\field{rational}
		\type{input.number}
		\begin{variables}
			\randint{v1}{-5}{5}
			\randint{v2}{-5}{5}
			\randint{v3}{-5}{0}
			\randint{w1}{-5}{5}
			\randint{w2}{-5}{5}
			\randint{w3}{1}{5}
			\randint{u1}{-5}{5}
			\randint{u2}{-5}{5}
			\randint{u3}{-5}{0}
			\function[calculate]{t1}{w1-v1}
			\function[calculate]{t2}{w2-v2}
			\function[calculate]{t3}{w3-v3}
		\end{variables}
		
			\text{\lang{de}{
      Bestimmen Sie eine Parameterform der Geraden $g$ durch den Punkt 
      $P=(\var{v1}; \var{v2}; \var{v3})$, die parallel zur Geraden
      }
      \lang{en}{
      Determine a parametrised form of the line $g$ that is parallel to the line
      }
			\[ h: \vec{x}=\begin{pmatrix} \var{u1}\\ \var{u2} \\ \var{u3} \end{pmatrix}+
			r \cdot \begin{pmatrix} \var{t1}\\ \var{t2} \\ \var{t3} \end{pmatrix}, \quad r\in \R \]
			\lang{de}{
      verläuft.
			\\\\
			Man erhält am einfachsten die Parameterform
      }
      \lang{en}{
      and goes through the point $P=(\var{v1}; \var{v2}; \var{v3})$.
      \\\\
      The easiest such parametrised form to obtain is
      }
			\begin{table}[\class{no-padding}]
			\rowspan[l][m]{3} $g:\vec{x}=
			\left(\begin{matrix} \\ \\ \\ \\ \end{matrix}\right.$ &  
			\ansref & \rowspan[l][m]{3} $\left.\begin{matrix} \\ \\ \\ \\  \end{matrix}\right)
			+\lambda \cdot \left(\begin{matrix} \\ \\ \\ \\ \end{matrix}\right.$& 
			\ansref & \rowspan[l][m]{3} $\left.\begin{matrix} \\ \\ \\ \\  \end{matrix}\right), \quad \lambda\in \R $. & \\ 
			\ansref & \ansref & \\ 
			\ansref & \ansref & 
			\end{table}			
			}
		
		\begin{answer}
			\solution{v1}
		\end{answer}
		\begin{answer}
			\solution{t1}
		\end{answer}
		\begin{answer}
			\solution{v2}
		\end{answer}
		\begin{answer}
			\solution{t2}
		\end{answer}
		\begin{answer}
			\solution{v3}
		\end{answer}
		\begin{answer}
			\solution{t3}
		\end{answer}
		\explanation{\lang{de}{
    Als Stützvektor sollte man den Ortsvektor von $P$ nehmen, also 
		$\begin{pmatrix} \var{v1}\\ \var{v2} \\ \var{v3} \end{pmatrix}$ und 
		als Richtungsvektor den Richtungsvektor von $h$, also
		$\begin{pmatrix}\var{t1}\\ \var{t2} \\ \var{t3} \end{pmatrix}$.
    }
		\lang{en}{
    The position vector of $P$ can be taken as the constant vector, 
    $\begin{pmatrix} \var{v1}\\ \var{v2} \\ \var{v3} \end{pmatrix}$ and as the lines are to be 
    parallel, the direction vector can be the same as the direction vector of $h$, 
		$\begin{pmatrix}\var{t1}\\ \var{t2} \\ \var{t3} \end{pmatrix}$.
    }}
	\end{quickcheck}



\section{\lang{de}{Ebenen}\lang{en}{Planes}}
\label{sec:ebenen}

\lang{de}{
Drei paarweise verschiedene Punkte $P$, $Q$ und $R$ im Raum, die nicht auf einer Geraden liegen, definieren eindeutig eine Ebene $E$, die durch diese drei Punkte verläuft.
}
\lang{en}{
Three pairwise distinct points $P$, $Q$, and $R$ in $\R^n$ that do not lie on one line uniquely 
determine a plane $E$ that contains these three points.
}

\begin{center}
	\image{T110_ThreePointPlane}
\end{center}


\begin{rule}[\lang{de}{Drei-Punkte-Darstellung einer Ebene}\lang{en}{Three-point form of a plane}]\label{rule:drei_pkt_ebenen}
   \lang{de}{
   Sind $P$, $Q$ und $R$ drei paarweise verschiedene Punkte, die nicht auf einer (einzigen) Geraden 
   liegen, so ist die Ebene $E$ durch diese drei Punkte gegeben durch die Menge aller Punkte, deren 
   Ortsvektor sich schreiben lässt als
   }
   \lang{en}{
   If $P$, $Q$, and $R$ are three pairwise distinct points that do not lie on a (single) line, then 
   the plane $E$ that goes through these points is the set of all points whose position vectors have 
   the form 
   }
	 \[\vec{x} = 
     \vec{OP} + \lambda\cdot \big(\vec{OQ} - \vec{OP}\big) + \mu\cdot \big(\vec{OR} - \vec{OP}\big), 
     \quad\lambda,\mu\in\R.\]
   \lang{de}{
   Wegen $\vec{OQ} - \vec{OP} = \vec{PQ}$ und $\vec{OR} - \vec{OP} = \vec{PR}$ lässt sich die Ebene 
   auch schreiben als
   }
   \lang{en}{
   Because $\vec{OQ} - \vec{OP} = \vec{PQ}$ and $\vec{OR} - \vec{OP} = \vec{PR}$, the plane can also 
   be written as
   }
	 \[E: \vec{x} = \vec{OP} + \lambda\cdot \vec{PQ} + \mu\cdot \vec{PR}, \quad\lambda,\mu\in\R.\]	
   \lang{de}{Die Ebene $E$ ist die durch diese Gleichung beschriebene Menge, d. h.}
   \lang{en}{We often write the set of points with these position vectors simply as a set of vectors}
	 \[E = \{\vec{OP} + \lambda\cdot \vec{PQ} + \mu\cdot \vec{PR} \mid \lambda,\mu\in\R\}.\]
	 \lang{de}{
   Diese Form der Darstellung einer Ebene nennt man \emph{Parameterform der Ebene}.\\
   \floatright{\href{https://www.hm-kompakt.de/video?watch=731}{\image[75]{00_Videobutton_schwarz}}}\\~
   }
   \lang{en}{This is called a \emph{parametric} representation of the plane.}
\end{rule}

\lang{de}{
Die Tatsache, dass $P$, $Q$ und $R$ nicht auf einer Geraden liegen dürfen, entspricht genau der 
Tatsache, dass $\vec{PQ}$ und $\vec{PR}$ nicht parallel sind.
\\\\
Alternativ dazu lässt sich eine Ebene im Raum auch mittels eines Punktes und zwei vom Nullvektor verschiedener nichtparalleler Richtungen beschreiben.
}
\lang{en}{
The fact that $P$, $Q$, and $R$ should not lie on a single line corresponds to the need for 
$\vec{PQ}$ and $\vec{PR}$ to not be parallel.
\\\\
Alternatively, planes in 3D can be described using a point and two different non-zero directions.
}

\begin{center}
	\image{T110_PointTwoDirectionsPlane}
\end{center}

\begin{rule}[\lang{de}{Punkt-Richtungs-Darstellung einer Ebene}
             \lang{en}{Point-direction form of a plane}]\label{rule:pkt_richt_ebenen}
  \lang{de}{
  Zu einem Punkt $P$ in $\R^3$ und zwei vom Nullvektor verschiedenen nichtparallelen Vektoren 
  $\vec{v}$ und $\vec{w}$ in $\R^3$ ist die Ebene $E$ durch den Punkt $P$ mit den Richtungen 
  $\vec{v}$ und $\vec{w}$ beschrieben durch
  }
  \lang{en}{
  Given a point $P$ in $\R^3$ and two distinct non-zero non-parallel vectors $\vec{v}$ and $\vec{w}$ 
  in $\R^3$, the plane that goes through the point $P$ with directions $\vec{v}$ and $\vec{w}$ is 
  described by
  }
	\[\vec{x} = \vec{OP} + \lambda\cdot \vec{v} + \mu\cdot \vec{w}, \quad\lambda,\mu\in\R.\]
  \lang{de}{
	Dabei heißt dann $\vec{OP}$ \emph{Stützvektor} und die Vektoren $\vec{v}$ und $\vec{w}$ 
  \emph{Richtungsvektoren} der Ebene.
  \\\\
  \floatright{\href{https://www.hm-kompakt.de/video?watch=730}{\image[75]{00_Videobutton_schwarz}}}\\~
  }
  \lang{en}{
  The vector $\vec{OP}$ is called the \emph{support vector} and the vectors $\vec{v}$ and $\vec{w}$ 
  are called \emph{direction vectors} of the plane.
   }\\
\end{rule}

\begin{example}
  \lang{de}{
  Wir bestimmen die Ebene durch die Punkte $P = (1;1;1)$, $Q = (2;3;3)$ und $R = (3;1;-1)$. Wegen 
  $\vec{PQ} = \begin{pmatrix} 1\\2\\2\end{pmatrix}$ und 
  $\vec{PR} = \begin{pmatrix} 2\\0\\-2\end{pmatrix}$ sind $\vec{PQ}$ und $\vec{PR}$ nicht parallel. 
	Die Ebene durch $P$, $Q$ und $R$ ist damit gegeben durch
  }
	\lang{en}{
  Let us find the plane containing the points $P = (1,1,1)$, $Q = (2,3,3)$ and $R = (3,1,-1)$. 
  Because $\vec{PQ} = \begin{pmatrix} 1\\2\\2\end{pmatrix}$ and 
  $\vec{PR} = \begin{pmatrix} 2\\0\\-2\end{pmatrix}$, $\vec{PQ}$ and $\vec{PR}$ are not parallel. 
	The plane through $P$, $Q$, and $R$ is thus given by
  }
	\[E: \vec{x} = \vec{OP} + \lambda\cdot \vec{PQ} + \mu\cdot\vec{PR} = \begin{pmatrix} 1\\1\\1\end{pmatrix} + \lambda\cdot \begin{pmatrix} 1\\2\\2\end{pmatrix} + \mu\cdot\begin{pmatrix}2\\0\\-2\end{pmatrix},\quad \lambda,\mu\in\R.\]
\end{example}

% \begin{quickcheckcontainer}
% \randomquickcheckpool{1}{1}
% \begin{quickcheck}
% 		\field{rational}
% 		\type{input.number}
% 		\begin{variables}
% 			\randint[Z]{a}{-5}{5}
% 			\randint[Z]{b}{1}{4}
% 			\randint{c}{-4}{4}
% 			\randint[Z]{d}{1}{4}
% 		    \function[normalize]{f}{(a/b)*x+c/d}
% 			\function[calculate]{ns}{-(c*b)/(a*d)}
% 		\end{variables}
% 		
% 			\text{Die Nullstelle der linearen Funktion $f(x)=\var{f}$ ist \ansref.}
% 		
% 		\begin{answer}
% 			\solution{ns}
% 		\end{answer}
% 	\end{quickcheck}
% \end{quickcheckcontainer}
% 
% 
% 	\begin{genericGWTVisualization}[550][1000]{mathlet1}
% 		\begin{variables}
% 			\randint{randomA}{1}{2}
% 
% 			\point[editable]{P}{rational}{var(randomA),var(randomA)}
% 		\end{variables}
% 		\color{P}{BLUE}
% 		\label{P}{$\textcolor{BLUE}{P}$}
% 
% 		\begin{canvas}
% 			\plotSize{300}
% 			\plotLeft{-3}
% 			\plotRight{3}
% 			\plot[coordinateSystem]{P}
% 		\end{canvas}
% 		\text{Der Punkt hat die Koordinaten $(\var{P}[x],\var{P}[y])$.}
% 	    	\end{genericGWTVisualization}

\end{visualizationwrapper}


\end{content}