%$Id:  $
\documentclass{mumie.article}
%$Id$
\begin{metainfo}
  \name{
  \lang{en}{...}
  \lang{de}{Einführende Begriffserklärungen}
  }
  \begin{description} 
 This work is licensed under the Creative Commons License Attribution 4.0 International (CC-BY 4.0)   
 https://creativecommons.org/licenses/by/4.0/legalcode 

    \lang{en}{...}
    \lang{de}{...}
  \end{description}
  \begin{components}
  \end{components}
  \begin{links}
  \end{links}
  \creategeneric
\end{metainfo}
\begin{content}
\begin{block}[annotation]
	Im Ticket-System: \href{https://team.mumie.net/issues/32728}{Ticket 32728}
\end{block}

\usepackage{mumie.ombplus}
\ombchapter{4}
\ombarticle{1}

\lang{de}{\title{Einführende Begriffserklärung}}
\begin{block}[info-box]
  \tableofcontents
\end{block}
% \section{Statistik}\label{sec:statistik}

% Der Begriff \notion{Statistik} kommt vom Lateinischen \notion{status}: Staat, Zustand.
% Ursprünglich war damit eine vergleichende Staatsbeschreibung gemeint.
% In der heutigen Zeit sind vor allem zwei Bedeutungen gebräuchlich,
% \begin{itemize}
% \item die \notion{amtliche Statistik}, die teilweise durch gesetzliche Regelungen definiert ist, und
% \item die \notion{(mathematische/angewandte) Statistik}, um die es sich in den folgenden Kapiteln dreht.
% \end{itemize}

% \subsection{Amtliche Statistik}

% Amtliche Statistiken sind Statistiken die von zuständigen Behörden durchgeführt
% werden. Sie sind auch teilweise durch gesetzliche Regelungen definiert. Die 
% durchgeführten Statistiken werden regelmäßig veröffentlicht, z.B. in sogenannten
% statistischen Jahrbüchern.

% \begin{example}
% Einige Ämter die Statistiken durchführen:
% \begin{itemize}
% \item Kraftfahrtbundesamt
% \item Bundesministerien,
% \item Statistisches Bundesamt, Wiesbaden (www.destatis.de)
% \end{itemize}
% \end{example}

% \subsection{Nicht-amtliche Statistik}

% Unter dem Begriff \notion{nicht-amtliche Statistik} versteht man Statistiken, die
% von Meinungsforschungsinstituten oder unabhängigen wissenschaftlichen
% Instituten durchgeführt werden, zum Beispiel Statistiken aufgrund von Umfragen vor Wahlen.

\section{Grundgesamtheit und Stichprobe}

Hauptziel der angewandten Statistik ist es, Daten zu erheben, sie aufzubereiten,
zu beschreiben und zu analysieren. Ein wichtiger Aspekt der Statistik ist dabei das
Entdecken von Strukturen und Zusammenhängen in Datenmaterial unter Nutzung von Werkzeugen
der \notion{Beschreibenden Statistik} (oder \notion{Deskriptiven Statistik}).

Um damit zu Arbeiten, ist es daher unumgänglich die notwendigen Begrifflichkeiten zu 
kennen.

\begin{definition}
Als \notion{Grundgesamtheit} oder \notion{Population} wird ein Menge von räumlich und zeitlich
eindeutig definierten Objekten bezeichnet, die hinsichtlich bestimmter Kriterien übereinstimmen.
Dies ist die Menge von Objekten über die in einer statistischen Untersuchung Daten ermittelt werden.
Die Kriterien für die Auswahl der Menge hängen üblicherweise vom Ziel der Untersuchung ab.

Ein Element der Grundgesamtheit nennt man \notion{statistische Einheit}. Alternative Bezeichnungen sind
\notion{Merkmalsträger}, \notion{Untersuchungseinheit} oder \notion{Messobjekt}.

Eine Teilmenge der Grundgesamtheit nennt man eine \notion{Stichprobe}.
\end{definition}

\begin{example}
Wird eine Untersuchung über das Mobilitätsverhalten von Eltern in einer gewissen Großstadt gewünscht,
so besteht die Grundgesamtheit aus allen Eltern, die zum Zeitpunkt der Erhebung in dieser besagten
Großstadt wohnen. Natürlich sind für eine solche Untersuchung noch viele weitere Fragen zu
klären, insbesondere welche weiteren Daten, z.B. Jahreseinkommen, Anzahl der Kinder etc.,
für eine aussagekräftige Analyse in diesem Zusammenhang erhoben werden sollen.

Ein Elternpaar oder ein alleinerziehendes Elternteil ist dann in diesem Sinne eine statistische Einheit.

Um etwas über das Mobilitätsverhalten zu erfahren, werden Umfragen, Zählungen und ähnliches durchgeführt.
Hierbei alle Eltern mit einzubeziehen, würde jeglichen Rahmen sprengen, weshalb man sich bei der
Datenerhebung auf eine Teilmenge aller Eltern, die \emph{Stichprobe}, beschränkt. 

\end{example}

\section{Merkmale, Merkmalsausprägungen und Datensätze}

\begin{definition}
Als \notion{Merkmal} einer statistischen Einheit bezeichnet man eine spezielle Eigenschaft, die im
Hinblick auf das Ziel einer konkreten Untersuchung von Interesse ist. Den Wert eines Merkmals einer 
statistischen Einheit nennt man \notion{Merkmalsausprägung} und die Menge aller Merkmalsausprägungen nennt man
\notion{Wertebereich}.

Im abstrakten Kontext verwendet man für Merkmale lateinische Großbuchstaben $X,Y,Z,\ldots$ und für ihre Werte, d.h.
zugehörige Merkmalsausprägungen, die entsprechenden Kleinbuchstaben $x,y,z,\ldots$.

Die an einer statistischen Einheit gemessenen Merkmalsausprägung nennt man \notion{Datum} oder auch 
\notion{Messwert} bzw. \notion{Beobachtungswert}.
\end{definition}

\begin{example}\label{example:Merkmale-und-Auspraegungen}
Beim obigen Beispiel des Mobilitätsverhaltens von Eltern wären Merkmale einer statistischen Einheit, also eines
"{}Elternpaars"  oder ein alleinerziehendes Elternteil,
zum Beispiel die Frage, ob es sich um alleinerziehende Eltern handelt, das Jahreseinkommen, die Anzahl der Kinder, 
die Anzahl eigener PKW, die Entfernung zur nächsten Bushaltestelle etc.

\begin{table} 
\head Merkmal & Mögliche Ausprägung & Datum für Eltern A.\\
\body 
Alleinerziehend &  ja, nein & nein \\ 
Jahreseinkommen & 0-5.000, 5.000-10.000,\ldots, 65.000-70.000, 
70.000-125.000,125.000-250.000,\ldots, 1.000.000+ (in Euro)  & 50.000-55.000\\  
Anzahl Kinder & 1,2,3,\ldots  & 2 \\
Anzahl eigener PKW & 0,1,\ldots & 1 \\
Entfernung zur Bushaltestelle & alle reellen Zahlen $\geq 0$ (in Meter) & 134
\end{table}
\end{example}

\begin{remark}
Sofern nicht das subjektive Empfinden Ziel einer Untersuchung ist, sollten die Ausprägungen
möglichst objektiv festgelegt werden!

Man kann auch Merkmale definieren, die aus mehreren einzelnen Eigenschaften/Merkmalen bestehen.
In diesem Fall spricht man von \notion{mehrdimensionalen Merkmalen} oder \notion{multivariaten Merkmalen},
und verwendet zur Bezeichnung Tupel $(X_1,\ldots, X_m)$, wobei die $X_1,\ldots, X_m$ die einzelnen
Merkmale bezeichnet. Die Anzahl $m$ der einzelnen Merkmale wird \notion{Dimension} des Merkmals genannt.

In diesem Zusammenhang spricht man bei Merkmalen, die wirklich nur aus einer Eigenschaft bestehen, 
auch von \notion{univariaten Merkmalen}.
\end{remark}

\begin{definition}
Die Liste aller erhobenen/gemessenen Daten zu einem bestimmten Merkmal wird \notion{Urliste} oder 
\notion{Datensatz} genannt. Üblicherweise werden die statistischen Einheiten durchnummeriert und die
zugehörigen Daten für Merkmal $X$ bekommen die Bezeichnung $x_1,x_2,\ldots, x_n$. Die Anzahl der
Messwerte (hier $n$) wird \notion{Stichprobenumfang} genannt.
\end{definition}

Das folgende Beispiel soll diese Begriffe erläutern.

\begin{example}
Beispielhafte Datensätze zu den Merkmalen "{}Anzahl Kinder" und "{}Alleinerziehend" 
\begin{table}
Merkmal: & Anzahl Kinder  & Alleinerziehend\\
Ausprägungen: & 1,2,3,\ldots & ja, nein \\
Urliste: & 2,2,1,4,3,1,1,2 & nein, nein, ja, ja, nein, nein\\
Stichprobenumfang: & 8 & 6
\end{table}
\end{example}


\section{Merkmalstypen und Skalen}

Wie in Beispiel \ref{example:Merkmale-und-Auspraegungen} gesehen, gibt es verschiedene Möglichkeiten, welcher Art die
Werte der Ausprägungen sind. Dementsprechend unterscheidet man verschiedene Typen von Merkmalen, die
\notion{qualitativen} Merkmale und \notion{quantitativen} Merkmale. Die qualitativen Merkmale werden weiter
unterschieden in \notion{nominal} und in \notion{ordinal}, während die quantitativen Merkmale
in \notion{diskret} und \notion{stetig} weiter unterteilt werden.

\begin{definition}
Bei einem \notion{qualitatives Merkmal} gibt die Ausprägung nur eine Zugehörigkeit oder Beurteilung wider.
Sie dienen zu Unterscheidung verschiedener Arten von Eigenschaften.

Bei einem \notion{quantitativen Merkmal} werden die Ausprägungen durch Zahlen erfasst, die Größen (Quantitäten)
widerspiegeln. Bei ihnen ist die Differenz zweier Zahlen sinnvoll interpretierbar.
\end{definition}

\begin{example}
In Beispiel \ref{example:Merkmale-und-Auspraegungen} ist das Merkmal "{}Alleinerziehend" offensichtlich ein quantitatives Merkmal. Es dient der
Unterscheidung verschiedener Eigenschaften. Auch das Merkmal "Jahreseinkommen" ist ein quantitatives Merkmal, da durch die Einteilung in Bereich lediglich
eine Abstufung vorgenommen wird. Die Differenz zweier Bereiche ist nicht wirklich sinnvoll.

Die Merkmale "{}Anzahl Kinder", "{}Anzahl eigener PKW" und "{}Entfernung zur nächsten Bushaltestelle" sind
hingegen quantitative Merkmale. Die Differenz zweier Ausprägungen beim Merkmal "{}Anzahl Kinder" gibt zum
Beispiel an, wie viele Kinder die einen Eltern mehr haben als die anderen. 
\end{example}

Auch wenn beim Merkmal "Jahreseinkommen" die Differenz zweier Bereiche keinen sinnvollen Wert ergibt, gibt es dennoch
eine Ordnung in den Ausprägungen, weil ein Bereich ein größeres Einkommen angibt, als ein anderer Bereich.
Dagegen liefert das Merkmal "{}Alleinerziehend" lediglich eine Unterscheidung ohne natürliche Rangfolge.

\begin{definition}
Ein qualitatives Merkmal heißt \notion{ordinales} Merkmal, wenn die Ausprägungen einer Rangfolge genügen.
Ein qualitatives Merkmal heißt \notion{nominales} Merkmal, wenn die Ausprägungen keiner natürlichen Rangfolge
genügen. Speziell nennt man ein nominales Merkmal auch \notion{dichotomes} Merkmal, wenn es genau zwei Ausprägungen gibt.
\end{definition}

\begin{example}
Das Merkmal "{}Alleinerziehend" ist ein nominales Merkmal und sogar ein dichotomes Merkmal, da es nur genau die 
zwei Ausprägungen "ja" und "nein" gibt. 

Das Merkmal "Jahreseinkommen", das wir oben schon als qualitativ diagnostiziert haben, ist ein typisches Beispiel
eines ordinalen Merkmals, da die einzelnen Ausprägungen eine natürliche Rangfolge haben. 
\end{example}

\begin{warning}
Auch wenn die Ausprägungen eines Merkmals Zahlen sind oder durch Zahlen kodiert sind, kann
es sich um ein nominal qualitatives Merkmal handeln. Zum Beispiel wäre die Postleitzahl der Eltern ein solches 
nominal qualitatives Merkmal. Ein Größenvergleich der Postleitzahlen ist nicht sinnvoll.

Auch wenn die Ausprägungen keine Zahlen (oder Zahlenbereiche) sind, kann es sich um ordinale Merkmale handeln.
Zum Beispiel bilden die Schulnoten mit den Ausprägungen "sehr gut", "gut", "befriedigend", "{}ausreichend", 
"mangelhaft" und "{}ungenügend" ein ordinales Merkmal, da sie von Anfang zum Ende eine immer schlechtere Leistung 
bezeichnen.
\end{warning}




\end{content}
