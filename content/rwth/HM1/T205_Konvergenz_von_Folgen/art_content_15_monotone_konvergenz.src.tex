%$Id:  $
\documentclass{mumie.article}
%$Id$
\begin{metainfo}
  \name{
    \lang{de}{Monotone Konvergenz}
    \lang{en}{}
  }
  \begin{description} 
 This work is licensed under the Creative Commons License Attribution 4.0 International (CC-BY 4.0)   
 https://creativecommons.org/licenses/by/4.0/legalcode 

    \lang{de}{Beschreibung}
    \lang{en}{}
  \end{description}
  \begin{components}
    \component{generic_image}{content/rwth/HM1/images/g_img_00_Videobutton_schwarz.meta.xml}{00_Videobutton_schwarz}
    \component{generic_image}{content/rwth/HM1/images/g_img_00_video_button_schwarz-blau.meta.xml}{00_video_button_schwarz-blau}
  \end{components}
  \begin{links}
    \link{generic_article}{content/rwth/HM1/T205_Konvergenz_von_Folgen/g_art_content_15_monotone_konvergenz.meta.xml}{content_15_monotone_konvergenz}
    \link{generic_article}{content/rwth/HM1/T202_Reelle_Zahlen_axiomatisch/g_art_content_07_vollstaendigkeit.meta.xml}{vollst}
    \link{generic_article}{content/rwth/HM1/T205_Konvergenz_von_Folgen/g_art_content_14_konvergenz.meta.xml}{konv-folgen}
    \link{generic_article}{content/rwth/HM1/T205_Konvergenz_von_Folgen/g_art_content_13_reelle_folgen.meta.xml}{reelle-folgen}
    \link{generic_article}{content/rwth/HM1/T201neu_Vollstaendige_Induktion/g_art_content_02_vollstaendige_induktion.meta.xml}{vollst-ind}
    \link{generic_article}{content/rwth/HM1/T201neu_Vollstaendige_Induktion/g_art_content_03_binomischer_lehrsatz.meta.xml}{binom}
  \end{links}
  \creategeneric
\end{metainfo}
\begin{content}
\usepackage{mumie.ombplus}
\ombchapter{5}
\ombarticle{3}

\lang{de}{\title{Monotone Konvergenz}}
 
\begin{block}[annotation]
  Beschränkte monotone Folgen konvergieren; Grenzwertberechnung bei rekursiv definierten monotonen Folgen;
  eulersche Zahl $e$ als $\lim_{n\to \infty} (1+1/n)^n$.
    
\end{block}
\begin{block}[annotation]
  Im Ticket-System: \href{http://team.mumie.net/issues/9674}{Ticket 9674}\\
\end{block}

\begin{block}[info-box]
\tableofcontents
\end{block}

\section{Beschränkte monotone Folgen}

Im letzten Abschnitt hatten wir bemerkt, dass konvergente Folgen beschränkt sind.
Für monotone Folgen gilt aber auch die Umkehrung.

\begin{theorem}\label{the:beschmon-folge}
Ist $(a_n)_{n\in \N}$ eine beschränkte monotone Folge, dann ist $(a_n)_{n\in \N}$ konvergent und es gilt
\begin{align*}
 \lim_{n\to \infty} a_n &= \sup \{ a_n| n\in \N\}, & \text{ falls }(a_n)_{n\in \N}\text{ monoton wachsend ist,}\\
 \lim_{n\to \infty} a_n &= \inf \{ a_n| n\in \N\}, & \text{ falls }(a_n)_{n\in \N}\text{ monoton fallend ist.}
\end{align*}
\floatright{\href{https://www.hm-kompakt.de/video?watch=304}{\image[75]{00_Videobutton_schwarz}}}\\\\
\end{theorem}

\begin{remark}
Im Allgemeinen ist es schwierig, das Supremum bzw. das Infimum direkt zu bestimmen. Wir werden aber unten eine
Methode sehen, wie man über Umwege den Grenzwert dennoch bestimmen kann.
\end{remark}

\begin{proof*}[Erklärung]
Für eine beschränkte monoton wachsende Folge $(a_n)_{n\in \N}$ ist zu zeigen, dass $a=\sup \{ a_n| n\in \N\}$ die Bedingung eines Grenzwertes
erfüllt, dass es also zu jedem $\epsilon>0$ ein $N_\epsilon\in \N$ gibt, so dass $|a_n-a|<\epsilon$ für alle $n\geq N_\epsilon$.\\
Sei $W=\{ a_n| n\in \N\}$.
Nach einer \ref[vollst][Charakterisierung des Supremums]{thm:charakterisierung-sup} ist $a_m\leq a$ für alle $a_m\in W$ und es gibt 
für jedes $\epsilon>0$ ein Element $a_m\in W$, so dass $a-\epsilon< a_m$. Setzen wir $N_\epsilon=m$, so gilt wegen der Monotonie
$a_n\geq a_m>a-\epsilon$ für alle $n\geq N_\epsilon$, sowie $a_n\leq a$, und daher
\[ |a_n-a|=a-a_n< \epsilon\quad \text{für alle }n\geq N_\epsilon. \]

Für monoton fallende Folgen funktioniert der Beweis ganz analog. Eine ähnliche Beweisführung findet sich auch im folgenden Video.\\
\floatright{\href{https://api.stream24.net/vod/getVideo.php?id=10962-2-11281&mode=iframe&speed=true}
{\image[75]{00_video_button_schwarz-blau}}}\\\\
\end{proof*}

\begin{example}\label{ex:beispiele}
\begin{enumerate}
\item Die Folge $(a_n)_{n\geq 1}$ mit $a_1=2$ und
$a_{n+1}=\frac{a_n}{2}+\frac{1}{a_n}$ für alle $n\in \N $
aus dem Abschnitt \ref[reelle-folgen][Reelle Folgen]{ex:sqrt-2} ist monoton fallend und nach unten durch $1$ beschränkt, wie
dort gezeigt wurde. Die Folge konvergiert also nach obigem Satz und es gilt 
\[ 2\geq \lim_{n\to \infty} a_n \geq 1. \]
Zur Berechnung des Grenzwerts $a=\lim_{n\to \infty} a_n$ verwendet man folgenden Trick:\\
Die Folge $(a_{n+1})_{n\geq 1}$ besitzt die gleichen Folgeglieder (außer $a_1$), nur dass der Index verschoben ist. Deshalb
konvergiert auch sie und hat denselben Grenzwert $a$. Aus der Rekursionsgleichung erhält man dann mit den
\ref[konv-folgen][Grenzwertregeln]{sec:grenzwertregeln}:
\[ a= \lim_{n\to \infty} a_{n+1} =\lim_{n\to \infty} \left( \frac{a_n}{2}+\frac{1}{a_n} \right)
=\frac{a}{2}+\frac{1}{a}. \]
Diese Gleichung löst man nach $a$ auf:
\begin{eqnarray*}
  a &=& \frac{a}{2}+\frac{1}{a} \\
 \Leftrightarrow \qquad a^2 &=& \frac{a^2}{2}+1 \\
 \Leftrightarrow \qquad \frac{a^2}{2} &=& 1\\
 \Leftrightarrow\qquad  a^2 &=& 2 \\
 \Leftrightarrow  a=\sqrt{2}  &\text{ oder }& a=-\sqrt{2} 
\end{eqnarray*}
Da aber $a\geq 1$ ist, ist daher $a=\sqrt{2}$.
\item Die geometrische Folge $(a_n)_{n\geq 0}= (q^n)_{n\geq 0}$ mit $0<q<1$ ist nach unten beschränkt, denn
$a_n>0$ für alle $n\geq 0$, und monoton fallend, denn 
$a_{n+1}=q^{n+1}=qa_n<a_n$ für alle $n\geq 0$. Damit ist die Folge konvergent. Für den Grenzwert $a$ gilt:
\[ a= \lim_{n\to \infty} a_{n+1} =\lim_{n\to \infty} qa_n =qa, \]
und daher $a(1-q)=0$, d.h. $a=0$. Es ist also
\[ \lim_{n\to \infty} q^n =0, \]
wenn $0<q<1$. 
Für $-1<q<0$ erhält man ebenfalls $\lim_{n\to \infty} q^n =0$, denn dies ist äquivalent dazu, dass 
$\lim_{n\to \infty} |q^n-0| =0$ (vgl. \ref[konv-folgen][Regel aus vorigem Abschnitt]{rule:folge--limes-nullfolge}), aber 
$|q^n-0|=|q|^n$ ist eine geometrische Folge mit $0<|q|<1$.

Mit den Grenzwertregeln erhält man nun schließlich für beliebiges $u\in \R^*$, dass auch die 
geometrische Folge $(uq^n)_{n\geq 0}$ für $|q|<1$ gegen $0$ konvergiert.
\item Für eine natürliche Zahl $m\in \N$ betrachten wir die Folge $\left(\frac{1}{\sqrt[m]{n}}\right)_{n\in \N}$.

Wegen $0<\frac{1}{\sqrt[m]{n}}$ für alle $n\in \N$ ist die Folge nach unten beschränkt. Außerdem gilt wegen der Monotonie der
Wurzel, dass $\sqrt[m]{n}<\sqrt[m]{n+1}$ für alle $n$, d.h. $\frac{1}{\sqrt[m]{n}}>\frac{1}{\sqrt[m]{n+1}}$ für alle $n\in \N$.
Die Folge ist also monoton fallend und beschränkt und besitzt daher einen Grenzwert.\\
Ist $a$ dieser Grenzwert, dann gilt
\[   a^m=\left( \lim_{n\to \infty} \frac{1}{\sqrt[m]{n}} \right)^m=\lim_{n\to \infty}\left(\frac{1}{\sqrt[m]{n}}\right)^m
=\lim_{n\to \infty}\frac{1}{n}=0. \]
Also ist $a=0$, d.h.
\[   \lim_{n\to \infty} \frac{1}{\sqrt[m]{n}}  =0. \]
\end{enumerate}
\end{example}

Diese Beispiele werden auch im folgenden Video anschaulich erläutert:\\\\
\floatright{\href{https://api.stream24.net/vod/getVideo.php?id=10962-2-11282&mode=iframe&speed=true}
{\image[75]{00_video_button_schwarz-blau}}}\\\\

\begin{block}[warning]
Die in den vorigen Beispielen angewendete Methode, den Grenzwert zu berechnen, funktioniert nur, wenn man schon weiß, 
dass es einen Grenzwert gibt.
Ansonsten liefert sie höchstens eine Möglichkeit, die Werte für den Grenzwert einzuschränken.

Man betrachte zum Beispiel die Folge $(a_n)_{n\geq 1}$ mit $a_1=-1$ und $a_{n+1}=-a_n$. Für einen Grenzwert $a$ hätte man dann:
\[ a= \lim_{n\to \infty}a_{n+1} =\lim_{n\to \infty} (-a_n)=-a, \]
also $a=0$. Dies besagt, dass $a=0$ der einzige mögliche Grenzwert ist. In der Tat hat diese Folge aber keinen Grenzwert, denn man 
kann leicht sehen, dass es genau die Folge $((-1)^n)_{n\in \N}$ ist.
\end{block}

\begin{quickcheckcontainer}
\randomquickcheckpool{1}{1}
\begin{quickcheck}
\type{input.number}
		\field{rational}
            
       	\begin{variables}
			%\randint{a}{2}{6} 						
			%\function[normalize]{loes0}{a}
            \drawFromSet{loes0}{0}
        \end{variables}
	
		\lang{de}{
			\text{Gegeben sei die Folge $(a_n)_{n\in\N}$ mit $a_0=1$ sowie $a_{n+1}=\frac{2a_n}{2+a_n}$ für 
            $n\geq 1$. Überlegen Sie sich, wie sie die Konvergenz der Folge begründen können und geben Sie den Grenzwert $a$ der Folge an.\\\\
                        
            Die Folge $(a_n)_{n\in\N}$ konvergiert gegen $a=$\ansref.}}
		
    	\begin{answer}
		\solution{loes0}
		\end{answer}
            
		\explanation{Verwenden Sie obigen \ref[content_15_monotone_konvergenz][Satz]{the:beschmon-folge} und dass für den Grenzwert $a$ gilt: $a=\lim\limits_{n\to\infty}a_n=\lim\limits_{n\to\infty}a_{n+1}$}
	\end{quickcheck}
\end{quickcheckcontainer}

\section{Die eulersche Zahl $e$}\label{sec:eulersche-zahl}

Es gibt mehrere Möglichkeiten, die Eulersche Zahl $e=2,7182818284\ldots$ einzuführen, da sie in mehreren Bereichen der Mathematik
eine Rolle spielt.
Hier definieren wir sie als Grenzwert einer beschränkten monoton wachsenden Folge.

\begin{theorem}\label{thm:ezahl}
Die Folge $\left( (1+\frac{1}{n})^n \right)_{n\in \N}$ ist monoton wachsend und erfüllt
\[   2\leq \left(1+\frac{1}{n}\right)^n < 3\quad \text{für alle }n\in \N. \]

Ihr Grenzwert 
\[  e:= \lim_{n\to \infty} \left(1+\frac{1}{n}\right)^n \approx 2,7182818284 \]
wird \notion{Eulersche Zahl} genannt.
\end{theorem}

\begin{proof*}[Erklärung]
Sei $a_n=(1+\frac{1}{n})^n$, dann ist $a_1=2$ und $a_n\geq 1$ für alle $n\in \N$. Wenn wir die Monotonie zeigen, ist
also auch die linke Ungleichung $2=a_1\leq a_n$ für alle $n\in \N$ gezeigt.\\
Dazu betrachten wir den Quotienten $\frac{a_{n+1}}{a_n}$:
\begin{eqnarray*}
\frac{a_{n+1}}{a_n} &=& \frac{(1+\frac{1}{n+1})^{n+1}}{(1+\frac{1}{n})^n} 
= (1+\frac{1}{n})\left( \frac{1+\frac{1}{n+1}}{1+\frac{1}{n}}\right)^{n+1} \\
&=& \frac{n+1}{n} \left(  \frac{\,\,\frac{n+2}{n+1}\,\,}{\frac{n+1}{n}} \right)^{n+1}
=\frac{n+1}{n} \left(  \frac{ (n+2)n}{(n+1)(n+1)} \right)^{n+1} \\
&=& \frac{n+1}{n} \left(  \frac{ ((n+1)+1)((n+1)-1)}{(n+1)^2} \right)^{n+1} 
= \frac{n+1}{n} \left(  \frac{ (n+1)^2-1}{(n+1)^2} \right)^{n+1} \\
&=& \frac{n+1}{n} \left( 1- \frac{ 1}{(n+1)^2} \right)^{n+1} \\
&\geq & \frac{n+1}{n}\left( 1-  \frac{ n+1}{(n+1)^2} \right)\quad \text{nach Bernoulli-Ungl.}\\
&=& \frac{n+1}{n}\left( 1- \frac{ 1}{n+1} \right)= \frac{n+1}{n} \frac{n+1 -1}{n+1} =1. 
\end{eqnarray*}
Also ist $a_{n+1}\geq a_n$ für alle $n\in \N$.\\
Für die Abschätzung $a_n<3$ verwenden wir die \ref[binom][allgemeine binomische Formel]{thm:binom}
\[ a_n = \left(1+\frac{1}{n}\right)^n=\sum_{k=0}^n \binom{n}{k} \left(\frac{1}{n}\right)^k 
= 1+ \sum_{k=1}^n \frac{n(n-1)\cdot (n-k+1)}{1\cdot 2\cdots k \cdot n^k}. \]
Für die einzelnen Summanden gilt nun 
\[ \frac{n(n-1)\cdot (n-k+1)}{1\cdot 2\cdots k \cdot n^k}\leq \frac{1}{1\cdot 2\cdots k}\leq \frac{1}{1\cdot 2^{k-1}}=\left(\frac{1}{2}\right)^{k-1}, \]
und daher mit Hilfe der \ref[vollst-ind][geometrischen Summenformel]{rule:geom-summe}
\[ a_n \leq 1+ \sum_{k=1}^n \left(\frac{1}{2}\right)^{k-1}=1+\sum_{j=0}^{n-1} \left(\frac{1}{2}\right)^j=1+ \frac{1-\left(\frac{1}{2}\right)^n}{1-\frac{1}{2}}
=1+  2-2\cdot (\frac{1}{2})^n<3.\]
\floatright{\href{https://api.stream24.net/vod/getVideo.php?id=10962-2-11283&mode=iframe&speed=true}
{\image[75]{00_video_button_schwarz-blau}}}\\\\
\end{proof*}



\end{content}