%$Id:  $
\documentclass{mumie.article}
%$Id$
\begin{metainfo}
  \name{
    \lang{de}{Reelle Folgen}
    \lang{en}{}
  }
  \begin{description} 
 This work is licensed under the Creative Commons License Attribution 4.0 International (CC-BY 4.0)   
 https://creativecommons.org/licenses/by/4.0/legalcode 

    \lang{de}{Beschreibung}
    \lang{en}{}
  \end{description}
  \begin{components}
\component{generic_image}{content/rwth/HM1/images/g_tkz_T205_Inventory.meta.xml}{T205_Inventory}
\component{generic_image}{content/rwth/HM1/images/g_img_00_Videobutton_schwarz.meta.xml}{00_Videobutton_schwarz}
\component{generic_image}{content/rwth/HM1/images/g_img_00_Videobutton_blau.meta.xml}{00_Videobutton_blau}
\component{generic_image}{content/rwth/HM1/images/g_img_00_video_button_schwarz-blau.meta.xml}{00_video_button_schwarz-blau}
%\component{generic_image}{content/rwth/HM1/T205_Konvergenz_von_Folgen/Audios/g_img_Button_schwarz.meta.xml}{Button_schwarz}
%\component{generic_image}{content/rwth/HM1/T205_Konvergenz_von_Folgen/Audios/g_img_Button_3.meta.xml}{Button_3}
%\component{generic_image}{content/rwth/HM1/T205_Konvergenz_von_Folgen/Audios/g_img_Button_2.meta.xml}{Button_2}
%\component{generic_image}{content/rwth/HM1/T205_Konvergenz_von_Folgen/Audios/g_img_Button_1.meta.xml}{Button_1}
%\component{generic_sound}{content/rwth/HM1/T205_Konvergenz_von_Folgen/Audios/g_snd_Folgendefinition_Kurztest_neu.meta.xml}{Folgendefinition_Kurztest_neu}
%\component{generic_sound}{content/rwth/HM1/T205_Konvergenz_von_Folgen/Audios/g_snd_Folgendefinition_Beispiel_2_neu.meta.xml}{Folgendefinition_Beispiel_2_neu}
%\component{generic_sound}{content/rwth/HM1/T205_Konvergenz_von_Folgen/Audios/g_snd_Folgendefinition_Beispiel_1_neu.meta.xml}{Folgendefinition_Beispiel_1_neu}
%\component{generic_sound}{content/rwth/HM1/T205_Konvergenz_von_Folgen/Audios/g_snd_Folgendefinition_Bemerkung_neu.meta.xml}{Folgendefinition_Bemerkung_neu}
%\component{generic_sound}{content/rwth/HM1/T205_Konvergenz_von_Folgen/Audios/g_snd_Folgendefinition_neu.meta.xml}{Folgendefinition_neu}
%\component{generic_sound}{content/rwth/HM1/T205_Konvergenz_von_Folgen/Audios/g_snd_Test_1.meta.xml}{Test_1}
\end{components}
  \begin{links}
\link{generic_article}{content/rwth/HM1/T208_Reihen/g_art_content_24_reihen_und_konvergenz.meta.xml}{content_24_reihen_und_konvergenz}
\link{generic_article}{content/rwth/HM1/T204_Abbildungen_und_Funktionen/g_art_content_12_reelle_funktionen_monotonie.meta.xml}{content_12_reelle_funktionen_monotonie}
\link{generic_article}{content/rwth/HM1/T303_Approximationen/g_art_content_05_newtonverfahren.meta.xml}{content_05_newtonverfahren}
\link{generic_article}{content/rwth/HM1/T205_Konvergenz_von_Folgen/g_art_content_13_reelle_folgen.meta.xml}{content_13_reelle_folgen}
\end{links}
  \creategeneric
\end{metainfo}
\begin{content}
\usepackage{mumie.ombplus}
\ombchapter{5}
\ombarticle{1}

\lang{de}{\title{Reelle Folgen}}
 
\begin{block}[annotation]
  
  
\end{block}
\begin{block}[annotation]
  Im Ticket-System: \href{http://team.mumie.net/issues/9655}{Ticket 9655}\\
\end{block}

\begin{block}[info-box]
\tableofcontents
\end{block}


\begin{example}
  \begin{tabs*}[\initialtab{0}]
    \tab{Lagerhaltung 1a}
  Ein Logistikunternehmen lagert unzählige Materilaien und Stoffe in seinen Hallen und auf dem
  Außengelände der Firma. Dabei führen sie digital ausführlich Buch darüber, wo genau welche und wie viele 
  Materialien gelagert werden. Intern bezeichnet das Unternehmen diesen Vorgang als Lagerhaltung.\\
  
 In den nächsten Tagen soll ihr Bestand erweitert werden, weshalb das Unternehmen eine große
 Warenanlieferung eines Produkts erwartet. Dafür muss eine neue Lagerstätte vorbereitet werden.
 Die zuständigen Mitarbeiter fertigen große Regalsysteme, die bei Bedarf nach oben hin um weitere
 Fächer ausgebaut werden können. Die weitere Planung beschränkt sich auf eines dieser Regalsysteme 
 - wie nachfolgend dargestellt.
 
\begin{center}
\image{T205_Inventory}
\end{center}

Aus der Masse eines jeden Kartons, der in das Regalsystem eingelagert werden soll, ermitteln die Mitarbeiter
die höchstzulässige Lagermenge an Kartons für das unterste Fach: $1024$ Stück. Sie entscheiden sich dafür, 
in ein jeweils darüber liegendes Fach $50$ Kartons weniger einräumen zu wollen als in das darunter liegende Fach.
Denn es ist bekannt, dass jeweils die Gesamtmasse aller Fächer, die über einem Fach liegen, auf die Stabilität des
Fachs einwirken. In das System zur Lagerhaltung notieren sie folgendes:\\

 Lagerstätte: Außenbereich, $C3$, Regalsystem $A10$:
    \begin{itemize}
        \item Fach Nummer 0: Produkt $A$, $1024$ Stück
        \item Fach Nummer 1: Produkt $A$, $974$ Stück
        \item Fach Nummer 2: Produkt $A$, $924$ Stück
        \item Fach Nummer 3: Produkt $A$, $874$ Stück
    \end{itemize}

Wir sehen hier also eine Durchnummerierung der Fächer, wobei jedem Fach mit der Nummer $n$ eine bestimmte
Anzahl an Produkten der Sorte $A$ in Abhängigkeit der Fachnummer $n$ zugeordnet wird. In diesem Sinne kann
sofort die Anzahl an Produkten in einem bestimmten Fach ermittelt werden.\\\\
Steht $A(n)$ für die Anzahl an Produkten in Fach $n$, so erhalten wir sofort die Anzahl der Produkte durch die folgende
Berechnungsformel:\\\\
$\quad\quad\quad\quad\quad\quad\quad\quad A(n)=1024-50\cdot n$\\\\
Die Aufstellung derartiger Zuordnungen geschieht mit Hilfe von (arithmetischen) \notion{Folgen}.

 \tab{Lagerhaltung 1b}
 
 Kurz vor der Warenanlieferung bekommt das Unternehmen eine unangekündigte Kontrolle durch einen Gutachter.
 Dieser schaut sich das neue Regalsystem an und bemängelt folgendes:\\
 
 "Bei dieser Art Regalsystem, darf die Gesamtmasse der eingelagerten Produkte aller Fächer, die über
 einem bestimmten Fach liegen, die Masse der eingelagerten Produkte dieses (unteren) Fachs nicht überschreiten."\\
 
 Der zuständige Mitarbeiter schlägt daraufhin vor, ein jedes Fach mit der halben Masse des darunter liegenden Fachs
 zu füllen. Dadurch kommt er zu folgender Lagerhaltung:\\
 
  Lagerstätte: Außenbereich, $C3$, Regalsystem $A10$:
    \begin{itemize}
        \item Fach Nummer 0: Produkt $A$, $1024$ Stück
        \item Fach Nummer 1: Produkt $A$, $512$ Stück
        \item Fach Nummer 2: Produkt $A$, $256$ Stück
        \item Fach Nummer 3: Produkt $A$, $128$ Stück
    \end{itemize}
 
Auch hier findet eine Durchnummerierung der Fächer statt, wobei jedem Fach mit der Nummer $n$ eine bestimmte
Anzahl an Produkten der Sorte $A$ in Abhängigkeit der Fachnummer $n$ zugeordnet wird.\\\\
Steht $A(n)$ für die Anzahl an Produkten in Fach $n$, so erhalten wir sofort die Anzahl der Produkte durch die folgende
Berechnungsformel:\\\\
$\quad\quad\quad\quad\quad\quad\quad\quad A(n)=1024\cdot (\frac{1}{2})^n$\\\\
Die Aufstellung derartiger Zuordnungen geschieht mit Hilfe von \notion{geometrischen Folgen}.\\

Die Antwort auf die Frage, ob der Vorschlag des Mitarbeiters ausreicht, um die Vorschriften des Gutachters 
einhalten zu können, wird im Kapitel \ref[content_24_reihen_und_konvergenz][Reihen]{def:reihe} beantwortet.

    \end{tabs*}
\end{example}


In diesem Themenblock befassen wir uns mit reellen Folgen. 
Folgen sind für die Anwendung wichtig, insbesondere wenn es darum geht, sukzessive bessere Näherungswerte (z. B. für die Lösung einer Gleichung) zu bestimmen.
Wir werden sie aber auch benötigen, um die Begriffe "Stetigkeit" und "Differenzierbarkeit" definieren zu können.

\section{Folgen und ihre Wertemengen}

\begin{definition}
%\audio[1,0.5,0.8,1.25,1.5]{Folgendefinition_neu}\\
Eine reelle (oder reellwertige) \notion{Folge} ist eine reelle Funktion mit Definitionsbereich $\N$.\\
Statt $a:\N\to \R,n\mapsto a(n)$ bezeichnet man eine solche Folge üblicherweise mit 
\[ (a_n)_{n \in \N}, \] 
wobei $a_n:=a(n)$ der Funktionswert an der Stelle $n$ ist. Man nennt $a_n$ auch \notion{n-tes
Folgeglied} der Folge $(a_n)_{n \in \N}$.\\

Die \notion{Wertemenge} einer Folge $(a_n)_{n \in \N}$ ist die Wertemenge als Funktion,
d.h. die Menge $\{ a_n | n \in \N\}$.\\
\floatright{\href{https://www.hm-kompakt.de/video?watch=300}{\image[75]{00_Videobutton_schwarz}}}\\\\
\end{definition}

Da wir in diesem Themenblock nur reelle Folgen behandeln, werden wir oft das Wort \emph{reell} weglassen
und nur von \emph{Folgen} reden, auch wenn \emph{reelle Folgen} gemeint sind.

\begin{remark}
%\audio[1,0.5,0.8,1.25,1.5]{Folgendefinition_Bemerkung_neu}\\
\begin{enumerate}
\item Oft wird als Folge auch allgemeiner eine reelle Funktion bezeichnet, deren Definitionsbereich
eine Menge $\{n\in \Z | n\geq n_0\}$ ist für eine feste ganze Zahl $n_0$. Insbesondere
taucht oft als Definitionsmenge bzw. in der üblichen Schreibweise als Indexmenge die Menge $\Nzero$
auf.
\item Für den Index kann eine beliebige Variable verwendet werden. 
\[ (a_n)_{n\in \N},\quad  (a_k)_{k\in \N}\quad \text{und} \quad (a_l)_{l\in \N} \]
bezeichnen also dieselbe Folge.
\item Andere Bezeichnungen für eine Folge $(a_n)_{n\in \N}$ sind auch $(a_n)_{n\geq 1}$
oder $(a_n)_{n=1}^\infty$.
\end{enumerate}
\end{remark}

Es gibt im Wesentlichen zwei Weisen, in denen Folgen beschrieben werden: Zum einen
durch eine \notion{explizite Formel} für $a_n$ in Abhängigkeit von $n$, oder zum anderen
\notion{rekursiv}, d.h. "`ein paar"' Folgeglieder werden explizit angegeben, und das
allgemeine Folgeglied $a_n$ wird ausgedrückt durch $n$ und Folgeglieder $a_k$ mit $k<n$
 (s. Beispiele).

\begin{example}\label{ex:22}
%\href{https://www.hm-kompakt.de/video?watch=300}{\image[75]{Button_schwarz}}\\
%\audio[1,0.5,0.8,1.25,1.5]{Folgendefinition_Beispiel_1_neu}\\
Beispiele mit expliziten Formeln:
\begin{enumerate}
\item \emph{Konstante Folgen:} Für $c\in \R$ ist die Folge $(a_n)_{n\in \N}$ mit $a_n=c$ für alle $n$
eine sogenannte konstante Folge. Die Wertemenge ist lediglich $\{c\}$.
\item $(a_n)_{n\in \N}=((-1)^n)_{n\in \N}$, d.h. $a_n=(-1)^n$ für alle $n\in \N$ ist eine Folge
mit Wertemenge $\{-1;1\}$. Diese Folge lässt sich auch beschreiben durch
\[ a_n= \begin{cases} 1, & \text{falls }n \text{ gerade} \\ 
-1, & \text{falls }n \text{ ungerade}.\end{cases}\]
\item Durch
\[ a_n= \begin{cases} \frac{n}{2}, &\text{falls } n \text{ gerade} \\ 
\frac{1-n}{2} &\text{falls } n, \text{ ungerade}\end{cases}\]
wird eine Folge $(a_n)_{n\geq 1}$ mit Wertemenge $\Z$ definiert.
\item \emph{Arithmetische Folgen:} Für $c,d\in \R$ nennt man die Folge $(a_n)_{n\geq 0}$ mit
\[  a_n=c+nd \quad \text{für alle } n\in \Nzero \]
eine \emph{arithmetische Folge}. Arithmetische Folgen sind dadurch gekennzeichnet, dass die Differenz
aufeinanderfolgender Folgenglieder stets dieselbe ist, hier $a_{n+1}-a_n=d$ für alle $n\in \Nzero$.\\
Konstante Folgen sind spezielle arithmetische Folgen mit $d=0$.
\item \emph{Geometrische Folgen:} Für $u,q\in \R^*$ nennt man die Folge $(a_n)_{n\geq 0}$ mit
\[  a_n=u\cdot q^n \quad \text{für alle } n\in \Nzero \]
eine \emph{geometrische Folge}. Geometrische Folgen sind dadurch gekennzeichnet, dass der Quotient
aufeinanderfolgender Folgenglieder stets derselbe ist, hier $\frac{a_{n+1}}{a_n}=q$ für alle $n\in \Nzero$.\\
Konstante Folgen sind spezielle geometrische Folgen mit $q=1$.
\end{enumerate}
\end{example}

Bevor rekursiv definierte Folgen vorgestellt werden, kann sich ein Video angeschaut werden, dass die bisherige
 Definition einer Folge und zwei einfache Beispiele noch einmal näher erläutert:\\
 \floatright{\href{https://api.stream24.net/vod/getVideo.php?id=10962-2-11265&mode=iframe&speed=true}
{\image[75]{00_video_button_schwarz-blau}}}\\\\

\begin{example}\label{ex:33}
%\audio[1,0.5,0.8,1.25,1.5]{Folgendefinition_Beispiel_2_neu}\\
Beispiele mit rekursiven Definitionen:
\begin{enumerate}
\item \label{ex:sqrt-2} Die Folge $(a_n)_{n\geq 1}$ mit $a_1=2$ und für 
alle $n\in\mathbb{N}$ gelte $a_{n+1}=\frac{a_n}{2}+\frac{1}{a_n}$ ist eine rekursiv definierte Folge.\\\\
Die ersten Folgeglieder sind also $a_1=2$, sowie
\begin{align*} a_2 &= \frac{a_1}{2}+\frac{1}{a_1} &=\frac{2}{2}+\frac{1}{2} &=\frac{3}{2} &=1,5\, , \\
a_3 &= \frac{a_2}{2}+\frac{1}{a_2} &=\frac{3}{4}+\frac{2}{3}&=\frac{17}{12}&=1,41\bar{6}\, , \\
a_4 &= \frac{a_3}{2}+\frac{1}{a_3}&=\frac{17}{24}+\frac{12}{17}&=\frac{577}{408} &\approx 1,4142157\, , \\
a_5 &= \frac{a_4}{2}+\frac{1}{a_4}&=\frac{577}{816}+\frac{408}{577}
&=\frac{665857}{470832} &\approx 1,414213562375\, .
\end{align*}
Diese Folge taucht beim \href{https://de.wikipedia.org/wiki/Heron-Verfahren}{Heron-Verfahren} zur
näherungsweisen Berechnung von $\sqrt{2}\approx 1,414213562373$ auf. Das 
\ref[content_05_newtonverfahren][Heron-Verfahren]{rem:heron-verfahren} stellt einen 
Spezialfall des \ref[content_05_newtonverfahren][Newton-Verfahrens]{the:newton-verfahren} für $n=2$ dar.
\item \label{ex:fibonacci} \emph{Fibonacci-Folge:} Die Definition $f_0 = f_1 = 1$ und
\[f_{n+2} = f_{n+1} + f_n,\quad n \in \Nzero\]
definiert eine Folge $(f_n)_{n\geq 0}$, die sogenannte \emph{Fibonacci-Folge}.
Die ersten Folgeglieder sind also $f_0=1$ und $f_1=1$, sowie
\begin{align*} f_2 &= f_1+f_0 &= 1+1 &= 2\, ,\\
f_3 &= f_2+f_1 &= 2+1 &= 3\, ,\\
f_4 &= f_3+f_2 &= 3+2 &= 5\, ,\\
f_5 &= f_4+f_3 &= 5+3 &= 8\, .
\end{align*}
\item Arithmetische Folgen können auch rekursiv definiert werden durch
$a_0=c$ und $a_{n+1}=a_n+d$ für alle $n\in \Nzero$, wobei wieder $c,d\in \R$ festgewählte Zahlen 
sind.\\
Ebenso kann für $u,q\in \R^*$ die geometrische Folge $(a_n)_{n\geq 0}=(uq^n)_{n\geq 0}$ 
rekursiv definiert werden durch $a_0=u$ und $a_{n+1}=qa_n$.
\end{enumerate}
\end{example}

\begin{quickcheckcontainer}
\randomquickcheckpool{1}{1}
\begin{quickcheck}
\type{input.number}
		\field{rational}
            
       	\begin{variables}
			\randint{x2}{2}{4}	
            \randint{a}{1}{4}   						
			\function[normalize]{afo}{(1/x2)^n}
            \function[normalize]{loes0}{1*a}
            \function[normalize]{loes1}{a+1/x2}
            \function[normalize]{loes2}{a+1/x2+1/x2^2}
                                         
		\end{variables}
	
		\lang{de}{
			\text{
            %\audio[1,0.5,0.8,1.25,1.5]{Folgendefinition_Kurztest_neu}\\
            Geben Sie die ersten drei Folgenglieder der rekursiven Folge 
            $(a_n)_{n\in\N}$ mit $a_1=\var{a}$ und $a_{n+1}= a_n + \var{afo}$ für $n\geq 1$ an.
		\\\\
        Geben Sie die Werte exakt und so weit gekürzt wie möglich ein.\\\\
        
        Die ersten drei Folgenglieder lauten: $a_1=$\ansref, $a_2=$\ansref und $a_3=$\ansref.
 		}
       		}
		
    	\begin{answer}
		\solution{loes0}
		\end{answer}
            
		\begin{answer}
		\solution{loes1}
		\end{answer}
        
        \begin{answer}
		\solution{loes2}
		\end{answer}
     
        \explanation{Verwenden Sie das jeweils zuvor berechnete Folgenglied, um das nächste Folgenglied zu bestimmen.} 
	\end{quickcheck}
\end{quickcheckcontainer}


\section{Monotonie und Beschränktheit}

Da reelle Folgen spezielle reelle Funktionen sind, können wir die Begriffe der Monotonie und Beschränktheit
direkt auf Folgen übertragen.


\begin{definition}\label{def:beschränkte_Folge}
Eine Folge $(a_n)_{n\in \N}$ heißt \notion{nach oben beschränkt} bzw. \notion{nach unten beschränkt}
bzw. \notion{beschränkt}, wenn ihre Wertemenge $\{ a_n | n\in \N\}$ nach oben beschränkt bzw. 
nach unten beschränkt bzw. beschränkt ist.
\end{definition}


\begin{definition}\label{def:monotone_Folge}
Eine Folge $(a_n)_{n\in \N}$ heißt \notion{monoton wachsend}, wenn $a_n\leq a_m$ für 
alle $n,m\in \N$ mit $n\leq m$ gilt bzw. \notion{streng monoton wachsend}, wenn $a_n< a_m$ für 
alle $n,m\in \N$ mit $n< m$ gilt.\\
Sie heißt \notion{monoton fallend}, wenn $a_n\geq a_m$ für 
alle $n,m\in \N$ mit $n\leq m$ gilt bzw. \notion{streng monoton fallend}, wenn $a_n> a_m$ für 
alle $n,m\in \N$ mit $n< m$ gilt.
\end{definition}

\begin{remark}
\begin{enumerate}
\item Die Monotonie einer Folge $(a_n)_{n\in\N}$ ist gleichbedeutend mit der Monotonie der entsprechenden
Funktion $a:\N\to \R,n\mapsto a(n)$ im Sinne der Definition der \ref[content_12_reelle_funktionen_monotonie][Monotonie einer Funktion]{Monotonie}.  
\item Für die Monotonie von Folgen reicht es jeweils ein Folgeglied mit dem nächsten zu vergleichen.\\
Eine Folge $(a_n)_{n\in \N}$ ist genau dann (streng) monoton wachsend, wenn $a_n\leq a_{n+1}$ (bzw.
$a_n< a_{n+1}$) für alle $n\in \N$ gilt.\\
Entsprechendes gilt für monoton fallende Folgen.
\item Monoton wachsende Folgen sind stets nach unten beschränkt, denn das erste Folgeglied ist
das kleinste Element der Wertemenge. Ebenso sind monoton fallende Folgen nach oben beschränkt,
da das erste Folgeglied das größte Element der Wertemenge ist. 
\item Eine Folge $(a_n)_{n\in\N}$ ist genau dann nach oben unbeschränkt, wenn es zu jedem $M>0$ ein
$N\in\N$ gibt mit $a_N>M$. Analoges gilt für Unbeschränktheit nach unten.
\end{enumerate}
\end{remark}


\begin{example}\label{ex:monotone-folgen}

\begin{tabs*}[\initialtab{0}] 
  
\tab{\lang{de}{Monotonie}}
\lang{de}{
\begin{enumerate}
\item Wie auch schon konstante Funktionen sind konstante Folgen sowohl monoton wachsend 
als auch monoton fallend.
\item Die Folge $(a_n)_{n\in \N}=((-1)^n)_{n\in \N}$ ist weder monoton wachsend noch monoton fallend, denn für
alle geraden $n\in \N$ gilt $a_n=1>-1=a_{n+1}$ und für alle ungeraden $n\in \N$ gilt $a_n=-1<1=a_{n+1}$.
\item \ref[content_13_reelle_folgen][Arithmetische Folgen]{ex:22} $(c+dn)_{n\geq 0}$ sind streng monoton wachsend, wenn $d>0$ ist, und streng monoton
fallend, wenn $d<0$ ist, denn für alle $n\in \Nzero$ ist 
\begin{align*} 
a_{n+1} &=a_n+d >a_n,& \text{ falls } d>0\\ 
a_{n+1} &=a_n+d <a_n,& \text{ falls } d<0.
\end{align*}
\item Bei \ref[content_13_reelle_folgen][geometrischen Folgen]{ex:22} $(a_n)_{n\geq 0}=(uq^n)_{n\geq 0}$ hängt die Monotonie sowohl von 
$q$ als auch von $u$ ab.\\
Ist $q<0$ so ist die Folge nicht monoton, denn die Folgeglieder sind abwechselnd positiv und
negativ (vgl. 2.Beispiel, wo $u=1$ und $q=-1$ ist).
Für $q>0$ haben alle $a_n$ das gleiche Vorzeichen wie $u$.\\
Wegen $a_{n+1}=qa_n$ ist also die Frage, ob $qa_n$ größer oder kleiner als
$a_n$ ist, was zum einen von der Größe von $q$ und vom Vorzeichen von $a_n$ abhängt, und man erhält
folgende $5$ Fälle:
\begin{itemize}
\item $u>0$, $q>1$: Dann ist $a_n>0$ und $a_{n+1}=qa_n>a_n$ für alle $n\geq 0$. Also ist die 
Folge streng monoton wachsend.
\item $u>0$, $0<q<1$: Dann ist $a_n>0$ und $a_{n+1}=qa_n<a_n$ für alle $n\geq 0$. Also ist die 
Folge streng monoton fallend.
\item $u<0$, $q>1$: Dann ist $a_n<0$ und $a_{n+1}=qa_n<a_n$ für alle $n\geq 0$. Also ist die 
Folge streng monoton fallend.
\item $u<0$, $0<q<1$: Dann ist $a_n<0$ und $a_{n+1}=qa_n>a_n$ für alle $n\geq 0$. Also ist die 
Folge streng monoton wachsend.
\item  $q=1$: Dann ist die Folge konstant (s.o.).
\end{itemize}
\item Die Folge $(a_n)_{n\geq 1}$ mit $a_1=2$ und
\[  a_{n+1}=\frac{a_n}{2}+\frac{1}{a_n}\quad \forall n\in \N. \]
ist streng monoton fallend. Dies zeigt man mit vollständiger Induktion.
\begin{incremental}
\step Vielmehr zeigt man mit vollständiger Induktion für alle $n\in \N$ die Aussage:
\[   A(n):\, 1< a_n\leq 2< a_n^2\text{ und } a_{n+1}<a_n. \]
  \begin{enumerate}
    \item[(IA)] $n=1$: Es ist $a_1=2$ und $a_2=\frac{3}{2}$. Also ist $1< a_1\leq 2< a_1^2$ erfüllt
    und auch $a_2<a_1$.
    \item[(IS)] $n\rightarrow n+1$: Man hat also die Induktionsvoraussetzung f"ur ein festes $n\in\N$:
    \begin{center}(IV) $A(n):\, 1< a_n\leq 2< a_n^2\text{ und } a_{n+1}<a_n$.\end{center}
    und muss f"ur dieses $n$ die Aussage 
    \[ A(n+1): \, 1< a_{n+1}\leq 2< a_{n+1}^2\text{ und } a_{n+2}<a_{n+1}. \] zeigen:

	Nach Definition ist $a_{n+1}=\frac{a_n}{2}+\frac{1}{a_n}=\frac{1}{2}\left( a_n+ \frac{2}{a_n}\right)$, d.h.
	$a_{n+1}$ ist das arithmetische Mittel der Zahlen $a_n$ und $\frac{2}{a_n}$.
	Nach Induktionsvoraussetzung ist $a_n\leq 2<a_n^2$ und daher $1\leq \frac{2}{a_n}<a_n$.\\
	Da $a_{n+1}$ das arithmetische Mittel ist, ist $\frac{2}{a_n}<a_{n+1}<a_n$ und insbesondere
	$1 < a_{n+1}\leq 2$. Außerdem ist das arithmetische Mittel verschiedener Zahlen strikt größer als
	ihr geometrisches Mittel, weshalb außerdem
	\[ a_{n+1}>\sqrt{a_n\cdot \frac{2}{a_n}} =\sqrt{2}, \]
	d.h. $a_{n+1}^2>2$.\\
	Schließlich folgt $a_{n+2}<a_{n+1}$ wieder aus der Tatsache, dass $\frac{2}{a_{n+1}}<a_{n+1}$ und dass
	$a_{n+2}$ das arithmetische Mittel der beiden Zahlen ist.
  \end{enumerate}

\end{incremental}
\item Die \lref{ex:fibonacci}{Fibonacci-Folge} $(f_n)_{n\geq 0}$ ist monoton wachsend, aber nicht streng monoton
wachsend, denn $f_0=f_1$.\\ Betrachtet man jedoch nur die Folge $(f_n)_{n\geq 1}$ (lässt also
$f_0$ weg), erhält man eine streng monoton wachsende Folge.\\
\begin{incremental}
\step Formal zeigt man dies mit vollständiger Induktion. Und zwar betrachte man für alle 
$n\in \N$ die Aussage
\[   A(n): \, f_{n-1}>0,\quad f_n>0 \text{ und } f_{n+1}>f_n.  \]
  welche man mit vollständiger Induktion beweist:
  \begin{enumerate}
    \item[(IA)] $n=1$: Es ist $f_0=1>0$ und $f_1=1>0$, sowie $f_2=2>1=f_1$. Also ist $A(1)$ wahr.
    \item[(IS)] $n\rightarrow n+1$: Man hat also die Induktionsvoraussetzung f"ur ein festes $n\in\N$:
    \begin{center}(IV) $A(n):\, f_{n-1}>0,\quad f_n>0 \text{ und } f_{n+1}>f_n$\end{center}
    und muss f"ur dieses $n$ die Aussage 
    \[ A(n+1):f_{n}>0,\quad f_{n+1}>0 \text{ und } f_{n+2}>f_{n+1} \] zeigen:

	$f_n>0$ ist nach Induktionsvoraussetzung wahr und ebenfalls nach Induktionsvoraussetzung
	ist $f_{n+1}>f_n>0$. Schließlich ist
	
    \begin{align*}
      f_{n+2} &= f_{n+1}+f_n &\quad \text{nach Definition der Folge} \\
      &> f_{n+1}+0 &\quad \text{da }f_n>0. \\
    \end{align*}
  \end{enumerate}
\end{incremental}
\end{enumerate}
}
\tab{\lang{de}{Beschränktheit}}
\lang{de}{
\begin{enumerate}
\item Die Folge $(a_n)_{n\in \N}=((-1)^n)_{n\in \N}$ hat als Wertemenge $\{-1;1\}$. Sie ist also beschränkt.
\item \ref[content_13_reelle_folgen][Arithmetische Folgen]{ex:22} sind zwar monoton und daher in eine Richtung beschränkt, aber nicht in die andere. 
\item Bei \ref[content_13_reelle_folgen][geometrischen Folgen]{ex:22} hängt die Frage nach der Beschränktheit von $u$ und $q$ ab. \\
Ist $|q|\leq 1$ so ist die Folge beschränkt, denn $|a_n|=|uq^n|=|u|\cdot |q|^n \leq |u|$ für alle $n\in\Nzero$, 
weshalb die Wertemenge durch $|u|$ beschränkt ist.\\
Für $q>1$ ist die Folge zwar monoton, und daher in eine Richtung beschränkt, aber nicht in die andere. Und für $q<-1$
ist die Folge weder nach oben noch nach unten beschränkt. 
\item Für die Folge $(a_n)_{n\geq 1}$ mit $a_1=2$ und
\[  a_{n+1}=\frac{a_n}{2}+\frac{1}{a_n}\quad \forall n\in \N \]
hatten wir oben gezeigt, dass $1<a_n\leq 2$ für alle $n\in \N$ gilt. Diese Folge ist also beschränkt.
\item Die \ref[content_13_reelle_folgen][Fibonacci-Folge]{ex:33} ist monoton wachsend und daher nach unten beschränkt. Sie ist jedoch nicht
nach oben beschränkt. Vielmehr lässt sich leicht zeigen, dass $f_n\geq n$ für alle $n\in \N$ gilt.
\end{enumerate}
}
\end{tabs*}
\end{example}

Im folgenden Video wird die Definition von Monotonie und Beschränktheit einer Folge noch einmal anhand von Beispielen erläutert:\\\\ 
 \floatright{\href{https://www.hm-kompakt.de/video?watch=301}{\image[75]{00_Videobutton_schwarz}}}\\\\

\begin{quickcheckcontainer}
\randomquickcheckpool{1}{1}
\begin{quickcheck}
\type{input.number}
		\field{rational}
            
       	\begin{variables}
			\randint{a}{2}{9} 						
			\function[normalize]{loes0}{a+1}
        \end{variables}
	
		\lang{de}{
			\text{Gegeben sei die rekursive Folge $(a_n)_{n\in\N}$ mit $a_0=\var{a}$ sowie $a_{n+1}=a_n+\frac{1}{2^{n+1}}$ für 
            $n\geq 1$. Geben sie die kleinstmögliche obere Schranke $c\in\N$ an, durch die die Folge $(a_n)_{n\in\N}$ nach oben beschränkt 
            ist.\\\\
            
            Die kleinstmögliche natürliche Zahl $c$, durch die die Folge $(a_n)_{n\in\N}$ nach oben beschränkt ist, lautet $c=$\ansref.}}
		
    	\begin{answer}
		\solution{loes0}
		\end{answer}
            
		\explanation{Verwenden Sie das jeweils zuvor berechnete Folgenglied, um das nächste Folgenglied zu bestimmen und machen Sie
        sich klar, wie das jeweils nächste Folgenglied rechnerisch zustande kommt.} 
	\end{quickcheck}
\end{quickcheckcontainer}


\end{content}