\documentclass{mumie.problem.gwtmathlet}
%$Id$
\begin{metainfo}
  \name{
    \lang{de}{A05: Häufungspunkte}
    \lang{en}{mc yes-no}
  }
  \begin{description} 
 This work is licensed under the Creative Commons License Attribution 4.0 International (CC-BY 4.0)   
 https://creativecommons.org/licenses/by/4.0/legalcode 

    \lang{de}{Beschreibung}
    \lang{en}{description}
  \end{description}
  \corrector{system/problem/GenericCorrector.meta.xml}
  \begin{components}
    \component{js_lib}{system/problem/GenericMathlet.meta.xml}{gwtmathlet}
  \end{components}
  \begin{links}
  \end{links}
  \creategeneric
\end{metainfo}
\begin{content}
\usepackage{mumie.ombplus}
\usepackage{mumie.genericproblem}

\lang{de}{\title{A05: Häufungspunkte}}
\lang{en}{\title{Problem 5}}

\begin{block}[annotation]
	Im Ticket-System: \href{http://team.mumie.net/issues/9862}{Ticket 9862}
\end{block}


\begin{problem}

  	\begin{question}
        \begin{variables}
			\drawFromSet{a}{3,5,7}
            \randint{b}{6}{8}
            \randint{c}{2}{3}
            \randint{d}{2}{8}
            \function{loes1}{1*e}
            \number{loes2}{1}
            \function[calculate]{loes3}{d*e}
        \end{variables}
		\type{input.finite-number-set}
		\field{real}
        \correctorprecision{2}
        
		\lang{de}{\text{Bestimmen Sie die Häufungspunkte bzw. den Grenzwert der nachstehenden beiden Folgen $(a_n)_{n\in\mathbb{N}}$
        und $(b_n)_{n\in\mathbb{N}}$ mit\\\\
        $a_{n}=\begin{cases}
         \left(1+\frac{1}{\var{a}n+1}\right)^{\var{a}n+1}&\text{ , falls }n \text{ gerade}\\
	     1+\var{b}^{-n}&\text{ , falls }n \text{ ungerade.}
        \end{cases} $\\\\
        sowie\\\\
         $b_{n}=\var{d}~(1+\frac{1}{n^2+\var{c}})^{n^2}$
        }}
        \explanation{Untersuchen Sie die Folge $a_n$ auf Teilfolgen. Denken Sie an Folgen, die den Grenzwert $e$ besitzen.}
    
		\begin{answer}
			\text{Die Häufungspunkte bzw. der Grenzwert von $a_n$ lauten/lautet:}
			\solution{loes1,loes2}
		\end{answer}
		\begin{answer}
			\text{Die Häufungspunkte bzw. der Grenzwert von $b_n$ lauten/lautet:}
			\solution{loes3}
		\end{answer}
		
	\end{question}

  \begin{question}
    \lang{de}{\text{Kreuzen Sie alle Aussagen an, die auf die Folgen $(a_n)_{n\in\mathbb{N}}$ und $(b_n)_{n\in\mathbb{N}}$ von Aufgabenteil $a)$ zutreffen.}}
    \explanation{Beachten Sie, dass der Grenzwert einer konvergenten Folge eindeutig bestimmt ist.}
    \type{mc.multiple}
    \permutechoices{1}{5}
    
       \begin{choice}
      \lang{de}{\text{
      Die Folge $(a_n)_{n\in\mathbb{N}}$ konvergiert.
      }}
      \solution{false}
    \end{choice}  
    
    \begin{choice}
      \lang{de}{\text{
      Die Folge $(a_n)_{n\in\mathbb{N}}$ hat zwei Häufungspunkte.
      }}
      \solution{true}
    \end{choice}
    
      \begin{choice}
      \lang{de}{\text{
      Die Folge $(b_n)_{n\in\mathbb{N}}$ hat genau einen Häufungspunkt und konvergiert.
      }}
      \solution{true}
    \end{choice}
    
      \begin{choice}
      \lang{de}{\text{
      Die Folge $(b_n)_{n\in\mathbb{N}}$ besitzt keinen Häufungspunkt.
      }}
      \solution{false}
    \end{choice}
    
      \begin{choice}
      \lang{de}{\text{
      Beide Folgen besitzen mindestens eine konvergente Teilfolge.
      }}
      \solution{true}
    \end{choice}
       
   \end{question}

\end{problem}

\embedmathlet{gwtmathlet}

\end{content}