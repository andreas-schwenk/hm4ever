\documentclass{mumie.problem.gwtmathlet}
%$Id$
\begin{metainfo}
  \name{
    \lang{de}{A03: Folgenabschätzung}
    \lang{en}{input numbers}
  }
  \begin{description} 
 This work is licensed under the Creative Commons License Attribution 4.0 International (CC-BY 4.0)   
 https://creativecommons.org/licenses/by/4.0/legalcode 

    \lang{de}{Die Beschreibung}
    \lang{en}{}
  \end{description}
  \corrector{system/problem/GenericCorrector.meta.xml}
  \begin{components}
    \component{js_lib}{system/problem/GenericMathlet.meta.xml}{gwtmathlet}
  \end{components}
  \begin{links}
  \end{links}
  \creategeneric
\end{metainfo}
\begin{content}
\usepackage{mumie.ombplus}
\usepackage{mumie.genericproblem}

\lang{de}{\title{A03: Folgenabschätzung}}
\lang{en}{\title{Problem 3}}

\begin{block}[annotation]
	Im Ticket-System: \href{http://team.mumie.net/issues/9861}{Ticket 9861}
\end{block}

\begin{problem}

	\begin{question}
		\lang{de}{
			\text{Bestimmen Sie für jede der Folgen $(a_{n})_{n\in\N}$ das minimale $N\in\N$, so dass $|a_{n}|<\frac{1}{10}$ für alle $n\geq N$.
		\\
 		a) Ist $a_{n}=\frac{\var{x}}{n}$ für alle $n\in\N$, so ist das minimale $N$ gerade $N=$\ansref.\\
  		b) Ist $a_{n}=(\var{q})^{n}$ für alle $n\in\N$, so ist das minimale $N$ gerade $N=$\ansref.\\
 		c) Ist $a_{n}=\frac{\var{k}}{n^{2}}$ für alle $n\in\N$, so ist das minimale $N$ gerade $N=$\ansref.\\
		
				}}
		
		\explanation{Formen Sie die Ungleichung nach $n$ um, nachem Sie die Folgendefinition für $a_n$ eingesetzt haben.}
		\type{input.number}
		\field{rational}
		

    \precision[false]{3}
    \displayprecision{3}
    \correctorprecision{3}
		
		\begin{variables}
			\randint[Z]{x}{-10}{10}
			\function[calculate]{xl}{10*|x|+1}
			
			\randint{qq}{2}{10}
			\randint[Z]{pmeins}{-1}{1}
			\function[calculate]{q}{pmeins/qq}
			\function[calculate]{ql}{floor(-ln(10)/ln(pmeins*q))+1}
			
			\randint[Z]{k}{-10}{10}
			\function[calculate]{kl}{floor(sqrt(10*|k|))+1}
			
		\end{variables}
		
		\begin{answer}
			\solution{xl}
			
		\end{answer}
		\begin{answer}
			\solution{ql}
		\end{answer}
		\begin{answer}
			\solution{kl}
		\end{answer}
	\end{question}
	
\end{problem}

\embedmathlet{gwtmathlet}


\end{content}