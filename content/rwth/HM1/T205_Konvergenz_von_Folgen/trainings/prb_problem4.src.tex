\documentclass{mumie.problem.gwtmathlet}
%$Id$
\begin{metainfo}
  \name{
    \lang{de}{A06: Grenzwertberechnung}
    \lang{en}{input numbers}
  }
  \begin{description} 
 This work is licensed under the Creative Commons License Attribution 4.0 International (CC-BY 4.0)   
 https://creativecommons.org/licenses/by/4.0/legalcode 

    \lang{de}{Die Beschreibung}
    \lang{en}{}
  \end{description}
  \corrector{system/problem/GenericCorrector.meta.xml}
  \begin{components}
    \component{js_lib}{system/problem/GenericMathlet.meta.xml}{gwtmathlet}
  \end{components}
  \begin{links}
  \end{links}
  \creategeneric
\end{metainfo}
\begin{content}
\usepackage{mumie.ombplus}
\usepackage{mumie.genericproblem}

\lang{de}{\title{A06: Grenzwertberechnung}}
\lang{en}{\title{Problem 6}}

\begin{block}[annotation]
	Im Ticket-System: \href{http://team.mumie.net/issues/9863}{Ticket 9863}
\end{block}

\begin{problem}

% Erweiterung um Folgen insb. exp>polyn (a), d))
% Teil b) ist neu (mit unendlich)

	\begin{question}
		\lang{de}{
			\text{Bestimmen Sie den Grenzwert $a$ der Folgen $(a_{n})_{n\in\N}$ und geben Sie die Lösung vollständig gekürzt an.
		\\
 		1. $a_{n}= \frac{\var{x22} \cdot n^2}{(-\var{x33}n+\var{bnenner})^\var{x44}} $ für alle $n\in\N$\\ 
        2. $a_{n}= \frac{\var{azahler}}{\var{anenner}} $ für alle $n\in\N$ \\ 
 		3. $a_{n}= \frac{\var{x00} \cdot (\var{bzahler})^4}{(n^2+\var{bnenner})^2} $ für alle $n\in\N$\\
        4. $a_{n}= \frac{\var{x22}\cdot n^\var{x33}}{\var{x44}^n} $ für alle $n\in\N$\\
		}
		}
		%\frac{\var{x0}n^{2}+(\var{x1})n+(\var{x2})}{\var{y0}n^{2}+(\var{y1})n+(\var{y2})}
		%\frac{\var{x00}(n+(\var{x11}))^{4}}{(n^{2}+(\var{y}))^{2}}
		\type{input.number}
		\field{rational}

    \precision[false]{3}
    \displayprecision{3}
    \correctorprecision{3}
		
		\begin{variables}
			\randint[Z]{x0}{-5}{5}
			\randint{x1}{-5}{5}
			\randint{x2}{-5}{5}
			
			\randint[Z]{y0}{1}{10}
			\randint{y1}{1}{10}
			\randint[Z]{y2}{1}{10}
			
			\randint[Z]{y}{-10}{10}
			
			\randint[Z]{x00}{-10}{10}
			\randint{x11}{-10}{10}
            
            \randint{x22}{2}{8}
            \randint{x33}{2}{8}
            \randint{x44}{3}{5}
			
			\function{al}{x0/y0}
			\function{bl}{x00}
            \number{cl}{0}
			
			\function[normalize]{azahler}{x0*n*n+ x1 * n+ x2}
			\function[normalize]{anenner}{y0 * n*n+y1 * n+ y2 }
			
			\function[normalize]{bzahler}{n+x11}
			
			\function[normalize]{bnenner}{y*y}
		
		\end{variables}
		
		\begin{answer}
			\text{1. $a = $}
			\solution{cl}
		\end{answer}
		
		\begin{answer}
			\text{2. $a = $}
			\solution{al}
		\end{answer}
    
        \begin{answer}
			\text{3. $a = $}
			\solution{bl}
		\end{answer}
        
        \begin{answer}
			\text{4. $a = $}
			\solution{cl}
		\end{answer}
	
    \end{question}
    
    \begin{question}
		\lang{de}{
			\text{Bestimmen Sie den Grenzwert $a$ der unten stehenden Folgen - vollständig gekürzt -
            in $\R$, sofern diese konvergieren. Wenn die Folge divergiert, so geben Sie 
            $\{\pm\infty\}$ ein (Eingabe als \textit{infty}).\\
        1. $(\frac{\var{azahler}}{\var{anenner}})_{n\in\N}$\\\\ 
 		2. $(\frac{\var{bzahler}}{\var{bnenner}})_{n\in\N}$\\\\
        3. $(\frac{\var{czahler}}{\var{cnenner}})_{n\in\N}$\\\\
		4. $(\frac{\var{dzahler}}{\var{dnenner}})_{n\in\N}$\\\\
		}
		}
		%\frac{\var{x0}n^{2}+(\var{x1})n+(\var{x2})}{\var{y0}n^{2}+(\var{y1})n+(\var{y2})}
		%\frac{\var{x00}(n+(\var{x11}))^{4}}{(n^{2}+(\var{y}))^{2}}
		\type{input.number}
		\field{rational}
        \explanation{Merkregel: Der Teil des Bruchs (Zähler oder Nenner) ist stärker, der den größeren Exponenten hat!}

    %\precision[false]{3}
    %\displayprecision{3}
    %\correctorprecision{3}
		
		\begin{variables}
			\randint[Z]{x0}{-5}{-1}
			\randint{x1}{-5}{5}
			\randint{x2}{2}{6}
            \randint{x3}{-5}{5}
			\randint{x4}{2}{8}
            \randint{x5}{2}{8}
            \randint{x6}{2}{3}
			
			\randint[Z]{y0}{1}{10}
			\randint{y1}{1}{10}
					
			\function[normalize]{azahler}{x0*n+x1}
			\function[normalize]{anenner}{y0*n*n+y1}
			
			\function[normalize]{bzahler}{x2*n^5-x3}
			\function[normalize]{bnenner}{x4*n^4+5}
            
            \function[normalize]{czahler}{3*n^(2*x5)}
			\function[normalize]{cnenner}{x4*(n^2+2)^x5}
		
            \function[normalize]{dzahler}{n*(2*n-5)^x6}
			\function[normalize]{dnenner}{(3*n+1)^(x6+1)}
        
            \function[normalize]{aloes}{0}
            \function[normalize]{bloes}{infty}
		    \function[normalize]{cloes}{3/x4}
		    \function[normalize]{dloes}{2^x6/3^(x6+1)}
		
      
		\end{variables}
		
		\begin{answer}
			\text{1. $a = $}
			\solution{aloes}
		\end{answer}
		
		\begin{answer}
			\text{2. $a = $}
			\solution{bloes}
		\end{answer}
        
        \begin{answer}
			\text{3. $a = $}
			\solution{cloes}
		\end{answer}
        
          \begin{answer}
			\text{4. $a = $}
			\solution{dloes}
		\end{answer}
	
    \end{question}
	
	
\end{problem}

\embedmathlet{gwtmathlet}

\end{content}