\documentclass{mumie.problem.gwtmathlet}
%$Id$
\begin{metainfo}
  \name{
    \lang{de}{A04: Folgeneigenschaften}
    \lang{en}{Problem 4}
  }
  \begin{description} 
 This work is licensed under the Creative Commons License Attribution 4.0 International (CC-BY 4.0)   
 https://creativecommons.org/licenses/by/4.0/legalcode 

    \lang{de}{Beschreibung}
    \lang{en}{}
  \end{description}
  \corrector{system/problem/GenericCorrector.meta.xml}
  \begin{components}
    \component{js_lib}{system/problem/GenericMathlet.meta.xml}{mathlet}
  \end{components}
  \begin{links}
  \end{links}
  \creategeneric
\end{metainfo}
\begin{content}
\begin{block}[annotation]
	Im Ticket-System: \href{https://team.mumie.net/issues/18865}{Ticket 18865}
\end{block}
\usepackage{mumie.genericproblem}
\lang{de}{\title{A04: Folgeneigenschaften}}
\lang{en}{\title{Problem 4}}

\begin{problem}
  \begin{question}
    \lang{de}{\text{
    Kreuzen Sie alle wahren Aussagen an.
    }}
    \lang{de}{\explanation{Betrachten Sie die divergenten Folgen $(a_n)_{n\in\N}=(-1)^n$ und $(b_n)_{n\in\N}=n$ im Hinblick auf Beschränktheit.\\
    Schauen sie sich insbesondere die Theoreme aus den ersten beiden Abschnitten zu Folgen noch mal genauer an.
    }}
    \type{mc.multiple}
    \permutechoices{1}{5}
    \begin{choice}
      \lang{de}{\text{
      Jede konvergente Folge ist beschränkt.
      }}
      \solution{true}
    \end{choice}

    \begin{choice}
      \lang{de}{\text{
      Eine Folge $(a_n)_{n\in\N}$ konvergiert genau dann gegen
      $a\in\R$, wenn die Folge $(a_n-a)_{n\in\N}$ eine Nullfolge ist.
      }}
      \solution{true}
    \end{choice}

    \begin{choice}
      \lang{de}{\text{
      Jede streng monotone wachsende Folge ist konvergent.
      }}
      \solution{false}
    \end{choice}

    \begin{choice}
      \lang{de}{\text{
      Jede beschränkte monotone Folge ist konvergent.
      }}
      \solution{true}
    \end{choice}

    \begin{choice}
      \lang{de}{\text{
      Jede divergente Folge ist unbeschränkt.
      }}
      \solution{false}
    \end{choice}

  \end{question}

\end{problem}

\embedmathlet{mathlet}
\end{content}