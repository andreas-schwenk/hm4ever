\documentclass{mumie.problem.gwtmathlet}
%$Id$
\begin{metainfo}
  \name{
    \lang{de}{A07: Grenzwertberechnung}
    \lang{en}{input numbers}
  }
  \begin{description} 
 This work is licensed under the Creative Commons License Attribution 4.0 International (CC-BY 4.0)   
 https://creativecommons.org/licenses/by/4.0/legalcode 

    \lang{de}{Die Beschreibung}
    \lang{en}{}
  \end{description}
  \corrector{system/problem/GenericCorrector.meta.xml}
  \begin{components}
    \component{js_lib}{system/problem/GenericMathlet.meta.xml}{gwtmathlet}
  \end{components}
  \begin{links}
  \end{links}
  \creategeneric
\end{metainfo}
\begin{content}
\usepackage{mumie.ombplus}
\usepackage{mumie.genericproblem}

\lang{de}{\title{A07: Grenzwertberechnung}}
\lang{en}{\title{Problem 7}}

\begin{block}[annotation]
	Im Ticket-System: \href{http://team.mumie.net/issues/9864}{Ticket 9864}
\end{block}

\begin{problem}

	\begin{question}
		\lang{de}{
			\text{
			Gegeben sei die Folge $(a_{n})_{n\in\N}$ mit $a_{1}=1$ und $a_{n+1}=\frac{\var{xx}}{a_n+\var{x}}$ für alle $n\in\N$. 
			Geben Sie den Grenzwert $a$ der konvergenten Folge an.\\\\
            \textit{Hinweis: Geben Sie den Grenzwert $a$ der Folge exakt ein. Verwenden Sie dazu ggfls. Brüche und Wurzelausdrücke 
            (Eingabe: sqrt(Zahl))}
		}
		}
		\explanation{Verwenden Sie statt der Folgenglieder $a_{n+1}$ und $a_n$ den gesuchten Grenzwert 
        $a$ in der Gleichung und formen Sie nach $a$ um.}
		\type{input.function}
		\field{rational}
		

    
		
		\begin{variables}
			
			\randint{x}{1}{20}
			\function[calculate]{xx}{x*x}
			\function{null}{x/2*(sqrt(5)-1)}
		
			
		\end{variables}
		
		\begin{answer}
			\text{$a = $}
			\solution{null}
            \checkAsFunction{c}{-10}{10}{100}
		\end{answer}
		
	\end{question}
	
\end{problem}

\embedmathlet{gwtmathlet}

\end{content}