\documentclass{mumie.problem.gwtmathlet}
%$Id$
\begin{metainfo}
  \name{
    \lang{de}{A01: Folgenglieder}
    \lang{en}{input numbers}
  }
  \begin{description} 
 This work is licensed under the Creative Commons License Attribution 4.0 International (CC-BY 4.0)   
 https://creativecommons.org/licenses/by/4.0/legalcode 

    \lang{de}{Die Beschreibung}
    \lang{en}{}
  \end{description}
  \corrector{system/problem/GenericCorrector.meta.xml}
  \begin{components}
    \component{js_lib}{system/problem/GenericMathlet.meta.xml}{gwtmathlet}
  \end{components}
  \begin{links}
  \end{links}
  \creategeneric
\end{metainfo}
\begin{content}
\begin{block}[annotation]
	Im Ticket-System: \href{https://team.mumie.net/issues/18864}{Ticket 18864}
\end{block}

\usepackage{mumie.ombplus}
\usepackage{mumie.genericproblem}

\lang{de}{\title{A01: Folgenglieder}}
\lang{en}{\title{Problem 1}}

\begin{problem}

% neue TA1

	\begin{question}
		\lang{de}{
			\text{Berechnen Sie die ersten vier Folgenglieder der Folge 
            $(a_n)_{n\in\N}$ mit $a_{n}= \var{afo}$.
		\\\\
        Geben Sie die Werte so weit gekürzt wie möglich ein.
 		}
       		}
		\type{input.number}
		\field{rational}

             \explanation{Setzen Sie in die Definition der Folge die ersten vier $n\in\N$ ein.}

    	\begin{variables}
			\randint[Z]{x1}{2}{8}
            \randint[Z]{x2}{2}{4}	
            \randint[Z]{x3}{1}{3}
          						
			\function[normalize]{afo}{(x1*n)/(x2*n+x3)}	
            \function[normalize]{loes1}{x1/(x2*1+x3)}
            \function[normalize]{loes2}{(x1*2)/(x2*2+x3)}
            \function[normalize]{loes3}{(x1*3)/(x2*3+x3)}
            \function[normalize]{loes4}{(x1*4)/(x2*4+x3)}
                      
		\end{variables}
		
		\begin{answer}
			\text{$a_1 = $}
			\solution{loes1}
		\end{answer}
        
        \begin{answer}
			\text{$a_2 = $}
			\solution{loes2}
		\end{answer}
        
        \begin{answer}
			\text{$a_3 = $}
			\solution{loes3}
		\end{answer}
        
        \begin{answer}
			\text{$a_4 = $}
			\solution{loes4}
		\end{answer}
		
	\end{question}

\begin{question}
		\lang{de}{
			\text{Geben Sie die ersten drei Folgenglieder der rekursiven Folge 
            $(b_n)_{n\in\N}$ an mit $b_1=1$ und $b_{n+1}= b_n + \var{afo}$ für $n\geq 1$.
		\\\\
        Geben Sie die Werte so weit gekürzt wie möglich ein.
 		}
       		}
		\type{input.number}
		\field{rational}
            
            \explanation{Da es sich um eine rekursive Folge handelt, brauchen Sie bei der 
            Berechnung des nächsten Folgenglieds das vorherige. }


    	\begin{variables}
			\randint[Z]{x1}{2}{5}
            \randint[Z]{x2}{2}{4}	
               						
			\function[normalize]{afo}{x1/(x2^n)}
            \function{loes0}{1}
            \function[normalize]{loes1}{1+x1/x2^1}
            \function[normalize]{loes2}{(1+x1/x2^1)+x1/x2^2}
                                         
		\end{variables}
	
    	\begin{answer}
			\text{$b_1 = $}
			\solution{loes0}
		\end{answer}
            
		\begin{answer}
			\text{$b_2 = $}
			\solution{loes1}
		\end{answer}
        
        \begin{answer}
			\text{$b_3 = $}
			\solution{loes2}
		\end{answer}
        
       % \begin{answer}
		%	\text{$b_4 = $}
		%	\solution{loes3}
		%\end{answer}
        
    \end{question}

\end{problem}

\embedmathlet{gwtmathlet}

\end{content}