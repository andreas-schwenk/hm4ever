%$Id:  $
\documentclass{mumie.article}
%$Id$
\begin{metainfo}
  \name{
    \lang{de}{Überblick: Konvergenz von Folgen}
    \lang{en}{overview: }
  }
  \begin{description} 
 This work is licensed under the Creative Commons License Attribution 4.0 International (CC-BY 4.0)   
 https://creativecommons.org/licenses/by/4.0/legalcode 

    \lang{de}{Beschreibung}
    \lang{en}{}
  \end{description}
  \begin{components}
  \end{components}
  \begin{links}
\link{generic_article}{content/rwth/HM1/T205_Konvergenz_von_Folgen/g_art_content_16_konvergenzkriterien.meta.xml}{content_16_konvergenzkriterien}
\link{generic_article}{content/rwth/HM1/T205_Konvergenz_von_Folgen/g_art_content_15_monotone_konvergenz.meta.xml}{content_15_monotone_konvergenz}
\link{generic_article}{content/rwth/HM1/T205_Konvergenz_von_Folgen/g_art_content_14_konvergenz.meta.xml}{content_14_konvergenz}
\link{generic_article}{content/rwth/HM1/T205_Konvergenz_von_Folgen/g_art_content_13_reelle_folgen.meta.xml}{content_13_reelle_folgen}
\end{links}
  \creategeneric
\end{metainfo}
\begin{content}
\begin{block}[annotation]
	Im Ticket-System: \href{https://team.mumie.net/issues/30132}{Ticket 30132}
\end{block}
\begin{block}[annotation]
Copy of : /home/mumie/checkin/content/rwth/HM1/T206_Folgen_II/art_T206_overview.src.tex
\end{block}





\begin{block}[annotation]
Im Entstehen: Überblicksseite für Kapitel Konvergenz von Folgen
\end{block}

\usepackage{mumie.ombplus}
\ombchapter{1}
\lang{de}{\title{Überblick: Konvergenz von Folgen}}
\lang{en}{\title{}}



\begin{block}[info-box]
\lang{de}{\strong{Inhalt}}
\lang{en}{\strong{Contents}}


\lang{de}{
    \begin{enumerate}%[arabic chapter-overview]
   \item[5.1] \link{content_13_reelle_folgen}{Reelle Folgen}
   \item[5.2] \link{content_14_konvergenz}{Konvergenz von Folgen}
   \item[5.3] \link{content_15_monotone_konvergenz}{Monotone Konvergenz}
   \item[5.4] \link{content_16_konvergenzkriterien}{Konvergenzkriterien}
   \end{enumerate}
} %lang

\end{block}

\begin{zusammenfassung}

\lang{de}{Reelle Folgen sind Abbildungen von den natürlichen Zahlen in die reellen, für die wir die Schreibweise $(a_n)_{n\in\N}$ prägen.
Folgen können insbesondere explizit durch eine Abbildungsvorschrift oder rekursiv definiert werden.

Wir präsentieren berühmte Folgen wie die harmonische und die geometrischen Folgen, und wir studieren mögliche Eigenschaften wie Monotonie und Beschränktheit.

Folgen betrachtet man meist, um ihre Werteentwicklung zu studieren. Wir führen die Begriffe Konvergenz und Grenzwert sowie Divergenz ein und untersuchen elementare Beispiele.

Mit den Grenzwertregeln stellen wir das wichtigste Werkzeug zur Konvergenzuntersuchung zur Verfügung.

Weitere Konvergenzkriterien wie die der monotonen Beschränktheit, 
das Sandwich-Lemma und Teilfolgenkonvergenz komplettieren das Grundwissen über Folgen.
}


\end{zusammenfassung}

\begin{block}[info]\lang{de}{\strong{Lernziele}}
\lang{en}{\strong{Learning Goals}} 
\begin{itemize}[square]
\item \lang{de}{Sie kennen den Begriff einer Folge und wichtige Beispiele.}
\item \lang{de}{Sie bestimmen die Abbildungsvorschrift aus rekursiv definierten Folgen.}
\item \lang{de}{Sie prüfen Folgen auf Monotonie und Beschränktheit.}
\item \lang{de}{Sie kennen den Begriff der Folgenkonvergenz, des Grenzwerts und der Divergenz und erläutern ihn an Beispielen.}
\item \lang{de}{Sie kennen die Grenzwertsätze und die Konvergenzkriterien, die durch monotone Beschränktheit, das Sandwich-Lemma und Teilfolgenkonvergenz gegeben werden.}
\item \lang{de}{Sie wenden diese Kriterien an, um Grenzwerte von Folgen zu bestimmen bzw. diese auf Konvergenz/Divergenz zu untersuchen.}
\end{itemize}
\end{block}




\end{content}
