%$Id:  $
\documentclass{mumie.article}
%$Id$
\begin{metainfo}
  \name{
    \lang{de}{Folgenkonvergenz}
    \lang{en}{}
  }
  \begin{description} 
 This work is licensed under the Creative Commons License Attribution 4.0 International (CC-BY 4.0)   
 https://creativecommons.org/licenses/by/4.0/legalcode 

    \lang{de}{Beschreibung}
    \lang{en}{}
  \end{description}
  \begin{components}
    \component{generic_image}{content/rwth/HM1/images/g_img_00_Videobutton_schwarz.meta.xml}{00_Videobutton_schwarz}
    \component{generic_image}{content/rwth/HM1/images/g_img_00_video_button_schwarz-blau.meta.xml}{00_video_button_schwarz-blau}
    \component{js_lib}{system/media/mathlets/GWTGenericVisualization.meta.xml}{mathlet1}
  \end{components}
  \begin{links}
    \link{generic_article}{content/rwth/HM1/T205_Konvergenz_von_Folgen/g_art_content_14_konvergenz.meta.xml}{content_14_konvergenz}
    \link{generic_article}{content/rwth/HM1/T205_Konvergenz_von_Folgen/g_art_content_13_reelle_folgen.meta.xml}{reelle-folgen}
    \link{generic_article}{content/rwth/HM1/T205_Konvergenz_von_Folgen/g_art_content_15_monotone_konvergenz.meta.xml}{monotone-konvergenz}
    \link{generic_article}{content/rwth/HM1/T205_Konvergenz_von_Folgen/g_art_content_16_konvergenzkriterien.meta.xml}{konv-krit}
    \link{generic_article}{content/rwth/HM1/T210_Stetigkeit/g_art_content_30_elem_funktionen.meta.xml}{elem-funk}
  \end{links}
  \creategeneric
\end{metainfo}
\begin{content}
\usepackage{mumie.ombplus}
\ombchapter{5}
\ombarticle{2}
\usepackage{mumie.genericvisualization}

\begin{visualizationwrapper}

\lang{de}{\title{Konvergenz von Folgen}}
 
\begin{block}[annotation]
  Inhalt: $\epsilon$-Definition;Bsp. $1/n$; Eindeutigkeit des Grenzwerts; 
  Beschränktheit konvergenter Folgen; Grenzwertsätze
  
  TODO: In Mathlet: Nur die Folgenpunkte zeichnen statt $f$? Aber wie?
  
\end{block}
\begin{block}[annotation]
  Im Ticket-System: \href{http://team.mumie.net/issues/9656}{Ticket 9656}\\
\end{block}

\begin{block}[info-box]
\tableofcontents
\end{block}

\section{Konvergenz und Divergenz}

\begin{definition}\label{def:folgenkonvergent}
Eine Folge reeller Zahlen $(a_n)_{n\in \N}$ heißt \notion{konvergent}, wenn eine Zahl $a \in \R$ existiert, so dass
die folgende Eigenschaft erfüllt ist:

 F"ur jedes $\epsilon > 0$ gibt es eine Zahl  \nowrap{$N_{\epsilon} \in \N $,}
  so dass \[ | a_n - a | < \epsilon \text{ f"ur alle } n \geq N_{\epsilon}. \]

Eine solche Zahl $a$ (sofern sie existiert) nennt man \notion{Limes} oder \notion{Grenzwert} der 
Folge  $(a_n)_{n\in \N}$ und schreibt
\[  \lim_{n\to\infty} a_n =a. \]
Man sagt auch: Die Folge $(a_n)_{n\in \N}$ konvergiert gegen $a$.

Eine Folge, die gegen $0$ konvergiert, nennt man \notion{Nullfolge}.

Eine Folge, die nicht konvergiert, nennt man \notion{divergent}. \\
\floatright{\href{https://www.hm-kompakt.de/video?watch=302}{\image[75]{00_Videobutton_schwarz}}}\\\\
\end{definition}

\begin{remark}
Grob gesagt bedeutet die Konvergenzbedingung, dass die Abstände der Folgeglieder zu $a$ immer kleiner
werden. Sie müssen nicht kontinuierlich kleiner werden, aber wenn man sich einen Abstand $\epsilon$ zu $a$ vorgibt,
dann haben fast alle Folgenglieder, d. h. alle bis auf endlich viele, einen kleineren Abstand zu $a$.

\begin{genericGWTVisualization}[550][1000]{mathlet1}
		\begin{variables}
			\function{f}{rational}{(2*x^2-3*x+1)/(x^2+3x-3)+sqrt(x-0.5)-(x-0.5)/sqrt(x-0.5)}  % +-sqrt(x-0.5), um auf >=0.5 einzuschränken
			\function{g}{rational}{var(f)*theta(sqrt(0.03-sin(pi*x)^2))}
%			\function{g}{rational}{var(f)}
%			\pointOnCurve{a1}{rational}{var(f)}{1}
%			\number{a}{rational}{2}
			\parametricFunction{ax}{real}{0, 2+t, 0, 2.02, 201}
			\pointOnParametricCurve[editable]{P}{real}{var(ax)}{1}		
			\number{ups}{real}{var(P)[y]-2}
%-------------------variante mit editierbarem epsilon
%			\number[editable]{ups}{real}{1}
%			\point{P}{real}{0,var(ups)}
%-----------------------------------------------------
			\function{gr}{real}{2}
			\function{l1}{real}{2+var(ups)}
			\function{l2}{real}{2-var(ups)}
			\number{b}{real}{9/(2*var(ups))-3/2}
			\number{N}{real}{floor(var(b)+sqrt(var(b)^2+3-7/var(ups) )+1.0001) }
%			\point{c0}{real}{var(N),2-var(ups)}
			\point{c0}{real}{var(N),0}
			\point{c1}{real}{var(N),2+var(ups)}
			\segment{v}{real}{var(c0), var(c1)}
			\set{s}{real}{|y-2|<var(ups) AND x>var(N)}
			\point{O}{real}{0,2}
			\segment{vups}{real}{var(O),var(P)}
		\end{variables}
		\color{P}{#CC00CC}
		\label{P}{\textcolor{#CC00CC}{$\epsilon$}}
		\color{s}{#CC00CC}
		\color{l1}{#CC00CC}
		\color{l2}{#CC00CC}
		\color{gr}{#0066CC}
		\color{v}{#CC00CC}
		\color{g}{#CC6600}
		\color{ups}{#CC00CC}
		\color{N}{#CC00CC}
		\color{vups}{#CC00CC}

		\begin{canvas}
			\plotLeft{-0.3}
			\plotRight{6.3}
			\plotSize{800,270}
			\plot[coordinateSystem]{vups, s, gr,l1,l2, v, P,g}
			%\plotRatio{keep}
		\end{canvas}
		\text{Gegeben ist die Folge \textcolor{#CC6600}{$a_n=\frac{2n^2-3n+1}{n^2+3n-3}$} (=y-Werte der orangenen Punkte), 
		welche den Grenzwert \textcolor{#0066CC}{$a=2$} hat. 
		Für \textcolor{#CC00CC}{$\epsilon= $}\var{ups} ist 
		\textcolor{#CC00CC}{$N_\epsilon=$}\var{N} eine Zahl, so dass
		$|a_n-a|<\var{ups}$ für alle $n\geq N_\epsilon$.\\
		Verschieben Sie den pinken Punkt, um $\epsilon$ zu variieren. Die Zahl $N_\epsilon$ wird automatisch angepasst.}
   	\end{genericGWTVisualization}


%-------------------alte Version mit durchgezogenem Graphen -------------------------
%------------------------------------------------------------------------------------
% \begin{genericGWTVisualization}[550][1000]{mathlet1}
% 		\begin{variables}
% 			\function{f}{rational}{(2*x^2-3*x+1)/(x^2+3x-3)+sqrt(x-1)-(x-1)/sqrt(x-1)}  % +-sqrt(x-1), um auf >=1 einzuschränken
% %			\pointOnCurve{a1}{rational}{var(f)}{1}
% %			\number{a}{rational}{2}
% 			\parametricFunction{ax}{real}{0, 2+t, 0, 3, 299}
% 			\pointOnParametricCurve[editable]{P}{real}{var(ax)}{1}		
% 			\number{ups}{real}{var(P)[y]-2}
% %-------------------variante mit editierbarem epsilon
% %			\number[editable]{ups}{real}{1}
% %			\point{P}{real}{0,var(ups)}
% %-----------------------------------------------------
% 			\function{gr}{real}{2}
% 			\function{l1}{real}{2+var(ups)}
% 			\function{l2}{real}{2-var(ups)}
% 			\number{b}{real}{9/(2*var(ups))-3/2}
% 			\number{N}{real}{floor(var(b)+sqrt(var(b)^2+1-7/var(ups) )+1) }
% 			\point{c0}{real}{var(N),2-var(ups)}
% 			\point{c1}{real}{var(N),2+var(ups)}
% 			\segment{v}{real}{var(c0), var(c1)}
% 			\set{s}{real}{|y-2|<var(ups) AND x>var(N)}
% 		\end{variables}
% 		\color{P}{LIGHT_GREEN}
% 		%\label{P}{$\textcolor{BLUE}{P}$}
% 		\color{s}{LIGHT_GREEN}
% 		\color{l1}{LIGHT_GREEN}
% 		\color{l2}{LIGHT_GREEN}
% 		\color{gr}{BLUE}
% 		\color{v}{LIGHT_GREEN}
% 
% 
% 		\begin{canvas}
% 			\plotLeft{-0.3}
% 			\plotRight{6.3}
% 			\plotSize{600,200}
% 			\plot[coordinateSystem]{s, gr,l1,l2, v, P, f}
% 		\end{canvas}
% 		\text{Gegeben ist die Folge $a_n=\frac{2n^2-3n+1}{n^2+3n-3}$ ($y$-Koordinaten der Punkte $(n,a_n)$ in obiger Kurve),
% 		welche den Grenzwert $a=2$ hat. Für $\epsilon= \var{ups}$ ist $N_\epsilon=\var{N}$ eine Zahl, so dass
% 		$|a_n-a|<\var{ups}$ für alle $n\geq N_\epsilon$.}
%    	\end{genericGWTVisualization}


\end{remark}

\begin{example}\label{ex:einfache-folgen}
\begin{enumerate}
\item Für $c\in \R$ konvergiert die konstante Folge $(a_n)_{n\in \N}$ mit $a_n=c$ für alle $n\in \N$ 
gegen die Zahl $c$. Man kann nämlich für jedes $\epsilon>0$ die Zahl $N_\epsilon=1$ wählen und erhält für alle
$n\geq N_\epsilon=1$:
\[  | a_n -c|=|c-c|=0<\epsilon, \]
wie gefordert.
\item Die Folge $(a_n)_{n\in \N}=((-1)^n)_{n\in \N}$, d.h. $a_n=(-1)^n$ für alle $n\in \N$, konvergiert nicht, ist also
divergent. Wäre nämlich $a$ ein Grenzwert, so ist $|(-1)-a|\geq 1$ oder $|1-a|\geq 1$. Das bedeutet, für
 $\epsilon>0$ klein genug, z.B. $\epsilon=\frac{1}{2}$ ist $|a_n-a|\geq 1>\epsilon$ für alle geraden $n$ oder alle
 ungeraden $n$, weshalb es für $\epsilon=\frac{1}{2}$ kein gefordertes $N_{\epsilon}$ gibt.
\item \notion{Harmonische Folge} Die Folge $(a_n)_{n\in \N}$ mit $a_n=\frac{1}{n}$ ist eine Nullfolge, d.h. $\lim_{n\to \infty} \frac{1}{n}=0$.\\ 
Zu $\epsilon>0$ kann man nämlich eine Zahl $N_\epsilon>\frac{1}{\epsilon}$ wählen, 
dann gilt für alle $n\geq N_\epsilon$:
\[ {|a_n-0|}={|\frac{1}{n}|}=\frac{1}{n}<\frac{1}{N_\epsilon}<\epsilon. \]
\item Die Folge $(a_n)_{n\geq 1}$ mit $a_1=2$ und
$a_{n+1}=\frac{a_n}{2}+\frac{1}{a_n}$ für alle $n\in \N $
aus dem letzten \ref[reelle-folgen][Abschnitt]{ex:sqrt-2} konvergiert gegen $\sqrt{2}$.\\
Die Berechnung der ersten Folgeglieder 
\begin{align*}
 a_1 &=2,\quad a_2=1,5,\quad a_3= 1,41\bar{6},\\ a_4 &\approx 1,4142157,\quad
  a_5 \approx 1,414213562375
\end{align*}
deutet schon darauf hin, da für die Abstände zu $\sqrt{2}\approx 1,414213562373$ gilt:
\begin{align*}
&|a_1-\sqrt{2}|<1, \quad |a_2-\sqrt{2}|<0,1, \quad |a_3-\sqrt{2}|<0,003,\\
&|a_4-\sqrt{2}|<0,000003, \quad |a_5-\sqrt{2}|<0,000000000004
\end{align*}
Allerdings beweisen die ersten Folgeglieder noch lange nicht, dass $\sqrt{2}$ der Grenzwert ist. Zum einen könnte
der Grenzwert knapp daneben liegen, aber vor allem könnte die Folge sich noch ganz anders entwickeln.\\
Dass $\sqrt{2}$ wirklich der Grenzwert ist, werden wir \link{monotone-konvergenz}{später} zeigen.
\item Nicht-konstante arithmetische Folgen und geometrische Folgen $(uq^n)_{n\in \Nzero}$ mit $|q|>1$ konvergieren
nicht. Dies folgt aus dem zweiten Teil des nächsten Satzes.\\
Geometrische Folgen $(uq^n)_{n\in \Nzero}$ mit $|q|<1$ konvergieren, wie wir im \link{monotone-konvergenz}{nächsten Abschnitt}
zeigen werden.
\end{enumerate}
\end{example}

Die oben aufgeführten Begriffe werden im folgenden Video noch einmal genauer anhand von Beispielen erläutert:\\\\
\floatright{\href{https://api.stream24.net/vod/getVideo.php?id=10962-2-11266&mode=iframe&speed=true}
{\image[75]{00_video_button_schwarz-blau}}}\\\\\\

\begin{remark}\label{rem:n-geq-n_0}
Für Fragen zur Konvergenz von Folgen sind die Folgenglieder "`am Anfang"' unerheblich. Die Konvergenz einer Folge ändert sich
also nicht, wenn man \emph{endlich} viele Folgenglieder abändert, oder gar die Folge erst später beginnen lässt, also statt einer
Folge $(a_n)_{n\geq 1}$ die Folge $(a_n)_{n\geq n_0}$ mit einer festen natürlichen Zahl $n_0\in \N$ betrachtet.

Insbesondere lassen sich bei den nachfolgenden Grenzwertregeln und Grenzwertkriterien Bedingungen an die Folgenglieder abschwächen von
"`für alle $n\in \N$"' zu "`für alle $n\geq n_0$"' (bei einem fest gewählten $n_0\in \N$), ohne dass die Aussagen falsch werden.
\end{remark}



\begin{theorem}\label{thm:eindeutigkeitgw}
\begin{enumerate}
\item Eine Folge besitzt höchstens einen Grenzwert. Das heißt, wenn $(a_n)_{n\in \N}$ konvergiert, ist der
Grenzwert eindeutig bestimmt.
\item Eine konvergente Folge ist beschränkt.
\end{enumerate}
\end{theorem}

\begin{proof*}[Erklärung]
\begin{enumerate}
\item Seien $a$ und $a'$ zwei Grenzwerte einer Folge $(a_n)_{n\in \N}$. Für jedes beliebige $\epsilon>0$ gibt
es dann $N_{\epsilon}, M_{\epsilon}\in \N$ so dass $|a_n-a|<\epsilon$ für alle $n\geq N_{\epsilon}$ und
$|a_n-a'|<\epsilon$ für alle $n\geq M_{\epsilon}$. Insbesondere gilt dies für alle 
$n\geq \max\{N_{\epsilon},M_{\epsilon}\}$.\\
Mit der Dreiecksungleichung erhält man nun für $n\geq \max\{N_{\epsilon},M_{\epsilon}\}$
\[  |a-a'|\leq |a-a_n|+|a_n-a'|< \epsilon+\epsilon<2\epsilon. \]
Wären nun $a$ und $a'$ verschieden, könnte man  $\epsilon=\frac{1}{2}|a-a'|>0$ wählen und erhielte dennoch
aus der letzten Zeile $|a-a'|<2\epsilon=|a-a'|$. Widerspruch.
\item Sei $(a_n)_{n\in \N}$ eine Folge, die gegen eine Zahl $a\in \R$ konvergiert. Dann gibt es nach Definition
für jedes beliebige $\epsilon>0$ ein $N_{\epsilon}\in \N$ so dass $|a_n-a|<\epsilon$ für alle $n\geq N_{\epsilon}$.\\
Insbesondere für $\epsilon=1$ gilt dann $|a_n-a|<1$ für alle $n\geq N_1$.\\
Die Wertemenge von $(a_n)_{n\in \N}$ ist daher eine Teilmenge der Menge
\[  \{ a_k | k<N_1 \} \cup [a-1,a+1]. \]
Da sowohl die endliche Menge $\{ a_k | k<N_1 \}$ als auch das Intervall $[a-1,a+1]$ beschränkt sind, ist insbesondere die
Wertemenge beschränkt.\\
\end{enumerate}
\floatright{\href{https://api.stream24.net/vod/getVideo.php?id=10962-2-11267&mode=iframe&speed=true}
{\image[75]{00_video_button_schwarz-blau}}}\\\\
\end{proof*}

\begin{remark}
Jede konvergente Folge ist nach obigem Satz beschränkt. Jedoch muss nicht jede beschränkte Folge konvergent sein, wie
das Beispiel $((-1)^n)_{n\in \N}$ zeigt.
\end{remark}



\section{Grenzwertregeln}\label{sec:grenzwertregeln}

Um Grenzwerte zu berechnen, sind einige Regeln sehr hilfreich.

\begin{rule}[Grenzwertregeln]\label{grenzwertregeln}
Seien  $(a_n)_{n\in \N}$, $(b_n)_{n\in \N}$ konvergente (reelle) Folgen mit
$\lim_{n\to\infty} a_n =a$ und $\lim_{n\to\infty} b_n =b$. Dann gilt für alle $\alpha, \beta \in \R$:
\begin{enumerate}
\item Die Folge $( \alpha a_n + \beta b_n )_{n\in \N}$ konvergiert und \[ \lim_{n\to\infty} ( \alpha a_n + \beta b_n )
= \alpha \cdot a + \beta \cdot b. \]
\item Die Folge $( a_n \cdot b_n )_{n\in \N}$ konvergiert und \[\lim_{n\to\infty} ( a_n \cdot b_n ) = a \cdot b. \]
\item Falls $b_n \neq 0$ für alle $n \in \N$ und $b \neq 0$, so konvergiert die Folge $( a_n /b_n )_{n\in \N}$ und
\[  \lim_{n\to\infty} ( \frac{a_n}{b_n} )=\frac{a}{b}.\]
\end{enumerate}
\floatright{\href{https://www.hm-kompakt.de/video?watch=305}{\image[75]{00_Videobutton_schwarz}}}\\\\
\end{rule}


\begin{proof*}[Erklärung]
Wir zeigen die erste Aussage. Die anderen beiden sind etwas komplizierter zu zeigen, folgen aber dem gleichen Prinzip.

Nach Definition der Konvergenz von $(a_n)_{n\in \N}$ und $(b_n)_{n\in \N}$ gibt es für beliebige $\epsilon_1>0$ und $\epsilon_2>0$ 
natürliche Zahlen $K_{\epsilon_1}, M_{\epsilon_2}\in \N$, so dass $|a_n-a|<\epsilon_1$ für alle $n\geq K_{\epsilon_1}$ und 
$|b_n-b|<\epsilon_2$ für alle $n\geq M_{\epsilon_2}$. Insbesondere gilt dies für alle 
$n\geq \max\{K_{\epsilon_1},M_{\epsilon_2}\}$.\\
Damit gilt für alle $n\geq \max\{K_{\epsilon_1},M_{\epsilon_2}\}$
\[ |( \alpha a_n + \beta b_n )-(\alpha  a + \beta  b)| =|\alpha (a_n-a)+\beta (b_n-b)|
\leq |\alpha|\cdot |a_n-a| + |\beta|\cdot |b_n-b| <  |\alpha|\epsilon_1 + |\beta|\epsilon_2. \]
Für beliebiges $\epsilon>0$ kann man also $\epsilon_1=\frac{\epsilon}{2|\alpha|}>0$ (bzw. $\epsilon_1$ beliebig, falls $\alpha=0$)
und $\epsilon_2=\frac{\epsilon}{2|\beta|}>0$ (bzw. $\epsilon_2$ beliebig, falls $\beta=0$) wählen, sowie $N_\epsilon=
\max\{K_{\epsilon_1},M_{\epsilon_2}\}$, und erhält
für alle $n\geq N_\epsilon$
\[ |( \alpha a_n + \beta b_n )-(\alpha  a + \beta  b)| 
<  |\alpha|\epsilon_1 + |\beta|\epsilon_2\leq \frac{\epsilon}{2}+\frac{\epsilon}{2}=\epsilon.\]
\floatright{\href{https://api.stream24.net/vod/getVideo.php?id=10962-2-10800&mode=iframe&speed=true}
{\image[75]{00_video_button_schwarz-blau}}}\\\\
\end{proof*}

\begin{quickcheckcontainer}
\randomquickcheckpool{1}{1}
\begin{quickcheck}
		\field{real}
		\type{input.number}
		\begin{variables}
			\randint{a}{2}{6}
			\randint{b}{2}{8}
			\randint{c}{2}{6}
			\randint{d}{3}{6}
		    \function{ff}{a/n+b)}
            \function{gg}{c+(1/d)^n}
            \function{loes}{b/c}
		\end{variables}


         \lang{de}{\text{Bestimmen Sie den Grenzwert $a$ der konvergenten Folge
         $(a_n)_{n\in\N}$ definiert durch $a_n:=
         \frac{\frac{\var{a}}{n}+\var{b}}{\var{c}+(\frac{1}{\var{d}})^n}$\\\\
         
         Der Grenzwert von $(a_n)_{n\in\N}$ lautet: $a=$\ansref}}
			

		\begin{answer}
        	\solution{loes}
	    \end{answer}
		\explanation{Betrachten Sie die Grenzwerte der einzelnen konvergenten Folgen $a_{n,1}=
        \frac{\var{a}}{n}$, $a_{n,2}=a_{n,1}+\var{b}$, $a_{n,3}=(\frac{1}{\var{d}})^n$ 
        sowie $a_{n,4}=\var{c}+a_{n,3}$ und verwenden Sie dann die Grenzwertregeln. } 
	\end{quickcheck}
\end{quickcheckcontainer}

\begin{remark}
Für den Quotienten kann man die Voraussetzung gemäß obiger \lref{rem:n-geq-n_0}{Bemerkung} auch abschwächen:\\
Sind für ein festes $n_0\in \N$ die Folgenglieder $b_n\neq 0$ für alle $n\geq n_0$, dann ist die
Folge $( a_n /b_n )_{n\geq n_0}$ konvergent und
\[  \lim_{n\to\infty} ( \frac{a_n}{b_n} )=\frac{a}{b}.\]
\end{remark}


\begin{example}
\begin{enumerate}
\item Der Grenzwert der Folge $(a_n)_{n\in\N}$ definiert durch $a_n:=\frac{1}{n^2}$
lässt sich mit Hilfe der \ref[content_14_konvergenz][Grenzwertregeln]{grenzwertregeln} bestimmen:
\[ \lim_{n\to \infty} \frac{1}{n^2}= (\lim_{n\to \infty} \frac{1}{n})\cdot (\lim_{n\to \infty} \frac{1}{n})=0\cdot 0=0 \]
Allgemeiner gilt sogar für alle $k\in\N$:
\[\lim_{n\to \infty} \frac{1}{n^k}=(\lim_{n\to \infty} \frac{1}{n})^k =0^k=0\]
\item Für die Folge $(a_n)_{n\in \N}$ mit $a_n=\frac{2n^2-3n+1}{n^2+3n-3}$ lassen sich die Grenzwertregeln nicht direkt anwenden,
da die Folgen $(n^2)_{n\in \N}$ und $(n)_{n\in \N}$ nicht konvergieren. Wenn wir den Bruch jedoch mit der höchsten Potenz
$n^2$ kürzen, erhalten wir 
\[ a_n=\frac{2n^2-3n+1}{n^2+3n-3}=\frac{2-3\cdot \frac{1}{n}+\frac{1}{n^2}}{1+3\cdot \frac{1}{n}-3\cdot \frac{1}{n^2}}.\]
Für den Zähler gilt
\[ \lim_{n\to \infty} \left(2-3\cdot \frac{1}{n}+\frac{1}{n^2}\right)= \lim_{n\to \infty} 2 -3\cdot\lim_{n\to \infty}\frac{1}{n}+
 \lim_{n\to \infty} \frac{1}{n^2} =2-3\cdot 0+0 =2, \]
und für den Nenner
\[ \lim_{n\to \infty} \left(1+3\cdot \frac{1}{n}-3\cdot \frac{1}{n^2} \right)=1+3\cdot 0 -3\cdot 0=1.\]
Man kann also die Regel für Quotienten anwenden und bekommt:
\[ \lim_{n\to \infty} a_n =\lim_{n\to \infty} \frac{2-3\cdot \frac{1}{n}+\frac{1}{n^2}}{1+3\cdot \frac{1}{n}-3\cdot \frac{1}{n^2}}
= \frac{2}{1}=2. \]
\item Um den Grenzwert für die Folge $(a_n)_{n\in \N}$ mit $a_n=\frac{2n^2-3n+1}{n^3+2n^2-4}$ zu berechnen, geht man wie im vorigen
Beispiel vor und kürzt den Bruch zunächst mit der höchsten Potenz des Nenners, hier also $n^3$:
\[ \lim_{n\to \infty} a_n =\lim_{n\to \infty} \frac{2n^2-3n+1}{n^3+2n^2-4}
=\lim_{n\to \infty}\frac{2\cdot \frac{1}{n} -3\cdot \frac{1}{n^2}+\frac{1}{n^3}}{1+2\cdot \frac{1}{n}-4\cdot \frac{1}{n^3}}
=\frac{2\cdot 0-3\cdot 0+0}{1+2\cdot 0-4\cdot 0}=\frac{0}{1}=0.\]
Die Folge ist also eine Nullfolge.
\item Folgen von obigem Typ, bei denen die höchste Potenz im Zähler größer als die höchste Potenz im Nenner ist, konvergieren nicht,
wie wir am Beispiel $a_n=\frac{2n^4-3n+1}{n^3+2n^2-4}$ erläutern:\\
Zunächst kürzt man wieder mit der höchsten Potenz des Nenners, hier $n^3$ und erhält
$a_n=\frac{2\cdot n -3\cdot \frac{1}{n^2}+\frac{1}{n^3}}{1+2\cdot \frac{1}{n}-4\cdot \frac{1}{n^3}}$.
Die Folge der Nenner $b_n=1+2\cdot \frac{1}{n}-4\cdot \frac{1}{n^3}$ ist nun konvergent gegen $1$.\\
Wäre die Folge $(a_n)_{n\in \N}$ konvergent, so wäre auch die Folge $(a_nb_n)_{n\in \N}$ konvergent, aber wegen
$a_nb_n=2\cdot n -3\cdot \frac{1}{n^2}+\frac{1}{n^3}>2\cdot n -3$ ist dies eine unbeschränkte Folge im Widerspruch zur Konvergenz.
\end{enumerate}
\end{example}

Ähnliche Beispiele werden im folgenden Video veranschaulicht und damit die Grenzwertregeln erläutert:\\\\
\floatright{\href{https://api.stream24.net/vod/getVideo.php?id=10962-2-10893&mode=iframe&speed=true}
{\image[75]{00_video_button_schwarz-blau}}}\\\\\\

Auch für das Wurzelziehen gibt es eine Grenzwertregel. Den Beweis dazu werden wir jedoch erst indirekt durch die \ref[elem-funk][Stetigkeit 
der Wurzelfunktion]{sec:wurzel-und-ln} bekommen.

\begin{rule}
Sei $(a_n)_{n\in \N}$ eine konvergente Folge mit  $a_n\geq 0$ für alle $n\in \N$
und mit Grenzwert
$a=\lim_{n\to \infty}  a_n$. Des Weiteren sei $k\in \N$ eine natürliche Zahl.
Dann ist die Folge der $k$-ten Wurzeln $(\sqrt[k]{a_n})_{n\in \N}$ konvergent und es gilt
\[ \lim_{n\to \infty} \sqrt[k]{a_n} = \sqrt[k]{a}. \]
\end{rule}

An dieser Stelle sollen noch Regeln und Anmerkungen zu Grenzwerten, Konvergenz und bestimmter Divergenz bei Folgen gegeben werden, die u. a. die Laufvariable im Exponenten haben.
Insbesondere wird das Konvergenzverhalten von "exponentiellen" Folgen im Vergleich zu "polynomialen" Folgen betrachtet.

\begin{theorem}\label{thm:exp-schlaegt-pol}
Sei $n\in\mathbb{N}$ sowie $a, q\in\mathbb{R}$. Dann gilt für $|q|<1$:
\[\lim_{n\to\infty} n^a\cdot q^n=0\]
\floatright{\href{https://www.hm-kompakt.de/video?watch=310}{\image[75]{00_Videobutton_schwarz}}}\\\\
\end{theorem}
\begin{proof*}[Beweis des  \ref[content_14_konvergenz][Theorems]{{thm:exp-schlaegt-pol}}]
\begin{incremental}
\step
Wir schreiben $b_n= n^a\cdot q^n$ und zeigen: Es gibt eine Konstante $c$ mit $| q| <c<1$ und ein $N\in\N$   so,
dass für alle $n\geq N$ gilt
\[| b_{n+1}|<c |b_n|.\]
Dann folgt nämlich 
\[\lim_{n\to\infty}\abs{b_n} <\lim_{n\to\infty}c^{n-N}\cdot\abs{b_N}=0,\]
und somit auch $\lim_{n\to\infty} n^a\cdot q^n=0$.
\step
Die Bedingung $| b_{n+1}|<c| b_n|$ ist gleichbedeutend mit $(n+1)^a | q|^{n+1}<c\cdot n^a | q|^n$ bzw. mit
\[ (1+\frac{1}{n})^a{| q|}<c.\]
Für $a\leq 0$ ist $(1+\frac{1}{n})^a\leq 1$, und somit die Bedingung erfüllt für jedes $| q|<c<1$.

Für $a>0$ ist $(1+\frac{1}{n})^a$ eine monoton fallende Folge mit Grenzwert $1$. Nun wählen wir $\vert q\vert <\tilde{c}<1$ und dazu ein $\epsilon>0$ so klein,
dass immer noch $c=(1+\epsilon)\tilde{c}<1$ erfüllt ist. Dann wählen wir $N\in\N$ so groß, dass $(1+\frac{1}{N})^a<1+\epsilon$. 
Dann gilt für alle $n\geq N$
\[(1+\frac{1}{n})^a \abs{q}<c,\]
und somit ist wieder obige Bedingung erfüllt.
\end{incremental}
\end{proof*}

\begin{remark}\label{rem:expstrong}
Die Eigenschaft $n^a \cdot q^n \rightarrow 0$ bei $|q|<1$ passt zur \textbf{Merkregel}:\\
\[\text{Exponentiell ist stärker als polynomial!}\]
\end{remark}

\begin{example}
\begin{enumerate}
\item Die Folge $(a_n)_{n\in\mathbb{N}}=n^7\cdot 0,9^n$ konvergiert und ist eine Nullfolge. 
\item Die Folge $(b_n)_{n\in\N}=(n+2)^8\cdot (\frac{3}{7})^n+c$ konvegiert für $c\in\R$ gegen $c$, da $((n+2)^8\cdot (\frac{3}{7})^n)_{n\in\N}$ eine Nullfolge ist. 
\end{enumerate}
\end{example}


\begin{remark}
Die angegebenen Grenzwertregeln helfen also Grenzwerte zu berechnen. Weitere Möglichkeiten werden im Abschnitt 
\link{konv-krit}{Konvergenzkriterien} behandelt. 
\end{remark}

Die folgende Regel hilft manchmal Grenzwerte abzuschätzen.

\begin{rule}\label{rule:grenzwerte-vergleichen}
Seien $(a_n)_{n\in \N}$ und $(b_n)_{n\in \N}$ konvergente Folgen mit $a_n\leq b_n$ für alle $n\in \N$. Dann gilt
für die Grenzwerte
\[ \lim_{n\to \infty}  a_n \leq \lim_{n\to \infty} b_n.\]
\end{rule}

\begin{proof*}[Erklärung]
Sei $a=\lim_{n\to \infty}  a_n $ und $b=\lim_{n\to \infty} b_n$. Für beliebiges $\epsilon>0$ gibt es dann eine 
natürliche Zahl $N_\epsilon$ mit
\[  |a_n-a|<\epsilon \quad \text{und} \quad |b_n-b|<\epsilon \]
für alle $n\geq N_\epsilon$. Damit folgt für alle $n\geq N_\epsilon$
\[  a-\epsilon < a_n \leq b_n < b+\epsilon. \]
Also gilt $a< b+2\epsilon$ für beliebiges $\epsilon>0$, weshalb $a\leq b$.
\end{block}

\begin{block}[warning]
Zu beachten ist, dass die stärkere Voraussetzung $a_n<b_n$ für alle $n\in \N$ \emph{nicht} die strikte
Ungleichung $\lim_{n\to \infty}  a_n < \lim_{n\to \infty} b_n$ impliziert.

Dies ist am Beispiel $(a_n)_{n\in \N}=(-\frac{1}{n})_{n\in \N}$ und $(b_n)_{n\in \N}=(\frac{1}{n})_{n\in \N}$
gut zu sehen, wo $a_n<0<b_n$ gilt für alle $n\in \N$, aber
\[ \lim_{n\to \infty}  a_n =0 = \lim_{n\to \infty}  b_n. \]
\end{proof*}

Zuletzt bemerken wir noch folgende Tatsache, die direkt aus den Definitionen folgt.

\begin{rule}\label{rule:folge--limes-nullfolge}
Eine Folge $(a_n)_{n\in \N}$ konvergiert genau dann gegen eine Zahl $a\in \R$, wenn die Folge
$(a_n-a)_{n\in \N}$ eine Nullfolge ist, und dies ist genau dann der Fall, wenn $\big({|a_n-a|}\big)_{n\in \N}$ eine Nullfolge ist.
\end{rule}


\end{visualizationwrapper}

\end{content}