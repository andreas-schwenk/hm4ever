\documentclass{mumie.element.exercise}
%$Id$
\begin{metainfo}
  \name{
    \lang{de}{Ü06: Konvergenzkriterien}
    \lang{en}{}
  }
  \begin{description} 
 This work is licensed under the Creative Commons License Attribution 4.0 International (CC-BY 4.0)   
 https://creativecommons.org/licenses/by/4.0/legalcode 

    \lang{de}{Hier die Beschreibung}
    \lang{en}{}
  \end{description}
  \begin{components}
  \end{components}
  \begin{links}
\link{generic_article}{content/rwth/HM1/T205_Konvergenz_von_Folgen/g_art_content_16_konvergenzkriterien.meta.xml}{content_16_konvergenzkriterien}
\end{links}
  \creategeneric
\end{metainfo}
\begin{content}
\title{
\lang{de}{Ü06: Konvergenzkriterien}
}
\begin{block}[annotation]
  Im Ticket-System: \href{http://team.mumie.net/issues/9857}{Ticket 9857}
\end{block}
 
\lang{de}{Es sei $k\in\N$ und gegeben sei die Folge $(a_{n})_{n\in\N}$ mit $a_{n}=\frac{1}{(n+k)^{k}}$. 
Untersuchen Sie die Folge auf Konvergenz und bestimmen Sie gegebenenfalls den Grenzwert.}

\begin{tabs*}[\initialtab{0}\class{exercise}]

  \tab{
  \lang{de}{Lösung }}
  
  \begin{incremental}[\initialsteps{1}]
    \step 
    \lang{de}{Wir verwenden das \ref[content_16_konvergenzkriterien][Sandwich-Lemma]{thm:sandwich}. 
    Zunächst gilt für jedes $n\in\N$ die Abschätzung
	\[0\leq a_{n}\,.\]}
     
    \step \lang{de}{Weiter gilt
	\[\frac{1}{a_{n}}=(n+k)^{k}\geq  n+k\geq n\]
	 für jedes $n\in\N$. Durch den Übergang zum Kehrwert erhalten wir so die Abschätzung
	\[\forall n\in\N\,:\,0\leq a_{n}\leq \frac{1}{n}\,.\]
	Die konstante Folge $(0)_{n\in\N}$ konvergiert gegen 0 und die Folge 
	$(x_{m})_{m\in\N}$ mit $x_{m}=\frac{1}{m}$ konvergiert ebenfalls gegen 0. 
	Daher existiert auch $\lim_{n\to\infty}{a_{n}}=0$. }
  \end{incremental}

  
\end{tabs*}


\end{content}