\documentclass{mumie.element.exercise}
%$Id$
\begin{metainfo}
  \name{
   \lang{de}{Ü10: Grenzwertberechnung}
   \lang{en}{Exercise 10}
 }
 \begin{description} 
 This work is licensed under the Creative Commons License Attribution 4.0 International (CC-BY 4.0)   
 https://creativecommons.org/licenses/by/4.0/legalcode 

   \lang{de}{Hier die Beschreibung}
   \lang{en}{Description here}
 \end{description}
 \begin{components}
 \end{components}
 \begin{links}
\link{generic_article}{content/rwth/HM1/T205_Konvergenz_von_Folgen/g_art_content_14_konvergenz.meta.xml}{content_14_konvergenz}
\end{links}
 \creategeneric
 \end{metainfo}
 \begin{content}
\begin{block}[annotation]
	Im Ticket-System: \href{https://team.mumie.net/issues/15907}{Ticket 15907}
\end{block}
   \title{
    \lang{de}{Übung 10}
    \lang{en}{Exercise 10}
   }
  
  \begin{enumerate}
  
\item Bestimmen Sie den Grenzwert der Folge $(a_n)_{n \in \N}$ mit $a_n = \sqrt{n^2+4n}-\sqrt{n^2+1}$.
\item Bestimmen Sie die Grenzwerte von
\begin{enumerate}
\item[a)] $(\sqrt{n+1}-\sqrt{n-1})_{n\in\mathbb{N}}$
\item[b)] $(\sqrt{n^2+n}-n)_{n\in\mathbb{N}}$
\end{enumerate}

\end{enumerate}

 \begin{tabs*}[\initialtab{0}\class{exercise}]
 
 \tab{
   \lang{de}{Antworten}
 }
 
 1.\\
 Der Grenzwert lautet $a=2$.\\\\
 
 2.\\
 a) Der Grenzwert lautet $a=0$.\\
 b) Der Grenzwert lautet $a=\frac{1}{2}$.
 
 \tab{
   \lang{de}{Lösung 1}
   \lang{en}{Solution}
} \begin{incremental}[\initialsteps{1}]
    \step 
     \lang{de}{Wir erweitern den Term so, dass wir die dritte binomische Formel anwenden 
     können. Im Zähler fallen dadurch die Wurzelterme weg. Danach werden wir die
    \ref[content_14_konvergenz][Grenzwertsätze]{sec:grenzwertregeln} aus der Vorlesung anwenden, um den Grenzwert zu berechnen.}
       \step \begin{align*} a_n &= \sqrt{n^2+4n}-\sqrt{n^2+1} \\
       &= \frac{\textcolor{#CC6600}{(\sqrt{n^2+4n}-\sqrt{n^2+1})(\sqrt{n^2+4n}+\sqrt{n^2+1})}}{\sqrt{n^2+4n}+\sqrt{n^2+1}}\\
       &= \frac{\textcolor{#CC6600}{(n^2+4n)-(n^2+1)}}{\sqrt{n^2(1+\frac{4}{n})}+\sqrt{n^2(1+\frac{1}{n^2})}} \\
       &= \frac{4n-1}{\sqrt{n^2}(\sqrt{1+\frac{4}{n}}+\sqrt{1+\frac{1}{n^2}})} \\
       &= \frac{n (4-\frac{1}{n})}{n(\sqrt{1+\frac{4}{n}}+\sqrt{1+\frac{1}{n^2}})} \\
       &= \frac{4-\frac{1}{n}}{\sqrt{1+\frac{4}{n}}+\sqrt{1+\frac{1}{n^2}}}.
       \end{align*}
       \step Nun nutzen wir die Grenzwertsätze. Es gilt $\frac{1}{n} \to 0$ für $n \to \infty$ und beispielsweise
       $\lim_{n \to \infty} \sqrt{1+\frac{4}{n}} = \sqrt{\lim_{n \to \infty} (1+\frac{4}{n})} = \sqrt{1}=1.$
       Damit erhalten wir
       \[
       \lim_{n \to \infty} a_n = \frac{4-0}{\sqrt{1+0}+\sqrt{1+0}} = \frac{4}{1+1} = 2.
       \]
       Die Folge konvergiert also und ihr Grenzwert ist $2$.
 \end{incremental}
 
 
     \tab{\lang{de}{Lösungsvideo 2 a) + b)}}	
    \youtubevideo[500][300]{GbEJdq0JV3g}\\
 
 \end{tabs*}
 \end{content}
