\documentclass{mumie.element.exercise}
%$Id$
\begin{metainfo}
  \name{
    \lang{de}{Ü05: Folgengrenzwerte}
    \lang{en}{}
  }
  \begin{description} 
 This work is licensed under the Creative Commons License Attribution 4.0 International (CC-BY 4.0)   
 https://creativecommons.org/licenses/by/4.0/legalcode 

    \lang{de}{Hier die Beschreibung}
    \lang{en}{}
  \end{description}
  \begin{components}
  \end{components}
  \begin{links}
\link{generic_article}{content/rwth/HM1/T206_Folgen_II/g_art_content_19_bestimmte_divergenz.meta.xml}{content_19_bestimmte_divergenz}
\link{generic_article}{content/rwth/HM1/T205_Konvergenz_von_Folgen/g_art_content_14_konvergenz.meta.xml}{content_14_konvergenz}
\end{links}
  \creategeneric
\end{metainfo}
\begin{content}
\title{
\lang{de}{Ü05: Folgengrenzwerte}
}
\begin{block}[annotation]
  Im Ticket-System: \href{http://team.mumie.net/issues/9856}{Ticket 9856}
\end{block}
 
% Erweitere BA4 durch Teil d und e: Merkregel exp>polyn (Satz und Bemerkung/Merkregel in VL ergänzen)

1. Untersuchen Sie die folgenden Folgen auf Konvergenz und bestimmen Sie gegebenenfalls den Grenzwert.
 \begin{table}[\class{items}]
       \nowrap{a) $a_{n}=\frac{n^{2}+3}{(n+2)^{2}}$} 
     & \nowrap{b) $b_{n}=\frac{n+\frac{1}{n}}{n^{3}+3n+2}$} 
     & \nowrap{c) $c_{n}=\frac{n^{3}+n^{2}+2n+1}{n^{2}+4n+5}$} \\
       \nowrap{d) $d_{n}=\frac{n^{3}+2^nn}{3^n}$}
     & \nowrap{e) $e_{n}=\frac{1-2^n}{n^4+2}$}
     &\\
    \end{table}
    
      
\begin{tabs*}[\initialtab{0}\class{exercise}]
  \tab{
  \lang{de}{Antworten}
  \lang{en}{Answers}
  }


\begin{table}[\class{items}]
1.\\
a) $a_n$ konvergiert gegen den Grenzwert $1$.\\
b) $b_n$ konvergiert gegen den Grenzwert $0$.\\
c) $c_n$ divergiert.\\
d) $d_n$ konvergiert gegen den Grenzwert $0$.\\
e) $e_n$ divergiert.

\end{table}
  \tab{
  \lang{de}{Lösung 1 a)}}
  
  \begin{incremental}[\initialsteps{1}]
    \step 
    \lang{de}{Durch Ausklammern der Potenz $n^{2}$ in Zähler und Nenner und anschließendes Kürzen erhalten wir für jedes $n\in\N$
	\[a_{n}=\frac{n^{2}+3}{n^{2}+4n+4}=\frac{n^{2}}{n^{2}}\cdot \frac{1+\frac{3}{n^2}}{1+\frac{4}{n}+\frac{4}{n^{2}}}=\frac{1+\frac{3}{n^2}}{1+\frac{4}{n}+\frac{4}{n^{2}}}\,.\] }
     
    \step \lang{de}{Nach den Grenzwertsätzen ist diese Folge also konvergent und der Grenzwert ergibt sich als Quotient der Grenzwerte der Terme in Zähler und Nenner. Wir erhalten also 
	\[\lim_{n\to\infty}{a_{n}}=\frac{\lim_{n\to\infty}{(1+\frac{3}{n^2}})}{\lim_{n\to\infty}{(1+\frac{4}{n}+\frac{4}{n^{2}}})}=\frac{1}{1}=1\,.\]}
  \end{incremental}

  \tab{
  \lang{de}{Lösung 1 b)}
  }
  \begin{incremental}[\initialsteps{1}]
     \step \lang{de}{Durch Ausklammern der Potenz $n^{3}$ in Zähler und Nenner und anschließendes Kürzen erhalten wir für jedes $n\in\N$
	\[b_{n}=\frac{n^{3}}{n^{3}}\cdot \frac{\frac{1}{n^{2}}+\frac{1}{n^{4}}}{1+\frac{3}{n^{2}}+\frac{2}{n^{3}}}=\frac{\frac{1}{n^{2}}+\frac{1}{n^{4}}}{1+\frac{3}{n^{2}}+\frac{2}{n^{3}}}\,.\] }
     \step \lang{de}{Nach den Grenzwertsätzen ist diese Folge also konvergent und der Grenzwert ergibt sich als Quotient der Grenzwerte der Terme in Zähler und Nenner. Wir erhalten also 
	\[\lim_{n\to\infty}{b_{n}}=\frac{\lim_{n\to\infty}{(\frac{1}{n^{2}}+\frac{1}{n^{3}})}}{\lim_{n\to\infty}{(1+\frac{3}{n^{2}}+\frac{2}{n^{3}})}}=\frac{0}{1}=0\,.\]}
    
  \end{incremental}

  \tab{
  \lang{de}{Lösung 1 c)}
  }
  
   \lang{de}{Die höchste Potenz von $n$ im Zählerterm ist $n^{3}$, während die höchste im Nennerterm auftretende Potenz $n^{2}$ ist. 
   Also ist die Folge divergent.}
    
  \tab{
  \lang{de}{Lösung 1 d)}
  }
   \lang{de}{Wir spalten den Bruch auf, denn es gilt: $\lim_{n\to\infty} q^n=0$ sowie mit 
   \ref[content_14_konvergenz][Merkregel]{rem:expstrong} $\lim_{n\to\infty} n^a q^n=0$ für $|q|<1$.
   \[\frac{n^2+2^nn}{3^n}=n^2\cdot(\frac{1}{3})^n+(\frac{2}{3})^n\cdot n\xrightarrow{n\to\infty} 0 + 0 =0\]
   }
    
  
   \tab{
  \lang{de}{Lösung 1 e)}
  }
    \begin{incremental}[\initialsteps{1}]
   \step\lang{de}{Der Zähler strebt gegen $-\infty$, das polynomiale Wachstum des Nenners kann das nicht bremsen, so dass die Folge $e_n$ divergiert.
   }
   \end{incremental}
    
\end{tabs*}
    

2. Geben Sie den Grenzwert der folgenden Folgen in $\mathbb{R}\subset\{\pm\infty\}$ an.\\

 \begin{table}[\class{items}]
       \nowrap{a) $\big{(}\frac{2n-1}{4n+3}\big{)}_{n\in\mathbb{N}}$} 
     & \nowrap{b) $\big{(}\frac{3n}{n^2-3}\big{)}_{n\in\mathbb{N}}$} 
     & \nowrap{c) $\big{(}\frac{2n^2+3}{2-n^2}\big{)}_{n\in\mathbb{N}}$} \\
       \nowrap{d) $\big{(}\frac{n^2}{n+1}\big{)}_{n\in\mathbb{N}}$}
     & \nowrap{e) $\big{(}\frac{(n+2)^2}{2n^2+1}\big{)}_{n\in\mathbb{N}}$}
     & \nowrap{f) $\big{(}\frac{n^3-3n^2+1}{1-n^2}\big{)}_{n\in\mathbb{N}}$} \\
       \nowrap{g) $\big{(}\frac{4n+2}{(3n-1)^2}\big{)}_{n\in\mathbb{N}}$}
     & \nowrap{h) $\big{(}\frac{3n^2}{(2n+1)^2}\big{)}_{n\in\mathbb{N}}$}
     & \nowrap{i) $\big{(}\frac{n(4n-1)^2}{(2n+1)^3}\big{)}_{n\in\mathbb{N}}$} \\
    \end{table}
    
    

3. Geben Sie den Grenzwert der folgenden Folgen in $\mathbb{R}\subset\{\pm\infty\}$ an.\\

 \begin{table}[\class{items}]
       \nowrap{a) $\big{(}\frac{n^4}{2^n}\big{)}_{n\in\mathbb{N}}$} 
     & \nowrap{b) $\big{(}\frac{3^n}{n^3}\big{)}_{n\in\mathbb{N}}$} 
     & \nowrap{c) $\big{(}n^4\cdot (\frac{1}{3})^n\big{)}_{n\in\mathbb{N}}$} \\
       \nowrap{d) $\big{(}\frac{1}{\sqrt{n}+1}\big{)}_{n\in\mathbb{N}}$}
     & \nowrap{e) $\big{(}\frac{\sqrt{n}}{\sqrt[3]{n}+1}\big{)}_{n\in\mathbb{N}}$}
     & \nowrap{f) $\big{(}\frac{n^2}{\sqrt{2n^4+n}}\big{)}_{n\in\mathbb{N}}$} \\
       \nowrap{g) $\big{(}\sqrt[3]{n}\big{)}_{n\in\mathbb{N}}$}
     & \nowrap{h) $\big{(}\frac{n^2+n\cdot 2^n}{3^n}\big{)}_{n\in\mathbb{N}}$}
     & \nowrap{i) $\big{(}\frac{1-2^n}{n^3+1}\big{)}_{n\in\mathbb{N}}$} \\
    \end{table}

\textit{Hinweis: Einige der angegebenen Folgen divergieren bestimmt gegen unendlich, was erst \link{content_19_bestimmte_divergenz}{später} 
thematisiert wird.}\\\\
    
\begin{tabs*}[\initialtab{0}\class{exercise}]
  \tab{
  \lang{de}{Antworten}
  \lang{en}{Answers}
  }


\begin{table}[\class{items}]

2.\\
a) Folge konvergiert gegen $\frac{1}{2}$.\\
b) Folge konvergiert gegen $0$.\\
c) Folge konvergiert gegen $-2$.\\
d) Folge divergiert.\\
e) Folge konvergiert gegen $\frac{1}{2}$.\\
f) Folge divergiert.\\
g) Folge konvergiert gegen $0$.\\
h) Folge konvergiert gegen $\frac{3}{4}$.\\
i) Folge konvergiert gegen $2$.\\\\

3.\\
a) Folge konvergiert gegen $0$.\\
b) Folge divergiert.\\
c) Folge konvergiert gegen $0$.\\
d) Folge konvergiert gegen $0$.\\
e) Folge divergiert.\\
f) Folge konvergiert gegen $\frac{1}{\sqrt{2}}$.\\
g) Folge divergiert.\\
h) Folge konvergiert gegen $0$.\\
i) Folge divergiert.\\\\

\end{table}
 
    \tab{\lang{de}{Lösungsvideo 2 a) - i)}}	
    \youtubevideo[500][300]{f8KXBEqj1pg}
    
      \tab{\lang{de}{Lösungsvideo 3 a) - i)}}	

    \youtubevideo[500][300]{Om-E0R3AV2M}
    
\end{tabs*}


\end{content}