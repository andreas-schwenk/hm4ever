\documentclass{mumie.element.exercise}
%$Id$
\begin{metainfo}
  \name{
    \lang{de}{Ü07: Grenzwertberechnung}
    \lang{en}{}
  }
  \begin{description} 
 This work is licensed under the Creative Commons License Attribution 4.0 International (CC-BY 4.0)   
 https://creativecommons.org/licenses/by/4.0/legalcode 

    \lang{de}{Hier die Beschreibung}
    \lang{en}{}
  \end{description}
  \begin{components}
  \end{components}
  \begin{links}
\link{generic_article}{content/rwth/HM1/T205_Konvergenz_von_Folgen/g_art_content_14_konvergenz.meta.xml}{content_14_konvergenz}
\end{links}
  \creategeneric
\end{metainfo}
\begin{content}
\title{
\lang{de}{Ü07: Grenzwertberechnung}
}
\begin{block}[annotation]
  Im Ticket-System: \href{http://team.mumie.net/issues/9858}{Ticket 9858}
\end{block}
 
\begin{enumerate}
\item Betrachten Sie noch einmal die Folge $(a_{n})_{n\in\N}$ mit $a_{1}=0$ und $a_{n+1}=\frac{1}{2-a_{n}}$ für alle $n\in\N$.  
Welche Gleichung erfüllt ein möglicher Grenzwert? 
Zeigen Sie, dass die Folge konvergiert und bestimmen Sie den Grenzwert.

\item 
\begin{enumerate}
\item[a)] Die Folge $(a_n)_{n\in\mathbb{N}}$ erfülle $a_{n+1}=\frac{1}{2}a_n+1$.\\
Welchen Grenzwert hat die Folge, falls sie konvergiert?
\item[b)] Die Folge $(a_n)_{n\in\mathbb{N}}$ mit\\
$a_1=1,~~~~~~a_{n+1}=\frac{a_n}{2}+\frac{c}{2a_n}$\\
mit einem Parameter $c\in\mathbb{R}^{>0}$ konvergiert. Welchen Grenzwert hat die Folge?
\end{enumerate}

\end{enumerate}

\begin{tabs*}[\initialtab{0}\class{exercise}]

   \tab{
   \lang{de}{Antworten}}
   
   1.\\
   Die Gleichung lautet: $a=\frac{1}{2-a}$.\\
   Der Grenzwert ist $a=1$.\\\\
   
   2.\\
   a) Der Grenzwert ist $a=2$.\\
   b) Der Grenzwert ist $a=\sqrt{c}$.

  \tab{
  \lang{de}{Lösung 1}}
  
  \begin{incremental}[\initialsteps{1}]
    \step 
    \lang{de}{Wir haben in einer vorherigen Aufgabe bereits gesehen, dass die Folge beschränkt und alle Folgeglieder im Intervall $[0,1]$ liegen. 
    Weiter haben wir verifiziert, dass die Folge streng monoton steigend ist. 
    Das Monotoniekriterium für Folgen garantiert uns daher die Konvergenz der Folge.}
     
    \step \lang{de}{Da mit $(a_{n})_{n\geq 1}$ auch jede Teilfolge gegen denselben Grenzwert konvergiert, 
    erhalten wir für den Grenzwert $a$ die bestimmende Gleichung 
	\[a=\lim_{n\to\infty}{a_{n}}=\lim_{n\to\infty}{a_{n+1}}=\lim_{n\to\infty}{\frac{1}{2-a_{n}}}=\frac{1}{2-a}\,.\]}
    \step \lang{de}{Beachte, dass hier nach Vorlesung  $a\in [0,1]$ gilt.
	Dies ist äquivalent zu der Gleichung
	\[(2-a)a=1\iff 2a-a^{2}-1=0\iff (a-1)^{2}=a^{2}-2a+1=0\,.\]
	Also folgt $a=1$.}
  \end{incremental}

 
    \tab{\lang{de}{Lösungsvideo 2 a) + b)}}	
    \youtubevideo[500][300]{PFoQPbucukc}\\

\end{tabs*}


\end{content}