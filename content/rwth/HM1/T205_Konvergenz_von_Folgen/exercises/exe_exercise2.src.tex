\documentclass{mumie.element.exercise}
%$Id$
\begin{metainfo}
  \name{
    \lang{de}{Ü02: Folgeneigenschaften}
    \lang{en}{}
  }
  \begin{description} 
 This work is licensed under the Creative Commons License Attribution 4.0 International (CC-BY 4.0)   
 https://creativecommons.org/licenses/by/4.0/legalcode 

    \lang{de}{Hier die Beschreibung}
    \lang{en}{}
  \end{description}
  \begin{components}
  \end{components}
  \begin{links}
  \end{links}
  \creategeneric
\end{metainfo}
\begin{content}
\title{
\lang{de}{Ü02: Folgeneigenschaften}
}
\begin{block}[annotation]
  Im Ticket-System: \href{http://team.mumie.net/issues/9854}{Ticket 9854}
\end{block}
 
 
 \begin{enumerate}
% Video
  \item \lang{de}{Mit einer Konstanten $c$ wird die Folge $(a_n)_{n\in\N}$ rekursiv definiert durch\\\\
  $a_1=0$~~~~und~~~~$a_{n+1}=a_n^2+c$ ~~~~für $n\geq 1$.
  }
    \begin{enumerate}
        \item[a)] Berechnen Sie für verschiedene Werte von $c$, z. B. für $c=\pm 0.5$, $c=-1$, $c=-1.5$, $c=-2$, $c=-2.5$, die ersten Folgenglieder.
        \item[b)] Zeigen Sie:\\
        Für $c>\frac{1}{4}$ ist die Folge $(a_n)_{n\in\N}$ streng monoton wachsend.\\
        Tipp: Es kann helfen, sich zu überlegen, dass die Funktion $f(x)=x^2-x+c$ für die betrachteten Werte $c$ immer positiv ist.
        \item[c)] Zeigen Sie:\\
        Ist $c\in[-\frac{1}{2};\frac{1}{4}]$, und gilt $\abs{a_n}\leq\frac{1}{2}$, so ist auch $\abs{a_{n+1}}\leq\frac{1}{2}$.\\
        Überlegen Sie sich weiter, dass daraus folgt, dass die Folge für $c\in[-\frac{1}{2};\frac{1}{4}]$ beschränkt durch $C=\frac{1}{2}$ ist.
    \end{enumerate}
%    
  \item \lang{de}{Gegeben sei die rekursiv definierte Folge $(a_{n})_{n\in\N}$ mit 
$a_{1}=0$ und $a_{n+1}=\frac{1}{2-a_{n}}$ für alle $n\in\N$.\\ 
Zeigen Sie, dass die Folge monoton und beschränkt ist.}
\end{enumerate}
 
\begin{tabs*}[\initialtab{0}\class{exercise}]

  \tab{\lang{de}{Lösungsvideo $1$ $a) - c)$}}	
    \youtubevideo[500][300]{Mjbm1X96Vik}

  \tab{
  \lang{de}{Lösung 2}}
  
  \begin{incremental}[\initialsteps{1}]
    \step 
    \lang{de}{Wir weisen zunächst die Beschränktheit nach. Wir behaupten, dass für alle $n\in\N$ gilt $a_{n}\in [0,1]$. Dazu verwenden wir eine Induktion nach $n$. Für $n=1$ gilt $a_{1}=0\in [0,1]$. Die Behauptung gelte nun für ein beliebiges aber festes $n\in\N$. Daraus folgern wir
	\[a_{n+1}=\frac{1}{2-a_{n}}\leq \frac{1}{2-1}=1\]
	und
	\[a_{n+1}= \frac{1}{2-a_{n}}\geq \frac{1}{2-0}=\frac{1}{2}\geq 0\,.\]
	Daher ist auch $a_{n+1}\in[0,1]$. Mit dem Prinzip der vollständigen Induktion  folgt somit die Behauptung.}
     
    \step \lang{de}{Wir wollen weiter zeigen, dass die Folge streng monoton steigend ist. Dazu verwenden wir erneut eine Induktion über $n$. Zunächst gilt 
	\[a_{1}=0\leq \frac{1}{2}=a_{2}\,.\] 
	Wir nehmen nun an, dass die Aussage für ein $n\in\N$ bereits gezeigt sei, dass also $a_{n+1}>a_{n}$ gelte. Daraus folgern wir 
	\[a_{n+2}=\frac{1}{2-a_{n+1}}>\frac{1}{2-a_{n}}=a_{n+1}\,,\]
	wobei wir zusätzlich die Beschränktheit $a_{n}\in[0,1]$ für alle $n\in\N$ verwendet haben. Somit erhalten wir auch die Monotonie der Folge mit Hilfe des Induktionsprinzips.
	}
  \end{incremental}

\end{tabs*}


\end{content}