\documentclass{mumie.element.exercise}
%$Id$
\begin{metainfo}
  \name{
    \lang{de}{Ü09: Häufungspunkte}
    \lang{en}{}
  }
  \begin{description} 
 This work is licensed under the Creative Commons License Attribution 4.0 International (CC-BY 4.0)   
 https://creativecommons.org/licenses/by/4.0/legalcode 

    \lang{de}{Hier die Beschreibung}
    \lang{en}{}
  \end{description}
  \begin{components}
  \end{components}
  \begin{links}
\link{generic_article}{content/rwth/HM1/T205_Konvergenz_von_Folgen/g_art_content_15_monotone_konvergenz.meta.xml}{content_15_monotone_konvergenz}
\link{generic_article}{content/rwth/HM1/T205_Konvergenz_von_Folgen/g_art_content_14_konvergenz.meta.xml}{content_14_konvergenz}
\end{links}
  \creategeneric
\end{metainfo}
\begin{content}
\title{
\lang{de}{Ü09: Häufungspunkte}
}
\begin{block}[annotation]
  Im Ticket-System: \href{http://team.mumie.net/issues/9860}{Ticket 9860}
\end{block}
 
\lang{de}{\begin{enumerate}[(a)]
 \item a) Gegeben sei die Folge $(a_{n})_{n\in\N}$ mit 
\[a_{n}=\begin{cases}
         \left(1+\frac{1}{n}\right)^{n}&\text{ , falls }n \text{ gerade}\\
	  1+2^{-n}&\text{ , falls }n \text{ ungerade.}
        \end{cases}\] 
Untersuchen Sie, ob $(a_{n})_{n\in\N}$ konvergiert und bestimmen Sie die Häufungspunkte und gegebenenfalls den Grenzwert.
  \item b) Ist die Folge $(b_{n})_{n\in\N}$ mit $b_{n}=\left(1+\frac{1}{n^{2}+1}\right)^{n^{2}}$ konvergent? Bestimmen Sie die Häufungspunkte und gegebenenfalls den Grenzwert.
\end{enumerate}}

\begin{tabs*}[\initialtab{0}\class{exercise}]

 \tab{
  \lang{de}{Antworten}}
  
  a) $a_n$ konvergiert nicht. Die Häufungspunkte lauten $\{1,e\}$.\\
  b) $b_n$ konvergiert gegen $e$, was demnach auch dem einzigen Häufungspunkt entspricht.

  \tab{
  \lang{de}{Lösung a)}}
 
    
    \lang{de}{Wir betrachten die beiden Teilfolgen $(x_n)_{n\in\N}$ definiert durch $x_n:=a_{2n}$ (gerade Indizes) und $(y_n)_{n\in\N}$
    definiert durch $y_n:=a_{2n-1}$ (ungerade Indizes) für alle $n\in\N$. $x_n=(1+\frac{1}{2n})^{2n}$ konvergiert gegen die 
    \ref[content_15_monotone_konvergenz][eulersche Zahl $e$]{thm:ezahl} und $y_n=1+2^{-(2n-1)}=1+\frac{1}{2}\cdot \frac{1}{2^{2n}}$ konvergiert gegen $1$.\\\\
    Somit gibt es zwei Teilfolgen von $(a_n)_{n\in\N}$, die gegen verschiedene Grenzwerte konvergieren. Aufgrund der 
    \ref[content_14_konvergenz][Eindeutigkeit]{thm:eindeutigkeitgw} des Grenzwerts, konvergiert $(a_n)_{n\in\N}$ nicht.\\\\
    Die Menge der Häufungspunkte ist $\{1,e\}$.}

  \tab{
  \lang{de}{Lösung b)}
  }
  \begin{incremental}[\initialsteps{1}]
     \step \lang{de}{Wir schreiben zunächst den definierenden Ausdruck der Folge ein wenig um. Für jedes $n\in\N$ gilt nämlich
	\[b_{n}=\left(1+\frac{1}{n^{2}+1}\right)^{n^{2}}=\left(1+\frac{1}{n^{2}+1}\right)^{-1}\cdot \left(1+\frac{1}{n^{2}+1}\right)^{n^{2}+1}\,.\] }
     \step \lang{de}{Die Folge $(x_{m})_{m\in\N}$ mit $x_{m}=\left(1+\frac{1}{m^{2}+1}\right)^{m^{2}+1}$ für alle $m\in\N$ ist eine Teilfolge von $\left(\left(1+\frac{1}{n}\right)^{n}\right)_{n\in\N}$ und konvergiert daher gegen $e$. Nach den Grenzwertsätzen ist somit auch $(b_{n})_{n\in\N}$
 	konvergent und es gilt 
	\[\lim_{n\to\infty}{b_{n}}=\lim_{n\to\infty}{\left(1+\frac{1}{n^{2}+1}\right)^{-1}}\cdot \lim_{n\to\infty}{x_{n}}=\frac{1}{1}\cdot e=e\,.\]
    
    Da die Folge konvergiert, ist der einzige Häufungspunkt $e$.}
  \end{incremental}

 


\end{tabs*}


\end{content}