\documentclass{mumie.element.exercise}
%$Id$
\begin{metainfo}
  \name{
    \lang{de}{Ü03: Grenzwertabschätzung}
    \lang{en}{}
  }
  \begin{description} 
 This work is licensed under the Creative Commons License Attribution 4.0 International (CC-BY 4.0)   
 https://creativecommons.org/licenses/by/4.0/legalcode 

    \lang{de}{Hier die Beschreibung}
    \lang{en}{}
  \end{description}
  \begin{components}
\component{generic_image}{content/rwth/HM1/T205_Konvergenz_von_Folgen/Audios/g_img_Button_schwarz.meta.xml}{Button_schwarz}
\end{components}
  \begin{links}
  \end{links}
  \creategeneric
\end{metainfo}
\begin{content}
\begin{block}[annotation]
	Im Ticket-System: \href{https://team.mumie.net/issues/18863}{Ticket 18863}
\end{block}

\title{
\lang{de}{Ü03: Grenzwertabschätzung}
}

 \begin{enumerate}

  \item[a)] Die Folge $(a_n)_{n\in\N}=\frac{3n}{n+2}$ konvergiert gegen den Grenzwert $a=3$. 
Geben Sie das kleinstmögliche $N$ an, so dass für $n\geq N$ für die Folgenglieder
$a_n$ gilt:\\

$|a_n-a|<\frac{1}{100}$
    
  \item[b)] Die Folgen $(\frac{2n}{n+1})_{n\in\N}$ und $(1+\frac{1}{2^n})_{n\in\N}$ konvergieren.\\
  Stellen Sie jeweils eine Vermutung bezüglich des Grenzwertes $a$ auf und geben Sie dann jeweils ein $N$ an, so dass für $n\geq N$ für die Folgenglieder
  $a_n$ gilt:\\\\
  $\abs{a_n-a}<\frac{1}{100}$~~~~~~~~ bzw.~~~~~~~~ $\abs{a_n-a}<10^{-8}$.
  
\end{enumerate}
 
\begin{tabs*}[\initialtab{0}\class{exercise}]

  \tab{
  \lang{de}{Antworten}}
    a) $N=599$\\\\
    b) Zur ersten Folge: $a=2$, $N=200$ bzw. $N=2\cdot 10^8$\\
    ~~~ Zur zweiten Folge: $a=1$, $N=7$ bzw. $N=27$

  \tab{
  \lang{de}{Lösung a)}}
  
  \begin{incremental}[\initialsteps{1}]
    \step 
    \lang{de}{Es gilt:
    \begin{align*}
        |a_n-a|&=|\frac{3n}{n+2}-3|\\
            &=|\frac{3n-3(n+2)}{n+2}|\\
            &=|\frac{-6}{n+2}|\\
            &=\frac{6}{n+2}
    \end{align*}
    } 
    \step 
    \lang{de}{Daraus folgt: 
    \begin{align*}
    |a_n-a|<\frac{1}{100}&~\Leftrightarrow \frac{6}{n+2}<\frac{1}{100}\\&~\Leftrightarrow 600<n+2\\&~\Leftrightarrow n>598\\
    &~\Leftrightarrow n\geq 599
    \end{align*}
    Demnach ist das kleinstmögliche $N=599$, damit $|a_n-a|<\frac{1}{100}$ gilt.
    }
  \end{incremental}
  
      \tab{\lang{de}{Lösungsvideo b)}}	
    \youtubevideo[500][300]{H1Zmfmi9cSA}\\
 
 % \end{incremental}

\end{tabs*}


\end{content}