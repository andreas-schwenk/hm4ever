\documentclass{mumie.element.exercise}
%$Id$
\begin{metainfo}
  \name{
    \lang{de}{Ü04: Folgenkonvergenz}
    \lang{en}{}
  }
  \begin{description} 
 This work is licensed under the Creative Commons License Attribution 4.0 International (CC-BY 4.0)   
 https://creativecommons.org/licenses/by/4.0/legalcode 

    \lang{de}{Hier die Beschreibung}
    \lang{en}{}
  \end{description}
  \begin{components}
  \end{components}
  \begin{links}
  \end{links}
  \creategeneric
\end{metainfo}
\begin{content}
\title{
\lang{de}{Ü04: Folgenkonvergenz}
}
\begin{block}[annotation]
  Im Ticket-System: \href{http://team.mumie.net/issues/9855}{Ticket 9855}
\end{block}
 
  \begin{enumerate}
  \item Zeichnen Sie einige Folgenglieder zu den durch die folgenden Ausdrücke beschriebenen Folgen auf der Zahlengerade.
 \begin{table}[\class{items}]
       \nowrap{a) $a_n=\frac{2n}{n+1}$} 
     & \nowrap{b) $b_n=(-1)^n\cdot (1+\frac{1}{n})$} \\
      \nowrap{c) $c_n=\sin(n)$} 
     &  \nowrap{d) $d_n=\cos(2-\frac{1}{n})$} \\
      \nowrap{e) $e_1=1,~~e_{n+1}=\frac{e_n}{2}+\frac{2}{e_n}$} 
     & \nowrap{f) $s_0=0,~~s_{n+1}=s_n+\frac{1}{2^n}$} \\
    \end{table}
  
Sind die Folgen konvergent? (Sie brauchen keine exakte Begründung anzugeben.)
 
\item 
\begin{enumerate}[(a)]
 \item a) Ist die Folge $(a_{n})_{n\in\N}$ gegeben durch $a_{n}=\frac{2^{n}}{n}$ konvergent?
 \item b) Ist die Folge $(b_{n})_{n\in\N}$ gegeben durch 
\[b_{n}=\begin{cases}
         \frac{1}{n}&\text{ , falls }n \text{ gerade}\\
	  1&\text{ , falls }n \text{ ungerade}
        \end{cases}\] 
konvergent?
\item c) Ist die Folge $(c_{n})_{n\in\N}$ gegeben durch $c_{n}=(-1)^{n}+\frac{1}{n}$ konvergent?
\end{enumerate}

\end{enumerate}

\begin{tabs*}[\initialtab{0}\class{exercise}]
  \tab{
  \lang{de}{Antworten}
  \lang{en}{Answers}
  }


\begin{table}[\class{items}]
1.\\
a) $a_n$ ist konvergent gegen $2$.\\
b) $b_n$ konvergiert nicht.\\ 
c) $c_n$ konvergiert nicht.\\
d) $d_n$ konvergiert gegen $cos(2)$.\\
e) $e_n$ konvergiert gegen $2$.\\
f) $s_n$ konvergiert gegen $2$.\\

2. \\
a) $a_n$ ist nicht beschränkt und damit auch nicht konvergent.\\
b) $b_n$ ist nicht konvergent.\\
c) $c_n$ ist nicht konvergent.
\end{table}
      \tab{\lang{de}{Lösungsvideo  1 a) - f)}}	
    \youtubevideo[500][300]{DLpg3wd7lpA}
  
  \tab{
  \lang{de}{Lösung 2 a)}}
    
  \begin{incremental}[\initialsteps{1}]
    \step 
    \lang{de}{Wir behaupten, dass die Folge nicht konvergiert. Offenbar genügt es zu zeigen, dass für alle $n\in\N$ die Aussage $2^{n}\geq \frac{n^{2}-n}{2}$ gilt. Nach Division durch $n$ erhalten wir daraus die Aussage 
	\[a_{n}\geq \frac{n-1}{2}\] 
	für alle $n\in\N$. Dies zeigt nämlich, dass die Folge nicht beschränkt, und damit auch nicht konvergent ist.}
     
    \step \lang{de}{Um die Hilfsaussage zu beweisen, erinnern wir uns an den binomischen Lehrsatz, der besagt, dass
	\[(a+b)^{n}=\sum_{k=0}^{n}{\binom{n}{k}a^{k}b^{n-k}}\quad\text{ für }a,b\in \R\text{ und }n\in\N_{0}\]
	gilt. Setzen wir $a=b=1$ ein, so erhalten wir daraus für $n \geq 2$ die Aussage
	\[2^{n}=(1+1)^{n}=\sum_{k=0}^{n}{\binom{n}{k}}\geq \binom{n}{2}=\frac{n!}{2! (n-2)!}=\frac{n(n-1)}{2}=\frac{n^{2}-n}{2}\,,\]
	denn für jedes $k\in\N_{0}$ mit $k\leq n$ gilt $\binom{n}{k}\geq 0$. Insgesamt folgt die Divergenz der Folge.}
  \end{incremental}

  \tab{
  \lang{de}{Lösung 2 b)}
  }
  
      \lang{de}{Der Grenzwert einer konvergenten Folge ist eindeutig bestimmt.  Die beiden Teilfolgen $(b_{2n})_{n\in\N}$ und $(b_{2n-1})_{n\in\N}$ sind konvergent und es gilt nach Definition
	\[\lim_{n\to\infty}{b_{2n-1}}=\lim_{n\to\infty}{1}=1\quad\text{ und }\quad \lim_{n\to\infty}{b_{2n}}=\lim_{n\to\infty}\frac{1}{2n}=0\,.\]
	Also ist die Folge nicht konvergent, denn ansonsten müsste jede Teilfolge gegen den Grenzwert der Folge konvergieren.}

 

  \tab{
  \lang{de}{Lösung 2 c)}
  }
  \begin{incremental}[\initialsteps{1}]
    \step \lang{de}{Wir nehmen an, dass die Folge konvergent ist. 
    Wie wir bereits wissen, ist die Folge $(u_{m})_{m\in\N}$ mit $u_{m}=\frac{1}{m}$ für alle $m\in\N$ konvergent mit Grenzwert 0. 
    Nach den Grenzwertsätzen ist dann auch die Differenz $(d_{n})_{n\geq 1}=(c_{n})_{n\in \N}-(u_{n})_{n\in\N}$ konvergent. 
    Es gilt jedoch für alle $n\in\N$ die Gleichung $d_{n}=(-1)^{n}$ und diese Folge ist divergent. 
    Also erhalten wir einen Widerspruch. 
    Daher war die Annahme, dass $(c_{n})_{n\geq 1}$ konvergiert, falsch, und die Folge ist divergent. }
    
    
  \end{incremental}

\end{tabs*}


\end{content}