\documentclass{mumie.problem.gwtmathlet}
%$Id$
\begin{metainfo}
  \name{
    \lang{en}{...}
    \lang{de}{A03: AWA}
  }
  \begin{description} 
 This work is licensed under the Creative Commons License Attribution 4.0 International (CC-BY 4.0)   
 https://creativecommons.org/licenses/by/4.0/legalcode 

    \lang{en}{...}
    \lang{de}{...}
  \end{description}
  \corrector{system/problem/GenericCorrector.meta.xml}
  \begin{components}
    \component{js_lib}{system/problem/GenericMathlet.meta.xml}{gwtmathlet}
  \end{components}
  \begin{links}
  \end{links}
  \creategeneric
\end{metainfo}
\begin{content}
\lang{de}{\title{A03: AWA}}
\begin{block}[annotation]
	Im Ticket-System: \href{https://team.mumie.net/issues/21956}{Ticket 21956}
\end{block}
\usepackage{mumie.genericproblem}

\begin{block}[annotation]
Pool mit zwei Aufgaben zu separierten, nicht linearen Anfangswertproblemen mit unterschiedlichen
Koeffizienten.

Differenziertes Feedback enthalten, verschiedene mögliche Wahlen der Funktionen (können sich um multiplikative Konstanten unterscheiden) berücksichtigt.
\end{block}
\begin{problem}
\randomquestionpool{1}{2}
%% Erste Aufgabe: g=exp(-bx), f=alpha x+beta, also DGL y'=f *g(y) mit AW y(0)=y0
	\begin{question}
		
		\begin{variables}
    %Anfangswert:
    %x0=0
    \randint[Z]{y0}{1}{5}
    %Koeffizienten von g
    \randint[Z]{b}{2}{10}
    %Funktion g in y
    \function{g}{exp(-b*y)}
    %Koeffizient von f
    \randint[Z]{alpha}{1}{10}
    \randint[Z]{beta}{-10}{10}
    \randadjustIf{beta}{beta^2-2*alpha*e^(b*y0)>0}%Stellt sicher, dass Diskriminante<0, also das Polynom im ln-Argument der Lösung keine Nullstellen hat.
    %Funktion f
    \function{f}{alpha*x+beta}
    % Stammfunktion H von 1/g
    \function{H}{1/b*exp(b*y)-1/b*exp(b*y0)}
    %Stammfunktion F von f:
    \function{F}{alpha/2*x^2+beta*x}
    %Lösung selbst:
    \function{phi}{1/b*ln(b*(alpha/2*x^2+beta*x)+exp(b*y0))}
    %Intervallgrenzen von J' und I'
    \function{l1}{-infty}
    \function{l2}{infty}
    \function{l3}{-1/b*exp(b*y0)}
    \function{null}{0}
        \end{variables}
        
      \lang{de}{\text{Lösen Sie das Anfangswertproblem $y'=(\var{f})\cdot \var{g}\:$ mit $y(0)=\var{y0}$.
      \\
      (a) Geben Sie dazu zunächst  Funktionen $f$ und $g$ an, so dass die DGL die Form $y'=f\cdot g(y)$ hat, 
      sowie die durch das Problem bestimmten Stammfunktionen $F$ von $f$ und $H$ von $\frac{1}{g}$.\\
      $f(x)=$\ansref mit $F(x)=$\ansref, sowie $g(y)=$\ansref mit $H(y)=$\ansref.
      \\
      (b) Bestimmen Sie nun die Lösung $\phi$ des Anfangswertproblems $\phi(x)=$\ansref.
       \\
       (c) Bestimmen Sie das maximale Definitionsintervall $J'$ von $H$, sein Bild $H(J')$ und das maximale Definitionsintervall $I'$
       der Lösung $\phi$.
       $J'=$\ansref, $H(J')=$\ansref, $I'=$\ansref.
       }}
      \explanation[NOT[edited]]{Es handelt sich um ein separierbares Anfangswertproblem.}
    %Teil (a)
	\begin{answer}    
        \type{input.function}
		\solution{f}
        \inputAsFunction{x}{f1}
        \checkFuncForZero{f1-f}{-1}{1}{10}
        \explanation[edited]{Ihre Wahl von $f$ ist nicht die einfachste, möglicherweise aber trotzdem korrekt.
        Ihre Wahl wird im folgenden berücksichtigt.}
     \end{answer}
     \begin{answer}
        \type{input.function}
        \solution{F}
        \inputAsFunction{x}{F1}
        % Prüfe, ob F1 mit Erwartung übereinstimmt oder  ausgehend von der Eingabe f1 die korrekte Stammfunktion mit Wert 0 in x0=0 ist:
        \checkFuncForZero{(F-F1)*(|D[F1,x]-f1|+|F1[null,x]|)}{-1}{1}{10}
        \explanation[edited]{Die Stammfunktion $F$ ergibt sich eindeutig durch die Forderung $F(x_0)=0$.}
     \end{answer}
     \begin{answer}
        \type{input.function}
        \solution{g}
        \inputAsFunction{y}{g1}
        % Prüfe, ob g1 mit Erwartung übereinstimmt oder eine mit Eingabe f1 konsistente Wahl ist:
        \checkFuncForZero{(g-g1)*(f*g-f1*g1)}{-1}{1}{10}
        \explanation[edited]{Ihre Wahl von $g$ ist nicht kompatibel mit Ihrer Wahl von $f$.}
     \end{answer}
     \begin{answer}
        \type{input.function}
        \solution{H}
        \inputAsFunction{y}{H1}
        % Prüfe, ob H1 mit Erwartung übereinstimmt oder ausgehend von der Eingabe g1  die korrekte Stammfunktion mit Wert 0 in y0 ist:
        \checkFuncForZero{(H-H1)*(|D[H1,y]-1/g1|+|H1[y0,y]|)}{-1}{1}{10}     
        \explanation[edited]{Die Stammfunktion $H$ ergibt sich eindeutig durch die Forderung $H(y_0)=0$.}
	\end{answer}
    %Teil (b)
    \begin{answer}    
        \type{input.function}
        \solution{phi}
        \inputAsFunction{x}{phi1}
        \checkFuncForZero{phi-phi1}{-1}{1}{10}
        \explanation[edited]{Die Lösung $\phi$ eines separierbaren Anfangswertproblems ergibt sich als $H^{-1}\circ F(x)$.}
    \end{answer}
    %Teil (c)
    \begin{answer}    
        \type{input.interval}
        \solution{(l1;l2)}
        \explanation{Das Definitionsintervall $J'$ von $H$ ist das größte Intervall, auf dem $g$ definiert ist und das $y_0$ enthält, aber keine Nullstelle von $g$.}   
     \end{answer}  
     \begin{answer}    
        \type{input.interval}
        \solution{(l3;l2)}
     \end{answer} 
     \begin{answer}    
        \type{input.interval}
        \solution{(l1;l2)}
        \explanation{Das maximale Lösungsintervall $I'$ von $\phi$ ist maximal mit der Eigenschaft, dass $x_0\in I'$ und $F(I')\subset H(J')$.}
     \end{answer} 
    \end{question}
    
    
%% Zweite Aufgabe: g=1/(y^2^k), f=a*exp(bx), also DGL y'=f *g(y) mit AW y(0)=y0
	\begin{question}
		
		\begin{variables}
    %Anfangswert:
    %x0=0; y0 soll groß genug sein, so dass das Existenzintervall nicht zu schwierig wird:
    \randint[Z]{y0}{2}{10}
    %Exponent von g
    \randint[Z]{k}{1}{10}
    \function[normalize]{l}{2*k+1}
    %Funktion g in y
    \function[normalize]{g}{1/y^(2*k)}
    %Koeffizient von f: Es soll a<=b gelten, damit Existenzintervall nicht zu schwierig wird
    \randint[Z]{a}{1}{5}
    \randint[Z]{b}{5}{10}
    %Funktion f
    \function{f}{a*exp(b*x)}
    % Stammfunktion H von 1/g
    \function{H}{1/l*(y^l-y0^l)}
    %Stammfunktion F von f:
    \function{F}{a/b*(exp(b*x)-1)}
    %Lösung selbst:
    \function{phi}{(l*a/b*(exp(b*x)-1)+y0^l)^(1/l)}
    %Intervallgrenzen von J' und I'
    \function{l1}{-infty}
    \function{l2}{infty}
    \function{l3}{-1/l*y0^l}
    \function{null}{0}
        \end{variables}
        
      \lang{de}{\text{Lösen Sie das Anfangswertproblem $y'=\var{f}\cdot \var{g}\:$ mit $y(0)=\var{y0}$.
      \\
      (a) Geben Sie dazu zunächst  Funktionen $f$ und $g$ an, so dass die DGL die Form $y'=f\cdot g(y)$ hat, 
      sowie die durch das Problem bestimmten Stammfunktionen $F$ von $f$ und $H$ von $\frac{1}{g}$.\\
      $f(x)=$\ansref mit $F(x)=$\ansref, sowie $g(y)=$\ansref mit $H(y)=$\ansref.
      \\
      (b) Bestimmen Sie nun die Lösung $\phi$ des Anfangswertproblems $\phi(x)=$\ansref.
       \\
       (c) Bestimmen Sie das maximale Definitionsintervall $J'$ von $H$, sein Bild $H(J')$ und das maximale Definitionsintervall $I'$
       der Lösung $\phi$.
       $J'=$\ansref, $H(J')=$\ansref, $I'=$\ansref.
       }}
      \explanation[NOT[edited]]{Es handelt sich um ein separierbares Anfangswertproblem.}
    %Teil (a)
	\begin{answer}    
        \type{input.function}
		\solution{f}
        \inputAsFunction{x}{f1}
        \checkFuncForZero{f1-f}{-1}{1}{10}
        \explanation[edited]{Ihre Wahl von $f$ ist nicht die einfachste, möglicherweise aber trotzdem korrekt.
        Ihre Wahl wird im folgenden berücksichtigt.}
     \end{answer}
     \begin{answer}
        \type{input.function}
        \solution{F}
        \inputAsFunction{x}{F1}
        % Prüfe, ob F1 mit Erwartung übereinstimmt oder  ausgehend von der Eingabe f1 die korrekte Stammfunktion mit Wert 0 in x0=0 ist:
        \checkFuncForZero{(F-F1)*(|D[F1,x]-f1|+|F1[null,x]|)}{-1}{1}{10}
        \explanation[edited]{Die Stammfunktion $F$ ergibt sich eindeutig durch die Forderung $F(x_0)=0$.}
     \end{answer}
     \begin{answer}
        \type{input.function}
        \solution{g}
        \inputAsFunction{y}{g1}
        % Prüfe, ob g1 mit Erwartung übereinstimmt oder eine mit Eingabe f1 konsistente Wahl ist:
        \checkFuncForZero{(g-g1)*(f*g-f1*g1)}{-1}{1}{10}
        \explanation[edited]{Ihre Wahl von $g$ ist nicht kompatibel mit Ihrer Wahl von $f$.}
     \end{answer}
     \begin{answer}
        \type{input.function}
        \solution{H}
        \inputAsFunction{y}{H1}
        % Prüfe, ob H1 mit Erwartung übereinstimmt oder ausgehend von der Eingabe g1  die korrekte Stammfunktion mit Wert 0 in y0 ist:
        \checkFuncForZero{(H-H1)*(|D[H1,y]-1/g1|+|H1[y0,y]|)}{-1}{1}{10}     
        \explanation[edited]{Die Stammfunktion $H$ ergibt sich eindeutig durch die Forderung $H(y_0)=0$.}
	\end{answer}
    %Teil (b)
    \begin{answer}    
        \type{input.function}
        \solution{phi}
        \inputAsFunction{x}{phi1}
        \checkFuncForZero{phi-phi1}{-1}{1}{10}
        \explanation[edited]{Die Lösung $\phi$ eines separierbaren Anfangswertproblems ergibt sich als $H^{-1}\circ F(x)$.}
    \end{answer}
    %Teil (c)
    \begin{answer}    
        \type{input.interval}
        \solution{(null;l2)}
        \explanation{Das Definitionsintervall $J'$ von $H$ ist das größte Intervall, auf dem $g$ definiert ist und das $y_0$ enthält, aber keine Nullstelle von $g$.}   
     \end{answer}  
     \begin{answer}    
        \type{input.interval}
        \solution{(l3;l2)}
     \end{answer} 
     \begin{answer}    
        \type{input.interval}
        \solution{(l1;l2)}
        \explanation{Das maximale Lösungsintervall $I'$ von $\phi$ ist maximal mit der Eigenschaft, dass $x_0\in I'$ und $F(I')\subset H(J')$.}
     \end{answer} 
    \end{question}    
\end{problem}
\embedmathlet{gwtmathlet}

\end{content}
