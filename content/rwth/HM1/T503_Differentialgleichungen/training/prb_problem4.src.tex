\documentclass{mumie.problem.gwtmathlet}
%$Id$
\begin{metainfo}
  \name{
    \lang{en}{...}
    \lang{de}{A04: Pandemie}
  }
  \begin{description} 
 This work is licensed under the Creative Commons License Attribution 4.0 International (CC-BY 4.0)   
 https://creativecommons.org/licenses/by/4.0/legalcode 

    \lang{en}{...}
    \lang{de}{...}
  \end{description}
  \corrector{system/problem/GenericCorrector.meta.xml}
  \begin{components}
    \component{js_lib}{system/problem/GenericMathlet.meta.xml}{gwtmathlet}
  \end{components}
  \begin{links}
  \end{links}
  \creategeneric
\end{metainfo}
\begin{content}
\usepackage{mumie.genericproblem}
\lang{de}{\title{A04: Pandemie}}
\begin{block}[annotation]
	Im Ticket-System: \href{https://team.mumie.net/issues/21997}{Ticket 21997}
\end{block}


\begin{block}[annotation]
Textaufgabe zum langfristigen, ungedämpften Pandemieverlauf.
Lösungsfunktion ist eine logistische Funktion. 
Bei Einbindung in einen Kurs bietet es sich an, den Graph der Lösung zu diskutieren.
\end{block}
\begin{problem}
%\randomquestionpool{1}{2}
%%  DGL y'=k*y(1-y) mit AW y(0)=y0
	\begin{question}
    
    \begin{variables}
    %Feste Proportionalitätskonstante
    \randint{j}{1}{50}
    \function[normalize]{k}{j/10}
    %Anfangswert y0
    \function{y0}{z}
    %Rechte Seite der DGL
    \function{RS}{k*y*(1-y)}
    %Lösung
    \function{phi}{1/(1+z/(1-z)*exp(-k*x))}
    \end{variables}
    \text{Die langfristige Änderung des Anteils der infizierten Bevölkerung während einer Pandemie ist proportional abhängig 
    vom Anteil der infizierten Bevölkerung selbst, als auch vom Anteil der (noch) nicht infizierten Bevölkerung.
    Werden keine Maßnahmen gegen die Pandemie ergriffen (Infektionsschutz, Impfung o.ä.) und wird von einer Immunität nach einmaliger Ansteckung ausgegangen,
    dann lassen sich die übrigen Einflüsse auf den Pandemieverlauf in einer einzigen Proportionalitätskonstante zusammen fassen. Nehmen Sie an,
    diese Konstante sei $\var{k}$. Stellen Sie die Differentialgleichung für den Anteil der infizierten Bevölkerung auf und lösen Sie
    sie unter der Annahme, dass zum Zeitpunkt $t_0=0$ ein Anteil $1>y_0:=z>0$ der Bevölkerung infiziert ist.}
    \begin{answer}
            \type{input.function}
		    \text{Die Differentialgleichung lautet $y'=$}
		    \solution{RS}
            \inputAsFunction{y}{t}
            \checkFuncForZero{RS-t}{-1}{1}{10}
        \explanation{Die rechte Seite der Differentialgleichung ist das Produkt der obengenannten Faktoren. 
        Bezeichnen Sie die Gesamtbevölkerung mit $1$,  den  Anteil der infizierten Bevölkerung mit $y$.}
    \end{answer}
    \begin{answer}
            \type{input.function}
		    \text{Die Lösung des Anfangswertproblems lautet $\phi(x)=$}
		    \solution{phi}
            \inputAsFunction{x,z}{f1}
            \checkFuncForZero{phi-f1}{-1}{1}{10}
            \explanation{Es handelt sich um eine separierbare Differentialgleichung, für die Sie ein Lösungsverfahren kennen sollten.}
    \end{answer}
    \end{question}
\end{problem}
\embedmathlet{gwtmathlet}

\end{content}
