\documentclass{mumie.problem.gwtmathlet}
%$Id$
\begin{metainfo}
  \name{
    \lang{en}{...}
    \lang{de}{A01: AWA}
  }
  \begin{description} 
 This work is licensed under the Creative Commons License Attribution 4.0 International (CC-BY 4.0)   
 https://creativecommons.org/licenses/by/4.0/legalcode 

    \lang{en}{...}
    \lang{de}{...}
  \end{description}
  \corrector{system/problem/GenericCorrector.meta.xml}
  \begin{components}
    \component{js_lib}{system/problem/GenericMathlet.meta.xml}{gwtmathlet}
  \end{components}
  \begin{links}
  \end{links}
  \creategeneric
\end{metainfo}
\begin{content}
\begin{block}[annotation]
	Im Ticket-System: \href{https://team.mumie.net/issues/21700}{Ticket 21700}
\end{block}
\usepackage{mumie.genericproblem}

\lang{de}{\title{A01: AWA}}
\begin{block}[annotation]
Eine einfache Anfangswertaufgabe: linear, homogen, konstanter Koeffizient.
\\
Hier ist differenziertes Feedback bereits eingebaut, aber nicht abrufbar: 
Dazu ist die Weiterverarbeitung von Eingabefunktionen noch zu arm implementiert.
\\
tbc
\end{block}
\begin{problem}
  \begin{question}
    \begin{variables}
    %Anfangswert:
    \randint[Z]{vz}{-1}{1} 
    \randint{X0}{1}{10}
    \function[calculate]{x0}{vz*X0}
    %Für Kosmetik der Aufgabenstellung:
    \begin{switch}
        \begin{case}{vz=-1}
        \string{VZ}{-}
        \end{case}
        \begin{default}
        \string{VZ}{}
        \end{default}
    \end{switch}
    \randint{x00}{1}{10}     
    \randint[Z]{y0}{-10}{10}
    %der Koeffizienten der Lösung alpha*exp(a*x) der DGL y'=ay zu obigem Anfangswert:
	\function{alpha}{y0/e}
    \function{a}{1/x0} 
    %Lösung selbst:
    \function{phi}{alpha*exp(a*x)}
    %ausgewertet in Null:
    \function{null}{0}
    \substitute{const}{phi}{x}{null}
    %Intervall:
    \function{l1}{-infty}
    \function{l2}{infty}
    \end{variables}
        
    \lang{de}{\text{Lösen Sie das Anfangswertproblem $y'=\var{VZ}\frac{1}{\var{X0}}\cdot y$ mit $y(\var{x0})=\var{y0}$.\\
                    \textit{Bestimmen Sie die Lösung exakt, nicht gerundet.}}}
    
	\begin{answer}
        \type{input.function}
		\text{ Die Lösung des Anfangswertproblems ist $\phi(x)=$}
		\solution{phi}
        \inputAsFunction{x}{t1}
        \checkFuncForZero{phi-t1}{-1}{1}{10}
        %Die bedingten Explanations funktionieren noch nicht, weil die Weiterverarbeitung der Funktion t1 noch nicht ausgereift implementiert ist.
        %Falls dies einmal, der Fall sein sollte, ist differenziertes Feedback möglich.
        %Wenn DGL erfüllt, aber nicht Anfangswert:
        \explanation[NOT[equal(y0,t1[x0,x])]AND [equal(t1,a*D[t1,x])]]{Ihre Funktion erfüllt die Differentialgleichung, aber nicht die Anfangswertbedingung.}
        %Wenn Anfangswert erfüllt, aber nicht DGL:
        \explanation[[equal(y0,t1[x0,x])]AND NOT[equal(t1,a*D[t1,x])]]{Ihre Funktion erfüllt die Anfangswertbedingung, aber nicht die Differentialgleichung.}
        %Wenn weder Anfangswert noch DGL erfüllt:
        \explanation[NOT[equal(y0,t1[x0,x])]AND NOT[equal(t1,a*D[t1,x])]]{Es handelt sich um ein lineares, homogenes Anfangswertproblem. Weil der Koeffizient $\var{a}$
        konstant ist, hat die Lösung die Form $a\cdot \exp(b x)$. Wie müssen $a$ und $b$ gewählt werden?}
	\end{answer}
    \begin{answer}
        \type{input.interval}
        \text{Was ist das maximale Existenzintervall dieser Lösung?}
        \solution{(l1;l2)}
        \explanation{Das maximale Lösungsintervall einer linearen Differentialgleichung ist gegeben durch das Definitionsintervall $I$ seiner Koeffizienten mit $x_0\in I$.}
    \end{answer}
    \end{question}
\end{problem}
\embedmathlet{gwtmathlet}
\end{content}
