%$Id:  $
\documentclass{mumie.article}
%$Id$
\begin{metainfo}
  \name{
    \lang{de}{Überblick: Differentialgleichungen}
    \lang{en}{overview: }
  }
  \begin{description} 
 This work is licensed under the Creative Commons License Attribution 4.0 International (CC-BY 4.0)   
 https://creativecommons.org/licenses/by/4.0/legalcode 

    \lang{de}{Beschreibung}
    \lang{en}{}
  \end{description}
  \begin{components}
  \end{components}
  \begin{links}
\link{generic_article}{content/rwth/HM1/T503_Differentialgleichungen/g_art_content_57_gewoehnliche_DGL_erster_Ordnung.meta.xml}{content_57_gewoehnliche_DGL_erster_Ordnung}
\link{generic_article}{content/rwth/HM1/T503_Differentialgleichungen/g_art_content_56_Separierbare_Differentialgleichungen.meta.xml}{content_56_Separierbare_Differentialgleichungen}

  \end{links}
  \creategeneric
\end{metainfo}
\begin{content}
\begin{block}[annotation]
	Im Ticket-System: \href{https://team.mumie.net/issues/30103}{Ticket 30103}
\end{block}


\begin{block}[annotation]
Im Entstehen: Überblicksseite für Kapitel gewöhnliche DGL
\end{block}

\usepackage{mumie.ombplus}
\ombchapter{1}
\lang{de}{\title{Überblick: Differentialgleichungen}}
\lang{en}{\title{}}



\begin{block}[info-box]
\lang{de}{\strong{Inhalt}}
\lang{en}{\strong{Contents}}


\lang{de}{
    \begin{enumerate}%[arabic chapter-overview]
   \item[3.1] \link{content_56_Separierbare_Differentialgleichungen}{Separierbare und lineare Differentailgleichungen}
   \item[3.2] \link{content_57_gewoehnliche_DGL_erster_Ordnung}{Gewöhnliche Differentialgleichungen erster Ordnung}
  \end{enumerate}
} %lang

\end{block}

\begin{zusammenfassung}

\lang{de}{
Differentialgleichungen treten überall in Natur und Technik auf, 
denn in fast allen Prozessen bedingt das Vorhandene (die Funktion) seine Änderungen (die Ableitungen).
Differentialgleichungen sind im allgemeinen sehr schwierig zu lösen. 

Hier werden die konzeptionell einfachsten Differentialgleichungen vorgestellt, die gewöhnlichen Differentialgleichungen erster Ordnung.
Von diesen Differentialgleichungen weiß man, dass sie, wenn man einen sogenannten Anfangswert festlegt, eine eindeutige Lösung besitzen.
Für die einfachsten Sorten von solchen Differentialgleichungen geben wir Lösungsverfahren an, nämlich für 
linear homogene und linear inhomogene, sowie für separiebare Anfangwertprobleme erster Ordnung.
}


\end{zusammenfassung}

\begin{block}[info]\lang{de}{\strong{Lernziele}}
\lang{en}{\strong{Learning Goals}} 
\begin{itemize}[square]
\item \lang{de}{Sie kennen die Begriffe gewöhnliche Differentialgleichung erster Ordnung  und deren Anfangswertprobleme und wissen um deren Lösbarkeit.}
\item \lang{de}{Sie erkennen, klassifizieren und lösen separierbare, homogene lineare und inhomogene lineare Anfangswertprobleme erster Ordnung.}
%\item \lang{de}{}
\end{itemize}

\end{block}




\end{content}
