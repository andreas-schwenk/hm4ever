%$Id:  $
\documentclass{mumie.article}
%$Id$
\begin{metainfo}
  \name{
    \lang{en}{...}
    \lang{de}{Separierbare und lineare Differentialgleichungen}
   }
  \begin{description} 
 This work is licensed under the Creative Commons License Attribution 4.0 International (CC-BY 4.0)   
 https://creativecommons.org/licenses/by/4.0/legalcode 

    \lang{en}{...}
    \lang{de}{...}
  \end{description}
  \begin{components}
\component{generic_image}{content/rwth/HM1/images/g_img_00_video_button_schwarz-blau.meta.xml}{00_video_button_schwarz-blau}
  \end{components}
  \begin{links}
\link{generic_article}{content/rwth/HM1/T305_Integrationstechniken/g_art_content_12_substitutionsregel.meta.xml}{content_12_substitutionsregel}
\link{generic_article}{content/rwth/HM1/T305_Integrationstechniken/g_art_content_13_partialbruchzerlegung.meta.xml}{content_13_partialbruchzerlegung}
\end{links}
  \creategeneric
\end{metainfo}


\begin{content}
\begin{block}[annotation]
	Im Ticket-System: \href{https://team.mumie.net/issues/21645}{Ticket 21645}
\end{block}
\usepackage{mumie.ombplus}
\ombchapter{3}
\ombarticle{1}
\usepackage{mumie.genericvisualization}
\title{Separierbare und lineare Differentialgleichungen}
\begin{block}[info-box]
\tableofcontents
\end{block}
Differentialgleichungen stellen Funktionen und ihre Ableitungen miteinander in Verbindung.
Die einfachste Differentialgleichung ist
\[f'=f.\]
Sie kennen bereits eine Lösung dieser Differentialgleichung: $f(x)=\exp(x)$. 
Ebenso ist auch $f_a(x)=a\exp(x)$ eine Lösung für jedes $a\in\R$.
Differentialgleichungen beschreiben wesentliche Zusammenhänge in allen Natur- und Lebenswissenschaften
und damit in Konsequenz auch in Ihrem Fachbereich. 
Obige Gleichung gibt zum Beispiel ein exponentielles Wachstum an, 
das heißt der Zuwachs einer Population (o.ä.)
ist genau so groß wie die Polulation selbst. Ist dann die Größe der Population etwa zur  Zeit $x_0=0$ 
durch $a=f_a(x_0)=a\exp(0)$ gegeben, so ist sie zu jeder anderen Zeit $x$ gleich $f_a(x)=a\exp(x)$.
Wir werden unten sehen, dass dieses Anfangswertproblem eine eindeutige Lösung hat. 
Das heißt, es gibt eine einzige  Funktion, die sowohl $f'=f$ als auch $f(x_0)=a$ erfüllt.
Der Anfangspunkt $x_0$ liegt dabei nicht notwendig auf dem Rand des Definitionsbereichs.

In diesem Kapitel stellen wir einige Typen von Differentialgleichungen (kurz: DGL) vor. 
Die Lösungsmethoden für Differentialgleichungen sind sehr vielfältig, reichen von sehr einfach bis äußerst kompliziert
und bilden einen ganzen Zweig der mathematischen Forschung.
Oft kann man Differentialgleichungen gar nicht explizit lösen, sondern bestenfalls die Existenz einer Lösung garantieren.
Dann kann man aber mit numerischen Methoden Näherungen für die Lösung berechnen, die für praktische Zwecke völlig ausreichen.

Dabei benutzen wir eine übliche Konvention und schreiben die Differentialgleichung in der Variablen $y$ statt $f$.
Unsere Differentialgleichung für exponentielles Wachstum lautet also
\[y'=y.\]
Damit bringen wir zum Ausdruck, dass $y$ eine bislang unbestimmte Funktion ist, 
halten aber die Variable $x$ als Funktionsvariable  vor. 
%Video
\center{\href{https://api.stream24.net/vod/getVideo.php?id=10962-2-10751&mode=iframe&speed=true}{\image[75]{00_video_button_schwarz-blau}}}\\
\\

%%
%\section{Separierbare und lineare Differentialgleichungen}
\section{Separierbare Differentialgleichungen}\label{sec:sep_lin_DGL}
Separierbare (oder: trennbare) Differentialgleichungen sind solche, in denen die übrige Abhängigkeit der
Differentialgleichung von der Funktionsvariablen $x$ von der Abhängigkeit in $y$ als Faktor abgespalten werden kann.
Wir beginnen ganz formal mit der genauen Definition.
\begin{definition}[Separierbare DGL]\label{def:sep_DGL}
Es seien $I$ und $J$ in $\R$ nicht ausgeartete Intervalle. Weiter seien $f:I\to\R$ und $g:J\to \R$ stetige Funktionen.
\begin{enumerate}
\item[(i)]
Eine Gleichung der Form
\[y'=f(x)\cdot g(y)\]
heißt \notion{separierbare} Differentialgleichung auf $I\times J$.
Ein \notion{Anfangswert} ist die Vorgabe eines Wertes $y(x_0)=y_0\in J$ für ein $x_0\in I$.
\item[(ii)]
Ist $I'$ ein nicht ausgeartetes Teilintervall von $I$ und ist $\phi:I'\to \R$ eine differenzierbare Funktion
mit $\phi(I')\subset J$, dann ist $\phi$ eine \notion{Lösung} der Differentialgleichung, wenn
\[\phi'(x)=f(x)g(\phi(x))\]
für alle $x\in I'$ erfüllt ist.
\item[(iii)]
Gilt zusätzlich $y_0=\phi(x_0)$, dann ist $\phi$ eine Lösung des  \notion{Anfangswertproblems}
\[y'=f(x)\cdot g(y), \quad y(x_0)=y_0.\]
\end{enumerate}
\end{definition}
%%
\begin{example}\label{ex:sep_DG}
\begin{tabs*}[\initialtab{0}]
\tab{Beispiel a)}
$y'=x^2\cdot y^2$ ist eine separierbare Differentialgleichung auf $\R\times\R$ mit $f(x)=x^2$ und $g(y)=y^2$.
\tab{Beispiel b)}
$y'=y-y^2$ ist ebenfalls eine separierbare Differentialgleichung auf $\R\times \R$ mit $f\equiv 1$.
\tab{Beispiel c)}
$y'=y^2+x^2$ ist keine separierbare Differentialgleichung.
\end{tabs*}
\end{example}
Die Lösung separierbarer Gleichungen beruht im wesentlichen auf Angabe geeigneter Stammfunktionen,
also auf Integralrechung.
\begin{theorem}[Separierbares Anfangswertproblem]\label{thm:sep_DG}
Es sei 
\[y'=f(x)g(y) \text{ mit } y(x_0)=y_0\]
ein separierbares Anfangswertproblem, und weiter sei $g(y_0)\neq 0$.
(Insbesondere sind also $f:I\to\R$ und $g:J\to \R$ stetige Funktionen mit $x_0\in I$ und $y_0\in J$.)

Dann gibt es ein nichtausgeartetes Intervall $I'\subset I$, auf dem das Anfangswertproblem 
eine eindeutige Lösung $\phi$ besitzt.

Genauer: Es sei $F:I\to\R$ die Stammfunktion von $f$ mit $F(x_0)=0$, das heißt $F(x)=\int_{x_0}^xf(t)~dt$.
Sei $J'\subset J$ ein nichtausgeartetes Intervall, das $y_0$ enthält und auf dem $g$ keine Nullstelle hat,
und sei weiter $H:J'\to\R$ eine Stammfunktion von $\frac{1}{g}$ mit $H(y_0)=0$, das heißt $H(y)=\int_{y_0}^y\frac{1}{g(s)}ds$.
Dann gibt es ein Teilintervall $I'\subset I$ so, dass für alle $x\in I'$ gilt
\[H(\phi(x))=F(x).\]
Außerdem besitzt $H$ eine Umkehrfunktion $H^{-1}$, und mit dieser gilt für alle $x\in H(J')$
\[\phi(x)=H^{-1}(F(x)).\]
\end{theorem}
%%
\begin{proof*}
\begin{incremental}[\initialsteps{0}]
\step
Wir nehmen zunächst an, es gibt eine Lösung $\phi$ des Anfangswertproblems.
Dann gilt für alle $x\in I'$
\[\frac{\phi'(x)}{g(\phi(x))}=f(x).\]
Wir benutzen die \ref[content_12_substitutionsregel][Substitutionsregel]{thm:substitutionsregel} und erhalten
\[\int_{y_0}^{\phi(x)}\frac{ds}{g(s)}=\int_{x_0}^x\frac{\phi'(t)}{g(\phi(t))}dt=\int_{x_0}^xf(t)dt.\]
Also: Wenn eine Lösung existiert, dann erfüllt sie
\[H(\phi(x))=F(x).\]
\step
Nun zeigen wir die Existenz der Lösung.
Weil $\frac{1}{g}$ auf $J'$ das Vorzeichen nicht wechselt, ist $H$ streng monoton und als Stammfunktion stetig. 
Also ist $H$ umkehrbar auf seinem Bild.
Nun \emph{definieren} wir
\[\phi(x):=H^{-1}(F(x)).\]
Dann ist $\phi(x_0)=H^{-1}(0)=y_0$, und mit Hilfe der Ableitungsregeln für Umkehrfunktion und Komposition erhalten wir
\begin{align*}
\phi'(x)&=(H^{-1})'(F(x))\cdot F'(x)=\frac{1}{H'((H^{-1})(F(x)))}\cdot f(x)\\
&= \frac{1}{H'(\phi(x))}\cdot f(x)=g(\phi(x))\cdot f(x).
\end{align*}
\step
Zuletzt zeigen wir, dass dieses $\phi$ die einzige Lösung ist.
Diese Lösung erfüllt $H(\phi(x))=F(x)$. Weil $H$ umkehrbar ist, kann nur die Funktion $\phi$ diese Gleichung erfüllen.
\end{incremental}
\end{proof*}
%%
\begin{remark}\label{rem:sep_AW-probleme}
\begin{enumerate}
\item[(a)]
Man kann die größtmöglichen Teilintervalle $I'$ und $J'$, deren Existenz durch Satz \ref{thm:sep_DG} gewährleistet ist, 
genauer beschreiben:
 $J'\subset J$ ist das größte Teilintervall von $J$, das $y_0$ enthält aber keine Nullstelle von $g$.
Dann ist $I'\subset I$ das größte Teilintervall von $I$, das  $x_0$ enthält und $F(I')\subset H(J')$ erfüllt.\\
In der Praxis erhält man das maximale Existenzintervall einer lokal gefundenen Lösung oft  dadurch, 
dass man die Gültigkeitsgrenzen dieser Lösung retrospektiv bestimmt.
\item[(b)]
Der Satz macht keine Aussage über Anfangswerte $(x_0,y_0)$ mit $g(y_0)=0$. Für solche Anfangswerte ist die konstante Funktion
\[\phi:I\to\R,\quad x\mapsto \phi(x)=y_0\]
eine Lösung -- aber es kann im allgemeinen weitere Lösungen geben. 
Allerdings kann man zeigen: Wenn $g$ differenzierbar ist in $x_0$, dann ist diese konstante Lösung die einzige.
\end{enumerate}
\end{remark}
%Video
\floatright{\href{https://api.stream24.net/vod/getVideo.php?id=10962-2-10752&mode=iframe&speed=true}{\image[75]{00_video_button_schwarz-blau}}}\\
\\

%%
\begin{example}\label{ex:sep_AW-probleme}
Wir lösen Anfangswertprobleme zu den separierbaren Differentialgleichungen aus Beispiel \ref{ex:sep_DG} a) und b).
\begin{tabs*}[\initialtab{0}]
\tab{Beispiel a)}
Wir lösen das Anfangswertproblem $y'=x^2\cdot y^2$ zu $y(1)=1$.
Hier ist $f(x)=x^2$ und $g(x)=y^2$ sowie $g(y_0)=g(1)\neq 0$. Damit finden wir die Stammfunktionen
\[F(x)=\int_{1}^x f(t)\:dt=\frac{1}{3}(x^3-1)\]
und
\[H(y)=\int_1^y \frac{ds}{g(s)}=\int_1^y\frac{ds}{s^2}=-\frac{1}{y}+1.\]
Nun lösen wir
\begin{align*}
& \: H(\phi(x))=F(x)\\
\Leftrightarrow&\: -\frac{1}{\phi(x)}+1=\frac{1}{3}(x^3-1)\\
\Leftrightarrow&\:\frac{1}{\phi(x)}=\frac{1}{3}(4-x^3)\\
\Leftrightarrow&\:\phi(x)=\frac{3}{4-x^3}.
\end{align*}
Die maximalen Existenzintervalle sind $J'=(0;\infty)$, denn $1\in J'$ und die einzige Nullstelle von $g$ ist null.
Und folglich $I'=(-\infty;\sqrt[3]{4})$.

Ist man lediglich am maximalen Existenzintervall der Lösung $\phi$ selbst interessiert, so kann man auch wie folgt argumentieren.
Das maximale Definitionsintervall der Funktion $\phi$, das $x_0=1$ enthält, ist $I'=(-\infty;\sqrt[3]{4})$. 
(Der Definitionsbereich selbst ist größer.) Weil $\phi$ offenbar die Differentialgleichung für alle $x\in I'$ erfüllt und die Anfangswertbedingung erfüllt,
muss $I'$ das maximale Lösungsintervall sein.
\tab{Beispiel b)}
Wir lösen das Anfangswertproblem $y'=y-y^2$ mit $y(0)=1$. Hier ist $f\equiv 1$ und $g(y)=y-y^2$ als Polynom differenzierbar. 
Weil $g(y)=0$, hat das Anfangswertproblem nach Bemerkung \ref{rem:sep_AW-probleme} (b) nur die konstante Lösung $\phi\equiv 1$.
Das maximale Existenzintervall ist $I'=\R$.
\tab{Beispiel b')}
Wir lösen das Anfangswertproblem $y'=y-y^2$ mit $y(0)=\frac{1}{2}$.
Die Stammfunktionen sind $F(x)=\int_0^x 1\:dt=x$ und 
\[H(y)=\int_{\frac{1}{2}}^y\frac{ds}{s(1-s)}.\]
Dieses Integral lösen wir durch \ref[content_13_partialbruchzerlegung][Partialbruchzerlegung]{sec:partialbruch_int}. Es ist
\[\frac{1}{s(1-s)}=\frac{1}{s}+\frac{1}{(1-s)}\]
und somit $\tilde{H}(y)=\ln y -\ln(1-y)=\ln\frac{y}{1-y}$ eine Stammfunktion. Dann ist
\[H(y)=\tilde{H}(y)-\tilde{H}(\frac{1}{2})=\ln\frac{y}{1-y}-\ln 1= \ln\frac{y}{1-y}.\]
Damit berechnen wir die Lösung $\phi$.
\begin{align*}
&\:\ln\frac{\phi(x)}{1-\phi(x)}=x\\
\Leftrightarrow&\:\frac{\phi(x)}{1-\phi(x)}=e^x\\
\Leftrightarrow& \:
\phi(x)(1+e^x)=e^x\\
\Leftrightarrow&\:
\phi(x)=\frac{e^x}{1+e^x}.
\end{align*}
Dabei ist das maximale Intervall $J'=(0;1)\ni\frac{1}{2}$, weil $0$ und $1$ die Nullstellen von $g$ sind.
Dann ist 
\[H(J')=(\lim_{s\to 0}\ln\frac{s}{1-s};\lim_{s\to 0}\ln\frac{s}{1-s})=(-\infty;\infty)=\R.\]
Also ist das maximale Existenzintervall $I'=\R$, denn $F(\R)=J'$.
\end{tabs*}
\end{example}
%%
\begin{quickcheck}
\text{Markiere alle wahren Aussagen.}
\begin{choices}{multiple}
    \begin{choice}
      \text{$y'=x^2y^2+xy^2$ ist eine separierbare Differentialgleichung.}
      \solution{true}
    \end{choice} 
    \begin{choice}
      \text{$y'=\cos(x\cdot y)$ ist eine separierbare Differentialgleichung.}
      \solution{false}
    \end{choice}
    \begin{choice}
      \text{Die Funktion $\phi:\R_{>0}\to\R$, $x\mapsto 2x$, ist eine Lösung des Anfangswertproblems
      $y'=\frac{y^2}{x^2}$, $y(1)=2$.}
      \solution{false}
    \end{choice}
    \begin{choice}
      \text{Die Funktion $\phi:\R\to\R$, $x\mapsto e^{2x}$, ist eine Lösung des Anfangswertproblems
      $y'=2y$, $y(0)=1$.}
      \solution{true}
    \end{choice}      
  \end{choices}
\end{quickcheck}
%%
\section{Lineare Differentialgleichungen}
\begin{definition}[Lineare DGL]\label{def:lin_DGL}
Es seien $a,b:I\to\R$ stetige Funktionen auf einem nichtausgearteten Intervall $I\subset\R$.
Eine Differentialgleichung der Form
\[y'=a(x)\cdot y+b(x)\]
heißt \notion{linear}.
Ist dabei die Funktion $b$ identisch null, so heißt die Gleichung \notion{homogen}, anderenfalls \notion{inhomogen}.
\end{definition}
Homogene lineare Differentialgleichungen sind Spezialfälle von separierbaren Differentialgleichungen.
Daher können wir leicht die Lösungen linearer Differentialgleichungen bestimmen.
%%
\begin{theorem}[Lineare DGL]\label{thm:lin_DG}
Es sei $I\subset \R$ ein nichtausgeartetes Intervall, $x_0\in I$ ein Punkt, und es sei $a:I\to\R$ eine stetige Funktion.
\begin{enumerate}
\item[(a)] \notion{Homogene lineare Differentialgleichung}
Für jedes $y_0\in \R$ existiert genau eine Lösung $\phi:I\to\R$ des Anfangswertproblems 
\[y'=a(x)\cdot y, \quad y(x_0)=y_0.\]
Sie  wird gegeben durch
\begin{equation}\label{eq:lsg_homogene_lin_DG}
\phi(x)=y_0\cdot\exp\big(\int_{x_0}^x a(t)\:dt\big).
\end{equation}
\item[(b)]\notion{Inhomogene lineare Differentialgleichung}
Sei  $b:I\to\R$ eine weitere stetige Funktion. Dann besitzt das Anfangswertproblem
\[y'=a(x)\cdot y+b(x), \quad y(x_0)=y_0\]
genau eine Lösung $\psi:I\to\R$. Sie wird gegeben durch
\[\psi(x)=\phi_1(x)\cdot u(x),\]
wobei $\phi_1:I\to\R$,
\[\phi_1(x)=\exp\big(\int_{x_0}^x a(t)\:dt\big),\]
eine Lösung des homogenen Problems $y'=a(x)\cdot y$ zum Anfangswert $y(x_0)=1$ ist, und $u:I\to\R$ durch
\[u(x)=y_0+\int_{x_0}^x\frac{b(t)}{\phi_1(t)}\:dt\]
bestimmt wird.
\end{enumerate}
\end{theorem}
%%
\begin{proof*}
Die Lösung der homogenen linearen Differentialgleichung bestimmt sich aus Satz \ref{thm:sep_DG}, die
der inhomogenen durch Variation der Konstanten. Genauer:
\begin{incremental}[\initialsteps{0}]
\step
Zu (a): In dieser separierbaren Differentialgleichung ist (mit den Bezeichnungen aus Satz \ref{thm:sep_DG})
$f(x)=a(x)$ und $g(y)=y$. Ist $y_0=0$, dann ist nach Bemerkung \ref{rem:sep_AW-probleme} (b) die Lösung $\phi\equiv 0$
eindeutig bestimmt und entspricht (\ref{eq:lsg_homogene_lin_DG}).
Ist $y_0>0$, dann ist $J'=(0;\infty)$, und falls $y_0<0$ ist $J'=(-\infty;0)$. In beiden Fällen
erhält man $F(x)=\int_{x_0}^xa(t)\:dt$ und 
\[H(y)=\int_{y_0}^y\frac{ds}{s}=\ln {\vert s\vert} -\ln {\vert y_0\vert}=\ln\frac{\vert y\vert}{\vert y_0\vert}=\ln\frac{y}{y_0}.\]
Damit ergibt sich aus $H(\phi(x))=\ln\frac{\phi(x)}{y_0}=F(x)$ die Lösung $\phi$ durch
\[\phi(x)=y_0\cdot\exp(F(x))=y_0\cdot\exp\big(\int_{x_0}^xa(t)\:dt\big).\]
Das maximale Existenzintervall ist das maximale Intervall $I'$, das $x_0$ enthält und $F(I')=H(J')=\R$ erfüllt, also $I'=\R$.
\step
Zu (b): Wir zeigen zunächst, dass es höchstens eine Lösung des inhomogenen linearen Anfangswertproblems geben kann.
Denn sind $\psi$ und $\tilde{\psi}$ zwei Lösungen, dann ist $\psi-\tilde{\psi}$ eine Lösung der homogenen linearen Differentialgleichung
$y'=a(x)\cdot y$ zum Anfangswert $y(x_0)=0$. Nach Teil (a) ist $\psi-\tilde{\psi}\equiv 0$ die einzige Lösung davon. 
Also stimmen $\psi$ und $\tilde{\psi}$ überein.
\step
Um eine Lösung $\psi$ zu konstruieren, machen wir den Ansatz (die sogenannte Variation der Konstanten)
\[\psi(x)=\phi_1(x)\cdot u(x),\]
mit einer noch zu bestimmenden differenzierbaren Funktion $u$ und einer Lösung des homogenen Anfangswertproblems $y'=a(x)y$ zum Anfangswert $y(x_0)=1$, 
also $ \phi_1(x)=\exp(\int_{x_0}^x a(t)\:dt)$.
Wir differenzieren $\psi$ mit der Produktregel $\psi'(x)=\phi_1'(x)\cdot u(x)+\phi_1(x)\cdot u'(x)$ und setzen in die Differentialgleichung ein
\[\phi_1'(x)\cdot u(x)+\phi_1(x)\cdot u'(x)=a(x)\cdot \phi_1(x)\cdot u(x)+b(x).\]
Weil  $\phi_1'(x)=a(x)\phi_1(x)$, stimmen die ersten Terme der linken und rechten Seite überein, und wir erhalten
\[u'(x)=\frac{b(x)}{\phi_1(x)}.\]
(Als Exponentialfunktion hat $\phi_1$ keine Nullstellen, der Bruch ist also wohldefiniert.)
Um eine Lösung des Anfangswertproblems zu erhalten, müssen wir nur noch die Stammfunktion $u(x)=\int_{x_0}^x \frac{b(t)}{\phi_1(t)}dt+c$
jetzt passend wählen:
Es ist
\[\phi_1(x_0)\cdot( \int_{x_0}^{x_0} \frac{b(t)}{\phi_1(t)}dt+c)=c,\]
also müssen wir $c=y_0$ setzen und haben eine Lösung $\psi=\phi_1\cdot u:\R\to\R$ des Anfangswertproblems gefunden.
\end{incremental}
\end{proof*}
%%
Wir bemerken noch, dass sich die Lösung eines homogenen linearen Anfangswertproblems als Spezialfall der Lösung
des  inhomogenen linearen Anfangswertproblems gewinnen lässt, indem wir dort $b=0$ setzen.
\begin{example}
\begin{tabs*}[\initialtab{0}]
\tab{$y'=ay$}
In der homogenen linearen Differentialgleichung $y'=ay$ finden wir unser Eingangsbeispiel wieder für $a=1$.
Ist $a=0$, dann ist jede konstante Funktion Lösung der Differentialgleichung.
Beachten wir den Anfangswert $y(x_0)=y_0$, dann ist die nun eindeutige Lösung des Anfangswertproblems gegeben durch $\phi(x)=y_0$.
\\
Für $a\neq 0$ und einen Anfangswert $y(x_0)=y_0$,  ist die eindeutig bestimmte Lösung $\phi:\R\to\R$ dieses Anfangswertproblems
gegeben durch
\[\phi(x)=y_0\cdot\exp(\int_{x_0}^x a\:dt)=y_0\cdot e^{a(x-x_0)}.\]

\tab{$y'=\cos(x)\cdot y$}
Wir lösen das homogene lineare Anfangswertproblem $y'=\cos(x)\cdot y$ zum Anfangswert $y(0)=1$.
Die Lösung $\phi:\R\to\R$ ist gegeben durch
\[\phi(x)=1\cdot\exp\big(\int_0^x \cos t\:dt\big)=\exp(\sin(x)-\sin(0))=e^{\sin x}.\]
\tab{$y'=\frac{y}{x}$}
Die homogene lineare Differentialgleichung $y'=\frac{y}{x}$ betrachten wir auf dem Intervall $I=(0;\infty)$. Wir wählen den Anfangswert $y(1)=1$.
Wir finden die Lösung $\phi:I\to \R$ des Anfangswertproblems als
\[x\mapsto\phi(x)=\exp\big(\int_1^x\frac{1}{t}\:dt\big)=\exp(\ln x-\ln 1)=x.\]
Das ist auch die Lösung, die man durch einen scharfen Blick auf die Gleichung erwartet.
\tab{$y'=\frac{y}{x}+x^2+1$}
Die inhomogene lineare Differentialgleichung $ y'=\frac{y}{x}+x^2+1$ betrachten wir zum Anfangswert $y(1)=1$
auf dem Intervall $I=(0;\infty)$. Im vorigen Beispiel haben wir schon die Lösung $\phi_1:I\to\R$, $x\mapsto x$,
des homogenen Anfangswertproblems $y'=\frac{y}{x}$, $y(1)=1$ gefunden.
Somit ist die Lösung $\psi:I\to\R$ des inhomogenen Anfangswertproblems bestimmt durch
\begin{align*}\psi(x)\:&=\:x\cdot(1+\int_1^x\frac{t^2+1}{t}dt)=x\cdot(1+\int_1^xt\:dt+\int_1^x\frac{dt}{t})\\
&=\:x\cdot(1+\frac{1}{2}x^2-\frac{1}{2}+\ln x-\ln 1)\\
&=\:\frac{1}{2}x^3+x\cdot\ln x+\frac{1}{2}x.
\end{align*}
\end{tabs*}
\end{example}
%Video
\floatright{\href{https://api.stream24.net/vod/getVideo.php?id=10962-2-10753&mode=iframe&speed=true}{\image[75]{00_video_button_schwarz-blau}}}\\
\\

%%
\begin{quickcheck}
\type{input.function}
  \begin{variables}
  \function{f1}{exp(x)}
  \end{variables}
\text{Bestimme die Lösung $\phi:\R\to\R$ des Anfangswertproblems $y'=y$ mit $y(1)=e$.
Es ist $\phi(x)=$\ansref.}
\begin{answer}
\solution{f1}
\end{answer}
\end{quickcheck}
%%
\end{content}