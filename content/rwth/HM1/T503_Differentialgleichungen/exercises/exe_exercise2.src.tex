\documentclass{mumie.element.exercise}
%$Id$
\begin{metainfo}
  \name{
    \lang{en}{...}
    \lang{de}{Ü02: AWA}
  }
  \begin{description} 
 This work is licensed under the Creative Commons License Attribution 4.0 International (CC-BY 4.0)   
 https://creativecommons.org/licenses/by/4.0/legalcode 

    \lang{en}{...}
    \lang{de}{...}
  \end{description}
  \begin{components}
  \end{components}
  \begin{links}
  \end{links}
  \creategeneric
\end{metainfo}
\begin{content}
\begin{block}[annotation]
	Im Ticket-System: \href{https://team.mumie.net/issues/21697}{Ticket 21697}
\end{block}
\usepackage{mumie.ombplus}

\title{\lang{de}{Ü02: AWA}}
Bestimmen Sie die Lösung des Anfangswertproblems
\[y'=\frac{y}{x+1}+x^2+2\quad \text{mit Anfangswert }\: y(1)=2.\]
%%
\begin{tabs*}[\initialtab{0}\class{exercise}]
\tab{Lösung}
Die Funktion $\psi:(-1;\infty)\to\R$,
\[x\mapsto\psi(x)=\frac{1}{4}(x+1)\big(x^2-2x+5+3\ln(x+1)-3\ln(2)\big)\] 
ist die eindeutig bestimmte Lösung des Anfangswertproblems mit maximalem Existenzintervall $(-1;\infty)$.
\begin{incremental}[\initialsteps{1}]
\step
\notion{Begründung:}
\step
Es handelt sich um eine inhomogene lineare Differentialgleichung. 
Diese besitzt eine eindeutig bestimmte Lösung auf dem maximalen Definitionsintervall $I$ ihrer Koeffizienten, das den Anfangspunkt $x_0$ enthält.
Weil der Anfangspunkt $x_0=1>-1$, ist dieses Intervall gegeben durch den Bereich \glqq rechts\grqq{} von der Definitionlücke von
$\frac{1}{x+1}$, also $I=(-1;\infty)$.
\step
Die Lösung $\psi=\phi_1\cdot u$ setzt sich zusammen aus den Funktionen
\[\phi_1:\:x\mapsto\phi_1(x)=\exp\big(\int_1^x\frac{dt}{t+1}\big)\]
und
\[u:\:x\mapsto u(x)=2+\int_1^x\frac{t^2+2}{\phi_1(t)}\:dt.\]
\step
Wir berechnen diese Funktionen
\[\phi_1(x)=\exp\big(\ln(x+1)-\ln(2)\big)=\frac{1}{2}(x+1),\]
sowie
\begin{align*}
u(x)\:&=\: 2+2\cdot\int_1^x\frac{t^2+2}{t+1}=2+2\int_1^x(t-1)\:dt+2\int_1^x\frac{3}{t+1}\:dt\\
&=\: 2+x^2-2x-2(\frac{1}{2}-1)+6\big(\ln(x+1)-\ln 2\big)\\
&=\: x^2-2x+3+6\ln(x+1)-6\ln 2.
\end{align*}
\step
Damit ergibt sich die Lösung $\psi$ durch
\[\psi(x)=\frac{x+1}{2}\big(x^2-2x+3+6\ln(x+1)-6\ln 2\big).\]
\step
\notion{Zur Probe} setzen wir den Anfangspunkt ein, $\psi(1)=\frac{2}{2}(1-2+3+6\ln 2-6\ln 2)=2$,
und berechnen die Ableitung
\begin{align*}
\psi'(x)\:&=\:\frac{1}{2}\big(x^2-2x+3+6\ln(x+1)-6\ln 2\big)+\frac{x+1}{2}\big(2x-2+\frac{6}{x+1}\big)\\
&=\:\frac{1}{2}\big(x^2-2x+3+6\ln(x+1)-6\ln 2\big)+x^2+2\\
&=\:\frac{\psi(x)}{x+1}+x^2+2.
\end{align*}
Also haben wir mit $\psi$ in der Tat die Lösung des Anfangswertproblems gefunden.
\end{incremental}
\end{tabs*}
\end{content}

