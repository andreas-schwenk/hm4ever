\documentclass{mumie.element.exercise}
%$Id$
\begin{metainfo}
  \name{
    \lang{en}{...}
    \lang{de}{Ü01: DGL prüfen}
  }
  \begin{description} 
 This work is licensed under the Creative Commons License Attribution 4.0 International (CC-BY 4.0)   
 https://creativecommons.org/licenses/by/4.0/legalcode 

    \lang{en}{...}
    \lang{de}{...}
  \end{description}
  \begin{components}
  \end{components}
  \begin{links}
  \end{links}
  \creategeneric
\end{metainfo}
\begin{content}
\begin{block}[annotation]
	Im Ticket-System: \href{https://team.mumie.net/issues/21688}{Ticket 21688}
\end{block}
\usepackage{mumie.ombplus}

\title{\lang{de}{Ü01: DGL prüfen}}

Welche Funktion $\psi_j:\R_{>0}\to\R$ löst das Anfangswertproblem $y'=\frac{y}{x}+\frac{1}{x^2}$ mit $y(1)=\frac{3}{2}$?
\begin{align*}
\psi_1(x)\:&=\:2x^2-\frac{1}{2x^2},\\
\psi_2(x)\:&=\:2x-\frac{1}{2x}, \\
\psi_3(x)\:&=\:x+\frac{1}{2x}  \text{ oder}\\
\psi_4(x)\:&=\:x-\frac{1}{2x}  \text{ ?}
\end{align*}

\begin{tabs*}[\initialtab{0}\class{exercise}]
\tab{Antwort}
Die Funktion $\psi_2$ löst das Anfangswertproblem.
\tab{Lösung}
\begin{incremental}[\initialsteps{1}]
\step
Eine Lösung $\psi$ des Anfangswertproblems ist dadurch gekennzeichnet, 
dass sie sowohl die Anfansgwertbedingung $\psi(1)=\frac{3}{2}$ 
als auch die Differentialgleichung $\psi'=\frac{\psi}{x}+\frac{1}{x^2}$ erfüllt.
\step
Es ist also zu prüfen, welche der Funktionen beide Bedingungen erfüllt/erfüllen.
\step 
Hier erfüllen offensichtlich die ersten drei Funktionen $\psi_1$, $\psi_2$, $\psi_3$ die Anfangswertbedingung.
$\psi_4$ erfüllt die Anfangswertbedingung nicht und ist damit keine Lösung. (Aber $\psi_4$ erfüllt die Differentialgleichung.)
Also leiten wir die verbliebenen Funktionen ab, um die Differentialgleichung zu überprüfen.
\step
Es gilt
\[\psi'_1(x)=4x+\frac{1}{x^3} \text{ sowie } \frac{\psi_1(x)}{x}+\frac{1}{x^2}=2x-\frac{1}{2x^3}+\frac{1}{2x^3}.\]
Also ist $\psi_1'(x)\neq \frac{\psi_1(x)}{x}+\frac{1}{x^2}$ , und damit ist $\psi_1$ keine Lösung.
\step
Es gilt
\[\psi'_2(x)=2+\frac{1}{2x^2} \text{ sowie } \frac{\psi_2(x)}{x}+\frac{1}{x^2}=2+\frac{1}{2x^2}.\]
Also ist $\psi_2'(x)= \frac{\psi_2(x)}{x}+\frac{1}{x^2}$ , und damit ist $\psi_2$ eine die Lösung.
\step
Weil das Anfangswertproblem dieser linearen Differentialgleichung eindeutig lösbar ist, können wir an dieser Stelle aufhören:
$\psi_3\neq \psi_2$ kann nicht auch noch eine Lösung sein.
Wenn wir uns an diese Eigenschaft nicht mehr erinnern, können wir auch noch direkt nachprüfen, dass $\psi_3$ die Differentialgleichung
nicht erfüllt.
\step
Es gilt
\[\psi'_3(x)=x-\frac{1}{2x^2} \text{ sowie } \frac{\psi_1(x)}{x}+\frac{1}{x^2}=1+\frac{3}{2x^3}.\]
Also ist $\psi_3'(x)\neq \frac{\psi_3(x)}{x}+\frac{1}{x^2}$ , und damit ist $\psi_3$ keine Lösung.
\end{incremental}
\end{tabs*}
\end{content}

