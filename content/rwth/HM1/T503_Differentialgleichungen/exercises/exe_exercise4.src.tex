\documentclass{mumie.element.exercise}
%$Id$
\begin{metainfo}
  \name{
    \lang{en}{...}
    \lang{de}{Ü04: AWA}
  }
  \begin{description} 
 This work is licensed under the Creative Commons License Attribution 4.0 International (CC-BY 4.0)   
 https://creativecommons.org/licenses/by/4.0/legalcode 

    \lang{en}{...}
    \lang{de}{...}
  \end{description}
  \begin{components}
  \end{components}
  \begin{links}
  \end{links}
  \creategeneric
\end{metainfo}
\begin{content}
\begin{block}[annotation]
	Im Ticket-System: \href{https://team.mumie.net/issues/21986}{Ticket 21986}
\end{block}

\title{\lang{de}{Ü04: AWA}}
Bestimmen Sie die Lösung des Anfangswertproblems
\[y'=y+x\cdot y^2\quad \text{mit Anfangswert }\: y(1)=1.\]
%%
\begin{tabs*}[\initialtab{0}\class{exercise}]
\tab{Lösung}
Die Lösung $\phi:I\to\R$ dieser Bernoulli-Anfangswertaufgabe ist gegeben durch
\[\phi(x)=(e^{1-x}-x+1)^{-1}\]
mit maximalem Existenzintervall $I=(-infty;x_1)$, wobei $x_1$ 
die eindeutig bestimmte Lösung der Gleichung $e^{1-x}=x-1$ ist.
\begin{incremental}[\initialsteps{1}]
\step
\notion{Begründung:}
\step
Die Differentialgleichung ist vom Bernoulli-Typ $y'=a(x)y+b(x)y^k$ mit Koeffizienten $a(x)=1$ und $b(x)=x$ sowie Exponent $k=2$.
Wir setzen formal $z=y^{-1}$ und erhalten ein lineares Anfangswertproblem
\[z'=-z-x,\:\text{ mit Anfangswert } z(x_0)=\frac{1}{y_0}=1.\]
Das lösen wir zunächst: Die Lösung des homogenen Problems $z'=-z$ mit $z(1)=1$ ist
\[\phi_1(x)=e^{-x+1}.\]
Damit berechnen wir den zweiten Faktor der Lösung des inhomogenen Problems mit partieller Integration
\begin{align*}
u(x)\:&=&\:1+\int_1^x\frac{-t}{e^{-x+1}}dt=1-e^{-1}\int_1^xte^t\:dt\\
&=&\:1-e^{-1}([te^t]_1^x-\int_1^x e^t\:dt)\\
&=&\:1-(x-1)e^{x-1}.
\end{align*}
Das lineare Anfangswertproblem in $z$ hat also die Lösung
\[\psi(x)=\phi_1(x)\cdot u(x)=e^{1-x}-(x-1).\]
\step
Daraus erhalten wir die formale Lösung der Bernoulli-Anfangswertaufgabe als
\[\phi(x)=\frac{1}{\psi(x)}.\]
\step
Wir müssen allerdings noch ein Intervall $I$ finden, das $x_0=1$ enthält und auf dem $\phi$ wohldefiniert ist.
Die Funktion $\psi$ hat eine Nullstelle $x_1$. Sie begrenzt das Existenzintervall der Lösung $\phi$.
Es gilt $x_0=1<x_1$, denn $\psi(x_0)=1>0$
und $\psi$ ist streng monoton fallend (denn $\psi'(x)=-e^{1-x}-1<0$). Also ist $\phi$ definiert auf $(-\infty;x_1)$.
\end{incremental}
\end{tabs*}

\end{content}

