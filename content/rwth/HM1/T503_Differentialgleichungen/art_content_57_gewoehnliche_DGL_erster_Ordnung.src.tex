%$Id:  $
\documentclass{mumie.article}
%$Id$
\begin{metainfo}
  \name{
    \lang{en}{...}
    \lang{de}{Gewöhnliche Differentialgleichungen erster Ordnung}
   }
  \begin{description} 
 This work is licensed under the Creative Commons License Attribution 4.0 International (CC-BY 4.0)   
 https://creativecommons.org/licenses/by/4.0/legalcode 

    \lang{en}{...}
    \lang{de}{...}
  \end{description}
  \begin{components}
\component{generic_image}{content/rwth/HM1/images/g_img_00_video_button_schwarz-blau.meta.xml}{00_video_button_schwarz-blau}
\component{generic_image}{content/rwth/HM1/images/g_tkz_T503_Pendulum.meta.xml}{T501_Pendulum}
\end{components}
  \begin{links}
\link{generic_article}{content/rwth/HM1/T404_Eigenwerte,_Eigenvektoren/g_art_content_11_eigenwerte.meta.xml}{content_11_eigenwerte}
\link{generic_article}{content/rwth/HM1/T503_Differentialgleichungen/g_art_content_56_Separierbare_Differentialgleichungen.meta.xml}{content_56_Separierbare_Differentialgleichungen}
\end{links}
  \creategeneric
\end{metainfo}

\begin{content}
\begin{block}[annotation]
	Im Ticket-System: \href{https://team.mumie.net/issues/21659}{Ticket 21659}
\end{block}
\usepackage{mumie.ombplus}
\ombchapter{3}
\ombarticle{2}
\usepackage{mumie.genericvisualization}
\begin{block}[info-box]
\tableofcontents
\end{block}

\title{Gewöhnliche Differentialgleichungen erster Ordnung}
\section{Gewöhnliche Differentialgleichungen erster Ordnung}
Im Abschnitt über \ref[content_56_Separierbare_Differentialgleichungen][separierbare und lineare Differentialgleichungen]{sec:sep_lin_DGL} haben wir bereits
einige einfache Typen von Differentialgleichungen kennengelernt.
Jetzt werden wir allgemein Systeme von gewöhnlichen Differentialgleichungen erster Ordnung definieren, 
eine fundamentale Aussage
über ihre Lösbarkeit treffen und Beispiele angeben.
\begin{definition}[Gewöhnliche DGL erster Ordnung]\label{def:ODE}
\begin{enumerate}
\item[(a)]
Es sei $D\subset \R\times \R^n$ eine offene Teilmenge, und es sei $f:D\to\R^n$, $(x,y)\mapsto f(x,y)$ eine stetige Funktion.
(Dabei seien $x\in\R$ und $y\in\R^n$.)
Dann heißt
\[y'=f(x,y)\]
ein \notion{System von $n$ gewöhnlichen Differentialgleichungen erster Ordnung}.
\item[(b)]
Eine  \notion{Lösung} dieses Systems ist eine Kurve $\phi:I\to\R^n$, wobei $I$ ein nichtausgeartetes Intervall ist, mit den folgenden Eigenschaften.
\begin{enumerate}
\item[(i)] $\phi$ ist differenzierbar.
\item[(ii)] Für alle $x\in I$ ist $(x,\phi(x))\in D$.
\item[(iii)] Für alle $x\in I$ gilt $\phi'(x)=f(x,\phi(x))$.
\end{enumerate}
\item[(c)]
Für $(x_0,y_0)\in D$ ist $\phi$ eine Lösung des \notion{Anfangwertproblems}
\[y'=f(x,y), \quad y(x_0)=y_0,\]
wenn $\phi$ eine Lösung des Systems ist und $\phi(x_0)=y_0$ erfüllt.
\end{enumerate}
\end{definition}
Oft spricht man lediglich von einer (mehrdimensionalen) gewöhnlichen Differentialgleichung ersten Grades, statt von einem System.
Dabei bezieht sich der Begriff \glqq erster Ordnung\grqq{} darauf, dass die Differentialgleichung nur erste Ableitungen enthält.
\glqq Gewöhnlich\grqq{} heißt eine Differentialgleichung, wenn wie hier nur Ableitungen nach einer Variablen vorkommen. 
%%
\begin{quickcheck}
\text{Markiere alle (Systeme von) gewöhnlichen Differentialgleichungen erster Ordnung.}
\begin{choices}{multiple}
    \begin{choice}
      \text{$y'=x^2y^2+xy^2$.}
      \solution{true}
    \end{choice} 
    \begin{choice}
      \text{$\begin{pmatrix}\frac{\partial y}{\partial x_1}\\\frac{\partial y}{\partial x_2}\\
      \frac{\partial y}{\partial x_3}\end{pmatrix}=(x_1^2+x_2x_3+\cos x_3)\cdot y+18x_2^5$.}
      \solution{false}
    \end{choice}
    \begin{choice}
      \text{$\begin{pmatrix}y_1'\\y_2'\\y_3'\end{pmatrix}=\begin{pmatrix}18x^2y_3\\x_1x_3y_2
      \\x_2^3y_3y_1+1\end{pmatrix}$.}
      \solution{true}
    \end{choice}
    \begin{choice}
      \text{$y'-y''=y$.}
      \solution{false}
    \end{choice}      
  \end{choices}
\end{quickcheck}
%%
Das Fundament zur Untersuchung gewöhnlicher Differentialgleichungen ist der folgende Satz.
\begin{theorem}[Lösbarkeit des gewöhnlichen Anfangswertproblems erster Ordnung]\label{thm:ODE}
Es sei $D\subset\R\times\R^n$ eine offene Teilmenge, und es sei $f:D\to\R^n$, $(x,y)\mapsto f(x,y)$, eine stetige Funktion.
Wenn alle partiellen Ableitungen $\frac{\partial f_i}{\partial y_j}$, $1\leq i,j\leq n$, existieren und stetig sind,
dann besitzt das Anfangswertproblem 
\[y'=f(x,y),\quad y(x_0)=y_0\]
für jeden Anfangswert $(x_0,y_0)\in D$ eine eindeutig bestimmte Lösung $\phi:I\to\R^n$ auf einem geeigneten Intervall $I$, das $x_0$ enthält.
\end{theorem}
%%
Der Beweis des Satzes liegt jenseits der Möglichkeiten dieses Kurses. Er ist aber äußerst nützlich für die Anwendung. 
Als Existenz- und Eindeutigkeitsaussage legt er die Grundlage für eine näherungsweise, numerische Bestimmung der Lösung. 
%%
Im folgenden illustrieren wir anhand verschiedener Beispiele die Vielfalt gewöhnlicher Differentialgleichungen.
Bei allen ist die Existenz und Eindeutigkeit der Lösung durch Satz \ref{thm:ODE} garantiert. 
Die Lösungsmethoden sind allerdings verschieden,
manchmal ist es gar nicht möglich, eine explizite Lösungsformel anzugeben.
%
\begin{remark}\label{rem:existenz_auf_intervallen}
Weshalb beschreiben wir die Lösung eines Anfangswertproblems stets auf einem Intervall $I$,
obwohl der Definitionsbereich des Funktionsterms der Lösung oft viel größer ist als dieses Intervall und auch dort
die Differentialgleichung erfüllt ist?
Der Grund hierfür liegt in der Eindeutigkeit der Lösung, die nur durch Erfüllung der Anfangswertbedingung gegeben ist. Die Differentialgleichung selbst
hat viele Lösungen, durch die Wahl des Anfangswertes suchen wir eine davon aus.
Wollen wir das Anfangswertproblem auf $I\cup I_1$ betrachten, wobei $I_1$ ein weiteres Intervall
mit $I_1\cap I=\emptyset$ ist, dann können wir auf $I_1$ \emph{irgendeine} Lösung wählen und durch sie die (eindeutige) 
Lösung auf $I$ des Anfangswertproblems  fortsetzen auf $I_1$. 
Auf solchen \glqq nicht zusammenhängenden\grqq{} Mengen wird die Lösung also niemals eindeutig bestimmt sein.
\end{remark}
%Video
\floatright{\href{https://api.stream24.net/vod/getVideo.php?id=10962-2-10754&mode=iframe&speed=true}{\image[75]{00_video_button_schwarz-blau}}}\\
\\

%%
\begin{example}[Separierbare und lineare Differentialgleichungen]
Separierbare und lineare Differentialgleichungen sind gewöhnliche Differentialgleichungen erster Ordnung.
Ihre Lösungsmethodik  haben wir im 
\ref[content_56_Separierbare_Differentialgleichungen][vorigen Abschnitt]{sec:sep_lin_DGL} kennengelernt.
\end{example}
%%
\begin{example}[Bernoullische Differentialgleichungen]\label{ex:bernoullische-dgl}
Es sei $D\subset \R\times \R$.
Bernoullische Differentialgleichungen sind vom Typ 
\[y'=a(x)\cdot y+b(x)\cdot y^k,\]
also $f:D\to\R$, $(x,y)\mapsto a(x)\cdot y+b(x)\cdot y^k$, wobei $a,b:I\to\R$ stetige Funktionen auf einem nichtausgearteten
Intervall sind und der Exponent $k\in\Z\setminus\{0,1\}$.
Insbesondere sind Bernoullische Differentialgleichungen weder linear noch separierbar (außer im Fall $a=\text{const.}\cdot b$).

Das Anfangswertproblem mit $(x_0,y_0)\in D$ und $y_0>0$ wird formal gelöst, indem man substituiert $z=y^{1-k}$.
So erhält man  mit Hilfe der formalen Ableitung $z'=\frac{(1-k)y'}{y^{k}}$
eine lineare Differentialgleichung in $z$
\[z'=(1-k)a(x)z+(1-k)b(x),\]
für die man mit dem \ref[content_56_Separierbare_Differentialgleichungen][entsprechenden Satz]{thm:lin_DG} eine Lösung $\psi$ zum Anfangswert 
$(x_0,y_0^{1-k})$ findet. 
Schränkt man den Definitionsbereich
von $\psi$ so ein, dass $\psi$ keine Nullstellen hat, so ist -- wenn wohldefiniert -- $\phi=\psi^{\frac{1}{1-k}}$ eine Lösung des ursprünglichen Problems. 

%Video
Das folgende Video enthält neben der Einführung in die Bernoullischen Differentialgleichungen 
ein Beispiel dafür.

\center{\href{https://api.stream24.net/vod/getVideo.php?id=10962-2-10755&mode=iframe&speed=true}
{\image[75]{00_video_button_schwarz-blau}}}\\
\\

\end{example}

%%
\begin{quickcheck}
\type{input.function}
  \begin{variables}
  \function{Z}{1/y}
  \function{DG}{-x*z-1}
  \end{variables}
\text{Wir betrachten die Bernoullische Differentialgleichung $y'=xy+y^2$.
Gib eine Funktion $z$ an, mit deren Hilfe die Differentialgleichung in eine lineare überführt werden kann.\\
Mit $z=$\ansref wird die Differentialgleichung zu $z'=$\ansref.}
\begin{answer}
\solution{Z}
\end{answer}
\begin{answer}
\solution{DG}
\end{answer}
\end{quickcheck}
%%
\begin{definition}[Autonome DGL]\label{def:autonom_DGL}
(Systeme von) Differentialgleichungen $y'=f(x,y)$ heißen \notion{autonom}, wenn der Funktionsterm $f(x,y)$ nicht explizit von $x$ abhängt.
\end{definition}
%%
\begin{example}[Autonome lineare Differentialgleichungen]
Es sei $A\in M(n,n,;\R)$ eine quadratische Matrix. 
Dann ist
\[y'=A\cdot y.\]
eine autonome lineare Differentialgleichung.
Hier stellt die lineare Algebra (in voller Allgemeinheit allerdings weiterführende  als die, 
die in diesem Kurs behandelt  wird)  eine allgemeine Lösungsmethodik bereit.
Wir beschränken uns auf einen speziellen Fall, der mit den Kenntnissen aus dem 
\ref[content_11_eigenwerte][Kapitel 
Eigenwerte und Eigenvektoren]{sec:eigenwerte_eigenvektoren} des Vertiefungsteils 3b Lineare Algebra verstanden werden kann.
Beginnen wir mit einem Beispiel:
\begin{incremental}[\initialsteps{0}]
\step
 Sei
\[A=\begin{pmatrix}0&1\\4&0\end{pmatrix}\in\R^2.\]
Wir bestimmen die Eigenwerte und Eigenvektoren von $A$. Das charakteristische Polynom
ist
\[\det(A-X\cdot E_2)=X^2-4=(X-2)(X+2).\]
Also sind die Eigenwerte $\lambda=2$ und $\mu=-2$. Dazu findet man Eigenverktoren
$v_\lambda=\begin{pmatrix}1\\2\end{pmatrix}$ und $v_\mu=\begin{pmatrix}1\\-2\end{pmatrix}$.
Man rechnet sofort nach, dass für alle $\alpha,\beta\in\R$
\[\phi_{\alpha,\beta}(x)=\alpha e^{\lambda x}\cdot v_\lambda +\beta e^{\mu x}\cdot v_\mu\]
eine Lösung der Gleichung $y'=A\cdot y$ ist.
Die Koeffizienten $\alpha$ und $\beta$ können dann dafür genutzt werden, eine Anfangswertbedingung $y(x_0)=y_0$ 
zu erfüllen: Das lineare Gleichungssystem 
\[\alpha e^{\lambda x_0}\cdot v_\lambda +\beta e^{\mu x_0}\cdot v_\mu=y_0\]
besitzt eine eindeutige Lösung $(\alpha,\beta)\in\R^2$, weil Eigenvektoren zu verschiedenen Eigenwerten linear unabhängig sind.
\step
Dieses Konzept funktioniert immer dann, wenn die Matrix $A\in M(n,n;\R)$ über  $n$ linear unabhängige Eigenvektoren verfügt, 
also zum Beispiel, wenn $A=A^T$ symmetrisch ist.
Sind $v_{\lambda_j}$ linear unabhängige Eigenvektoren zu reellen Eigenwerten $\lambda_j$, $j=1,\ldots, n$,
dann ist
\[\phi(x)=\sum_{j=1}^n\alpha_j e^{\lambda_j x}v_{\lambda_j}\]
die Lösung des Anfangswertproblems $y'=Ay$, $y(x_0)=y_0$, wenn $(\alpha_1,\ldots,\alpha_n)\in\R^n$ die eindeutig bestimmte
Lösung des LGS $\sum_{j=1}^n\alpha_j e^{\lambda_j x_0}v_{\lambda_j}=y_0$ ist.
\end{incremental}

%Video
\center{\href{https://api.stream24.net/vod/getVideo.php?id=10962-2-10756&mode=iframe&speed=true}{\image[75]{00_video_button_schwarz-blau}}}\\
\\
\end{example}

%%
\begin{quickcheck}
\type{input.function}
\begin{variables}
\function{phi}{exp(3*x)*v+2exp(x)*w}
\end{variables}
\text{Bestimme die Lösung $\phi:\R\to\R^2$ des Anfangswertproblems $y'=\begin{pmatrix}3&0\\0&1\end{pmatrix}\cdot y\:$ mit $y(0)=v+2w$.
Schreibe sie bezüglich der Basis $v=\begin{pmatrix}1\\0\end{pmatrix}$, $w=\begin{pmatrix}0\\1\end{pmatrix}$.
Es ist $\phi(x)=$\ansref.}
\begin{answer}
\solution{phi}
\end{answer}


\end{quickcheck}
%%
\begin{example}[Fadenpendel]\label{ex:fadenpendel}
\begin{center}
\image{T501_Pendulum}
\end{center}
Die Bewegung eines ungedämpften Fadenpendels wird beschrieben durch den Winkel $\phi(t)$ und die 
Winkelgeschwindigkeit $\omega(t)$, die die Änderung des Winkels mit der Zeit $t$ angibt: $\dot{\phi}(t)=\omega(t)$.
(Hier ersetzt also die Zeitvariable $t$ unsere übliche Variable $x$, 
und wir  verwenden wie in der Physik üblich die Schreibweise $\dot{\phi}$ für die Ableitung nach $t$.)
Die Änderung der Winkelgeschwindigkeit ist proportional zu dem Anteil $F_{tan}$ der Gewichtskraft $F_G$, 
der tangential an die Pendelbahn wirkt.
Setzen wir die Proportionalitätskonstante der Einfachheit halber gleich eins, dann erhalten wir $\dot{\omega}(t)=-\sin(\phi(t))$.
Definieren wir nun $y=(y_1,y_2)^T=(\phi,\omega)^T$, dann ist die Differentialgleichung des Fadenpendels $y'=f(t,y)$ mit  
\[f(t,y)=\begin{pmatrix}y_2\\-\sin y_1\end{pmatrix}\]
mit $D=\R\times\R^2$. 
\\
Während man für sehr kleine Auslenkungen des Pendels noch einfache Näherungslösungen angeben kann, ist für realistische
größere Auslenkungen die Lösung der Differentialgleichung nicht mehr elementar erreichbar, 
sondern benötigt elliptische Funktionen.
%%
% Infinitesimale Auslenkungen: Lösung diskutieren??
%
\end{example}
%%
\begin{example}[Lorenz-Gleichung]
Es sei $D=\R\times \R^2$. Für positive Parameter $\sigma,\rho,\beta>0$ sei $f:D\to \R$
\[f(x,y)=\begin{pmatrix}\sigma(y_2-y_1)\\\rho y_1-y_2-y_1y_3\\y_1y_2-\beta y_3\end{pmatrix}.\]
Die Differentialgleichung $y'=f(x,y)$ ist autonom. 
Sie heißt Lorenz-Gleichung und wurde bei der Modellierung von Strömungsphänomenen entdeckt. 
Für bestimmte Parameterbereiche zeigen die Lösungen ein sehr wildes, quasi chaotisches Verhalten.
Explizite Lösungsformeln gibt es für diese Differentialgleichung nicht, obwohl die Funktion $f$ nur polynomial vom Grad $2$ ist.
\end{example}\end{content}