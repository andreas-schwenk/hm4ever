%$Id:  $
\documentclass{mumie.article}
%$Id$
\begin{metainfo}
  \name{
  \lang{en}{...}
  \lang{de}{Matrizen}
  }
  \begin{description} 
 This work is licensed under the Creative Commons License Attribution 4.0 International (CC-BY 4.0)   
 https://creativecommons.org/licenses/by/4.0/legalcode 

    \lang{en}{...}
    \lang{de}{...}
  \end{description}
  \begin{components}
  \end{components}
  \begin{links}
    \link{generic_article}{content/rwth/HM1/T605_Lineare_Algebra/g_art_content_16_vektoren.meta.xml}{content_16_vektoren}
  \end{links}
  \creategeneric
\end{metainfo}
\begin{content}
\begin{block}[annotation]
	Im Ticket-System: \href{https://team.mumie.net/issues/32568}{Ticket 32568}
\end{block}
\usepackage{mumie.ombplus}
\ombchapter{4}
\ombarticle{1}
\lang{de}{\title{Matrizen}}
\begin{block}[info-box]
  \tableofcontents
\end{block}
\section{Matrizen}
\begin{definition}
\begin{enumerate}
\item[(i)]
Eine \notion{Matrix (Pl. Matrizen)} ist ein rechteckiges Zahlenschema 
\[
A=\begin{pmatrix}a_{11}&a_{12}&\cdots&a_{1n}\\a_{21}&a_{22}&\cdots&a_{2n}\\
\vdots&\vdots&&\vdots\\
a_{m1}&a_{m2}&\cdots&a_{mn}\end{pmatrix}
\]
mit $m$ Zeilen und $n$ Spalten. Dabei heißt $m\times n$ die \notion{Größe} (oder auch Dimension) der Matrix.
Man schreibt $A=(a_{ij})_{1\leq i\leq m,1\leq j\leq n}$, 
bzw. kurz $A=(a_{ij})_{ij}=(a_{ij})$, wenn die Größe der Matrix klar ist.
\item[(ii)]
Die Menge aller $(m\times n)$-Matrizen wird mit $M_{m,n}(\R)$ bezeichnet.
\item[(iii)] Die \notion{Koeffizienten oder Einträge} $a_{ij}$ sind reelle Zahlen.
\item[(iv)] Für eine Matrix $A=(a_{ij})$ heißt $i$ der \notion{Zeilenindex} und $j$ der \notion{Spaltenindex}.\\
Merkregel: "Zeilen zuerst, Spalten später."
Der Vektor $\begin{pmatrix}a_{1j}\\\vdots\\a_{mj}\end{pmatrix}$ heißt der $j$-te \notion{Spaltenvektor} von $A=(a_{ij})$,
und $\begin{pmatrix}a_{i1}&\cdots&a_{in}\end{pmatrix}$ der $i$-te \notion{Zeilenvektor}.
\end{enumerate}
\end{definition}
\begin{example}
\begin{enumerate}
\item[(i)]
Es ist $A=\begin{pmatrix}1&2&3\\4&5&6\end{pmatrix}$ eine $(2\times 3)$-Matrix. 
Der Eintrag in der ersten Zeile und zweiten Spalte ist $a_{12}=2$, 
während Koeffizient $a_{21}=4$ der Eintrag in der zweiten Zeile und ersten Spalte bezeichnet.
\item[(ii)]
Eine $(1\times n)$-Matrix ist eine Matrix mit einer Zeile und $n$ Spalten, also ein Zeilenvektor.
\item[(iii)] 
Eine $(m\times 1)$-Matrix hat $m$ Zeilen und eine Spalte, ist ein Spaltenvektor der Dimension $m$. 
Somit ist $M_{m,1}(\R)=\R^m$.
\end{enumerate}
\end{example}
\begin{quickcheck}
\type{input.number}
\begin{variables}
\function{m}{4}
\function{n}{3}
\function{a}{7}
\end{variables}
\text{Die Matrix $A=\begin{pmatrix}2&3&1\\4&-1&5\\3&7&-2\\0&1&4\end{pmatrix}$ ist eine $($\ansref$\times$\ansref$)$-Matrix mit
Koeffizient $a_{32}=$\ansref.}
\begin{answer}
\solution{m}
\end{answer}
\begin{answer}
\solution{n}
\end{answer}
\begin{answer}
\solution{a}
\end{answer}
\end{quickcheck}
Einige häufig benutzte Matrizen erhalten eigene Namen:
\begin{definition}
\begin{enumerate}
\item[(i)] Die $(m\times n)$-Matrix, deren Koeffizienten alle gleich Null sind heißt \notion{Nullmatrix}
\[
0_{mn}=\begin{pmatrix}0&\cdots&0\\\vdots&&\vdots\\0&\cdots&0\end{pmatrix}.
\]
\item[(ii)] Gilt $m=n$, so spricht man von einer \notion{quadratischen} Matrix.
\item[(iii)] Eine spezielle quadratische $(n\times n)$-Matrix ist die \notion{Einheitsmatrix} $E_n=(e_{ij})$. 
Sie hat Einsen auf der (Haupt-)Diagonalen, $e_{ii}=1$ (für $i=1,\ldots,n$), und Nullen sonst, $e_{ij}=0$ für $i\neq j$. 
\[
E_n=\begin{pmatrix} 1&0&\cdots&\cdots&0\\0&1&0&\cdots&0\\\vdots&0&\ddots&\ddots&0\\\vdots&&\ddots&\ddots&\vdots\\
0&\cdots&\cdots&0&1\end{pmatrix}.
\]
\item[(iv)]
Allgemeiner ist eine \notion{Diagonalmatrix} eine quadratische Matrix, 
falls $a_{ij}=0$ für $i\neq j$ für (für $i,j=1,\ldots,n$), das heißt
\[
A=\begin{pmatrix} a_{11}&0&\cdots&\cdots&0\\0&a_{22}&0&\cdots&0\\\vdots&0&\ddots&\ddots&0\\\vdots&&\ddots&\ddots&\vdots\\
0&\cdots&\cdots&0&a_{nn}\end{pmatrix}=\text{diag}(a_{11},\ldots,a_{nn})\in M_{n,n}(\R).
\]
\item[(v)]
Eine quadratische Matrix $A=(a_{ij})\in M_{n,n}(\R)$ heißt \notion{obere Dreiecksmatrix}, 
wenn alle Koeffizienten unterhalb der Diagonalen verschwinden, also $a_{ij}=0$ für $i>j$.
Eine quadratische Matrix $B=(b_{ij})\in M_{n,n}(\R)$ heißt \notion{untere Dreiecksmatrix}, 
wenn alle Koeffizienten oberhalb der Diagonalen verschwinden, also $b_{ij}=0$ für $i>j$.
\[
A=\begin{pmatrix} \textcolor{#0066CC}{a_{11}}&\textcolor{#0066CC}{a_{12}}&\textcolor{#0066CC}{\cdots}&\textcolor{#0066CC}{\cdots}&\textcolor{#0066CC}{a_{1n}}
\\0&\textcolor{#0066CC}{a_{22}}&&&\textcolor{#0066CC}{\ldots}\\0&\ddots&\textcolor{#0066CC}{\ddots}&&\textcolor{#0066CC}{\vdots}
\\
\vdots&&\ddots&\textcolor{#0066CC}{\ddots}&\textcolor{#0066CC}{\vdots}\\
0&\cdots&\cdots&0&\textcolor{#0066CC}{a_{nn}}\end{pmatrix}
\text{ bzw. }
B=\begin{pmatrix} \textcolor{#0066CC}{b_{11}}&0&\cdots&\cdots&0\\\textcolor{#0066CC}{a_{21}}&\textcolor{#0066CC}{a_{22}}&0&&\vdots\\
\textcolor{#0066CC}{\vdots}&&\textcolor{#0066CC}{\ddots}&\ddots&\vdots\\
\textcolor{#0066CC}{\vdots}&&&\textcolor{#0066CC}{\ddots}&0\\
\textcolor{#0066CC}{a_{n1}}&\textcolor{#0066CC}{\cdots}&\textcolor{#0066CC}{\cdots}&\textcolor{#0066CC}{a_{n,n-1}}&\textcolor{#0066CC}{a_{nn}}\end{pmatrix}.
\]
\end{enumerate}
\end{definition}
\begin{example}
\begin{itemize}
\item Es ist $0_{3,2}=\begin{pmatrix}0&0\\0&0\\0&0\end{pmatrix}$ die Nullmatrix in $\mathcal{M}_{3,2}$.
\item Die Einheitsmatrix in $\mathcal{M}_{3,3}$ ist $E_3=\begin{pmatrix}1&0&0\\0&1&0\\0&0&1\end{pmatrix}$.
\item Die Matrix $\begin{pmatrix}2&0&0&0\\0&1&0&0\\0&0&0&0\\0&0&0&5\end{pmatrix}\in M_{4,4}(\R)$ ist eine Diagonalmatrix.
Die Matrix $\begin{pmatrix}0&0&0&2\\0&0&1&0\\0&1&0&0\\5&0&0&0\end{pmatrix}\in M_{4,4}(\R)$ ist keine Diagonalmatrix.
Hier stehen die Einträge nicht auf der (Haupt-)Diagonalen der Matrix, die dadurch gekennzeichnet ist, 
dass Zeilen- und Spaltenindices übereinstimmen.
\end{itemize}
\end{example}
Matrizen kann man ähnlich wie Vektoren \notion{komponentenweise} (koeffizientenweise) addieren und mit Skalaren multiplizieren.
Die einzige Änderung dabei ist, dass die Koeffizienten bei Matrizen rechteckig angeordnet und bei Vektoren in Spalten.
\begin{definition}
Es sei $\lambda\in\R$ ein Skalar und $A=(a_{ij})$ und $B=(b_{ij})$ Matrizen in $M_{m,n}(\R)$ Matrizen derselben Größe.
\begin{enumerate}
\item[(i)]
Die Summe der Matrizen $A$ und $B$ ist
\[
A+B=(a_{ij}+b_{ij})=
\begin{pmatrix}a_{11}+b_{11}&a_{12}+b_{12}&\cdots&a_{1n}+b_{1n}\\a_{21}+b_{21}&a_{22}+b_{22}&\cdots&a_{2n}+b_{2n}\\
\vdots&\vdots&&\vdots\\
a_{m1}+b_{m1}&a_{m2}+b_{m2}&\cdots&a_{mn}+b_{mn}\end{pmatrix}.
\]
\item[(ii)]
Das $\lambda$-fache der Matrix $A$ ist
\[
\lambda A=\begin{pmatrix}\lambda a_{11}&\lambda a_{12}&\cdots&\lambda a_{1n}\\
\lambda a_{21}&\lambda a_{22}&\cdots&\lambda a_{2n}\\
\vdots&\vdots&&\vdots\\
\lambda a_{m1}&\lambda a_{m2}&\cdots&\lambda a_{mn}\end{pmatrix}.
\]
\end{enumerate}
\end{definition}
\begin{block}[warning]
Man kann nur Matrizen derselben Größe addieren!
\end{block}
\begin{example}
\begin{enumerate}
\item[(i)]$
\begin{pmatrix}1&2&3\\4&5&6\end{pmatrix}=\begin{pmatrix}2&-1&0\\4&0&-2\end{pmatrix}
=\begin{pmatrix}3&1&3\\8&5&4\end{pmatrix}$
\item[(ii)]
$5\begin{pmatrix}1&2&3\\4&5&6\end{pmatrix}=\begin{pmatrix}5&10&15\\20&25&30\end{pmatrix}$.
\item[(iii)]
Man kan die Regel für die Multiplikation mit einem Skalar auch benutzen, um eine Matrix zu vereinfachen, 
indem man einen gemeinsamen Faktor aller Koeffizienten ausklammert. Zum Beispiel ist in der folgenden Matrix
jeder Koeffizient durch $13$ teilbar:
\[
\begin{pmatrix}26&39\\169&-13\end{pmatrix}=13\begin{pmatrix}2&3\\13&-1\end{pmatrix}.
\]
\end{enumerate}
\end{example}
%%
%% Füge Quickcheck zu Matrixaddition und Skalarmultiplikation ein!
%%
Die Matrixaddition und die Multiplikation mit Skalaren gehorcht wie die Vektorrechnung folgenden Rechenregeln.
\begin{rule}[Eigenschaften der Matrizenrechnung]
Es seien $A,B,C \in M_{m,n}(\R)$ und $\lambda,\mu\in\R$.
\begin{enumerate}
\item[(i)] $A=(a_{ij})$ und $B=(b_{ij})$ sind genau dann gleich, 
wenn die Koeffizienten an jeder Stelle übereinstimmen: $a_{ij}=b_{ij}$ für $i=1,\ldots,m$ und $j=1,\ldots,n$.
Zwei Matrizen können höchstens dann gleich sein, wenn sie dieselbe Größe haben.
\item[(ii)]
Die Matrixaddition  ist \emph{kommutativ}
\[A+B=B+A.\]
\item[(iii)]
Für die Nullmatrix $0_{m,n}\in\R^n$ gilt
\[A+0_{m,n}=A=0_{m,n}+A.\]
\item[(iv)]
Die Matrixaddition ist \emph{assoziativ}
\[A+(B+C)=A+B+C=(A+B)+C.\]
\item[(v)]
Die Multiplikation mit Skalaren ist assoziativ
\[\lambda(\mu A)=\lambda\mu A=(\lambda\mu)A.\]
\item[(vi)]
Die Multiplikation mit Skalaren und die Matrixaddition sind distributiv
\[(\lambda+\mu)A=\lambda A+\mu A\quad\text{ sowie }\quad \lambda(A+B)=\lambda A+\lambda B.\]
\end{enumerate}
\end{rule}
%%
%% Quickcheck zu einer Rechnung aA+bB=??.
%%
\section{Transponierte Matrix}
Ähnlich wie einen Vektor kann man auch Matrizen transponieren:
\begin{definition}
\begin{enumerate}
\item[(i)]
Die \notion{Transponierte} einer Matrix $A=(a_{ij})\in M_{m,n}(\R)$ ist die Matrix
\[
A^T=(a_{ji})=\begin{pmatrix}a_{11}&a_{21}&\cdots&a_{1m}\\a_{12}&a_{22}&\cdots&a_{m2}\\
\vdots&\vdots&&\vdots\\
a_{1n}&a_{2n}&\cdots&a_{mn}\end{pmatrix}\in M_{n,m}(\R).
\]
Die zu $A$ transponierte Matrix $A^T$ entsteht also aus $A$, indem man die Zeilen von $A$ zu Spalten von $A^T$ macht. 
(Damit werden dann auch die Spalten von $A$ zu den Zeilen von $A^T$.)
\item[(ii)]
Eine Matrix heißt \notion{symmetrisch}, wenn $A=A^T$. Eine symmetrische Matrix ist also notweniger Weise quadratisch.
\end{enumerate}
\end{definition}
\begin{example}
\begin{itemize}
\item
Die zu
$A=\begin{pmatrix} 
       \textcolor{#CC6600}{1} & \textcolor{#CC6600}{2} & \textcolor{#CC6600}{3}  \\ 
       \textcolor{#0066CC}{4} & \textcolor{#0066CC}{5} & \textcolor{#0066CC}{6} 
       \end{pmatrix} \in M_{2,3}(\R)$
      \lang{de}{ist}
      \lang{en}{is}
      $A^T=\begin{pmatrix} \textcolor{#CC6600}{1} & \textcolor{#0066CC}{4} \\ 
                       \textcolor{#CC6600}{2} & \textcolor{#0066CC}{5} \\ 
                       \textcolor{#CC6600}{3} & \textcolor{#0066CC}{6} \end{pmatrix}\in M_{2,3}(\R) .$
% Die zu $A=\begin{pmatrix}1&2&3\\4&5&6\end{pmatrix}\in M_{2,3}(\R)$ transponierte Matrix ist
% $A^T=\begin{pmatrix}1&4\\2&5\\3&6\end{pmatrix}\in M_{3,2}(\R)$.
\item
Die Matrix $B=\begin{pmatrix} 2&1\\1&3\end{pmatrix}$ ist symmetrisch, denn es ist
$B^T=\begin{pmatrix} 2&1\\1&3\end{pmatrix}=B$.
\end{itemize}
\end{example}
%%
%%
%% Quickcheck zur Transponierten.
%%
%%
\begin{remark}[Regeln für transponierte Matrizen]
Es seien $A,B \in M_{m,n}(\R)$ und $\lambda\in\R$.
\begin{enumerate}
\item[(i)] 
Zweifaches Transponieren liefert die ursprüngliche Matrix
\[ (A^T)^T=A.\]
\item[(ii)] Transponieren ist verträglich mit Matrixaddition und Multiplikation mit Skalaren
\[(A+B)^T=A^T+B^T \quad \text{ sowie }\quad (\lambda A)^T=\lambda A^T.\]
\end{enumerate}
\end{remark}
%%
%%
%%
\section{Matrix-Vektor-Multiplikation}
\begin{definition}
Das Produkt einer Matrix $A=(a_{ij})\in M_{m,\textcolor{#CC6600}{n}}(\R)$ mit einem Spaltenvektor 
$x=\begin{pmatrix}x_1\\\vdots\\x_n\end{pmatrix}\in\R^n=M_{\textcolor{#CC6600}{n},1}(\R)$ ist der Vektor
\[
A\cdot x=\underbrace{\begin{pmatrix}a_{11}&a_{12}&\cdots&a_{1n}\\a_{21}&a_{22}&\cdots&a_{2n}\\
\vdots&\vdots&&\vdots\\
a_{m1}&a_{m2}&\cdots&a_{mn}\end{pmatrix}}_{\textcolor{#0066CC}{m}\times \textcolor{#CC6600}{n}}\cdot
\underbrace{\begin{pmatrix}x_1\\\vdots\\x_n\end{pmatrix}}_{\textcolor{#CC6600}{n}\times 1}=
\underbrace{\begin{pmatrix}b_1\\\vdots\\b_m\end{pmatrix}}_{\textcolor{#0066CC}{m}\times 1}=b\in \R^m=M_{\textcolor{#0066CC}{m},1}(\R)
\]
mit den Koordinaten (für $i=1,\ldots,m$)
\[
b_i=\underbrace{\begin{pmatrix}a_{i1}&\ldots&a_{in}\end{pmatrix}}_{i-\text{ter Zeilenvektor}}\cdot\begin{pmatrix}x_1\\\vdots\\x_n\end{pmatrix}
=\sum_{j=1}^n
a_{ij}x_j=a_{i1}x_1+\ldots+a_{in}x_n.
\]
\end{definition}
Die Matrix-Vektor-Multiplikation $A\cdot x$ wird also aufs Skalarprodukt zurückgeführt: 
Die $i$-te Koordinate $b_i$ des Ergebnisvektors ist das Skalarprodukt des $i$-ten Zeilenvektors von $A$
mit $x$.
\begin{remark}
Die Matrix-Vektor-Multiplikation $A\cdot x=b$ ist nur definiert für eine Matrix $A$, 
die genauso viele Spalten hat wie $x$ Zeilen.\\
Das Ergebnis $b$ hat genauso viele Zeilen wie die Matrix $A$ selbst.
\end{remark}
\begin{example}
\begin{enumerate}
\item[(i)]
\lang{de}{Wir berechnen}
\lang{en}{Consider}
\[ \underbrace{\begin{pmatrix} 2 & -3 & 1 \\ 0 & 4 & 5 \end{pmatrix}}_{\textcolor{#0066CC}{2}\times \textcolor{#CC6600}{3}} \cdot
\underbrace{\begin{pmatrix} 1 \\ 3 \\ -2\end{pmatrix}}_{\textcolor{#CC6600}{3}\times 1}=
\begin{pmatrix} 2\cdot 1+(-3)\cdot 3+1\cdot (-2) \\ 0\cdot 1+4\cdot 3 +5\cdot (-2)\end{pmatrix} 
=\underbrace{\begin{pmatrix} -9\\ 2\end{pmatrix}}_{\textcolor{#0066CC}{2}\times 1}.
\]
\item[(ii)]
Für eine Matrix $A=(a_{ij})\in M_{m,n}(\R)$ und den $j$-ten Standardvektor $e_j\in R^n$ ist
\[A\cdot e_i= \begin{pmatrix}a_{11}&a_{12}&\cdots&a_{1n}\\a_{21}&a_{22}&\cdots&a_{2n}\\
\vdots&\vdots&&\vdots\\
a_{m1}&a_{m2}&\cdots&a_{mn}\end{pmatrix}\cdot \begin{pmatrix}0\\\vdots\\0\\1\\0\\\vdots\\0\end{pmatrix}
= \begin{pmatrix} 
0+\ldots+a_{1j}+0+\ldots+0\\\vdots\\0+\ldots+a_{mj}+0+\ldots+0\end{pmatrix}
=\begin{pmatrix} 
a_{1j}\\\vdots\\a_{mj}\end{pmatrix}
\]
der $j$-te Spaltenvektor von $A$.
\end{enumerate}
\end{example}
\begin{quickcheck}
       \precision{3}
      \field{real}
      \begin{variables}
           \randint[Z]{a}{2}{4}
           \randint[Z]{b}{2}{5}
           \randint[Z]{m}{2}{4}
           \randint[Z]{n}{5}{6}
           \randint[Z]{k}{7}{8}
           \randint[Z]{l}{2}{4}
           \randint[Z]{o}{5}{6}
           \randint[Z]{p}{7}{8}
           \randint[Z]{l1}{3}{3}
           \function[expand]{l2}{o*a+l*b}
           var
       \end{variables}
      \text{\lang{de}{
      Für 
      $A=\begin{pmatrix} \var{n} & \var{m} \\ \var{o} &\var{l}\\ \var{k} & \var{p} \end{pmatrix}$ und 
      $x=\begin{pmatrix} \var{a} \\ \var{b} \end{pmatrix}$ hat der resultierende Vektor $A\cdot x$ genau 
      \ansref Koordinaten. Die zweite Koordinate dieses Vektors lautet \ansref.
      }
      \lang{en}{
      Let 
      $A=\begin{pmatrix} \var{n} & \var{m} \\ \var{o} &\var{l}\\ \var{k} & \var{p} \end{pmatrix}$ and 
      $x=\begin{pmatrix} \var{a} \\ \var{b} \end{pmatrix}$. The vector $A\cdot x$ has exactly \ansref 
      coordinates. The second coordinate of this vector is \ansref.
      }}
%      \explanation{}
      \begin{answer}
         \type{input.number}
            \solution{l1}
      \end{answer}
      \begin{answer}
      \type{input.function}
            \solution{l2}
      \end{answer}
\end{quickcheck}
\begin{remark}
Ebenso erhalten wir das Produkt  eines Zeilenvektors
$x=\begin{pmatrix} x_1&\ldots&x_m\end{pmatrix}\in M_{1,m}(\R)$ mit einer Matrix $A\in M_{m,n}(\R)$ als
Zeilenvektor $b\in M_{1,n}(\R)$:
\[x\cdot A=\underbrace{\begin{pmatrix} x_1&\ldots&x_m\end{pmatrix}}_{1\times \textcolor{#CC6600}{m}}\cdot 
\underbrace{\begin{pmatrix}a_{11}&a_{12}&\cdots&a_{1n}\\a_{21}&a_{22}&\cdots&a_{2n}\\
\vdots&\vdots&&\vdots\\
a_{m1}&a_{m2}&\cdots&a_{mn}\end{pmatrix}}_{\textcolor{#CC6600}{m}\times \textcolor{#0066CC}{ n}}
=\underbrace{\begin{pmatrix}b_1\\\vdots\\b_n\end{pmatrix}}_{1\times\textcolor{#0066CC}{n}}=b
\]
hat die Koordinaten (für $j=1,\ldots, n$)
$b_j=\begin{pmatrix} x_1&\ldots&x_m\end{pmatrix}\cdot \begin{pmatrix}a_{1j}\\\vdots\\a_{mj}\end{pmatrix}=
x_1a_{1j}+\ldots+x_ma_{mj}$.
\\
Hier ist $b_j$ das Skalarprodukt von $x$ mit dem $j$-ten Spaltenvektor von $A$.
\end{remark}
\begin{example}
Wir berechnen
\begin{align*}
\underbrace{\begin{pmatrix}1&-1\end{pmatrix}}_{1\times \textcolor{#CC6600}{2}}\cdot
\underbrace{\begin{pmatrix}2&3&-4&1\\0&2&1&-3\end{pmatrix}}_{\textcolor{#CC6600}{2}\times \textcolor{#0066CC}{4}}\:=&
\begin{pmatrix}1\cdot 2-1\cdot 0,&1\cdot 3-1\cdot 2,&1\cdot(-4)-1\cdot 1,&1\cdot 1-1\cdot(-3)\end{pmatrix}\\
=&
\underbrace{\begin{pmatrix}2&1&-5&4\end{pmatrix}}_{1\times \textcolor{#0066CC}{4}}.
\end{align*}
Nebenbei: Rutschen die Einträge in einer Spalte optisch so eng zusammen, dass die Lesbarkeit leidet, 
so trennen wir sie wie oben durch Kommata.
\end{example}
\begin{example}
In der \ref[content_16_vektoren][Prozessanalyse]{sec:prozessanalyse} war der Gewinn gegeben durch
$G=U-K=p^T\cdot x-k^T\cdot  y=p^T\cdot x-k^T\cdot A\cdot x=(p+A^T\cdot k)^T\cdot x$.
\end{example}
\begin{example}[Einstufiger Produktionsprozess]\label{ex:einstufiger_prod_prozess}
Die Produkte $P_1, P_2, P_3$ werden aus den Zutaten $Z_1,Z_2,Z_3,Z_4$ hergestellt.
%%
%% Bildchen einfügen!!
%%
Dabei werden pro Liter fertigem Produkt $P_j$ die Zutaten in (in ml angegebenen, neben Wasser) den Mengen benötigt
\begin{table}
    Verbrauch & $P_1$ & $P_2$ &$P_3$ \\ 
     $Z_1$& $200$ &$300$  &$350$  \\
    $Z_2$& $0$ &$450$  &$200$  \\
    $Z_3$&$350$ & $150$& $200$\\
    $Z_4$& $250$& $40$& $0$
    \end{table}
Die gewünschten Produktionsmenge von $P_j$ bezeichnen wir mit $x_j$ (für $j=1,\ldots,3$). 
Damit erhalten wir Formeln für die benötigten Mengen $b_i$ der Zutaten $Z_i$ (für $i=1,\ldots,4$):
\begin{align*}
200x_1&\:+\:&300x_2&\:+\:&350x_3&\:=\:&b_1,\\
&&450x_2&\:+\:&200x_3&\:=\:&b_2,\\
350x_1&\:+\:&150x_2&\:+\:&200x_3&\:=\:&b_3,\\
250x_1&\:+\:&40x_2&& &\:=\:&b_4.
\end{align*}
Dies notieren wir kürzer mit
\[
A=\begin{pmatrix}200&300&350\\0&450&200\\350&150&200\\250&40&0\end{pmatrix},\quad x=\begin{pmatrix}x_1\\x_2\\x_3\end{pmatrix}
\quad \text{ und }\quad b=\begin{pmatrix}b_1\\b_2\\b_3\\b_4\end{pmatrix},
\]
so dass wir dafür kompakt schreiben können
\[
A\cdot x=b.
\]
sollen zum Beispiel $x_1=1000$ Liter von $P_1$, $x_2=500$ Liter von $P_2$ 
und $x_3=7500$ Liter von $P_3$ hergestellt werden, dann berechnet sich der Bedarf $b_j$ der Zutat $Z_j$ zu
\begin{align*}
\begin{pmatrix}200&300&350\\0&450&200\\350&150&200\\250&40&0\end{pmatrix}
\cdot \begin{pmatrix}1000\\500\\7500\end{pmatrix}
&\:=\:&\begin{pmatrix}200\cdot 1000+300\cdot 500+350\cdot 7500\\
0\cdot 1000+450\cdot 500+200\cdot 7500\\
350\cdot 1000+150\cdot500+200\cdot 7500\\
250\cdot 1000+40\cdot 500+0\cdot 7500\end{pmatrix}\\
&\:=\:&
\begin{pmatrix}2975000\\1725000\\575000\\270000\end{pmatrix}=
1000\cdot\begin{pmatrix}2975\\1725\\575\\270\end{pmatrix}
=\begin{pmatrix}b_1\\b_2\\b_3\\b_4\end{pmatrix}.
\end{align*}
Der Vektor $\begin{pmatrix}2975\\1725\\575\\270\end{pmatrix}$ gibt 
den Bedarf in Litern statt Millilitern an.
Geben wir sinnvoller bereits die Matrix $A$ in Litern an, so
 ersetzen wir $A$ durch 
$\tilde{A}= \begin{pmatrix}0,2&0,3&0,35\\0&0,45&0,2\\0,35&0,15&0,2\\0,25&0,04&0\end{pmatrix}$. 
Es gilt also $A=1000 \cdot\tilde{A}$.
Dann ist $\tilde{A}\cdot x=\tilde{b}$, also in unserem Beispiel
\[\begin{pmatrix}0,2&0,3&0,35\\0&0,45&0,2\\0,35&0,15&0,2\\0,25&0,04&0\end{pmatrix}
\cdot \begin{pmatrix}1000\\500\\7500\end{pmatrix}=\begin{pmatrix}2975\\1725\\575\\270\end{pmatrix}.
\]
\end{example}
%%
%%
\section{Matrix-Matrix-Produkt}
\begin{definition}
Es seien $A\in M_{m,n}(\R)$ und $B\in M_{n,r}(\R)$ Matrizen.
Das \notion{Matrixprodukt}  $A\cdot B$ ist eine Matrix $C\in M_{m,r}(\R)$,
\[
A\cdot B=\begin{pmatrix}a_{11}&\cdots &a_{1n}\\
\vdots&&\vdots\\
a_{m1}&\cdots &a_{mn}\end{pmatrix}
\cdot
\begin{pmatrix}b_{11}&\cdots &b_{1r}\\
\vdots &&\vdots\\
b_{n1}&\cdots &b_{nr}
\end{pmatrix}
=
\begin{pmatrix}c_{11}&\cdots&c_{1r}\\
\vdots&&\vdots\\
c_{m1}&\cdots &c_{mr}\end{pmatrix},
\]
deren Einträge $c_{ik}$ (für $i=1,\ldots,m$ und $k=1,\ldots,r$) gegeben werden durch
\[
c_{ik}=\underbrace{\begin{pmatrix}a_{i1}&\ldots& a_{in}\end{pmatrix}}_{i-\text{te Zeile von }A}
\cdot
\underbrace{\begin{pmatrix}b_{1k}\\\vdots\\b_{nk}\end{pmatrix}}_{k-\text{te Spalte von }B}
=a_{i1}b_{1k}+\ldots+a_{in}c_{nk}=\sum_{j=1}^n a_{ij}b_{jk}.
\]
\end{definition}
\begin{remark}
Zwei Matrizen $A$ und $B$ kann man nur dann multiplizieren, wenn $A$ genauso viele Spalten hat wie $B$ Zeilen, 
wenn also  $A\in M_{m,\textcolor{#CC6600}{n}}(\R)$ und $B\in M_{\textcolor{#CC6600}{n},r}(\R)$.\\
Die entstehende Matrix $C=A\cdot B$ hat so viele Zeilen wie $A$ und soviele Spalten wie $B$.\\
Die der Koeffizient $c_{ik}$ der Produktmatrix $C=(c_{ik})=A\cdot B$ ist das Skalarprodukt des $i$-ten Zeilenvektors von $A$
mit dem $k$-ten Spaltenvektor von $B$.
\[
\begin{pmatrix}a_{11}&\cdots &a_{1n}\\
\vdots&&\vdots\\
\textcolor{#CC6600}{a_{i1}}&\textcolor{#CC6600}{\ldots}&\textcolor{#CC6600}{ a_{in}}\\
\vdots&&\vdots\\
a_{m1}&\cdots &a_{mn}\end{pmatrix}
\cdot
\begin{pmatrix}b_{11}&\cdots&\textcolor{#CC6600}{b_{1k}}&\cdots &b_{1r}\\
\vdots &&\textcolor{#CC6600}{\vdots}&&\vdots\\
b_{n1}&\cdots&\textcolor{#CC6600}{b_{nk}}&\cdots &b_{nr}
\end{pmatrix}
=
\begin{pmatrix}c_{11}&\cdots&\cdots&\cdots&c_{1r}\\
\vdots&&&&\vdots\\
\vdots&&\textcolor{#CC6600}{\sum_{j=1}^na_{ij}b_{jk}}&&\vdots\\
\vdots&&&&\vdots\\
c_{m1}&\cdots&\cdots&\cdots &c_{mr}\end{pmatrix}
\]
\end{remark}
\begin{quickcheck}
    \type{input.number}
      \precision{3}
      \field{real}
      \begin{variables}
           \randint[Z]{m}{2}{4}
           \randint[Z]{n}{5}{6}
           \randint[Z]{k}{7}{8}
       \end{variables}
      \text{\lang{de}{
            Wird eine $(\var{m}\times \var{n})$-Matrix
            mit einer $(\var{n}\times \var{k})$-Matrix multipliziert,\\
            so ist das Ergebnis eine \ansref $\times$ \ansref-Matrix.
            }
            \lang{en}{
            If a $(\var{m}\times \var{n})$-matrix
            is multiplied with a $(\var{n}\times \var{k})$-matrix,\\
            the result is a \ansref $\times$ \ansref-matrix.
            }}
%      \explanation{}
      \begin{answer}
            \solution{m}
      \end{answer}
       \begin{answer}
            \solution{k}
      \end{answer}
\end{quickcheck}

\begin{example}
\begin{enumerate}
\item[(i)]
Wir berechnen
\begin{align*}
&\begin{pmatrix}1&2&3\\4&5&6\end{pmatrix}\cdot
\begin{pmatrix}1&0&1&0\\0&2&-1&3\\0&1&0&1\end{pmatrix}
\\
&=\:\begin{pmatrix} 
1\cdot 1+ 0+0,& 0+2\cdot 2+3\cdot 1,&
1\cdot 1+2\cdot(-1)+ 0,& 0+2\cdot 3+3\cdot1\\
4\cdot 1+ 0+ 0,& 0+5\cdot 2+6\cdot 1,&
4\cdot 1+5\cdot(-1)+ 0,& 0+5\cdot 3+6\cdot1
\end{pmatrix}\\
&=\:
\begin{pmatrix}1&7&-1&9\\4&16&-1&21\end{pmatrix}.
\end{align*}
\item[(ii)]
Die Matrix-Vektor-Multiplikation ist ein Spezialfall der Matrix-Matrix-Multiplikation,
bei der eine der beteiligten Matrizen nur eine Spalte (oder nur eine Zeile) hat.
\end{enumerate}
\end{example}
\begin{block}[warning]
Selbst wenn sowohl das Matrixprodukt $A\cdot B$ als auch das Matrixprodukt $B\cdot A$
sinnvoll definiert sind, ist meist $A\cdot B\neq B\cdot A$.
\begin{itemize}
\item
Damit beide Produkte sinnvoll sind, muss  $A\in M_{m,n}(\R)$ und $B\in M_{n,m}(\R)$ gelten.
Dann ist $A\cdot B\in M_{m,m}(\R)$ eine $(m\times m)$-Matrix, aber
$B\cdot A\in M_{n,n}(\R)$ eine $(n\times n)$-Matrix.
Ist nun $m\neq n$, dann sind die Matrizen offenbar verschieden.
\item 
Selbst wenn $m=n$ gelten sollte, ist zu erwarten, dass die Produkte $A\cdot B$ und $B\cdot A$
verschieden sind:
\notion{Matrizenmultiplikation ist nicht kommutativ!}
\\
Beispiel: Es ist 
$\begin{pmatrix} 1&2\\0&1\end{pmatrix}\cdot \begin{pmatrix}0&1\\1&1\end{pmatrix}=
\begin{pmatrix}2&3\\1&1\end{pmatrix}$, 
hingegen ist das Produkt in umgekehrter Reihenfolge
$ \begin{pmatrix}0&1\\1&1\end{pmatrix}\cdot \begin{pmatrix} 1&2\\0&1\end{pmatrix}=
\begin{pmatrix}0&1\\1&3\end{pmatrix}$.
\end{itemize}
\end{block}
Weitere Beispiele finden sich im 
%\ref{entsprechenden Kapitel des HM4mint.nrw}{https://hm4mint.nrw/hm1/link/HoeherMathem1/Teil1Grundl/11Matriz/Matrix/1#anchor-section-Multiplikation-zweier-Matrizen-1.


Die meisten anderen Rechenregeln gelten aber auch für die Matrix-Matrix-Multiplikation.
Weil die Matrix-Vektor-Multiplikation ein Spezialfall davon ist, gelten diese Regeln natürlich auch dafür.
\begin{rule}[Rechenregeln der Matrix-Matrix-Multiplikation]
Es seien $A\in M_{m,n}(\R)$, $B,C\in M_{n,r}(\R)$ und $D\in M_{r,s}(\R)$ Matrizen
und $\lambda\in\R$ ein Skalar. Dann gilt:
\begin{enumerate}
\item[(i)] Multiplikation mit einer (passenden) Einheitsmatrix ändert die Matrix nicht
\[ E_m\cdot A=A=A\cdot E_n.\]
\item[(ii)] Multiplikation irgendeiner Matrix mit einer passenden Nullmatrix ergibt eine Nullmatrix
\[A\cdot 0_{n,r}=0_{m,p}\quad \text{ sowie }\quad 0_{k,m}\cdot A=0_{k,n}.\]
\item 8(iii) Matrixmultiplikation ist assoziativ
\[A\cdot(B\cdot D)=A\cdot B\cdot D=(A\cdot B)\cdot D.\]
\item[(iv)]
Matrixmultiplikation und Multiplikation mit Skalaren sind assoziativ und vertauschbar
\[(\lambda A)\cdot B=\lambda A\cdot B=\lambda(A\cdot B)=A\cdot(\lambda B).\]
\item[(v)] Matrixmultiplikation und -addition sind distributiv
\[A\cdot(B+C)=A\cdot B+A\cdot C\quad \text{ sowie } (B+C)\cdot D=B\cdot D+C\cdot D. \]
\item[(vi)] Transponieren vertauscht die Reihenfolge in einem Matrix-Matrix-Produkt
\[(A\cdot B)^T=B^T\cdot A^T.\]
\item[(vii)] Wenn $B=C$, dann ist $B\cdot D=C\cdot D$ und $A\cdot B=A\cdot C$.\\
Aber die umgekehrte Aussage von (vii) ist falsch: Aus $A\cdot B=A\cdot C$ folgt im allgemeinen nicht $B=C$. 
Zum Beispiel ist für $A=0_{m,n}$ nach (ii) $A\cdot B=0_{m,r}=A\cdot C$ für \emph{alle} $B,C\in M_{n,r}(\R)$.
\end{enumerate}
\end{rule}
%%
%%
%%
%%
\section{Ein mehrstufiger Produktionsprozess}
Der einstufige Produktionsprozess aus Beispiel \ref{ex:einstufiger_prod_prozess} soll erweitert werden, 
indem aus den Produkten $P_1, P_2, P_3$ anschließend Endprodukte $E_1$ und $E_2$ hergestellten werden.
Die Produkte  werden aus den Zutaten $Z_1,Z_2,Z_3,Z_4$ hergestellt.
%%
%% Bildchen einfügen!! erweitern
%%
Die erste Produktionsstufe hatten wir bereits durch die Matrix (in Litern)
\[A=\begin{pmatrix}0,2&0,3&0,35\\0&0,45&0,2\\0,35&0,15&0,2\\0,25&0,04&0\end{pmatrix}\]
beschrieben. 
Für die  zweite Produktionsstufe erhalten wir die Produktionsmatrix 
\[
B=\begin{pmatrix}
0,2&0,1\\0,5&0,3\\0,25&0,45
\end{pmatrix}.
\]
Die Gesamtproduktionsmatrix $G$ ergibt sich aus dem Produkt von $A$ und $B$
\begin{align*}G&\:=\:&%A\cdot B\\
%&\:=\:&
\begin{pmatrix}0,2&0,3&0,35\\0&0,45&0,2\\0,35&0,15&0,2\\0,25&0,04&0\end{pmatrix}
\cdot
\begin{pmatrix}
0,2&0,1\\0,5&0,3\\0,25&0,45
\end{pmatrix}=
%&\:=\:&
\begin{pmatrix}
0,2775&0,2675\\
0,275&0,225\\
0,195&0,17\\
0,07&0,037
\end{pmatrix}.
\end{align*}
Durch die Matrix-Schreibweise haben wir die Produktionsschritte übersichtlich beschrieben 
und den gesamten Produktionsprozess in der Matrix $G$ zusammengefasst. 

Für die Herstellung von jeweils $100$ Litern der beiden Endprodukte $E_1$ und $E_2$
benötigt man somit die durch den folgenden Vektor angegebenen Mengen (in Litern) der Ausgangsstoffe $Z_1,\ldots,Z_4$:
\[G\cdot\begin{pmatrix}100\\100\end{pmatrix}=\begin{pmatrix}
0,2775&0,2675\\
0,275&0,225\\
0,195&0,17\\
0,07&0,037
\end{pmatrix}\cdot \begin{pmatrix}100\\100\end{pmatrix}=
\begin{pmatrix}
54,5\\50\\36,5\\4,4
\end{pmatrix}.
\]
Wir könnten auch anders fragen: Wennn  die Mengen $b_1,\ldots, b_4$ der 
Zutaten $Z_1,\ldots, Z_4$ vorhanden sind, 
wie viel der Endprodukte $E_1$, $E_2$ kann man dann herstellen?
Dann gilt es das \emph{lineare Gleichungssystem $G\cdot x =b$} in $x=\begin{pmatrix}x_1\\x_2\end{pmatrix}$ 
zulösen,
wobei $b=\begin{pmatrix}b_1\\b_2\\b_3\\b_4\end{pmatrix}$.

Lineare Gleichungssysteme werden im nächsten Kapitel besprochen.


\end{content}
