%$Id:  $
\documentclass{mumie.article}
%$Id$
\begin{metainfo}
  \name{
  \lang{en}{...}
  \lang{de}{Lineare Gleichungssysteme}
  }
  \begin{description} 
 This work is licensed under the Creative Commons License Attribution 4.0 International (CC-BY 4.0)   
 https://creativecommons.org/licenses/by/4.0/legalcode 

    \lang{en}{...}
    \lang{de}{...}
  \end{description}
  \begin{components}
  \end{components}
  \begin{links}
  \end{links}
  \creategeneric
\end{metainfo}
\begin{content}
\begin{block}[annotation]
	Im Ticket-System: \href{https://team.mumie.net/issues/32610}{Ticket 32610}
\end{block}
\usepackage{mumie.ombplus}
\ombchapter{4}
\ombarticle{1}
\lang{de}{\title{Lineare Gleichungssysteme}}
\begin{block}[info-box]
  \tableofcontents
\end{block}
\section{Ein Beispiel}
\section{Lineare Gleichungssysteme}
Wir betrachten ein allgemeines lineares Gleichungssystem
\[
\begin{mtable}[\cellaligns{ccccccccc}]
 a_{11} x_1 & + & a_{12} x_2 & + & \cdots & + & a_{1n} x_n & = & b_1 \\
 a_{21} x_1 & + & a_{22} x_2 & + & \cdots & + & a_{2n} x_n & = & b_2 \\
 \vdots     &   & \vdots     &   &        &   & \vdots     &   & \vdots \\
 a_{m1} x_1 & + & a_{m2} x_2 & + & \cdots & + & a_{mn} x_n & = & b_m
\end{mtable}
\]
mit $m$ Gleichungen und $n$ Unbekannten (Unbestimmten) $x_1,\ldots, x_n$.
Die Koeffizienten $a_{ij}$ und $b_i$ mit $1 \leq i \leq m$ und $1 \leq j \leq n$ 
sind reelle Zahlen.
\begin{example}
Das lineare Gleichungssystem
\[ \begin{mtable}[\cellaligns{ccrcrcrcr}]
&&x_{1}&+&3x_{2}&-&x_{3}&=&4\\
&&2x_{1}&-&2x_{2}&+&5x_{3}&=&-7
\end{mtable} \]
besteht aus zwei Gleichungen und hat drei Unbekannte.
Die Koeffizienten sind $a_{11}=1$, $a_{12}=3$, $a_{13}=-1$, $a_{21}=2$, $a_{22}=-2$, $a_{23}=5$,
sowie $b_1=4$ und $b_2=-7$.
\end{example}
\begin{quickcheck}
\type{input.number}
\begin{variables}
\function{z}{3}
\function{n}{4}
\function{a13}{-5}
\function{a31}{0}
\function{b2}{2}
\end{variables}
\text{Das lineare Gleichungssystem\\ $\begin{matrix}
   x_{1}&&&-&5x_{3}&+&x_4&=&0\\
    2x_{1}&+&3x_{2}&&& -&x_{4}&=&2\\
    &&4x_{2}&&& +&3x_{4}&=&-1\\
  \end{matrix}$\\
  hat \ansref Zeilen und \ansref Unbekannte. Unter den Koeffizienten sind $a_{13}=$\ansref
  und $a_{31}=$\ansref sowie $b_2=$\ansref.
}
\begin{answer}
\solution{z}
\end{answer}
\begin{answer}
\solution{n}
\end{answer}
\begin{answer}
\solution{a13}
\end{answer}
\begin{answer}
\solution{a31}
\end{answer}
\begin{answer}
\solution{b2}
\end{answer}
\end{quickcheck}
Mit Hilfe von Vektoren und Matrizen können wir das lineare Gleichungssystem kompakt schreiben.
Dazu definieren wir die \notion{Koeffizientenmatrix}
\[
A=\left( \begin{smallmatrix}
a_{11} & a_{12} & \cdots & a_{1n} \\
a_{21} & a_{22} & \cdots & a_{2n} \\
\vdots & \vdots & \ddots & \vdots \\
a_{m1} & a_{m2} & \cdots & a_{mn}
\end{smallmatrix} \right)\:\in\: M_{m,n}(\R),
\]
den \notion{Koeffizientenvektor der rechten Seite} $b$ und den Vektor der Unbestimmten $x$,
\[
b=\left( \begin{smallmatrix}
b_{1}  \\
b_{2} \\
\vdots \\
b_{m} 
\end{smallmatrix} \right)\:\in\: \R^m\quad \text{ bzw. }\quad x=\left( \begin{smallmatrix}
x_{1}  \\
x_{2} \\
\vdots \\
x_{n} 
\end{smallmatrix} \right)\:.
\]
Das lineare Gleichungssystem ist dann gleichbedeutend zur Matrix-Vektor-Gleichung
\[A\cdot x=b.\]
\begin{example}\label{ex:LGS_coefficient_matrix}
Das lineare Gleichungssystem
\[ \begin{mtable}[\cellaligns{ccrcrcrcr}]
&&x_{1}&+&2x_{2}&+&x_{3}&=&4\\
&&x_{1}&-&x_{2}&+&\frac{3}{2}x_{3}&=&-7\\
&&-4x_{1}&+&2x_{2}&&&=&-2
\end{mtable} \]
hat die Koeffizientenmatrix 
$ A=\begin{pmatrix} 1 & 2 & 1 \\ 1 & -1 & \frac{3}{2} \\ -4 & 2 & 0\end{pmatrix} $
und die rechte Seite $b=\begin{pmatrix} 4 \\ -7 \\ -2 \end{pmatrix} $. Die Gleichung $A\cdot x=b$ ist also
 \[ \begin{pmatrix} 1 & 2 & 1 \\ 1 & -1 & \frac{3}{2} \\ -4 & 2 & 0\end{pmatrix}\cdot \begin{pmatrix}x_1 \\x_2\\ x_3\end{pmatrix}
= \begin{pmatrix} 4 \\ -7 \\ -2 \end{pmatrix}. \]
\end{example}
Die \notion{Lösung} des linearen Gleichungssystems $A\cdot x=b$ ist die Menge alle konkreten $x\in\R^n$
\[
\mathbb{L}=\{x=\left( \begin{smallmatrix}
x_{1}  \\
x_{2} \\
\vdots \\
x_{n} 
\end{smallmatrix} \right)\in\R^n\::\: A\cdot x=b\}.
\]
Um die Lösung eines linearen Gleichungssystems zu bestimmen, benützen wir im folgenden oft die
\notion{erweiterte Koeffizientenmatrix}
\[
(A ~\mid~ b) = \left(  \begin{smallmatrix}
                                      a_{11} & a_{12} & \cdots & a_{1n} & | & b_1\\
                                      a_{21} & a_{22} & \cdots & a_{2n} & | &b_2\\
                                      \vdots & \vdots & \ddots & \vdots & \mid &\vdots\\
                                      a_{m1} & a_{m2} & \cdots & a_{mn} &| & b_m
                                      \end{smallmatrix} \right) \in M_{m,n+1}(\R),
\]
die  zusammengesetzt wird aus der Koeffizientenmatrix $A$ und der rechten Seite $b$.
\begin{example}
Die erweitere Koeffizientenmatrix des linearen Gleichungssystems aus Beispiel \ref{ex:LGS_coefficient_matrix}
ist
\[ (A \mid b )= 
\begin{pmatrix} 1 & 2 & 1 & | & 4 \\
1 & -1 & \frac{3}{2}& | & -7\\ -4 & 2 & 0& | &-2\end{pmatrix}.\]
\end{example}
\section{Elementare Zeilenumformungen}
Um ein lineares Gleichungssystem zu lösen, müssen wir es so lange umformen, 
bis wir die Lösung ablesen können. Bei den Umformungen darf keine Information verloren gehen,
aber auch keine Information dazu \glqq erdichtet\grqq{} werden, denn beides
verfäscht die Lösung.\\
Um das sicher zustellen, lassen wir nur bestimmte Umformungen zu:
\begin{definition}[\lang{de}{Elementare Zeilenumformungenmformungen}
\lang{en}{Elementary row operations}]\label{def:elementare_Zeilenumformungen}
\lang{de}{
Die sogenannten \notion{elementaren Zeilenumformungen} sind:
}
\begin{itemize}
\item \lang{de}{Vertauschen zweier Gleichungen.}
      \lang{en}{Exchanging/swapping two equations.}
\item \lang{de}{Multiplikation einer Gleichung mit einer Konstanten $c\neq0$.}
      \lang{en}{Multiplication of an equation by a number $c\neq0$.}
\item \lang{de}{
      Addition (Subtraktion) eines $c$-fachen einer Gleichung zu (von) einer anderen.
      }
      \lang{en}{
      Addition (or subtraction) of a multiple of one row to (from) another.
      }
\end{itemize}
\end{definition}
\begin{theorem}
Elementare Zeilenumformungen verändern die Lösungsmenge eines LGS nicht.
\end{theorem}
\begin{example}
\lang{de}{Wir formen das folgende lineare Gleichungssystem mit Hilfe der elementaren Zeilenumformungen um.
Unser Vorgehen notieren wir rechts vom linearen Gleichungssystem.}
\[  \begin{mtable}[\cellaligns{crcrcrcrcrl}]
(I)&\qquad x_1 &+& 2x_2 &+ &\phantom{3}x_3 &-  &x_4 & = & 7 & \phantom{\qquad|\,\, -2\cdot \text{(II)}}\\
(II)&  \textcolor{#CC6600}{x_1} &+ &3x_2 &  &    & -  & x_4 & = & 8&\qquad|\,\, -\text{(I)} \\
(III)& \textcolor{#CC6600}{-x_1}  &  &      &  -   &3x_3 &+   &5x_4  & = & -9& \qquad|\,\, +\text{(I)}
\end{mtable} \]
\lang{de}{Zunächst eliminieren wir  die Variable $x_1$ aus der zweiten und dritten Gleichung und erhalten:}
\lang{en}{First we eliminate  the variably $x_1$ from the second and third equations and obtain:}
\[  \begin{mtable}[\cellaligns{crcrcrcrcrl}]
(I)&\qquad \phantom{+}x_1 &+& 2x_2 &+ &\phantom{3}x_3 &-  &x_4 & = & 7 & \\
(II)&   & &x_2 & - &  x_3  &   & & = & 1 & \\
(III)&  &  &   \textcolor{#CC6600}{2x_2}   &  -   &2x_3 &+   &4x_4  & = & -2& \qquad|\,\, -2\cdot \text{(II)}
\end{mtable} \]
\lang{de}{Anschließend eliminieren wir $x_2$ aus der dritten Gleichung, und erhalten:}
\lang{en}{Finally we eliminate $x_2$ from the third equation}
\[  \begin{mtable}[\cellaligns{crcrcrcrcrl}]
(I)&\qquad \phantom{+}x_1 &+& 2x_2 &+ &\phantom{3}x_3 &-  &x_4 & = & 7 & \phantom{\qquad|\,\, -2\cdot \text{(II)}} \\
(II)&  & &x_2 & - &  x_3  &   & & = & 1 & \\
(III)&  &  &    &     & &   &4x_4  & = & -4& \qquad|\,\, \cdot\frac{1}{4}
\end{mtable} \]
\lang{de}{
Wir können noch weiter vereinfachen, indem wir von unten nach oben möglichst viele Unbekannte eleminieren.
Das ist diesem Fall $x_4$ sowie $x_2$ in der ersten  Gleichung.
 Dafür ist es praktisch, zunächst die dritte Gleichung durch $4$ zu teilen.
}
\[  \begin{mtable}[\cellaligns{crcrcrcrcrl}]
(I)&\qquad \phantom{+}x_1 &+& 2x_2 &+ &\phantom{3}x_3 &\textcolor{#CC6600}{-}  &\textcolor{#CC6600}{x_4} & = & 7 & \qquad|\,\,+ \text{(III)} \\
(II)&   & &x_2 & - &  x_3  &   & & = & 1 & \\
(III)&  &  &    &     & &   &\phantom{4}x_4  & = & -1& \phantom{\qquad|\,\, -2\cdot \text{(II)}}
\end{mtable} \]
\lang{de}{Dann die dritte Gleichung zur ersten addieren:}
\lang{en}{Then we add the third equation to the first}
\[  \begin{mtable}[\cellaligns{crcrcrcrcrl}]
(I)&\qquad \phantom{+}x_1 &+& \textcolor{#CC6600}{2x_2} &+ &\phantom{3}x_3 &\phantom{-}  & & = & 6 & \qquad|\,\,-2\cdot \text{(II)} \\
(II)&   & &x_2 & - &  x_3  &   & & = & 1 & \\
(III)&  &  &    &     & &   &\phantom{4}x_4  & = & -1& \phantom{\qquad|\,\, -2\cdot \text{(II)}}
\end{mtable} \]
\lang{de}{und zuletzt das Doppelte der zweiten Gleichung von der ersten subtrahieren:}
\lang{en}{and finally subtract twice the second equation from the first:}
\[  \begin{mtable}[\cellaligns{crcrcrcrcrl}]
(I)&\qquad \phantom{+}x_1 &\phantom{+}& \phantom{2x_2} &+ &3x_3 &\phantom{-}  & & = & 4 &  \\
(II)&   & &x_2 & - &  x_3  &   & & = & 1 & \\
(III)&  &  &    &     & &   &\phantom{4}x_4  & = & -1& \phantom{\qquad|\,\, -2\cdot \text{(II)}}
\end{mtable} \]
\lang{de}{
Damit ist $x_4=-1$ eindeutig festgelegt durch Gleichung (III). 
Die ersten beiden Gleichungen bestimmen  $x_1,x_2,x_3$ nur in Abhängigkeit von einander.
Durch Setzen von $x_3=t$ und Auflösen der ersten beiden Gleichungen nach $x_1$ bzw. $x_2$ erhalten wir nun
 als Lösungsmenge
}
\[ \mathbb{L}
= \left\{ \begin{pmatrix} 4-3t \\ 1+t \\t \\ -1\end{pmatrix} \, \big| \, t\in \R \right\}
= \left\{ \begin{pmatrix} 4 \\ 1 \\ 0 \\ -1\end{pmatrix}+ t\cdot \begin{pmatrix} -3\\ 1\\ 1\\ 0 
\end{pmatrix} \, \big| \, t\in \R \right\}\:. \]
\lang{de}{Weil wir das letzte LGS nur durch elementare Zeilenumformungen aus dem ursprünglichen erhalten haben,
ist $\mathbb{L}$ auch die Lösungsmenge des ursprünglichen.\\
Mit Hilfe von Begleitmatrizen können wir dies wesentlich kompakter fassen:}
\[  \left(\begin{mtable}
1 & 2 &1 &-1 & | & 7 \\
 \textcolor{#CC6600}{1} &3 &   0   & -1 & | & 8 \\
\textcolor{#CC6600}{-1}  & 0  &  - 3  &5  & | & -9
\end{mtable} \right)\begin{mtable}
 \phantom{|\,\, -2\cdot \text{(II)}}\\
|\,\, -\text{(I)} \\
|\,\, +\text{(I)}
\end{mtable}
\Leftrightarrow
\left(\begin{mtable}
1 &2 &1 &-1 & |& 7 \\
0 &1 & -1 &0 & | & 1  \\
0 &   \textcolor{#CC6600}{2}   &  - 2 &+ 4  & |& -2
\end{mtable}\right)
\begin{mtable}
\\\\\quad|\,\, -2\cdot \text{(II)}
\end{mtable} 
\]
\[
\Leftrightarrow
\left(\begin{mtable}
1 & 2 &1 &-1 & | & 7 \\
0&1&-1  & 0& | & 1\\ 
0&0&0  &4  & | & -4
\end{mtable}\right)
\begin{mtable}
 \\
 \\
\quad|\,\, \cdot\frac{1}{4}
\end{mtable}
\Leftrightarrow
\left(\begin{mtable}
1 &2 &1 &\textcolor{#CC6600}{-1}   & | & 7\\ 
0&1 & -1& 0&| & 1\\ 
0&0&0   &1  & |& -1
\end{mtable} \right)
\begin{mtable}
\quad|\,\,+ \text{(III)} \\
\phantom{|\,\, -2\cdot \text{(II)}} \\
\phantom{|\,\, -2\cdot \text{(II)}}
\end{mtable} 
\]
\[  \Leftrightarrow
\left(\begin{mtable}
1 & \textcolor{#CC6600}{2} &1 &0 & | & 6 \\ 
0&1 & - 1  & 0 & | & 1\\ 
0&0&0&1    & | & -1  
\end{mtable}\right) 
\begin{mtable}
 \quad|\,\,-2\cdot \text{(II)} \\
  \phantom{\qquad|\,\, -2\cdot \text{(II)}}\\
 \phantom{\qquad|\,\, -2\cdot \text{(II)}}
\end{mtable}
\Leftrightarrow
 \left(\begin{mtable}[\cellaligns{crcrcrcrcrl}]
1 &0 &3 &0 & | & 4  \\
0 &1 & -1  & 0& | & 1  \\
0&0&0 &1 & |& -1
\end{mtable}\right)
\]
\end{example}
\begin{remark}
Wesentlich an den elementaren Zeilenumformungen ist, dass sie \emph{reversibel} sind: 
Aus dem finalen linearen Gleichungssystem kann das ursprüngliche durch elementare Zeilenumformungen 
zurück gewonnen werden.
\\
Es gibt viele ähnliche Umformungen, die dies nicht leisten. Einige Beispiele:
\begin{enumerate}
\item[(i)]
Multiplikation einer Gleichung mit der Zahl $0$ ist irreversibel. 
Diese Operation entspricht dem Weglassen einer Gleichung. Deren Information geht dadurch verloren.
\begin{showhide}
Wir betrachten das LGS
\[  \begin{mtable}
(I)& x_1 &+ x_2 & = & 0 \\ 
(II)&  x_1 & +x_2 & = & 1
\end{mtable}\quad.\]
multiplizieren wir  die zweite Zeile mit $0$, so erhalten wir
\[  \begin{mtable}
(I)& x_1 &+ x_2 & = & 0 \\ 
(II^')&  &0 & = & 0
\end{mtable}\quad.\]
Die Gleichung $(II')$  ist immer erfüllt. Die Lösung des zweiten Gleichungssystems ist offenbar
$\mathbb{L}_2=\{\begin{pmatrix}t\\-t\end{pmatrix}\::\:t\in\R\}$. 
Das ursprüngliche Gleichungssystem hat hingegen gar keine Lösung, $\mathbb{L}_1=\emptyset$, 
denn wenn wie in $(I)$ $x_1+x_2=0$ erfüllt ist,
dann ist eben $x_1+x_2\neq 1$, d.h. $(II)$ kann nie gleichzeit mit $(I)$ erfüllt sein.
% Offenbar ist jedes $x=\begin{pmatrix}\frac{t}{3}\\1-2t\\t\end{pmatrix}$ mit $t\in\R$ eine Lösung.
% Setzen wir speziell $t=3$ und setzt $\begin{pmatrix}1\\-5\\3\end{pmatrix}$ in das ursprüngliche LGS ein,
% so erhalten wir
% \[ \left. \begin{mtable}%[\cellaligns{crcrcrcrcrl}]
% (I)&3 \cdot 1 &-5 &+ 3  & = & 1 \\ %& \phantom{\qquad|\,\, -2\cdot \text{(II)}}\\
% (II)&  &-5& +2\cdot 3 & = & 1\\ %&\qquad|\,\, -\text{(I)} \\
% (III)&  &  &        3  & = & 1 %& \qquad|\,\, +\text{(I)}
% \end{mtable}\right\}
% \Leftrightarrow \left\{
% \begin{mtable}
% (I)1 & = & 1 \\ %& \phantom{\qquad|\,\, -2\cdot \text{(II)}}\\
% (II)1& = & 1\\ %&\qquad|\,\, -\text{(I)} \\
% (III)      3  & = & 1 %& \qquad|\,\, +\text{(I)}
% \end{mtable}
% \right.
% .\]
% Es ist (III) offensichtlich falsch. Das zweite LGS hat mehr Lösungen als das ursprüngliche.
% Bei der Multiplikation mit Null geht die Information (III) $x_3=1$ verloren. Denn aus der
% Gleichung (III') $0=0$, die schlicht immer erfüllt ist, können wir diese spezielle Bedingung nicht zurückgewinnen.
\end{showhide}
\item[(ii)]
Addieren wir Gleichung zu einer anderen und gleichzeitig diese andere zu der einen,
so können wir ebenfalls die ursprünglichen Gleichungen nicht mehr rekonstruieren.
\begin{showhide}
Zum Beispiel entsteht so aus
\[  \begin{mtable}
(I)& x_1 &+ \phantom{2}x_2 & = & 1 &\qquad +(II)\\ 
(II)&  x_1 & -2x_2 & = & 2 &\qquad +(I)
\end{mtable}\quad\]
das LGS mit zwei gleichen Zeilen 
\[  \begin{mtable}
(I)& 2x_1 &-x_2 & = & 3 \\ 
(II)&  2x_1 &-x_2 & = & 3
\end{mtable}\quad.\]
Die Lösungsmenge des zweiten LGS ist $\mathbb{L}_2=\{\begin{pmatrix}t\\2t-3\end{pmatrix}\::\:t\in\R\}$, 
während das ursprüngliche LGS die Lösung 
$\mathbb{L}_1=\{\begin{pmatrix}\frac{4}{3}\\\frac{-1}{3}\end{pmatrix}\}$ hat.
\end{showhide}
\item[(iii)]
Bei vorzeitige Spezifizierung einer Koordinate schränkt man die Lösung stärker ein als notwendig.
Man stellt dadurch eine zusätzliche Bedingung auf.
\begin{showhide}
Untersucht man zum Beispiel das LGS
\[  \begin{mtable}
(I)& x_1 &+ \phantom{2}x_2 & = & 1 \\ 
(II)&  x_1 & -2x_2 & = & 2 
\end{mtable}\quad\]
nur im Fall $x_1=1$, so erhält man $(I')$ $1+x_2=1$, also $x_2=0$, und $(II')$ $1-2x_2=2$, also $x_2=-\frac{1}{2}$.
Diese Bedingung sind widersprüchlich, also gibt es keine Lösung der Form $\begin{pmatrix}1\\x_2\end{pmatrix}$.
Hingegen hat das LGS aber die Lösung (siehe (ii))
$\mathbb{L}_1=\{\begin{pmatrix}\frac{4}{3}\\\frac{-1}{3}\end{pmatrix}\}$.
\end{showhide}
\item[(iv)] Führt man die elementaren Operationen für Spalten statt für Zeilen aus, so verändert man 
den Unbestimmtenvektor $x$, also das LGS. Die Lösungsmenge ist dann eine völlig andere. 
Dieser Fehler wird häufig gemacht, wenn mit der Begleitmatrix gerechnet wird.
\begin{showhide}
Das LGS
\[  \begin{mtable}
(I)& x_1 &+ \:x_2 & = & 1 \\ 
(II)&  x_1 & -\:x_2 & = & 2 
\end{mtable}\quad\]
hat die Begleitmatrix $\begin{pmatrix}1&1&|&1\\1&-1&|&2\end{pmatrix}$ und die 
Lösung $\mathbb{L}=\{\begin{pmatrix}\frac{3}{2}\\-\frac{1}{2}\end{pmatrix}\}$.
Addieren wir hierin die zweite \emph{Spalte} zur ersten, so entsteht die Begleitmatrix
$\begin{pmatrix}2&1&|&1\\0&-1&|&2\end{pmatrix}$, also das LGS
\[  \begin{mtable}
(I^')& 2\tilde{x_1} &+ \:\tilde{x_2} & = & 1 \\ 
(II^')&  & -\:\tilde{x_2} & = & 2 
\end{mtable}\quad\]
mit der Lösung $\mathbb{L'}=\{\begin{pmatrix}\frac{3}{2}\\-2\end{pmatrix}\}$.
Die Unbestimmten stehen durch $\begin{pmatrix}x_1\\x_2
\end{pmatrix}=\begin{pmatrix}\tilde{x_1}\\\tilde{x_1}+\tilde{x_2}\end{pmatrix}$ in Beziehung.

\end{showhide}
\end{enumerate}
\end{remark}
\section{Gauß-Algorithmus}
\[ \begin{pmatrix}
    a_{1} & \star & \cdots & \star & \star&| & b_1\\ 
     0 & a_{2} & \cdots & \star & \star&| & b_2\\ 
    \vdots & \, & \ddots & \, & \vdots&| & \vdots\\ 
 0 & 0 & \cdots & a_{k} & \star&| & b_k\\ 
   0 & 0 & \cdots & 0 & 0&| & \vdots \\ 
  0 & 0 & \cdots & 0 & 0&| & b_{m}\\ 
    \end{pmatrix},\]
\section{Rang}
\section{Inverse Matrix}

\end{content}
