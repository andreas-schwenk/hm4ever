\documentclass{mumie.problem.gwtmathlet}
%$Id$
\begin{metainfo}
  \name{
    \lang{de}{A02: Matrixmultiplikation}
    \lang{en}{P02: Matrix multiplication}
  }
  \begin{description} 
 This work is licensed under the Creative Commons License Attribution 4.0 International (CC-BY 4.0)   
 https://creativecommons.org/licenses/by/4.0/legalcode 

    \lang{de}{Rechnen mit Matrizen}
    \lang{en}{Calculating with matrices}
  \end{description}
  \corrector{system/problem/GenericCorrector.meta.xml}
  \begin{components}
    \component{js_lib}{system/problem/GenericMathlet.meta.xml}{gwtmathlet}
  \end{components}
  \begin{links}
  \end{links}
  \creategeneric
\end{metainfo}
\begin{content}
\usepackage{mumie.genericproblem}

\lang{de}{
	\title{A02: Matrixmultiplikation}
}

\lang{en}{
	\title{P02: Matrix multiplication}
}

\begin{block}[annotation]
	Im Ticket-System: \href{http://team.mumie.net/issues/11102}{Ticket 11102}
\end{block}



\begin{problem}
    
\begin{question}

	\begin{variables}
		\randint[Z]{a}{-5}{5}
		\randint[Z]{b}{-5}{5}
		\randint[Z]{c}{-5}{5}
		\randint[Z]{d}{-5}{5}
		
		\randint[Z]{ee}{-5}{5}
		\randint[Z]{f}{-5}{5}
		\randint[Z]{g}{-5}{5}
		\randint[Z]{h}{-5}{5}
		
			\matrix[calculate]{aa}{
  			a & b \\ 
  			c & d
      	}
      	
      	\matrix[calculate]{bb}{
  			ee & f \\ 
  			g & h
      	}
		
		\matrix[calculate]{m}{
		a*ee+b*g & a*f+b*h \\
		c*ee+d*g & c*f+d*h
      } 
	\end{variables}

	\type{input.matrix}
	\displayprecision{3}
    \correctorprecision{2}
    \field{integer}
    
    \lang{de}{
	    \text{Bestimmen Sie das Produkt der folgenden Matrizen über $\R$.\\
$\var{aa}\cdot\var{bb}$}}

\lang{en}{
	    \text{Determine the product of the following matrices over $\R$.\\
$\var{aa}\cdot\var{bb}$}}
    
    \begin{answer}
	    \solution{m}
      \format{2}{2}
      \lang{de}{
      \explanation{
            Die Matrixmultiplikation zweier quadratischer Matrizen,
            also Matrizen mit gleicher Zeilen- und Spaltenanzahl ist 
            immer erlaubt. Die Ergebnismatrix hat dann ebensoviele
            Zeilen und Spalten.
            Man bestimmt die Ergebnismatrix, indem man für den
            $(i,j)$-ten Eintrag das Skalarprodukt aus der
            $i$-ten Zeile der ersten Matrix und der $j$-ten
            Spalte der zweiten Matrix bestimmt.}}
      \lang{en}{
      \explanation{
      The given matrices are square matrices (same number of rows and columns), which is why matrix multiplication is defined here.
      The result matrix will then have the same number of rows and columns. Its entries are calculated as follows:
      The $(i,j)$th entry in the result matrix is the scalar product (dot product) of the $i$th row of the first matrix with the
      $j$th column of the second matrix.}}
	\end{answer}
    
\end{question}

\begin{question}

	\begin{variables}
		\randint[Z]{a}{-3}{3}
		\randint[Z]{b}{-3}{3}
		\randint[Z]{c}{-3}{3}
		\randint[Z]{d}{-3}{3}
		\randint[Z]{ai}{-3}{3}
		\randint[Z]{bi}{-3}{3}
		\randint[Z]{ci}{-3}{3}
		\randint[Z]{di}{-3}{3}
		
		\randint[Z]{ee}{-3}{3}
		\randint[Z]{f}{-3}{3}
		\randint[Z]{g}{-3}{3}
		\randint[Z]{h}{-3}{3}
		\randint[Z]{eei}{-3}{3}
		\randint[Z]{fi}{-3}{3}
		\randint[Z]{gi}{-3}{3}
		\randint[Z]{hi}{-3}{3}
		
			\matrix[calculate]{aa}{
  			a+ai*i & b+bi*i \\ 
  			c+ci*i & d+di*i
      	}
      	
      	\matrix[calculate]{bb}{
  			ee+eei*i & f+fi*i \\ 
  			g+gi*i & h+hi*i
      	}
		
		\matrix[calculate]{m}{
		(a+ai*i)*(ee+eei*i)+(b+bi*i)*(g+gi*i) & (a+ai*i)*(f+fi*i)+(b+bi*i)*(h+hi*i) \\
		(c+ci*i)*(ee+eei*i)+(d+di*i)*(g+gi*i) & (c+ci*i)*(f+fi*i)+(d+di*i)*(h+hi*i)
      } 
	\end{variables}

	\type{input.matrix}
	\displayprecision{3}
    \correctorprecision{2}
    \field{complex}
    
    \lang{de}{
	    \text{
	    Bestimmen Sie das Produkt der folgenden Matrizen über $\C$:\\
$\var{aa}\cdot\var{bb}$}}

\lang{en}{
	    \text{
	    Determine the product of the following matrices over $\C$:\\
$\var{aa}\cdot\var{bb}$}}
    
    \begin{answer}
	    \solution{m}
      \format{2}{2}
      \lang{de}{
      \explanation{
            Die Matrixmultiplikation zweier quadratischer Matrizen,
            also Matrizen mit gleicher Zeilen- und Spaltenanzahl ist 
            immer erlaubt. Die Ergebnismatrix hat dann ebensoviele
            Zeilen und Spalten.
            Man bestimmt die Ergebnismatrix, indem man für den
            $(i,j)$-ten Eintrag das Skalarprodukt aus der
            $i$-ten Zeile der ersten Matrix und der $j$-ten
            Spalte der zweiten Matrix bestimmt.
            In dieser Aufgabe wird über dem Körper der komplexen
            Zahlen gearbeit. Also ist jede Multiplikation
            auch komplexwertig durchzuführen.}}
      \lang{en}{
      \explanation{
      The given matrices are square matrices (same number of rows and columns), which is why matrix multiplication is defined here.
      The result matrix will then have the same number of rows and columns. Its entries are calculated as follows:
      The $(i,j)$th entry in the result matrix is the scalar product (dot product) of the $i$th row of the first matrix with the
      $j$th column of the second matrix. 
      Since the matrices contain complex entries, each multiplication must be done as known for complex numbers.}}
	\end{answer}
    
\end{question}

%\begin{question}
%
%	\begin{variables}
%		\randint[Z]{x11}{-5}{5}
%		\randint[Z]{x12}{-5}{5}
%		\randint[Z]{x13}{-5}{5}
%		\randint[Z]{x21}{-5}{5}
%		\randint[Z]{x22}{-5}{5}
%		\randint[Z]{x23}{-5}{5}
%		\randint[Z]{x31}{-5}{5}
%		\randint[Z]{x32}{-5}{5}
%		\randint[Z]{x33}{-5}{5}
%	
%		\randint[Z]{y11}{-5}{5}
%		\randint[Z]{y12}{-5}{5}
%		\randint[Z]{y13}{-5}{5}
%		\randint[Z]{y21}{-5}{5}
%		\randint[Z]{y22}{-5}{5}
%		\randint[Z]{y23}{-5}{5}
%		\randint[Z]{y31}{-5}{5}
%		\randint[Z]{y32}{-5}{5}
%		\randint[Z]{y33}{-5}{5}
%		
%			\matrix[calculate]{a}{
%  			x11 & x12 & x13 \\ 
%  			x21 & x22 & x23 \\
%  			x31 & x32 & x33
%      	}
%      		\matrix[calculate]{b}{
%  			y11 & y12 & y13 \\ 
%  			y21 & y22 & y23 \\
%  			y31 & y32 & y33
%      	}
%      		\matrix[calculate]{c}{
%  			x11*y11 +x12*y21 + x13*y31 & x11*y12 +x12*y22+x13*y32 & x11*y13 +x12*y23+x13*y33\\ 
%  x21*y11 +x22*y21 + x23*y31 & x21*y12 +x22*y22 + x23*y32 & x21*y13 +x22*y23 + x23*y33\\
%  x31*y11 +x32*y21 + x33*y31 & x31*y12 +x32*y22 + x33*y32 & x31*y13 +x32*y23 + x33*y33
%      	}
%	\end{variables}
%
%	\type{input.matrix}
%	\displayprecision{3}
%    \correctorprecision{2}
%    \field{integer}
%    
%    \lang{de}{
%	    \text{Bestimmen Sie das Produkt der folgenden Matrizen über $\R$:\\
%$\var{a}\cdot\var{b}$}}
%    
%    \begin{answer}
%	    \solution{c}
%      \format{3}{3}
%	\end{answer}
%    
%\end{question}

\begin{question}

	\begin{variables}
		\randint[Z]{x11}{-5}{5}
		\randint[Z]{x12}{-5}{5}
		\randint[Z]{x21}{-5}{5}
		\randint[Z]{x22}{-5}{5}
		\randint[Z]{x31}{-5}{5}
		\randint[Z]{x32}{-5}{5}
	
		\randint[Z]{y11}{-5}{5}
		\randint[Z]{y12}{-5}{5}
		\randint[Z]{y13}{-5}{5}
		\randint[Z]{y21}{-5}{5}
		\randint[Z]{y22}{-5}{5}
		\randint[Z]{y23}{-5}{5}
		
			\matrix[calculate]{a}{
  			x11 & x12  \\ 
  			x21 & x22  \\
  			x31 & x32 
      	}
      		\matrix[calculate]{b}{
  			y11 & y12 & y13 \\ 
  			y21 & y22 & y23
      	}
      		\matrix[calculate]{c}{
  			x11*y11 +x12*y21  & x11*y12 +x12*y22 & x11*y13 +x12*y23\\ 
  x21*y11 +x22*y21 & x21*y12 +x22*y22 & x21*y13 +x22*y23\\
  x31*y11 +x32*y21 & x31*y12 +x32*y22 & x31*y13 +x32*y23	}
	\end{variables}

	\type{input.matrix}
	\displayprecision{3}
    \correctorprecision{2}
    \field{integer}
    
    \lang{de}{
	    \text{Bestimmen Sie das Produkt der folgenden Matrizen über $\R$:\\
$\var{a}\cdot\var{b}$}}

\lang{en}{
	    \text{Determine the product of the following matrices over $\R$:\\
$\var{a}\cdot\var{b}$}}
    
    \begin{answer}
	    \solution{c}
        \lang{de}{
        \explanation{
            Die Matrixmultiplikation ist dann erlaubt,
            wenn die Anzahl der Spalten der ersten Matrix
            mit der Anzahl der Zeilen der zweiten Matrix
            übereinstimmt. Dies ist hier gegeben.
            Nun muss das Format bestimmt werden:
            Die Anzahl der Zeilen der Ergebnismatrix
            entspricht der Anzahl der Zeilen der ersten Matrix,
            also hier 3.
            Die Anzahl der Spalten der Ergebnismatrix
            entspricht der Anzahl der Zeilen der zweiten Matrix,
            also hier 3.
            Man bestimmt die Komponenten der Ergebnismatrix,
            indem man für den $(i,j)$-ten Eintrag das Skalarprodukt 
            aus der $i$-ten Zeile der ersten Matrix und der $j$-ten
            Spalte der zweiten Matrix bestimmt.}}
      \lang{en}{
      \explanation{
      Matrix multiplication is defined, wenn the number of columns of the first matrix is the same as the number of rows 
      of the second matrix. This is given here. Now the dimensions need to be determined:
      The number of rows in the result matrix is the number of rows in the first matrix, so 3.
      The number of columns in the result matrix is the number of rows in the second matrix, so 3.
      The entries are calculated as follows:
      The $(i,j)$th entry in the result matrix is the scalar product (dot product) of the $i$th row of the first matrix with the
      $j$th column of the second matrix. }}
	\end{answer}
    
\end{question}

\end{problem}


\embedmathlet{gwtmathlet}

\end{content}