\documentclass{mumie.problem.gwtmathlet}
%$Id$
\begin{metainfo}
  \name{
    \lang{de}{A05: Matrixoperation}
    \lang{en}{P05: Matrix operation}
  }
  \begin{description} 
 This work is licensed under the Creative Commons License Attribution 4.0 International (CC-BY 4.0)   
 https://creativecommons.org/licenses/by/4.0/legalcode 

    \lang{de}{Rechnen mit Matrizen}
    \lang{en}{Calculating with matrices}
  \end{description}
  \corrector{system/problem/GenericCorrector.meta.xml}
  \begin{components}
    \component{js_lib}{system/problem/GenericMathlet.meta.xml}{mathlet}
  \end{components}
  \begin{links}
  \end{links}
  \creategeneric
\end{metainfo}
\begin{content}
\usepackage{mumie.genericproblem}

\lang{de}{\title{A05: Matrixoperation}}
\lang{en}{\title{P05: Matrix operations}}

\begin{block}[annotation]
	Im Ticket-System: \href{http://team.mumie.net/issues/11521}{Ticket 11521}
\end{block}

\begin{problem}
	%\randomquestionpool{}{}
	\begin{question}
		
		\begin{variables}
			\randint{c2}{1}{4}
			\randint{a2}{2}{4}
			\randint{r}{-1}{1}
			\randint{t}{0}{1}
			\function{b1}{a2}
			\function{c1}{a2}
			\function[calculate]{a1}{a2+r}
			\function[calculate]{b2}{t*c1+(1-t)*c2}
			\function[calculate]{s1}{(a2-b1)^2+(a1-c1)^2+(b2-c2)^2}
			\function[calculate]{s2}{(a2-b1)^2+(a2-c1)^2+(b2-c2)^2}
			\function[calculate]{s3}{(a2-b2)^2+(a1-c2)^2+(b1-c1)^2}
			\function[calculate]{s4}{(a2-b1)^2+(b2-c1)^2}
		\end{variables}
		
		\type{input.generic}
        \field{real}
		\correctorprecision{3}
		\displayprecision{3}
		\lang{de}{
	    \text{Gegeben seien eine $(\var{a1}\times \var{a2})$-Matrix $A$, eine
  $(\var{b1}\times \var{b2})$-Matrix $B$ und eine  $(\var{c1}\times \var{c2})$-Matrix $C$. 
  Die Einträge aller Matrizen seien Elemente eines Körpers $\mathbb{K}$. Welche der folgenden Ausdrücke sind definiert?}
	    }
     \lang{en}{
	    \text{Let a$(\var{a1}\times \var{a2})$-matrix $A$, a
  $(\var{b1}\times \var{b2})$-matrix $B$ und eine  $(\var{c1}\times \var{c2})$-matrix $C$ be given. 
  The entries of all matrices are elements of a field $\mathbb{K}$. Which of the following terms are defined?}
	    }
	    %\permuteAnswers{1, 2, 3} 
	    %http://team.mumie.net/projects/support/wiki/DifferentAnswerType
	    \begin{answer}
            \type{mc.multiple}

            \begin{choice}
	    	\text{$A\cdot B+C$}
	    	\solution{compute}
	    	\iscorrect{s1}{=}{0} 
	    	\end{choice}
            
	    	\begin{choice}
	    	\text{$A\cdot (B+C)$}
	    	\solution{compute}
	    	\iscorrect{s2}{=}{0} 
	    	\end{choice}
            
	    	\begin{choice}
	    	\text{$B+C\cdot A$}
	    	\solution{compute}
	    	\iscorrect{s3}{=}{0} 
	    	\end{choice}
            
	    	\begin{choice}
	    	\text{$A\cdot B\cdot C$}
	    	\solution{compute}
	    	\iscorrect{s4}{=}{0}
	    	\end{choice}
            
            \lang{de}{
            \explanation {
                $A\cdot B+C$:
                Die Spaltenanzahl von $A$ muss
                gleich der Zeilenanzahl von $B$ sein.
                Das Produkt muss die gleiche Dimensionierung wie Matrix $C$ haben.
                ~\\~\\

                $A\cdot (B+C)$:
                Die Matrizen $B$ und $C$ müssen gleich dimensioniert sein.
                Die Spaltenanzahl von $A$ muss mit der Zeilenanzahl von $B$
                (bzw. $C$) übereinstimmen.
                ~\\~\\

                $B+C\cdot A$:
                Die Spaltenanzahl von $C$ muss gleich der Zeilenanzahl von $A$ sein.
                Das Produkt muss die gleiche Dimensionierung wie Matrix $B$ haben.
                ~\\~\\

                $A\cdot B\cdot C$:
                Die Spaltenanzahl von $A$ muss 
                gleich der Zeilenanzahl von $B$ sein.
                Weiterhin muss die Spaltenanzahl von $B$ 
                gleich der Zeilenanzahl von $C$ sein.
            }}

      \lang{en}{
            \explanation {
                $A\cdot B+C$:
                The number of columns in $A$ must be the same as the number of rows in $B$. The product must have the
                same dimensions as matrix $C$.
                ~\\~\\

                $A\cdot (B+C)$:
                The matrices $B$ and $C$ must have the same dimensions.
                The number of columns in $A$ must be the same as the number of rows in $B$ (and $C$).
                ~\\~\\

                $B+C\cdot A$:
                The number of columns in $C$ must be the same as the number of rows in $A$.
                The product must have the same dimensions as matrix $B$.
                ~\\~\\

                $A\cdot B\cdot C$:
                The number of columns in $A$ must be the same as the numbers of $rows$ in $B$.
                Furthermore, the number of columns in $B$ must be the same as the number of rows in $C$.
            }}
            
	    \end{answer}    
        

	    
	    %generic viz: http://team.mumie.net/projects/support/wiki/Example
	    
	\end{question}
\end{problem}

\embedmathlet{mathlet}

\end{content}