\documentclass{mumie.problem.gwtmathlet}
%$Id$
\begin{metainfo}
  \name{
    \lang{de}{A04: Transponierte}
    \lang{en}{P04: The transpose}
  }
  \begin{description} 
 This work is licensed under the Creative Commons License Attribution 4.0 International (CC-BY 4.0)   
 https://creativecommons.org/licenses/by/4.0/legalcode 

    \lang{de}{Beschreibung}
    \lang{en}{Description}
  \end{description}
  \corrector{system/problem/GenericCorrector.meta.xml}
  \begin{components}
    \component{js_lib}{system/problem/GenericMathlet.meta.xml}{gwtmathlet}
  \end{components}
  \begin{links}
  \end{links}
  \creategeneric
\end{metainfo}
\begin{content}
\usepackage{mumie.genericproblem}

\lang{de}{
	\title{A04: Transponierte}
}

\lang{de}{
	\title{P04: The transpose}
}


\begin{block}[annotation]
	Im Ticket-System: \href{http://team.mumie.net/issues/11104}{Ticket 11104}
\end{block}


%Bestimmen Sie für die folgenden Matrizen jeweils die transponierte Matrix:
\lang{de}{
Lösen Sie die folgenden Aufgaben:}
\lang{en}{
Solve the following exercises:}

\begin{problem}

    
\begin{question}

	\begin{variables}
		\randint[Z]{a}{-9}{9}
		\randint[Z]{b}{-9}{9}
		\randint[Z]{c}{-9}{9}
		\randint[Z]{d}{-9}{9}
		
		
			\matrix[calculate]{aa}{
  			a & b \\ 
  			c & d
      	}
      	
		\matrix[calculate]{m}{
		a & c \\
		b & d
      } 
	\end{variables}

	\type{input.matrix}
	\displayprecision{3}
    \correctorprecision{2}
    \field{integer}
    
    \lang{de}{
	    \text{
$\var{aa}^T$}}
    
    \begin{answer}
	    \solution{m}
      \format{2}{2}
      \lang{de}{
      \explanation{
            Für die Bestimmung der Transponierten einer Matrix,
            wird die Rolle der Zeilen und Spalten vertauscht.}}
      \lang{en}{
      \explanation{
      To determine the transpose of a matrix,
            the roles of the rows and columns are reversed. }}
	\end{answer}
    
\end{question}

\begin{question}

	\begin{variables}
		\randint[Z]{x11}{-9}{9}
		\randint[Z]{x12}{-9}{9}
		\randint[Z]{x13}{-9}{9}
		\randint[Z]{x21}{-9}{9}
		\randint[Z]{x22}{-9}{9}
		\randint[Z]{x23}{-9}{9}
		\randint[Z]{x31}{-9}{9}
		\randint[Z]{x32}{-9}{9}
		\randint[Z]{x33}{-9}{9}
		
			\matrix[calculate]{a}{
  			x11 & x12 & x13 \\ 
  			x21 & x22 & x23 \\
  			x31 & x32 & x33
      	}
      	\matrix[calculate]{c}{
  			x11 & x21 & x31 \\ 
  			x12 & x22 & x32 \\
  			x13 & x23 & x33
      	}
	\end{variables}

	\type{input.matrix}
	\displayprecision{3}
    \correctorprecision{2}
    \field{integer}
    
    \lang{de}{
	    \text{
$\var{a}^T$}}
    
    \begin{answer}
	    \solution{c}
      \format{3}{3}
      \lang{de}{
      \explanation{
            Für die Bestimmung der Transponierten einer Matrix,
            wird die Rolle der Zeilen und Spalten vertauscht.
        }}
        \lang{en}{
      \explanation{
      To determine the transpose of a matrix,
            the roles of the rows and columns are reversed. }}
	\end{answer}
    
\end{question}

\begin{question}

	\begin{variables}
		\randint[Z]{x11}{-9}{9}
		\randint[Z]{x12}{-9}{9}
		\randint[Z]{x21}{-9}{9}
		\randint[Z]{x22}{-9}{9}
		\randint[Z]{x31}{-9}{9}
		\randint[Z]{x32}{-9}{9}
			\matrix[calculate]{a}{
  			x11 & x12  \\ 
  			x21 & x22  \\
  			x31 & x32 
      	}
      		\matrix[calculate]{b}{
  			x11 & x21 & x31 \\ 
  			x12 & x22 & x32
      	}
	\end{variables}

	\type{input.matrix}
	\displayprecision{3}
    \correctorprecision{2}
    \field{integer}
    
    \lang{de}{
	    \text{
$\var{a}^T$}}
    
    \begin{answer}
	    \solution{b}
     \lang{de}{
        \explanation{
            Für die Bestimmung der Transponierten einer Matrix,
            wird die Rolle der Zeilen und Spalten vertauscht.
        }}
        \lang{en}{
      \explanation{
      To determine the transpose of a matrix,
            the roles of the rows and columns are reversed. }}
	\end{answer}
    
\end{question}


\begin{question}

	\begin{variables}
		\randint[Z]{x11}{-9}{9}
		\randint[Z]{x12}{-9}{9}
		\randint[Z]{x21}{-9}{9}
		\randint[Z]{x22}{-9}{9}
		\randint[Z]{x31}{-9}{9}
		\randint[Z]{x32}{-9}{9}
			\matrix[calculate]{a}{
  			x11 & x12  \\ 
  			x21 & x22  \\
  			x31 & x32 
      	}
      		\matrix[calculate]{b}{
  			x11 & x21 & x31 \\ 
  			x12 & x22 & x32
      	}
	\end{variables}

	\type{input.matrix}
	\displayprecision{3}
    \correctorprecision{2}
    \field{integer}
    
    \lang{de}{
	    \text{
$\var{b}^T$}}
    
    \begin{answer}
	    \solution{a}
     \lang{de}{
        \explanation{
            Für die Bestimmung der Transponierten einer Matrix,
            wird die Rolle der Zeilen und Spalten vertauscht.}}
        
        \lang{en}{
      \explanation{
      To determine the transpose of a matrix,
            the roles of the rows and columns are reversed. }}
	\end{answer}
    
\end{question}




\begin{question}

	\begin{variables}
		\randint[Z]{x11}{-9}{9}
		\randint[Z]{x12}{-9}{9}
        \randint[Z]{x12i}{-9}{9}
		\randint[Z]{x21}{-9}{9}
		\randint[Z]{x22}{-9}{9}
		\randint[Z]{x31}{-9}{9}
		\randint[Z]{x32}{-9}{9}
		\randint[Z]{x32i}{-9}{9}
			\matrix[calculate]{a}{
  			x11 & x12+x12i*i  \\ 
  			x21 & x22  \\
  			x31 & x32+x32i*i 
      	}
      		\matrix[calculate]{b}{
  			x11 & x21 & x31 \\ 
  			x12+x12i*i & x22 & x32+x32i*i
      	}
	\end{variables}

	\type{input.matrix}
	\displayprecision{3}
    \correctorprecision{2}
    \field{complex}
    
    \lang{de}{
	    \text{
$\var{b}^T$}}
    
    \begin{answer}
	    \solution{a}
     \lang{de}{
        \explanation{
            Für die Bestimmung der Transponierten einer Matrix,
            wird die Rolle der Zeilen und Spalten vertauscht.
        }}
        \lang{en}{
      \explanation{
      To determine the transpose of a matrix,
            the roles of the rows and columns are reversed. }}
	\end{answer}
    
\end{question}



\begin{question}

	\begin{variables}
		\randint[Z]{x11}{-5}{5}
		\randint[Z]{x12}{-5}{5}
		\randint[Z]{x21}{-5}{5}
		\randint[Z]{x22}{-5}{5}
		\randint[Z]{x31}{-5}{5}
		\randint[Z]{x32}{-5}{5}
		\randint[Z]{y11}{-5}{5}
		\randint[Z]{y12}{-5}{5}
		\randint[Z]{y21}{-5}{5}
		\randint[Z]{y22}{-5}{5}
		\randint[Z]{y31}{-5}{5}
		\randint[Z]{y32}{-5}{5}
        \randint[Z]{f}{2}{4}
			\matrix[calculate]{a}{
  			x11 & x12  \\ 
  			x21 & x22  \\
  			x31 & x32 
      	}
      		\matrix[calculate]{b}{
  			y11 & y21 & y31 \\ 
  			y12 & y22 & y32
      	}
        \matrix[calculate]{c}{
  			f*y11+x11 & f*y12+x12 \\ 
  			f*y21+x21 & f*y22+x22 \\ 
            f*y31+x31 & f*y32+x32
      	}
	\end{variables}

	\type{input.matrix}
	\displayprecision{3}
    \correctorprecision{2}
    \field{integer}
    
    \lang{de}{
	    \text{
$\var{f} \cdot \var{b}^T + \var{a}$}}
    
    \begin{answer}
	    \solution{c}
     \lang{de}{
        \explanation{
            Für die Bestimmung der Transponierten einer Matrix,
            wird die Rolle der Zeilen und Spalten vertauscht.
            Weiterhin wird bei dieser Aufgabe die skalare Multiplikation
            mit einer Matrix, sowie die Addition zweier Matrizen
            durchgeführt.
        }}
        \lang{en}{
      \explanation{
      To determine the transpose of a matrix,
            the roles of the rows and columns are reversed.
            Furthermore, we need scalar multiplication and matrix addition in this exercise.}}
	\end{answer}
    
\end{question}


\end{problem}


%TODO:

%\begin{problem}
%\end{problem}



\embedmathlet{gwtmathlet}

\end{content}