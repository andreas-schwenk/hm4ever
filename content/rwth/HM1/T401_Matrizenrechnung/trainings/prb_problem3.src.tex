\documentclass{mumie.problem.gwtmathlet}
%$Id$
\begin{metainfo}
  \name{
    \lang{de}{A03: Matrizenoperation}
    \lang{en}{P03: Matrix operations}
  }
  \begin{description} 
 This work is licensed under the Creative Commons License Attribution 4.0 International (CC-BY 4.0)   
 https://creativecommons.org/licenses/by/4.0/legalcode 

    \lang{de}{Rechnen mit Matrizen}
    \lang{en}{Calculating with matrices}
  \end{description}
  \corrector{system/problem/GenericCorrector.meta.xml}
  \begin{components}
    \component{js_lib}{system/problem/GenericMathlet.meta.xml}{gwtmathlet}
  \end{components}
  \begin{links}
  \end{links}
  \creategeneric
\end{metainfo}
\begin{content}
\usepackage{mumie.genericproblem}

\lang{de}{
	\title{A03: Matrizenoperation}
}

\lang{en}{
	\title{P03: Matrix operations}
}

\begin{block}[annotation]
	Im Ticket-System: \href{http://team.mumie.net/issues/11103}{Ticket 11103}
\end{block}



\begin{problem}

\begin{question}

	\begin{variables}
		\randint{a}{-2}{2}
		\randint{b}{-2}{2}
		\randint{c}{-2}{2}
		\randint{d}{-2}{2}
		
		\randint{ee}{-2}{2}
		\randint{f}{-2}{2}
		\randint{g}{-2}{2}
		\randint{h}{-2}{2}
		
		\randint{u}{-2}{2}
		\randint{v}{-2}{2}
		\randint{w}{-2}{2}
		\randint{x}{-2}{2}
		
		\randint{y}{-2}{2}
		\randint{z}{-2}{2}
		
		\randint{r}{-2}{-1}
		
		\matrix[calculate]{aa}{
  			a & b \\ 
  			c & d
      	}
      	
      	\matrix[calculate]{bb}{
  			ee & f \\ 
  			g & h
      	}
		
		\matrix[calculate]{zz}{
			u & v\\
			w & x
		}
			
		\matrix[calculate]{uu}{
			y\\
			z
		}	
		
		\matrix[calculate]{m}{
  			((a+ee*r)*u+(b+f*r)*w)*y+((a+ee*r)*v+(b+f*r)*x)*z\\
  			((c+g*r)*u+(d+h*r)*w)*y+((c+g*r)*v+(d+h*r)*x)*z
      	} 
		
		
		
	\end{variables}

	\type{input.matrix}
	
    \field{rational}
    
    \lang{de}{
	    \text{Bestimmen Sie:\\$
\left(\var{aa}+(\var{r})\cdot\var{bb}\right)\cdot\var{zz}\cdot\var{uu} =$}}

\lang{en}{
	    \text{Determine:\\$
\left(\var{aa}+(\var{r})\cdot\var{bb}\right)\cdot\var{zz}\cdot\var{uu} =$}}
    
    \begin{answer}
	    \solution{m}
      \format{2}{1}
      
      \lang{de}{
      \explanation{
            In dieser Aufgabe werden die Operationen Matrixmultiplikation,
            Matrixaddition und die Skalarmultiplikation durchgeführt.
            Zunächst berechnet man den Term in den Klammern
            also die skalare Multiplikation, sowie die Summe.
            Man erhält eine $2 \times 2$-Matrix.
            Danach kann die Matrixmultiplikation mit den verbleibenden
            beiden Matrizen nacheinander durchgeführt werden.
            Man erhält zunächst eine $2 \times 2$-Matrix und als
            Ergebnis eine eine $2 \times 1$-Matrix.
            Da die Matrixmultiplikation assoziativ ist, 
            darf man alternativ auch zunächst die beiden hinteren
            Matrizen miteinander multiplizieren, denn: 
            $(A \cdot B) \cdot C = A \cdot (B \cdot C)$.
            Man beachte jedoch stets, dass die Matrixmultiplikation nicht 
            kommutativ ist. Im Allgemeinen gilt: $A \cdot B \neq B \cdot A$.}}
            
      \lang{en}{
      \explanation{
      In this exercise, the operations matrix addition, matrix multiplication and scalar multiplication are perfomed combined.
      First of all, the term in the brackets with scalar multiplication and then addition of the matrices, will be calculated.
      The result is a $(2\times 2)$-matrix. 
      After that, the two matrix multiplications need to be performed. First, we get a $(2\times 2)$-matrix and after multiplying with
      the column vector, we end up with a $(2\times 1)$-matrix.
      Since matrix multiplication is associativ, it is allowed to multiply back matrices first, because:
      $(A \cdot B) \cdot C = A \cdot (B \cdot C)$.
      Consider, that matrix multiplication is not commutative, in general: $A \cdot B \neq B \cdot A$. }}
      
	\end{answer}
    
\end{question}

\end{problem}


\embedmathlet{gwtmathlet}

\end{content}