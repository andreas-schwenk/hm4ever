\documentclass{mumie.problem.gwtmathlet}
%$Id$
\begin{metainfo}
  \name{
    \lang{de}{A01: Matrixaddition}
    \lang{en}{P01: Matrix addition}
  }
  \begin{description} 
 This work is licensed under the Creative Commons License Attribution 4.0 International (CC-BY 4.0)   
 https://creativecommons.org/licenses/by/4.0/legalcode 

    \lang{de}{Rechnen mit Matrizen}
    \lang{en}{Calculating with matrices}
  \end{description}
  \corrector{system/problem/GenericCorrector.meta.xml}
  \begin{components}
    \component{js_lib}{system/problem/GenericMathlet.meta.xml}{gwtmathlet}
  \end{components}
  \begin{links}
  \end{links}
  \creategeneric
\end{metainfo}
\begin{content}
\usepackage{mumie.genericproblem}


	\title{\lang{de}{A01: Matrixaddition} \lang{en}{P01: Matrix addition}}


\begin{block}[annotation]
	Im Ticket-System: \href{http://team.mumie.net/issues/11101}{Ticket 11101}
\end{block}



\begin{problem}

    
\begin{question}

	\begin{variables}
		\randint[Z]{x11}{-9}{9}
		\randint[Z]{x12}{-9}{9}
		\randint[Z]{x13}{-9}{9}
		\randint[Z]{x21}{-9}{9}
		\randint[Z]{x22}{-9}{9}
		\randint[Z]{x23}{-9}{9}
		\randint[Z]{x31}{-9}{9}
		\randint[Z]{x32}{-9}{9}
		\randint[Z]{x33}{-9}{9}
	
		\randint[Z]{y11}{-9}{9}
		\randint[Z]{y12}{-9}{9}
		\randint[Z]{y13}{-9}{9}
		\randint[Z]{y21}{-9}{9}
		\randint[Z]{y22}{-9}{9}
		\randint[Z]{y23}{-9}{9}
		\randint[Z]{y31}{-9}{9}
		\randint[Z]{y32}{-9}{9}
		\randint[Z]{y33}{-9}{9}
		
			\matrix[calculate]{a}{
  			x11 & x12 & x13 \\ 
  			x21 & x22 & x23 \\
  			x31 & x32 & x33
      	}
      		\matrix[calculate]{b}{
  			y11 & y12 & y13 \\ 
  			y21 & y22 & y23 \\
  			y31 & y32 & y33
      	}
      		\matrix[calculate]{c}{
  			x11+y11 & x12+y12 & x13+y13 \\ 
  			x21+y21 & x22+y22 & x23+y23 \\
  			x31+y31 & x32+y32 & x33+y33
      	}
	\end{variables}

	\type{input.matrix}
	\displayprecision{3}
    \correctorprecision{2}
    \field{integer}
    
    \lang{de}{
	    \text{
	    Bestimmen Sie die Summe der folgenden beiden Matrizen über $\R$:\\
$\var{a}+\var{b}$}}

\lang{en}{
	    \text{
	    Determine the sum of the following matrices over $\R$:\\
$\var{a}+\var{b}$}}
    
    \begin{answer}
	    \solution{c}
      \format{3}{3}
      \lang{de}{
      \explanation{
            Beide Matrizen haben dasselbe Format, 
            also die gleiche Anzahl an Zeilen und Spalten. 
            Folglich ist die Matrizenaddition erlaubt. 
            Die Ergebnismatrix hat ebenso dieses Format.
            Man bestimmt die Ergebnismatrix, 
            indem man die Summe komponentenweise bestimmt:
            Der $(i,j)$-te Koeffizient der Summe ist
            gleich der Summe der $(i,j)$-ten Koeffizienten
            der beiden zu addierenden Matrizen.}}
      \lang{en}{
      \explanation{
      Both matrices have the same number of rows and columns, which is why matrix addition is possible.
      Since matrix addition is defined componentwise, the sum has also the same number of rows and columns.
      The entries of the result matrix are calculated as follows: The $(i,j)$th entry of the sum is the sum of the $(i,j)$th
      entries of the matrices added.}}
	\end{answer}
    
\end{question}



\begin{question}

	\begin{variables}
		\randint[Z]{x11}{-5}{5}
		\randint[Z]{x12}{-5}{5}
		\randint[Z]{x13}{-5}{5}
		\randint[Z]{x21}{-5}{5}
		\randint[Z]{x22}{-5}{5}
		\randint[Z]{x23}{-5}{5}
		\randint[Z]{x31}{-5}{5}
		\randint[Z]{x32}{-5}{5}
		\randint[Z]{x33}{-5}{5}

        \randint[Z]{x11i}{-5}{5}
		\randint[Z]{x12i}{-5}{5}
		\randint[Z]{x13i}{-5}{5}
		\randint[Z]{x21i}{-5}{5}
		\randint[Z]{x22i}{-5}{5}
		\randint[Z]{x23i}{-5}{5}
		\randint[Z]{x31i}{-5}{5}
		\randint[Z]{x32i}{-5}{5}
		\randint[Z]{x33i}{-5}{5}
	
		\randint[Z]{y11}{-5}{5}
		\randint[Z]{y12}{-5}{5}
		\randint[Z]{y13}{-5}{5}
		\randint[Z]{y21}{-5}{5}
		\randint[Z]{y22}{-5}{5}
		\randint[Z]{y23}{-5}{5}
		\randint[Z]{y31}{-5}{5}
		\randint[Z]{y32}{-5}{5}
		\randint[Z]{y33}{-5}{5}

		\randint[Z]{y11i}{-5}{5}
		\randint[Z]{y12i}{-5}{5}
		\randint[Z]{y13i}{-5}{5}
		\randint[Z]{y21i}{-5}{5}
		\randint[Z]{y22i}{-5}{5}
		\randint[Z]{y23i}{-5}{5}
		\randint[Z]{y31i}{-5}{5}
		\randint[Z]{y32i}{-5}{5}
		\randint[Z]{y33i}{-5}{5}
        
			\matrix[calculate]{a}{
  			x11+x11i*i & x12+x12i*i \\ 
  			x21+x21i*i & x22+x22i*i
      	}
      		\matrix[calculate]{b}{
  			y11+y11i*i & y12+y12i*i \\ 
  			y21+y21i*i & y22+y22i*i
      	}
      		\matrix[calculate]{c}{
  			x11+y11+x11i*i+y11i*i & x12+y12+x12i*i+y12i*i \\ 
  			x21+y21+x21i*i+y21i*i & x22+y22+x22i*i+y22i*i
      	}
	\end{variables}

	\type{input.matrix}
	\displayprecision{3}
    \correctorprecision{2}
    \field{complex}
    
    \lang{de}{
	    \text{
	    Bestimmen Sie die Summe der folgenden beiden Matrizen über $\C$:\\
$\var{a}+\var{b}$}}

\lang{en}{
	    \text{
	    Determine the sum of the following matrices over $\C$:\\
$\var{a}+\var{b}$}}
    
    \begin{answer}
	    \solution{c}
      \format{2}{2}
      \lang{de}{
      \explanation{
            Beide Matrizen haben dasselbe Format, 
            also die gleiche Anzahl an Zeilen und Spalten. 
            Folglich ist die Matrizenaddition erlaubt. 
            Die Ergebnismatrix hat ebenso dieses Format.
            Man bestimmt die Ergebnismatrix, 
            indem man die Summe komponentenweise bestimmt:
            Der $(i,j)$-te Koeffizient der Summe ist
            gleich der Summe der $(i,j)$-ten Koeffizienten
            der beiden zu addierenden Matrizen.
            Da hier über dem Körper der komplexen Zahlen
            gearbeitet wird, ist die Addition jeweils
            komplexwertig durchzuführen:
            Man addiert also jeweils den Real- und den
            Imaginäranteil.}}
      \lang{en}{
      \explanation{
      Both matrices have the same number of rows and columns, which is why matrix addition is possible.
      Since matrix addition is defined componentwise, the sum has also the same number of rows and columns.
      The entries of the result matrix are calculated as follows: The $(i,j)$th entry of the sum is the sum of the $(i,j)$th
      entries of the matrices added.
      Because the entries are elements of $\C$, they are added by separately adding their real and imaginary parts.}}
	\end{answer}
    
\end{question}


\end{problem}


\embedmathlet{gwtmathlet}

\end{content}