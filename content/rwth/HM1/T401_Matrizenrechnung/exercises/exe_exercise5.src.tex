\documentclass{mumie.element.exercise}
%$Id$
\begin{metainfo}
  \name{
    \lang{de}{Ü06: Matrizenoperation}
    \lang{en}{Ex06: Matrix operations}
  }
  \begin{description} 
 This work is licensed under the Creative Commons License Attribution 4.0 International (CC-BY 4.0)   
 https://creativecommons.org/licenses/by/4.0/legalcode 

    \lang{de}{Rechnen mit Matritzen}
    \lang{en}{Calculating with matrices}
  \end{description}
  \begin{components}
  \end{components}
  \begin{links}
  \end{links}
  \creategeneric
\end{metainfo}
\begin{content}
\usepackage{mumie.ombplus}

\title{\lang{de}{Ü06: Matrizenoperation} \lang{en}{Ex06: Matrix operations}}

\begin{block}[annotation]
  Im Ticket-System: \href{http://team.mumie.net/issues/11515}{Ticket 11515}
\end{block}

%######################################################FRAGE_TEXT
\lang{en}{ We consider the matrices 
\[ A=\begin{pmatrix}
2 & \frac{1}{2} \\ 0 & 3i \end{pmatrix}, \quad B=\begin{pmatrix}
\frac{3}{2} & 2 & 1 \\ -1 & 2 & 0
\end{pmatrix}\quad \text{und}\quad C=\begin{pmatrix}
0 & 2 \\ 3 & 1 \\ 4 & 0 \end{pmatrix}. \]
Which of the following forms are defined? When defined, determine the result.
\begin{enumerate}
\item[(a)] $A+ \frac{2}{3}\cdot B$ \item[(b)] $2A+B\cdot C$ \item[(c)] $C\cdot (A+B)$
\item[(d)] $B\cdot A-A\cdot C$ \item[(e)] $C\cdot A\cdot B$
\end{enumerate}}

\lang{de}{ Wir betrachten die Matrizen 
\[ A=\begin{pmatrix}
2 & \frac{1}{2} \\ 0 & 3i \end{pmatrix}, \quad B=\begin{pmatrix}
\frac{3}{2} & 2 & 1 \\ -1 & 2 & 0
\end{pmatrix}\quad \text{und}\quad C=\begin{pmatrix}
0 & 2 \\ 3 & 1 \\ 4 & 0 \end{pmatrix}. \]
Welche der folgenden Ausdrücke sind definiert? Bestimmen Sie gegebenenfalls das Ergebnis.
\begin{enumerate}
\item[(a)] $A+ \frac{2}{3}\cdot B$ \item[(b)] $2A+B\cdot C$ \item[(c)] $C\cdot (A+B)$
\item[(d)] $B\cdot A-A\cdot C$ \item[(e)] $C\cdot A\cdot B$
\end{enumerate}}

%##################################################ANTWORTEN_TEXT
\begin{tabs*}[\initialtab{0}\class{exercise}]

  %++++++++++++++++++++++++++++++++++++++++++START_TAB_X
  \tab{\lang{de}{    Antwort    } \lang{en}{Answer}}
  \begin{incremental}[\initialsteps{1}]
  
  	 %----------------------------------START_STEP_X
    \step 
    \lang{de}{   Die Ausdrücke (b) und (e) sind definiert, die anderen nicht.\\ Es sind
\[  2A+B\cdot C=\begin{pmatrix}
14 & 6\\ 6 & 6i \end{pmatrix}\]
und
\[ C\cdot A\cdot B=\begin{pmatrix}
 -6i & 12i & 0\\
\frac{15}{2}-3i & 15+6i & 6\\
 10 & 20 & 8
\end{pmatrix} .\]}

\lang{en}{   The terms (b) and (e) are defined, the others are not.\\ We have
\[  2A+B\cdot C=\begin{pmatrix}
14 & 6\\ 6 & 6i \end{pmatrix}\]
and
\[ C\cdot A\cdot B=\begin{pmatrix}
 -6i & 12i & 0\\
\frac{15}{2}-3i & 15+6i & 6\\
 10 & 20 & 8
\end{pmatrix} .\]}
  	 %------------------------------------END_STEP_X
 
  \end{incremental}
  %++++++++++++++++++++++++++++++++++++++++++++END_TAB_X


  %++++++++++++++++++++++++++++++++++++++++++START_TAB_X
  \tab{\lang{de}{    Lösung    } \lang{en}{Solution}}
  \begin{incremental}[\initialsteps{1}]
  
  	 %----------------------------------START_STEP_X
    \step 
    \lang{de}{Additionen von Matrizen sind nur definiert, wenn die Matrizen gleich dimensioniert sind (also die gleiche Anzahl an Zeilen und Spalten haben). 
Da $\frac{2}{3}\cdot B$ eine $(2\times 3)$-Matrix
und $A$ eine $(2\times 2)$-Matrix ist, ist $A+\frac{2}{3}\cdot B$ nicht definiert.

Ebenso ist $A+B$ nicht definiert, weshalb auch der Ausdruck in (c) $C\cdot (A+B)$ nicht definiert ist.

Das Produkt zweier Matrizen ist nur definiert, wenn die Anzahl der Spalten der ersten Matrix gleich der Anzahl der Zeilen der zweiten Matrix ist. Das Produkt $B\cdot A$ ist daher nicht definiert, weshalb auch
der gesamte Ausdruck in (d) $B\cdot A-A\cdot C$ nicht definiert ist.

Um zu sehen, dass der Ausdruck in (b) definiert ist, bemerken wir zunächst, dass
das Produkt der $(2\times 3)$-Matrix $B$ mit der $(3\times 2)$-Matrix $C$ definiert ist und das Ergebnis
eine $(2\times 2)$-Matrix ist. 
Dieses kann mit $2A$ addiert werden. 
Das gesamte Ergebnis berechnet sich als
\begin{eqnarray*}
  2A+B\cdot C &=& 2\cdot \begin{pmatrix}
2 & \frac{1}{2} \\ 0 & 3i \end{pmatrix}+\begin{pmatrix}
\frac{3}{2} & 2 & 1 \\ -1 & 2 & 0
\end{pmatrix} \cdot \begin{pmatrix}
0 & 2 \\ 3 & 1 \\ 4 & 0 \end{pmatrix} \\
&=& \begin{pmatrix}
4 & 1 \\ 0 & 6i \end{pmatrix}+  \begin{pmatrix}
10 & 5 \\ 6 & 0 \end{pmatrix} \\
&=& \begin{pmatrix}
14 & 6\\ 6 & 6i \end{pmatrix}.
\end{eqnarray*}
}\step 

  \lang{en}{Matrices can only be added, when they have the same dimension (same number of columns and rows). 
 $\frac{2}{3}\cdot B$ is a $(2\times 3)$-matrix
and $A$ a $(2\times 2)$-matrix, so $A+\frac{2}{3}\cdot B$ is not defined.

Likewise, $A+B$ is not defined, which is why the form in (c) $C\cdot (A+B)$ is not defined.

The product of two matrices is only defined, when the number of columns in the first matrix is the same as the number
of rows in the second matrix. Because of that, $B\cdot A$ is not defined, which is why the entire term in (d) $B\cdot A-A\cdot C$ is not defined.

We see, that the term in (b) is defined, because the product of the $(2\times 3)$-matrix $B$ with $(3\times 2)$-matrix $C$ is defined
and the result is a $(2\times 2)$-matrix, which can be added to $2A$. 
Hence we can calculate
\begin{eqnarray*}
  2A+B\cdot C &=& 2\cdot \begin{pmatrix}
2 & \frac{1}{2} \\ 0 & 3i \end{pmatrix}+\begin{pmatrix}
\frac{3}{2} & 2 & 1 \\ -1 & 2 & 0
\end{pmatrix} \cdot \begin{pmatrix}
0 & 2 \\ 3 & 1 \\ 4 & 0 \end{pmatrix} \\
&=& \begin{pmatrix}
4 & 1 \\ 0 & 6i \end{pmatrix}+  \begin{pmatrix}
10 & 5 \\ 6 & 0 \end{pmatrix} \\
&=& \begin{pmatrix}
14 & 6\\ 6 & 6i \end{pmatrix}.
\end{eqnarray*}
}\step 

    \lang{de}{  
Auch der Ausdruck in (e) $C\cdot A\cdot B$ ist definiert und das Ergebnis ist
\begin{eqnarray*}
 C\cdot A\cdot B &=& \begin{pmatrix}
0 & 2 \\ 3 & 1 \\ 4 & 0 \end{pmatrix}\cdot  \begin{pmatrix}
2 & \frac{1}{2} \\ 0 & 3i \end{pmatrix}\cdot \begin{pmatrix}
\frac{3}{2} & 2 & 1 \\ -1 & 2 & 0
\end{pmatrix}  \\
&=&  \begin{pmatrix}
0 & 6i \\ 6 & \frac{3}{2}+3i\\ 8 & 2 \end{pmatrix}\cdot  \begin{pmatrix}
\frac{3}{2} & 2 & 1 \\ -1 & 2 & 0
\end{pmatrix}  \\
&=& \begin{pmatrix}
 -6i & 12i &  0\\
\frac{15}{2}-3i & 15+6i & 6\\
 10 & 20 &  8
\end{pmatrix}.
\end{eqnarray*}    }

\lang{en}{  
The term in (e) $C\cdot A\cdot B$ is defined, because for each multiplication the number of the coloumns of the first matrix
is the same as the amount of the rows in the second matrix. The product can be determined to
\begin{eqnarray*}
 C\cdot A\cdot B &=& \begin{pmatrix}
0 & 2 \\ 3 & 1 \\ 4 & 0 \end{pmatrix}\cdot  \begin{pmatrix}
2 & \frac{1}{2} \\ 0 & 3i \end{pmatrix}\cdot \begin{pmatrix}
\frac{3}{2} & 2 & 1 \\ -1 & 2 & 0
\end{pmatrix}  \\
&=&  \begin{pmatrix}
0 & 6i \\ 6 & \frac{3}{2}+3i\\ 8 & 2 \end{pmatrix}\cdot  \begin{pmatrix}
\frac{3}{2} & 2 & 1 \\ -1 & 2 & 0
\end{pmatrix}  \\
&=& \begin{pmatrix}
 -6i & 12i &  0\\
\frac{15}{2}-3i & 15+6i & 6\\
 10 & 20 &  8
\end{pmatrix}.
\end{eqnarray*}    }
  	 %------------------------------------END_STEP_X
 
  \end{incremental}
  %++++++++++++++++++++++++++++++++++++++++++++END_TAB_X



%#############################################################ENDE
\end{tabs*}
\end{content}