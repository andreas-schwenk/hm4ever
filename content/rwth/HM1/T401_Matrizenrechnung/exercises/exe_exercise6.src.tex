\documentclass{mumie.element.exercise}
%$Id$
\begin{metainfo}
  \name{
    \lang{de}{Ü07: Matrixgleichung}
    \lang{en}{Ex07: Matrix equation}
  }
  \begin{description} 
 This work is licensed under the Creative Commons License Attribution 4.0 International (CC-BY 4.0)   
 https://creativecommons.org/licenses/by/4.0/legalcode 

    \lang{de}{Rechnen mit Matritzen}
    \lang{en}{Calculating with matrices}
  \end{description}
  \begin{components}
  \end{components}
  \begin{links}
  \end{links}
  \creategeneric
\end{metainfo}
\begin{content}
\usepackage{mumie.ombplus}

\title{\lang{de}{Ü07: Matrixgleichung} \lang{en}{Ex07: Matrix equation}}

\begin{block}[annotation]
  Im Ticket-System: \href{http://team.mumie.net/issues/11514}{Ticket 11514}
\end{block}

%######################################################FRAGE_TEXT
\lang{de}{ $A$ und $B$ seien $(3\times 2)$-Matrizen über $\R$ und $C$ eine $(2\times 4)$-Matrix über $\C$. Welche der folgenden Ausdrücke sind gleich der transponierten Matrix zu $(A+3B)\cdot C$?

\begin{enumerate}[(a)]
\item (a) $(A^T+3B^T)\cdot C^T$,
\item (b) $3C^T\cdot B^T+C^T\cdot A^T$,
\item (c) $C^T\cdot (A^T+3B^T)$,
\item (d) $(A\cdot C)^T+3B^T\cdot C^T$.
\end{enumerate} }

\lang{en}{ Let $A$ and $B$ be $(3\times 2)$-matrices over $\R$ and $C$ a $(2\times 4)$-matrix over $\C$. Which of the following terms are equivalent to the transpose of the matrix?

\begin{enumerate}[(a)]
\item (a) $(A^T+3B^T)\cdot C^T$,
\item (b) $3C^T\cdot B^T+C^T\cdot A^T$,
\item (c) $C^T\cdot (A^T+3B^T)$,
\item (d) $(A\cdot C)^T+3B^T\cdot C^T$.
\end{enumerate} }

%##################################################ANTWORTEN_TEXT
\begin{tabs*}[\initialtab{0}\class{exercise}]

  %++++++++++++++++++++++++++++++++++++++++++START_TAB_X
  \tab{\lang{de}{    Antwort    } \lang{en}{Answer}}
  \begin{incremental}[\initialsteps{1}]
  
  	 %----------------------------------START_STEP_X
    \step 
    \lang{de}{   Die Ausdrücke in (b) und (c) liefern die richtige Transponierte, die anderen beiden nicht.    }
     \lang{en}{ The terms in (b) and (c) give the right transpose, the other do not.    }
  	 %------------------------------------END_STEP_X
 
  \end{incremental}
  %++++++++++++++++++++++++++++++++++++++++++++END_TAB_X


  %++++++++++++++++++++++++++++++++++++++++++START_TAB_X
  \tab{\lang{de}{    Lösung    } \lang{en}{Solution}}
  \begin{incremental}[\initialsteps{1}]
  
  	 %----------------------------------START_STEP_X
    \step 
    \lang{de}{   Beim Transponieren einer Summe erhält man die Summe der transponierten Matrizen. Die Reihenfolge ist unerheblich, da die Addition kommutativ ist.
Beim Transponieren eines Produkts erhält man zwar das Produkt der transponierten Matrizen, aber die Reihenfolge der Matrizen muss umgekehrt werden.

Wir erhalten also:
\[  \big( (A+3B)\cdot C \big)^T= C^T\cdot (A+3B)^T
=C^T\cdot (A^T+3B^T).\]
Antwort (c) ist also richtig.

Bei Antwort (a) wurde jedoch die Reihenfolge im Produkt nicht vertauscht. Diese ist also falsch. Der angegebene Ausdruck ist nicht einmal definiert!

Auch der Ausdruck in (d) ist nicht definiert, da $B^T$ eine $(2\times 3)$-Matrix ist, welche man nicht mit der $(4\times 2)$-Matrix $C^T$ in dieser Reihenfolge multiplizieren kann.

Um einzusehen, dass der Ausdruck in (b) auch das richtige Ergebnis liefert, muss noch das Distributivgesetz und das Kommutativgesetz der Addition angewendet werden:
\[ C^T\cdot (A^T+3B^T)= C^T\cdot A^T+ C^T\cdot (3B^T)
=C^T\cdot A^T+3 C^T\cdot B^T=3 C^T\cdot B^T+ C^T\cdot A^T .\]

    }

\lang{en}{  The transpose of a sum is the sum of the transposes. The order does not matter, as the addition is commutative.
However, transposing a product rotates the order of the matrices within the product.

We have:
\[  \big( (A+3B)\cdot C \big)^T= C^T\cdot (A+3B)^T
=C^T\cdot (A^T+3B^T).\]
So, answer (c) is correct.

Answer (a) is wrong, because the matrices within the product were not rotated. The given term is not even defined!

Also answer (d) is not defined, because $B^T$ is a $(2\times 3)$-matrix, which cannot be multitplied in this order with the $(4\times 2)$-matrix $C^T$.

Answer (b) ist correct. This can been seen, when applying the distrubitive law and the commmutativity of the addition:
\[ C^T\cdot (A^T+3B^T)= C^T\cdot A^T+ C^T\cdot (3B^T)
=C^T\cdot A^T+3 C^T\cdot B^T=3 C^T\cdot B^T+ C^T\cdot A^T .\]

    }
  	 %------------------------------------END_STEP_X
 
  \end{incremental}
  %++++++++++++++++++++++++++++++++++++++++++++END_TAB_X

%#############################################################ENDE
\end{tabs*}
\end{content}