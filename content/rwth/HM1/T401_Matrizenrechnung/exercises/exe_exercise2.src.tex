\documentclass{mumie.element.exercise}
%$Id$
\begin{metainfo}
  \name{
    \lang{de}{Ü02: Matrixmultiplikation}
    \lang{en}{Ex02: Matrix multiplication}
  }
  \begin{description} 
 This work is licensed under the Creative Commons License Attribution 4.0 International (CC-BY 4.0)   
 https://creativecommons.org/licenses/by/4.0/legalcode 

    \lang{de}{Rechnen mit Matritzen}
    \lang{en}{}
  \end{description}
  \begin{components}
  \end{components}
  \begin{links}
  \end{links}
  \creategeneric
\end{metainfo}
\begin{content}
\usepackage{mumie.ombplus}

\title{
  \lang{de}{Ü02: Matrixmultiplikation}
  \lang{en}{Ex02: Matrix multiplication}
}

\begin{block}[annotation]
  Rechnen mit Matritzen
     
\end{block}
\begin{block}[annotation]
  Im Ticket-System: \href{http://team.mumie.net/issues/11098}{Ticket 11098}
\end{block}



\lang{de}{ 
Bestimmen Sie: \\
\begin{enumerate}[(a)]
 \item (a)
\[
  \begin{pmatrix}
  1 & 2 \\ 
  4 & -6
 \end{pmatrix}
\cdot
 \begin{pmatrix}
  2 & -3 \\ 
  1 & 7
 \end{pmatrix}
\]
 \item (b)
\[
  \begin{pmatrix}
  1 & 0 & -3\\ 
  -2 & 5 & 4\\
  0 & 7 & 9
 \end{pmatrix}
\cdot
 \begin{pmatrix}
  0 & 2 & 5\\ 
  3 & 3 & -2\\
  1 & 0 & 4
 \end{pmatrix}
\]
 \item (c)
\[
  \begin{pmatrix}
  i & 0 \\ 
  -4 & 6 \\
  2 & 8i 
 \end{pmatrix}
\cdot
 \begin{pmatrix}
  7 & -2 & -3\\ 
  0 & 6 & -5
 \end{pmatrix}
\]
 \item (d)
\[
  \begin{pmatrix}
  1 & 4 & 5
 \end{pmatrix}
\cdot
 \begin{pmatrix}
  2\\ 
  1 \\
  4
 \end{pmatrix}
\]
\item (e)
Es sei\\
\[A=\begin{pmatrix} 2 & 1 \\ -1 & 1 \end{pmatrix} \; , \;
B=\begin{pmatrix} 1 & 3 & 0 \\ 5 & -1 & 2 \end{pmatrix} \; , \;
C=\begin{pmatrix} 3 & 0 \\ 2 & 1 \\ -1 & 2 \end{pmatrix} \; , \;
D=\begin{pmatrix} 1 & -1 & 0 & 2 \\ 3 & 0 & 1 & -1 \\ 1 & 1 & 2 & 0\end{pmatrix} \; .\]
Welche Matrixprodukte kann man mit diesen Matrizen bilden? Welche Dimensionen haben die Produkte?
Berechnen Sie die Produkte.
\end{enumerate}
}

\lang{en}{ 
Determine: \\
\begin{enumerate}[(a)]
 \item (a)
\[
  \begin{pmatrix}
  1 & 2 \\ 
  4 & -6
 \end{pmatrix}
\cdot
 \begin{pmatrix}
  2 & -3 \\ 
  1 & 7
 \end{pmatrix}
\]
 \item (b)
\[
  \begin{pmatrix}
  1 & 0 & -3\\ 
  -2 & 5 & 4\\
  0 & 7 & 9
 \end{pmatrix}
\cdot
 \begin{pmatrix}
  0 & 2 & 5\\ 
  3 & 3 & -2\\
  1 & 0 & 4
 \end{pmatrix}
\]
 \item (c)
\[
  \begin{pmatrix}
  i & 0 \\ 
  -4 & 6 \\
  2 & 8i 
 \end{pmatrix}
\cdot
 \begin{pmatrix}
  7 & -2 & -3\\ 
  0 & 6 & -5
 \end{pmatrix}
\]
 \item (d)
\[
  \begin{pmatrix}
  1 & 4 & 5
 \end{pmatrix}
\cdot
 \begin{pmatrix}
  2\\ 
  1 \\
  4
 \end{pmatrix}
\]
\item[(e)]
Set\\
\[A=\begin{pmatrix} 2 & 1 \\ -1 & 1 \end{pmatrix} \; , \;
B=\begin{pmatrix} 1 & 3 & 0 \\ 5 & -1 & 2 \end{pmatrix} \; , \;
C=\begin{pmatrix} 3 & 0 \\ 2 & 1 \\ -1 & 2 \end{pmatrix} \; , \;
D=\begin{pmatrix} 1 & -1 & 0 & 2 \\ 3 & 0 & 1 & -1 \\ 1 & 1 & 2 & 0\end{pmatrix} \; .\]
Which matrix products can be formed this these matrices? Which dimensions do the products have?
Determine these products.
\end{enumerate}
}


\begin{tabs*}[\initialtab{0}\class{exercise}]
  \tab{\lang{de}{Lösung (a)} \lang{en}{Solution for (a)}}
  \lang{de}{\[
  \begin{pmatrix}
  1 & 2 \\ 
  4 & -6
 \end{pmatrix}
\cdot
 \begin{pmatrix}
  2 & -3 \\ 
  1 & 7
 \end{pmatrix}
=
  \begin{pmatrix}
  1\cdot 2 +2\cdot 1 & 1\cdot (-3) +2\cdot 7 \\ 
  4\cdot 2 +(-6)\cdot 1 & 4\cdot (-3) +(-6)\cdot 7
 \end{pmatrix}
=
  \begin{pmatrix}
  4 & 11 \\ 
  2 & -54
 \end{pmatrix}
\]}

\lang{en}{\[
  \begin{pmatrix}
  1 & 2 \\ 
  4 & -6
 \end{pmatrix}
\cdot
 \begin{pmatrix}
  2 & -3 \\ 
  1 & 7
 \end{pmatrix}
=
  \begin{pmatrix}
  1\cdot 2 +2\cdot 1 & 1\cdot (-3) +2\cdot 7 \\ 
  4\cdot 2 +(-6)\cdot 1 & 4\cdot (-3) +(-6)\cdot 7
 \end{pmatrix}
=
  \begin{pmatrix}
  4 & 11 \\ 
  2 & -54
 \end{pmatrix}
\]}

 \tab{\lang{de}{Lösung (b)} \lang{en}{Solution for (b)}}
  \lang{de}{\begin{align*}
  &\begin{pmatrix}
  1 & 0 & -3\\ 
  -2 & 5 & 4\\
  0 & 7 & 9
 \end{pmatrix}
\cdot
 \begin{pmatrix}
  0 & 2 & 5\\ 
  3 & 3 & -2\\
  1 & 0 & 4
 \end{pmatrix}\\
=&
 \begin{pmatrix}
  1\cdot 0 + 0 \cdot 3+(-3) \cdot 1& 1\cdot 2 + 0 \cdot 3+ (-3) \cdot 0& 1\cdot 5 + 0 \cdot (-2)+(-3) \cdot 4\\ 
  -2 \cdot 0+ 5 \cdot  3+ 4\cdot 1& -2 \cdot 2+ 5 \cdot 3+ 4\cdot 0 & -2 \cdot 5+ 5 \cdot (-2)+ 4\cdot 4\\
  0 \cdot 0+ 7 \cdot 3+ 9 \cdot 1 & 0 \cdot 2 + 7 \cdot 3 + 9 \cdot 0 & 0 \cdot 5 + 7 \cdot (-2)+ 9 \cdot 4
 \end{pmatrix}\\
=&
 \begin{pmatrix}
  -3 & 2 & -7\\ 
  19 & 11 & -4\\
  30 & 21 & 22
 \end{pmatrix}
\end{align*}}

\lang{en}{\begin{align*}
  &\begin{pmatrix}
  1 & 0 & -3\\ 
  -2 & 5 & 4\\
  0 & 7 & 9
 \end{pmatrix}
\cdot
 \begin{pmatrix}
  0 & 2 & 5\\ 
  3 & 3 & -2\\
  1 & 0 & 4
 \end{pmatrix}\\
=&
 \begin{pmatrix}
  1\cdot 0 + 0 \cdot 3+(-3) \cdot 1& 1\cdot 2 + 0 \cdot 3+ (-3) \cdot 0& 1\cdot 5 + 0 \cdot (-2)+(-3) \cdot 4\\ 
  -2 \cdot 0+ 5 \cdot  3+ 4\cdot 1& -2 \cdot 2+ 5 \cdot 3+ 4\cdot 0 & -2 \cdot 5+ 5 \cdot (-2)+ 4\cdot 4\\
  0 \cdot 0+ 7 \cdot 3+ 9 \cdot 1 & 0 \cdot 2 + 7 \cdot 3 + 9 \cdot 0 & 0 \cdot 5 + 7 \cdot (-2)+ 9 \cdot 4
 \end{pmatrix}\\
=&
 \begin{pmatrix}
  -3 & 2 & -7\\ 
  19 & 11 & -4\\
  30 & 21 & 22
 \end{pmatrix}
\end{align*}}

 \tab{\lang{de}{Lösung (c)} \lang{en}{Solution for (c)}}
  \lang{de}{\begin{align*}
  \begin{pmatrix}
  i & 0 \\ 
  -4 & 6 \\
  2 & 8i 
 \end{pmatrix}
\cdot
 \begin{pmatrix}
  7 & -2 & -3\\ 
  0 & 6 & -5
 \end{pmatrix}
&=  
 \begin{pmatrix}
  i \cdot 7 + 0 \cdot  0 & i \cdot (-2) + 0 \cdot 6 & i \cdot (-3) + 0 \cdot (-5)\\ 
  -4 \cdot 7 + 6 \cdot 0 & -4 \cdot (-2) + 6 \cdot 6 & -4 \cdot (-3) + 6 \cdot (-5)\\
  2 \cdot 7 + 8i \cdot 0 & 2 \cdot (-2) + 8i \cdot 6 & 2 \cdot (-3) + 8i \cdot (-5)
 \end{pmatrix}\\
&=
 \begin{pmatrix}
  7i & -2i & -3i\\ 
  -28 & 44 & -18\\
  14 & -4+48i & -6-40i
 \end{pmatrix}
\end{align*}
  }

\lang{en}{\begin{align*}
  \begin{pmatrix}
  i & 0 \\ 
  -4 & 6 \\
  2 & 8i 
 \end{pmatrix}
\cdot
 \begin{pmatrix}
  7 & -2 & -3\\ 
  0 & 6 & -5
 \end{pmatrix}
&=  
 \begin{pmatrix}
  i \cdot 7 + 0 \cdot  0 & i \cdot (-2) + 0 \cdot 6 & i \cdot (-3) + 0 \cdot (-5)\\ 
  -4 \cdot 7 + 6 \cdot 0 & -4 \cdot (-2) + 6 \cdot 6 & -4 \cdot (-3) + 6 \cdot (-5)\\
  2 \cdot 7 + 8i \cdot 0 & 2 \cdot (-2) + 8i \cdot 6 & 2 \cdot (-3) + 8i \cdot (-5)
 \end{pmatrix}\\
&=
 \begin{pmatrix}
  7i & -2i & -3i\\ 
  -28 & 44 & -18\\
  14 & -4+48i & -6-40i
 \end{pmatrix}
\end{align*}
  }

  
   \tab{\lang{de}{Lösung (d)}\lang{en}{Solution for (d)}}
  \lang{en}{\[
  \begin{pmatrix}
  1 & 4 & 5
 \end{pmatrix}
\cdot
 \begin{pmatrix}
  2\\ 
  1 \\
  4
 \end{pmatrix}
=(1\cdot 2 +4 \cdot 1 + 5\cdot 4)= 26
\]}

 \lang{de}{\[
  \begin{pmatrix}
  1 & 4 & 5
 \end{pmatrix}
\cdot
 \begin{pmatrix}
  2\\ 
  1 \\
  4
 \end{pmatrix}
=(1\cdot 2 +4 \cdot 1 + 5\cdot 4)= 26
\]}
  
  \tab{\lang{de}{Lösungsvideo (e)} \lang{en}{Solution for (e)}}	
   \lang{de}{\youtubevideo[500][300]{WT_nRhJzga0}}\\
   \lang{en}{
The following products of matrices can be built:
\begin{itemize}
\item $A$ has the same number of columns as $B$ has rows. So, the product $A\cdot B$ is defined and a $(2\times 3)$-matrix.\\
\begin{align*} A\cdot B= \begin{pmatrix} 2 & 1 \\ -1 & 1 \end{pmatrix} \cdot \begin{pmatrix} 1 & 3 & 0 \\ 5 & -1 & 2 \end{pmatrix}
              = \begin{pmatrix} 2\cdot1+1\cdot5 & 2\cdot 3+1\cdot(-1) & 2\cdot0+1\cdot2 \\ (-1)\cdot1+1\cdot5 & (-1)\cdot3+1\cdot(-1) & (-1)\cdot3+1\cdot(-1) \end{pmatrix}
              = \begin{pmatrix} 7 & 5 & 2 \\ 4 & -4 & 2 \end{pmatrix} \end{align*}
\item $B$ has the same number of columns as $C$ has rows. So, the product $B\cdot C$ is defined and a $(2\times 2)$-matrix.\\
\begin{align*} B\cdot C= \begin{pmatrix} 1 & 3 & 0 \\ 5 & -1 & 2 \end{pmatrix} \cdot \begin{pmatrix} 3 & 0 \\ 2 & 1 \\ -1 & 2 \end{pmatrix} 
             = \begin{pmatrix} 1\cdot3+3\cdot2+0\cdot(-1) & 1\cdot0+3\cdot1+0\cdot2 \\ 5\cdot3+(-1)\cdot2+2\cdot(-1) & 5\cdot0+(-1)\cdot1+2\cdot 2\end{pmatrix}
             = \begin{pmatrix} 9 & 3 \\ 11 & 3 \end{pmatrix} \end{align*}
\item $B$ has the same number of columns as $D$ has rows. So, the product $B\cdot  D$ is defined and a $(2\times 4)$-matrix.\\
\begin{align*} B\cdot D=\begin{pmatrix} 1 & 3 & 0 \\ 5 & -1 & 2 \end{pmatrix}\cdot \begin{pmatrix} 1 & -1 & 0 & 2 \\ 3 & 0 & 1 & -1 \\ 1 & 1 & 2 & 0\end{pmatrix}
              = \begin{pmatrix} 1\cdot1+3\cdot3+1\cdot0 & 1\cdot(-1)+3\cdot0+0\cdot1 & 1\cdot0+3\cdot1+0\cdot2 & 1\cdot2+3\cdot(-1)+0\cdot0 \\ 5\cdot1+(-1)\cdot3+2\cdot1
              & 5\cdot(-1)+(-1)\cdot0+2\cdot1 &5\cdot0+(-1)\cdot1+2\cdot2
              & 5\cdot+2+(-1)\cdot(-1)+2\cdot0 \end{pmatrix}
              = \begin{pmatrix} 10 & -1 & 3 & -1 \\ 4 & -3 & 3 & 11 \end{pmatrix} \end{align*}
\item $C$ has the same number of columns as $A$ has rows. So, the product $C\cdot A$ is defined and a $(3\times 2)$-matrix.\\
\begin{align*}=C\cdot A=\begin{pmatrix} 3 & 0 \\ 2 & 1 \\ -1 & 2 \end{pmatrix}\cdot \begin{pmatrix} 2 & 1 \\ -1 & 1 \end{pmatrix}
              = \begin{pmatrix} 3\cdot2+0\cdot(-1) & 3\cdot1+0\cdot1 \\ 2\cdot2+1\cdot(-1) & 2\cdot1+1\cdot1 \\ (-1)\cdot2+2\cdot(-1) & (-1)\cdot1+2\cdot1 \end{pmatrix}
              = \begin{pmatrix} 6 & 3 \\ 3 & 3 \\ -4 & 1 \end{pmatrix} \end{align*}
\item $C$ has the same number of columns as $B$ has rows. So, the product $C\cdot B$ is defined and a $(3\times 3)$-matrix.\\
\begin{align*}C\cdot B= \begin{pmatrix} 3 & 0 \\ 2 & 1 \\ -1 & 2 \end{pmatrix} \cdot \begin{pmatrix} 1 & 3 & 0 \\ 5 & -1 & 2 \end{pmatrix}
              = \begin{pmatrix} 3\cdot1+0\cdot5 & 3\cdot3+0\cdot(-1) & 3\cdot0+0\cdot2 \\ 2\cdot1+1\cdot5 & 2\cdot3+1\cdot(-1) & 2\cdot0+1\cdot2 \\ -1\cdot1+2\cdot5 & (-1)\cdot3+2\cdot(-1) & (-1)\cdot0+2\cdot2 \end{pmatrix}
              = \begin{pmatrix} 3 & 9 & 0 \\ 7 & 5 & 2 \\ 9 & -5 & 4 \end{pmatrix} \end{align*}
            Comparing this with $B\cdot C$ illustrates again, that matrix multiplication is not commutative. Not only the entries are different, but the products even have different dimensions!
\item $A$ is a square matrix, which means the number of rows is the same as the number of columns. So, the product $A\cdot A$ is defined and a $(2\times2)$-matrix.\\
\begin{align*}A\cdot A=\begin{pmatrix} 2 & 1 \\ -1 & 1 \end{pmatrix} \cdot \begin{pmatrix} 2 & 1 \\ -1 & 1 \end{pmatrix} 
              = \begin{pmatrix} 2\cdot2+(-1)\cdot1 & 2\cdot1+1\cdot1 \\ (-1)\cdot2+1\cdot(-1) & (-1)\cdot1+1\cdot1 \end{pmatrix} 
              =\begin{pmatrix} 3 & 3 \\ -3 & 0 \end{pmatrix} \end{align*}
\end{itemize}}

\end{tabs*}
\end{content}