\documentclass{mumie.element.exercise}
%$Id$
\begin{metainfo}
  \name{
    \lang{de}{Ü08: Anwendung}
    \lang{en}{Ex08: Application}
  }
  \begin{description} 
 This work is licensed under the Creative Commons License Attribution 4.0 International (CC-BY 4.0)   
 https://creativecommons.org/licenses/by/4.0/legalcode 

    \lang{de}{Rechnen mit Matritzen}
    \lang{en}{Calculating with matrices}
  \end{description}
  \begin{components}
  \end{components}
  \begin{links}
  \end{links}
  \creategeneric
\end{metainfo}
\begin{content}
\begin{block}[annotation]
	Im Ticket-System: \href{https://team.mumie.net/issues/19603}{Ticket 19603}
\end{block}
\usepackage{mumie.ombplus}

\title{\lang{de}{Ü08: Anwendung} \lang{en}{Ex08: Application}}


\lang{de}{
  Ein Unternehmen fertigt die \textit{Endprodukte} $P_1$, $P_2$ und $P_3$.
  Es soll ermittelt werden, welcher Bedarf an Rohstoffen je Endprodukt besteht.
    \begin{itemize}
        \item Jedes Endprodukt setzt sich aus den \textit{Zwischenprodukten} $Z_1$, $Z_2$ und $Z_3$ zusammen.
        \item Die Zwischenprodukte bestehen aus den \textit{Rohstoffen} $R_1$, $R_2$, $R_3$ und $R_4$.
    \end{itemize}
Die folgende Tabelle 1 gibt den Rohstoffbedarf für jedes Zwischenprodukt in der Einheit Kilogramm an:
\begin{center}
\begin{table}[\cellaligns{ccc}]
    \head
      & $Z_1$ & $Z_2$ & $Z_3$
    \body
  $R_1$ & 0 & 4 & 3 \\
  $R_2$ & 3 & 2 & 2 \\
  $R_3$ & 0 & 2 & 3 \\
  $R_4$ & 3 & 2 & 4
\end{table}
\end{center}
Weiterhin liegt eine Tabelle 2 vor, welche die Stückzahl der Zwischenprodukte je Endprodukt auflistet:
\begin{table}[\cellaligns{ccc}]
    \head
      & $P_1$ & $P_2$ & $P_3$
    \body
  $Z_1$ & 2 & 1 & 5 \\
  $Z_2$ & 1 & 9 & 3 \\
  $Z_3$ & 4 & 7 & 2
\end{table}}

\lang{en}{
  A company manufactures the \textit{final products} $P_1$, $P_2$ and $P_3$.
  The raw material requirements for each end product should be determined.
    \begin{itemize}
        \item Each final product consists of the \textit{intermediates} $Z_1$, $Z_2$ and $Z_3$.
        \item The intermediate consist of the \textit{raw materials} $R_1$, $R_2$, $R_3$ and $R_4$.
    \end{itemize}
The following table describes the raw material requirement for each intermediate in the unit kilogramme:
\begin{center}
\begin{table}[\cellaligns{ccc}]
    \head
      & $Z_1$ & $Z_2$ & $Z_3$
    \body
  $R_1$ & 0 & 4 & 3 \\
  $R_2$ & 3 & 2 & 2 \\
  $R_3$ & 0 & 2 & 3 \\
  $R_4$ & 3 & 2 & 4
\end{table}
\end{center}
Furthermore there is table 2, which describes the amount of intermediates in each final product:
\begin{table}[\cellaligns{ccc}]
    \head
      & $P_1$ & $P_2$ & $P_3$
    \body
  $Z_1$ & 2 & 1 & 5 \\
  $Z_2$ & 1 & 9 & 3 \\
  $Z_3$ & 4 & 7 & 2
\end{table}}

\lang{de}{
\begin{enumerate}[(a)]
  \item (a) Stellen Sie die beiden Tabellen als Matrizen $A$ und $B$ dar.
  \item (b) Bestimmen Sie den Rohstoffbedarf für jedes Endprodukt. Geben Sie die Lösung als Matrix $C$ an.
     %Die $i$-te Zeile der Lösungsmatrix beschreibt, wie viele Einheiten vom Rohstoff $R_i$ für die Endprodukte $E_1, E_2, E_3$ benötigt werden.
  \item (c) Für die Erstellung der Rohstoffbestellung möchte man nun aus Matrix $C$ wieder eine Tabelle erstellen. 
     Dazu sollen die benötigten Rohstoffmengen für Endprodukt $P_k$ in Zeile $k$ stehen ($1 \leq k \leq 3$).
\end{enumerate}}

\lang{en}{
\begin{enumerate}[(a)]
  \item (a) Display the tables as matrices.
  \item (b) Determine the raw material requirement for each final product. Display the result in a matrix $C$.
     %Die $i$-te Zeile der Lösungsmatrix beschreibt, wie viele Einheiten vom Rohstoff $R_i$ für die Endprodukte $E_1, E_2, E_3$ benötigt werden.
  \item (c) For ordering the raw materials, the matrix $C$ should be displayed as a table. The row $k$ indicates the needed amount of raw material
  for the final product $P_k$.
\end{enumerate}}

\begin{tabs*}[\initialtab{0}\class{exercise}]

  \tab{\lang{de}{Lösung (a)} \lang{en}{Solution for (a)}}
  \lang{de}{
    Wir stellen die Matrix $A \in M(4,3;\R)$ auf, deren Eintrag $a_{i,j}$ den Rohstoffbedarf am Rohstoff $R_i$ für das Zwischenprodukt $Z_j$ angibt.
    Weiter soll die Matrix $B \in M(3,3;\R)$ im Eintrag $b_{i,j}$ angeben, welche Stückzahl des Zwischenprodukts $Z_i$ je Endprodukt $P_j$ notwendig ist.
    Damit sind die Matrizen wie folgt gegeben:}

  \lang{en}{
We set up the matrix $A \in M(4,3;\R)$. The $(i,J)$th entry indicates the requirement of the raw material $R_i$ for the intermediate $Z_j$.
  The $(i,j)$th entry of matrix $B \in M(3,3;\R)$ indicates the quantity of the intermediate $Z_i$ needed for the final product $P_j$.
 The matrices are then given as follows:}
    
    \begin{equation*}
A = 
	\begin{pmatrix}
		0 & 4 & 3 \\
		3 & 2 & 2 \\
		0 & 2 & 3 \\
		3 & 2 & 4
	\end{pmatrix}
 ~~~~~ B=
	\begin{pmatrix}
		2 & 1 & 5 \\
		1 & 9 & 3 \\
		4 & 7 & 2
	\end{pmatrix}
\end{equation*}


  \tab{\lang{de}{Lösung (b)} \lang{en}{Solution for (b)}}
  \lang{de}{
  Die Lösung erhält man durch Multiplikation der beiden Matrizen $A$ und $B$:}
  \lang{en}{
  The solution is obtained by multiplying the two matrices $A$ and $B$}
  
\begin{eqnarray*}
	C = A \cdot B
	&=&
	\begin{pmatrix}
		0 & 4 & 3 \\
		3 & 2 & 2 \\
		0 & 2 & 3 \\
		3 & 2 & 4
	\end{pmatrix}
	\cdot
	\begin{pmatrix}
		2 & 1 & 5 \\
		1 & 9 & 3 \\
		4 & 7 & 2
	\end{pmatrix}
    \\
	&=&
	\begin{pmatrix}
0 \cdot 2 + 4 \cdot 1 + 3 \cdot 4  &  0 \cdot 1 + 4 \cdot 9 + 3 \cdot 7  &  0 \cdot 5 + 4 \cdot 3 + 3 \cdot 2 \\
3 \cdot 2 + 2 \cdot 1 + 2 \cdot 4  &  3 \cdot 1 + 2 \cdot 9 + 2 \cdot 7  &  3 \cdot 5 + 2 \cdot 3 + 2 \cdot 2 \\
0 \cdot 2 + 2 \cdot 1 + 3 \cdot 4  &  0 \cdot 1 + 2 \cdot 9 + 3 \cdot 7  &  0 \cdot 5 + 2 \cdot 3 + 3 \cdot 2 \\
3 \cdot 2 + 2 \cdot 1 + 4 \cdot 4  &  3 \cdot 1 + 2 \cdot 9 + 4 \cdot 7  &  3 \cdot 5 + 2 \cdot 3 + 4 \cdot 2
	\end{pmatrix}
    \\
	&=&
    \begin{pmatrix}
   16 &  57  & 18 \\
   16 &  35  & 25 \\
   14 &  39  & 12 \\
   24 &  49  & 29
	\end{pmatrix}
\end{eqnarray*}
\lang{de}{
Der Eintrag $c_{ij}$ gibt nun an, wie viele Einheiten des Rohstoffs $R_i$ für das Endprodukt $P_j$ benötigt werden.}
\lang{en}{
The entry $c_{ij}$ indicates, how many units of the raw material $R_i$ are needed for the final product $P_j$.}

  \tab{\lang{de}{Lösung (c)} \lang{en}{Solution for (c)}}#
  \lang{de}{
  In (b) haben wir gesehen, dass in der $i$-ten Zeile die benötigte Rohstoffmenge des Rohstoffes $R_i$ für das Endprodukt $P_j$ abgelesen werden kann.
  Die Aufgabenstellung erwartet aber in den Zeilen jeweils das Endprodukt. Dazu ist das Transponieren der Matrix $C$ nötig:}
  \lang{en}{
As seen in (b), the $i$th row indicates the needed amount of the raw material $R_i$ for the final product $P_j$.
But as asked in the exercise, the final product should be listed in the row. Therefore, it is necessary to transpose the matrix $C$:}

\[
    C^T = 
    \begin{pmatrix}
   16 &  57  & 18 \\
   16 &  35  & 25 \\
   14 &  39  & 12 \\
   24 &  49  & 29
	\end{pmatrix}^T
    =
    \begin{pmatrix}
   16 &  16 &  14 &  24 \\
   57 &  35 &  39 &  49 \\
   18 &  25 &  12 &  29
	\end{pmatrix}
\]

\lang{de}{Für die Bestellung werden die Elemente aus Matrix $C^T$ nun wieder in eine Tabelle übertragen:}
\lang{en}{For the order, the elements of $C^T$ will be transfered into a table:}

\begin{center}
\begin{table}[\cellaligns{ccc}]
    \head
      & $R_1$ & $R_2$ & $R_3$ & $R_4$
    \body
  $P_1$ & 16 &  16 &  14 &  24 \\
  $P_2$ & 57 &  35 &  39 &  49 \\
  $P_3$ & 18 &  25 &  12 &  29
\end{table}
\end{center}

   \tab{\lang{de}{Video: ähnliche Übungsaufgabe}}	
    \youtubevideo[500][300]{cwCgtYMaFCg}\\

\end{tabs*}
\end{content}
