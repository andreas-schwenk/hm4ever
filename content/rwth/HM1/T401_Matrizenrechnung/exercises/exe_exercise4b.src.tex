\documentclass{mumie.element.exercise}
%$Id$
\begin{metainfo}
  \name{
    \lang{de}{Ü05: Transponierte}
    \lang{en}{Ex05: The transpose}
  }
  \begin{description} 
 This work is licensed under the Creative Commons License Attribution 4.0 International (CC-BY 4.0)   
 https://creativecommons.org/licenses/by/4.0/legalcode 

    \lang{de}{Rechnen mit Matrizen}
    \lang{en}{Calculating with matrices}
  \end{description}
  \begin{components}
  \end{components}
  \begin{links}
  \end{links}
  \creategeneric
\end{metainfo}
\begin{content}
\begin{block}[annotation]
	Im Ticket-System: \href{https://team.mumie.net/issues/28453}{Ticket 28453}
\end{block}
\begin{block}[annotation]
Copy of \href{https://team.mumie.net/issues/28452}{Ticket 28452}: content/rwth/HM1/T111neu_Matrizen/exercises/exe_exercise6b.src.tex
\end{block}

\usepackage{mumie.ombplus}

\title{
  \lang{de}{Ü05: Transponierte}
  \lang{en}{Ex05: The transpose}
}

\begin{block}[annotation]
  Rechnen mit Matrizen
     
\end{block}


\lang{de}{
\begin{enumerate}
 \item[(a)]
 Berechnen Sie $M_1=A \cdot A^T$ und $M_2=A^T \cdot A$ zu 
$ A=
 \begin{pmatrix}
  1 & -2 & 0\\
  1 & 3 & 1
 \end{pmatrix}.
$
 \item[(b)]
 Überlegen Sie sich, dass man zu $A \in \mathbb{R}^{m \times n}$ 
 stets die Produkte $A \cdot A^T$ und $A^T \cdot A$ bilden kann.
 Welche Dimensionen ergeben sich?
 \item[(c)] 
Die Produkte $M_1$ und $M_2$ aus (a) sind symmetrisch bzgl. der Hauptdiagonalen, also $M_1^T=M_1$ 
und $ M_2^T=M_2$. Ist das Zufall?
\end{enumerate}}


\lang{en}{
\begin{enumerate}
 \item[(a)]
 Determine $M_1=A \cdot A^T$ and $M_2=A^T \cdot A$ for
$ A=
 \begin{pmatrix}
  1 & -2 & 0\\
  1 & 3 & 1
 \end{pmatrix}.
$
 \item[(b)]
 Consider, that it is possible to determine $A \cdot A^T$ and $A^T \cdot A$ for any $A \in \mathbb{R}^{m \times n}$. 
 Which dimensions do the products have?
 \item[(c)] 
 The products $M_1$ and $M_2$ from (a) are symmetrical with respect to the main diagonal. This is, that $M_1^T=M_1$ and $ M_2^T=M_2$.
 Is that a coincidence?
\end{enumerate}}

\lang{de}{
\begin{tabs*}[\initialtab{0}\class{exercise}]
  \tab{Lösungsvideo}
  \youtubevideo[500][300]{WQlRMQ08w-k}\\
\end{tabs*}}
  
\lang{en}{
\begin{tabs*}[\initialtab{0}\class{exercise}]
\tab{Solution for (a)}
\begin{align*}
M_1=A \cdot A^T=\begin{pmatrix} 1 & -2 & 0\\ 1 & 3 & 1 \end{pmatrix} \cdot \begin{pmatrix} -1 & 1 \\ 2 & 3 \\ 0 & 1  \end{pmatrix}
= \begin{pmatrix} 5 & 5 \\ 5 & 11 \end{pmatrix}\\
M_2=A^T \cdot A= \begin{pmatrix} -1 & 1 \\ 2 & 3 \\ 0 & 1  \end{pmatrix} \cdot \begin{pmatrix} 1 & -2 & 0\\ 1 & 3 & 1 \end{pmatrix}
= \begin{pmatrix} 2 & 1 & 1 \\ 1 & 13 & 3 \\ 1 & 3 & 1 \end{pmatrix}
\end{align*}

\tab{Solution for (b)}
For any $A\in \mathbb{R}^{m \times n}$, the transpose is $A^T\mathbb{R}^{n \times m}$. So, $A$ has $n$ columns and $m$ rows. $A^T$ has $m$ columns and $n$ rows.\\
The condition for mmatrix-matrix multiplication is: The number of columns in the first matrix must be the same as the number of rows in 
the second matrix.\\
This is fulfilled for the product $A\cdot A^T$ for any $A\in \mathbb{R}^{m \times n}$, which is why the product is
always defined. The analogue consideration can be done for the product $A^T\cdot A$, as $A^T$ has the same number of columns as $A$ has rows
When multiplying any $(m\times n)$-matrix with a $(n\times k)$-matrix, the result is a $(m\times k)$-matrix.\\
If we transfer this to $A\cdot A^T$ and $A^T \cdot A$, we see that $A\cdot A^T$ is a $(m\times m)$-matrix and $A^T \cdot A$ a $(n\times n )$-matrix.

\tab{Solution for (c)}
We can specify the term symmetrical with respect to the main diagonal as follows:
\begin{align*}
\text{A square matrix M is symmetrical.} \Leftrightarrow M^T=M
\end{align*}
To check whether the symmetry is a coincidence or not, we need to examine this criterion.
for the general matrix $A$.\\
Let $A\in \mathbb{R}^{m \times n}$ be given. We then determine $M_1^T$:
\begin{align*}
M_1^T=(A\cdot A^T)^T=(A^T)^T\cdot A^T=A\cdot A^T=M_1
\end{align*}
This equation uses the calculation rules for transposes and shows, that $M_1=A\cdot A^T$ is always symmetrical.\\
Analogous, we find, that $M_2=A^T\cdot A$ is also symmetrical.
\end{tabs*}}

\end{content}

