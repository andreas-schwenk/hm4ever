%$Id:  $
\documentclass{mumie.article}
%$Id$
\begin{metainfo}
  \name{
    \lang{de}{Rechnen mit transponierten Matrizen}
    \lang{en}{Calculating with Transposes}
  }
  \begin{description} 
 This work is licensed under the Creative Commons License Attribution 4.0 International (CC-BY 4.0)   
 https://creativecommons.org/licenses/by/4.0/legalcode 

    \lang{de}{Beschreibung}
    \lang{en}{Description}
  \end{description}
  \begin{components}
  \component{generic_image}{content/rwth/HM1/images/g_img_00_video_button_schwarz-blau.meta.xml}{00_video_button_schwarz-blau}
\component{generic_image}{content/rwth/HM1/images/g_img_00_Videobutton_schwarz.meta.xml}{00_Videobutton_schwarz}
\end{components}
  \begin{links}
\link{generic_article}{content/rwth/HM1/T111neu_Matrizen/g_art_content_44_transponierte_matrix.meta.xml}{content_44_transponierte_matrix}
\end{links}
  \creategeneric
\end{metainfo}
\begin{content}
\usepackage{mumie.ombplus}
\ombchapter{1}
\ombarticle{3}
\usepackage{mumie.genericvisualization}

\begin{visualizationwrapper}

\title{\lang{de}{Transponierte Matrix und Rechenregeln} \lang{en}{Transpose of a matrix and calculation rules}}

\begin{block}[annotation]
  
\end{block}
\begin{block}[annotation]
  Im Ticket-System: \href{http://team.mumie.net/issues/11068}{Ticket 11068}\\
\end{block}

\begin{block}[info-box]
\tableofcontents
\end{block}

%Wie schon in den vorigen Abschnitten bezeichnet $\mathbb{K}$ einen beliebigen Körper.
\lang{de}{
Transponierte Matrizen über den reellen Zahlen $\R$ sind aus dem \link{content_44_transponierte_matrix}{Grundlagenteil} bekannt.
In diesem Kapitel werden transponierte Matrizen nun über beliebige Körper $\mathbb{K}$ betrachtet.
Weiterhin lernen wir nun Rechenregeln für transponierte Matrizen kennen.}

\lang{en}{We already had a look at transposes of matrices in the \link{content_44_transponierte_matrix}{basic part}. In this chapter we
consider them again, but now over any field $\mathbb{K}$. Further, we will learn calculation rules for transposes.}

\section{\lang{de}{Transponierte Matrix} \lang{en}{Matrix transposition}}

%\begin{definition}[Transponierte Matrix]
%Sei $A=\left(a_{ij}\right) \in M(m,n;\mathbb{K})$ eine Matrix
%\begin{equation*}
%A :=
%\begin{pmatrix}
%a_{11} & a_{12} & \cdots & a_{1n} \\
%a_{21} & a_{22} & \cdots & a_{2n} \\
%\vdots & \vdots & \ddots & \vdots \\
%a_{m1} & a_{m2} & \cdots & a_{mn}
%\end{pmatrix}
%\end{equation*}
%Die Matrix
%\begin{equation*}
%A^{T} :=
%\begin{pmatrix}
%a_{11} & a_{21} & \cdots & a_{m1} \\
%a_{12} & a_{22} & \cdots & a_{m2} \\
%\vdots & \vdots & \ddots & \vdots \\
%a_{1n} & a_{2n} & \cdots & a_{mn}
%\end{pmatrix} = (a_{ji}) \in M(n,m;\mathbb{K})
%\end{equation*}
%heißt dann Transponierte der Matrix $A$. 
%
%Die Spalten von $A^T$ sind also die Zeilen der Matrix $A$ und die Zeilen von $A^T$ sind die Spalten von $A$. 
%\end{definition}


\begin{definition}[\lang{de}{Transponierte Matrix} \lang{en}{Matrix transposition}]\label{def:transponierte-matrix}

\lang{de}{
Sei $A=\left( a_{ij} \right)_{1 \leq i \leq m, 1 \leq j \leq n} \in M(m,n;\mathbb{K})$:
%Sei $A=\left(a_{ij}\right) \in M(m,n;\R)$ eine Matrix
\begin{equation*}
A :=
\begin{pmatrix}
a_{11} & a_{12} & \cdots & a_{1n} \\
a_{21} & a_{22} & \cdots & a_{2n} \\
\vdots & \vdots & \ddots & \vdots \\
a_{m1} & a_{m2} & \cdots & a_{mn}
\end{pmatrix}
\end{equation*}
%Die Matrix $A^T=\left( a_{ji} \right)_{1 \leq j \leq n, 1 \leq i \leq m} \in M(n,m;\mathbb{K})$ heißt dann Transponierte der Matrix $A$:
%Die Matrix $A^T = B=\left( b_{ij}=a_{ji} \right)_{1 \leq i \leq n, 1 \leq j \leq m} \in M(n,m;\mathbb{K})$ heißt dann Transponierte der Matrix $A$:
%Die Matrix $B=\left( b_{ij} \right)_{1 \leq i \leq n, 1 \leq j \leq m} \in M(n,m;\mathbb{K})$ mit $\left( b_{ij} = a_{ji} \right)_{1 \leq i \leq n, 1 \leq j \leq m}$ heißt dann Transponierte der Matrix $A$. 
%Wir schreiben kurz $A^T$. Ausführliche Schreibweise:
%Die Matrix $A^T=\left( a_{ij} \right)^T = \left( a_{ji} \right)_{1 \leq j \leq n, 1 \leq i \leq m} \in M(n,m;\mathbb{K})$ heißt dann Transponierte der Matrix $A$:
Die Matrix $A^T \in M(n,m;\mathbb{K})$ heißt dann Transponierte der Matrix $A$:
\begin{equation*}
A^{T} = 
%(a_{ij})^T_{1 \leq i \leq m, 1 \leq j \leq n} :=
\begin{pmatrix}
a_{11} & a_{21} & \cdots & a_{m1} \\
a_{12} & a_{22} & \cdots & a_{m2} \\
\vdots & \vdots & \ddots & \vdots \\
a_{1n} & a_{2n} & \cdots & a_{mn}
\end{pmatrix}% = (a_{ji}) \in M(n,m;\R)
\end{equation*}
Die Spalten von $A^T$ sind also die Zeilen der Matrix $A$ und die Zeilen von $A^T$ sind die Spalten von $A$.\\
\floatright{\href{https://www.hm-kompakt.de/video?watch=824}{\image[75]{00_Videobutton_schwarz}}}}\\\\

\lang{en}{
Set $A=\left( a_{ij} \right)_{1 \leq i \leq m, 1 \leq j \leq n} \in M(m,n;\mathbb{K})$:
%Sei $A=\left(a_{ij}\right) \in M(m,n;\R)$ eine Matrix
\begin{equation*}
A :=
\begin{pmatrix}
a_{11} & a_{12} & \cdots & a_{1n} \\
a_{21} & a_{22} & \cdots & a_{2n} \\
\vdots & \vdots & \ddots & \vdots \\
a_{m1} & a_{m2} & \cdots & a_{mn}
\end{pmatrix}
\end{equation*}
%Die Matrix $A^T=\left( a_{ji} \right)_{1 \leq j \leq n, 1 \leq i \leq m} \in M(n,m;\mathbb{K})$ heißt dann Transponierte der Matrix $A$:
%Die Matrix $A^T = B=\left( b_{ij}=a_{ji} \right)_{1 \leq i \leq n, 1 \leq j \leq m} \in M(n,m;\mathbb{K})$ heißt dann Transponierte der Matrix $A$:
%Die Matrix $B=\left( b_{ij} \right)_{1 \leq i \leq n, 1 \leq j \leq m} \in M(n,m;\mathbb{K})$ mit $\left( b_{ij} = a_{ji} \right)_{1 \leq i \leq n, 1 \leq j \leq m}$ heißt dann Transponierte der Matrix $A$. 
%Wir schreiben kurz $A^T$. Ausführliche Schreibweise:
%Die Matrix $A^T=\left( a_{ij} \right)^T = \left( a_{ji} \right)_{1 \leq j \leq n, 1 \leq i \leq m} \in M(n,m;\mathbb{K})$ heißt dann Transponierte der Matrix $A$:
The matrix $A^T \in M(n,m;\mathbb{K})$ is called the transpose of matrix $A$:
\begin{equation*}
A^{T} = 
%(a_{ij})^T_{1 \leq i \leq m, 1 \leq j \leq n} :=
\begin{pmatrix}
a_{11} & a_{21} & \cdots & a_{m1} \\
a_{12} & a_{22} & \cdots & a_{m2} \\
\vdots & \vdots & \ddots & \vdots \\
a_{1n} & a_{2n} & \cdots & a_{mn}
\end{pmatrix}% = (a_{ji}) \in M(n,m;\R)
\end{equation*}
The columns of $A^T$ are the rows of $A$ and the rows of $A^T$ are the columns of $A$.}\\

\end{definition}



\begin{example}
\lang{de}{
\begin{enumerate}
\item Die Transponierte Matrix zu
$  \begin{pmatrix} 1 & 2 & 3  \\ 4 & 5 & 6 \end{pmatrix} $
ist
$  \begin{pmatrix} 1 & 4 \\ 2 & 5 \\ 3 & 6 \end{pmatrix} .$ \\
Die Transponierte Matrix zu $ \begin{pmatrix} 1 & 4+i \\ 2+2i & 5+3i \\ 3+4i & 6+5i \end{pmatrix} $  ist $  \begin{pmatrix} 1 & 2+2i & 3+4i  \\ 4+i & 5+3i & 6+5i \end{pmatrix}$.
\item Die Transponierte eines Spaltenvektors (also einer $(m\times 1)$-Matrix) ist ein Zeilenvektor, also eine $(1\times m)$-Matrix und umgekehrt:
\[  \begin{pmatrix} 1 \\ -3 \\ 2 \end{pmatrix}^T =\begin{pmatrix} 1 & -3 & 2 \end{pmatrix}. \]
\end{enumerate}}

\lang{en}{
\begin{enumerate}
\item The transpose of
$  \begin{pmatrix} 1 & 2 & 3  \\ 4 & 5 & 6 \end{pmatrix} $
is
$  \begin{pmatrix} 1 & 4 \\ 2 & 5 \\ 3 & 6 \end{pmatrix} .$ \\
The transpose of $ \begin{pmatrix} 1 & 4+i \\ 2+2i & 5+3i \\ 3+4i & 6+5i \end{pmatrix} $  is $  \begin{pmatrix} 1 & 2+2i & 3+4i  \\ 4+i & 5+3i & 6+5i \end{pmatrix}$.
\item The transpose of a column vector (considered as a $(m\times 1)$-matrix) is a row vector (a $(1\times m)$-matrix) and converserly:
\[  \begin{pmatrix} 1 \\ -3 \\ 2 \end{pmatrix}^T =\begin{pmatrix} 1 & -3 & 2 \end{pmatrix}. \]
\end{enumerate}}
\end{example}

\begin{rule}
\lang{de}{
Es ist leicht zu sehen, dass die Transponierte der Transponierten einer jeden Matrix wieder die Matrix selbst ist, d.h. für jede Matrix $A$ ist
\[ (A^{T})^{T} =A. \]}
\lang{en}{
It is easy to see, that the transpose of a transpose is just the matrix herself again. This is for every matrix $A$ we have
\[ (A^{T})^{T} =A. \]}
\end{rule}

\section{\lang{de}{Rechenregeln für transponierte Matrizen} \lang{en}{Calculation rules for transposes}}\label{sec:rechenregeln}
\lang{de}{
Im Folgenden wird erklärt, wie sich das Transponieren von Matrizen mit den Rechenoperationen verträgt.}
\lang{en}{
The following explains what calculation rules apply to transposes.}

\begin{rule}
\lang{de}{
\begin{enumerate}
\item Für alle Matrizen $A,B\in M(m,n;\mathbb{K})$ gilt
\[   (A+B)^T=A^T+B^T. \]
Das Transponierte der Summe ist also die Summe der transponierten Matrizen.
\item Für alle Matrizen $A\in M(m,n;\mathbb{K})$ und Skalare $r\in\mathbb{K}$ gilt
\[ (rA)^T=r\cdot A^T.\]
\item Für Matrizen $A\in M(m,n;\mathbb{K})$ und $B\in M(n,k;\mathbb{K})$ gilt
\[  (AB)^T =B^T\cdot A^T. \]
Beim Transponieren von Matrizenprodukten dreht sich also die Reihenfolge der Faktoren um.
\end{enumerate}}

\lang{en}{
\begin{enumerate}
\item For all matrices $A,B\in M(m,n;\mathbb{K})$ we have
\[   (A+B)^T=A^T+B^T. \]
The transpose of a sum is the sum of the transposes.
\item For all matrices $A\in M(m,n;\mathbb{K})$ and scalars $r\in\mathbb{K}$ we have
\[ (rA)^T=r\cdot A^T.\]
\item For $A\in M(m,n;\mathbb{K})$ and $B\in M(n,k;\mathbb{K})$ we have
\[  (AB)^T =B^T\cdot A^T. \]
The transposition of a matrix product rotates the order of the factors.
\end{enumerate}}
\end{rule}

\lang{de}{
\begin{tabs*}[\initialtab{0}]
\tab{Herleitung der dritten Regel}%%5HERLEITUNG???!!!!!!
\[
    (A B)^T = \left( \sum_{\ell=1}^n a_{i\ell} \cdot b_{\ell j} \right)^T_{1 \leq i \leq n, 1 \leq j \leq k}
\]
\[
    = \left( \sum_{\ell=1}^n a_{j\ell} \cdot b_{\ell i} \right)_{1 \leq i \leq n, 1 \leq j \leq k}
    = \left( \sum_{\ell=1}^n b_{\ell i} \cdot a_{j\ell} \right)_{1 \leq i \leq n, 1 \leq j \leq k}
    = B^T \cdot A^T
\]
\end{tabs*}}

\lang{en}{
\begin{tabs*}[\initialtab{0}]
\tab{Derivation of the third rule}%%5HERLEITUNG???!!!!!!
\[
    (A B)^T = \left( \sum_{\ell=1}^n a_{i\ell} \cdot b_{\ell j} \right)^T_{1 \leq i \leq n, 1 \leq j \leq k}
\]
\[
    = \left( \sum_{\ell=1}^n a_{j\ell} \cdot b_{\ell i} \right)_{1 \leq i \leq n, 1 \leq j \leq k}
    = \left( \sum_{\ell=1}^n b_{\ell i} \cdot a_{j\ell} \right)_{1 \leq i \leq n, 1 \leq j \leq k}
    = B^T \cdot A^T
\]
\end{tabs*}}

\lang{de}{
\begin{example}
F\"ur
\[A = \begin{pmatrix}
12 & 11\\ 10 & 9 \\8 & 7 \\ 6& 5
\end{pmatrix}\in M(4,2;\R)  \,\,\text{und } B = \begin{pmatrix}
1 & 3 & 5 \\ 2i & 4 & 6
\end{pmatrix}\in M(2,3;\C)
\]
ist das Produkt die $(4\times 3)$-Matrix
\[ AB= \begin{pmatrix}
12+22i & 36+44 & 60+66\\ 10+18i & 30+36 & 50+54 \\ 8+14i & 24+28 & 40+42 \\ 6+10i & 18+20 & 30+30
\end{pmatrix} =\begin{pmatrix}  
12+22i & 80 & 126 \\ 10+18i & 66 & 104 \\ 8+14i & 52 & 82 \\ 6+10i & 38 & 60
\end{pmatrix} \in M(4, 3;\C), \]
und daher 
\[ (AB)^T= \begin{pmatrix} 12+22i & 10+18i & 8+14i & 6+10i\\ 80 & 66 & 52 & 38 \\ 126 & 104 & 82 & 60
\end{pmatrix}\in M(3, 4;\C).\]
Das Produkt $B^T\cdot A^T$ ist gegeben durch
\begin{eqnarray*}
 B^T\cdot A^T &=& \begin{pmatrix}
1 & 2i\\ 3& 4\\ 5 & 6
\end{pmatrix} \cdot  \begin{pmatrix}
12  & 10 & 8 & 6\\ 11 & 9 & 7 & 5
\end{pmatrix} \\
&=& \begin{pmatrix}
12+22i & 10+18i & 8+14i & 6+10i \\ 36+44 & 30+36 & 24+28 & 18+20 \\ 60+66 & 50+54  & 40+42 & 30+30
\end{pmatrix} =\begin{pmatrix} 12+22i & 10+18i & 8+14i & 6+10i \\ 80 & 66 & 52 & 38 \\ 126 & 104 & 82 & 60
\end{pmatrix} = (AB)^T
\end{eqnarray*}
\end{example}}

\lang{en}{
\begin{example}
For
\[A = \begin{pmatrix}
12 & 11\\ 10 & 9 \\8 & 7 \\ 6& 5
\end{pmatrix}\in M(4,2;\R)  \,\,\text{and } B = \begin{pmatrix}
1 & 3 & 5 \\ 2i & 4 & 6
\end{pmatrix}\in M(2,3;\C)
\]
we have, that the product is the $(4\times 3)$-matrix
\[ AB= \begin{pmatrix}
12+22i & 36+44 & 60+66\\ 10+18i & 30+36 & 50+54 \\ 8+14i & 24+28 & 40+42 \\ 6+10i & 18+20 & 30+30
\end{pmatrix} =\begin{pmatrix}  
12+22i & 80 & 126 \\ 10+18i & 66 & 104 \\ 8+14i & 52 & 82 \\ 6+10i & 38 & 60
\end{pmatrix} \in M(4, 3;\C), \]
hence 
\[ (AB)^T= \begin{pmatrix} 12+22i & 10+18i & 8+14i & 6+10i\\ 80 & 66 & 52 & 38 \\ 126 & 104 & 82 & 60
\end{pmatrix}\in M(3, 4;\C).\]
The product $B^T\cdot A^T$ is given by
\begin{eqnarray*}
 B^T\cdot A^T &=& \begin{pmatrix}
1 & 2i\\ 3& 4\\ 5 & 6
\end{pmatrix} \cdot  \begin{pmatrix}
12  & 10 & 8 & 6\\ 11 & 9 & 7 & 5
\end{pmatrix} \\
&=& \begin{pmatrix}
12+22i & 10+18i & 8+14i & 6+10i \\ 36+44 & 30+36 & 24+28 & 18+20 \\ 60+66 & 50+54  & 40+42 & 30+30
\end{pmatrix} =\begin{pmatrix} 12+22i & 10+18i & 8+14i & 6+10i \\ 80 & 66 & 52 & 38 \\ 126 & 104 & 82 & 60
\end{pmatrix} = (AB)^T
\end{eqnarray*}
\end{example}}


\end{visualizationwrapper}

\end{content}