%$Id:  $
\documentclass{mumie.article}
%$Id$
\begin{metainfo}
  \name{
    \lang{de}{Matrixmultiplikation}
    \lang{en}{Matrix multiplication}
  }
  \begin{description} 
 This work is licensed under the Creative Commons License Attribution 4.0 International (CC-BY 4.0)   
 https://creativecommons.org/licenses/by/4.0/legalcode 

    \lang{de}{Beschreibung}
    \lang{en}{Description}
  \end{description}
  \begin{components}
\component{generic_image}{content/rwth/HM1/images/g_tkz_T111_FalksScheme.meta.xml}{T111_FalksScheme}
\component{generic_image}{content/rwth/HM1/images/g_tkz_T111_MatrixProduct.meta.xml}{T111_MatrixProduct}
\component{generic_image}{content/rwth/HM1/images/g_img_00_video_button_schwarz-blau.meta.xml}{00_video_button_schwarz-blau}
\component{generic_image}{content/rwth/HM1/images/g_img_00_Videobutton_schwarz.meta.xml}{00_Videobutton_schwarz}
\end{components}
  \begin{links}
    \link{generic_article}{content/rwth/HM1/T401_Matrizenrechnung/g_art_content_02_matrizenmultiplikation.meta.xml}{content_02_matrizenmultiplikation}
    \link{generic_article}{content/rwth/HM1/T403_Quadratische_Matrizen,_Determinanten/g_art_content_07_quadratische_matrizen.meta.xml}{content_07_quadratische_matrizen}
    \link{generic_article}{content/rwth/HM1/T403a_Vektorraum/g_art_content_10b_lineare_abb.meta.xml}{content_10b_lineare_abb}
    \link{generic_article}{content/rwth/HM1/T401_Matrizenrechnung/g_art_content_01_matrizen.meta.xml}{matrizen}
  \end{links}
  \creategeneric
\end{metainfo}
\begin{content}
\usepackage{mumie.ombplus}
\ombchapter{1}
\ombarticle{2}
\usepackage{mumie.genericvisualization}

\begin{visualizationwrapper}

\title{\lang{de}{Matrixmultiplikation} \lang{en}{Matrix multiplication}}

\begin{block}[annotation]
  
\end{block}
\begin{block}[annotation]
  Im Ticket-System: \href{http://team.mumie.net/issues/11067}{Ticket 11067}\\
\end{block}

\begin{block}[info-box]
\tableofcontents
\end{block}

\lang{de}{
Abgesehen von der \link{matrizen}{Addition zweier Matrizen und der skalaren Multiplikation eines Körperelements mit  einer Matrix}, gibt es auch noch die Multiplikation von Matrizen und als Spezialfall die
Multiplikation einer Matrix mit einem Spaltenvektor.\\
Im Gegensatz zu den anderen Operationen werden die Multiplikationen hier nicht koeffizientenweise durchgeführt.
Insbesondere sind die Größen der zu multiplizierenden Matrizen im Allgemeinen verschieden, müssen aber "`zueinander
passen"'.}

\lang{en}{
Besides  \link{matrizen}{addition of two matrices and scalar multiplication with a field element}, there is multiplication 
of two matrices, including multiplication of matrix and a column vector. In contrast to the other operations, multiplications here 
will not be done componentwise. The size of the matrices to be multiplied can in general be different, but still need "`to fit together"'.}

\lang{de}{Im ganzen Abschnitt sei $\mathbb{K}$ ein Körper und alle Matrizen und Spaltenvektoren haben Einträge in 
$\mathbb{K}$.}

\lang{en}{In the entire section, let $\mathbb{K}$ be a field and all matrices and column vectors have entries in $\mathbb{K}$.}



%~\\
\lang{de}{
Neben einigen Wiederholungen aus Teil 1 sind die folgenden Inhalte neu:
\begin{itemize}
    \item Diverse Beipiele über den komplexen Zahlen $\C$.
    \item Beweisskizze zu den \ref[content_02_matrizenmultiplikation][Rechenregeln]{beweisskizzeRechenregelnMatMul}.
\end{itemize}}

\lang{en}{
Besides some repetition from Part 1, the following content will be new:
\begin{itemize}
\item Examples over the complex numbers $\C$.
\item Proof outlines for the \ref[content_02_matrizenmultiplikation][calculation rules]{beweisskizzeRechenregelnMatMul}.
\end{itemize}}



\section{\lang{de}{Matrix-Vektor-Multiplikation} \lang{en}{Matrix-vector multiplication} }\label{sec:matrix-vektor-mult}

\begin{definition}\label{def:matrix-vektor-multiplikation}
\lang{de}{
Eine $(m\times n)$-Matrix $A=(a_{ij})_{1 \leq i \leq m, 1 \leq j \leq n}$ kann mit einem Spaltenvektor $x=\left(  \begin{smallmatrix}
                                     x_1 \\      x_2\\   \vdots\\ x_n  \end{smallmatrix} \right)$ der L"ange $n$ multipliziert werden.
                                     Das Ergebnis ist dann ein Spaltenvektor der L"ange $m$:}

\lang{en}{A $(m\times n)$-matrix $A=(a_{ij})_{1 \leq i \leq m, 1 \leq j \leq n}$ can be multiplied with a column vector $x=\left(  \begin{smallmatrix}
                                     x_1 \\      x_2\\   \vdots\\ x_n  \end{smallmatrix} \right)$. The result is a column vector of length $m$:}

\[ A\cdot x=
\begin{pmatrix}
a_{11} & a_{12} & \cdots & a_{1n} \\
a_{21} & a_{22} & \cdots & a_{2n} \\
\vdots & \vdots & \ddots & \vdots \\
a_{m1} & a_{m2} & \cdots & a_{mn}
\end{pmatrix} \cdot  \begin{pmatrix}
                                     x_1 \\      x_2\\ \vdots\\  \vdots\\  x_n  \end{pmatrix} 
                                     = \begin{pmatrix}
                                     a_{11} x_1 + a_{12} x_2 + \cdots + a_{1n} x_n \\
                                     a_{21} x_1 + a_{22} x_2 + \cdots + a_{2n} x_n \\
                                     \vdots \\
                                     a_{m1} x_1 + a_{m2} x_2 + \cdots + a_{mn} x_n
                                     \end{pmatrix}
\]
\lang{en}{As often done with multiplication, we usually omit the multiplication symbol and write $Ax$ instead of $A\cdot x$.}
\lang{de}{Wie bei Multiplikationen üblich, wird der Mal-Punkt auch oft weggelassen und $Ax$ statt $A\cdot x$ geschrieben.\\
\floatright{\href{https://api.stream24.net/vod/getVideo.php?id=10962-2-11010&mode=iframe&speed=true}{\image[75]{00_video_button_schwarz-blau}}}
\floatright{\href{https://www.hm-kompakt.de/video?watch=801}{\image[75]{00_Videobutton_schwarz}}}}\\\\
\end{definition}

\begin{block}[warning]
\lang{de}{Das Produkt einer Matrix mit einem Spaltenvektor ist nur definiert, wenn die Länge des Spaltenvektors mit der Anzahl der Spalten
der Matrix übereinstimmt!}
\lang{en}{The result of multiplying a matrix with a column vector ist only defined, when the number of rows in the matrix 
is the length of the vector!}
\end{block}


\begin{example}
\lang{de}{
Wir betrachten 
\[ A= \begin{pmatrix} 2 & -3 & i \\ 0 & \frac{4}{3} & 5 \end{pmatrix} \quad \text{und} \quad x=\begin{pmatrix} 1 \\ 3 \\ -2i\end{pmatrix}.\]
$A$ ist eine $(2\times 3)$-Matrix über $\C$ und $x$ ist ein Spaltenvektor der Länge $3$. Seine Länge stimmt also mit der Spaltenzahl von $A$ überein,
weshalb man das Produkt der Matrix $A$ mit dem Spaltenvektor $x$ bilden kann. Das Ergebnis ist der Spaltenvektor
\[ A\cdot x= \begin{pmatrix} 2 & -3 & i \\ 0 & \frac{4}{3} & 5 \end{pmatrix} \cdot \begin{pmatrix} 1 \\ 3 \\ -2i\end{pmatrix}
=\begin{pmatrix} 2\cdot 1+(-3)\cdot 3+i\cdot (-2i) \\ 0\cdot 1+\frac{4}{3}\cdot 3 +5\cdot (-2i)\end{pmatrix} =\begin{pmatrix} -5\\ 4-10i\end{pmatrix}.
\]}

\lang{en}{Consider
\[ A= \begin{pmatrix} 2 & -3 & i \\ 0 & \frac{4}{3} & 5 \end{pmatrix} \quad \text{and} \quad x=\begin{pmatrix} 1 \\ 3 \\ -2i\end{pmatrix}.\]
$A$ is a $(2\times 3)$-matrix over $\C$ and $x$ is a column vector of length $3$. The number of columns in the matrix matches the length of the vector,
so the product of $A$ and $x$ is defined, giving
\[ A\cdot x= \begin{pmatrix} 2 & -3 & i \\ 0 & \frac{4}{3} & 5 \end{pmatrix} \cdot \begin{pmatrix} 1 \\ 3 \\ -2i\end{pmatrix}
=\begin{pmatrix} 2\cdot 1+(-3)\cdot 3+i\cdot (-2i) \\ 0\cdot 1+\frac{4}{3}\cdot 3 +5\cdot (-2i)\end{pmatrix} =\begin{pmatrix} -5\\ 4-10i\end{pmatrix}.
\]}
\end{example}


\section{\lang{de}{Multiplikation zweier Matrizen} \lang{en}{Multiplication of two matrices}} \label{sec:matrix-matrix-mult}

\lang{de}{Die Multiplikation von Matrizen stellt eine Verallgemeinerung der obigen \lref{sec:matrix-vektor-mult}{Matrix-Vektor-Multiplikation} dar.}
\lang{en}{Multiplication of two matrices is a generalisation of \lref{sec:matrix-vektor-mult}{matrix-vector-multiplikation}. }


\begin{definition}\label{def:matrix-multiplikation}
\lang{de}{
Eine $(m\times n)$-Matrix $A=(a_{ij})_{1 \leq i \leq m, 1 \leq j \leq n}$ kann mit einer $(n\times k)$-Matrix $B=(b_{ij})_{1 \leq i \leq n, 1 \leq j \leq k}$ multipliziert werden.}
\lang{en}{
A $(m\times n)$-matrix $A=(a_{ij})_{1 \leq i \leq m, 1 \leq j \leq n}$ can be multiplied with a $(n\times k)$-matrix $B=(b_{ij})_{1 \leq i \leq n, 1 \leq j \leq k}$.}

\lang{de}{
Das Ergebnis ist dann eine $(m\times k)$-Matrix % $C=(c_{ij})_{1 \leq i \leq m, 1 \leq j \leq k}$:
\[ A \cdot B = \left( \sum_{\ell=1}^n a_{i\ell} \cdot b_{\ell j} \right)_{1 \leq i \leq m, 1 \leq j \leq k} \]}
\lang{en}{
The result is a $(m\times k)$-matrix
\[ A \cdot B = \left( \sum_{\ell=1}^n a_{i\ell} \cdot b_{\ell j} \right)_{1 \leq i \leq m, 1 \leq j \leq k} \]}

\lang{de}{Ausführliche Schreibweise:}
\lang{en}{In more detailed notation:}

\begin{eqnarray*}
 A\cdot B &=&
\begin{pmatrix}
a_{11} & a_{12} & \cdots & a_{1n} \\
a_{21} & a_{22} & \cdots & a_{2n} \\
\vdots & \vdots & \ddots & \vdots \\
a_{m1} & a_{m2} & \cdots & a_{mn}
\end{pmatrix} \cdot  \begin{pmatrix}
b_{11} & b_{12} & \cdots & b_{1k} \\
b_{21} & b_{22} & \cdots & b_{2k} \\
\vdots & \vdots & \ddots & \vdots \\
b_{n1} & b_{n2} & \cdots & b_{nk}
\end{pmatrix} \\ && \\
&=& \begin{pmatrix}
\sum_{j=1}^n a_{1j}b_{j1}\, & \sum_{j=1}^n a_{1j}b_{j2}\, & \cdots & \,\sum_{j=1}^n a_{1j}b_{jk} \\
\sum_{j=1}^n a_{2j}b_{j1}\, & \sum_{j=1}^n a_{2j}b_{j2}\, & \cdots & \,\sum_{j=1}^n a_{2j}b_{jk} \\
\vdots & \vdots & \ddots & \vdots \\
\sum_{j=1}^n a_{mj}b_{j1}\, & \sum_{j=1}^n a_{mj}b_{j2}\, & \cdots &\, \sum_{j=1}^n a_{mj}b_{jk}
\end{pmatrix}
%                                      = \begin{pmatrix}
%                                      a_{11} b_{11} + a_{12} b_{21} + \cdots + a_{1n} b_{n1} &  a_{11} b_{12} + a_{12} b_{22} + \cdots + a_{1n} b_{n2} & \cdots &
%                                      a_{11} b_{1k} + a_{12} b_{2k} + \cdots + a_{1n} b_{nk}
%                                      \\
%                                      a_{21} b_{11} + a_{22} b_{21} + \cdots + a_{2n} b_{n1} & & & \vdots \\
%                                      \vdots  & & & \vdots \\
%                                      a_{m1} b_{11} + a_{m2} b_{21} + \cdots + a_{mn} b_{n1} & \cdots  & \cdots & a_{m1} b_{1k} + a_{m2} b_{2k} + \cdots + a_{mn} b_{nk}
%                                      \end{pmatrix}.
\end{eqnarray*}

\lang{en}{As often done with multiplication, we usually omit the multiplication symbol and write $AB$ instead of $A\cdot B$.}
\lang{de}{
Wie bei Multiplikationen üblich wird der Mal-Punkt auch oft weggelassen und $AB$ statt $A\cdot B$ geschrieben.\\
\floatright{\href{https://www.hm-kompakt.de/video?watch=819}{\image[75]{00_Videobutton_schwarz}}}}\\\\
\end{definition}

\lang{de}{In dem folgenden Video wird die Matrix-Multiplikation mit Hilfe \link{content_10b_lineare_abb}{linearer Abbilungen} eingeführt. 
Diese werden erst später im Kurs behandelt.
\floatright{\href{https://api.stream24.net/vod/getVideo.php?id=10962-2-11290&mode=iframe&speed=true}{\image[75]{00_video_button_schwarz-blau}}}}\\\\

%\begin{remark}  % nach unten verschoben + erweitert
%Die $j$-te Spalte des Matrizenprodukts $A\cdot B$ ist also nichts anderes als das Matrix-Vektor-Produkt von $A$ mit der $j$-ten Spalte von $B$.
%\end{remark}



\begin{block}[warning]
\lang{de}{
\begin{enumerate}
\item Das Produkt $AB$ zweier Matrizen $A$ und $B$ ist nur definiert, wenn die Anzahl der Spalten von $A$ mit der Anzahl der Zeilen von $B$ übereinstimmt.
\begin{center}
\image{T111_MatrixProduct}
\end{center}
\item Die Matrixultiplikation ist nicht kommutativ, d.h. im Allgemeinen gilt $AB\ne BA$, selbst wenn beide Seiten definiert sind.
\end{enumerate}}

\lang{en}{
\begin{enumerate}
\item The product $AB$ of two matrices $A$ und $B$ is only defined, when the number of columns in $A$ matches the number of rows in $B$.
\begin{center}
\image{T111_MatrixProduct}
\end{center}
\item The multiplication of two matrices is not commutative, that is, we do not have $AB\ne BA$, even if both sides are defined.
\end{enumerate}}
\end{block}


\begin{remark}
\label{falksches-schema}
\lang{de}{
Die Multiplikation zweier Matrizen kann über das \textit{Falksche Schema} veranschaulicht werden:}
\lang{en}{
The multiplication is illustrated by the following table:}
\begin{center}
\image{T111_FalksScheme}
\end{center}
\lang{de}{\floatright{\href{https://www.hm-kompakt.de/video?watch=820}{\image[75]{00_Videobutton_schwarz}}}}\\\\

\end{remark}



\begin{example}
\begin{tabs*}[\initialtab{0}]
\tab{\lang{de}{Produkt einer $(2\times 3)$-Matrix und einer $(3\times 2)$-Matrix} \lang{en}{Product of a $(2\times 3)$-matrix
with a $(3\times 2)$-matrix}} 
\lang{de}{
F\"ur
\[A = \begin{pmatrix}
1 & 3 & 5 \\ 2 & 4 & 6
\end{pmatrix} \,\,\text{und }
B = \begin{pmatrix}
12 & 11\\ 10 & 9 \\8 & 7
\end{pmatrix} 
\]
ist das Produkt $A\cdot B$ definiert, da $A$ drei Spalten hat und $B$ genauso viele Zeilen. 
Das Produkt einer $(2\times 3)$-Matrix mit einer $(3\times 2)$-Matrix ergibt dann eine $(2\times 2)$-Matrix und es ist
\begin{eqnarray*} AB &=& \begin{pmatrix} 1\cdot 12+3\cdot 10+5\cdot 8 & 1\cdot 11+3\cdot 9+5\cdot 7 \\
  2\cdot 12+4\cdot 10+6\cdot 8 & 2\cdot 11+4\cdot 9+6\cdot 7  \end{pmatrix}\\
  &=&  \begin{pmatrix}
12+30+40 & 11+27+35 \\ 24+40+48 & 22+36+42
\end{pmatrix}
 =  \begin{pmatrix}
82 & 73 \\ 112 & 100
\end{pmatrix}.\end{eqnarray*}}

\lang{en}{
The product $A\cdot B$ of the matrices
\[A = \begin{pmatrix}
1 & 3 & 5 \\ 2 & 4 & 6
\end{pmatrix} \,\,\text{und }
B = \begin{pmatrix}
12 & 11\\ 10 & 9 \\8 & 7
\end{pmatrix} 
\]
is defined, because $A$ has three columns and $B$ three rows. 
The product of a $(2\times 3)$-matrix with a $(3\times 2)$-matrix yields a $(2\times 2)$-matrix,
\begin{eqnarray*} AB &=& \begin{pmatrix} 1\cdot 12+3\cdot 10+5\cdot 8 & 1\cdot 11+3\cdot 9+5\cdot 7 \\
  2\cdot 12+4\cdot 10+6\cdot 8 & 2\cdot 11+4\cdot 9+6\cdot 7  \end{pmatrix}\\
  &=&  \begin{pmatrix}
12+30+40 & 11+27+35 \\ 24+40+48 & 22+36+42
\end{pmatrix}
 =  \begin{pmatrix}
82 & 73 \\ 112 & 100
\end{pmatrix}.\end{eqnarray*}}

\tab{\lang{de}{Fortsetzung} \lang{en}{Continuation}}
\lang{de}{
 F\"ur die beiden Matrizen
\[A = \begin{pmatrix}
1 & 3 & 5 \\ 2 & 4 & 6
\end{pmatrix} \,\,\text{und }
B = \begin{pmatrix}
12 & 11\\ 10 & 9 \\8 & 7
\end{pmatrix} 
\]
ist auch das Produkt $B\cdot A$ definiert, da $B$ zwei Spalten hat und $A$ genauso viele Zeilen. Das Ergebnis ist
dann aber die $(3\times 3)$-Matrix
\[ BA= \begin{pmatrix}12+22 & 36+44 & 60+66\\ 10+18 & 30+36 & 50+54 \\ 8+14 & 24+28 & 40+42 \end{pmatrix} 
=\begin{pmatrix}  
34 & 80 & 126 \\ 28 & 66 & 104 \\ 22 & 52 & 82\end{pmatrix}. \]}

\lang{en}{
 The product $B\cdot A$ for the matrices
\[A = \begin{pmatrix}
1 & 3 & 5 \\ 2 & 4 & 6
\end{pmatrix} \,\,\text{und }
B = \begin{pmatrix}
12 & 11\\ 10 & 9 \\8 & 7
\end{pmatrix} 
\]
is also defined, because $B$ have two columns and $A$ two rows. However, the result is a $(3\times 3)$-matrix
\[ BA= \begin{pmatrix}12+22 & 36+44 & 60+66\\ 10+18 & 30+36 & 50+54 \\ 8+14 & 24+28 & 40+42 \end{pmatrix} 
=\begin{pmatrix}  
34 & 80 & 126 \\ 28 & 66 & 104 \\ 22 & 52 & 82\end{pmatrix}. \]}

\end{tabs*}


\begin{tabs*}[\initialtab{0}]
\tab{\lang{de}{Produkt einer $(3\times 1)$-Matrix und einer $(1\times 2)$-Matrix} \lang{en}{Product of a $(3\times 1)$-matrix with a $(1\times 2)$-matrix}} 
\lang{de}{
F\"ur
\[ B=\begin{pmatrix} 1 \\ 3 \\ -2\end{pmatrix}\quad
 \text{und} \quad C=\begin{pmatrix} 2 & -3i \end{pmatrix}
 \]
 ist das Produkt $B\cdot C$ definiert, da $B$ eine Spalte hat und $C$ genauso viele Zeilen. Das Produkt ist dann eine $(3\times 2)$-Matrix,
 da $B$ drei Zeilen hat und $C$ zwei Spalten hat. Es ist
 \[ B\cdot C= \begin{pmatrix} 1\cdot 2 & 1\cdot (-3i) \\ 3\cdot 2 & 3\cdot (-3i)\\ -2\cdot 2 & -2\cdot (-3i) \end{pmatrix}
 = \begin{pmatrix} 2 & -3i \\ 6 & -9i\\ -4 & 6i \end{pmatrix}.\]}

 \lang{en}{
The product $B\cdot C$ of the matrices
\[ B=\begin{pmatrix} 1 \\ 3 \\ -2\end{pmatrix}\quad
 \text{und} \quad C=\begin{pmatrix} 2 & -3i \end{pmatrix}
 \]
 is defined, because $B$ has one column and $C$ one row. $B\cdot C$ is a $(3\times 2)$-matrix,
 because $B$ has three rows and $C$ two columns. It is
 \[ B\cdot C= \begin{pmatrix} 1\cdot 2 & 1\cdot (-3i) \\ 3\cdot 2 & 3\cdot (-3i)\\ -2\cdot 2 & -2\cdot (-3i) \end{pmatrix}
 = \begin{pmatrix} 2 & -3i \\ 6 & -9i\\ -4 & 6i \end{pmatrix}.\]}
\end{tabs*}

\begin{tabs*}[\initialtab{0}]
\tab{\lang{de}{$AB$ nicht definiert, aber $BA$ definiert} \lang{en}{$AB$ not defined, but $BA$ defined}} 
\lang{de}{F\"ur
\[A = \begin{pmatrix}
1 & 3 & 5 \\ 2 & 4 & 6
\end{pmatrix} \,\,\text{und } B = \begin{pmatrix}
12 & 11\\ 10 & 9 \\8 & 7 \\ 6& 5
\end{pmatrix} 
\]
ist das Produkt $A\cdot B$ nicht definiert, da $A$ nur $3$ Spalten hat, $B$ jedoch $4$ Zeilen.}

\lang{en}{
For the matrices
\[A = \begin{pmatrix}
1 & 3 & 5 \\ 2 & 4 & 6
\end{pmatrix} \,\,\text{und } B = \begin{pmatrix}
12 & 11\\ 10 & 9 \\8 & 7 \\ 6& 5
\end{pmatrix} 
\]
the product $A\cdot B$ is not defined, as $A$ has only three columns, but $B$ has four rows.}

\lang{de}{Das Produkt $B\cdot A$ ist jedoch definiert, da $B$ zwei Spalten hat und $A$ zwei Zeilen, und es gilt
\begin{eqnarray*} B\cdot A &=& \begin{pmatrix}
12 & 11\\ 10 & 9 \\8 & 7 \\ 6& 5
\end{pmatrix} \cdot \begin{pmatrix}
1 & 3 & 5 \\ 2 & 4 & 6
\end{pmatrix}\\
&=& \begin{pmatrix}
12+22 & 36+44 & 60+66\\ 10+18 & 30+36 & 50+54 \\ 8+14 & 24+28 & 40+42 \\ 6+10 & 18+20 & 30+30
\end{pmatrix} =\begin{pmatrix}  
34 & 80 & 126 \\ 28 & 66 & 104 \\ 22 & 52 & 82 \\ 16 & 38 & 60
\end{pmatrix} \in M(4, 3;\mathbb{R}). \end{eqnarray*}}

\lang{en}{However, $B\cdot A$ is defined, as $B$ has two columns and $A$ two rows, giving
\begin{eqnarray*} B\cdot A &=& \begin{pmatrix}
12 & 11\\ 10 & 9 \\8 & 7 \\ 6& 5
\end{pmatrix} \cdot \begin{pmatrix}
1 & 3 & 5 \\ 2 & 4 & 6
\end{pmatrix}\\
&=& \begin{pmatrix}
12+22 & 36+44 & 60+66\\ 10+18 & 30+36 & 50+54 \\ 8+14 & 24+28 & 40+42 \\ 6+10 & 18+20 & 30+30
\end{pmatrix} =\begin{pmatrix}  
34 & 80 & 126 \\ 28 & 66 & 104 \\ 22 & 52 & 82 \\ 16 & 38 & 60
\end{pmatrix} \in M(4, 3;\mathbb{R}). \end{eqnarray*}}
\end{tabs*}


\begin{tabs*}[\initialtab{0}]
\tab{\lang{de}{Quadratische Matrizen $A,B$ mit $AB\neq BA$} \lang{en}{Square matrices $A,B$ with $AB\neq BA$}}
\lang{de}{Selbst für \emph{quadratische} Matrizen (d.h. mit gleicher Zeilenzahl wie Spaltenzahl) sind $AB$ und $BA$ meist verschieden. Zum Beispiel gilt
f\"ur}
\lang{en}{Even for square matrices, $AB$ and $BA$ are usually different. It is, for example, }
$A = \begin{pmatrix}
1 & 2 \\ 0& 1
\end{pmatrix} \,\,\text{und }
B = \begin{pmatrix}
2 & 0\\ 1&1 \end{pmatrix} \in M(2, 2;\R) 
$
\[ AB=\begin{pmatrix}
1 & 2 \\ 0& 1
\end{pmatrix}\cdot \begin{pmatrix}
2 & 0\\ 1&1 \end{pmatrix}=\begin{pmatrix} 4&2\\ 1&1\end{pmatrix}, \]
\lang{de}{aber} \lang{en}{but}
\[ BA=\begin{pmatrix}
2 & 0\\ 1&1 \end{pmatrix}\cdot\begin{pmatrix}
1 & 2 \\ 0& 1
\end{pmatrix}=\begin{pmatrix} 2&4\\ 1&3\end{pmatrix}. \]
\end{tabs*}
\end{example}


\begin{remark}
\lang{de}{
Die Matrix-Vektor-Multiplikation wurde zu Beginn dieses Kapitels eingeführt.
Die $j$-te Spalte des Matrizenprodukts $A\cdot B$ ist also nichts anderes als das Matrix-Vektor-Produkt von $A$ mit der $j$-ten Spalte von $B$:}
\lang{en}{
At the start of this Chapter, we had a look at matrix-vector multiplication. The $j$th column of the matrix product $A\cdot B$ ist nothing more
than the matrix-vector product of $A$ with the $j$th column of $B$:}

\begin{itemize}
    \item 
        \lang{de}{Eine Matrix $A \in {M}(m,n;\K)$ besitze die Spaltenvektoren $a_1, a_2, \cdots, a_n \in \K^m$.
    \[
        A =
        \begin{pmatrix}
            | & | & & | \\
            a_1 & a_2 & \cdots & a_n \\
            | & | & & |
        \end{pmatrix}
    \]
        Ein Spaltenvektor $x \in \K^n$ habe die Komponenten $x_1, x_2, \cdots, x_n$.
Dann ist 
    \[
        A \cdot x = x_1 a_1 + x_2 a_2 + \cdots + x_n a_n \in \K^m
    \] 
eine Linearkombination der Spaltenvektoren von $A$.
Die Gewichte der einzelnen Vektoren sind durch den Vektor $x$ gegeben.}

\lang{en}{
A matrix $A\in {M}(m,n;\K)$ 'contains' the column vectors $a_1, a_2,\cdots, a_n \in\K^m$.
 \[
        A =
        \begin{pmatrix}
            | & | & & | \\
            a_1 & a_2 & \cdots & a_n \\
            | & | & & |
        \end{pmatrix}
    \]
Consider a column vector $x\in\K^n$ with components $x_1, x_2,\cdots, x_n$. Then
\[
        A \cdot x = x_1 a_1 + x_2 a_2 + \cdots + x_n a_n \in \K^m
    \] 
    is a linear combination of the column vectors in $A$. The 'weights' (coefficients) of each vector are given by the components of $x$.}

\textit{
\lang{de}{
    Ausblick: $f(x)=A \cdot x$ definiert eine lineare Abbildung $f$ von $~\K^n$ nach $~\K^m$. 
    Lineare Abbildungen werden \link{content_10b_lineare_abb}{hier} vertieft.
}
\lang{en}{
Preview: $f(x)=A\cdot x$ defines a linear map $f$ from $~\K^n$ to $~\K^m$. Linear maps are covered \link{content_10b_lineare_abb}{here}.}
}

    \item 
\lang{de}{ Eine Matrix $B \in {M}(n,k;\K)$ besitze die Spaltenvektoren $b_1, b_2, \cdots, b_k \in \K^n$,
also 
\[
B=\begin{pmatrix}
| & | & & | \\
b_1 & b_2 & \cdots & b_k \\
| & | & & | 
\end{pmatrix}.
\]
Dann ist:
    \[
        A \cdot B =
        \begin{pmatrix}
        | & | & & | \\
            A \cdot b_1 & A \cdot b_2 & \cdots & A \cdot b_k \\
            | & | & & |
        \end{pmatrix}
    \]}

\lang{en}{
A matrix $B\in{M}(n,m;\K)$ contains the column vectors $b_1, b_2,\cdots, b_k\in\K^n$, so
\[
B=\begin{pmatrix}
| & | & & | \\
b_1 & b_2 & \cdots & b_k \\
| & | & & | 
\end{pmatrix}.
\]
Hence
\[
        A \cdot B =
        \begin{pmatrix}
        | & | & & | \\
            A \cdot b_1 & A \cdot b_2 & \cdots & A \cdot b_k \\
            | & | & & |
        \end{pmatrix}
    \]
    }
\end{itemize}

\end{remark}

\begin{example}
  \begin{tabs*}[\initialtab{0}]
  
    \tab{\lang{de}{Beispiel zur Bemerkung} \lang{en}{Example for the remark}}
  
   \lang{de}{Wir verdeutlichen die Bemerkung mit einem Beispiel.
   Wir möchten
   \[
   \begin{pmatrix}
              1 & 2 \\
              3 & 4
          \end{pmatrix}
          \cdot
          \begin{pmatrix}
              \textcolor{#CC6600}{5} & \textcolor{#0066CC}{6} \\
              \textcolor{#CC6600}{7} & \textcolor{#0066CC}{8}
          \end{pmatrix}
   \]
   berechnen.
   
   Die erste Spalte des Matrizenprodukts können wir durch
   \[
   \begin{pmatrix}
                  1 & 2 \\
                  3 & 4
              \end{pmatrix}
              \cdot
              \textcolor{#CC6600}{\begin{pmatrix}
                  5 \\
                  7
              \end{pmatrix}} = \textcolor{#CC6600}{\begin{pmatrix}
            1 \cdot 5 + 2 \cdot 7 \\
            3 \cdot 5 + 4 \cdot 7 
        \end{pmatrix}}= \textcolor{#CC6600}{\begin{pmatrix}
            19 \\ 43
        \end{pmatrix}}
   \]
   berechnen und die zweite Spalte ist durch
   \[
    \begin{pmatrix}
                  1 & 2 \\
                  3 & 4
              \end{pmatrix}
              \cdot\textcolor{#0066CC}{\begin{pmatrix}
                  6 \\
                  8
              \end{pmatrix}}
               = \textcolor{#0066CC}{\begin{pmatrix}
              1 \cdot 6 + 2 \cdot 8 \\
              3 \cdot 6 + 4 \cdot 8 
        \end{pmatrix}}= \textcolor{#0066CC}{\begin{pmatrix}
             22 \\ 50
        \end{pmatrix}}
   \]
   gegeben.
   
   Also ist
   \[
   \begin{pmatrix}
              1 & 2 \\
              3 & 4
          \end{pmatrix}
          \cdot
          \begin{pmatrix}
              5 & 6 \\
              7 & 8
          \end{pmatrix} = 
          \begin{pmatrix}
             \textcolor{#CC6600}{19} & \textcolor{#0066CC}{22} \\
            \textcolor{#CC6600}{43} & \textcolor{#0066CC}{50}
          \end{pmatrix}.
   \]}
   
\lang{en}{We clarify the remark with an example.
   We want to calculate
   \[
   \begin{pmatrix}
              1 & 2 \\
              3 & 4
          \end{pmatrix}
          \cdot
          \begin{pmatrix}
              \textcolor{#CC6600}{5} & \textcolor{#0066CC}{6} \\
              \textcolor{#CC6600}{7} & \textcolor{#0066CC}{8}
          \end{pmatrix}
   \]
   .
   
   The first column of the matrix product can be calculated to be
   \[
   \begin{pmatrix}
                  1 & 2 \\
                  3 & 4
              \end{pmatrix}
              \cdot
              \textcolor{#CC6600}{\begin{pmatrix}
                  5 \\
                  7
              \end{pmatrix}} = \textcolor{#CC6600}{\begin{pmatrix}
            1 \cdot 5 + 2 \cdot 7 \\
            3 \cdot 5 + 4 \cdot 7 
        \end{pmatrix}}= \textcolor{#CC6600}{\begin{pmatrix}
            19 \\ 43
        \end{pmatrix}}
   \]
   and the second column to be
   \[
    \begin{pmatrix}
                  1 & 2 \\
                  3 & 4
              \end{pmatrix}
              \cdot\textcolor{#0066CC}{\begin{pmatrix}
                  6 \\
                  8
              \end{pmatrix}}
               = \textcolor{#0066CC}{\begin{pmatrix}
              1 \cdot 6 + 2 \cdot 8 \\
              3 \cdot 6 + 4 \cdot 8 
        \end{pmatrix}}= \textcolor{#0066CC}{\begin{pmatrix}
             22 \\ 50
        \end{pmatrix}}
   \]
   
   
   Therefore
   \[
   \begin{pmatrix}
              1 & 2 \\
              3 & 4
          \end{pmatrix}
          \cdot
          \begin{pmatrix}
              5 & 6 \\
              7 & 8
          \end{pmatrix} = 
          \begin{pmatrix}
             \textcolor{#CC6600}{19} & \textcolor{#0066CC}{22} \\
            \textcolor{#CC6600}{43} & \textcolor{#0066CC}{50}
          \end{pmatrix}.
   \]}
  
%       \begin{eqnarray*}
%           \begin{pmatrix}
%               1 & 2 \\
%               3 & 4
%           \end{pmatrix}
%           \cdot
%           \begin{pmatrix}
%               5 & 6 \\
%               7 & 8
%           \end{pmatrix}
%           &=&
%           \begin{pmatrix}
%               \begin{pmatrix}
%                   1 & 2 \\
%                   3 & 4
%               \end{pmatrix}
%               \cdot
%               \begin{pmatrix}
%                   5 \\
%                   7
%               \end{pmatrix}
%               &
%               \begin{pmatrix}
%                   1 & 2 \\
%                   3 & 4
%               \end{pmatrix}
%               \cdot
%               \begin{pmatrix}
%                   6 \\
%                   8
%               \end{pmatrix}
%           \end{pmatrix}
%         \\
%         &&
%         \begin{pmatrix}
%             1 \cdot 5 + 2 \cdot 7 & 1 \cdot 6 + 2 \cdot 8 \\
%             3 \cdot 5 + 4 \cdot 7 & 3 \cdot 6 + 4 \cdot 8 
%         \end{pmatrix}
%         =
%         \begin{pmatrix}
%             19 & 22 \\
%             43 & 50
%         \end{pmatrix}
%       \end{eqnarray*}
      
    \end{tabs*}
\end{example}


\section{\lang{de}{Rechenregeln} \lang{en}{Rules for matrix multplication}}\label{sec:rechenregeln}

\lang{de}{
Abgesehen von der Vertauschung gelten für die Matrizen-Multiplikation 
die üblichen Regeln.}

\lang{en}{Besides the commutativity, the usual rules also apply for matrix multiplication.}

\begin{rule}\label{rule:rechenregeln}
\lang{de}{
\begin{enumerate}
\item \emph{(Assoziativregel)} Für alle $A\in M(m,n;\mathbb{K})$, $B \in M(n,k;\mathbb{K})$ und $C \in M(k,l;\mathbb{K})$ gilt
\[ A \cdot (B\cdot C) = (A\cdot B)\cdot C \in M(m,l;\mathbb{K}). \]
\item \emph{(Distributivregeln)} Für alle $A,D \in M(m,n;\mathbb{K})$ und $B,C \in M(n,k;\mathbb{K})$  gelten
\[ A\cdot (B+C) =AB+ AC,\quad \text{und}\quad (A+D)B=AB+DB. \]
\item \emph{(Verträglichkeit mit Skalaren)} Für alle $A\in M(m,n;\mathbb{K})$, $B \in M(n,k;\mathbb{K})$ und reelle Zahlen $r\in \mathbb{K}$ gilt
\[ r(AB)=(rA)\cdot B= A\cdot (rB). \]
\end{enumerate}
\floatright{\href{https://www.hm-kompakt.de/video?watch=823}{\image[75]{00_Videobutton_schwarz}}}}\\\\

\lang{en}{
\begin{enumerate}
\item \emph{(Associativity)} For all $A\in M(m,n;\mathbb{K})$, $B \in M(n,k;\mathbb{K})$ and $C \in M(k,l;\mathbb{K})$ we have
\[ A \cdot (B\cdot C) = (A\cdot B)\cdot C \in M(m,l;\mathbb{K}). \]
\item \emph{(Distributivity over addition)} For all $A,D \in M(m,n;\mathbb{K})$ and $B,C \in M(n,k;\mathbb{K})$  we have
\[ A\cdot (B+C) =AB+ AC,\quad \text{und}\quad (A+D)B=AB+DB. \]
\item \emph{(Compability with scalar multiplication)} For all $A\in M(m,n;\mathbb{K})$, $B \in M(n,k;\mathbb{K})$ and real numbers $r\in \mathbb{K}$ we have
\[ r(AB)=(rA)\cdot B= A\cdot (rB). \]
\end{enumerate}}
%\textit{Anmerkung: $E$ bezeichnet hier nicht die \link{content_07_quadratische_matrizen}{Einheitsmatrix} $E_n$. }
\end{rule}

\lang{de}{Das folgende Video stellt die Regel wieder im Zusammenhang mit  \link{content_10b_lineare_abb}{linearer Abbilungen} dar. 
\floatright{\href{https://api.stream24.net/vod/getVideo.php?id=10962-2-11262&mode=iframe&speed=true}{\image[75]{00_video_button_schwarz-blau}}}}\\\\


\begin{block}[warning]
\lang{de}{
Bei der letzten Regel muss die Reihenfolge von $A$ und $B$ stets gleich bleiben. Lediglich der Skalar $r$ darf mit $A$ vertauscht werden.}
\lang{en}{
In the last rule, the order of $A$ and $B$ must remain the same. Only the scalar $r$ may commutate with $A$.}
\end{block}

\label{beweisskizzeRechenregelnMatMul}

%\begin{block}[explanation]
\begin{proof*}[\lang{de}{Beweisskizze} \lang{en}{Proof outline}]
\lang{de}{
Es müssen auch hier lediglich die Definitionen der Multiplikationen und Additionen eingesetzt werden, und die
Rechenregeln in Körpern ausgenutzt werden. Dies soll beispielhaft an der Assoziativregel demonstriert werden.
 Die linke Seite bei der Assoziativregel ist das Produkt der Matrix $A$ mit dem Produkt $B\cdot C$.
Letztere hat als $(h,j)$-Eintrag die Summe $\sum_{i=1}^k b_{hi}c_{ij}$. Damit ist der Eintrag von
$A(BC)$ an der Stelle $(g,j)$ gegeben durch
\[   \sum_{h=1}^n a_{gh}\cdot \left( \sum_{i=1}^k b_{hi}c_{ij}\right). \]
Entsprechend ist der Eintrag von $(AB)C$ an der Stelle $(g,j)$ gegeben durch
\[ \sum_{i=1}^k  \left( \sum_{h=1}^n a_{gh}b_{hi} \right)\cdot c_{ij}.\]
Mit dem Distributivgesetz und der Kommutativität der Addition in $\mathbb{K}$ ist beides gleich
\[ \sum_{i=1}^k \sum_{h=1}^n a_{gh}b_{hi}c_{ij}. \]}

\lang{en}{
Like before, we only need to use the definitions for multiplication and addition and apply the calculation rules that hold in fields.
This will be demonstrated exemplary with the associativity. The left side of the associative rule is the product of matrix $A$ with the 
product $B\cdot C$. The latter has the sum $\sum_{i=1}^k b_{hi}c_{ij}$ as its $(h,j)$th entry. The entry $(g,j)$ in $A(BC)$ is given by
\[   \sum_{h=1}^n a_{gh}\cdot \left( \sum_{i=1}^k b_{hi}c_{ij}\right). \]
Equivalently, the entry $(g,j)$ in $(AB)C$ is given by
\[ \sum_{i=1}^k  \left( \sum_{h=1}^n a_{gh}b_{hi} \right)\cdot c_{ij}.\]
Using the distributivity and the commutativity of addition in $\mathbb{K}$, both terms equal
\[ \sum_{i=1}^k \sum_{h=1}^n a_{gh}b_{hi}c_{ij}. \]}
%\end{block}
\end{proof*}

\begin{example}
\lang{de}{
\begin{tabs*}
\tab{$A (B C) = (A B) C$}
Wir betrachten
\[ A= \begin{pmatrix} 2 & -3 & 1 \\ 0 & \frac{4}{3} & 5 \end{pmatrix}, \quad B=\begin{pmatrix} 1 \\ 3 \\ -2\end{pmatrix}\quad
 \text{und} \quad C=\begin{pmatrix} 2 & -3 \end{pmatrix}.\]
Dann sind
 \begin{eqnarray*} A(BC)&=& \begin{pmatrix} 2 & -3 & 1 \\ 0 & \frac{4}{3} & 5 \end{pmatrix}\cdot \Big( \begin{pmatrix} 1 \\ 3 \\ -2\end{pmatrix} \cdot
 \begin{pmatrix} 2 & -3 \end{pmatrix} \Big) \\
 &=& \begin{pmatrix} 2 & -3 & 1 \\ 0 & \frac{4}{3} & 5 \end{pmatrix}\cdot \begin{pmatrix} 2 & -3 \\ 6 & -9\\ -4 & 6 \end{pmatrix} \\
 &=& \begin{pmatrix} -18 & 27 \\ -12 & 18 \end{pmatrix}
 \end{eqnarray*}
 und
 \begin{eqnarray*} (AB)C &=& \Big( \begin{pmatrix} 2 & -3 & 1 \\ 0 & \frac{4}{3} & 5 \end{pmatrix}\cdot \begin{pmatrix} 1 \\ 3 \\ -2\end{pmatrix}\Big) \cdot
 \begin{pmatrix} 2 & -3 \end{pmatrix}  \\
 &=& \begin{pmatrix} 2\cdot 1 + (-3)\cdot 3 + 1\cdot (-2) \\ 0\cdot 1 + \frac{4}{3} \cdot 3 + 5\cdot (-2) \end{pmatrix}\cdot
 \begin{pmatrix} 2 & -3 \end{pmatrix} = \begin{pmatrix} -9\\ -6 \end{pmatrix} \cdot
 \begin{pmatrix} 2 & -3 \end{pmatrix} \\
 &=& \begin{pmatrix} -18 & 27 \\ -12 & 18 \end{pmatrix}
 \end{eqnarray*}
\tab{$A(B+C)=AB+AC$}
Wir betrachten
\[ A= \begin{pmatrix} 2 & -3 & 1 \\ 0 & \frac{4}{3} & 5 \end{pmatrix}, \quad B=\begin{pmatrix} 1 \\ 3 \\ -2\end{pmatrix}\quad
 \text{und} \quad C=\begin{pmatrix} 2i \\ -3 \\ 1\end{pmatrix}.\]
 Dann sind
 \begin{eqnarray*} A(B+C) &=& \begin{pmatrix} 2 & -3 & 1 \\ 0 & \frac{4}{3} & 5 \end{pmatrix}\cdot \begin{pmatrix} 1+2i \\ 3-3 \\ -2+1\end{pmatrix}\\
 &=& \begin{pmatrix} 2 & -3 & 1 \\ 0 & \frac{4}{3} & 5 \end{pmatrix}\cdot \begin{pmatrix} 1+2i \\ 0 \\ -1\end{pmatrix}\\
 &=& \begin{pmatrix} 2\cdot (1+2i)+(-3)\cdot 0+1\cdot (-1) \\ 0\cdot (1+2i)+\frac{4}{3} \cdot 0+5\cdot (-1) \end{pmatrix} =\begin{pmatrix} 1+4i\\ -5 \end{pmatrix}
 \end{eqnarray*}
 und 
  \begin{eqnarray*} AB+AC &=& \begin{pmatrix} 2\cdot 1 + (-3)\cdot 3 + 1\cdot (-2) \\ 0\cdot 1 + \frac{4}{3} \cdot 3 + 5\cdot (-2) \end{pmatrix}
   +\begin{pmatrix} 2\cdot 2i + (-3)\cdot (-3) + 1\cdot 1 \\ 0\cdot 2i + \frac{4}{3}\cdot (-3)+ 5\cdot 1 \end{pmatrix} \\
 &=& \begin{pmatrix} -9\\ -6 \end{pmatrix}+ \begin{pmatrix} 10+4i \\ 1\end{pmatrix} = \begin{pmatrix} 1+4i\\ -5 \end{pmatrix}.
 \end{eqnarray*}
 \end{tabs*}}

 \lang{en}{
\begin{tabs*}
\tab{$A (B C) = (A B) C$}
Consider
\[ A= \begin{pmatrix} 2 & -3 & 1 \\ 0 & \frac{4}{3} & 5 \end{pmatrix}, \quad B=\begin{pmatrix} 1 \\ 3 \\ -2\end{pmatrix}\quad
 \text{und} \quad C=\begin{pmatrix} 2 & -3 \end{pmatrix}.\]
Then we have
 \begin{eqnarray*} A(BC)&=& \begin{pmatrix} 2 & -3 & 1 \\ 0 & \frac{4}{3} & 5 \end{pmatrix}\cdot \Big( \begin{pmatrix} 1 \\ 3 \\ -2\end{pmatrix} \cdot
 \begin{pmatrix} 2 & -3 \end{pmatrix} \Big) \\
 &=& \begin{pmatrix} 2 & -3 & 1 \\ 0 & \frac{4}{3} & 5 \end{pmatrix}\cdot \begin{pmatrix} 2 & -3 \\ 6 & -9\\ -4 & 6 \end{pmatrix} \\
 &=& \begin{pmatrix} -18 & 27 \\ -12 & 18 \end{pmatrix}
 \end{eqnarray*}
 and
 \begin{eqnarray*} (AB)C &=& \Big( \begin{pmatrix} 2 & -3 & 1 \\ 0 & \frac{4}{3} & 5 \end{pmatrix}\cdot \begin{pmatrix} 1 \\ 3 \\ -2\end{pmatrix}\Big) \cdot
 \begin{pmatrix} 2 & -3 \end{pmatrix}  \\
 &=& \begin{pmatrix} 2\cdot 1 + (-3)\cdot 3 + 1\cdot (-2) \\ 0\cdot 1 + \frac{4}{3} \cdot 3 + 5\cdot (-2) \end{pmatrix}\cdot
 \begin{pmatrix} 2 & -3 \end{pmatrix} = \begin{pmatrix} -9\\ -6 \end{pmatrix} \cdot
 \begin{pmatrix} 2 & -3 \end{pmatrix} \\
 &=& \begin{pmatrix} -18 & 27 \\ -12 & 18 \end{pmatrix}
 \end{eqnarray*}
\tab{$A(B+C)=AB+AC$}
Consider
\[ A= \begin{pmatrix} 2 & -3 & 1 \\ 0 & \frac{4}{3} & 5 \end{pmatrix}, \quad B=\begin{pmatrix} 1 \\ 3 \\ -2\end{pmatrix}\quad
 \text{und} \quad C=\begin{pmatrix} 2i \\ -3 \\ 1\end{pmatrix}.\]
 Then we have
 \begin{eqnarray*} A(B+C) &=& \begin{pmatrix} 2 & -3 & 1 \\ 0 & \frac{4}{3} & 5 \end{pmatrix}\cdot \begin{pmatrix} 1+2i \\ 3-3 \\ -2+1\end{pmatrix}\\
 &=& \begin{pmatrix} 2 & -3 & 1 \\ 0 & \frac{4}{3} & 5 \end{pmatrix}\cdot \begin{pmatrix} 1+2i \\ 0 \\ -1\end{pmatrix}\\
 &=& \begin{pmatrix} 2\cdot (1+2i)+(-3)\cdot 0+1\cdot (-1) \\ 0\cdot (1+2i)+\frac{4}{3} \cdot 0+5\cdot (-1) \end{pmatrix} =\begin{pmatrix} 1+4i\\ -5 \end{pmatrix}
 \end{eqnarray*}
 and 
  \begin{eqnarray*} AB+AC &=& \begin{pmatrix} 2\cdot 1 + (-3)\cdot 3 + 1\cdot (-2) \\ 0\cdot 1 + \frac{4}{3} \cdot 3 + 5\cdot (-2) \end{pmatrix}
   +\begin{pmatrix} 2\cdot 2i + (-3)\cdot (-3) + 1\cdot 1 \\ 0\cdot 2i + \frac{4}{3}\cdot (-3)+ 5\cdot 1 \end{pmatrix} \\
 &=& \begin{pmatrix} -9\\ -6 \end{pmatrix}+ \begin{pmatrix} 10+4i \\ 1\end{pmatrix} = \begin{pmatrix} 1+4i\\ -5 \end{pmatrix}.
 \end{eqnarray*}
 \end{tabs*}}
 \end{example}

\begin{remark}
\lang{de}{
Als Spezialfall der Matrixmultiplikation erfüllt auch die Matrix-Vektor-Multiplikation 
die oben aufgezeigten Regeln.
Für den Vektor nehme man jeweils eine Matrix mit der Spaltenzahl $1$ an.}
\lang{en}{
Matrix-vector multiplication as a special case of matrix multiplication also fulfils the rules above. For that, we consider
the vector as a matrix with only one column.
}
%entsprechende Regeln, die man dadurch erhält, dass man jeweils bei der letzten Matrix einer Gleichung als Spaltenzahl $1$ setzt. 
\end{remark}



%% TODO: Das nachfolgend auskommentierte Hadamardprodukt könnte dem Kurskanon 
%% zukünftig hinzugefügt werden. Z.Zt. gibt es hierzu nur die oben stehende Bemerkung.
%% Sollten die folgenden Zeilen einbezogen werden, so müssten auch Beispiel- und
%% Trainingsaufgaben ausgearbeitet werden.

%\section{Hadamard-Produkt}
%
%Neben der oben definierten Matrixmultiplikation gibt es noch weitere multiplikative Verknüpfungen von Matrizen.
%
%\begin{definition}
%    Das Hadamard-Produkt ist das elementweise Produkt zweier $(m \times n)$-Matrizen $A=(a_{ij})_{1\leq i\leq m,1\leq j\leq n}$ und $B=(b_{ij})_{1\leq i\leq m,1\leq j\leq n}$. % ist wie folgt definiert. 
%    Das Ergebnis ist wieder eine $(m \times n)$-Matrix:
%       \[ A \circ B = \left( a_{ij} \cdot b_{ij} \right)_{1 \leq i \leq m, 1 \leq j \leq n} \]
%    Ausführliche Schreibweise:
%    \begin{eqnarray*}
%        A \circ B &=&
%        \begin{pmatrix}
%            a_{11} & a_{12} & \cdots & a_{1n} \\
%            a_{21} & a_{22} & \cdots & a_{2n} \\
%            \vdots & \vdots & \ddots & \vdots \\
%            a_{m1} & a_{m2} & \cdots & a_{mn}
%        \end{pmatrix} 
%        \quad\circ\quad
%        \begin{pmatrix}
%            b_{11} & b_{12} & \cdots & b_{1n} \\
%            b_{21} & b_{22} & \cdots & b_{2n} \\
%            \vdots & \vdots & \ddots & \vdots \\
%            b_{m1} & b_{m2} & \cdots & b_{mn}
%        \end{pmatrix}
%        =\\&&
%        \begin{pmatrix}
%            a_{11} b_{11} & a_{12} b_{12} & \cdots & a_{1n} b_{1n} \\
%            a_{21} b_{21} & a_{22} b_{22} & \cdots & a_{2n} b_{2n} \\
%            \vdots & \vdots & \ddots & \vdots \\
%            a_{m1} b_{m1} & a_{m2} b_{m2} & \cdots & a_{mn} b_{mn}
%        \end{pmatrix} 
%    \end{eqnarray*}
%    %Jeder Eintrag $(a_{ij} \cdot b_{ij})$ berechnet sich also durch eine komponentenweise Multiplikation.
%\end{definition}
%
%\begin{block}[warning]
%    Der Operator $\circ$ wird beim Hadamard Produkt immer explizit angegeben.
%    Die verkürzende Schreibweise $AB$ bezeichnet immer die Matrixmultiplikation aus 
%    Definition 2.3. % TODO: statischer Verweis
%\end{block}
%
%\begin{example}
%\begin{tabs*}%[\initialtab{0}]
%\tab{a)} 
%Für
%    \begin{equation*}
%        A=
%        \begin{pmatrix}
%            1 & 2 \\
%            3 & 4 \\
%        \end{pmatrix}
%        \text{ und }
%        B=
%        \begin{pmatrix}
%            2 & 3 \\
%            3 & 1 \\
%        \end{pmatrix}
%    \end{equation*}
%ist das Hadamard-Produkt $A \circ B$ definiert, da die Anzahl der Zeilen und Spalten von $A$ und $B$ übereinstimmen.
%Das Hadamard-Produkt ist dann ebenfalls eine $(2 \times 2)$-Matrix:
%    \begin{equation*}
%        A \circ B =
%        \begin{pmatrix}
%            1 & 2 \\
%            3 & 4 \\
%        \end{pmatrix}
%        \circ
%        \begin{pmatrix}
%            2 & 3 \\
%            3 & 1 \\
%        \end{pmatrix}
%        =
%        \begin{pmatrix}
%            1 \cdot 2 & 2 \cdot 3 \\
%            3 \cdot 3 & 4 \cdot 1 \\
%        \end{pmatrix}
%        =
%        \begin{pmatrix}
%            2 & 6\\
%            9 & 4 \\
%        \end{pmatrix}
%    \end{equation*}
%
%\tab{b)} 
%Für
%    \begin{equation*}
%        A=
%        \begin{pmatrix}
%            1 & 2 & 1 \\
%            3 & 4 & 0 \\
%        \end{pmatrix}
%        \text{ und }
%        B=
%        \begin{pmatrix}
%            2 & 3 \\
%            3 & 1 \\
%            1 & 7 \\
%        \end{pmatrix}
%    \end{equation*}
%ist das Hadamard-Produkt $A \circ B$ nicht definiert.
%Die Anzahl der Spalten von $A$ ist ungleich der Anzahl der Spalten von $B$. Gleiches gilt für die Anzahl der Zeilen von $A$ und $B$.
%
%\end{tabs*}
%\end{example}
%
%\begin{remark}
%Im Gegensatz zur Matrixmultiplikation ist das Hadamard-Produkt \textit{kommutativ}:
%\begin{itemize}
%    \item Für alle $A\in M(m,n;\mathbb{K})$, $B \in M(m,n;\mathbb{K})$ gilt
%\[ A \circ B = B \circ A \in M(m,n;\mathbb{K}). \]
%\end{itemize}
%\end{remark}


\end{visualizationwrapper}

\end{content}