%$Id:  $
\documentclass{mumie.article}
%$Id$
\begin{metainfo}
  \name{
    \lang{de}{Überblick: Matrizenrechnung}
    \lang{en}{overview: }
  }
  \begin{description} 
 This work is licensed under the Creative Commons License Attribution 4.0 International (CC-BY 4.0)   
 https://creativecommons.org/licenses/by/4.0/legalcode 

    \lang{de}{Beschreibung}
    \lang{en}{}
  \end{description}
  \begin{components}
  \end{components}
  \begin{links}
\link{generic_article}{content/rwth/HM1/T401_Matrizenrechnung/g_art_content_03_transponierte.meta.xml}{content_03_transponierte}
\link{generic_article}{content/rwth/HM1/T401_Matrizenrechnung/g_art_content_02_matrizenmultiplikation.meta.xml}{content_02_matrizenmultiplikation}
\link{generic_article}{content/rwth/HM1/T401_Matrizenrechnung/g_art_content_01_matrizen.meta.xml}{content_01_matrizen}
\end{links}
  \creategeneric
\end{metainfo}
\begin{content}
\begin{block}[annotation]
	Im Ticket-System: \href{https://team.mumie.net/issues/30119}{Ticket 30119}
\end{block}




\begin{block}[annotation]
Im Entstehen: Überblicksseite für Kapitel Matrizenrechnung
\end{block}

\usepackage{mumie.ombplus}
\ombchapter{1}
\lang{de}{\title{Überblick: Matrizenrechnung}}
\lang{en}{\title{}}



\begin{block}[info-box]
\lang{de}{\strong{Inhalt}}
\lang{en}{\strong{Contents}}


\lang{de}{
    \begin{enumerate}%[arabic chapter-overview]
   \item[1.1] \link{content_01_matrizen}{Matrizen}
   \item[1.2] \link{content_02_matrizenmultiplikation}{Matrizen-Multiplikation}
   \item[1.3] \link{content_03_transponierte}{Rechnen mit transponierten Matrizen}
   \end{enumerate}
} %lang

\lang{en}{
    \begin{enumerate}%[arabic chapter-overview]
   \item[1.1] \link{content_01_matrizen}{Matrices}
   \item[1.2] \link{content_02_matrizenmultiplikation}{Matrix multiplication}
   \item[1.3] \link{content_03_transponierte}{Calculating with transposes}
   \end{enumerate}
} %lang


\end{block}

\begin{zusammenfassung}
\lang{de}{
Matrizen über den reellen Zahlen haben wir bereits in Kursteil 1 kennengelernt. 
Nun interessieren wir uns für Matrizen, deren Elemente über anderen Körpern, wie zum Beispiel den rationalen und den komplexen Zahlen, definiert sind. 
Anstelle einer konkreten Spezialisierung betrachten wir die Definitionen, Sätze und Rechenregeln immer allgemein über einen Körper $\K$.
Als Basisoperationen begegnen uns wieder die Matrix-Addition, die Multiplikation mit Skalaren, die Matrix-Vektormultiplikation und die Multiplikation zweier Matrizen.
Zum Schluss wird die transponierte Matrix nun auch über Körpern definiert.
}

\lang{en}{
We already got to know to matrices over the real numbers in Part 1 of the course. 
Now we are interested in matrices with elements defined over any field, for example the rational or the complex numbers. 
Instead of a specific case, we will consider definitions, theorems and calculation rules over a general field $\K$.
We will come across the basic operations matrix addition, multiplication with scalars, matrix-vector multiplication and the multiplicaton of two matrices.
In the end, we will also define the transposition of matrices.
}
\end{zusammenfassung}

\begin{block}[info]\lang{de}{\strong{Lernziele}}
\lang{en}{\strong{Learning Goals}} 
\begin{itemize}[square]
\item \lang{de}{Sie beherrschen die Matrixaddition und -multiplikation über einen beliebigen Körpern.}
      \lang{en}{Being able to add and multiply matrices over any field.}
\item \lang{de}{Sie erkennen, welche Operationen auf konkreten Matrizen erlaubt sind.}
      \lang{en}{Being able to recognize which operations are allowed for specific matrices.}
\item \lang{de}{Sie bestimmen die Transponierte einer Matrix.}
      \lang{en}{Being able to determine the transpose of a matrix.}
\item \lang{de}{Sie wenden die erworbenen Kenntnisse auf praktische Beispiele der Rohstoffkalkulation an.}
      \lang{en}{Being able to apply the aquired knowledge on practical examples like raw material calculation.}
\end{itemize}
\end{block}


\end{content}
