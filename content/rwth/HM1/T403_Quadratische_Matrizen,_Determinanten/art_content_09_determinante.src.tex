%$Id:  $
\documentclass{mumie.article}
%$Id$
\begin{metainfo}
  \name{
    \lang{de}{Determinante}
    \lang{en}{The determinant}
  }
  \begin{description} 
 This work is licensed under the Creative Commons License Attribution 4.0 International (CC-BY 4.0)   
 https://creativecommons.org/licenses/by/4.0/legalcode 

    \lang{de}{Beschreibung}
    \lang{en}{}
  \end{description}
  \begin{components}
    \component{generic_image}{content/rwth/HM1/images/g_tkz_T306_Sarrus.meta.xml}{T306_Sarrus}
    \component{generic_image}{content/rwth/HM1/images/g_img_00_video_button_schwarz-blau.meta.xml}{00_video_button_schwarz-blau}
\end{components}
  \begin{links}
    \link{generic_article}{content/rwth/HM1/T401_Matrizenrechnung/g_art_content_03_transponierte.meta.xml}{content_03_transponierte}
    \link{generic_article}{content/rwth/HM1/T403_Quadratische_Matrizen,_Determinanten/g_art_content_09_determinante.meta.xml}{content_09_determinante}
    \link{generic_article}{content/rwth/HM1/T402_Lineare_Gleichungssysteme/g_art_content_06_umformungen_rang.meta.xml}{umformungen}
    \link{generic_article}{content/rwth/HM1/T403_Quadratische_Matrizen,_Determinanten/g_art_content_08_inverse_matrix.meta.xml}{inverse-matrix}
  \end{links}
  \creategeneric
\end{metainfo}
\begin{content}
\usepackage{mumie.ombplus}
\ombchapter{3}
\ombarticle{3}
\usepackage{mumie.genericvisualization}

\begin{visualizationwrapper}

\title{\lang{de}{Determinante} \lang{en}{The determinant}}

\begin{block}[annotation]
 
  
\end{block}
\begin{block}[annotation]
  Im Ticket-System: \href{http://team.mumie.net/issues/11284}{Ticket 11284}\\
\end{block}

\begin{block}[info-box]
\tableofcontents
\end{block}

\lang{de}{
Die Determinante einer quadratischen Matrix $A\in M(n;\K)$ gilt als Kenngröße für die Invertierbarkeit von $A$. 
Es gibt mehrere äquivalente Möglichkeiten, die Determinante zu definieren. Hier soll eine Definition verwendet werden, die zugleich
eine Möglichkeit der Berechnung liefert.}
\lang{en}{
The determinant of a square matrix $A\in M(n;\K)$ is a characteristic for the invertibility of $A$. 
There are several equivalent ways of defining the determinant. Here we will choose a definition, that gives us a way of determining the determinant right away.}

\section{\lang{de}{Definition der Determinante} \lang{en}{Definition of the determinant}}

\begin{definition}\label{def:determinante}
\lang{de}{
Die \notion{Determinante} einer Matrix $A\in M(n;\K)$, bezeichnet mit $\det(A)$, ist ein Element des Körpers $\K$, welches durch folgende Eigenschaften charakterisiert ist:}
\lang{en}{
The \notion{determinant} of a matrix $A\in M(n;\K)$, denoted with $\det(A)$, is an element of the field $\K$ and characterised by the following properties:}
\begin{enumerate}
\item \lang{de}{Ist $A \in M(n;\K) $ eine obere Dreiecksmatrix oder eine untere Dreiecksmatrix mit Diagonaleintr"agen $a_{11}, a_{22}, \ldots, a_{nn}$, so ist
\[ \det(A)=a_{11}\cdot a_{22} \cdots a_{nn} \]
das Produkt der Diagonalelemente.}
\lang{en}{Let $A \in M(n;\K) $ be a lower or upper triangular matrix with diagonal entries $a_{11}, a_{22}, \ldots, a_{nn}$, the determinant is the product
of the diagonal elements:
\[ \det(A)=a_{11}\cdot a_{22} \cdots a_{nn} \]}
 \item \lang{de}{Erh\"alt man $B \in M(n;\K)$ aus $A$ durch Addition des $r$-fachen der $j$-ten Zeile zur $i$-ten Zeile ($i \neq j$), so gilt f\"ur die Determinanten:
       \[
       \det(A)=\det(B).
       \]}
       \lang{en}{If matrix $B \in M(n;\K)$ is the result of adding in $A$ $r$-times $j$th row to the $i$th row ($i \neq j$), we have for the determinants:
       \[
       \det(A)=\det(B).
       \]}
 \item \lang{de}{Erh\"alt man $B$ aus $A$ durch Multiplikation der $i$-ten Zeile mit 
 $r \in \K \setminus \{ 0 \}$, so gilt
       \[
       \det(A) = \frac{1}{r} \cdot  \det(B).
       \]}
       \lang{en}{If $B$ is the result of multiplying the $i$th row of $A$ with 
 $r \in \K \setminus \{ 0 \}$, we have
       \[
       \det(A) = \frac{1}{r} \cdot  \det(B).
       \]}
 \item \lang{de}{Erh\"alt man $B$ aus $A$ durch Vertauschen der $i$-ten und $j$-ten Zeile, so gilt
       \[
       \det(A)=- \det(B).
       \]}
       \lang{en}{If $B$ is the result of swapping the $i$th and $j$th row if $A$, we have
       \[
       \det(A)=- \det(B).
       \]}
\end{enumerate}

\lang{de}{
\floatright{\href{https://api.stream24.net/vod/getVideo.php?id=10962-2-11380&mode=iframe&speed=true}{\image[75]{00_video_button_schwarz-blau}}}}

\end{definition}

\begin{quickcheck}
\type{input.number}
  \field{complex}
  \displayprecision{3}
  \correctorprecision{4}
 
  \begin{variables}
   \function[calculate]{a}{6}
   \function[calculate]{b}{-i}
  \end{variables}
\text{\lang{de}{Bestimmen Sie die Determinanten von $A=\begin{pmatrix}1&0&0\\2&2&0\\1&0&3\end{pmatrix}$ und 
$B=\begin{pmatrix}0&1\\i&0\end{pmatrix}$.

Antwort: $\det(A)=$\ansref und $\det(B)=$\ansref.}
\lang{en}{Determine the determinants of $A=\begin{pmatrix}1&0&0\\2&2&0\\1&0&3\end{pmatrix}$ and 
$B=\begin{pmatrix}0&1\\i&0\end{pmatrix}$.

Answer: $\det(A)=$\ansref and $\det(B)=$\ansref.}
}
 \begin{answer}
    \solution{a}
  \end{answer}
   \begin{answer}
    \solution{b}
   \end{answer}
   \explanation{\lang{de}{$A$ ist eine untere Dreiecksmatrix. 
   Die Determinante berechnet sich also als Produkt der Diagonalglieder $\det(A)=1\cdot 2\cdot 3=6$. 
   Vertauscht man in $B$ die zwei Zeilen, so erhält man die Diagonalmatrix $\begin{pmatrix}1&0\\0&i\end{pmatrix}$,
   deren Determinante gleich $i$ ist. 
   Die Determinante von $B$ erhält noch das Vorzeichen, das durch die Vertauschung aufgenommen wird: $\det(B)=-i$.}
   \lang{en}{$A$ is a lower triangular matrix.
   The determinant is the product of the diagonal entries $\det(A)=1\cdot 2\cdot 3=6$. 
   By swapping the two rows of $B$, we get the diagonal matrix $\begin{pmatrix}1&0\\0&i\end{pmatrix}$,
   which has the determinant $i$. 
   The determinant of $B$ needs the sign, which results from swapping the rows: $\det(B)=-i$.}}
\end{quickcheck}

\begin{remark}
\begin{itemize}
\item \lang{de}{Da man mittels Zeilenumformungen jede quadratische Matrix auf obere Dreiecksgestalt bringen kann (z.B. mit dem Gauß-Verfahren), ist somit für jede Matrix $A$ ein Wert $\det(A)$ definiert. 
Auch zeigt diese Definition, wie man die Determinante berechnet, nämlich durch Zeilenumformungen.}
\lang{en}{Every square matrix can be transformed into an upper triangular matrix by using  Gaussian elimination. Therefore $\det(A)$ is defined for every square matrix 
$A$. Also, this definition shows how to calculate the determinant, namely by row transformations.}
\item \lang{de}{Genau genommen müsste man noch nachprüfen, dass dieses Verfahren auch tatsächlich für jede Matrix $A$ einen eindeutigen Wert $\det(A)$ liefert,
und nicht davon abhängt, über welche Zeilenumformungen man zu einer oberen Dreiecksmatrix kommt. Darauf soll hier aber verzichtet werden.}
\lang{en}{Strictly speaking, we would also need to check, that the value $\det(A)$ is unambigous for every matrix $A$ and not dependend on the
performed row transformations. This can be omitted here.}
\end{itemize}
\end{remark}

\begin{example}
\begin{enumerate}
\item \lang{de}{Die Einheitsmatrix $E_n$ hat als Determinante
\[ \det(E_n)=1\cdot 1\cdots 1=1. \]}
\lang{en}{The indentity matrix $I_n$ has the determinant
\[ \det(I_n)=1\cdot 1\cdots 1=1. \]}
\item \lang{de}{Wir betrachten die reelle $(3\times 3)$-Matrix 
\[ A= \begin{pmatrix}
-2 & 4 & 6 \\ 1 & -1 & 1 \\ 2 & -3 & -4 \end{pmatrix}. \]
Mit Gauß-Verfahren erhalten wir die Umformungen
\begin{eqnarray*}
&& \begin{pmatrix} -2 & 4 & 6 \\ 1 & -1 & 1  \\ 2 & -3 & -4  \end{pmatrix} 
\begin{matrix} / \cdot (-1/2) \\ \phantom{1}\\ \phantom{1} \end{matrix} \rightsquigarrow 
\begin{pmatrix} 1 & -2 & -3  \\ 1 & -1 & 1  \\ 2 & -3 & -4  \end{pmatrix} 
\begin{matrix} \phantom{1} \\ / - 1\cdot \text{(I)}\\ / -2\cdot \text{(I)} \end{matrix} \rightsquigarrow  \\ &&\\
& \rightsquigarrow & \begin{pmatrix} 1 & -2 & -3  \\ 0 & 1 & 4  \\ 0 & 1 & 2 \end{pmatrix} 
\begin{matrix} \phantom{1} \\  \phantom{1}\\ / -1\cdot \text{(II)} \end{matrix} \rightsquigarrow 
\begin{pmatrix} 1 & -2 & -3  \\ 0 & 1 & 4  \\ 0 & 0 & -2 \end{pmatrix} .
\end{eqnarray*}
Damit gilt unter Verwendung der Eigenschaften aus Definition \ref{def:determinante}
\begin{eqnarray*} \det(A) 
&=& (-2)\cdot \det\Big(\left( \begin{smallmatrix} 1 & -2 & -3  \\ 1 & -1 & 1  \\ 2 & -3 & -4  \end{smallmatrix}\right)\Big) 
=-2\cdot \det\Big(\left(  \begin{smallmatrix} 1 & -2 & -3  \\ 0 & 1 & 4  \\ 0 & 1 & 2 \end{smallmatrix} \right)\Big)\\
&=& -2\cdot \det\Big(\left(  \begin{smallmatrix} 1 & -2 & -3  \\ 0 & 1 & 4  \\ 0 & 0 & -2 \end{smallmatrix} \right)\Big)
=-2\cdot \big( 1\cdot 1\cdot (-2)\big)=4
\end{eqnarray*}}

\lang{en}{We consider the real $(3\times 3)$-matrix 
\[ A= \begin{pmatrix}
-2 & 4 & 6 \\ 1 & -1 & 1 \\ 2 & -3 & -4 \end{pmatrix}. \]
We receive the following transformations by using Gaussian elimination
\begin{eqnarray*}
&& \begin{pmatrix} -2 & 4 & 6 \\ 1 & -1 & 1  \\ 2 & -3 & -4  \end{pmatrix} 
\begin{matrix} / \cdot (-1/2) \\ \phantom{1}\\ \phantom{1} \end{matrix} \rightsquigarrow 
\begin{pmatrix} 1 & -2 & -3  \\ 1 & -1 & 1  \\ 2 & -3 & -4  \end{pmatrix} 
\begin{matrix} \phantom{1} \\ / - 1\cdot \text{(I)}\\ / -2\cdot \text{(I)} \end{matrix} \rightsquigarrow  \\ &&\\
& \rightsquigarrow & \begin{pmatrix} 1 & -2 & -3  \\ 0 & 1 & 4  \\ 0 & 1 & 2 \end{pmatrix} 
\begin{matrix} \phantom{1} \\  \phantom{1}\\ / -1\cdot \text{(II)} \end{matrix} \rightsquigarrow 
\begin{pmatrix} 1 & -2 & -3  \\ 0 & 1 & 4  \\ 0 & 0 & -2 \end{pmatrix} .
\end{eqnarray*}
Then we have using the properties from definition \ref{def:determinante}
\begin{eqnarray*} \det(A) 
&=& (-2)\cdot \det\Big(\left( \begin{smallmatrix} 1 & -2 & -3  \\ 1 & -1 & 1  \\ 2 & -3 & -4  \end{smallmatrix}\right)\Big) 
=-2\cdot \det\Big(\left(  \begin{smallmatrix} 1 & -2 & -3  \\ 0 & 1 & 4  \\ 0 & 1 & 2 \end{smallmatrix} \right)\Big)\\
&=& -2\cdot \det\Big(\left(  \begin{smallmatrix} 1 & -2 & -3  \\ 0 & 1 & 4  \\ 0 & 0 & -2 \end{smallmatrix} \right)\Big)
=-2\cdot \big( 1\cdot 1\cdot (-2)\big)=4.
\end{eqnarray*}}
\end{enumerate}
\end{example}

\begin{example}
\lang{de}{
Die Determinanten der \ref[umformungen][Elementarmatrizen]{rule:elementarmatrizen} lassen sich leicht berechnen:}
\lang{en}{
The determinant of \ref[umformungen][elementary matrices]{rule:elementarmatrizen} can be calculated easily:}
\begin{incremental}
\step
%\begin{itemize}
% \item 
\lang{de}{
Die Matrix $A_{ij}(r)$, $r \in \K$, $i\neq j$, deren Diagonalelemente alle $1$ sind, deren Eintrag an der Stelle $(i,j)$ gleich $r$ ist und die sonst nur Nulleinträge hat, ist insbesondere eine untere oder obere Dreiecksmatrix. Daher ist
       \[
       \det(A_{ij}(r))=1\cdot 1\cdots 1=1.
       \] }
\lang{en}{
The diagonal entries of matrix $A_{ij}(r)$, $r \in \K$, $i\neq j$ are all equal to $1$, the entry $a_{,j}$ is equal to $r$ and the other entries are zero-entries. Then $A$ is a lower or
upper triangular matrix with determinant
       \[
       \det(A_{ij}(r))=1\cdot 1\cdots 1=1.
       \] }
\step
% \item 
\lang{de}{
Die Matrix $M_{i}(r)$, $r \in \R \setminus \lbrace 0 \rbrace $, $i\in\{ 1,\ldots,n\}$,  deren $i$-tes Diagonalelement gleich $r$ ist, alle anderen Diagonalelemente gleich $1$, und die sonst nur Nulleinträge hat, ist sogar eine Diagonalmatrix. Daher ist:
       \[
       \det(M_{i}(r))=1\cdots 1\cdot r\cdot 1\cdots 1=r.
       \]}
\lang{en}{ The matrix$M_{i}(r)$, $r \in \R \setminus \lbrace 0 \rbrace $, $i\in\{ 1,\ldots,n\}$ is a diagonal matrix, because all diagonal elements besides
the $i$th entry are equal to $1$. The $i$th entry is equal to $r$. Therefore the determinant is
       \[
       \det(M_{i}(r))=1\cdots 1\cdot r\cdot 1\cdots 1=r.
       \]}
\step
% \item 
\lang{de}{
Die Matrix $V_{ij}$ unterscheidet sich von der Einheitsmatrix $E_n$ nur dadurch, dass die $i$-te und die $j$-te Zeile vertauscht sind. Daher ist 
       \[
       \det(V_{ij})=-\det(E_n)=-1.
       \]}
       \lang{en}{
The matrix $V_{ij}$ differs from the identity matrix $I_n$ only by swapped $i$th and $j$th rows. Therefore the determinant is 
       \[
       \det(V_{ij})=-\det(I_n)=-1.
       \]}
\end{incremental}
%\end{itemize}

\end{example}

\lang{de}{
Ein wichtiger Grund, wofür Determinanten verwendet werden, ist der folgende.}
\lang{en}{
The following theorem is another important application of determinants.}

\begin{theorem}
\lang{de}{
Eine quadratische Matrix $A\in M(n;\K)$ ist genau dann invertierbar, wenn ihre Determinante von $0$ verschieden ist.}
\lang{en}{
A square matrix $A\in M(n;\K)$ is invertible if and only if the determinant is unequal to $0$.}
\end{theorem}

\begin{proof*}[\lang{de}{Beweis des Satzes} \lang{en}{Proof of the theorem}]
\begin{showhide}
\lang{de}{
Durch Anwenden elementarer Zeilenumformungen kann sich zwar der Wert der Determinante ändern, jedoch nicht die Eigenschaft, ob sie $0$ ist oder nicht.
Denn dabei wird nur mit Elementen ungleich $0$ multipliziert.
Bezeichnen wir mit $S$ die reduzierte Zeilenstufenform von $A$, dann gilt also:
\begin{eqnarray*}
\det(A)=0 &\Leftrightarrow & \det(S)=0 \\
&\Leftrightarrow & S\text{ besitzt weniger als $n$ Stufen} \quad (\text{da $S$ obere Dreiecksmatrix})\\
&\Leftrightarrow & S\text{ hat Rang kleiner als }n\quad (\text{da Rang($S$)=Anzahl der Stufen})\\
&\Leftrightarrow & A\text{ hat Rang kleiner als }n\quad (\text{da Rang($A$)=Rang($S$)})\\
&\Leftrightarrow & A\text{ ist nicht invertierbar}.
\end{eqnarray*}}

\lang{en}{
Elementary row transformations may change the value of the determinant, but not the property whether it is equal to $0$ or not,
because row transformations only change the determinant by a factor unequal to zero.
If we denote the reduced row echelon form of $A$ with $S$, we have:
\begin{eqnarray*}
\det(A)=0 &\Leftrightarrow & \det(S)=0 \\
&\Leftrightarrow & S\text{ has less than $n$ echelons} \quad (\text{because $S$ is upper triangular matrix})\\
&\Leftrightarrow & \text{rank of $S$ is less than }n\quad (\text{because rank($S$)=number of echelons})\\
&\Leftrightarrow & \text{ rank of $A$ is less than}n\quad (\text{because rank($A$)=rank($S$)})\\
&\Leftrightarrow & A\text{ is not invertible}.
\end{eqnarray*}}
\end{showhide}
\end{proof*}

%\floatright{\href{https://api.stream24.net/vod/getVideo.php?id=10962-2-10879&mode=iframe&speed=true}{\image[75]{00_video_button_schwarz-blau}}}\\

\lang{de}{
Das folgende Video zeigt zunächst den Beweis zu Satz 3.3.5. Im Anschluss wird ein Beispiel vorgerechnet:
\floatright{\href{https://api.stream24.net/vod/getVideo.php?id=10962-2-11381&mode=iframe&speed=true}{\image[75]{00_video_button_schwarz-blau}}}}


\section{\lang{de}{Determinanten kleiner Matrizen} \lang{en}{Determinants of small matrices}}\label{sec:determinanten-kleine-matrizen}

\lang{de}{
Für Matrizen kleiner Größe gibt es einfache Formeln, die Determinante zu berechnen.}
\lang{en}{There are easy formulas to calculate the determinant of small matrices.}

\begin{rule}
\begin{enumerate}
\item \lang{de}{Für $(1\times 1)$-Matrizen $A=(a)\in M(1;\K)$ gilt nach Definition $\det(A)=a\in \K$.}
\lang{en}{For $(1\times 1)$-matrices $A=(a)\in M(1;\K)$ we have, according to the definition, $\det(A)=a\in \K$.}

\item \lang{de}{Für $(2\times 2)$-Matrizen $A= \begin{pmatrix}
a & b \\ c& d \end{pmatrix} \in M(2;\K) $ ist
\[ \det(A)=ad-bc.\]}
\lang{en}{For $(2\times 2)$-matrices $A= \begin{pmatrix}
a & b \\ c& d \end{pmatrix} \in M(2;\K) $ is
\[ \det(A)=ad-bc.\]}

\item \lang{de}{Für $(3\times 3)$-Matrizen $A=\left( \begin{smallmatrix}
a_{11} & a_{12} &a_{13} \\ a_{21}& a_{22} &a_{23} \\ a_{31}& a_{32}& a_{33}
\end{smallmatrix}\right)$ ist
\begin{eqnarray*}
\det(A) &=& \ a_{11} a_{22} a_{33} + a_{12} a_{23} a_{31} + a_{13} a_{21}a_{32} \\
                   & & - a_{13} a_{22} a_{31} - a_{11} a_{23} a_{32} - a_{12} a_{21} a_{33}.
\end{eqnarray*}}
\lang{en}{For $(3\times 3)$-matrices $A=\left( \begin{smallmatrix}
a_{11} & a_{12} &a_{13} \\ a_{21}& a_{22} &a_{23} \\ a_{31}& a_{32}& a_{33}
\end{smallmatrix}\right)$ is
\begin{eqnarray*}
\det(A) &=& \ a_{11} a_{22} a_{33} + a_{12} a_{23} a_{31} + a_{13} a_{21}a_{32} \\
                   & & - a_{13} a_{22} a_{31} - a_{11} a_{23} a_{32} - a_{12} a_{21} a_{33}.
\end{eqnarray*}}
\end{enumerate}
\end{rule}

\begin{remark}
\begin{enumerate}
\item \lang{de}{Den Ausdruck $ad-bc$ für $(2\times 2)$-Matrizen hatten wir im 
\ref[inverse-matrix][vorigen Abschnitt]{rule:inverse-2x2} bei der Berechnung der inversen Matrix als Kriterium 
für die Invertierbarkeit entdeckt.} 
\lang{en}{We already discovered the term $ad-bc$ for $(2\times 2)$-matrices in the 
\ref[inverse-matrix][previous section]{rule:inverse-2x2}, while calculating the inverse matrix
as a criteria for invertibility.}
\item \lang{de}{Die Formel für $(3\times 3)$-Ma
trizen wird auch Regel von Sarrus genannt. Sie lässt sich mit dem folgenden Schema 
leicht merken:
\begin{center}
\image{T306_Sarrus}
\end{center}}

\lang{en}{The formula for $(3\times 3)$-matrices is also called rule of Sarrus. It can easily be remembered with the following scheme:
\begin{center}
\image{T306_Sarrus}
\end{center}}

\lang{de}{
Erg\"anze die ersten beiden Spalten der Matrix $A$ noch einmal zur Matrix $A$ als 4. und 5. Spalte. Dann addiere man die 
Produkte der Diagonalen von links oben nach rechts unten und subtrahiere
die Produkte der Diagonalen von links unten nach rechts oben.}
\lang{en}{
We complement the first two columns of $A$ behind it as a $4$th and $5$th column. We then add the products of the diagonals
from top left to bottom right and substract the products of the 
diagonals vom bottom left to top right.}
\end{enumerate}
\end{remark}

%%%%%%%%%%%%%%%%%%%%%%%%%%%%%%%%%%%%%%%%%%%%%%%%%%%%%%%%%%%%%%%%%%%%%%%%%%%%%%
%%%%%%%%%%%%%%%%%%%%%%%%%%%%%%%% HIER GEHT ES WEITER %%%%%%%%%%%%%%%%%%%%%%%%%
%%%%%%%%%%%%%%%%%%%%%%%%%%%%%%%%%%%%%%%%%%%%%%%%%%%%%%%%%%%%%%%%%%%%%%%%%%%%%%
\begin{example}
\begin{tabs*}[\initialtab{0}]
\tab{\lang{de}{Beispiel 1} \lang{en}{1. Example}}
\lang{de}{
Wir betrachten wieder die reelle $(3\times 3)$-Matrix 
\[ A= \begin{pmatrix}
-2 & 4 & 6 \\ 1 & -1 & 1 \\ 2 & -3 & -4 \end{pmatrix}. \]
Schreibt man die ersten zwei Spalten nochmal daneben, erhält man das Schema
\[ \begin{matrix}
-2 & 4 & 6 &: &-2 & 4\\ 1 & -1 & 1&: &1 & -1 \\ 2 & -3 & -4&: &2 & -3
\end{matrix}. \]
Damit berechnen wir also:
\begin{eqnarray*}
 \det(A)&=&(-2)\cdot (-1)\cdot (-4)+4\cdot 1\cdot 2
+ 6\cdot 1\cdot (-3) \\ 
&& - 2\cdot (-1)\cdot 6- (-3)\cdot 1\cdot (-2)
- (-4)\cdot 1\cdot 4 \\
&=& -8+8-18+12-6+16=4
\end{eqnarray*}}
\lang{en}{
We consider again the real $(3\times 3)$-matrix 
\[ A= \begin{pmatrix}
-2 & 4 & 6 \\ 1 & -1 & 1 \\ 2 & -3 & -4 \end{pmatrix}. \]
We write the first two columns behind the matrix and receive the scheme
\[ \begin{matrix}
-2 & 4 & 6 &: &-2 & 4\\ 1 & -1 & 1&: &1 & -1 \\ 2 & -3 & -4&: &2 & -3
\end{matrix}. \]
Using this we calculate:
\begin{eqnarray*}
 \det(A)&=&(-2)\cdot (-1)\cdot (-4)+4\cdot 1\cdot 2
+ 6\cdot 1\cdot (-3) \\ 
&& - 2\cdot (-1)\cdot 6- (-3)\cdot 1\cdot (-2)
- (-4)\cdot 1\cdot 4 \\
&=& -8+8-18+12-6+16=4
\end{eqnarray*}}

\tab{\lang{de}{Beispiel 2} \lang{en}{2. Example}}
\lang{de}{
Wir betrachten die Matrix
\[ B= \begin{pmatrix}
-1 & i & 2+i \\ i & -1 & 0 \\ 3-i & -1+i & 5i \end{pmatrix}\in M(3;\C). \]
Schreibt man die ersten zwei Spalten nochmal daneben, erhält man das Schema
\[ \begin{matrix}
-1 & i & 2+i &: &-1 & i\\ i & -1 & 0&: &i & -1 \\ 3-i & -1+i & 5i&: &3-i & -1+i
\end{matrix}. \]
Somit berechnen wir:
\begin{eqnarray*}
 \det(B)&=&(-1)\cdot(-1)\cdot5i+i\cdot 0\cdot(3-i)+(2+i)\cdot i\cdot (-1+i) \\ 
&& - (3-i)\cdot (-1)\cdot (2+i)- (-1+i)\cdot 0\cdot (-1) - (5i)\cdot i\cdot i \\
&=& 5i+0+i(-3+i)+(7+i)-0+5i=6+8i.
\end{eqnarray*}}
\lang{en}{
We consider the matrix
\[ B= \begin{pmatrix}
-1 & i & 2+i \\ i & -1 & 0 \\ 3-i & -1+i & 5i \end{pmatrix}\in M(3;\C). \]
We write the first two columns behind the matrix and receive the scheme
\[ \begin{matrix}
-1 & i & 2+i &: &-1 & i\\ i & -1 & 0&: &i & -1 \\ 3-i & -1+i & 5i&: &3-i & -1+i
\end{matrix}. \]
With that we calculate:
\begin{eqnarray*}
 \det(B)&=&(-1)\cdot(-1)\cdot5i+i\cdot 0\cdot(3-i)+(2+i)\cdot i\cdot (-1+i) \\ 
&& - (3-i)\cdot (-1)\cdot (2+i)- (-1+i)\cdot 0\cdot (-1) - (5i)\cdot i\cdot i \\
&=& 5i+0+i(-3+i)+(7+i)-0+5i=6+8i.
\end{eqnarray*}}
\end{tabs*}
\end{example}


\begin{block}[warning]
\lang{de}{
Die Regel von Sarrus gilt ausschließlich für $(3\times 3)$-Matrizen. Für größere Matrizen ist
eine derartige Regel im Allgemeinen falsch!
\\
Für $(2\times 2)$-Matrizen sieht man das sofort ein: Eine übertragene Sarrus-Regel würde hier
den Term $ad+bc-bc-ad=0$ liefern.}
\lang{en}{
The rule of Sarrus applies exclusively for $(3\times 3$)-matrices. 
For larger matrices a rule like that is generally wrong!
\\
This is easy to understand, when we look at $(2\times 2)$-matrices. A transfered rule of Sarrus would result in the term $ad+bc-bc-ad=0$, which is not 
the determinant of a $(2\times 2)$-matrix.}
\end{block}



\section{\lang{de}{Rechenregeln für Determinanten} \lang{en}{Calculating rules for determinants}}\label{sec:determinante-rechenregeln}
\lang{de}{
Auch Determinanten von quadratischen Matrizen beliebigen Ausmaßes sind gut berechenbar. Wir geben hier nur zwei Möglichkeiten an.}
\lang{en}{
We can  easily calculate the determinants for square matrices of any dimension. Here we will discuss only two options.
}

\begin{rule}[\lang{de}{Berechnung von allgemeinen Determinanten} \lang{en}{Calculating general determinants}]\label{rule:Laplace}
\lang{de}{Die Determinante einer Matrix $A=(a_{ij})_{ij}\in M(n;\K)$ kann wie folgt berechnet werden.}
\lang{en}{The determinant of a matrix $A=(a_{ij})_{ij}\in M(n;\K)$ can be calculated as follows.}
\begin{itemize}
\item[(i)] \lang{de}{Man wendet die \ref[content_09_determinante][Definition der Determinante]{def:determinante} an. 
Durch Multiplikation mit Elementarmatrizen bringt man $A$ also auf eine Stufenform $S$, die eine (obere) Dreiecksmatrix ist.
Deren Determinante kann man daher direkt ablesen, die der Elementarmatrizen kennt man.
Bezeichen $M_1\ldots,M_r$ die verwendeten Elementarmatrizen, dann gilt
\[\det(A)=\det(M_1)^{-1}\cdot\ldots\cdot\det(M_r)^{-1}\cdot\det(S).\]}
\lang{en}{We utilise the \ref[content_09_determinante][definition of the determinant]{def:determinante}.
By multiplying with elementary matrices we transform $A$ into row echelon form $S$, which is a (upper) triangular matrix.
Its matrix can be read right away, since we know the determinants of the elementary matrices.
We denote the used elementary matrices with $M_1\ldots,M_r$, then we have:
\[\det(A)=\det(M_1)^{-1}\cdot\ldots\cdot\det(M_r)^{-1}\cdot\det(S).\]}

\item[(ii)]
\lang{de}{
Es sei $A_{ij}\in M(n-1;\K)$ die Matrix, die aus $A$ entsteht, wenn man die $i$-te Zeile und $j$-te Spalte streicht.
\notion{Laplace-Entwicklung nach der $i$-ten Zeile:}
\[\det(A)=\sum_{j=1}^n(-1)^{i+j}a_{ij}\cdot\det(A_{ij}).\]
\notion{Laplace-Entwicklung nach der $j$-ten Spalte:}
\[\det(A)=\sum_{i=1}^n(-1)^{i+j}a_{ij}\cdot\det(A_{ij}).\]}
\lang{en}{
Let $A_{ij}\in M(n-1;\K)$ be the matrix, that results from $A$, after removing the $i$th row and the $j$th column.
\notion{Laplace-expansion along with the $i$th row:}
\[\det(A)=\sum_{j=1}^n(-1)^{i+j}a_{ij}\cdot\det(A_{ij}).\]
\notion{Laplace-expansion along with the $i$th column:}
\[\det(A)=\sum_{i=1}^n(-1)^{i+j}a_{ij}\cdot\det(A_{ij}).\]}
\end{itemize}
\end{rule}

\lang{de}{
Im nachfolgenden Video wird die Laplace-Entwicklung erklärt und durch ein Beispiel veranschaulicht:

\floatright{\href{https://api.stream24.net/vod/getVideo.php?id=10962-2-10882&mode=iframe&speed=true}{\image[75]{00_video_button_schwarz-blau}}}}\\


\lang{en}{
\begin{example}
Given is the matrix $A=\begin{pmatrix}2&0&0&1\\1&3&0&1\\-0&7&2&1\\0&0&0&5\end{pmatrix}$, for which we want to calculate the determinant.\\
Since the forth row has only one non-zero entry, we can utilise Laplace-expansion along with the $4$th row:
\[ \det(A)=\sum_{j=1}^{4} (-1)^{4+j}\cdot \underbrace{a_{4j}}_{=0\text{ except for }j=4}\cdot \det(A_{4j})=(-1)^{4+4}\cdot 5\cdot
\det(\begin{pmatrix}2&0&0\\1&3&0\\0&7&2\end{pmatrix}) \underbrace{=}_{\text{triangular matrix}}5\cdot 2\cdot 3 \cdot 2=60\].
We may also expand along with the third column, since this has also only one non-zero entry. With the Laplace-expansion we get:
\[\det(A)=\sum_{i=1}^{4}(-1)^{i+3}\cdot \underbrace{a_{i3}}_{\neq 0{\text{ only for }i=3}}\cdot 
\det(A_{i3})=2\cdot \det(\begin{pmatrix}2&0&1\\1&3&1\\0&0&5\end{pmatrix}) \underbrace{=}_{\text{Laplace for third row}} 2\cdot
(-1)^{3+3}\cdot 5\cdot \det(\begin{pmatrix} 2&0\\1&3\end{pmatrix})=2\cdot5\cdot6=60\]
So there are several options for the Laplace-expansion, although some take longer than others.
\end{example}
}


\begin{quickcheck}
\type{input.number}
  \field{rational}
  \displayprecision{3}
  \correctorprecision{4}
 
  \begin{variables}
   \function{a}{-6}
   \end{variables}
%\text{Es sei $A=\begin{pmatrix}1&2&3\\4&5&6\\-7&8&9\end{pmatrix}$. Dann ist $\det(A_{21})=$\ansref.}
\text{\lang{de}{Es sei $A=\begin{pmatrix}1&2&3\\4&5+3i&6\\-7i&8&9\end{pmatrix}$. Dann ist $\det(A_{21})=$\ansref.}
\lang{en}{Given $A=\begin{pmatrix}1&2&3\\4&5+3i&6\\-7i&8&9\end{pmatrix}$. Then we have $\det(A_{21})=$\ansref.}}
 \begin{answer}
    \solution{a}
  \end{answer}
   \explanation{\lang{de}{$A_{21}$ entsteht aus $A$ durch streichen der zweiten Zeile und ersten Spalte. 
   Also ist $A_{21}=\begin{pmatrix}2&3\\8&9\end{pmatrix}$ und $\det(A_{21})=2\cdot 9-8\cdot 3=-6$.}
   \lang{en}{$A_{21}$ is the result of removing the second row and first column in $A$. So we have $A_{21}=\begin{pmatrix}2&3\\8&9\end{pmatrix}$ und $\det(A_{21})=2\cdot 9-8\cdot 3=-6$.}}
\end{quickcheck}

\begin{remark}
\begin{itemize}
\item
\lang{de}{
Bei der Laplace-Entwicklung wird also die Berechnung der Determinante zurückgeführt auf die Berechnung von
$n$  Determinanten $\det(A_{ij})$ kleinerer Größe, sogenannten Unterdeterminanten von $A$.}
\lang{en}{
The Laplace-expansion reduces the calculation of the determinant to the calcuation of $n$ determinants $\det(A_{ij})$ with
a smaller size. They are called minors of $A$ and are the determinant of a smaller square matrix created from $A$.}

\item 
\lang{de}{Die Laplace-Entwicklung eignet sich besonders gut bei dünnbesetzten Zeilen oder Spalten, also solchen bei denen viele Einträge gleich null sind.
Ist nämlich ein $a_{ij}=0$, dann taucht der entsprechende Term in der Summe nicht auf. Man braucht dann $\det(A_{ij})$ gar nicht erst zu berechnen, was den Rechenaufwand erheblich reduziert.}
\lang{en}{The Laplace-expansion is especially useful, if the matrix has a lot of zero-entries. Because the term of a zero-entry, i.e. $a_{ij}=0$, will not appear in the sum and therefore
we do not need to calculate $\det(A_{ij})$. This reduces the calculation effort.}
\item 
\lang{de}{Wenn die Matrix sehr voll besetzt ist, dann ist Methode (i) meist schneller. Man braucht die Matrix dazu nicht einmal in reduzierte Stufenform zu bringen, irgendeine Dreiecksform reicht aus.}
\lang{en}{For a matrix with just a few or no zero-entries it is faster to use method (i). Keep in mind, that any triangular matrix is fine - you do not need to transform it into reduced row echelon form.}
\item
\lang{de}{
Es gibt allgemeinere Versionen der Laplace-Entwicklung und noch viele andere  Möglichkeiten, Determinanten zu berechnen.
Der tiefere mathematische Grund dafür liegt in den faszinierenden Eigenschaften, die die Determinante hat.
Deren Diskussion geht aber weit über das hinaus, was in der praktischen Anwendung benötigt wird. 
Deshalb verzichten wir hier ebenso darauf, wie auf einen Beweis der Laplace-Entwicklung nach der $i$-ten Zeile oder $j$-ten Spalte.}
\lang{en}{
There are more general versions of the Laplace-expansion and even more other methods to calculate determinants. Their discussion is not needed for practical
usage, which is why is will not be done here. Furthermore we refrain from proofing the Laplace-expansion along with the $i$th row or $j$ column.
}
\end{itemize}
\end{remark}


\begin{example}
\begin{incremental}[\initialsteps{1}]
\step \lang{de}{Wir berechnen die Determinante von $A=\begin{pmatrix} -1&0&0&2\\2&3&0&1\\1&5&0&0\\0&0&3&2\end{pmatrix}$
mit Hilfe der Laplace-Entwicklung. }
\lang{en}{
We calculate the determinant of $A=\begin{pmatrix} -1&0&0&2\\2&3&0&1\\1&5&0&0\\0&0&3&2\end{pmatrix}$ using the Laplace-expansion. }

\step
\lang{de}{
Am dünnbesetztesten ist hier die dritte Spalte, also entwickeln wir nach dieser:
\[\det(A)=\sum_{i=1}^4(-1)^{i+3}a_{i3}\det(A_{i3}),\]
worin alle $a_{i3}$ bis auf $a_{43}$ gleich null sind.
Es ist also
\[\det(A)=(-1)^{4+3}\cdot 3\cdot\det(A_{43})=(-3)\cdot\det\Big(\left(\begin{smallmatrix}
-1&0&2\\2&3&1\\1&5&0\end{smallmatrix}\right)\Big).\]}
\lang{en}{
We expand along the third column, because it has the most zero-entries:
\[\det(A)=\sum_{i=1}^4(-1)^{i+3}a_{i3}\det(A_{i3}),\]
in which all $a_{i3}$ except $a_{43}$ are equal to zero.
So we have
\[\det(A)=(-1)^{4+3}\cdot 3\cdot\det(A_{43})=(-3)\cdot\det\Big(\left(\begin{smallmatrix}
-1&0&2\\2&3&1\\1&5&0\end{smallmatrix}\right)\Big).\]}

\step
\lang{de}{
Diese Determinante könnten wir mit der Regel von Sarrus berechnen, aber wir entwickeln sie hier weiter nach der ersten Zeile:
 \begin{align*}
 \det\Big(\left(\begin{smallmatrix}
-1&0&2\\2&3&1\\1&5&0\end{smallmatrix}\right)\Big)&=(-1)^{1+1}\cdot(-1)\cdot\det\Big(\left(\begin{smallmatrix}3&1\\5&0\end{smallmatrix}\right)\Big)+0
 +(-1)^{1+3}\cdot 2\cdot\det\Big(\left(\begin{smallmatrix}2&3\\1&5\end{smallmatrix}\right)\Big)\\
 &=(-1)(-5)+2(10-3)=19,
 \end{align*}
 wobei wir die Determinantenregel für $(2\times 2)$-Matrizen benutzt haben.
Insgesamt finden wir also 
\[\det(A)=(-3)\cdot 19=-57.\]}
\lang{en}{
We could caluclate this determinant with the rule of Sarrus, but we continue with expanding, here along with the first row:
 \begin{align*}
 \det\Big(\left(\begin{smallmatrix}
-1&0&2\\2&3&1\\1&5&0\end{smallmatrix}\right)\Big)&=(-1)^{1+1}\cdot(-1)\cdot\det\Big(\left(\begin{smallmatrix}3&1\\5&0\end{smallmatrix}\right)\Big)+0
 +(-1)^{1+3}\cdot 2\cdot\det\Big(\left(\begin{smallmatrix}2&3\\1&5\end{smallmatrix}\right)\Big)\\
 &=(-1)(-5)+2(10-3)=19,
 \end{align*}
 at which we used the rule for the determinant of $(2\times 2)$-matrices.
Alltogether we have 
\[\det(A)=(-3)\cdot 19=-57.\]}
\end{incremental}
\end{example}


\lang{de}{Für Determinanten gelten weitere wichtige Rechenregeln}
\lang{en}{There are further rules applying for determinants}
\begin{rule}\label{rule:rechenregeln}
\begin{itemize}
 \item[(i)] \notion{\lang{de}{Determinantenmultiplikationssatz:} \lang{en}{Multiplicative property of determinants:}} 
 \lang{de}{Seien $A, B \in M(n;\K)$. Dann gilt}
  \lang{en}{Given are $A, B \in M(n;\K)$. Then we have}
       \[
       \det(A \cdot B)= \det(A) \cdot \det(B).
       \] 
 \lang{en}{The determinant of a matrix product is equal to the product of their determinants.}
 \item[(ii)] \lang{de}{Ist $A \in M(n;\K)$ mit $\det(A) \neq 0$ (d.h. $A$ ist invertierbar), dann gilt f\"ur die Determinante der inversen Matrix $A^{-1}$:
       \[
       \det(A^{-1})= \frac{1}{\det(A)}.
       \]}
       \lang{en}{Given is $A \in M(n;\K)$ with $\det(A) \neq 0$ (i.e. $A$ is invertible). Then it holds for the determinant of the inverse matrix $A^{-1}$:
       \[
       \det(A^{-1})= \frac{1}{\det(A)}.
       \]}
 \item[(iii)] \lang{de}{Die in der Definition angegebenen Gleichungen für die Determinanten nach Zeilenumformungen gelten genauso auch für Spaltenumformungen.}
 \lang{en}{The equations, listed in the definition, for the determinant after transforming the rows hold also for transforming columns.}
 \item[(iv)] \lang{de}{Kann man $A$ in der Blockmatrix-Form
       \[
       A= \begin{pmatrix} B & C \\  0_{r,s} & D \end{pmatrix} \text{ oder } A=\begin{pmatrix} B & 0_{s,r} \\ C & D \end{pmatrix}
       \]
       schreiben, wobei $B \in M(s;\K)$, $D \in M(r;\K)$, f\"ur $r$, $s \in \N$, dann berechnet sich die Determinante von $A$ wie folgt:
       \[
       \det(A)= \det(B) \det(D).
       \]}
       \lang{en}{If we can write $A$ as a block matrix       \[
       A= \begin{pmatrix} B & C \\  0_{r,s} & D \end{pmatrix} \text{ or } A=\begin{pmatrix} B & 0_{s,r} \\ C & D \end{pmatrix}
       \]
       with $B \in M(s;\K)$, $D \in M(r;\K)$, für $r$, $s \in \N$, then we can calculate the determinant of $A$ as follows:
       \[
       \det(A)= \det(B) \det(D).
       \]}
\item[(v)] \lang{de}{Ist $A^T\in M(n;\K)$ die transponierte Matrix zu $A$, so gilt  \[
       \det(A^T)= \det(A).
       \]}
       \lang{en}{Is $A^T\in M(n;\K)$ the transpose of $A$, we have  \[
       \det(A^T)= \det(A).
       \]}
\end{itemize}
\end{rule}
\begin{proof*}
\begin{showhide}
\begin{itemize}
\item[(i)]
\lang{de}{
Wir benutzen, dass jede invertierbare Matrix als Produkt von Elementarmatrizen geschrieben werden kann.
Ist $A$ also invertierbar, so reicht es zu zeigen, 
dass für jede Matrix $A\in M(n;\K)$ und jede Elementarmatrix $M\in M(n;\K)$ gilt
\[\det(M\cdot A)=\det(M)\cdot\det(A).\]
Das ist allerdings klar nach der \ref[content_09_determinante][Definition der Determinante]{def:determinante}.
Ist in dem Produkt $A\cdot B$ eine der Matrizen nicht invertierbar, hat also nicht vollen Rang, dann gilt das auch für das Produkt.
Dann ist 
\[\det(A\cdot B)=0=\det(A)\cdot\det(B).\]}
\lang{en}{
We utilise, that every invertible matrix can be written as a product of elementary matrices.
If $A$ is invertible, we only need to demonstrate, that for every matrix $A\in M(n;\K)$ 
and every elementary matrix $M\in M(n;\K)$ holds
\[\det(M\cdot A)=\det(M)\cdot\det(A).\]
This is easy to see according to the \ref[content_09_determinante][definition of the determinant]{def:determinante}.
If one of the matrices in the product $A\cdot B$ not invertible, does not have full rank, then this holds also for the product.
Then we have
\[\det(A\cdot B)=0=\det(A)\cdot\det(B).\]}

\item[(ii)]
\lang{de}{
Wir wenden den Determinantenmultiplikationssatz (i) an auf die Identität $A\cdot A^{-1}=E_n$ und erhalten
\[\det(A)\cdot\det(A^{-1})=\det(E_n)=1.\]
Weil $\det(A)\neq 0$, ist das gleichbedeutend zu
\[\det(A^{-1})=\frac{1}{\det(A)}.\]}
\lang{en}{
We can write the identity as $I_n=A\cdot A^{-1}$. Then we apply the multiplication property of the determinant (i) 
on the determinant of $I_n$ and we get
\[\det(A)\cdot\det(A^{-1})=\det(I_n)=1.\]
Because $\det(A)\neq 0$, the equation is equivalent to
\[\det(A^{-1})=\frac{1}{\det(A)}.\]}

\item[(iii)] \lang{de}{Die Spaltenumformungen erhalten wir durch Multiplikation mit Elementarmatrizen von rechts (statt von links wie bei Zeilenumformungen).
Somit liefert der Determinantenmultiplikationssatz (i) die Behauptung.}
\lang{en}{We create the column transformations by multiplying with the elementary matrices from the right (instead of from the left for the row
transformations). The multiplicative property of the determinant (i) then proves the claim.}

\item[(iv)]
\lang{de}{Um die Determinante dieser Blockmatrix zu berechnen, bringt man sie auf Stufenform. Die dazu nötigen elementaren Zeilenumformungen lassen sich aufteilen in die für den
oberen linken Block und die für den unteren rechten Block. Die dadurch entstehende Stufenform ist eine Dreiecksmatrix, und für diese gilt die Formel aus
(iv). Also gilt sie auch für die Ausgangsmatrix.}
\lang{en}{To calculate the determinant of a block matrix, we transform her into row echelon form. The used elementary row transformations can be split in the ones for the
upper left block and the ones for the lower right block. The resulting row echelon form is a triangular matrix, for which we can use
the formular (iv). Therefore we can also use it for the matrix we started with.}

\item[(v)] \lang{de}{Wieder benutzen wir, dass sich $A$ entweder als Produkt von Elementarmatrizen schreiben lässt oder nicht vollen Rang hat.
Im zweiten Fall hat auch $A^T$ nicht vollen Rang (Spaltenrang$=$Zeilenrang). Es gilt also $\det(A)=0=\det(A^T)$.
Im ersten Fall sei $A=M_1\cdots M_r$ mit Elementarmatrizen $M_j$, $j=1,\ldots,r$.
Für die Elementarmatrizen gilt aber offensichtlich $\det(M_j^T)=\det(M_j)$. Also gilt
\[\det(A^T)=\det(M_r^T\cdots M_1^T)=\det(M_r^T)\cdots\det(M_1^T)=
\det(M_1)\cdots\det(M_r)=\det(M_1\cdots M_r)=\det(A),\]
wobei wir die \ref[content_03_transponierte][Rechenregel für transponierte Matrizen]{sec:rechenregeln} und den Determinantenmultiplikationssatz (i) verwendet haben.}
\lang{en}{Here we use again the fact, that $A$ can either be written as a product of elementary matrices or $A$ does not have full rank.
In the second case $A^T$ does not have full rank either (row rank=column rank). It holds $\det(A)=0=\det(A^T)$.
In the first case let $A$ be $A=M_1\cdots M_r$ with the elementary matrices $M_j$, $j=1,\ldots,r$.
For the elementary matrices it obviously holds \[\det(A^T)=\det(M_r^T\cdots M_1^T)=\det(M_r^T)\cdots\det(M_1^T)=
\det(M_1)\cdots\det(M_r)=\det(M_1\cdots M_r)=\det(A).\] Note, that we use the \ref[content_03_transponierte][calculating rules for the transpose]{sec:rechenregeln}
and the multiplicative property of the determinant (i).}
\end{itemize}
\end{showhide}
\end{proof*}
\begin{quickcheck}
\type{input.number}
  \field{rational}
  \displayprecision{3}
  \correctorprecision{4}
 
  \begin{variables}
   \function{a}{5}
   \function{b}{3}
   \function{d}{1/5}
   \function{c}{15}
   \function{f}{5/3}
  \end{variables}
\text{\lang{de}{Für zwei Matrizen $A,B\in M(3;\C)$ sei $\det(A)=5$ und $\det(B)=3$.
Dann ist $\det(A\cdot B)=$\ansref, $\det A^{-1}=$\ansref, $\det(B^T)=$\ansref und $\det(B^{-1}A^T)=$\ansref.}
\lang{en}{Let $\det(A)=5$ and $\det(B)=3$ be the determinants of the matrices $A,B\in M(3;\C)$ .
Determine $\det(A\cdot B)=$\ansref, $\det A^{-1}=$\ansref, $\det(B^T)=$\ansref and $\det(B^{-1}A^T)=$\ansref.}}
 \begin{answer}
    \solution{c}
  \end{answer}
   \begin{answer}
    \solution{d}
   \end{answer}
   \begin{answer}
    \solution{b}
   \end{answer}
   \begin{answer}
    \solution{f}
   \end{answer}
   \end{quickcheck}
   

\lang{de}{
Das folgende Video zeigt die Berechnung von $2 \times2$ - Matrizen,
die Regel von Sarrus,
sowie den Determinantenmultiplikationssatz.
\floatright{\href{https://api.stream24.net/vod/getVideo.php?id=10962-2-10880&mode=iframe&speed=true}{\image[75]{00_video_button_schwarz-blau}}}}\\
   
   
\end{visualizationwrapper}

\end{content}