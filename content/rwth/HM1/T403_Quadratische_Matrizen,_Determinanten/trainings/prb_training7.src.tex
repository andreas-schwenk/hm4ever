\documentclass{mumie.problem.gwtmathlet}
%$Id$
\begin{metainfo}
  \name{
    \lang{de}{A08: Determinante}
    \lang{en}{P08: The determinant}
  }
  \begin{description} 
 This work is licensed under the Creative Commons License Attribution 4.0 International (CC-BY 4.0)   
 https://creativecommons.org/licenses/by/4.0/legalcode 

    \lang{de}{}
    \lang{en}{}
  \end{description}
  \corrector{system/problem/GenericCorrector.meta.xml}
  \begin{components}
    \component{js_lib}{system/problem/GenericMathlet.meta.xml}{mathlet}
  \end{components}
  \begin{links}
  \end{links}
  \creategeneric
\end{metainfo}
\begin{content}
\usepackage{mumie.genericproblem}

\lang{de}{\title{A08: Determinante}}
\lang{en}{\title{P08: The determinant}}

\begin{block}[annotation]
	Im Ticket-System: \href{http://team.mumie.net/issues/11441}{Ticket 11441}
\end{block}

\begin{problem}
	\randomquestionpool{1}{2}
	

	
	\begin{question}
% 	
% 	$a2,a3,a4,a5,b3,b4,b5,c4,c5,d5\in\{-5;\ldots;5\}$\\
%   $a1,b2,c3,d4,f5\in\{-2;\ldots;3\}\setminus\{0\}$\\
%   $r,t\in\{-2;\ldots;3\}\setminus\{0\}$\\
% 		
		\begin{variables}
		
		\randint{a2}{-5}{5}
		\randint{a3}{-5}{5}
		\randint{a4}{-5}{5}
		\randint{a5}{-5}{5}
		\randint{b3}{-5}{5}
		\randint{b4}{-5}{5}
		\randint{b5}{-5}{5}
		\randint{c4}{-5}{5}
		\randint{c5}{-5}{5}
		\randint{d5}{-5}{5}
	
		\randint[Z]{f5}{-2}{3}
		\randint[Z]{b2}{-2}{3}
		\randint[Z]{c3}{-2}{3}
		\randint[Z]{d4}{-2}{3}
		\randint[Z]{a1}{-2}{3}
		
		\randint[Z]{r}{-2}{3}
		\randint[Z]{t}{-2}{3}
		
			\begin{pool}
				\begin{variables}
% \item $b1=r, c2=0, d3=t, f4=0$,
% $s=(a1*b2-b1*a2)*(c3*d4-d3*c4)*f5$.
		\function[calculate]{b1}{r}
		\function[calculate]{c2}{0}
		\function[calculate]{d3}{t}
		\function[calculate]{f4}{0}
		
		\function[calculate]{s}{(a1*b2-b1*a2)*(c3*d4-d3*c4)*f5}
		
				\end{variables}
				
				\begin{variables} 				
% \item $b1=r, c2=0, d3=0, f4=t$,
% $s=(a1*b2-b1*a2)*c3*(d4*f5-d5*f4)$.
		\function[calculate]{b1}{r}
		\function[calculate]{c2}{0}
		\function[calculate]{d3}{0}
		\function[calculate]{f4}{t}
		
		\function[calculate]{s}{(a1*b2-b1*a2)*c3*(d4*f5-d5*f4)}
				\end{variables}
				
				\begin{variables}
% \item $b1=0, c2=r, d3=0, f4=t$,
%$s=a1*(b2*c3-b3*c2)*(d4*f5-d5*f4)$.
		\function[calculate]{b1}{0}
		\function[calculate]{c2}{r}
		\function[calculate]{d3}{0}
		\function[calculate]{f4}{t}
		
		\function[calculate]{s}{a1*(b2*c3-b3*c2)*(d4*f5-d5*f4)}
				
				\end{variables}
			\end{pool}

		
			\matrix{aa}{
a1 & a2 & a3 & a4& a5\\ b1 & b2 & b3 & b4 & b5 \\ 0& c2 & c3 & c4 & c5\\
0 & 0 & d3 & d4&d5 \\ 0 & 0 & 0 & f4 & f5 }

		\end{variables}
		
		\type{input.generic}
        \field{real}
		\correctorprecision{3}
		\displayprecision{3}
		\lang{de}{
	    \text{Bestimmen Sie die Determinante der  Matrix $A=\var{aa}$.}
	    }
     \lang{en}{
	    \text{Calculate the determinant of the matrix $A=\var{aa}$.}
	    }
	    %\permuteAnswers{1, 2, 3} 
	    %http://team.mumie.net/projects/support/wiki/DifferentAnswerType
     \lang{de}{
	    \begin{answer}
            \type{input.number}
	    	\text{$\det(A)= $}
			\solution{s}
            \explanation{Es handelt sich um eine Blockmatrix. Führen Sie die Berechnung auf die Determinanten der Diagonalblöcke zurück.}
	    \end{answer}} 

     \lang{en}{
	    \begin{answer}
            \type{input.number}
	    	\text{$\det(A)= $}
			\solution{s}
            \explanation{$A$ is a block matrix. Reduce the calculation to the calculation of determinants of diagonal blocks.}
	    \end{answer}} 
	    
	    %generic viz: http://team.mumie.net/projects/support/wiki/Example
	    
	\end{question}
	
	\begin{question}
		
		\begin{variables}
		
		\randint{a2}{-5}{5}
		\randint{a3}{-5}{5}
		\randint{a4}{-5}{5}
		\randint{a5}{-5}{5}
		\randint{b3}{-5}{5}
		\randint{b4}{-5}{5}
		\randint{b5}{-5}{5}
		\randint{c4}{-5}{5}
		\randint{c5}{-5}{5}
		\randint{d5}{-5}{5}
	
		\randint[Z]{f5}{-2}{3}
		\randint[Z]{b2}{-2}{3}
		\randint[Z]{c3}{-2}{3}
		\randint[Z]{d4}{-2}{3}
		\randint[Z]{a1}{-2}{3}
		
		\randint[Z]{r}{-2}{3}
		\randint[Z]{t}{-2}{3}
		
			\begin{pool}
				\begin{variables}
% \item $b1=r, c2=0, d3=t, f4=0$,
% $s=(a1*b2-b1*a2)*(c3*d4-d3*c4)*f5$.
		\function[calculate]{b1}{r}
		\function[calculate]{c2}{0}
		\function[calculate]{d3}{t}
		\function[calculate]{f4}{0}
		
		\function[calculate]{s}{(a1*b2-b1*a2)*(c3*d4-d3*c4)*f5}
		
				\end{variables}
				
				\begin{variables} 				
% \item $b1=r, c2=0, d3=0, f4=t$,
% $s=(a1*b2-b1*a2)*c3*(d4*f5-d5*f4)$.
		\function[calculate]{b1}{r}
		\function[calculate]{c2}{0}
		\function[calculate]{d3}{0}
		\function[calculate]{f4}{t}
		
		\function[calculate]{s}{(a1*b2-b1*a2)*c3*(d4*f5-d5*f4)}
				\end{variables}
				
				\begin{variables}
% \item $b1=0, c2=r, d3=0, f4=t$,
%$s=a1*(b2*c3-b3*c2)*(d4*f5-d5*f4)$.
		\function[calculate]{b1}{0}
		\function[calculate]{c2}{r}
		\function[calculate]{d3}{0}
		\function[calculate]{f4}{t}
		
		\function[calculate]{s}{a1*(b2*c3-b3*c2)*(d4*f5-d5*f4)}
				
				\end{variables}
			\end{pool}

			
				\matrix{aa}{
a1 & b1 & 0 & 0 &0\\ a2&b2&c2&0&0\\ a3&b3&c3&d3&0\\ a4&b4&c4&d4&f4\\
a5&b5&c5&d5&f5
}

		\end{variables}
		
		\type{input.generic}
        \field{real}
		\correctorprecision{3}
		\displayprecision{3}
		\lang{de}{
	    \text{Bestimmen Sie die Determinante der  Matrix $A=\var{aa}$.}
	    }
     \lang{en}{
	    \text{Calculate the determinant of the matrix $A=\var{aa}$.}
	    }
	    %\permuteAnswers{1, 2, 3} 
	    %http://team.mumie.net/projects/support/wiki/DifferentAnswerType
     \lang{de}{
	    \begin{answer}
            \type{input.number}
	    	\text{$\det(A)= $}
			\solution{s}
            \explanation{Es handelt sich um eine Blockmatrix. Führen Sie die Berechnung auf die Determinanten der Diagonalblöcke zurück.}
	    \end{answer}}  

     \lang{en}{
	    \begin{answer}
            \type{input.number}
	    	\text{$\det(A)= $}
			\solution{s}
            \explanation{$A$ is a block matrix. Reduce the calculation to the calculation of determinants of diagonal blocks.}
	    \end{answer}}  
	    
	    %generic viz: http://team.mumie.net/projects/support/wiki/Example
	    
	\end{question}
\end{problem}

\embedmathlet{mathlet}

\end{content}