\documentclass{mumie.problem.gwtmathlet}
%$Id$
\begin{metainfo}
  \name{
    \lang{de}{A09: Cramersche Regel}
    \lang{en}{P09: Cramer's rul}
  }
  \begin{description} 
 This work is licensed under the Creative Commons License Attribution 4.0 International (CC-BY 4.0)   
 https://creativecommons.org/licenses/by/4.0/legalcode 

    \lang{de}{}
    \lang{en}{}
  \end{description}
  \corrector{system/problem/GenericCorrector.meta.xml}
  \begin{components}
    \component{js_lib}{system/problem/GenericMathlet.meta.xml}{mathlet}
  \end{components}
  \begin{links}
  \end{links}
  \creategeneric
\end{metainfo}
\begin{content}
\usepackage{mumie.genericproblem}

\lang{de}{\title{A09: Cramersche Regel}}
\lang{en}{\title{P09: Cramer's rule}}

\begin{block}[annotation]
	Im Ticket-System: \href{http://team.mumie.net/issues/11442}{Ticket 11442}
\end{block}

\begin{problem}
	%\randomquestionpool{}{}
    
    		\begin{variables}
			
			\randint[Z]{a}{-3}{3}
			\randint[Z]{b}{-3}{3}
			\randint[Z]{c}{-3}{3}
			\randint[Z]{f}{-3}{3}
			\randint[Z]{r}{-3}{3}
			\randint[Z]{t}{-3}{3}
			
			\matrix[calculate]{aa}{a & 0 & 0 \\ c  & 2 & b \\ 0 & f & f*b } 
\matrix{bb}{r\\ 0 \\ t}
\function[calculate]{s}{a*f*b}
\function{s1}{s}
\function{l1}{(r*f*b)/s1}
\function{l2}{-(a*b*t+r*c*f*b)/s1}
\function{l3}{(2*a*t+r*c*f)/s1}
% \function{L1}{l1}
% \function{L2}{l2}
% \function{L2}{l2}
% % \function{A1}{r*f*b}
% \function{A2}{-(a*b*t+r*c*f*b)}
% \function{A3}{2*a*t+c*f*r}
\matrix[calculate]{m}{l1\\ l2\\ l3}
  
		\end{variables}
	\begin{question}
		
		\begin{variables}
			\earlierAnswer{s1}{-1,1}
% 			\randint[Z]{a}{-3}{3}
% 			\randint[Z]{b}{-3}{3}
% 			\randint[Z]{c}{-3}{3}
% 			\randint[Z]{f}{-3}{3}
% 			\randint[Z]{r}{-3}{3}
% 			\randint[Z]{t}{-3}{3}
			
% 			\matrix[calculate]{aa}{a & 0 & 0 \\ c  & 2 & b \\ 0 & f & f*b } 
% \matrix{bb}{r\\ 0 \\ t}
% \function[calculate]{s}{a*f*b}
% \function{l1}{r/a}
% \function{l2}{-(a*b*t+r*c*f*b)/s}
% \function{l3}{(2*a*t+r*c*f)/s}
% \function{L1}{l1}
% \function{L2}{l2}
% \function{L2}{l2}
% % \function{A1}{r*f*b}
% % \function{A2}{-(a*b*t+r*c*f*b)}
% % \function{A3}{2*a*t+c*f*r}
% \matrix[calculate]{m}{l1\\ l2\\ l3}
  
		\end{variables}
		
		\type{input.generic}
        \field{rational}
        \lang{de}{
	    \text{Bestimmen Sie mit Hilfe der Cramerschen Regel die eindeutige Lösung des
linearen Gleichungssystems
$    \var{aa}\cdot \begin{pmatrix}
x_1 \\ x_2 \\ x_3
\end{pmatrix} = \var{bb}. $}}

 \lang{en}{
	    \text{Determine with the help of Cramer's rule the unique solution of the linear system
$    \var{aa}\cdot \begin{pmatrix}
x_1 \\ x_2 \\ x_3
\end{pmatrix} = \var{bb}. $}}

     \lang{de}{
	    \begin{answer}
            \type{input.number}
	    	\text{Die Determinante der Koeffizientenmatrix ist: $\det(\var{aa})= $}
			\solution{s}
            \explanation{Benutzen Sie die Laplace-Entwicklung nach der ersten Zeile oder die Regel von Sarrus.}
	    \end{answer}    
	    \begin{answer}
            \type{input.matrix}
	    	\text{Die Lösung des LGS ist:  $\begin{pmatrix}
x_1 \\ x_2 \\ x_3
\end{pmatrix} =$}
			\solution{m}
			\format{3}{1}
         \explanation{Bezeichnet $A$ die obige Matrix und $b$ den Vektor der rechten Seite, so ergibt sich nach der Cramerschen Regel die
                       $j$-te Komponente der Lösung als $x_j=\frac{\det(A_{j|b})}{\det A}$.}
	    \end{answer}}

     \lang{en}{
	    \begin{answer}
            \type{input.number}
	    	\text{The determinant of the coefficient matrix is $\det(\var{aa})= $}
			\solution{s}
            \explanation{Use Laplace-expansion along with the first row or the rule of Sarrus.}
	    \end{answer}    
	    \begin{answer}
            \type{input.matrix}
	    	\text{The solution of the linear system is $\begin{pmatrix}
x_1 \\ x_2 \\ x_3
\end{pmatrix} =$}
			\solution{m}
			\format{3}{1}
         \explanation{If we denote with $A$ the matrix above and with $b$ the vector on the right side,
         we receive with Cramer's rule the $j$th component of the solution as $x_j=\frac{\det(A_{j|b})}{\det A}$.}
	    \end{answer}}
	    
	\end{question}
\end{problem}

\embedmathlet{mathlet}

\end{content}