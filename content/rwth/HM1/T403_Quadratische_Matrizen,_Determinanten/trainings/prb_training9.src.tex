\documentclass{mumie.problem.gwtmathlet}
%$Id$
\begin{metainfo}
  \name{
    \lang{en}{P10: Determinant and inverse matrix}
    \lang{de}{A10: Determinante, Inverse Matrix}
  }
  \begin{description} 
 This work is licensed under the Creative Commons License Attribution 4.0 International (CC-BY 4.0)   
 https://creativecommons.org/licenses/by/4.0/legalcode 

    \lang{en}{...}
    \lang{de}{Bestimmung von Determinante und Inverser eine komplexwertigen $(2\times 2)$-Matrix.}
  \end{description}
  \corrector{system/problem/GenericCorrector.meta.xml}
  \begin{components}
    \component{js_lib}{system/problem/GenericMathlet.meta.xml}{gwtmathlet}
  \end{components}
  \begin{links}
  \end{links}
  \creategeneric
\end{metainfo}
\begin{content}
\begin{block}[annotation]
	Im Ticket-System: \href{https://team.mumie.net/issues/21824}{Ticket 21824}
\end{block}
\usepackage{mumie.genericproblem}



\lang{de}{\title{A10: Determinante, Inverse Matrix}}
\lang{en}{\title{P10: Determinant and inverse matrix}}


\begin{problem}
	%\randomquestionpool{}{}
	
		
\begin{variables}

    % Einträge der (2x2)-Matrix:
    \randint[Z]{a1}{-2}{2}
    \randint[Z]{a2}{-2}{2}
    \randint[Z]{b1}{-2}{2}
    \randint[Z]{b2}{0}{3}
    \randint[Z]{c2}{0}{2}

    %Matrix A
    \matrix[calculate]{A}{a1+i*a2&b1-i*b2\\-b1-i*c2&a1-i*a2}
    % Determinante von A
    \function[calculate]{det}{a1^2+a2^2+b1^2+b2*c2+i*b1*(c2-b2)}
    %Inverses der Determinante von A
    \function[calculate]{dinv}{(a1^2+a2^2+b1^2+b2*c2-i*b1*(c2-b2))/((a1^2+a2^2+b1^2+b2*c2)^2+(b1*(c2-b2))^2)}
    % Inverse von A
    \matrix[calculate]{B}{(a1-i*a2)/det&(-b1+i*b2)/det\\(b1+i*c2)/det&(a1+i*a2)/det}

\end{variables}

\begin{question}		
		\type{input.function}
        \field{complex-rational}
		\correctorprecision{3}
		\displayprecision{3}
		\lang{de}{
	    \text{Berechnen Sie die Determinante und die inverse Matrix der Matrix $A=\var{A}\in\C$. 
        Geben Sie komplexen Zahlen in der Form $a+b \cdot i$ mit gekürzten Brüchen $a,b$ an.
        \\
        $\det(A)=$\ansref, also ist $\frac{1}{\det(A)}=$\ansref und
        \\
        $A^{-1}=$\ansref.}}
    \lang{en}{
	    \text{Calculate the determinant and the inverse matrix of the matrix $A=\var{A}\in\C$. 
        Write the complex numbers as $a+b \cdot i$ with reduced fractions $a,b$.
        \\
        $\det(A)=$\ansref, therefore it is $\frac{1}{\det(A)}=$\ansref and
        \\
        $A^{-1}=$\ansref.}}

        \lang{de}{
        \begin{answer}
         \type{input.number}
         \solution{det}
         \explanation{Benutzen Sie eine bekannte Formel für die Determinante einer $(2\times 2)$-Matrix.}
         \end{answer}
         \begin{answer}
         \type{input.number}
         \solution{dinv}
         \explanation{Erinnern Sie sich an die Formel $z^{-1}=\frac{\bar{z}}{z\cdot \bar{z}}$ für $z\neq 0$.}
         \end{answer}
	    \begin{answer}
            \type{input.matrix}
            \format{2}{2}
            \solution{B}
            \explanation{Auch für die Inverse einer $(2\times 2)$-Matrix gibt es eine bekannte Formel, 
            die Sie benutzen können.}
         \end{answer}}

         \lang{en}{
        \begin{answer}
         \type{input.number}
         \solution{det}
         \explanation{Use one of the formulas you know to calculate the determinant of a $(2\times 2)$-matrix.}
         \end{answer}
         \begin{answer}
         \type{input.number}
         \solution{dinv}
         \explanation{Recap the formula $z^{-1}=\frac{\bar{z}}{z\cdot \bar{z}}$ for $z\neq 0$.}
         \end{answer}
	    \begin{answer}
            \type{input.matrix}
            \format{2}{2}
            \solution{B}
            \explanation{There is also a familiar formular for the inverse matrix of a $(2\times 2)$-matrix.}
         \end{answer}}
          
\end{question}
\end{problem}
\embedmathlet{gwtmathlet}
\end{content}
