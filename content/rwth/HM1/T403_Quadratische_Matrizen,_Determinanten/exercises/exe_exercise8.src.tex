\documentclass{mumie.element.exercise}
%$Id$
\begin{metainfo}
  \name{
    \lang{de}{Ü11: Cramersche Regel}
    \lang{en}{Ex11: Cramer's rule}
  }
  \begin{description} 
 This work is licensed under the Creative Commons License Attribution 4.0 International (CC-BY 4.0)   
 https://creativecommons.org/licenses/by/4.0/legalcode 

    \lang{de}{}
    \lang{en}{}
  \end{description}
  \begin{components}
  \end{components}
  \begin{links}
  \end{links}
  \creategeneric
\end{metainfo}
\begin{content}
\usepackage{mumie.ombplus}

\title{\lang{de}{Ü11: Cramersche Regel} \lang{en}{Ex11: Cramer's rule}}

\begin{block}[annotation]
  Im Ticket-System: \href{http://team.mumie.net/issues/11363}{Ticket 11363}
\end{block}

%######################################################FRAGE_TEXT
\lang{de}{ 
Lösen Sie die folgenden linearen Gleichungssysteme mit der Cramerschen Regel.
\[
%\setlength{\arraycolsep}{1.5pt}%

\begin{mtable}[\cellaligns{lrl}]
\text{a)}\ & 2 x + y + z & = 0 \\
& x + y + 2 z & = -3 \\
& -x - 2 y + 2 z & = 2 \\
\end{mtable}\qquad\quad

\begin{mtable}[\cellaligns{lrl}]
\text{b)}\ & 3 x + 2 y - z & = 2 \\
& x + z & = 2 \\
& x - y + 2 z & = 3 \\
\end{mtable}
\] 

\[
\begin{mtable}[\cellaligns{lrl}]
\text{c)}\ & x_1 + x_2 & = 0 \\
& 2 x_1 + 3 x_2 & = 4 \\
\end{mtable}\qquad\quad

\begin{mtable}[\cellaligns{lrl}]
\text{d)}\ & x_1 - x_2 - x_3 & = 0 \\
& x_1 + x_2 + 3 x_3 & = 4 \\
& 2 x_1 + x_3 & = 3 \\
\end{mtable}
\] 

}


\lang{en}{ 
Solve the following systems of linear equations with the help of Cramer's rule.
\[
%\setlength{\arraycolsep}{1.5pt}%

\begin{mtable}[\cellaligns{lrl}]
\text{a)}\ & 2 x + y + z & = 0 \\
& x + y + 2 z & = -3 \\
& -x - 2 y + 2 z & = 2 \\
\end{mtable}\qquad\quad

\begin{mtable}[\cellaligns{lrl}]
\text{b)}\ & 3 x + 2 y - z & = 2 \\
& x + z & = 2 \\
& x - y + 2 z & = 3 \\
\end{mtable}
\] 

\[
\begin{mtable}[\cellaligns{lrl}]
\text{c)}\ & x_1 + x_2 & = 0 \\
& 2 x_1 + 3 x_2 & = 4 \\
\end{mtable}\qquad\quad

\begin{mtable}[\cellaligns{lrl}]
\text{d)}\ & x_1 - x_2 - x_3 & = 0 \\
& x_1 + x_2 + 3 x_3 & = 4 \\
& 2 x_1 + x_3 & = 3 \\
\end{mtable}
\] 

}
%##################################################ANTWORTEN_TEXT
\begin{tabs*}[\initialtab{0}\class{exercise}]

    \tab{\lang{de}{    Antwort    } \lang{en}{Answer}}
    \lang{de}{
        
        a) 
        ~$\left( \begin{mtable}[\cellaligns{r}]
            2 & -3 & -1
        \end{mtable} \right)^T$
        
        b)
        ~$\left( \begin{mtable}[\cellaligns{r}]
            1 & 0 & 1
        \end{mtable} \right)^T$
        
        c)
        ~$\left( \begin{mtable}[\cellaligns{r}]
            -4 & 4
        \end{mtable} \right)^T$
        
        d)
        ~$\left( \begin{mtable}[\cellaligns{r}]
            1 & 0 & 1
        \end{mtable} \right)^T$

    }
\lang{en}{
        
        a) 
        ~$\left( \begin{mtable}[\cellaligns{r}]
            2 & -3 & -1
        \end{mtable} \right)^T$
        
        b)
        ~$\left( \begin{mtable}[\cellaligns{r}]
            1 & 0 & 1
        \end{mtable} \right)^T$
        
        c)
        ~$\left( \begin{mtable}[\cellaligns{r}]
            -4 & 4
        \end{mtable} \right)^T$
        
        d)
        ~$\left( \begin{mtable}[\cellaligns{r}]
            1 & 0 & 1
        \end{mtable} \right)^T$

    }

  %++++++++++++++++++++++++++++++++++++++++++START_TAB_X
  \tab{\lang{de}{    Lösung a)    } \lang{en}{Solution a)}}
  \begin{incremental}[\initialsteps{1}]
  
  	 %----------------------------------START_STEP_X
    \step 
    \lang{de}{   Man hat
\begin{align*}
\det \left( \begin{mtable}[\cellaligns{rrr}]
2 & 1 & 1 \\
1 & 1 & 2 \\
-1 & -2 & 2 \\
\end{mtable} \right) = 7, \qquad
\det \left( \begin{mtable}[\cellaligns{rrr}]
0 & 1 & 1 \\
-3 & 1 & 2 \\
2 & -2 & 2 \\
\end{mtable} \right) = 14, \qquad
\det \left( \begin{mtable}[\cellaligns{rrr}]
2 & 0 & 1 \\
1 & -3 & 2 \\
-1 & 2 & 2 \\
\end{mtable} \right) = -21, \\
\det \left( \begin{mtable}[\cellaligns{rrr}]
2 & 1 & 0 \\
1 & 1 & -3 \\
-1 & -2 & 2 \\
\end{mtable} \right) = -7, \qquad \quad
\mbox{also} \quad
\left( \begin{mtable}[\cellaligns{c}] x \\ y \\ z \end{mtable} \right) =
\frac{1}{7} \left( \begin{mtable}[\cellaligns{r}] 14 \\ -21 \\ -7 \end{mtable} \right) =
\left( \begin{mtable}[\cellaligns{r}] 2 \\ -3 \\ -1 \end{mtable} \right)
\end{align*}
als eindeutige Lösung des linearen Gleichungssystems.
    }

    \lang{de}{   We have
\begin{align*}
\det \left( \begin{mtable}[\cellaligns{rrr}]
2 & 1 & 1 \\
1 & 1 & 2 \\
-1 & -2 & 2 \\
\end{mtable} \right) = 7, \qquad
\det \left( \begin{mtable}[\cellaligns{rrr}]
0 & 1 & 1 \\
-3 & 1 & 2 \\
2 & -2 & 2 \\
\end{mtable} \right) = 14, \qquad
\det \left( \begin{mtable}[\cellaligns{rrr}]
2 & 0 & 1 \\
1 & -3 & 2 \\
-1 & 2 & 2 \\
\end{mtable} \right) = -21, \\
\det \left( \begin{mtable}[\cellaligns{rrr}]
2 & 1 & 0 \\
1 & 1 & -3 \\
-1 & -2 & 2 \\
\end{mtable} \right) = -7, \qquad \quad
\mbox{also} \quad
\left( \begin{mtable}[\cellaligns{c}] x \\ y \\ z \end{mtable} \right) =
\frac{1}{7} \left( \begin{mtable}[\cellaligns{r}] 14 \\ -21 \\ -7 \end{mtable} \right) =
\left( \begin{mtable}[\cellaligns{r}] 2 \\ -3 \\ -1 \end{mtable} \right)
\end{align*}
as the unique solution of the linear system.}
  	 %------------------------------------END_STEP_X
 
  \end{incremental}
  %++++++++++++++++++++++++++++++++++++++++++++END_TAB_X


  %++++++++++++++++++++++++++++++++++++++++++START_TAB_X
  \tab{\lang{de}{    Lösung b)    } \lang{en}{Solution b)}}
  \begin{incremental}[\initialsteps{1}]
  
  	 %----------------------------------START_STEP_X
    \step 
    \lang{de}{   Man hat
\begin{align*}
\det \left( \begin{mtable}[\cellaligns{rrr}]
3 & 2 & -1 \\
1 & 0 & 1 \\
1 & -1 & 2 \\
\end{mtable} \right) = 2, \qquad
\det \left( \begin{mtable}[\cellaligns{rrr}]
2 & 2 & -1 \\
2 & 0 & 1 \\
3 & -1 & 2 \\
\end{mtable} \right) = 2, \qquad
\det \left( \begin{mtable}[\cellaligns{rrr}]
3 & 2 & -1 \\
1 & 2 & 1 \\
1 & 3 & 2 \\
\end{mtable} \right) = 0, \\
\det \left( \begin{mtable}[\cellaligns{rrr}]
3 & 2 & 2 \\
1 & 0 & 2 \\
1 & -1 & 3 \\
\end{mtable} \right) = 2, \qquad \quad
\mbox{also} \quad
\begin{pmatrix} x \\ y \\ z \end{pmatrix} = \frac{1}{2} \begin{pmatrix} 2 \\ 0 \\ 2 \end{pmatrix} = \begin{pmatrix} 1 \\ 0 \\ 1 \end{pmatrix}
\end{align*}
als eindeutige Lösung des linearen Gleichungssystems.
    }
    \lang{en}{   We get
\begin{align*}
\det \left( \begin{mtable}[\cellaligns{rrr}]
3 & 2 & -1 \\
1 & 0 & 1 \\
1 & -1 & 2 \\
\end{mtable} \right) = 2, \qquad
\det \left( \begin{mtable}[\cellaligns{rrr}]
2 & 2 & -1 \\
2 & 0 & 1 \\
3 & -1 & 2 \\
\end{mtable} \right) = 2, \qquad
\det \left( \begin{mtable}[\cellaligns{rrr}]
3 & 2 & -1 \\
1 & 2 & 1 \\
1 & 3 & 2 \\
\end{mtable} \right) = 0, \\
\det \left( \begin{mtable}[\cellaligns{rrr}]
3 & 2 & 2 \\
1 & 0 & 2 \\
1 & -1 & 3 \\
\end{mtable} \right) = 2, \qquad \quad
\mbox{also} \quad
\begin{pmatrix} x \\ y \\ z \end{pmatrix} = \frac{1}{2} \begin{pmatrix} 2 \\ 0 \\ 2 \end{pmatrix} = \begin{pmatrix} 1 \\ 0 \\ 1 \end{pmatrix}
\end{align*}
as unique solution of the linear system.
    }
  	 %------------------------------------END_STEP_X
 
  \end{incremental}
  %++++++++++++++++++++++++++++++++++++++++++++END_TAB_X

  %++++++++++++++++++++++++++++++++++++++++++START_TAB_X
  \tab{\lang{de}{    Lösungsvideo c) und d)    }}
  
\lang{de}{  Bemerkung: Im Video werden c), d) mit a), b) bezeichnet.
  
\youtubevideo[500][300]{e8dvelvD9aM}}\\


%####################ENDE

% \tab{\lang{de}{Video: ähnliche Übungsaufgabe}}	
%    \youtubevideo[500][300]{e8dvelvD9aM}\\
    
\end{tabs*}
\end{content}