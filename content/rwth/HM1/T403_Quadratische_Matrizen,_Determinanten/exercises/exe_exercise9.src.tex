\documentclass{mumie.element.exercise}
%$Id$
\begin{metainfo}
  \name{
    \lang{de}{Ü12: Cramersche Regel}
    \lang{en}{Ex12: Cramer's rule}
  }
  \begin{description} 
 This work is licensed under the Creative Commons License Attribution 4.0 International (CC-BY 4.0)   
 https://creativecommons.org/licenses/by/4.0/legalcode 

    \lang{de}{}
    \lang{en}{}
  \end{description}
  \begin{components}
  \end{components}
  \begin{links}
  \end{links}
  \creategeneric
\end{metainfo}
\begin{content}
\usepackage{mumie.ombplus}

\title{\lang{de}{Ü12: Cramersche Regel}  \lang{en}{Ex12: Cramer's rule}}

\begin{block}[annotation]
  Im Ticket-System: \href{http://team.mumie.net/issues/11364}{Ticket 11364}
\end{block}

%######################################################FRAGE_TEXT
\lang{de}{ 
Bestimmen Sie  zu $A$ und zu $B$ jeweils die inverse Matrix mit Hilfe der Cramerschen Regel 
\[
A= \begin{pmatrix} 3 & 0 & 1 \\ 1 & 2 & 1 \\ 1 & -1 & 2 \end{pmatrix}, \quad 
B=\begin{pmatrix} 1 & 1 & 2 \\ -1 & 4 & 5 \\ 1 & 2 & 3 \end{pmatrix}.
\]
 }

\lang{en}{ 
Determine the inverse matrices for $A$ and $B$ with the help of Cramer's rule
\[
A= \begin{pmatrix} 3 & 0 & 1 \\ 1 & 2 & 1 \\ 1 & -1 & 2 \end{pmatrix}, \quad 
B=\begin{pmatrix} 1 & 1 & 2 \\ -1 & 4 & 5 \\ 1 & 2 & 3 \end{pmatrix}.
\]
 }
%##################################################ANTWORTEN_TEXT
\begin{tabs*}[\initialtab{0}\class{exercise}]

  %++++++++++++++++++++++++++++++++++++++++++START_TAB_X
  \tab{\lang{de}{    Lösung zu $A$    } \lang{en}{Solution for $A$}}
  \begin{incremental}[\initialsteps{1}]
  
  	 %----------------------------------START_STEP_X
    \step 
    \lang{de}{   
Wir beginnen mit der Matrix $A$ und berechnen zunächst die Determinante (zum Beispiel mit der Regel von Sarrus)
\[
\det(A)= 12-1-2+3=12.
\]
Nun l\"osen wir mit Hilfe der Cramerschen Regel die folgenden $3$ Gleichungssysteme
\[
A x= \begin{pmatrix} 1 \\ 0 \\ 0 \end{pmatrix}, \quad A y= \begin{pmatrix} 0 \\ 1 \\ 0 \end{pmatrix}, \quad  A z= \begin{pmatrix} 0 \\ 0 \\ 1 \end{pmatrix}. 
\]
Beachten wir, dass $\det(A)=12$ gilt, so berechnen wir weiter
\[
x_1 = \frac{\det \left( \left( \begin{smallmatrix} 1 & 0 & 1 \\ 0 & 2 & 1 \\ 0 & -1 & 2   \end{smallmatrix} \right) \right)}{\det(A)}  =  \frac{5}{12},
\quad
x_2 = \frac{ \det \left( \left( \begin{smallmatrix} 3 & 1 & 1 \\ 1 & 0 & 1 \\ 1 & 0 & 2   \end{smallmatrix}  \right) \right)}{\det(A)} = \frac{-1}{12},
\quad
x_3 = \frac{ \det \left( \left( \begin{smallmatrix} 3 & 0 & 1 \\ 1 & 2 & 0 \\ 1 & -1 & 0   \end{smallmatrix} \right) \right)}{\det(A)} = \frac{-3}{12}, 
\]
\[
y_1 = \frac{\det \left( \left( \begin{smallmatrix} 0 & 0 & 1 \\ 1 & 2 & 1 \\ 0 & -1 & 2   \end{smallmatrix} \right) \right)}{\det(A)}  = \frac{-1}{12} ,
\quad
y_2 = \frac{ \det \left( \left( \begin{smallmatrix} 3 & 0 & 1 \\ 1 & 1 & 1 \\ 1 & 0 & 2   \end{smallmatrix}  \right) \right)}{\det(A)} = \frac{5}{12},
\quad
y_3 = \frac{ \det \left( \left( \begin{smallmatrix} 3 & 0 & 0 \\ 1 & 2 & 1 \\ 1 & -1 & 0   \end{smallmatrix} \right) \right)}{\det(A)} =  \frac{3}{12}, 
\]
\[
z_1 = \frac{\det \left( \left( \begin{smallmatrix} 0 & 0 & 1 \\ 0 & 2 & 1 \\ 1 & -1 & 2   \end{smallmatrix} \right) \right)}{\det(A)}  =\frac{-2}{12} ,
\quad
z_2 = \frac{ \det \left( \left( \begin{smallmatrix} 3 & 0 & 1 \\ 1 & 0 & 1 \\ 1 & 1 & 2   \end{smallmatrix}  \right) \right)}{\det(A)} = \frac{-2}{12},
\quad
z_3 = \frac{ \det \left( \left( \begin{smallmatrix} 3 & 0 & 0 \\ 1 & 2 & 0 \\ 1 & -1 & 1   \end{smallmatrix} \right) \right)}{\det(A)} =  \frac{6}{12}, 
\]
Damit ist $A^{-1}$ gegeben durch
\[
A^{-1} = \frac{1}{12} \begin{pmatrix} 5 & -1 & -2 \\ -1 & 5 & -2 \\ -3 & 3 & 6 \end{pmatrix}.
\]
Wir machen noch einmal die Probe:
\[
A \cdot A^{-1} = \begin{pmatrix} 3 & 0 & 1 \\ 1 & 2 & 1 \\ 1 & -1 & 2 \end{pmatrix} \cdot \frac{1}{12} \begin{pmatrix} 5 & -1 & -2 \\ -1 & 5 & -2 \\ -3 & 3 & 6 \end{pmatrix}
               = \frac{1}{12} \begin{pmatrix} 12 & 0 & 0 \\ 0 & 12 & 0 \\ 0 & 0 & 12 \end{pmatrix} = \begin{pmatrix} 1 & 0 & 0 \\ 0 & 1 & 0 \\ 0 & 0 & 1 \end{pmatrix}.
\]
    }


    \lang{en}{  
    We start with the matrix $A$ and determine the determinant (for example with the rule of Sarrus)
\[
\det(A)= 12-1-2+3=12.
\]
Now we solve the following $3$ linear system using Cramer's rule
\[
A x= \begin{pmatrix} 1 \\ 0 \\ 0 \end{pmatrix}, \quad A y= \begin{pmatrix} 0 \\ 1 \\ 0 \end{pmatrix}, \quad  A z= \begin{pmatrix} 0 \\ 0 \\ 1 \end{pmatrix}. 
\]
Considering $\det(A)=12$, we continue to calculate
\[
x_1 = \frac{\det \left( \left( \begin{smallmatrix} 1 & 0 & 1 \\ 0 & 2 & 1 \\ 0 & -1 & 2   \end{smallmatrix} \right) \right)}{\det(A)}  =  \frac{5}{12},
\quad
x_2 = \frac{ \det \left( \left( \begin{smallmatrix} 3 & 1 & 1 \\ 1 & 0 & 1 \\ 1 & 0 & 2   \end{smallmatrix}  \right) \right)}{\det(A)} = \frac{-1}{12},
\quad
x_3 = \frac{ \det \left( \left( \begin{smallmatrix} 3 & 0 & 1 \\ 1 & 2 & 0 \\ 1 & -1 & 0   \end{smallmatrix} \right) \right)}{\det(A)} = \frac{-3}{12}, 
\]
\[
y_1 = \frac{\det \left( \left( \begin{smallmatrix} 0 & 0 & 1 \\ 1 & 2 & 1 \\ 0 & -1 & 2   \end{smallmatrix} \right) \right)}{\det(A)}  = \frac{-1}{12} ,
\quad
y_2 = \frac{ \det \left( \left( \begin{smallmatrix} 3 & 0 & 1 \\ 1 & 1 & 1 \\ 1 & 0 & 2   \end{smallmatrix}  \right) \right)}{\det(A)} = \frac{5}{12},
\quad
y_3 = \frac{ \det \left( \left( \begin{smallmatrix} 3 & 0 & 0 \\ 1 & 2 & 1 \\ 1 & -1 & 0   \end{smallmatrix} \right) \right)}{\det(A)} =  \frac{3}{12}, 
\]
\[
z_1 = \frac{\det \left( \left( \begin{smallmatrix} 0 & 0 & 1 \\ 0 & 2 & 1 \\ 1 & -1 & 2   \end{smallmatrix} \right) \right)}{\det(A)}  =\frac{-2}{12} ,
\quad
z_2 = \frac{ \det \left( \left( \begin{smallmatrix} 3 & 0 & 1 \\ 1 & 0 & 1 \\ 1 & 1 & 2   \end{smallmatrix}  \right) \right)}{\det(A)} = \frac{-2}{12},
\quad
z_3 = \frac{ \det \left( \left( \begin{smallmatrix} 3 & 0 & 0 \\ 1 & 2 & 0 \\ 1 & -1 & 1   \end{smallmatrix} \right) \right)}{\det(A)} =  \frac{6}{12}, 
\]
So $A^{-1}$ is
\[
A^{-1} = \frac{1}{12} \begin{pmatrix} 5 & -1 & -2 \\ -1 & 5 & -2 \\ -3 & 3 & 6 \end{pmatrix}.
\]
Let us check the calculation:
\[
A \cdot A^{-1} = \begin{pmatrix} 3 & 0 & 1 \\ 1 & 2 & 1 \\ 1 & -1 & 2 \end{pmatrix} \cdot \frac{1}{12} \begin{pmatrix} 5 & -1 & -2 \\ -1 & 5 & -2 \\ -3 & 3 & 6 \end{pmatrix}
               = \frac{1}{12} \begin{pmatrix} 12 & 0 & 0 \\ 0 & 12 & 0 \\ 0 & 0 & 12 \end{pmatrix} = \begin{pmatrix} 1 & 0 & 0 \\ 0 & 1 & 0 \\ 0 & 0 & 1 \end{pmatrix}.
\]
    }
  	 %------------------------------------END_STEP_X
 
  \end{incremental}
  %++++++++++++++++++++++++++++++++++++++++++++END_TAB_X


  %++++++++++++++++++++++++++++++++++++++++++START_TAB_X
  \tab{\lang{de}{    Lösung zu $B$    } \lang{en}{Solution for $B$}}
  \begin{incremental}[\initialsteps{1}]
  
  	 %----------------------------------START_STEP_X
    \step 
    \lang{de}{   

Betrachten wir nun die Matrix $B$. Wir berechnen
\[
\det(B)=12 + 5 -4 -8 +3 -10=-2.
\]
Nun l\"osen wir mit Hilfe der Cramerschen Regel die folgenden $3$ Gleichungssysteme
\[
B x= \begin{pmatrix} 1 \\ 0 \\ 0 \end{pmatrix}, \quad B y= \begin{pmatrix} 0 \\ 1 \\ 0 \end{pmatrix}, \quad  B z= \begin{pmatrix} 0 \\ 0 \\ 1 \end{pmatrix}. 
\]
Beachten wir, dass $\det(B)=-2$ gilt, so berechnen wir weiter
\[
x_1 = \frac{\det \left( \left( \begin{smallmatrix} 1 & 1 & 2 \\ 0 & 4 & 5 \\ 0 & 2 & 3   \end{smallmatrix} \right) \right)}{\det(B)}  =  \frac{2}{-2},
\quad
x_2 = \frac{ \det \left( \left( \begin{smallmatrix} 1 & 1 & 2 \\ -1 & 0 & 5 \\ 1 & 0 & 3   \end{smallmatrix}  \right) \right)}{\det(B)} = \frac{8}{-2},
\quad
x_3 = \frac{ \det \left( \left( \begin{smallmatrix} 1 & 1 & 1 \\ -1 & 4 & 0 \\ 1 & 2 & 0   \end{smallmatrix} \right) \right)}{\det(B)} = \frac{-6}{-2}, 
\]
\[
y_1 = \frac{\det \left( \left( \begin{smallmatrix} 0 & 1 & 2 \\ 1 & 4 & 5 \\ 0 & 2 & 3   \end{smallmatrix} \right) \right)}{\det(B)}  = \frac{1}{-2} ,
\quad
y_2 = \frac{ \det \left( \left( \begin{smallmatrix} 1 & 0 & 2 \\ -1 & 1 & 5 \\ 1 & 0 & 3   \end{smallmatrix}  \right) \right)}{\det(B)} = \frac{1}{-2},
\quad
y_3 = \frac{ \det \left( \left( \begin{smallmatrix} 1 & 1 & 0 \\ -1 & 4 & 1 \\ 1 & 2 & 0   \end{smallmatrix} \right) \right)}{\det(B)} =  \frac{-1}{-2}, 
\]
\[
z_1 = \frac{\det \left( \left( \begin{smallmatrix} 0 & 1 & 2 \\ 0 & 4 & 5 \\ 1 & 2 & 3   \end{smallmatrix} \right) \right)}{\det(B)}  =\frac{-3}{-2} ,
\quad
z_2 = \frac{ \det \left( \left( \begin{smallmatrix} 1 & 0 & 2 \\ -1 & 0 & 5 \\ 1 & 1 & 3   \end{smallmatrix}  \right) \right)}{\det(B)} = \frac{-7}{-2},
\quad
z_3 = \frac{ \det \left( \left( \begin{smallmatrix} 1 & 1 & 0 \\ -1 & 4 & 0 \\ 1 & 2 & 1   \end{smallmatrix} \right) \right)}{\det(B)} =  \frac{5}{-2}, 
\]
Damit ist $B^{-1}$ gegeben durch
\[
B^{-1} = \frac{1}{-2} \begin{pmatrix} 2 & 1 & -3 \\ 8 & 1 & -7 \\ -6 & -1 & 5 \end{pmatrix}.
\]
Wir machen noch einmal die Probe:
\[
B \cdot B^{-1} = \begin{pmatrix} 1 & 1 & 2 \\ -1 & 4 & 5 \\ 1 & 2 & 3 \end{pmatrix} \cdot \frac{1}{-2} \begin{pmatrix} 2 & 1 & -3 \\ 8 & 1 & -7 \\ -6 & -1 & 5 \end{pmatrix}
               = -\frac{1}{2} \begin{pmatrix} -2 & 0 & 0 \\ 0 & -2 & 0 \\ 0 & 0 & -2 \end{pmatrix} = \begin{pmatrix} 1 & 0 & 0 \\ 0 & 1 & 0 \\ 0 & 0 & 1 \end{pmatrix}.
\]
    }


        \lang{en}{   
Now we consider the matrix $B$. We calculate the determinant
\[
\det(B)=12 + 5 -4 -8 +3 -10=-2.
\]
Now we solve the following $3$ linear systems with the help of Cramer's rule
\[
B x= \begin{pmatrix} 1 \\ 0 \\ 0 \end{pmatrix}, \quad B y= \begin{pmatrix} 0 \\ 1 \\ 0 \end{pmatrix}, \quad  B z= \begin{pmatrix} 0 \\ 0 \\ 1 \end{pmatrix}. 
\]
With keeping in mind, that $\det(B)=-2$ we continue calculating
\[
x_1 = \frac{\det \left( \left( \begin{smallmatrix} 1 & 1 & 2 \\ 0 & 4 & 5 \\ 0 & 2 & 3   \end{smallmatrix} \right) \right)}{\det(B)}  =  \frac{2}{-2},
\quad
x_2 = \frac{ \det \left( \left( \begin{smallmatrix} 1 & 1 & 2 \\ -1 & 0 & 5 \\ 1 & 0 & 3   \end{smallmatrix}  \right) \right)}{\det(B)} = \frac{8}{-2},
\quad
x_3 = \frac{ \det \left( \left( \begin{smallmatrix} 1 & 1 & 1 \\ -1 & 4 & 0 \\ 1 & 2 & 0   \end{smallmatrix} \right) \right)}{\det(B)} = \frac{-6}{-2}, 
\]
\[
y_1 = \frac{\det \left( \left( \begin{smallmatrix} 0 & 1 & 2 \\ 1 & 4 & 5 \\ 0 & 2 & 3   \end{smallmatrix} \right) \right)}{\det(B)}  = \frac{1}{-2} ,
\quad
y_2 = \frac{ \det \left( \left( \begin{smallmatrix} 1 & 0 & 2 \\ -1 & 1 & 5 \\ 1 & 0 & 3   \end{smallmatrix}  \right) \right)}{\det(B)} = \frac{1}{-2},
\quad
y_3 = \frac{ \det \left( \left( \begin{smallmatrix} 1 & 1 & 0 \\ -1 & 4 & 1 \\ 1 & 2 & 0   \end{smallmatrix} \right) \right)}{\det(B)} =  \frac{-1}{-2}, 
\]
\[
z_1 = \frac{\det \left( \left( \begin{smallmatrix} 0 & 1 & 2 \\ 0 & 4 & 5 \\ 1 & 2 & 3   \end{smallmatrix} \right) \right)}{\det(B)}  =\frac{-3}{-2} ,
\quad
z_2 = \frac{ \det \left( \left( \begin{smallmatrix} 1 & 0 & 2 \\ -1 & 0 & 5 \\ 1 & 1 & 3   \end{smallmatrix}  \right) \right)}{\det(B)} = \frac{-7}{-2},
\quad
z_3 = \frac{ \det \left( \left( \begin{smallmatrix} 1 & 1 & 0 \\ -1 & 4 & 0 \\ 1 & 2 & 1   \end{smallmatrix} \right) \right)}{\det(B)} =  \frac{5}{-2}, 
\]
So $B^{-1}$ is
\[
B^{-1} = \frac{1}{-2} \begin{pmatrix} 2 & 1 & -3 \\ 8 & 1 & -7 \\ -6 & -1 & 5 \end{pmatrix}.
\]
We check our calculations:
\[
B \cdot B^{-1} = \begin{pmatrix} 1 & 1 & 2 \\ -1 & 4 & 5 \\ 1 & 2 & 3 \end{pmatrix} \cdot \frac{1}{-2} \begin{pmatrix} 2 & 1 & -3 \\ 8 & 1 & -7 \\ -6 & -1 & 5 \end{pmatrix}
               = -\frac{1}{2} \begin{pmatrix} -2 & 0 & 0 \\ 0 & -2 & 0 \\ 0 & 0 & -2 \end{pmatrix} = \begin{pmatrix} 1 & 0 & 0 \\ 0 & 1 & 0 \\ 0 & 0 & 1 \end{pmatrix}.
\]
    }
  	 %------------------------------------END_STEP_X
 
  \end{incremental}
  %++++++++++++++++++++++++++++++++++++++++++++END_TAB_X


%#############################################################ENDE



\end{tabs*}
\end{content}