\documentclass{mumie.element.exercise}
%$Id$
\begin{metainfo}
  \name{
    \lang{de}{Ü03: Inverse Matrix}
    \lang{en}{Ex03: The inverse matrix}
  }
  \begin{description} 
 This work is licensed under the Creative Commons License Attribution 4.0 International (CC-BY 4.0)   
 https://creativecommons.org/licenses/by/4.0/legalcode 

    \lang{de}{}
    \lang{en}{}
  \end{description}
  \begin{components}
  \end{components}
  \begin{links}
  \end{links}
  \creategeneric
\end{metainfo}
\begin{content}
\usepackage{mumie.ombplus}

\title{\lang{de}{Ü03: Inverse Matrix} \lang{en}{Ex03: The inverse matrix}}

\begin{block}[annotation]
  Im Ticket-System: \href{http://team.mumie.net/issues/11358}{Ticket 11358}
\end{block}

%######################################################FRAGE_TEXT
\lang{de}{ 
Wir betrachten die Matrix $A=\left(\begin{smallmatrix}  1 & 0 & 0 \\ 2 & 1 & 1 \\ -1 & 1 & 2 \end{smallmatrix}\right) $.
Welche der folgenden Matrizen ist die inverse Matrix zu $A$?

\[  \text{a) } B_1=\left(\begin{smallmatrix}  1 & 2 & -1 \\ 0 & 1 & 1 \\ 0 & 1 & 2 \end{smallmatrix}\right), \quad
 \text{b) } B_2=\left(\begin{smallmatrix}  1 & 0 & 1 \\ -1 & 1 & 0 \\ -1 & 0 & 1 \end{smallmatrix}\right), \quad
 \text{c) } B_3=\left(\begin{smallmatrix}  1 & 0 & 0 \\ -5 & 2 & -1 \\ 3 & -1 & 1 \end{smallmatrix}\right)  \]
 }


 \lang{en}{ 
We consider the matrix $A=\left(\begin{smallmatrix}  1 & 0 & 0 \\ 2 & 1 & 1 \\ -1 & 1 & 2 \end{smallmatrix}\right) $.
Which of the following matrices is the inverse matrix of $A$?

\[  \text{a) } B_1=\left(\begin{smallmatrix}  1 & 2 & -1 \\ 0 & 1 & 1 \\ 0 & 1 & 2 \end{smallmatrix}\right), \quad
 \text{b) } B_2=\left(\begin{smallmatrix}  1 & 0 & 1 \\ -1 & 1 & 0 \\ -1 & 0 & 1 \end{smallmatrix}\right), \quad
 \text{c) } B_3=\left(\begin{smallmatrix}  1 & 0 & 0 \\ -5 & 2 & -1 \\ 3 & -1 & 1 \end{smallmatrix}\right)  \]
 }

%##################################################ANTWORTEN_TEXT
\begin{tabs*}[\initialtab{0}\class{exercise}]

  %++++++++++++++++++++++++++++++++++++++++++START_TAB_X
  \tab{\lang{de}{    Antwort    } \lang{en}{ Answer }}
    \lang{de}{   $B_3=\left(\begin{smallmatrix} 1 & 0 & 0 \\ -5 & 2 & -1 \\ 3 & -1 & 1 \end{smallmatrix}\right)$ ist die inverse Matrix
zu $A$.    }
 \lang{en}{   $B_3=\left(\begin{smallmatrix} 1 & 0 & 0 \\ -5 & 2 & -1 \\ 3 & -1 & 1 \end{smallmatrix}\right)$ is the inverse matrix
of $A$.    }

  \tab{\lang{de}{    Lösung    }  \lang{en}{Solution}}
  \begin{incremental}[\initialsteps{1}]
  
  	 %----------------------------------START_STEP_X
    \step 
    \lang{de}{
Die inverse Matrix zu $A$ ist dadurch gekennzeichnet, dass das Produkt von $A$ mit ihr die Einheitsmatrix ergibt.
Um herauszufinden, ob eine der drei Matrizen die inverse Matrix zu $A$ ist, sind also lediglich die Produkte
$A\cdot B_1$, $A\cdot B_2$ und $A\cdot B_3$ zu berechnen und mit der Einheitsmatrix
\[ E_3=\left(\begin{smallmatrix} 1 & 0 & 0 \\ 0 & 1 & 0 \\ 0 & 0 & 1 \end{smallmatrix}\right) \]
zu vergleichen.}
 \lang{en}{
 The inverse matrix of $A$ is characterised by the fact that, the product of the inverse with $A$ equals the identity matrix.
 To check if one of the three matrices is the inverse matrix of $A$, we only need to calculate the products
$A\cdot B_1$, $A\cdot B_2$ and $A\cdot B_3$ and then compare with the identity matrix
\[ I_3=\left(\begin{smallmatrix} 1 & 0 & 0 \\ 0 & 1 & 0 \\ 0 & 0 & 1 \end{smallmatrix}\right) \]}


\step
\begin{eqnarray*}
A\cdot B_1 &=& \left(\begin{smallmatrix}  1 & 0 & 0 \\ 2 & 1 & 1 \\ -1 & 1 & 2 \end{smallmatrix}\right)
\cdot \left(\begin{smallmatrix}  1 & 2 & -1 \\ 0 & 1 & 1 \\ 0 & 1 & 2 \end{smallmatrix}\right)
=  \left(\begin{smallmatrix} 1 & 2 & -1 \\ 2 & 6 & 1 \\ -1 & 1 & 6 \end{smallmatrix}\right),\\
A\cdot B_2 &=&\left(\begin{smallmatrix}  1 & 0 & 0 \\ 2 & 1 & 1 \\ -1 & 1 & 2 \end{smallmatrix}\right)
\cdot \left(\begin{smallmatrix}  1 & 0 & 1 \\ -1 & 1 & 0 \\ -1 & 0 & 1  \end{smallmatrix}\right)
=  \left(\begin{smallmatrix} 1 & 0 & 1 \\ 0 & 1 & 3 \\ -4 & 1 & 1 \end{smallmatrix}\right), \\
A\cdot B_3 &=&\left(\begin{smallmatrix}  1 & 0 & 0 \\ 2 & 1 & 1 \\ -1 & 1 & 2 \end{smallmatrix}\right)
\cdot \left(\begin{smallmatrix} 1 & 0 & 0 \\ -5 & 2 & -1 \\ 3 & -1 & 1 \end{smallmatrix}\right)
= \left(\begin{smallmatrix} 1 & 0 & 0 \\ 0 & 1 & 0 \\ 0 & 0 & 1 \end{smallmatrix}\right).
\end{eqnarray*}

\lang{de}{Also ist $B_3$ die inverse Matrix zu $A$.}
\lang{en}{Therefore $B_3$ is the inverse matrix of $A$.}
    
  	 %------------------------------------END_STEP_X
 
  \end{incremental}
  %++++++++++++++++++++++++++++++++++++++++++++END_TAB_X


%#############################################################ENDE
\end{tabs*}
\end{content}