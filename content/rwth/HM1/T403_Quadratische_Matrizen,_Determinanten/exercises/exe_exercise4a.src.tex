\documentclass{mumie.element.exercise}
%$Id$
\begin{metainfo}
  \name{
    \lang{en}{Ü05: The inverse Matrix}
    \lang{de}{Ü05: Inverse Matrix}
  }
  \begin{description} 
 This work is licensed under the Creative Commons License Attribution 4.0 International (CC-BY 4.0)   
 https://creativecommons.org/licenses/by/4.0/legalcode 

    \lang{en}{}
    \lang{de}{}
  \end{description}
  \begin{components}
  \end{components}
  \begin{links}
  \end{links}
  \creategeneric
\end{metainfo}
\begin{content}
\begin{block}[annotation]
	Im Ticket-System: \href{https://team.mumie.net/issues/28417}{Ticket 28417}
\end{block}

\usepackage{mumie.ombplus}

\title{\lang{de}{Ü05: Inverse Matrix} \lang{en}{Ex05: The inverse matrix}}

\lang{de}{
a) Für welche Werte $c \in \R$ ist die folgende Matrix invertierbar?
\begin{align*}
A=
\left( \begin{smallmatrix}
1 & 3 & 0 \\
0 & 1 & 2 \\
-1 & -2 & c 
\end{smallmatrix} \right)
\end{align*}

b) Wie lautet die Inverse $A^{-1}$?}

\lang{en}{
a) For which values $c \in \R$ is the following matrix invertible?
\begin{align*}
A=
\left( \begin{smallmatrix}
1 & 3 & 0 \\
0 & 1 & 2 \\
-1 & -2 & c 
\end{smallmatrix} \right)
\end{align*}

b) What is the inverse matrix $A^{-1}$?}

%##################################################ANTWORTEN_TEXT
\begin{tabs*}[\initialtab{0}\class{exercise}]

  %++++++++++++++++++++++++++++++++++++++++++START_TAB_X
  \tab{\lang{de}{   Antwort   } \lang{en}{Answer}}
  %\begin{incremental}[\initialsteps{1}]
  
  	 %----------------------------------START_STEP_X
    %\step 
    \lang{de}{
a) 
        
\begin{align*}
    c \neq 2
\end{align*}


b) 
        
\begin{align*}
A^{-1}=
\frac{1}{c-2}
\left( \begin{smallmatrix}
c+4 & -3c & 6 \\
-2 & c & -2 \\
1 & -1 & 1 
\end{smallmatrix} \right)
\end{align*}
        
    }


      \lang{en}{
a) 
        
\begin{align*}
    c \neq 2
\end{align*}


b) 
        
\begin{align*}
A^{-1}=
\frac{1}{c-2}
\left( \begin{smallmatrix}
c+4 & -3c & 6 \\
-2 & c & -2 \\
1 & -1 & 1 
\end{smallmatrix} \right)
\end{align*}
        
    }
  	 %------------------------------------END_STEP_X
 
  %\end{incremental}
  %++++++++++++++++++++++++++++++++++++++++++START_TAB_X
  \tab{\lang{de}{    Lösungsvideo    }}
  %\begin{incremental}[\initialsteps{1}]
  
  	 %----------------------------------START_STEP_X
    %\step 
    \lang{de}{   

        \youtubevideo[500][300]{jbJK9fbMf8M}\\

    }
  	 %------------------------------------END_STEP_X
 
  %\end{incremental}
  %++++++++++++++++++++++++++++++++++++++++++++END_TAB_X


\end{tabs*}
%#############################################################ENDE


\end{content}
