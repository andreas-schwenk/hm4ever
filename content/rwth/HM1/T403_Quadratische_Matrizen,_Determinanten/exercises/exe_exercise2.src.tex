\documentclass{mumie.element.exercise}
%$Id$
\begin{metainfo}
  \name{
    \lang{de}{Ü02: Matrixoperationen}
    \lang{en}{Ex02: Matrix operations}
  }
  \begin{description} 
 This work is licensed under the Creative Commons License Attribution 4.0 International (CC-BY 4.0)   
 https://creativecommons.org/licenses/by/4.0/legalcode 

    \lang{de}{}
    \lang{en}{}
  \end{description}
  \begin{components}
  \end{components}
  \begin{links}
  \end{links}
  \creategeneric
\end{metainfo}
\begin{content}
\usepackage{mumie.ombplus}

\title{\lang{de}{Ü02: Matrixoperationen} \lang{en}{Ex02: Matrix operations}}

\begin{block}[annotation]
  Im Ticket-System: \href{http://team.mumie.net/issues/11357}{Ticket 11357}
\end{block}

%######################################################FRAGE_TEXT
\lang{de}{ 
Bestimmen Sie für die Matrizen $A=\left(\begin{smallmatrix}  -2 & 0 & 0 \\ 0 & 0 & 1 \\ 0 & 0 & 2 \end{smallmatrix}\right) $ und $B=\left(\begin{smallmatrix}  1 & 1 & 0 \\ 0 & 0 & 0 \\ -1 & 1 & 3 \end{smallmatrix}\right) $
die folgenden Matrizen:
\[ A^2, \quad A\cdot B, \quad B\cdot A,  \quad B^2\quad \text{und}\quad A+B.\]
Bestimmen Sie weiter die Matrizen
\[ (A+B)^2\quad \text{und} \quad A^2+2\cdot A\cdot B+B^2.\]
Warum sind die letzten beiden Matrizen verschieden?

Hierbei ist $A^2$ etc. die übliche Kurzschreibweise für $A\cdot A$ etc. }



\lang{en}{ 
Determine for the matrices $A=\left(\begin{smallmatrix}  -2 & 0 & 0 \\ 0 & 0 & 1 \\ 0 & 0 & 2 \end{smallmatrix}\right) $ and $B=\left(\begin{smallmatrix}  1 & 1 & 0 \\ 0 & 0 & 0 \\ -1 & 1 & 3 \end{smallmatrix}\right) $
the following matrices:
\[ A^2, \quad A\cdot B, \quad B\cdot A,  \quad B^2\quad \text{und}\quad A+B.\]
Determine also the matrices:
\[ (A+B)^2\quad \text{ and } \quad A^2+2\cdot A\cdot B+B^2.\]
Why are the last two matrices different?

 $A^2$ is the usual notation for für $A\cdot A$. }

%##################################################ANTWORTEN_TEXT
\begin{tabs*}[\initialtab{0}\class{exercise}]

  %++++++++++++++++++++++++++++++++++++++++++START_TAB_X
  \tab{\lang{de}{    Lösung  } \lang{en}{Solution}}
  \begin{incremental}[\initialsteps{1}]
  
  	 %----------------------------------START_STEP_X
    \step 
    \lang{de}{   
Die Matrizen berechnet man mit der übliche Matrizen-Multiplikation und Addition
\begin{eqnarray*}
A^2 &=& \left(\begin{smallmatrix} -2 & 0 & 0 \\ 0 & 0 & 1 \\ 0 & 0 & 2 \end{smallmatrix}\right)\cdot \left(\begin{smallmatrix} -2 & 0 & 0 \\ 0 & 0 & 1 \\ 0 & 0 & 2 \end{smallmatrix}\right) \\
&=& \left(\begin{smallmatrix} 
(-2)\cdot (-2)+0\cdot 0+0\cdot 0 & (-2)\cdot 0+0\cdot 0+0\cdot 0 & (-2)\cdot 0+0\cdot 1+0\cdot 2 \\ 
0\cdot (-2)+0\cdot 0+1\cdot 0 & 0\cdot 0+0\cdot 0+1\cdot 0 & 0\cdot 0+0\cdot 1+1\cdot 2 \\ 
0\cdot (-2)+0\cdot 0+2\cdot 0 & 0\cdot 0+0\cdot 0+2\cdot 0 & 0\cdot 0+0\cdot 1+2\cdot 2 
 \end{smallmatrix}\right) \\
&=& \left(\begin{smallmatrix} 4&0 &0 \\ 0& 0& 2\\ 0& 0& 4 \end{smallmatrix}\right),  \\
A\cdot B &=&\left(\begin{smallmatrix} -2 & 0 & 0 \\ 0 & 0 & 1 \\ 0 & 0 & 2 \end{smallmatrix}\right)\cdot
\left(\begin{smallmatrix}  1 & 1 & 0 \\ 0 & 0 & 0 \\ -1 & 1 & 3 \end{smallmatrix}\right) 
= \left(\begin{smallmatrix} -2& -2&0 \\ -1& 1& 3\\ -2& 2& 6 \end{smallmatrix}\right),  \\
B\cdot A &=& \left(\begin{smallmatrix} -2& 0& 1\\ 0& 0&0 \\ 2&0 &7  \end{smallmatrix}\right),  \\
B^2 &=& \left(\begin{smallmatrix} 1&1 &0 \\ 0&0 &0 \\ -4&2 &9  \end{smallmatrix}\right),  \\
A+B &=& \left(\begin{smallmatrix} -2 & 0 & 0 \\ 0 & 0 & 1 \\ 0 & 0 & 2 \end{smallmatrix}\right)+
\left(\begin{smallmatrix}  1 & 1 & 0 \\ 0 & 0 & 0 \\ -1 & 1 & 3 \end{smallmatrix}\right) \\
&=& \left(\begin{smallmatrix} -2+1 & 0+1 & 0+0 \\ 0+0 & 0+0 & 1+0 \\ 0-1 & 0+1 & 2+3 \end{smallmatrix}\right)
=\left(\begin{smallmatrix} -1&1 &0 \\ 0&0 &1 \\ -1&1 &5  \end{smallmatrix}\right).
\end{eqnarray*}
}  

  \lang{en}{   
We calculate the matrices with the usual matrix multiplication and addition
\begin{eqnarray*}
A^2 &=& \left(\begin{smallmatrix} -2 & 0 & 0 \\ 0 & 0 & 1 \\ 0 & 0 & 2 \end{smallmatrix}\right)\cdot \left(\begin{smallmatrix} -2 & 0 & 0 \\ 0 & 0 & 1 \\ 0 & 0 & 2 \end{smallmatrix}\right) \\
&=& \left(\begin{smallmatrix} 
(-2)\cdot (-2)+0\cdot 0+0\cdot 0 & (-2)\cdot 0+0\cdot 0+0\cdot 0 & (-2)\cdot 0+0\cdot 1+0\cdot 2 \\ 
0\cdot (-2)+0\cdot 0+1\cdot 0 & 0\cdot 0+0\cdot 0+1\cdot 0 & 0\cdot 0+0\cdot 1+1\cdot 2 \\ 
0\cdot (-2)+0\cdot 0+2\cdot 0 & 0\cdot 0+0\cdot 0+2\cdot 0 & 0\cdot 0+0\cdot 1+2\cdot 2 
 \end{smallmatrix}\right) \\
&=& \left(\begin{smallmatrix} 4&0 &0 \\ 0& 0& 2\\ 0& 0& 4 \end{smallmatrix}\right),  \\
A\cdot B &=&\left(\begin{smallmatrix} -2 & 0 & 0 \\ 0 & 0 & 1 \\ 0 & 0 & 2 \end{smallmatrix}\right)\cdot
\left(\begin{smallmatrix}  1 & 1 & 0 \\ 0 & 0 & 0 \\ -1 & 1 & 3 \end{smallmatrix}\right) 
= \left(\begin{smallmatrix} -2& -2&0 \\ -1& 1& 3\\ -2& 2& 6 \end{smallmatrix}\right),  \\
B\cdot A &=& \left(\begin{smallmatrix} -2& 0& 1\\ 0& 0&0 \\ 2&0 &7  \end{smallmatrix}\right),  \\
B^2 &=& \left(\begin{smallmatrix} 1&1 &0 \\ 0&0 &0 \\ -4&2 &9  \end{smallmatrix}\right),  \\
A+B &=& \left(\begin{smallmatrix} -2 & 0 & 0 \\ 0 & 0 & 1 \\ 0 & 0 & 2 \end{smallmatrix}\right)+
\left(\begin{smallmatrix}  1 & 1 & 0 \\ 0 & 0 & 0 \\ -1 & 1 & 3 \end{smallmatrix}\right) \\
&=& \left(\begin{smallmatrix} -2+1 & 0+1 & 0+0 \\ 0+0 & 0+0 & 1+0 \\ 0-1 & 0+1 & 2+3 \end{smallmatrix}\right)
=\left(\begin{smallmatrix} -1&1 &0 \\ 0&0 &1 \\ -1&1 &5  \end{smallmatrix}\right).
\end{eqnarray*}
}


\step 
    \lang{de}{  
Damit berechnet man dann weiter:
\begin{eqnarray*}
(A+B)^2 &=& \left(\begin{smallmatrix} -1&1 &0 \\ 0&0 &1 \\ -1&1 &5  \end{smallmatrix}\right)\cdot\left(\begin{smallmatrix} -1&1 &0 \\ 0&0 &1 \\ -1&1 &5  \end{smallmatrix}\right) 
= \left(\begin{smallmatrix} 1& -1& 1\\ -1&1 &5 \\ -4&4 &26  \end{smallmatrix}\right),  \\
A^2+2\cdot A\cdot B+B^2  &=& \left(\begin{smallmatrix} 4&0 &0 \\ 0& 0& 2\\ 0& 0& 4 \end{smallmatrix}\right)
+ 2\cdot  \left(\begin{smallmatrix} -2& -2&0 \\ -1& 1& 3\\ -2& 2& 6 \end{smallmatrix}\right)
+  \left(\begin{smallmatrix} 1&1 &0 \\ 0&0 &0 \\ -4&2 &9  \end{smallmatrix}\right) \\
&=& \left(\begin{smallmatrix} 1& -3&0 \\ -2&2 &8 \\ -8&6 &25  \end{smallmatrix}\right).
\end{eqnarray*}
}  


  \lang{en}{  
We continue to calculate with that:
\begin{eqnarray*}
(A+B)^2 &=& \left(\begin{smallmatrix} -1&1 &0 \\ 0&0 &1 \\ -1&1 &5  \end{smallmatrix}\right)\cdot\left(\begin{smallmatrix} -1&1 &0 \\ 0&0 &1 \\ -1&1 &5  \end{smallmatrix}\right) 
= \left(\begin{smallmatrix} 1& -1& 1\\ -1&1 &5 \\ -4&4 &26  \end{smallmatrix}\right),  \\
A^2+2\cdot A\cdot B+B^2  &=& \left(\begin{smallmatrix} 4&0 &0 \\ 0& 0& 2\\ 0& 0& 4 \end{smallmatrix}\right)
+ 2\cdot  \left(\begin{smallmatrix} -2& -2&0 \\ -1& 1& 3\\ -2& 2& 6 \end{smallmatrix}\right)
+  \left(\begin{smallmatrix} 1&1 &0 \\ 0&0 &0 \\ -4&2 &9  \end{smallmatrix}\right) \\
&=& \left(\begin{smallmatrix} 1& -3&0 \\ -2&2 &8 \\ -8&6 &25  \end{smallmatrix}\right).
\end{eqnarray*}
}    


\step 
    \lang{de}{  
Die letzten beiden Matrizen sind verschieden, weil die binomischen Formeln für Matrizen nicht gelten!
Wendet man nämlich auf $(A+B)^2$ die Distributivgesetze an, erhält man:

\begin{eqnarray*}
(A+B)^2 &=& (A+B)\cdot  (A+B) \\
&=& (A+B)\cdot A \quad +\quad (A+B)\cdot B \\
&=&  \, A\cdot A + B\cdot A \ +\ A\cdot B+ B\cdot B \ .
\end{eqnarray*}
Der Unterschied zwischen $A^2+2\cdot A\cdot B+B^2$ und $(A+B)^2$ besteht also darin, dass  
$2A \cdot B \neq A \cdot B + B\cdot A$ ist. Dies wiederum haben wir bereits eingesehen: 
Die Matrix-Multiplikation ist im Allgemeinen nicht kommutativ, d.h. $A \cdot B \neq B \cdot A$.
    }

\lang{en}{ 
The last two matrices are different, because the binomial formulas do not apply for matrices!
We we use the distributive property on  $(A+B)^2$, we get:
\begin{eqnarray*}
(A+B)^2 &=& (A+B)\cdot  (A+B) \\
&=& (A+B)\cdot A \quad +\quad (A+B)\cdot B \\
&=&  \, A\cdot A + B\cdot A \ +\ A\cdot B+ B\cdot B \ .
\end{eqnarray*}
The difference between $A^2+2\cdot A\cdot B+B^2$ and $(A+B)^2$ is, that $2A \cdot B \neq A \cdot B + B\cdot A$ is.
This we already discussed:
In general matrix multiplication is not commutative, so $A \cdot B \neq B \cdot A$.
    }
  	 %------------------------------------END_STEP_X
 
 
  \end{incremental}
  %++++++++++++++++++++++++++++++++++++++++++++END_TAB_X


%#############################################################ENDE
\end{tabs*}
\end{content}