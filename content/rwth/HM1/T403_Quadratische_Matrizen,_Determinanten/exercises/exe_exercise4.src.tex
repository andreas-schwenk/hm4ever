\documentclass{mumie.element.exercise}
%$Id$
\begin{metainfo}
  \name{
    \lang{de}{Ü04: Inverse Matrix}
    \lang{en}{Ex04: The inverse matrix}
  }
  \begin{description} 
 This work is licensed under the Creative Commons License Attribution 4.0 International (CC-BY 4.0)   
 https://creativecommons.org/licenses/by/4.0/legalcode 

    \lang{de}{}
    \lang{en}{}
  \end{description}
  \begin{components}
  \end{components}
  \begin{links}
  \end{links}
  \creategeneric
\end{metainfo}
\begin{content}
\usepackage{mumie.ombplus}

\title{\lang{de}{Ü04: Inverse Matrix}  \lang{en}{Ex04: The inverse matrix}}

\begin{block}[annotation]
  Im Ticket-System: \href{http://team.mumie.net/issues/11359}{Ticket 11359}
\end{block}

%######################################################FRAGE_TEXT
\lang{de}{ 
Untersuchen Sie jede der folgenden Matrizen auf Invertierbarkeit und
bestimmen Sie gegebenenfalls die inverse Matrix.
\begin{align*}
\text{a) }
\left( \begin{smallmatrix}
0 & 0 & 1 & 0 \\
0 & -1 & 0 & -3 \\
1 & 2 & 0 & 6 \\
0 & 0 & 0 & 1 
\end{smallmatrix} \right), \qquad
\text{b) }
\left( \begin{smallmatrix}
3 & 4 \\
-1 & 2 
\end{smallmatrix} \right) \\
\text{c) }
\left( \begin{smallmatrix}
2 & 3 & -4 \\
3 & 3 & -1 \\
0 & 3 & -10 
\end{smallmatrix} \right), \qquad
\text{d) }
\left( \begin{smallmatrix}
2 & -1 & 3 \\
7 & 3 & 0 \\
-1 & 2 & -4 
\end{smallmatrix} \right).
\end{align*}
}

\lang{en}{ 
Examine each of the following matrices for invertibility and determine if applicable the inverse matrix.
\begin{align*}
\text{a) }
\left( \begin{smallmatrix}
0 & 0 & 1 & 0 \\
0 & -1 & 0 & -3 \\
1 & 2 & 0 & 6 \\
0 & 0 & 0 & 1 
\end{smallmatrix} \right), \qquad
\text{b) }
\left( \begin{smallmatrix}
3 & 4 \\
-1 & 2 
\end{smallmatrix} \right) \\
\text{c) }
\left( \begin{smallmatrix}
2 & 3 & -4 \\
3 & 3 & -1 \\
0 & 3 & -10 
\end{smallmatrix} \right), \qquad
\text{d) }
\left( \begin{smallmatrix}
2 & -1 & 3 \\
7 & 3 & 0 \\
-1 & 2 & -4 
\end{smallmatrix} \right).
\end{align*}
}

%##################################################ANTWORTEN_TEXT
\begin{tabs*}[\initialtab{0}\class{exercise}]

  %++++++++++++++++++++++++++++++++++++++++++START_TAB_X
  \tab{\lang{de}{   Antwort   } \lang{en}{Answer}}
  \begin{incremental}[\initialsteps{1}]
  
  	 %----------------------------------START_STEP_X
    \step 
    \lang{de}{   Die Matrix in c) ist nicht invertierbar.
Die Inversen der übrigen Matrizen sind:
\[
\text{a) } \left( \begin{smallmatrix}0 & 2 & 1 & 0 \\ 0 & -1 & 0 & -3 \\ 1 & 0 & 0 & 0 \\ 0 & 0 & 0 & 1 \end{smallmatrix} \right), \quad
\text{b) } \left( \begin{smallmatrix} \frac{1}{5} & -\frac{2}{5} \\ \frac{1}{10} & \frac{3}{10} \end{smallmatrix} \right), \quad
\text{d) } \left( \begin{smallmatrix} 12 & -2 & 9 \\ -28 & 5 & -21 \\ -17 & 3 & -13  \end{smallmatrix} \right).
\]
    }
\lang{en}{   The matrix in c) is not invertible.
The inverse matrices of the other matrices are:
\[
\text{a) } \left( \begin{smallmatrix}0 & 2 & 1 & 0 \\ 0 & -1 & 0 & -3 \\ 1 & 0 & 0 & 0 \\ 0 & 0 & 0 & 1 \end{smallmatrix} \right), \quad
\text{b) } \left( \begin{smallmatrix} \frac{1}{5} & -\frac{2}{5} \\ \frac{1}{10} & \frac{3}{10} \end{smallmatrix} \right), \quad
\text{d) } \left( \begin{smallmatrix} 12 & -2 & 9 \\ -28 & 5 & -21 \\ -17 & 3 & -13  \end{smallmatrix} \right).
\]
    }

  	 %------------------------------------END_STEP_X
 
  \end{incremental}
  %++++++++++++++++++++++++++++++++++++++++++START_TAB_X
  \tab{\lang{de}{    Lösung a)    }   \lang{en}{Solution a)}}
  \begin{incremental}[\initialsteps{1}]
  
  	 %----------------------------------START_STEP_X
    \step 
    \lang{de}{   
Um Invertierbarkeit zu überprüfen und gegebenfalls die Inverse zu bestimmen, wendet man das Gauß-Verfahren auf 
die um die Einheitsmatrix erweiterte Matrix an:
\begin{eqnarray*}
&& \begin{pmatrix} 
0 & 0 & 1 & 0 &|&  1 & 0 & 0 & 0\\
0 & -1 & 0 & -3 &|&  0 & 1 & 0 & 0\\
1 & 2 & 0 & 6 &|&  0 & 0 & 1& 0\\
0 & 0 & 0 & 1  &|&  0 & 0 & 0& 1
\end{pmatrix} \begin{matrix} \uparrow \\  \downarrow \\   \phantom{1}\end{matrix} \\
& \rightsquigarrow & 
 \begin{pmatrix} 
1 & 2 & 0 & 6 &|&  0 & 0 & 1& 0\\
0 & -1 & 0 & -3 &|&  0 & 1 & 0 & 0\\
0 & 0 & 1 & 0 &|&  1 & 0 & 0 & 0\\
0 & 0 & 0 & 1  &|&  0 & 0 & 0& 1
\end{pmatrix} \begin{matrix}/  -6\cdot \text{(IV)} \\  /  +3\cdot \text{(IV)} \\  \phantom{1} \\  \phantom{1} \end{matrix} \\
& \rightsquigarrow & 
 \begin{pmatrix} 
1 & 2 & 0 & 0 &|&  0 & 0 & 1& -6\\
0 & -1 & 0 & 0 &|&  0 & 1 & 0 & 3\\
0 & 0 & 1 & 0 &|&  1 & 0 & 0 & 0\\
0 & 0 & 0 & 1  &|&  0 & 0 & 0& 1
\end{pmatrix} \begin{matrix}/  +2\cdot \text{(II)} \\  \phantom{1}  \\  \phantom{1} \\  \phantom{1} \end{matrix} \\
& \rightsquigarrow & 
 \begin{pmatrix} 
1 & 0 & 0 & 0 &|&  0 & 2 & 1& 0\\
0 & -1 & 0 & 0 &|&  0 & 1 & 0 & 3\\
0 & 0 & 1 & 0 &|&  1 & 0 & 0 & 0\\
0 & 0 & 0 & 1  &|&  0 & 0 & 0& 1
\end{pmatrix} \begin{matrix} \phantom{1} \\ / \cdot (-1) \\  \phantom{1} \\  \phantom{1} \end{matrix} \\
& \rightsquigarrow & 
\begin{pmatrix} 
1 & 0 & 0 & 0 &|&  0 & 2 & 1& 0\\
0 & 1 & 0 & 0 &|&  0 & -1 & 0 & -3\\
0 & 0 & 1 & 0 &|&  1 & 0 & 0 & 0\\
0 & 0 & 0 & 1  &|&  0 & 0 & 0& 1
\end{pmatrix}
\end{eqnarray*}
Wir haben bei der reduzierten Zeilenstufenform auf der linken Seite die Einheitsmatrix erreicht. Daher ist die
ursprüngliche Matrix invertierbar und ihre inverse Matrix steht auf der rechten Seite:
\[ \left( \begin{smallmatrix}
0 & 0 & 1 & 0 \\
0 & -1 & 0 & -3 \\
1 & 2 & 0 & 6 \\
0 & 0 & 0 & 1 
\end{smallmatrix} \right)^{-1} = \left( \begin{smallmatrix}0 & 2 & 1 & 0 \\ 0 & -1 & 0 & -3 \\ 1 & 0 & 0 & 0 \\ 0 & 0 & 0 & 1 \end{smallmatrix} \right). \]
    }

\lang{en}{   
We use Gaussian elimination for the matrix augmented with the identity matrix. In doing so we check if
the matrix is invertible and if applicable we already determine the inverse:
\begin{eqnarray*}
&& \begin{pmatrix} 
0 & 0 & 1 & 0 &|&  1 & 0 & 0 & 0\\
0 & -1 & 0 & -3 &|&  0 & 1 & 0 & 0\\
1 & 2 & 0 & 6 &|&  0 & 0 & 1& 0\\
0 & 0 & 0 & 1  &|&  0 & 0 & 0& 1
\end{pmatrix} \begin{matrix} \uparrow \\  \downarrow \\   \phantom{1}\end{matrix} \\
& \rightsquigarrow & 
 \begin{pmatrix} 
1 & 2 & 0 & 6 &|&  0 & 0 & 1& 0\\
0 & -1 & 0 & -3 &|&  0 & 1 & 0 & 0\\
0 & 0 & 1 & 0 &|&  1 & 0 & 0 & 0\\
0 & 0 & 0 & 1  &|&  0 & 0 & 0& 1
\end{pmatrix} \begin{matrix}/  -6\cdot \text{(IV)} \\  /  +3\cdot \text{(IV)} \\  \phantom{1} \\  \phantom{1} \end{matrix} \\
& \rightsquigarrow & 
 \begin{pmatrix} 
1 & 2 & 0 & 0 &|&  0 & 0 & 1& -6\\
0 & -1 & 0 & 0 &|&  0 & 1 & 0 & 3\\
0 & 0 & 1 & 0 &|&  1 & 0 & 0 & 0\\
0 & 0 & 0 & 1  &|&  0 & 0 & 0& 1
\end{pmatrix} \begin{matrix}/  +2\cdot \text{(II)} \\  \phantom{1}  \\  \phantom{1} \\  \phantom{1} \end{matrix} \\
& \rightsquigarrow & 
 \begin{pmatrix} 
1 & 0 & 0 & 0 &|&  0 & 2 & 1& 0\\
0 & -1 & 0 & 0 &|&  0 & 1 & 0 & 3\\
0 & 0 & 1 & 0 &|&  1 & 0 & 0 & 0\\
0 & 0 & 0 & 1  &|&  0 & 0 & 0& 1
\end{pmatrix} \begin{matrix} \phantom{1} \\ / \cdot (-1) \\  \phantom{1} \\  \phantom{1} \end{matrix} \\
& \rightsquigarrow & 
\begin{pmatrix} 
1 & 0 & 0 & 0 &|&  0 & 2 & 1& 0\\
0 & 1 & 0 & 0 &|&  0 & -1 & 0 & -3\\
0 & 0 & 1 & 0 &|&  1 & 0 & 0 & 0\\
0 & 0 & 0 & 1  &|&  0 & 0 & 0& 1
\end{pmatrix}
\end{eqnarray*}
In the reduced row echelon form on the left side we have the identity matrix. Therefore the original matrix is invertible and
its inverse matrix is on the right side:
\[ \left( \begin{smallmatrix}
0 & 0 & 1 & 0 \\
0 & -1 & 0 & -3 \\
1 & 2 & 0 & 6 \\
0 & 0 & 0 & 1 
\end{smallmatrix} \right)^{-1} = \left( \begin{smallmatrix}0 & 2 & 1 & 0 \\ 0 & -1 & 0 & -3 \\ 1 & 0 & 0 & 0 \\ 0 & 0 & 0 & 1 \end{smallmatrix} \right). \]
    }
    
  	 %------------------------------------END_STEP_X
 
  \end{incremental}
  %++++++++++++++++++++++++++++++++++++++++++++END_TAB_X


  %++++++++++++++++++++++++++++++++++++++++++START_TAB_X
  \tab{\lang{de}{    Lösung b)    } \lang{en}{Solution b)}}
  \begin{incremental}[\initialsteps{1}]
  
  	 %----------------------------------START_STEP_X
    \step 
    \lang{de}{   
Auch hier wenden wir das Gauß-Verfahren auf 
die um die Einheitsmatrix erweiterte Matrix an:
\begin{eqnarray*}
&& \begin{pmatrix} 
3 & 4 &|&  1 & 0  \\
-1 & 2 &|&  0 & 1 
\end{pmatrix} \begin{matrix} \updownarrow \end{matrix} \\
& \rightsquigarrow & 
 \begin{pmatrix} 
-1 & 2 &|&  0 & 1 \\
3 & 4 &|&  1 & 0  
\end{pmatrix} \begin{matrix} \phantom{1}\\  /  +3\cdot \text{(I)} \end{matrix} \\
& \rightsquigarrow & 
 \begin{pmatrix} 
-1 & 2 &|&  0 & 1 \\
0 & 10 &|&  1 & 3  
\end{pmatrix} \begin{matrix}  /  \cdot (-1) \\  /  \cdot \frac{1}{10} \end{matrix} \\
& \rightsquigarrow & 
 \begin{pmatrix} 
1 & -2 &|&  0 & -1 \\
0 & 1 &|&  1/10 & 3/10  
\end{pmatrix} \begin{matrix} /  +2\cdot \text{(II)} \\ \phantom{1} \end{matrix} \\
& \rightsquigarrow & 
 \begin{pmatrix} 
1 & 0 &|&  1/5 & -2/5 \\
0 & 1 &|&  1/10 & 3/10  
\end{pmatrix}
\end{eqnarray*}
Wir haben bei der reduzierten Zeilenstufenform auf der linken Seite die Einheitsmatrix erreicht. Daher ist die
ursprüngliche Matrix invertierbar und ihre inverse Matrix steht auf der rechten Seite:
\[ \begin{pmatrix}
3 & 4 \\
-1 & 2 
 \end{pmatrix} ^{-1} =  \begin{pmatrix}\, \frac{1}{5} & -\frac{2}{5} \\ \frac{1}{10} & \frac{3}{10} \end{pmatrix}. \]
    }

 \lang{en}{
 Here we also use Gaussian elimination for the matrix augmented with the identity matrix:
\begin{eqnarray*}
&& \begin{pmatrix} 
3 & 4 &|&  1 & 0  \\
-1 & 2 &|&  0 & 1 
\end{pmatrix} \begin{matrix} \updownarrow \end{matrix} \\
& \rightsquigarrow & 
 \begin{pmatrix} 
-1 & 2 &|&  0 & 1 \\
3 & 4 &|&  1 & 0  
\end{pmatrix} \begin{matrix} \phantom{1}\\  /  +3\cdot \text{(I)} \end{matrix} \\
& \rightsquigarrow & 
 \begin{pmatrix} 
-1 & 2 &|&  0 & 1 \\
0 & 10 &|&  1 & 3  
\end{pmatrix} \begin{matrix}  /  \cdot (-1) \\  /  \cdot \frac{1}{10} \end{matrix} \\
& \rightsquigarrow & 
 \begin{pmatrix} 
1 & -2 &|&  0 & -1 \\
0 & 1 &|&  1/10 & 3/10  
\end{pmatrix} \begin{matrix} /  +2\cdot \text{(II)} \\ \phantom{1} \end{matrix} \\
& \rightsquigarrow & 
 \begin{pmatrix} 
1 & 0 &|&  1/5 & -2/5 \\
0 & 1 &|&  1/10 & 3/10  
\end{pmatrix}
\end{eqnarray*}
In the reduced row echelon form on the left side we have the identity matrix. Therefore the original matrix is invertible and
its inverse matrix is on the right side:
\[ \begin{pmatrix}
3 & 4 \\
-1 & 2 
 \end{pmatrix} ^{-1} =  \begin{pmatrix}\, \frac{1}{5} & -\frac{2}{5} \\ \frac{1}{10} & \frac{3}{10} \end{pmatrix}. \]
    }
    
  	 %------------------------------------END_STEP_X
 
  \end{incremental}
  %++++++++++++++++++++++++++++++++++++++++++++END_TAB_X


  %++++++++++++++++++++++++++++++++++++++++++START_TAB_X
  \tab{\lang{de}{    Lösung c)    }  \lang{en}{ Solution c)}}
  \begin{incremental}[\initialsteps{1}]
  
  	 %----------------------------------START_STEP_X
    \step 
    \lang{de}{   
Wie vorher wendet man das Gauß-Verfahren auf 
die um die Einheitsmatrix erweiterte Matrix an:
\begin{eqnarray*}
&& \begin{pmatrix} 
2 & 3 & -4 &|&  1 & 0 & 0 \\
3 & 3 & -1  &|&  0 & 1 & 0 \\
0 & 3 & -10  &|&  0 & 0 & 1
\end{pmatrix} \begin{matrix} \phantom{1}\\  /  -\frac{3}{2}\cdot \text{(I)} \\   \phantom{1}\end{matrix} \\
& \rightsquigarrow & 
 \begin{pmatrix} 
2 & 3 & -4 &|&  1 & 0 & 0 \\
0 & -\frac{3}{2} & 5  &|&  -\frac{3}{2} & 1 & 0 \\
0 & 3 & -10  &|&  0 & 0 & 1
\end{pmatrix} \begin{matrix} \phantom{1}\\ \phantom{1}\\ /  +2\cdot \text{(II)} \end{matrix} \\
& \rightsquigarrow & 
 \begin{pmatrix} 
2 & 3 & -4 &|&  1 & 0 & 0 \\
0 & -\frac{3}{2} & 5  &|&  -\frac{3}{2} & 1 & 0 \\
0 & 0 & 0  &|&  -3 & 2 & 1
\end{pmatrix}
\end{eqnarray*}
Auf der linken Seite ist eine Nullzeile entstanden. Die ursprüngliche Matrix hat daher keinen vollen Rang und ist
daher nicht invertierbar.    }

\lang{en}{   
Here we also use Gaussian elimination for the matrix augmented with the identity matrix:
\begin{eqnarray*}
&& \begin{pmatrix} 
2 & 3 & -4 &|&  1 & 0 & 0 \\
3 & 3 & -1  &|&  0 & 1 & 0 \\
0 & 3 & -10  &|&  0 & 0 & 1
\end{pmatrix} \begin{matrix} \phantom{1}\\  /  -\frac{3}{2}\cdot \text{(I)} \\   \phantom{1}\end{matrix} \\
& \rightsquigarrow & 
 \begin{pmatrix} 
2 & 3 & -4 &|&  1 & 0 & 0 \\
0 & -\frac{3}{2} & 5  &|&  -\frac{3}{2} & 1 & 0 \\
0 & 3 & -10  &|&  0 & 0 & 1
\end{pmatrix} \begin{matrix} \phantom{1}\\ \phantom{1}\\ /  +2\cdot \text{(II)} \end{matrix} \\
& \rightsquigarrow & 
 \begin{pmatrix} 
2 & 3 & -4 &|&  1 & 0 & 0 \\
0 & -\frac{3}{2} & 5  &|&  -\frac{3}{2} & 1 & 0 \\
0 & 0 & 0  &|&  -3 & 2 & 1
\end{pmatrix}
\end{eqnarray*}
We have a zero-row on the left side. The original matrix does not have full rank and therefore the matrix
is not invertible.    }
  	 %------------------------------------END_STEP_X
 
  \end{incremental}
  %++++++++++++++++++++++++++++++++++++++++++++END_TAB_X
  %++++++++++++++++++++++++++++++++++++++++++START_TAB_X
  \tab{\lang{de}{    Lösung d)    }  \lang{en}{Solution d)}}
  \begin{incremental}[\initialsteps{1}]
  
  	 %----------------------------------START_STEP_X
    \step 
    \lang{de}{   
Wie in den anderen Teilen wendet man das Gauß-Verfahren auf 
die um die Einheitsmatrix erweiterte Matrix an:
\begin{eqnarray*}
&& \begin{pmatrix} 
2 & -1 & 3 &|&  1 & 0 & 0 \\
7 & 3 & 0  &|&  0 & 1 & 0 \\
-1 & 2 & -4  &|&  0 & 0 & 1
\end{pmatrix} \begin{matrix} \phantom{1}\\  /  -\frac{7}{2}\cdot \text{(I)} \\  /  +\frac{1}{2}\cdot \text{(I)}\end{matrix} \\
& \rightsquigarrow & 
\begin{pmatrix} 
2 & -1 & 3 &|&  1 & 0 & 0 \\
0 & 13/2 & -21/2  &|&  -7/2 & 1 & 0 \\
0 & 3/2 & -5/2  &|&  1/2 & 0 & 1
\end{pmatrix} \begin{matrix} \phantom{1}\\  /  \cdot 2 \\  /  \cdot \frac{2}{3}\end{matrix} \\
& \rightsquigarrow & 
\begin{pmatrix} 
2 & -1 & 3 &|&  1 & 0 & 0 \\
0 & 13 & -21  &|&  -7 & 2 & 0 \\
0 & 1 & -5/3  &|&  1/3 & 0 & 2/3
\end{pmatrix} \begin{matrix} \\   \updownarrow \\ \end{matrix} \\
& \rightsquigarrow & 
\begin{pmatrix} 
2 & -1 & 3 &|&  1 & 0 & 0 \\
0 & 1 & -5/3  &|&  1/3 & 0 & 2/3\\
0 & 13 & -21  &|&  -7 & 2 & 0 
\end{pmatrix} \begin{matrix} \phantom{1} \\ \phantom{1} \\  /  -13\cdot \text{(II)} \end{matrix} \\
& \rightsquigarrow & 
\begin{pmatrix} 
2 & -1 & 3 &|&  1 & 0 & 0 \\
0 & 1 & -5/3  &|&  1/3 & 0 & 2/3\\
0 & 0 & 2/3  &|&  -34/3 & 2 & -26/3 
\end{pmatrix} \begin{matrix} /  \cdot \frac{1}{2} \\ \phantom{1} \\  /  \cdot \frac{3}{2} \end{matrix} \\
& \rightsquigarrow & 
\begin{pmatrix} 
1 & -1/2 & 3/2 &|&  1/2 & 0 & 0 \\
0 & 1 & -5/3  &|&  1/3 & 0 & 2/3\\
0 & 0 & 1   &|&  -17 & 3 & -13 
\end{pmatrix} \begin{matrix} /  -\frac{3}{2}\cdot \text{(III)}\\  /  +\frac{5}{3}\cdot \text{(III)} \\ \phantom{1}  \end{matrix} \\
& \rightsquigarrow & 
\begin{pmatrix} 
1 & -1/2 & 0 &|&  26 & -9/2 & 39/2 \\
0 & 1 & 0  &|&  -28 & 5 & -21\\
0 & 0 & 1   &|&  -17 & 3 & -13 
\end{pmatrix} \begin{matrix} /  +\frac{1}{2}\cdot \text{(II)}\\  \phantom{1}  \\ \phantom{1}  \end{matrix} \\
& \rightsquigarrow & 
\begin{pmatrix} 
1 & 0 & 0 &|&  12 & -2 & 9 \\
0 & 1 & 0  &|&  -28 & 5 & -21\\
0 & 0 & 1   &|&  -17 & 3 & -13 
\end{pmatrix} 
\end{eqnarray*}


Also ist die gesuchte inverse Matrix: 
$\begin{pmatrix} 12 & -2 & 9 \\  -28 & 5 & -21\\  -17 & 3 & -13 \end{pmatrix}$.
    }


\lang{en}{  
Here we also use Gaussian elimination for the matrix augmented with the identity matrix:
\begin{eqnarray*}
&& \begin{pmatrix} 
2 & -1 & 3 &|&  1 & 0 & 0 \\
7 & 3 & 0  &|&  0 & 1 & 0 \\
-1 & 2 & -4  &|&  0 & 0 & 1
\end{pmatrix} \begin{matrix} \phantom{1}\\  /  -\frac{7}{2}\cdot \text{(I)} \\  /  +\frac{1}{2}\cdot \text{(I)}\end{matrix} \\
& \rightsquigarrow & 
\begin{pmatrix} 
2 & -1 & 3 &|&  1 & 0 & 0 \\
0 & 13/2 & -21/2  &|&  -7/2 & 1 & 0 \\
0 & 3/2 & -5/2  &|&  1/2 & 0 & 1
\end{pmatrix} \begin{matrix} \phantom{1}\\  /  \cdot 2 \\  /  \cdot \frac{2}{3}\end{matrix} \\
& \rightsquigarrow & 
\begin{pmatrix} 
2 & -1 & 3 &|&  1 & 0 & 0 \\
0 & 13 & -21  &|&  -7 & 2 & 0 \\
0 & 1 & -5/3  &|&  1/3 & 0 & 2/3
\end{pmatrix} \begin{matrix} \\   \updownarrow \\ \end{matrix} \\
& \rightsquigarrow & 
\begin{pmatrix} 
2 & -1 & 3 &|&  1 & 0 & 0 \\
0 & 1 & -5/3  &|&  1/3 & 0 & 2/3\\
0 & 13 & -21  &|&  -7 & 2 & 0 
\end{pmatrix} \begin{matrix} \phantom{1} \\ \phantom{1} \\  /  -13\cdot \text{(II)} \end{matrix} \\
& \rightsquigarrow & 
\begin{pmatrix} 
2 & -1 & 3 &|&  1 & 0 & 0 \\
0 & 1 & -5/3  &|&  1/3 & 0 & 2/3\\
0 & 0 & 2/3  &|&  -34/3 & 2 & -26/3 
\end{pmatrix} \begin{matrix} /  \cdot \frac{1}{2} \\ \phantom{1} \\  /  \cdot \frac{3}{2} \end{matrix} \\
& \rightsquigarrow & 
\begin{pmatrix} 
1 & -1/2 & 3/2 &|&  1/2 & 0 & 0 \\
0 & 1 & -5/3  &|&  1/3 & 0 & 2/3\\
0 & 0 & 1   &|&  -17 & 3 & -13 
\end{pmatrix} \begin{matrix} /  -\frac{3}{2}\cdot \text{(III)}\\  /  +\frac{5}{3}\cdot \text{(III)} \\ \phantom{1}  \end{matrix} \\
& \rightsquigarrow & 
\begin{pmatrix} 
1 & -1/2 & 0 &|&  26 & -9/2 & 39/2 \\
0 & 1 & 0  &|&  -28 & 5 & -21\\
0 & 0 & 1   &|&  -17 & 3 & -13 
\end{pmatrix} \begin{matrix} /  +\frac{1}{2}\cdot \text{(II)}\\  \phantom{1}  \\ \phantom{1}  \end{matrix} \\
& \rightsquigarrow & 
\begin{pmatrix} 
1 & 0 & 0 &|&  12 & -2 & 9 \\
0 & 1 & 0  &|&  -28 & 5 & -21\\
0 & 0 & 1   &|&  -17 & 3 & -13 
\end{pmatrix} 
\end{eqnarray*}

So the inverse matrix is:
$\begin{pmatrix} 12 & -2 & 9 \\  -28 & 5 & -21\\  -17 & 3 & -13 \end{pmatrix}$.
    }
  	 %------------------------------------END_STEP_X
 
  \end{incremental}
  %++++++++++++++++++++++++++++++++++++++++++++END_TAB_X


%#############################################################ENDE

% \tab{\lang{de}{Video: ähnliche Übungsaufgabe}}	
%    \youtubevideo[500][300]{jbJK9fbMf8M}\\

\end{tabs*}
\end{content}