\documentclass{mumie.element.exercise}
%$Id$
\begin{metainfo}
  \name{
    \lang{de}{Ü09: Determinante}
    \lang{en}{Ex09: Determinant}
  }
  \begin{description} 
 This work is licensed under the Creative Commons License Attribution 4.0 International (CC-BY 4.0)   
 https://creativecommons.org/licenses/by/4.0/legalcode 

    \lang{de}{}
    \lang{en}{}
  \end{description}
  \begin{components}
  \end{components}
  \begin{links}
  \end{links}
  \creategeneric
\end{metainfo}
\begin{content}
\usepackage{mumie.ombplus}

\title{\lang{de}{Ü09: Determinante} \lang{en}{Ex09: Determinants}}

\begin{block}[annotation]
  Im Ticket-System: \href{http://team.mumie.net/issues/11362}{Ticket 11362}
\end{block}

%######################################################FRAGE_TEXT
\lang{de}{ 
Bestimmen Sie mit Hilfe von Zeilenumformungen die Determinanten der folgenden Matrizen:
\begin{align*}
\text{a) }
A=\left( \begin{smallmatrix}
0 & 0 & 1 & 0 \\
0 & -1 & 0 & -3 \\
1 & 2 & 0 & 6 \\
0 & 0 & 0 & 1 
\end{smallmatrix} \right), \qquad
\text{b) }
B=\left( \begin{smallmatrix}
2 & 3 & -4 \\
3 & 3 & -1 \\
0 & 3 & -9 
\end{smallmatrix} \right),\qquad
\text{c) }
C=\left( \begin{smallmatrix}
1&3& 3 & -1 & -2 \\
2 &4&3&0&-1\\
0&0&0&0&5\\
0&0&1&2&1\\
2&6&6&2&4 
\end{smallmatrix} \right).
\end{align*}
 }

 \lang{en}{ 
Determine with the help of row transformations the determinans of the following matrices:
\begin{align*}
\text{a) }
A=\left( \begin{smallmatrix}
0 & 0 & 1 & 0 \\
0 & -1 & 0 & -3 \\
1 & 2 & 0 & 6 \\
0 & 0 & 0 & 1 
\end{smallmatrix} \right), \qquad
\text{b) }
B=\left( \begin{smallmatrix}
2 & 3 & -4 \\
3 & 3 & -1 \\
0 & 3 & -9 
\end{smallmatrix} \right),\qquad
\text{c) }
C=\left( \begin{smallmatrix}
1&3& 3 & -1 & -2 \\
2 &4&3&0&-1\\
0&0&0&0&5\\
0&0&1&2&1\\
2&6&6&2&4 
\end{smallmatrix} \right).
\end{align*}
 }

%##################################################ANTWORTEN_TEXT
\begin{tabs*}[\initialtab{0}\class{exercise}]

  %++++++++++++++++++++++++++++++++++++++++++START_TAB_X
  \tab{\lang{de}{    Antwort    } \lang{en}{ Answer}}
  \begin{incremental}[\initialsteps{1}]
  
  	 %----------------------------------START_STEP_X
    \step 
    \lang{de}{   
\[ \text{a) } \det(A)=1,\quad \text{b) } \det(B)=-3,\quad \text{c) } \det(C)=-40 .\]    }

\lang{en}{   
\[ \text{a) } \det(A)=1,\quad \text{b) } \det(B)=-3,\quad \text{c) } \det(C)=-40 .\]    }
  	 %------------------------------------END_STEP_X
 
  \end{incremental}
  %++++++++++++++++++++++++++++++++++++++++++++END_TAB_X


  %++++++++++++++++++++++++++++++++++++++++++START_TAB_X
  \tab{\lang{de}{    Lösung a)    } \lang{en}{ Solution a)}}
  \begin{incremental}[\initialsteps{1}]
  
  	 %----------------------------------START_STEP_X
    \step 
    \lang{de}{   
Durch Vertauschen der ersten mit der dritten Zeile erhalten wir die Matrix
\[  \begin{pmatrix} 
1 & 2 & 0 & 6 \\
0 & -1 & 0 & -3 \\
0 & 0 & 1 & 0 \\
0 & 0 & 0 & 1 
\end{pmatrix} \]
Diese ist eine obere Dreiecksmatrix und daher ist ihre Determinante das Produkt der Diagonalelemente, d.h.
\[ \det \Big( \left( \begin{smallmatrix}
1 & 2 & 0 & 6 \\
0 & -1 & 0 & -3 \\
0 & 0 & 1 & 0 \\
0 & 0 & 0 & 1 
\end{smallmatrix} \right) \Big) = 1\cdot (-1)\cdot 1\cdot 1=-1.\]
Da sich beim Vertauschen zweier Zeilen lediglich das Vorzeichen der Determinante ändert, gilt also
\[  \det (A)= - \det \Big( \left( \begin{smallmatrix}
1 & 2 & 0 & 6 \\
0 & -1 & 0 & -3 \\
0 & 0 & 1 & 0 \\
0 & 0 & 0 & 1 
\end{smallmatrix} \right) \Big) =-(-1)=1.\]
    }

    \lang{en}{ 
    By swapping the first and the third row we get the matrix
\[  \begin{pmatrix} 
1 & 2 & 0 & 6 \\
0 & -1 & 0 & -3 \\
0 & 0 & 1 & 0 \\
0 & 0 & 0 & 1 
\end{pmatrix} \]
This is a upper triangular matrix and therefore its determinant is the product of the diagonal entries:
\[ \det \Big( \left( \begin{smallmatrix}
1 & 2 & 0 & 6 \\
0 & -1 & 0 & -3 \\
0 & 0 & 1 & 0 \\
0 & 0 & 0 & 1 
\end{smallmatrix} \right) \Big) = 1\cdot (-1)\cdot 1\cdot 1=-1.\]
Since swapping two rows only changes the sign of the determinant, we have
\[  \det (A)= - \det \Big( \left( \begin{smallmatrix}
1 & 2 & 0 & 6 \\
0 & -1 & 0 & -3 \\
0 & 0 & 1 & 0 \\
0 & 0 & 0 & 1 
\end{smallmatrix} \right) \Big) =-(-1)=1.\]
    }
  	 %------------------------------------END_STEP_X
 
  \end{incremental}
  %++++++++++++++++++++++++++++++++++++++++++++END_TAB_X


  %++++++++++++++++++++++++++++++++++++++++++START_TAB_X
  \tab{\lang{de}{    Lösung b)    } \lang{en}{Solution b)}}
  \begin{incremental}[\initialsteps{1}]
  
  	 %----------------------------------START_STEP_X
    \step 
    \lang{de}{   
Wir bringen auch hier die Matrix durch Zeilenumformungen auf Dreiecksgestalt:
\begin{eqnarray*}
&& \begin{pmatrix} 
2 & 3 & -4 \\
3 & 3 & -1  \\
0 & 3 & -9 
\end{pmatrix} \begin{matrix} \phantom{1}\\  /  -\frac{3}{2}\cdot \text{(I)} \\   \phantom{1}\end{matrix}
 \rightsquigarrow  
 \begin{pmatrix} 
2 & 3 & -4  \\
0 & -3/2 & 5   \\
0 & 3 & -9 
\end{pmatrix} \begin{matrix} \phantom{1}\\ \phantom{1}\\ /  +2\cdot \text{(II)} \end{matrix} \\
& \rightsquigarrow & 
 \begin{pmatrix} 
2 & 3 & -4  \\
0 & -3/2 & 5  \\
0 & 0 & 1 
\end{pmatrix}
\end{eqnarray*}
Die letzte Matrix hat nun die Determinante
\[ \det \Big( \left( \begin{smallmatrix}2 & 3 & -4  \\
0 & -3/2 & 5  \\
0 & 0 & 1  \end{smallmatrix} \right) \Big) =2\cdot (-3/2)\cdot 1=-3.\]
Da sich die Determinante nicht ändert, wenn man Vielfache einer Zeile zu einer anderen addiert, ist auch
\[ \det(B)=-3.\]
    }


    \lang{en}{ 
    Here we also transform the matrix into triangular form with the help of row transformations:
\begin{eqnarray*}
&& \begin{pmatrix} 
2 & 3 & -4 \\
3 & 3 & -1  \\
0 & 3 & -9 
\end{pmatrix} \begin{matrix} \phantom{1}\\  /  -\frac{3}{2}\cdot \text{(I)} \\   \phantom{1}\end{matrix}
 \rightsquigarrow  
 \begin{pmatrix} 
2 & 3 & -4  \\
0 & -3/2 & 5   \\
0 & 3 & -9 
\end{pmatrix} \begin{matrix} \phantom{1}\\ \phantom{1}\\ /  +2\cdot \text{(II)} \end{matrix} \\
& \rightsquigarrow & 
 \begin{pmatrix} 
2 & 3 & -4  \\
0 & -3/2 & 5  \\
0 & 0 & 1 
\end{pmatrix}
\end{eqnarray*}
The latter matrix has the determinant
\[ \det \Big( \left( \begin{smallmatrix}2 & 3 & -4  \\
0 & -3/2 & 5  \\
0 & 0 & 1  \end{smallmatrix} \right) \Big) =2\cdot (-3/2)\cdot 1=-3.\]
The determinant does not change, wenn we add a multiple of a row to another, so we have
\[ \det(B)=-3.\]
    }
  	 %------------------------------------END_STEP_X
 
  \end{incremental}
  %++++++++++++++++++++++++++++++++++++++++++++END_TAB_X


  %++++++++++++++++++++++++++++++++++++++++++START_TAB_X
  \tab{\lang{de}{    Lösung c)    }  \lang{en}{Solution c)}}
  \begin{incremental}[\initialsteps{1}]
  
  	 %----------------------------------START_STEP_X
    \step 
    \lang{de}{   Wir bringen auch hier die Matrix durch Zeilenumformungen auf Dreiecksgestalt:
\begin{eqnarray*}
&C=& \begin{pmatrix} 
1&3& 3 & -1 & -2 \\
2 &4&3&0&-1\\
0&0&0&0&5\\
0&0&1&2&1\\
2&6&6&2&4 
\end{pmatrix} \begin{matrix} \phantom{1}\\  /  -2\cdot \text{(I)} \\   \phantom{1}\\   \phantom{1} \\  /  -2\cdot \text{(I)} \end{matrix}\\
& \rightsquigarrow  &
 \begin{pmatrix} 
1&3& 3 & -1 & -2 \\
0 &-2&-3&2&3\\
0&0&0&0&5\\
0&0&1&2&1\\
0&0&0&4&8 
\end{pmatrix} \begin{matrix} \phantom{1}\\  \phantom{1}\\ \updownarrow \\   \phantom{1} \end{matrix}\\
& \rightsquigarrow  &
\begin{pmatrix} 
1&3& 3 & -1 & -2 \\
0 &-2&-3&2&3\\
0&0&1&2&1\\
0&0&0&0&5\\
0&0&0&4&8 
\end{pmatrix} \begin{matrix} \phantom{1}\\  \phantom{1} \\   \phantom{1} \\ \updownarrow \end{matrix}\\
& \rightsquigarrow  &
\begin{pmatrix} 
1&3& 3 & -1 & -2 \\
0 &-2&-3&2&3\\
0&0&1&2&1\\
0&0&0&4&8 \\
0&0&0&0&5
\end{pmatrix}
\end{eqnarray*}
Bei der ersten Zeilenumformung (Addition der Vielfachen einer Zeile zu anderen) ändert sich die Determinante nicht,
bei den Zeilenvertauschungen ändert sich die Determinante jeweils um das Vorzeichen. Die Determinante der
Matrix $C$ ist also gleich der Determinante der letzten Matrix. Also
\[ \det(C)= \det\Big( \left( \begin{smallmatrix}1&3& 3 & -1 & -2 \\
0 &-2&-3&2&3\\
0&0&1&2&1\\
0&0&0&4&8 \\
0&0&0&0&5
 \end{smallmatrix} \right) \Big) =1\cdot (-2)\cdot 1\cdot 4\cdot 5=-40.\]
    }

     \lang{en}{  We transform the matrix into triangular form using row transformations:
\begin{eqnarray*}
&C=& \begin{pmatrix} 
1&3& 3 & -1 & -2 \\
2 &4&3&0&-1\\
0&0&0&0&5\\
0&0&1&2&1\\
2&6&6&2&4 
\end{pmatrix} \begin{matrix} \phantom{1}\\  /  -2\cdot \text{(I)} \\   \phantom{1}\\   \phantom{1} \\  /  -2\cdot \text{(I)} \end{matrix}\\
& \rightsquigarrow  &
 \begin{pmatrix} 
1&3& 3 & -1 & -2 \\
0 &-2&-3&2&3\\
0&0&0&0&5\\
0&0&1&2&1\\
0&0&0&4&8 
\end{pmatrix} \begin{matrix} \phantom{1}\\  \phantom{1}\\ \updownarrow \\   \phantom{1} \end{matrix}\\
& \rightsquigarrow  &
\begin{pmatrix} 
1&3& 3 & -1 & -2 \\
0 &-2&-3&2&3\\
0&0&1&2&1\\
0&0&0&0&5\\
0&0&0&4&8 
\end{pmatrix} \begin{matrix} \phantom{1}\\  \phantom{1} \\   \phantom{1} \\ \updownarrow \end{matrix}\\
& \rightsquigarrow  &
\begin{pmatrix} 
1&3& 3 & -1 & -2 \\
0 &-2&-3&2&3\\
0&0&1&2&1\\
0&0&0&4&8 \\
0&0&0&0&5
\end{pmatrix}
\end{eqnarray*}
The first transformation (adding a multiple of a row to another) does not change the determinant. Swapping rows change
the determinants sign. The determinant of the matrix $C$ is equal to the determinant of the last matrix, so
\[ \det(C)= \det\Big( \left( \begin{smallmatrix}1&3& 3 & -1 & -2 \\
0 &-2&-3&2&3\\
0&0&1&2&1\\
0&0&0&4&8 \\
0&0&0&0&5
 \end{smallmatrix} \right) \Big) =1\cdot (-2)\cdot 1\cdot 4\cdot 5=-40.\]
    }
  	 %------------------------------------END_STEP_X
 
  \end{incremental}
  %++++++++++++++++++++++++++++++++++++++++++++END_TAB_X


%#############################################################ENDE
\end{tabs*}
\end{content}