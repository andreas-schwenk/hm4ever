\documentclass{mumie.element.exercise}
%$Id$
\begin{metainfo}
  \name{
    \lang{de}{Ü07: Determinante}
    \lang{en}{Ex07: The determinant}
  }
  \begin{description} 
 This work is licensed under the Creative Commons License Attribution 4.0 International (CC-BY 4.0)   
 https://creativecommons.org/licenses/by/4.0/legalcode 

    \lang{de}{}
    \lang{en}{}
  \end{description}
  \begin{components}
  \end{components}
  \begin{links}
  \end{links}
  \creategeneric
\end{metainfo}
\begin{content}
\usepackage{mumie.ombplus}

\title{\lang{de}{Ü07: Determinante}  \lang{en}{Ex07: The determinant}}

\begin{block}[annotation]
  Im Ticket-System: \href{http://team.mumie.net/issues/11361}{Ticket 11361}
\end{block}

%######################################################FRAGE_TEXT
\lang{de}{ 
Berechnen Sie mit Hilfe der Formeln für $(2\times 2)$-Matrizen und $(3\times 3)$-Matrizen die Determinanten der folgenden Matrizen.
\[
\text{a) }
\begin{pmatrix}
6 & 1 \\
5 & 3 \\
\end{pmatrix}, \qquad
\text{b) }
\begin{pmatrix}
2 & 3 & -4 \\
1 & 0 & 3 \\
0 & 3 & -10 \\
\end{pmatrix}, \qquad
\text{c) }
\begin{pmatrix}
2 & -1 & 2 \\
3 & 3 & 1 \\
-4 & 2 & 1 \\
\end{pmatrix}, \qquad
\text{d) }
\begin{pmatrix}
5 & 2 & 3 \\
-1 & 2 & 1 \\
1 & 1 & 4 \\
\end{pmatrix}.
\] }

\lang{en}{ 
Determine with the help of the formulas for $(2\times 2)$-matrices and $(3\times 3)$-matrices the determinants of the following matrices.
\[
\text{a) }
\begin{pmatrix}
6 & 1 \\
5 & 3 \\
\end{pmatrix}, \qquad
\text{b) }
\begin{pmatrix}
2 & 3 & -4 \\
1 & 0 & 3 \\
0 & 3 & -10 \\
\end{pmatrix}, \qquad
\text{c) }
\begin{pmatrix}
2 & -1 & 2 \\
3 & 3 & 1 \\
-4 & 2 & 1 \\
\end{pmatrix}, \qquad
\text{d) }
\begin{pmatrix}
5 & 2 & 3 \\
-1 & 2 & 1 \\
1 & 1 & 4 \\
\end{pmatrix}.
\] }

%##################################################ANTWORTEN_TEXT
\begin{tabs*}[\initialtab{0}\class{exercise}]

  %++++++++++++++++++++++++++++++++++++++++++START_TAB_X
  \tab{\lang{de}{    Antwort   } \lang{en}{ Answer}}
  \begin{incremental}[\initialsteps{1}]
  
  	 %----------------------------------START_STEP_X
    \step 
    \lang{de}{   
Die Determinanten sind:\\
a) $13$, \quad b) $0$, \quad c) $45$, \quad d) $36$.
    }
    \lang{en}{   
The determinants are:\\
a) $13$, \quad b) $0$, \quad c) $45$, \quad d) $36$.
    }
  	 %------------------------------------END_STEP_X
 
  \end{incremental}

  %++++++++++++++++++++++++++++++++++++++++++START_TAB_X
  \tab{\lang{de}{    Lösung a)    } \lang{en}{ Solution a)}}
  \begin{incremental}[\initialsteps{1}]
  
  	 %----------------------------------START_STEP_X
    \step 
    \lang{de}{   
Die Determinante einer $(2\times 2)$-Matrix $\left(\begin{smallmatrix}
a & b\\ c & d \end{smallmatrix}\right)$ ist gegeben durch $ad-bc$. In diesem Beispiel also

\[ \det\Big( \begin{pmatrix}
6 & 1 \\
5 & 3 \\
\end{pmatrix} \Big) = 6\cdot 3-1\cdot 5=18-5=13.\]    }
\lang{en}{   
The determinant of a $(2\times 2)$-matrix $\left(\begin{smallmatrix}
a & b\\ c & d \end{smallmatrix}\right)$ is $ad-bc$. So for this example

\[ \det\Big( \begin{pmatrix}
6 & 1 \\
5 & 3 \\
\end{pmatrix} \Big) = 6\cdot 3-1\cdot 5=18-5=13.\]    }
  	 %------------------------------------END_STEP_X
 
  \end{incremental}
  %++++++++++++++++++++++++++++++++++++++++++++END_TAB_X


  %++++++++++++++++++++++++++++++++++++++++++START_TAB_X
  \tab{\lang{de}{    Lösung b)    } \lang{en}{ Solution b)}}
  \begin{incremental}[\initialsteps{1}]
  
  	 %----------------------------------START_STEP_X
    \step 
    \lang{de}{   
Nach der Regel von Sarrus schreibt man zunächst die ersten beiden Spalten der Matrix nochmals rechts daneben
\[ \begin{vmatrix}
2 & 3 & -4 &: &2 & 3\\ 1 & 0 & 3&: &1 & 0 \\ 0 & 3 & -10&: &0 & 3
\end{vmatrix}. \]
Dann addiert man die Produkte der Einträge der Diagonalen von links oben nach rechts unten und subtrahiert
anschließend die Produkte der Einträge der Diagonalen von links unten nach rechts oben:
\begin{eqnarray*}
 \det \Big( \begin{pmatrix}
2 & 3 & -4 \\
1 & 0 & 3 \\
0 & 3 & -10 \\
\end{pmatrix} \Big)&=& 2\cdot 0\cdot (-10)+3\cdot 3\cdot 0
+ (-4)\cdot 1\cdot 3 \\ 
&& - 0\cdot 0\cdot (-4)- 3\cdot 3\cdot 2
- (-10)\cdot 1\cdot 3 \\
&=& 0+0-12-0-18+30=0
\end{eqnarray*}    }

\lang{en}{ 
According to the rule of Sarrus we write the first two columns of the matrix behind it:
\[ \begin{vmatrix}
2 & 3 & -4 &: &2 & 3\\ 1 & 0 & 3&: &1 & 0 \\ 0 & 3 & -10&: &0 & 3
\end{vmatrix}. \]
Then we add the products of the diagonal entries from the upper left to the bottom right and subtract
the products of the diagonal entries from the bottom left to the upper right:
\begin{eqnarray*}
 \det \Big( \begin{pmatrix}
2 & 3 & -4 \\
1 & 0 & 3 \\
0 & 3 & -10 \\
\end{pmatrix} \Big)&=& 2\cdot 0\cdot (-10)+3\cdot 3\cdot 0
+ (-4)\cdot 1\cdot 3 \\ 
&& - 0\cdot 0\cdot (-4)- 3\cdot 3\cdot 2
- (-10)\cdot 1\cdot 3 \\
&=& 0+0-12-0-18+30=0
\end{eqnarray*}    }
  	 %------------------------------------END_STEP_X
 
  \end{incremental}
  %++++++++++++++++++++++++++++++++++++++++++++END_TAB_X


  %++++++++++++++++++++++++++++++++++++++++++START_TAB_X
  \tab{\lang{de}{    Lösung c)    } \lang{en}{Solution c)}}
  \begin{incremental}[\initialsteps{1}]
  
  	 %----------------------------------START_STEP_X
    \step 
    \lang{de}{   
Wie in b) berechnet man mit der Regel von Sarrus:
\begin{eqnarray*}
 \det \Big( \begin{pmatrix}
2 & -1 & 2 \\
3 & 3 & 1 \\
-4 & 2 & 1 \\
\end{pmatrix} \Big)&=& 2\cdot 3\cdot 1+(-1)\cdot 1\cdot (-4)
+ 2\cdot 3\cdot 2 \\ 
&& - (-4)\cdot 3\cdot 2- 2\cdot 1\cdot 2
- 1\cdot 3\cdot (-1) \\
&=& 6+4+12+24-4+3=45
\end{eqnarray*}    }
\lang{en}{   
Like in b), we calculate the determinant with the rule of Sarrus:
\begin{eqnarray*}
 \det \Big( \begin{pmatrix}
2 & -1 & 2 \\
3 & 3 & 1 \\
-4 & 2 & 1 \\
\end{pmatrix} \Big)&=& 2\cdot 3\cdot 1+(-1)\cdot 1\cdot (-4)
+ 2\cdot 3\cdot 2 \\ 
&& - (-4)\cdot 3\cdot 2- 2\cdot 1\cdot 2
- 1\cdot 3\cdot (-1) \\
&=& 6+4+12+24-4+3=45
\end{eqnarray*}    }
  	 %------------------------------------END_STEP_X
 
  \end{incremental}
  %++++++++++++++++++++++++++++++++++++++++++++END_TAB_X

  %++++++++++++++++++++++++++++++++++++++++++START_TAB_X
  \tab{\lang{de}{    Lösung d)    } \lang{en}{Solution d)}}
  \begin{incremental}[\initialsteps{1}]
  
  	 %----------------------------------START_STEP_X
    \step 
    \lang{de}{   
Wie in b) berechnet man mit der Regel von Sarrus:
\begin{eqnarray*}
 \det \Big( \begin{pmatrix}
5 & 2 & 3 \\
-1 & 2 & 1 \\
1 & 1 & 4 \\
\end{pmatrix} \Big)&=& 5\cdot 2\cdot 4+2\cdot 1\cdot 1
+ 3\cdot (-1)\cdot 1 \\ 
&& - 1\cdot 2\cdot 3- 1\cdot 1\cdot 5
- 4\cdot (-1)\cdot 2 \\
&=& 40+2-3-6-5+8=36
\end{eqnarray*}    }
\lang{en}{   
As done in b), we utilise the rule of Sarrus:
\begin{eqnarray*}
 \det \Big( \begin{pmatrix}
5 & 2 & 3 \\
-1 & 2 & 1 \\
1 & 1 & 4 \\
\end{pmatrix} \Big)&=& 5\cdot 2\cdot 4+2\cdot 1\cdot 1
+ 3\cdot (-1)\cdot 1 \\ 
&& - 1\cdot 2\cdot 3- 1\cdot 1\cdot 5
- 4\cdot (-1)\cdot 2 \\
&=& 40+2-3-6-5+8=36
\end{eqnarray*}    }
  	 %------------------------------------END_STEP_X
 
  \end{incremental}

%#############################################################ENDE

% \tab{\lang{de}{Video: ähnliche Übungsaufgabe}}	
%    \youtubevideo[500][300]{aEyex3-vFbs}\\

\end{tabs*}
\end{content}