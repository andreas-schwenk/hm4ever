\documentclass{mumie.element.exercise}
%$Id$
\begin{metainfo}
  \name{
    \lang{de}{Ü01: Matrixtypen}
    \lang{en}{Ex01: Types of matrices}
  }
  \begin{description} 
 This work is licensed under the Creative Commons License Attribution 4.0 International (CC-BY 4.0)   
 https://creativecommons.org/licenses/by/4.0/legalcode 

    \lang{de}{}
    \lang{en}{}
  \end{description}
  \begin{components}
  \end{components}
  \begin{links}
  \end{links}
  \creategeneric
\end{metainfo}
\begin{content}
\usepackage{mumie.ombplus}

\title{\lang{de}{Ü01: Matrixtypen} \lang{en}{Ex01: Types of matrices}}

\begin{block}[annotation]
  Im Ticket-System: \href{http://team.mumie.net/issues/11356}{Ticket 11356}
\end{block}

%######################################################FRAGE_TEXT
\lang{de}{ Geben Sie bei den folgenden Matrizen jeweils an, ob es sich um eine 
Nullmatrix, eine Einheitsmatrix, eine Diagonalmatrix, eine obere Dreiecksmatrix und/oder untere
Dreiecksmatrix handelt:
\begin{align*}
A=\begin{pmatrix} 0 & 0 \\ 0 & 0 \end{pmatrix}, \quad 
B=\begin{pmatrix} 1 \end{pmatrix}, \quad 
C=\begin{pmatrix} 1 & 0 & 0 \\ 7 & 2 & 0 \\ 0 & 1 & 3 \end{pmatrix}, \quad
D=\begin{pmatrix} 1 & 2 \\ 0 & 0 \end{pmatrix}, \\
F=\begin{pmatrix} 1 & 0 & 0 & 0 \\ 0 & 1 & 0 & 0 \\ 0 & 0 & 1 & 0 \\ 0 & 0 & 0 & 1 \end{pmatrix}, \quad
G=\begin{pmatrix} 1 & 0 \\ 0 & 2 \end{pmatrix} \quad \text{ und } \quad 
H=\begin{pmatrix} 1 & 2 & 3 \\ 2 & 0 & 4 \\ 3 & 4 & 0 \end{pmatrix}.
\end{align*} }

\lang{en}{ Decide for the following matrices if they are a zero matrix, a identity matrix, a diagonal matrix, a lower and/or 
upper triangular matrix:
\begin{align*}
A=\begin{pmatrix} 0 & 0 \\ 0 & 0 \end{pmatrix}, \quad 
B=\begin{pmatrix} 1 \end{pmatrix}, \quad 
C=\begin{pmatrix} 1 & 0 & 0 \\ 7 & 2 & 0 \\ 0 & 1 & 3 \end{pmatrix}, \quad
D=\begin{pmatrix} 1 & 2 \\ 0 & 0 \end{pmatrix}, \\
F=\begin{pmatrix} 1 & 0 & 0 & 0 \\ 0 & 1 & 0 & 0 \\ 0 & 0 & 1 & 0 \\ 0 & 0 & 0 & 1 \end{pmatrix}, \quad
G=\begin{pmatrix} 1 & 0 \\ 0 & 2 \end{pmatrix} \quad \text{ and } \quad 
H=\begin{pmatrix} 1 & 2 & 3 \\ 2 & 0 & 4 \\ 3 & 4 & 0 \end{pmatrix}.
\end{align*} }

%##################################################ANTWORTEN_TEXT
\begin{tabs*}[\initialtab{0}\class{exercise}]

  %++++++++++++++++++++++++++++++++++++++++++START_TAB_X
  \tab{\lang{de}{    Antwort    } \lang{en}{Answer}}
  
    \lang{de}{   \begin{table}

Typ                              &   Nullmatrix      & Einheitsmatrix      & Diagonalmatrix      & untere Dreiecksmatrix          & obere Dreiecksmatrix         \\
                                             
$ A=\left(\begin{smallmatrix}  0 & 0 \\ 0 & 0\end{smallmatrix}\right)$
&   $ \checkmark$   & X          & $ \checkmark$          & $ \checkmark$      & $ \checkmark$                   \\ 
$ B=\left(\begin{smallmatrix} 1 \end{smallmatrix}\right) $
  &   X       & $ \checkmark$        & $ \checkmark$        & $ \checkmark$       & $ \checkmark$                \\ 
$ C=\left(\begin{smallmatrix}  1 & 0 & 0 \\ 7 & 2 & 0 \\ 0 & 1 & 3  \end{smallmatrix}\right)$
&   X       & X           & X          & $ \checkmark$                     & X            \\ 
$ D=\left(\begin{smallmatrix} 1 & 2 \\ 0 & 0 \end{smallmatrix}\right) $
  &  X  & X           & X           & X    &  $ \checkmark$                   \\ 
$ F=\left(\begin{smallmatrix} 1 & 0 & 0 & 0 \\ 0 & 1 & 0 & 0 \\ 0 & 0 & 1 & 0 \\ 0 & 0 & 0 & 1 \end{smallmatrix}\right)$
 & X  &   $ \checkmark$     & $ \checkmark$     & $ \checkmark$       & $ \checkmark$            \\ 
 $ G=\left(\begin{smallmatrix}  1 & 0 \\ 0 & 2 \end{smallmatrix}\right) $
  &  X  & X           &   $ \checkmark$            &   $ \checkmark$     &  $ \checkmark$                   \\ 
 $ H=\left(\begin{smallmatrix}  1 & 2 & 3 \\ 2 & 0 & 4 \\ 3 & 4 & 0 \end{smallmatrix}\right) $
  &  X  & X           &   X     &  X   & X           \\ 
  \end{table}

\begin{itemize}
\item $A$  ist eine Nullmatrix und damit insbesondere eine Diagonalmatrix, obere und untere Dreiecksmatrix.
\item $B$ und $F$ sind Einheitsmatrizen und damit insbesondere Diagonalmatrizen, obere und untere Dreiecksmatrizen.
\item $C$  ist eine untere Dreiecksmatrix, gehört aber sonst zu keiner der anderen Kategorien.
\item $D$  ist eine obere Dreiecksmatrix, gehört aber sonst zu keiner der anderen Kategorien.
\item $G$  ist eine Diagonalmatrix und daher insbesondere eine obere und untere Dreiecksmatrix.
\item $H$  erfüllt keine der Kriterien.
\end{itemize}}




  \lang{en}{   \begin{table}

Typ                              &   Zero matrix      & identity matrix      & diagonal matrix      & lower triangular matrix          & upper triangular matrix        \\
                                             
$ A=\left(\begin{smallmatrix}  0 & 0 \\ 0 & 0\end{smallmatrix}\right)$
&   $ \checkmark$   & X          & $ \checkmark$          & $ \checkmark$      & $ \checkmark$                   \\ 
$ B=\left(\begin{smallmatrix} 1 \end{smallmatrix}\right) $
  &   X       & $ \checkmark$        & $ \checkmark$        & $ \checkmark$       & $ \checkmark$                \\ 
$ C=\left(\begin{smallmatrix}  1 & 0 & 0 \\ 7 & 2 & 0 \\ 0 & 1 & 3  \end{smallmatrix}\right)$
&   X       & X           & X          & $ \checkmark$                     & X            \\ 
$ D=\left(\begin{smallmatrix} 1 & 2 \\ 0 & 0 \end{smallmatrix}\right) $
  &  X  & X           & X           & X    &  $ \checkmark$                   \\ 
$ F=\left(\begin{smallmatrix} 1 & 0 & 0 & 0 \\ 0 & 1 & 0 & 0 \\ 0 & 0 & 1 & 0 \\ 0 & 0 & 0 & 1 \end{smallmatrix}\right)$
 & X  &   $ \checkmark$     & $ \checkmark$     & $ \checkmark$       & $ \checkmark$            \\ 
 $ G=\left(\begin{smallmatrix}  1 & 0 \\ 0 & 2 \end{smallmatrix}\right) $
  &  X  & X           &   $ \checkmark$            &   $ \checkmark$     &  $ \checkmark$                   \\ 
 $ H=\left(\begin{smallmatrix}  1 & 2 & 3 \\ 2 & 0 & 4 \\ 3 & 4 & 0 \end{smallmatrix}\right) $
  &  X  & X           &   X     &  X   & X           \\ 
  \end{table}

\begin{itemize}
\item $A$  is a zero matrix and therefore especially a diagonal matrix, lower and upper triangular matrix.
\item $B$ and $F$ are identity matrices and therefore especially diagonal matrices, lower and upper triangular matrices.
\item $C$  is only a lower triangular matrix.
\item $D$  is only a upper triangular matrix.
\item $G$  is a diangonal matrix and therefore especially a lower and upper triangular matrix.
\item $H$  does not fulfill any of the criteria.
\end{itemize}}


\tab{\lang{de}{Nullmatrizen} \lang{en}{Zero matrices}}
\lang{de}{Eine Matrix ist eine Nullmatrix, wenn \textbf{alle} Einträge $0$ sind. Dies ist bei $A$ der Fall, bei den anderen
nicht, weshalb $A$ die einzige dieser Matrizen ist, welche eine Nullmatrix ist.}
\lang{en}{A matrix is a zero matrix, when \textbf{all} entries are equal to $0$. This is only the case for matrix $A$, which is why
$A$ is the only zero matrix.}

\tab{\lang{de}{Einheitsmatrizen} \lang{en}{Identity matrices}}
\lang{de}{
Eine quadratische Matrix ist eine Einheitsmatrix, wenn die Diagonaleinträge \textbf{alle} gleich $1$ sind, und \textbf{alle} anderen gleich $0$. Dies ist bei $F$ der Fall. Auch die $(1\times 1)$-Matrix $B$ erfüllt dieses Kriterium,
da sie nur einen Eintrag $b_{11}=1$ hat, welcher ein Diagonaleintrag ist.}
\lang{en}{
A square matrix is a identity matrix, wenn the diagonal entries are \textbf{all} equal to $1$ and \textbf{all} the others are equal to $0$. This is the case for $F$. Also the $(1\times 1)$-Matrix $B$ fulfills this criteria since
the only entry $b_{11}=1$ is a diagonal entry.}

\tab{\lang{de}{Diagonalmatrizen} \lang{en}{Diagonal matrices}}
\lang{de}{
Eine quadratische Matrix ist eine Diagonalmatrix, wenn alle Einträge außerhalb der Diagonalen gleich $0$ sind. Die Werte auf der Diagonalen dürfen beliebig sein ($0$ ist dabei nicht ausgeschlossen).

Die Matrizen $C$, $D$ und $H$ besitzen auch außerhalb der Diagonalen Einträge, die nicht Null sind, weshalb dies keine
Diagonalmatrizen sind. Alle anderen, also $A$, $B$, $F$ und $G$, sind Diagonalmatrizen.}

\lang{en}{
A square matrix is a diagonal matrix, when all entries beyond the diagonal are equal to $0$. The value of the diagonal entries may be any (including $0$).

The matrices $C$, $D$ and $H$ have non-zero entries beyond their diagonal, so they are no diagonal matrices.
But all the others, so $A$, $B$, $F$ and $G$ are diagonal matrices.}

\tab{\lang{de}{Obere Dreiecksmatrizen} \lang{en}{Upper triangular matrices}}

\lang{en}{A square matrix is a upper triangular matrix, if all the entries below the diagonal are equal to $0$.
The value of the entries on and above the diagonal may be any, including $0$.
Only the matrices $C$ and $H$ do not fulfill this criteria, which is why they are no upper triangular matrices.
But all the others, so $A$, $B$, $D$, $F$ and $G$ are upper triangular matrices.}

\lang{de}{Eine quadratische Matrix ist eine obere Dreiecksmatrix, wenn alle Einträge unterhalb der Diagonalen gleich $0$ sind.
Die Werte auf und über der Diagonalen dürfen beliebig sein ($0$ ist dabei nicht ausgeschlossen).
Lediglich die Matrizen $C$ und $H$ erfüllen dieses Kriterium nicht, weshalb es keine oberen Dreiecksmatrizen sind.
Alle anderen, also $A$, $B$, $D$, $F$ und $G$, sind obere Dreiecksmatrizen.}

\tab{\lang{de}{Untere Dreiecksmatrizen} \lang{en}{Lower triangular matrices}}

\lang{de}{Eine quadratische Matrix ist eine untere Dreiecksmatrix, wenn alle Einträge oberhalb der Diagonalen gleich $0$ sind.
Die Werte auf und unter der Diagonalen dürfen beliebig sein ($0$ ist dabei nicht ausgeschlossen).
Lediglich die Matrizen $D$ und $H$ erfüllen dieses Kriterium nicht, weshalb es keine unteren Dreiecksmatrizen sind.
Alle anderen, also $A$, $B$, $C$, $F$ und $G$, sind untere Dreiecksmatrizen.}

\lang{en}{A square matrix is a lower triangular matrix, if all the entries above the diagonal are equal to $0$.
The value of the entries on and below the diagonal may be any, including $0$.
Only the matrices $D$ and $H$ do not fulfill this criteria, which is why they are no lower triangular matrices.
But all the others, so $A$, $B$, $C$, $F$ and $G$ are lower triangular matrices.}
  %++++++++++++++++++++++++++++++++++++++++++++END_TAB_X

%#############################################################ENDE
\end{tabs*}
\end{content}