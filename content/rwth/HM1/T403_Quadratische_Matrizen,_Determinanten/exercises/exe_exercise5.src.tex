\documentclass{mumie.element.exercise}
%$Id$
\begin{metainfo}
  \name{
    \lang{de}{Ü06: Inverse Matrix}
    \lang{en}{Ex06: The inverse matrix}
  }
  \begin{description} 
 This work is licensed under the Creative Commons License Attribution 4.0 International (CC-BY 4.0)   
 https://creativecommons.org/licenses/by/4.0/legalcode 

    \lang{de}{}
    \lang{en}{}
  \end{description}
  \begin{components}
  \end{components}
  \begin{links}
  \end{links}
  \creategeneric
\end{metainfo}
\begin{content}
\usepackage{mumie.ombplus}

\title{\lang{de}{Ü06: Inverse Matrix} \lang{en}{Ex06: The inverse matrix}}

\begin{block}[annotation]
  Im Ticket-System: \href{http://team.mumie.net/issues/11360}{Ticket 11360}
\end{block}

%######################################################FRAGE_TEXT
\lang{de}{ 
Es seien $A,B$ und $C$ invertierbare $(n\times n)$-Matrizen über einem Körper $\mathbb{K}$.
Verwenden Sie die Rechenregeln für inverse Matrizen, um die folgenden Ausdrücke 
ohne Klammern 
%als Produkte in $A$, $B$, $C$, $A^{-1}$, $B^{-1}$ und $C^{-1}$ 
zu schreiben:
\begin{enumerate} 
\item[a)] $(ABC)^{-1}$,
\item[b)] $(AB^{-1})^{-1}$,
\item[c)] $(AB)^{-1}\cdot AC\cdot (BC)^{-1}$.
\end{enumerate}
 }

 \lang{en}{
 Let $A,B$ and $C$ be invertible $(n\times n)$-matrices over a field $\mathbb{K}$.
 Use the calculating rules for inverse matrices to write the following terms withouth brackets:
\begin{enumerate} 
\item[a)] $(ABC)^{-1}$,
\item[b)] $(AB^{-1})^{-1}$,
\item[c)] $(AB)^{-1}\cdot AC\cdot (BC)^{-1}$.
\end{enumerate}
 }
 

%##################################################ANTWORTEN_TEXT
\begin{tabs*}[\initialtab{0}\class{exercise}]

  %++++++++++++++++++++++++++++++++++++++++++START_TAB_X
  \tab{\lang{de}{    Antwort  } \lang{en}{Answer}}
  \begin{incremental}[\initialsteps{1}]
  
  	 %----------------------------------START_STEP_X
    \step 
    \lang{de}{   
\begin{enumerate}
\item[a)] $(ABC)^{-1}=C^{-1}B^{-1}A^{-1}$,
\item[b)] $(AB^{-1})^{-1}=BA^{-1}$,
\item[c)] $(AB)^{-1} AC (BC)^{-1}=B^{-1}B^{-1}=B^{-2}$.
\end{enumerate}
    }
    \lang{en}{   
\begin{enumerate}
\item[a)] $(ABC)^{-1}=C^{-1}B^{-1}A^{-1}$,
\item[b)] $(AB^{-1})^{-1}=BA^{-1}$,
\item[c)] $(AB)^{-1} AC (BC)^{-1}=B^{-1}B^{-1}=B^{-2}$.
\end{enumerate}
    }
  	 %------------------------------------END_STEP_X
 
  \end{incremental}
  %++++++++++++++++++++++++++++++++++++++++++++END_TAB_X
  %++++++++++++++++++++++++++++++++++++++++++START_TAB_X
  \tab{\lang{de}{    Lösung a)    } \lang{en}{Solution a)}}
  \begin{incremental}[\initialsteps{1}]
  
  	 %----------------------------------START_STEP_X
    \step 
    \lang{de}{   
Das Inverse des Produkts zweier Matrizen ist ja das Produkt der Inversen, jedoch in umgekehrter Reihenfolge:
\[  (AB)^{-1}=B^{-1}A^{-1}\]
für invertierbare Matrizen $A$ und $B$ gleicher Größe.
Damit rechnet man
\[ (ABC)^{-1}=((AB)C)^{-1}=C^{-1}(AB)^{-1}=C^{-1}B^{-1}A^{-1}.\]    }

\lang{en}{
The inverse of the product of two matrices is the product of the inverses, but in reversed order:
\[  (AB)^{-1}=B^{-1}A^{-1}\]
for invertible matrices $A$ and $B$ with the same size.
Using that, we calculate:
\[ (ABC)^{-1}=((AB)C)^{-1}=C^{-1}(AB)^{-1}=C^{-1}B^{-1}A^{-1}.\]    }
  	 %------------------------------------END_STEP_X
 
  \end{incremental}
  %++++++++++++++++++++++++++++++++++++++++++++END_TAB_X


  %++++++++++++++++++++++++++++++++++++++++++START_TAB_X
  \tab{\lang{de}{    Lösung b)    } \lang{en}{Solution b)}}
  \begin{incremental}[\initialsteps{1}]
  
  	 %----------------------------------START_STEP_X
    \step 
    \lang{de}{   
Wir wenden zunächst die Formel $(AC)^{-1}=C^{-1}A^{-1}$ für $C=B^{-1}$ an und erhalten
\[ (AB^{-1})^{-1}=(B^{-1})^{-1}A^{-1}.\]
Da die Inverse der Inversen wieder die Matrix selbst ist, ist nun $(B^{-1})^{-1}=B$ und somit 
\[ (AB^{-1})^{-1}=(B^{-1})^{-1}A^{-1}=BA^{-1}.\]
    }

  \lang{en}{
  First of all we use the formula $(AC)^{-1}=C^{-1}A^{-1}$ for $C=B^{-1}$ and get
\[ (AB^{-1})^{-1}=(B^{-1})^{-1}A^{-1}.\]
Because the inverse of the inverse is the matrix itself, it is $(B^{-1})^{-1}=B$ and therefore 
\[ (AB^{-1})^{-1}=(B^{-1})^{-1}A^{-1}=BA^{-1}.\]
    }
  	 %------------------------------------END_STEP_X
 
  \end{incremental}
  %++++++++++++++++++++++++++++++++++++++++++++END_TAB_X


  %++++++++++++++++++++++++++++++++++++++++++START_TAB_X
  \tab{\lang{de}{    Lösung c)    } \lang{en}{Solution c)}}
  \begin{incremental}[\initialsteps{1}]
  
  	 %----------------------------------START_STEP_X
    \step 
    \lang{de}{   
Auch hier wenden wir wieder die Regel für die Inverse des Produkts an:
\[ (AB)^{-1} AC (BC)^{-1}=B^{-1}A^{-1} AC C^{-1}B^{-1}.\]
Nun sind aber $A^{-1} A=E_n$ und $C C^{-1}=E_n$, sowie $E_n\cdot D=D$ und $D\cdot E_n=D$ für jede $(n\times n)$-Matrix $D$, also
\[ B^{-1}A^{-1} AC C^{-1}B^{-1}=B^{-1}E_nE_nB^{-1}=B^{-1}B^{-1}.\]
Letzteres schreibt man auch - wie bei den negativen Potenzen von Zahlen - als $B^{-2}$.
    }

    \lang{en}{   
    Here we also use the rule for the inverse of a product:
\[ (AB)^{-1} AC (BC)^{-1}=B^{-1}A^{-1} AC C^{-1}B^{-1}.\]
But it is $A^{-1} A=I_n$ and $C C^{-1}=I_n$, plus $I_n\cdot D=D$ and $D\cdot I_n=D$ for each $(n\times n)$-matrix $D$, so
\[ B^{-1}A^{-1} AC C^{-1}B^{-1}=B^{-1}I_n I_nB^{-1}=B^{-1}B^{-1}.\]
The latter we also write - as with the negative powers of numbers - as $B^{-2}$.
    }
  	 %------------------------------------END_STEP_X
 
  \end{incremental}
  %++++++++++++++++++++++++++++++++++++++++++++END_TAB_X


%#############################################################ENDE
\end{tabs*}
\end{content}