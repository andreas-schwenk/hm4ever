\documentclass{mumie.element.exercise}
%$Id$
\begin{metainfo}
  \name{
    \lang{de}{Ü10: Determinante}
    \lang{en}{Ex10: Determinant}
  }
  \begin{description} 
 This work is licensed under the Creative Commons License Attribution 4.0 International (CC-BY 4.0)   
 https://creativecommons.org/licenses/by/4.0/legalcode 

    \lang{de}{}
    \lang{en}{}
  \end{description}
  \begin{components}
  \end{components}
  \begin{links}
  \end{links}
  \creategeneric
\end{metainfo}
\begin{content}
\begin{block}[annotation]
	Im Ticket-System: \href{https://team.mumie.net/issues/21087}{Ticket 21087}
\end{block}
\begin{block}[annotation]
Abwandlung von \href{http://team.mumie.net/issues/11362}{Ticket 11362}: content/rwth/HM1/T403_Quadratische_Matrizen,_Determinanten/exercises/exe_exercise7.src.tex

Dieselben Matrizen wie in Aufgabe 7, nur wird diesmal Laplace-entwickelt.
\end{block}

\usepackage{mumie.ombplus}

\title{\lang{de}{Ü10: Determinante} \lang{en}{Ex10: Determinants}}


%######################################################FRAGE_TEXT
\lang{de}{ 
Bestimmen Sie mit Hilfe des Laplace-Entwicklungssatzes die Determinanten der folgenden Matrizen:
\begin{align*}
\text{a) }
A=\left( \begin{smallmatrix}
0 & 0 & 1 & 0 \\
0 & -1 & 0 & -3 \\
1 & 2 & 0 & 6 \\
0 & 0 & 0 & 1 
\end{smallmatrix} \right), \qquad
\text{b) }
B=\left( \begin{smallmatrix}
2 & 3 & -4 \\
3 & 3 & -1 \\
0 & 3 & -9 
\end{smallmatrix} \right),\qquad
\text{c) }
C=\left( \begin{smallmatrix}
1&3& 3 & -1 & -2 \\
2 &4&3&0&-1\\
0&0&0&0&5\\
0&0&1&2&1\\
2&6&6&2&4 
\end{smallmatrix} \right).
\end{align*}
 }

\lang{en}{
Determine with the help of Laplace expansion the determinants of the following matrices:
\begin{align*}
\text{a) }
A=\left( \begin{smallmatrix}
0 & 0 & 1 & 0 \\
0 & -1 & 0 & -3 \\
1 & 2 & 0 & 6 \\
0 & 0 & 0 & 1 
\end{smallmatrix} \right), \qquad
\text{b) }
B=\left( \begin{smallmatrix}
2 & 3 & -4 \\
3 & 3 & -1 \\
0 & 3 & -9 
\end{smallmatrix} \right),\qquad
\text{c) }
C=\left( \begin{smallmatrix}
1&3& 3 & -1 & -2 \\
2 &4&3&0&-1\\
0&0&0&0&5\\
0&0&1&2&1\\
2&6&6&2&4 
\end{smallmatrix} \right).
\end{align*}
 }

%##################################################ANTWORTEN_TEXT
\begin{tabs*}[\initialtab{0}\class{exercise}]

  %++++++++++++++++++++++++++++++++++++++++++START_TAB_X
  \tab{\lang{de}{    Antwort    } \lang{en}{Answer}}
  \begin{incremental}[\initialsteps{1}]
  
  	 %----------------------------------START_STEP_X
    \step 
    \lang{de}{   
\[ \text{a) } \det(A)=1,\quad \text{b) } \det(B)=-3,\quad \text{c) } \det(C)=-40 .\]    }
\lang{en}{   
\[ \text{a) } \det(A)=1,\quad \text{b) } \det(B)=-3,\quad \text{c) } \det(C)=-40 .\]    }
  	 %------------------------------------END_STEP_X
 
  \end{incremental}
  %++++++++++++++++++++++++++++++++++++++++++++END_TAB_X


  %++++++++++++++++++++++++++++++++++++++++++START_TAB_X
  \tab{\lang{de}{    Lösung a)    } \lang{en}{Solution a)}}
  \begin{incremental}[\initialsteps{1}]
  
  	 %----------------------------------START_STEP_X
    \step 
    \lang{de}{   
    Besonders dünnbesetzt sind hier die erste und die vierte Zeile sowie die dritte Spalte.
    Wir wählen die Entwicklung nach der ersten Zeile. Die anderen Entwicklungen sind selbstverständlich ebenso möglich.
    Weil alle Einträge der ersten Zeile außer dem dritten null sind, erhalten wir zunächst
\[\det(A)=(-1)^{1+3}\cdot 1\cdot \det(A_{13})=\det\Big( \left( \begin{smallmatrix}0&-1&-3\\1&2&6\\0&0&1\end{smallmatrix} \right) \Big).\]
Die noch zu bestimmende $(3\times 3)$-Determinante berechnen wir durch Laplace-Entwicklung nach der ersten Spalte 
(ebenso möglich wäre  die Entwicklung nach der dritten Zeile o.ä.)
\[\det\Big( \left( \begin{smallmatrix}0&-1&-3\\1&2&6\\0&0&1\end{smallmatrix} \right) \Big)=(-1)^{1+2}\cdot 1\cdot
\det\Big( \left( \begin{smallmatrix}-1&-3\\0&1\end{smallmatrix} \right) \Big)=(-1)\cdot(-1)=1.\]
Somit ist auch $\det(A) =1$.}

\lang{en}{   
The first and fourth row as well as the third column contain a lot of zeros.
We choose the expansions along the first row. The other expansion may be chosen as well.
Because all the entries in the first row exept the third are equal to zero, we receive
\[\det(A)=(-1)^{1+3}\cdot 1\cdot \det(A_{13})=\det\Big( \left( \begin{smallmatrix}0&-1&-3\\1&2&6\\0&0&1\end{smallmatrix} \right) \Big).\]
We still need to calculate the determinant of the $(3\times 3)$-matrix. We do that with the Laplace expansion along with the first column.
(You may also expand along with the third row.)
\[\det\Big( \left( \begin{smallmatrix}0&-1&-3\\1&2&6\\0&0&1\end{smallmatrix} \right) \Big)=(-1)^{1+2}\cdot 1\cdot
\det\Big( \left( \begin{smallmatrix}-1&-3\\0&1\end{smallmatrix} \right) \Big)=(-1)\cdot(-1)=1.\]
Therefore it is $\det(A) =1$.}
  	 %------------------------------------END_STEP_X
 
  \end{incremental}
  %++++++++++++++++++++++++++++++++++++++++++++END_TAB_X


  %++++++++++++++++++++++++++++++++++++++++++START_TAB_X
  \tab{\lang{de}{    Lösung b)    } \lang{en}{ Solution b)}}
  \begin{incremental}[\initialsteps{1}]
  
  	 %----------------------------------START_STEP_X
    \step 
    \lang{de}{   
    Hier bietet sich die Entwicklung nach der ersten Spalte oder der letzten Zeile an. 
    Wir entscheiden uns für die erste Spalte und erhalten im ersten Schritt
    \[\det(B)=(-1)^{1+1}\cdot2\cdot\det\Big( \left( \begin{smallmatrix} 3&-1\\3&-9\end{smallmatrix} \right) \Big) +(-1)^{2+1}\cdot 3\cdot\det\Big( \left( \begin{smallmatrix} 3&-4\\3&-9\end{smallmatrix} \right) \Big)\]
    Mit der Formel für $(2\times 2)$-Determinanten rechnen wir weiter
    \[ \det(B)=2(-27+3)-3(-27+12)=-48+45=-3.\]
    }

        \lang{en}{   
    It makes sense to expand along with the first column or the last row.
    We decide to expand along with the first column and get in the first step
    \[\det(B)=(-1)^{1+1}\cdot2\cdot\det\Big( \left( \begin{smallmatrix} 3&-1\\3&-9\end{smallmatrix} \right) \Big) +(-1)^{2+1}\cdot 3\cdot\det\Big( \left( \begin{smallmatrix} 3&-4\\3&-9\end{smallmatrix} \right) \Big)\]
    We continue with the formula for $(2\times 2)$-determinants
    \[ \det(B)=2(-27+3)-3(-27+12)=-48+45=-3.\]
    }
  	 %------------------------------------END_STEP_X
 
  \end{incremental}
  %++++++++++++++++++++++++++++++++++++++++++++END_TAB_X


  %++++++++++++++++++++++++++++++++++++++++++START_TAB_X
  \tab{\lang{de}{    Lösung c)    } \lang{en}{Solution c)}}
  \begin{incremental}[\initialsteps{1}]
  
  	 %----------------------------------START_STEP_X
    \step 
    \lang{de}{ Hier entwickeln wir am besten nach der dritten Zeile. Dann ist
    \[\det(C)=(-1)^{3+5}\cdot 5\cdot
    \det\Big( \left( \begin{smallmatrix} 1&3&3&-1\\2&4&3&0\\0&0&1&2\\2&6&6&2\end{smallmatrix} \right) \Big).\]
    Diese $(4\times 4)$-Determinante entwickeln wir nach der dritten Zeile und erhalten
    \[ \det(C)=5\cdot\left[(-1)^{3+3}\cdot 1\cdot 
    \det\Big( \left( \begin{smallmatrix} 1&3&-1\\2&4&0\\2&6&2\end{smallmatrix} \right) \Big)
    +(-1)^{3+4}\cdot 2\cdot
    \det\Big( \left( \begin{smallmatrix} 1&3&3\\2&4&3\\2&6&6\end{smallmatrix} \right) \Big)\right].\]
    Darin entwickeln wir die erste Determinante nach der letzten Spalte. 
    Die zweite Determinante ist null, denn die letzte Zeile ist das Doppelte der ersten, also ist der Rang höchstens zwei.
    Damit erhalten wir
    \begin{align*}
    \det(A)&=5\cdot\left[(-1)^{1+3}\cdot(-1)\cdot\det\Big( \left( \begin{smallmatrix} 2&4\\2&6\end{smallmatrix} \right) \Big)
    +(-1)^{3+3}\cdot 2\cdot
    \det\Big( \left( \begin{smallmatrix} 1&3\\2&4\end{smallmatrix} \right) \Big)+0\right]\\
    &=5\cdot\left[-(12-8)+2(4-6)\right]=5\cdot(-4-4)=-40.
    \end{align*}
    }
    \lang{en}{ It is reasonable to expand along with the third row. Then it is
    \[\det(C)=(-1)^{3+5}\cdot 5\cdot
    \det\Big( \left( \begin{smallmatrix} 1&3&3&-1\\2&4&3&0\\0&0&1&2\\2&6&6&2\end{smallmatrix} \right) \Big).\]
    We expand this $(4\times 4)$-determinant along with the third row and get
    \[ \det(C)=5\cdot\left[(-1)^{3+3}\cdot 1\cdot 
    \det\Big( \left( \begin{smallmatrix} 1&3&-1\\2&4&0\\2&6&2\end{smallmatrix} \right) \Big)
    +(-1)^{3+4}\cdot 2\cdot
    \det\Big( \left( \begin{smallmatrix} 1&3&3\\2&4&3\\2&6&6\end{smallmatrix} \right) \Big)\right].\]
    We expand the first determinant along with the last column. 
    The second determinant is zero, because the last row is twice the first, so the rank is at most two.
    With this we get
    \begin{align*}
    \det(A)&=5\cdot\left[(-1)^{1+3}\cdot(-1)\cdot\det\Big( \left( \begin{smallmatrix} 2&4\\2&6\end{smallmatrix} \right) \Big)
    +(-1)^{3+3}\cdot 2\cdot
    \det\Big( \left( \begin{smallmatrix} 1&3\\2&4\end{smallmatrix} \right) \Big)+0\right]\\
    &=5\cdot\left[-(12-8)+2(4-6)\right]=5\cdot(-4-4)=-40.
    \end{align*}
    }
  	 %------------------------------------END_STEP_X
 
  \end{incremental}
  %++++++++++++++++++++++++++++++++++++++++++++END_TAB_X


%#############################################################ENDE
\end{tabs*}
\end{content}