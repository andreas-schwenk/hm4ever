
%$Id:  $
\documentclass{mumie.article}
%$Id$
\begin{metainfo}
  \name{
    \lang{de}{Überblick: Einfache Funktionen}
    \lang{en}{Overview: Basic functions}
  }
  \begin{description} 
 This work is licensed under the Creative Commons License Attribution 4.0 International (CC-BY 4.0)   
 https://creativecommons.org/licenses/by/4.0/legalcode 

    \lang{de}{Beschreibung}
    \lang{en}{Description}
  \end{description}
  \begin{components}
  \end{components}
  \begin{links}
\link{generic_article}{content/rwth/HM1/T102neu_Einfache_Reelle_Funktionen/g_art_content_08_quadratische_funktionen.meta.xml}{content_08_quadratische_funktionen}
\link{generic_article}{content/rwth/HM1/T102neu_Einfache_Reelle_Funktionen/g_art_content_07_geradenformen.meta.xml}{content_07_geradenformen}
\link{generic_article}{content/rwth/HM1/T102neu_Einfache_Reelle_Funktionen/g_art_content_06_funktionsbegriff_und_lineare_funktionen.meta.xml}{content_06_funktionsbegriff_und_lineare_funktionen}
\end{links}
  \creategeneric
\end{metainfo}
\begin{content}
\begin{block}[annotation]
	Im Ticket-System: \href{https://team.mumie.net/issues/30147}{Ticket 30147}
\end{block}


\begin{block}[annotation]
Im Entstehen: Überblicksseite für Kapitel Einfache Funktionen
\end{block}

\usepackage{mumie.ombplus}
\ombchapter{1}
\lang{de}{\title{Überblick: Einfache Funktionen}}
\lang{en}{\title{Overview: Basic functions}}



\begin{block}[info-box]
\lang{de}{\strong{Inhalt}}
\lang{en}{\strong{Contents}}


\lang{de}{
    \begin{enumerate}%[arabic chapter-overview]
   \item[2.1]  \link{content_06_funktionsbegriff_und_lineare_funktionen}{Reelle Funktionen allgemein und lineare Funktionen}
   \item[2.2] \link{content_07_geradenformen}{Darstellungsformen von Geraden}
   \item[2.3] \link{content_08_quadratische_funktionen}{Quadratische Funktionen}
     \end{enumerate}
}
\lang{en}{
    \begin{enumerate}%[arabic chapter-overview]
   \item[2.1]  \link{content_06_funktionsbegriff_und_lineare_funktionen}{Real functions and linear real functions}
   \item[2.2] \link{content_07_geradenformen}{Equations of a line}
   \item[2.3] \link{content_08_quadratische_funktionen}{Quadratic functions and parabolas}
     \end{enumerate}
}

\end{block}

\begin{zusammenfassung}

\lang{de}{
Funktionen gehören zu den wichtigsten mathematischen Objekten überhaupt.
Der Funktionsbegriff ist universell und beschreibt Prozesse in den Anwendungen \textit{aller} Bereiche, in denen mathematische Konzepte benutzt werden.
Deshalb geben wir hier (obwohl wir uns auf den Fall einer reellen Funktion in einer Variablen beschränken) eine
strenge mathematische Definition bestehend aus Definitionsmenge und eindeutiger Zuordnung eines Wertes (Bildes) für jedes Element der Definitionsmenge im Zielbereich.
\\
Wir bestimmen den Graph einer Funktion und ordnen die Bestandteile linearer und quadratischer Abbildungen Eigenschaften ihrer Graphen zu.
}
\lang{en}{
Functions are one of the most important mathematical objects. 
The definition of a function is universal and describes processes in applications through 
\textit{every} field in which mathematical concepts are put to use. 
For this reason we give a formal definition of a function, including its domain, codomain and graph 
(although we restrict ourselves to real functions in a single variable). Especially important is that 
functions must map every element in the domain to precisely one element in the codomain.
\\
We discuss the graphs of linear and quadratic functions, and consider the effect of the coefficients 
on the shape and position of the line or parabola respectively.
}


\end{zusammenfassung}

\begin{block}[info]\lang{de}{\strong{Lernziele}}
\lang{en}{\strong{Learning Goals}} 
\begin{itemize}[square]
\item \lang{de}{
Sie kennen die Begriffe Definitions- und Zielbereich, Werte- bzw. Bildmenge und Abbildungsvorschrift 
und deren Bedeutung für den Funktionsbegriff.
}
\lang{en}{
Knowing the definitions of a function, domain, codomain, image and graph.
}
\item \lang{de}{
Sie skizzieren Graphen linearer und quadratischer Funktionen und lesen Nullstellen 
und weitere Funktionseigenschaften aus Graphen und Funktionsvorschriften ab.
}
\lang{en}{
Being able to sketch graphs of linear and quadratic functions and find characteristics such as its 
gradient, $y$-intercept, roots and turning point from both its definition and its graph.
}
\item \lang{de}{Sie bestimmen maximale Definitionsbereiche und Bildbereiche einfacher Funktionen.}
\lang{en}{Being able to find the maximal domain and image of a given basic function.}
\end{itemize}
\end{block}




\end{content}
