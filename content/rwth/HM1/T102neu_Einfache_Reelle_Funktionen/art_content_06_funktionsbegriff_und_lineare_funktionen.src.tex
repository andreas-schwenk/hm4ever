%$Id:  $
\documentclass{mumie.article}
%$Id$
\begin{metainfo}
  \name{
    \lang{de}{Lineare reelle Funktionen}
%   \lang{de}{Reelle Funktionen allgemein und Lineare Funktionen}
    \lang{en}{Linear real functions}
  }
  \begin{description} 
 This work is licensed under the Creative Commons License Attribution 4.0 International (CC-BY 4.0)   
 https://creativecommons.org/licenses/by/4.0/legalcode 

    \lang{de}{Beschreibung}
    \lang{en}{Description}
  \end{description}
  \begin{components}
    \component{generic_image}{content/rwth/HM1/images/g_tkz_T102_Graph_B.meta.xml}{T102_Graph_B}
    \component{generic_image}{content/rwth/HM1/images/g_tkz_T102_Graph_A.meta.xml}{T102_Graph_A}
    \component{generic_image}{content/rwth/HM1/images/g_tkz_T102_FunctionGraph.meta.xml}{T102_FunctionGraph}
    \component{generic_image}{content/rwth/HM1/images/g_img_00_Videobutton_schwarz.meta.xml}{00_Videobutton_schwarz}
    \component{generic_image}{content/rwth/HM1/images/g_img_00_Videobutton_blau.meta.xml}{00_Videobutton_blau}
    %\component{generic_image}{content/rwth/HM1/images/g_img_T102_funktionsgraph_geraden.meta.xml}{funktionsgraph_geraden}
    \component{js_lib}{system/media/mathlets/GWTGenericVisualization.meta.xml}{mathlet1}
  \end{components}
  \begin{links}
    \link{generic_article}{content/rwth/HM1/T101neu_Elementare_Rechengrundlagen/g_art_content_05_loesen_gleichungen_und_lgs.meta.xml}{content_05_loesen_gleichungen_und_lgs}
    \link{generic_article}{content/rwth/HM1/T102neu_Einfache_Reelle_Funktionen/g_art_content_06_funktionsbegriff_und_lineare_funktionen.meta.xml}{content_06_funktionsbegriff_und_lineare_funktionen}
    \link{generic_article}{content/rwth/HM1/T102neu_Einfache_Reelle_Funktionen/g_art_content_07_geradenformen.meta.xml}{content_07_geradenformen}
    \link{generic_article}{content/rwth/HM1/T102neu_Einfache_Reelle_Funktionen/g_art_content_08_quadratische_funktionen.meta.xml}{content_08_quadratische_funktionen}
    \link{generic_article}{content/rwth/HM1/T103_Polynomfunktionen/g_art_content_09_polynome.meta.xml}{content_09_polynome}
    \link{generic_article}{content/rwth/HM1/T403a_Vektorraum/g_art_content_10b_lineare_abb.meta.xml}{content_10b_lineare_abb}
    \link{generic_article}{content/rwth/HM1/T204_Abbildungen_und_Funktionen/g_art_content_10_abbildungen_verkettung.meta.xml}{content_10_abbildungen_verkettung}
\end{links}
  \creategeneric
\end{metainfo}

\begin{content}
%
% ursprüngliche Version:
%
%   \href{http://team.mumie.net/issues/8972}{Ticket 8972}: 
%           content/rwth/HM1/T102_Einfache_Funktionen,_grundlegende_Begriffe/art_content_04_lineare_funktionen.src.tex
%
% \begin{block}[annotation]
%    Im Ticket-System: \href{http://team.mumie.net/issues/8973}{Ticket 8972}\\
% \end{block}
% 
% Ticket neu: 
%
\begin{block}[annotation]
	Im Ticket-System: \href{https://team.mumie.net/issues/21161}{Ticket 21161}
\end{block}

\usepackage{mumie.ombplus}
\ombchapter{2}
\ombarticle{1}

\usepackage{mumie.genericvisualization}

\begin{visualizationwrapper}

\title{\lang{de}{Reelle Funktionen allgemein und lineare Funktionen}
       \lang{en}{Real functions and linear real functions}}

 
\begin{block}[annotation]
 reelle Funktion, Definitionsbereich und Wertemenge, Lineare Funktionen, Steigung, Nullstelle
\end{block}


% \begin{block}[important]
%    Dieses Kapitel wird derzeit überarbeitet und ist somit nur eine Arbeitsversion -
%    ohne Anspruch auf Vollständigkeit, Konsistenz und Fehlerfreiheit.
% \end{block}

\begin{block}[info-box]
\tableofcontents
\end{block}


\section{\lang{de}{Funktionsbegriff reeller Funktionen}
         \lang{en}{Definition of a real function}}\label{sec:funktion} 

\lang{de}{
Bevor wir uns mit den einfachen reellen Funktionen und ihren Eigenschaften befassen, definieren wir zunächst, 
was wir generell unter einer Funktion verstehen und welche Begriffe wir im Zusammenhang
mit Funktionen verwenden werden. Dabei befassen wir uns hier im Grundlagen-Teil vorerst ausschließlich
mit Funktionen über den \emph{reellen Zahlen}. Im Analysis-Teil dieses Kurses werden Funktionen allgemeiner als 
\ref[content_10_abbildungen_verkettung][Abbildungen]{sec:abbildungen} zwischen beliebigen Mengen betrachtet.
}
\lang{en}{
Before we look at specific real functions and their properties, we explain what is meant by 
\emph{function}, and provide terminology that is used in reference to functions. For this, we begin by 
considering only functions over the \emph{real numbers}. In the analysis part of this course, functions are more generally considered as 
\ref[content_10_abbildungen_verkettung][operators]{sec:abbildungen} between two arbitrary given sets.
}


\begin{definition}[\lang{de}{Reelle Funktion}\lang{en}{Real function}] \label{def:reelle_funktion}
     \lang{de}{
	   Eine \emph{reelle Funktion} $\,f\,$ ist eine Vorschrift, die jeder 
     Zahl $x$ aus einer vorgegebenen, nichtleeren Teilmenge $D$ der reellen Zahlen 
     eindeutig eine reelle Zahl $\,y=:f(x)\,$ zuordnet. Man sagt auch \emph{"`$x\,$ wird 
     durch $\,f\,$ abgebildet auf $\,f(x)$"'}$\,$ und man schreibt formal
     \[f:D\to \R, \; x\mapsto f(x).\]
  	 \\
     Dabei nennt man $\,x\,$ die \notion{\emph{Variable}} oder das 
     \notion{\emph{Argument}} der Funktion $\,f\,$
     und $\,y\,$ bzw. $\, f(x) \,$ den eindeutig zugeordneten \notion{\emph{Funktionswert}} oder auch
     das \notion{\emph{Bild}} von $\,x\,$ unter $\,f$.
     }
  	 \lang{en}{
     A \emph{real function} $\,f\,$ is a rule which sends \emph{('maps')} each number $x$ in a nonempty 
     subset of the real numbers to a unique real number $\,y=:f(x)\,$. We also say that '$x\,$ is mapped 
     to $\,f(x)$ by $\,f\,$', and formally write 
     \[f:D\to \R, \; x\mapsto f(x).\]
  	 \\
     Here $\,x\,$ is called the \notion{\emph{independent variable}} or \notion{\emph{argument}} of the 
     function $\,f\,$ and $\,y\,$ or $\,f(x)\,$ is called the \notion{\emph{dependent variable}}, 
     \notion{\emph{value}} or \notion{\emph{image}} of $\,x\,$ under $\,f$.
     }
\end{definition}


\begin{definition}[\lang{de}{Definitionsbereich, Zielbereich, Bildmenge, Wertemenge}
                   \lang{en}{Domain, codomain, image and range}]\label{dd}
    \lang{de}{
    Für eine reelle Funktion $\, f:D \to Z \,$ ist 
    $\, D \, $ (auch $D_f$) der \notion{\emph{Definitionsbereich}} 
    und $\, Z$\,  der \notion{\emph{Zielbereich}} von $f$.
    \\              
    Die Menge aller Funktionswerte (Bilder) von Elementen aus $D$ unter $f$ 
    \[ 
    W_f := \{f(x) \mid x\in D\} \subseteq Z 
		\]
    ist eine Teilmenge des Zielbereichs von $f$ und wird bezeichnet als die \notion{\emph{Bildmenge}}
    oder auch die \notion{\emph{Wertemenge}} von $f$.
    }
    \lang{en}{
    For a real function $\, f:D \to Z \,$, we call $\, D \, $ (or $D_f$) the \notion{\emph{domain}} of $f$ 
    and $\, Z$\, the \notion{\emph{codomain}} of $f$.
    \\
    The set of all function values of elements in $D$ under $f$ 
    \[ 
    W_f := \{f(x) \mid x\in D\} \subseteq Z 
		\]
    is a subset of the codomain of $f$ and is called the \notion{\emph{image}} of $f$. Sometimes it is 
    also called the \notion{\emph{range}} of $f$, but this term can also refer to the codomain.
    }
\end{definition} 	

\lang{de}{
Die \emph{Bildmenge von $f$} bezeichnet man auch als das \emph{Bild von $D$ 
unter $f$} und schreibt kurz $f(D)$.
}
\lang{en}{
The \emph{image of $f$} is also called the \emph{image of $D$ under $f$}, and can be written as $f(D)$.
}

\begin{example}\label{example:bereiche}
\begin{enumerate}

\item \lang{de}{
Die Funktion $f:\R\to \R,\, x\mapsto x^2$ hat als Definitionsbereich die Menge $\,D_f=\R$. 
Das \emph{Bild von $\,\R$ unter $\,f$} besteht jedoch nur aus den nicht-negativen reellen Zahlen, 
denn für jede relle Zahl $\, x\,$ ist $\, x^2 \geq 0$. Deshalb gilt für die Wertemenge $W_f=[0,\infty)$.
}
\lang{en}{
The function $f:\R\to \R,\, x\mapsto x^2$ has domain $\,D_f=\R$. The \emph{image of $\,\R$ under $\,f$} 
consists of only the non-negative real numbers, as for every real number $\, x\,$, we have 
$\, x^2 \geq 0$. Hence the image of $f$ is $W_f=[0,\infty)$.
}

\item \lang{de}{
Die Funktion $g:\R\setminus \{1\}\to \R,\, x\mapsto \frac{1}{1-x}$ hat als Definitionsbereich die
Menge $\,D_g=\R\setminus \{1\}$, da die Division durch $0$ nicht erlaubt ist. Das \emph{Bild
von $\,D_g\,$ unter $\,g\,$} liefert dann alle reellen Zahlen außer $\,0$. Die Wertemenge 
ist folglich $\, W_g=\R\setminus \{0\}$.
}
\lang{en}{
The function $g:\R\setminus \{1\}\to \R,\, x\mapsto \frac{1}{1-x}$ has domain $\,D_g=\R\setminus \{1\}$, 
as division by $0$ is not possible, hence the exclusion of $1$ from the domain to avoid this. The 
\emph{image of $\,D_g\,$ under $\,g\,$} is therefore all real numbers besides $\,0$. That is, the image of 
$f$ is $\, W_g=\R\setminus \{0\}$.
}

% weil jede Zahl $y\ne 0$ der Funktionswert einer Zahl $x$ aus der Menge $\R\setminus \{1\}$ ist, nämlich von der Zahl $x=1-\frac{1}{y}$.

\end{enumerate}
\end{example}

% \begin{block}[warning]\label{max-defbereich}
\begin{remark}     \label{max-defbereich}
\begin{itemize}  
 \item  \lang{de}{
        Wie in \ref{dd} definiert, entspricht die \emph{Wertemenge} einer Funktion $\,f\,$
        dem \emph{Bild des Definitionsbereichs unter $f$}. Das bedeutet in der Umkehrung
        aber auch, dass es zu jedem Element $\,y\,$ der \emph{Wertemenge} eine Zahl $\,x\,$ 
        im Definitionsbereich gibt mit $\,f(x)=y.$ Ist dies nicht sichergestellt, so spricht man
        allgemeiner von dem \emph{Zielbereich} einer Funktion $f$, der eine Obermenge
        der \emph{Werte-} bzw. \emph{Bildmenge} ist.
        \\
        Im 2. Beispiel von \ref{example:bereiche} lässt sich nachweisen, dass 
        $\, W_g =\R\setminus \{0\} \,$ die \emph{Werte-} bzw. \emph{Bildmenge} von $\,g\,$ ist. 
        Jede Zahl $\,y\in \R\setminus \{0\}\,$ ist nämlich Funktionswert einer Zahl $\,x\,$ aus 
        dem Definitionsbereich $D_g=\R\setminus \{1\}$, und zwar $\,x=1-\frac{1}{y}$.
        }
        \lang{en}{
        As defined in \ref{dd}, the \emph{image} of a function $\,f\,$ can also be defined as the image 
        of the domain under $\,f\,$, which is the subset of the codomain containing the image of every 
        element of the domain under $\,f\,$. This means that for every element $\,y\,$ in the image of 
        $\,f\,$ there exists a number $\,x\,$ such that $\,f(x)=y$. This is not guaranteed in the 
        \emph{codomain} of $\,f\,$, as $\,f\,$ does not need to map to every element of the codomain. 
        This is why the image of $\,f\,$ is a subset of the codomain rather than equal to it.
        \\
        In the second example of \ref{example:bereiche}, it is easy to check that 
        $\, W_g =\R\setminus \{0\} \,$ is the \emph{image} of $\,g\,$. For each number 
        $\,y\in \R\setminus \{0\}\,$, we can find a number $\,x=1-\frac{1}{y}$ in the domain 
        $D_g=\R\setminus \{1\}$, which clearly maps to $\,y\,$ under $\,g\,$.
        }
 
 \item  \lang{de}{
      Beim Rechnen mit reellen Funktionen wird oft nur die Funktionsvorschrift angegeben, 
      z.\,B. $f(x)=x^2$ oder $g(x)=\frac{1}{1-x}$, ohne den Definitionsbereich explizit zu nennen. 
      Man impliziert damit dann, dass der Definitionsbereich der sogenannte \notion{\emph{maximale 
      Definitionsbereich}} ist, d.\,h. die Teilmenge \emph{aller} reellen Zahlen, die als Wert f"ur 
      die Variable $\,x\,$ in die Funktionsvorschrift $\,f(x)\,$ bzw. $\,g(x)\,$ eingesetzt werden
      d"urfen.
      \\
      In den Beispielen \ref{example:bereiche} wäre der maximale Definitionsbereich von $\,f\,$ somit 
      $\,D_f=\R\,$ und der von $\,g\,$ wäre $\,D_g=\R\setminus \{1\}$.
      }
      \lang{en}{
      When real functions are used in mathematics, often only part of their definition is given, for 
      example $f(x)=x^2$ or $g(x)=\frac{1}{1-x}$, without explicitly mentioning the domain or codomain. 
      The underlying assumption is that the domain is the so-called \notion{\emph{maximal domain}}, which 
      is the subset of \emph{all} real numbers which can be substituted into $x$ in the given function 
      $\,f(x)\,$ or $\,g(x)\,$ without breaking any rules (division by zero, etc.).
      \\
      In example \ref{example:bereiche}, the maximal domain of $\,f\,$ is therefore 
      $\,D_f=\R\,$ and the maximal domain of $\,g\,$ is $\,D_g=\R\setminus \{1\}$.
      }

\end{itemize}
\end{remark}
% \end{block}

\begin{quickcheck}
		\field{rational}
		\type{input.number}
		\begin{variables}
			\randint[Z]{d}{-5}{5}
			\randint{c}{-4}{4}
			\randint[Z]{b}{1}{4}
			\function[calculate]{a}{b*sign(d)*sign(c)}
			\function[calculate]{y}{d}
		    \function[normalize]{f}{a*x+d}
		    \function{g}{c-x}		    
		\end{variables}
		
			\text{\lang{de}{
      Welche reelle Zahl liegt nicht im maximalen Definitionsbereich der Funktion
			$f(x)=\frac{\var{f}}{\var{g}}$?\\ Die Zahl \ansref.
      }
      \lang{en}{
      Which real number lies in the maximal domain of the function 
      $f(x)=\frac{\var{f}}{\var{g}}$?\\ The number \ansref.
      }}
		
		\begin{answer}
			\solution{c}
		\end{answer}
		\explanation{\lang{de}{
    Die einzige Zahl, die in die Vorschrift nicht eingesetzt werden darf, ist die, für die
		der Nenner $\var{g}$ gleich $0$ wird, hier also $\,x=\var{c}$.
    }
    \lang{en}{
    The only number that is not allowed to be substituted into the function definition is the one for 
    which the denominator $\var{g}$ is equal to $0$, so $\,x=\var{c}$.
    }}
	\end{quickcheck}


\section{\lang{de}{Graph einer reellen Funktion}
         \lang{en}{Graph of a real function}}\label{sec:graph}
    \lang{de}{
    Funktionale Zusammenhänge werden in der Praxis oft in einer \emph{(Werte-) Tabelle}
    oder \emph{graphisch} in einem zweidimensionalen Koordinatensystem dargestellt.  
    Eine \emph{Wertetabelle} ist dabei eine Tabelle mit zwei Zeilen (oder Spalten), in die einzelne diskrete
    Werte des Definitionsbereichs und die dazugehörigen Funktionswerte eingetragen werden.
    }
    \lang{en}{
    Functions are in practice often represented by \emph{(function) tables} or two-dimensional 
    \emph{graphs}. A \emph{function table} is a table with two rows (or columns), in which individual, 
    discrete values in the domain are matched with their images under the function.
    }
    
    \begin{table}[\align{c} \cellaligns{ccccccc}]
        $\;\mathbf{x}\;$ && $\,x_1\,$ & $\,x_2\,$ & $\,x_3\,$ & $\,x_4\,$ & ...\\
        $\;\mathbf{f(x)}\;$ && $\,y_1\,$ & $\,y_2\,$ & $\,y_3\,$ & $\,y_4\,$ & ...\\
     \end{table}

   \lang{de}{
   Sie dient in der Regel zur Erstellung der \emph{graphischen} Darstellung der Funktion im
   zweidimensionalen Koordinatensystem.
   }
   \lang{en}{
   It is often useful to have such a table in order to then draw the \emph{graph} of the function in  
   two-dimensional coordinates.
   }

    \begin{center}
\image{T102_FunctionGraph}
    \end{center}

 \lang{de}{
 Die waagerechte Achse des Koordinatensystems heißt \emph{x-Achse} oder auch \notion{\emph{Abszisse}} 
 und die senkrechte Achse ist die \emph{y-Achse} oder auch die \notion{\emph{Ordinate}}.
 }
 \lang{en}{
 The horizontal axis of the graph is called the \emph{$x$-axis} or also the \notion{\emph{abscissa}}, and 
 the vertical axis is the \emph{$y$-axis} or also the \notion{\emph{ordinate}}.
 }

\begin{definition} \label{def:graph}

    \lang{de}{
    Als \notion{\emph{Funktionsgraph}} oder kurz als \notion{\emph{Graph}} der reellen Funktion 
    $f:D\to \R$ bezeichnet man die Menge aller geordneten Paare $\,(x;f(x))\,$ aus Elementen $\,x\,$
    des Definitionsbereichs $\,D\,$ und ihrem Bild $\,f(x)\,$.
    }
    \lang{en}{
    We define the \notion{\emph{graph}} of the real function $f:D\to \R$ as the set of all ordered pairs 
    $\,(x;f(x))\,$ of elements $\,x\,$ in the domain $\,D\,$ and their image $\,f(x)\,$.
    }
	\[ \text{Graph}(f):=\{ (x;y) \mid x\in D, y=f(x) \}=\{ (x;f(x)) \mid x\in D \} \subseteq \R^2.\] 
    
\end{definition}   

\lang{de}{
Überträgt man alle geordneten Paare der Menge \notion{Graph}$(f)\,$ in ein Koordinatensystem, mit
dem Argument $\,x\,$ als Wert auf der \emph{x-Achse} und dem zugehörigen Funktionswert $\,f(x)\,$
als Wert auf der \emph{y-Achse}, so erhält man die graphische Darstellung der Funktion $f$.
}
\lang{en}{
Plotting all ordered pairs in the set \notion{Graph}$(f)\,$ onto a coordinate system, with the independent 
variable $\,x\,$ on the \emph{$x$-axis} andn the corresponding image $\,f(x)\,$ on the \emph{$y$-axis} gives 
us a graphical representation of the function $f$.
}

\begin{example}
  \lang{de}{
  Wir schauen uns die graphische Darstellung für unsere Funktionen $f$ und $g$ 
  aus Beispiel \ref{example:bereiche} an:
  }
  \lang{en}{
  We look at the graphical representation for the functions $f$ and $g$ from example 
  \ref{example:bereiche}:
  }
  
  \begin{tabs*}[\initialtab{1}\class{example}]
   \tab{$f(x)=x^2$}  
          \lang{de}{
          Der Funktionsgraph der Funktion $f:\R\to \R,\, x\mapsto x^2$ wird beschrieben durch die Menge 
          }
          \lang{en}{
          The graph of the function $f:\R\to \R,\, x\mapsto x^2$ is given by the set 
          }
           \[ \text{Graph}(f) =\{ (x;x^2) \mid x\in \R \}\]
          \lang{de}{
          oder beispielhaft durch die Wertetabelle 
          }
          \lang{en}{
          or with some example columns of the table 
          }
          \begin{table}[\align{c} \cellaligns{ccrrrrr}]
              $\;\mathbf{x}\;$ && $-2$ & $\textcolor{\#4169E1}{-1}$ & $\,\textcolor{\#4169E1}{0}\,$ & $\,1\,$ & $\,\textcolor{\#4169E1}{2}\,$ \\
              $\;\mathbf{x^2}\;$ && $\,4\,$ & $\,\textcolor{\#4169E1}{1}\,$ & $\,\textcolor{\#4169E1}{0}\,$ & $\,1\,$ & $\,\textcolor{\#4169E1}{4}\,$ \\
          \end{table}
          \lang{de}{
          und hat die folgende Darstellung im zweidimensionalen Koordinatensystem:
          }
          \lang{en}{
          and has the following representation in the two dimensional coordinate system:
          }
            \begin{center}
               \image{T102_Graph_A}
            \end{center}


   \tab{$g(x)=\frac{1}{1-x}$}
        \lang{de}{
        Der Funktionsgraph von $\,g:\R\setminus \{1\}\to \R\setminus \{0\},\, x\mapsto \frac{1}{1-x}\,$ 
        wird beschrieben durch die Menge 
        }
        \lang{en}{
        The graph of $\,g:\R\setminus \{1\}\to \R\setminus \{0\},\, x\mapsto \frac{1}{1-x}\,$ is given by 
        the set 
        }
        \[ \text{Graph}(g) =\{\displaystyle \Big(x;\frac{1}{1-x}\Big) \mid x\in \R \setminus \{1\} \} \]
        \lang{de}{
        und hat die folgende Darstellung im zweidimensionalen Koordinatensystem:
        }
        \lang{en}{
        and has the following representation in the two dimensional coordinate system:
        }

            \begin{center}
               \image{T102_Graph_B}
            \end{center}
  \end{tabs*}
\end{example}

\lang{de}{
Wir werden uns im Folgenden die elementarsten reellen Funktionen anschauen, nämlich 
\ref[content_06_funktionsbegriff_und_lineare_funktionen]["`Lineare Funktionen"']{sec:linear}  und 
\ref[content_08_quadratische_funktionen]["`Quadratische Funktionen"']{sec:quadratic},
und deren charakteristische Eigenschaften näher untersuchen.
}
\lang{en}{
In the following section, we will look at the most elementary real functions, namely 
\ref[content_06_funktionsbegriff_und_lineare_funktionen]['linear functions']{sec:linear} and 
\ref[content_08_quadratische_funktionen]['quadratic functions']{sec:quadratic}, and then investigate 
their characteristic properties.
}

\section{\lang{de}{Lineare Funktionen}\lang{en}{Linear functions}} \label{sec:linear}

\begin{definition}[\lang{de}{Lineare Funktion}\lang{en}{Linear function}] \label{def:linear_func}
    \lang{de}{
    Eine Funktion $f:\R\to \R\,$ der Form 
    }
    \lang{en}{
    A function $f:\R\to \R\,$ of the form 
    }
    \begin{align*}
		f(x)=mx+b 
  	\end{align*}
    \lang{de}{
  	mit beliebigen, aber festen reellen Zahlen $m$ und $b$ hei\"st \notion{\emph{lineare Funktion.}}
    % oder \notion{\emph{Gerade}}.
    }
    \lang{en}{
    with arbitrary but fixed real numbers $m$ and $b$ is called a \notion{\emph{linear function}}.
    }
\end{definition}   


%%%%%%%%%%%%%%%%%%%%%%%%%%%%%%%%%%%%%%%%%%%%%%%%%%%%%%%%%%%%%%%%%%%%%%%%%%%%%%%%%%%%%%%%%%%
% Die folgende Warnung wird gestrichen/umgeschrieben, da sie an dieser Stelle nicht verständlich ist.
% Der Hinweis erfolgt in T403a_Vektorraum\content_10b_lineare_abb in der Warnung nach
% Definition 4.2.1 Lineare Abbildungen, dass nämlich lineare Funktionen nicht mit linearen 
% Abbildungen gleichzusetzen sind!
%%%%%%%%%%%%%%%%%%%%%%%%%%%%%%%%%%%%%%%%%%%%%%%%%%%%%%%%%%%%%%%%%%%%%%%%%%%%%%%%%%%%%%%%%%%

%
\lang{de}{
\begin{block}[warning]
	\begin{showhide}
	
     Es sei an dieser Stelle kurz bemerkt, dass es sich bei einer \emph{"`linearen Funktion"'} 
     gemäß Definition \ref{def:linear_func} nicht um eine \emph{"`lineare Abbildung"'} im 
     Sinne der \notion{linearen Algebra} handelt, die wir später im Kapitel über
     \ref[content_10b_lineare_abb][Vektorräume]{def_lin_abb} noch kennenlernen werden. Eine 
     \emph{"`lineare Funktion"'} würde im Sinne der linearen Algebra einer sogenannten 
     \emph{"`affinen Abbildung"'} entsprechen und nur im Spezialfall $\,b=0\,$ den Anforderungen 
     einer \emph{"`linearen Abbildung"'} genügen.
     
     In der \notion{Analysis} hingegen bezeichnet eine \emph{"`lineare Funktion"'}  eine Funktion, 
     deren Graph eine gerade Linie ist und entspricht als solche einer 
     \ref[content_09_polynome][Polynomfunktion]{polynomial} vom Grad Null oder Eins.
        
%			\lang{de}{"'Lineare Funktionen"' im Sinne der obigen Definition \ref{geraden.definition.1} sind nicht "'linear"' im Sinne der Hochschulmathematik. Dort bezeichnet man 
%			Funktionen der Form $y =mx +b$ als "'affin"' und nur dann als "'linear"', falls $b=0$ ist.}
%			\lang{en}{"Linear Functions" in the sense of the above Definition \ref{geraden.definition.1} are not 'linear' in the sense of linear algebra. In linear algebra (as part of university mathematics)
%			we refer to functions of the form $y=mx+b$ as "{a}ffine" and call functions of the form $y=mx$ linear functions (i.e. when $b=0$).}
	\end{showhide}
\end{block}
}
%

\begin{definition}[\lang{de}{Gerade und Geradengleichung}
                   \lang{en}{Lines and equations of a line}] \label{def:gerade}
    \lang{de}{
    Der Graph einer linearen Funktion $f:\R\to \R, \, x \mapsto mx+b$, 
    also die Menge $\{ (x;y) \in \R^2 \mid y=mx+b \},$ beschreibt eine 
    \notion{\emph{Gerade} $\,g,$} die durch die Wahl der reellen Konstanten $m$ und $b$, 
    auch \emph{Parameter} genannt, charakterisiert wird.
    \\
    Man bezeichnet $\,y=mx+b\,$ auch als die \notion{\emph{Geradengleichung}} 
    von $g$ und schreibt statt der Mengenschreibweise 
    \[g: \; y = m x + b, \: x \in \R .\]
    Der Parameter \emph{$m$} beschreibt dabei die \notion{\emph{Steigung}} von $g$ und
    \emph{$b$} den \notion{\emph{Ordinaten-}} bzw. \notion{\emph{$y-$Achsenabschnitt}}, da 
    die Gerade $g$ die $y-$Achse im Punkt $(0;b)$ schneidet.
    }
    \lang{en}{
    The graph of a linear function $f:\R\to \R, \, x \mapsto mx+b$, that is the set 
    $\{ (x;y) \in \R^2 \mid y=mx+b \}$, describes a  \notion{\emph{line} $\,g,$} which is characterised 
    by the choice of the real constants $m$ and $b$, which are also called \emph{parameters} in this case.
    \\
    We call $\,y=mx+b\,$ the \notion{\emph{equation}} of $g$ and instead of set notation, we write 
    \[g: \; y = m x + b, \: x \in \R .\]
    The parameter \emph{$m$} is called the \notion{\emph{gradient}} or \notion{\emph{slope}} of $g$ and 
    \emph{$b$} is called the \notion{\emph{$y-$intercept}}, as the line $g$ crosses the $y$-axis at the 
    point $(0;b)$.
    }
\end{definition}

\lang{de}{
Da somit die Geradengleichung durch ihre Steigung und ihren Schnittpunkt 
mit der $y-$Achse eindeutig bestimmt ist, und umgekehrt die Steigung und 
der Schnittpunkt mit der $y-$Achse direkt aus der Geradengleichung abgeleitet 
werden können, bezeichnet man die Darstellung $\,g: \; y = m x + b\,$ auch als  
eine \ref[content_07_geradenformen][Punkt-Steigungsform]{sec:punkt_steig_form}
der Geraden $g$. Hierauf werden wir im folgenden Kapitel noch näher eingehen.
\\
Zunächst schauen wir uns an, welche Informationen uns diese Parameter
über die Lage der Geraden $g$ im zweidimensionalen Koordinatensystem liefern.
\\
%
%%% Video Hoever
% 
Eine anschauliche Erläuterung hierzu, illustriert durch einige Beispiele, finden Sie im folgenden Video.
\\
    \floatright{\href{https://www.hm-kompakt.de/video?watch=101}{\image[75]{00_Videobutton_schwarz}}}
\\\\
Das wichtigste fassen wir hier nochmal zusammen:
%
}
\lang{en}{
As a line equation is determined uniquely by its gradient and $y$-intercept, and conversely the gradient 
and $y$-intercept can be uniquely determined from a line equation, we call $\,g: \; y = m x + b\,$ the 
\ref[content_07_geradenformen][slope-intercept form]{sec:punkt_steig_form} of the line equation $g$. This will 
be further examined in the next chapter.
\\
Firstly we examine what information these parameters provide about the graphical representation of the line 
$g$ in two-dimensional coordinates.
}



\begin{remark}[\lang{de}{Bedeutung der Parameter}\lang{en}{Meaning of the parameters}]
    \lang{de}{
    Der $y-$Achsenabschnitt $b$ gibt an, dass die Gerade $g$ die $y-$Achse
    im Punkt $(0;b)$ schneidet.
    }
    \lang{en}{
    From the $y$-intercept $b$ we know the point at which the line $g$ crosses, or 'intercepts' the 
    $y$-axis, $(0;b)$.
    }
    \begin{itemize}
    \item \lang{de}{
      Ist $b=0$, so verläuft die Gerade durch den \textit{Ursprung} 
      $(0;0)$ des Koordinatensystems (es ist $f(x)=mx$).
      }
      \lang{en}{
      If $b=0$, the line goes through the \textit{origin} $(0;0)$ of the coordinate system (it is of the 
      form $f(x)=mx$).
      }
    \item \lang{de}{
      Ist $b>0$, so schneidet die Gerade die $y-$Achse oberhalb 
      der $x$-Achse.
      }
      \lang{en}{
      If $b>0$, the line intercepts the $y$-axis in the upper half of the plane, above the $x$-axis.
      }
    \item \lang{de}{
      Ist $b<0$, so schneidet die Gerade die $y-$Achse unterhalb 
      der $x$-Achse.
      }
      \lang{en}{
      If $b<0$, the line intercepts the $y$-axis in the lower half of the plane, below the $x$-axis.
      }
    \end{itemize}
    \lang{de}{
    Die Steigung $m$ gibt an, um wie viel die Gerade steigt, wenn man $x$ % um $1$ 
    erh"oht.
    }
    \lang{en}{
    The gradient $m$ reveals how steep the line is, hence the alternative term \emph{'slope'}. That is, 
    how fast $y$ increases as $x$ is increased.
    }
    \begin{itemize}
    \item \lang{de}{Ist $m=0$, so ist die Gerade \emph{parallel} zur $x$-Achse (es ist $f(x)=b$).}
          \lang{en}{If $m=0$, the line is \emph{parallel} to the $x$-axis (it is $f(x)=b$).}
    \item \lang{de}{Ist $m>0$, so \emph{steigt} die Gerade und zwar je \emph{größer} $m$ ist, desto 
    \emph{steiler}.}
          \lang{en}{If $m>0$, the line \emph{increases}; the \emph{larger} $m$ is, the \emph{steeper} the 
          line.}
% Genauer:
% \begin{itemize}
% \item Für $m=1$ hat die Gerade \glqq{}diagonale\grqq{} Steigung (aufw"arts),
% \item f"ur $m>1$ ist die Steigung \emph{steiler} als diagonal,
% \item f"ur $0<m<1$  ist die Steigung \emph{flacher} als diagonal.
% \end{itemize}
    \item \lang{de}{Ist $m<0$, so \emph{f"allt} die Gerade, wobei das Gefälle 
    mit wachsendem $m (<0)$ abflacht, also geringer wird.}
    \lang{en}{If $m<0$, the line \emph{decreases}, with a \emph{less steep} descent for larger $m$, that is 
    $m$ closer to $0$.}
% Genauer:
% \begin{itemize}
% \item Für $m=-1$ hat die Gerade \glqq{}diagonale\grqq{} Steigung abw"arts,
% \item f"ur $m<-1$ ist die Steigung \emph{steiler} als diagonal,
% \item f"ur $-1<m<0$  ist die Steigung \emph{flacher} als diagonal.
%  \end{itemize}
\end{itemize}
\end{remark}


\begin{example}\label{geraden.example.1}
%
%%% Video Hoever
%
\lang{de}{
\begin{enumerate}[alph] 
  \item Wir lösen zunächst die Abschlussfrage aus dem vorherigen Video: 
   $\quad$ \floatright{\href{https://www.hm-kompakt.de/video?watch=101lsg}{\image[75]{00_Videobutton_schwarz}}}\\
%
  \item Das folgende Beispiel veranschaulicht, wie sich Veränderungen der charakterisierenden Parameter
        $m$ und $b$ der linearen Funktion $f:\R\to \R, \, x \mapsto mx+b$ auf ihre graphische 
        Darstellung der Geraden auswirken.
\end{enumerate}
}
\lang{en}{
The following example shows how changes in the characteristic parameters $m$ and $b$ affect the 
graphical representation of the linear function $f:\R\to \R, \, x \mapsto mx+b$.
}\\
        

  \lang{de}{
	\begin{genericGWTVisualization}[450][800]{mathlet1}
		\begin{variables}
			\randint{randomA}{1}{2}
			\randint{randomB}{-2}{2}
			\number[editable]{m}{real}{var(randomA)}
			\number[editable]{b}{real}{var(randomB)}
			\slider{sm}{m}{-5,editable}{5,editable}
			\slider{sb}{b}{-5,editable}{5,editable}
			\number{opm}{operation}{var(m)}
			\number{opb}{operation}{var(b)}
%			\function{diag1}{real}{sign(var(m))*x}
%			\function{diag2}{real}{-x}			
			\function{F}{real}{var(m) * x + var(b)}
			\point{P}{real}{0,var(b)}
			\function[calculate]{F1}{real}{var(m) * x + var(b)}
		\end{variables}
		\label{F}{@2d[$\textcolor{BLACK}{g}$]}
		\label{m}{$m=$}
		\label{b}{$b=$}
%		\color{diag1}{LIGHT_GRAY}
%		\color{diag2}{LIGHT_GRAY}
		\color{P}{#0066CC}
		\begin{canvas}
			\plotSize{540}
			\plotLeft{-3}
			\plotRight{3}
			\plot[coordinateSystem]{diag1,F,P}
			\slider{sm,sb}
		\end{canvas}
	    \text{Der Graph der reellen Funktion $f$, definiert durch}
	    \\
	    \phantom{}
		\text{$f(x)=\var{F1}\;$  f\"ur alle $x \in \R$}
		\\
		\phantom{}
	    \text{
            Er hat die Steigung $\; m= \var{opm}$, den $y-$Achsenabschnitt 
            $\; b= \var{opb}$ und entspricht der Geraden
            }
        \\
		\phantom{}
		\text{$ g = \{ (x;y) \in \R^2 \: | \: y = \var{F1}  \}$ .}
    \text{\IFELSE{var(b)>0}
                       {Da $\; b=\var{opb}$ größer als $0$ ist, schneidet die Gerade die 
                        $y-$Achse oberhalb der $x-$Achse.}
                       {\IFELSE{var(b)=0}
                               {Da $\; b=\var{opb}$ ist, geht die Gerade durch den Ursprung.}
                               {Da $\; b=\var{opb}$ kleiner als $0$ ist, schneidet die Gerade die 
                                $y-$Achse unterhalb der $x-$Achse.}}
  		     }
		\text{  \IFELSE{var(m)=0}
                        {Da $\; m=\var{opm}$ ist, ist die Steigung $0$, d.h. die Gerade ist 
                        \IFELSE{var(b)=0}{diem}{parallel zur } x-Achse.}
                    	 {\IFELSE{var(m)>0}
                               {Da $\; m=\var{opm}$ größer als $0$ ist, steigt die Gerade.}
                               {Da $\; m=\var{opm}$ kleiner als $0$ ist, fällt die Gerade.}}
		     }
	\end{genericGWTVisualization}
  }
  \lang{en}{
  \begin{genericGWTVisualization}[450][800]{mathlet1}
		\begin{variables}
			\randint{randomA}{1}{2}
			\randint{randomB}{-2}{2}
			\number[editable]{m}{real}{var(randomA)}
			\number[editable]{b}{real}{var(randomB)}
			\slider{sm}{m}{-5,editable}{5,editable}
			\slider{sb}{b}{-5,editable}{5,editable}
			\number{opm}{operation}{var(m)}
			\number{opb}{operation}{var(b)}
%			\function{diag1}{real}{sign(var(m))*x}
%			\function{diag2}{real}{-x}			
			\function{F}{real}{var(m) * x + var(b)}
			\point{P}{real}{0,var(b)}
			\function[calculate]{F1}{real}{var(m) * x + var(b)}
		\end{variables}
		\label{F}{@2d[$\textcolor{BLACK}{g}$]}
		\label{m}{$m=$}
		\label{b}{$b=$}
%		\color{diag1}{LIGHT_GRAY}
%		\color{diag2}{LIGHT_GRAY}
		\color{P}{#0066CC}
		\begin{canvas}
			\plotSize{540}
			\plotLeft{-3}
			\plotRight{3}
			\plot[coordinateSystem]{diag1,F,P}
			\slider{sm,sb}
		\end{canvas}
      \text{The graph of the real function $f$, defined by}
	    \\
	    \phantom{}
		\text{$f(x)=\var{F1}\;$ for all $x \in \R$}
		\\
		\phantom{}
	    \text{It has gradient $\; m= \var{opm}$ and $y$-intercept $\; b= \var{opb}$, and corresponds to 
            the line}
        \\
		\phantom{}
		\text{$ g = \{ (x;y) \in \R^2 \: | \: y = \var{F1}  \}$ .}
       \text{\IFELSE{var(b)>0}
                       {As $\; b=\var{opb}$ is greater than $0$, the line intercepts the 
                        $y$-axis above the $x$-axis.}
                       {\IFELSE{var(b)=0}
                               {As $\; b=\var{opb}$, it goes through the origin.}
                               {As $\; b=\var{opb}$ is less than $0$, the line intercepts the 
                                $y$-axis below the $x$-axis.}}
  		     }
		\text{  \IFELSE{var(m)=0}
                       {As $\; m=\var{opm}$, the gradient is $0$, so the line is 
                        \IFELSE{var(b)=0}{precisely }{parallel to } the $x$-axis.
                        }
                    	 {\IFELSE{var(m)>0}
                               {As $\; m=\var{opm}$ is greater than $0$, the line slopes 
                                upwards.}
                               {As $\; m=\var{opm}$ is less than $0$, the line slopes 
                                downwards.}}
		     }
	\end{genericGWTVisualization}
    }
\end{example}

                %       \\ Die Gerade
                %		\IFELSE{var(m)=1}{ist \IFELSE{var(b)=0 }{die 1. Winkelhalbierende }{parallel zur 1. Winkelhalbierenden}, weil die Steigung $m$ gleich $1$ ist.}
                %					{\IFELSE{var(m)>1}{ist steiler als die 1. Winkelhalbierende, weil $m$ sogar größer $1$ ist.}
                %									  {ist flacher als die 1. Winkelhalbierende, weil $m$ kleiner als $1$ ist.}}}                                 
                %           {Da $m=\var{opm}$ kleiner als $0$ ist, fällt die Gerade.}                                  
                %       \\ Die Gerade
                %		\IFELSE{var(m)=-1}{ist \IFELSE{var(b)=0}{die 2. Winkelhalbierende }{parallel zur 2. Winkelhalbierenden}, weil die Steigung $m$ gleich $-1$ ist.}{\IFELSE{var(m)<-1}
                %		{ist steiler als die 2. Winkelhalbierende, weil $m$ sogar kleiner $-1$ ist.}{ist flacher als die 2. Winkelhalbierende, weil $m$ größer als $-1$ ist.}}}
                
%%%%%%%%%%%%%%%%%%%%%%%%%%%%%%%%%%%%%%%%%%%%%%%%%%%%%%%%%%
% Das Beispiel ist obsolet, da die Darstellung 
% im nächsten Beispiel ausreicht.
%%%%%%%%%%%%%%%%%%%%%%%%%%%%%%%%%%%%%%%%%%%%%%%%%%%%%%%%%%
%
% Wir wollen uns das an einem Beispiel veranschaulichen
%  \begin{example}
%  $g_1\,$ sei der Graph der linearen Funktion $f_1:\R\to \R, \, x \mapsto 2x-1\,$  und \\
%  $\,g_2\,$ sei der Graph von $f_2:\R\to \R, \, x \mapsto -\frac{1}{2}x+2.$ \\
%
%              \begin{center}
%                 \image[500]{funktionsgraph_geraden}
%              \end{center}
%  Die Steigung der Geraden $g_1$ ist also $m=2$ und der Schnittpunkt mit der $y-$Achse 
%  ist $(0;-1).$ \\ Die Steigung der Geraden $g_2$ ist hingegen $m=-\frac{1}{2}$ und der 
%  Schnittpunkt mit der $y-$Achse ist $(0;2).$
%  \end{example}
%%%%%%%%%%%%%%%%%%%%%%%%%%%%%%%%%%%%%%%%%%%%%%%%%%%%%%%%%%




\begin{quickcheckcontainer}
\randomquickcheckpool{1}{2}
\randomquickcheckpool{3}{3}
\begin{quickcheck}
		\field{rational}
		\type{input.number}
		\begin{variables}
			\randint[Z]{a}{-5}{5}
			\randint[Z]{b}{1}{4}
			\randint{c}{-4}{4}
			\randint[Z]{d}{1}{4}
			\function[calculate]{m}{a/b}
			\function[calculate]{y}{c/d}
		    \function[normalize]{f}{m*x+y}
		\end{variables}
		
			\text{\lang{de}{
       Für die lineare Funktion $f(x)=\var{f}$ hat die zugehörige Gerade
			 die Steigung \ansref und den $y-$Achsenabschnitt \ansref.
       }
       \lang{en}{
       The line corresponding to the linear funtion $f(x)=\var{f}$ has gradient \ansref and 
       $y$-intercept \ansref.
       }}
		
		\begin{answer}
			\solution{m}
		\end{answer}
		\begin{answer}
			\solution{y}
		\end{answer}
		\explanation{\lang{de}{
    Die Steigung ist der Koeffizient von $x$, also der Faktor, mit dem $x$ multipliziert wird.\\
		Der $y-$Achsenabschnitt ist die Zahl, die hinzuaddiert wird, d.\,h. der Wert von $f$ an der Stelle $0$.}
    \lang{en}{
    The gradient is the coefficient of $x$, that is, the factor by which $x$ is multiplied.\\
    The $y$-intercept is the number which is added as a constant, that is, the value of $f$ at $x=0$.
    }}
	\end{quickcheck}

\begin{quickcheck}
		\field{rational}
		\type{input.number}
		\begin{variables}
			\randint[Z]{a}{-5}{5}
			\randint[Z]{b}{1}{4}
			\randint{c}{-4}{4}
			\function[calculate]{m}{a/b}
			\function[calculate]{y}{c/b}
		    \function[normalize]{f0}{a*x+c}
		    \function{f}{f0/b}
		\end{variables}
		
			\text{\lang{de}{
       Für die lineare Funktion $f(x)=\var{f}$ hat die zugehörige Gerade
			 die Steigung \ansref und den $y-$Achsenabschnitt \ansref.
       }
       \lang{en}{
       The line corresponding to the linear function $f(x)=\var{f}$ has gradient \ansref and 
       $y$-intercept \ansref.
       }}
		
		\begin{answer}
			\solution{m}
		\end{answer}
		\begin{answer}
			\solution{y}
		\end{answer}
		\explanation{\lang{de}{
    Die Steigung ist der Koeffizient von $x$, also der Faktor, mit dem $x$ multipliziert wird.\\
		Der $y-$Achsenabschnitt ist die Zahl die hinzuaddiert wird, d.\,h. der Wert von $f$ an der Stelle $0$.}
    \lang{en}{
    The gradient is the coefficient of $x$, that is, the factor by which $x$ is multiplied.\\
    The $y$-intercept is the number which is added as a constant, that is, the value of $f$ at $x=0$.
    }}
	\end{quickcheck}

\begin{quickcheck}
		\field{rational}
		\type{input.function}
		\begin{variables}
			\randint[Z]{a}{-5}{5}
			\randint[Z]{b}{1}{4}
			\randint{c}{-4}{4}
			\randint[Z]{d}{1}{4}
			\function[calculate]{m}{a/b}
			\function[calculate]{y}{c/d}
		    \function[normalize]{f}{m*x+y}
		\end{variables}
		
			\text{\lang{de}{
       Für die Gerade mit Steigung $\var{m}$ und $y-$Achsenabschnitt $\var{y}$
			 ist die zugehörige Funktionsgleichung $y=$\ansref.}
       \lang{en}{
       The line with gradient $\var{m}$ and $y$-intercept $\var{y}$ is the graph of $y=$\ansref.
       }}
		
		\begin{answer}
			\solution{f}
			\checkAsFunction{x}{-10}{10}{100}
		\end{answer}
		\explanation{\lang{de}{
    Die Funktionsgleichung ergibt sich als $y=mx+b$, wobei $m$ die Steigung und $b$ der 
    $y-$Achsenabschnitt ist.
    }
    \lang{en}{
    The line equation is of the form $y=mx+b$, where $m$ is the gradient and $b$ is the 
    $y$-intercept.
    }}
	\end{quickcheck}

\end{quickcheckcontainer}




\section{\lang{de}{Nullstellen}\lang{en}{Roots of a real function}}\label{Nullstellen}

\lang{de}{
Wir definieren zunächst den Begriff der \emph{Nullstelle} allgemein 
für beliebige reelle Funktionen.
}
\lang{en}{
We will define the concept of \emph{roots} generally, for all real functions.
}
	 
\begin{definition}[\lang{de}{Nullstellen einer reeller Funktion}\lang{en}{Roots}]\label{def:nullstelle}
  \lang{de}{
  Die \emph{\notion{Nullstellen}} einer reellen Funktion $f:D_f\to \R$ sind die Stellen $x\in D_f$, 
	an denen die Funktion den Wert $f(x)=0$ annimmt.
  }
  \lang{en}{
	The \emph{roots} or \emph{zeros} of a function $f$ are the points $x\in D_f$ in the domain at which 
  the function takes on the value $f(x)=0$.
  }
\end{definition}
 
\lang{de}{
   Bei der Darstellung der Funktion als Graph im zweidimensionalen
	 Koordinatensystem sind die Nullstellen die \nowrap{$x$-Koordinaten}
	 der Schnittpunkte des Funktionsgraphen mit der $x$-Achse.
	 \\
	 Eine Funktion kann eine, mehrere oder keine Nullstelle haben.
   }
\lang{en}{
  In the represention of a function as a graph in a two-dimensional coordinate system,
	the roots are the \nowrap{$x$-coordinates} where the graph of the function intersects the $x$-axis.
	\\
	A function can have one, multiple, or no roots.
  }

\lang{de}{
	\begin{genericGWTVisualization}[450][800]{mathlet1}
		\begin{variables}
% 			\point[editable]{P1}{real}{-1,-1}
% 			\point[editable]{P2}{real}{1,1}
% 			\number{x1}{real}{var(P1)[x]}
% 			\number{y1}{real}{var(P1)[y]}
% 			\number{x2}{real}{var(P2)[x]}
% 			\number{y2}{real}{var(P2)[y]}
% 			%% wollen Polynom f(x)=x^4+ax^3+bx^2+cx+d mit Extrempunkten P1 und P2; Formel für Koeffizienten:
% 			\number{nen}{real}{(var(x1)^3 - 3*var(x1)^2*var(x2) + 3*var(x1)*var(x2)^2 - var(x2)^3)}
% 			\number{a}{real}{(-2*var(x1)^4 + 4*var(x1)^3*var(x2) - 4*var(x1)*var(x2)^3 + 2*var(x2)^4 - 2*var(y1) + 2*var(y2))/var(nen)}
% 			\number{b}{real}{(var(x1)^5 + var(x1)^4*var(x2) - 8*var(x1)^3*var(x2)^2 + 8*var(x1)^2*var(x2)^3 - var(x1)*var(x2)^4 + 3*var(x1)*var(y1) - 3*var(x1)*var(y2) - var(x2)^5 + 3*var(x2)*var(y1) - 3*var(x2)*var(y2))/var(nen)}
% 			\number{c}{real}{(-2*var(x1)^5*var(x2) + 4*var(x1)^4*var(x2)^2 - 4*var(x1)^2*var(x2)^4 + 2*var(x1)*var(x2)^5 - 6*var(x1)*var(x2)*var(y1) + 6*var(x1)*var(x2)*var(y2))/var(nen)}
% 			\number{d}{real}{(var(x1)^5*var(x2)^2 - 3*var(x1)^4*var(x2)^3 + 3*var(x1)^3*var(x2)^4 + var(x1)^3*var(y2) - var(x1)^2*var(x2)^5 - 3*var(x1)^2*var(x2)*var(y2) + 3*var(x1)*var(x2)^2*var(y1) - var(x2)^3*var(y1))/var(nen)}
%     		\function{f}{real}{x^4+var(a)*x^3+var(b)*x^2+var(c)*x+var(d)}
%     		% dritter Extrempunkt:
% 			\number{x3}{real}{-3/4*var(a)-var(x1)-var(x2)}
% 			\number{y3}{real}{var(x3)^4+var(a)*var(x3)^3+var(b)*var(x3)^2+var(c)*var(x3)+var(d)}
% 			%Anzahl der Nullstellen ergibt sich aus der Lage der Extrempunkte:
% 			%1. x3=x1 (analog x3=x2):
% 			%	keine NS für y2>0, eine NS für y2=0, 2 NS für y2<0
% 			%2. x1\ne x3\ne x2:
% 			%	keine NS für y1,y2,y3>0
% 			%	1 NS für min(y1,y2,y3)=0, aber nur ein y gleich 0
% 			%	2 NS für zwei y's gleich 0 oder y1*y2*y3<0
% 			%	3 NS für min(y1,y2,y3)<0 und y1*y2*y3=0
% 			%	4 NS für min(y1,y2,y3)<0 und y1*y2*y3>0		
			\function{pol1}{rational}{x^3-2*x^2+x-2}
			\function{pol2}{rational}{x^4-1/4*x^3-2*x^2+3/4*x-1/2}
			\function{pol3}{rational}{x^4-2*x^2+1/2}
			\function{pol4}{rational}{x^4-x^2+1/2*x+1}	
		\end{variables}
% 		\color{f}{BLACK}
% 		\color{P1}{BLUE}
% 		\color{P2}{BLUE}
		\color{pol1}{#00CC00}
		\color{pol2}{#0066CC}
		\color{pol3}{#0066CC}
		\color{pol4}{#CC6600}
%		\label{P1}{$\textcolor{BLUE}{P_1}$}
\\
		\begin{canvas}
			\plotSize{300}
			\plotLeft{-3}
			\plotRight{3}
%			\plot[coordinateSystem]{f,P1,P2}
			\plot[coordinateSystem]{pol1,pol2,pol3,pol4}
		\end{canvas}
%		\text{Es ist $P_1=(\var{P1})$ und $P_2=\var{P2}$, die Kurve gehört zu $f(x)=\var{f}$.}
		\text{Das Beispiel zeigt Funktionen mit \textcolor{#00CC00}{genau einer}, mit \textcolor{#0066CC}{mehreren}
		 und mit \textcolor{#CC6600}{keiner} Nullstelle.}
	    	\end{genericGWTVisualization}
}
\lang{en}{
	\begin{genericGWTVisualization}[450][800]{mathlet1}
		\begin{variables}
			\function{pol1}{rational}{x^3-2*x^2+x-2}
			\function{pol2}{rational}{x^4-1/4*x^3-2*x^2+3/4*x-1/2}
			\function{pol3}{rational}{x^4-2*x^2+1/2}
			\function{pol4}{rational}{x^4-x^2+1/2*x+1}
		\end{variables}
		\color{pol1}{#00CC00}
		\color{pol2}{#0066CC}
		\color{pol3}{#0066CC}
		\color{pol4}{#CC6600}
		\begin{canvas}
			\plotSize{300}
			\plotLeft{-3}
			\plotRight{3}
			\plot[coordinateSystem]{pol1,pol2,pol3,pol4}
		\end{canvas}
		\text{This example shows functions with \textcolor{#00CC00}{exactly one}, with 
          \textcolor{#0066CC}{multiple} and with \textcolor{#CC6600}{no} roots.}
	    	\end{genericGWTVisualization}
}

\lang{de}{
Die \notion{Nullstellen einer linearen Funktion} $f(x)=mx+b$ entprechen also nach vorstehender 
Definition \ref{def:nullstelle} der  
% \ref[content_05_loesen_gleichungen_und_lgs][Lösungen der linearen Gleichung]{sec:linear}
\ref[content_05_loesen_gleichungen_und_lgs][Lösungsmenge der linearen Gleichung]{alg:lin_gleichung}
$\; mx+b=0$ und folglich gilt: 
}
\lang{en}{
The set of \notion{roots of a linear function} $f(x)=mx+b$ is, from definition \ref{def:nullstelle}, the
\ref[content_05_loesen_gleichungen_und_lgs][solution set of a linear equation]{alg:lin_gleichung} 
$\; mx+b=0$ and hence: 
}

% \begin{theorem}[Nullstellen Linearer Funktionen]\label{theorem:nullst_lin_fkt}
% Eine lineare Funktion $f(x)=mx+b$ mit $m\ne 0$ besitzt genau eine Nullstelle. Diese l"asst sich berechnen, indem man die
% Gleichung $f(x)=0$, also $mx+b=0$ nach $x$ aufl"ost:
% \[ mx+b=0 \Leftrightarrow mx=-b \Leftrightarrow x=-\frac{b}{m}. \]
% \end{theorem}

\begin{theorem}[\lang{de}{Nullstellen Linearer Funktionen}
                \lang{en}{Roots of a linear function}]\label{theorem:nullst_lin_fkt}
    \lang{de}{
    Eine lineare Funktion $f: \R \to \R\;$ mit $\; f(x)=mx+b\;$ und $\; m, b \in \R$
    besitzt, sofern \notion{$m\ne 0$} ist, \notion{genau eine} Nullstelle, nämlich
    }
    \lang{en}{
    A linear function $f: \R \to \R\;$ with $\; f(x)=mx+b\;$ and $\; m, b \in \R$ has, as long as 
    \notion{$m\ne 0$}, \notion{exactly one} root, namely
    }
    \[x=-\frac{b}{m}. \]
    \lang{de}{Falls $m=0$ ist und }
    \lang{en}{If $m=0$ and }
    \begin{enumerate}
		\item[$b \neq 0, \;$]
        \lang{de}{
		    liegt $f$ parallel zur $x-$Achse und besitzt folglich \notion{keine} Nullstelle. 
        }
        \lang{en}{
        then $f$ is parallel to the $x$-axis and therefore has \notion{no} roots.
        }\\ 
		\item[$b = 0, \;$]
        \lang{de}{
		    liegt $f$ genau auf der $x-$Achse und besitzt somit \notion{unendlich viele} Nullstellen.
        }
        \lang{en}{
        then  $f$ describes the $x$-axis and therefore has \notion{infinitely many} roots.
        }\\
    \end{enumerate}

\end{theorem}

\begin{example}
\begin{enumerate}
\item \lang{de}{
    Wir bestimmen die Nullstelle der lineare Funktion $f(x)=3x-1$, indem wir die lineare Gleichung
    $3x-1=0$ lösen.
    }
    \lang{en}{
    We find the root of the linear funtion $f(x)=3x-1$ by solving the linear equation $3x-1=0$.
    }
    \[ 3x-1=0 \Leftrightarrow 3x=1 \Leftrightarrow x=\frac{1}{3}.\]
    \lang{de}{
    Also liegt die Nullstelle von $f$ in $\frac{1}{3}$.
    }
    \lang{en}{
    So the root of $f$ is at $\frac{1}{3}$.
    }

\item \lang{de}{
    Die Nullstelle der lineare Funktion $f(x)=\frac{2}{3}x +\frac{1}{2}$ ist
    $x=-\frac{3}{4}$, denn
    }
    \lang{en}{
    The root of the linear function $f(x)=\frac{2}{3}x +\frac{1}{2}$ is $x=-\frac{3}{4}$, as
    }
    \[ \frac{2}{3}x +\frac{1}{2}=0 \Leftrightarrow \frac{2}{3}x=-\frac{1}{2} \Leftrightarrow x=-\frac{1}{2}\cdot \frac{3}{2}=-\frac{3}{4}. \]
    \end{enumerate}
\end{example}

\begin{quickcheck}
		\field{rational}
		\type{input.number}
		\begin{variables}
			\randint[Z]{a}{-5}{5}
			\randint[Z]{b}{1}{4}
			\randint{c}{-4}{4}
			\randint[Z]{d}{1}{4}
			\function[calculate]{m}{a/b}
			\function[calculate]{rezm}{b/a}
			\function[calculate]{r}{-c/d}
		    \function[normalize]{f}{(a/b)*x+c/d}
			\function[calculate]{ns}{-(c*b)/(a*d)}
		\end{variables}
		
			\text{\lang{de}{Die Nullstelle der linearen Funktion $f(x)=\var{f}$ ist \ansref.}
            \lang{en}{The root of the linear function $f(x)=\var{f}$ is \ansref.}}
		
		\begin{answer}
			\solution{ns}
		\end{answer}
		\explanation{\lang{de}{Um die Nullstelle zu berechnen, ist die Gleichung $f(x)=0$ zu lösen:}
                 \lang{en}{To find the root, we solve the equation $f(x)=0$:}
			\[ \var{f}=0 \Leftrightarrow \var{m}\cdot x=\var{r} \Leftrightarrow x=\var{r}\cdot \var{rezm}=\var{ns}. \]
		}
	\end{quickcheck}

\end{visualizationwrapper}

\end{content}

