%$Id:  $
\documentclass{mumie.article}
%$Id$
\begin{metainfo}
  \name{
    \lang{de}{Darstellungsformen von Geraden}
    \lang{en}{Equations of a line}
  }
  \begin{description} 
 This work is licensed under the Creative Commons License Attribution 4.0 International (CC-BY 4.0)   
 https://creativecommons.org/licenses/by/4.0/legalcode 

    \lang{de}{Beschreibung}
    \lang{en}{Description}
  \end{description}
  \begin{components}
    \component{generic_image}{content/rwth/HM1/images/g_tkz_T102_LinearFunction.meta.xml}{T102_LinearFunction}
    \component{generic_image}{content/rwth/HM1/images/g_img_00_Videobutton_blau.meta.xml}{00_Videobutton_blau}
    \component{generic_image}{content/rwth/HM1/images/g_img_00_Videobutton_schwarz.meta.xml}{00_Videobutton_schwarz}
    \component{js_lib}{system/media/mathlets/GWTGenericVisualization.meta.xml}{gwtviz}
  \end{components}
  \begin{links}
\link{generic_article}{content/rwth/HM1/T102neu_Einfache_Reelle_Funktionen/g_art_content_06_funktionsbegriff_und_lineare_funktionen.meta.xml}{content_06_funktionsbegriff_und_lineare_funktionen}
\end{links}
  \creategeneric
\end{metainfo}

\begin{content}

%
% ursprüngliche Version:
%
% Copy of \href{http://team.mumie.net/issues/8973}{Ticket 8973}: 
%           content/rwth/HM1/T102_Einfache_Funktionen,_grundlegende_Begriffe/art_content_05_geradenformen.src.tex
%
% \begin{block}[annotation]
%    Im Ticket-System: \href{http://team.mumie.net/issues/8973}{Ticket 8973}\\
% \end{block}
% 
% Ticket neu: 
%
\begin{block}[annotation]
	Im Ticket-System: \href{https://team.mumie.net/issues/21155}{Ticket 21155}
\end{block}

\usepackage{mumie.ombplus}
\ombchapter{2}
\ombarticle{2}

\usepackage{mumie.genericvisualization}

\begin{visualizationwrapper}

\title{\lang{de}{Darstellungsformen von Geraden}
       \lang{en}{Equations of a line}}

\begin{block}[annotation]
 Zweipunktform, Punkt-Steigungsform, Umrechnungen von einer in die andere Form  
\end{block}

\begin{block}[info-box]
\tableofcontents
\end{block}

\lang{de}{
Im vorangegangenen Kapitel haben wir Geraden als  
\ref[content_06_funktionsbegriff_und_lineare_funktionen][Graphen linearer Funktionen]{def:gerade} 
mit der Darstellung $\,g: \; y=mx+b\,$ definiert und festgestellt, dass diese durch die
Steigung $m$ und den $y-$Achsenabschnitt $b$ eindeutig bestimmt sind.
\\
Wir werden in diesem Kapitel sehen, dass es im zweidimensionalen 
Koordinatenraum $\R^2$ noch weitere Größen geben kann, durch die eine Gerade 
eindeutig zu bestimmen ist. Statt des Schnittpunkts mit der $y-$Achse kann nämlich
ebenso ein beliebiger anderer Punkt auf der Geraden gewählt werden, um in 
Kombination mit der Steigung $m$ eine Gerade eindeutig zu beschreiben. 
Ohne Kenntnis der Steigung sind zwei verschiedene Punkte erforderlich, 
um eine Gerade eindeutig zu bestimmen.
}
\lang{en}{
In the previous chapter we defined lines as 
\ref[content_06_funktionsbegriff_und_lineare_funktionen][graphs of linear functions]{def:gerade}, 
written in the 'slope-intercept' form $\,g: \; y=mx+b\,$. We determined that the gradient $m$ and 
$y$-intercept $b$ uniquely define such a line.
\\
In this chapter we will see that in two-dimensional coordinates $\R^2$, there are other parameters 
which can uniquely characterise a line. Any other point on the line can be chosen instead of the 
$y$-intercept, and in combination with the gradient $m$ used to uniquely describe the line. Even 
without knowing the gradient, we can simply use two distinct points on the line to uniquely describe 
it.
}

    \begin{center}
        \image{T102_LinearFunction}
    \end{center}

\lang{de}{
Hieraus ergeben sich die folgenden weiteren Darstellungsformen für Geraden.
}
\lang{en}{
From this we can represent lines in the following ways.
}

\section{\lang{de}{Zweipunktform einer Geraden}\lang{en}{Two-point form of a line}} \label{sec:zweipunktform}

\lang{de}{
Sind $P_1=(x_1 ; y_1)$ und $P_2=(x_2 ; y_2 )$ zwei beliebige Koordinatenpunkte in $\R^2$ mit $x_1 \neq x_2$, 
so gibt es genau eine Gerade $g$, die durch $P_1$ und $P_2$ verläuft. Ihre Steigung $m$ kann
folgendermaßen berechnet werden:
}
\lang{en}{
Suppose $P_1=(x_1 ; y_1)$ and $P_2=(x_2 ; y_2 )$ are two arbitrary coordinates in $\R^2$ with 
$x_1 \neq x_2$. Then there is exactly one line $g$ that goes through both $P_1$ und $P_2$. Its 
graient $m$ can be calculated using these coordinates:
}

  \begin{align*} 
      m= \frac{y_2 -y_1}{x_2 - x_1} =: \frac{\Delta y}{ \Delta x } \: . &\qquad (1)
  \end{align*}

\lang{de}{
Zur Bestimmung des $\,y-$Achsenabschnitts $\,b\,$ lösen wir die Geradengleichung $\,y=mx+b\,$ nach $\,b\,$ auf 
}
\lang{en}{
To calculate the $\,y$-intercept $\,b\,$ we rearrange the linear equation $\,y=mx+b\,$ for $\,b\,$
}
  \begin{align*} \label{b=y-mx}
      b=y-mx \: 
  \end{align*}

\lang{de}{
und setzen zum Beispiel den Punkt $P_1$, der auf der Geraden $g$ liegt, hier ein. Wir erhalten 
}
\lang{en}{
and substitute for example the coordinates of the point $P_1$, which lies on the line $g$. We obtain 
}
\begin{align*}
	b=y_1 - m x_1 \, \underset{(1)}{=} \,  y_1 - \frac{y_2 - y_1}{x_2 -x_1} x_1 \: . &\qquad (2)
\end{align*}
\lang{de}{
Setzt man die in $(1)$ und $(2)$ berechneten Werte für $\,m\,$ und $\,b\,$ in die Geradengleichung  
$\,y=mx + b\,$ ein, dann erhält man die sogenannte \emph{\notion{Zweipunktform}} der Geraden, nämlich
}
\lang{en}{
Substituting into $\,y=mx + b\,$ the values calculated in $(1)$ und $(2)$ for $\,m\,$ und $\,b\,$ 
gives us the so-called \emph{\notion{two-point form}} of the line:
}

\begin{align*}
	g:\; y= \frac{y_2 - y_1}{x_2 - x_1} (x - x_1) + y_1 \qquad \text{für alle} \,x \in \R. 
\end{align*}
\lang{de}{Diese kann man sich besonders einfach in der folgenden Form merken:}
\lang{en}{In particular, this can be easily remembered in the following form:}

\begin{align*}
	\frac{y - y_1}{x - x_1} = \frac{y_2 - y_1}{x_2 - x_1}\: \text{ 
    \lang{de}{für $x \neq x_1$.}
    \lang{en}{for $x \neq x_1$.}
     }
\end{align*}

\lang{de}{
Diese Formeln ermöglichen es, zwischen der Zweipunktform einer Geraden und ihrer Geradengleichung hin- 
und herzuwechseln.
}
\lang{en}{
These formulas make it possible to switch between the two-point form of a line's equation and its 
slope-intercept form.
}

\begin{rule} \label{rule:zweipunktform}
\begin{enumerate}[alph]
    \item \lang{de}{
    \textbf{Bestimmung der Geradengleichung aus zwei Punkten}\\
    Ist eine Gerade gegeben durch zwei Punkte $\,P_1=(x_1 ; y_1) \,$ und $\, P_2=(x_2 ; y_2) \,$ 
    (mit $\, x_1 \ne x_2 $), so lassen sich mittels folgender Formeln die Parameter für
    ihre Geradengleichung bestimmen:
    }
    \lang{en}{
    \textbf{Finding the slope-intercept line equation using two points}\\
    If a line is given by two points $\,P_1=(x_1 ; y_1) \,$ and $\, P_2=(x_2 ; y_2) \,$ that lie on 
    it (with $\, x_1 \ne x_2 $), the following formulas give the gradient and $y$-intercept:
    }
         \begin{align*}
            m = \frac{\Delta y}{ \Delta x} = \frac{y_2 -y_1}{x_2 - x_1 } 
            \quad \text{\lang{de}{und}\lang{en}{and}} \quad
            b =  y_1 - \frac{\Delta y}{ \Delta x} x_1 = y_1 - \frac{y_2 - y_1}{x_2 -x_1} x_1 
          \end{align*}  
          
    \item \lang{de}{
    \textbf{Bestimmung der Zweipunktform aus der Geradengleichung}\\
    Umgekehrt lässt sich, bei gegebenen Parametern $\, m, b \in \R\,$, die Zweipunkteform einer
    Geraden aus ihrer Geradengleichung herleiten, indem zu zwei beliebig gewählten $x_1, x_2 \in \R$
    mit $\, x_1 \ne x_2 \,$ die zugehörigen Punkte $\, P_1=(x_1 ; y_1) \,$ und $\, P_2=(x_2 ; y_2) \,$
    auf der Geraden bestimmt werden, also 
    }
    \lang{en}{
    \textbf{Finding the two-line form using the slope-intercept equation of a line}\\
    In the other direction, we can use the parameters $\, m, b \in \R\,$ to find the two-point form 
    of a line. We can choose arbitrary  $x_1, x_2 \in \R$ with $\, x_1 \ne x_2 \,$, and easily get 
    corresponding points $\, P_1=(x_1 ; y_1) \,$ and $\, P_2=(x_2 ; y_2) \,$ on the line, with 
    }
	\begin{align*}
		y_1=mx_1 +b \qquad \text{\lang{de}{und}\lang{en}{and}} \quad y_2= mx_2 +b \: . 
	\end{align*}
\end{enumerate}  
\end{rule}

\begin{quickcheck}
		\type{input.number}
		\field{rational}
		\begin{variables}
			\randint{x1}{-5}{5}
			\randint{x2}{-4}{4}
			\randadjustIf{x1,x2}{x1 = x2}
			\randint[Z]{y1}{-4}{4}
			\randint[Z]{y2}{-4}{4}
	
			\function[calculate]{m}{(y2-y1)/(x2-x1)}
			\function[calculate]{b}{y1-m*x1}
		\end{variables}
		
			\text{\lang{de}{
      Die Geradengleichung der Geraden durch die Punkte $P_1=(\var{x1};\var{y1})$
			und $P_2=(\var{x2};\var{y2})$ lautet:
      }
      \lang{en}{
      The slope-intercept equation of the line through the points $P_1=(\var{x1};\var{y1})$ 
			and $P_2=(\var{x2};\var{y2})$ is:
      }\\
			$f(x)= $\ansref $x + $\ansref.}
		
		\begin{answer}
			\solution{m}
		\end{answer}
		\begin{answer}
			\solution{b}
		\end{answer}
		
	\end{quickcheck}

\begin{remark}
\lang{de}{
Sind $P_1=(x_1 ; y_1)$ und $P_2=(x_1 ; y_2 )$ zwei verschiedene Koordinatenpunkte mit 
derselben $x$-Koordinate und $y_1 \neq y_2$, so gibt es \textbf{keine} lineare Funktion $f(x)=mx+b$, die
eine Gerade durch die beiden Punkte beschreibt, da eine Funktion $f(x)$  
\ref[content_06_funktionsbegriff_und_lineare_funktionen][per Definition]{def:reelle_funktion}
jedem $x$ nur genau eine Zahl $y$ zuordnet, und nicht zwei verschiedene Zahlen $y_1$ und $y_2$.
}
\lang{en}{
Suppose $P_1=(x_1 ; y_1)$ and $P_2=(x_1 ; y_2 )$ are two distinct points with the same $x$ coordinate 
and $y_1 \neq y_2$. Then there is \textbf{no} linear function $f{x}=mx+b$ that contains both 
points, as a function $f(x)$ 
\ref[content_06_funktionsbegriff_und_lineare_funktionen][by definition ]{def:reelle_funktion} 
can only map each $x$ to one number $y$, not to two seperate numbers $y_1$ and $y_2$.
}

% In den Formeln sieht man dies auch daran, dass der allgemeine Ausdruck f"ur $m$, $\frac{y_2-y_1}{x_2-x_1}$,
% in diesem Fall ein Ausdruck $\frac{\Delta y}{0}$ w"are, man aber durch $0$ nicht teilen kann!}
\end{remark}

%
%%% Video Hoever
%
\lang{de}{
In folgendem Video wird die Herleitung der vorstehenden Regel \ref{rule:zweipunktform} nocheinmal anschaulich 
wiederholt und die Bestimmung der Geradengleichung zu 2 vorgegebenen Punkten anhand eines Beispiels erläutert.
\\
\floatright{\href{https://www.hm-kompakt.de/video?watch=103}{\image[75]{00_Videobutton_schwarz}}}\\\\ }
\lang{en}{
\\
}

%

\begin{example}\label{geraden.example.2}
%
%%% Video Hoever
%
\lang{de}{
\begin{enumerate}[alph]
  \item Wir lösen zunächst die Abschlussfrage aus dem vorherigen Video: 
   $\quad$ \floatright{\href{https://www.hm-kompakt.de/video?watch=103lsg}{\image[75]{00_Videobutton_schwarz}}}\\\\
%
  \item 
\end{enumerate}
}
\lang{en}{
\\
}
\begin{genericGWTVisualization}[550][1000]{gwtviz}
	\begin{variables}
		\point[editable]{p1}{rational}{0,0}
		\point[editable]{p2}{rational}{1,1}
		\line{g}{rational}{var(p1),var(p2)}
		\number{p1x}{rational}{var(p1)[x]}
		\number{p1y}{rational}{var(p1)[y]}
		\number{p2x}{rational}{var(p2)[x]}
		\number{p2y}{rational}{var(p2)[y]}
		\function{m}{rational}{(var(p2y)-var(p1y))/(var(p2x)-var(p1x))}
		\number{mRes}{rational}{var(m)}
		\function{op1}{rational}{var(p1y) - (var(mRes) * var(p1x))}
		\point{p3}{rational}{var(p2)[x],var(p1)[y]}
		\segment{mx}{rational}{var(p1),var(p3)}
		\segment{my}{rational}{var(p2),var(p3)}
		\function{g2}{rational}{var(mRes)* x +var(op1)}
	\end{variables}

	\color{p1}{#0066CC}
	\color{p2}{#CC6600}
%	\color{g}{BLACK}
	\color{my}{#00CC00}
	\color{mx}{#00CC00}
	\label{my}{@2d[$\textcolor{#00CC00}{m= \frac{\Delta y}{ \Delta x}}$]}
	\label{p1}{@2d[$\textcolor{#0066CC}{P_1}$]}
	\label{p2}{@2d[$\textcolor{#CC6600}{P_2}$]}
	\label{g}{@2d[$\textcolor{BLACK}{g}$]}
	\evaluate{op1}
	\evaluate{g2}
	\begin{canvas}
		\plotSize{540}
		\plotLeft{-3}
		\plotRight{3}
		\plot[coordinateSystem]{g,p1,p2,mx,my}
	\end{canvas}

    \lang{de}{\text{Sei $g$ eine Gerade durch $P_1 = (\var{p1}[x];\var{p1}[y])$ und $P_2$ = (\var{p2}[x];\var{p2}[y]).}}
    \lang{en}{\text{Let $g$ be a line through $P_1 = (\var{p1}[x],\var{p1}[y])$ and $P_2$ = (\var{p2}[x],\var{p2}[y]).}}
    \lang{de}{\text{Die in der Geradengleichung $y = mx + b$ auftretenden Konstanten}}
    \lang{en}{\text{In the equation $y = mx + b$, the constants}}
    \lang{de}{\text{\IFELSE{var(p1x)=var(p2x)}{$m$ (\emph{Steigung}) und $b$ (\emph{Ordinatenabschnitt}) sind nicht definiert (Division durch Null).}{$m$ (\emph{Steigung}) und $b$ (\emph{Ordinatenabschnitt}) werden so berechnet: }}}
	\lang{de}{\text{\IFELSE{var(p1x)=var(p2x)}{\IF{var(p1y)>var(p2y) OR var(p1y)<var(p2y)}{Die Geradengleichung für $g$ lautet: $x = \var{p1x}$}}{$m = \var{m} = \var{mRes},$}}}
    \lang{en}{\text{\IFELSE{var(p1x)=var(p2x)}{$m$ (the \emph{gradient}) and $b$ (the \emph{y-intercept}) are not defined (division by zero).}{$m$ (the \emph{gradient}) and $b$ (the \emph{y-intercept}) are calculated as follows: }}}
	\lang{en}{\text{\IFELSE{var(p1x)=var(p2x)}{\IF{var(p1y)>var(p2y) OR var(p1y)<var(p2y)}{The equation for $g$ is: $x = \var{p1x}$}}{$m = \var{m} = \var{mRes},$}}}
	\lang{de}{\text{\IFELSE{var(p1x)=var(p2x)}{}{$b = \var{p1y} - \var{m} \cdot (\var{p1x}) = \var{p1y} - (\var{mRes}) \cdot (\var{p1x}) = \var{op1}.$\\ \phantom{}}}}
	\lang{en}{\text{\IFELSE{var(p1x)=var(p2x)}{}{$b = \var{p1y} - \var{m} \cdot (\var{p1x}) = \var{p1y} - (\var{mRes}) \cdot (\var{p1x}) = \var{op1}.$\\ \phantom{}}}}
	
	\lang{de}{\text{\IFELSE{var(p1x)=var(p2x)}{\IFELSE{var(p1y)=var(p2y)}{Nur zwei \emph{verschiedene} Punkte definieren eine Gerade.}{}}{Die Geradengleichung für $g$ lautet: $y=\var{g2}$}}}
    \lang{en}{\text{\IFELSE{var(p1x)=var(p2x)}{\IFELSE{var(p1y)=var(p2y)}{Only two \emph{different} points define a line.}{}}{The equation for $g$ is: $y=\var{g2}$}}}
\end{genericGWTVisualization}
\end{example}

% \begin{center}
% \lang{de}{\iframe[400][225][S]{https://www.stream24.net/vod/getVideo.php?id=10962-1-6077&mode=iframe}}
% \end{center}

\section{\lang{de}{Punkt-Steigungsform einer Geraden}
         \lang{en}{Point-slope form of a line}} \label{sec:punkt_steig_form}

\lang{de}{
Kennt man nur einen Punkt $\, P=(x_{p} ; y_{p})\,$ einer Geraden $\,g\,$ und ihre
Steigung $\, m \,$, so gilt für jeden weiteren Punkt $\, (x ; y)\,$ dieser Geraden 
mit $\, x \neq x_{p} \,$ gemäß Regel \ref{rule:zweipunktform}, dass
}
\lang{en}{
Suppose we only know one point $\, P=(x_{p} ; y_{p})\,$ on a line $\,g\,$, and the gradient $\, m \,$ 
of the line. By rule \ref{rule:zweipunktform}, for every other point $\, (x ; y)\,$ on the line 
with $\, x \neq x_{p} \,$ we have 
}
\begin{align*}
m = \frac{y -y_{p}}{x - x_{p} }.
\end{align*}
%
% Eine Gerade durch den Punkt $P=(x_{p} ; y_{p})$ mit der Steigung $m$ erfüllt
%
% \begin{align}
%	y-y_{p}=m (x-x_{p}) \: .
% \end{align}

\lang{de}{
Löst man diese Gleichung nach $y$ auf, so erhält man die sogenannte 
\emph{\notion{Punkt-Steigungsform}} der Geraden, nämlich 
}
\lang{en}{
Rearranging this equation for $y$ yields the so-called \emph{\notion{point-slope form}} of the line, 
namely 
}
	\begin{align*}
		g: \; y = m \, (x-x_{p}) + y_{p} \qquad \text{\lang{de}{für alle}\lang{en}{for all}} \,x \in \R . 
	\end{align*}

%
%%% Video Hoever
%
\lang{de}{
Im folgenden Video wird diese Herleitung nochmal anhand eines Beispiel erläutert.
\floatright{\href{https://www.hm-kompakt.de/video?watch=104}{\image[75]{00_Videobutton_schwarz}}}
\\\\\\
%
Hieraus ergibt sich die folgende Formel zur \textbf{Bestimmung der Geradengleichung aus Punkt und Steigung}.
}
\lang{en}{From this we obtain the following formula for \textbf{finding the slope-intercept form 
using the point-slope form of a line}
}

\begin{rule} \label{rule:punk_steig_form}
\lang{de}{
Ist eine Gerade gegeben durch einen Punkt $\,P=(x_{p} ; y_{p}) \,$ und 
ihre Steigung $\, m $, so lässt sich der noch fehlende Parameter  $\,b$,
also der $y-$Achsenabschnitt, mittels folgender Formel bestimmen:
}
\lang{en}{
If a line is given by a point $\,P=(x_{p} ; y_{p}) \,$ and a gradient $\, m $, then the $y$-intercept 
$\,b$ can be determined using the following formula:
}
	\begin{align*}
		b=y_{p} - m \, x_{p} \, .
	\end{align*}
\end{rule}

\lang{de}{
Im Spezialfall, dass $P=(0;y_{p})$ die Schnittstelle der Geraden mit der $y-$Achse ist, 
liefert die Formel \ref{rule:punk_steig_form} $\; b=y_{p}. $
}
\lang{en}{
In the special case that $P=(0;y_{p})$ is the $y$-intercept of the line, the formula in 
\ref{rule:punk_steig_form} yields $\; b=y_{p}$.
}

\begin{proof*}
  \begin{showhide}
  \lang{de}{
  $\,P=(x_{p} ; y_{p}) \,$ und $\, m $ liefern die Punkt-Steigungsform
  der Geraden. Multipliziert man diese aus, so erhält man
  }
  \lang{en}{
  $\,P=(x_{p} ; y_{p}) \,$ and $\, m $ give the point-slope form of a line. We simply multiply this 
  out and compare it with the slope-intercept form:
  }
      \begin{align*}
          y \;&= m \, (x-x_{p}) + y_{p} \\
              &= m\, x + (\underbrace{-m\, x_p +y_p}) \\
              &=m\,x+ \phantom{-m\,x} b .
      \end{align*}
  \end{showhide}
\end{proof*}

\begin{example}\label{geraden.example.3}
%
%%% Video Hoever
%
\lang{de}{
\begin{enumerate}[alph]
  \item Wir lösen zunächst die Abschlussfrage aus dem vorherigen Video: 
   $\quad$ \floatright{\href{https://www.hm-kompakt.de/video?watch=104lsg}{\image[75]{00_Videobutton_schwarz}}}\\
%
  \item 
\end{enumerate}
}
\lang{en}{
\\
}

	\begin{genericGWTVisualization}[550][1000]{gwtviz}
		\begin{variables}
			\randint{randomx}{-1}{1}
			\randint{randomy}{-1}{1}
			\randint[Z]{randomm}{-2}{2}
			\point[editable]{p1}{rational}{var(randomx),var(randomy)}
			\number{p1x}{rational}{var(p1)[x]}
			\number{p1y}{rational}{var(p1)[y]}
			\number[editable]{m}{rational}{var(randomm)}
			\number{opm}{rational}{var(m)}
			\function{opb}{rational}{var(p1y)  - (var(opm) * var(p1x) ) }
			\number{b}{rational}{var(opb)}
			\number{opy}{rational}{var(p1y)}
			\number{p2x}{operation}{var(p1x) + 1}
			\number{p2y}{operation}{var(p1y) +  var(opm)}
			\point{p2}{rational}{var(p2x),var(p2y)}
			\point{p3}{rational}{var(p2x),var(p1)[y]}
			\line{g}{rational}{var(p1),var(p2)}
			\segment{mx}{rational}{var(p1),var(p3)}
			\segment{my}{rational}{var(p2),var(p3)}
			\function{F}{rational}{var(opm) * x + var(b)}
		\end{variables}
		\evaluate{F}
		\color{p1}{#0066CC}
		\color{my}{#00CC00}
		\color{mx}{#00CC00}
		\label{p1}{@2d[$\textcolor{#0066CC}{P}$]}
		\label{g}{@2d[$\textcolor{BLACK}{g}$]}
		\label{my}{@2d[$\textcolor{#00CC00}{m= \frac{\Delta y}{ \Delta x}}$]}
		\begin{canvas}
			\plotSize{540}
			\plotLeft{-3}
			\plotRight{3}
			\plot[coordinateSystem]{g,p1,mx,my}
		\end{canvas}
	    \lang{de}{\text{Sei $g$ die Gerade durch den Punkt $P= (\var{p1}[x];\var{p1}[y])$ mit der Steigung $m=\var{m}$.}}
	    \lang{en}{\text{Let $g$ be the line that goes through the point $P= (\var{p1}[x],\var{p1}[y])$ with gradient $m=\var{m}$.}}
	    \\
		\phantom{}
	    \lang{de}{\text{Die Geradengleichung lautet dann }}
	    \lang{en}{\text{The equation of the line is then}}
		\phantom{}
		\text{
		$y=\var{opm} x + ( \var{opy}  - (\var{opm}) \cdot (\var{p1x})\; ) = \var{F}$.\\
		\phantom{}}
	    \lang{de}{\text{Die Gerade schneidet somit die $y$-Achse in dem Punkt $(0;\var{b})$, d.h. $b=\var{b}$.}}
	    \lang{en}{\text{The line intersects the $y$-axis at the point $(0,\var{b})$, i.e. $b=\var{b}$.}}  
	\end{genericGWTVisualization}
\end{example}

\begin{quickcheck}
		\type{input.number}
		\field{rational}
		\begin{variables}
			\randint{x1}{-5}{5}
			\randint{x2}{-4}{4}
			\randadjustIf{x1,x2}{x1 = x2}
			\randint[Z]{y1}{-4}{4}
			\randint[Z]{y2}{-4}{4}
	
			\function[calculate]{m}{(y2-y1)/(x2-x1)}
			\function[calculate]{b}{y1-m*x1}
		\end{variables}
		
			\text{\lang{de}{
      Die Punkt-Steigungsform der Geraden mit Steigung $\var{m}$ durch den Punkt 
			$P_1=(\var{x1};\var{y1})$ lautet:\\
			$y= $\ansref $(x- $\ansref$) + $\ansref.\\
			Die zugeh"orige Geradengleichung ist:\\
			$y= $\ansref $x + $\ansref.
      }
      \lang{en}{
      The point-slope form of the line with gradient $\var{m}$ through the point 
      $P_1=(\var{x1};\var{y1})$ is:\\
      $y= $\ansref $(x- $\ansref$) + $\ansref.\\
      The corresponding line-intercept form of the equation is:\\
			$y= $\ansref $x + $\ansref.
      }
			}
		
		\begin{answer}
			\solution{m}
		\end{answer}
		\begin{answer}
			\solution{x1}
		\end{answer}
		\begin{answer}
			\solution{y1}
		\end{answer}
		\begin{answer}
			\solution{m}
		\end{answer}
		\begin{answer}
			\solution{b}
		\end{answer}
		
	\end{quickcheck}

\end{visualizationwrapper}

\end{content}

