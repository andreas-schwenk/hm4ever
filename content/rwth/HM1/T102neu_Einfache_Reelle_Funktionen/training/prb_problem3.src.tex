\documentclass{mumie.problem.gwtmathlet}
%$Id$
\begin{metainfo}
  \name{
    \lang{en}{...}
    \lang{de}{A03: Lage einer Geraden}
  }
  \begin{description} 
 This work is licensed under the Creative Commons License Attribution 4.0 International (CC-BY 4.0)   
 https://creativecommons.org/licenses/by/4.0/legalcode 

    \lang{en}{...}
    \lang{de}{Bestimmung der Parameter einer Geraden}
  \end{description}
  \corrector{system/problem/GenericCorrector.meta.xml}
  \begin{components}
    \component{js_lib}{system/problem/GenericMathlet.meta.xml}{gwtmathlet}
  \end{components}
  \begin{links}
  \end{links}
  \creategeneric
\end{metainfo}

%
\begin{content}
\begin{block}[annotation]
	Im Ticket-System: \href{https://team.mumie.net/issues/22072}{Ticket 22072}
\end{block}

\usepackage{mumie.genericproblem}

\begin{block}[annotation]

 Bestimmung der Parameter und der Nullstelle einer Geraden aus 
 gegebener Geradengleichung 
        
\end{block}

\lang{de}{\title{A03: Lage einer Geraden}}
\lang{en}{\title{Problem 3}}


\begin{problem}
%######################################################QUESTION_START	
%++++++++++++++++++_Frage_1_von_2_+++++++++++++++++++++++++++++	

 
      \begin{variables}
          \randint[Z]{p}{-10}{10}
          \randint[Z]{q}{-10}{10}
          \randadjustIf{q}{q=1}
          \randint[Z]{c}{-10}{10} 	       	       	  	   	
          \function[calculate]{b}{-(c/q)}  
          \function[calculate]{m}{-(p/q)}
          \function[calculate]{nst}{-(b/m)}
          \function[normalize]{f}{p * x + q * y + c}
          \function[normalize]{g}{m * x + b}   	           	        	
      \end{variables}

  \begin{question}  
  
      \text{Überführen Sie die lineare Gleichung $\,\var{f}= 0\,$
            in die Punkt-Steigungsform einer Geraden. 
            (Geben Sie dabei alle Zahlen als Brüche oder ganze Zahlen an.)
            }
            
        \type{input.function}
        \field{rational} 
 
%++++++++++++++++++++++ANSWERS++++++++++++++++++++++++    

      
	\begin{answer}
        \text{$\qquad g: y=mx+b=$}
        \solution{g}
        \checkAsFunction{x}{-10}{10}{100}
        \explanation{Die Geradengleichung erhält man durch Aufl"osen von $\var{f}= 0$ nach $y$.}
   \end{answer}

 \end{question}		

%++++++++++++++++++_Frage_2_von_2_+++++++++++++++++++++++++++++	

  \begin{question}  
  
      \text{Beschreiben Sie die Lage dieser Geraden anhand der
            charakteristischen Eigenschaften, die sich aus der 
            Geradengleichung ableiten lassen.
      }


    \type{input.number} %input.text %input.cases.function %input.finite-number-set %input.interval %...http://team.mumie.net/projects/support/wiki/DifferentAnswerType
    \field{rational} 
    \precision{2}
           

 	\begin{answer}
       \text{$\quad$ Die Gerade hat die Steigung $\,m=$}
       \solution{m}
       \explanation{Die Steigung der Geraden kann direkt aus der Geradengleichung abgelesen werden.}        
   \end{answer}
%
 	\begin{answer}
        \text{$\quad$ Sie schneidet die $y-$Achse in $\,y_0=$}
        \solution{b}       
        \explanation{Die Schnittstelle mit der $y-$Achse entspricht dem $y-$Achsenabschnitt $b$ 
        der Geraden und kann daher direkt aus der Geradengleichung abgelesen werden.}        
    \end{answer}
%
 	\begin{answer}
       \text{$\quad$ und schneidet die $x-$Achse in $\,x_0=$}
       \explanation{Die Schnittstelle mit der $x-$Achse entspricht der Nullstelle der Geraden. 
        Sie berchnet sich aus $\,-\frac{b}{m}.$}        
        \solution{nst}
    \end{answer}
      
\end{question}		
	

%######################################################QUESTION_END


\end{problem}

\embedmathlet{gwtmathlet}                                         

\end{content}
