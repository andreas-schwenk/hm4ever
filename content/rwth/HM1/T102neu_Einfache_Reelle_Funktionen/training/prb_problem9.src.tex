\documentclass{mumie.problem.gwtmathlet}
%$Id$
\begin{metainfo}
  \name{
    \lang{en}{...}
    \lang{de}{A09: Parabelbestimmung}
  }
  \begin{description} 
 This work is licensed under the Creative Commons License Attribution 4.0 International (CC-BY 4.0)   
 https://creativecommons.org/licenses/by/4.0/legalcode 

    \lang{en}{...}
    \lang{de}{Interpretation der Parameter einer Parabel}
  \end{description}
  \corrector{system/problem/GenericCorrector.meta.xml}
  \begin{components}
    \component{js_lib}{system/problem/GenericMathlet.meta.xml}{gwtmathlet}
  \end{components}
  \begin{links}
  \end{links}
  \creategeneric
\end{metainfo}

%
\begin{content}
\begin{block}[annotation]
	Im Ticket-System: \href{https://team.mumie.net/issues/22082}{Ticket 22082}
\end{block}

\usepackage{mumie.genericproblem}

\begin{block}[annotation]

	Erkennen der Lage einer Parabel aus graphischer Darstellung 
    durch Interpretation der Parameter (Öffnung, Streckung und 
    y-Achsen Schnitt)\\
        
\end{block}

\lang{de}{\title{A09: Parabelbestimmung}}
\lang{en}{\title{Problem 9}}


\begin{problem}
 
      \begin{variables}
        \randint[Z]{xs}{-3}{3}                                  % Scheitelpunkt ALLER Parabeln
        \randint{ys}{-3}{3}                                     % Scheitelpunkt ALLER Parabeln

        \randint[Z]{a1}{-10}{10}                                % Steckfaktor für Parabel f1
        \function[calculate, normalize]{c1}{a1*xs^2+ys}         % y-Achsenschnitt von f1 
        \randadjustIf{a1}{a1=1 OR a1=-1}
        
        \randint[Z]{k}{-1}{1}         
        \function[calculate]{a2}{k/a1}                          % Steckfaktor für Parabel f2
        \function[calculate, normalize]{c2}{a2*xs^2+ys}         % y-Achsenschnitt von f2         
        \function[calculate, normalize]{c0}{xs^2+ys}            % y-Achsenschnitt von f0
        \function[calculate, normalize]{h}{ys+1}                % Hilfsvariable
        
        \randint{c3}{-5}{5}                                     % y-Achsenschnitt von f3 
        \randadjustIf{c3}{c3=ys OR c3=h OR c3=c0 OR c3=c2 OR c3=c1 }        % da ansonsten
                                                                            % a3=0 oder a4=0 oder
                                                                            % c3=c0 (dann wäre f3=f0) oder
                                                                            % c3=c2 (dann wäre f3=f2) oder
                                                                            % c3=c1 (dann wäre f3=f1)
        \function[calculate, normalize]{a3}{(c3-ys)/xs^2}       % Steckfaktor für Parabel f3 
        \function[calculate, normalize]{a4}{(c3-(ys+1))/xs^2}    % Steckfaktor für Parabel f4 


        \function[normalize]{f0}{(x-xs)^2+ys}                   % Normalparabel mit S=(xs;ys)
        \function[normalize]{f1}{a1*(x-xs)^2+ys}                % steiler als f0
        \function[normalize]{f2}{a2*(x-xs)^2+ys}                % flacher als f0
        \function[normalize]{f3}{a3*(x-xs)^2+ys}                % Parabel mit S=(xs;ys) u. 
                                                                % y-Achsenschnitt in c3

        \function[normalize, expand]{p0}{(x-xs)^2+ys}           % Normalpüarabelitelpunktform Normalparabel
        \function[normalize, expand]{p1}{a1*(x-xs)^2+ys}        % steiler
        \function[normalize, expand]{p2}{a2*(x-xs)^2+ys}        % flacher 
        \function[normalize, expand]{p22}{k*a2*(x-xs)^2+ys}     % Variante für falsche Antwort         
        \function[normalize, expand]{p3}{a3*(x-xs)^2+ys}        % y-Achsenschnitt in c3
        \function[normalize, expand]{p32}{a4*(x-xs)^2+ys}       % y-Achsenschnitt in (c3-1) 
                                                                % als Variante für falsche Antwort         
               
    \end{variables}
%         
%
  \begin{question} 
  
	\type{mc.multiple}     
    \field{rational}  
     
     \text{Die folgende Grafik zeigt verschiedene Parabeln mit einem
           gemeinsamen Scheitelpunkt, darunter in schwarz die 
           Normalparabel $\,f(x)=\var{p0} .$ Ordnen Sie den farbigen 
           Parabeln jeweils ihre beschreibende quadratische Funktion zu.\\
           Kreuzen Sie alle richtigen Antworten an.}
          
     \explanation{Verwenden Sie die Umrechnungsformeln zwischen Standard- 
                und Scheitelpunktform einer Parabel und beachten Sie die
                Öffnung, die Streckung und die Schnittstelle der 
                jeweiligen Parabel mit der $y-$Achse.}

%++++++++++++++++VISUALISATION+++++++++++++++++++++++++++

        \plotF{1}{f0} 
        \plotF{2}{f1} 
        \plotF{3}{f2} 
        \plotF{4}{f3}
        \plotFrom{1}{-6} 
        \plotTo{1}{6} 
        \plotColor{1}{black} 
        \plotColor{2}{red}
        \plotColor{3}{blue}
        \plotColor{4}{green}
        \plotLeft{-6}
        \plotRight{6}
        \plotSize{600}
        \plotRatio{keep}        

%+++++++++++++++++++++++++++++++++++++++++++++++++++++++

 	\permutechoices{1}{6}
% 1von6 
	\begin{choice}
	    \text{Die quadratische Funktion  
        $\, p_1(x) = \var{p1}$ hat den roten Funktionsgraphen.}
	  	\solution{true}    
    \end{choice}
% 2von6    
	\begin{choice}
	    \text{Die quadratische Funktion  
        $\, p_1(x) = \var{p1}$ hat den blauen Funktionsgraphen.}
	  	\solution{false}    
    \end{choice}
% 3von6     
	\begin{choice}
	    \text{Die quadratische Funktion  
        $\, p_3(x) = \var{p2}$ hat den blauen Funktionsgraphen.}
	  	\solution{true}    
    \end{choice}
% 4von6 		
	\begin{choice}
	    \text{Die quadratische Funktion  
        $\, p_4(x) = \var{p3}$ hat den grünen Funktionsgraphen.}
	  	\solution{true}    
    \end{choice}
% 5von6 		
	\begin{choice}
	    \text{Die quadratische Funktion  
        $\, p_5(x) = \var{p32}$ hat den grünen Funktionsgraphen.}
	  	\solution{false}    
    \end{choice}
% 6von6
	\begin{choice}
	    \text{Die quadratische Funktion  
        $\, p_2(x) = \var{p22}$ hat den roten Funktionsgraphen.}
	  	\solution{false}    
    \end{choice}
        
   
\end{question}

%######################################################QUESTION_END


\end{problem}

\embedmathlet{gwtmathlet}                                         

\end{content}
