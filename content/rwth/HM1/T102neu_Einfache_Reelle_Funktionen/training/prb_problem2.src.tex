\documentclass{mumie.problem.gwtmathlet}
%$Id$
\begin{metainfo}
  \name{
    \lang{en}{...}
    \lang{de}{A02: Lineare Funktionen}
  }
  \begin{description} 
 This work is licensed under the Creative Commons License Attribution 4.0 International (CC-BY 4.0)   
 https://creativecommons.org/licenses/by/4.0/legalcode 

    \lang{en}{...}
    \lang{de}{Bestimmung der Parameter einer Geraden}
  \end{description}
  \corrector{system/problem/GenericCorrector.meta.xml}
  \begin{components}
    \component{js_lib}{system/problem/GenericMathlet.meta.xml}{gwtmathlet}
  \end{components}
  \begin{links}
  \end{links}
  \creategeneric
\end{metainfo}

%
\begin{content}
\begin{block}[annotation]
	Im Ticket-System: \href{https://team.mumie.net/issues/22071}{Ticket 22071}
\end{block}

\usepackage{mumie.genericproblem}

\begin{block}[annotation]

	Bestimmung der linearen Funktion oder auch der Geradengleichung 
    aus graphischer Darstellung durch Ablesen der Parameter und der
    Nullstelle (NEU)\\
        
\end{block}

\lang{de}{\title{A02: Lineare Funktionen}}
\lang{en}{\title{Problem 2}}

\begin{problem}
%######################################################QUESTION_START	

%+++++++++++++++++++++++QUESTION+++++++++++++++++++++++++++++		

  \begin{question}  
     
     \text{Beschreiben Sie die Lage der nachfolgend dargestellten Geraden $g$ durch
      ihre bekannten Charakteristika und bestimmen Sie die lineare Funktion 
      $\,f:\R\to \R, \; x\mapsto mx+b$, 
      deren Graph durch die Gerade $\,g\,$ dargestellt wird. }
 
      \begin{variables}
        \randint[Z]{b}{-5}{5}
        \randint[Z]{nst}{-5}{5}
        \function[calculate]{m}{-b/nst}
       	\function{g}{m*x+b}    
    \end{variables}


%++++++++++++++++VISUALISATION+++++++++++++++++++++++++++

        \plotF{1}{g} % % the function a1 is defined in 'variables' in the usual way
        \plotFrom{1}{-6} % % and is plotted starting from 0.0
        \plotTo{1}{6} % % and ending in 1.0 , it's a quarter of a circle
        \plotColor{1}{blue} % % colored blu
        \plotLeft{-6} % % defines the canvas bound left
        \plotRight{6}
        \plotSize{600}
        \plotRatio{keep}


%++++++++++++++++++++++ANSWERS++++++++++++++++++++++++    

     \type{input.number} 
     \field{rational}            

%        
	    \begin{answer}
%            \type{input.number} %input.text %input.cases.function %input.finite-number-set %input.interval %...http://team.mumie.net/projects/support/wiki/DifferentAnswerType
%            \field{rational} 
        	\text{1.$\quad$ Der $y-$Achsenabschnitt von $g$ ist $\;b=$}
			\solution{b}
            \explanation{Der $y-$Achsenabschnitt ist die Schnittstelle der 
                        Geraden mit der $y-$Achse. Bitte erneut ablesen.}
	    \end{answer}  
%        
        \begin{answer}
%            \type{input.number} %input.text %input.cases.function %input.finite-number-set %input.interval %...http://team.mumie.net/projects/support/wiki/DifferentAnswerType
%            \field{rational} 
           	\text{2.$\quad$ Die Nullstelle von $g$ liegt in $\;x_0=$}
			\solution{nst}
            \explanation{Die Nullstelle ist die Schnittstelle der Geraden
                        mit der $x-$Achse. Bitte erneut ablesen.}
	    \end{answer}  
%
	    \begin{answer}   
%            \type{input.number} %input.text %input.cases.function %input.finite-number-set %input.interval %...http://team.mumie.net/projects/support/wiki/DifferentAnswerType
%            \field{rational} 
           	\text{3.$\quad$  Die Steigung von $g$ ist $\;m=$}
			\solution{m}
            \explanation{Die Steigung kann aus der Nullstelle und dem $y-$Achsenabschnitt
                        berechnet werden, denn wegen $g(x_0)=0$ gilt $\,m=-\frac{b}{x_0}$.}
	    \end{answer} 
%
        \begin{answer}
	    	\type{input.function} %input.text %input.cases.function %input.finite-number-set %input.interval %...http://team.mumie.net/projects/support/wiki/DifferentAnswerType
            \text{Die lineare Funktion, die diese Gerade beschreibt, lautet
                 $\; f(x)=$}
			\solution{g}
            \checkAsFunction{x}{-10}{10}{100}  
          \explanation{Die Funktionsvorschrift für die lineare Funktion $\,f\,$ entspricht 
                    der allgemeinen Form der Geradengleichung $\,y=mx+b.$}
	    \end{answer} 

	\end{question}	

%######################################################QUESTION_END


\end{problem}

\embedmathlet{gwtmathlet}                                         

\end{content}
