\documentclass{mumie.problem.gwtmathlet}
%$Id$
\begin{metainfo}
  \name{
    \lang{de}{A04: Geradenbestimmung}
    \lang{en}{}
  }
  \begin{description} 
 This work is licensed under the Creative Commons License Attribution 4.0 International (CC-BY 4.0)   
 https://creativecommons.org/licenses/by/4.0/legalcode 

    \lang{de}{Beschreibung}
    \lang{en}{}
  \end{description}
  \corrector{system/problem/GenericCorrector.meta.xml}
  \begin{components}
    \component{js_lib}{system/problem/GenericMathlet.meta.xml}{mathlet}
  \end{components}
  \begin{links}
  \end{links}
  \creategeneric
\end{metainfo}
\begin{content}
\begin{block}[annotation]
	Im Ticket-System: \href{https://team.mumie.net/issues/22077}{Ticket 22077}
\end{block}

\usepackage{mumie.genericproblem}

\begin{block}[annotation]
	Trainingsaufgaben zur Zweipunktform einer Geraden \\
    Bestimmung einer Geraden aus zwei Punkten
         
\end{block}

\lang{de}{\title{A04: Geradenbestimmung}}
\lang{en}{\title{Problem 4}}

\begin{problem}
%
% variablen
%
    \begin{variables}
       	\randint[Z]{deltax}{-8}{8}
       	\randint[Z]{p1x}{-5}{5}
       	\function[calculate]{p2x}{p1x+deltax}
      	\randint[Z]{deltay}{-8}{8}
       	\randint{p1y}{-5}{5}
       	\function[calculate]{p2y}{p1y+deltay}
       	\function[normalize]{m}{((p2y)-(p1y))/((p2x)-(p1x))}   
       	\function[normalize]{b}{p1y-(((p2y)-(p1y))/((p2x)-(p1x)))p1x}         	        	
    	\function{f1}{(p2y-p1y)/(p2x-p1x)}
    	\function{f2}{p1y-m*p1x}
        \function{g}{m*x+b}
    \end{variables}
			
%Frage 1 von 2
    \begin{question}  
	 	\text{Gesucht sei die Gerade, die durch die Punkte 
 		$\,P_1 = (x_1 ; y_1) = (\var{p1x} \, ; \, \var{p1y})\,$ und
 		$\,P_2 = (x_2 ; y_2) = (\var{p2x} \, ; \, \var{p2y})\,$ 
 		verläuft. Bestimmen Sie über die Zweipunktform der Geraden
        ihre Steigung $\,m\,$ und den Ordinatenabschnitt $\,b$. \\
        (Geben Sie die gesuchten Zahlen als Brüche oder ganze Zahlen an.)}

% tpye    
	\type{input.number}
 	\field{rational}
%    
% answer 1 von 2
%
 	\begin{answer}
    	\lang{de}{\text{$\qquad m=$}}
        \solution{m}
		\explanation{Die Steigung $\,m\,$ ergibt sich aus dem Steigungsdreieck 
        $\,\frac{y_2 - y_1}{x_2 - x_1}.$}
    \end{answer}
%    
% answer 2 von 2
%
 	\begin{answer}
      	\lang{de}{\text{$\qquad b=$}}
        \solution{b}        
 		\explanation{Der Ordinatenabschnitt $\,b\,$ berechnet sich aus 
                    $\,b= y_1-\frac{y_2 - y_1}{x_2 - x_1} \cdot x_1.$}                
	\end{answer}

\end{question}

%Frage 2 von 2

\begin{question} 
 	\text{Wie lautet die Geradengleichung in der Punkt-Steigungsform?}


    \type{input.function}
 	\field{rational}
%    
% answer
%    
 	\begin{answer}
      	\lang{de}{\text{$\qquad y=$}}
        \solution{g}
        \checkAsFunction{x}{-10}{10}{100} 
        \explanation{Die Punkt-Steigungsform der Geraden ist $\,y=mx+b$.}
        
    \end{answer}

\end{question}
	
\end{problem}

\embedmathlet{mathlet}

\end{content}