\documentclass{mumie.problem.gwtmathlet}
%$Id$
\begin{metainfo}
 \name{
  \lang{de}{A01: Definitionsbereich}
  }
  \begin{description} 
 This work is licensed under the Creative Commons License Attribution 4.0 International (CC-BY 4.0)   
 https://creativecommons.org/licenses/by/4.0/legalcode 

    \lang{de}{...}
  \end{description}
  \corrector{system/problem/GenericCorrector.meta.xml}
  \begin{components}
    \component{js_lib}{system/problem/GenericMathlet.meta.xml}{gwtmathlet}
  \end{components}
  \begin{links}
  \end{links}
  \creategeneric
\end{metainfo}

\begin{content}
\begin{block}[annotation]
	Im Ticket-System: \href{https://team.mumie.net/issues/22070}{Ticket 22070}
\end{block}

\usepackage{mumie.genericproblem}

\begin{block}[annotation]
	Bestimmung des Definitionsbereich einer reellen Funktion
\end{block}

\lang{de}{
	\title{A01: Definitionsbereich}
}


\begin{problem}
\randomquestionpool{1}{4}
\randomquestionpool{5}{5}

%Frage 1 von 4: a=0 (Nenner linear), a=1 (Nenner quadrat. Term), n1=1 (es gibt genau 1 Nullstelle des Nenners)
\begin{question}
%
		\text{Bestimmen Sie den maximalen Definitionsbereich der Funktion 
	     $\,f(x)=\var{f}$ \\
         (Geben Sie alle Zahlen als Brüche oder ganze Zahlen an und kürzen 
         Sie soweit wie möglich. Ist der Definitionsbereich nicht eingeschränkt,
         geben Sie als Lösung $D_f=\R \setminus \{\} \,$ an.)}
%         
	    \explanation{Die Nullstellen des Nenners sind auszuschließen.}
%

    \type{input.finite-number-set}
    \field{rational}
      
    \begin{variables}
    
        \randint{zero}{0}{0}
    	\randint{a}{0}{1}    
    	\randint[Z]{b}{-5}{5}
	    \randrat{c}{0}{5}{1}{10}
 	    \randint{n1}{1}{1}        
	    \randint[Z]{d}{-4}{4}
        \function[normalize]{p}{b/n1}
        \function[normalize]{q}{(1-a)*c+a*(b/2)^2}
 		\function[normalize]{f}{d/(a*x^2-p*x+q)}   
		  	  
		\function{f1}{(1-a)*q/p+a*b/2}
	\end{variables}
%
%    \debug[a,b,n1]
%
     \begin{answer}
        \type{input.finite-number-set}
        \text{$D_f=\R \setminus $}
        \solution{f1}
    \end{answer}

\end{question}

%Frage 2 von 4: a=0 (Nenner linear), n1 >= 1 (es gibt genau 1 Nullstelle des Nenners)
\begin{question}
%
		\text{Bestimmen Sie den maximalen Definitionsbereich der Funktion 
	     $\,f(x)=\var{f}$ \\
         (Geben Sie alle Zahlen als Brüche oder ganze Zahlen an und kürzen 
         Sie soweit wie möglich. Ist der Definitionsbereich nicht eingeschränkt,
         geben Sie als Lösung $D_f=\R \setminus \{\} \,$ an.)}
%         
	    \explanation{Die Nullstellen des Nenners sind auszuschließen.}
%

    \type{input.finite-number-set}
    \field{rational}
      
    \begin{variables}
    
        \randint{zero}{0}{0}
    	\randint{a}{0}{0}    
    	\randint[Z]{b}{-5}{5}
	    \randrat{c}{0}{5}{1}{10}
 	    \randint{n1}{1}{10}        
	    \randint[Z]{d}{-4}{4}
        \function[normalize]{p}{b/n1}
        \function[normalize]{q}{(1-a)*c+a*(b/2)^2}
 		\function[normalize]{f}{d/(a*x^2-p*x+q)}   
		  	  
		\function{f1}{(1-a)*q/p+a*b/2}
	\end{variables}
%
%    \debug[a,b,n1]
%
     \begin{answer}
        \type{input.finite-number-set}
        \text{$D_f=\R \setminus $}
        \solution{f1}
    \end{answer}

\end{question}

%Frage 3 von 4:  a=1 (Nenner quadrat. Term), n1>1 (es gibt keine Nullstelle des Nenners, D_f = \R)
\begin{question}
%
		\text{Bestimmen Sie den maximalen Definitionsbereich der Funktion 
	     $\,f(x)=\var{f}$ \\
         (Geben Sie alle Zahlen als Brüche oder ganze Zahlen an und kürzen 
         Sie soweit wie möglich. Ist der Definitionsbereich nicht eingeschränkt,
         geben Sie als Lösung $D_f=\R \setminus \{\} \,$ an.)}
%         
	    \explanation{Die Nullstellen des Nenners sind auszuschließen.}
%

    \type{input.finite-number-set}
    \field{rational}
      
    \begin{variables}
    
        \randint{zero}{0}{0}
    	\randint{a}{1}{1}    
    	\randint[Z]{b}{-5}{5}
	    \randrat{c}{0}{5}{1}{10}
 	    \randint{n1}{2}{10}        
	    \randint[Z]{d}{-4}{4}
        \function[normalize]{p}{b/n1}
        \function[normalize]{q}{(1-a)*c+a*(b/2)^2}
 		\function[normalize]{f}{d/(a*x^2-p*x+q)}   
		  	  
%		\function{f1}{(1-a)*q/p+a*b/2}    % Nenner quadratischer Term ohne Nullstelle
	\end{variables}
%
%    \debug[a,b,n1]
%
     \begin{answer}
        \type{input.finite-number-set}
        \text{$D_f=\R \setminus $}
        \solution{}
    \end{answer}

\end{question}

%Frage 4 von 4:  a=1, n1=1 ABER q=-3(p/2)^2 statt q=(p/2)^2 (=> Diskrimitnante des quadr.
%                               Nennerterms>0, es gibt also 2 Nullstellen des Nenners)
\begin{question}
%
		\text{Bestimmen Sie den maximalen Definitionsbereich der Funktion 
	     $\,f(x)=\var{f}$ \\
         (Geben Sie alle Zahlen als Brüche oder ganze Zahlen an und kürzen 
         Sie soweit wie möglich. Ist der Definitionsbereich nicht eingeschränkt,
         geben Sie als Lösung $D_f=\R \setminus \{\} \,$ an.)}
%         
	    \explanation{Die Nullstellen des Nenners sind auszuschließen.}
%

    \type{input.finite-number-set}
    \field{rational}
      
    \begin{variables}
    
        \randint{zero}{0}{0}
    	\randint{a}{1}{1}    
    	\randint[Z]{b}{-5}{5}
	    \randrat{c}{0}{5}{1}{10}
 	    \randint{n1}{1}{1}        
	    \randint[Z]{d}{-4}{4}
        \function[normalize]{p}{b/n1}                   % p=b (da n1=1)
        \function[normalize]{q}{(1-a)*c-a*3*(b/2)^2}    % q=-3(p/2)^2
 		\function[normalize]{f}{d/(a*x^2-p*x+q)}       
		  	  
		\function{f1}{(1-a)*q/p+a*(-b/2)}               % Nenner-Nst_1=(p/2)-(2p/2)=-p/2
        \function{f2}{(1-a)*q/p+a*(3*b/2)}              % Nenner-Nst_2=(p/2)+(2p/2)=3p/2
	\end{variables}
%
%    \debug[a,b,n1]
%
     \begin{answer}
        \type{input.finite-number-set}
        \text{$D_f=\R \setminus $}
        \solution{f1,f2}
    \end{answer}

\end{question}


%Frage 5
\begin{question}
%
		\text{Bestimmen Sie den maximalen Definitionsbereich der Funktion 
	    $f(x)=\sqrt{\var{f}}$ \\
        (Geben Sie alle Zahlen als Brüche oder ganze Zahlen an und kürzen
        Sie soweit wie möglich.)}
%        
	    \explanation{Der Term unter der Wurzel darf nicht negativ sein.}
%

    \type{input.interval}
    \field{rational}
      
    \begin{variables}
    	\randint{a}{-7}{-3}
	    \randint{b}{-5}{-1}
	    \randint{d}{3}{7}
	    \randint{c}{-5}{-1}
		\function[expand, normalize]{f}{(a*x-b)*(c*x-d)}
		  
		\function[calculate]{f1}{d/c}
		\function[calculate]{f2}{b/a}
	\end{variables}
   
    \begin{answer}
	    
		    \text{$D_f=\R \setminus$}
	    
     \solution{(f1;f2)}
    \end{answer} 
\end{question}

\end{problem}

\embedmathlet{gwtmathlet}

\end{content}
