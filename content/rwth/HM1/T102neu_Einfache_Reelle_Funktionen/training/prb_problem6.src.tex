\documentclass{mumie.problem.gwtmathlet}
%$Id$
\begin{metainfo}
  \name{
    \lang{de}{A06: Quadratische Funktionen}
    \lang{en}{}
  }
  \begin{description} 
 This work is licensed under the Creative Commons License Attribution 4.0 International (CC-BY 4.0)   
 https://creativecommons.org/licenses/by/4.0/legalcode 

    \lang{de}{Beschreibung}
    \lang{en}{}
  \end{description}
  \corrector{system/problem/GenericCorrector.meta.xml}
  \begin{components}
    \component{js_lib}{system/problem/GenericMathlet.meta.xml}{mathlet}
  \end{components}
  \begin{links}
  \end{links}
  \creategeneric
\end{metainfo}
\begin{content}
\begin{block}[annotation]
	Im Ticket-System: \href{https://team.mumie.net/issues/22079}{Ticket 22079}
\end{block}

\begin{block}[annotation]
	Trainingsaufgaben zur Bestimmung der Nullstellen quadratischer Funktionen
    mittels pq-Formel, Mitternachtsformel oder quadratischer Ergänzung
\end{block}

\usepackage{mumie.genericproblem}

\lang{de}{
	\title{A06: Quadratische Funktionen}
}
\lang{en}{
	\title{Problem 6}
}

\begin{problem}

\randomquestionpool{1}{2}

% Frage 1 (1 von 2 im Questionpool)
\begin{question}
	
\begin{variables}
	\randint[Z]{b}{-10}{10}
	\randint[Z]{c}{-10}{10}
	\function[calculate]{d}{b*b-4*c}
	\randadjustIf{b,c}{d<=0}
	
	\function[calculate]{s}{(-b+sqrt(d))/(2)}
	\function[calculate]{y}{(-b-sqrt(d))/(2)}
	
	\function[expand, normalize]{aa}{x^2+b*x+c}
	
\end{variables}

	\type{input.number}
	\field{real} 
	\precision{3}
%
	\text{
	    Bestimmen Sie die Nullstellen 
        der folgenden quadratischen Funktion.\\
	    $f:\, \R \rightarrow  \R \, $, $f(x)=\var{aa}$\\
	    Geben Sie die gerundeten Ergebnisse bis auf drei Nachkommastellen an.}
%
    \explanation{Die Nullstellen können z.B. durch quadratische Ergänzung oder mit der p-q-Formel
                 berechnet werden. Runden Sie das Ergebnis auf drei Nachkommastellen. 
                 } 
%
   \permuteAnswers{1, 2}
%
    \begin{answer}
	    \text{$x_1 =$}
	    \solution{s}
    \end{answer}
        \begin{answer}
	    \text{$x_2 =$}
	    \solution{y}
    \end{answer}
    
\end{question}

% Frage 2 (2 von 2 im Questionpool)
\begin{question}

\begin{variables}
    \randint[Z]{v}{-1}{1}
	\randint[Z]{a}{1}{10}
	\randint[Z]{b}{-10}{10}
	\randint[Z]{c}{-10}{10}
	\function[calculate]{d}{b^2-4*a*c}
	
	\randadjustIf{a,b,c}{d<=0}
	
	\function[calculate]{s}{(-b+sqrt(d))/(2*a)}
	\function[calculate]{y}{(-b-sqrt(d))/(2*a)}
	
	\function[expand, normalize]{ab}{v*(a*x^2+b*x+c)}
	
\end{variables}

	\type{input.number}
	\field{real} 
	\precision{3}
%
	    \text{
	    Bestimmen Sie die Nullstellen 
        der folgenden quadratischen Funktion.\\
	    $f:\, \R \rightarrow  \R\ $, $f(x)=\var{ab}$\\
	    Geben Sie die gerundeten Ergebnisse bis auf drei Nachkommastellen an.} 
%
    \explanation{Die Nullstellen können z.B. mit der Mitternachtsformel berechnet werden. 
                Runden Sie das Ergebnis auf drei Nachkommastellen.
                }
%
   \permuteAnswers{1, 2}
%    
    \begin{answer}
	    \text{$x_1 =$}
	    \solution{s}
    \end{answer}
        \begin{answer}
	    \text{$x_2 =$}
	    \solution{y}
    \end{answer}
    
\end{question}

\end{problem}


\embedmathlet{mathlet}

\end{content}