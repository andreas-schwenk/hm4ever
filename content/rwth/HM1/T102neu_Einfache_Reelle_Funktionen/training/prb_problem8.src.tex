\documentclass{mumie.problem.gwtmathlet}
%$Id$
\begin{metainfo}
 \name{
  \lang{de}{A08: Parabeleigenschaften}
  }
  \begin{description} 
 This work is licensed under the Creative Commons License Attribution 4.0 International (CC-BY 4.0)   
 https://creativecommons.org/licenses/by/4.0/legalcode 

    \lang{de}{...}
  \end{description}
  \corrector{system/problem/GenericCorrector.meta.xml}
  \begin{components}
    \component{js_lib}{system/problem/GenericMathlet.meta.xml}{gwtmathlet}
  \end{components}
  \begin{links}
  \end{links}
  \creategeneric
\end{metainfo}

\begin{content}

\begin{block}[annotation]
	Im Ticket-System: \href{https://team.mumie.net/issues/22081}{Ticket 22081}
\end{block}

\usepackage{mumie.genericproblem}


\begin{block}[annotation]
    Eigenschaften einer Parabel und Bestimmung 
    des Scheitelpunkt und der Scheitelpunktform
    
\end{block}

\lang{de}{
	\title{A08: Parabeleigenschaften}
}


\begin{problem}
%
%        
    \begin{variables}
% Streckungsfaktor 
	    \randint[Z]{k}{-1}{1}
	    \randrat{aa}{1}{5}{1}{5}      
	    \function[calculate]{a}{k*aa}     

% Parameter allgemeine Form
	    \randint{b}{-10}{10}
	    \randint{c}{-10}{10}


% Scheitelpunkt
        \function[calculate]{xs}{-b/(2*a)}
        \function[calculate]{ys}{c-(b^2/(4*a))}
        
% Anzahl Nullstellen      
        \function[calculate]{d}{b^2-4*a*c}      % Diskriminante Mitternachtsformel 
        \randint{nst}{0}{2}                     % Anzahl Nullstellen
        \begin{switch}  
         \begin{case}{d<0}
          \number{nst}{0}
         \end{case} 
         \begin{case}{d=0}
          \number{nst}{1}
         \end{case} 
         \begin{default}
          \number{nst}{2}
          \end{default}         
        \end{switch}        
        
	    \function{f}{a*x^2+b*x+c}    % Parabel in allgemeiner Form     
        \function{fs}{a*(x-xs)^2+ys} % Scheitelpunktform 
        \function[normalize]{ffs}{(x-xs)}       

	\end{variables}
%
\begin{question}
%
    \text{Untersucht wird die durch $\;f(x)= \var{f}\,$ beschriebene Parabel. \\
         Überführen Sie die quadratische Funktion zunächst in die Scheitelpunktform.\\
         \\
         
         $\quad f(x)=$\ansref$\cdot \,($\ansref$)^2+$\ansref
         }
%
        \explanation{Die Überführung von $\,f\,$ in ihre Scheitelpunktform erfolgt durch
                    quadratische Ergänzung. }
%
    \type{input.generic}
    \field{rational}
%
%\debug[fs,xs,ys]
%
    \begin{answer}
        \type{input.number}
      	\solution{a}
    \end{answer} 
%
    \begin{answer}
     	\type{input.function} 
      	\solution{ffs}
	    \checkAsFunction{x}{-10}{10}{100}        
    \end{answer}
%
    \begin{answer}
      	\type{input.number}
        \solution{ys}
    \end{answer}  
%    
\end{question}
%
\begin{question}
% 
    \type{input.number}
    \field{rational} 
%    
    \text{Der Scheitelpunkt der Parabel liegt in $\,S=($\ansref $; $\ansref $).$}
%
    \explanation{Der Scheitelpunkt kann aus der Scheitelpunktform 
                direkt abgelesen werden. }
%
    \begin{answer}
      	\solution{xs}
   \end{answer}  
%
    \begin{answer}
       	\solution{ys}
   \end{answer}  
\end{question}

%
\begin{question}
% 
    \type{input.number}
    \field{rational} 
%       
    \text{Die Parabel schneidet die $y-$Achse in $\,y_0=$\ansref 
          und sie hat $\,$\ansref Nullstellen.}
%	
    \begin{answer}
      	\solution{c}
        \explanation{Die Schnittstelle mit der $y-$Achse errechnet sich aus $\,f(0)$. }
    \end{answer}  
\%	
    \begin{answer}
        \solution{nst}
        \explanation{Die Anzahl der Nullstellen können aus der Diskriminante der
                    Mitternachtsformel $\,(D=b^2-4ac)\,$ bestimmt werden. }
    \end{answer}  
\end{question}

% 
\begin{question}
%
    \text{Welche der folgenden Aussagen trifft zu? 
          Wählen Sie die ALLE richtigen Antworten aus.}
%
    \explanation{Untersuchen Sie hierzu den Leitkoeffizienten 
    der quadratischen Funktion. }
%
    \permutechoices{1}{4}
    \type{mc.multiple}
%    
% c1   
       \begin{choice}
          \lang{de}{\text{Die Parabel ist nach oben geöffnet.}}
          \solution{compute}
          \iscorrect{k}{>}{0}
      \end{choice}
% c2   
         \begin{choice}
            \lang{de}{\text{Die Parabel verläuft steiler als die Normalparabel.}}
            \solution{compute}
            \iscorrect{aa}{>}{1}           
        \end{choice}
% c3   
       \begin{choice}
          \lang{de}{\text{Die Parabel ist nach unten geöffnet.}}
          \solution{compute}
          \iscorrect{k}{<}{0}
      \end{choice}
% c4   
         \begin{choice}
            \lang{de}{\text{Die Parabel verläuft flacher als die Normalparabel.}}
            \solution{compute}
            \iscorrect{aa}{<}{1}           
        \end{choice}
        
\end{question}
%

\end{problem}

\embedmathlet{gwtmathlet}

\end{content}
