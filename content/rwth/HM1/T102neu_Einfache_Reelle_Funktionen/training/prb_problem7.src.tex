\documentclass{mumie.problem.gwtmathlet}
%$Id$
\begin{metainfo}
  \name{
    \lang{de}{A07: Nullstellen/Linearfaktoren}
    \lang{en}{}
  }
  \begin{description} 
 This work is licensed under the Creative Commons License Attribution 4.0 International (CC-BY 4.0)   
 https://creativecommons.org/licenses/by/4.0/legalcode 

    \lang{de}{Beschreibung}
    \lang{en}{}
  \end{description}
  \corrector{system/problem/GenericCorrector.meta.xml}
  \begin{components}
    \component{js_lib}{system/problem/GenericMathlet.meta.xml}{mathlet}
  \end{components}
  \begin{links}
  \end{links}
  \creategeneric
\end{metainfo}
\begin{content}
\begin{block}[annotation]
	Im Ticket-System: \href{https://team.mumie.net/issues/22080}{Ticket 22080}
\end{block}

\begin{block}[annotation]
	Trainingsaufgaben zur Nullstellenbestimmung, hier speziell mit 
    Anwendung des Satzes von Viëta und zur Linearfaktorzerlegung.
\end{block}

\usepackage{mumie.genericproblem}

\lang{de}{
	\title{A07: Nullstellen/Linearfaktoren}
}
\lang{en}{
	\title{Problem 7}
}


\begin{problem}

\randomquestionpool{1}{2}
\randomquestionpool{3}{3}


% Frage 1 von 2 im 1. Questionpool)
\begin{question}
%
 	\type{input.number}
    \field{rational} 
%    
    \text{Die quadratische Funktion $f:\, \R \rightarrow \R$, $\;f(x)= \var{f1}+c\;$ liegt
          in Normalform vor und besitzt eine Nullstellen in $\, x_1=\var{ns1}. \;$
          Bestimmen Sie den Parameter $\,c\,$ sowie die zweite Nullstelle $\,x_2\,$ 
          von $\,f\,$ unter Verwendung des Satzes von Viëta. 
          Geben Sie die Ergebnisse in ganzen Zahlen oder in Brüchen an.}
%
%
%
    \begin{variables}
	    \randint{a}{1}{1}   %  quadratische Funktion in Normalform !
	    \randint{b}{1}{6}  
	    \randint{c}{1}{1}   %  quadratische Funktion in Normalform !
	    \randint{d}{1}{6}
	    \function[calculate]{ns1}{-b/a}
	    \function[calculate]{ns2}{-d/c}
	    \function{c_0}{b*d}
	    \function[calculate]{p}{c*b+d*a}
	    \function[calculate]{q}{b*d}          
	    \function[expand, normalize]{f}{(c*x+d)*(a*x+b)}
	    \function[expand, normalize]{f1}{(c*x+d)*(a*x+b)-b*d}
	\end{variables}
%    
    \begin{answer}
    	\lang{de}{\text{Die zweite Nullstelle liegt in $\;x_2=$}}
     	\solution{ns2}
        \explanation{Nach dem Satz von Viëta gilt für den Koeffizienten 
                    vor $\,x\,$ in $\,f(x)\,$: $\; b=-(x_1+x_2).$ }
    \end{answer}
%	
    \begin{answer}
    	\lang{de}{\text{Der Parameter ist $c=$}}
   		\solution{c_0}
        \explanation{Nach dem Satz von Viëta ist $\,c =x_1 \cdot x_2 .\,$ 
                    Berechnen Sie also zunächst $\,x_2 .$}
    \end{answer}

\end{question}


% Frage 2 von 2 im 1. Questionpool)
\begin{question}
%
 	\type{input.number}
    \field{rational} 
%    
    \text{Die quadratische Funktion $f:\, \R \rightarrow \R$, $\;f(x)= \var{f1}+c\,$ 
          hat eine Nullstellen in $\, x_1=\var{ns1}$. 
          Bestimmen Sie den Parameter $\,c\,$ sowie die zweite Nullstelle $\,x_2\,$ 
          dieser Funktion. \\
          Geben Sie die Ergebnisse in ganzen Zahlen oder in Brüchen an.  }
%
    \explanation{Berechnen Sie zunächst den Parameter $\,c\,$ aus $f(x_1)=f(\var{ns1})=0$.
                Die zweite Nullstelle kann dann nach einem der bekannten 
                Verfahren bestimmt werden.}
%
    \begin{variables}
	    \randint{a}{1}{6}
	    \randint{b}{1}{6}
	    \randint{c}{1}{6}
	    \randint{d}{1}{6}
	    \function[calculate]{ns1}{-b/a}
	    \function{ns2}{-d/c}
	    \function{c_0}{b*d}
	    \function[expand, normalize]{f}{(c*x+d)*(a*x+b)}
	    \function[expand, normalize]{f1}{(c*x+d)*(a*x+b)-b*d}
	\end{variables}
	
    \begin{answer}
    	\lang{de}{\text{Der Parameter ist $\, c=$}}
   		\solution{c_0}
    \end{answer}
    
    \begin{answer}
    	\lang{de}{\text{Die zweite Nullstelle liegt in $\,x_2=$}}
     	\solution{ns2}
    \end{answer}
\end{question}

% Frage 3
\begin{question}
%
 	\type{input.function}
    \field{rational} 
%    
    \text{$f:\, \R \rightarrow \R$, $\;f(x)= \var{f}\,$ ist eine quadratische Funktion
          in Normalform. Bestimmen Sie die Linearfaktorzerlegung von $\,f\,$.\\
          
          $f(x)=($\ansref$)\cdot ($\ansref$)$
          }
%
    \explanation{Die Linearfaktorzerlegung ist 
                $\, f(x)=(x-x_1) \cdot (x-x_2)\,$, wobei $\,x_1\,$ und $\,x_2\,$
                die Nullstellen von $\,f\,$ sind.  }
%
%
    \begin{variables}
        \randrat{a}{-10}{10}{1}{8}
        \randrat{b}{-10}{10}{1}{8}
	    \function[normalize]{p}{-(b+a)}
	    \function[normalize]{q}{b*a}          
	    \function[normalize]{f}{x^2+p*x+q}
	    \function[normalize]{linfak1}{x-(a)}
        \function[normalize]{linfak2}{x-(b)}
	\end{variables}
    
    \permuteAnswers{1, 2} 
    \begin{answer}
    	\lang{de}{\text{$\qquad f(x)=$}}
     	\solution{linfak1}
	    \checkAsFunction{x}{-10}{10}{100}        
    \end{answer}
    
    \begin{answer}
    	\lang{de}{\text{$\qquad f(x)=$}}
     	\solution{linfak2}
	    \checkAsFunction{x}{-10}{10}{100}        
    \end{answer}
	
\end{question}

\end{problem}


\embedmathlet{mathlet}

\end{content}