\documentclass{mumie.element.exercise}
%$Id$
\begin{metainfo}
  \name{
    \lang{de}{Ü05: Linearfaktorzerlegung}
    \lang{en}{Exercise 5}
  }
  \begin{description} 
 This work is licensed under the Creative Commons License Attribution 4.0 International (CC-BY 4.0)   
 https://creativecommons.org/licenses/by/4.0/legalcode 

    \lang{de}{Linearfaktorzerlegung quadratischer Funktionen}
    \lang{en}{}
  \end{description}
  \begin{components}
  \end{components}
  \begin{links}
  \end{links}
  \creategeneric
\end{metainfo}
\begin{content}
\begin{block}[annotation]
	Im Ticket-System: \href{https://team.mumie.net/issues/21996}{Ticket 21996}
\end{block}

\begin{block}[annotation]
	Linearfaktorzerlegung quadratischer Funktionen
    und Bestimmung der Nullstellen
\end{block}


\title{
	\lang{de}{Ü05: Linearfaktorzerlegung}
  	\lang{en}{Exercise 5}
}

%
% Aufgabe
%
%\begin{block}[exercise]
%  Bestimmen Sie die Linearfaktorzerlegung der folgenden quadratischen 
%  Funktionen und ermitteln Sie ihre Nullstellen (falls vorhanden).
%   \begin{enumerate}[alph]
%     \item $f\colon \R\to\R,x \mapsto ~x^2+x-56$ 
%     \item $f\colon \R\to\R,x \mapsto ~2x^2+28x+98$
%     \item $f\colon \R\to\R,x\mapsto ~x^2+1$
%     \item $f\colon \R\to\R,x \mapsto ~x^2-9$
%   \end{enumerate}

Bestimmen Sie die Linearfaktorzerlegung der folgenden quadratischen Funktionen (falls vorhanden).
Bestimmen Sie hierzu zunächst die Nullstellen der Funktionen und faktorisieren 
Sie anschließend die Darstellung.

  \begin{table}[\class{items}]
% Video  
  \nowrap{a) $f\colon \R\to\R,x \mapsto ~x^2-x-2$} 
  & \nowrap{b) $h\colon \R\to\R,x \mapsto ~-x^2+2x+8$}\\
   \nowrap{c) $g\colon \R\to\R,z\mapsto ~-\frac{1}{2}z^2-3z-4$} 
  & \nowrap{d) $f\colon \R\to\R,a \mapsto ~a^2-2a+3$}\\
  \nowrap{e) $g\colon \R\to\R,c \mapsto ~-0,75 +c+c^2$} 
  & \nowrap{f) $h\colon \R\to\R,r \mapsto ~6r+3r^2$}\\
%   
  \nowrap{g) $f\colon \R\to\R,x \mapsto ~x^2+x-56$} 
  & \nowrap{h) $f\colon \R\to\R,x \mapsto ~2x^2+28x+98$}\\
   \nowrap{i) $f\colon \R\to\R,x\mapsto ~x^2+1$} 
  & \nowrap{j) $f\colon \R\to\R,x \mapsto ~x^2-9$}\\
  \end{table}

%\end{block}

%
% Lösungen
%

\begin{tabs*}[\initialtab{0}\class{exercise}]

  \tab{Antworten}
    \begin{enumerate}[alph]
% Video  
      \item $f(x)= (x-2)(x+1) $ 
      \item $h(x)= -(x+2)(x-4) $
      \item $g(z)= -\frac{1}{2}(z+2)(z+4) $ 
      \item Es exisitert keine Linearfaktorzerlegung für $f(a)=a^2-2a+3.$
      \item $g(c)= (c+\frac{3}{2})(c-\frac{1}{2}) $ 
      \item $h(r)= 3(r-0)(r+2)=3r(r+2)$
%       
      \item $ f(x)=(x-7)(x-(-8))=(x-7)(x+8)$ 
      \item $ f(x)=2(x-(-7))(x-(-7))=2(x+7)(x+7)$
      \item Es exisitert keine Linearfaktorzerlegung für $\,f(x)=x^2+1.$
      \item $ f(x)=(x-3)(x-(-3))=(x-3)(x+3)$
    \end{enumerate}
     
     \tab{\lang{de}{Lösungsvideo a) - f)}}	
\youtubevideo[500][300]{AUj1LiB2gb4}\\
     
   \tab{\lang{de}{Lösung g)}}
	\begin{incremental}[\initialsteps{1}]
	
     \step 
        Wir bestimmen zunächst die Nullstellen mit Hilfe der die pq-Formel
        \begin{eqnarray*}
         x_{1,2}&=&-\frac{1}{2}\pm \sqrt{\left(-\frac{1}{2}\right)^2+56}=-\frac{1}{2}\pm \sqrt{\frac{1}{4}+\frac{4\cdot56}{4}}\\
                &=&-\frac{1}{2}\pm \sqrt{\frac{225}{4}}=-\frac{1}{2}\pm \frac{15}{2}.
        \end{eqnarray*} 
     
     \step 
        Wir erhalten also die zwei Nullstellen $x_1=7$ und $x_2=-8\,$
        und damit die Linearfaktorzerlegung \[f(x)=(x-7)(x-(-8))=(x-7)(x+8).\]
    \end{incremental}
    
  	\tab{\lang{de}{Lösung h)}}
    \begin{incremental}[\initialsteps{1}]
     \step 
        Um die Nullstellen zu bestimmen, suchen wir die Lösungen von
        \[
         f(x)=0 \iff 2x^2+28x+98=0.
        \]
      \step 
        Damit wir die pq-Formel anwenden können, müssen wir die quadratische 
        Gleichung zuvor in Normalform bringen
        \[
         2x^2+28x+98=0 \iff x^2+14x+49=0.
        \]
        Nun können wir die pq-Formel anwenden und erhalten
        \begin{align*}
         x_{1,2}&=-7\pm \sqrt{7^2-49}\\
                &=-7\pm \sqrt{0}=-7,
        \end{align*}
        
      \step
        Die quadratische Gleichung hat also genau eine Nullstelle $x=-7$.
        Wir können aber, bei Erkennen der \emph{binomischen Formel}, den
        quadratischen Term auch direkt umformen zu
        \[
         x^2+14x+49=(x+7)^2=0 .
        \]
                
      \step
        Da der Leitkoeffizient $2$  zuvor ausgeklammert wurde, müssen wir
        ihn zur Linearfaktorzerlegung von $f$ wieder hinzufügen und 
        erhalten schließlich
        \[  f(x)=2(x+7))^2=2(x+7)(x+7)2=2(x-(-7))(x-(-7)).\]
 \end{incremental}
 
 \tab{\lang{de}{Lösung i)}}
    \begin{incremental}[\initialsteps{1}]
     \step 
        In diesem Fall existiert keine Nullstelle. \\
        
        Dies erkennt man z.B. durch
        \[
         f(x)=0\iff x^2+1=0 \iff x^2=-1,
        \]
       was einen Widerspruch darstellt, da für alle reellen Zahlen $x$ stets
       $\, x^2>0\,$ gilt.\\
       
       Formal kann man aber auch die Diskriminante aus der pq-Formel für diese
       quadratische Gleichung betrachten. Dann gilt
       \[
         D=0-1=-1<0
       \]
       und man erhält damit folglich die gleiche Aussage.
     \step
        Da die quadratischen Funktion $\,f(x)=x^2+1\,$ keine Nullstellen besitzt,
        gibt es folglich auch keine Linearfaktorzerlegung für $f$.
        
    \end{incremental}

    \tab{\lang{de}{Lösung j)}}
        
        Die Anwendung der 3. binomischen Formel liefert direkt die
        Linearfaktorzerlegung von $f:$
        \[f(x)=x^2-9=(x-3)(x+3)=(x-3)(x-(-3)).\] 
        Die Nullstellen können wir nun ablesen. Es sind $\,x_1=3\,$ und $\,x_1=-3.$
     
\end{tabs*}
\end{content}
    