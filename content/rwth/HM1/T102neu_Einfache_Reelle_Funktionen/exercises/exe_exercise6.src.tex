\documentclass{mumie.element.exercise}
%$Id$
\begin{metainfo}
  \name{
    \lang{de}{Ü06: Parabeleigenschaften}
  }
  \begin{description} 
 This work is licensed under the Creative Commons License Attribution 4.0 International (CC-BY 4.0)   
 https://creativecommons.org/licenses/by/4.0/legalcode 

    \lang{de}{Eigenschaften einer Parabel}
  \end{description}
  \begin{components}
  \end{components}
  \begin{links}
    \link{generic_article}{content/rwth/HM1/T102neu_Einfache_Reelle_Funktionen/g_art_content_08_quadratische_funktionen.meta.xml}{content_08_quadratische_funktionen}
  \end{links}
  \creategeneric
\end{metainfo}
%
\begin{content}
\begin{block}[annotation]
	Im Ticket-System: \href{https://team.mumie.net/issues/21998}{Ticket 21998}
\end{block}
%
 \begin{block}[annotation]
   Neue Übung zur Beschreibung einer Parabel \\
   -> Bedeutung d. Parameter (Öffnung, Streckung und Schnitt mit y-Achse)\\
   -> Bestimmung Nullstellen (Schnitt mit x-Achse) \\
   -> Bestimmung Scheitelpunkt und Scheitelpunktform 
\end{block}
%

\title{
    \lang{de}{Ü06: Parabeleigenschaften}
  }
 
%
% Aufgabenstellung
%
Beschreiben Sie die Lage der folgenden Parabeln im zweidimensionalen Koordinatenraum.
Bestimmen Sie ihre Nullstellen, die Scheitelpunktform und den Scheitelpunkt.
\begin{enumerate}[alph]
  \item $ f(x)=3x^2-30x+63$
  \item $ f(x)=-\frac{1}{2}x^2-6x-18$
\end{enumerate}
%
% Lösungen
%
  \begin{tabs*}[\initialtab{0}\class{exercise}]
  
    \tab{Antworten}
      \begin{enumerate}[alph]
        \item Die Parabel $\, f(x)=3x^2-30x+63\,$ mit der Scheitelpunktform $\, f(x)=3(x-5)^2-12\,$        
         \begin{itemize}
           \item ist nach oben geöffnet und verläuft steiler als die Normalparabel,
           \item schneidet die $y$-Achse in $\,(0;63),$
           \item hat zwei Nullstellen,  $\;x_1=3\;$ und $\;x_2=7,$
           \item und ihren Scheitelpunkt in  $\; S=(5;-12).$
        \end{itemize}
        
        \item Die Parabel $\, f(x)=-\frac{1}{2}x^2-6x-18\,$ mit der Scheitelpunktform 
                $\, f(x)=-\frac{1}{2} (x+6)^2\,$        
         \begin{itemize}
           \item ist nach unten geöffnet und verläuft flacher als die Normalparabel,
           \item schneidet die $y$-Achse in $\,(0;-18),$
           \item hat hat nur eine Nullstelle in $\;x_0=-6\;$ und \emph{berührt} damit die
                $x-$Achse genau in ihrem Scheitelpunkt $\; S=(-6;0).$
        \end{itemize}
        
      \end{enumerate}
%
    \tab{Lösung a)} 
     \begin{incremental}[\initialsteps{1}]
      \step 
        Wir untersuchen zunächst die Parameter der Parabelfunktion.
        Der Leitkoeffizient  ist $\,3 >0,\,$ daher ist die Parabel 
        nach oben geöffnet. Am Streckfaktor $\,3 >1 \,$ erkennen
        wir, dass die Parabel im Vergleich zur Normalparabel 
        steiler verläuft.\\
        
        Der Koeffizient \emph{"`ohne $x$"'}, hier $\,63$, liefert uns den Schnittpunkt 
        $\,(0;63)\,$ der Parabel mit der $y$-Achse.   
        
      \step  
        Zur Berechnung der Nullstellen der Parabel verwenden wir die 
        \ref[content_08_quadratische_funktionen][Mitternachtsformel:]{rule:quadratic_nst}
        \[
        \displaystyle{
          \begin{mtable}[\cellaligns{cccl}]
               & x_{1,2} &\,=\,& \frac{30 \pm \sqrt{30^2-4 \cdot 3 \cdot 63}}{2 \cdot 3} \\
               &   &\,=\,& \frac{30 \pm \sqrt{900-756}}{6}     \\ 
               &   &\,=\,& \frac{30}{6} \pm \frac{\sqrt{144}}{6}     \\
               &   &\,=\,& 5 \pm 2   
          \end{mtable}
          }
        \]
        Die Nullstellen der Parabel liegen also in $\, x_1=3\,$ und $\, x_2=7.$
      
      \step 
        Wir bestimmen nun noch des Scheitelpunkt der Parabel und formen hierzu
        die Parabelgleichung in die Scheitelpunktform um.
        \[
          \begin{mtable}[\cellaligns{cccl}]
           & f(x)=3x^2-30x+63 &\,=\,& 3(x^2-10x+21) &\vert \text{quadr. Erg"anzung} \\
           &   &\,               =\,& 3(x^2-10x+25-25+21) &  \\ 
           &   &\,               =\,& 3 \big( (x-5)^2 -4 \big)  &  \\ 
           &   &\,               =\,& 3 (x-5)^2 -12  &  
          \end{mtable}
        \]
        Aus der Scheitelpunktform $\,f(x)=3 (x-5)^2 -12  \,$ können wir den Scheitelpunkt
        der Parabel direkt ablesen. Er liegt in $\,S=(5;-12).$ 
 %     
   \end{incremental}

    \tab{Lösung b) } 
     \begin{incremental}[\initialsteps{1}]    
      \step 
        Wir untersuchen zunächst die Parameter der Parabelfunktion.
        Der Leitkoeffizient  ist $\,-\frac{1}{2} <0,\,$ daher ist die Parabel 
        nach unten geöffnet. Als Streckfaktor gibt
        $\,\abs{-\frac{1}{2}} =\frac{1}{2} <1 \,$ uns an, dass die Parabel im 
        Vergleich zur Normalparabel flacher verläuft.\\
        
        Der Koeffizient \emph{"`ohne $x$"'}, hier $\,(-18)$, liefert uns den Schnittpunkt 
        $\,(0;-18)\,$ der Parabel mit der $y$-Achse.   
        
      \step  
        Zur Berechnung der Nullstellen der Parabel verwenden wir die 
        \ref[content_08_quadratische_funktionen][Mitternachtsformel:]{rule:quadratic_nst}
        \[
        \displaystyle{
          \begin{mtable}[\cellaligns{cccl}]
               & x_{1,2} &\,=\,& \frac{-(-6) \pm \sqrt{(-6)^2-4 \cdot (-\frac{1}{2}) \cdot (-18)}}{2 \cdot (-\frac{1}{2})} \\
               &   &\,=\,& \frac{6 \pm \sqrt{36-4\cdot 9}}{(-1)}     \\ 
               &   &\,=\,& \frac{6 \pm 0}{(-1)}     \\
               &   &\,=\,& -6 
          \end{mtable}
          }
        \]
        Die Parabel hat nur eine Nullstelle, und zwar in $\, x_0=-6.$
      
      \step 
        Wir bestimmen nun noch des Scheitelpunkt der Parabel und formen hierzu
        die Parabelgleichung in die Scheitelpunktform um.
        \[
          \begin{mtable}[\cellaligns{cccl}]
           & f(x)=-\frac{1}{2}x^2-6x-18 &\,=\,& -\frac{1}{2}(x^2+12x+36) &\vert \text{1. bin. Formel} \\
           &   &\,               =\,&  -\frac{1}{2} (x+6)^2   &  
          \end{mtable}
        \]
        Aus der Scheitelpunktform $\,f(x)=-\frac{1}{2} (x+6)^2  \,$ können wir den Scheitelpunkt
        der Parabel direkt ablesen. Er liegt in $\,S=(-6;0)$, also in der Nullstelle der Parabel.
        Das bedeutet, dass die Parabel die $x$-Achse in diesem Punkt \emph{berührt}, ansonsten aber,
        da sie nach unten geöffnet ist, komplett \emph{"`unterhalb"'} der $x-$Achse liegt.
%     
   \end{incremental}

%
%        \tab{\lang{de}{Video: ähnliche Übungsaufgabe}}	
%        \youtubevideo[500][300]{379wH3cSCkg}\\
%  
        
  \end{tabs*} 

\end{content}


