\documentclass{mumie.element.exercise}
%$Id$
\begin{metainfo}
  \name{
    \lang{de}{Ü07: Parabelbestimmung}
  }
  \begin{description} 
 This work is licensed under the Creative Commons License Attribution 4.0 International (CC-BY 4.0)   
 https://creativecommons.org/licenses/by/4.0/legalcode 

    \lang{de}{Bestimmung einer Parabel aus gegebenen Punkten}
  \end{description}
  \begin{components}
  \end{components}
  \begin{links}
\link{generic_article}{content/rwth/HM1/T102neu_Einfache_Reelle_Funktionen/g_art_content_08_quadratische_funktionen.meta.xml}{content_08_quadratische_funktionen}
\end{links}
  \creategeneric
\end{metainfo}
%
\begin{content}
\begin{block}[annotation]
	Im Ticket-System: \href{https://team.mumie.net/issues/21999}{Ticket 21999}
\end{block}
 
%
 \begin{block}[annotation]
   Bestimmung einer Parabel aus gegebenen Punkten
 \end{block}

  \title{
    \lang{de}{Ü07: Parabelbestimmung}
  }

%
% Aufgabenstellung
%
Bestimmen Sie die Funktionsgleichung der folgenden Parabeln.
\begin{enumerate}[alph]
% Video
  \item Gesucht ist eine Parabel, die durch die Punkte $\,(-1;1)$, 
        $\,(3;4)\,$ verläuft, und deren Scheitelpunkt auf der $y$-Achse liegt.
 \item Gesucht ist eine Parabel, die durch die Punkte $\,(0;\frac{4}{3})$, 
        $\,(5;3)\,$ verläuft, und deren Scheitelpunkt auf der $x$-Achse liegt.
%
  \item Gesucht ist eine Parabel, die durch die Punkte $\,P_1=(4;-3)$, 
        $\,P_2=(2;-1)\,$ und $\,P_3=(8;5)\,$ verläuft.

  \item Gesucht ist eine Parabel mit Scheitelpunkt in $\,S=(4;-3),\,$ 
        die die $y-$Achse im Punkt $\,(0;5)\,$ schneidet.     
\end{enumerate}
%
% Lösungen
%
  \begin{tabs*}[\initialtab{0}\class{exercise}]
  
    \tab{Antworten}
      \begin{enumerate}[alph]
% Video
        \item $f(x)=\frac{3}{8}x^2+\frac{5}{8}$         
        \item Es gibt 2 Lösungen:\\ 
        
              $f_1(x)=\frac{1}{3}(x-2)^2=\frac{1}{3}x^2-\frac{4}{3}x+\frac{4}{3} \quad$ und \\
              
              $f_2(x)=\frac{1}{75}(x+10)^2=\frac{1}{75}x^2+\frac{20}{75}x+\frac{100}{75}$
%
        \item $f(x)=\frac{1}{2}x^2-4x+5$         
        \item $f(x)=\frac{1}{2}(x-4)^2-3=\frac{1}{2}x^2-4x+5$
      \end{enumerate}
% Video   
   \tab{\lang{de}{Lösungsvideo a) und b)}}	
            \youtubevideo[500][300]{wkZ5gNb4ba0}\\
%
    \tab{Lösung c)} 
     \begin{incremental}[\initialsteps{1}]
      \step
        Gesucht sind die reellen Parameter $a, b\,$ und $\,c\,$ der Funktionsgleichung
        $\,f(x)=ax^2+bx+c$, die eine Parabel durch die gegebenen Punkte $\,P_1, P_2\,$ 
        und $\,P_3\,$ beschreibt. 
        
      \step 
        Die Bedingung, dass die drei Punkte $\,P_1, P_2\,$ und $\,P_3\,$ auf der Parabel
        liegen, liefert das folgende lineare Gleichungssystem in den Unbekannten $a, b\,$
        und $\,c:$
        \begin{align*}
            \text{(I)}   &\qquad f(4)= a\cdot 16+ b\cdot 4+c    &\,=\,& -3     & \\
            \text{(II)}  &\qquad f(2)= a\cdot \, 4+ b\cdot 2+c  &\,=\,& -1     & \\
            \text{(III)} &\qquad f(8)= a\cdot 64+ b\cdot 8+c    &\,=\,& \, 5   & \\
        \end{align*}
        Wir lösen dieses Gleichungssystem nach dem Additionsverfahren.
        
      \step
        Zuerst eliminieren wir die Variable $c$ und die Gleichung (II), indem wir 
        diese jeweils von der Gleichung (I) und der Gleichung (III) subtrahieren.
        Wir erhalten 
        \begin{align*}
          \text{(I)-(II)}\;&=\text{(I*)}   &\qquad   a\cdot 12+ b\cdot 2 &\,=\,& -2 &\qquad &\vert :2\\
          \text{(III)-(II)}\;&=\text{(III*)} &\qquad   a\cdot 60+ b\cdot 6 &\,=\,& 6 &\qquad &\vert :6\\
        \end{align*}
        \begin{center}
        _____________________________________________________
        \end{center}
         \begin{align*}
            \text{(I*)}   &\qquad   a\cdot 6+ b\cdot 1 &\,=\,& -1 & \\
            \text{(III*)} &\qquad   a\cdot 10+ b\cdot 1 &\,=\,& 1 & \\
        \end{align*}
        Nun subtrahieren wir (I*) von (III*), um die Variable $b$ zu eliminieren.
        Wir erhalten
         \[4a=2 \quad \Leftrightarrow \quad a=\frac{1}{2}.\]
        $a=\frac{1}{2}$ eingesetzt in (I*) liefert
        \[\frac{1}{2}\cdot 6+ b = -1 \quad \Leftrightarrow \quad b=-1-3=-4.\]
        Den dritten Parameter $c$ können wir nun zum Beispiel aus Gleichung (II)
        berechnen, indem wir $a=\frac{1}{2}$ und $b=-4$ dort einsetzen. Wir erhalten
        \[  4+ b\cdot 2+c =-1 \quad \Leftrightarrow \quad c=-1- \frac{1}{2}\cdot  4- (-4)\cdot 2=5\]
        Die Funktionsgleichung der Parabel durch die Punkte $\,P_1, P_2\,$ und $\,P_3\,$ 
        lautet also
        \[f(x)=\frac{1}{2}x^2-4x+5.\]
 
   \end{incremental}
%
  \tab{Lösung d) } 
   \begin{incremental}[\initialsteps{1}]
    \step
        Da der Scheitelpunkt der gesuchten Parabel bekannt ist, bestimmen wir zunächst
        ihre Scheitelpunktform. Diese ist gegeben durch
        \[
            f(x)=a \cdot (x-4)^2-3, 
        \]
        wobei $a$ der Streckfaktor der Parabel ist.
      
      \step
        Dieser lässt sich durch Einsetzen des bekannten Schnittpunkts $(0;5)$ mit der 
        $y-$Achse aus der Scheitelpunktform berechnen.
        \[
            f(0)=a \cdot (0-4)^2-3=5 \quad \Leftrightarrow \quad 16a=8 
                                     \quad \Leftrightarrow \quad a=\frac{1}{2}
        \]
        Die Scheitelpunktform ist somit $\; f(x)=\frac{1}{2} \cdot (x-4)^2-3.$
      
      \step
        Mithilfe der 
        \ref[content_08_quadratische_funktionen][Umrechnungsformeln]{rule:umrechnung_scheitelpkt}
        können wir nun die fehlenden Parameter $b$ und $c$ für die Parabelgleichung berechnen.
         \begin{align*}
            b= -2as_{x} \quad &= \frac{1}{2} \cdot 4=-4 \quad &\text{und} \\
            c= as_{x}^2+s_{y} \,&= \frac{1}{2} \cdot 4^2+(-3)=5  &
        \end{align*}
        
        Damit ist die Funktionsgleichung für die gesuchte Parabel bestimmt. Sie lautet
        \[f(x)= \frac{1}{2}x^2-4x+5, \]
        es handelt sich also um dieselbe Parabel wie in Teilaufgabe b).
   \end{incremental}

  \end{tabs*} 




\end{content}

