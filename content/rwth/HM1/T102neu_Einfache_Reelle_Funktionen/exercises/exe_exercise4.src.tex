\documentclass{mumie.element.exercise}
%$Id$
\begin{metainfo}
  \name{
    \lang{de}{Ü04: Geradenbestimmung}
    \lang{en}{Exercise 4}
  }
  \begin{description} 
 This work is licensed under the Creative Commons License Attribution 4.0 International (CC-BY 4.0)   
 https://creativecommons.org/licenses/by/4.0/legalcode 

    \lang{de}{Bestimmung einer Geraden aus Punkt + Steigung oder aus zwei Punkten}
    \lang{en}{}
  \end{description}
  \begin{components}
  \end{components}
  \begin{links}
    \link{generic_article}{content/rwth/HM1/T102neu_Einfache_Reelle_Funktionen/g_art_content_07_geradenformen.meta.xml}{content_07_geradenformen}
  \end{links}
  \creategeneric
\end{metainfo}
\begin{content}

\begin{block}[annotation]
	Im Ticket-System: \href{https://team.mumie.net/issues/21995}{Ticket 21995}
\end{block}

\begin{block}[annotation]
	Steigungsform einer Geraden aus einem bekannten Punkt und der Steigung 
    oder aus zwei vorgegebenen Punkten bestimmen
\end{block}

\title{
	\lang{de}{Ü04: Geradenbestimmung}
  	\lang{en}{Exercise 4}
}

%
% Aufgabenstellung
%
Bestimmen Sie die Geradengleichung zu einer Geraden
\begin{enumerate}[alph]
    \item die durch die Punkte $\,(-1;0)\,$ und $\,(1;2)\,$ führt,
    \item die durch den Punkt $\,(2;-1)\,$ führt und die Steigung $m=-2\,$ hat,
    \item die die $x$-Achse bei $-2$ und die $y$-Achse bei $1$ schneidet,
    \item die durch den Punkt $\;(1;-2)\,$ führt und senkrecht zu der Geraden ist,
          die durch $\, y=\frac{1}{3} x-1\,$ beschrieben wird,    
    \item mit der Steigung $m=-2$, die durch den Punkt $P=(1;9)$ verläuft.
\end{enumerate}
 
 Geben Sie jeweils die Geradengleichung in der \emph{Punkt-Steigungsform} an.
   
%
% Lösung
%

\begin{tabs*}[\initialtab{0}\class{exercise}]
   
  \tab{\lang{de}{Antworten}}
  
    \begin{enumerate}[alph]
        \item $y=x+1$,
        \item $y=-2x+3$,
        \item $y=\frac{1}{2}x+1$,
        \item $y=-3x+1$,  
        \item $y=-2x+11\,.$
    \end{enumerate}  

  \tab{\lang{de}{Lösungsvideo a) - d)}}	
    \youtubevideo[500][300]{UbXskYhBsM0}\\
    
  \tab{\lang{de}{Lösung e)}}	

    Die Gerade ist durch den Punkt $P$ und die Steigung $m$ eindeutig festgelegt. 
    Die Bestimmung der 
    \ref[content_07_geradenformen][Punkt-Steigungsform]{rule:punk_steig_form}
    für die Geradengleichung erfolgt durch die Formel
    \[y=m \cdot x+(y_{P}-m \cdot x_{P})\quad,\quad \text{ wobei }P=(x_{P};y_{P}).\]
    Einsetzen der Werte liefert uns die gewünschte Darstellung der gesuchten Gerade
    \[y=(-2)\cdot x+(9-(-2)\cdot 1)=-2x+11\,.\]

\end{tabs*}
\end{content}

