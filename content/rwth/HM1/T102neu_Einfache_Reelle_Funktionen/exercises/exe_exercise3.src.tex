\documentclass{mumie.element.exercise}
%$Id$
\begin{metainfo}
  \name{
    \lang{de}{Ü03: Geradenbestimmung}
    \lang{en}{Exercise 3}
  }
  \begin{description} 
 This work is licensed under the Creative Commons License Attribution 4.0 International (CC-BY 4.0)   
 https://creativecommons.org/licenses/by/4.0/legalcode 

    \lang{de}{Bestimmung einer Geraden aus vorgegebenen Punkten}
    \lang{en}{}
  \end{description}
  \begin{components}
  \end{components}
  \begin{links}
    \link{generic_article}{content/rwth/HM1/T102neu_Einfache_Reelle_Funktionen/g_art_content_07_geradenformen.meta.xml}{content_07_geradenformen}
  \end{links}
  \creategeneric
\end{metainfo}
\begin{content}
\begin{block}[annotation]
	Im Ticket-System: \href{https://team.mumie.net/issues/21994}{Ticket 21994}
\end{block}

\begin{block}[annotation]
	Steigungsform einer Geraden aus 2 Punkten bestimmen
\end{block}

\title{
	\lang{de}{Ü03: Geradenbestimmung}
  	\lang{en}{Exercise 3}
}
%
% Aufgabenstellung
%
\begin{enumerate}
  \item Bestimmen Sie die Gerade, die durch die Punkte $\,P_1=(10;3)$ und $P_2=(5;4)\,$ 
    verläuft. Geben Sie die Geradengleichung in der \emph{Punkt-Steigungsform} an. 
  \item Gibt es eine Gerade, die durch die 3 angegebenen Punkte führt?
  \begin{table}[\class{items}]
  \nowrap{a) $\; (-1;2), (1;-1), (3;-3) $}
  & \nowrap{b) $\; (-1;3), (1;1), (2;0) $}\\
   \nowrap{c) $\; (-1;3), (1;2), (5;0) $} 
  & \nowrap{d) $\; (-1;2), (2;1), (4;2) $}\\
  \end{table}

\end{enumerate}
%
% Lösung
%

\begin{tabs*}[\initialtab{0}\class{exercise}]

  \tab{\lang{de}{Antworten}}
    \begin{enumerate}
      \item $y=-\frac{1}{5}x+5$ 
      
      \item a) $\;$ nein $\qquad$ b) $\;$ ja $\qquad$ c) $\;$ ja $\qquad$ d) $\;$ nein
      
  \end{enumerate}

  \tab{\lang{de}{Lösung 1.}}	
	\begin{incremental}[\initialsteps{1}]
     \step 
        Die Gerade ist durch die zwei angegebenen Punkte $\,P_{1}\,$ und $\,P_{2}\,$
        eindeutig festgelegt und wird als   
        \ref[content_07_geradenformen][Zweipunktform]{rule:zweipunktform} beschrieben durch  
        \[
            y=\frac{y_{2}-y_{1}}{x_{2}-x_{1}}(x-x_{1})+y_{1}\, , \quad
            \text{ wobei }P_{1}=(x_{1};y_{1}),P_{2}=(x_{2};y_{2})\,.
        \]
      \step 
        Setzen wir die Werte für $\,P_1$ und $P_2\,$ aus der Aufgabenstellung in diese Formel
        ein, so erhalten wir
        \[y=\frac{4-3}{5-10}(x-10)+3.\]
        Durch geeignete Umformung lässt sich hieraus schließlich die \emph{Punkt-Steigungsform}
        der gesuchten Geraden ableiten.
        \[
          \begin{mtable}[\cellaligns{cccl}]
               & y &\,=\,& \frac{4-3}{5-10}(x-10)+3 \\
               &   &\,=\,& \frac{-1}{5}(x-10)+3      \\ 
               &   &\,=\,& -\frac{1}{5}x+5  
          \end{mtable}
        \]
     \step   
        Die Berechnung der Parameter $\,m\,$ und $\,b\,$ kann auch direkt mittels 
        der bekannten Formeln erfolgen:
        \[m=\frac{-1}{5} \quad \text{und} \quad b=\frac{-1}{5}\cdot(-10)+3=5.\]
        
        Die \emph{Punkt-Steigungsform} der Geraden ist dann
        \[y=mx+b=-\frac{1}{5}x+5\,.\]
     
  \end{incremental}
  
    
  \tab{\lang{de}{Lösungsvideo 2.}}
    \youtubevideo[500][300]{9CEEw8nw7qA}\\
  
\end{tabs*}
\end{content}