\documentclass{mumie.element.exercise}
%$Id$
\begin{metainfo}
  \name{
    \lang{de}{Ü02: Geraden (TA)}
  }
  \begin{description} 
 This work is licensed under the Creative Commons License Attribution 4.0 International (CC-BY 4.0)   
 https://creativecommons.org/licenses/by/4.0/legalcode 

    \lang{de}{Eigenschaften von Geraden}
  \end{description}
  \begin{components}
    \component{generic_image}{content/rwth/HM1/images/g_tkz_T102_Exercise02.meta.xml}{T102_Exercise02}
  \end{components}
  \begin{links}
  \end{links}
  \creategeneric
\end{metainfo}
%
\begin{content}
\begin{block}[annotation]
	Im Ticket-System: \href{https://team.mumie.net/issues/21992}{Ticket 21992}
\end{block}

%
 \begin{block}[annotation]
%
      Bedeutung der Parameter und Lage einer Geraden - Textaufgaben mit Video
%
\end{block}

  \title{
    \lang{de}{Ü02: Geraden}
  }
%
% Aufgabenstellung
%
Lösen Sie die folgenden Textaufgaben mit Hilfe von linearen Funktionen.

\begin{enumerate}[alph]
    \item Auf Meereshöhe ist der Luftdruck 1013 hPa (Hektopascal). Pro 8 m Höhe nimmt er um ca. 1 hPa ab.\\
          Geben Sie den funktionalen Zusammenhang zwischen Höhe und Luftdruck an. Wie groß ist der Druck in 
          500 m Höhe?
    \item Herr Müller gründet ein Gewerbe: Er produziert und verkauft Lebkuchen. Die Anschaffung der Produktionsmaschine 
          kostet 10.000 €. Jeder verkaufte Lebkuchen bringt ihm einen Gewinn von 0,58 €.\\
          Wie hoch ist sein Gesamtgewinn/-verlust in Abhängigkeit von der Anzahl der verkauften Lebkuchen? Wieviel Lebkuchen
          muss er verkaufen, um den break-even (Gesamtgewinn/-verlust = 0) zu erreichen?
    \item Ein Joghurtbecher hat eine Höhge von 8 cm, einen unteren Radius von 2 cm und einen oberen Radius von 3 cm.\\
          Wie groß ist der Radius in Abhängigkeit von der Höhe?
    \item Tom und Max machen eine Wettrennen über eine Strecke von 10 km. 
          Tom läuft dabei mit einer konstanten Geschwindigkeit von 12 km pro Stunde 
          und Max, der deutlich langsamer ist, läuft konstant 6 km pro Stunde.
          Welchen Vorsprung bräuchte Max, um zeitgleich mit Tom im Ziel anzukommen?
\end{enumerate}
%%%%%%%%%%%%%%%%%%%%%%%%%%%%%%%%%%%%%%%%%%%%%%%%%%%%%%%%%%%%%%%%%%%%%%%%%%%%%%%%%%%%%%%%%%%%%%%%%%%%%%%%%%%%%%%%%%%%%%%
%
% Lösung
%

\begin{tabs*}[\initialtab{0}\class{exercise}]
%%%%%%%%%%%%%%%%%%%%%%%%%%%%%%%%%%%%%%%%%%%%%%%%%%%%%%%%%%%%%%%%%%%%%%%%%%%%%%%%%%%%%%%%%%%%%%%%%%%%%%%%%%%%%%%%%%%%%%%
\tab{\lang{de}{Antworten}}
  \begin{enumerate}[alph]
      \item In 500 m Höhe beträgt der Druck $\, 950,5$ hPa (Hektopascal).
      
      \item Um den break-even zu erreichen, muss Herr Müller mindestens $17.242$ Lebkuchen verkaufen.
      
      \item Der Radius $r$ lässt sich in Abhängigkeit von der Höhe $h$ berechnen aus $\; r(h)= \frac{1}{8} h+2 .$
            
      \item Max muss im Vergleich zu Tom $\,50$ min früher starten.
  \end{enumerate}
%
\tab{\lang{de}{Lösungsvideo a) - c)}}
  \youtubevideo[500][300]{1Uo-Ijvzl_c}\\

\tab{\lang{de}{Lösung d)}}	
%%%%%%%%%%%%%%%%%%%%%%%%%%%%%%%%%%%%%%%%%%%%%%%%%%%%%%%%%%%%%%%%%%%%%%%%%%%%%%%%%%%%%%%%%%%%%%%%%%%%%%%%%%%%%%%%%%%%%%%
	\begin{incremental}[\initialsteps{1}]    
     \step 
        Die Läufe von Tom und Max, also die jeweils gelaufenen Kilometer in Abhängigheit von der Zeit,
        lassen sich aufgrund der konstanten Geschwindigkeit jeweils als Gerade darstellen. Die 
        individuelle Laufgeschwindigkeit gibt dabei die Steigung der jeweiligen Geraden an. \\
        
        Bei einem Start ab km 0 wird also der Lauf von Tom beschrieben durch  $\,g_{Tom}(t)=12t \,$ und der Lauf von Max durch 
        $\,g_{Max}(t)=6t$, wobei $t$ die Laufzeit in Stunden angibt und $g_{...}(t)$ die gelaufenen Kilometer.
          
     \step   
        Wenn beide gleichzeitig loslaufen, erreicht Tom das $10$ km entfernte Ziel nach $50$ min 
        (bzw. $\frac{5}{6}$ h), denn $\,g_{Tom}(\frac{5}{6})=12\cdot \frac{5}{6}=10.$ \\
        
        Max hingegen braucht für dieselbe Strecke $100$ min (bzw. $\frac{10}{6}$ h), denn 
        $\,g_{Max}(\frac{10}{6})=6\cdot \frac{10}{6}=10.$
          
     \step 
        Zur Ermittlung des Vorsprungs, den Max benötigt, um gleichzeitig mit Tom das Ziel zu erreichen, 
        verschieben wir die Gerade $g_{Max}$ bei unveränderter Steigung so, dass sie ebenfalls den 
        Zielpunkt $\,Z=(\frac{5}{6};10)\,$ durchläuft. \\
        
                Graphisch bedeutet das 
        \begin{center}
          \image{T102_Exercise02}
        \end{center}
        
        Der neue Laufweg von Max wird also beschrieben durch $\,g_{Max-neu}(t)=6t +c. \,$ Da der
        Zielpunkt $\,Z\,$ auf dieser Geraden liegen soll, gilt 
        $\,g_{Max-neu}(\frac{5}{6})=6 \cdot \frac{5}{6}+c=10. \,$ Folglich ist $\, c=5 \,$ 
        und damit \[g_{Max-neu}(t)=6t +5.\]
                  
     \step 
         Die neue Geradengleichung $\,g_{Max-neu}(t)=6t +5 \,$ können wir wie folgt interpretieren: \\
    
         Der $y-$Achsen-Schnitt der neuen Gerade $\,g_{Max-neu}\,$ gibt den notwendigen km-Vorsprung
         an, den Max benötigt, um mit Tom gleichzeitig das Ziel zu erreichen, denn wenn Max zum Zeitpunkt 
         $0$ bereits $5$ km ab Start zurückgelegt hat, erreicht er $\,\frac{5}{6}\,$ Stunden später das Ziel,
         während Tom in derselben Zeit die komplette Strecke läuft.\\

        Der Schnittpunkt der neuen Geraden $\,g_{Max-neu}\,$ mit der $t-$Achse liefert den zeitlichen
        Vorsprung, den Max benötigt, um mit Tom gleichzeitig das Ziel zu erreichen.
        \[g_{Max-neu}(t)=6t +5 =0 \quad \Leftrightarrow \quad t=-\frac{5}{6}\]
        Max muss demnach im Vergleich zu Tom $\frac{5}{6}$ Stunden, also $50$ min früher starten.
     
  \end{incremental}
  
  \end{tabs*}
\end{content}