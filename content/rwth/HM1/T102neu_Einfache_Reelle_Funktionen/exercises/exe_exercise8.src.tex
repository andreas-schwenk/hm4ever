\documentclass{mumie.element.exercise}
%$Id$
\begin{metainfo}
  \name{
    \lang{de}{Ü08: Parabeln (TA)}
  }
  \begin{description} 
 This work is licensed under the Creative Commons License Attribution 4.0 International (CC-BY 4.0)   
 https://creativecommons.org/licenses/by/4.0/legalcode 

    \lang{de}{Anwendungsübung zur Bestimmung einer Parabel}
  \end{description}
  \begin{components}
\component{generic_image}{content/rwth/HM1/images/g_tkz_T102_Bridge_A.meta.xml}{T102_Bridge_A}

    \component{generic_image}{content/rwth/HM1/images/g_tkz_T102_Bridge_B.meta.xml}{T102_Bridge_B}    
  \end{components}
  \begin{links}
\link{generic_article}{content/rwth/HM1/T102neu_Einfache_Reelle_Funktionen/g_art_content_08_quadratische_funktionen.meta.xml}{content_08_quadratische_funktionen}
\end{links}
  \creategeneric
\end{metainfo}
%
\begin{content}
\begin{block}[annotation]
	Im Ticket-System: \href{https://team.mumie.net/issues/22000}{Ticket 22000}
\end{block}

 \begin{block}[annotation]
      Textaufgabe / Anwendung \\
      Bedeutung der Parameter und Lage einer Parabel
\end{block}

  \title{
    \lang{de}{Ü08: Parabeln}
  }
 
%
%
% Aufgabenstellung
%
Lösen Sie die folgenden Textaufgaben mit Hilfe quadratischer Funktionen.

\begin{enumerate}[alph]
    \item Die Fahrbahn einer Hängebrücke wird von zwei parabelförmigen Stahlseilen gehalten, 
        die durch mehrere Pfeiler mit der Fahrban verbunden sind und an ihrem tiefsten Punkt 
        die Fahrbahn berühren. An den jeweils äußeren Hauptpfeilern ist das Stahlseil in einer
        Höhe von 8 m befestigt und hat in dieser Höhe eine Spannweite von 40 m.
%        Graphisch lässt sich die Situation im zweidimensionalen Koordinatensystem wie folgt darstellen:
%        
%        \begin{figure}
%        \image{T102_Bridge_A}
%        \end{figure}
%        
%        Wie lautet die Parabelgleichung, die den Verlauf des Stahlseils beschreibt?
        Geben Sie eine Funktionsbeschreibung für die Parabel an, die den Verlauf des Stahlseils beschreibt.

    \item Über einen 30 m breiten Fluss, bei dem das rechte Ufer 15 m tiefer liegt als das linke,
        soll eine Brücke mit parabelförmiger Unterseite gebaut werden. Der Scheitelpunkt der 
        Parabel soll dabei 10 m vom Ufer entfernt sein. 
%        Dies lässt sich graphisch im zweidimensionalen 
%        Koordinatenraum wie folgt darstellen:
%
%
%        \begin{figure}
%        \image{T102_Bridge_B}
%        \end{figure}
%
        Wie kann man den Brückenbogen funktional beschreiben?


\end{enumerate}
%
% Lösung
%
%        \begin{center}
%          \image[800]{haengebruecke}
%        \end{center}
%
\begin{tabs*}[\initialtab{0}\class{exercise}]
  \tab{\lang{de}{Antworten}}	
    \begin{enumerate}[alph]
      \item $f(x)=0,02\cdot x^2 \quad$ 
      
      \item Abhängig von der Lage der Parabel im Koordinatensystem:
      \[    
        \begin{mtable} %[\class{items}]
                  & f(x)&=&-\frac{1}{20} x^2 +x   \\
          \text{oder} \quad & f(x)&=&-\frac{1}{20} x^2 +5 \\
          \text{oder} \quad & f(x)&=&-\frac{1}{20} x^2 -x  \\ 
          \text{oder} \quad & f(x)&=&-\frac{1}{20} x^2 -2x .
        \end{mtable}
      \]

    \end{enumerate}
  
  
  
  \tab{\lang{de}{Lösung a)}}	
  	\begin{incremental}[\initialsteps{1}]    
     \step 
      Wir stellen die in der Aufgabenstellung beschriebene Situation zunächst graphisch
      in einem zweidimensionalen Koordinatensystem dar und legen dabei der Einfachheit 
      halber den Scheitelpunkt der Parabel in den Koordinatenursprung. 
      
       \begin{figure}
        \image{T102_Bridge_A}
       \end{figure}
        
     \step 
%     Bei der graphischen Darstellung der in der Aufgabenstellung beschriebenen
%     Hängebrücke im zweidimensionalen Koordinatensystem haben wir den
%     Scheitelpunkt der Parabel der Einfachheit in den Koordinatenursprung $\,(0;0)\,$ gelegt.
       Da der Scheitelpunkt der Parabel in $(0;0)$ liegt, ist die gesuchte Parabel von 
       der Form $\; f(x)=ax^2.\,$ Wir müssen daher nur noch den Streckfaktor $a$ bestimmen. 
       
       Dieser lässt sich aus der Höhe des Pfeilers (8 m) und der 
       dazugehörigen Spannweite des Stahlseils (20 m) berechnen, denn dies ist genau der Punkt,
       wo das Seil am Pfeiler befestigt ist. Der Punkt $\,(20;8)\,$ liegt also auf der Parabel
       und daher gilt: 
       \[ f(20)=a \cdot 20^2=8 \quad \Rightarrow \quad a=\frac{8}{400}=0,02.\]    
     \step 
     Die gesuchte Parabel wird beschrieben durch    $\quad f(x)=0,02\cdot x^2.$
     \\
 
     Bemerkung: Die funktionale Beschreibung der Parabel ist nicht eindeutig. Sie hängt von der Lage
     der Parabel im Koordinatensystem ab. Eindeutig ist nur der Streckfaktor $a=0,02$, da dieser den
     Verlauf des Stahlseils bestimmt.
  \end{incremental}
  
  \tab{\lang{de}{Lösungsvideo b)}}
  	\begin{incremental}[\initialsteps{1}]    
     \step 
      Wir skizzieren zunächst die in der Aufgabenstellung beschriebene Situation 
      im zweidimensionalen Raum. 
        
        \begin{figure}
        \image{T102_Bridge_B}
        \end{figure}

     \step
      In unserer Skizze liegt die parabelförmigen Brücke noch frei im zweidimensionalen 
      Raum. Zur Bestimmung einer Funktionsgleichung für die Parabel benötig man nun 
      \ref[content_08_quadratische_funktionen][mindestens drei Punkte,]{{sec:3_P_Parabel}} 
      die auf der Parabel liegen. 
      
     \step Diese Punkte $P_1$, $P_2$ und $P_3$ können aus den bekannten Größen berechnet werden, vorausgesetzt, 
      die Parabel wird in einem zweidimensionalen Koordinatensystem fest "`verankert"'. Schauen Sie hierzu
      das folgende Video.
      
       \begin{center} 
        \youtubevideo[500][300]{8hQPYlo7o00}\\
       \end{center}    
       

     \step     
     Abhängig von der Lage der Parabel im Koordinatensystem ergeben sich also verschiedene
     Parabelgleichungen, die alle denselben Streckfaktor $a=-\frac{1}{20}$
     haben, da sie denselben Bogen beschreiben. Dies sind z.B.
\[    
  \begin{mtable} %[\class{items}]
            & f(x)&=&-\frac{1}{20} x^2 +x   \\
    \text{oder} \quad & f(x)&=&-\frac{1}{20} x^2 +5 \\
    \text{oder} \quad & f(x)&=&-\frac{1}{20} x^2 -x  \\ 
    \text{oder} \quad & f(x)&=&-\frac{1}{20} x^2 -2x .
  \end{mtable}
\]
            
  \end{incremental} 
  \end{tabs*}
\end{content}