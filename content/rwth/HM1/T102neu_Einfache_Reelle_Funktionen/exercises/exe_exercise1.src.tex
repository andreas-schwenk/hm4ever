\documentclass{mumie.element.exercise}
%$Id$
\begin{metainfo}
  \name{
    \lang{de}{Ü01: Funktionsbeschreibung}
  }
  \begin{description} 
 This work is licensed under the Creative Commons License Attribution 4.0 International (CC-BY 4.0)   
 https://creativecommons.org/licenses/by/4.0/legalcode 

    \lang{de}{Eigenschaften von Funktionen}
  \end{description}
  \begin{components}
  \end{components}
  \begin{links}
    \link{generic_article}{content/rwth/HM1/T101neu_Elementare_Rechengrundlagen/g_art_content_05_loesen_gleichungen_und_lgs.meta.xml}{content_05_loesen_gleichungen_und_lgs}
    \link{generic_article}{content/rwth/HM1/T102neu_Einfache_Reelle_Funktionen/g_art_content_06_funktionsbegriff_und_lineare_funktionen.meta.xml}{content_06_funktionsbegriff_und_lineare_funktionen}
  \end{links}
  \creategeneric
\end{metainfo}
%
\begin{content}
\begin{block}[annotation]
	Im Ticket-System: \href{https://team.mumie.net/issues/21990}{Ticket 21990}
\end{block}
%
 \begin{block}[annotation]
%
   Neue Übung zum Funktionsbegriff reeller Funktionen mit
   Festlegung des Definitionsbereichs und Bestimmung der Nullstellen und wo möglich auch der Wertemenge
%
\end{block}
%

  \title{
    \lang{de}{Ü01: Funktionsbeschreibung}
  }
 
% Aufgabenstellung
%

  Handelt es sich bei den folgenden Zuordnungen für reelle Zahlen $\,x\,$ um Funktionen? \\
  Wenn ja, bestimmen sie jeweils den Definitionsbereich, die Nullstellen und die Wertemenge der Funktion.
  
    \begin{enumerate}[alph]   
      \item Der Zahl $\,x\,$ wird $\,4x-3\,$ zugeordnet.
      \item Der Zahl $\,x\,$ wird $\,-x\,$ und $\,x\,$ zugeordnet.
      \item Der Zahl $\,x\,$ wird der Term $\,\frac{x^2}{x^2+2}\,$ zugeordnet.
      \item Der Zahl $\,x\,$ wird der Term $\,\frac{x^2+2x+1}{x-5}\,$ zugeordnet.         
    \end{enumerate}
%
% Lösungen
%
  \begin{tabs*}[\initialtab{0}\class{exercise}]
  
    \tab{Antworten}
     \begin{enumerate}[alph]
      \item Ja $\quad$ mit $\,D_f=\R, \;$ $\,W_f=\R \;$
                       und Nullstelle in $\,x=\frac{3}{4}.$
      \item Nein
      \item Ja $\quad$ mit $\,D_f=\R, \;$ $\,W_f=[0;1) \;$ 
                       und Nullstelle in $\,x=0.$
      \item Ja $\quad$ mit $\,D_f=\R\setminus \{5\} $, $\,W_f=(-\infty;0] \cup [24;\infty) \;$  
                       und Nullstelle in $\,x=-1.$       
     \end{enumerate}
%
    \tab{Lösung a)} 
     \begin{incremental}[\initialsteps{1}]
      \step Jeder reellen Zahl $\,x\,$ wird mit $\,f(x)=4x-3\,$ auf eindeutige Weise ein fester 
            reeller Wert zugeordnet, denn man kann $\,4x-3\,$ ausrechnen. Daher ist durch diese 
            Zuordnung eine Funktion gegeben.      
      \step Da der Term $\,4x-3\,$ für jede reellen Zahl $\,x\,$ ohne Einschränkung wieder eine 
            reelle Zahl ergibt, und umgekehrt auch jede reelle Zahl über diese Vorschrift erreicht wird,
            ist sowohl der Definitionsbereich $\,D_f=\R,\,$ als auch die Wertemenge $\,W_f=\R.$
      \step Die Nullstellen der Funktion $\,f(x)=4x-3\,$ sind die Stellen $\,x,\,$ die die lineare
            Gleichung $\,4x-3=0\,$ erfüllen, also $\,x=\frac{3}{4}.$
      
      \end{incremental} 
 %     
    \tab{Lösung b) } 
    
        Diese Zuordnung ist nicht eindeutig, denn sie ordnet jeder rellen Zahl $\,x\,$ außer 
        der $\,0\,$ zwei verschiedene Werte zu. Daher handelt es sich gemäß der 
        \ref[content_06_funktionsbegriff_und_lineare_funktionen][Definition]{def:reelle_funktion}
        nicht um eine Funktion. 
%
    \tab{Lösung c)} 
     \begin{incremental}[\initialsteps{1}]
      \step Einer reellen Zahl $\,x\,$ wird mit $\,f(x)=\frac{x^2}{x^2+2}\,$ auf eindeutige Weise 
            ein fester reeller Wert zugeordnet, sofern der Nenner für $x$ ungleich Null ist. Daher 
            ist durch diese Zuordnung eine Funktion gegeben. 
            
      \step Bestimmung des Definitionsbereichs von $f:$\\
      
            Die Einschränkung, dass der Nenner nicht Null sein darf, wird über die Festlegung
            des Definitionsbereichs für $f$ geregelt. Die Nullstellen des Nenner-Terms 
            $\,x^2+2\,$ sind daher aus dem Definitionsbereich auszuschließen.\\
            
            Da $\,x^2 \,$ für alle reellen Zahlen $\,x\,$ immer größer oder gleich $0$ ist, 
            gilt für den Nenner $\,x^2 +2 \geq 2 >0.\,$
            Der Nenner besitzt also keine Nullstellen, folglich darf jede reelle Zahl $\,x\,$ 
            in $\,f\,$ eingesetzt werden und es gilt $\,D_f=\R.\,$
            
      \step Die Nullstellen der Funktion $\,f(x)=\frac{x^2}{x^2+2} \,$ sind die Stellen $\,x,\,$
            für die der Zähler des Bruchterms Null und der Nenner ungleich Null ist. \\
            
            Wir lösen also die quadratische Gleichung $\,x^2=0\,$ und erhalten die eindeutige Lösung
            $\,x=0.\,$ Eingesetzt im Nenner gilt $\,0^2+2=2 \neq 0.\,$ 
            $f$ besitzt also genau eine Nullstelle in $\,x=0.\,$
            
      \step Zur Bestimmung der Wertemenge von $f$ setzen wir $y := f(x)=\frac{x^2}{x^2+2}. \,$ 
      
            Da für alle $x \in \R \quad x^2 \geq 0\,$ ist, gilt auch stets $\, y \geq 0$.\\
            Da darüber hinaus $0 \leq x^2 \leq x^2+2$ gilt, ist auch $y<1$.
            
            Um zu sehen, dass der Wertebereich dann wirklich $[0;1)$ ist, 
            lösen wir die Gleichung $\, y=f(x)=\frac{x^2}{x^2+2}\,$ nach $x$ auf.

            \begin{align*}
            &&\quad	                y    \,&=\,& \frac{x^2}{x^2+2} \quad &\vert \cdot (x^2+2) \\
            &\Leftrightarrow &\quad   y \cdot x^2 + 2y \; &=\,& x^2      &\vert -x^2 -2y \\
            &\Leftrightarrow &\quad  x^2 \cdot  (y-1) \; &=\,& -2y       &\vert :(y-1)\\
            &\Leftrightarrow &\quad  x^2    \,&=\,& \frac{-2y}{y-1}      &
            \end{align*}
            
            Für $\, y\in [0;1)\,$ ist die rechte Seite $\, \frac{-2y}{y-1} \geq 0 \,$ und wir können die Wurzel ziehen. 
            Es gibt also für jedes $y\in [0;1)$ ein $x$ mit $f(x)=y \;$ (meistens sogar zwei $x$). Folglich           
            ist die Wertemenge für $f$ \[ W_f= [0;1).\]
            
      
      \end{incremental} 
    
%
    \tab{Lösung d)} 
     \begin{incremental}[\initialsteps{1}]
       \step Einer reellen Zahl $\,x\,$ wird mit $\,f(x)=\frac{x^2+2x+1}{x-5}\,$ auf eindeutige Weise 
            ein fester reeller Wert zugeordnet, sofern der Nenner für $x$ ungleich Null ist. Daher 
            ist durch diese Zuordnung eine Funktion gegeben.  
            
      \step Bestimmung des Definitionsbereichs von $f:$\\
      
            Die Einschränkung, dass der Nenner nicht Null sein darf, wird über die Festlegung
            des Definitionsbereichs für $f$ geregelt. Die Nullstellen des Nenner-Terms 
            $\,x-5\,$ sind daher aus dem Definitionsbereich auszuschließen.\\
            
            Wir lösen also die Gleichung $\,x-5=0\,\Leftrightarrow \, x=5.\,$ und schließen die
            berechnete Nenner-Nullstelle aus dem Definitionsbereich aus. 
            Damit ist $\,D_f=\R\setminus \{5\}.$
            
      \step Die Nullstellen der Funktion $\,f(x)=\frac{x^2+2x+1}{x-5}\,$ sind die Stellen $\,x,\,$
            für die der Zähler des Bruchterms Null und der Nenner ungleich Null ist. \\
            
            Wir lösen also die quadratische Gleichung $\,x^2+2x+1=0\,$ 
          \[
          \begin{mtable}[\cellaligns{crcll}]
                           &          x^2+2x+1  &\,=\,& 0   &\vert \text{1. bin. Formel} \\
           \Leftrightarrow &\qquad   (x+1)^2    &\,=\,& 0   &  \\
           \Leftrightarrow &\qquad    x+1       &\,=\,& 0   & \vert -1 \\
           \Leftrightarrow &\qquad    x         &\,=\,& -1   &
          \end{mtable}
          \]
            und erhalten die eindeutige Lösung $\,x=-1.\,$ Eingesetzt im Nenner gilt 
            $\,-1-5=-6 \neq 0.\,$ $f$ besitzt also genau eine Nullstelle in $\,x=-1.\,$

      \step Zur Bestimmung der Wertemenge von $f$ setzen wir $\; y := \frac{x^2+2x+1}{x-5} \;$ und
            prüfen, für welche $y$ die Gleichung eine reelle Lösung $x$ hat.
           \[
           \begin{mtable}[\cellaligns{cccll}]
            &\quad	                 y       \,&=\,& \frac{x^2+2x+1}{x-5} \quad &\vert \cdot (x-5) \\
            \Leftrightarrow &\quad  yx - 5y \; &=\,& x^2+2x+1                  &\\
            \Leftrightarrow &\quad  0       \; &=\,& x^2 + (2-y)x + (1+5y)     &\\
           \end{mtable}
           \]
           
           Diese Gleichung muss mindestens eine Lösung für $x$ in $\R$ haben, d.h. die Diskriminante,
           die sich durch Anwendung der \ref[content_05_loesen_gleichungen_und_lgs][pq-Formel]{rule:pqFormel} 
           aus dieser quadratischen Gleichung ergibt, muss größer oder gleich Null sein. Es muss also gelten:
           \[
           \begin{mtable}[\cellaligns{cccll}]
            &\quad	                 \frac{(2-y)^2}{2^2}-(1+5y) \,&\geq\,& 0 \quad &\\
            \Leftrightarrow &\quad  (2-y)^2 - 4(1+5y) \;       \,&\geq\,& 0        &\\
            \Leftrightarrow &\quad   y^2-4y+4 -4-20y \;    \,&\geq\,& 0            &\\ 
            \Leftrightarrow &\quad   y \cdot (y-24) \;           \,&\geq\,& 0      &\\            
            \Leftrightarrow &\quad   y \geq 0 \wedge  (y-24)\geq 0   \; &\vee \,& y \leq 0 \wedge  (y-24)\leq 0     &\\
            \Rightarrow     &\quad   y \geq 24   \; &\vee \,& y \leq 0     &\\            
           \end{mtable}
           \]
            Damit ist die Wertemenge für $f$ bestimmt: $\qquad W_f= (-\infty;0] \cup [24;\infty).$

      \end{incremental} 
        
  \end{tabs*} 

\end{content}

