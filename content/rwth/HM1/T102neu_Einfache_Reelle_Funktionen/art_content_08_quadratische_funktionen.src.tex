%$Id:  $
\documentclass{mumie.article}
%$Id$
\begin{metainfo}
  \name{
    \lang{de}{Quadratische Funktionen, Parabeln}
    \lang{en}{Quadratic functions and parabolas}
  }
  \begin{description} 
 This work is licensed under the Creative Commons License Attribution 4.0 International (CC-BY 4.0)   
 https://creativecommons.org/licenses/by/4.0/legalcode 

    \lang{de}{Beschreibung}
    \lang{en}{Description}
  \end{description}
  \begin{components}
    \component{generic_image}{content/rwth/HM1/images/g_img_00_video_button_schwarz-blau.meta.xml}{00_video_button_schwarz-blau}  
    \component{generic_image}{content/rwth/HM1/images/g_tkz_T102_ParabolaVertex.meta.xml}{T102_ParabolaVertex}
    \component{generic_image}{content/rwth/HM1/images/g_tkz_T102_ParabolaThreePoints.meta.xml}{T102_ParabolaThreePoints}
    \component{generic_image}{content/rwth/HM1/images/g_tkz_T102_Parabolas.meta.xml}{T102_Parabolas}
    \component{generic_image}{content/rwth/HM1/images/g_img_00_Videobutton_schwarz.meta.xml}{00_Videobutton_schwarz}         
    \component{js_lib}{system/media/mathlets/GWTGenericVisualization.meta.xml}{mathlet1}
  \end{components}
  \begin{links}
    \link{generic_article}{content/rwth/HM1/T102neu_Einfache_Reelle_Funktionen/g_art_content_08_quadratische_funktionen.meta.xml}{content_08_quadratische_funktionen}
    \link{generic_article}{content/rwth/HM1/T101neu_Elementare_Rechengrundlagen/g_art_content_02_rechengrundlagen_terme.meta.xml}{content_02_rechengrundlagen_terme}
    \link{generic_article}{content/rwth/HM1/T102neu_Einfache_Reelle_Funktionen/g_art_content_07_geradenformen.meta.xml}{content_07_geradenformen}
    \link{generic_article}{content/rwth/HM1/T101neu_Elementare_Rechengrundlagen/g_art_content_05_loesen_gleichungen_und_lgs.meta.xml}{link-04-lin-funk}
  \end{links}
  \creategeneric
\end{metainfo}

\begin{content}
\begin{block}[annotation]
	Im Ticket-System: \href{https://team.mumie.net/issues/21160}{Ticket 21160}
\end{block}

%
% ursprüngliche Version:
%
% Copy of \href{http://team.mumie.net/issues/8974}{Ticket 8974}: 
%          content/rwth/HM1/T102_Einfache_Funktionen,_grundlegende_Begriffe/art_content_06_quadratische_funktionen.src.tex
% 
% \begin{block}[annotation]
%   Im Ticket-System: \href{http://team.mumie.net/issues/8974}{Ticket 8974}\\
% \end{block}
%
% Ticket neu: 
%

\usepackage{mumie.ombplus}
\ombchapter{2}
\ombarticle{3}

\usepackage{mumie.genericvisualization}

\begin{visualizationwrapper}

\title{\lang{de}{Quadratische Funktionen und Parabeln}
       \lang{en}{Quadratic functions and parabolas}}

\begin{block}[annotation]
 quadratische Funktionen, Parameter einer Parabel und deren Bedeutung, Scheitelpunktform, 
 Festlegung einer quadratischen Funktion durch drei Punkte\\

\end{block} 


\begin{block}[info-box]
\tableofcontents
\end{block}

\section{Quadratische Funktionen}  \label{sec:quadratic}
%
%%% Video Hoever
%
\lang{de}{
Wir beginnen mit einer anschaulichen Einführung in quadratische Funktionen.
\floatright{\href{https://www.hm-kompakt.de/video?watch=105}{\image[75]{00_Videobutton_schwarz}}}
\\\\
Fassen wir nun die wesentlichen Definitionen und Merkmale zusammen.
}
\lang{en}{
We begin by giving the essential definitions for this section.
}
% 

\begin{definition}[\lang{de}{Quadratische Funktion}\lang{en}{Quadratic Function}] \label{def:quadratic_func}
\lang{de}{
Eine Funktion $f:\R\to \R\,$ der Form 
}
\lang{en}{
A function $f:\R\to \R\,$ of the form 
}
\begin{align*}
f(x) = a x^2 + b \, x + c
\end{align*}
\lang{de}{
mit Koeffizienten $a,\, b,\,c \in \mathbb{R}$, wobei $a \neq 0$ ist, heißt 
\notion{\emph{quadratische Funktion}} oder \notion{\emph{Parabelfunktion}}.
\\ 
Den Koeffizienten $a$ bezeichnet man als \notion{\emph{Leitkoeffizienten}.}
\\
Der Definitionsbereich ist $D_f = \mathbb{R}$.
}
\lang{en}{
with coefficients $a,\, b,\,c \in \mathbb{R}$, where $a \neq 0$, is called a 
\notion{\emph{quadratic function}} or sometimes just a \notion{\emph{quadratic}}.
\\
The coefficient $a$ is called the \notion{\emph{leading coefficient}} of the quadratic.
\\
If $a = 1$, the quadratic function is called \notion{\emph{monic}}.
\\
The domain is $D_f = \mathbb{R}$.
}
\end{definition}

\begin{definition}[\lang{de}{Parabel und Normalparabel}\lang{en}{Parabola}] \label{def:parabel}
\lang{de}{
Der Graph einer quadratischen Funktion $f:\R\to \R, \; x \mapsto a x^2 + b \, x + c$, \\
also die Menge $\{ (x;y) \in \R^2 \mid y=a x^2 + b \, x + c \},$ beschreibt eine 
\notion{\emph{Parabel}.}
\\
Ist der Leitkoeffizient $a = 1$, so bezeichnet man den Graphen von $f$ als 
\notion{\emph{Normalparabel}.}
}
\lang{en}{
The graph of a quadratic function $f:\R\to \R, \; x \mapsto a x^2 + b \, x + c$, \\
so the set $\{ (x;y) \in \R^2 \mid y=a x^2 + b \, x + c \},$ describes a curve called a 
\notion{\emph{parabola}}.
}
\end{definition}

%%%%%%%%%%%%%%%%%%%%%%%%%%%%%%%%%%%%%%%%%%%%%%%%%%%%%%%%%%%%%%%%%%%%%%%%%%%%%%%%%%%%%%%%%%%%%%%%%%%%%%
%
% 
% \begin{center}
%   \lang{de}{\iframe[400][225][S]{https://www.stream24.net/vod/getVideo.php?id=10962-1-5531&mode=iframe}}
% \end{center}
% 
% \lang{de}{Man bezeichnet eine ganzrationale Gleichung als quadratisch, 
% wenn die Unbekannte mit dem Exponenten 2 auftritt (z.B. als $x^2$), 
% und auch nicht durch Umformungen wegfällt. 
% Außerdem tritt sie nicht mit irgendwelchen Wurzeln oder sonstigen Funktionen auf.}
%
%
%%%%%%%%%%%%%%%%%%%%%%%%%%%%%%%%%%%%%%%%%%%%%%%%%%%%%%%%%%%%%%%%%%%%%%%%%%%%%%%%%%%%%%%%%%%%%%%%%%%%%%

\begin{remark}[\lang{de}{Bedeutung der Parameter}
               \lang{en}{Interpreting the parameters}] \label{rem:parabel_parameter}

  \lang{de}{
  In der Darstellung $\,f(x)=ax^2+bx+c\,$ bestimmt der Leitkoeffizient $a$ die Form der Parabel:
  }
  \lang{en}{
  In the form $\,f(x)=ax^2+bx+c\,$, the leading coefficient $a$ determines the shape of the parabola: 
  }
    \begin{itemize}
    \item   \lang{de}{
            F"ur $a>0$ ist die Parabel nach \emph{oben} geöffnet und\\ 
            f"ur $a<0$ ist die Parabel nach \emph{unten} geöffnet.
            }
            \lang{en}{
            For $a>0$ the parabola opens to the \emph{top}, and \\ 
            for $a<0$ the parabola opens to the \emph{bottom}.
            }
    \item   \lang{de}{
            Man bezeichnet $a$ auch als \notion{\emph{Streckfaktor}} der Parabel, denn\\
            f"ur $|a|>1$ ist die Parabel \emph{steiler} bzw. \emph{spitzer} als die Normalparabel und\\ 
            f"ur $|a|<1$ ist die Parabel \emph{flacher} bzw. \emph{stumpfer} als die Normalparabel.
            }
            \lang{en}{
            We sometimes call $a$ the \notion{\emph{stretch factor}} of the parabola, as \\
            for $|a|>1$ the parabola is \emph{steeper} or \emph{'sharper'} than for a monic 
            quadratic.\\
            For $|a|<1$ the parabola is \emph{more flat} or \emph{'blunter'} than for a monic 
            quadratic.
            }
    \end{itemize}
  \lang{de}{
  Der Koeffizient $\,c\,$ kennzeichnet den Schnittpunkt mit der $y-$Achse, also $\,(0;c)$, denn $\,f(0)=c.$ 
  \\
  Die Bedeutung des Koeffizienten $b$ ist nicht so offensichtlich, wird aber später erläutert.
  }
  \lang{en}{
  The coefficient $\,c\,$ once more denotes the $y$-intercept, that is, $\,(0;c)$ such that 
  $\,f(0)=c.$ 
  }
 
\end{remark}

\lang{de}{
Die folgenden Beispiele veranschaulichen, wie sich Veränderungen der Koeffizienten einer  
quadratischen Funktion $\, f:\R\to \R, \; x \mapsto ax^2+bx+c\,$ auf ihre graphische 
Darstellung als Parabel auswirken.
}
\lang{en}{
The following examples demonstrate how changing the coefficients of a quadratic function 
$\, f:\R\to \R, \; x \mapsto ax^2+bx+c\,$ affects the parabola when the function is graphed.
}

\begin{example}
%

\lang{de}{
Wir beobachten zunächst, wie sich die Parabel allein durch Variation des \notion{Streckfaktors} 
$\,a\,$ verändert. Hierzu wählen wir eine Parabel der Form  $\textcolor{\#0066CC}{f(x)}= a \cdot x^2$,
in der also die Koeffizienten $c$ und $b$ gleich Null sind, und vergleichen diese jeweils mit der 
Normalparabel $\textcolor{\#CC6600}{f_0 (x)}=x^2$.
}
\lang{en}{
We firstly observe how changing the \notion{stretch factor} $\,a\,$ modifies the parabola. To 
demonstrate this, we consider a parabola of the form $\textcolor{\#0066CC}{f(x)}= a \cdot x^2$, 
which has the coefficients $c$ and $b$ both equal to zero, and compare it to the parabola given 
by the monic quadratic function $\textcolor{\#CC6600}{f_0 (x)}=x^2$.
}
%
\lang{de}{
 \begin{genericGWTVisualization}[480][600]{mathlet1}
  \begin{variables}
      % editierbare Variablen (Koeffizienten für die Parabelfunktion)
      \number[editable]{a}{real}{-1}
      %\number[editable]{b}{real}{0}
      %\number[editable]{c}{real}{0}
      % Definition der Parabelfunktionen
      \function{P}{rational}{var(a)*x^2}  % Parabel zu obigen Koeffizienten
      \function{NP}{rational}{x^2}        % Normalparabel
  \end{variables}
    \label{P}{$f$}
    \label{NP}{$f_0$}   
	% COLOR:
	\color{P}{#0066CC}
	\color{NP}{#CC6600}
	\text{Variieren Sie den Leitkoeffizienten der Parabel 
    $\textcolor{#0066CC}{f(x)}=\var{a} \cdot x^2$ mit beliebigen reellen Werten 
    und beobachten Sie dabei die Veränderung der Parabel.}
	\begin{canvas}
		\plotSize{400}
		\plotLeft{-6}
		\plotRight{6}
		\plot[coordinateSystem]{NP,P}
%        \slider{sla}
	\end{canvas}
 \end{genericGWTVisualization}
}
\lang{en}{
 \begin{genericGWTVisualization}[480][600]{mathlet1}
  \begin{variables}
      % editierbare Variablen (Koeffizienten für die Parabelfunktion)
      \number[editable]{a}{real}{-1}
      %\number[editable]{b}{real}{0}
      %\number[editable]{c}{real}{0}
      % Definition der Parabelfunktionen
      \function{P}{rational}{var(a)*x^2}  % Parabel zu obigen Koeffizienten
      \function{NP}{rational}{x^2}        % Normalparabel
  \end{variables}
    \label{P}{$f$}
    \label{NP}{$f_0$}
	% COLOR:
	\color{P}{#0066CC}
	\color{NP}{#CC6600}
	\text{Vary the coefficients of the parabola 
    $\textcolor{#0066CC}{f(x)}=\var{a} \cdot x^2$ in the real numbers and observe the changes in 
    the graph.}
	\begin{canvas}
		\plotSize{400}
		\plotLeft{-6}
		\plotRight{6}
		\plot[coordinateSystem]{NP,P}
%        \slider{sla}
	\end{canvas}
 \end{genericGWTVisualization}
}
\end{example}

\begin{example}
%
\lang{de}{
Wir betrachten nun eine Parabel in der allgemeinen Form $\textcolor{#0066CC}{f(x)}= a x^2 + b x + c\,$ 
und betrachten die Veränderungen, die durch Variation der Koeffizienten $a, b\,$ und $\,c\,$ entstehen.
}
\lang{en}{
We consider a general parabola of the form $\textcolor{#0066CC}{f(x)}= a x^2 + b x + c\,$ 
and observe the changes in the parabola caused by variation of the coefficients $a, b\,$ and $\,c\,$.
}
%
\lang{de}{
 \begin{genericGWTVisualization}[480][600]{mathlet1}
  \begin{variables}
      % editierbare Variablen (Koeffizienten für die Parabelfunktion)
      \number[editable]{a}{real}{0.5}
      \number[editable]{b}{real}{-2}
      \number[editable]{c}{real}{2}
      % Definition der Parabelfunktionen
      \function{P}{rational}{var(a)*x^2+var(b)*x+var(c)}  % Parabel zu obigen Koeffizienten
  \end{variables}
    \label{P}{$f$}
	% COLOR:
	\color{P}{#0066CC}
	\text{Variieren Sie nun alle drei Koeffizienten der Parabel 
    $\textcolor{#0066CC}{f(x)}=\var{a} \cdot x^2 + \var{b} \cdot x + \var{c}$ mit beliebigen reellen Werten 
    und beobachten Sie erneut die Veränderung der Parabel. Beachten Sie insbesondere, dass der dritte 
    Koeffizient, also der Koeffizient ohne $\,x\,$, stets den Schnittpunkt mit der $y-$Achse kennzeichnet.}
	\begin{canvas}
		\plotSize{400}
		\plotLeft{-6}
		\plotRight{6}
		\plot[coordinateSystem]{NP,P}
%        \slider{sla}
	\end{canvas}
 \end{genericGWTVisualization}
}
\lang{en}{
 \begin{genericGWTVisualization}[480][600]{mathlet1}
  \begin{variables}
      % editierbare Variablen (Koeffizienten für die Parabelfunktion)
      \number[editable]{a}{real}{0.5}
      \number[editable]{b}{real}{-2}
      \number[editable]{c}{real}{2}
      % Definition der Parabelfunktionen
      \function{P}{rational}{var(a)*x^2+var(b)*x+var(c)}  % Parabel zu obigen Koeffizienten
  \end{variables}
    \label{P}{$f$}
	% COLOR:
	\color{P}{#0066CC}
	\text{Vary the coefficients of the parabola 
    $\textcolor{#0066CC}{f(x)}=\var{a} \cdot x^2 + \var{b} \cdot x + \var{c}$ in the real numbers 
    and observe the changes in the graph. Pay particular attention to the third coefficient, the 
    constant without $\,x\,$, and the intersection of the parabola with the $y$-axis.}
	\begin{canvas}
		\plotSize{400}
		\plotLeft{-6}
		\plotRight{6}
		\plot[coordinateSystem]{NP,P}
%        \slider{sla}
	\end{canvas}
 \end{genericGWTVisualization}
}
\end{example}


%%%%%%%%%%%%%%%%%%%%%%%%%%%%%%%%%%%%%%%%%%%%%%%%%%%%%%%%%%%%%%%%%%%%%%%%%%%%%%%%%%%%%%%%%%%%

\section{\lang{de}{Nullstellenbestimmung}\lang{en}{Finding roots}} \label{sec:nullstellen}

\lang{de}{
Um die Nullstellen einer quadratischen Funktion $\,f(x)=ax^2+bx+c\;$ (mit $a\ne 0$) zu bestimmen,
ist die \emph{quadratische Gleichung} $\, ax^2+bx+c=0 \,$ zu lösen. Hierzu liefert uns 
zum Beispiel die \ref[link-04-lin-funk][Mitternachtsformel]{rem:mitternachtsformel} direkt
die Lösungen.
}
\lang{en}{
To find the roots of a quadratic function $\,f(x)=ax^2+bx+c\;$ (with $a\ne 0$), we solve the 
\emph{quadratic equation} $\, ax^2+bx+c=0 \,$. Using a technique such as the 
\ref[link-04-lin-funk][quadratic formula]{rem:mitternachtsformel} immediately gives us the 
solutions.
}

\begin{rule}[\lang{de}{Nullstellen quadratischer Funktionen}
             \lang{en}{Roots of a quadratic function}]  \label{rule:quadratic_nst}

 \lang{de}{
 Eine quadratischen Funktion $f:\R\to \R$, definiert in allgemeiner Form durch 
 }
 \lang{en}{
 A quadratic function $f:\R\to \R$, defined in general in the form 
 }
 \[f(x)= a x^2 + b \, x + c, \quad \text{\lang{de}{mit}\lang{en}{ with }} 
 \,a, b,c \in \mathbb{R}\, \text{\lang{de}{und}\lang{en}{ and }} \,a \neq 0,\]
 \lang{de}{besitzt}
 \lang{en}{has}
 
 \begin{enumerate}
    \item \lang{de}{\emph{keine} reelle Nullstelle, falls $\,b^2-4ac<0,$}
          \lang{en}{\emph{no} real roots if $\,b^2-4ac<0,$}
    
    \item \lang{de}{
          \emph{eine} reelle Nullstelle, falls $\,b^2-4ac=0$, nämlich an $\, x = -\frac{b}{2a},$
          }
          \lang{en}{
          \emph{one} real root if $\,b^2-4ac=0$, namely $\, x = -\frac{b}{2a},$
          }
    \item \lang{de}{
          \emph{zwei} reelle Nullstellen,  falls $\,b^2-4ac>0.\;$  Diese liegen an \\
          $\,x_1=\frac{-b - \sqrt{b^2-4ac}}{2a}\,$ und $\, x_2=\frac{-b + \sqrt{b^2-4ac}}{2a}$.
          }
          \lang{en}{
          \emph{two} real roots if $\,b^2-4ac>0.\;$ These lie at \\
          $\,x_1=\frac{-b - \sqrt{b^2-4ac}}{2a}\,$ and $\, x_2=\frac{-b + \sqrt{b^2-4ac}}{2a}$.
          }
 \end{enumerate}
\end{rule}

\begin{remark} \label{rem:quadratic_nst}
\lang{de}{
Handelt es sich bei der quadratischen Funktion speziell um eine \emph{Normalparabel}, d.h. 
ihr Leitkoeffizient ist $1$, dann ist zur Bestimmung der Nullstellen eine
\emph{quadratische Gleichung in Normalform} zu lösen, also 
}
\lang{en}{
If we are dealing with a monic quadratic function, whose leading coefficient is $1$, then to find 
its roots we solve a monic quadratic equation 
}
\[f(x)= x^2 + p \, x + q=0, \quad \text{mit} \,p, q \in \mathbb{R}.\]
\lang{de}{
Hierzu verwendet man die \ref[link-04-lin-funk][p-q-Formel]{rule:pqFormel} 
und erhält als Ergebnis, dass die Parabel
}
\lang{en}{
This is suited to the \ref[link-04-lin-funk][p-q-formula]{rule:pqFormel} which shows that the 
parabola has
}
 \begin{description} 
 This work is licensed under the Creative Commons License Attribution 4.0 International (CC-BY 4.0)   
 https://creativecommons.org/licenses/by/4.0/legalcode 

    \item \lang{de}{entweder \emph{keine} reelle Nullstelle besitzt, falls $\,\Big{(}\frac{p}{2}\Big{)}^2-q<0,$}
          \lang{en}{either \emph{no} real roots, if $\,\Big{(}\frac{p}{2}\Big{)}^2-q<0,$}
    \item \lang{de}{
          oder \emph{eine} reelle Nullstelle besitzt, falls $\,\Big{(}\frac{p}{2}\Big{)}^2-q=0$, 
          nämlich $\, x = -\frac{p}{2},$
          }
          \lang{en}{\emph{one} real root, if $\,\Big{(}\frac{p}{2}\Big{)}^2-q=0$, namely
          $\, x = -\frac{p}{2},$
          }
    \item \lang{de}{
          oder \emph{zwei} reelle Nullstellen besitzt, falls $\,\Big{(}\frac{p}{2}\Big{)}^2-q>0.\;$
          Diese liegen in
          }
          \lang{en}{
          or \emph{two} real roots, if $\,\Big{(}\frac{p}{2}\Big{)}^2-q>0.\;$ These lie at 
          }
   \[ x_1=-\frac{p}{2}-\sqrt{\Big{(}\frac{p}{2}\Big{)}^2-q}\quad 
      \text{\lang{de}{ und }\lang{en}{ and }} \quad
      x_2=-\frac{p}{2}+\sqrt{\Big{(}\frac{p}{2}\Big{)}^2-q}.\]
 \end{description}

\end{remark}

\begin{example}\label{quadratic.example.1}
\lang{de}{
Wir untersuchen die folgenden quadratischen Funktionen auf ihre Nullstellen.
}
\lang{en}{
We find the roots of the following quadratic functions.
}

\begin{tabs*}[\initialtab{1}\class{example}]

% f_1 hat keine Nst  
  \tab{\small $\displaystyle{\textcolor{\#0066CC}{f_1(x)}= \frac{1}{2}\, x^2 - 6\,x +19}$}
    \begin{description} 
 This work is licensed under the Creative Commons License Attribution 4.0 International (CC-BY 4.0)   
 https://creativecommons.org/licenses/by/4.0/legalcode 

      \item \lang{de}{
            Der \emph{Leitkoeffizient} $\,$ von $\, f_1\,$ ist % $\,\frac{1}{2} \; (\neq 1)$, 
            $\, \neq 1$, also verwenden wir die \emph{Mitternachtsformel} und
            betrachten zunächst die Diskriminante $(b^2-4ac)$. Mit den 
            Koeffizienten von $\, f_1\,$ gilt   
            $\quad 6^2 - 4\cdot 0,5 \cdot 19=-2\,<0.$
            }
            \lang{en}{
            The \emph{leading coefficient} $\,$ of $\, f_1\,$ is $\, \neq 1$, so we use the 
            \emph{quadratic formula} and consider the discriminant $(b^2-4ac)$. With the coefficients 
            of $\, f_1\,$ this is $\quad 6^2 - 4\cdot 0,5 \cdot 19=-2\,<0.$
            }
      \item \lang{de}{
            Nach Regel \ref{rule:quadratic_nst} folgt hieraus, 
            dass $f_1$ \emph{keine} Nullstelle hat. 
            }
            \lang{en}{
            By rule \ref{rule:quadratic_nst}, $f_1$ has \emph{no} roots.
            }
    \end{description}
% f_2 hat 1 Nst    
  \tab{\small $\displaystyle{\textcolor{\#CC6600}{f_2(x)}= -x^2 - 4x - 4}$}     
    \begin{description} 
 This work is licensed under the Creative Commons License Attribution 4.0 International (CC-BY 4.0)   
 https://creativecommons.org/licenses/by/4.0/legalcode 

      \item \lang{de}{
            Der \emph{Leitkoeffizient} $\,$ von $\, f_2\,$ ist % $\,\frac{1}{2} \; (\neq 1)$, 
            $\, \neq 1$, also verwenden wir die \emph{Mitternachtsformel} und
            betrachten zunächst die Diskriminante $(b^2-4ac)$. Mit den 
            Koeffizienten von $\, f_2\,$ gilt 
            $\quad (-4)^2 - 4\cdot (-1) \cdot (-4)\,=0$
            }
            \lang{en}{
            The \emph{leading coefficient} $\,$ of $\, f_2\,$ is $\, \neq 1$, so we use the 
            \emph{quadratic formula} and consider the discriminant $(b^2-4ac)$. With the coefficients 
            of $\, f_2\,$ this is $\quad (-4)^2 - 4\cdot (-1) \cdot (-4)\,=0$.
            }
      \item \lang{de}{
            Nach Regel \ref{rule:quadratic_nst} folgt hieraus, dass $f_2$ 
            \emph{genau eine} Nullstelle hat und zwar in $\;x=-\frac{(-4)}{2 \cdot(-1)}=-2.$
            }
            \lang{en}{
            By rule \ref{rule:quadratic_nst}, $f_2$ has \emph{exactly one} root, namely 
            $\;x=-\frac{(-4)}{2 \cdot(-1)}=-2$.
            }
     \end{description}

%% f_3 hat 2 Nst
  \tab{\small $\displaystyle{\textcolor{\#00CC00}{f_3(x)}=x^2 - x - 6}$}
    \begin{description} 
 This work is licensed under the Creative Commons License Attribution 4.0 International (CC-BY 4.0)   
 https://creativecommons.org/licenses/by/4.0/legalcode 

      \item \lang{de}{
            Der \emph{Leitkoeffizient} $\,$ von $\, f_3$ ist $\,1$, wir
            können hier also die \emph{p-q-Formel} gemäß Bemerkung \ref{rem:quadratic_nst}
            anwenden und erhalten
            $\quad \big( \frac{(-1)}{2} \big)^2 - (-6)= \frac{25}{4}\,>0$
            }
            \lang{en}{
            The \emph{leading coefficient} $\,$ of $\, f_3$ is $\,1$, so we can use the 
            \emph{p-q-formula} and remark \ref{rem:quadratic_nst} as 
            $\quad \big( \frac{(-1)}{2} \big)^2 - (-6)= \frac{25}{4}\,>0$.
            }
      \item \lang{de}{
            Nach Regel \ref{rem:quadratic_nst} folgt hieraus, dass $f_3$ 
            \emph{zwei} Nullstellen hat und zwar in 
            $\;x_1=-\frac{(-1)}{2}-\sqrt{\frac{25}{4}}=-2 \;$ und in
            $\;x_2=-\frac{(-1)}{2}+\sqrt{\frac{25}{4}}=3.$
            }
            \lang{en}{
            By remark \ref{rem:quadratic_nst}, $f_3$ has \emph{two} roots, namely 
            $\;x_1=-\frac{(-1)}{2}-\sqrt{\frac{25}{4}}=-2 \;$ and 
            $\;x_2=-\frac{(-1)}{2}+\sqrt{\frac{25}{4}}=3$.
            }
    \end{description}
% \tab{\small \textbf{Videos zur Nullstellenbestimmung}}
%    \begin{description} 
 This work is licensed under the Creative Commons License Attribution 4.0 International (CC-BY 4.0)   
 https://creativecommons.org/licenses/by/4.0/legalcode 

%      \item Nullstellenbestimmung mittels \emph{p-q-Formel}:\\\href{https://www.hm-kompakt.de/video?watch=109}{\image[75]{00_Videobutton_schwarz}}
%      \item Nullstellenbestimmung mittels \emph{Mitternachtsformel}:\\\href{https://www.hm-kompakt.de/video?watch=110}{\image[75]{00_Videobutton_schwarz}}   
%    \end{description}
\end{tabs*}
  
  
    \begin{center}
        \image{T102_Parabolas}
    \end{center}

\end{example}

%
%%% Video K.M. *** NEU: 11274 ((Un-)Gleichungen_3_b) ***
%
\lang{de}{
Durch die Bestimmung der Nullstellen einer Parabel erhalten wir eine grobe Vorstellung über ihre Lage im Koordinatensystem. 
\\
\floatright{\href{https://api.stream24.net/vod/getVideo.php?id=10962-2-11274&mode=iframe&speed=true}
{\image[75]{00_video_button_schwarz-blau}}}\\\\
}
\lang{en}{
\\
}
\\
% Weitere Hinweise zur Lage einer Parabel liefern uns gemäß Bemerkung \ref{rem:parabel_parameter} die Parameter der Parabel 
% in der Darstellung $\,f(x)=ax^2+bx+c\,$, insbesondere ihr Leitkoeffizient $a$. 
%
\begin{remark}\label{rem:factorised_form}
\lang{de}{
Die quadratische Funktion $\,f(x)=ax^2+bx+c\,$ lässt sich zudem mithilfe ihrer Nullstellen $x_1$ und $x_2$ in
einer sogenannten \notion{\emph{faktorisierten Form}} $\,f(x)=a(x-x_1)(x-x_2)\,$ darstellen.
}
\lang{en}{
If we know the roots $x_1$ and $x_2$ of the quadratic function $\,f(x)=ax^2+bx+c\,$, it can be 
written in a \notion{\emph{factorised form}} $\,f(x)=a(x-x_1)(x-x_2)\,$.
}
\end{remark}

\section{\lang{de}{Scheitelpunktform}\lang{en}{Turning point form}} \label{sec:scheitel}

\lang{de}{
Eine weitere Darstellungsform für quadratische Funktionen ist, wie im Video bereits erwähnt, die sogenannte 
\emph{Scheitelpunktform}. Sie gibt ebenfalls Auskunft über die Lage der Parabel im zweidimensionalen Koordinatenraum.
}
\lang{en}{
A further way of writing quadratic functions is the so-called \emph{turning point} or 
\emph{completed square} form, which gives some information about the position of the parabola in 
two-dimensional coordinates.
}

\begin{definition}
\lang{de}{
Der \emph{\notion{Scheitelpunkt}} einer Parabel ist der Punkt mit der kleinsten 
$\,y-$Koordinate, falls die Parabel nach oben geöffnet ist, und der Punkt mit der 
größten $\,y-$Koordinate, falls die Parabel nach unten geöffnet ist.
}
\lang{en}{
The \emph{\notion{turning point}} of a parabola is the point with the smallest $y$-coordinate if the 
parabola is open upwards, or the point with the largest $y$-coordinate if the parabola opens 
downwards.
}
\end{definition}

\begin{remark}  %%% oder doch Definition ?????
\lang{de}{
Zu jeder quadratischen Funktion $\,f(x)= ax^2+bx+c\;$ (mit $a\ne 0$) gibt es eine
alternative Darstellungsform, die wie folgt durch die Koordinaten ihres Scheitelpunkts
und den Streckfaktor $\,a\,$ bestimmt ist
}
\lang{en}{
Every quadratic function $\,f(x)= ax^2+bx+c\;$ (with $a\ne 0$) can be written in a form using the 
coordinates of the turning point and the leading coefficient $\,a\,$
}
\[ f(x)=a(x-s_{x})^2+s_{y}.\]
\lang{de}{
Man bezeichnet sie deshalb auch als die \emph{\notion{Scheitelpunktform}} der Parabel.
Der Streckfaktor $\,a\,$ ist in beiden Darstellungsformen identsich.
}
\lang{en}{
This is called the \emph{\notion{turning point form}} or \emph{\notion{completed square form}} of the 
quadratic. The stretch factor $\,a\,$ is the same in both forms.
}
\end{remark}

\begin{example}
\lang{de}{
Die quadratische Funktion $\,f(x)= x^2-4x+5\,$ hat ihren Scheitelpunkt in 
$\; \textcolor{\#CC6600}{S=(2;1)}\,$ und ihre Scheitelpunktform lautet 
$\, \textcolor{\#0066CC}{f(x)= (x-2)^2+1}.$
}
\lang{en}{
The quadratic function $\,f(x)= x^2-4x+5\,$ has a turning point at 
$\; \textcolor{\#CC6600}{S=(2;1)}\,$ and can be written in turning point form as 
$\, \textcolor{\#0066CC}{f(x)= (x-2)^2+1}$.
}

    \begin{center}
        \image{T102_ParabolaVertex}
    \end{center}

\end{example}

\begin{rule}[\lang{de}{Umrechnung Scheitelpunktform}
             \lang{en}{Finding the turning point form}] \label{rule:umrechnung_scheitelpkt}
\lang{de}{
Die allgemeine Standardform einer quadratischen Funktion $\,f(x)=ax^2+bx+c\,$ 
(mit $a\ne 0$) und ihre Scheitelpunktform $\,f(x)=a(x-s_{x})^2+s_{y}\,$ sind durch folgende
Umrechnungsformeln ineinender überführbar:
}
\lang{en}{
The general standard form of a quadratic function $\,f(x)=ax^2+bx+c\,$ (with $a\ne 0$) and the 
turning point form $\,f(x)=a(x-s_{x})^2+s_{y}\,$ are related as follows:
}
  
 \begin{enumerate}[alph] 
  \item \lang{de}{\notion{Umrechnung von der Scheitelpunktform in die Standardform }}
        \lang{en}{\notion{Manipulating the turning point form into the standard form}}
  \[b=-2as_{x} \qquad \text{\lang{de}{und}\lang{en}{and}} \qquad c=as_{x}^2+s_{y}.\]
  \item \lang{de}{\notion{Umrechnung von der Standardform in die Scheitelpunktform }}
        \lang{en}{\notion{Manipulating the standard form into the turning point form}}
    \[s_{x}=-\frac{b}{2a} \qquad \text{\lang{de}{und}\lang{en}{and}} \qquad s_{y}=c- \frac{b^2}{4a}.\]
  \end{enumerate}
\end{rule}


\begin{proof}[\lang{de}{Herleitung}\lang{en}{Derivation}]
\begin{showhide}
 
 \begin{enumerate}[alph] 
%
  \item \lang{de}{
    Die \notion{Umrechnung von der Scheitelpunktform in die Standardform }
    erfolgt durch Anwendung der 
    \ref[content_02_rechengrundlagen_terme][2. binomischen Formel]{rule:binomische_formeln}
    und Sortieren der Terme nach Exponenten:
    }
    \lang{en}{
    \notion{Manipulating the turning point form into the standard form} is achieved by expanding the 
    \ref[content_02_rechengrundlagen_terme][square of the binomial]{rule:binomische_formeln} 
    and sorting the terms by exponents:
    }
    \[ a(x-s_{x})^2+s_{y}=a(x^2-2s_{x}x+s_{x}^2)+s_{y}=ax^2-2as_{x}x+(as_{x}^2+s_{y}). \]
    \lang{de}{
    Der \emph{Leitkoeffizient} $a$ ist als \emph{Streckfaktor} in beiden Darstellungen 
    gleich.\\Für die Koeffizienten $b$ und $c$ ergeben sich durch einen Koeffizientenvergleich
    folgende Umrechnungsfomeln:
    }
    \lang{en}{
    The \emph{leading coefficient} $a$ is the same \emph{stretch factor} in both forms.
    \\
    The coefficients $b$ and $c$ are found by comparing coefficients:
    }
    \[b=-2as_{x} \quad \text{\lang{de}{und}\lang{en}{and}} \quad c=as_{x}^2+s_{y}.\]
%
  \item \lang{de}{
    Zur \notion{Umrechnung von der Standardform in die Scheitelpunktform }
    wird zunächst der \emph{Leitkoeffizient} $\,a\,$ ausgeklammert 
    }
    \lang{en}{
    \notion{Manipulating the standard form into the turning point form} is achieved by factoring 
    out the leading coefficient $\,a\,$:
    }
    \[
        ax^2+bx+c=a(x^2+\frac{b}{a}x+\frac{c}{a}),
    \]
    \lang{de}{
    und anschließend innerhalb der Klammer der quadratische Term in Normalform
    \ref[link-04-lin-funk][quadratisch ergänzt,]{alg:quadr_erg} um über die
    binomische Formel einen Term der Form $(x-s_{x})^2$ zu erhalten:
    }
    \lang{en}{
    and we \ref[link-04-lin-funk][complete the square]{alg:quadr_erg} to obtain a term of the form 
    $(x-s_{x})^2$ using the formula for the square of a binomial:
    }
    \begin{align*} a\bigg(x^2+\frac{b}{a}x+\frac{c}{a}\bigg)
        & =a\bigg(x^2+\frac{b}{a}x+\big(\frac{b}{2a}\big)^2-\big(\frac{b}{2a}\big)^2+\frac{c}{a}\bigg)\\
        & =a\bigg( \big( x+\frac{b}{2a} \big)^2 -\big(\frac{b}{2a}\big)^2+\frac{c}{a}\bigg)\\
        & =a\big( x+\frac{b}{2a} \big)^2 - a\big(\frac{b}{2a}\big)^2+a\frac{c}{a} \\
        & = a\big( x-\underbrace{(-\frac{b}{2a})}_{=s_{x}} \big)^2 +\underbrace{c- \frac{b^2}{4a}}_{=s_{y}}
    \end{align*}
    \lang{de}{
    Der letzte Term hat die gewünschte Form der Scheitelpunktform und liefert somit die 
    folgende Umrechnungsformel:
    }
    \lang{en}{
    The final expression is in turning point form and hence we obtain the following 
    identities:
    }
    \[s_{x}=-\frac{b}{2a} \quad \text{\lang{de}{und}\lang{en}{and}} \quad s_{y}=c- \frac{b^2}{4a}.\]

  \end{enumerate}
\end{showhide}  
\end{proof}

\begin{example}
\lang{de}{
\begin{itemize}
%
%%% Video Hoever
%
  \item 
      Überführung quadratischer Funktionen mit Leitkoeffizient $a=1$ in ihre Scheitelpunktform 
        \floatright{\href{https://www.hm-kompakt.de/video?watch=107lsg}{\image[75]{00_Videobutton_schwarz}}}
        \floatright{\href{https://www.hm-kompakt.de/video?watch=107}{\image[75]{00_Videobutton_schwarz}}}\\\\ 
  \item 
      Überführung quadratischer Funktionen mit Leitkoeffizient $a \neq 1$ in ihre Scheitelpunktform 
        \floatright{\href{https://www.hm-kompakt.de/video?watch=108lsg}{\image[75]{00_Videobutton_schwarz}}}
        \floatright{\href{https://www.hm-kompakt.de/video?watch=108}{\image[75]{00_Videobutton_schwarz}}}\\
%
  \item 
      Gesucht ist die Scheitelpunktform der quadratischen Funktion \[f(x)=2x^2+6x-5.\]
      Dazu gehen wir die einzelnen, in der Herleitung beschriebenen Schritte durch:
      Wir klammern zunächst den Leitkoeffizienten, hier die $\,2\,$, aus und bestimmen
      anschließend den quadratischen Term der Scheitelpunktform durch quadratische Ergänzung, 
      hier $\,\Big( +  (\frac{3}{2})^2 -(\frac{3}{2})^2\Big), \,$ innerhalb der Klammer:
      \begin{align*}
      f(x) & =2x^2+6x-5 = 2(x^2+3x-\frac{5}{2}) \\
      &= 2\big( x^2+2\cdot \frac{3}{2}x +  (\frac{3}{2})^2 -(\frac{3}{2})^2 -\frac{5}{2}\big) \\
      &=  2\big( (x+\frac{3}{2})^2 -\frac{9}{4}-\frac{5}{2}\big) \\
      &= 2(x+\frac{3}{2})^2 - \frac{9}{2}-5 \\
      &= 2\Big(x- (-\frac{3}{2})\Big)^2 - \frac{19}{2}
      \end{align*}
      Aus dem letzten Term können wir auch den Scheitelpunkt ablesen, n"amlich $(-\frac{3}{2}; - \frac{19}{2})$.
      Beachte, dass in der Scheitelpunktform $x-d$ quadriert wird, sodass in diesem Beispiel $d=-\frac{3}{2}$ ist und
      \textbf{nicht} $\frac{3}{2}$.
\end{itemize}
}
\lang{en}{
      Suppose we want to write the quadratic function \[f(x)=2x^2+6x-5.\] in turning point form, 
      by repeating the derivation rather than by using the identities. Firstly we factor out the 
      leading coefficient $\,2\,$, and by completing the square we find the square term of the 
      turning point form, which here is $\,\Big( +  (\frac{3}{2})^2 -(\frac{3}{2})^2\Big)\,$.
      \begin{align*}
      f(x) & =2x^2+6x-5 = 2(x^2+3x-\frac{5}{2}) \\
      &= 2\big( x^2+2\cdot \frac{3}{2}x +  (\frac{3}{2})^2 -(\frac{3}{2})^2 -\frac{5}{2}\big) \\
      &=  2\big( (x+\frac{3}{2})^2 -\frac{9}{4}-\frac{5}{2}\big) \\
      &= 2(x+\frac{3}{2})^2 - \frac{9}{2}-5 \\
      &= 2\Big(x- (-\frac{3}{2})\Big)^2 - \frac{19}{2}
      \end{align*}
      We can immediately read the turning point coordinates from the last expression, 
      $(-\frac{3}{2}; - \frac{19}{2})$. Note that $x-d$ is squared in the turning point form, so in 
      this example $d=-\frac{3}{2}$ and \textbf{not} $\frac{3}{2}$.
}

\end{example}

\begin{quickcheckcontainer}
\randomquickcheckpool{1}{2}
\begin{quickcheck}
		\type{input.number}
		\begin{variables}
			\randint[Z]{a}{-5}{5}			
			\randint{b}{-4}{4}
			\randint{c}{-4}{4}
		    \function[normalize]{f}{a*x^2+b*x+c}
			\function[calculate]{d}{-b/(2*a)}
			\function[calculate]{e}{c-b^2/(4*a)}
% Änderung, da es Probleme mit dem Variablen-Namen "e" gibt:
			\function[calculate]{xs}{-b/(2*a)}
			\function[calculate]{ys}{c-b^2/(4*a)}            
		\end{variables}
		
			\text{\lang{de}{
       Bestimmen Sie die Scheitelpunktform der quadratischen Funktion $f(x)=\var{f}$,
			 und geben Sie den Scheitelpunkt an.}
            \lang{en}{
       Determine the turning point form of the quadratic function $f(x)=\var{f}$, and give the 
       coordinates of the turning point.
            }}
			\text{$f(x)=$\ansref$(x-$\ansref$)^2+$\ansref.\\ 
      \lang{de}{Der Scheitelpunkt ist}\lang{en}{The turning point is} $S=$(\ansref;\ansref)}
		
		\begin{answer}
			\solution{a}
		\end{answer}
%		\begin{answer}
%			\solution{d}
%		\end{answer}
%		\begin{answer}
%			\solution{e}
%		\end{answer}
%		\begin{answer}
%			\solution{d}
%		\end{answer}
%		\begin{answer}
%			\solution{e}
%		\end{answer}
% Änderung, da es Probleme mit dem Variablen-Namen "e" gibt:
		\begin{answer}
			\solution{xs}
		\end{answer}
		\begin{answer}
			\solution{ys}
		\end{answer}
		\begin{answer}
			\solution{xs}
		\end{answer}
		\begin{answer}
			\solution{ys}
		\end{answer}        
	\end{quickcheck}
	
\begin{quickcheck}
		\type{input.number}
		\begin{variables}
			\randint[Z]{a}{-3}{3}			
%			\randint{d}{-3}{3}
%			\randint{e}{-4}{4}
%		    \function[normalize]{f}{a*(x-d)^2+e}
%			\function[calculate]{b}{-2*a*d}
%			\function[calculate]{c}{a*d^2+e}
            % Änderung, da es Probleme mit dem Variablen-Namen "e" gibt:
			\randint{xs}{-3}{3}
			\randint{ys}{-4}{4}
		    \function[normalize]{f}{a*(x-xs)^2+ys}
			\function[calculate]{b}{-2*a*xs}
			\function[calculate]{c}{a*xs^2+ys}
		\end{variables}
		
			\text{\lang{de}{Bestimmen Sie die Normalform der quadratischen Funktion $f(x)=\var{f}$.}
            \lang{en}{Determine the standard form of the quadratic function $f(x)=\var{f}$.}}
			\text{$f(x)=$\ansref$ x^2+$\ansref$x+$\ansref.}
		
		\begin{answer}
			\solution{a}
		\end{answer}
		\begin{answer}
			\solution{b}
		\end{answer}
		\begin{answer}
			\solution{c}
		\end{answer}
	\end{quickcheck}
        
\end{quickcheckcontainer}


%%%%%%%%%%%%%%%%%%%%%%%%%%%%%%%%%%%%%%%%%%%%%%%%%%%%%%%%%%%%%%%%%%%%%%%%%%%%%%%%%%%%%%%%%%%%%

\lang{de}{
In den letzten beiden Abschnitten \ref{sec:nullstellen} und \ref{sec:scheitel} haben wir gesehen, 
wie sich die Lage einer Parabel aus ihren Nullstellen und ihrem Scheitelpunkt ableiten lässt. 
Im folgenden Abschnitt werden wir untersuchen, wie man zu vorgegebenen Punkten im zweidimensionalen
Koordinatenraum eine Parabel bestimmen kann, die diese Punkte durchläuft.
}
\lang{en}{
In the previous two sections \ref{sec:nullstellen} and \ref{sec:scheitel}, we have seen how the 
shape and position of a parabola can be determined by its roots and turning point. In the next 
section we will examine how to find a parabola that contains given points in the two-dimensional 
plane.
}

\section{\lang{de}{Festlegung einer Parabel}
         \lang{en}{Finding a parabola containing given points}}\label{sec:3_P_Parabel}

\lang{de}{
Zur Bestimmung einer Geraden werden, wie im Kapitel 
\ref[content_07_geradenformen]["`Darstellungsformen von Geraden"']{sec:zweipunktform}
beschrieben, zwei Punkte mit unterschiedlichen $x$-Werten benötigt. 
Eine Parabel ist erst durch \textbf{drei} Punkte, die in $x$ verschiedenen sind und zudem
nicht auf einer Geraden liegen dürfen, eindeutig festgelegt.
Setzt man die Koordinaten dieser drei Punkte in die Parabelgleichung $\,f(x)=ax^2+bx+c\,$
ein, so erhält man ein \emph{lineares Gleichungssystem} mit drei Gleichungen, in dem die 
drei Koeffizienten $\,a, b\,$ und $\, c\,$ die Unbekannten sind. Durch die eindeutige 
\ref[link-04-lin-funk][Lösung dieses LGS]{sec:lgs} ist dann auch die Parabel eindeutig bestimmen.
}
\lang{en}{
To find the equation of a line, like in chapter 
\ref[content_07_geradenformen]['The equation of a line']{sec:zweipunktform}, two points with 
different $x$-coordinates are required. A parabola is only uniquely determined by \textbf{three} 
points, all with different $x$-coordinates and not all on a common straight line. 
Substituting the coordinates of such three points into the quadratic $\,f(x)=ax^2+bx+c\,$ gives a 
\emph{linear system} of three equations, in which the three coefficients $\,a, b\,$ and $\, c\,$ 
are the variables. As there is a 
\ref[link-04-lin-funk][unique solution of this linear system]{sec:lgs}, we can uniquely determine 
the parabola.
}

\begin{example}\label{geraden.example.2}
%
%
% \begin{block}[important]
%    Die generische Version ist noch fehlerhaft und wird noch bearbeitet ...
% \end{block}
%
% \begin{genericGWTVisualization}[450][800]{mathlet1}
%	\begin{variables}
%        % Die editierbaren Variablen ...%
%		\number{p1x}{rational}{0}
%		\number[editable]{p1y}{rational}{1}
%		\number[editable]{p2x}{rational}{1}
%		\number[editable]{p2y}{rational}{2}
%		\number[editable]{p3x}{rational}{2}
%		\number[editable]{p3y}{rational}{0}
 %       % ... für die folgenden drei Punkte P_1, P_2 und P_3		
%		\point{p1}{rational}{var(p1x),var(p1y)}
%		\point{p2}{rational}{var(p2x),var(p2y)}
%		\point{p3}{rational}{var(p3x),var(p3y)}
%
%		% für Rechnung:
%		% nicht-editierbare Kopien der obigen Variablen für das Lösen des LGS:
%		\number{q1x}{rational}{var(p1x)}
%		\number{q1y}{rational}{var(p1y)}
%		\number{q2x}{rational}{var(p2x)}
%		\number{q2y}{rational}{var(p2y)}
%		\number{q3x}{rational}{var(p3x)}
%		\number{q3y}{rational}{var(p3y)}
%		% in Rechnung:
%		\number{c}{rational}{var(p1y)}
%		\number{f2a}{rational}{var(q2x)^2}
%		\number{f3a}{rational}{var(q3x)^2}
%		\number{l2}{rational}{var(q2y)-var(c)}
%		\number{l3}{rational}{var(q3y)-var(c)}
%		\number{nl2}{rational}{(var(q2y)-var(c))/var(q2x)}
%		\number{nl3}{rational}{(var(q3y)-var(c))/var(q3x)}
%		\number{vl}{rational}{var(nl3)-var(nl2)}
%		\number{vr}{rational}{var(q3x)-var(q2x)}
%		\number{a}{rational}{var(vl)/var(vr)}
%		\number{b}{rational}{var(nl2)-var(q2x)*var(a)}
%		\function{f}{rational}{var(a)*x^2+var(b)*x+var(c)}  %quadr. Funktion durch die drei Punkte p1,p2 und p3.
%	\end{variables}
%
%	\color{p1}{BLUE}
%	\color{p2}{RED}
%	\color{p3}{DARK_ORANGE}
%	\text{Gesucht ist die Funktionsgleichung $f(x)=ax^2+bx+c$ für die Parabel durch die drei Punkte $\textcolor{blue}{P_1}=(\var{p1x};\var{p1y})$,
%	$\textcolor{red}{P_2}=(\var{p2x};\var{p2y})$ und $\textcolor{darkorange}{P_3}=(\var{p3x};\var{p3y})$.}
%
%	\begin{canvas}
%		\plotSize{240}
%		\plotLeft{-3}
%		\plotRight{3}
%		\plot[coordinateSystem]{f,p1,p2,p3}
%	\end{canvas}
%
%	\text{ Einsetzen der Punkte in die Funktionsvorschrift liefert:}
% \text[c]{\begin{align}
% \text{(I)}&\quad \var{q1y} &= f(\var{q1x}) &= a\cdot (\var{q1x})^2+ b\cdot (\var{q1x})+c &=  &&\phantom{+}c \\
% \text{(II)}&\quad \var{q2y} &= f(\var{q2x}) &= a\cdot (\var{q2x})^2+ b\cdot (\var{q2x})+c &=\var{f2a}\cdot a&+(\var{q2x})\cdot b&+c \\
% \text{(III)}&\quad \var{q3y} &= f(\var{q3x}) &= a\cdot (\var{q3x})^2+ b\cdot (\var{q3x})+c &=\var{f3a}\cdot a&+(\var{q3x})\cdot b&+c \\
% \end{align} }
%\text{Dieses Gleichungssystem lässt sich nach dem \textbf{Einsetzungsverfahren} lösen.
%Setzt man $c=\var{q1y}$ aus der ersten Gleichung \textbf{(I)} in die 
%Gleichungen \textbf{(II)} und \textbf{(III)} ein, erhält man die Gleichungen }
% \text[c]{\begin{align}
%    \text{(II*)} &\quad \var{l2} &= \var{f2a}a+(\var{q2x})b \\
%    \text{(III*)}&\quad \var{l3} &= \var{f3a}a+(\var{q3x})b.\\
% \end{align} }
%\text[c]{ 
% \IF{var(q2x) != 1}{$(II*)\,$ ist äquivalent zu $\var{nl2}=\var{q2x}a+b$\quad (teilen durch $\var{q2x}$).\\ }
% \IF{var(q3x) != 1}{$(III*)\;$ ist äquivalent zu $\var{nl3}=\var{q3x}a+b$\quad (teilen durch $\var{q3x}$).\\ }
%}
%\text{Zieht man die erste der beiden Gleichungen von der zweiten ab, erhält man}
% \text[c]{ $\var{nl3}-\var{nl2} = (\var{q3x}a+b)-(\var{q2x}a+b)$
%  \IFELSE{var(vr) = 1}{$\Leftrightarrow \var{a} = a.$}
%  {$\Leftrightarrow  \var{vl}  = \var{vr}a \Leftrightarrow \var{a} = a.$ }
% }
%\text{Daraus berechnet man $b$ mittels: $\var{nl2}=\var{q2x}a+b$, d.h. $b=\var{b}$.}
%\text{Insgesamt also: $f(x)=\var{f}$.}
%
%\end{genericGWTVisualization}
%\end{example}


%\begin{example}  %Gleiches Beispiel, aber nicht generisch.
\lang{de}{
Gesucht ist die Funktionsgleichung $\, f(x)=ax^2+bx+c\,$ f"ur die Parabel, die durch 
die drei Punkte $P_1=(0;1)$, $P_2=(1;2)$ 
und $P_3=(2;7)$ verläuft.
}
\lang{en}{
Suppose we want to find the quadratic function $\, f(x)=ax^2+bx+c\,$ whose graph intersects the 
points $P_1=(0;1)$, $P_2=(1;2)$ and $P_3=(2;7)$.
}

    \begin{center}
        \image{T102_ParabolaThreePoints}
    \end{center}

\lang{de}{
Wir setzen die Koordinaten der gegebenen Punkte in die Funktionsgleichung ein und erhalten so 
das folgende lineare Gleichungssystem mit 3 Gleichungen und den 3 Unbekannten: 
}
\lang{en}{We substitute the coordinates of the given points into the general form of the function 
and hence obtain the following linear system with three solutions and three indeterminates: 
}
$a, b\,$ und $\,c:$
\begin{align*}
    \text{(I)}   &\qquad  1 &\,=\,& f(0) &\,=\,& a\cdot 0^2+ b\cdot 0+c &\,=\,& && && \, c \\
    \text{(II)}  &\qquad  2 &\,=\,& f(1) &\,=\,& a\cdot 1^2+ b\cdot 1+c &\,=\,&\, a &\,+\,&\, b &\,+\,& c \\
    \text{(III)} &\qquad  7 &\,=\,& f(2) &\,=\,& a\cdot 2^2+ b\cdot 2+c &\,=\,&4 a &\,+\,&2b &\,+\,& c \\
\end{align*}
\lang{de}{
Da  die Variable $\,c\,$ in der ersten Gleichung bereits \emph{"`isoliert"'} ist, bietet 
sich im ersten Schritt die Anwendung des  
\ref[link-04-lin-funk][Einsetzungsverfahrens]{alg:einsetzungsverfahren} an.
Wir setzen also $c=1$ aus der Gleichung \textbf{(I)} in die 
Gleichungen \textbf{(II)} und \textbf{(III)} ein und erhalten das folgende
reduzierte LGS mit 2 Gleichungen und 2 Unbekannten:
}
\lang{en}{
As the variable $\,c\,$ is \emph{'isolated'} in the first equation, the opportunity arises to solve 
the system by \ref[link-04-lin-funk][substitution]{alg:einsetzungsverfahren}. We therefore 
substitute $c=1$ from equation \textbf{(I)} into equations \textbf{(II)} und \textbf{(III)} and 
obtain the following reduced linear system with two equations and two indeterminates.
}
\begin{align*}
    \text{(II*)}  &\qquad  1 &\,=\,& \,a+ \,b   &\\
    \text{(III*)} &\qquad  6 &\,=\,& 4a+2b &\qquad \vert :2 \\
\end{align*}
\begin{center}
_____________________________________________________
\end{center}
\begin{align*}
    \text{(II*)}  &\qquad  1 &\,=\,& \, a+b  \\
    \text{(III*)} &\qquad  3 &\,=\,& 2a+ b  \\
\end{align*}

\lang{de}{
Hier ziehen wir nun nach dem  
\ref[link-04-lin-funk][Additionsverfahren]{alg:additionsverfahren}
die Gleichung \textbf{(II*)} von der Gleichung \textbf{(III*)} ab und erhalten 
}
\lang{en}{
Using the \ref[link-04-lin-funk][addition method]{alg:additionsverfahren}, we now subtract equation 
\textbf{(II*)} from equation \textbf{(III*)} and obtain 
}
 \[  3-1 = 2a-a+b-b  \Leftrightarrow  2  = a. \]
\lang{de}{$a=2\,$ eingesetzt in Gleichung \textbf{(II*)} liefert: }
\lang{en}{Now $a=2\,$ substituted into equation \textbf{(II*)} yields}
 \[  1=2+b  \Leftrightarrow  b=-1. \]

\lang{de}{Die Funktionsgleichung für die gesuchte Parabel lautet somit:}
\lang{en}{The equation of the parabola is therefore:}
\[ f(x)=2x^2-x+1. \]
\end{example}

\begin{remark}
\begin{enumerate}
\item \lang{de}{
Die Rechnung im vorigen Beispiel wird komplizierter, wenn keiner der Punkte $\,0\,$ als 
$\, x-$Koordinate hat, denn dann lässt sich der Wert von $c$ nicht mehr direkt ablesen. \\
Allgemeine Verfahren zur Lösung linearer Gleichungssysteme mit 3 Gleichungen und 3 Unbekannten 
sind in Kapitel 1 anhand eines Beispiels sowohl mit dem
\ref[link-04-lin-funk][Additionsverfahren,]{ex:additionsverfahren_3_3} als auch mit dem
\ref[link-04-lin-funk][Einsetzungsverfahren]{ex:einsetzungsverfahren_3_3} beschrieben.
}
\lang{en}{
The calculation in the previous example becomes more complicated when none of the points has 
$x$-coordinate $\,0\,$, as then the system cannot be immediately reduced to a smaller one. \\
General methods for solving linear systems with three equations and three indeterminates are 
described in chapter 1, for example the 
\ref[link-04-lin-funk][addition method]{ex:additionsverfahren_3_3} and the 
\ref[link-04-lin-funk][substitution method]{ex:einsetzungsverfahren_3_3}.
}

\item \lang{de}{
Die Berechnung von $a$, $b$ und $c$ funktioniert übrigens auf diese Weise auch, wenn die 
drei gegebenen Punkte auf einer Geraden liegen. In dem Fall erhält man bei der Lösung für $a$ 
den Wert $0$. Die Funktionsgleichung beschreibt also keine Parabel, sondern eine lineare Funktion
$f(x)=bx+c$.
}
\lang{en}{
Calculating $a$, $b$ and $c$ in this way also works if the three given points lie on one line. In 
this case the value for $a$ will turn out to be $0$. The equation will therefore not describe a 
parabola, but rather a line with equation $f(x)=bx+c$.
}

\item \lang{de}{
Handelt es sich bei einem der vorgegebenen Punkte um den Scheitelpunkt der Parabel, 
empfiehlt es sich von der \emph{Scheitelpunktform} auszugehen, um die Parameter der zugehörigen
quadratischen Funktion zu bestimmen. \\
Sind die Nullstellen der Parabel vorgegeben, ist ein Ansatz über die \emph{faktorisierte Form} 
der Parabel geeignet.
}
\lang{en}{
If one of the given points is the turning point of the parabola, it may be easier to derive the 
quadratic function by starting from the \emph{turning point form}.\\
If the roots of the parabola are given, it makes sense to substitute them into the \emph{factorised 
form} of the parabola.
}
\end{enumerate}
\end{remark}
%
%%% Video Hoever
%
\lang{de}{
Schauen Sie sich zu den in diesem Abschnitt erworbenen Erkenntnissen weitere anschauliche
Erklärungen und Beispiele im Video an:
\\
\floatright{\href{https://www.hm-kompakt.de/video?watch=112lsg}{\image[75]{00_Videobutton_schwarz}}}
\floatright{\href{https://www.hm-kompakt.de/video?watch=112}{\image[75]{00_Videobutton_schwarz}}}\\\\
}
\lang{en}{
\\
}

% 

\end{visualizationwrapper}

\end{content}

