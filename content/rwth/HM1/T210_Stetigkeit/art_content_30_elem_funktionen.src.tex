%$Id:  $
\documentclass{mumie.article}
%$Id$
\begin{metainfo}
  \name{
    \lang{de}{Stetigkeit elementarer Funktionen}
    \lang{en}{}
  }
  \begin{description} 
 This work is licensed under the Creative Commons License Attribution 4.0 International (CC-BY 4.0)   
 https://creativecommons.org/licenses/by/4.0/legalcode 

    \lang{de}{Beschreibung}
    \lang{en}{}
  \end{description}
  \begin{components}
\component{generic_image}{content/rwth/HM1/images/g_tkz_T210_Discontinuity.meta.xml}{T210_Discontinuity}
\end{components}
  \begin{links}
    \link{generic_article}{content/rwth/HM1/T205_Konvergenz_von_Folgen/g_art_content_14_konvergenz.meta.xml}{content_14_konvergenz}
    \link{generic_article}{content/rwth/HM1/T210_Stetigkeit/g_art_content_29_stetigkeit_definitionen.meta.xml}{stetigkeit}
    \link{generic_article}{content/rwth/HM1/T209_Potenzreihen/g_art_content_27_konvergenzradius.meta.xml}{potenzreihen}
    \link{generic_article}{content/rwth/HM1/T204_Abbildungen_und_Funktionen/g_art_content_12_reelle_funktionen_monotonie.meta.xml}{reelle-funktionen}
    \link{generic_article}{content/rwth/HM1/T209_Potenzreihen/g_art_content_28_exponentialreihe.meta.xml}{exp}
    \link{generic_article}{content/rwth/HM1/T104_weitere_elementare_Funktionen/g_art_content_15_exponentialfunktionen.meta.xml}{exponentialfunktion}
    \link{generic_article}{content/rwth/HM1/T211_Eigenschaften_stetiger_Funktionen/g_art_content_33_zwischenwertsatz.meta.xml}{zwischenwertsatz}
  \end{links}
  \creategeneric
\end{metainfo}
\begin{content}
\usepackage{mumie.ombplus}
\ombchapter{10}
\ombarticle{2}

\lang{de}{\title{Stetigkeit elementarer Funktionen}}
 
\begin{block}[annotation]
  
  
\end{block}
\begin{block}[annotation]
  Im Ticket-System: \href{http://team.mumie.net/issues/9790}{Ticket 9790}\\
\end{block}

\begin{block}[info-box]
\tableofcontents
\end{block}


In diesem Abschnitt werden wir zeigen, dass die unten aufgeführten elementaren Funktionen 
alle auf ihrem Definitionsbereich stetig sind. Anschaulich ist das aus der Schule bekannt, weil
ihre Graphen (außer bei Definitionslücken) keine Sprünge aufweisen.


\section{Polynomfunktionen und rationale Funktionen}\label{sec:polynome-und-rationale-funk}

Die "`einfachsten"' Funktionen sind die Polynomfunktionen und rationalen Funktionen.
 
\begin{theorem}\label{thm:polynomstetig}
Jede Polynomfunktion $f:\R\to \R, x\mapsto a_0+a_1x+\ldots+a_kx^k$ mit reellen Zahlen $a_0,a_1,\ldots, a_k$ ist auf ganz $\R$ stetig.
\end{theorem}

\begin{proof*}
\begin{incremental}{0}
\step
Um dies zu sehen, verwenden wir das \ref[stetigkeit][Folgenkriterium]{sec:folgenkriterium} für Stetigkeit.

Ist nämlich $x^*\in \R$ und $( x_n)_{n \geq 1}$ eine Folge, die gegen $x^*$ konvergiert, so gilt für die Funktionswerte:
\begin{eqnarray*}
 \lim_{n\to \infty} f(x_n) &=& \lim_{n\to \infty} (a_0+a_1x_n+\ldots+a_kx_n^k)
=a_0+a_1 (\lim_{n\to \infty} x_n)+\ldots +a_k (\lim_{n\to \infty} x_n^k)\\ 
&=&a_0+a_1 (\lim_{n\to \infty} x_n)+\ldots +a_k (\lim_{n\to \infty} x_n)^k= a_0+a_1x^*+\ldots+a_k(x^*)^k\\
&=&f(x^*).
\end{eqnarray*}
Benutzt wurden hierbei die \ref[content_14_konvergenz][Grenzwertregeln]{sec:grenzwertregeln}.
\end{incremental}
\end{proof*}

\begin{theorem}\label{thm:rationalstetig}
Jede rationale Funktion $\frac{f}{g}$ mit Polynomfunktionen $f$ und $g$ ist stetig auf ihrem Definitionsbereich
$D=\{ x\in \R | g(x)\neq 0 \}$. 
\end{theorem}

\begin{proof*}
Mit der Stetigkeit der Polynomfunktionen erhält man dies direkt aus der \ref[stetigkeit][Regel zur Stetigkeit zusammengesetzter Funktionen]{rule:zusammengesetzte-funktionen}:
\begin{incremental}{0}
\step
Die Polynomfunktionen $f$ und $g$ sind auf ganz $\R$ stetig, und daher ist ihr Quotient $\frac{f}{g}$ an
allen Stellen $x^*$ stetig, an denen $g(x^*)\neq 0$ ist. Also auf dem kompletten Definitionsbereich.
\end{incremental}
\end{proof*}


\section{Potenzreihen und Exponentialfunktion}\label{sec:potenzreihen}
Die Exponentialfunktion ist ebenso stetig. Das wird über die Potenzreihe mit Konvergenzradius deutlich.
\begin{theorem}\label{thm:ptenzreihen_stetig}
Sei $x_0 \in \R$ und $\sum_{n = 0}^\infty  a_n ( x - x_0 )^n$ eine reelle \link{potenzreihen}{Potenzreihe} mit 
\ref[potenzreihen][Konvergenzradius]{sec:konvergenzradius} $R > 0$. Dann ist die Funktion
$f : U_R ( x_0 ) \to \R, x \mapsto \sum_{n = 0}^\infty  a_n ( x - x_0 )^n$
stetig.
Für $R=\infty$ ist hierbei $U_\infty ( x_0 )  = \R$ zu setzen.
\end{theorem}

\begin{proof*}
Die Grenzwertregeln können nicht direkt wie bei den Polynomen angewandt werden, da es sich bei der
Potenzreihe um eine unendliche Summe handelt.
\begin{incremental}{0}
\step
Sei $x^*\in U_R ( x_0 )$, d.h. es gilt $|x^*-x_0|<R$. Wie schon beim Beweis zur absoluten Konvergenz
der Potenzreihe wählen wir $r\in \R$ mit $|x^*-x_0|<r<R$ und zeigen zunächst, dass es ein $C\in \R$ gibt,
so dass $|f(x_1)-f(x^*)|\leq C\cdot |x_1-x^*|$ für alle $x_1\in \R$ mit $|x_1-x_0|<r$:
\step
Nach der geometrischen Summenformel gilt für jedes $n\in \N$:
\begin{eqnarray*}
(x_1-x_0)^n -(x^*-x_0)^n &=& (x^*-x_0)^n\cdot \left( (\frac{x_1-x_0}{x^*-x_0})^n -1\right) \\
&=& (x^*-x_0)^n\cdot  \left( \frac{x_1-x_0}{x^*-x_0} -1\right) \cdot 
\sum_{k=0}^{n-1} (\frac{x_1-x_0}{x^*-x_0})^k \\
&=& \big( (x_1-x_0)-(x^*-x_0) \big) \cdot \sum_{k=0}^{n-1} (x_1-x_0)^k (x^*-x_0)^{n-1-k} \\
&=& (x_1-x^*)\cdot \sum_{k=0}^{n-1} (x_1-x_0)^k (x^*-x_0)^{n-1-k}.
\end{eqnarray*}

Daher gilt für jedes $N\in \N$:
\begin{eqnarray*}
| \sum_{n = 0}^N  a_n ( x_1 - x_0 )^n - \sum_{n = 0}^N  a_n ( x^* - x_0 )^n |
&=& |\sum_{n = 1}^N  a_n \left( (x_1-x_0)^n -(x^*-x_0)^n\right) | \\
&\leq & \sum_{n = 1}^N  {|a_n|} |(x_1-x_0)^n -(x^*-x_0)^n | \\
&=& \sum_{n = 1}^N  {|a_n|} | (x_1-x^*)\cdot \sum_{k=0}^{n-1} (x_1-x_0)^k (x^*-x_0)^{n-1-k} |\\
&\leq & |x_1-x^*| \cdot \sum_{n = 1}^N   {|a_n|} \sum_{k=0}^{n-1} |x_1-x_0|^k |x^*-x_0|^{n-1-k} \\
&\leq & |x_1-x^*| \cdot \sum_{n = 1}^N   {|a_n|} \sum_{k=0}^{n-1} r^{n-1}\\
&=&  |x_1-x^*| \cdot \frac{1}{r}\cdot \sum_{n = 1}^N  {|a_n|}nr^n.
\end{eqnarray*}
\step
Nach dem \ref[potenzreihen][Abelschen Lemma]{thm:abelsches-lemma} konvergiert die
Reihe $\sum_{n = 1}^\infty  {|a_n|}nr^n$ absolut, da die Folge $(|a_n|s^n)$ für beliebiges $s$ mit $r<s<R$ beschränkt ist.\\
Setzen wir $C= \frac{1}{r}\cdot \sum_{n = 1}^\infty  {|a_n|}nr^n$, so erhalten wir aus obiger Rechnung
\[ | \sum_{n = 0}^N  a_n ( x_1 - x_0 )^n - \sum_{n = 0}^N  a_n ( x^* - x_0 )^n | \leq C\cdot  {|x_1-x^*|}, \]
und damit auch für den Grenzwert
\begin{eqnarray*}
 |f(x_1)-f(x^*)| &=&  |  \lim_{n\to \infty}  \sum_{n = 0}^N  a_n ( x_1 - x_0 )^n -  \lim_{n\to \infty} \sum_{n = 0}^N  a_n ( x^* - x_0 )^n |  \\
&=& \lim_{n\to \infty} |  \sum_{n = 0}^N  a_n ( x_1 - x_0 )^n - \sum_{n = 0}^N  a_n ( x^* - x_0 )^n | \leq C \cdot  {|x_1-x^*|}
\end{eqnarray*}
für alle $x_1\in \R$ mit $|x_1-x_0|<r$.

Ist nun $( x_n)_{n \geq 1}$ eine beliebige Folge, die gegen $x^*$ konvergiert, so gibt es ein $N\in \N$ mit 
$|x_n-x^*|< r- |x^* - x_0|$ für alle $n\geq N$ und insbesondere mit 
$|x_n-x_0|<r$ für alle $n\geq N$.

Daher gilt:
\[  \lim_{n\to \infty} |f(x_n)-f(x^*)| \leq  \lim_{n\to \infty} C\cdot  |x_n-x^*| =0, \]
also
\[  \lim_{n\to \infty} f(x_n) =f(x^*). \]
Die Funktion $f$ ist somit in $x^*$ stetig.
\end{incremental}
\end{proof*}

Im Abschnitt \link{exp}{Exponentialreihe} hatten wir die (komplexe) Exponentialfunktion $\exp$ definiert als die Funktion
\[ \exp:\C\to \C, z \mapsto \sum_{n=0}^\infty \frac{z^n}{n!}. \]
Durch Einschränkung auf reelle Werte erhalten wir die reelle Exponentialreihe
\[ \exp:\R\to \R, x \mapsto \sum_{n=0}^\infty \frac{x^n}{n!} \]
mit Wertemenge $\R_+^*=\{ x\in \R | x>0\}$.

Nach dem Satz über die Stetigkeit von Potenzreihen gilt also:
\begin{theorem}
Die Exponentialfunktion $\exp$ ist auf ganz $\R$ stetig.
\end{theorem}

\begin{remark}
Für rationale Zahlen $x=\frac{p}{q}\in \Q$ ist 
\[ \exp(x)=\exp(\frac{p}{q})=e^x=e^{\frac{p}{q}}=\sqrt[q]{e^p} \]
(vgl. Abschnitt \link{exp}{Exponentialreihe}), weshalb man auch für reelle Zahlen $x$ oft
$e^x$ statt $\exp(x)$ schreibt.

Die in der Schule übliche \link{exponentialfunktion}{intuitive Einführung} von $e^x$ bedeutet nichts anderes, als dass man für reelle Zahlen $x$, den Wert $e^x$ definiert als
\[ e^x= \lim_{n\to \infty} e^{x_n} \]
für eine beliebige Folge $(x_n)_{n\geq 1}$ von rationalen Zahlen, die gegen $x$ konvergiert.

Damit diese Definition überhaupt sinnvoll ist, ist natürlich zu klären, dass der Grenzwert nicht von der gewählten Folge abhängt. Dies impliziert insbesondere zu zeigen, dass die $e$-Funktion als Funktion auf den rationalen Zahlen stetig ist, d.h. dass $\Q\to \R, x\mapsto e^x$ stetig ist.

Darauf soll hier aber nicht näher eingegangen werden.
\end{remark}

\begin{quickcheck}
    \field{real}
        \type{input.function}
            \begin{variables}
                \function{a}{ln(2)}
                \drawFromSet{b}{3,4,5,6}
                \function[calculate]{bb}{b+b}
            \end{variables}
            \text{Berechnen Sie eine reelle Zahl a so, dass folgende Funktion für $x\in \R$
                  stetig ist:\\
                  \[f(x)=\begin{cases}
                  \var{b}e^{ax}& \text{für } x<1\\
                  \var{b}(x^2+1)& \text{für }x\geq 1
                  
                  \end{cases}\]
                  Antwort:\quad $a=$\ansref}
            \begin{answer}
                \solution{a}
                \end{answer}
            \explanation{Die beiden Teilfunktionen sind auf $\R$ stetig, d.h. es muss lediglich die
            Anschlussstelle $x=1$ untersucht werden: $f(1)=\var{bb}$, also muss gelten:\\
            $\var{b}e^{a\cdot 1}=\var{bb} \iff e^a=2$ und somit $a=\ln(2)$.}

\end{quickcheck}



\section{Wurzelfunktionen und Logarithmusfunktion}\label{sec:wurzel-und-ln}

Die $n$-te \ref[reelle-funktionen][Wurzelfunktion]{ex:n-te-wurzel} 
$g:\R_+\to \R_+, x\mapsto \sqrt[n]{x}$ ist die Umkehrfunktion zur (eingeschränkten) $n$-ten Potenzfunktion $f:\R_+\to \R_+,x\mapsto x^n$.

Ebenso ist die Logarithmusfunktion $\ln:\R_+^*\to \R$ die Umkehrfunktion zur Exponentialfunktion.

Die Stetigkeit dieser beiden Funktionen folgt aus dem folgenden allgemeineren Satz über die Stetigkeit von Umkehrfunktionen streng monotoner Funktionen.

\begin{theorem}[Stetigkeit der Umkehrfunktion]\label{thm:inverse-stetig}
Sei $I \subseteq \R$ ein (evtl. unendliches) Intervall und $f : I \to \R$ eine streng monotone, stetige Funktion mit Wertemenge $W$. Dann ist die Umkehrfunktion $f^{-1}: W \to I$
ebenfalls stetig.
\end{theorem}

\begin{proof*}
%Wir betrachten den Fall, dass $f$ streng monoton wachsend ist. Der Fall einer streng monoton fallenden Funktion wird ganz ähnlich gezeigt.
%
Da $f$ stetig ist, ist nach dem % \link{zwischenwertsatz}{Zwischenwertsatz} 
\ref[zwischenwertsatz][Zwischenwertsatz]{thm:zwischenwertsatz}
das Bild des Intervalls $I$ ein Intervall, d.h. $W$ ist ein Intervall.
\begin{incremental}{0}
\step
Da der Zwischenwertsatz erst in einem späteren Kapitel 
behandelt wird, sei hier der Anschaulichkeit halber mit einem Gegenbeispiel gearbeitet: die folgende Abbildung
zeigt eine \emph{unstetige} Funktion, deren Wertebereich des Intervalls $I=[1;3]$ eben kein Intervall ist,
da die Funktionswerte $y\in [0;1)$ fehlen.

\begin{center}
\image{T210_Discontinuity}
\end{center}

Wegen der Monotonie von $f$ ist $x\in I$ genau dann ein Randpunkt von $I$, wenn $f(x)$ ein Randpunkt von $W$ ist.
\step
Sei nun zunächst $y^*\in W$ kein Randpunkt, sowie  $\epsilon>0$ beliebig.
$x^*=f^{-1}(y^*)$ ist dann kein Randpunkt von $I$ und damit ist $x^*$ auch innerer Punkt
des Intervalls $U_{\epsilon}(x^*)\cap I$. Wegen der Stetigkeit und Monotonie von $f$ ist dann $y^*=f(x^*)$ auch innerer Punkt des Bildes $J=f(U_{\epsilon}(x^*)\cap I)$.
Es gibt daher ein $\delta>0$ mit $U_\delta(y^*)\subseteq J$. Damit gilt für alle $y\in U_\delta(y^*)$:
\[  f^{-1}(y)\in f^{-1}(J)\subseteq \left( U_{\epsilon}(x^*)\cap I\right) \subseteq U_\epsilon(x^*)=U_\epsilon(f^{-1}(y^*)).\]
Also ist $f^{-1}$ in $y^*$ stetig.

Ist $y^*\in W$ ein unterer Randpunkt (d.h. $y\geq y^*$ für alle $y\in W$), so ändert sich lediglich, dass es ein $\delta>0$ gibt mit
$[y^*; y^*+\delta)\subseteq J$. Da jedoch in diesem Fall $[y^*; y^*+\delta)=U_\delta(y^*)\cap W$ ist, folgt ebenfalls
\[  f^{-1}(y) \in U_\epsilon(x^*)\quad \text{für alle }y \in U_\delta(y^*)\cap W. \]
Daher ist $f^{-1}$ auch am unteren Randpunkt stetig.

Ganz entsprechend folgt die Stetigkeit am oberen Randpunkt von $W$.
\end{incremental}
\end{proof*}


\section{Kompositionen stetiger Funktionen}\label{sec:kompositionen}

In der Schule hat man auch kompliziertere Funktionsvorschriften wie $h(x)=\exp(\cos(x))$ kennengelernt. 
Dass auch diese auf ihrem Definitionsbereich stetige Funktionen beschreiben,
ergibt sich aus dem Satz über die Komposition (Verkettung) stetiger Funktionen.


\begin{theorem}[Komposition stetiger Funktionen]\label{thm:komposition_stetiger_fkt}
Seien $f:D_f\to \R$ und $g:D_g\to \R$ reelle Funktionen mit $f(D_f)\subseteq D_g$. Weiter sei
$f$ stetig an einer Stelle $x^*$ und $g$ stetig an der Stelle $y^*=f(x^*)$.

Dann ist die Komposition $g\circ f:D_f\to \R$ stetig an der Stelle $x^*$.

Sind $f$ und $g$ stetig auf ihrem ganzen Definitionsbereich, so ist auch $g\circ f$ auf dem ganzen Definitionsbereich $D_f$ stetig.
\end{theorem}

\begin{proof*}
Dies zeigen wir mit dem Folgenkriterium für Stetigkeit:
\begin{incremental}{0}
\step
Sei $( x_n)_{n \geq 1}$ eine beliebige Folge in $D_f$, die gegen $x^*$ konvergiert, und
$y_n=f(x_n)$ für alle $n\in \N$.
Wegen der Stetigkeit von $f$ bei $x^*$ folgt: 
\[ \lim_{n\to \infty} y_n=\lim_{n\to \infty} f(x_n)=f(x^*)=y^*.\]
Wegen der Stetigkeit von $g$ bei $y^*$ folgt weiter:
\[ \lim_{n\to \infty} g(y_n)= g(y^*). \]
Damit gilt:
\[ \lim_{n\to \infty} (g\circ f)(x_n)=  \lim_{n\to \infty} g(y_n)=g(y^*)=g(f(x^*))=(g\circ f)(x^*).\]
Also ist $g\circ f$ bei $x^*$ stetig.
\end{incremental}
\end{proof*}

\begin{example}
Die Funktion $h:\R\to \R, x\mapsto \exp(\cos(x))$ ist stetig.\\
Wählt man nämlich $g:\R\to \R,x\mapsto \exp(x)$ und $f:\R\to \R, x\mapsto \cos(x)$, so ist $h$ genau die Komposition $h=g\circ f$, denn
\[  (g\circ f)(x)=g(f(x))=g(\cos(x))=\exp(\cos(x))=h(x) \]
für alle $x\in \R$.
\end{example}


\end{content}