%$Id:  $
\documentclass{mumie.article}
%$Id$
\begin{metainfo}
  \name{
    \lang{de}{Asymptoten}
    \lang{en}{}
  }
  \begin{description} 
 This work is licensed under the Creative Commons License Attribution 4.0 International (CC-BY 4.0)   
 https://creativecommons.org/licenses/by/4.0/legalcode 

    \lang{de}{Beschreibung}
    \lang{en}{}
  \end{description}
    \begin{components}
    \component{generic_image}{content/rwth/HM1/images/g_tkz_T210_Sine1x.meta.xml}{T210_Sine1x1}
    \component{generic_image}{content/rwth/HM1/images/g_tkz_T210_xSine1x.meta.xml}{T210_xSine1x1}
    \component{js_lib}{system/media/mathlets/GWTGenericVisualization.meta.xml}{mathlet1}
  \end{components}
  \begin{links}
    \link{generic_article}{content/rwth/HM1/T103_Polynomfunktionen/g_art_content_10_polynomdivision.meta.xml}{content_10_polynomdivision}
    \link{generic_article}{content/rwth/HM1/T205_Konvergenz_von_Folgen/g_art_content_16_konvergenzkriterien.meta.xml}{content_16_konvergenzkriterien}
    \link{generic_article}{content/rwth/HM1/T210_Stetigkeit/g_art_content_31_grenzwerte_von_funktionen.meta.xml}{content_31_grenzwerte_von_funktionen}
    \link{generic_article}{content/rwth/HM1/T206_Folgen_II/g_art_content_19_bestimmte_divergenz.meta.xml}{bestimmte-divergenz}
  \end{links}
  \creategeneric

  % Alte Links/Components  
  %\begin{components}
    %   \component{js_lib}{system/media/mathlets/GWTGenericVisualization.meta.xml}{mathlet1}
  % \end{components}
  % \begin{links}
  %   \link{generic_article}{content/rwth/HM1/T206_Folgen_II/g_art_content_19_bestimmte_divergenz.meta.xml}{bestimmte-divergenz}
  % \end{links}
  % \creategeneric
\end{metainfo}
\begin{content}
\usepackage{mumie.ombplus}
\ombchapter{10}
\ombarticle{4}
\usepackage{mumie.genericvisualization}

\begin{visualizationwrapper}

\lang{de}{\title{Asymptoten}}

\begin{block}[annotation]
  Kapitel wurde um geschrieben, viele JSX-Graphs ergänzt
  und dem Speziallfall der gebrochen rationalen Funktionen ein eigener Abschnitt gewidmet.
Weiter wurden die Farben für die Visualisierungen bewusst gewählt: 
Weil die recht verbreitete Rot-Grün-Schwäche die Unterscheidung von grün-orange-magenta
unmöglich macht, wurden die Graphen meist schwarz gesetzt, die Asymptoten grün bzw. blau.
 \end{block}
\begin{block}[annotation]
  Im Ticket-System: \href{http://team.mumie.net/issues/9792}{Ticket 9792}\\
\end{block}

\begin{block}[info-box]
\tableofcontents
\end{block}
Hier wenden wir das Wissen über  Funktionsgrenzwerte an, um das 
Verhalten von Funktionen am Rand ihrer Definitionsbereiche zu untersuchen.\\
Sei $f:D\to\R$ eine reelle Funktion. Wir interessieren uns für das Verhalten von $f$ in  einem Randpunkt
$x^\ast$ von $D$, der nicht zu $D$ gehört, zum Beispiel dem Endpunkt eines offenen Intervalls.
Randpunkte sind grundsätzlich endlich, also $x^\ast\in\R$, aber für Funktionen mit unbeschränktem Definitionsbereich
ist auch (oder gerade) das Verhalten für $x$ gegen $\pm\infty$ interessant. Deshalb lassen wir im letzten Abschnitt wiederum die Schreibweise $x^\ast=\pm \infty$ zu.
\\
Oftmals kann das Verhalten der Funktion gegen einen solchen Randpunkt oder $\pm\infty$ durch eine Asymptote beschrieben werden: 
Der Funktionsgraph schmiegt sich dann an eine Gerade oder den Graphen einer einfacheren Funktion an, wenn $x$ gegen  $x^\ast$ läuft.
% Im letzten Abschnitt hatten wir Grenzwerte gegen Definitionslücken betrachtet. In diesem Abschnitt geht es um die Grenzwerte für $x$ gegen $\infty$ oder gegen $-\infty$, wenn man also Folgen $(x_n)$ betrachtet, die \link{bestimmte-divergenz}{bestimmt divergent} gegen  $\infty$ oder gegen $-\infty$ sind.

% Des weiteren geht es um den Fall, dass für links- bzw. rechtsseitige Folgen an Definitionslücken die Folgen der Funktionswerte bestimmt divergieren.

% In beiden Fällen erhält man grafisch sogenannte \emph{Asymptoten}.


% Die Ergebnisse sind ganz ähnlich wie im vorigen Abschnitt, wenn man bestimmt divergent gegen $\infty$ bzw. gegen $-\infty$ als Konvergenz ansieht.


\section{Verhalten an endlichen Randpunkten}
In diesem Abschnitt sei stets $x^\ast\in\overline{D}\setminus D\subset\R$ ein 
endlicher Randpunkt des Definitionsbereichs $D$ einer Funktion $f$.
\begin{definition}[Stetige Fortsetzbarkeit]
Es sei $f:D\to\R$ eine reelle Funktion und $x^\ast\in\overline{D}\setminus D\subset\R$.
Die Funktion $f$ heißt \notion{stetig fortsetzbar in} $x^\ast$, wenn der Funktionsgrenzwert $\lim_{x\to x^\ast}f(x)\in\R$ existiert.
\\
In diesem Fall heißt die Funktion 
\[\tilde{f}:D\cup{\{x^\ast\}}\to\R, x\mapsto\begin{cases}f(x)& \text{ falls } x\in D,\\ \lim_{x\to x^\ast}f(x)& \text{ falls } x=x^\ast,\end{cases}\]
die stetige Fortsetzung der Funktion $f$ auf $D\cup\{x^\ast\}$.
\end{definition}
\begin{example}
Im \ref[content_31_grenzwerte_von_funktionen][allerersten Beispiel (b)]{ex:erstes_bsp_Grenzwert_von_Funktionswerten}
hatten wir gesehen,
dass der Grenzwert der Funktion $g:\R\setminus\{1\}\to\R$, $x\mapsto \frac{x^2-1}{x-1}$, in $x^\ast=1$ existiert,
\[\lim_{x\to 1}g(x)=\lim_{x\to 1}(x+1)=2.\]
Die Funktion $g$ ist also auf ganz $\R$ stetig fortsetzbar durch die Funktionsvorschrift $x\mapsto x+1$.
\end{example}

\begin{definition}\label{def:best_div_fkt}
Es sei $f:D\to \R$ eine reelle Funktion und es sei $x^\ast\in\overline{D}\setminus D\subset\R$ ein Randpunkt.
\begin{enumerate}
\item[(i)]
  Die Funktion $f$ heißt \notion{bestimmt divergent gegen $\infty$ für $x \to x^*$}, wenn
  \[ \lim_{n\to \infty} f(x_n)= \infty \]
  für jede Folge $(x_n)_{n\geq 1}$ mit $x_n\in D\setminus \{x^*\}$ und $\lim_{n\to \infty} x_n=x^*$.
\\
Wir schreiben dann kurz
\[ \lim_{x\to x^*} f(x)= \infty. \]
\item[(ii)]
  Die Funktion $f$ heißt \notion{bestimmt divergent gegen $-\infty$ für $x \to x^*$}, wenn
  \[ \lim_{n\to \infty} f(x_n)= -\infty \]
  für jede Folge $(x_n)_{n\geq 1}$ mit $x_n\in D\setminus \{x^*\}$ und $\lim_{n\to \infty} x_n=x^*$.
\\
Wir schreiben dann kurz
\[ \lim_{x\to x^*} f(x)= -\infty. \]
\end{enumerate}
\end{definition}

\begin{remark}
Man definiert ebenso \notion{linksseitig bestimmte Divergenz} von $f$ über Folgen
 $(x_n)_{n\geq 1}$ mit $x_n<x^*$ und schreibt dann
 \[ \lim_{x\nearrow x^*} f(x)= \infty\quad \text{bzw.} \quad \lim_{x\nearrow x^*} f(x)=-\infty, \]
 und auch \notion{rechtsseitig bestimmte Divergenz} von $f$ über Folgen
 $(x_n)_{n\geq 1}$ mit $x_n>x^*$ und schreibt dann
 \[ \lim_{x\searrow x^*} f(x)= \infty\quad \text{bzw.} \quad \lim_{x\searrow x^*} f(x)=-\infty. \]
\end{remark}


\begin{remark}\label{rem:senkrechte-asymptote}
Anschaulich bedeutet die Eigenschaft $\lim_{x\nearrow x^*} f(x)= \infty$, dass der Graph von $f$
immer steiler wird, wenn man sich der Stelle $x^*$ von links nähert, und sich der Graph an die senkrechte Gerade $x=x^*$ "`anschmiegt"'.
Diese senkrechte Gerade $x=x^*$ nennt man daher \notion{(senkrechte) Asymptote} an den Graphen von $f$.

Entsprechend fällt der Graph von $f$ bei Annäherung an $x^*$ von links immer stärker, wenn 
$\lim_{x\nearrow x^*} f(x)= -\infty$. Auch in diesem Fall "`schmiegt"' sich der Graph an die senkrechte Gerade $x=x^*$ an.

Für die rechtsseitige bestimmte Divergenz gilt entsprechendes.
\end{remark}


\begin{example}
\begin{tabs*}
\tab{$f(x)=\frac{x+1}{x-1}$ für $x\to 1$}
Wir betrachten die Funktion $f:\R\setminus {\{1\}}\to \R, x\mapsto \frac{x+1}{x-1}$ und untersuchen
 das Verhalten bei $x^*=1$.\\
Für jede Folge $(x_n)_{n\geq 1}$ mit $\lim_{n\to \infty} x_n=1$ ist
\[ \lim_{n\to \infty} \frac{1}{f(x_n)}=\lim_{n\to \infty} \frac{x_n-1}{x_n+1}
= \frac{\lim_{n\to \infty} x_n-1}{\lim_{n\to \infty} x_n+1}=\frac{1-1}{1+1}=0.\]
Sind nun alle $x_n<1$, dann sind sie auch alle ab einem bestimmten $N$ zwischen $-1$ und $1$ und daher $f(x_n)<0$ für alle $n\geq N$. 
Nach den \ref[bestimmte-divergenz][Rechenregeln mit bestimmter Divergenz]{sec:rechenregeln} gilt
daher für gegen $1$ konvergente Folgen $(x_n)_{n\geq 1}$ mit $x_n<1$:
\[ \lim_{n\to \infty} f(x_n)=-\infty. \]
Also ist $ \lim_{x\nearrow 1} f(x)=-\infty.$

Entsprechend ist $f(x)>0$ für alle $x>1$ und man erhält entsprechend
\[ \lim_{x\searrow 1} f(x)= \infty. \]

Der Graph (schwarz) hat also bei $x=1$ eine senkrechte Asymptote (grün).

\begin{genericJSXVisualization}[550][800]{example_a}
 		\begin{variables}
 			\function{f}{real}{(x+1)/(x-1)}  
			\point{O}{real}{1,0} 			
			\point{P}{real}{1,1} 			
 			\line{l}{real}{var(O),var(P)}
 			
 		\end{variables}
 		\color{l}{#00CC00}
   \color{f}{black}
 
 		\begin{canvas}
 			\plotSize{400,300}
 			\plotLeft{-4}
 			\plotRight{4}
 			\plot[coordinateSystem]{f,l }
 		\end{canvas}
 		% \text{Die schwarze Kurve ist der Graph der Funktion $f:\R\setminus \{1\}\to \R$ mit 
 		% $f(x)=\frac{x+1}{x-1}$. Die grüne Gerade ist die senkrechte Asymptote $x=1$.
 		%}
\end{genericJSXVisualization}
\tab{$f(x)=\frac{x+1}{(x-1)^2}$ für $x\to 1$} Wir betrachten die Funktion $f:\R\setminus {\{1\} }\to \R, x\mapsto \frac{x+1}{(x-1)^2}$ und untersuchen
 das Verhalten bei $x^*=1$.\\
Für jede Folge $(x_n)_{n\geq 1}$ mit $\lim_{n\to \infty} x_n=1$ ist
\[ \lim_{n\to \infty} \frac{1}{f(x_n)}=\lim_{n\to \infty} \frac{(x_n-1)^2}{x_n+1}
= \frac{(\lim_{n\to \infty} x_n-1)^2}{\lim_{n\to \infty} x_n+1}=\frac{(1-1)^2}{1+1}=0.\]
Für $x>-1$ (und $x\neq 1$) ist auch $f(x)>0$. Da ab einem bestimmten $N$ alle $x_n$ größer als $-1$ sind (da die Folge gegen $1$ konvergiert),
gilt nach den \ref[bestimmte-divergenz][Rechenregeln mit bestimmter Divergenz]{sec:rechenregeln} 
\[ \lim_{n\to \infty} f(x_n)=\infty. \]
Also ist $ \lim_{x\to 1} f(x)=\infty.$

Der Graph (schwarz) hat also bei $x=1$ eine senkrechte Asymptote (grün) und er schmiegt sich von beiden Seiten gegen $\infty$ an.

\begin{genericJSXVisualization}[550][800]{example_b}
 		\begin{variables}
 			\function{f}{real}{(x+1)/(x-1)^2}  
			\point{O}{real}{1,0} 			
			\point{P}{real}{1,1} 			
 			\line{l}{real}{var(O),var(P)}
 			
 		\end{variables}
 		\color{l}{#00CC00}
   \color{f}{black}
   
 		\begin{canvas}
 			\plotSize{400,300}
 			\plotLeft{-4}
 			\plotRight{4}
 			\plot[coordinateSystem]{f,l }
 		\end{canvas}
 		% \text{Die schwarze Kurve ist der Graph der Funktion $f:\R\setminus \{1\}\to \R$ mit 
 		% $f(x)=\frac{x+1}{(x-1)^2}$. Die grüne Gerade ist die senkrechte Asymptote $x=1$.
 		%}
\end{genericJSXVisualization}
\end{tabs*}
\end{example}

\begin{example}
Wir betrachten $f:\R\setminus\{1\}\to\R$, $x\mapsto \exp(\frac{1}{x-1})$.
Das Verhalten in der Nähe der Definitionslücke $x^\ast=1$ ergibt sich aus dem Verhalten der inneren Funktion 
$x\mapsto\frac{1}{x-1}$ und dem der Exponentialfunktion 
wie folgt:
Es ist $\lim_{x\nearrow 1}\frac{1}{x-1}=-\infty$, und weil die Exponentialfunktion stetig ist und abklingt 
(also $\exp(y)\to 0$ für $y\to -\infty$), gilt
\[\lim_{x\nearrow 1}\exp(\frac{1}{x-1})=\exp(\lim_{x\nearrow 1}\frac{1}{x-1})=0.\]
Dagegen ist $\lim_{x\searrow 1}\frac{1}{x-1}=+\infty$. Aus der Stetigkeit der Exponentialfunktion und ihrem Wachstum 
(also $\exp(y)\to \infty$ für $y\to \infty$) folgt nun
\[\lim_{x\searrow 1}\exp(\frac{1}{x-1})=\exp(\lim_{x\searrow 1}\frac{1}{x-1})=+\infty.\]
Die Funktion $f$ kann also in $x^\ast=1$ durch $f(1)=0$ \emph{linksseitig} stetig fortgesetzt werden. Sie hat dennoch
eine senkrechte Asymptote in $x^\ast=1$, an die sich der rechte Zweig des Graphen immer mehr anschmiegt.
\begin{genericJSXVisualization}[550][800]{example_b}
 		\begin{variables}
 			\function{f}{real}{exp(1/(x-1))}  
			\point{O}{real}{1,0} 			
			\point{P}{real}{1,1} 			
 			\line{l}{real}{var(O),var(P)}
 			
 		\end{variables}
 		\color{l}{#00CC00}
   \color{f}{black}
 
 		\begin{canvas}
 			\plotSize{400,300}
 			\plotLeft{-4}
 			\plotRight{4}
 			\plot[coordinateSystem]{f,l }
 		\end{canvas}
 		%\text{Die schwarze Kurve ist der Graph der Funktion $f:\R\setminus \{1\}\to \R$ mit 
 		%$f(x)=\frac{x+1}{(x-1)^2}$. Die grüne Gerade ist die senkrechte Asymptote $x=1$.
 		%}
\end{genericJSXVisualization}
\end{example}

Wie die obigen Beispiele schon vermuten lassen, gibt es analog zu den 
\ref[bestimmte-divergenz][Rechenregeln für Folgen mit bestimmter Divergenz]{sec:rechenregeln} auch Rechenregeln für 
die bestimmte Divergenz von Funktionswerten $\lim_{x\to x^*} f(x)$.

\begin{rule}
Es sei $f:D\to \R$ eine reelle Funktion,
und es sei $x^\ast\in\overline{D}\setminus D\subset\R$ ein Randpunkt.

Dann gilt:
\begin{enumerate}
\item[(i)] $ \lim_{x\nearrow x^*} f(x)=+\infty  $
  genau dann, wenn $\,\lim_{x\nearrow x^*} \frac{1}{f(x)}=0\,$ und wenn die Funktionswerte in einem kleinen Intervall links von $x^\ast$ positiv sind,
  d.h. wenn es $\delta>0$ gibt so, dass $f(x)>0$ für alle $x\in (x^*-\delta; x^*)\cap D$. 
%\[ \exists \delta>0 \text{ mit }\quad f(x)>0\text{ für alle } x\in (x^*-\delta; x^*)\cap D. \]
\item[(ii)] $  \lim_{x\nearrow x^*} f(x)=-\infty$ 
  genau dann, wenn $\,\lim_{x\nearrow x^*} \frac{1}{f(x)}=0\,$ und wenn die Funktionswerte in einem kleinen Intervall links von $x^\ast$ negativ sind,
  d.h. wenn es $\delta>0$ gibt so, dass $f(x)<0$ für alle $x\in (x^*-\delta; x^*)\cap D$.
%\[ \exists \delta>0\text{ mit }\quad f(x)<0\text{ für alle } x\in (x^*-\delta; x^*)\cap D. \]
\item[(iii)] $  \lim_{x\searrow x^*} f(x)=\infty $
  genau dann, wenn $\,\lim_{x\searrow x^*} \frac{1}{f(x)}=0\,$ und wenn die Funktionswerte in einem kleinen Intervall rechts von $x^\ast$ positiv sind,
  d.h. wenn es $\delta>0$ gibt so, dass $f(x)>0$ für alle $x\in (x^*; x^*+\delta)\cap D$.
%\[ \exists \delta>0\text{ mit }\quad f(x)>0\text{ für alle } x\in (x^*; x^*+\delta)\cap D. \]
\item[(iv)]$  \lim_{x\searrow x^*} f(x)=-\infty  $ genau dann, wenn  $\,\lim_{x\searrow x^*} \frac{1}{f(x)}=0\,$ und wenn
die Funktionswerte in einem kleinen Intervall rechts von $x^\ast$ negativ sind,
d.h. wenn es $\delta>0$ gibt so, dass $f(x)<0$ für alle $x\in (x^*; x^*+\delta)\cap D$.
%\[ \exists \delta>0\text{ mit }\quad f(x)<0\text{ für alle } x\in (x^*; x^*+\delta)\cap D. \]
\end{enumerate}  
\end{rule}
\begin{example}
Für $f:\R\setminus\{0\}\to\R$, $x\mapsto \frac{1}{x^5}$
erhalten wir wegen $\lim_{x\to\infty}x^5=+\infty$ und $\lim_{x\to-\infty} x^5=-\infty$
\[\lim_{x\searrow 0}\frac{1}{x^5}=+\infty\quad \text{ \lang{de}{ sowie } \lang{en}{ and }}\quad
\lim_{x\nearrow 0}\frac{1}{x^5}=-\infty.\]
Die Gerade $x=0$ ist eine senkrechte Asymptote des Funktionsgraphen von $f$. Die Funktion divergiert bestimmt, und zwar von links gegen $-\infty$ und von rechts gegen $+\infty$.
\end{example}




\begin{quickcheck}
    \field{real}
    \type{input.number}
    \begin{variables}
        \number{n}{0}
    \end{variables}
    \text{Berechnen Sie den Grenzwert der Funktion $f(x)=x^2\ln(x^2)$. Es ist 
    $\lim_{x\to0}f(x)=$\ansref.
    \\
    Tipp: Substituieren Sie $x=e^y$, passen Sie "$x\to 0$" im Limes entsprechend an und berücksichtigen Sie,
    dass für jedes Polynom $p(x)$ gilt $\lim_{x\to\infty}\frac{p(x)}{e^{cx}}=0$ für $c>0$.\\
    }
    \begin{answer}
        \solution{n}
    \end{answer}
    \explanation{$\lim_{x\to0}x^2\ln(x^2)=\lim_{y\to-\infty}(e^y)^2\ln((e^y)^2)=
    \lim_{y\to-\infty}e^{2y}\ln(e^{2y})=\lim_{y\to-\infty}e^{2y}\cdot2y=0$}

\end{quickcheck}

In den bisherigen Beispielen waren die Funktionen in Definitionslücken stetig fortsetzbar oder hatten
senkrechte Asymptoten. Natürlich gibt es auch Funktionen, die sich an Lücken im Definitionsbereich nicht so schön verhalten.

\begin{example}
Wir haben bereits in einem \ref[content_31_grenzwerte_von_funktionen][früheren Beispiel]{ex:Funktionsgrenzwerte}, dass
für die Funktion 
$f(x)=\sin\left(\frac{1}{x}\right)$ kein Grenzwert gegen $x^*=0$ existiert, weil verschiedene Nullfolgen  Funktionswertfolgen produzieren, die
gegen verschiedene Grenzwerte oder gar nicht konvergieren. Demnach gibt es keine Möglichkeit, die Funktion in $x^\ast=0$ stetig fortzusetzen oder
eine Asymptotik abzuleiten.\\
Die folgende Abbildung stellt $f(x)=\sin\left(\frac{1}{x}\right)$ grafisch dar. 
Setzen wir $y=\frac{1}{x}$, dann liegt zwischen $y$ und $y+2\pi$
genau eine volle Schwingung von $\sin(y)$. Nun entspricht aber die Änderung um den Summanden $2\pi$ in $y$ der Änderung in $x$ um den Faktor $\frac{1}{1+2\pi}$.
Also werden die Funktionswerte $[-1;1]$
in der Nähe von $x^\ast=0$ auf einen beliebig schmalen Abszissenbereich zusammengedrängt: $f(x)$ nimmt in jeder noch so 
 kleinen Umgebung des Nullpunktes jeden Wert zwischen $-1$ und $+1$ an.
% der Funktionswerte (Ordinate) der Hyperbel $x'=\frac{1}{x}$ eine Abnahme der $x$-Werte (Abszisse), die zugleich
% mit $x$ selbst gegen 0 konvergiert, und das heißt, daß eine volle Schwingung von $\sin(\frac{1}{x})$
% in der Nähe von $x=0$ auf einen beliebig schmalen Abszissenbereich zusammengedrängt wird. $f(x)$ nimmt in jeder noch so 
% kleinen Umgebung des Nullpunktes jeden Wert zwischen $-1$ und $+1$ an und kann sich daher
% keinem Grenzwert nähern.
% \begin{center}
% \image{T210_Sine1x1}
% \end{center}
\begin{center}
\begin{genericJSXVisualization}[550][800]{example_sin_1x}
 		\begin{variables}
 			\function{f}{real}{sin(1/x)}  
			%\point{O}{real}{1,0} 			
			%\point{P}{real}{1,1} 			
 			%\line{l}{real}{var(O),var(P)}
 			
 		\end{variables}
 		%\color{l}{#00CC00}
 
 		\begin{canvas}
 			\plotSize{400,300}
 			\plotLeft{-4}
 			\plotRight{4}
 			\plot[coordinateSystem]{f,l }
 		\end{canvas}
 		%\text{Die schwarze Kurve ist der Graph der Funktion $f:\R\setminus \{1\}\to \R$ mit 
 		%$f(x)=\frac{x+1}{(x-1)^2}$. Die grüne Gerade ist die senkrechte Asymptote $x=1$.
 		%}
\end{genericJSXVisualization}
\end{center}

Anders verhält sich allerdings die Funktion $g(x)=x\cdot\sin(\frac{1}{x})$. Auch sie hat eine Definitionslücke in $x^*=0$, 
besitzt aber  den Grenzwert $\lim_{x \to  0}f(x)=0$. Dazu betrachten wir
\[{\vert g(x)-0\vert}={\vert x\cdot \sin\left(\frac{1}{x}\right)\vert}=
{\vert x\vert}\cdot {\vert\sin\left(\frac{1}{x}\right)\vert}\leq{\vert x\vert}\cdot 1={\vert x\vert} \]
Für jede Nullfolge $(x_n)_{n\in\N}$ mit $x_n\neq 0$ für alle $n\in\N$ gilt also
\[0\leq {\vert g(x_n)\vert}\leq {\vert x_n\vert.} \]
Nach dem \ref[content_16_konvergenzkriterien][Sandwich-Lemma]{thm:sandwich} gilt somit auch $\lim_{n\to\infty}{\vert g(x_n)\vert=0}$. 
Und weil eine Folge genau dann gegen Null
konvergiert, wenn es die Folge ihrer Beträge tut, folgt $\lim_{n\to\infty}g(x_n)=0$.
Weil dies für beliebige solche Folgen gilt, ist $\lim_{x\to 0}g(x)=0$.
Die Funktion lässt sich also in $x^\ast=0$ stetig fortsetzen.
Die Multiplikation mit $x$ erzwingt hier also den Verlauf durch den Nullpunkt.
% \begin{center}
% \image{T210_xSine1x1}
% \end{center}
\begin{center}
\begin{genericJSXVisualization}[550][800]{example_sin_1x}
 		\begin{variables}
 			\function{f}{real}{x*sin(1/x)}  	
 		\end{variables}
 		%\color{l}{#00CC00}
 
 		\begin{canvas}
 			\plotSize{400,300}
 			\plotLeft{-4}
 			\plotRight{4}
 			\plot[coordinateSystem]{f,l }
 		\end{canvas}
 		%\text{Die schwarze Kurve ist der Graph der Funktion $f:\R\setminus \{1\}\to \R$ mit 
 		%$f(x)=\frac{x+1}{(x-1)^2}$. Die grüne Gerade ist die senkrechte Asymptote $x=1$.
 		%}
\end{genericJSXVisualization}
\end{center}
\end{example}
%%
%%
%%
\section{Verhalten gegen Unendlich}
Nun befassen wir uns mit dem Verhalten reeller Funktionen an den uneigentlichen \glqq Rändern\grqq{} $\pm\infty$ von Definitionsbereichen.
Wir haben bereits \ref[content_31_grenzwerte_von_funktionen][Funktionsgrenzwerte für $x$ gegen $\pm\infty$]{rem:funktionsgrenzwert}
 definiert.
 Wir wiederholen diese und fügen den Begriff der bestimmten Divergenz hinzu.
\begin{definition}
\begin{enumerate}
\item[(i)]
Sei $f:D\to \R$ eine Funktion, deren Definitionsbereich $D$ nach oben (bzw. nach unten) unbeschränkt ist.
%und es gebe mindestens eine Folge $(x_n)_{n\geq 1}$ mit $x_n\in D$, die bestimmt gegen $\infty$ divergiert, d.h. $D$ ist nach oben unbeschränkt.
\\
Die Funktion $f$ heißt \notion{konvergent gegen $L \in \R$ für $x \to \infty$ (bzw. $x \to -\infty$)}, wenn
  \[ \lim_{n\to \infty} f(x_n)= L \]
  für alle Folgen $(x_n)_{n\geq 1}$ in $D$ mit $\lim_{n\to \infty} x_n=\infty$ (bzw. $=-\infty$) erfüllt ist.
In diesem Fall schreiben wir kurz
\[ \lim_{x\to \infty} f(x)= L \quad \text{ bzw. }\quad \lim_{x\to -\infty} f(x)= L . \]
%%
\item[(ii)] Sei $f:D\to \R$ eine Funktion, deren Definitionsbereich $D$ nach oben (bzw. nach unten) unbeschränkt ist.
\\
Die Funktion $f$ heißt \notion{bestimmt divergent  gegen $+\infty$ (oder $-\infty$) für $x \to \infty$ (bzw. $x \to -\infty$)}, 
wenn
  \[ \lim_{n\to \infty} f(x_n)= +\infty \:\text{ (oder }-\infty) \]
  für alle Folgen $(x_n)_{n\geq 1}$ in $D$ mit $\lim_{n\to \infty} x_n=\infty$ (bzw. $=-\infty$) erfüllt ist.
In diesem Fall schreiben wir kurz
\[ \lim_{x\to \infty} f(x)= +\infty \:\text{ (oder }-\infty) \quad \text{ bzw. }\quad \lim_{x\to -\infty} f(x)= +\infty\:\text{ (oder }-\infty) . \]
\end{enumerate}
\end{definition}

% \begin{remark}
% Auch hier gibt es wieder ein $\epsilon$-Kriterium. Da die Nähe zu $\infty$ aber nicht durch eine $\delta$-Umgebung gegeben ist, sondern durch Schranken, die überschritten werden, lautet die entsprechende Definition:

% Sei $f:D\to \R$ eine Funktion, für welche $D$ nach oben unbeschränkt ist. Die Funktion $f$ ist konvergent gegen $L \in \R$ für $x \to \infty$, wenn es für jedes $\epsilon>0$ ein $R_\epsilon\in \R$ gibt,
% sodass
% \[ |f ( x ) - L| < \epsilon\quad \text{für alle }x\in D \text{ mit }x > R_\epsilon. \]

% \\

% Entsprechend gilt für $-\infty$:

% Sei $f:D\to \R$ eine Funktion, für welche $D$ nach unten unbeschränkt ist. Die Funktion $f$ ist konvergent gegen $L \in \R$ für $x \to -\infty$, wenn es für jedes $\epsilon>0$ ein $R_\epsilon\in \R$ gibt,
% sodass
% \[ |f ( x ) - L| < \epsilon\quad \text{für alle }x\in D \text{ mit }x < R_\epsilon. \]
% \end{remark}

\begin{remark}\label{rem:waagerechte-asymptote}
\begin{enumerate}
\item[(i)]
Anschaulich bedeutet die Existenz des Grenzwerts $\lim_{x\to \infty} f(x)= L$, dass sich der Graph von $f$ für
immer größer werdende $x$-Werte immer mehr der waagerechten Geraden $y=L$ annähert.\\
Diese waagerechte Gerade $y=L$ nennt man daher \notion{(waagerechte) Asymptote} an den Graphen von $f$.

Die  Existenz des Grenzwerts $\lim_{x\to -\infty} f(x)= L$ bedeutet ganz entsprechend, dass sich der Graph von $f$ für
immer kleiner werdende $x$-Werte immer mehr der waagerechten Geraden $y=L$ annähert.\\
Auch in diesem Fall nennt man  die Gerade $y=L$ \notion{(waagerechte) Asymptote} an den Graphen von $f$.
\item[(ii)] Die bestimmte Divergenz gegen $+\infty$ für $x\to+\infty$ (bzw. für $x\to -\infty$) bedeutet, dass der Graph über jede Grenze
hinaus wächst.\\
Analog bedeutet die bestimmte Divergenz gegen $-\infty$ für $x\to+\infty$ (bzw. für $x\to -\infty$), 
dass der Graph unter jede Grenze fällt.
\end{enumerate}
\end{remark}

\begin{example}
\begin{tabs*}
\tab{$e^x$}
Es ist bereits bekannt, dass
\[\lim_{x\to-\infty}e^x=0 \quad\text{ sowie }\quad \lim_{x\to+\infty}e^x=+\infty\:.\]
\begin{center}
\begin{genericJSXVisualization}[550][800]{example_exp_asymp}
 		\begin{variables}
 			\function{f}{real}{e^x}  	
    	\point{O}{real}{0,0} 			
			\point{Q}{real}{4,0}
     \line{m}{real}{var(Q),var(O)}
 		\end{variables}
 		\color{m}{blue}
   \color{f}{black}
 
 		\begin{canvas}
 			\plotSize{400,300}
      \plotTop{4.4}
 			\plotLeft{-4}
 			\plotRight{4}
 			\plot[coordinateSystem]{f,m }
 		\end{canvas}
 		%\text{Die schwarze Kurve ist der Graph der Funktion $f:\R\setminus \{1\}\to \R$ mit 
 		%$f(x)=\frac{x+1}{(x-1)^2}$. Die grüne Gerade ist die senkrechte Asymptote $x=1$.
 		%}
\end{genericJSXVisualization}
\end{center}
\tab{$\frac{4x^3+x^2+1}{x^3-2}$}
Für das Verhalten von $f:\R\setminus\{\sqrt[3]{2}\}\to\R$, $x\mapsto \frac{4x^3+x^2+1}{x^3-2}$, für $x\to\pm\infty$ 
bemerken wir zunächst,
dass $\lim_{x\to\pm\infty}\frac{1}{x^k}=0$ für alle $k>0$.
Somit gilt mit den \ref[bestimmte-divergenz][Rechenregeln für bestimmte Divergenz]{sec:rechenregeln}
\[\lim_{x\to\infty}f(x)=
\lim_{x\to\infty}\frac{x^3}{x^3}\frac{4+\frac{1}{x}+\frac{1}{x^3}}{1-\frac{2}{x^3}}=
\lim_{x\to\infty}\frac{4+\frac{1}{x}+\frac{1}{x^3}}{1-\frac{2}{x^3}}=\frac{4}{1}=4,\]
und ganz analog
\[\lim_{x\to-\infty}\frac{4x^3+x^2+1}{x^3-2}=4.\]
Der Funktionsgraph (schwarz) hat die waagerechte Asymptote $y=4$ (blau).
Zusätzlich hat die Funktion eine senkrechte Asymptote (grün) in $x=\sqrt[3]{2}$ mit folgender links- bzw. rechtsseitiger bestimmter Divergenz (beachte, dass der Zähler für $x>0$ stets größer als Null ist)
\[\lim_{x\nearrow \sqrt[3]{2}}\frac{4x^3+x^2+1}{x^3-2}=-\infty \quad \text{ und }\quad
\lim_{x\searrow \sqrt[3]{2}}\frac{4x^3+x^2+1}{x^3-2}=+\infty .
\]
\begin{center}
\begin{genericJSXVisualization}[550][800]{example_gebro_rat_1}
 		\begin{variables}
 			\function{f}{real}{(4x^3+x^2+1)/(x^3-2)}  	
    	\point{O}{real}{cbrt(2),0} 			
			\point{P}{real}{cbrt(2),1} 			
       \point{Q}{real}{0,4}
    \point{R}{real}{1,4}
    \line{l}{real}{var(O),var(P)}
    \line{m}{real}{var(Q),var(R)}
 		\end{variables}
 		\color{l}{#00CC00}
   \color{m}{blue}
   \color{f}{black}
 
 		\begin{canvas}
 			\plotSize{400,300}
      \plotTop{4.4}
 			\plotLeft{-4}
 			\plotRight{4}
 			\plot[coordinateSystem]{f,l,m }
 		\end{canvas}
 		%\text{Die schwarze Kurve ist der Graph der Funktion $f:\R\setminus \{1\}\to \R$ mit 
 		%$f(x)=\frac{x+1}{(x-1)^2}$. Die grüne Gerade ist die senkrechte Asymptote $x=1$.
 		%}
\end{genericJSXVisualization}
\end{center}
\tab{$ \frac{|x-2|}{x^2-2x}$}
Sei $f:\R\setminus \{0; 2\}\to \R$ gegeben durch
\[ f(x)= \frac{|x-2|}{x^2-2x} \]
für alle $x\in \R\setminus \{0; 2\}$.

Ist $(x_n)_{n\geq 1}$ eine Folge in $\R\setminus \{0; 2\}$, die bestimmt divergent gegen $\infty$ ist, also mit $\lim_{n\to \infty} x_n=\infty$, so gibt es ein $N\in \N$ mit $x_n>2$ für alle $n\geq N$.\\
Dementsprechend gilt für alle $n\geq N$: $f(x_n)=\frac{x-2}{x^2-2x}=\frac{1}{x}$.

Mit den \ref[bestimmte-divergenz][Rechenregeln mit bestimmter Divergenz]{sec:rechenregeln} gilt daher:
\[ \lim_{n\to \infty} f(x_n)= \lim_{n\to \infty} \frac{1}{x_n}=0. \]

Da dies für alle gegen $\infty$ bestimmt divergenten Folgen $(x_n)_{n\geq 1}$ gilt, ist somit
\[ \lim_{x\to \infty} f(x)= 0.\]
Die Gerade $y=0$ ist also eine waagerechte Asymptote an den Graphen für $x\to \infty$. \\ \\


Folgen $(x_n)_{n\geq 1}$ in $\R\setminus \{0; 2\}$, die bestimmt divergent gegen $-\infty$ sind,
erfüllen $x_n<0$ für alle $n\geq N$ bei geeignetem festen $N\in \N$ und dementsprechend gilt:
\[ f(x_n)=\frac{-(x_n-2)}{x_n^2-2x_n}=-\frac{1}{x_n} \quad \text{ für alle }n\geq N.\]

Also ist auch hier:
\[ \lim_{n\to \infty} f(x_n)= \lim_{n\to \infty} -\frac{1}{x_n}=0. \]

Da dies für alle gegen $-\infty$ bestimmt divergenten Folgen $(x_n)_{n\geq 1}$ gilt, ist somit
\[ \lim_{x\to -\infty} f(x)= 0.\]
Die Gerade $y=0$ ist also auch eine waagerechte Asymptote (blau) an den Graphen (schwarz) für $x\to -\infty$.

 	\begin{genericJSXVisualization}[550][800]{waagerecht_1}
 		\begin{variables}
 			\function{f}{real}{|x-2|/(x^2-2*x)}  
			\point{O}{real}{0,0} 			
			\point{P}{real}{1,0} 			
 			\line{l}{real}{var(O),var(P)}
       	\point{Q}{real}{0,0} 			
			\point{R}{real}{0,1} 	
     \line{m}{real}{var(Q),var(R)}
 			
 		\end{variables}
 		\color{l}{blue}
     \color{m}{#00CC00}
      \color{f}{black}
 
 		\begin{canvas}
 			\plotSize{400,300}
 			\plotLeft{-4},
 			\plotRight{4}
 			\plot[coordinateSystem]{f,l,m }
 		\end{canvas}
 \end{genericJSXVisualization}
 In der Definitionslücke $x=0$ hat die Funktion eine senkrechte Asymptote (grün). Es gilt
  \[\lim_{x\nearrow 0}f(x)=\lim_{x\nearrow 0}-\frac{1}{x}=+\infty \quad\text{ sowie }
 \lim_{x\searrow 0}f(x)=\lim_{x\searrow 0}-\frac{1}{x}=-\infty .\]
 Zusätzlich hat die Funktion eine Definitionslücke in $x=2$. Es gilt
 \[\lim_{x\nearrow 2}f(x)=\lim_{x\nearrow 2}-\frac{1}{x}=-\frac{1}{2} \quad\text{ sowie }
 \lim_{x\searrow 2}f(x)=\lim_{x\searrow 2}\frac{1}{x}=+\frac{1}{2} .\]
 \end{tabs*}
\end{example}
Natürlich gibt es auch Funktionen, die für $x\to\pm\infty$ weder konvergieren noch bestimmt divergieren:
\begin{example}
\begin{tabs*}
\tab{$\cos$ und $\sin$}
Die Funktionen $\cos(x)$ und $\sin(x)$ sind $2\pi$-periodisch. Sie nehmen also auch für $\vert x\vert$ größer als jede beliebige Grenze jeden Wert im Intervall $[-1;1]$ an.
Demnach kann weder ein Grenzwert für $x\to\pm\infty$ existieren, noch kann bestimmte Divergenz für $x\to\pm\infty$ vorliegen.
\end{tabs*}
\end{example}
\begin{quickcheck}
    \field{real}
        \type{input.number}
        \begin{variables}
            \drawFromSet{a}{2,3,4,5}
            \function[calculate]{b}{2*a}
            \number{n}{0}
        \end{variables}
        \text{Berechnen Sie den Grenzwert  $\:\lim_{x\to -\infty}\frac{\var{b}}{\var{a}+\sqrt{|x|}}=$\ansref
        }
        \begin{answer}
            \solution{n}
        \end{answer}
        \explanation{Für jede Folge $(x_n)_{n\in\N}$ mit $\lim_{n\to\infty}x_n=-\infty$  gilt 
        $\lim_{n\to\infty}\sqrt{\vert x_n\vert}=\infty$. Somit gilt nach den Rechenregeln für bestimmt Divergenz 
        $\lim_{n\to\\infty}\frac{\var{b}}{\var{a}+\sqrt{|x_n|}}=0$.
        }
\end{quickcheck}
\section{Gebrochen rationale Funktionen}\label{sec:asymptotics_rational}
Für gebrochen rationale Funktionen $r(x)=\frac{p(x)}{q(x)}$ mit Polynomen $p(x)$ und $q(x)$
lassen sich allgemeine Aussagen über  Asymptotiken treffen. 
Diese Asymptotiken sind sehr nützlich dafür, Funktionsgraphen  zu skizzieren.
\begin{rule}[Verhalten an Definitionslücken]\label{rule:rat_fct_def_gaps}
Eine rationale Funktion $r(x)=\frac{p(x)}{q(x)}$ mit Polynomen $p(x)$ und $q(x)$
hat genau an den Nullstellen des Nennerpolynoms $q(x)$ Definitionslücken.
Für jede solche Definitionslücke $x_\ast$ gibt es Polynome $p_\ast(x)$ und $q_\ast(x)$ 
und eine Zahl
$s\in\Z$ so,
dass $p_\ast(x_\ast)\neq 0$ und $q_\ast(x_\ast)\neq 0$ und 
\[r(x)=\frac{p(x)}{q(x)}=\frac{p_\ast(x)}{q_\ast(x)}\cdot (x-x_\ast)^s.\]
Dann gilt:
\begin{itemize}
\item[(i)] Ist $s\geq 0$ so lässt $r(x)$ sich in $x_\ast$ stetig fortsetzen durch den Wert
\[\frac{p_\ast(x_\ast)}{q_\ast(x_\ast)}\cdot (x_\ast-x_\ast)^s=\begin{cases}0&\text{ für } s>0\\
                                                              \frac{p_\ast(x_\ast)}{q_\ast(x_\ast)}&\text{ für } s=0
                                                              \end{cases}.\]
In diesem Fall nennt man $x_\ast$ eine \notion{hebbare Singularität} von $r(x)$.
\item[(ii)] Ist $s<0$, dann nennt man $x_\ast$ eine \notion{Polstelle} von $r(x)$. 
Die Gerade $x=x_\ast$ ist dann eine senkrechte Asymptote des Graphen.
\end{itemize}
\end{rule}
\begin{proof*}
\begin{incremental}[\initialsteps{0}]
\step
Durch \ref[content_10_polynomdivision][Polynomdivision]{sec:poly-div-factoring} kann man den Linearfaktor $(x-x_\ast)$ 
so oft von $q(x)$ abspalten, wie es der Vielfachheit $l\in\N$ der Nullstelle $x_\ast$ entspricht.
Es gibt also ein Polynom $q_\ast(x)$ mit $q_\ast(x_\ast)\neq 0$ und $q(x)=q_\ast(x)\cdot(x-x_\ast)^l$.
\\
Ebenso zerlegt man $p(x)=p_\ast(x)\cdot(x-x_\ast)^k$ in ein Polynom $p_\ast(x)$ mit $p_\ast(x_\ast)\neq 0$
und die $k$-te Potenz des Linearfaktors $(x-x_\ast)$. Wenn schon $p(x_\ast)\neq 0$ gilt,
 dann ist $k=0$ und $p_\ast(x)=p(x)$.
Somit ist die Darstellung
\[r(x)=\frac{p_\ast(x)}{q_\ast(x)}\cdot (x-x_\ast)^s\]
gezeigt mit $s:=k-l\in\Z$.
Weil nun $\lim_{x\to x_\ast}\frac{p_\ast(x)}{q_\ast(x)}=\frac{p_\ast(x_\ast)}{q_\ast(x_\ast)}\neq 0$ existiert,
hängt nach den Grenzwertregeln die Existenz des Grenzwerts $\lim_{x\to x_\ast}r(x)$ nur ab von der Existenz von $\lim_{x\to x_\ast}(x-x_\ast)^s$.
\\
Ist $s\geq 0$, dann ist $(x-x_\ast)^s$ ein Polynom, und der Grenzwert existiert: 
Für $s=0$ ist $\lim_{x\to x_\ast}(x-x_\ast)^s=1$, und für $s>0$ ist $\lim_{x\to x_\ast}(x-x_\ast)^s=0$.
Wiederum nach den Grenzwertsätzen folgt die Existenz des Grenzwerts
\[\lim_{x\to x_\ast}r(x)=\frac{p_\ast(x_\ast)}{q_\ast(x_\ast)}\cdot \lim_{x\to x_\ast}(x-x_\ast)^k=
\begin{cases} \frac{p_\ast(x_\ast)}{q_\ast(x_\ast)} &\text{ falls } s=0\\
0&\text{ falls } s>0\end{cases}.\]
\\
Ist $s<0$, so setzen wir $y=x-x_\ast$ und erhalten $\lim_{x\nearrow x_\ast}(x-x_\ast)^s=\lim_{y\nearrow 0}\frac{1}{y^{\vert s\vert}}=(-1)^s\cdot\infty$
und $\lim_{x\searrow x_\ast}(x-x_\ast)^s=\lim_{y\searrow 0}\frac{1}{y^{\vert s\vert}}=\infty$. Also ist $x=x_\ast$ senkrechte Asymptote.
Das Vorzeichen der bestimmen Divergenz von $r(x)$ ist zusätzlich abhängig vom Vorzeichen des Faktors $\frac{p_\ast(x_\ast)}{q_\ast(x_\ast)}$.
\end{incremental}
\end{proof*}
\begin{rule}[Verhalten gegen Unendlich]\label{rule:rat_fun_infty}
Es seien $p(x)=a_nx^n+a_{n-1}x^{n-1}+\ldots+a_1x+a_0$ und $q(x)=b_mx^m+b_{m-1}x^{m-1}+\ldots+b_1x+b_0$ reelle Polynome vom Grad $n$ bzw. $m$. Insbesondere seien also 
die höchsten Koeffizienten $a_n$ und $b_m$ ungleich Null.
Es sei weiter 
\[
d:=\text{grad } p(x) - \text{grad } q(x)=n-m
\]
die Differenz der Polynomgrade. Dann gilt für die gebrochen rationale Funktion $r(x)=\frac{p(x)}{q(x)}$:
\begin{itemize}
\item[(i)] Ist $d>0$, dann divergiert $r(x)$ bestimmt für $x\to\pm\infty$. Genauer gilt
\[\lim_{x\to\infty}r(x)=\text{VZ}\left(\frac{a_n}{b_m}\right)\cdot\infty\quad\text{ und }\quad \lim_{x\to\infty}r(x)=\text{VZ}\left(\frac{a_n}{b_m}\right)\cdot (-1)^d\cdot\infty.\]
\item[(ii)]  Ist $d=0$, dann konvergiert $r(x)$ für $x\to\pm\infty$ gegen den Grenzwert
\[\lim_{x\to\infty}r(x)=\frac{a_n}{b_m}.\]
\item[(iii)] Ist $d<0$, dann konvergiert $r(x)$ für $x\to\pm\infty$ gegen den Grenzwert
\[\lim_{x\to\infty}r(x)=0.\]
\end{itemize}
In Fall (ii) und (iii) hat der Funktionsgraph eine waagerechte Asymptote in Höhe des Grenzwerts.
\end{rule}
\begin{proof*}
\begin{incremental}[\initialsteps{0}]
\step
Wir schreiben die Funktion $r(x)$ um zu
\[
r(x)=\frac{x^n}{x^m}\cdot\frac{a_n+\frac{a_{n-1}}{x}+\ldots+\frac{a_{1}}{x^{n-1}}+\frac{a_{0}}{x^n}}
{b_m+\frac{b_{m-1}}{x}+\ldots+\frac{b_{1}}{x^{m-1}}+\frac{b_{0}}{x^m}}.\]
 Wir beachten $\lim_{x\to\pm\infty}\frac{1}{x}=0$ und erhalten unter Beachtung der Grenzwertregeln
\[
\lim_{x\to\pm\infty}\frac{a_n+\frac{a_{n-1}}{x}+\ldots+\frac{a_{1}}{x^{n-1}}+\frac{a_{0}}{x^n}}
{b_m+\frac{b_{m-1}}{x}+\ldots+\frac{b_{1}}{x^{m-1}}+\frac{b_{0}}{x^m}}=\frac{a_n}{b_m}.
\]
Somit hängt die Existenz der Grenzwerte $\lim_{x\to\pm\infty}r(x)$ nur ab von der Existenz der Grenzwerte
 $\lim_{x\to\pm\infty}\frac{x^n}{x^m}=\lim_{x\to\pm\infty} x^d$.
 \step
(i) Im Fall $d>0$ ist $\lim_{x\to+\infty} x^d=\infty$ und $\lim_{x\to-\infty}x^d=(-1)^d\cdot\infty$.
Mit den Grenzwertregeln folgt die bestimmte Divergenz von $r(x)$ für $x\to\pm\infty$
mit
\[\lim_{x\to\infty}r(x)=\text{VZ}\left(\frac{a_n}{b_m}\right)\cdot\infty\quad\text{ und }\quad \lim_{x\to\infty}r(x)=\text{VZ}\left(\frac{a_n}{b_m}\right)\cdot (-1)^d\cdot\infty.\]
\step
(ii)
Im Fall $d=0$ ist $\lim_{x\to\pm\infty }x^d=\lim_{x\to\pm\infty}1=1$, also folgt
\[\lim_{x\to\infty}r(x)=1\cdot \lim_{x\to\pm\infty}\frac{a_n+\frac{a_{n-1}}{x}+\ldots+\frac{a_{1}}{x^{n-1}}+\frac{a_{0}}{x^n}}
{b_m+\frac{b_{m-1}}{x}+\ldots+\frac{b_{1}}{x^{m-1}}+\frac{b_{0}}{x^m}}=\frac{a_n}{b_m}.\]
\step
(iii)
Im Fall $d<0$ ist $\lim_{x\to\pm\infty }x^d=0$, also folgt
\[\lim_{x\to\pm\infty}r(x)=\lim_{x\to\pm\infty }x^d\cdot \lim_{x\to\pm\infty}\frac{a_n+\frac{a_{n-1}}{x}+\ldots+\frac{a_{1}}{x^{n-1}}+\frac{a_{0}}{x^n}}
{b_m+\frac{b_{m-1}}{x}+\ldots+\frac{b_{1}}{x^{m-1}}+\frac{b_{0}}{x^m}}=0\cdot \frac{a_n}{b_m}=0.\]
\end{incremental}
\end{proof*}
Im Fall (i) der vorgehenden Regel können wir sogar noch eine bessere Beschreibung der bestimmten Divergenz
erhalten:
\begin{remark}\label{rem:polynomial_asypt}
Ist für die rationale Funktion $r(x)=\frac{p(x)}{q(x)}$ der Zählergrad größer als der Nennergrad, 
dann gibt es (eindeutig bestimmte) Polynome $s(x)$ mit $\text{grad }s(x)=\text{grad } p(x) - \text{grad } q(x)$
und $t(x)$ mit $\text{grad }t(x)<\text{grad } q(x)$ oder $t=0$, so dass
\[r(x)=\frac{p(x)}{q(x)}=s(x)+\frac{t(x)}{q(x)}.\]
Das Polynom $s(x)$ beschreibt das asymptotische Verhalten von $r(x)$ für $x\to\pm\infty$, d.h. für $x\to\pm\infty$ schmiegt
sich der Graph von $r(x)$ an den von $s(x)$ an.
\end{remark}
\begin{proof*}
\begin{incremental}[\initialsteps{0}]
\step
Die Existenz eindeutig bestimmter solcher Polynome $s(x)$ und $t(x)$ erhält man aus der \ref[content_10_polynomdivision][Polynomdivision]{sec:poly-div-factoring}
\[ p(x)=s(x)\cdot q(x)+t(x).\]
Dabei ist $\text{grad } p(x)=\text{grad } s(x)+\text{grad } q(x)$ und $\text{grad } t(x)<\text{grad } q(x)$ oder $t$ ist bereits das Nullpolynom.
Es gilt damit
\[r(x)=\frac{p(x)}{q(x)}=\frac{s(x)\cdot q(x)+t(x)}{q(x)}=s(x)+\frac{t(x)}{q(x)}.\]
Nach Regel \ref{rule:rat_fun_infty}(iii) gilt $\lim_{x\to\pm\infty}\frac{t(x)}{q(x)}=0$.
Also gilt 
\[\lim_{x\to\pm\infty}r(x)-s(x)=\lim_{x\to\pm\infty}\frac{t(x)}{q(x)}=0.\]
\end{incremental}
\end{proof*}
\begin{example}
Die gebrochen rationale Funktion $r(x)=\frac{x^4+x^3-x^2+x-2}{5x^2-10x-5}$
hat eine einzige Definitionslücke $x_\ast=-1$, denn das Nennerpolynom $q(x)=5(x^2-2x+1)=5(x-1)^2$ hat eine doppelte Nullstelle.
Auch das Zählerpolynom $p(x)=x^4+x^3-x^2+x-2$ hat $x_\ast$ zur Nullstelle.
Es ist $p(x)=(x^3+2x^2+x+2)(x-1)$ (Polynomdivision!), und $x_\ast$ ist keine Nullstelle des erstes Faktors mehr.
Somit ergibt sich
\[r(x)=\frac{x^3+2x^2+x+2}{5}\cdot (x-1)^{-1}.\]
Die Funktion hat also in $x=1$ eine senkrechte Asymptote (grün) und divergiert dort bestimmt. Genauer ist
\[\lim_{x\nearrow 1}r(x)=\lim_{x\nearrow 1}\frac{x^3+2x^2+x+2}{5}\cdot \lim_{x\nearrow 1}(x-1)^{-1}=-\infty,\]
sowie
\[\lim_{x\searrow 1}r(x)=\lim_{x\searrow 1}\frac{x^3+2x^2+x+2}{5}\cdot \lim_{x\searrow 1}(x-1)^{-1}=+\infty.\]
Weil der Zählergrad vier ist und der Nennergrad zwei, divergiert die Funktion für $x\to\pm\infty$ bestimmt.
Durch Polynomdivision erhalten wir weiter eine Zerlegung der Form $r(x)=s(x)+\frac{t(x)}{q(x)}$, nämlich
\[r(x)=\frac{1}{5}(x^2+3x+4)+\frac{6(x-1)}{5(x-1)^2}=\frac{1}{5}(x^2+3x+4)+\frac{6}{5(x-1)}.\]
Es ist also $s(x)=\frac{1}{5}(x^2+3x+4)$ und $t(x)=6(x-1)$ wie in Bemerkung \ref{rem:polynomial_asypt}.
Das Verhalten für $x$ gegen $+\infty$ und $-\infty$ ist somit gegeben durch
\[
\lim_{x\to\pm\infty} r(x)=\lim_{x\to\pm\infty}\frac{1}{5}(x^2+3x+4)=+\infty,\]
und der Funktionsgraph von $x\mapsto \frac{1}{5}(x^2+3x+4)$ ist eine Asymptote (blau) an den Graphen (schwarz) von $r(x)$ für $x\to\pm\infty$.

 	\begin{genericJSXVisualization}[550][800]{example_gebr_rat_final}
 		\begin{variables}
 			\function{f}{real}{(x^3+2*x^2+x+2)/(5*(x-1))}  
			\point{O}{real}{1,0} 			
			\point{P}{real}{1,1} 			
 			\line{l}{real}{var(O),var(P)}
    \function{g}{real}{(x^2+3*x+4)/5}
     
 			
 		\end{variables}
 		\color{l}{green}
     \color{g}{blue}%{magenta}
     \color{f}{black}
 
 		\begin{canvas}
 			\plotSize{400,300}
 			\plotLeft{-4},
 			\plotRight{4}
 			\plot[coordinateSystem]{f,l,g }
 		\end{canvas}
 \end{genericJSXVisualization}
\end{example}

% Im letzten Abschnitt hatten wir Grenzwerte gegen Definitionslücken betrachtet. In diesem Abschnitt geht es um die Grenzwerte für $x$ gegen $\infty$ oder gegen $-\infty$, wenn man also Folgen $(x_n)$ betrachtet, die \link{bestimmte-divergenz}{bestimmt divergent} gegen  $\infty$ oder gegen $-\infty$ sind.

% Des weiteren geht es um den Fall, dass für links- bzw. rechtsseitige Folgen an Definitionslücken die Folgen der Funktionswerte bestimmt divergieren.

% In beiden Fällen erhält man grafisch sogenannte \emph{Asymptoten}.


% Die Ergebnisse sind ganz ähnlich wie im vorigen Abschnitt, wenn man bestimmt divergent gegen $\infty$ bzw. gegen $-\infty$ als Konvergenz ansieht.


% \section{Grenzwert gegen Unendlich}

% \begin{definition}
% \begin{enumerate}
% \item
% Sei $f:D\to \R$ eine Funktion und es gebe mindestens eine Folge $(x_n)_{n\geq 1}$ mit $x_n\in D$, die bestimmt gegen $\infty$ divergiert, d.h. $D$ ist nach oben unbeschränkt.

% Die Funktion $f$ heißt \notion{konvergent gegen $L \in \R$ für $x \to \infty$}, wenn
%   \[ \lim_{n\to \infty} f(x_n)= L \]
%   für jede Folge $(x_n)_{n\geq 1}$ mit $x_n\in D$ und $\lim_{n\to \infty} x_n=\infty$.

% Wir schreiben dann kurz:
% \[ \lim_{x\to \infty} f(x)= L. \]

% \item Sei $f:D\to \R$ eine Funktion und es gebe mindestens eine Folge $(x_n)_{n\geq 1}$ mit $x_n\in D$, die bestimmt gegen $-\infty$ divergiert, d.h. $D$ ist nach unten unbeschränkt.

% Die Funktion $f$ heißt \notion{konvergent gegen $L \in \R$ für $x \to -\infty$}, wenn
%   \[ \lim_{n\to \infty} f(x_n)= L \]
%   für jede Folge $(x_n)_{n\geq 1}$ mit $x_n\in D$ und $\lim_{n\to \infty} x_n=-\infty$.

% Wir schreiben dann kurz:
% \[ \lim_{x\to -\infty} f(x)= L. \]
% \end{enumerate}
% \end{definition}

% \begin{remark}
% Auch hier gibt es wieder ein $\epsilon$-Kriterium. Da die Nähe zu $\infty$ aber nicht durch eine $\delta$-Umgebung gegeben ist, sondern durch Schranken, die überschritten werden, lautet die entsprechende Definition:

% Sei $f:D\to \R$ eine Funktion, für welche $D$ nach oben unbeschränkt ist. Die Funktion $f$ ist konvergent gegen $L \in \R$ für $x \to \infty$, wenn es für jedes $\epsilon>0$ ein $R_\epsilon\in \R$ gibt,
% sodass
% \[ |f ( x ) - L| < \epsilon\quad \text{für alle }x\in D \text{ mit }x > R_\epsilon. \]

% \\

% Entsprechend gilt für $-\infty$:

% Sei $f:D\to \R$ eine Funktion, für welche $D$ nach unten unbeschränkt ist. Die Funktion $f$ ist konvergent gegen $L \in \R$ für $x \to -\infty$, wenn es für jedes $\epsilon>0$ ein $R_\epsilon\in \R$ gibt,
% sodass
% \[ |f ( x ) - L| < \epsilon\quad \text{für alle }x\in D \text{ mit }x < R_\epsilon. \]
% \end{remark}

% \begin{remark}\label{rem:waagerechte-asymptote}
% Anschaulich bedeutet die Eigenschaft $\lim_{x\to \infty} f(x)= L$, dass sich der Graph von $f$ für
% immer größer werdende $x$-Werte immer mehr der waagerechten Geraden $y=L$ annähert.\\
% Diese waagerechte Gerade $y=L$ nennt man daher \notion{(waagerechte) Asymptote} an den Graphen von $f$.

% Die  Eigenschaft $\lim_{x\to -\infty} f(x)= L$ bedeutet ganz entsprechend, dass sich der Graph von $f$ für
% immer kleiner werdende $x$-Werte immer mehr der waagerechten Geraden $y=L$ annähert.\\
% Auch in diesem Fall nennt man  die Gerade $y=L$ \notion{(waagerechte) Asymptote} an den Graphen von $f$.
% \end{remark}

% \begin{example}
% Sei $f:\R\setminus \{0; 2\}\to \R$ gegeben durch
% \[ f(x)= \frac{|x-2|}{x^2-2x} \]
% für alle $x\in \R\setminus \{0; 2\}$.

% Ist $(x_n)_{n\geq 1}$ eine Folge in $\R\setminus \{0; 2\}$, die bestimmt divergent gegen $\infty$ ist, also mit $\lim_{n\to \infty} x_n=\infty$, so gibt es ein $N\in \N$ mit $x_n>2$ für alle $n\geq N$.\\
% Dementsprechend gilt für alle $n\geq N$: $f(x_n)=\frac{x-2}{x^2-2x}=\frac{1}{x}$.

% Mit den \ref[bestimmte-divergenz][Rechenregeln mit bestimmter Divergenz]{sec:rechenregeln} gilt daher:
% \[ \lim_{n\to \infty} f(x_n)= \lim_{n\to \infty} \frac{1}{x_n}=0. \]

% Da dies für alle gegen $\infty$ bestimmt divergenten Folgen $(x_n)_{n\geq 1}$ gilt, ist somit
% \[ \lim_{x\to \infty} f(x)= 0.\]
% Die Gerade $y=0$ ist also eine waagerechte Asymptote an den Graphen für $x\to \infty$. \\ \\


% Folgen $(x_n)_{n\geq 1}$ in $\R\setminus \{0; 2\}$, die bestimmt divergent gegen $-\infty$ sind,
% erfüllen $x_n<0$ für alle $n\geq N$ bei geeignetem festen $N\in \N$ und dementsprechend gilt:
% \[ f(x_n)=\frac{-(x_n-2)}{x_n^2-2x_n}=-\frac{1}{x_n} \quad \text{ für alle }n\geq N.\]

% Also ist auch hier:
% \[ \lim_{n\to \infty} f(x_n)= \lim_{n\to \infty} -\frac{1}{x_n}=0. \]

% Da dies für alle gegen $-\infty$ bestimmt divergenten Folgen $(x_n)_{n\geq 1}$ gilt, ist somit
% \[ \lim_{x\to -\infty} f(x)= 0.\]
% Die Gerade $y=0$ ist also auch eine waagerechte Asymptote an den Graphen für $x\to -\infty$.

%  	\begin{genericGWTVisualization}[550][800]{mathlet1}
%  		\begin{variables}
%  			\function{f}{real}{|x-2|/(x^2-2*x)}  
% 			\point{O}{real}{0,0} 			
% 			\point{P}{real}{1,0} 			
%  			\line{l}{real}{var(O),var(P)}
 			
%  		\end{variables}
%  		\color{l}{#00CC00}
 
%  		\begin{canvas}
%  			\plotSize{400,300}
%  			\plotLeft{-4}
%  			\plotRight{4}
%  			\plot[coordinateSystem]{f,l }
%  		\end{canvas}
%  		\text{Die schwarze Kurve ist der Graph der Funktion $f:\R\setminus \{0;2\}\to \R$ mit 
%  		$f(x)=\frac{|x-2|}{x^2-2x}$. Die grüne Gerade ist die waagerechte Asymptote $y=0$.
%  		}
%  	    	\end{genericGWTVisualization}

% \end{example}

% \begin{quickcheck}
%     \field{real}
%         \type{input.number}
%         \begin{variables}
%             \drawFromSet{a}{2,3,4,5}
%             \function[calculate]{b}{2*a}
%             \number{n}{0}
%         \end{variables}
%         \text{Berechnen Sie den Grenzwert für die Funktion $f(x)=\frac{\var{b}}{\var{a}+\sqrt{|x|}}$\\
%         $\lim_{x\to -\infty}f(x)=$\ansref
%         }
%         \begin{answer}
%             \solution{n}
%         \end{answer}
%         \explanation{Mit $x_n=-n^2$ ist $\lim_{n\to\infty}x_n=-\infty$ und somit gilt: 
%         \[\lim_{x\to -\infty}f(x)=\lim_{n\to \infty}f(x_n)=\frac{\var{b}}{\var{a}+\sqrt{|-n^2|}}
%         =\frac{\var{b}}{\var{a}+n}=0\]
%         }
% \end{quickcheck}

% \section{Bestimmte Divergenz an Definitionslücken}


% \begin{definition}
% Sei $f:D\to \R$ eine reelle Funktion und sei $x^*\in\R$ der Grenzwert 
%   von mindestens einer Folge $(x_n)_{n\geq 1}$ mit $x_n\in D\setminus \{x^*\}$.
  
% \begin{enumerate}
% \item
%   Die Funktion $f$ heißt \notion{bestimmt divergent gegen $\infty$ für $x \to x^*$}, wenn
%   \[ \lim_{n\to \infty} f(x_n)= \infty \]
%   für jede Folge $(x_n)_{n\geq 1}$ mit $x_n\in D\setminus \{x^*\}$ und $\lim_{n\to \infty} x_n=x^*$.

% Wir schreiben dann kurz:
% \[ \lim_{x\to x^*} f(x)= \infty. \]
% \item
%   Die Funktion $f$ heißt \notion{bestimmt divergent gegen $-\infty$ für $x \to x^*$}, wenn
%   \[ \lim_{n\to \infty} f(x_n)= -\infty \]
%   für jede Folge $(x_n)_{n\geq 1}$ mit $x_n\in D\setminus \{x^*\}$ und $\lim_{n\to \infty} x_n=x^*$.

% Wir schreiben dann kurz:
% \[ \lim_{x\to x^*} f(x)= -\infty. \]
% \end{enumerate}
% \end{definition}

% \begin{remark}
% Man definiert entsprechend \notion{linksseitig bestimmte Divergenz} von $f$ über Folgen
%  $(x_n)_{n\geq 1}$ mit $x_n<x^*$ und schreibt dann
%  \[ \lim_{x\nearrow x^*} f(x)= \infty\quad \text{bzw.} \quad \lim_{x\nearrow x^*} f(x)=-\infty, \]
%  und auch \notion{rechtsseitig bestimmte Divergenz} von $f$ über Folgen
%  $(x_n)_{n\geq 1}$ mit $x_n>x^*$ und schreibt dann
%  \[ \lim_{x\searrow x^*} f(x)= \infty\quad \text{bzw.} \quad \lim_{x\searrow x^*} f(x)=-\infty. \]
% \end{remark}


% \begin{remark}\label{rem:senkrechte-asymptote}
% Anschaulich bedeutet die Eigenschaft $\lim_{x\nearrow x^*} f(x)= \infty$, dass der Graph von $f$
% immer steiler wird, wenn man sich der Stelle $x^*$ von links nähert, und sich der Graph an die senkrechte Gerade $x=x^*$ "`anschmiegt"'.
% Diese senkrechte Gerade $x=x^*$ nennt man daher \notion{(senkrechte) Asymptote} an den Graphen von $f$.

% Entsprechend fällt der Graph von $f$ bei Annäherung an $x^*$ von links immer stärker, wenn 
% $\lim_{x\nearrow x^*} f(x)= -\infty$. Auch in diesem Fall "`schmiegt"' sich der Graph an die senkrechte Gerade $x=x^*$ an.

% Für die rechtsseitige bestimmte Divergenz gilt entsprechendes.
% \end{remark}


% \begin{example}
% \begin{enumerate}
% \item Wir betrachten die Funktion $f:\R\setminus {\{1\}}\to \R, x\mapsto \frac{x+1}{x-1}$ und untersuchen
%  das Verhalten bei $x^*=1$.\\
% Für jede Folge $(x_n)_{n\geq 1}$ mit $\lim_{n\to \infty} x_n=1$ ist
% \[ \lim_{n\to \infty} \frac{1}{f(x_n)}=\lim_{n\to \infty} \frac{x_n-1}{x_n+1}
% = \frac{\lim_{n\to \infty} x_n-1}{\lim_{n\to \infty} x_n+1}=\frac{1-1}{1+1}=0.\]
% Sind nun alle $x_n<1$, dann sind sie auch alle ab einem bestimmten $N$ zwischen $-1$ und $1$ und daher $f(x_n)<0$ für alle $n\geq N$. 
% Nach den \ref[bestimmte-divergenz][Rechenregeln mit bestimmter Divergenz]{sec:rechenregeln} gilt
% daher für gegen $1$ konvergente Folgen $(x_n)_{n\geq 1}$ mit $x_n<1$:
% \[ \lim_{n\to \infty} f(x_n)=-\infty. \]
% Also ist $ \lim_{x\nearrow 1} f(x)=-\infty.$

% Entsprechend ist $f(x)>0$ für alle $x>1$ und man erhält entsprechend
% \[ \lim_{x\searrow 1} f(x)= \infty. \]

% Der Graph hat also bei $x=1$ eine senkrechte Asymptote.

% \begin{genericGWTVisualization}[550][800]{mathlet1}
%  		\begin{variables}
%  			\function{f}{real}{(x+1)/(x-1)}  
% 			\point{O}{real}{1,0} 			
% 			\point{P}{real}{1,1} 			
%  			\line{l}{real}{var(O),var(P)}
 			
%  		\end{variables}
%  		\color{l}{#00CC00}
 
%  		\begin{canvas}
%  			\plotSize{400,300}
%  			\plotLeft{-4}
%  			\plotRight{4}
%  			\plot[coordinateSystem]{f,l }
%  		\end{canvas}
%  		\text{Die schwarze Kurve ist der Graph der Funktion $f:\R\setminus \{1\}\to \R$ mit 
%  		$f(x)=\frac{x+1}{x-1}$. Die grüne Gerade ist die senkrechte Asymptote $x=1$.
%  		}
%  	    	\end{genericGWTVisualization}
% \item Wir betrachten die Funktion $f:\R\setminus {\{1\} }\to \R, x\mapsto \frac{x+1}{(x-1)^2}$ und untersuchen
%  das Verhalten bei $x^*=1$.\\
% Für jede Folge $(x_n)_{n\geq 1}$ mit $\lim_{n\to \infty} x_n=1$ ist
% \[ \lim_{n\to \infty} \frac{1}{f(x_n)}=\lim_{n\to \infty} \frac{(x_n-1)^2}{x_n+1}
% = \frac{(\lim_{n\to \infty} x_n-1)^2}{\lim_{n\to \infty} x_n+1}=\frac{(1-1)^2}{1+1}=0.\]
% Für $x>-1$ (und $x\neq 1$) ist auch $f(x)>0$. Da ab einem bestimmten $N$ alle $x_n$ größer als $-1$ sind (da die Folge gegen $1$ konvergiert),
% gilt nach den \ref[bestimmte-divergenz][Rechenregeln mit bestimmter Divergenz]{sec:rechenregeln} 
% \[ \lim_{n\to \infty} f(x_n)=\infty. \]
% Also ist $ \lim_{x\to 1} f(x)=\infty.$

% Der Graph hat also bei $x=1$ eine senkrechte Asymptote und er schmiegt sich von beiden Seiten gegen $\infty$ an.

% \begin{genericGWTVisualization}[550][800]{mathlet1}
%  		\begin{variables}
%  			\function{f}{real}{(x+1)/(x-1)^2}  
% 			\point{O}{real}{1,0} 			
% 			\point{P}{real}{1,1} 			
%  			\line{l}{real}{var(O),var(P)}
 			
%  		\end{variables}
%  		\color{l}{#00CC00}
 
%  		\begin{canvas}
%  			\plotSize{400,300}
%  			\plotLeft{-4}
%  			\plotRight{4}
%  			\plot[coordinateSystem]{f,l }
%  		\end{canvas}
%  		\text{Die schwarze Kurve ist der Graph der Funktion $f:\R\setminus \{1\}\to \R$ mit 
%  		$f(x)=\frac{x+1}{(x-1)^2}$. Die grüne Gerade ist die senkrechte Asymptote $x=1$.
%  		}
%  	    	\end{genericGWTVisualization}
% \end{enumerate}
% \end{example}

% Wie die obigen Beispiele schon vermuten lassen, gibt es analog zu den \ref[bestimmte-divergenz][Rechenregeln für Folgen mit bestimmter Divergenz]{sec:rechenregeln} auch Rechenregeln für 
% $\lim_{x\to x^*} f(x)$.

% \begin{rule}
% Sei $f:D\to \R$ eine reelle Funktion und sei $x^*\in\R$ der Grenzwert 
%   von mindestens einer Folge $(x_n)_{n\geq 1}$ mit $x_n\in D\setminus \{x^*\}$.

% Dann gelten:
% \begin{enumerate}
% \item \[   \lim_{x\nearrow x^*} f(x)=\infty   \]
%   genau dann, wenn $\,\lim_{x\nearrow x^*} \frac{1}{f(x)}=0\,$ und 
% \[ \exists \delta>0 \text{ mit }\quad f(x)>0\text{ für alle } x\in (x^*-\delta; x^*)\cap D. \]
% \item \[   \lim_{x\nearrow x^*} f(x)=-\infty     \]
%   genau dann, wenn $\,\lim_{x\nearrow x^*} \frac{1}{f(x)}=0\,$ und
% \[ \exists \delta>0\text{ mit }\quad f(x)<0\text{ für alle } x\in (x^*-\delta; x^*)\cap D. \]
% \item \[   \lim_{x\searrow x^*} f(x)=\infty  \]
%   genau dann, wenn $\,\lim_{x\searrow x^*} \frac{1}{f(x)}=0\,$ und
% \[ \exists \delta>0\text{ mit }\quad f(x)>0\text{ für alle } x\in (x^*; x^*+\delta)\cap D. \]
% \item \[   \lim_{x\searrow x^*} f(x)=-\infty   \]
%   genau dann, wenn $\,\lim_{x\searrow x^*} \frac{1}{f(x)}=0\,$ und
% \[ \exists \delta>0\text{ mit }\quad f(x)<0\text{ für alle } x\in (x^*; x^*+\delta)\cap D. \]
% \end{enumerate}  
% \end{rule}

% \end{visualizationwrapper}

% \begin{quickcheck}
%     \field{real}
%     \type{input.number}
%     \begin{variables}
%         \number{n}{0}
%     \end{variables}
%     \text{Berechen Sie den Grenzwert der Funktion $f(x)=x^2\ln(x^2)$:
%     \[\lim_{x\to0}f(x)=\]
%     Tipp: Substituieren Sie: $x=e^y$, passen Sie "$x\to 0$" im limes entsprechend an und berücksichtigen Sie,
%     dass für jedes Polynom $p(x)$ gilt: $\lim_{x\to\infty}\frac{p(x)}{e^{cx}}=0$ für $c>0$.\\
%     Antwort:\ansref
%     }
%     \begin{answer}
%         \solution{n}
%     \end{answer}
%     \explanation{$\lim_{x\to0}x^2\ln(x^2)=\lim_{y\to-\infty}(e^y)^2\ln((e^y)^2)=
%     \lim_{y\to-\infty}e^{2y}\ln(e^{2y})=\lim_{y\to-\infty}e^{2y}\cdot2y=0$}

% \end{quickcheck}


\end{visualizationwrapper}
\end{content}