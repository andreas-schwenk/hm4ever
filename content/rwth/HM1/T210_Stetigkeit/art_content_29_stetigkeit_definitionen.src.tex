%$Id:  $
\documentclass{mumie.article}
%$Id$
\begin{metainfo}
  \name{
    \lang{de}{Stetigkeitsbegriff}
    \lang{en}{}
  }
  \begin{description} 
 This work is licensed under the Creative Commons License Attribution 4.0 International (CC-BY 4.0)   
 https://creativecommons.org/licenses/by/4.0/legalcode 

    \lang{de}{Beschreibung}
    \lang{en}{}
  \end{description}
  %Links alte Version:
  % \begin{components}
  %   \component{generic_image}{content/rwth/HM1/images/g_tkz_T204_Example_A.meta.xml}{T204_Example_A}
  %   \component{generic_image}{content/rwth/HM1/images/g_img_00_Videobutton_schwarz.meta.xml}{00_Videobutton_schwarz}
  %   \component{generic_image}{content/rwth/HM1/images/g_img_00_video_button_schwarz-blau.meta.xml}{00_video_button_schwarz-blau}
  %   \component{js_lib}{system/media/mathlets/GWTGenericVisualization.meta.xml}{mathlet1}
  % \end{components}
  % \begin{links}
   %     \link{generic_article}{content/rwth/HM1/T210_Stetigkeit/g_art_content_31_grenzwerte_von_funktionen.meta.xml}{content_31_grenzwerte_von_funktionen1}
  %   \link{generic_article}{content/rwth/HM1/T210_Stetigkeit/g_art_content_30_elem_funktionen.meta.xml}{elem-funk-stetig}
  %   \link{generic_article}{content/rwth/HM1/T204_Abbildungen_und_Funktionen/g_art_content_12_reelle_funktionen_monotonie.meta.xml}{reelle-funk}
  %   \link{generic_article}{content/rwth/HM1/T205_Konvergenz_von_Folgen/g_art_content_16_konvergenzkriterien.meta.xml}{konv-krit}
  % \end{links}
    \begin{components}
    \component{generic_image}{content/rwth/HM1/images/g_tkz_T210_Heaviside.meta.xml}{T210_Heaviside}
    \component{generic_image}{content/rwth/HM1/images/g_tkz_T210_sprung.meta.xml}{T210_sprung}
    \component{generic_image}{content/rwth/HM1/images/g_tkz_T204_Example_A.meta.xml}{T204_Example_A}
    \component{generic_image}{content/rwth/HM1/images/g_img_00_Videobutton_schwarz.meta.xml}{00_Videobutton_schwarz}
    \component{generic_image}{content/rwth/HM1/images/g_img_00_video_button_schwarz-blau.meta.xml}{00_video_button_schwarz-blau}
    \component{js_lib}{system/media/mathlets/GWTGenericVisualization.meta.xml}{mathlet1}
  \end{components}
  \begin{links}
   % \link{generic_article}{content/playground/Kathrin_Maurischat/g_art_content_31_grenzwerte_von_funktionen_neu.meta.xml}{content_31_grenzwerte_von_funktionen_neu}
    \link{generic_article}{content/rwth/HM1/T205_Konvergenz_von_Folgen/g_art_content_14_konvergenz.meta.xml}{content_14_konvergenz1}
    \link{generic_article}{content/rwth/HM1/T202_Reelle_Zahlen_axiomatisch/g_art_content_07_vollstaendigkeit.meta.xml}{content_07_vollstaendigkeit}
    \link{generic_article}{content/rwth/HM1/T205_Konvergenz_von_Folgen/g_art_content_14_konvergenz.meta.xml}{content_14_konvergenz}
    \link{generic_article}{content/rwth/HM1/T210_Stetigkeit/g_art_content_31_grenzwerte_von_funktionen.meta.xml}{content_31_grenzwerte_von_funktionen}
    \link{generic_article}{content/rwth/HM1/T210_Stetigkeit/g_art_content_30_elem_funktionen.meta.xml}{elem-funk-stetig}
    \link{generic_article}{content/rwth/HM1/T204_Abbildungen_und_Funktionen/g_art_content_12_reelle_funktionen_monotonie.meta.xml}{reelle-funk}
    \link{generic_article}{content/rwth/HM1/T205_Konvergenz_von_Folgen/g_art_content_16_konvergenzkriterien.meta.xml}{konv-krit}
  \end{links}
  \creategeneric
\end{metainfo}
\begin{content}
\usepackage{mumie.ombplus}
\ombchapter{10}
\ombarticle{1}
\usepackage{mumie.genericvisualization}

\begin{visualizationwrapper}

\lang{de}{\title{Stetigkeitsbegriff}}
 
\begin{block}[annotation]
  Dieses Kapitel ist das neue zweite Kapitel im großen Kapitel "Stetigkeit" und steht dort nach den Funktionsgrenzwerten. 
  Es ist umgestellt worden: Zunächst wird das Folgenkriterium zur Einführung in die Stetigkeit benutzt. Erst anschließend folgt das epsilon-delta-Kriterium.

  JSX-Graphs: Alle Visualisierungen werden ersetzt. Die Farben werden für rot-grün-Sehschwäche unterscheidbar gewählt.
\end{block}
\begin{block}[annotation]
  Im Ticket-System: \href{http://team.mumie.net/issues/9789}{Ticket 9789}\\
\end{block}

\begin{block}[info-box]
\tableofcontents
\end{block}


Viele Vorgänge in der Natur lassen sich mit stetigen Funktionen beschreiben. Betrachtet
man z.B. in einem Weg-Zeit-Diagramm die zurückgelegte Wegstrecke als Funktion
der Zeit, so wird es hier keine Sprünge geben. Die Weg-Zeit-Funktion ist stetig.

Es gibt im wesentlichen zwei Möglichkeiten, die Stetigkeit von Funktionen zu definieren. Die eine verwendet ein sogenanntes
$\epsilon$-$\delta$-Kriterium, die andere verwendet Folgen. Beide Definitionen sind gleichwertig, da der 
identische Begriff definiert wird. Stetigkeit ist ein punktweiser Begriff, von dem sich Stetigkeit auf 
einer Menge ableitet.

Anschaulich wird eine stetige Funktion manchmal dadurch charakterisiert, dass man ihren Graphen
„ohne Absetzten durchzeichnen“ kann, dass also keine Sprünge auftreten. Diese Anschauung ist höchstens auf \emph{Intervallen} gerechtfertigt.

 


\section{Folgenstetigkeit}\label{sec:folgenkriterium}
In Abschnitt \link{content_31_grenzwerte_von_funktionen}{Folgen von Funktionswerten} haben wir
bereits den Begriff des 
\ref[content_31_grenzwerte_von_funktionen][Funktionsgrenzwerts]{def:funktionsgrenzwert} für $x$ gegen $x^\ast$ kennen gelernt.
Existiert dieser Funktionsgrenzwert, so beschreibt er das Verhalten der Funktion um die Stelle $x^\ast$ herum. 
Stimmt der Funktionsgrenzwert $\lim_{x\to x^\ast}f(x)$ zusätzlich mit dem Funktionswert $f(x^\ast)$ überein, 
so sprechen wir von Stetigkeit an der Stelle $x^\ast$:
\begin{definition}[Folgenkriterium für Stetigkeit]\label{def:Folgenstetigkeit}
Es sei $f : D \to \R$ eine Funktion und $x^* \in D$. Die Funktion
$f$ heißt  \notion{stetig an der Stelle $x^*$}, wenn die folgende Bedingung erfüllt ist:
\\
Für jede Folge $( x_n)_{n \in\N}$ in $D$ mit 
$\lim_{n \to \infty} x_n = x^*$ gilt
\[ \lim_{n \to \infty} f ( x_n ) = f ( \lim_{n \to \infty} x_n) 
= f ( x^* ) .\]

Eine Funktion, die in \notion{jeder} Stelle $x^\ast\in D$ stetig ist, heißt (überall) \notion{stetig}.
Allgemeiner heißt $f$ stetig auf einer Teilmenge $\tilde{D}\subset D$, wenn $f$ an jeder Stelle $x^\ast\in \tilde{D}$ stetig ist.
\end{definition}
\begin{remark}
Eine Funktion $f:D\to\R$ ist also genau dann stetig, wenn für alle in $D$ konvergenten Folgen $(x_n)_{n\in\N}$ gilt
\[
\lim_{n\to\infty}f(x_n)=f(\lim_{n\to\infty}x_n).
\]
\end{remark}
\begin{remark}
Das Folgenkriterium für Stetigkeit beinhaltet zwei Bedingungen:
\begin{enumerate}
\item[(i)] Für jede Folge $( x_n)_{n \geq 1}$, die gegen $x^*$ konvergiert, ist die Folge $(f(x_n))_{n \geq 1}$ konvergent.
\item[(ii)] Jede der nach (i) konvergenten Folgen $(f(x_n))_{n \geq 1}$ hat tatsächlich den Funktionswert $f(x^*)$ als Grenzwert.
\end{enumerate}
Um zu zeigen, dass eine Funktion unstetig (nicht stetig) an einer Stelle ist, 
reicht es somit zu zeigen, dass eine dieser beiden Bedingungen verletzt ist.
\end{remark}

\begin{example}[Einfache Beispiele]\label{ex:einfach_stetig}
Mithilfe der \ref[content_14_konvergenz][Grenzwertregeln]{sec:grenzwertregeln} 
für Folgenkonvergenz lassen sicht leicht Beispiele für stetige Funktionen finden.
\begin{tabs*}[\initialtab{0}]
\tab{Lineare und konstante Funktionen} Für feste reelle Zahlen $a,b\in \R$ sei $f: \R \to \R$, $x \mapsto ax + b$, eine 
lineare Funktion (falls $a\neq 0$) oder eine konstante Funktion (falls $a=0$).
Für eine beliebige feste Stelle $x^* \in \R$ und jede Folge $( x_n)_{n \geq 1}$ mit $\lim_{n\to \infty} x_n=x^*$ gilt dann:
\[  \lim_{n\to \infty} f(x_n)= \lim_{n\to \infty} (ax_n+b)=a \lim_{n\to \infty} x_n + b =ax^*+b=f(x^*). \]
Also sind lineare und konstante Funktionen stetig in jedem Punkt.
\tab{$f(x)=x^2$} Die Funktion $f : \R \to \R, x \mapsto x^2$ ist auf ganz $\R$ stetig. Denn für beliebiges $x^*\in \R$ und
 jede Folge $( x_n)_{n \geq 1}$ mit $\lim_{n\to \infty} x_n=x^*$ gilt:
 \[  \lim_{n\to \infty} f(x_n)= \lim_{n\to \infty} x_n^2 =\left( \lim_{n\to \infty} x_n\right)^2 =(x^*)^2 =f(x^*). \]
\end{tabs*}
\end{example}

\begin{example}[Elementare Funktionen] %\label{ex:einfach_stetig}
Die meisten elementaren Funktionen sind auf ihrer gesamten Definitionsmenge stetig, 
was wir im \link{elem-funk-stetig}{nächsten Abschnitt} zeigen.
\end{example}
\begin{example}[Gegenbeispiele]\label{ex:unstetig}
\begin{tabs*}[\initialtab{0}]
\tab{Heaviside-Funktion} Die Heaviside-Funktion 
\[
H:\R\to\R,\: x\mapsto\begin{cases}0& \text{ falls } x<0\\ 1& \text{ falls } x\geq 0\end{cases},
\]
ist an der Stelle $x^*=0$ nicht stetig.
Dazu zeigen wir, dass der linksseitige Funktionsgrenzwert gegen $x^\ast=0$ existiert, 
aber nicht mit dem Funktionswert $f(x^\ast)=1$ übereinstimmt.
\\
Es sei  $( x_n)_{n \in\N}$ eine beliebige Folge mit $x_n<x^*$ für alle $n\in\N$ und $\lim_{n\to\infty}x_n=0$.
Dann ist
\[ \lim_{n\to \infty} H(x_n) =\lim_{n\to\infty} 0=0\neq 1 =f(0).\]
Also ist $H$ nicht stetig in $x^\ast=0$.
\begin{center}
\image[350]{T210_Heaviside}
\end{center}
\tab{Dirichletsche Sprungfunktion} Es sei 
\[  D:\R\to {\{0;1\}}, x \mapsto \begin{cases} 1 & \text{falls }x\in \Q \\
0 & \text{falls }x\notin \Q \end{cases}.\]
die \ref[reelle-funk][Dirichletsche Sprungfunktion]{ex:unstetige-funktionen}.
Sie ist in keinem einzigen Punkt stetig, nämlich:
\\
Für jede irrationale Stelle $x^*$ gibt es  eine Folge rationaler Zahlen $( x_n)_{n \geq 1}$, die gegen $x^*$ konvergiert. 
Dies ist eine Eigenschaft der \link{content_07_vollstaendigkeit}{Vollständigkeit der reellen Zahlen}.
Für diese Folge gilt dann
\[ \lim_{n\to \infty} D(x_n)= \lim_{n\to \infty} 1\neq 0=D(x^*),\]
weshalb $D$ in $x^*\in\R\setminus\Q$ nicht stetig ist.
\\
Ebenso gibt es für jede rationale Stelle $x^*$ eine Folge irrationaler Zahlen 
$( x_n)_{n \geq 1}$, die gegen $x^*$ konvergiert. Für diese gilt dann
\[ \lim_{n\to \infty} D(x_n)= \lim_{n\to \infty} 0\neq 1=D(x^*),\]
weshalb $D$ in $x^*\in\Q$ nicht stetig ist.
\end{tabs*}
\end{example}

\begin{example}[Noch eine Sprungfunktion]
Die Funktion
\[
f:\R\setminus{\{0\}}\to\R,\: x\mapsto\begin{cases}0& \text{ falls } x<0\\ 1& \text{ falls } x>0\end{cases},
\]
ist überall stetig. Denn sei $x^\ast\in\R\setminus\{0\}$ beliebig und $(x_n)_{n\in\N}$ eine Folge mit $\lim_{n\to\infty}=x^\ast$.
Weil die Folge gegen $x^\ast$ \ref[content_14_konvergenz1][konvergiert]{def:folgenkonvergent}, gibt es zu $\epsilon:=\frac{\vert x^\ast\vert}{2}$ ein $N\in\N$ so, dass für alle $n\geq N$ gilt
$\vert x^\ast-x_n\vert<\epsilon$. Insbesondere haben alle diese $x_n$ dasselbe Vorzeichen wie $x^\ast$.
Somit gilt
\[
\lim_{n\to\infty}f(x_n)=\lim_{n\to\infty, n\geq N}f(x_n)=f(x^\ast),
\]
d.h. $f$ ist stetig in $x^\ast$.\\
Anders formuliert: Die Unstetigkeitsstelle der Heaviside-Funktion wurde aus dem Definitionsbereich entfernt.
Für $x>0$ wie für $x<0$ ist $f$ konstant, also stetig. Somit ist $f$ insgesamt stetig -- obwohl ihr Graph einen Sprung macht.
\begin{center}
\image[350]{T210_sprung}
\end{center}
\end{example}





% \begin{remark}
% In den Beispielen hatten wir zum Beweis der Unstetigkeit Folgen verwendet, bei denen die Folge der 
% Funktionswerte gegen einen Wert konvergieren, der vom Funktionswert der Stelle verschieden ist. 
% Im Allgemeinen muss aber die Folge der Funktionswerte nicht einmal konvergieren.
% \end{remark}

Die \ref[content_14_konvergenz][Grenzwertregeln]{sec:grenzwertregeln} zeigen ganz allgemein (also allgemeiner als in obigen Beispielen)
die Stetigkeit zusammengesetzter Funktionen:

\begin{theorem}\label{rule:zusammengesetzte-funktionen}
Seien $f:D\to \R$ und $g:D\to \R$ reelle Funktionen, die an einer Stelle $x^*\in D$ stetig sind, sowie $r\in \R$.

Dann sind auch die Funktionen $f+g$, $f-g$, $f\cdot g$ und $rf$ in $x^*$ stetig.

Ist zudem $g(x^*)\neq 0$, dann ist auch die Funktion $\frac{f}{g}$ an der Stelle $x^*$ stetig. \\

\end{theorem}
  
\begin{quickcheck}
%\type{mc.multiple}%
\lang{de}{\text{Wählen Sie alle richtigen Aussagen aus:}
}

\begin{choices}{multiple}
    \begin{choice}
        \text{1) Eine Funktion, die an der Stelle $x_0$ nicht definiert ist, 
            kann dort stetig sein.}
        \solution{false}
    \end{choice}
    \begin{choice}
        \text{2) Die Funktion $f:\R\to\R$, $x\mapsto x^4-1$ ist überall stetig.}
        \solution{true}
    \end{choice}
    % \begin{choice}
    %     \text{Am ersten Tag geht ein Bergwanderer morgens um 8 Uhr los und erreicht den Berggipfel um 17 Uhr.
    %     Am zweiten Tag läuft er von 8 bis 17 Uhr den Berg über dieselbe Strecke wieder hinunter. 
    %     Es gibt auf dem Weg eine Stelle, an der er 
    %     zur selben Tageszeit wie am Vortag wieder vorbeikommt.}
    %     \solution{true}
    % \end{choice}
\end{choices}

\explanation{Aussage 1) ist falsch, denn damit $f(x)$ an einer Stelle $x_0$ stetig ist, 
            muss $x_0 \in D$ gelten.\\
            Aussage 2) ist wahr, denn wie gesehen sind die Funktionen $g:x\mapsto x^2$ und 
            $h:x\mapsto -1$ auf ganz $\R$ stetig. Satz \ref{rule:zusammengesetzte-funktionen} liefert
            die Stetigkeit der zusammengesetzten Funktion $f(x)=g(x)^2+h(x)$.}
%   zu 2) Trägt man die Höhenmeter über der Zeit auf, so ist der Aufstieg eine stetige, monoton steigende Funktion und der Abstieg
%   eine stetige, monoton fallende Funktion, d.h. es gibt genau einen Kreuzungspunkt. \\Aussage 2) ist damit richtig.}

 \end{quickcheck}  


Das Folgenkriterium für Stetigkeit wird mit vielen Beispielen im folgenden Video erläutert:
\floatright{\href{https://api.stream24.net/vod/getVideo.php?id=10962-2-10900&mode=iframe&speed=true}{\image[75]{00_video_button_schwarz-blau}}}\\









\section{Das $\epsilon$-$\delta$-Kriterium für Stetigkeit}\label{sec:eps-delta}

\begin{definition}[$\epsilon$-$\delta$-Kriterium]
Sei $f$ eine auf $D\subset\R$ definierte reelle Funktion und sei $x^*\in D$.
 \lang{de}{Man sagt $f$ \notion{erfüllt das $\epsilon$-$\delta$-Kriterium an der Stelle} $x^*\in D$, 
 wenn es zu jedem $\epsilon >0$ ein \nowrap{$\delta = \delta(\epsilon) >0$} gibt, so dass }
  \[ |f(x) - f(x^*)| < \epsilon \quad \text{f"ur alle }
  \nowrap{x\in D\text{ mit } |x-x^*|<\delta(\epsilon).}
  \]
\end{definition}

\begin{remark}
Da die Menge aller $x$ mit $|x-x^*|<\delta$ genau die $\delta$-Umgebung $U_\delta(x^*)$ von $x^*$ ist, lässt sich diese Definition auch mit Umgebungen formulieren:

$f:D\to \R$ erfüllt das $\epsilon$-$\delta$-Kriterium an der Stelle $x^*\in D$, wenn es f"ur jedes $\epsilon >0$ ein \nowrap{$\delta = \delta(\epsilon) >0$} gibt, so dass
  \[ f(x)\in U_{\epsilon}(f(x^*))\quad \text{f"ur alle }
x\in D\cap U_\delta(x^*).
  \]
% Oder in Quantoren-Schreibweise:
% \[
% \forall \epsilon>0\:\exists \delta=\delta(\epsilon)>0\:\forall x\in D\cap U_\delta(x^\ast): f(x)\in U_\epsilon(f(x^\ast))
% \]
\end{remark}

Das folgende Video erläutert  das $\epsilon$-$\delta$-Kriterium und beweist die Unstetigkeit
der Dirichlet'schen Sprungfunktion.
\floatright{\href{https://api.stream24.net/vod/getVideo.php?id=10962-2-10915&mode=iframe&speed=true}{\image[75]{00_video_button_schwarz-blau}}}\\

Bevor wir wichtige Visualisierungen des $\epsilon$-$\delta$-Kriteriums betrachten, zeigt der folgende Satz seine Relevanz.


\begin{theorem}\label{thm:stetigkeit_folgenkrit_aequiv_epsilon-delta}
Sei $f : D \to \R$ eine Funktion und $x^* \in D$. Die Funktion
$f$ ist genau dann an der Stelle $x^*$ stetig, wenn sie dort das $\epsilon$-$\delta$-Kriterium erfüllt.
\end{theorem}


\begin{proof*}%[Beweis (Stetigkeit)]

Beweisskizze: Die Äquivalenz des Satzes wird in zwei Schritten gezeigt: Die Hin-Richtung
benutzt nur das $\epsilon$-$\delta$-Kriterium, um daraus die Stetigkeit, also die Konvergenz
der Folgen der Funktionswerte gegen den Funktionswert, zu zeigen. 
\\Die Rück-Richtung wird mittels indirektem Beweis gezeigt: Unter der
Annahme, dass $f$ an der Stelle $x^*$ das $\epsilon$-$\delta$-Kriterium verletzt, finden wir, dass die Folge der Funktionswerte
nicht gegen $f(x^*)$ konvergiert. Dies ist im Widerspruch zur  Stetigkeit der Funktion in $x^*$.\\
\begin{incremental}{0}
\step
Wir nehmen zunächst an, dass $f$ in $x^*\in D$ das  $\epsilon$-$\delta$-Kriterium erfüllt, und betrachten eine beliebige Folge $( x_n)_{n \geq 1}$ in $D$ mit 
$\lim_{n \to \infty} x_n = x^*$.
Dann ist zu zeigen, dass  $\lim_{n \to \infty} f ( x_n ) =f(x^*)$,
d.h. dass es zu jedem $\epsilon>0$ ein $N\in \N$ gibt mit
\[ |f(x_n)-f(x^*)|<\epsilon\quad \text{für alle }n\geq N.\]

Da $f$ das  $\epsilon$-$\delta$-Kriterium erfüllt ist, gibt es zu $\epsilon>0$ ein $\delta>0$ mit
\[ |f(x)-f(x^*)|<\epsilon\quad\text{für alle }x\in D\text{ mit }|x-x^*|<\delta. \]
Wegen $\lim_{n \to \infty} x_n = x^*$ gibt es zu diesem $\delta>0$ ein $N\in \N$ mit
\[ |x_n-x^*|<\delta\quad\text{für alle }n\geq N,\]
und daher auch
\[ |f(x_n)-f(x^*)|<\epsilon\quad\text{für alle }n\geq N.\]
Also ist auch das Folgenkriterium für Stetigkeit erfüllt.
\step

Nimmt man umgekehrt an, dass $f$ an der Stelle $x^*$ das  $\epsilon$-$\delta$-Kriterium verletzt, so ist zu zeigen, 
dass es eine Folge $( x_n)_{n \geq 1}$ in $D$ mit 
$\lim_{n \to \infty} x_n = x^*$ gibt, so dass die Folge $(f(x_n))_{n \geq 1}$ nicht gegen $f(x^*)$ konvergiert.

Da $f$  in $x^*$ das  $\epsilon$-$\delta$-Kriterium verletzt, gibt es (mindestens) ein $\epsilon>0$, 
% Vorschlag: %
zu dem kein $\delta>0$ existiert, welches das $\epsilon$-$\delta$-Kriterium erfüllt.
% statt: % so, dass kein gewünschtes $\delta>0$ existiert. 
Für dieses $\epsilon$ gilt also insbesondere, dass für $\delta=\frac{1}{n}$ ein $x_n\in D\cap U_\delta(x^*)$ existiert mit $|f(x_n)-f(x^*)|\geq \epsilon$.
Diese Wahl liefert also eine Folge $( x_n)_{n \geq 1}$ mit
\[ \lim_{n\to \infty} |x_n-x^*| \leq \lim_{n\to \infty} \frac{1}{n}
=0,\]
d.h. $\lim_{n\to \infty} x_n=x^*$.
Andererseits ist $|f(x_n)-f(x^*)|\geq \epsilon$ für alle $n\in \N$, weshalb die Folge $(f(x_n))_{n \geq 1}$ nicht gegen $f(x^*)$ konvergiert.
\end{incremental}
\end{proof*}


\begin{example}
Visualisierungen zur Stetigkeit einer Funktion an einer Stelle.
\begin{tabs*}
\tab{1. Visualisierung}
\begin{genericJSXVisualization}[550][800]{epsilon1}
		
		\begin{variables}
			\function{f}{rational}{x^2/2-x+1}  
%			\pointOnCurve{a1}{rational}{var(f)}{1}
%			\number{a}{rational}{2}
			\parametricFunction{ax}{real}{0, 1+t, 0.02, 2, 100}
			\pointOnParametricCurve[editable]{P}{real}{var(ax)}{0.8}
			\number{ups}{real}{var(P)[y]-1}
%-------------------variante mit editierbarem epsilon
%			\number[editable]{ups}{real}{1}
%			\point{P}{real}{0,var(ups)}
%-----------------------------------------------------
			\function{gr}{real}{1}
%			\function{l1}{real}{1+var(ups)}
%			\function{l2}{real}{1-var(ups)}
%			\number{b}{real}{1+2*var(ups)}
			\number{d}{real}{2*var(ups)/(2+var(ups))}
			\point{c00}{real}{2-var(d),1-var(ups)}
			\point{c01}{real}{2-var(d),1+var(ups)}
			\point{c10}{real}{2+var(d),1-var(ups)}
			\point{c11}{real}{2+var(d),1+var(ups)}
			% \line{v0}{real}{var(c00), var(c01)}
			% \line{v1}{real}{var(c10), var(c11)}
%			\set{s}{real}{|y-1|<var(ups) AND |x-2|<var(d)}
%			\set{s}{real}{|x-2|<var(d)}
      \verticalStrip{s}{real}{2,d}
      \horizontalStrip{H}{real}{1,ups}
			\point{m}{real}{2,0}
			\point{pc}{real}{2,1}
			
		\end{variables}
		\color{P}{#CC6600}
		%\label{P}{$\textcolor{BLUE}{P}$}
		\color{s}{#0066CC}%{#00CC00}
	%	\color{l1}{#CC6600}
	%	\color{l2}{#CC6600}
   \color{H}{#CC6600}
		\color{gr}{black}%{#0066CC}
    \style{gr}{dash:3}
		\color{m}{#0066CC}
		\color{pc}{BLACK}%{#0066CC}
		% \color{v0}{#00CC00}
		% \color{v1}{#00CC00}
		\color{f}{BLACK}
        
		\begin{canvas}
			\plotLeft{-0.3}
			\plotRight{4.3}
      \plotTop{2.5}
			\plotSize{400,300}
			\plot[coordinateSystem]{m, pc, s, gr,  P, f, H}%l1,l2, v0, v1,
		\end{canvas}
\end{genericJSXVisualization}
Die schwarze Kurve ist der Graph der Funktion $f:\R\to \R, x\mapsto \frac{x^2}{2}-x+1$, die an der Stelle $x^*=2$ den
		Funktionswert $f(x^*)=f(2)=1$ hat. Zur Visualisierung der Stetigkeit von $f$ an der Stelle $x^*=2$ wird hier für jedes $\epsilon>0$ ein
		passendes $\delta$ angegeben. 
  Graphisch ist ein passendes $\delta$ daran zu erkennen, dass innerhalb des $x$-Bereiches, der durch
		$2-\delta$ und $2+\delta$ beschränkt wird (blauer vertikaler Balken), der Funktionsgraph 
  im horizontalen orangenen $\epsilon$-Balken liegt.\\
	Für $\epsilon$ ist $\delta(\epsilon)$ eine Zahl, so dass
		$|f(x)-f(x^*)|<\epsilon$ für alle $x$ mit $|x-x^*|<\delta(\epsilon)$ erfüllt ist.\\ \\
		\textcolor{#0066CC}{Sie können den Wert von $\epsilon$ ändern, indem Sie den orangenen Punkt auf der $y$-Achse verschieben.}
	


\tab{2. Visualisierung}   	
%    	\begin{genericJSXVisualization}[550][800]{epsilon2}
		
% 		\begin{variables}
% 			\function{f}{rational}{x^2/2-x+1}  
% %			\pointOnCurve{a1}{rational}{var(f)}{1}
% %			\number{a}{rational}{2}
% 			\parametricFunction{ax}{real}{2+t, 0, 0.02, 2, 100}
% 			\pointOnParametricCurve[editable]{Q}{real}{var(ax)}{1.5}	
% 			\number{ups}{real}{var(Q)[x]-2}
% %-------------------variante mit editierbarem epsilon
% 			% \number[editable]{ups}{real}{0.8}
% 			% \number{ups1}{real}{var(ups)}
% 			% \point{P}{real}{0,var(ups)}
% %-----------------------------------------------------
%  			\function{gr}{real}{1}
% 			\function{l1}{real}{1.5}
% 			\function{l2}{real}{0.5}
% %			\number{b}{real}{1+2*var(ups)}
% 			\number{d}{real}{2*var(ups)/(2+var(ups))}
%       \point{c00}{real}{2-var(ups),1-var(ups)}
%       \point{c01}{real}{2-var(ups),1+var(ups)}
% 			\point{c10}{real}{2+var(ups),1-var(ups)}
% 			\point{c11}{real}{2+var(ups),1+var(ups)}
%  		  \line{v0}{real}{var(c00), var(c01)}
%   		\line{v1}{real}{var(c10), var(c11)}
% % %			\set{s}{real}{|y-1|<var(ups) AND |x-2|<var(d)}
%  			\set{s}{real}{|x-2|<var(ups)}
%  			\point{m}{real}{2,0}
%  			\point{pc}{real}{2,1}

% 			% \number{x1}{real}{2+var(d)-0.01}  % -0.01 als Toleranz. 
% 			% \number{fx1}{real}{(var(x1))^2/2-var(x1)+1}
% 			% \number{t}{real}{var(fx1)-1-var(ups)-0.01} % -0.01 als Toleranz. 
% 			% \number{lb}{real}{2-var(d)}
% 			% \number{ub}{real}{2+var(d)}

			
% 		\end{variables}
% 		\color{P}{#CC6600}
% 		\color{Q}{#00CC00}
% 		%\label{P}{$\textcolor{BLUE}{P}$}
% 		\color{s}{#00CC00}
% 		\color{l1}{#CC6600}
% 		\color{l2}{#CC6600}
% 		\color{gr}{#0066CC}
% 		\color{m}{#0066CC}
% 		\color{pc}{#0066CC}
% 		\color{v0}{#00CC00}
% 		\color{v1}{#00CC00}
%         \color{f}{BLACK}

% 		\begin{canvas}
% 			\updateOnDrag[false]
% 			\plotLeft{-0.3}
% 			\plotRight{4.3}
% 			\plotSize{400,300}
% 			\plot[coordinateSystem]{m, pc, s, gr,l1,l2, v0, v1, Q, f}
% 		\end{canvas}
% \end{genericJSXVisualization}
%% Kathrin neu:
  	\begin{genericJSXVisualization}[550][800]{stetigkeit_2}
		
		\begin{variables}
			\function{f}{rational}{x^2/2-x+1}  
%			\pointOnCurve{a1}{rational}{var(f)}{1}
%			\number{a}{rational}{2}
			\parametricFunction{ax}{real}{2+t, 0, 0.02, 2}
			\pointOnParametricCurve[editable]{Q}{real}{var(ax)}{1.5}		
			\number{d}{real}{var(Q)[x]-2}
%-------------------variante mit editierbarem epsilon
			\number[editable]{ups}{real}{0.8}
			\number{ups1}{real}{var(ups)}
			\point{P}{real}{0,var(ups)}
%-----------------------------------------------------
			\function{gr}{real}{1}
%			\function{l1}{real}{1+var(ups)}
%			\function{l2}{real}{1-var(ups)}
      \horizontalStrip{H}{real}{1,ups}
%			\number{b}{real}{1+2*var(ups)}
%			\number{d}{real}{2*var(ups)/(2+var(ups))}
			% \point{c00}{real}{2-var(d),1-var(ups)}
			% \point{c01}{real}{2-var(d),1+var(ups)}
			% \point{c10}{real}{2+var(d),1-var(ups)}
			% \point{c11}{real}{2+var(d),1+var(ups)}
			% \line{v0}{real}{var(c00), var(c01)}
			% \line{v1}{real}{var(c10), var(c11)}
%			\set{s}{real}{|y-1|<var(ups) AND |x-2|<var(d)}
%			\set{s}{real}{|x-2|<var(d)}
      \verticalStrip{s}{real}{2,d}
			\point{m}{real}{2,0}
			\point{pc}{real}{2,1}

			\number{x1}{real}{2+var(d)-0.01}  % -0.01 als Toleranz. 
			\number{fx1}{real}{(var(x1))^2/2-var(x1)+1}
			\number{T}{real}{var(fx1)-1-var(ups)-0.01} % -0.01 als Toleranz. 
			\number{lb}{real}{2-var(d)}
			\number{ub}{real}{2+var(d)}
   %für die roten Abschnitte des Graphen, wenn das Kriterium nicht erfüllt ist:
       %linke Grenze des rechten roten Abschnitts:
       \number{minrechts}{real}{(1+sqrt(1+2*ups)+2+d-|1+sqrt(1+2*ups)-2-d|)/2}
       \parametricFunction{frechts}{real}{x,f,minrechts,2+d}
      %\number{minimum}{real}{1-d+theta(1-2*ups)*(d-1+1+sqrt(|1-2*ups|))}
      %rechte Grenze des linken roten Abschnitts wird nur angezeigt, wenn epsilon<1/2. 
      \number{minimum}{real}{2-d+(1- sqrt(|1-2*ups|)-2+d)*theta(1-2*ups)*theta(1-sqrt(|1-2*ups|)-2+d)}
      \parametricFunction{flinks}{real}{x,f,2-d,minimum}
      %Kosmetik dient dazu, dass auf Rand der delta-Umgebung kein roter Punkt erscheint.
      \point{kosmetikrechts}{real}{2+d,(2+d)^2/2-(2+d)+1}
      \point{kosmetiklinks}{real}{2-d,(2-d)^2/2-(2-d)+1}

			
		\end{variables}
		\color{P}{#CC6600}
		\color{Q}{#0066CC}%{#00CC00}
		%\label{P}{$\textcolor{BLUE}{P}$}
		\color{s}{#0066CC}%{#00CC00}
		% \color{l1}{#CC6600}
		% \color{l2}{#CC6600}
    \color{H}{#CC6600}
		\color{gr}{BLACK}
    \style{gr}{dash:3}
		\color{m}{#0066CC}%
		\color{pc}{BLACK}%{#0066CC}
		% \color{v0}{#00CC00}
		% \color{v1}{#00CC00}
    \color{f}{BLACK}
    \color{frechts}{RED}
    \color{flinks}{RED}
    \color[0.1]{kosmetikrechts}{#0066CC}
    \style{kosmetikrechts}{size:0.1,strokeWidth:0}
    \color[0.1]{kosmetiklinks}{#0066CC}
    \style{kosmetiklinks}{size:0.1,strokeWidth:0}
    

		\begin{canvas}
			\updateOnDrag[false]
			\plotLeft{-0.3}
			\plotRight{4.3}
     \plotTop{2.5}
			\plotSize{400,300}
			\plot[coordinateSystem]{m, pc, s, gr, H,  Q, f, frechts, flinks, kosmetikrechts, kosmetiklinks} %l1,l2, v0, v1,
		\end{canvas}
		\text{Die schwarze Kurve ist wieder der Graph der Funktion $f:\mathbb{R}\to \mathbb{R}, x\mapsto \frac{x^2}{2}-x+1$, die an der Stelle $x^*=2$ den
		Funktionswert $f(x^*)=f(2)=1$ hat.
  	Nun sollen Sie selbst zu gegebenem $\epsilon>0$ ein passendes $\delta>0$ finden, so dass
		$|f(x)-f(x^*)|<\epsilon$ für alle $x$ mit $|x-x^*|<\delta$.}
  \text{Wählen Sie dazu zunächst den Wert: $\epsilon= \var{ups}$.}
  \text{Passen Sie anschließend $\delta$ durch Verschieben des rechten blauen Punktes in der Graphik an.}
  \text{	\IFELSE{var(T) < 0}{
   	 Richtig! Sie haben ein geeignetes $\delta$ gefunden, 
    denn für $\delta=\var{d}$ gilt: Für alle $x$ zwischen $x^*-\delta=\var{lb}$ und $x^*+\delta=\var{ub}$
		ist $|f(x)-f(x^*)|<\epsilon=\var{ups1}$.}{
  Für Ihre Wahl 
   $\delta=\var{d}$ 
  ist die Bedingung
		$|f(x)-f(x^*)|<\epsilon=\var{ups1}$ nicht für alle $x$ mit $|x-x^*|<\delta$
		
  erfüllt. Es gibt $x$ im blauen vertikalen $\delta$-Streifen, deren Werte $f(x)$ nicht 
  im orangenen horizontalen $\epsilon$-Streifen liegen,
    für die also $\vert f(x)-f(x^\ast)\vert\geq\epsilon$ gilt. Entsprechende Graphenstücke sind rot markiert. 
    Passen Sie $\delta$ an, um das zu vermeiden.
  		}}
   	\end{genericJSXVisualization}
% Die schwarze Kurve ist wieder der Graph der Funktion $f:\R\to \R, x\mapsto \frac{x^2}{2}-x+1$, welche an der Stelle $x^*=2$ den
% 		Funktionswert $f(x^*)=f(2)=1$ hat.\\
% 		In dieser Visualisierung sollen Sie selbst zu gegebenem $\epsilon>0$ ein passendes $\delta>0$ finden, so dass
% 		$|f(x)-f(x^*)|<\epsilon$ für alle $x$ mit $|x-x^*|<\delta$.\\ 
 	%	Wählen Sie dazu zunächst den Wert: $\epsilon= \var{ups}$.\\ 
 	% 	Passen Sie anschließend $\delta$ durch Verschieben des grünen Punktes in der Graphik an.\\ \\		 
		% \IFELSE{var(t) < 0}{\textcolor{#0066CC}{Richtig!} Für $\delta=\var{d}$ gilt: Für alle $x$ zwischen $x^*-\delta=\var{lb}$ und $x^*+\delta=\var{ub}$
		% gilt $|f(x)-f(x^*)|<\epsilon=\var{ups1}$}{\textcolor{#CC6600}{Für }$\delta=\var{d}$ \textcolor{#CC6600}{ ist die Bedingung }\\
		% $|f(x)-f(x^*)|<\epsilon=\var{ups1}$ für alle $x$ mit $|x-x^*|<\delta$ \\
		% \textcolor{#CC6600}{nicht erfüllt. Zum Beispiel ist für }$x=\var{x1}<2+\delta=\var{ub}$:\\  $f(x)\approx\var{fx1}\geq 1+\var{ups1}= f(x^*)+\epsilon$.  	
   	\end{tabs*}
\end{example}
   	
\begin{example}   	
Visualisierung zur Unstetigkeit einer Funktion an einer Stelle.
\begin{genericJSXVisualization}[550][800]{unstetig}
		
		\begin{variables}
			\function{f}{real}{abs(x-3)/(x^2-3*x)+sqrt(x)-x/sqrt(x)}   % +-sqrt(x) um auf R_+ einzuschränken.
%			\pointOnCurve{a1}{rational}{var(f)}{1}
%			\number{a}{rational}{2}
			\parametricFunction{ax}{real}{3+t, 0, 0.02, 2}
			\pointOnParametricCurve[editable]{Q}{real}{var(ax)}{2}		
			\number{d}{real}{var(Q)[x]-3}
%-------------------variante mit editierbarem epsilon
			\number[editable]{ups}{real}{1}
			\number{ups1}{real}{var(ups)}
			\point{P}{real}{0,var(ups)}
%-----------------------------------------------------
			\function{gr}{real}{1/3}
			% \function{l1}{real}{1/3+var(ups)}
			% \function{l2}{real}{1/3-var(ups)}
       \horizontalStrip{H}{real}{1/3,ups}
%			\number{b}{real}{1+2*var(ups)}
%			\number{d}{real}{2*var(ups)/(2+var(ups))}
			\point{c00}{real}{3-var(d),1/3-var(ups)}
			\point{c01}{real}{3-var(d),1/3+var(ups)}
			\point{c10}{real}{3+var(d),1/3-var(ups)}
			\point{c11}{real}{3+var(d),1/3+var(ups)}
			\line{v0}{real}{var(c00), var(c01)}
			\line{v1}{real}{var(c10), var(c11)}
%			\set{s}{real}{|y-1|<var(ups) AND |x-2|<var(d)}
%			\set{s}{real}{|x-3|<var(d)}
       \verticalStrip{s}{real}{3,d}
			\point{m}{real}{3, 0}
			
			\number{ed}{real}{1/3}
			\point{pc}{real}{3, var(ed)}

			\number{x1}{real}{3-var(d)+0.01}  % -0.01 als Toleranz. 
			\number{fx1}{real}{(3-var(x1))/(var(x1)^2-3*var(x1))}
			\number{T}{real}{1/3-var(fx1)-var(ups)-0.01} % -0.01 als Toleranz. 
			\number{lb}{real}{3-var(d)}
			\number{ub}{real}{3+var(d)}
			\number{x2}{real}{3-var(d)/2}
			\number{fx2}{real}{(3-var(x2))/(var(x2)^2-3*var(x2))}
     %Nur für große ups>=2/3 wird fex rot:
      \number{hilfswert}{real}{ups-1/3+dirac(ups-1/3)*1/3}
       \number{untererWert}{real}{1/(ups-1/3)}
     %\number{minimum}{real}{3+theta(3/2*ups-1)*(1/(hilfswert)-3)}
     \number{minimum}{real}{3-d+theta(1/(ups-1/3)-3+d)*(1/(ups-1/3)-3+d)}
     %\number{minimum}{real}{3+theta(3/2*ups-1)*theta(1/(ups-1/3)-3+d)*(1/(ups-1/3)-3)}
     \parametricFunction{fex}{real}{x, f,  3-d,minimum}
     %Nur für kleinere ups<2/3 wird fex2 rot:
     \number{maximum}{real}{3+theta(3/2*ups-1)*(-d)}
     \parametricFunction{fex2}{real}{x, f,  3-d,maximum}
      %Kosmetik dient dazu, dass auf Rand der delta-Umgebung kein roter Punkt erscheint.
      \point{kosmetik}{real}{3-d,-1/(3-d)}
			
		\end{variables}
		\color{P}{#CC6600}
		\color{Q}{BLUE}
		%\label{P}{$\textcolor{BLUE}{P}$}
		\color[0.9]{s}{#0066CC}%grün:{#00CC00}
		% \color{l1}{#CC6600}
		% \color{l2}{#CC6600}
      \color[0.9]{H}{#CC6600}
		\color{gr}{BLACK}% blau:{#0066CC}
    \style{gr}{dash:3}
		\color{m}{#0066CC}
		\color{pc}{BLACK}
		% \color{v0}{#00CC00}
		% \color{v1}{#00CC00}
    \color{fex}{RED}
    \color{fex2}{RED}
    \color{f}{BLACK}
    \color[0.1]{kosmetik}{#0066CC}
    \style{kosmetik}{size:0.1,strokeWidth:0}    

		\begin{canvas}
			\updateOnDrag[false]
			\plotLeft{0.7}
			\plotRight{5.3}
			\plotSize{400,300}
			\plot[coordinateSystem]{m, pc, s, gr, H,  Q, f, fex, fex2, kosmetik}%l1,l2,v0, v1,
		\end{canvas}
		\text{Die schwarze Kurve ist der Graph der Funktion $f:\mathbb{R}_{>0}\to \mathbb{R}$ mit $f(x)=\frac{|x-3|}{x^2-3x}$ für $x\neq 3$ und $f(3)=\frac{1}{3}$. 
		Wir betrachten die Stelle $x^*=3$ mit dem Funktionswert $f(x^*)=f(3)=\frac{1}{3}$, an der die Funktion nicht stetig ist.}
  \text{		In dieser Visualisierung sollen Sie selbst zu gegebenem $\epsilon>0$ ein passendes $\delta>0$ suchen, so dass
		$|f(x)-f(x^*)|<\epsilon$ für alle $x$ mit $|x-x^*|<\delta$.}
  \text{
 		Wählen Sie dazu zunächst den Wert: $\epsilon= \var{ups}$. (Sie sollten insbesondere auch kleine Werte wählen.)}
   \text{Passen Sie anschließend $\delta$ durch Verschieben des rechten blauen Punktes in der Graphik an.}	
   \text{
  \IFELSE{var(ups)<= 2/3}{Für dieses $\epsilon=\var{ups1}$ können Sie ausprobieren, so lange Sie wollen: 
     Sie finden kein $\delta$ so, dass die Bedingung $f(x)\in (\frac{1}{3}-\epsilon;\frac{1}{3}+\epsilon)=(\frac{1}{3}-\var{ups1};\frac{1}{3}+\var{ups1})$
     erfüllt ist für alle $x$ in im Intervall $(x^*-\delta;x^*+\delta)$. Für alle $x$ aus dem Teilintervall $(x^\ast-\delta;x^\ast)$ 
     ist nämlich  $f(x)<\frac{1}{3}-\var{ups1})\frac{1}{3}-\epsilon$. 
     Der entsprechende Abschnitt des Graphen ist rot markiert.
 		Da es für dieses $\epsilon$ kein passendes $\delta$ gibt, ist diese Funktion $f$ an der Stelle $x^*=3$ nicht stetig.
 		}{
		\IFELSE{var(T) < 0}{Sie haben für dieses $\epsilon=\var{ups1}$ ein geeignetes  $\delta=\var{d}$ gefunden: 
    Für alle $x$ im Intervall  $(x^*-\delta=\var{lb};x^*+\delta=\var{ub})$
		gilt $|f(x)-f(x^*)|<\epsilon=\var{ups1}$. Wählen Sie nun ein kleineres $\epsilon$ und versuchen Sie erneut,
    $\delta$ anzupassen.
    }{Für $\epsilon=\var{ups1}$ ist Ihre Wahl von $\delta=\var{d}$ nicht ausreichend. 
    Im Intervall $(x^*-\delta;x^*+\delta)=(x^*-\var{lb};x^*+\var{ub})$ gibt es Funktionswerte, 
    die nicht zu $(\frac{1}{3}-\epsilon;\frac{1}{3}+\epsilon)=(\frac{1}{3}-\var{ups1};\frac{1}{3}+\var{ups1})$ gehören. 
    Der entsprechende Abschnitt des Graphen ist rot markiert. 
    Verkleinern Sie $\delta$ solange, bis der rote Abschnitt verschwindet.}
		} }
  	\end{genericJSXVisualization}
\end{example}

Das folgende Video erläutert den Begriff der Stetigkeit sehr anschaulich und verdeutlicht ihn an Beispielen:
\floatright{\href{https://www.hm-kompakt.de/video?watch=407}{\image[75]{00_Videobutton_schwarz}}}\\\\
 
\begin{example}
Wir zeigen das $\epsilon$-$\delta$-Kriterium für die Funktionen aus Beispiel \ref{ex:einfach_stetig}.
\begin{tabs*}
\tab{konstante Funktionen} Für festes $c\in \R$ ist die konstante Funktion $f:\R\to \R,x\mapsto c$ stetig auf ganz $D=\R$.\\
Ist $x^*\in \R$ eine beliebige Stelle, so gilt für jedes $\epsilon>0$:
\[   |f(x)-f(x^*)|=|c-c|=0<\epsilon \]
für alle $x\in \R$. Insbesondere kann man stets $\delta=1$ wählen (oder irgendeinen anderen Wert $>0$) und erhält:
\[   |f(x)-f(x^*)|=|c-c|=0<\epsilon \quad \text{f"ur alle }
  \nowrap{x\in D,\text{ mit } |x-x^*|<\delta}. \]
\tab{lineare Funktionen} Für feste reelle Zahlen $a,b\in \R$ mit
$a \neq 0$ sei $f: \R \to \R, x \mapsto ax + b$. Sei $x^* \in \R$ und $\epsilon > 0$. Wir wählen
$\delta  = \frac{\epsilon}{| a |}$ . Für alle $x \in \R$ mit $| x - x^* | < \delta$ gilt dann
\[ |f ( x ) - f ( x^* )| = |ax + b - ax^* - b| = | a | \cdot | x - x^* | < | a | \cdot \delta = \epsilon. \]
Also ist $f$ stetig auf $\R$.
\tab{$f(x)=x^2$} Sei $f : [-1; 1 ] \to \R, x \mapsto x^2$ .\\
Für $x^* \in [-1; 1 ]$ und $\epsilon > 0$ sei $\delta =\frac{\epsilon}{2}> 0$.\\
Dann gilt für alle
$x \in [-1; 1 ]$ mit $| x - x^* | < \delta$:
\[ |f ( x ) - f ( x^* )| = |x^2 - (x^*)^2| = | x+x^* |\cdot | x - x^* | 
 \leq 2 | x - x^* | < 2\delta = \epsilon.\]
Also ist $f$ stetig auf $[-1; 1 ]$.
\end{tabs*}
\end{example}   	

\begin{example}\label{ex:gauss-klammer}
Wir zeigen, dass das $\epsilon$-$\delta$-Kriterium in den Fällen aus Beispiel \ref{ex:unstetig} verletzt ist.
\begin{tabs*}
\tab{Gauß-Klammer-Funktion} 
Allgemeiner als für die Heaviside-Funktionen zeigen wir die Unstetigkeit für die durch die Gauß-Klammer gegebene Funktion
\[   f:\R\to \Z, x\mapsto [x]:=\max\{ n\in \Z | n\leq x\}. \]
Deren Graph weist sehr viele \emph{Sprünge} auf. 

\begin{center}
\image{T204_Example_A}
\end{center}

An den \emph{Sprungstellen} ist diese Funktion nicht stetig:\\
Ist nämlich $x^*=n\in \Z$, so ist $[x^*]=n$ und für jedes $x\in \R$ mit $x<x^*$ gilt $[x]\leq n-1$.\\
Insbesondere ist $|f(x)-f(x^*)|=|[x]-n|\geq 1$ für alle $x<x^*$.

Wählt man also ein $\epsilon\in (0;1]$ (z.B. $\epsilon=\frac{1}{2}$), so gilt für alle $x<x^*$
\[  |f(x)-f(x^*)|\geq 1\geq \epsilon. \]
Da es aber für jedes $\delta>0$ auch ein $x<x^*$ mit $|x-x^*|<\delta$ gibt (z.B. $x=x^*-\delta/2$), gibt es also zu keinem $\epsilon\in (0;1]$ ein gewünschtes $\delta$. Die Gauß-Klammer-Funktion ist also an der Stelle $x^*\in \Z$ nicht stetig.

Für $x^*\in \R\setminus \Z$ ist die Funktion an der Stelle $x^*$ jedoch stetig. In einer kleinen Umgebung von $x^*$ ist die Funktion nämlich konstant.

\begin{incremental}[\initialsteps{0}]
\step Genauer: Wählt man $\delta=\min\{x^*-[x^*]; [x^*]+1-x^*\}$, so gilt
\[ f(x)=[x]=[x^*]=f(x^*)\quad \text{für alle }x\text{ mit }|x-x^*|<\delta \]
(denn $x< x^*+\delta\leq x^*+[x^*]+1-x^*=[x^*]+1$ und
$x>x^*-\delta\geq x^*-(x^*-[x^*])=[x^*]$ nach Definition von $\delta$.)
Insbesondere gilt für beliebiges $\epsilon>0$
\[ |f(x)-f(x^*)|=0<\epsilon\quad \text{für alle }x\text{ mit }|x-x^*|<\delta. \]
\end{incremental}
\tab{Dirichletsche Sprungfunktion} 
Sei 
\[  D:\R\to {\{0;1\}}, x \mapsto \begin{cases} 1 & \text{falls }x\in \Q \\
0 & \text{falls }x\notin \Q \end{cases}.\]
die \ref[reelle-funk][Dirichletsche Sprungfunktion]{ex:unstetige-funktionen}.

Ist $x^*$ eine rationale Zahl, so gibt es in jeder $\delta$-Umgebung $U_{\delta}(x^*)$ von $x^*$ eine irrationale Zahl $x_1$. Das bedeutet, dass es zu jedem $\delta>0$ ein $x_1$ mit $|x_1-x^*|<\delta$ gibt, so dass
\[ |D(x_1)-D(x^*)| =|0 -1|=1. \]
Wählt man also zum Beispiel $\epsilon = \frac{1}{2}$, dann gibt es
kein $\delta>0$ mit
\[ |D(x)-D(x^*)|<\epsilon\quad\text{für alle }x\text{ mit }|x-x^*|<\delta, \]
da es für die irrationalen $x$ nicht erfüllt ist.

Also ist die Dirichletsche Sprungfunktion an jeder rationalen Stelle nicht stetig.

Ganz entsprechend sieht man, dass diese Funktion auch an den irrationalen Stellen $x^* \in \R \setminus \Q$ nicht stetig ist, da es in jeder $\delta$-Umgebung einer irrationalen Zahl auch immer eine rationale Zahl gibt.
\end{tabs*}
\end{example}

\begin{remark}
Sei $x^*$ ein \emph{isolierter Punkt} von $D$, d. h. ein Punkt, für den es ein $\delta > 0$ gibt mit $U_{\delta}( x^*)\cap D =
\{ x^* \}$. Dann ist $f$ stetig in $x^*$ (weil man für jedes $\epsilon > 0$ dieses $\delta$ wählen kann). Dies wird in der Literatur zum Teil anders definiert.
\end{remark}

\section{Links- und Rechtsseitige Stetigkeit}\label{sec:einseitige-stetigkeit}

Als Verallgemeinerung zur Stetigkeit führen wir noch die links- bzw. rechtsseitige Stetigkeit ein. Hierbei werden nur Folgen betrachtet, deren Folgeglieder auf dem Zahlenstrahl alle links bzw. alle rechts der betrachteten Stelle liegen.
Dies steht in Analogie zum Begriff des 
\ref[content_31_grenzwerte_von_funktionen][einseitigen Grenzwerts]{sec:einseitige-grenzwerte}
\begin{definition} 
Sei $f : D \to \R$ eine Funktion und $x^* \in D$.
\begin{enumerate}
\item[(i)] $f$ ist \notion{linksseitig stetig} in $x^*$, wenn für jede Folge $( x_n )_{n \geq 1}$ in $D$ mit
$\lim_{n \to \infty} x_n = x^*$ und $x_n \leq x^*$  für alle $n \in \N$ gilt 
\[ \lim_{n \to \infty} f ( x_n ) = f ( x^* ) .\]
\item[(ii)] $f$ ist \notion{rechtsseitig stetig} in $x^*$, wenn für jede Folge $( x_n )_{n \geq 1}$ in $D$ mit
$\lim_{n \to \infty} x_n = x^*$ und $x_n \geq x^*$  für alle $n \in \N$ gilt 
\[ \lim_{n \to \infty} f ( x_n ) = f ( x^* ) .\]
\end{enumerate}
\end{definition}

\begin{remark}
\begin{itemize}
\item[(a)]
Mit dem Begriff des \ref[content_31_grenzwerte_von_funktionen][einseitigen Funktionsgrenzwerts]{sec:einseitige-grenzwerte}
lässt sich die einseitige Stetigkeit kompakt formulieren:
\begin{enumerate}
\item[(i)] 
$f$ ist linksseitig stetig in $x^\ast$, genau dann wenn der linksseitige Grenzwert gegen $x^\ast$ existiert und mit dem
Funktionswert übereinstimmt, $\lim_{x\nearrow x^*} f(x)=f(x^\ast)$.
\item[(ii)]
$f$ ist rechtsseitig stetig in $x^\ast$ genau dann, wenn der rechtsseitige Grenzwert gegen $x^\ast$ existiert und mit dem
Funktionswert übereinstimmt, $\lim_{x\searrow x^*} f(x)=f(x^\ast)$.
\end{enumerate}
\item[(b)]
Wie auch die Stetigkeit, so lässt sich die einseitige Stetigkeit durch ein 
$\epsilon$-$\delta$-Kriterium darstellen:
\begin{enumerate}
\item[(i)]
$f$ ist linksseitig stetig in $x^*$, wenn es f"ur jedes $\epsilon >0$ ein \nowrap{$\delta = \delta(\epsilon) >0$} gibt, so dass 
  \[ |f(x) - f(x^*)| < \epsilon \quad \text{f"ur alle }
  \nowrap{x\in D,\text{ mit } x^*-\delta(\epsilon)< x\leq x^*.}
  \]
  \item[(ii)]
$f$ ist rechtsseitig stetig in $x^*$, wenn es f"ur jedes $\epsilon >0$ ein \nowrap{$\delta = \delta(\epsilon) >0$} gibt, so dass 
  \[ |f(x) - f(x^*)| < \epsilon \quad \text{f"ur alle }
  \nowrap{x\in D,\text{ mit } x^*\leq x< x^*+\delta(\epsilon).}
  \]
  \end{enumerate}
\end{itemize}
\end{remark}

\begin{example}
\begin{enumerate}
\item
Ist eine Funktion $f$ stetig in $x^*$ ist sie natürlich auch linksseitig und rechtsseitig stetig, da die Bedingung für die Folgen der Funktionswerte lediglich für weniger Folgen $( x_n )_{n \geq 1}$ erfüllt sein müssen.
\item Die Gauß-Klammer-Funktion $[\cdot]$ ist an allen Stellen rechtsseitig stetig. An den ganzzahligen Stellen ist sie jedoch nicht linksseitig stetig, wie oben gezeigt wurde.
\\
Ebenso ist die Heaviside-Funktion in $x^\ast=0$ nur rechtsseitig stetig.
\item Die Dirichletsche Sprungfunktion ist an keiner Stelle linksseitig stetig und auch nicht rechtsseitig stetig.
\end{enumerate}
\end{example}

Die Bedeutung der links- und rechtsseitigen Stetigkeit wird ersichtlich durch

\begin{theorem}
$f:D\to \R$ ist genau dann stetig in $x^*\in D$ , wenn $f$ linksseitig und rechtsseitig stetig in $x^*$ ist.
\end{theorem}

\begin{proof*}
Dass Stetigkeit die links- und rechtsseitige Stetigkeit impliziert, wurde ja schon erklärt.
\begin{incremental}{0}
\step
Sei nun $f$ in $x^*$ links- und rechtsseitig stetig. Für  
$\epsilon>0$ muss dann ein passendes $\delta$ gefunden werden.
Da $f$ in $x^*$ links- und rechtsseitig stetig ist, gibt es $\delta_1>0$ mit
  \[ |f(x) - f(x^*)| < \epsilon \quad \text{f"ur alle }
  x\in D,\text{ mit } x^*-\delta_1< x\leq x^*
  \]
 und $\delta_2>0$ mit
  \[ |f(x) - f(x^*)| < \epsilon \quad \text{f"ur alle }
  x\in D,\text{ mit } x^*\leq x< x^*+\delta_2.
  \]
Daher ist
  \[ |f(x) - f(x^*)| < \epsilon \quad \text{f"ur alle }
  x\in D,\text{ mit } x^*-\delta_1< x< x^*+\delta_2.
  \]
Wählt man nun $\delta=\min\{\delta_1;\delta_2\}$, so ist insbesondere
  \[ |f(x) - f(x^*)| < \epsilon \quad \text{f"ur alle }
  x\in D,\text{ mit } x^*-\delta< x< x^*+\delta,
  \]
  d.h. für alle $x\in D$ mit $|x-x^*|<\delta$. 
\end{incremental}
\end{proof*}

\begin{example}
Wir  zeigen, dass die Betragsfunktion $f:\R\to \R, x\mapsto |x|$ auf ganz $\R$ stetig ist.

Betrachten wir zunächst die Stelle $x^*=0$.\\ Für alle gegen $0$ konvergente Folgen $( x_n )_{n \geq 1}$ mit $x_n\leq 0$ gilt hier
\[ \lim_{n \to \infty} {|x_n|} =\lim_{n \to \infty} -x_n =- \lim_{n \to \infty} x_n=-0={|0|}. \]
Also ist die Betragsfunktion in $0$ linksseitig stetig.

Für alle gegen $0$ konvergente Folgen $( x_n )_{n \geq 1}$ mit $x_n\geq 0$ gilt hier
\[ \lim_{n \to \infty} {|x_n|} =\lim_{n \to \infty} x_n =0={|0|}. \]
Also ist die Betragsfunktion in $0$ rechtsseitig stetig.

Damit ist die Betragsfunktion in $0$ auch stetig.\\ \\

Für alle anderen Stellen verwenden wir ein Mittel, das häufig hilfreich ist: Um Stetigkeit an einer bestimmten Stelle 
$x^*$ zu untersuchen, reicht es gegen $x^*$ konvergente Folgen $( x_n )_{n \geq 1}$ zu betrachten, deren
Folgeglieder alle in einer kleinen Umgebung von $x^*$ liegen. Das liegt daran, dass wir nur an dem Grenzwert
der Folge $( f(x_n) )_{n \geq 1}$ interessiert sind, und daher die "`ersten Folgeglieder"' uninteressant sind.
Da die Folge $( x_n )_{n \geq 1}$ aber gegen $x^*$ konvergiert, liegen ab einem bestimmten $N\in \N$ alle Folgeglieder
$x_n$ mit $n\geq N$ in dieser kleinen Umgebung von $x^*$, und wir hätten direkt mit der Folge $( x_n )_{n \geq N}$ starten
können.

Betrachten wir nun zunächst eine Stelle $x^*>0$: In einer Umgebung von $x^*$ (genauer sogar auf
dem Intervall $(0;2x^*)=U_{x^*}(x^*)$ ist die Betragsfunktion identisch mit der linearen Funktion $x\mapsto x$.
Da letztere bei $x^*$ stetig ist, ist die Betragsfunktion bei $x^*$ stetig.

Entsprechend ist die Betragfunktion an allen negativen Stellen $x^*$ stetig, weil sie auf der Umgebung $(2x^*;0)=U_{-x^*}(x^*)$
mit der linearen Funktion $x\mapsto -x$ identisch ist.

Somit ist die Betragsfunktion auf ganz $\R$ stetig.
\end{example}

\begin{quickcheck}
%\type{mc.multiple}%
    \lang{de}{\text{Unter Beispiel \ref{ex:gauss-klammer} findet sich der Graph der Gauß-Klammer-Funktion. Wir betrachten jetzt:
    \[   f:\R\to \Z, x\mapsto [-x] \]\\Welche der folgenden Aussagen ist wahr?}}
    \begin{choices}{multiple}
        \begin{choice}
            \text{Die Funktion $f(x)$ ist unstetig an $x_0=1$.}
            \solution{true}
        \end{choice}
        \begin{choice}
            \text{Die Funktion $f(x)$ ist linksseitig stetig an $x_0=1$.}
            \solution{true}
        \end{choice}
        \begin{choice}
            \text{Die Funktion $f(x)$ ist rechtsseitig stetig an $x_0=1$.}
            \solution{false}
        \end{choice}
        \begin{choice}
            \text{Die Funktion $f(x)$ ist stetig an $x_0=1$.}
            \solution{false}
        \end{choice}
        
     \end{choices}
     \explanation{Die hier betrachtete Funktion ist die an der y-Achse gespiegelte Gauß-Klammer-Funktion.
     Deshalb ist sie an $x_0=1$ linksseitig stetig, aber nicht rechtsseitig stetig. Demnach ist sie insgesamt unstetig.}
\end{quickcheck}

Wie sich der Stetigkeitsbegriff auf komplexe Funktionen erweitern lässt, zeigt das folgende Video:
\floatright{\href{https://api.stream24.net/vod/getVideo.php?id=10962-2-10921&mode=iframe&speed=true}{\image[75]{00_video_button_schwarz-blau}}}\\


% Alte Version
% Viele Vorgänge in der Natur lassen sich mit stetigen Funktionen beschreiben. Betrachtet
% man z.B. in einem Weg-Zeit-Diagramm die zurückgelegte Wegstrecke als Funktion
% der Zeit, so wird es hier keine Sprünge geben. Die Weg-Zeit-Funktion ist stetig.

% Es gibt im wesentlichen zwei Möglichkeiten, die Stetigkeit von Funktionen zu definieren. Die eine verwendet ein sogenanntes
% $\epsilon$-$\delta$-Kriterium, die andere verwendet Folgen. Beide Definitionen sind gleichwertig, da der 
% identische Begriff definiert wird. Stetigkeit ist ein punktweiser Begriff, von dem sich Stetigkeit auf 
% einer Menge ableitet.

% Anschaulich gesprochen wird eine stetige Funktion auf einem \emph{Intervall} dadurch charakterisiert, dass man ihren Graphen
% „ohne Absetzten durchzeichnen“ kann, dass also keine Sprünge auftreten.


% \section{$\epsilon$-$\delta$-Kriterium}\label{sec:eps-delta}

% \begin{definition}
% Sei $f$ eine auf $D\subset\R$ definierte reelle Funktion und sei $x^*\in D$.
%  \lang{de}{$f$ heißt \notion{stetig an der Stelle} $x^*\in D$, wenn es f"ur jedes $\epsilon >0$ ein \nowrap{$\delta = \delta(\epsilon) >0$} gibt, so dass }
%   \[ |f(x) - f(x^*)| < \epsilon \quad \text{f"ur alle }
%   \nowrap{x\in D,\text{ mit } |x-x^*|<\delta(\epsilon).}
%   \]

%   \lang{de}{$f$ ist \notion{stetig}, falls
%     $f$ f\"ur alle \nowrap{$x^*\in D$} stetig ist.}
    
%  Allgemeiner sagt man für eine Teilmenge $E$ des Definitionsbereichs $D$:
   
%   \lang{de}{$f$ ist \notion{stetig auf $E$}, falls
%     $f$ f\"ur alle \nowrap{$x^*\in E$} stetig ist.}
    
% \end{definition}

% \begin{remark}
% Da die Menge aller $x$ mit $|x-x^*|<\delta$ genau die $\delta$-Umgebung $U_\delta(x^*)$ von $x^*$ ist, lässt sich diese Definition auch mit Umgebungen formulieren:

% $f:D\to \R$ ist stetig an der Stelle $x^*\in D$, wenn es f"ur jedes $\epsilon >0$ ein \nowrap{$\delta = \delta(\epsilon) >0$} gibt, so dass
%   \[ f(x)\in U_{\epsilon}(f(x^*))\quad \text{f"ur alle }
% x\in D\cap U_\delta(x^*).
%   \]
% \end{remark}

% Das folgende Video erläutert nochmal das $\epsilon$-$\delta$-Kriterium und beweist die Unstetigkeit
% der Dirichlet'schen Sprungfunktion.
% \floatright{\href{https://api.stream24.net/vod/getVideo.php?id=10962-2-10915&mode=iframe&speed=true}{\image[75]{00_video_button_schwarz-blau}}}\\



% \begin{example}
% Visualisierungen zur Stetigkeit einer Funktion an einer Stelle.
% \begin{tabs*}
% \tab{1. Visualisierung}
% \begin{genericGWTVisualization}[550][800]{mathlet1}
		
% 		\begin{variables}
% 			\function{f}{rational}{x^2/2-x+1}  
% %			\pointOnCurve{a1}{rational}{var(f)}{1}
% %			\number{a}{rational}{2}
% 			\parametricFunction{ax}{real}{0, 1+t, 0.02, 2, 100}
% 			\pointOnParametricCurve[editable]{P}{real}{var(ax)}{0.8}		
% 			\number{ups}{real}{var(P)[y]-1}
% %-------------------variante mit editierbarem epsilon
% %			\number[editable]{ups}{real}{1}
% %			\point{P}{real}{0,var(ups)}
% %-----------------------------------------------------
% 			\function{gr}{real}{1}
% 			\function{l1}{real}{1+var(ups)}
% 			\function{l2}{real}{1-var(ups)}
% %			\number{b}{real}{1+2*var(ups)}
% 			\number{d}{real}{2*var(ups)/(2+var(ups))}
% 			\point{c00}{real}{2-var(d),1-var(ups)}
% 			\point{c01}{real}{2-var(d),1+var(ups)}
% 			\point{c10}{real}{2+var(d),1-var(ups)}
% 			\point{c11}{real}{2+var(d),1+var(ups)}
% 			\line{v0}{real}{var(c00), var(c01)}
% 			\line{v1}{real}{var(c10), var(c11)}
% %			\set{s}{real}{|y-1|<var(ups) AND |x-2|<var(d)}
% 			\set{s}{real}{|x-2|<var(d)}
% 			\point{m}{real}{2,0}
% 			\point{pc}{real}{2,1}
			
% 		\end{variables}
% 		\color{P}{#CC6600}
% 		%\label{P}{$\textcolor{BLUE}{P}$}
% 		\color{s}{#00CC00}
% 		\color{l1}{#CC6600}
% 		\color{l2}{#CC6600}
% 		\color{gr}{#0066CC}
% 		\color{m}{#0066CC}
% 		\color{pc}{#0066CC}
% 		\color{v0}{#00CC00}
% 		\color{v1}{#00CC00}
% 		\color{f}{BLACK}
        
% 		\begin{canvas}
% 			\plotLeft{-0.3}
% 			\plotRight{4.3}
% 			\plotSize{400,300}
% 			\plot[coordinateSystem]{m, pc, s, gr,l1,l2, v0, v1, P, f}
% 		\end{canvas}
% 		\text{Die schwarze Kurve ist der Graph der Funktion $f:\R\to \R, x\mapsto \frac{x^2}{2}-x+1$, welche an der Stelle $x^*=2$ den
% 		Funktionswert $f(x^*)=f(2)=1$ hat. Zur Untersuchung der Stetigkeit von $f$ an der Stelle $x^*=2$ werden hier für jedes $\epsilon>0$ ein
% 		passendes $\delta$ berechnet. Graphisch ist ein passendes $\delta$ daran zu erkennen, dass innerhalb des $x$-Bereiches, der durch
% 		$2-\delta$ und $2+\delta$ beschränkt wird (schraffierter Bereich), der Funktionsgraph zwischen den horizontalen grünen Geraden liegt.\\
% 		Für $\epsilon= \var{ups}$ ist $\delta(\epsilon)=\var{d}$ eine Zahl, so dass
% 		$|f(x)-f(x^*)|<\var{ups}$ für alle $x$ mit $|x-x^*|<\var{d}$.\\ \\
% 		\textcolor{#0066CC}{Sie können den Wert von $\epsilon$ ändern, indem Sie den orangenen Punkt auf der $y$-Achse verschieben.}
% 		}
%    	\end{genericGWTVisualization}

% \tab{2. Visualisierung}   	
%    	\begin{genericGWTVisualization}[550][800]{mathlet1}
		
% 		\begin{variables}
% 			\function{f}{rational}{x^2/2-x+1}  
% %			\pointOnCurve{a1}{rational}{var(f)}{1}
% %			\number{a}{rational}{2}
% 			\parametricFunction{ax}{real}{2+t, 0, 0.02, 2, 100}
% 			\pointOnParametricCurve[editable]{Q}{real}{var(ax)}{1.5}		
% 			\number{d}{real}{var(Q)[x]-2}
% %-------------------variante mit editierbarem epsilon
% 			\number[editable]{ups}{real}{0.8}
% 			\number{ups1}{real}{var(ups)}
% 			\point{P}{real}{0,var(ups)}
% %-----------------------------------------------------
% 			\function{gr}{real}{1}
% 			\function{l1}{real}{1+var(ups)}
% 			\function{l2}{real}{1-var(ups)}
% %			\number{b}{real}{1+2*var(ups)}
% %			\number{d}{real}{2*var(ups)/(2+var(ups))}
% 			\point{c00}{real}{2-var(d),1-var(ups)}
% 			\point{c01}{real}{2-var(d),1+var(ups)}
% 			\point{c10}{real}{2+var(d),1-var(ups)}
% 			\point{c11}{real}{2+var(d),1+var(ups)}
% 			\line{v0}{real}{var(c00), var(c01)}
% 			\line{v1}{real}{var(c10), var(c11)}
% %			\set{s}{real}{|y-1|<var(ups) AND |x-2|<var(d)}
% 			\set{s}{real}{|x-2|<var(d)}
% 			\point{m}{real}{2,0}
% 			\point{pc}{real}{2,1}

% 			\number{x1}{real}{2+var(d)-0.01}  % -0.01 als Toleranz. 
% 			\number{fx1}{real}{(var(x1))^2/2-var(x1)+1}
% 			\number{t}{real}{var(fx1)-1-var(ups)-0.01} % -0.01 als Toleranz. 
% 			\number{lb}{real}{2-var(d)}
% 			\number{ub}{real}{2+var(d)}

			
% 		\end{variables}
% 		\color{P}{#CC6600}
% 		\color{Q}{#00CC00}
% 		%\label{P}{$\textcolor{BLUE}{P}$}
% 		\color{s}{#00CC00}
% 		\color{l1}{#CC6600}
% 		\color{l2}{#CC6600}
% 		\color{gr}{#0066CC}
% 		\color{m}{#0066CC}
% 		\color{pc}{#0066CC}
% 		\color{v0}{#00CC00}
% 		\color{v1}{#00CC00}
%         \color{f}{BLACK}

% 		\begin{canvas}
% 			\updateOnDrag[false]
% 			\plotLeft{-0.3}
% 			\plotRight{4.3}
% 			\plotSize{400,300}
% 			\plot[coordinateSystem]{m, pc, s, gr,l1,l2, v0, v1, Q, f}
% 		\end{canvas}
% 		\text{Die schwarze Kurve ist wieder der Graph der Funktion $f:\R\to \R, x\mapsto \frac{x^2}{2}-x+1$, welche an der Stelle $x^*=2$ den
% 		Funktionswert $f(x^*)=f(2)=1$ hat.\\
% 		In dieser Visualisierung sollen Sie selbst zu gegebenem $\epsilon>0$ ein passendes $\delta>0$ finden, so dass
% 		$|f(x)-f(x^*)|<\epsilon$ für alle $x$ mit $|x-x^*|<\delta$.\\ 
%  		Wählen Sie dazu zunächst den Wert: $\epsilon= \var{ups}$.\\ 
%  		Passen Sie anschließend $\delta$ durch Verschieben des grünen Punktes in der Graphik an.\\ \\		 
% 		\IFELSE{var(t) < 0}{\textcolor{#0066CC}{Richtig!} Für $\delta=\var{d}$ gilt: Für alle $x$ zwischen $x^*-\delta=\var{lb}$ und $x^*+\delta=\var{ub}$
% 		gilt $|f(x)-f(x^*)|<\epsilon=\var{ups1}$}{\textcolor{#CC6600}{Für }$\delta=\var{d}$ \textcolor{#CC6600}{ ist die Bedingung }\\
% 		$|f(x)-f(x^*)|<\epsilon=\var{ups1}$ für alle $x$ mit $|x-x^*|<\delta$ \\
% 		\textcolor{#CC6600}{nicht erfüllt. Zum Beispiel ist für }$x=\var{x1}<2+\delta=\var{ub}$:\\  $f(x)\approx\var{fx1}\geq 1+\var{ups1}= f(x^*)+\epsilon$.}
% 		}
%    	\end{genericGWTVisualization}
%    	\end{tabs*}
% \end{example}
   	
% \begin{example}   	
% Visualisierung zur Unstetigkeit einer Funktion an einer Stelle.
% \begin{genericGWTVisualization}[550][800]{mathlet1}
		
% 		\begin{variables}
% 			\function{f}{real}{|x-3|/(x^2-3*x)+sqrt(x)-x/sqrt(x)}   % +-sqrt(x) um auf R_+ einzuschränken.
% %			\pointOnCurve{a1}{rational}{var(f)}{1}
% %			\number{a}{rational}{2}
% 			\parametricFunction{ax}{real}{3+t, 0, 0.02, 2, 100}
% 			\pointOnParametricCurve[editable]{Q}{real}{var(ax)}{2}		
% 			\number{d}{real}{var(Q)[x]-3}
% %-------------------variante mit editierbarem epsilon
% 			\number[editable]{ups}{real}{1}
% 			\number{ups1}{real}{var(ups)}
% 			\point{P}{real}{0,var(ups)}
% %-----------------------------------------------------
% 			\function{gr}{real}{1/3}
% 			\function{l1}{real}{1/3+var(ups)}
% 			\function{l2}{real}{1/3-var(ups)}
% %			\number{b}{real}{1+2*var(ups)}
% %			\number{d}{real}{2*var(ups)/(2+var(ups))}
% 			\point{c00}{real}{3-var(d),1/3-var(ups)}
% 			\point{c01}{real}{3-var(d),1/3+var(ups)}
% 			\point{c10}{real}{3+var(d),1/3-var(ups)}
% 			\point{c11}{real}{3+var(d),1/3+var(ups)}
% 			\line{v0}{real}{var(c00), var(c01)}
% 			\line{v1}{real}{var(c10), var(c11)}
% %			\set{s}{real}{|y-1|<var(ups) AND |x-2|<var(d)}
% 			\set{s}{real}{|x-3|<var(d)}
% 			\point{m}{real}{3, 0}
			
% 			\number{ed}{real}{1/3}
% 			\point{pc}{real}{3, var(ed)}

% 			\number{x1}{real}{3-var(d)+0.01}  % -0.01 als Toleranz. 
% 			\number{fx1}{real}{(3-var(x1))/(var(x1)^2-3*var(x1))}
% 			\number{t}{real}{1/3-var(fx1)-var(ups)-0.01} % -0.01 als Toleranz. 
% 			\number{lb}{real}{3-var(d)}
% 			\number{ub}{real}{3+var(d)}
% 			\number{x2}{real}{3-var(d)/2}
% 			\number{fx2}{real}{(3-var(x2))/(var(x2)^2-3*var(x2))}

			
% 		\end{variables}
% 		\color{P}{#CC6600}
% 		\color{Q}{#00CC00}
% 		%\label{P}{$\textcolor{BLUE}{P}$}
% 		\color{s}{#00CC00}
% 		\color{l1}{#CC6600}
% 		\color{l2}{#CC6600}
% 		\color{gr}{#0066CC}
% 		\color{m}{#0066CC}
% 		\color{pc}{BLACK}
% 		\color{v0}{#00CC00}
% 		\color{v1}{#00CC00}
%         \color{f}{BLACK}
        

% 		\begin{canvas}
% 			\updateOnDrag[false]
% 			\plotLeft{0.7}
% 			\plotRight{5.3}
% 			\plotSize{400,300}
% 			\plot[coordinateSystem]{m, pc, s, gr,l1,l2, v0, v1, Q, f}
% 		\end{canvas}
% 		\text{Die schwarze Kurve ist der Graph der Funktion $f:\R_+\to \R$ mit $f(x)=\frac{|x-3|}{x^2-3x}$ für $x\neq 3$ und $f(3)=\frac{1}{3}$. 
% 		Wir betrachten die Stelle $x^*=3$ mit dem Funktionswert $f(x^*)=f(3)=\frac{1}{3}$, an der die Funktion nicht stetig ist.\\
% 		In dieser Visualisierung sollen Sie selbst zu gegebenem $\epsilon>0$ ein passendes $\delta>0$ suchen, so dass
% 		$|f(x)-f(x^*)|<\epsilon$ für alle $x$ mit $|x-x^*|<\delta$.\\ 
%  		Wählen Sie dazu zunächst den Wert: $\epsilon= \var{ups}$. (Sie sollten insbesondere auch kleine Werte wählen.) \\ 
%  		Passen Sie anschließend $\delta$ durch Verschieben des grünen Punktes in der Graphik an.\\ \\	
%  		\IF{var(ups)<= 2/3}{\textcolor{red}{Für dieses $\epsilon$ können Sie ausprobieren, so lange Sie wollen.}\\ 
%  		Links von $x^*=3$ sind die Funktionswerte zu weit von $f(3)= \frac{1}{3}$ entfernt.\\
%  		Zum Beispiel für $x=\var{x2}>3-\delta=\var{lb}$ ist $f(x)\approx \var{fx2}\leq \frac{1}{3}-\var{ups1}=f(x^*)-\epsilon$.\\
%  		Da es hier kein passendes $\delta$ gibt, ist diese Funktion $f$ an der Stelle $x^*=3$ nicht stetig.
%  		}
%  		\IF{var(ups)> 2/3}{
% 		\IFELSE{var(t) < 0}{\textcolor{#0066CC}{Richtig!} Für $\delta=\var{d}$ gilt: Für alle $x$ zwischen $x^*-\delta=\var{lb}$ und $x^*+\delta=\var{ub}$
% 		ist $|f(x)-f(x^*)|<\epsilon=\var{ups1}$}{\textcolor{#CC6600}{Für }$\delta=\var{d}$ \textcolor{#CC6600}{ ist die Bedingung }\\
% 		$|f(x)-f(x^*)|<\epsilon=\var{ups1}$ für alle $x$ mit $|x-x^*|<\delta$ \\
% 		\textcolor{#CC6600}{nicht erfüllt. Zum Beispiel ist für }$x=\var{x1}>3-\delta=\var{lb}$:\\  $f(x)\approx\var{fx1}\leq \frac{1}{3}-\var{ups1}
% 		= f(x^*)-\epsilon$.}
% 		}
% 		}
%    	\end{genericGWTVisualization}
% \end{example}

% Das folgende Video erläutert den Begriff der Stetigkeit sehr anschaulich und verdeutlicht ihn an Beispielen:
% \floatright{\href{https://www.hm-kompakt.de/video?watch=407}{\image[75]{00_Videobutton_schwarz}}}\\\\
 
% \begin{example}
% Die meisten elementaren Funktionen sind auf ihrer gesamten Definitionsmenge stetig (vgl. auch den \link{elem-funk-stetig}{nächsten Abschnitt}).
% \begin{tabs*}
% \tab{konstante Funktionen} Für festes $c\in \R$ ist die konstante Funktion $f:\R\to \R,x\mapsto c$ stetig auf ganz $D=\R$.\\
% Ist $x^*\in \R$ eine beliebige Stelle, so gilt für jedes $\epsilon>0$:
% \[   |f(x)-f(x^*)|=|c-c|=0<\epsilon \]
% für alle $x\in \R$. Insbesondere kann man stets $\delta=1$ wählen (oder irgendeinen anderen Wert $>0$) und erhält:
% \[   |f(x)-f(x^*)|=|c-c|=0<\epsilon \quad \text{f"ur alle }
%   \nowrap{x\in D,\text{ mit } |x-x^*|<\delta}. \]
% \tab{lineare Funktionen} Für feste reelle Zahlen $a,b\in \R$ mit
% $a \neq 0$ sei $f: \R \to \R, x \mapsto ax + b$. Sei $x^* \in \R$ und $\epsilon > 0$. Wir wählen
% $\delta  = \frac{\epsilon}{| a |}$ . Für alle $x \in \R$ mit $| x - x^* | < \delta$ gilt dann
% \[ |f ( x ) - f ( x^* )| = |ax + b - ax^* - b| = | a | \cdot | x - x^* | < | a | \cdot \delta = \epsilon. \]
% Also ist $f$ stetig auf $\R$.
% \tab{$f(x)=x^2$} Sei $f : [-1; 1 ] \to \R, x \mapsto x^2$ .\\
% Für $x^* \in [-1; 1 ]$ und $\epsilon > 0$ sei $\delta =\frac{\epsilon}{2}> 0$.\\
% Dann gilt für alle
% $x \in [-1; 1 ]$ mit $| x - x^* | < \delta$:
% \[ |f ( x ) - f ( x^* )| = |x^2 - (x^*)^2| = | x+x^* |\cdot | x - x^* | 
%  \leq 2 | x - x^* | < 2\delta = \epsilon.\]
% Also ist $f$ stetig auf $[-1; 1 ]$.
% \end{tabs*}
% \end{example}   	

% \begin{example}
% Wir haben auch schon Funktionen kennengelernt, die nicht an allen Stellen stetig sind.
% \begin{tabs*}
% \tab{Gauß-Klammer-Funktion} 
% Die durch die Gauß-Klammer gegebene Funktion
% \[   f:\R\to \Z, x\mapsto [x]:=\max\{ n\in \Z | n\leq x\} \]
% hat einen Graph, der viele \emph{Sprünge} aufweist. 

% \begin{center}
% \image{T204_Example_A}
% \end{center}

% An den \emph{Sprungstellen} ist diese Funktion nicht stetig:\\
% Ist nämlich $x^*=n\in \Z$, so ist $[x^*]=n$ und für jedes $x\in \R$ mit $x<x^*$ gilt $[x]\leq n-1$.\\
% Insbesondere ist $|f(x)-f(x^*)|=|[x]-n|\geq 1$ für alle $x<x^*$.

% Wählt man also ein $\epsilon\in (0;1]$ (z.B. $\epsilon=\frac{1}{2}$), so gilt für alle $x<x^*$
% \[  |f(x)-f(x^*)|\geq 1\geq \epsilon. \]
% Da es aber für jedes $\delta>0$ auch ein $x<x^*$ mit $|x-x^*|<\delta$ gibt (z.B. $x=x^*-\delta/2$), gibt es also zu keinem $\epsilon\in (0;1]$ ein gewünschtes $\delta$. Die Gauß-Klammer-Funktion ist also an der Stelle $x^*\in \Z$ nicht stetig.

% Für $x^*\in \R\setminus \Z$ ist die Funktion an der Stelle $x^*$ jedoch stetig. In einer kleinen Umgebung von $x^*$ ist die Funktion nämlich konstant.

% \begin{incremental}[\initialsteps{0}]
% \step Genauer: Wählt man $\delta=\min\{x^*-[x^*]; [x^*]+1-x^*\}$, so gilt
% \[ f(x)=[x]=[x^*]=f(x^*)\quad \text{für alle }x\text{ mit }|x-x^*|<\delta \]
% (denn $x< x^*+\delta\leq x^*+[x^*]+1-x^*=[x^*]+1$ und
% $x>x^*-\delta\geq x^*-(x^*-[x^*])=[x^*]$ nach Definition von $\delta$.)
% Insbesondere gilt für beliebiges $\epsilon>0$
% \[ |f(x)-f(x^*)|=0<\epsilon\quad \text{für alle }x\text{ mit }|x-x^*|<\delta. \]
% \end{incremental}
% \tab{Dirichlet'sche Sprungfunktion} 
% Sei 
% \[  D:\R\to {\{0;1\}}, x \mapsto \begin{cases} 1 & \text{falls }x\in \Q \\
% 0 & \text{falls }x\notin \Q \end{cases}.\]
% die \ref[reelle-funk][Dirichlet'sche Sprungfunktion]{ex:unstetige-funktionen}.

% Ist $x^*$ eine rationale Zahl, so gibt es in jeder $\delta$-Umgebung $U_{\delta}(x^*)$ von $x^*$ eine irrationale Zahl $x_1$. Das bedeutet, dass es zu jedem $\delta>0$ ein $x_1$ mit $|x_1-x^*|<\delta$ gibt, so dass
% \[ |D(x_1)-D(x^*)| =|0 -1|=1. \]
% Wählt man also zum Beispiel $\epsilon = \frac{1}{2}$, dann gibt es
% kein $\delta>0$ mit
% \[ |D(x)-D(x^*)|<\epsilon\quad\text{für alle }x\text{ mit }|x-x^*|<\delta, \]
% da es für die irrationalen $x$ nicht erfüllt ist.

% Also ist die Dirichlet'sche Sprungfunktion an jeder rationalen Stelle nicht stetig.

% Ganz entsprechend sieht man, dass diese Funktion auch an den irrationalen Stellen $x^* \in \R \setminus \Q$ nicht stetig ist, da es in jeder $\delta$-Umgebung einer irrationalen Zahl auch immer eine rationale Zahl gibt.
% \end{tabs*}
% \end{example}

% \begin{remark}
% Sei $x^*$ ein \emph{isolierter Punkt} von $D$, d. h. ein Punkt, für den es ein $\delta > 0$ gibt mit $U_{\delta}( x^*)\cap D =
% \{ x^* \}$. Dann ist $f$ stetig in $x^*$ (weil man für jedes $\epsilon > 0$ dieses $\delta$ wählen kann). Dies wird in der Literatur zum Teil anders definiert.
% \end{remark}

% \section{Folgen-Kriterium}\label{sec:folgenkriterium}

% Anstelle des $\epsilon$-$\delta$-Kriteriums von oben, wird die Stetigkeit von Funktionen auch oft über Folgen definiert.
% Wir formulieren dies als Satz.

% \begin{theorem}\label{thm:stetigkeit_folgenkrit_aequiv_epsilon-delta}
% Sei $f : D \to \R$ eine Funktion und $x^* \in D$. Die Funktion
% $f$ ist genau dann an der Stelle $x^*$ stetig, wenn die folgende Bedingung erfüllt ist:

% Für jede Folge $( x_n)_{n \geq 1}$ in $D$ mit 
% $\lim_{n \to \infty} x_n = x^*$ gilt
% \[ \lim_{n \to \infty} f ( x_n ) = f ( \lim_{n \to \infty} x_n) 
% = f ( x^* ) .\]
% \end{theorem}


% \begin{proof*}[Beweis (Stetigkeit)]

% Beweisskizze: Die Äquivalenz des Satzes wird in zwei Schritten gezeigt: Die Hin-Richtung
% benutzt nur die $\epsilon-\delta-$Definition der Stetigkeit und es folgt daraus die Konvergenz
% der Folge der Funktionswerte. Die Rückrichtung wird mittels indirektem Beweis gezeigt: unter der
% Annahme, dass $f$ an der Stelle $x^*$ nicht stetig ist, finden wir, dass dann die Folge der Funktionswerte
% nicht gegen $f(x^*)$ konvergiert. Dies ist im Widerspruch zur Voraussetzung der konvergenten Folge
% der Funktionswerte und somit folgt eben daraus die Stetigkeit an $x^*$.\\
% \begin{incremental}{0}
% \step
% Wir nehmen zunächst an, dass $f$ in $x^*\in D$ stetig ist, und betrachten eine beliebige Folge $( x_n)_{n \geq 1}$ in $D$ mit 
% $\lim_{n \to \infty} x_n = x^*$.
% Dann ist zu zeigen, dass  $\lim_{n \to \infty} f ( x_n ) =f(x^*)$,
% d.h. dass es zu jedem $\epsilon>0$ ein $N\in \N$ gibt mit
% \[ |f(x_n)-f(x^*)|<\epsilon\quad \text{für alle }n\geq N.\]

% Da $f$ stetig ist, gibt es zu $\epsilon>0$ ein $\delta>0$ mit
% \[ |f(x)-f(x^*)|<\epsilon\quad\text{für alle }x\in D\text{ mit }|x-x^*|<\delta. \]
% Wegen $\lim_{n \to \infty} x_n = x^*$ gibt es zu diesem $\delta>0$ ein $N\in \N$ mit
% \[ |x_n-x^*|<\delta\quad\text{für alle }n\geq N,\]
% und daher auch
% \[ |f(x_n)-f(x^*)|<\epsilon\quad\text{für alle }n\geq N.\]

% \step

% Nimmt man umgekehrt an, dass $f$ nicht stetig an der Stelle $x^*$ ist, so ist zu zeigen, dass es eine Folge $( x_n)_{n \geq 1}$ in $D$ mit 
% $\lim_{n \to \infty} x_n = x^*$ gibt, so dass die Folge $(f(x_n))_{n \geq 1}$ nicht gegen $f(x^*)$ konvergiert.

% Da $f$ nicht stetig in $x^*$ ist, gibt es ein $\epsilon>0$ so, dass kein gewünschtes $\delta>0$ existiert. Für dieses $\epsilon$ gilt also insbesondere, dass für $\delta=\frac{1}{n}$ ein $x_n\in D\cap U_\delta(x^*)$ existiert mit $|f(x_n)-f(x^*)|\geq \epsilon$.
% Diese Wahl liefert also eine Folge $( x_n)_{n \geq 1}$ mit
% \[ \lim_{n\to \infty} |x_n-x^*| \leq \lim_{n\to \infty} \frac{1}{n}
% =0,\]
% d.h. $\lim_{n\to \infty} x_n=x^*$.
% Andererseits ist $|f(x_n)-f(x^*)|\geq \epsilon$ für alle $n\in \N$, weshalb die Folge $(f(x_n))_{n \geq 1}$ nicht gegen $f(x^*)$ konvergiert.
% \end{incremental}
% \end{proof*}

% \begin{remark}
% Das Folgenkriterium beinhaltet zwei Bedingungen:
% \begin{enumerate}
% \item Für jede Folge $( x_n)_{n \geq 1}$, die gegen $x^*$ konvergiert, ist die Folge $(f(x_n))_{n \geq 1}$ konvergent.
% \item Die nach 1. konvergente Folge $(f(x_n))_{n \geq 1}$ hat tatsächlich $f(x^*)$ als Grenzwert.
% \end{enumerate}
% In den nachfolgenden Beispielen, die nicht stetig sind, werden wir auch sehen, dass schon 1. verletzt sein kann.
% \end{remark}


% \begin{example}
% Die obigen Beispiele lassen sich über dieses Kriterium mit Hilfe der \ref[konv-krit][Grenzwertregeln]{sec:grenzwertregeln} leichter erfassen.
% \begin{tabs*}[\initialtab{0}]
% \tab{Lineare und konstante Funktionen} Für feste reelle Zahlen $a,b\in \R$ sei $f: \R \to \R, x \mapsto ax + b$ eine lineare Funktion (falls $a\neq 0$) oder konstante Funktion (falls $a=0$).
% Für eine beliebige Stelle $x^* \in \R$ und jede Folge $( x_n)_{n \geq 1}$ mit $\lim_{n\to \infty} x_n=x^*$ gilt dann:
% \[  \lim_{n\to \infty} f(x_n)= \lim_{n\to \infty} (ax_n+b)=a \lim_{n\to \infty} x_n + b =ax^*+b=f(x^*). \]
% Also sind lineare und konstante Funktionen stetig.
% \tab{$f(x)=x^2$} Die Funktion $f : \R \to \R, x \mapsto x^2$ ist auf ganz $\R$ stetig. Denn für beliebiges $x^*\in \R$ und
%  jede Folge $( x_n)_{n \geq 1}$ mit $\lim_{n\to \infty} x_n=x^*$ gilt:
%  \[  \lim_{n\to \infty} f(x_n)= \lim_{n\to \infty} x_n^2 =\left( \lim_{n\to \infty} x_n\right)^2 =(x^*)^2 =f(x^*). \]
% \tab{Gauß-Klammer-Funktion} Die Gauß-Klammerfunktion ist an ganzzahligen Stellen $k=x^*\in \Z$ nicht stetig, denn für jede Folge $( x_n)_{n \geq 1}$ mit $x_n<x^*$, 
% die gegen $x^*$ konvergiert, ist
% \[ \lim_{n\to \infty} [x_n] =k-1\neq k=[x^*].\]
% \begin{incremental}[\initialsteps{0}]
% \step Es gibt hier auch Folgen $( x_n)_{n \geq 1}$, die gegen $k=x^*\in \Z$ konvergieren, für welche jedoch der Grenzwert $\lim_{n\to \infty} [x_n]$ nicht 
% existiert. Dies ist zum Beispiel bei der durch $x_n=k+\frac{(-1)^n}{n}$ definierten Folge der Fall. Hier ist
% \[  [x_n]= \begin{cases} k-1 & \text{falls }n\text{ ungerade} \\
% k & \text{falls }n\text{ gerade}. \end{cases}\]
% \end{incremental}
% \tab{Dirichletsche Sprungfunktion} Für die Dirichletsche Sprungfunktion $D$ sieht man es auch direkt:
% Für jede reelle Stelle $x^*$ kann man eine Folge $( x_n)_{n \geq 1}$ konstruieren, die gegen $x^*$ konvergiert und deren Glieder abwechselnd
% rational und irrational sind. Die Folge der Funktionswerte $( D(x_n))_{n \geq 1}$ ist dann aber die Folge, deren Glieder abwechselnd $0$ und $1$ sind.
% Insbesondere konvergiert die Folge $( D(x_n))_{n \geq 1}$ nicht.\\
% Also ist die Dirichletsche Sprungfunktion an keiner Stelle stetig.
% \begin{incremental}[\initialsteps{0}]
% \step Alternativ kann man auch folgendermaßen vorgehen:\\
% Für jede rationale Stelle $x^*$ gibt es eine Folge irrationaler Zahlen $( x_n)_{n \geq 1}$, die gegen $x^*$ konvergiert. Für diese gilt dann
% \[ \lim_{n\to \infty} D(x_n)= \lim_{n\to \infty} 0\neq 1=D(x^*),\]
% weshalb $D$ in $x^*$ nicht stetig ist.

% Ebenso gibt es für eine irrationale Stelle $x^*$ eine Folge rationaler Zahlen $( x_n)_{n \geq 1}$, die gegen $x^*$ konvergiert. Für diese gilt dann
% \[ \lim_{n\to \infty} D(x_n)= \lim_{n\to \infty} 1\neq 0=D(x^*),\]
% weshalb $D$ in $x^*$ nicht stetig ist.
% \end{incremental}
% \end{tabs*}
% \end{example}

% % \begin{remark}
% % In den Beispielen hatten wir zum Beweis der Unstetigkeit Folgen verwendet, bei denen die Folge der 
% % Funktionswerte gegen einen Wert konvergieren, der vom Funktionswert der Stelle verschieden ist. 
% % Im Allgemeinen muss aber die Folge der Funktionswerte nicht einmal konvergieren.
% % \end{remark}

% Wir hatten die \ref[konv-krit][Grenzwertregeln]{sec:grenzwertregeln} für Folgen schon verwendet, um die Stetigkeit obiger Beispiele zu zeigen. 
% Allgemeiner folgert man daraus direkt die Stetigkeit zusammengesetzter Funktionen:

% \begin{rule}\label{rule:zusammengesetzte-funktionen}
% Seien $f:D\to \R$ und $g:D\to \R$ reelle Funktionen, die an einer Stelle $x^*\in D$ stetig sind, sowie $r\in \R$.

% Dann sind auch die Funktionen $f+g$, $f-g$, $f\cdot g$ und $rf$ in $x^*$ stetig.

% Ist zudem $g(x^*)\neq 0$, dann ist auch die Funktion $\frac{f}{g}$ an der Stelle $x^*$ stetig. \\

% \end{rule}
  
% \begin{quickcheck}
% %\type{mc.multiple}%
% \lang{de}{\text{Wählen Sie alle richtigen Aussagen aus:}
% }

% \begin{choices}{multiple}
%     \begin{choice}
%         \text{Eine Funktion, die an der Stelle $x_0$ nicht definiert ist, 
%             kann dort stetig sein.}
%         \solution{false}
%     \end{choice}
%     \begin{choice}
%         \text{Am ersten Tag geht ein Bergwanderer morgens um 8 Uhr los und erreicht den Berggipfel um 17 Uhr.
%         Am zweiten Tag läuft er von 8 bis 17 Uhr den Berg über dieselbe Strecke wieder hinunter. 
%         Es gibt auf dem Weg eine Stelle, an der er 
%         zur selben Tageszeit wie am Vortag wieder vorbeikommt.}
%         \solution{true}
%     \end{choice}
% \end{choices}

% \explanation{zu 1) Damit $f(x)$ an einer Stelle $x_0$ stetig ist, muss $x_0 \in D$ sein. \\Aussage 1) ist also falsch.\\
%   zu 2) Trägt man die Höhenmeter über der Zeit auf, so ist der Aufstieg eine stetige, monoton steigende Funktion und der Abstieg
%   eine stetige, monoton fallende Funktion, d.h. es gibt genau einen Kreuzungspunkt. \\Aussage 2) ist damit richtig.}

% \end{quickcheck}  


% Wie Stetigkeit mit dem Folgenkriterium zu beweisen ist, erläutert das folgende Video:
% \floatright{\href{https://api.stream24.net/vod/getVideo.php?id=10962-2-10900&mode=iframe&speed=true}{\image[75]{00_video_button_schwarz-blau}}}\\


% \section{Links- und Rechtsseitige Stetigkeit}\label{sec:einseitige-stetigkeit}

% Als Verallgemeinerung zur Stetigkeit führen wir noch die links- bzw. rechtsseitige Stetigkeit ein. Hierbei werden nur Folgen betrachtet, deren Folgeglieder auf dem Zahlenstrahl alle links bzw. alle rechts der betrachteten Stelle liegen.

% \begin{definition} 
% Sei $f : D \to \R$ eine Funktion und $x^* \in D$.
% \begin{enumerate}
% \item $f$ ist \notion{linksseitig stetig} in $x^*$, wenn für jede Folge $( x_n )_{n \geq 1}$ in $D$ mit
% $\lim_{n \to \infty} x_n = x^*$ und $x_n \leq x^*$  für alle $n \in \N$ gilt 
% \[ \lim_{n \to \infty} f ( x_n ) = f ( x^* ) .\]
% \item $f$ ist \notion{rechtsseitig stetig} in $x^*$, wenn für jede Folge $( x_n )_{n \geq 1}$ in $D$ mit
% $\lim_{n \to \infty} x_n = x^*$ und $x_n \geq x^*$  für alle $n \in \N$ gilt 
% \[ \lim_{n \to \infty} f ( x_n ) = f ( x^* ) .\]
% \end{enumerate}
% \end{definition}

% \begin{remark}
% Wie auch schon bei Stetigkeit, lässt sich die links- und rechtsseitige Stetigkeit auch über ein $\epsilon$-$\delta$-Kriterium definieren:

% $f$ ist linksseitig stetig in $x^*$, wenn es f"ur jedes $\epsilon >0$ ein \nowrap{$\delta = \delta(\epsilon) >0$} gibt, so dass 
%   \[ |f(x) - f(x^*)| < \epsilon \quad \text{f"ur alle }
%   \nowrap{x\in D,\text{ mit } x^*-\delta(\epsilon)< x\leq x^*.}
%   \]
% $f$ ist rechtsseitig stetig in $x^*$, wenn es f"ur jedes $\epsilon >0$ ein \nowrap{$\delta = \delta(\epsilon) >0$} gibt, so dass 
%   \[ |f(x) - f(x^*)| < \epsilon \quad \text{f"ur alle }
%   \nowrap{x\in D,\text{ mit } x^*\leq x< x^*+\delta(\epsilon).}
%   \]
% \end{remark}

% \begin{example}
% \begin{enumerate}
% \item
% Ist eine Funktion $f$ stetig in $x^*$ ist sie natürlich auch linksseitig und rechtsseitig stetig, da die Bedingung für die Folgen der Funktionswerte lediglich für weniger Folgen $( x_n )_{n \geq 1}$ erfüllt sein müssen.
% \item Die Gauß-Klammer-Funktion $[\cdot]$ ist an allen Stellen rechtsseitig stetig. An den ganzzahligen Stellen ist sie jedoch nicht linksseitig stetig, wie oben gezeigt wurde.
% \item Die Dirichletsche Sprungfunktion ist an keiner Stelle linksseitig stetig und auch nicht rechtsseitig stetig.
% \end{enumerate}
% \end{example}

% Die Bedeutung der links- und rechtsseitigen Stetigkeit wird ersichtlich durch

% \begin{theorem}
% $f:D\to \R$ ist genau dann stetig in $x^*\in D$ , wenn $f$ linksseitig und rechtsseitig stetig in $x^*$ ist.
% \end{theorem}

% \begin{proof*}
% Dass Stetigkeit die links- und rechtsseitige Stetigkeit impliziert, wurde ja schon erklärt.
% \begin{incremental}{0}
% \step
% Sei nun $f$ in $x^*$ links- und rechtsseitig stetig. Für  
% $\epsilon>0$ muss dann ein passendes $\delta$ gefunden werden.
% Da $f$ in $x^*$ links- und rechtsseitig stetig ist, gibt es $\delta_1>0$ mit
%   \[ |f(x) - f(x^*)| < \epsilon \quad \text{f"ur alle }
%   x\in D,\text{ mit } x^*-\delta_1< x\leq x^*
%   \]
%  und $\delta_2>0$ mit
%   \[ |f(x) - f(x^*)| < \epsilon \quad \text{f"ur alle }
%   x\in D,\text{ mit } x^*\leq x< x^*+\delta_2.
%   \]
% Daher ist
%   \[ |f(x) - f(x^*)| < \epsilon \quad \text{f"ur alle }
%   x\in D,\text{ mit } x^*-\delta_1< x< x^*+\delta_2.
%   \]
% Wählt man nun $\delta=\min\{\delta_1;\delta_2\}$, so ist insbesondere
%   \[ |f(x) - f(x^*)| < \epsilon \quad \text{f"ur alle }
%   x\in D,\text{ mit } x^*-\delta< x< x^*+\delta,
%   \]
%   d.h. für alle $x\in D$ mit $|x-x^*|<\delta$. 
% \end{incremental}
% \end{proof*}

% \begin{example}
% Wir wollen zeigen, dass die Betragsfunktion $f:\R\to \R, x\mapsto |x|$ auf ganz $\R$ stetig ist.

% Betrachten wir zunächst die Stelle $x^*=0$.\\ Für alle gegen $0$ konvergente Folgen $( x_n )_{n \geq 1}$ mit $x_n\leq 0$ gilt hier
% \[ \lim_{n \to \infty} |x_n| =\lim_{n \to \infty} -x_n =- \lim_{n \to \infty} x_n=-0=|0|. \]
% Also ist die Betragsfunktion in $0$ linksseitig stetig.

% Für alle gegen $0$ konvergente Folgen $( x_n )_{n \geq 1}$ mit $x_n\geq 0$ gilt hier
% \[ \lim_{n \to \infty} |x_n| =\lim_{n \to \infty} x_n =0=|0|. \]
% Also ist die Betragsfunktion in $0$ rechtsseitig stetig.

% Damit ist die Betragsfunktion in $0$ auch stetig.\\ \\

% Für alle anderen Stellen verwenden wir eine Mittel, das häufig hilfreich ist: Um Stetigkeit an einer bestimmten Stelle 
% $x^*$ zu untersuchen, reicht es gegen $x^*$ konvergente Folgen $( x_n )_{n \geq 1}$ zu betrachten, deren
% Folgeglieder alle in einer kleinen Umgebung von $x^*$ liegen. Das liegt daran, dass wir nur an dem Grenzwert
% der Folge $( f(x_n) )_{n \geq 1}$ interessiert sind, und daher die "`ersten Folgeglieder"' uninteressant sind.
% Da die Folge $( x_n )_{n \geq 1}$ aber gegen $x^*$ konvergiert, liegen ab einem bestimmten $N\in \N$ alle Folgeglieder
% $x_n$ mit $n\geq N$ in dieser kleinen Umgebung von $x^*$, und wir hätten direkt mit der Folge $( x_n )_{n \geq N}$ starten
% können.

% Betrachten wir nun zunächst eine Stelle $x^*>0$: In einer Umgebung von $x^*$ (genauer sogar auf
% dem Intervall $(0;2x^*)=U_{x^*}(x^*)$ ist die Betragsfunktion identisch mit der linearen Funktion $x\mapsto x$.
% Da letztere bei $x^*$ stetig ist, ist die Betragsfunktion bei $x^*$ stetig.

% Entsprechend ist die Betragfunktion an allen negativen Stellen $x^*$ stetig, weil sie auf der Umgebung $(2x^*;0)=U_{-x^*}(x^*)$
% mit der linearen Funktion $x\mapsto -x$ identisch ist.

% Somit ist die Betragsfunktion auf ganz $\R$ stetig.
% \end{example}

% \begin{quickcheck}
% %\type{mc.multiple}%
%     \lang{de}{\text{Unter Beispiel 1.6 findet sich der Graph der Gauß-Klammer-Funktion. Wir betrachten jetzt:
%     \[   f:\R\to \Z, x\mapsto [-x] \]\\Welche der folgenden Aussagen ist wahr?}}
%     \begin{choices}{multiple}
%         \begin{choice}
%             \text{Die Funktion $f(x)$ ist unstetig an $x_0=1$.}
%             \solution{false}
%         \end{choice}
%         \begin{choice}
%             \text{Die Funktion $f(x)$ ist linksseitig stetig an $x_0=1$.}
%             \solution{true}
%         \end{choice}
%         \begin{choice}
%             \text{Die Funktion $f(x)$ ist rechtsseitig stetig an $x_0=1$.}
%             \solution{false}
%         \end{choice}
%         \begin{choice}
%             \text{Die Funktion $f(x)$ ist stetig an $x_0=1$.}
%             \solution{false}
%         \end{choice}
        
%      \end{choices}
%      \explanation{Die hier betrachtete Funktion ist die an der y-Achse gespiegelte Gauß-Klammer-Funktion.
%      Deshalb ist sie an $x_0=1$ linksseitig stetig.}
% \end{quickcheck}

% Wie sich der Stetigkeitsbegriff für komplexe Funktionen erweitern lässt, zeigt das folgende Video:
% \floatright{\href{https://api.stream24.net/vod/getVideo.php?id=10962-2-10921&mode=iframe&speed=true}{\image[75]{00_video_button_schwarz-blau}}}\\


\end{visualizationwrapper}

\end{content}