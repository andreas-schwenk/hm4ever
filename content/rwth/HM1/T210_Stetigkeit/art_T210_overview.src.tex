%$Id:  $
\documentclass{mumie.article}
%$Id$
\begin{metainfo}
  \name{
    \lang{de}{Überblick: Stetigkeit}
    \lang{en}{overview: }
  }
  \begin{description} 
 This work is licensed under the Creative Commons License Attribution 4.0 International (CC-BY 4.0)   
 https://creativecommons.org/licenses/by/4.0/legalcode 

    \lang{de}{Beschreibung}
    \lang{en}{}
  \end{description}
  \begin{components}
  \end{components}
  \begin{links}
\link{generic_article}{content/rwth/HM1/T210_Stetigkeit/g_art_content_32_grenzwert_gegen_unendlich.meta.xml}{content_32_grenzwert_gegen_unendlich}
\link{generic_article}{content/rwth/HM1/T210_Stetigkeit/g_art_content_31_grenzwerte_von_funktionen.meta.xml}{content_31_grenzwerte_von_funktionen}
\link{generic_article}{content/rwth/HM1/T210_Stetigkeit/g_art_content_30_elem_funktionen.meta.xml}{content_30_elem_funktionen}
\link{generic_article}{content/rwth/HM1/T210_Stetigkeit/g_art_content_29_stetigkeit_definitionen.meta.xml}{content_29_stetigkeit_definitionen}
\end{links}
  \creategeneric
\end{metainfo}
\begin{content}
\begin{block}[annotation]
	Im Ticket-System: \href{https://team.mumie.net/issues/30127}{Ticket 30127}
\end{block}




\begin{block}[annotation]
Im Entstehen: Überblicksseite für Kapitel  Stetigkeit
\end{block}

\usepackage{mumie.ombplus}
\ombchapter{1}
\lang{de}{\title{Überblick: Stetigkeit}}
\lang{en}{\title{}}



\begin{block}[info-box]
\lang{de}{\strong{Inhalt}}
\lang{en}{\strong{Contents}}


\lang{de}{
    \begin{enumerate}%[arabic chapter-overview]
   \item[10.1] \link{content_31_grenzwerte_von_funktionen}{Folgen von Funktionswerten}
   \item[10.2] \link{content_29_stetigkeit_definitionen}{Stetigkeitsbegriff}
   \item[10.3] \link{content_30_elem_funktionen}{Stetigkeit elementarer Funktionen}
   \item[10.4] \link{content_32_grenzwert_gegen_unendlich}{Asymptoten}
   \end{enumerate}
} %lang

\end{block}

\begin{zusammenfassung}

\lang{de}{Stetigkeit ist eine äußerst erstrebenswerte Eigenschaft von Funktionen:
Kleine Änderungen in der Eingabe sollten lediglich kleine Änderungen der Ausgabe zur Folge haben.

Wir entwickeln hier zwei äquivalente Stetigkeitsbegriffe, das Folgenkriterium und das $\epsilon$-$\delta$-Kriterium.
Diese benutzen wir, um die Stetigkeit der meisten bekannten elementaren Funktionen und ihrer Verkettungen zu zeigen. 
Auch Umkehrfunktionen sind auf Intervallen stetig, und Potenzreihen innerhalb ihres Konvergenzradius.


Folgen von Funktionswerten helfen uns auch, das Verhalten von Funktionen an den Rändern ihrer Definitionsbereiche
zu charakterisieren.}


\end{zusammenfassung}

\begin{block}[info]\lang{de}{\strong{Lernziele}}
\lang{en}{\strong{Learning Goals}} 
\begin{itemize}[square]
\item \lang{de}{Sie verfügen über eine visuelle Anschauung des Stetigkeitsbegriffs.}
\item \lang{de}{Sie verfügen über einen großen Fundus stetiger Funktionen.}
\item \lang{de}{Sie identifizieren mögliche Unstetigkeitsstellen von Funktionen und wenden das Folgenkriterium an,
                um über die Stetigkeit dort zu entscheiden.}
\item \lang{de}{Sie kennen das $\epsilon$-$\delta$-Kriterium und setzen es in einfachen Beispielen um.}
\item \lang{de}{Sie kennen die Begriffe links- und rechtsseitiger Grenzwert sowie links- und rechtsseitige Stetigkeit.}
\item \lang{de}{Sie untersuchen das Verhalten von Funktionen an den Rändern ihres Definitionsbereichs und bestimmen ggf. Asymptoten.}
\end{itemize}
\end{block}




\end{content}
