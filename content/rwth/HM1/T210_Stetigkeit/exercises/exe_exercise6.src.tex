\documentclass{mumie.element.exercise}
%$Id$
\begin{metainfo}
  \name{
    \lang{de}{Ü06: Stetigkeit an Stelle }
    \lang{en}{}
  }
  \begin{description} 
 This work is licensed under the Creative Commons License Attribution 4.0 International (CC-BY 4.0)   
 https://creativecommons.org/licenses/by/4.0/legalcode 

    \lang{de}{Hier die Beschreibung}
    \lang{en}{}
  \end{description}
  \begin{components}
  \end{components}
  \begin{links}
  \end{links}
  \creategeneric
\end{metainfo}
\begin{content}
\title{
\lang{de}{Ü06: Stetigkeit an Stelle}
}
\begin{block}[annotation]
  Im Ticket-System: \href{http://team.mumie.net/issues/9990}{Ticket 9990}
\end{block}

\begin{enumerate}
\item
\lang{de}{Untersuchen Sie, für welche Werte von $a,b\in\R$ die Funktion $f\colon\R\rightarrow\R$ mit
\[f(x)=\begin{cases}
        ae^{bx},&\text{falls } x<0\\
        1+x ,&\text{falls }x \geq 0\\
       \end{cases}\]
im Nullpunkt stetig ist.}
\item
Wie verhalten sich die Funktionen $f$ und $g$, wenn man sich mit dem Argument x der Null 
von rechts annähert?\\
Existieren $\displaystyle\lim_{x\to\downarrow 0}f(x)$ 
bzw. $\displaystyle\lim_{x\to\downarrow 0}g(x)$?\\
$f\colon\R^{>0}\rightarrow\R$ mit
\[f(x)=\sin\left(\frac{1}{x}\right)\]
$g\colon\R^{>0}\rightarrow\R$ mit
\[g(x)=x\cdot\sin\left(\frac{1}{x}\right)  \]
\end{enumerate}




\begin{tabs*}[\initialtab{0}\class{exercise}]
 \tab{
  \lang{de}{Antworten }}
  \begin{enumerate}
  \item \[ a= 1, b\in \R.\]
  \item $\displaystyle\lim_{x\to\downarrow 0}f(x)\text{ existiert nicht.}$\\
   \[\displaystyle\lim_{x\to\downarrow 0}g(x)=0\]
  \end{enumerate}
  \tab{
  \lang{de}{Lösung 1) }}
  \begin{incremental}[\initialsteps{1}]
      \step
        \lang{en}{...}
        \lang{de}{\textbf{Antwort:}\[ a= 1, b\in \R.\]}
      \step
        \lang{en}{...}
        \lang{de}{ Da $f_{1}(x)=ae^{bx}$ und $f_{2}(x)=1+x$ auf $\R$ stetig sind, 
muss der rechtsseitige und linksseitige Limes für $x$ gegen $0$ 
übereinstimmen, wir dürfen $x=0$ in beide Komponentenfunktionen einsetzen.
Es gilt nun $f_{1}(0)=a$ und $f(0)=f_{2}(0)=1$. Also ist $f$ genau für $a=1$ und alle $b\in\R$ stetig.}
    \end{incremental}
  
	
   \tab{\lang{de}{Lösungsvideo 2)}}	
    \youtubevideo[500][300]{DBHd8qUiDDg}\\
  
\end{tabs*}

\end{content}