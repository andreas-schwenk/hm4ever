\documentclass{mumie.element.exercise}
%$Id$
\begin{metainfo}
  \name{
    \lang{de}{Ü03: einseitige Grenzwerte}
    \lang{en}{}
  }
  \begin{description} 
 This work is licensed under the Creative Commons License Attribution 4.0 International (CC-BY 4.0)   
 https://creativecommons.org/licenses/by/4.0/legalcode 

    \lang{de}{Hier die Beschreibung}
    \lang{en}{}
  \end{description}
  \begin{components}
  \end{components}
  \begin{links}
  \end{links}
  \creategeneric
\end{metainfo}
\begin{content}
\title{
\lang{de}{Ü3: einseitige Grenzwerte}
}
\begin{block}[annotation]
  Im Ticket-System: \href{http://team.mumie.net/issues/9987}{Ticket 9987}
\end{block}
\begin{enumerate}
\item
\lang{de}{Gegeben sei die Funktionsvorschrift $\displaystyle{f(x)=\frac{x^2-9}{x^2+x-12}}$.
\begin{enumerate}[alph]
\item Bestimmen Sie den maximalen Definitionsbereich.
\item Bestimmen Sie den links- und rechtsseitigen Grenzwert an den Nullstellen des Nenners.
\end{enumerate}}
\item
Die Funktion \[f(x)=
\begin{cases}
1, & x<0\\
4-x, & x>2
\end{cases}
\]
soll für $x\in[0,2]$ so definiert werden, dass $f(x)$ stetig ist. 
\end{enumerate}

\begin{tabs*}[\initialtab{0}\class{exercise}]
  \tab{
    \lang{en}{...}
    \lang{de}{Antworten}
  }
\begin{enumerate}
\item 
\begin{enumerate}
 \item Es ist $D_{f}=\R\setminus\{-4,3\}.$
 \item Es gilt
 \begin{align*}
  &\lim_{x\downarrow -4} f(x) = - \infty \\
  &\lim_{x\uparrow -4} f(x) = \infty\\
  &\lim_{x\downarrow 3} f(x) = \frac{6}{7}\\
  &\lim_{x\uparrow 3} f(x) =\frac{6}{7}.
 \end{align*}
\end{enumerate}

\item  
Die Funktion ist stetig für $f(x)=\frac{1}{2}x+1$ für $x\in[0,2]$.

\end{enumerate}
        
  \tab{
    \lang{en}{...}
    \lang{de}{Lösung 1)}
  }
    \begin{incremental}[\initialsteps{1}]

      \step
        \lang{en}{...}
        \lang{de}{\textbf{(a)} Da $f$ eine rationale Funktion ist, ist der Definitionsbereich gleich $\R$ ohne die Nullstellen des Nenners. Wir bestimmen daher die Nullstellen des         
       Nennerpolynoms durch Faktorisieren und erhalten:
       \begin{align*}
        x^2+x-12 =0 ~\Leftrightarrow& ~(x-3)(x+4)= 0 
        \\ \phantom{x^2+x-12 =0} \Leftrightarrow& ~x-3=0 ~\vee ~ x+4=0 
        \\ \phantom{x^2+x-12 =0} \Leftrightarrow& ~ x=3 ~\vee ~x=-4.
       \end{align*}}
       \step
        \lang{en}{...}
        \lang{de}{
      \textbf{(b)} Die Funktion $f$ vereinfachen wir wie folgt (Merke: Wir dürfen Kürzen, da $x=3$ nicht im Definitionsbereich liegt.):
       \begin{align}
        f(x)=\frac{x^2-9}{x^2+x-12}=\frac{(x-3)(x+3)}{(x-3)(x+4)}=\frac{x+3}{x+4}.  \label{2}
       \end{align}
       Bestimme nun die links- und rechtsseitigen Grenzwerte an den Stellen $-4$ und $3$:
        Es gilt 
  \[
   \lim_{x\nearrow-4}(x+4)=\lim_{x\searrow-4}(x+4)=0
  \]
 und 
  \[
   \lim_{x\nearrow-4}(x+3)=\lim_{x\searrow -4}(x+3)=-1<0
  \]
 Da $x+4<0$ für $x<-4$ sowie $x+4>0$ für $x>-4$ gilt, erhalten wir 
              \begin{align*}
               &\lim_{x\searrow -4}f(x) \underset{(1)}{=}\lim_{x\searrow -4} \frac{x+3}{x+4} = -\infty\\
               &\lim_{x\nearrow -4}f(x) \underset{(1)}{=}\lim_{x\nearrow -4} \frac{x+3}{x+4} = +\infty.
              \end{align*}
       Weiter gilt 
	\begin{align*}
	 &\lim_{x\searrow 3}(x+3)=6=\lim_{x\nearrow 3}(x+3)\\
	 &\lim_{x\searrow 3}(x+4)=7=\lim_{x\nearrow 3}(x+4),
	\end{align*}
       also
	\[
	  \lim_{x\searrow 3}f(x) \underset{(1)}{=}\lim_{x\searrow 3} \frac{x+3}{x+4} = \frac{6}{7} = \lim_{x\nearrow 3}f(x).
	\] }
    \end{incremental}
  \tab{
    \lang{en}{...}
    \lang{de}{Lösungsvideo 2)}
  }
  \youtubevideo[500][300]{7FNmcJ3Vbtk}\\
\end{tabs*}




\end{content}