\documentclass{mumie.element.exercise}
%$Id$
\begin{metainfo}
  \name{
    \lang{de}{Ü04: Stetigkeit einer Funktion}
    \lang{en}{}
  }
  \begin{description} 
 This work is licensed under the Creative Commons License Attribution 4.0 International (CC-BY 4.0)   
 https://creativecommons.org/licenses/by/4.0/legalcode 

    \lang{de}{Hier die Beschreibung}
    \lang{en}{}
  \end{description}
  \begin{components}
  \end{components}
  \begin{links}
  \end{links}
  \creategeneric
\end{metainfo}
\begin{content}
\title{
\lang{de}{Ü04: Stetigkeit einer Funktion }
}
\begin{block}[annotation]
  Im Ticket-System: \href{http://team.mumie.net/issues/9988}{Ticket 9988}
\end{block}
 
\begin{enumerate}
\item
\lang{de}{Untersuchen Sie die Funktion $f\colon\mathbb{R} \rightarrow\mathbb{R}$, gegeben durch
\[f(x)=\begin{cases}
        \frac{1}{2}x+1,&\text{falls } x\leq 1\\
        \frac {1}{2} e^{x-1}+2,&\text{falls }x>1\\
       \end{cases}
       \]
auf Stetigkeit.}
\item
für welche Kombinationen von Parametern c und a ist die Funktion $f\colon\mathbb{R} \rightarrow\mathbb{R}$, gegeben durch
\[f(x)=\begin{cases}
        c\cdot x^2,&\text{falls } x<2\\
        \frac {1}{2}x+a,&\text{falls }x\geq 2\\
       \end{cases}
       \]
       stetig?\\
       Gibt es auch eine Kombination mit $c=a$?
\end{enumerate}

\begin{tabs*}[\initialtab{0}\class{exercise}]
 \tab{
    \lang{en}{...}
    \lang{de}{Antworten}
  }
  \begin{enumerate}
  \item  Die Funktion ist stetig auf $\R\setminus \{ 1\}$.
  \item Die Funktion $f(x)$ ist stetig für $4c=1+a$. 
  Insbesondere ist sie stetig für $c=a=\frac{1}{3}$.
  \end{enumerate}
  \tab{
    \lang{en}{...}
    \lang{de}{Lösung 1)}
  }
    \begin{incremental}[\initialsteps{1}]
      \step
        \lang{en}{...}
        \lang{de}{\textbf{Antwort:}\\ Die Funktion ist stetig auf $\R\setminus \{ 1\}$.}
      \step 
    \lang{de}{Die Funktion $f_1(x):=\frac{1}{2}x+1$ ist als Polynom auf ganz $\R$ stetig, 
 also ist $f$ auf $(-\infty,1) $ stetig. 
 Da auch die Exponentialfunktion $e^x$ auf ganz $\R$ stetig ist 
 und da die Funktionen $g_1(x):=x-1$ und $g_2(x):=\frac{1}{2}x+2$ als Polynome auf ganz
 $\R$ stetig sind, ist $f_2(x):=g_2(e^{g_1(x)})=\frac {1}{2} e^{x-1}+2$ 
 als Komposition stetiger Funktionen auf $\R$ stetig. 
 Also ist $f$ stetig in $(1,\infty)$. Es bleibt die Stelle $x_0=1$ zu untersuchen. }
     
    \step \lang{de}{Da die Komponentenfunktionen $f_1$ und $f_2$ auf $\R$ 
    stetig sind müssen wir nur die linksseitigen und rechtsseitigen Grenzwerte betrachten, 
    man sieht aber direkt, dass wir $x_0=1$ in beide Komponentenfunktionen einsetzen können.
 Also da $f_1(1)=\frac{3}{2}$ und $f_2(1)=\frac{5}{2}$, ist $f$ in $x_0=1$ nicht stetig. 
 Insgesamt erhalten wir daher die Stetigkeit von $f$ auf $\R\setminus\{1\}$. }
    \end{incremental}
  \tab{
    \lang{en}{...}
    \lang{de}{Lösungsvideo 2)}
  }
  \youtubevideo[500][300]{e4B7tDb-R5Q}
    
\end{tabs*}




\end{content}