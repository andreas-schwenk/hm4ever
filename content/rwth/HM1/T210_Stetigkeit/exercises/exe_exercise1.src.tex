\documentclass{mumie.element.exercise}
%$Id$
\begin{metainfo}
  \name{
    \lang{de}{Ü01: $\epsilon-\delta$-Kriterium}
    \lang{en}{}
  }
  \begin{description} 
 This work is licensed under the Creative Commons License Attribution 4.0 International (CC-BY 4.0)   
 https://creativecommons.org/licenses/by/4.0/legalcode 

    \lang{de}{Hier die Beschreibung}
    \lang{en}{}
  \end{description}
  \begin{components}
  \end{components}
  \begin{links}
  \end{links}
  \creategeneric
\end{metainfo}
\begin{content}
\title{
\lang{de}{Ü01: $\epsilon-\delta$-Kriterium}
}
\begin{block}[annotation]
  Im Ticket-System: \href{http://team.mumie.net/issues/9985}{Ticket 9985}
\end{block}
 
\lang{de}{\begin{enumerate}
 \item Zeigen Sie mit Hilfe des $\epsilon-\delta-$Kriteriums, dass
\[ f: [0, \infty) \to \R, x\mapsto \sqrt{x} \]
in $x_0=0$ stetig ist. 
\item  Zeigen Sie mit Hilfe des $\epsilon-\delta-$Kriteriums, dass
\[ f: \R\to \R, x\mapsto x^2+4x\]
stetig in $x_0=1$ ist. Begründen Sie außerdem, warum $f$ auf ganz $\R$ stetig ist.
\end{enumerate}}

\begin{tabs*}[\initialtab{0}\class{exercise}]

  \tab{
  \lang{de}{Lösung 1}}
  
    Es sei $x_0=0$ und $\epsilon>0$ beliebig. Wähle $\delta:= \epsilon^2 >0$. Dann gilt für 
 	alle $x\in [0,\infty)$ mit $|x-x_0|=|x| < \delta = \epsilon^2$:
	\[ |f(x) - f(x_0)| = |\sqrt{x}-0 | = \sqrt{x} < \sqrt{\epsilon^2} =\epsilon, \]
	da die Wurzelfunktion monoton ist und $\epsilon>0$ vorausgesetzt war. \\
	Also ist $f$ nach dem $\epsilon-\delta-$Kriterium in $x_0=0$ stetig.
  
    \tab{
  \lang{de}{Lösung 2}}
  
  Sei $x_0=1$ und $\epsilon>0$ beliebig. Wähle $\delta:= \min\{ 1, \frac{\epsilon}{7} \} >0$. 
	Dann gilt für alle $x\in \R$ mit $|x-x_0|= |x-1| < \delta$ bereits 
	\[ |x-1| < 1 \Leftrightarrow -1 < x-1 < 1 \Leftrightarrow 0 < x <2. \]
	Damit erhalten wir
\begin{align*} 
|f(x)-f(x_0)| &= |x^2+4x - (1^2 + 4\cdot 1) | \\
&= | x^2- 1^2 + 4x - 4 \cdot 1 | \\
&= | (x- 1) (x+1+4) | \\
&= | (x-1) (x+5) |\\
&< \delta |x+5| \\
&< \delta \cdot 7 \\
&\le \frac{\epsilon}{7}\cdot 7 = \epsilon.
\end{align*}
	Also ist $f$ in $x_0=1 $ stetig. \\
	Zusatz: Die Funktion ist als Polynom stetig auf ganz $\R$. Alternativ kann das 
	im Beweis durch allgemeines $x_0$ bewiesen werden, oder mit dem Folgenkriterium der Stetigkeit. 
  
\end{tabs*}

\end{content}