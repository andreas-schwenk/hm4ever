\documentclass{mumie.element.exercise}
%$Id$
\begin{metainfo}
  \name{
    \lang{de}{Ü02: Stetigkeit einer Funktion}
    \lang{en}{}
  }
  \begin{description} 
 This work is licensed under the Creative Commons License Attribution 4.0 International (CC-BY 4.0)   
 https://creativecommons.org/licenses/by/4.0/legalcode 

    \lang{de}{Hier die Beschreibung}
    \lang{en}{}
  \end{description}
  \begin{components}
  \end{components}
  \begin{links}
  \end{links}
  \creategeneric
\end{metainfo}
\begin{content}
\title{
\lang{de}{Ü02: Stetigkeit einer Funktion}
}
\begin{block}[annotation]
  Im Ticket-System: \href{http://team.mumie.net/issues/9986}{Ticket 9986}
\end{block}
 
\lang{de}{Zeigen Sie, dass
    \[
     f\colon \left(0,\frac{3}{2}\right)\to\R,\ f(x)=\begin{cases}
                                       \frac{x^2-1}{x^2-3x+2}, &x\neq 1\\
                                       0, & x=1
                                      \end{cases}
    \]
  für alle $x\in \left(0,\frac{3}{2}\right)$ definiert ist und untersuchen Sie $f$ auf Stetigkeit.}

\begin{tabs*}[\initialtab{0}\class{exercise}]

  \tab{
  \lang{de}{Antwort }}
  Die Funktion $f$ ist in $x_0=1$ nicht stetig.
  
  
  \tab{
  \lang{de}{Lösung }}
  \begin{incremental}[\initialsteps{1}]
    \step 
    \lang{de}{Wir suchen die Nullstellen des Nenners und bestimmen diese mit der $pq$-Formel. Es ergeben sich die beiden Lösungen:
    \[
     x_{1,2}=\frac{3}{2}\pm\sqrt{\frac{9}{4}-2}=\frac{3}{2}\pm\frac{1}{2},
    \]
  also $x_1=1$ und $x_2=2$. Da nur $x_1$ im Definitionsbereich enthalten ist gilt zunächst, dass 
  $\frac{x^2-1}{x^2-3x+2}$ auf $\left(0,\frac{3}{2}\right)\setminus\{1\}$ definiert ist. 
  Da $f(1)=0$ definiert ist, ist $f$ für alle $x\in \left(0,\frac{3}{2}\right)$ (eindeutig) definiert.}
     
    \step \lang{de}{Als Komposition stetiger Funktionen ist $f$ nun (mindestens) auf $\left(0,\frac{3}{2}\right)\setminus\{1\}$ stetig. 
    Wir können aber noch etwas mehr sagen: 
    Mit der 3. binomischen Formel gilt $x^2-1=(x-1)(x+1)$. Wir erhalten also mit den Grenzwertsätzen
    \[
     \lim_{x\to1}f(x)=\lim_{x\to1}\frac{x^2-1}{x^2-3x+2}=\lim_{x\to1}\frac{(x-1)(x+1)}{(x-1)(x-2)}=\lim_{x\to1}\frac{x+1}{x-2}=\frac{2}{-1}=-2.
    \]
  Also gilt
    \[
     \lim_{x\to1}f(x)=-2\neq 0=f(1).
    \]
  Damit ist $f$ in $x_0=1$ nicht stetig.
 }
  \end{incremental}

  
\end{tabs*}

\end{content}