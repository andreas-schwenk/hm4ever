\documentclass{mumie.problem.gwtmathlet}
%$Id$
\begin{metainfo}
  \name{
    \lang{de}{A03: Polstellen}
    \lang{en}{}
  }
  \begin{description} 
 This work is licensed under the Creative Commons License Attribution 4.0 International (CC-BY 4.0)   
 https://creativecommons.org/licenses/by/4.0/legalcode 

    \lang{de}{Beschreibung}
    \lang{en}{}
  \end{description}
  \corrector{system/problem/GenericCorrector.meta.xml}
  \begin{components}
    \component{js_lib}{system/problem/GenericMathlet.meta.xml}{mathlet}
  \end{components}
  \begin{links}
  \end{links}
  \creategeneric
\end{metainfo}
\begin{content}
\usepackage{mumie.genericproblem}

\lang{de}{
	\title{A03: Polstellen}
}

\begin{block}[annotation]
	Im Ticket-System: \href{http://team.mumie.net/issues/9977}{Ticket 9977}
\end{block}



\begin{problem}


\begin{question}
	
\begin{variables}
	\randint[Z]{a}{1}{10}
	\randint[Z]{c}{1}{10}
	\randint[Z]{q}{1}{5}
	
	\function[calculate]{b}{q*a}
	\function{xy}{b/a}
	
	\function[calculate]{tmp}{a*c}
	\function[calculate]{tmp2}{b*c}
	
	\function{f}{(a*x+b)/((tmp)*x-(tmp2))}
	\function[calculate]{sol1}{xy}
	\function{sol2}{-infty}
	\function{sol3}{infty}
	
\end{variables}
    
	\type{input.number}
	\field{rational} 
	\precision{3}
    \lang{de}{
	    \text{
	    Bestimmen Sie den maximalen Definitionsbereich für 
 $f(x) = \var{f}$, sowie den links- bzw. rechtsseitigen Grenzwert der Funktion für die vorhandene Polstelle $p$.\\
 (Tipp: Für $\infty$ / -$\infty$ schreiben Sie bitte $infty$ / -$infty$)
    }}
    \explanation[edited]{für $x<\var{sol1}$ ist der Nenner negativ und damit der
    Grenzwert $\var{sol2}$.}
    \begin{answer}
    \text{Der Definitionsbereich ist $\R \setminus \{p\}$ mit der Polstelle $p$ =}
	    \solution{sol1}
    \end{answer}
    
    \begin{answer}
	    	\text{$ \lim_{x \nearrow p} f(x) = $}
		    \solution{sol2}
            
	    \end{answer}
	    
	    \begin{answer}
	    	\text{$ \lim_{x \searrow p} f(x) = $ }
		    \solution{sol3}
            
	    \end{answer}
    
\end{question}
    
\end{problem}


\embedmathlet{mathlet}

\end{content}