\documentclass{mumie.problem.gwtmathlet}
%$Id$
\begin{metainfo}
  \name{
    \lang{de}{A01: Aussagen zu Stetigkeit}
    \lang{en}{mc yes-no}
  }
  \begin{description} 
 This work is licensed under the Creative Commons License Attribution 4.0 International (CC-BY 4.0)   
 https://creativecommons.org/licenses/by/4.0/legalcode 

    \lang{de}{Beschreibung}
    \lang{en}{description}
  \end{description}
  \corrector{system/problem/GenericCorrector.meta.xml}
  \begin{components}
    \component{js_lib}{system/problem/GenericMathlet.meta.xml}{gwtmathlet}
  \end{components}
  \begin{links}
  \end{links}
  \creategeneric
\end{metainfo}
\begin{content}
\usepackage{mumie.ombplus}
\usepackage{mumie.genericproblem}

\lang{de}{\title{A01: Aussagen zu Stetigkeit}}
\lang{en}{\title{Problem 1}}

\begin{block}[annotation]
	Im Ticket-System: \href{http://team.mumie.net/issues/9996}{Ticket 9996}
\end{block}

\begin{problem}

\begin{variables}
      \randint{a}{-9}{9}
      \randint{b}{-9}{9}
      \randint{d}{-9}{9}
      \randint[Z]{c}{-9}{9}
      \randint{p}{1}{5}
      \randint{q}{-5}{5}
      \function[normalize]{funktion}{x - b}
      
      
      
      \function[normalize]{f1}{a*x*x+b*x+c}
      \function[normalize]{f2}{a*x+b}
      \function[normalize]{f31}{a*x*x + b*x +p}
      \function[normalize]{f32}{x-d}
      \function[normalize]{f4}{a*x+q}
      \end{variables}
      
		\randomquestionpool{1}{1}
		\randomquestionpool{2}{4}
		\randomquestionpool{5}{6}
		
		\randomquestionpool{7}{7}
		\randomquestionpool{8}{17}
		\randomquestionpool{18}{20}
		\randomquestionpool{21}{25}
      
      
     \begin{question} %1
     \lang{de}{ 
      	\text{Kreuzen Sie korrekte Aussagen an. Wenn die Frage lautet: \\
      	Ist $f$ stetig, so ist stets nach Stetigkeit auf dem gesamten Definitionsbereich gefragt.\\
      	Wenn nicht anders angegeben, sei $f:\R\to\R$.
      	}
      }
      
    	\type{mc.yesno}
    	\begin{choice}
    		\text{$f(x) = \var{f1}$ ist stetig.}
  			\solution{true}
            \explanation[edited]{Polynome sind auf $R$ stetig.}
		\end{choice}
		\end{question}
		
		\begin{question} %2
		\lang{de}{\text{ }}
		\type{mc.yesno}
		\begin{choice}
  			\text{$f(x) = e^{\var{c}x+\var{p}}(\var{f2})$ ist stetig.}
  			\solution{true}
            \explanation[edited]{Eine Komposition stetiger Funktionen (e-Funktion und 
            Polynome, beachten Sie auch das Polynom im Exponenten) ist stetig.}
		\end{choice}
		\end{question}
		
		\begin{question} %3
		\lang{de}{\text{ }}
		\type{mc.yesno}
		\begin{choice}
			\text{$f(x)= \var{c}+|\var{f2}|$ ist stetig.}
  			\solution{true}
            \explanation[edited]{Eine Komposition stetiger Funktionen 
            (Betragsfunktion und Polynom) ist stetig.}
		\end{choice}
		\end{question}
		
		\begin{question} %4
		\lang{de}{\text{ }}
		\type{mc.yesno}
		\begin{choice}
			\text{$f(x)= \begin{cases}
                       \var{funktion} &, \text{ falls } x\ge 0 \\
                       x^2 &, \text{ falls } x<0
                       \end{cases}$ ist stetig.}
  			\solution{compute}
  			\iscorrect{b}{=}{0}
            \explanation[edited]{Diese abschnittsweise definierte Funktion (gebildet aus stetigen
            Funktionen) kann auf $R$ nur stetig sein, wenn es sich für $x\geq 0$ um eine Ursprungsgerade 
            handelt.}
		\end{choice}
		\end{question}
		
		\begin{question} %5
		\lang{de}{\text{ }}
		\type{mc.yesno}
		\begin{choice}
			\text{$f: \R\setminus \{\var{d} \} \to \R, f(x) = \frac{\var{f31}}{\var{f32}}$ ist stetig.}
  			\solution{true}
            \explanation[edited]{Diese gebrochen rationale Funktion hat eine Polstelle bei
            $x=\var{d}$. Also ist sie auf $R\setminus \{\var{d} \}$ stetig.}
		\end{choice}
		\end{question}
		
		
		\begin{question} %6
		\lang{de}{\text{ }}
		\type{mc.yesno}
		\begin{choice}
			\text{$f(x) = \begin{cases}
                          \var{b}+x^2 &, \text{ falls } x \ge \var{p} \\
                          \var{f4} &, \text{ falls } x<\var{p}
                         \end{cases}$ ist stetig. }
  			\solution{compute}
  			\iscorrect{p^2+b}{=}{a*p+q}
            \explanation[edited]{Wenn beide (stetigen) Teilfunktionen für $x=\var{p}$ denselben
            Funktionswert besitzen, ist die Funktion $f(x)$ auf $R$ stetig.}
		\end{choice}
    \end{question}
    
    
    \begin{question} %7
    	\lang{de}{
	      	\text{Kreuzen Sie korrekte Aussagen an. Wenn die Frage lautet: \\
      	Ist $f$ stetig, so ist stets nach Stetigkeit auf dem gesamten Definitionsbereich gefragt.\\
      	Eventuell werden eine oder beide der wie folgt definierten Funktionen auftauchen.\\
  	Die sogenannte \textit{Signum-}Funktion (kurz: sgn) ist definiert durch\\
   $ \text{sgn}: \R\to\R, \text{sgn}(x) := \begin{cases} 
                             -1 &, \text{ falls } x < 0, \\
                             0 &, \text{ falls } x= 0, \\
                             1, &, \text{ falls } x>0.
                            \end{cases}$\\
	Außerdem definiert man die \textit{Heaviside-}Funktion durch\\
	$h:\R\to\R, h(x):= \begin{cases}
                       0 &, \text{ falls } x< 0 \\
                       1 &, \text{ falls } x\ge0
                       \end{cases}. $

    	}
    	}
    	\type{mc.yesno}

    	
    	\begin{choice}
    		\text{Die Gauß-Klammer-Funktion ist linksseitig stetig. }
  			\solution{false}
            \explanation[edited]{Die Gauß-Klammer-Funktion ist linksseitig für $x\in Z$ nicht stetig. 
            Beispielsweise gilt: $\lim_{x \nearrow 1}[x]=0 \neq 1=[1]$.}
		\end{choice}
		\end{question}
		
		
		\begin{question} %8
    	\lang{de}{
	      	\text{ }}
	      	\type{mc.yesno}
		\begin{choice}
			\text{Die Gauß-Klammer-Funktion ist rechtsseitig stetig.}
  			\solution{true}
            \explanation[edited]{Die Gauß-Klammer-Funktion ist rechtsseitig für $x\in Z$ stetig. 
            Beispielsweise gilt: $\lim_{x \searrow 1}[x]=1=[1]$.}
		\end{choice}
		\end{question}
		
		\begin{question} %9
    	\lang{de}{
	      	\text{ }}
	      	\type{mc.yesno}
		\begin{choice}
			\text{Die Gauß-Klammer-Funktion ist stetig.}
  			\solution{false}
            \explanation[edited]{Die Gauß-Klammer-Funktion hat Sprungstellen für $x\in Z$, 
            sie ist also nicht stetig.}
		\end{choice}
		\end{question}
		
		
		
		\begin{question} %10
    	\lang{de}{
	      	\text{ }}
	      	\type{mc.yesno}
		\begin{choice}
			\text{Die Dirichlet'sche Sprungfunktion ist rechtsseitig stetig.}
  			\solution{false}
            \explanation[edited]{Die Dirichlet'sche Sprungfunktion ist an jeder Stelle unstetig, 
            weil sie je nach Zugehörigkeit von $x$ zu $Q$ die Werte $0$ oder $1$ annimmt.}
		\end{choice}
		\end{question}
		
		\begin{question} %11
    	\lang{de}{
	      	\text{ }}
	      	\type{mc.yesno}
		\begin{choice}
			\text{Die Dirichlet'sche Sprungfunktion ist linksseitig stetig.}
  			\solution{false}
            \explanation[edited]{Die Dirichlet'sche Sprungfunktion ist an jeder Stelle unstetig, 
            weil sie je nach Zugehörigkeit von $x$ zu $Q$ die Werte $0$ oder $1$ annimmt.}
		\end{choice}
		\end{question}
		
		\begin{question} %12
    	\lang{de}{
	      	\text{ }}
	      	\type{mc.yesno}
		\begin{choice}
			\text{Die Signum-Funktion ist stetig.}
  			\solution{false}
            \explanation[edited]{Die Signum-Funktion hat bei $x=0$ eine Sprungstelle, ist also nicht 
            auf $R$ stetig.}
		\end{choice}
		\end{question}
		
		\begin{question} %13
    	\lang{de}{
	      	\text{ }}
	      	\begin{variables}
	      	\randint[Z]{c}{-9}{9}
	      	\end{variables}
	      	\type{mc.yesno}
		\begin{choice}
			\text{Die Signum-Funktion ist stetig in $x_0=\var{c}$. }
  			\solution{true}
            \explanation[edited]{Die Signum-Funktion hat nur bei $x=0$ eine Sprungstelle, 
            bei $\var{c}$ ist sie also stetig. }
		\end{choice}
		\end{question}
		
		\begin{question} %14
    	\lang{de}{
	      	\text{ }}
	      	\type{mc.yesno}
		\begin{choice}
			\text{Die Heaviside-Funktion ist stetig.}
  			\solution{false}
            \explanation[edited]{Die Heaviside-Funktion hat bei $x=0$ eine Sprungstelle,
            ist also nicht stetig auf $R$.}
		\end{choice}
		\end{question}
		
		\begin{question} %15
    	\lang{de}{
	      	\text{ }}
	      	\type{mc.yesno}
		\begin{choice}
			\text{Die Exponentialfunktion ist stetig ist auf $\R$.}
  			\solution{true}
            \explanation[edited]{Die Exponentialfunktion ist das 
            klassische Beispiel einer stetigen Funktion.}
		\end{choice}
		\end{question}
		
		\begin{question} %16
    	\lang{de}{
	      	\text{ }}
	      	\begin{variables}
	      	\randint[Z]{c}{-9}{9}
	      	\end{variables}
	      	\type{mc.yesno}
		\begin{choice}
			\text{Die Heaviside-Funktion ist stetig in $x_0=\var{c}$.}
  			\solution{true}
            \explanation[edited]{Die Heaviside-Funktion hat nur bei $x=0$ eine Sprungstelle, 
            d.h. in $x_0=\var{c}$ ist sie stetig.}
		\end{choice}
		\end{question}
		
		\begin{question} %17
    	\lang{de}{
	      	\text{ }}
	      	\type{mc.yesno}
		\begin{choice}
			\text{ Die Heaviside-Funktion ist rechtsseitig stetig.}
  			\solution{true}
            \explanation[edited]{Die Heaviside-Funktion ist rechtsseitig stetig, weil 
            $\lim_{x\searrow 0}h(x)=1=h(0)$.}
		\end{choice}
		\end{question}
		
		\begin{question} %18
    	\lang{de}{
	      	\text{ }}
	      	\type{mc.yesno}
		\begin{choice}
			\text{Wenn $f:\R\to\R, \ g: \R\to \R$ stetig sind, so auch $f+g:\R\to \R$.}
  			\solution{true}
            \explanation[edited]{Kompositionen stetiger Funktionen sind stetig. }
		\end{choice}
		\end{question}
		
		\begin{question} %19
    	\lang{de}{
	      	\text{ }}
	      	\type{mc.yesno}
		\begin{choice}
			\text{ Wenn $f:\R\to\R, \ g: \R\to \R$ stetig sind, so auch $f\cdot g:\R\to\R$.}
  			\solution{true}
            \explanation[edited]{Kompositionen stetiger Funktionen sind stetig. }
		\end{choice}
		\end{question}
		
		\begin{question} %20
    	\lang{de}{
	      	\text{ }}
	      	\type{mc.yesno}
		\begin{choice}
			\text{Wenn $f:\R\to\R$ stetig ist, so auch $r\cdot f:\R\to\R,\ r\in\R$.}
  			\solution{true}
            \explanation[edited]{Dies ist eine einfache Komposition stetiger Funktionen: $f(x)$ und 
            $g(x)=r$.}
		\end{choice}
		\end{question}
		
		\begin{question} %21
    	\lang{de}{
	      	\text{ }}
	      	\type{mc.yesno}
		\begin{choice}
			\text{Wenn $f$ linksseitig stetig ist, so ist $f$ stetig.}
  			\solution{false}
            \explanation[edited]{Für Stetigkeit muss $f$ auch noch rechtsseitig stetig sein.}
		\end{choice}
		\end{question}
		
		\begin{question} %22
    	\lang{de}{
	      	\text{ }}
	      	\type{mc.yesno}
		\begin{choice}
			\text{Wenn $f$ rechtsseitig stetig ist, so ist $f$ stetig.}
  			\solution{false}
            \explanation[edited]{Für Stetigkeit muss $f$ auch noch linksseitig stetig sein.}
		\end{choice}
		\end{question}
		
		\begin{question} %23
    	\lang{de}{
	      	\text{ }}
	      	\type{mc.yesno}
		\begin{choice}
			\text{Wenn $f$ links- und rechtsseitig stetig ist, so ist $f$ stetig.}
  			\solution{true}
            \explanation[edited]{Rechts- und linksseitige Stetigkeit ist hinreichend für Stetigkeit.}
		\end{choice}
		\end{question}
		
		\begin{question} %24
    	\lang{de}{
	      	\text{ }}
	      	\type{mc.yesno}
		\begin{choice}
			\text{Wenn $f$ stetig ist, so ist $f$ linksseitig stetig.}
  			\solution{true}
            \explanation[edited]{eine stetige Funktion ist sowieso linksseitig stetig.}
		\end{choice}
		\end{question}
		
		\begin{question} %25
    	\lang{de}{
	      	\text{ }}
	      	\type{mc.yesno}
		\begin{choice}
			\text{Wenn $f$ stetig ist, so ist $f$ rechtsseitig stetig.}
  			\solution{true}
            \explanation[edited]{eine stetige Funktion ist sowieso rechtsseitig stetig.}
		\end{choice}
		
    \end{question}

    
	
\end{problem}

\embedmathlet{gwtmathlet}



\end{content}