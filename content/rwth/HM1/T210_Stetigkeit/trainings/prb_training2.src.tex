\documentclass{mumie.problem.gwtmathlet}
%$Id$
\begin{metainfo}
  \name{
    \lang{de}{A02: Funktionengrenzwerte}
    \lang{en}{}
  }
  \begin{description} 
 This work is licensed under the Creative Commons License Attribution 4.0 International (CC-BY 4.0)   
 https://creativecommons.org/licenses/by/4.0/legalcode 

    \lang{de}{Beschreibung}
    \lang{en}{}
  \end{description}
  \corrector{system/problem/GenericCorrector.meta.xml}
  \begin{components}
    \component{js_lib}{system/problem/GenericMathlet.meta.xml}{mathlet}
  \end{components}
  \begin{links}
  \end{links}
  \creategeneric
\end{metainfo}
\begin{content}
\usepackage{mumie.genericproblem}

\lang{de}{
	\title{A02: Funktionengrenzwerte}
}

\begin{block}[annotation]
	Im Ticket-System: \href{http://team.mumie.net/issues/9976 }{Ticket 9976}
\end{block}



\begin{problem}

    \begin{question}
	
		
		\begin{variables}
		
			\randint[Z]{a}{1}{3}
		
			\randint[Z]{c}{-9}{9}
			\randadjustIf{c}{c=0}
			
			\function{xy}{-c}
			
			\randint[Z]{d}{-9}{9}
			\randadjustIf{d}{c=0 OR c+d=0}
			
			\function[calculate]{tmp}{c*d}
			
			\function[calculate]{h}{c+d}
			\function{f}{a*((x^2+h*x+(tmp))/(x+c))}
			\function{g}{a*(x+c)(x+d)}
            
			\function[calculate]{sol}{a*(-c+d)}
			
		\end{variables}
	    
        \explanation[edited]{Der Zähler von $f(x)$ kann aufgespalten werden in 
        $\var{g}$. Dann kann gekürzt werden und $f(\var{xy})$ berechnet
        werden.}
		\type{input.number}
		\field{real} 
		\precision{3}
	    \lang{de}{
		    \text{
	Betrachten Sie die Funktion 
	 $ f: \R \setminus\{\var{xy}\} \to \R; x \mapsto \var{f}.$\\
	 Bestimmen Sie: 
	    }}
	    
	    %up- und downarrow gehen nicht :S
	    \begin{answer}
	    	\text{$ \lim_{x \nearrow \var{xy}} f(x) = $}
		    \solution{sol}
	    \end{answer}
	    
	    \begin{answer}
	    	\text{$ \lim_{x \searrow \var{xy}} f(x) = $ }
		    \solution{sol}
	    \end{answer}
	    
	\end{question}
	
	\begin{question}
	
		
		\begin{variables}
		
			\randint[Z]{a}{1}{3}
		
			\randint[Z]{c}{-9}{9}
			\randadjustIf{c}{c=0}
			
			\function{xy}{-c}
			
			\randint[Z]{d}{-9}{9}
			\randadjustIf{d}{c=0 OR c+d=0}
			
			\function[normalize]{f}{a*((x^2-c^2)/abs(x+c))}
            \function{g}{a*(x+c)(x-c)}
			
			\function[calculate]{sol1}{2*c*a}
			\function[calculate]{sol2}{-2*c*a}
			
		\end{variables}
	    
        \explanation{Mit Hilfe der 3. binomischen Formel kann der Zähler der Funktion 
        geschrieben werden als $\var{g}$. Beim Kürzen müssen die Betragsstriche berücksichtigt werden: 
        will man $ \lim_{x \nearrow \var{xy}} f(x)  $ berechnen, kommt ein zusätzliches Minuszeichen hinzu:
        Der Term $(x+\var{c})$ ist für $x<\var{xy}$ negativ.
        }
		\type{input.number}
		\field{real} 
		\precision{3}
	    \lang{de}{
		    \text{
	Betrachten Sie die Funktion 
	 $ f: \R \setminus\{\var{xy}\} \to \R; x \mapsto \var{f}.$\\
	 Bestimmen Sie: 
	    }}
	    
	    \begin{answer}
	    	\text{$ \lim_{x \nearrow \var{xy}} f(x) = $}
		    \solution{sol1}
	    \end{answer}
	    
	    \begin{answer}
	    	\text{$ \lim_{x \searrow \var{xy}} f(x) = $ }
		    \solution{sol2}
	    \end{answer}
	    
	\end{question}
    
    \begin{question}
        
        
        \begin{variables}
       
            \randint[Z]{a}{2}{4}
            
            \function{f1}{x^2-a}
            \function{f2}{ln((1+x)/e^a)}
            
            \function[calculate]{sol}{-a}
            
        \end{variables}
        
        \explanation{Der Zahlenwert 0 kann in beide Teilfunktionen eingesetzt werden:
        $\ln \frac{1}{e^\var{a}}=\var{sol}$.}
        \type{input.number}
		\field{real} 
		\precision{3}
	    \lang{de}{
		    \text{
Betrachten Sie die Funktion
  
  $ f: \R \setminus\{0\} \to \R; x \mapsto 
	 
  
  
  \begin{cases}
  \var{f1}&x<0\\
  \var{f2}&x>0.\\     
  \end{cases}$\\
Bestimmen Sie:
    }}
    \begin{answer}
	    	\text{$ \lim_{x \searrow 0} f(x) = $}
		    \solution{sol}
	\end{answer}
    
    \begin{answer}
	    	\text{$ \lim_{x \nearrow 0} f(x) = $}
		    \solution{sol}
	\end{answer}
    
    
    
    \end{question}
    
\end{problem}


\embedmathlet{mathlet}

\end{content}