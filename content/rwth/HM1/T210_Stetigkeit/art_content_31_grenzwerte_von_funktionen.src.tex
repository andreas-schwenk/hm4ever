%$Id:  $
\documentclass{mumie.article}
%$Id$
\begin{metainfo}
  \name{
    \lang{de}{Folgen von Funktionswerten}
    \lang{en}{}
  }
  \begin{description} 
 This work is licensed under the Creative Commons License Attribution 4.0 International (CC-BY 4.0)   
 https://creativecommons.org/licenses/by/4.0/legalcode 

    \lang{de}{Beschreibung}
    \lang{en}{}
  \end{description}
  \begin{components}
    \component{generic_image}{content/rwth/HM1/images/g_tkz_T210_gebrochen_rat_2.meta.xml}{T210_gebrochen_rat_2}
    \component{generic_image}{content/rwth/HM1/images/g_tkz_T210_gebrochen_rat_1.meta.xml}{T210_gebrochen_rat_1}
    \component{generic_image}{content/rwth/HM1/images/g_tkz_T210_xSine1x.meta.xml}{T210_xSine1x}
    \component{generic_image}{content/rwth/HM1/images/g_tkz_T210_Sine1x.meta.xml}{T210_Sine1x}
    \component{generic_image}{content/rwth/HM1/images/g_img_00_Videobutton_schwarz.meta.xml}{00_Videobutton_schwarz}
    \component{generic_image}{content/rwth/HM1/images/g_img_00_video_button_schwarz-blau.meta.xml}{00_video_button_schwarz-blau}
    \component{js_lib}{system/media/mathlets/GWTGenericVisualization.meta.xml}{mathlet1}
  \end{components}
  \begin{links}
    \link{generic_article}{content/rwth/HM1/T206_Folgen_II/g_art_content_19_bestimmte_divergenz.meta.xml}{content_19_bestimmte_divergenz}
    \link{generic_article}{content/rwth/HM1/T205_Konvergenz_von_Folgen/g_art_content_14_konvergenz.meta.xml}{content_14_konvergenz}
    \link{generic_article}{content/rwth/HM1/T210_Stetigkeit/g_art_content_29_stetigkeit_definitionen.meta.xml}{content_29_stetigkeit_definitionen}
    \link{generic_article}{content/rwth/HM1/T205_Konvergenz_von_Folgen/g_art_content_16_konvergenzkriterien.meta.xml}{konv-krit}
  \end{links}
  
  % Alte Components/Links:  
  % \begin{components}
  %   \component{generic_image}{content/rwth/HM1/images/g_tkz_T210_xSine1x.meta.xml}{T210_xSine1x}
  %   \component{generic_image}{content/rwth/HM1/images/g_tkz_T210_Sine1x.meta.xml}{T210_Sine1x}
  %   \component{generic_image}{content/rwth/HM1/images/g_img_00_Videobutton_schwarz.meta.xml}{00_Videobutton_schwarz}
  %   \component{generic_image}{content/rwth/HM1/images/g_img_00_video_button_schwarz-blau.meta.xml}{00_video_button_schwarz-blau}
  %   \component{js_lib}{system/media/mathlets/GWTGenericVisualization.meta.xml}{mathlet1}
  % \end{components}
  % \begin{links}
  %   \link{generic_article}{content/rwth/HM1/T210_Stetigkeit/g_art_content_29_stetigkeit_definitionen.meta.xml}{content_29_stetigkeit_definitionen}
  %   \link{generic_article}{content/rwth/HM1/T205_Konvergenz_von_Folgen/g_art_content_16_konvergenzkriterien.meta.xml}{konv-krit}
  % \end{links}
  \creategeneric
\end{metainfo}
\begin{content}
\usepackage{mumie.ombplus}
\ombchapter{10}
\ombarticle{3}
\usepackage{mumie.genericvisualization}

\begin{visualizationwrapper}

\lang{de}{\title{Folgen von Funktionswerten}}
 
\begin{block}[annotation]
  Das alte Kapitel "Grenzwerte von Funktionswerten" wird umgewandelt in "Folgen von Funktionswerten". Es wird VOR das Kapitel Stetigkeit (content_29) gesetzt.
  
\end{block}
\begin{block}[annotation]
  Im Ticket-System: \href{http://team.mumie.net/issues/9791}{Ticket 9791}\\
\end{block}

\begin{block}[info-box]
\tableofcontents
\end{block}
Wir betrachten eine reelle Folge $(x_n)_{n\in\N}$ mit Grenzwert $x^\ast$, also $\lim_{n\to\infty}x_n=x^\ast$.
Haben wir nun zusätzlich eine reelle Funktion $f:D\to \R$, und gehören die Folgeglieder $x_n$ zum Definitionsbereich, 
d.h. $x_n\in D$ für alle $n\in\N$, dann liegt es nahe, die Funktionswerte $f(x_n)$ der Folgeglieder zu bilden.
\\
Diese Funktionswerte bilden wieder eine reelle Folge $(f(x_n))_{n\in\N}$. Wie wird sich diese Folge verhalten? 
Wird sie ebenso wie $(x_n)_{n\in\N}$ selbst konvergieren?\\
Die Antwort hängt stark von der Funktion $f$ ab. 
In der Tat können wir, indem wir das Verhalten vieler solcher Funktionswertfolgen $(f(x_n))_{n\in\N}$ studieren, umgekehrt auf
Eigenschaften der Funktion $f$ schließen.

\begin{example}\label{ex:erstes_bsp_Grenzwert_von_Funktionswerten}
\begin{tabs*}
\tab{Beispiel (a)}
Die gebrochen rationale Funktion $f:\R\setminus\{1\}\to\R$, $x\mapsto \frac{x^2+1}{x-1}$, hat in $x^\ast=1$ eine Definitionslücke.
Wir betrachten die Folge $x_n=1+\frac{1}{n}$ mit $\lim_{n\to\infty}x_n=1$.
Für die Folge der Funktionswerte $f(x_n)$ finden wir
\[
\lim_{n\to\infty}f(x_n)=\frac{(1+\frac{1}{n})^2+1}{1+\frac{1}{n}-1}=\frac {2+\frac{2}{n}+\frac{1}{n^2}}{\frac{1}{n}}=+\infty.
\]
Diese Folge divergiert also bestimmt gegen $+\infty$.

Wählen wir stattdessen die Folge $y_n=1-\frac{1}{n}$ ebenfalls mit $\lim_{n\to\infty}y_n=1$, dann erhalten wir
\[
\lim_{n\to\infty}f(y_n)=\frac{(1-\frac{1}{n})^2+1}{1-\frac{1}{n}-1}=\frac {2-\frac{2}{n}+\frac{1}{n^2}}{-\frac{1}{n}}=-\infty,
\]
also bestimmte Divergenz gegen $-\infty$.
Das Verhalten der beiden Funktionswertfolgen spiegelt das Verhalten des Funktionsgraphen in der Nähe der Definitionslücke wider.
\begin{genericJSXVisualization}[550][800]{example_b}
 		\begin{variables}
 			\function{f}{real}{(x^2+1)/(x-1)}  
			\point{O}{real}{1,0} 			
			\point{P}{real}{1,1} 			
 			\line{l}{real}{var(O),var(P)}
 			
 		\end{variables}
 		\color{l}{#00CC00}
 
 		\begin{canvas}
 			\plotSize{400,300}
 			\plotLeft{-4}
 			\plotRight{4}
 			\plot[coordinateSystem]{f,l }
 		\end{canvas}
 \end{genericJSXVisualization}
\tab{Beispiel (b)}
Auch die Funktion $g:\R\setminus\{1\}\to\R$, $x\mapsto \frac{x^2-1}{x-1}$, hat in $x^\ast=1$ eine Definitionslücke.
Für alle $x\neq 1$ gilt aber
\[
g(x)=\frac{x^2-1}{x-1}=\frac {(x-1)(x+1)}{x-1}=\frac{x+1}{1}=x+1.
\]
Somit gilt für eine Folge $(x_n)_{n\in\N}$ in $\R\setminus\{1\}$, die gegen $1$ konvergiert, $\lim_{n\to\infty}x_n=1$,
\[
\lim_{n\to\infty}g(x_n)=\lim_{n\to\infty}(x_n+1)=2,
\]
unabhängig davon, wie die Folge genau aussieht. Auch dies spiegelt das Verhalten des Funktionsgraphen wider.
\begin{center}
\image[400]{T210_gebrochen_rat_2}
\end{center}
\end{tabs*}
\end{example}

\section{Grenzwerte von Funktionswerten}\label{sec: Funktionsgrenzwerte}
In Beispiel \ref{ex:erstes_bsp_Grenzwert_von_Funktionswerten}(b) konnten wir die Grenzwerte von 
Funktionswertfolgen $(f(x_n))_{n\in\N}$ bestimmen unabhängig davon, wie die Folgen $(x_n)_{n\in\N}$ genau aussahen, 
wenn sie nur den gemeinsamen Grenzwert $x^\ast$ hatten. 
Diesen besonderen Fall beschreibt die folgende Definition.
\begin{definition}\label{def:funktionsgrenzwert}
Es sei $f:D\to \R$ eine reelle Funktion und es sei $L\in\R$ eine reelle Zahl.
Es existiere mindestens eine Folge $(x_n)_{n\geq 1}$ mit $x_n\in D\setminus \{x^*\}$ für alle $n\in\N$
und $\lim_{n\to\infty}x_n=x^\ast\in\R$.
Wenn nun für \notion{jede} Folge $(x_n)_{n\in\N}$ mit $x_n\in D\setminus \{x^*\}$, die gegen den Grenzwert $x^*$ 
konvergiert, gilt
  \[ \lim_{n\to \infty} f(x_n)= L\:, \]
dann heißt $L$ der \notion{Grenzwert der Funktion $f$ für $x$ gegen $x^\ast$}.
\\
Ist diese Bedingung erfüllt, dann schreiben wir abkürzend
\[ \lim_{x\to x^*} f(x)= L. \]
\floatright{\href{https://www.hm-kompakt.de/video?watch=400}{\image[75]{00_Videobutton_schwarz}}}\\\\
\end{definition}
 \begin{remark}\label{rem:funktionsgrenzwert}
\begin{enumerate}
\item[(a)]
Wir bemerken, dass in Definition \ref{def:funktionsgrenzwert} der Punkt $x^*$ gar nicht zum Definitionsbereich von $f$ gehören muss. 
Dies ist sogar ein sehr üblicher Fall!\\
Selbst wenn $f$ in $x^*$ definiert ist, ist für den Grenzwert von $f$ für $x$ gegen $x^*$ der Funktionswert $f(x^*)$ unerheblich, 
weil bei den Folgen stets $x_n\neq x^*$ vorausgesetzt ist. \\
Der Grenzwert einer Funktion gegen ein $x^\ast$ gibt also Aufschluss darüber, wie sich die Funktion um $x^\ast$ herum verhält, nicht in $x^\ast$ selbst.
\item[(b)]
Trotzdem ist $x^\ast$ nicht völlig beliebig, sondern es muss sich mindestens um einen Randpunkt (oder wie in (c) unten um $\pm\infty$) handeln.
Es macht zum Beispiel keinen Sinn, eine Funktion $f:(0;1)\to \R$ in der Nähe von $x^\ast=2$ zu untersuchen.
Das wird durch die Bedingung 
\emph{Es existiere mindestens eine Folge $(x_n)_{n\geq 1}$ mit $x_n\in D\setminus \{x^*\}$ für alle $n\in\N$
und $\lim_{n\to\infty}x_n=x^\ast\in\R$. } 
gewährleistet und erklärt deren Sinn.

\item[(c)]
Wir erweitern den Begriff des Funktionsgrenzwerts aus Definition \ref{def:funktionsgrenzwert}, indem wir auch $x^\ast=+\infty$ (oder $x^\ast=-\infty$) zulassen:
Gilt für jede Folge $(x_n)_{n\in\N}$ im Definitionsbereich, die bestimmt gegen $+\infty$ divergiert, dass
der Grenzwert $\lim_{x_n\to\infty}f(x_n)$ existiert und gleich $L$ ist, 
dann heißt $L$ der \notion{Grenzwert von $f$ für $x$ gegen $+\infty$}. (Analog für $-\infty$.)

\end{enumerate}
\end{remark}

\begin{example}\label{ex:Funktionsgrenzwerte}
\begin{tabs*}
\tab{Beispiel \ref{ex:erstes_bsp_Grenzwert_von_Funktionswerten}(b)}
Wir haben bereits gezeigt, dass unabhängig vom konkreten Aussehen der Folge $(x_n)_{n\in\N}$ mit Grenzwert $\lim_{n\to\infty}x_n=1$ gilt
$\lim_{n\to\infty}g(x_n)=2$. Also gilt das erst recht für alle solche Folgen mit $x_n\neq 1$ für alle $n\in\N$. Somit existiert der
Grenzwert von $f$ für $x$ gegen $x^\ast=1$, und er ist
\[
\lim_{x\to 1}g(x)=2\:.
\]
\tab{$\frac{1}{x}$ für $x\to\pm\infty$}
Für die Funktion $f:\R\setminus\{0\}\to\R$, $x\mapsto \frac{1}{x}$, gilt
\[\lim_{x\to\infty}\frac{1}{x}=0\quad\text{ sowie }\quad \lim_{x\to-\infty}\frac{1}{x}=0.\]
Denn für jede bestimmt divergente Folge $(x_n)_{n\in\N}$, also $\lim_{n\to\infty}x_n=+\infty$ (oder $-\infty$),
gilt nach den \ref[content_19_bestimmte_divergenz][Rechenregeln für bestimmte Divergenz]{sec:rechenregeln}
$\lim_{n\to\infty}\frac{1}{x_n}=0$. Also gilt auch $\lim_{x\to\pm\infty}\frac{1}{x}=0$.
\tab{$\sin\left(\frac{1}{x}\right)$}
Wir betrachten die Funktion \[ f:\R\setminus { \{0\} }\mapsto [-1;1], x\mapsto \sin\left(\frac{1}{x}\right), \]
und den Punkt $x^*=0$.\\
\begin{enumerate}
\item[(i)]
Für die spezielle Folge $x_n=\frac{1}{\pi n}$ ist $\lim_{n\to \infty} x_n=0=x^*$.
Der Grenzwert der Folge der Funktionswerte $f(x_n)$ ist
\[ \lim_{n\to \infty} f(x_n)=\lim_{n\to \infty} \sin\left(\frac{1}{x_n}\right)=\lim_{n\to \infty} \sin(\pi n)=\lim_{n\to \infty} 0=0. \]
\item[(ii)]
Für die Folge $x_n=\frac{2}{\pi(4n+1)}$ ist ebenfalls $\lim_{n\to \infty} x_n=0=x^*$.
Es gilt aber 
\[\]
Der Grenzwert der  Funktionswertfolge $(f(x_n))_{n\in\N}$ ist jetzt aber
\[ =\lim_{n\to \infty} \sin\left(\frac{1}{x_n}\right)=
\lim_{n\to\infty}\sin\left(\frac{\pi}{2}+2n\pi\right)
=\lim_{n\to\infty}1=1.
\]
\item[(iii)]
Für die Folge $x_n=\frac{2}{\pi(4n+3)}$ ist ebenfalls $\lim_{n\to \infty} x_n=0=x^*$.
Es gilt aber 
\[\]
Der Grenzwert der  Funktionswertfolge $(f(x_n))_{n\in\N}$ ist jetzt aber
\[ =\lim_{n\to \infty} \sin\left(\frac{1}{x_n}\right)=
\lim_{n\to\infty}\sin\left(\frac{\pi}{2}+(2n+1)\pi\right)
=\lim_{n\to\infty}-1=-1.
\]
\end{enumerate}
Es existiert also kein Funktionengrenzwert für $x\to 0$, denn die angegebenen Folgen zeigen, 
dass die Grenzwerte von Folgen aus Funktionswerten davon abhängt, wie die Folgen sich $x^*=0$ annähern.\\
So ist in obiger Definition "\notion{für jede Folge}" zu verstehen: Ein Funktionengrenzwert kann nur existieren,
wenn er \notion{unabhängig von der gewählten Folge} ist.

Der Funktionsgraph macht anschaulich, dass wir in diesem Beispiel sogar Folgen mit Grenzwert $x^\ast=0$ wählen
könnten, deren Funktionswerte konstant einer beliebigen Zahl zwischen $-1$ und $1$ sind.
\begin{center}
\image{T210_Sine1x}
\end{center}

\tab{$\frac{x^2-4}{x-2}$ für $x\to 2$}
Ähnlich zu Beispiel \ref{ex:erstes_bsp_Grenzwert_von_Funktionswerten}(b)
betrachten wir die Funktion 
\[ f:\R\setminus { \{2\} }\to \R, x\mapsto \frac{x^2-4}{x-2} \]
und $x^*=2\in \R\setminus D$.
Dann gilt für jede Folge $(x_n)_{n\geq 1}$ mit $x_n\neq 2$ und
$\lim_{n\to \infty} x_n=2$:
\[ \lim_{n\to \infty} f(x_n)=\lim_{n\to \infty} \frac{x_n^2-4}{x_n-2}
=\lim_{n\to \infty} \frac{(x_n-2)(x_n+2)}{x_n-2}
=\lim_{n\to \infty} x_n+2 =2+2=4.\]
Also ist der Funktionsgrenzwert $\lim_{x\to 2} f(x)=4$.
%%
\tab{$\frac{x^2-4}{|x-2|}$ für $x\to 2$}
Wir betrachten die Funktion 
\[ f:\R\setminus { \{2\} }\to \R, x\mapsto \frac{x^2-4}{|x-2|} \]
und $x^*=2\in \R\setminus D$.

Ist $(x_n)_{n\in\N}$ eine Folge mit $x_n> 2$ und
$\lim_{n\to \infty} x_n=2$, so gilt:
\[ \lim_{n\to \infty} f(x_n)=\lim_{n\to \infty} \frac{x_n^2-4}{|x_n-2|}
=\lim_{n\to \infty} \frac{(x_n-2)(x_n+2)}{x_n-2}
=\lim_{n\to \infty} x_n+2 =2+2=4.\]

Ist jedoch $(x_n)_{n\in\N}$ eine Folge mit $x_n< 2$ und
$\lim_{n\to \infty} x_n=2$, so gilt:
\[ \lim_{n\to \infty} f(x_n)=\lim_{n\to \infty} \frac{x_n^2-4}{|x_n-2|}
=\lim_{n\to \infty} \frac{(x_n-2)(x_n+2)}{-(x_n-2)}
=\lim_{n\to \infty} -(x_n+2) =-(2+2)=-4.\]
Da es Folgen $(x_n)_{n\in\N}$ mit $\lim_{n\in\N} x_n=2$ gibt, bei denen die Folge $(f(x_n))_{n\in\N}$ gegen $4$ konvergiert, und
andere Folgen, bei denen die Folge $(f(x_n))_{n\in\N}$ gegen $-4$ konvergiert,
existiert der Grenzwert $\lim_{x\to 2} f(x)$ also nicht. 
Denn für die Existenz des Grenzwerts $\lim_{x\to 2} f(x)$ müssen \textbf{alle} solche
Folgen konvergieren und sogar gegen den \textbf{gleichen} Grenzwert.
\tab{$x^2$ für $x\to 2$}
Wir betrachten die Funktion $f:\R\to\R$, $x\mapsto x^2$ und den Punkt $x^\ast=2$.
Für eine beliebige Folge $(x_n)_{n\in\N}$ mit $\lim_{n\to\infty}x_n=2$ gilt nach den 
\ref[content_14_konvergenz][Grenzwertregeln]{sec:grenzwertregeln} 
\[
\lim_{n\to\infty}f(x_n)=\lim_{n\to\infty}x_n^2 = \left(\lim_{n\to\infty}x_n\right)^2=2^2=4.
\]
Insbesondere gilt das auch für alle Folgen, die den Wert $2$ nie annehmen. 
Somit existiert der Grenzwert von f gegen $x^\ast=2$ und ist gleich vier, also $\lim_{x\to 2}x^2=4$. 
\end{tabs*}
\end{example}


Das nun folgende Video behandelt Grenzwerte von Funktionen und macht den Inhalt anhand vieler unterschiedlicher
Beispiele greifbar:
\floatright{\href{https://api.stream24.net/vod/getVideo.php?id=10962-2-10899&mode=iframe&speed=true}{\image[75]{00_video_button_schwarz-blau}}}\\

\section{Einseitige Grenzwerte}\label{sec:einseitige-grenzwerte}
Oft hängt das Verhalten gegen ein $x^\ast$  einer reellen Funktion nur davon ab,
ob man sich dem kritischen Punkt $x^\ast$ von links oder rechts annähert. 
Der vierte Tab von Beispiel \ref{ex:Funktionsgrenzwerte} ist so ein Beispiel.
Deshalb ist folgende Definition hilfreich.
%Wie schon bei der Stetigkeit, kann man auch bei den Grenzwerten nur links- oder rechtsseitige Grenzwerte betrachten.
\begin{definition}
\begin{enumerate}
\item[(i)]
Sei $f:D\to \R$ eine reelle Funktion und sei $x^*\in\R$ der Grenzwert 
  von mindestens einer Folge $(x_n)_{n\geq 1}$ in $D$ mit $\lim_{n\to\infty}x_n=x^\ast$ und $x_n<x^*$ für alle $n\in \N$.
  \\
  Die Funktion $f$ heißt \notion{linksseitig konvergent gegen $L \in \R$ für $x \to x^*$}\\ (wir sagen auch: 
  $f$ \notion{konvergiert von links gegen $L$ für  $x \to x^*$}), wenn gilt
  \[ \lim_{n\to \infty} f(x_n)= L \]
  für \notion{jede} Folge $(x_n)_{n\geq 1}$ in $D$ mit$\lim_{n\to \infty} x_n=x^*$ und  $x_n<x^*$ für alle $n\in\N$ .
\\
Wir schreiben dann kurz
\[ \lim_{x\nearrow x^*} f(x)= L, \]
und nennen $L$ den \notion{linksseitigen Grenzwert von $f$ für $x$ gegen $x^\ast$.}
\item[(ii)]
Sei $f:D\to \R$ eine reelle Funktion und sei $x^*\in\R$ der Grenzwert 
  von mindestens einer Folge $(x_n)_{n\geq 1}$ in $D$ mit $\lim_{n\to\infty}x_n=x^\ast$ und $x_n>x^*$ für alle $n\in\N$.
\\
  Die Funktion $f$ heißt \notion{rechtsseitig konvergent gegen $L \in \R$ für $x \to x^*$}\\ (wir sagen auch: 
  $f$ \notion{konvergiert von rechts gegen $L$ für  $x \to x^*$}), wenn
  \[ \lim_{n\to \infty} f(x_n)= L \]
  für jede Folge $(x_n)_{n\geq 1}$ in $D$ mit $x_n>x^*$ für alle $n\in\N$ und $\lim_{n\to \infty} x_n=x^*$.
\\
Wir schreiben dann kurz
\[ \lim_{x\searrow x^*} f(x)= L,\]
und nennen $L$ den \notion{rechtsseitigen Grenzwert von $f$ gegen $x^\ast$.}
\end{enumerate}
\end{definition}



\begin{remark}
In der Literatur werden oft auch die folgenden Notationen verwendet.
\begin{enumerate}
\item Für den linksseitigen Grenzwert:
\[\lim_{x \nearrow x^{*}} f(x)=\lim_{x \uparrow x^{*}} f(x)=\lim_{x \to x^{*-}} f(x)
\]
\item Für den rechtsseitigen Grenzwert:
\[\lim_{x \searrow x^{*}} f(x)=\lim_{x \downarrow x^{*}} f(x)=\lim_{x \to x^{*+}} f(x)
\]
\end{enumerate}
\end{remark}

\begin{example}
Sei $f:\R\setminus \{0; 2\}\to \R$ gegeben durch
\[ f(x)= \frac{|x-2|}{x^2-2x} \]
für alle $x\in \R\setminus \{0; 2\}$.

Betrachtet man die Stelle $x^*=2$, so gilt für alle $x<x^*$:
\[ f(x)= \frac{|x-2|}{x^2-2x} =\frac{-(x-2)}{(x-2)x}
=\frac{-1}{x},\]
und daher
\[ \lim_{x\nearrow 2} f(x)=\lim_{x\nearrow 2} \frac{-1}{x}=- \frac{1}{2}. \]
Für alle $x>x^*$ gilt hingegen:
\[ f(x)= \frac{|x-2|}{x^2-2x} =\frac{x-2}{(x-2)x}
=\frac{1}{x},\]
und daher
\[ \lim_{x\searrow 2} f(x)=\lim_{x\searrow 2} \frac{1}{x}=\frac{1}{2}. \]
Es existieren also sowohl der linksseitige Grenzwert als auch der rechtsseitige Grenzwert. 
Da die beiden jedoch verschieden sind, existiert der Grenzwert $\lim_{x\to 2} f(x)$ nicht.
In der folgenden Visualisierung ist der Funktionsgraph blau skizziert.

\begin{genericJSXVisualization}[550][800]{seitigerGW}
		
		\begin{variables}
			\function{f}{real}{|x-2|/(x^2-2*x)}%+sqrt(x)-x/sqrt(x)}   % +-sqrt(x) um auf R_+ einzuschränken.
			\point{m1}{real}{2, -0.5}
			\point{m2}{real}{2, 0.5}
			
		\end{variables}
		\color{m1}{#0066CC}
		\color{m2}{#0066CC}

		\begin{canvas}
			\updateOnDrag[false]
			\plotLeft{0.7}
			\plotRight{5.3}
			\plotSize{400,300}
			\plot[coordinateSystem]{m1, m2,f}
  		\end{canvas}
   	\end{genericJSXVisualization}
	
\end{example}

\begin{quickcheck}
    \field{real}
        \type{input.number}
            \begin{variables}
                \drawFromSet{a}{1,3,4}
                \function[calculate]{aa}{a^2}
                \function[calculate]{n}{2*a}
            \end{variables}
            \text{Berechnen Sie: $\:\lim_{x\to \var{a}}\frac{x^2-\var{aa}}{x-\var{a}}=$\ansref}
            \begin{answer}
                \solution{n}
            \end{answer}
            \explanation{Durch Anwenden der dritten binomischen Formel im Zähler
            kann gekürzt werden: $\frac{x^2-\var{aa}}{x-\var{a}}=
            \frac{(x+\var{a})(x-\var{a})}{x-\var{a}}=x+\var{a}$. Nun kann der
            Grenzwert leicht berechnet werden: $\lim_{x\to \var{a}}x+\var{a}=\var{n}$.}

\end{quickcheck}

Folgendes Video erläutert links- und rechtsseitige Grenzwerte sowie Schlussfolgerungen daraus für
Funktionsgrenzwerte:
\floatright{\href{https://api.stream24.net/vod/getVideo.php?id=10962-2-10901&mode=iframe&speed=true}{\image[75]{00_video_button_schwarz-blau}}}\\


\section{Grenzwertregeln}\label{sec:grenzwertregeln}

Die \ref[content_14_konvergenz][Grenzwertregeln]{sec:grenzwertregeln}  für Folgen
lassen sich direkt auf die Grenzwerte von Funktionen übertragen.

\begin{rule}\label{rule:summen_und_prod_stetiger_fkt}
Seien $f:D\to \R$ und $g:D\to \R$ reelle Funktionen, und $x^*\in \R$  (oder $x^\ast=\pm\infty$) so, dass die Grenzwerte von $f$ und $g$ für $x$ gegen $x^*$ existieren.
Sei weiter $r\in \R$.

Dann existieren auch die entsprechenden Grenzwerte für die Funktionen $f+g$, $f-g$, $f\cdot g$ und $rf$ und es gelten
\begin{eqnarray*}
\lim_{x\to x^*} (f+g)(x) &=& \lim_{x\to x^*} f(x)+\lim_{x\to x^*}g(x) \\
\lim_{x\to x^*} (f-g)(x) &=&\lim_{x\to x^*} f(x)-\lim_{x\to x^*}g(x)\\
\lim_{x\to x^*} (f\cdot g)(x) &=& \lim_{x\to x^*} f(x) \cdot \lim_{x\to x^*} g(x) \\
\lim_{x\to x^*} (rf)(x) &=& r\cdot \lim_{x\to x^*} f(x)
\end{eqnarray*}

Ist zudem der Grenzwert von $g$ für $x$ gegen $x^*$ von $0$ verschieden, dann existiert auch der entsprechende Grenzwert von $\frac{f}{g}$ und es ist
\[ \lim_{x\to x^*} \frac{f}{g}(x) = \frac{\lim_{x\to x^*}f(x)}{\lim_{x\to x^*}g(x)}.\]

Entsprechendes erhält man für die einseitigen Grenzwerte der zusammengesetzten Funktionen, wenn für $f$ und $g$ jeweils der linksseitige bzw. rechtsseitige Grenzwert existiert.
\end{rule}
\begin{example}
Für die Funktion $f:\R\to\R$, $x\mapsto\frac{x^2+2x+1}{4x^2+1}$ und die Stelle $x^\ast=1$ bemerken wir, dass
sowohl das Zähler- als auch das Nennerpolynom nach der Regel \ref{rule:summen_und_prod_stetiger_fkt} Grenzwerte
für $x\to 1$ besitzen,
\[\lim_{x\to 1}(x^2+2x+1)=1+2+1=4\quad \text{ sowie }\quad \lim_{x\to 1}(4x^2+1)=4+1=5.\]
Somit ist, wiederum nach Regel \ref{rule:summen_und_prod_stetiger_fkt},
\[
\lim_{x\to 1}\frac{x^2+2x+1}{4x^2+1}=\frac{\lim_{x\to 1}(x^2+2x+1)}{\lim_{x\to 1}(4x^2+1)}=\frac{4}{5}.
\]
\end{example}

% Bei der Stetigkeit von Funktionen konvergiert die Folge von Funktionswerten $f(x_n)$ gegen $f(x^*)$, wenn die Folge $x_n$ gegen
% den Wert $x^*$ konvergiert. Ist die Funktion bei $x^*$ definiert und stetig, dann wird der Grenzwert $\lim_{x\to x^*} f(x)$ gleich $f(x^*)$ sein. 
% Die Definition von Stetigkeit verwendet also bereits Funktionengrenzwerte, wobei der Grenzwert der Funktionswert an der
% betrachteten Stelle $x^*$ ist.\\ 

% In diesem Abschnitt geht es darum, allgemeiner einen Grenzwert von Funktionswerten $\lim_{x\to x^*} f(x)$ zu definieren, 
% auch (oder gerade dann) wenn $f$ bei $x^*$ nicht definiert ist. Wir erlauben jetzt beliebige Grenzwerte.\\


% Die gebrochen rationale Funktion $f(x)=\frac{x^2-1}{x-1}=\frac {(x-1)(x+1)}{x-1}$ hat bei $x=1$ eine Definitionslücke, 
% für $x\neq 1$ gilt $f(x)=x+1$. Die Kurve $f(x)$ ist also die bei $x=1$ unterbrochene Gerade $x+1$; $x$ darf zwar keinesfallls den Wert 1 annehmen,
% wohl aber kann man $x$ beliebig nahe an $x^*=1$ heranrücken lassen: $\lim_{x\to 1}f(x)=1+1=2$, 
% der Grenzwert der Funktion $f(x)$ an der Stelle $x^*=1$ ist also 2.\\

% Die Funktion $f(x)=\frac{1}{x}$ hat bei $x=0$ einen Pol, sie ist dort nicht erklärt. Es ist zwar verboten, $x=0$ einzusetzen, 
% aber keineswegs, $x$ in beliebiger Nähe der kritischen Stelle 0 anzunehmen. Die Funktion $f(x)=\frac{1}{x}$ hat für $x>0$ positive Funktionswerte, für $x<0$ negative
% Funktionswerte. In diesem Fall spielt es also eine Rolle, wie wir uns 0 nähern: kommen wir von "rechts" ($\searrow$), 
% also von positiven $x$-Werten oder von "links" ($\nearrow$), den negativen Funktionswerten. Wir unterscheiden also rechtsseitige Annäherung
% bzw. linksseitige Annäherung.

% Um die mathematische Korrektheit zu gewährleisten, gehen wir genauso wie bei der Stetigkeit vor:

% 1. Wir betrachten Folgen $(x_n)_{n\geq1}$, die gegen die kritische Stelle $x^*$
% konvergieren und untersuchen, wie sich die Folge der Funktionswerte $f(x_n)$ verhält:
% was ist $\lim_{n\to \infty}f(x_n)$?

% 2. Wir übernehmen die $\epsilon-\delta-$Methode und überprüfen, ob sich Toleranzen in Funktionswerten
% durch Toleranzen in den zugehörigen x-Werten widerspiegeln lassen (siehe auch \ref[content_29_stetigkeit_definitionen][Stetigkeit]{sec:eps-delta}).


% \section{Grenzwert von Funktionswerten gegen feste Stellen}

% \begin{definition}\label{def:funktionsgrenzwert}
% Sei $f:D\to \R$ eine reelle Funktion und sei $x^*\in\R$ der Grenzwert 
%   von mindestens einer Folge $(x_n)_{n\geq 1}$ mit $x_n\in D\setminus \{x^*\}$.
  
%   Die Funktion $f$ heißt \notion{konvergent gegen $L \in \R$ für $x \to x^*$}, wenn
%   \[ \lim_{n\to \infty} f(x_n)= L \]
%   für jede Folge $(x_n)_{n\geq 1}$ mit $x_n\in D\setminus \{x^*\}$ und $\lim_{n\to \infty} x_n=x^*$.

% Wir schreiben dann kurz:
% \[ \lim_{x\to x^*} f(x)= L. \]
% \floatright{\href{https://www.hm-kompakt.de/video?watch=400}{\image[75]{00_Videobutton_schwarz}}}\\\\
% \end{definition}
 
% \begin{example}
% Wir betrachten die Funktion \[ f:\R\setminus { \{0\} }\mapsto [-1;1], x\mapsto \sin\left(\frac{\pi}{x}\right) \]
% und $x^*=0$.\\
% \begin{enumerate}
% \item
% Für die Folge $x_n=\frac{1}{n}$ ist $\lim_{n\to \infty} x_n=0=x^*$.
% Der Funktionengrenzwert ist:
% \[ \lim_{n\to \infty} f(x_n)=\lim_{n\to \infty} \sin\left(\frac{\pi}{x_n}\right)=\lim_{n\to \infty} \sin(n\pi)=0. \]
% \item
% Für die Folge $x_n=\frac{2}{2n+1}$ ist ebenfalls $\lim_{n\to \infty} x_n=0=x^*$.
% Der Funktionengrenzwert ist jetzt aber:
% \[ \lim_{n\to \infty} f(x_n)=\lim_{n\to \infty} \sin\left(\frac{\pi}{x_n}\right)=\lim_{n\to\infty}\sin\left(\frac{2n+1}{2}\cdot\pi\right)
% =\lim_{n\to\infty}\sin\left(\left(n+\frac{1}{2}\right)\cdot\pi\right)=
% \begin{cases}
% +1&\text{für n gerade}\\
% -1&\text{für n ungerade}\\
% \end{cases}.
% \]
% \end{enumerate}
% Es existiert kein Funktionengrenzwert für $x\to 0$, denn der Funktionengrenzwert 
% darf natürlich nicht davon abhängen mit welcher Folge (wie unter 1. oder 2.) wir uns $x^*=0$ nähern.
% So ist in obiger Definition "\notion{für jede Folge}" zu verstehen: ein Funktionengrenzwert kann nur existieren,
% wenn er \notion{unabhängig von der gewählten Folge} ist.
% \end{example}

% \begin{remark}
% \begin{enumerate}
% \item
% Man beachte, dass $f$ in $x^*$ gar nicht definiert sein muss (dies ist sogar der übliche Fall). Selbst wenn $f$ in $x^*$ definiert ist, ist für den Grenzwert von $f$ gegen $x^*$ der Funktionswert $f(x^*)$ unerheblich, weil bei den Folgen stets $x_n\neq x^*$ vorausgesetzt ist. 
% \item Wie auch bei der Stetigkeit, kann man den Funktionsgrenzwert auch mit einem $\epsilon$-$\delta$-Kriterium definieren:

%   Die Funktion $f$ ist konvergent gegen $L \in \R$ für $x \to x^*$, wenn es f"ur jedes $\epsilon >0$ ein $\delta >0$ gibt, so dass 
%   \[ |f(x) - L| < \epsilon \quad \text{f"ur alle }
%   x\in D\text{ mit } 0<|x-x^*|<\delta.
%   \]
%   Dies ist eine gleichwertige Alternative, falls es z.B. keine Folge mit $\lim_{n\to \infty} x_n=x^*$ gibt.
% \end{enumerate}
% \end{remark}


% \begin{example}
% \begin{enumerate}
% \item
% Wir betrachten die Funktion 
% \[ f:\R\setminus { \{2\} }\to \R, x\mapsto \frac{x^2-4}{x-2} \]
% und $x^*=2\in \R\setminus D$.

% Dann gilt für jede Folge $(x_n)_{n\geq 1}$ mit $x_n\neq 2$ und
% $\lim_{n\to \infty} x_n=2$:
% \[ \lim_{n\to \infty} f(x_n)=\lim_{n\to \infty} \frac{x_n^2-4}{x_n-2}
% =\lim_{n\to \infty} \frac{(x_n-2)(x_n+2)}{x_n-2}
% =\lim_{n\to \infty} x_n+2 =2+2=4.\]

% Also ist $\lim_{x\to 2} f(x)=4$.
% \item Wir betrachten die Funktion 
% \[ f:\R\setminus { \{2\} }\to \R, x\mapsto \frac{x^2-4}{|x-2|} \]
% und $x^*=2\in \R\setminus D$.

% Ist $(x_n)_{n\geq 1}$ eine Folge mit $x_n> 2$ und
% $\lim_{n\to \infty} x_n=2$, so gilt:
% \[ \lim_{n\to \infty} f(x_n)=\lim_{n\to \infty} \frac{x_n^2-4}{|x_n-2|}
% =\lim_{n\to \infty} \frac{(x_n-2)(x_n+2)}{x_n-2}
% =\lim_{n\to \infty} x_n+2 =2+2=4.\]

% Ist jedoch $(x_n)_{n\geq 1}$ eine Folge mit $x_n< 2$ und
% $\lim_{n\to \infty} x_n=2$, so gilt:
% \[ \lim_{n\to \infty} f(x_n)=\lim_{n\to \infty} \frac{x_n^2-4}{|x_n-2|}
% =\lim_{n\to \infty} \frac{(x_n-2)(x_n+2)}{-(x_n-2)}
% =\lim_{n\to \infty} -(x_n+2) =-(2+2)=-4.\]
% Da es Folgen $(x_n)_{n\geq 1}$ mit $\lim_{n\to \infty} x_n=2$ gibt, bei denen die Folge $(f(x_n))_{n\geq 1}$ gegen $4$ konvergiert, und
% andere Folgen, bei denen die Folge $(f(x_n))_{n\geq 1}$ gegen $-4$ konvergiert,
% existiert der Grenzwert $\lim_{x\to 2} f(x)$ also nicht. Denn für die Existenz des Grenzwerts $\lim_{x\to 2} f(x)$ müssen \textbf{alle} solche
% Folgen konvergieren und sogar gegen den \textbf{gleichen} Grenzwert.
% \end{enumerate}
% \end{example}

% Direkt unmittelbar aus der Definition der Stetigkeit und des Funktionengrenzwerts ist die folgende Aussage.

% \begin{theorem}
% Sei $f:D\to \R$ eine reelle Funktion und $x^*\in D$ der Grenzwert 
%   von mindestens einer Folge $(x_n)_{n\geq 1}$ mit $x_n\in D\setminus \{x^*\}$. Dann ist $f$ genau dann stetig in $x^*$, wenn 
%   \[  \lim_{x\to x^*} f(x) =f(x^*). \]
% \end{theorem}

% \begin{example}
% Wie wir in Beispiel 3.2 gesehen haben, existiert für die Funktion 
% $f(x)=\sin\left(\frac{\pi}{x}\right)$ kein Grenzwert an $x^*=0$. Dies gilt natürlich ebenso für
% die Funktion $f(x)=\sin\left(\frac{1}{x}\right)$. (Die Folge wäre dann $x_n=\frac{1}{\pi\cdot n}$.)
% Die folgende Abbildung stellt $f(x)=\sin\left(\frac{1}{x}\right)$ grafisch dar. 

% \begin{center}
% \image{T210_Sine1x}
% \end{center}


% Wie verhält sich $f(x)=\sin(\frac{1}{x})$ für $x^*=0$? Wir setzen $x'=\frac{1}{x}$.
% Zwischen $x'$ und $x'+2\pi$ liegt genau eine volle Schwingung von $\sin(x')$. Nun entspricht aber dem Zuwachs $2\pi$
% der Funktionswerte (Ordinate) der Hyperbel $x'=\frac{1}{x}$ eine Abnahme der $x$-Werte (Abszisse), die zugleich
% mit $x$ selbst gegen 0 konvergiert, und das heißt, daß eine volle Schwingung von $\sin(\frac{1}{x})$
% in der Nähe von $x=0$ auf einen beliebig schmalen Abszissenbereich zusammengedrängt wird. $f(x)$ nimmt in jeder noch so 
% kleinen Umgebung des Nullpunktes jeden Wert zwischen $-1$ und $+1$ an und kann sich daher
% keinem Grenzwert nähern.

% Unser Interesse soll jetzt der Funktion $f(x)=x\cdot \sin(\frac{1}{x})$ gelten mit $x^*=0$.
% Diese Funktion besitzt den Grenzwert $\lim_{x \to  0}f(x)=0$. 

% Wir betrachten \[\vert f(x)-0\vert=\vert x\cdot \sin\left(\frac{1}{x}\right)\vert=\vert x\vert\cdot \vert\sin\left(\frac{1}{x}\right)\vert\leq\vert x\vert\cdot 1=\vert x-0\vert \]

% Wir wählen $\delta=\epsilon$. Dann gilt für $\vert x-0\vert<\delta=\epsilon\Rightarrow\vert f(x)-0\vert<\epsilon$.
% Der Grenzwert ist damit bewiesen. Die Multiplikation mit $x$ hat die Funktion $f(x)=\sin\left(\frac{1}{x}\right)$ durch den 
% Nullpunkt gezwungen. Dies zeigt die folgende Abbildung.

% \begin{center}
% \image{T210_xSine1x}
% \end{center}

% \end{example}

% Das nun folgende Video behandelt Grenzwerte von Funktionen und macht den Inhalt anhand vieler unterschiedlicher
% Beispiele greifbar:
% \floatright{\href{https://api.stream24.net/vod/getVideo.php?id=10962-2-10899&mode=iframe&speed=true}{\image[75]{00_video_button_schwarz-blau}}}\\

% \section{Einseitige Grenzwerte}\label{sec:einseitige-grenzwerte}

% Wie schon bei der Stetigkeit, kann man auch bei den Grenzwerten nur links- oder rechtsseitige Grenzwerte betrachten.

% \begin{definition}
% \begin{enumerate}
% \item
% Sei $f:D\to \R$ eine reelle Funktion und sei $x^*\in\R$ der Grenzwert 
%   von mindestens einer Folge $(x_n)_{n\geq 1}$ mit $x_n\in D$ und $x_n<x^*$.
  
%   Die Funktion $f$ heißt \notion{linksseitig konvergent gegen $L \in \R$ für $x \to x^*$}\\ (wir sagen auch: 
%   $f$ \notion{konvergiert von links gegen $L$ für  $x \to x^*$}), wenn
%   \[ \lim_{n\to \infty} f(x_n)= L \]
%   für jede Folge $(x_n)_{n\geq 1}$ mit $x_n\in D$, $x_n<x^*$ und $\lim_{n\to \infty} x_n=x^*$.

% Wir schreiben dann kurz:
% \[ \lim_{x\nearrow x^*} f(x)= L. \]

% \item
% Sei $f:D\to \R$ eine reelle Funktion und sei $x^*\in\R$ der Grenzwert 
%   von mindestens einer Folge $(x_n)_{n\geq 1}$ mit $x_n\in D$ und $x_n>x^*$.

%   Die Funktion $f$ heißt \notion{rechtsseitig konvergent gegen $L \in \R$ für $x \to x^*$}\\ (wir sagen auch: 
%   $f$ \notion{konvergiert von rechts gegen $L$ für  $x \to x^*$}), wenn
%   \[ \lim_{n\to \infty} f(x_n)= L \]
%   für jede Folge $(x_n)_{n\geq 1}$ mit $x_n\in D$, $x_n>x^*$ und $\lim_{n\to \infty} x_n=x^*$.

% Wir schreiben dann kurz:
% \[ \lim_{x\searrow x^*} f(x)= L. \]
% \end{enumerate}
% \end{definition}

% \begin{remark}
% \begin{enumerate}
% \item Wie auch schon bei der Stetigkeit gilt:
% Gibt es sowohl gegen $x^*$ konvergierende Folgen in $D$ mit $x_n>x^*$ für alle $n\in \N$ und welche mit $x_n<x^*$ für alle $n\in \N$, 
% so ist genau dann $\lim_{x\to x^*} f(x)=L$, wenn 
% \[ \lim_{x\nearrow x^*} f(x)=L\quad\text{und} \quad 
% \lim_{x\searrow x^*} f(x)=L. \]
% \item Ist zusätzlich $x^*\in D$, so ist $f$ genau dann in $x^*$ linksseitig stetig, wenn $\lim_{x\nearrow x^*} f(x)=f(x^*)$,\\
% und $f$ ist genau dann in $x^*$ rechtsseitig stetig, wenn $\lim_{x\searrow x^*} f(x)=f(x^*)$.
% \end{enumerate}
% \end{remark}

% \begin{remark}
% In der Literatur werden auch oft die folgenden Notationen verwendet.
% \begin{enumerate}
% \item Linksseitiger Grenzwert:
% \[L=\lim_{x \nearrow x^{*}} f(x)=\lim_{x \uparrow x^{*}} f(x)=\lim_{x \to x^{*-}} f(x)
% \]
% \item Rechtsseitiger Grenzwert:
% \[L=\lim_{x \searrow x^{*}} f(x)=\lim_{x \downarrow x^{*}} f(x)=\lim_{x \to x^{*+}} f(x)
% \]
% \end{enumerate}
% \end{remark}

% \begin{example}
% Sei $f:\R\setminus \{0; 2\}\to \R$ gegeben durch
% \[ f(x)= \frac{|x-2|}{x^2-2x} \]
% für alle $x\in \R\setminus \{0; 2\}$.

% Betrachtet man die Stelle $x^*=2$, so gilt für alle $x<x^*$:
% \[ f(x)= \frac{|x-2|}{x^2-2x} =\frac{-(x-2)}{(x-2)x}
% =\frac{-1}{x},\]
% und daher
% \[ \lim_{x\nearrow 2} f(x)=\lim_{x\nearrow 2} \frac{-1}{x}=- \frac{1}{2}. \]
% Für alle $x>x^*$ gilt hingegen:
% \[ f(x)= \frac{|x-2|}{x^2-2x} =\frac{x-2}{(x-2)x}
% =\frac{1}{x},\]
% und daher
% \[ \lim_{x\searrow 2} f(x)=\lim_{x\searrow 2} \frac{1}{x}=\frac{1}{2}. \]
% Es existieren also sowohl der linksseitige Grenzwert als auch der rechtsseitige Grenzwert. 
% Da die beiden jedoch verschieden sind, existiert der Grenzwert $\lim_{x\to 2} f(x)$ nicht.

% \begin{genericGWTVisualization}[550][800]{mathlet1}
		
% 		\begin{variables}
% 			\function{f}{real}{|x-2|/(x^2-2*x)+sqrt(x)-x/sqrt(x)}   % +-sqrt(x) um auf R_+ einzuschränken.
% 			\point{m1}{real}{2, -0.5}
% 			\point{m2}{real}{2, 0.5}
			
% 		\end{variables}
% 		\color{m1}{#0066CC}
% 		\color{m2}{#0066CC}

% 		\begin{canvas}
% 			\updateOnDrag[false]
% 			\plotLeft{0.7}
% 			\plotRight{5.3}
% 			\plotSize{400,300}
% 			\plot[coordinateSystem]{m1, m2,f}
% 		\end{canvas}
% 		\text{Die schwarze Kurve ist der Graph der Funktion $f:\R_+\setminus \{2\}\to \R$ mit $f(x)=\frac{|x-2|}{x^2-2x}$. 
% 		Nähert man sich auf dem Graphen von rechts an die Stelle $x^*=2$, so gelangt man zum Punkt $(2;\frac{1}{2})$. 
% 		Für eine Folge $(x_n)_{n\geq 1}$ mit $x_n>2$ und $\lim_{n\to \infty} x_n=2$ ist also der Grenzwert
% 		$\lim_{n\to \infty} f(x_n)$ gleich $\frac{1}{2}$. Also ist $\lim_{x\searrow 2} f(x)=\frac{1}{2}$. \\
% 		Nähert man sich auf dem Graphen von links an die Stelle $x^*=2$, so gelangt man zum Punkt $(2;-\frac{1}{2})$. 
% 		Für eine Folge $(x_n)_{n\geq 1}$ mit $x_n<2$ und $\lim_{n\to \infty} x_n=2$ ist also der Grenzwert
% 		$\lim_{n\to \infty} f(x_n)$ gleich $-\frac{1}{2}$. Also ist $\lim_{x\nearrow 2} f(x)=-\frac{1}{2}$.\\
% 		Der Grenzwert $\lim_{x\to 2} f(x)$ existiert also nicht.
% 		}
%    	\end{genericGWTVisualization}

% \end{example}

% \begin{quickcheck}
%     \field{real}
%         \type{input.number}
%             \begin{variables}
%                 \drawFromSet{a}{1,2,3,4}
%                 \function[calculate]{aa}{a^2}
%                 \function[calculate]{n}{2*a}
%             \end{variables}
%             \text{Berechnen Sie:
%                 \[\lim_{x\to \var{a}}\frac{x^2-\var{aa}}{x-\var{a}}\] \\Antwort:\ansref}
%             \begin{answer}
%                 \solution{n}
%             \end{answer}
%             \explanation{Durch Anwenden der dritten binomischen Formel im Zähler
%             kann gekürzt werden: $\frac{x^2-\var{aa}}{x-\var{a}}=
%             \frac{(x+\var{a})(x-\var{a})}{x-\var{a}}=x+\var{a}$. Nun kann der
%             Grenzwert leicht berechnet werden: $\lim_{x\to \var{a}}x+\var{a}=\var{n}$.}

% \end{quickcheck}

% Folgendes Video erläutert links- und rechtsseitige Grenzwerte sowie Schlussfolgerungen daraus für
% Funktionsgrenzwerte:
% \floatright{\href{https://api.stream24.net/vod/getVideo.php?id=10962-2-10901&mode=iframe&speed=true}{\image[75]{00_video_button_schwarz-blau}}}\\


% \section{Grenzwertregeln}\label{sec:grenzwertregeln}

% Die \ref[konv-krit][Grenzwertregeln]{sec:grenzwertregeln} für Folgen
% lassen sich direkt auf die Grenzwerte von Funktionen übertragen.

% \begin{rule}\label{rule:summen_und_prod_stetiger_fkt}
% Seien $f:D\to \R$ und $g:D\to \R$ reelle Funktionen, und $x^*\in \R$ so, dass die Grenzwerte von $f$ und $g$ für $x$ gegen $x^*$ existieren.
% Sei weiter $r\in \R$.

% Dann existieren auch die entsprechenden Grenzwerte für die Funktionen $f+g$, $f-g$, $f\cdot g$ und $rf$ und es gelten
% \begin{eqnarray*}
% \lim_{x\to x^*} (f+g)(x) &=& \lim_{x\to x^*} f(x)+\lim_{x\to x^*}g(x) \\
% \lim_{x\to x^*} (f-g)(x) &=&\lim_{x\to x^*} f(x)-\lim_{x\to x^*}g(x)\\
% \lim_{x\to x^*} (f\cdot g)(x) &=& \lim_{x\to x^*} f(x) \cdot \lim_{x\to x^*} g(x) \\
% \lim_{x\to x^*} (rf)(x) &=& r\cdot \lim_{x\to x^*} f(x)
% \end{eqnarray*}

% Ist zudem der Grenzwert von $g$ für $x$ gegen $x^*$ von $0$ verschieden, dann existiert auch der entsprechende Grenzwert von $\frac{f}{g}$ und es ist
% \[ \lim_{x\to x^*} \frac{f}{g}(x) = \frac{\lim_{x\to x^*}f(x)}{\lim_{x\to x^*}g(x)}.\]

% Entsprechendes erhält man für die einseitigen Grenzwerte der zusammengesetzten Funktionen, wenn für $f$ und $g$ jeweils der linksseitige bzw. rechtsseitige Grenzwert existiert.
% \end{rule}



\end{visualizationwrapper}
\end{content}
