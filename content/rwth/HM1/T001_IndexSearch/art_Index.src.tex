%$Id:  $
\documentclass{mumie.article}
%$Id$
\begin{metainfo}
  \name{
  \lang{en}{Index Search}
  \lang{de}{Index Suche}
  }
  \begin{description} 
 This work is licensed under the Creative Commons License Attribution 4.0 International (CC-BY 4.0)   
 https://creativecommons.org/licenses/by/4.0/legalcode 

    \lang{en}{...}
    \lang{de}{...}
  \end{description}
  \begin{components}
  \end{components}
  \begin{links}
    \link{generic_article}{content/rwth/HM1/T503_Differentialgleichungen/g_art_content_57_gewoehnliche_DGL_erster_Ordnung.meta.xml}{content_57_gewoehnliche_DGL_erster_Ordnung}
    \link{generic_article}{content/rwth/HM1/T503_Differentialgleichungen/g_art_content_56_Separierbare_Differentialgleichungen.meta.xml}{content_56_Separierbare_Differentialgleichungen}
    \link{generic_article}{content/rwth/HM1/T502_Stetigkeit_Diffbarkeit-n-dim/g_art_content_55_Lokale_Umkehrbarkeit.meta.xml}{content_55_Lokale_Umkehrbarkeit}
    \link{generic_article}{content/rwth/HM1/T502_Stetigkeit_Diffbarkeit-n-dim/g_art_content_54_Differentiation.meta.xml}{content_54_Differentiation}
    \link{generic_article}{content/rwth/HM1/T502_Stetigkeit_Diffbarkeit-n-dim/g_art_content_53_Stetigkeit.meta.xml}{content_53_Stetigkeit}
    \link{generic_article}{content/rwth/HM1/T501_Orientierung_im_n-dim_Raum/g_art_content_52_Abstaende.meta.xml}{content_52_Abstaende}
    \link{generic_article}{content/rwth/HM1/T501_Orientierung_im_n-dim_Raum/g_art_content_51_Kurven.meta.xml}{content_51_Kurven}
  
% Links für Teil 1   
    \link{generic_article}{content/rwth/HM1/T101neu_Elementare_Rechengrundlagen/g_art_content_01_zahlenmengen.meta.xml}{content_01_zahlenmengen}
    \link{generic_article}{content/rwth/HM1/T101neu_Elementare_Rechengrundlagen/g_art_content_02_rechengrundlagen_terme.meta.xml}{content_02_rechengrundlagen_terme}
    \link{generic_article}{content/rwth/HM1/T101neu_Elementare_Rechengrundlagen/g_art_content_03_bruchrechnung.meta.xml}{content_03_bruchrechnung}
    \link{generic_article}{content/rwth/HM1/T101neu_Elementare_Rechengrundlagen/g_art_content_04_aussagen_aequivalenzumformungen.meta.xml}{content_04_aussagen_aequivalenzumformungen}
    \link{generic_article}{content/rwth/HM1/T101neu_Elementare_Rechengrundlagen/g_art_content_05_loesen_gleichungen_und_lgs.meta.xml}{content_05_loesen_gleichungen_und_lgs}
%    
    \link{generic_article}{content/rwth/HM1/T102neu_Einfache_Reelle_Funktionen/g_art_content_06_funktionsbegriff_und_lineare_funktionen.meta.xml}{content_06_funktionsbegriff_und_lineare_funktionen}    
    \link{generic_article}{content/rwth/HM1/T102neu_Einfache_Reelle_Funktionen/g_art_content_07_geradenformen.meta.xml}{content_07_geradenformen}
    \link{generic_article}{content/rwth/HM1/T102neu_Einfache_Reelle_Funktionen/g_art_content_08_quadratische_funktionen.meta.xml}{content_08_quadratische_funktionen}
%    
    \link{generic_article}{content/rwth/HM1/T103_Polynomfunktionen/g_art_content_09_polynome.meta.xml}{content_09_polynome}
    \link{generic_article}{content/rwth/HM1/T103_Polynomfunktionen/g_art_content_10_polynomdivision.meta.xml}{content_10_polynomdivision}
%    
    \link{generic_article}{content/rwth/HM1/T104_weitere_elementare_Funktionen/g_art_content_13_wurzelfunktionen.meta.xml}{content_13_wurzelfunktionen}
    \link{generic_article}{content/rwth/HM1/T104_weitere_elementare_Funktionen/g_art_content_14_potenzregeln.meta.xml}{content_14_potenzregeln}   
    \link{generic_article}{content/rwth/HM1/T104_weitere_elementare_Funktionen/g_art_content_15_exponentialfunktionen.meta.xml}{content_15_exponentialfunktionen}
    \link{generic_article}{content/rwth/HM1/T104_weitere_elementare_Funktionen/g_art_content_16_logarithmen.meta.xml}{content_16_logarithmen}
%    
    \link{generic_article}{content/rwth/HM1/T105_Trigonometrische_Funktionen/g_art_content_17_trigonometrie_im_dreieck.meta.xml}{content_17_trigonometrie_im_dreieck}
    \link{generic_article}{content/rwth/HM1/T105_Trigonometrische_Funktionen/g_art_content_18_grad_und_bogenmass.meta.xml}{content_18_grad_und_bogenmass}
    \link{generic_article}{content/rwth/HM1/T105_Trigonometrische_Funktionen/g_art_content_19_allgemeiner_sinus_cosinus.meta.xml}{content_19_allgemeiner_sinus_cosinus}
%    
    \link{generic_article}{content/rwth/HM1/T106_Differentialrechnung/g_art_content_20_ableitung_als_tangentensteigung.meta.xml}{content_20_ableitung_als_tangentensteigung}
    \link{generic_article}{content/rwth/HM1/T106_Differentialrechnung/g_art_content_21_kettenregel.meta.xml}{content_21_kettenregel}
    \link{generic_article}{content/rwth/HM1/T106_Differentialrechnung/g_art_content_22_extremstellen.meta.xml}{content_22_extremstellen}
    \link{generic_article}{content/rwth/HM1/T106_Differentialrechnung/g_art_content_23_kurvendiskussion.meta.xml}{content_23_kurvendiskussion}
%    
    \link{generic_article}{content/rwth/HM1/T107_Integralrechnung/g_art_content_24_integral_als_flaeche.meta.xml}{content_24_integral_als_flaeche}
    \link{generic_article}{content/rwth/HM1/T107_Integralrechnung/g_art_content_25_stammfunktion.meta.xml}{content_25_stammfunktion}
    \link{generic_article}{content/rwth/HM1/T107_Integralrechnung/g_art_content_26_flaechen_zwischen_graphen.meta.xml}{content_26_flaechen_zwischen_graphen}
%    
    \link{generic_article}{content/rwth/HM1/T108_Vektorrechnung/g_art_content_27_vektoren.meta.xml}{content_27_vektoren}
    \link{generic_article}{content/rwth/HM1/T108_Vektorrechnung/g_art_content_29_linearkombination.meta.xml}{content_29_linearkombination}
    \link{generic_article}{content/rwth/HM1/T108_Vektorrechnung/g_art_content_30_basen_eigenschaften.meta.xml}{content_30_basen_eigenschaften}    
%    
    \link{generic_article}{content/rwth/HM1/T109_Skalar-_und_Vektorprodukt/g_art_content_31_skalarprodukt.meta.xml}{content_31_skalarprodukt}
    \link{generic_article}{content/rwth/HM1/T109_Skalar-_und_Vektorprodukt/g_art_content_32_laenge_norm.meta.xml}{content_32_laenge_norm}
    \link{generic_article}{content/rwth/HM1/T109_Skalar-_und_Vektorprodukt/g_art_content_33_winkel.meta.xml}{content_33_winkel}
    \link{generic_article}{content/rwth/HM1/T109_Skalar-_und_Vektorprodukt/g_art_content_34_vektorprodukt.meta.xml}{content_34_vektorprodukt}
%    
    \link{generic_article}{content/rwth/HM1/T110_Geraden,_Ebenen/g_art_content_35_parameterformen.meta.xml}{content_35_parameterformen}
    \link{generic_article}{content/rwth/HM1/T110_Geraden,_Ebenen/g_art_content_36_normalenformen.meta.xml}{content_36_normalenformen}    
    \link{generic_article}{content/rwth/HM1/T110_Geraden,_Ebenen/g_art_content_37_schnittpunkte.meta.xml}{content_37_schnittpunkte}
    \link{generic_article}{content/rwth/HM1/T110_Geraden,_Ebenen/g_art_content_38_abstaende.meta.xml}{content_38_abstaende}
%    
    \link{generic_article}{content/rwth/HM1/T111neu_Matrizen/g_art_content_39_matrizen.meta.xml}{content_39_matrizen}
    \link{generic_article}{content/rwth/HM1/T111neu_Matrizen/g_art_content_39b_matrizen.meta.xml}{content_39b_matrizen}    
    \link{generic_article}{content/rwth/HM1/T112neu_Lineare_Gleichungssysteme/g_art_content_40_lineare_gleichungssysteme.meta.xml}{content_40_lineare_gleichungssysteme}
    \link{generic_article}{content/rwth/HM1/T112neu_Lineare_Gleichungssysteme/g_art_content_41_gauss_verfahren.meta.xml}{content_41_gauss_verfahren}
    \link{generic_article}{content/rwth/HM1/T111neu_Matrizen/g_art_content_42_matrixaddition.meta.xml}{content_42_matrixaddition}    
    \link{generic_article}{content/rwth/HM1/T111neu_Matrizen/g_art_content_43_matrizenmultiplikation.meta.xml}{content_43_matrizenmultiplikation}
    \link{generic_article}{content/rwth/HM1/T111neu_Matrizen/g_art_content_44_transponierte_matrix.meta.xml}{content_44_transponierte_matrix}
    \link{generic_article}{content/rwth/HM1/T112neu_Lineare_Gleichungssysteme/g_art_content_45_matrixrang.meta.xml}{content_45_matrixrang}
%
%%%%%%%%%%%%
%  Teil 2
    
    \link{generic_article}{content/rwth/HM1/T201neu_Vollstaendige_Induktion/g_art_content_01_indirekter_widerspruchsbeweis.meta.xml}{content_01_indirekter_widerspruchsbeweis}
    \link{generic_article}{content/rwth/HM1/T201neu_Vollstaendige_Induktion/g_art_content_02_vollstaendige_induktion.meta.xml}{content_02_vollstaendige_induktion}
    \link{generic_article}{content/rwth/HM1/T201neu_Vollstaendige_Induktion/g_art_content_03_binomischer_lehrsatz.meta.xml}{content_03_binomischer_lehrsatz}
    \link{generic_article}{content/rwth/HM1/T202_Reelle_Zahlen_axiomatisch/g_art_content_04_koerperaxiome.meta.xml}{content_04_koerperaxiome}
    \link{generic_article}{content/rwth/HM1/T202_Reelle_Zahlen_axiomatisch/g_art_content_05_anordnungsaxiome.meta.xml}{content_05_anordnungsaxiome}
    \link{generic_article}{content/rwth/HM1/T202_Reelle_Zahlen_axiomatisch/g_art_content_06_supremum_infimum.meta.xml}{content_06_supremum_infimum}
    \link{generic_article}{content/rwth/HM1/T202_Reelle_Zahlen_axiomatisch/g_art_content_07_vollstaendigkeit.meta.xml}{content_07_vollstaendigkeit}
%
    \link{generic_article}{content/rwth/HM1/T203_komplexe_Zahlen/g_art_content_08aneu_komplexeZahlen_intro.meta.xml}{content_08aneu_komplexeZahlen_intro} 
    \link{generic_article}{content/rwth/HM1/T203_komplexe_Zahlen/g_art_content_08bneu_komplexeZahlen_geom.meta.xml}{content_08bneu_komplexeZahlen_geom} 
    \link{generic_article}{content/rwth/HM1/T203_komplexe_Zahlen/g_art_content_09neu_komplexeZahlen_hauptsatz.meta.xml}{content_09neu_komplexeZahlen_hauptsatz} 
%
    \link{generic_article}{content/rwth/HM1/T204_Abbildungen_und_Funktionen/g_art_content_10_abbildungen_verkettung.meta.xml}{content_10_abbildungen_verkettung} 
    \link{generic_article}{content/rwth/HM1/T204_Abbildungen_und_Funktionen/g_art_content_11_injektiv_surjektiv_bijektiv.meta.xml}{content_11_injektiv_surjektiv_bijektiv} 
    \link{generic_article}{content/rwth/HM1/T204_Abbildungen_und_Funktionen/g_art_content_12_reelle_funktionen_monotonie.meta.xml}{content_12_reelle_funktionen_monotonie} 
    \link{generic_article}{content/rwth/HM1/T204_Abbildungen_und_Funktionen/g_art_content_13_unabzaehlbarkeit.meta.xml}{content_13_unabzaehlbarkeit} 
%
    \link{generic_article}{content/rwth/HM1/T205_Konvergenz_von_Folgen/g_art_content_13_reelle_folgen.meta.xml}{content_13_reelle_folgen} 
    \link{generic_article}{content/rwth/HM1/T205_Konvergenz_von_Folgen/g_art_content_14_konvergenz.meta.xml}{content_14_konvergenz} 
    \link{generic_article}{content/rwth/HM1/T205_Konvergenz_von_Folgen/g_art_content_15_monotone_konvergenz.meta.xml}{content_15_monotone_konvergenz} 
    \link{generic_article}{content/rwth/HM1/T205_Konvergenz_von_Folgen/g_art_content_16_konvergenzkriterien.meta.xml}{content_16_konvergenzkriterien} 
%   
    \link{generic_article}{content/rwth/HM1/T206_Folgen_II/g_art_content_19_bestimmte_divergenz.meta.xml}{content_19_bestimmte_divergenz} 
    \link{generic_article}{content/rwth/HM1/T206_Folgen_II/g_art_content_20_komplexe_folgen.meta.xml}{content_20_komplexe_folgen} 
% 
    \link{generic_article}{content/rwth/HM1/T207_Intervall_Schachtelung/g_art_content_21_intervalle.meta.xml}{content_21_intervalle} 
    \link{generic_article}{content/rwth/HM1/T207_Intervall_Schachtelung/g_art_content_22_offene_abgeschlossene_teilmengen.meta.xml}{content_22_offene_abgeschlossene_teilmengen} 
    \link{generic_article}{content/rwth/HM1/T207_Intervall_Schachtelung/g_art_content_23_intervallschachtelung.meta.xml}{content_23_intervallschachtelung} 
%
    \link{generic_article}{content/rwth/HM1/T208_Reihen/g_art_content_24_reihen_und_konvergenz.meta.xml}{content_24_reihen_und_konvergenz} 
    \link{generic_article}{content/rwth/HM1/T208_Reihen/g_art_content_25_konvergenz_kriterien.meta.xml}{content_25_konvergenz_kriterien} 
    \link{generic_article}{content/rwth/HM1/T208_Reihen/g_art_content_26_produkt_von_reihen.meta.xml}{content_26_produkt_von_reihen} 
%
    \link{generic_article}{content/rwth/HM1/T209_Potenzreihen/g_art_content_27_konvergenzradius.meta.xml}{content_27_konvergenzradius} 
    \link{generic_article}{content/rwth/HM1/T209_Potenzreihen/g_art_content_28_exponentialreihe.meta.xml}{content_28_exponentialreihe} 
%
    \link{generic_article}{content/rwth/HM1/T210_Stetigkeit/g_art_content_29_stetigkeit_definitionen.meta.xml}{content_29_stetigkeit_definitionen} 
    \link{generic_article}{content/rwth/HM1/T210_Stetigkeit/g_art_content_30_elem_funktionen.meta.xml}{content_30_elem_funktionen} 
    \link{generic_article}{content/rwth/HM1/T210_Stetigkeit/g_art_content_31_grenzwerte_von_funktionen.meta.xml}{content_31_grenzwerte_von_funktionen} 
    \link{generic_article}{content/rwth/HM1/T210_Stetigkeit/g_art_content_32_grenzwert_gegen_unendlich.meta.xml}{content_32_grenzwert_gegen_unendlich} 
%
    \link{generic_article}{content/rwth/HM1/T211_Eigenschaften_stetiger_Funktionen/g_art_content_33_zwischenwertsatz.meta.xml}{content_33_zwischenwertsatz} 
    \link{generic_article}{content/rwth/HM1/T211_Eigenschaften_stetiger_Funktionen/g_art_content_34_exp_und_log.meta.xml}{content_34_exp_und_log} 
    \link{generic_article}{content/rwth/HM1/T211_Eigenschaften_stetiger_Funktionen/g_art_content_35_trigonom_funktionen.meta.xml}{content_35_trigonom_funktionen} 
    \link{generic_article}{content/rwth/HM1/T211_Eigenschaften_stetiger_Funktionen/g_art_content_36_anwendungen.meta.xml}{content_36_anwendungen} 
%
%%%%%%%% Teil 3a 
%  
    \link{generic_article}{content/rwth/HM1/T301_Differenzierbarkeit/g_art_content_01_differenzenquotient.meta.xml}{content_01_differenzenquotient}
    \link{generic_article}{content/rwth/HM1/T301_Differenzierbarkeit/g_art_content_02_ableitungsregeln.meta.xml}{content_02_ableitungsregeln}
    \link{generic_article}{content/rwth/HM1/T301_Differenzierbarkeit/g_art_content_03_hoehere_ableitungen.meta.xml}{content_03_hoehere_ableitungen}
%    
    \link{generic_article}{content/rwth/HM1/T303_Approximationen/g_art_content_04_taylor_polynom.meta.xml}{content_04_taylor_polynom}
    \link{generic_article}{content/rwth/HM1/T303_Approximationen/g_art_content_05_newtonverfahren.meta.xml}{content_05_newtonverfahren}
    \link{generic_article}{content/rwth/HM1/T303_Approximationen/g_art_content_06_de_l_hospital.meta.xml}{content_06_de_l_hospital}
%    
    \link{generic_article}{content/rwth/HM1/T304_Integrierbarkeit/g_art_content_07_ober_und_untersumme.meta.xml}{content_07_ober_und_untersumme}
    \link{generic_article}{content/rwth/HM1/T304_Integrierbarkeit/g_art_content_08_integral_eigenschaften.meta.xml}{content_08_integral_eigenschaften}
    \link{generic_article}{content/rwth/HM1/T304_Integrierbarkeit/g_art_content_09_integrierbare_funktionen.meta.xml}{content_09_integrierbare_funktionen}
    \link{generic_article}{content/rwth/HM1/T304_Integrierbarkeit/g_art_content_10_uneigentliches_integral.meta.xml}{content_10_uneigentliches_integral}
%    
    \link{generic_article}{content/rwth/HM1/T305_Integrationstechniken/g_art_content_11_partielle_integration.meta.xml}{content_11_partielle_integration}
    \link{generic_article}{content/rwth/HM1/T305_Integrationstechniken/g_art_content_12_substitutionsregel.meta.xml}{content_12_substitutionsregel}
    \link{generic_article}{content/rwth/HM1/T305_Integrationstechniken/g_art_content_13_partialbruchzerlegung.meta.xml}{content_13_partialbruchzerlegung}
%    
    \link{generic_article}{content/rwth/HM1/T306_Reelle_Quadratische_Matrizen/g_art_content_14_quadratische_matrizen.meta.xml}{content_14_quadratische_matrizen}
    \link{generic_article}{content/rwth/HM1/T306_Reelle_Quadratische_Matrizen/g_art_content_15_inverse_matrix.meta.xml}{content_15_inverse_matrix}
    \link{generic_article}{content/rwth/HM1/T306_Reelle_Quadratische_Matrizen/g_art_content_16_determinante.meta.xml}{content_16_determinante}
    \link{generic_article}{content/rwth/HM1/T306_Reelle_Quadratische_Matrizen/g_art_content_17_cramersche_regel.meta.xml}{content_17_cramersche_regel}
%

%%%%%%%%% Teil 3b
    \link{generic_article}{content/rwth/HM1/T401_Matrizenrechnung/g_art_content_01_matrizen.meta.xml}{content_01_matrizen}
    \link{generic_article}{content/rwth/HM1/T401_Matrizenrechnung/g_art_content_02_matrizenmultiplikation.meta.xml}{content_02_matrizenmultiplikation}
    \link{generic_article}{content/rwth/HM1/T401_Matrizenrechnung/g_art_content_03_transponierte.meta.xml}{content_03_transponierte}
    \link{generic_article}{content/rwth/HM1/T402_Lineare_Gleichungssysteme/g_art_content_04_lgs.meta.xml}{content_04_lgs}
    \link{generic_article}{content/rwth/HM1/T402_Lineare_Gleichungssysteme/g_art_content_05_gaussverfahren.meta.xml}{content_05_gaussverfahren}
    \link{generic_article}{content/rwth/HM1/T402_Lineare_Gleichungssysteme/g_art_content_06_umformungen_rang.meta.xml}{content_06_umformungen_rang}
    \link{generic_article}{content/rwth/HM1/T403_Quadratische_Matrizen,_Determinanten/g_art_content_07_quadratische_matrizen.meta.xml}{content_07_quadratische_matrizen}
    \link{generic_article}{content/rwth/HM1/T403_Quadratische_Matrizen,_Determinanten/g_art_content_08_inverse_matrix.meta.xml}{content_08_inverse_matrix}
    \link{generic_article}{content/rwth/HM1/T403_Quadratische_Matrizen,_Determinanten/g_art_content_09_determinante.meta.xml}{content_09_determinante}
    \link{generic_article}{content/rwth/HM1/T403_Quadratische_Matrizen,_Determinanten/g_art_content_10_cramersche_regel.meta.xml}{content_10_cramersche_regel}
    \link{generic_article}{content/rwth/HM1/T403a_Vektorraum/g_art_content_10a_vektorraum.meta.xml}{content_10a_vektorraum}
    \link{generic_article}{content/rwth/HM1/T403a_Vektorraum/g_art_content_10b_lineare_abb.meta.xml}{content_10b_lineare_abb}
    \link{generic_article}{content/rwth/HM1/T403a_Vektorraum/g_art_content_10c_Orthogonalbasen.meta.xml}{content_10c_Orthogonalbasen}
    \link{generic_article}{content/rwth/HM1/T404_Eigenwerte,_Eigenvektoren/g_art_content_11_eigenwerte.meta.xml}{content_11_eigenwerte}
    \link{generic_article}{content/rwth/HM1/T404_Eigenwerte,_Eigenvektoren/g_art_content_12_symmetrische_matrizen.meta.xml}{content_12_symmetrische_matrizen}
\end{links}
  \creategeneric
\end{metainfo}
\begin{content}
\begin{block}[annotation]
	Im Ticket-System: \href{https://team.mumie.net/issues/30917}{Ticket 30917}
\end{block}

\lref{sec:a}{a} \lref{sec:b}{b} \lref{sec:c}{c} \lref{sec:d}{d} \lref{sec:e}{e} \lref{sec:f}{f} \lref{sec:g}{g}
\lref{sec:h}{h} \lref{sec:i}{i} \lref{sec:j}{j} \lref{sec:k}{k} \lref{sec:l}{l} \lref{sec:m}{m} \lref{sec:n}{n}
\lref{sec:o}{o} \lref{sec:p}{p} \lref{sec:q}{q} \lref{sec:r}{r} \lref{sec:s}{s} \lref{sec:t}{t} \lref{sec:u}{u}
\lref{sec:v}{v} \lref{sec:w}{w} \lref{sec:x}{x} \lref{sec:y}{y} \lref{sec:z}{z}
 
%%%%%%%%%%%%%%%%%%%%%%%%%%%%%%%
%%%%%%%%%%%%%%%%%%%%%%%%%%%%%%%%%%%
%Alphabetische Schlagworte
%%%%%%%%%%%%%%%%%%%%%%%%%%%%%%%%%%%%%%%%%%%%
%%%%%%%%%%%%%%%%%%%%%%%%%%%%%%%%%%%%%%%%%%%%%%%%%%%%%%%%%%%%%%%%%%%%%%%%%%%%%%%%%%%%%%%%%%%%%%%%%%%%%%%%%%%%%%%%%%%%%
\anchor{sec:a}{\textbf{A}}
\\
%
\lang{de}{Abbildungen}
\begin{itemize}
\item \lang{de}{\ref[content_10_abbildungen_verkettung][Abbildungen, Zuordnungen zwischen Mengen]{sec:abbildungen}}
\item \lang{de}{\ref[content_10_abbildungen_verkettung][Definitionsbereich und Zielbereich]{def:Wertemenge}} 
%\item \lang{de}{\ref[content_10_abbildungen_verkettung][grafische Darstellung von Abbildungen]{sec:graphik}}
\item \lang{de}{\ref[content_11_injektiv_surjektiv_bijektiv][Injektivität, Surjektivität, Bijektivität]{sec:inj-sur-bi}}
\item \lang{de}{\ref[content_10_abbildungen_verkettung][Komposition, Hintereinanderausführung, Verkettung]{kompositionv}}
\item \lang{de}{\ref[content_11_injektiv_surjektiv_bijektiv][Umkehrabbildung]{sec:umkehrabbildung}}
%\item \lang{de}{\ref[content_10_abbildungen_verkettung][Urbild]{def:urbild}}
%\item \lang{de}{\ref[content_11_injektiv_surjektiv_bijektiv][Umkehrabbildung, partielle]{sec:umkehrabbildung}}
\end{itemize}
%
\lang{de}{Abbildungsmatrix}
\begin{itemize}
\item \lang{de}{\ref[content_10b_lineare_abb][Abbildungsmatrix]{sec:abbildungsmatrizen}}
\end{itemize}


%
\lang{de}{a,b,c-Formel}
\begin{itemize}
\item \lang{de}{\ref[content_05_loesen_gleichungen_und_lgs][a,b,c-Formel (Mitternachtsformel)]{rem:mitternachtsformel}}
\end{itemize}
%
\lang{de}{Abelsches Lemma}
\begin{itemize}
\item \lang{de}{\ref[content_27_konvergenzradius][Abelsches Lemma]{thm:abelsches-lemma}}
\end{itemize}
%
\lang{de}{abgeschlossen}
\begin{itemize}
\item \lang{de}{\ref[content_01_zahlenmengen][abgeschlossenes Intervall]{def:intervall}}
\item \lang{de}{\ref[content_22_offene_abgeschlossene_teilmengen][abgeschlossene Menge (in $\R$)]{sec:abgeschl-mengen}}
\item \lang{de}{\ref[content_52_Abstaende][abgeschlossene Menge (im $\R^n$)]{def:offen_abgeschlossen}}
\item \lang{de}{\ref[content_52_Abstaende][abgeschlossener Quader]{ex:quader}}
\end{itemize}


%
\lang{de}{Ableitung}
\begin{itemize}
\item \lang{de}{\ref[content_20_ableitung_als_tangentensteigung][einer reellen Funktion (Teil 1)]{def:differenzierbar}}
\item \lang{de}{\ref[content_01_differenzenquotient][einer reellen Funktion (Teil 3a)]{def:diff}}
\item \lang{de}{\ref[content_54_Differentiation][als lineare Abbildung]{rem:lin_Abb}}
\item \lang{de}{\ref[content_20_ableitung_als_tangentensteigung][elementarer Funktionen Übersicht (Teil 1)]{rule:elementare_abl},
                \ref[content_01_differenzenquotient][Übersicht (Teil 3a)]{sec:abl-elem-funk}}
\item \lang{de}{\ref[content_02_ableitungsregeln][der Umkehrfunktion]{rule:abl-umkehrfkt}}
\item \lang{de}{\ref[content_03_hoehere_ableitungen][höhere Ableitungen (Einführung)]{sec:hoehere-abl}}
\item \lang{de}{\link{content_54_Differentiation}{mehrdimensionale}}
\item \lang{de}{\ref[content_54_Differentiation][Richtungsableitung]{def:richtungsableitung}}
\item \lang{de}{\ref[content_54_Differentiation][totale]{def:total-diffbar}}
\end{itemize}
%
\lang{de}{Ableitungsregeln}
\begin{itemize}
\item \lang{de}{\ref[content_20_ableitung_als_tangentensteigung][Ableitungsregeln (Teil 1)]{sec:abl_regeln}}
\item \lang{de}{\link{content_20_ableitung_als_tangentensteigung}{Ableitung der Umkehrfunktion (Teil 3a)}}
\item \lang{de}{\ref[content_20_ableitung_als_tangentensteigung][Faktorregel (Teil 1)]{rule:const_factor},
                \ref[content_02_ableitungsregeln][(Teil 3a)]{rule:summenregel}}
\item \lang{de}{\ref[content_21_kettenregel][Kettenregel (Teil 1)]{rule:kettenregel},
                \ref[content_02_ableitungsregeln][(Teil 3a)]{sec:kettenregel}}
\item \lang{de}{\ref[content_20_ableitung_als_tangentensteigung][Produktregel (Teil 1)]{rule:produktregel},
                \ref[content_02_ableitungsregeln][(Teil 3a)]{rule:produkt_quotient_regel}}
\item \lang{de}{\ref[content_20_ableitung_als_tangentensteigung][Quotientenregel (Teil 1)]{rule:quotient_regel},
                \ref[content_02_ableitungsregeln][(Teil 3a)]{rule:produkt_quotient_regel}}
\item \lang{de}{\ref[content_20_ableitung_als_tangentensteigung][Summenregel (Teil 1)]{rule:additiv},
                \ref[content_02_ableitungsregeln][(Teil 3a)]{rule:summenregel}}
\item \lang{de}{\ref[content_54_Differentiation][Ableitungsregeln mehrdimensionaler Funktionen]{sec:Ableitungsregeln}}
\end{itemize}
%
%
\lang{de}{Absolutbetrag}
\begin{itemize}
\item \lang{de}{\ref[content_08bneu_komplexeZahlen_geom][komplexer Zahlen]{betragkz}}
\item \lang{de}{\ref[content_01_zahlenmengen][reeller Zahlen]{def:betrag}}
\end{itemize}
%
\lang{de}{Abstand}
\begin{itemize}
\item \lang{de}{\ref[content_38_abstaende][ von Gerade und Ebene]{rule:abst_gerade_ebene}}
\item \lang{de}{\ref[content_38_abstaende][von Punkt und Ebene]{rule:abst_pkt_ebene}}
\item \lang{de}{\ref[content_38_abstaende][von Punkt und Gerade]{rule:abst_pkt_gerade}}
%\item \lang{de}{\ref[content_38_abstaende][windschiefer Geraden]{def:abst_windschief_gerade_n}}
\item \lang{de}{\ref[content_38_abstaende][paralleler Ebenen]{rule:abst_ebene_ebene}}
\item \lang{de}{\ref[content_38_abstaende][paralleler Geraden]{rule:abst_gerade_gerade}}
\item \lang{de}{\ref[content_32_laenge_norm][von Punkten im $\R^n$]{thm:abst_pkte}}
\item \lang{de}{\ref[content_38_abstaende][windschiefer Geraden]{def:abst_windschief_gerade}}
\end{itemize}
%
\lang{de}{Abzählbarkeit}
\begin{itemize}
\item \lang{de}{\ref[content_13_unabzaehlbarkeit][Abzählbarkeit]{def:abzaehlbarkeit}}
\end{itemize}
% 
\lang{de}{Addition}
\begin{itemize}
\item \lang{de}{\ref[content_03_bruchrechnung][Brüche und Bruchterme]{add}}
\item \lang{de}{\ref[content_04_koerperaxiome][Körperaxiome]{sec:axiome}}
\item\lang{de}{\ref[content_08aneu_komplexeZahlen_intro][komplexe Zahlen]{sec:add-mult-kompl}}
\item \lang{de}{\ref[content_02_rechengrundlagen_terme][reelle Zahlen]{def:grundrechenarten}}
\item \lang{de}{\ref[content_42_matrixaddition][Matrizen]{def:matrix_add}}
\item\lang{de}{\ref[content_27_vektoren][Vektoren]{def:vec-addition}}
\end{itemize}
%
\lang{de}{Additionstheoreme}
\begin{itemize}
\item \lang{de}{\ref[content_28_exponentialreihe][Additionstheoreme]{rem:sin_cos_exp}}
\end{itemize}
%
\lang{de}{Äquivalenz}
\begin{itemize}
\item \lang{de}{\ref[content_04_aussagen_aequivalenzumformungen][Äquivalenz (Definition)]{def:aequvalenz}}
%\item \lang{de}{\ref[content_04_aussagen_aequivalenzumformungen][Äquivalenz als logische Verknüpfung von Aussagen]{def:objekte_aussagenlogik}}
%\item \lang{de}{\ref[content_04_aussagen_aequivalenzumformungen][Wahrheitstafel zur Äquivalenz]{ex:equivalence}}
\item \lang{de}{\ref[content_04_aussagen_aequivalenzumformungen][Äquivalenzumformung]{sec:aequivalenz}}
\item\lang{de}{\ref[content_52_Abstaende][Äquivalenz von Normen]{remark:norm-equivalence}}
\end{itemize}
%
\lang{de}{äußere Funktion}
\begin{itemize}
\item \lang{de}{\ref[content_21_kettenregel][äußere Funktion]{addition.definition.1}}
\end{itemize}
%
\lang{de}{Approximationsverfahren}
\begin{itemize}
\item \lang{de}{\ref[content_04_taylor_polynom][Taylor-Näherung]{sec:polynom}}
\item \lang{de}{\ref[content_05_newtonverfahren][Newton-Verfahren]{newton}}
\item \lang{de}{\ref[content_05_newtonverfahren][Approximation von Wurzeln]{sec:n-te-wurzeln}}
\item \lang{de}{\ref[content_06_de_l_hospital][Regel von de l'Hospital]{sec:lHospital}}
%\item \lang{de}{\ref[content_06_de_l_hospital][Zusammenhang von de l'Hospitalsregel zur Taylor-Approximation]{sec:lhopital-taylor}}
\end{itemize}
%
\lang{de}{Anfangswertproblem}
\begin{itemize}
\item \lang{de}{\ref[content_56_Separierbare_Differentialgleichungen][separierbares Anfangswertproblem]{def:sep_DGL}}
\item \lang{de}{\ref[content_57_gewoehnliche_DGL_erster_Ordnung][Lösbarkeit des gewöhnlichen Anfangswertproblems 1. Ordnung]{thm:ODE}}
\end{itemize}
%
\lang{de}{Ankathete}
\begin{itemize}
\item \lang{de}{\ref[content_17_trigonometrie_im_dreieck][Ankathete]{def:rechtw_Dreieck}}
\end{itemize}
%
\lang{de}{Anordnung}
\begin{itemize}
\item \lang{de}{\ref[content_01_zahlenmengen][Anordnung  reeller Zahlen]{def:ordering}}
\item \lang{de}{\ref[content_05_anordnungsaxiome][Anordnungsaxiome]{rule:anordnungsregeln}}
\end{itemize}
%
\lang{de}{Argument}
\begin{itemize}
\item \lang{de}{\ref[content_08bneu_komplexeZahlen_geom][Argument einer komplexen Zahl]{sec:polar}}
\end{itemize}
%
\lang{de}{arithmetisches Mittel}
\begin{itemize}
\item \lang{de}{\ref[content_02_vollstaendige_induktion][arithmetisches Mittel]{rule:means_ineq}}
\end{itemize}
%
\lang{de}{Assoziativgesetz}
\begin{itemize}
\item \lang{de}{\ref[content_02_rechengrundlagen_terme][Assoziativgesetz]{rule:rechengesetze}}
\end{itemize}
%
\lang{de}{Asymptoten}
\begin{itemize}
\item \lang{de}{\ref[content_32_grenzwert_gegen_unendlich][senkrechte]{rem:senkrechte-asymptote}}
\item \lang{de}{\ref[content_32_grenzwert_gegen_unendlich][waagerechte]{rem:waagerechte-asymptote}}
\end{itemize}
%
\lang{de}{Ausklammern}
\begin{itemize}
\item \lang{de}{\ref[content_02_rechengrundlagen_terme][Ausklammern, Ausmultiplizieren,  Faktorisieren]{rule:rechengesetze}}
%\item \lang{de}{\ref[content_02_rechengrundlagen_terme][Distributivgesetz]{rule:rechengesetze}}
\end{itemize}
%
\lang{de}{Aussage}
\begin{itemize}
\item \lang{de}{\ref[content_04_aussagen_aequivalenzumformungen][Aussage (Aussagenlogik)]{def:aussagen}}
 \item \lang{de}{\ref[content_04_aussagen_aequivalenzumformungen][Aussage als bewertete Aussageform, Formel]{def:aussagen_af}}
% \item \lang{de}{\ref[content_04_aussagen_aequivalenzumformungen][Aussagenvariable, Aussageform, Formel]{def:objekte_aussagenlogik}}
\end{itemize}

%
\anchor{sec:b}{\textbf{B}}
\\
%
\lang{de}{Basis}
\begin{itemize}
\item \lang{de}{\ref[content_30_basen_eigenschaften][des $\R^n$]{def_basis}}
\item \lang{de}{\ref[content_10a_vektorraum][eines $\K$-Vektorraums $\vectorspace{V}$]{def:basis}}
\item \lang{de}{\ref[content_14_potenzregeln][einer Potenz]{def:potenzen}}
\end{itemize}
%
\lang{de}{Basisergänzungssatz}
\begin{itemize}
\item \lang{de}{\ref[content_10a_vektorraum][Basisergänzungssatz]{thm:basisergaenzungssatz}}
\end{itemize}
%
\lang{de}{Basiswechsel}
\begin{itemize}
\item \lang{de}{\ref[content_10b_lineare_abb][Basiswechsel]{sec:basiswechsel}}
\end{itemize}


%
\lang{de}{Bernoullische Ungleichung}
\begin{itemize}
\item \lang{de}{\ref[content_02_vollstaendige_induktion][Bernoullische Ungleichung]{rule:bernoulli-ungleichung}}
\end{itemize}
%
\lang{de}{Beschränktheit}
\begin{itemize}
\item \lang{de}{\ref[content_06_supremum_infimum][beschränkte Menge]{def:beschraenkt}}
\item \lang{de}{\ref[content_12_reelle_funktionen_monotonie][beschränkte Funktion]{def:bounded}}
\end{itemize}
%
\lang{de}{Betrag}
\begin{itemize}
\item \lang{de}{\ref[content_08bneu_komplexeZahlen_geom][komplexer Zahlen]{betragkz}}
\item \lang{de}{\ref[content_01_zahlenmengen][reeller Zahlen]{def:betrag}}
\item eines Vektors, siehe \lang{de}{\ref[content_32_laenge_norm][Länge]{def:euklidische_norm}}
\end{itemize}
%
\lang{de}{Beweis}
\begin{itemize}
%\item \lang{de}{\ref[content_04_aussagen_aequivalenzumformungen][Aussagen: Äquivalenz und Folgerung]{sec:folgerung}}
\item \lang{de}{\ref[content_01_indirekter_widerspruchsbeweis][direkter]{sec:direkterbeweis}}
\item \lang{de}{\ref[content_01_indirekter_widerspruchsbeweis][indirekter, Widerspruchsbeweis]{sec:indirekterbeweis}}
\item \lang{de}{\ref[content_02_vollstaendige_induktion][durch vollständige Induktion]{sec:induktionsprinzip}}
\end{itemize}
%
\lang{de}{Bijektivität}
\begin{itemize}
\item \lang{de}{\ref[content_11_injektiv_surjektiv_bijektiv][bijektiv]{sec:inj-sur-bi}}
\end{itemize}
%
\lang{de}{Bild}
\begin{itemize}
\item \lang{de}{\ref[content_06_funktionsbegriff_und_lineare_funktionen][Bild, Bildmenge]{sec:funktion}}
\end{itemize}
%
\lang{de}{Binomialkoeffizienten}
\begin{itemize}
\item \lang{de}{\ref[content_03_binomischer_lehrsatz][Binomialkoeffizient]{def:Binomcoeff}}
%\ref[content_03_binomischer_lehrsatz][--, Fakultät]{def:fakultaet}
\item \lang{de}{\ref[content_03_binomischer_lehrsatz][
Identitäten von Binomialkoeffizienten]{thm:identitaeten}}
\end{itemize}
%
\lang{de}{Binomische Formeln}
\begin{itemize}
\item \lang{de}{\ref[content_02_rechengrundlagen_terme][Binomische Formeln]{rule:binomische_formeln}}
\end{itemize}

\lang{de}{binomischer Lehrsatz}
\begin{itemize}
\item \lang{de}{\ref[content_03_binomischer_lehrsatz][binomischer Lehrsatz]{thm:binom}}
\end{itemize}
%
\lang{de}{Bisektionsverfahren}
\begin{itemize}
\item \lang{de}{\ref[content_23_intervallschachtelung][Bisektionsverfahren]{sec:intervallhalbierung}}
\end{itemize}
%
\lang{de}{Bogenmaß}
\begin{itemize}
\item \lang{de}{\link{content_18_grad_und_bogenmass}{Bogenmaß}}
\end{itemize}
%
\lang{de}{Bolzano-Weierstrass}
\begin{itemize}
\item \lang{de}{\ref[content_16_konvergenzkriterien][Satz von Bolzano-Weierstrass]{thm:bolzano-weierstrass}}
\end{itemize}
%
\lang{de}{Bruchrechnung}
\begin{itemize}
\item \lang{de}{\ref[content_03_bruchrechnung][Addition und Subtraktion]{add}}
\item \lang{de}{\ref[content_03_bruchrechnung][Division]{sec:div}}
\item \lang{de}{\ref[content_03_bruchrechnung][Erweitern und Kürzen]{erweitern_kuerzen}}
\item \lang{de}{\ref[content_03_bruchrechnung][Multiplikation]{mul}}
% \item \lang{de}{\ref[content_03_bruchrechnung][ggT-Bestimmung mittels Primfaktorzerelegung]{ggT}}
% \item \lang{de}{\ref[content_03_bruchrechnung][ggT-Bestimmung mittels Euklidischer Algorithmus]{euklid_algorithmus}}
\item \lang{de}{\ref[content_03_bruchrechnung][Hauptnenner]{kgV}}
\end{itemize}

%
\anchor{sec:c}{\textbf{C}}
\\
%
$\C$
\begin{itemize}
\item \lang{de}{\ref[content_08aneu_komplexeZahlen_intro][$\C$ (komplexe Zahlen)]{thm:C_field}}
\end{itemize}
%
\lang{de}{Cauchy-Produkt}
\begin{itemize}
\item \lang{de}{\ref[content_26_produkt_von_reihen][Cauchy-Produkt von Reihen]{def:cauchy-prod}}
\end{itemize}
%
\lang{de}{Cauchy-Hadamard}
\begin{itemize}
\item \lang{de}{\ref[content_27_konvergenzradius][Satz von Cauchy-Hadamard]{thm:Cauchy-Hadamard}}
\end{itemize}
%
\lang{de}{Cauchy-Schwarz-Ungleichung}
\begin{itemize}
\item \lang{de}{\ref[content_32_laenge_norm][Cauchy-Schwarz-Ungleichung]{thm:eigenschaften-norm}}
\end{itemize}
%
\lang{de}{charakteristisches Polynom}
\begin{itemize}
\item \lang{de}{\ref[content_11_eigenwerte][charakteristisches Polynom]{sec:charakteristischesPolynom}}
\item \lang{de}{\ref[content_11_eigenwerte][Nullstellen]{def:char_pol}}
\end{itemize}
%

%
\lang{de}{Cramersche Regel}
\begin{itemize}
\item \lang{de}{Cramersche Regel \ref[content_17_cramersche_regel][(reell),]{cramersche_regel} \ref[content_10_cramersche_regel][ (über beliebigen Körpern)]{cramersche_regel}}
\end{itemize}
\\
%%
%%
\anchor{sec:d}{\textbf{D}}
\\
%
\lang{de}{Definitionsbereich}
\begin{itemize}
\item \lang{de}{\ref[content_10_abbildungen_verkettung][Definitionsbereich]{def:Wertemenge}} 
\end{itemize}
%
\lang{de}{Dezimalbruchentwicklung}
\begin{itemize}
\item \lang{de}{\ref[content_24_reihen_und_konvergenz][Dezimalbruchentwicklung]{ex:dezimalbruch-als-reihe}}
\item \lang{de}{\ref[content_24_reihen_und_konvergenz][$0,\overline{9}=1$]{ex:erste-reihen}}
\end{itemize}
%
%
\lang{de}{de l'Hospital}
\begin{itemize}
\item \lang{de}{\ref[content_06_de_l_hospital][Regel von de l'Hospital]{sec:lHospital}}
\end{itemize}
%
\lang{de}{Determinante}
\begin{itemize}
\item \lang{de}{Determinate (Definition) \ref[content_16_determinante][(reell), ]{def:determinante} \ref[content_09_determinante][(über beliebigen Körpern)]{def:determinante}}
\item \lang{de}{Rechenregeln \ref[content_16_determinante][(reell),]{sec:rechenregeln-determinante} \ref[content_09_determinante][(über beliebigen Körpern)]{sec:determinante-rechenregeln}}
\end{itemize}
%
\lang{de}{Diagonalmatrix}
\begin{itemize}
\item \lang{de}{Diagonalmatrix \ref[content_14_quadratische_matrizen][(reell),]{def:wichtige-quadr-mat} \ref[content_07_quadratische_matrizen][(über beliebigen Körpern)]{def:wichtige-quadratische-matrizen}}
\end{itemize}

%
\lang{de}{Differential}
\begin{itemize}
\item \lang{de}{\ref[content_24_integral_als_flaeche][Differential $dx$]{bestimmtes_Integral}}
\item \lang{de}{\ref[content_54_Differentiation][Differential]{def:part_abl}}
\end{itemize}
%
\lang{de}{Differenz}
\begin{itemize}
\item \lang{de}{\ref[content_01_zahlenmengen][Differenz, Differenzmenge]{def:mengenoperationen}}
\end{itemize}
%
\lang{de}{Differenzenquotient}
\begin{itemize}
\item \lang{de}{\ref[content_20_ableitung_als_tangentensteigung][Differenzenquotient (Teil 1)]{diffbarkeit},
                \ref[content_01_differenzenquotient][(Teil 3a)]{def:punkt-diff}}
\end{itemize}
%
\lang{de}{Differentialgleichung}
\begin{itemize}
\item \lang{de}{\ref[content_57_gewoehnliche_DGL_erster_Ordnung][autonome ]{def:autonom_DGL}}
\item \lang{de}{\ref[content_57_gewoehnliche_DGL_erster_Ordnung][Bernoullische]{ex:bernoullische-dgl}}
\item \lang{de}{\ref[content_57_gewoehnliche_DGL_erster_Ordnung][gewöhnliche Differentialgleichungen 1. Ordnung]{def:ODE}}
\item \lang{de}{\ref[content_56_Separierbare_Differentialgleichungen][homogene lineare ]{thm:lin_DG}}
\item \lang{de}{\ref[content_56_Separierbare_Differentialgleichungen][inhomogene lineare ]{thm:lin_DG}}
\item \lang{de}{\ref[content_56_Separierbare_Differentialgleichungen][lineare ]{def:lin_DGL}}
\item \lang{de}{\ref[content_56_Separierbare_Differentialgleichungen][separierbare ]{def:sep_DGL}}
\end{itemize}
%
\lang{de}{Differenzierbarkeit}
\begin{itemize}
\item \lang{de}{\ref[content_20_ableitung_als_tangentensteigung][einer reeller Funktion in einem Punkt (Teil 1)]{diffbarkeit},
                \ref[content_01_differenzenquotient][(Teil 3a)]{def:punkt-diff}}
\item \lang{de}{\ref[content_03_hoehere_ableitungen][$n$-fache Differenzierbarkeit]{def:higher_deriv}}
\item \lang{de}{\ref[content_20_ableitung_als_tangentensteigung][reeller Funktionen (Teil 1)]{def:differenzierbar},
                \ref[content_01_differenzenquotient][(Teil 3a)]{def:diff}}
\item \lang{de}{\ref[content_54_Differentiation][partielle Differenzierbarkeit]{def:part_abl}}
\item \lang{de}{\ref[content_54_Differentiation][stetige Differenzierbarkeit (mehrdimensional)]{def:stetig_diffbar}}
\item \lang{de}{\ref[content_54_Differentiation][stetig partielle Differenzierbarkeit]{def:part_abl}}
\item \lang{de}{\ref[content_54_Differentiation][totale Differenzierbarkeit]{def:total-diffbar}}
\item \lang{de}{\ref[content_03_hoehere_ableitungen][Zusammenhang zur Stetigkeit]{thm:diffstetig}}
\item \lang{de}{\ref[content_03_hoehere_ableitungen][Zusammenhang zur Monotonie]{thm:diffmonotonie}}
\end{itemize}
%
\lang{de}{Dimension}
\begin{itemize}
\item \lang{de}{\ref[content_30_basen_eigenschaften][Dimension des $\R^n$]{def:dimension}}
\item \lang{de}{\ref[content_10a_vektorraum][Dimension eines  $\K$-Vektorraums]{def:dimension}}
\end{itemize}
%

\lang{de}{Dirichletsche Sprungfunktion}
\begin{itemize}
\item \lang{de}{\ref[content_12_reelle_funktionen_monotonie][Dirichletsche Sprungfunktion]{ex:unstetige-funktionen}}
\item \lang{de}{\ref[content_29_stetigkeit_definitionen][unstetig]{ex:unstetig}}
\end{itemize}
%
\lang{de}{Disjunktion}
\begin{itemize}
\item \lang{de}{\ref[content_04_aussagen_aequivalenzumformungen][Disjunktion, logisches Oder]{def:objekte_aussagenlogik}}
%\item \lang{de}{\ref[content_04_aussagen_aequivalenzumformungen][--, Wahrheitstafel]{ex:disjuktion}}
\end{itemize}
%
\lang{de}{Distributivgesetz}
\begin{itemize}
%\item \lang{de}{\ref[content_02_rechengrundlagen_terme][Ausmultiplizieren, Ausklammern, Faktorisieren]{rule:rechengesetze}}
\item \lang{de}{\ref[content_02_rechengrundlagen_terme][Distributivgesetz]{rule:rechengesetze}}
\end{itemize}
%
\lang{de}{Divergenz}
\begin{itemize}
\item \lang{de}{\ref[content_14_konvergenz][Divergenz]{def:folgenkonvergent}}
\item \lang{de}{\ref[content_52_Abstaende][von Folgen im $\R^n$]{def:folge_im_R_n}}
\item \lang{de}{\ref[content_19_bestimmte_divergenz][Divergenz, bestimmte]{def:bestimmteDivergenz}}
\item \lang{de}{\ref[content_19_bestimmte_divergenz][Rechenregeln für bestimmte Divergenz]{sec:rechenregeln}}

\end{itemize}
%
\lang{de}{Division}
\begin{itemize}
\item \lang{de}{\ref[content_02_rechengrundlagen_terme][Division]{def:grundrechenarten}}
\item \lang{de}{\ref[content_03_bruchrechnung][Division von Brüchen]{sec:div}}
\end{itemize}

%
\lang{de}{Dreiecksmatrix}
\begin{itemize}
\item \lang{de}{oberer/untere Dreiecksmatrix \ref[content_14_quadratische_matrizen][(reell),]{def:wichtige-quadr-mat} \ref[content_07_quadratische_matrizen][(über beliebigen Körpern)]{def:wichtige-quadratische-matrizen}}
\end{itemize}

%
\lang{de}{Dreiecksungleichung}
\begin{itemize}
\item \lang{de}{\ref[content_08bneu_komplexeZahlen_geom][des Absolutbetrags]{rule:betragsregeln}}
\item \lang{de}{\ref[content_32_laenge_norm][der euklidischen Norm]{thm:eigenschaften-norm}}
\item \lang{de}{\ref[content_52_Abstaende][einer beliebigen Norm]{def:norm_allgemein}}
\item \lang{de}{\ref[content_25_konvergenz_kriterien][für Reihen]{ex:dreiecksungl-reihen}}
\end{itemize}
%
\lang{de}{Durchschnitt}
\begin{itemize}
\item \lang{de}{\ref[content_01_zahlenmengen][Durchschnitt, Schnittmenge]{def:mengenoperationen}}
\end{itemize}
%
\anchor{sec:e}{\textbf{E}}
\\
%
\lang{de}{$\in$}
\begin{itemize}
\item \lang{de}{\ref[content_01_zahlenmengen][$\in$, Element]{def:menge}}
\end{itemize}
%
\lang{de}{Ebene}
\begin{itemize}
\item \lang{de}{\ref[content_35_parameterformen][Drei-Punkte-Darstellung]{rule:drei_pkt_ebenen}}
\item \lang{de}{\ref[content_36_normalenformen][Einheitsnormalenvektor]{def:norm_vec_ebene}}
\item \lang{de}{\ref[content_36_normalenformen][Hessesche Normalenform]{def:hesse_nf_ebene}}
\item \lang{de}{\ref[content_36_normalenformen][Koordinatenform]{def:norm_form_ebene}}
\item \lang{de}{\ref[content_36_normalenformen][Normalenvektor]{def:norm_vec_ebene}}
\item \lang{de}{\ref[content_35_parameterformen][Parameterform]{rule:drei_pkt_ebenen}}
\item \lang{de}{\ref[content_36_normalenformen][Punkt-Normalen-Form]{def:norm_form_ebene}}
\item \lang{de}{\ref[content_35_parameterformen][Punkt-Richtungs-Darstellung]{rule:pkt_richt_ebenen}}
\item \lang{de}{\ref[content_36_normalenformen][Umrechnung Darstellungsformen]{sec:umrechng_ebenenform}}
\end{itemize}
%
\lang{de}{Eigenvektor}
\begin{itemize}
\item \lang{de}{\ref[content_11_eigenwerte][Eigenvektor]{sec:eigenwerte_eigenvektoren}}
\item \lang{de}{\ref[content_12_symmetrische_matrizen][Eigenvektoren reeller symmetrischer Matrizen]{sec:eigenvektorenRellerSymmetrischerMatrizen}}
\item \lang{de}{\ref[content_12_symmetrische_matrizen][Hauptachsentransformation]{sec:hauptaschsentransformation}}
\item \lang{de}{\ref[content_12_symmetrische_matrizen][Kegelschnitt]{rem:kegelschnitte}}
\end{itemize}

%
\lang{de}{Eigenwert}
\begin{itemize}
\item \lang{de}{\ref[content_11_eigenwerte][Eigenwert]{sec:eigenwerte_eigenvektoren}}
\item \lang{de}{\ref[content_11_eigenwerte][charakteristisches Polynom]{sec:charakteristischesPolynom}}
\item \lang{de}{\ref[content_12_symmetrische_matrizen][Eigenwerte reeller symmetrischer Matrizen]{sec:eigenwerteRellerSymmetrischerMatrizen}}
\end{itemize}
%
\lang{de}{Einheitsmatrix}
\begin{itemize}
\item \lang{de}{Einheitsmatrix \ref[content_14_quadratische_matrizen][(reell),]{def:wichtige-quadr-mat} \ref[content_07_quadratische_matrizen][(über beliebigen Körpern)]{def:wichtige-quadratische-matrizen}}
\end{itemize}

%
\lang{de}{Einheitsnormalenvektor}
\begin{itemize}
\item \lang{de}{\ref[content_36_normalenformen][einer Ebene]{def:norm_vec_ebene}}
\item \lang{de}{\ref[content_36_normalenformen][einer Geraden]{def:norm_vec_gerade}}
\end{itemize}
%
\lang{de}{elementare Zeilenumformung}
\begin{itemize}
\item \lang{de}{\ref[content_41_gauss_verfahren][elementare Zeilenumformung]{def:elementare_Zeilenumformungen}}
\item \lang{de}{\ref[content_06_umformungen_rang][durch Matrix-Multiplikation]{sec:zeilenumformungMitMatrixmul}}
\end{itemize}
%
\lang{de}{Elementarmatrix}
\begin{itemize}
\item \lang{de}{\ref[content_06_umformungen_rang][Elementarmatrix]{rule:elementarmatrizen}}
\end{itemize}
%
%
\lang{de}{$\epsilon$-$\delta$-Kriterium}
\begin{itemize}
\item \lang{de}{$\epsilon$-$\delta$-Kriterium \ref[content_29_stetigkeit_definitionen][(in $\R$), ]{sec:eps-delta}\ref[content_53_Stetigkeit][ (im $\R^n$)]{def:stetigkeit_n-dim}}
\end{itemize}
%
\lang{de}{Epsilon-Umgebung}
\begin{itemize}
\item \lang{de}{\ref[content_21_intervalle][Epsilon-Umgebung in den reellen Zahlen]{sec:epsilon-umg}}
\item \lang{de}{{\ref[content_52_Abstaende][Epsilon-Umgebung im $\R^n$ (offene Kugel)]{rem:offene_Kugeln}}}
%\item \lang{de}{\ref[content_21_intervalle][Intervalle, endlich]{section.intervals}}
% \item \lang{de}{\ref[content_21_intervalle][Intervalle, unendlich]{sec:unendl-intervalle}}
\end{itemize}

%
\lang{de}{Erweitern}
\begin{itemize}
\item \lang{de}{\ref[content_03_bruchrechnung][Erweitern von Brüchen und Bruchtermen]{erweitern_kuerzen}}
\end{itemize}
%
\lang{de}{Erzeugendensystem} 
\begin{itemize}
\item \lang{de}{\ref[content_30_basen_eigenschaften][Erzeugendensystem des $\R^n$]{def_ES}}
\item \lang{de}{\ref[content_10a_vektorraum][Erzeugendensystem eines  $\K$-Vektorraums]{def:erzeugendensystem}}
\end{itemize}
%
\lang{de}{euklidische Norm}
\begin{itemize}
\item \lang{de}{\ref[content_32_laenge_norm][euklidische Norm]{def:euklidische_norm}}
\end{itemize}
%
\lang{de}{euklidischer Algorithmus}
\begin{itemize}
\item \lang{de}{\ref[content_03_bruchrechnung][euklidischer Algorithmus]{euklid_algorithmus}}
\end{itemize}
%
\lang{de}{Eulersche Formel/Eulersche Identität}
\begin{itemize}
\item \lang{de}{\ref[content_28_exponentialreihe][Eulersche Formel/Eulersche Identität]{rem:euler_formel}}
\end{itemize}
%
\lang{de}{Eulersche Zahl}
\begin{itemize}
\item \lang{de}{\ref[content_15_monotone_konvergenz][Eulersche Zahl]{sec:eulersche-zahl}}
\item \lang{de}{\ref[content_15_exponentialfunktionen][Eulersche Zahl als Basis der natürlichen Exponentialfunktion]{sec:nat-exp-fct}}
\end{itemize}

%
\lang{de}{Exponent}
\begin{itemize}
\item \lang{de}{\ref[content_14_potenzregeln][Exponent, ganzzahliger]{def:potenzen}}
\item \lang{de}{\ref[content_14_potenzregeln][Exponent, rationaler]{def:rat_potenz}}
\end{itemize}
%
\lang{de}{Exponentialfunktion}
\begin{itemize}
\item \lang{de}{\ref[content_15_exponentialfunktionen][zu beliebigiger Basis]{sec:exp_fkt}}
\item \lang{de}{\ref[content_34_exp_und_log][Asymptotik]{thm:exp_asymptotik}}
\item \lang{de}{\ref[content_34_exp_und_log][Eigenschaften]{thm:eigenschaften_exp}}
\item \lang{de}{\ref[content_28_exponentialreihe][Eulersche Identität]{rem:sin_cos_exp}}
\item \lang{de}{\ref[content_26_produkt_von_reihen][Funktionalgleichung]{ex:funktionalgleichung-exp}}
\item \lang{de}{\ref[content_28_exponentialreihe][komplexe Exponentialfunktion, Exponentialreihe]{sec:exp-reihe}}
\item \lang{de}{\ref[content_15_exponentialfunktionen][natürliche Exponentialfunktion, e-Funktion]{sec:nat-exp-fct}}
\item \lang{de}{\ref[content_34_exp_und_log][als Funktion auf den rationalen Zahlen]{rule:exp-auf-q}}
\item \lang{de}{\ref[content_30_elem_funktionen][Stetigkeit]{sec:Potenzreihen}}
\end{itemize}
%
\lang{de}{Exponentialgleichung}
\begin{itemize}
\item \lang{de}{\ref[content_16_logarithmen][Exponentialgleichungen lösen]{rule:exp_glg}}
\end{itemize}
%
\lang{de}{Extremstelle, Extremum}
\begin{itemize}
\item \lang{de}{\ref[content_22_extremstellen][lokale und globale Extrema]{def:extremstellen}}
\item \lang{de}{\ref[content_54_Differentiation][lokale Extremstellen (mehrdimensionale Analysis)]{def:extremstellen}}
\item \lang{de}{\ref[content_22_extremstellen][notwendige Bedingung für lokale Extrema (Teil 1)]{thm:notw_bedg_lok_extremum},
                \ref[content_03_hoehere_ableitungen][(Teil 3a)]{thm:NB_extrema}}
\item \lang{de}{\ref[content_22_extremstellen][hinreichende Bedingung für lokale Extrema]{sec:hinr_bedg_lok_extremum}}
\end{itemize}
%
\anchor{sec:f}{\textbf{F}}
\\
%
\lang{de}{Faktorregel}
\begin{itemize}
\item \lang{de}{\ref[content_20_ableitung_als_tangentensteigung][Faktorregel (Teil 1)]{rule:const_factor},
                \ref[content_02_ableitungsregeln][(Teil 3a)]{rule:summenregel}}
\end{itemize}
%
\lang{de}{Falksches Schema}
\begin{itemize}
\item \lang{de}{\ref[content_43_matrizenmultiplikation][Falksches Schema zur Matrixmultiplikation]{rem:falk_schema}}
\end{itemize}
%
\lang{de}{Fakultät}
\begin{itemize}
\item \lang{de}{\ref[content_03_binomischer_lehrsatz][Fakultät]{def:fakultaet}}
\end{itemize}
%
\lang{de}{Fläche}
\begin{itemize}
\item \lang{de}{\ref[content_26_flaechen_zwischen_graphen][Fläche zwischen zwei Graphen]{flaeche}}
\item \lang{de}{\ref[content_24_integral_als_flaeche][Flächeninhalt mit Vorzeichen]{orient-flaeche}}
\item \lang{de}{\ref[content_34_vektorprodukt][Fläche eines Parallelogramms]{rem:parallelogram}}
\item \lang{de}{\ref[content_24_integral_als_flaeche][Integral als orientierte Fläche]{orient-flaeche}}
\end{itemize}
%
\lang{de}{Folge}
\begin{itemize}
\item \lang{de}{\ref[content_13_reelle_folgen][arithmetische]{ex:22}}
\item \lang{de}{\ref[content_13_reelle_folgen][beschränkte]{def:beschränkte_Folge}}
\item \lang{de}{\ref[content_15_monotone_konvergenz][beschränkt und monoton zugleich]{the:beschmon-folge}}
\item \lang{de}{\ref[content_19_bestimmte_divergenz][bestimmt divergente]{def:bestimmteDivergenz}}
\item \lang{de}{\ref[content_14_konvergenz][divergente]{def:folgenkonvergent}}
\item \lang{de}{\ref[content_13_reelle_folgen][explizit definierte]{ex:22}}
\item \lang{de}{\ref[content_15_monotone_konvergenz][Folgengrenzwert $e$]{sec:eulersche-zahl}}
\item \lang{de}{\ref[content_13_reelle_folgen][geometrische]{ex:22}}
\item \lang{de}{\ref[content_14_konvergenz][Grenzwert, Limes]{def:folgenkonvergent}}
\item \lang{de}{\ref[content_14_konvergenz][Grenzwertregeln]{sec:grenzwertregeln}}
\item \lang{de}{\ref[content_14_konvergenz][harmonische]{ex:einfache-folgen}}
\item \lang{de}{\ref[content_52_Abstaende][im $\R^n$]{def:folge_im_R_n}}
\item \lang{de}{\ref[content_14_konvergenz][konvergente]{def:folgenkonvergent}}
\item \lang{de}{\ref[content_20_komplexe_folgen][komplexe und konvergente]{def:konvergent-komplexe-folge}}
\item \lang{de}{\ref[content_20_komplexe_folgen][Konvergenzkriterium und Grenzwertregeln, komplex]{sec:konvergenzkriterium}}
%\item \lang{de}{\ref[content_13_reelle_folgen][Monotonie, Beschränktheit]{ex:monotone-folgen}}
\item \lang{de}{\ref[content_13_reelle_folgen][monotone]{def:monotone_Folge}}
\item \lang{de}{\ref[content_24_reihen_und_konvergenz][Partialsummenfolge]{def:reihe}}
\item \lang{de}{\ref[content_19_bestimmte_divergenz][Rechenregeln für bestimmt divergente ]{sec:rechenregeln}}
\item \lang{de}{\ref[content_13_reelle_folgen][rekursiv definierte]{ex:33}}
\item \lang{de}{\ref[content_16_konvergenzkriterien][Sandwich-Lemma]{thm:sandwich}}
\item \lang{de}{\ref[content_16_konvergenzkriterien][Satz von Bolzano-Weierstrass]{thm:bolzano-weierstrass}}
\item \lang{de}{\ref[content_16_konvergenzkriterien][Teilfolge]{def:teilfolge}}
\end{itemize}
%
%
\lang{de}{Folgenstetigkeit}
\begin{itemize}
\item \lang{de}{Folgenstetigkeit \ref[content_29_stetigkeit_definitionen][(in $\R$), ]{sec:folgenkriterium} \ref[content_53_Stetigkeit][ (im $\R^n$)]{thm:äquivalent-zu-stetig}}
\end{itemize}
%
\lang{de}{Folgerung}
\begin{itemize}
\item \lang{de}{\ref[content_04_aussagen_aequivalenzumformungen][Folgerung]{def:implikation}}
% \item \lang{de}{\ref[content_04_aussagen_aequivalenzumformungen][--, logische Verknüpfung von Aussagen]{def:objekte_aussagenlogik}}
% \item \lang{de}{\ref[content_04_aussagen_aequivalenzumformungen][--, Wahrheitstafel]{ex:implication}}
\end{itemize}
%
\lang{de}{Fundamentalsatz der Algebra}
\begin{itemize}
\item \lang{de}{\ref[content_36_anwendungen][Fundamentalsatz der Algebra]{sec:fundamentalsatzAlgebra}}
\item \lang{de}{\ref[content_09neu_komplexeZahlen_hauptsatz][mit Faktorisierung]{sec:fundamentalsatz-der-algebra}}
\end{itemize}
%
\lang{de}{Funktion}
\begin{itemize}
\item \lang{de}{\ref[content_04_taylor_polynom][analytische]{def:reell-analyt-Fktn}}
\item \lang{de}{\ref[content_12_reelle_funktionen_monotonie][beschränkte]{def:bounded}}
\item \lang{de}{\ref[content_10_polynomdivision][echt gebrochen rationale]{sec:rationale_fkt}}
\item \lang{de}{\ref[content_15_exponentialfunktionen][e-Funktion]{sec:nat-exp-fct}}
\item \lang{de}{\ref[content_15_exponentialfunktionen][Exponentialfunktion]{sec:exp_fkt}}
\item \lang{de}{\ref[content_06_funktionsbegriff_und_lineare_funktionen][Funktionsgraph]{def:graph}}
\item \lang{de}{\ref[content_10_polynomdivision][gebrochen rationale]{sec:rationale_fkt}}
\item \lang{de}{\ref[content_06_funktionsbegriff_und_lineare_funktionen][lineare (Gerade)]{sec:linear}}
\item \lang{de}{\ref[content_16_logarithmen][Logarithmusfunktion]{sec:log_fkt}}
\item \lang{de}{\ref[content_12_reelle_funktionen_monotonie][monotone]{def:monotonie}}
\item \lang{de}{\ref[content_09_polynome][Polynomfunktion]{sec:polynomfkt}}
\item \lang{de}{\ref[content_08_quadratische_funktionen][quadratische (Parabel)]{sec:quadratic}}
\item \lang{de}{\ref[content_06_funktionsbegriff_und_lineare_funktionen][reelle]{sec:funktion}}
\item \lang{de}{trigonometrische  \ref[content_19_allgemeiner_sinus_cosinus][(Sinus-, Kosinusfunktion)]{def_trig1}, 
\ref[content_19_allgemeiner_sinus_cosinus][Tangens-, Kotangensfunktion)]{def:tan_cot}}
%\item \lang{de}{\ref[content_06_funktionsbegriff_und_lineare_funktionen][Bildmenge]{sec:funktion}}
%\item \lang{de}{\ref[content_06_funktionsbegriff_und_lineare_funktionen][Definitionsbereich]{sec:funktion}}
\item \lang{de}{\ref[content_21_kettenregel][Verkettung von Funktionen]{addition.definition.1}}
%\item \lang{de}{\ref[content_06_funktionsbegriff_und_lineare_funktionen][Wertemenge]{sec:funktion}}
\item \lang{de}{\ref[content_13_wurzelfunktionen][Wurzelfunktion]{sec:wurzel_fkt}}
%\item \lang{de}{\ref[content_06_funktionsbegriff_und_lineare_funktionen][Zielbereich]{sec:funktion}}
\end{itemize}
%
\lang{de}{Funktionalgleichung}
\begin{itemize}
\item \lang{de}{\ref[content_26_produkt_von_reihen][Funktionalgleichung der Exponentialfunktion]{ex:funktionalgleichung-exp}}
\end{itemize}
%
\lang{de}{Funktionalmatrix}
\begin{itemize}
\item \lang{de}{\ref[content_54_Differentiation][Funktionalmatrix]{def:part_abl}}
\end{itemize}

%
\anchor{sec:g}{\textbf{G}}
\\
%
\lang{de}{Gauß-Algorithmus, Gauß-Verfahren}
\begin{itemize}
\item \lang{de}{\ref[content_41_gauss_verfahren][Gauß-Verfahren]{rule:gauss}}
\item \lang{de}{\ref[content_41_gauss_verfahren][elementare Zeilenumformungen]{def:elementare_Zeilenumformungen}}
%\ref[content_41_gauss_verfahren][-- mit elementaren Umformungen zur Lösung linearer Gleichungssysteme]{ezu}
\item \lang{de}{\ref[content_41_gauss_verfahren][erweiterte Koeffizientenmatrix]{sec:gauss-mit-matrizen}}
\end{itemize}
%

\lang{de}{Gauß-Klammer}
\begin{itemize}
\item \lang{de}{\ref[content_12_reelle_funktionen_monotonie][Gauß-Klammer-Funktion]{ex:unstetige-funktionen}}
\item \lang{de}{\ref[content_29_stetigkeit_definitionen][unstetig]{ex:gauss-klammer}}
\end{itemize}
%
\lang{de}{Gaußsche Zahlenebene}
\begin{itemize}
\item \lang{de}{\ref[content_08aneu_komplexeZahlen_intro][Gaußsche Zahlenebene]{def:Gausssche_Ebene}}
\end{itemize}
%
\lang{de}{gebrochen rationale Funktionen}
\begin{itemize}
\item \ref[content_31_grenzwerte_von_funktionen][Asymptoten]{sec:asymptotics_rational}
\item \lang{de}{\ref[content_10_polynomdivision][Definitionslücke]{def:singularity}}
\item \lang{de}{\ref[content_10_polynomdivision][echt gebrochen rationale Funktion]{def:gebr_rat_fkt}}
\item \lang{de}{\ref[content_32_grenzwert_gegen_unendlich][hebbare Singularität]{rule:rat_fct_def_gaps}}
\item \lang{de}{\ref[content_10_polynomdivision][gebrochen rationale Funktion, rationale Funktion]{def:gebr_rat_fkt}}
\item \lang{de}{\ref[content_10_polynomdivision][Polstelle]{def:pole}}
\end{itemize}
%
\lang{de}{Gegenkathete}
\begin{itemize}
\item \lang{de}{\ref[content_17_trigonometrie_im_dreieck][Gegenkathete]{def:rechtw_Dreieck}}
\end{itemize}
%
%
\lang{de}{geometrisches Mittel}
\begin{itemize}
\item \lang{de}{\ref[content_02_vollstaendige_induktion][geometrisches Mittel]{rule:means_ineq}}
\end{itemize}
%
\lang{de}{geometrische Reihe}
\begin{itemize}
\item \lang{de}{\ref[content_24_reihen_und_konvergenz][geometrische Reihe]{ex:konvergenz-geo-reihe}}
\item \lang{de}{\ref[content_27_konvergenzradius][Konvergenzradius]{ex:geom-und-exp-reihe}}
\end{itemize}
%
\lang{de}{geometrische Summenformel}
\begin{itemize}
\item \lang{de}{\ref[content_02_vollstaendige_induktion][geometrische Summenformel]{rule:geom-summe}}
\end{itemize}
%
\lang{de}{Geraden}
\begin{itemize}
\item \lang{de}{\ref[content_06_funktionsbegriff_und_lineare_funktionen][Geradengleichung]{sec:linear}}
\item \lang{de}{\ref[content_36_normalenformen][Hessesche Normalenform]{def:hesse_nf_gerade}}
\item \lang{de}{\ref[content_36_normalenformen][Koordinatenform]{def:koord_form_gerade}}
\item \lang{de}{\ref[content_36_normalenformen][Normalenvektor]{def:norm_vec_gerade}}
\item \lang{de}{\ref[content_36_normalenformen][Normaleneinheitsvektor]{def:norm_vec_gerade}}
\item \lang{de}{\ref[content_36_normalenformen][Parameterform]{rule:zwei_pkt_geraden}}
\item \lang{de}{\ref[content_36_normalenformen][Punkt-Normalenform]{def:pkt_norm_form_gerade}}
\item \lang{de}{\ref[content_07_geradenformen][Punkt-Steigungsform der Geradengleichung]{sec:punkt_steig_form}}
\item \lang{de}{\ref[content_36_normalenformen][Umrechnung Darstellungsformen Gerade]{sec:umrechng_geradenform}}
\item \lang{de}{\ref[content_35_parameterformen][vektorielle Punkt-Richtungsdarstellung]{rule:pkt_richt_geraden}}
\item \lang{de}{\ref[content_35_parameterformen][vektorielle Zwei-Punkte-Darstellung]{rule:zwei_pkt_geraden}}
\item \lang{de}{\ref[content_07_geradenformen][Zweipunktform der Geradengleichung]{sec:zweipunktform}}
\end{itemize}
%
\lang{de}{gewöhnliche Differentialgleichungen 1. Ordnung}
\begin{itemize}
\item \lang{de}{\ref[content_57_gewoehnliche_DGL_erster_Ordnung][gewöhnliche Differentialgleichungen 1. Ordnung]{def:ODE}}
\item \lang{de}{\ref[content_57_gewoehnliche_DGL_erster_Ordnung][Lösbarkeit des gewöhnlichen Anfangswertproblems 1. Ordnung]{thm:ODE}}
\end{itemize}

%
\lang{de}{gewöhnliches Anfangswertproblem 1. Ordnung}
\begin{itemize}
\item \lang{de}{\ref[content_57_gewoehnliche_DGL_erster_Ordnung][gewöhnliches Anfangswertproblem 1. Ordnung]{thm:ODE}}
\end{itemize}
%
\lang{de}{Gleichung}
\begin{itemize}
\item \lang{de}{\ref[content_16_logarithmen][Exponentialgleichung]{rule:exp_glg}}
\item \lang{de}{\ref[content_06_funktionsbegriff_und_lineare_funktionen][Geradengleichung]{def:gerade}}
\item \lang{de}{\ref[content_05_loesen_gleichungen_und_lgs][lineare Gleichung]{sec:linear}}
\item \lang{de}{\ref[content_16_logarithmen][Logarithmusgleichung]{rule:log_glg}}
\item \lang{de}{\ref[content_16_logarithmen][Potenzgleichung]{rule:potenz_glg}}
\item \lang{de}{\ref[content_05_loesen_gleichungen_und_lgs][quadratische Gleichung]{sec:quadratic}}
\end{itemize}
%
\lang{de}{Gleichungssystem, lineares}
\begin{itemize}
\item \lang{de}{\ref[content_05_loesen_gleichungen_und_lgs][--, Definition LGS]{def:lgs}}
\item \lang{de}{siehe auch lineares Gleichungssystem}
% \item \lang{de}{\ref[content_40_lineare_gleichungssysteme][--, homogen]{def:homog_lgs}}
% \item \lang{de}{\ref[content_40_lineare_gleichungssysteme][--, inhomogen]{def:homog_lgs}}
% \item \lang{de}{\ref[content_40_lineare_gleichungssysteme][--, Koeffizientenmatrix]{def:koeffizientenmatrix}}
% \item \lang{de}{\ref[content_41_gauss_verfahren][--, Lösbarkeit eines LGS]{rule:loesung-parametrisiert}}
% \item \lang{de}{\ref[content_45_matrixrang][--, Lösbarkeit eines LGS und Rang]{thm:rang_loesg_lgs}}
% \item \lang{de}{\ref[content_05_loesen_gleichungen_und_lgs][--, lösen mittels Additionsverfahren (einfach)]{alg:additionsverfahren}}
% \item \lang{de}{\ref[content_05_loesen_gleichungen_und_lgs][--, lösen mittels Einsetzungsverfahren]{alg:einsetzungsverfahren}}
% \item \lang{de}{\ref[content_41_gauss_verfahren][--, lösen mittels Gauß-Verfahren und elementaren Umformungen]{ezu}}
% \item \lang{de}{\ref[content_41_gauss_verfahren][--, lösen mittels Gauß-Verfahren mit Matrizen]{sec:gauss-mit-matrizen}}
% \item \lang{de}{\ref[content_05_loesen_gleichungen_und_lgs][--, lösen mittels Gleichsetzungsverfahren]{alg:gleichsetzungsverfahren}}
% \item \lang{de}{\ref[content_40_lineare_gleichungssysteme][--, Lösungsmenge allgemein]{def:loesmenge_lgs}}
% \item \lang{de}{\ref[content_40_lineare_gleichungssysteme][--, Lösungsmenge homogener LGS]{sec:homogen}}
% \item \lang{de}{\ref[content_40_lineare_gleichungssysteme][--, Lösungsmenge inhomogener LGS]{sec:inhomogen}}
% \item \lang{de}{\ref[content_41_gauss_verfahren][--, reduzierte Stufenform eines LGS]{def:stufenformen}}
% \item \lang{de}{\ref[content_41_gauss_verfahren][--, Stufenform eines LGS]{def:stufenformen}}
\end{itemize}
%
\lang{de}{Gradient}
\begin{itemize}
\item \lang{de}{ \ref[content_54_Differentiation][Gradient]{def:gradient} }
\end{itemize}
%
\lang{de}{Gradmaß}
\begin{itemize}
\item \lang{de}{\link{content_18_grad_und_bogenmass}{Gradmaß}}
\end{itemize}
%
\lang{de}{Gram-Schmidt-Verfahren}
\begin{itemize}
\item \lang{de}{\ref[content_10c_Orthogonalbasen][Gram-Schmidt-Verfahren]{gram_schmidt}}
\end{itemize}
%
\lang{de}{Graph}
\begin{itemize}
\item \lang{de}{\ref[content_10_abbildungen_verkettung][Graph]{sec:graphik}}
\end{itemize}
%
%
\lang{de}{Grenzwert}
\begin{itemize}
\item \lang{de}{\ref[content_31_grenzwerte_von_funktionen][einseitig, links- und rechtsseitig]{sec:einseitige-grenzwerte}}
\item \lang{de}{\ref[content_34_exp_und_log][e-Funktion, asymptotisch]{thm:exp_asymptotik}}
\item \lang{de}{\ref[content_14_konvergenz][Folgengrenzwert]{thm:eindeutigkeitgw}}
\item \lang{de}{\ref[content_31_grenzwerte_von_funktionen][Funktionengrenzwert]{def:funktionsgrenzwert}}
\item \lang{de}{\ref[content_31_grenzwerte_von_funktionen][Grenzwertregeln]{sec:grenzwertregeln}}
\end{itemize}

%
\lang{de}{Grenzwertregeln}
\begin{itemize}
\item \lang{de}{\ref[content_31_grenzwerte_von_funktionen][Grenzwertregeln (Grenzwertsätze)]{sec:grenzwertregeln}}
\item \lang{de}{\ref[content_19_bestimmte_divergenz][Rechenregeln für bestimmte Divergenz ]{sec:rechenregeln}}
 
\end{itemize}

%
\lang{de}{größter gemeinsamer Teiler (ggT)}
\begin{itemize}
\item \lang{de}{\ref[content_03_bruchrechnung][Bestimmung mittels Primfaktorzerelegung]{ggT}}
\item \lang{de}{\ref[content_03_bruchrechnung][Bestimmung mittels Euklidischem Algorithmus]{euklid_algorithmus}}
\end{itemize}
%
\lang{de}{Grundrechenarten}
\begin{itemize}
\item \lang{de}{\ref[content_02_rechengrundlagen_terme][Grundrechenarten]{sec:grundrechenarten}}
\end{itemize}
%
\anchor{sec:h}{\textbf{H}}
\\
%
\lang{de}{Häufungspunkt}
\begin{itemize}
\item \lang{de}{\ref[content_16_konvergenzkriterien][Häufungspunkt, Teilfolgen]{ex:teilfolge-hpunkt}}
\end{itemize}
%
\lang{de}{harmonische Folge}
\begin{itemize}
\item \lang{de}{\ref[content_14_konvergenz][harmonische Folge]{ex:einfache-folgen}}
\end{itemize}
%
\lang{de}{harmonische Reihe}
\begin{itemize}
\item \lang{de}{\ref[content_24_reihen_und_konvergenz][harmonische Reihe]{ex:erste-reihen}}
\item \lang{de}{\ref[content_24_reihen_und_konvergenz][alternierende harmonische Reihe]{ex:alter-harm-reihe}}
\end{itemize}
%
%
\lang{de}{Hauptachsentransformation}
\begin{itemize}
\item \lang{de}{\ref[content_12_symmetrische_matrizen][Hauptachsentransformation]{thm:hauptachsentransformation}}
\end{itemize}
%
\lang{de}{Hauptnenner}
\begin{itemize}
\item \lang{de}{\ref[content_03_bruchrechnung][Hauptnenner eines Bruchs, Bruchterms]{kgV}}
\end{itemize}
%
\lang{de}{Hauptsatz der Differential- und Integralrechnung}
\begin{itemize}
\item \lang{de}{\ref[content_25_stammfunktion][Hauptsatz der Differential- und Integralrechnung (Teil 1)]{bestimmt}}
\item \lang{de}{\ref[content_09_integrierbare_funktionen][Hauptsatz der Differential- und Integralrechnung (Teil 3a) mit Beweis]{thm:fundamental}}
\end{itemize}
%
\lang{de}{hebbare Singularität}
\begin{itemize}
\item \lang{de}{\ref[content_32_grenzwert_gegen_unendlich][hebbare Singularität]{rule:rat_fct_def_gaps}}
\end{itemize}
%
\lang{de}{Heron-Verfahren}
\begin{itemize}
\item \lang{de}{\ref[content_05_newtonverfahren][Heron-Verfahren]{rem:heron-verfahren}}
\end{itemize}
%
\lang{de}{Hessesche Normalenform}
\begin{itemize}
\item \lang{de}{\ref[content_36_normalenformen][einer Ebene]{def:hesse_nf_ebene}}
\item \lang{de}{\ref[content_36_normalenformen][einer Geraden]{def:hesse_nf_gerade}}
\end{itemize}
%
\lang{de}{hinreichende Bedingung für}
\begin{itemize}
\item \lang{de}{\ref[content_22_extremstellen][für lokale Extrema]{sec:hinr_bedg_lok_extremum}}
\item \lang{de}{\ref[content_23_kurvendiskussion][für  Wendestellen]{thm:notw_hinr_bedg_wendepkt}}
\end{itemize}
%
\lang{de}{homogenes lineares Gleichungssystem}
\begin{itemize}
\item \lang{de}{\ref[content_40_lineare_gleichungssysteme][homogenes LGS]{def:homog_lgs}}
\item \lang{de}{\ref[content_40_lineare_gleichungssysteme][Lösungsmenge homogener LGS]{sec:homogen}}
\item \lang{de}{\ref[content_40_lineare_gleichungssysteme][triviale Lösung homogener LGS]{sec:homogen}}
\end{itemize}
%
\lang{de}{Hospital}
\begin{itemize}
\item \lang{de}{\ref[content_06_de_l_hospital][Regel von de l'Hospital]{sec:lHospital}}
% \item \lang{de}{\ref[content_06_de_l_hospital][\glqq $\frac{0}{0}$\grqq-Regel von de l'Hospital (für $x$ gegen $x_0$ )]{thm:de-l-Hospital_1}}
% \item \lang{de}{\ref[content_06_de_l_hospital][\glqq $\frac{\infty}{\infty}$\grqq-Regel von de l'Hospital (für $x$ gegen $x_0$ )]{thm:de-l-hospital}}
% \item \lang{de}{\ref[content_06_de_l_hospital][Regel von de l'Hospital (für $x$ gegen $\pm\infty$ )]{thm:de-l-hospital}}
\end{itemize}
%
\lang{de}{Horner Schema}
\begin{itemize}
\item \lang{de}{\ref[content_10_polynomdivision][Horner-Schema]{sec:poly-div-mit-horner}}
\end{itemize}
%
\lang{de}{Hypotenuse}
\begin{itemize}
\item \lang{de}{\ref[content_17_trigonometrie_im_dreieck][Hypotenuse]{def:rechtw_Dreieck}}
\end{itemize}

%
\anchor{sec:i}{\textbf{I}}
\\
%
\lang{de}{$i$, imaginäre Einheit}
\begin{itemize}
\item \lang{de}{\ref[content_08aneu_komplexeZahlen_intro][imaginäre Einheit $i$]{def:imaginäre_Einheit}}
\end{itemize}
%
\lang{de}{Imaginärteil}
\begin{itemize}
\item \lang{de}{\ref[content_08aneu_komplexeZahlen_intro][Imaginärteil]{def:real-imaginaer-teil}}
\end{itemize}
%
\lang{de}{Implikation}
\begin{itemize}
\item \lang{de}{\ref[content_04_aussagen_aequivalenzumformungen][Implikation]{def:implikation}}
% \item \lang{de}{\ref[content_04_aussagen_aequivalenzumformungen][--, logische Verknüpfung von Aussagen]{def:objekte_aussagenlogik}}
% \item \lang{de}{\ref[content_04_aussagen_aequivalenzumformungen][--, Wahrheitstafel]{ex:implication}}
\end{itemize}
%
\lang{de}{Indexverschiebung}
\begin{itemize}
\item \lang{de}{\ref[content_02_rechengrundlagen_terme][Indexverschiebung]{Indexverschiebung}}
\end{itemize}
%
%
\lang{de}{Induktion}
\begin{itemize}
\item \lang{de}{\ref[content_02_vollstaendige_induktion][vollständige Induktion]{sec:induktionsprinzip}}
\end{itemize}
%
\lang{de}{Infimum}
\begin{itemize}
\item \lang{de}{\ref[content_07_vollstaendigkeit][Infimum]{def:sup_inf}}
\end{itemize}
%
\lang{de}{inhomogenes lineares Gleichungssystem}
\begin{itemize}
\item \lang{de}{\ref[content_40_lineare_gleichungssysteme][inhomogenes LGS]{def:homog_lgs}}
\item \lang{de}{\ref[content_40_lineare_gleichungssysteme][Lösungsmenge inhomogener LGS]{sec:inhomogen}}
\end{itemize}
%
\lang{de}{Injektivität}
\begin{itemize}
\item \lang{de}{\ref[content_11_injektiv_surjektiv_bijektiv][injektiv]{sec:inj-sur-bi}}
\end{itemize}
%
\lang{de}{innere Funktion}
\begin{itemize}
\item \lang{de}{\ref[content_21_kettenregel][innere Funktion]{addition.definition.1}}
\end{itemize}
%
\lang{de}{innerer Punkt}
\begin{itemize}
\item \lang{de}{\ref[content_22_offene_abgeschlossene_teilmengen][innerer Punkt]{sec:innere-randpunkte}}
\end{itemize}
%
\lang{de}{Integral}
\begin{itemize}
\item \lang{de}{\ref[content_24_integral_als_flaeche][bestimmtes]{bestimmtes_Integral}} 
\item \lang{de}{\ref[content_08_integral_eigenschaften][Integral einer Funktion]{def:integral}}
\item \lang{de}{\ref[content_10_uneigentliches_integral][uneigentliches (über unbeschränkte Intervall)]{def:uneig-int-interval}}
\item \lang{de}{\ref[content_10_uneigentliches_integral][uneigentliches (über Definitionslücke)]{def:uneig-int-def-luecke}}
\item \lang{de}{\ref[content_25_stammfunktion][unbestimmtes]{def:unbest_Integral}}
\end{itemize}
%
\lang{de}{Integraleigenschaften}
\begin{itemize}
\item \lang{de}{\ref[content_24_integral_als_flaeche][Additivität bzgl. der Integrationsgrenzen (Teil 1)]{thm:additiv},
                \ref[content_08_integral_eigenschaften][(Teil 3a)]{eigenschaften}}
\item \lang{de}{\ref[content_08_integral_eigenschaften][Linearität]{Linear_Integral}}                
\item \lang{de}{\ref[content_24_integral_als_flaeche][Tausch von Integrationsgrenzen]{def:Integr_grenz_tausch}}
\end{itemize}
%
\lang{de}{Integrand}
\begin{itemize}
\item \lang{de}{\ref[content_24_integral_als_flaeche][Integrand]{bestimmtes_Integral}}
\end{itemize}
%
\lang{de}{Integrationsgrenzen}
\begin{itemize}
\item \lang{de}{\ref[content_24_integral_als_flaeche][eines bestimmten Integrals]{bestimmtes_Integral}}
\end{itemize}
%
\lang{de}{Integrationsmethoden}
\begin{itemize}
\item \lang{de}{\ref[content_12_substitutionsregel][Substitutionsregel]{sec:allg-subst-reg}}
\item \lang{de}{\ref[content_26_flaechen_zwischen_graphen][lineare Substitution (Teil 1)]{subst},
                \ref[content_12_substitutionsregel][(Teil 3a)]{{rule:lin-subst}}}
\item \lang{de}{\ref[content_13_partialbruchzerlegung][Partialbruchzerlegung (Verfahren)]{sec:summary}}
\item \lang{de}{\ref[content_11_partielle_integration][partielle Integration]{sec:part-int}}
\item \lang{de}{\ref[content_11_partielle_integration][partielle Integration mit anschließender Gleichungsauflösung]{sec:part-int-gleich-aufl}}
\end{itemize}
%
\lang{de}{Integrationsvariable}
\begin{itemize}
\item \lang{de}{\ref[content_24_integral_als_flaeche][Integrationsvariable]{bestimmtes_Integral}}
\end{itemize}
%
\lang{de}{Integrierbarkeit}
\begin{itemize}
\item \lang{de}{\ref[content_08_integral_eigenschaften][einer Funktion auf einem Intervall]{def:integral}}
\item \lang{de}{\ref[content_08_integral_eigenschaften][stückweise stetiger Funktionen]{sec:stet-u-int}}
\end{itemize}
%
\lang{de}{Intervall}
\begin{itemize}
\item \lang{de}{\ref[content_01_zahlenmengen][abgeschlossenes]{def:intervall}}
\item \lang{de}{\ref[content_01_zahlenmengen][ausgeartetes]{rem:intervall}}
\item \lang{de}{\ref[content_01_zahlenmengen][beschränktes]{rem:intervall}}
\item \lang{de}{\ref[content_01_zahlenmengen][beidseitig unbeschränktes]{rem:intervall}}
\item \lang{de}{\ref[content_21_intervalle][endliches]{section.intervals}}
%\item \lang{de}{\ref[content_21_intervalle][Epsilon-Umgebung]{sec:epsilon-umg}}
\item \lang{de}{\ref[content_01_zahlenmengen][offenes]{def:intervall}}
\item \lang{de}{\ref[content_01_zahlenmengen][halboffenes (linksoffenes, rechtsoffenes)]{def:intervall}}
\item \lang{de}{\ref[content_01_zahlenmengen][nach oben/unten unbeschränktes]{rem:intervall}}
\item \lang{de}{\ref[content_21_intervalle][unendliches]{sec:unendl-intervalle}}
\item \lang{de}{\ref[content_21_intervalle][Intervalle, Durchschnitte von]{thm:intervallschnitt}}
\end{itemize}
%
\lang{de}{Intervallschachtelung}
\begin{itemize}
\item \lang{de}{\ref[content_23_intervallschachtelung][Bisektionsverfahren, Intervallhalbierung]{sec:intervallhalbierung}}
\item \lang{de}{\ref[content_23_intervallschachtelung][Intervallschachtelungsprinzip]{thm:intervallschachtelungsprinzip}}
\end{itemize}
%
\lang{de}{Intervall-Zerlegung}
\begin{itemize}
\item \lang{de}{\ref[content_07_ober_und_untersumme][Zerlegung eines Intervalls in Teilintervalle]{def:intervall-zer}}
\item \lang{de}{\ref[content_07_ober_und_untersumme][äquidistante Zerlegung eines Intervalls]{ex:aequidistante-zerlegung}}
\item \lang{de}{\ref[content_07_ober_und_untersumme][Ober- und Untersumme einer Funktion zu einer Intervall-Zerlegung]{def:Obersum-untersum}}
\item \lang{de}{\ref[content_07_ober_und_untersumme][Riemannsche Zwischensummen einer Funktion zu einer Intervall-Zerlegung]{def:riemann-zwisch-sum}}
\end{itemize}
%
\lang{de}{inverse Matrix}
\begin{itemize}
\item \lang{de}{inverse Matrix, Inverse \ref[content_14_quadratische_matrizen][(reell),]{def:invertierbar} \ref[content_08_inverse_matrix][(über beliebigen Körpern)]{rule:invertierbarkeit-rang}}
\item \lang{de}{Rechenregeln für inverse Matrizen \ref[content_15_inverse_matrix][(reell),]{sec:rechenregeln} \ref[content_08_inverse_matrix][(über beliebigen Körpern)]{sec:rechenregeln}}
\item \lang{de}{Cramersche Regel \ref[content_17_cramersche_regel][(reell),]{rule:inverse_via_cramer} \ref[content_10_cramersche_regel][ (über beliebigen Körpern)]{rule:inverse_via_cramer}}
\end{itemize}

%
\lang{de}{invertierbare Matrix}
\begin{itemize}
\item \lang{de}{invertierbare Matrix \ref[content_14_quadratische_matrizen][(reell),]{def:invertierbar} \ref[content_07_quadratische_matrizen][(über beliebigen Körpern)]{sec:invertierbare-matrizen}}
\end{itemize}
%
\anchor{sec:j}{\textbf{J}}
\\
%
\lang{de}{Jacobi-Matrix}
\begin{itemize}
\item \lang{de}{\ref[content_54_Differentiation][Jakobi-Matrix]{def:part_abl}}
\end{itemize}

%
\lang{de}{Junktor}
\begin{itemize}
%\item \lang{de}{\ref[content_04_aussagen_aequivalenzumformungen][Äquivalenz]{ex:equivalence}}
%\item \lang{de}{\ref[content_04_aussagen_aequivalenzumformungen][Disjunktion]{ex:disjunktion}}
%\item \lang{de}{\ref[content_04_aussagen_aequivalenzumformungen][Folgerung]{ex:implication}}
%\item \lang{de}{\ref[content_04_aussagen_aequivalenzumformungen][Implikation]{ex:implication}}
%\item \lang{de}{\ref[content_04_aussagen_aequivalenzumformungen][Konjunktion]{ex:konjunktion}}
\item \lang{de}{\ref[content_04_aussagen_aequivalenzumformungen][Junktor]{def:objekte_aussagenlogik}}
%\item \lang{de}{\ref[content_04_aussagen_aequivalenzumformungen][Negation]{ex:negation}}
\end{itemize}
%
%
\anchor{sec:k}{\textbf{K}}
\\
%
$\K$
\begin{itemize}
\item \lang{de}{\ref[content_04_koerperaxiome][$\K$ (Körper)]{def:koerper}}
\end{itemize}

\lang{de}{kartesisches Produkt}
\begin{itemize}
\item \lang{de}{\ref[content_01_zahlenmengen][kartesisches Produkt]{def:mengenoperationen}}
\end{itemize}
%
\lang{de}{Kathete}
\begin{itemize}
\item \lang{de}{\ref[content_17_trigonometrie_im_dreieck][Kathete]{def:rechtw_Dreieck}}
\end{itemize}
%
\lang{de}{Kegelschnitt}
\begin{itemize}
\item \lang{de}{\ref[content_12_symmetrische_matrizen][Kegelschnitt]{rem:kegelschnitte}}
\end{itemize}

%
\lang{de}{Kettenregel der Ableitung}
\begin{itemize}
\item \lang{de}{\ref[content_21_kettenregel][eindimensional (Teil 1)]{rule:kettenregel},
                \ref[content_02_ableitungsregeln][(Teil 3a)]{sec:kettenregel}}
\item\lang{de}{\ref[content_54_Differentiation][mehrdimensional]{thm:kettenregel_mehrdim}}
\end{itemize}
%


\lang{de}{kleinstes gemeinsames Vielfaches (kgV)}
\begin{itemize}
\item \lang{de}{\ref[content_03_bruchrechnung][mittels Primfaktorzerlegung bestimmen]{kgV}}
\end{itemize}
%
\lang{de}{Koeffizienten}
\begin{itemize}
\item \lang{de}{\ref[content_29_linearkombination][einer Linearkombination im $\R^n$]{def:linearcomb}}
\item \lang{de}{\ref[content_10a_vektorraum][einer Linearkombination in einem $\K$-Vektorraum]{def:linearkombination}}
\item \lang{de}{\ref[content_39_matrizen][einer Matrix]{Matrix}}
\end{itemize}
%
\lang{de}{Koeffizientenmatrix}
\begin{itemize}
\item \lang{de}{Koeffizientenmatrix \ref[content_40_lineare_gleichungssysteme][(Teil 1), ]{def:koeffizientenmatrix} \ref[content_04_lgs][ (Teil 3b)]{def:koeffizientenmatrix}}
\end{itemize}
%
\lang{de}{Koeffizientenvektor}
\begin{itemize}
\item \lang{de}{\ref[content_10a_vektorraum][Koeffizientenvektor]{rem:koeffizientenvektor}}
\end{itemize}
%
\lang{de}{Körper}
\begin{itemize}
\item \lang{de}{\ref[content_04_koerperaxiome][Körperaxiome]{sec:axiome}}
\item \lang{de}{\ref[content_05_anordnungsaxiome][angeordneter]{rule:anordnungsregeln}}
\item \lang{de}{\ref[content_08aneu_komplexeZahlen_intro][Körper der komplexen Zahlen]{sec:koerper-komplexe-zahlen}}
%\item \lang{de}{\ref[content_04_koerperaxiome][--Rechenregeln]{sec:rechenregeln}}
%\item \lang{de}{\ref[content_05_anordnungsaxiome][Ungleichungsregeln und Betrag]{rule:regeln}}
\end{itemize}
%
\lang{de}{kollinear}
\begin{itemize}
\item \lang{de}{\ref[content_27_vektoren][kollineare Vektoren]{def:kollinear}}
\end{itemize}
%
\lang{de}{Kommutativgesetz}
\begin{itemize}
\item \lang{de}{\ref[content_02_rechengrundlagen_terme][Kommutativgesetz]{rule:rechengesetze}}
\end{itemize}
%
\lang{de}{Komplement}
\begin{itemize}
\item \lang{de}{\ref[content_01_zahlenmengen][Komplement, Komplementärmenge]{def:mengenoperationen}}
\end{itemize}
%
\lang{de}{komplexe Zahl}
\begin{itemize}
\item \lang{de}{\ref[content_08bneu_komplexeZahlen_geom][Addition]{sec:add-mult-kompl}}
\item \lang{de}{\ref[content_08bneu_komplexeZahlen_geom][Argument (Phasenwinkel)]{sec:polar}}
\item \lang{de}{\ref[content_08bneu_komplexeZahlen_geom][Betrag]{def:betragkz}}
\item \lang{de}{\ref[content_08aneu_komplexeZahlen_intro][Gaußsche Zahlenebene]{def:Gausssche_Ebene}}
\item \lang{de}{\ref[content_08aneu_komplexeZahlen_intro][imaginäre Einheit $i$]{def:imaginäre_Einheit}}
\item \lang{de}{\ref[content_08aneu_komplexeZahlen_intro][Imaginärteil]{def:real-imaginaer-teil}}
\item \lang{de}{\ref[content_08bneu_komplexeZahlen_geom][Multiplikation]{sec:add-mult-kompl}}
\item \lang{de}{\ref[content_09neu_komplexeZahlen_hauptsatz][Multiplikation in Polarkoordinaten]{sec:multiplikation-polar}}
\item \lang{de}{\ref[content_08aneu_komplexeZahlen_intro][Körper der komplexen Zahlen]{sec:koerper-komplexe-zahlen}}
\item \lang{de}{\ref[content_08aneu_komplexeZahlen_intro][konjungiert komplexe Zahl]{def:real-imaginaer-teil}}
\item \lang{de}{\ref[content_36_anwendungen][Polardarstellung]{sec:polardarst}}
\item \lang{de}{\ref[content_08bneu_komplexeZahlen_geom][Polarkoordinaten]{sec:polar}}
\item \lang{de}{\ref[content_36_anwendungen][Potenz]{def:komplexe-potenz}}
\item \lang{de}{\ref[content_08aneu_komplexeZahlen_intro][Realteil]{def:real-imaginaer-teil}}
\item \lang{de}{\ref[content_08bneu_komplexeZahlen_geom][Tupelschreibweise, Zahlenpaar-Schreibweise]{sec:zahlenpaare}}
\item \lang{de}{\ref[content_36_anwendungen][Wurzeln]{def:komplexe-wurzel}}
\end{itemize}
%
\lang{de}{Komposition}
\begin{itemize}
\item \lang{de}{\ref[content_10_abbildungen_verkettung][Komposition (Hintereinanderausführung, Verkettung)]{kompositionv}}
\end{itemize}
%
\lang{de}{konjungiert komplex}
\begin{itemize}
\item \lang{de}{\ref[content_08aneu_komplexeZahlen_intro][konjungiert komplexe Zahl]{def:real-imaginaer-teil}}
\end{itemize}
%
\lang{de}{Konjunktion}
\begin{itemize}
\item \lang{de}{\ref[content_04_aussagen_aequivalenzumformungen][Konjuktion (logisches Und)]{def:objekte_aussagenlogik}}
%\item \lang{de}{\ref[content_04_aussagen_aequivalenzumformungen][--, Wahrheitstafel]{ex:konjuktion}}
\end{itemize}
%
\lang{de}{Konklusion}
\begin{itemize}
\item \lang{de}{\ref[content_04_aussagen_aequivalenzumformungen][Konklusion]{def:implikation}}
\end{itemize}
%
\lang{de}{Konvergenz}
\begin{itemize}
\item \lang{de}{\ref[content_25_konvergenz_kriterien][absolute Konvergenz]{ex:absolut-konvergenz}}
\item \lang{de}{\ref[content_24_reihen_und_konvergenz][von Reihen]{def:reihenkonvergenz}}
 \item \lang{de}{\ref[content_20_komplexe_folgen][ komplexer Folgen]{def:konvergent-komplexe-folge}}
\item \lang{de}{\ref[content_14_konvergenz][ reeller Folgen]{def:folgenkonvergent}}
\item \lang{de}{\ref[content_52_Abstaende][von Folgen im $\R^n$]{def:folge_im_R_n}}
\end{itemize}
%
%
\lang{de}{Konvergenzradius}
\begin{itemize}
\item \lang{de}{\ref[content_27_konvergenzradius][Konvergenzradius]{def:konvergenzradius}}
\item \lang{de}{\ref[content_27_konvergenzradius][Satz von Cauchy-Hadamard]{thm:Cauchy-Hadamard}}
\item \lang{de}{\ref[content_27_konvergenzradius][Quotientenregel]{thm:quot-regel}}
\end{itemize}
%
\lang{de}{Koordinatenform}
\begin{itemize}
\item \lang{de}{\ref[content_36_normalenformen][einer Ebene]{def:hesse_nf_ebene}}
\item \lang{de}{\ref[content_36_normalenformen][einer Gerade]{def:koord_form_gerade}}
\end{itemize}
%
\lang{de}{Koordinatenvektor}
\begin{itemize}
\item \lang{de}{\ref[content_10a_vektorraum][Koordinatenvektor]{rem:koordinatenvektor}}
\end{itemize}
%
\lang{de}{Kosinus}
\begin{itemize}
\item \lang{de}{\ref[content_17_trigonometrie_im_dreieck][Kosinus (geometrische Definition)]{sin-cos-im-dreieck}}
\item \lang{de}{\ref[content_19_allgemeiner_sinus_cosinus][Kosinus-Funktion]{def_trig1}}
\item \lang{de}{\ref[content_28_exponentialreihe][komplexe Kosinus-Funktion]{sec:sinus-kosinus}}
\item \lang{de}{\ref[content_19_allgemeiner_sinus_cosinus][Nullstellen]{sin_cos_roots}}
\item \lang{de}{\ref[content_19_allgemeiner_sinus_cosinus][Periodizität]{sin_cos_periodic}}
\item \lang{de}{\ref[content_19_allgemeiner_sinus_cosinus][Pythagoras, trigonometrischer]{trig-pythagoras}}
\item \lang{de}{\ref[content_19_allgemeiner_sinus_cosinus][Symmetrie]{sin_cos_symmetry}}
\item \lang{de}{\ref[content_19_allgemeiner_sinus_cosinus][Verschiebung]{sin_cos_verschiebung}}
\end{itemize}
%
\lang{de}{Koordinatentransformation}
\begin{itemize}
\item \lang{de}{\ref[content_10b_lineare_abb][Koordinatentransformation]{sec:basiswechsel}}
\end{itemize}

%
\lang{de}{Kosinus hyperbolicus}
\begin{itemize}
\item \lang{de}{\ref[content_28_exponentialreihe][Kosinus hyperbolicus]{sec:sinh-cosh}}
\end{itemize}
%
\lang{de}{Kotangens}
\begin{itemize}
\item \lang{de}{\ref[content_19_allgemeiner_sinus_cosinus][Kotangens-Funktion]{def:tan_cot}}
\item \lang{de}{\ref[content_19_allgemeiner_sinus_cosinus][Nullstellen der Kotangensfunktion]{tan_roots}}
\item \lang{de}{\ref[content_19_allgemeiner_sinus_cosinus][Periodizität]{tan_eigenschaften}}
\item \lang{de}{\ref[content_19_allgemeiner_sinus_cosinus][Symmetrie]{tan_eigenschaften}}
\item \lang{de}{\ref[content_19_allgemeiner_sinus_cosinus][Verschiebung]{tan_eigenschaften}}
\end{itemize}

%
\lang{de}{Kreuzprodukt}
\begin{itemize}
\item \lang{de}{\ref[content_34_vektorprodukt][Kreuzprodukt (Vektorprodukt)]{def:Vektorprodukt}}
\end{itemize}
%
\lang{de}{kritischer Punkt, kritische Stelle}
\begin{itemize}
\item \lang{de}{\ref[content_22_extremstellen][kritische/stationäre Stelle]{def:krit_pkt}}
\end{itemize}
%
\lang{de}{Krümmung}
\begin{itemize}
\item \lang{de}{\ref[content_23_kurvendiskussion][Krümmung eines Funktionsgraphen]{sec:kruemmung}}
\item \lang{de}{\ref[content_23_kurvendiskussion][Linkskrümmung, Linkskurve]{rule:recht_links_kruemmung}}
\item \lang{de}{\ref[content_23_kurvendiskussion][Rechtskrümmung, Rechtskurve]{rule:recht_links_kruemmung}}
\end{itemize}
%
\lang{de}{Kürzen}
\begin{itemize}
\item \lang{de}{\ref[content_03_bruchrechnung][Kürzen von Brüchen und Bruchtermen]{erweitern_kuerzen}}
\end{itemize}
%
\lang{de}{Kurve}
\begin{itemize}
\item \lang{de}{\ref[content_51_Kurven][Kurve im $\R^n$]{def:kurve}}
\item \lang{de}{\ref[content_51_Kurven][Tangentialvektor]{def:tangentialvektor}}
\item \lang{de}{\ref[content_51_Kurven][Parametrisierung]{rem:parametrisierung}}
\end{itemize}
%
\lang{de}{Kurvendiskussion}
\begin{itemize}
\item \lang{de}{\ref[content_23_kurvendiskussion][Kurvendiskussion]{sec:kurvendiskussion}}
\end{itemize}
%

\anchor{sec:l}{\textbf{L}}
\\
%
\lang{de}{Lagrange}
\begin{itemize}
\item \lang{de}{\ref[content_04_taylor_polynom][Lagrange-Restglied]{thm:rg-Lagrange}}
\end{itemize}
%
\lang{de}{Länge}
\begin{itemize}
\item \lang{de}{\ref[content_32_laenge_norm][Länge eines Vektors]{def:euklidische_norm}}
\end{itemize}
%
\lang{de}{Leibniz-Kriterium}
\begin{itemize}
\item \lang{de}{\ref[content_24_reihen_und_konvergenz][Leibniz-Kriterium]{thm:leibnizkriterium}}
\end{itemize}

%
\lang{de}{Leitkoeffizient}
\begin{itemize}
\item \lang{de}{\ref[content_08_quadratische_funktionen][Leitkoeffizient der Parabelfunktion]{def:quadratic_func}}
\end{itemize}
%
\lang{de}{l'Hospital}
\begin{itemize}
\item \lang{de}{\ref[content_06_de_l_hospital][Regel von de l'Hospital]{sec:lHospital}}
\end{itemize}
%
\lang{de}{Limes}
\begin{itemize}
\item \lang{de}{siehe Grenzwert}
\end{itemize}
%
\lang{de}{lineare Abbhängigkeit}
\begin{itemize}
\item \lang{de}{\ref[content_30_basen_eigenschaften][von Vektoren des $\R^n$]{def_lin_unabh}}
\item \lang{de}{\ref[content_10a_vektorraum][von Vektoren eines  $\K$-Vektorraums ]{def:lineareAbhaengigkeit}}
\end{itemize}
%
\lang{de}{lineare Abbildung}
\begin{itemize}
\item \lang{de}{\ref[content_10b_lineare_abb][lineare Abbildung (Definition)]{def_lin_abb}}
\item \lang{de}{\ref[content_10b_lineare_abb][Abbildungsmatrix]{sec:abbildungsmatrizen}}
\end{itemize}


%
\lang{de}{lineare Funktion}
\begin{itemize}
\item \lang{de}{\ref[content_06_funktionsbegriff_und_lineare_funktionen][lineare Funktion]{sec:linear}}
\end{itemize}
%
\lang{de}{lineare Gleichung}
\begin{itemize}
\item \lang{de}{\ref[content_05_loesen_gleichungen_und_lgs][lineare Gleichung]{sec:linear}}
\end{itemize}
%
\lang{de}{lineares Gleichungssystem}
\begin{itemize}
\item \lang{de}{\ref[content_05_loesen_gleichungen_und_lgs][Definition LGS]{def:lgs}}
\item \lang{de}{\ref[content_06_umformungen_rang][elementare Zeilenumformung durch Matrix-Multiplikation]{sec:zeilenumformungMitMatrixmul}}
\item \lang{de}{homogenes \ref[content_40_lineare_gleichungssysteme][(Teil 1), ]{def:homog_lgs}\ref[content_04_lgs][(Teil 3b)]{def:homogen-inhomogen}}
\item \lang{de}{inhomogenes\ref[content_40_lineare_gleichungssysteme][(Teil 1), ]{def:homog_lgs}\ref[content_04_lgs][(Teil 3b)]{def:homogen-inhomogen}}
\item \lang{de}{Koeffizientenmatrix\ref[content_40_lineare_gleichungssysteme][(Teil 1), ]{def:koeffizientenmatrix}\ref[content_04_lgs][ (Teil 3b)]{def:koeffizientenmatrix}}
\item \lang{de}{\ref[content_05_gaussverfahren][LGS mit mehreren rechten Seiten]{sec:mehrere-rechte-seiten}}
\item \lang{de}{\ref[content_41_gauss_verfahren][Lösbarkeit eines LGS]{rule:loesung-parametrisiert}}
\item \lang{de}{Lösbarkeit von LGS und Rang\ref[content_45_matrixrang][ (Teil 1),]{thm:rang_loesg_lgs}\ref[content_05_gaussverfahren][ (Teil 3b)]{rule:loesung-parametrisiert}}
\item \lang{de}{\ref[content_05_loesen_gleichungen_und_lgs][lösen mittels Additionsverfahren (elementar)]{alg:additionsverfahren}}
\item \lang{de}{lösen mittels Cramerscher Regel \ref[content_17_cramersche_regel][(reell),]{cramersche_regel} \ref[content_10_cramersche_regel][ (über beliebigen Körpern)]{cramersche_regel}}
\item \lang{de}{\ref[content_05_loesen_gleichungen_und_lgs][lösen mittels Einsetzungsverfahren]{alg:einsetzungsverfahren}}
\item \lang{de}{\ref[content_41_gauss_verfahren][lösen mittels Gauß-Verfahren und elementaren Zeilenumformungen]{ezu}}
\item \lang{de}{\ref[content_41_gauss_verfahren][lösen mittels Gauß-Verfahren mit Matrizen]{sec:gauss-mit-matrizen}}
\item \lang{de}{\ref[content_05_loesen_gleichungen_und_lgs][lösen mittels Gleichsetzungsverfahren]{alg:gleichsetzungsverfahren}}
\item \lang{de}{Lösungsmenge eines LGS \ref[content_40_lineare_gleichungssysteme][(Teil 1), ]{def:loesmenge_lgs}\ref[content_04_lgs][(Teil 3b)]{def:loesungsmenge}}
\item \lang{de}{Lösungsmenge homogener LGS \ref[content_40_lineare_gleichungssysteme][(Teil 1), ]{sec:homogen}\ref[content_04_lgs][(Teil 3b)]{sec:homogen}}
\item \lang{de}{Lösungsmenge inhomogener LGS \ref[content_40_lineare_gleichungssysteme][(Teil 1), ]{sec:inhomogen} \ref[content_04_lgs][(Teil 3b)]{sec:inhomogen}}
\item \lang{de}{Rang\ref[content_45_matrixrang][ (Teil 1),]{thm:rang_loesg_lgs}\ref[content_06_umformungen_rang][ (Teil 3b)]{def:vollerRang}}

\item \lang{de}{\ref[content_41_gauss_verfahren][reduzierte Stufenform]{def:stufenformen}}
\item \lang{de}{\ref[content_05_gaussverfahren][Rückwärtseinsetzen]{rule:bestimmungDerLoesungsmenge}}
\item \lang{de}{Stufenform \ref[content_41_gauss_verfahren][ (Teil 1), ]{def:stufenformen}\ref[content_05_gaussverfahren][ (Teil 3b)]{def:stufenformen}}
\end{itemize}
%



\lang{de}{Lineare Gleichungssysteme}
\begin{itemize}


\item \lang{de}{\ref[content_06_umformungen_rang][Elementarmatrix]{rule:elementarmatrizen}}
\item \lang{de}{\ref[content_06_umformungen_rang][Zeilenrang einer Matrix]{def:zeilenSpaltenRang}}
\item \lang{de}{\ref[content_06_umformungen_rang][Spaltenrang einer Matrix]{def:zeilenSpaltenRang}}
\item \lang{de}{\ref[content_06_umformungen_rang][Rang einer Matrix]{def:vollerRang}}

\end{itemize}

\lang{de}{lineare Unabhängigkeit}
\begin{itemize}
\item \lang{de}{\ref[content_30_basen_eigenschaften][von Vektoren des $\R^n$]{def_lin_unabh}}
\item \lang{de}{\ref[content_10a_vektorraum][von Vektoren eines  $\K$-Vektorraums ]{def:lineareAbhaengigkeit}}
\end{itemize}
%
\lang{de}{Linearfaktor}
\begin{itemize}
\item \lang{de}{\ref[content_05_loesen_gleichungen_und_lgs][Linearfaktor]{thm:Satz_von_Vieta}}
\item \lang{de}{\ref[content_36_anwendungen][komplexer Linearfaktor]{sec:fundamentalsatzAlgebra}}
\end{itemize}
%
\lang{de}{Linearfaktorzerlegung}
\begin{itemize}
\item \lang{de}{\ref[content_09neu_komplexeZahlen_hauptsatz][Linearfaktorzerlegung von Polynomen]{sec:fundamentalsatz-der-algebra}}
\end{itemize}
%
\lang{de}{Linearkombination}
\begin{itemize}
\item \lang{de}{\ref[content_29_linearkombination][von Vektoren des $\R^n$]{sec:lin-comb}}
\item \lang{de}{\ref[content_10a_vektorraum][von Vektoren eines $\K$-Vektorraums]{def:linearkombination}}
\end{itemize}
%
\lang{de}{Logarithmus}
\begin{itemize}
\item \lang{de}{\ref[content_34_exp_und_log][beliebiger Basen]{def:Log_arbitrary_base}}
\item \lang{de}{\ref[content_16_logarithmen][natürlicher]{ln}}
\item \lang{de}{\ref[content_16_logarithmen][dekadischer, Zehner-]{ln}}
\item \lang{de}{\ref[content_16_logarithmen][binärer, Zweier-]{ln}}
\item \lang{de}{\ref[content_16_logarithmen][Logarithmusfunktion]{sec:log_fkt}}
\item \lang{de}{\ref[content_34_exp_und_log][Logarithmus-Regeln ($\ln$)]{rule:ln}}

\end{itemize}
%
\lang{de}{Logarithmusregeln}
\begin{itemize}
\item \lang{de}{\ref[content_16_logarithmen][Multiplikationstheorem]{functional_eqn_log}}
\item \lang{de}{\ref[content_16_logarithmen][Potenzregel für Logarithmen]{thm:log-power_rule}}
%\item \lang{de}{\ref[content_16_logarithmen][Rechenregeln für Logarithmen]{sec:log_rules}}
\item \lang{de}{\ref[content_16_logarithmen][Transformationsformel zum Basiswechsel]{change-base}}
\end{itemize}
%
\lang{de}{Logarithmusfunktion}
\begin{itemize}
\item \lang{de}{\ref[content_16_logarithmen][Logarithmusfunktion]{sec:log_fkt}}
\item \lang{de}{\ref[content_27_konvergenzradius][Logarithmusreihe]{ex:konvergenz-auf-rand}}
\item \lang{de}{\ref[content_30_elem_funktionen][Stetigkeit]{sec:wurzel-und-ln}}
\end{itemize}
%
\lang{de}{Logarithmusgleichung}
\begin{itemize}
\item \lang{de}{\ref[content_16_logarithmen][Logarithmusgleichung lösen]{rule:log_glg}}
\end{itemize}
%
\lang{de}{logisches Oder}
\begin{itemize}
\item \lang{de}{\ref[content_04_aussagen_aequivalenzumformungen][logisches Oder, Disjunktion]{def:objekte_aussagenlogik}}
%\item \lang{de}{\ref[content_04_aussagen_aequivalenzumformungen][--, Wahrheitstafel]{ex:disjuktion}}
\end{itemize}
%
\lang{de}{logisches Und}
\begin{itemize}
\item \lang{de}{\ref[content_04_aussagen_aequivalenzumformungen][logisches Und, Konjunktion]{def:objekte_aussagenlogik}}
%\item \lang{de}{\ref[content_04_aussagen_aequivalenzumformungen][--, Wahrheitstafel]{ex:disjuktion}}
\end{itemize}
%
\lang{de}{lokales Extremum, lokale Extremalstelle}
\begin{itemize}
\item \lang{de}{\ref[content_22_extremstellen][lokales Extremum (eindimensional)]{def:extremstellen}}
\item \lang{de}{\ref[content_54_Differentiation][lokale Extremalstelle (im $\R^n$)]{def:extremstellen}}
\end{itemize}
%
\lang{de}{lokales Maximum, lokale Maximalstelle}
\begin{itemize}
\item \lang{de}{\ref[content_22_extremstellen][lokales Maximum (eindimensional)]{def:extremstellen}}
\item \lang{de}{\ref[content_54_Differentiation][lokale Maximalstelle (im $\R^n$)]{def:extremstellen}}
\end{itemize}
%
\lang{de}{lokales Minimum, lokale Minimalstelle}
\begin{itemize}
\item \lang{de}{\ref[content_22_extremstellen][lokale Minimum (eindimensional)]{def:extremstellen}}
\item  \lang{de}{\ref[content_54_Differentiation][lokale Minimalstelle (im $\R^n$)]{def:extremstellen}}
\end{itemize}
%
\lang{de}{Lot}
\begin{itemize}
\item \lang{de}{\ref[content_38_abstaende][Lot und Lotfußpunkt]{sec:lot}}
\end{itemize}
%
\anchor{sec:m}{\textbf{M}}
\\
%
\lang{de}{Mächtigkeit}
\begin{itemize}
\item \lang{de}{\ref[content_13_unabzaehlbarkeit][Mächtigkeit einer Menge]{def:maechtigkeit}}
\end{itemize}
%
\lang{de}{Majorantenkriterium}
\begin{itemize}
\item \lang{de}{\ref[content_25_konvergenz_kriterien][Majorantenkriterium]{thm:majoranten-krit}}
\item \lang{de}{\ref[content_25_konvergenz_kriterien][Majorantenkriterium, schwaches]{positive-reihen}}
\end{itemize}

%
\lang{de}{Matrix}
\begin{itemize}
\item \lang{de}{Matrix (Definition)\ref[content_39_matrizen][Definition]{ (Teil 1),} \ref[content_01_matrizen][ (Teil 3b)]{def:matrix}}
\item \lang{de}{invertierbare Matrix \ref[content_14_quadratische_matrizen][(reell),]{def:invertierbar} \ref[content_07_quadratische_matrizen][(über beliebigen Körpern)]{sec:invertierbare-matrizen}}
\item \lang{de}{inverse Matrix \ref[content_14_quadratische_matrizen][(reell),]{def:invertierbar} \ref[content_08_inverse_matrix][(über beliebigen Körpern)]{rule:invertierbarkeit-rang}}
\item \lang{de}{quadratische Matrix \ref[content_14_quadratische_matrizen][(reell),]{def:quadr-mat} \ref[content_07_quadratische_matrizen][(über beliebigen Körpern)]{def:quad_matrix}}
\item \lang{de}{\ref[content_45_matrixrang][Rang einer Matrix]{sec:rang}}
\item \lang{de}{\ref[content_45_matrixrang][reguläre ]{rem:sing_reg}}
\item \lang{de}{\ref[content_45_matrixrang][singuläre ]{rem:sing_reg}}
\item \lang{de}{symmetrische\ref[content_44_transponierte_matrix][ (Teil 1),]{def:symmetrische_matrix}
\ref[content_12_symmetrische_matrizen][(Teil 3b)]{def:symmetrischeMatrix}}
\item \lang{de}{transponierte Matrix\ref[content_44_transponierte_matrix][(Teil 1) ]{def:def:transp_matrix}\ref[content_03_transponierte][ (Teil 3b)]{def:transponierte-matrix}}
\item \lang{de}{\ref[content_40_lineare_gleichungssysteme][Koeffizientenmatrix]{def:koeffizientenmatrix}}
\end{itemize}


%
\lang{de}{Matrizen-Rechnung}
\begin{itemize}
\item \lang{de}{Falksches Schema zur Matrixmultiplikation\ref[content_43_matrizenmultiplikation][ (Teil 1), ]{rem:falk_schema}\ref[content_02_matrizenmultiplikation][ (Teil 3b)]{falksches-schema}}
\item \lang{de}{Matrix-Addition \ref[content_42_matrixaddition][ (Teil 1), ]{def:matrix_add} \ref[content_01_matrizen][(Teil 3b)]{def:matrix-addition}}
\item \lang{de}{ Multiplikation mit Skalaren \ref[content_42_matrixaddition][(Teil 1), ]{def:matrix_skalar_mult} \ref[content_01_matrizen][(Teil 3b)]{def:matrix-multiplikation-skalar}}
\item \lang{de}{Matrix-Vektor-Multiplikation\ref[content_02_matrizenmultiplikation][ (Teil 1), ]{def:matrix-vektor-multiplikation}\ref[content_02_matrizenmultiplikation][ (Teil 3b)]{def:matrix-vektor-multiplikation}}
%\item \lang{de}{\ref[content_39b_matrizen][Multiplikation mit einem Vektor]{sec:matrix-vektor-mult}}
\item \lang{de}{Matrix-Multiplikation \ref[content_43_matrizenmultiplikation][(Teil 1), ]{def:matizen_mult}\ref[content_02_matrizenmultiplikation][ (Teil 3b)]{def:matrix-multiplikation}}
\item \lang{de}{\ref[content_03_transponierte][transponierte Matrix]{def:transponierte-matrix}}
\item \lang{de}{\ref[content_42_matrixaddition][Regeln zur Addition und Skalar-Multiplikation]{rule:matrix_rechenregeln}}
\item \lang{de}{\ref[content_43_matrizenmultiplikation][Regeln zur Matrixmultiplikation]{sec:rechenregeln}}
\item \lang{de}{\ref[content_42_matrixaddition][Regeln zur Vektor-Multiplikation]{sec:rechenregeln}}
\end{itemize}
%
\lang{de}{Maximum}
\begin{itemize}
\item \lang{de}{\ref[content_06_supremum_infimum][Maximum einer Menge]{def:min_max}}
\item \lang{de}{\ref[content_22_extremstellen][lokale/globale Maximalstelle, lokales/globales Maximum]{def:extremstellen}}
\end{itemize}
%
\lang{de}{Menge}
\begin{itemize}
\item \lang{de}{\ref[content_01_zahlenmengen][Menge (Definition)]{def:menge}}
\item \lang{de}{abgeschlossene \ref[content_22_offene_abgeschlossene_teilmengen][ (in $\R$),]{sec:abgeschl-mengen} \ref[content_52_Abstaende][ (im $\R^n$)]{def:offen_abgeschlossen}}
\item \lang{de}{\ref[content_13_unabzaehlbarkeit][abzählbare]{def:abzaehlbarkeit}}
\item \lang{de}{\ref[content_06_supremum_infimum][beschränkte]{def:beschraenkt}}
\item \lang{de}{offene \ref[content_22_offene_abgeschlossene_teilmengen][(in $\R$), ]{sec:offene_mengen} \ref[content_52_Abstaende][ (im $\R^n$)]{def:offen_abgeschlossen}}
\item \lang{de}{\ref[content_13_unabzaehlbarkeit][Mächtigkeit einer Menge]{def:maechtigkeit}}
\item \lang{de}{\ref[content_01_zahlenmengen][Mengenrelationen]{def:mengenrelationen}}
\item \lang{de}{\ref[content_01_zahlenmengen][Mengenoperationen]{def:mengenoperationen}}
\item \lang{de}{\ref[content_13_unabzaehlbarkeit][überabzählbare]{def:abzaehlbarkeit}}
\item \lang{de}{\ref[content_01_zahlenmengen][Zahlenmengen]{def:zahlenmengen}}
\end{itemize}


%
\lang{de}{Minimum}
\begin{itemize}
\item \lang{de}{\ref[content_06_supremum_infimum][Minimum einer Menge]{def:min_max}}
\item \lang{de}{\ref[content_22_extremstellen][lokale/globale Minimalstelle, lokales/globales Minimum]{def:extremstellen}}
\end{itemize}
%
\lang{de}{Minorantenkriterium}
\begin{itemize}
\item \lang{de}{\ref[content_25_konvergenz_kriterien][Minorantenkriterium]{thm:majoranten-krit}}
\item \lang{de}{\ref[content_25_konvergenz_kriterien][Minorantenkriterium, schwaches]{positive-reihen}}
\end{itemize}
%
\lang{de}{Mittelwertsatz}
\begin{itemize}
\item \lang{de}{\ref[content_03_hoehere_ableitungen][der Differentialrechnung]{thm:mittelwertsatz}}
\item \lang{de}{\ref[content_09_integrierbare_funktionen][der Integralrechnung]{thm:mittelwertsatz}}
\item \lang{de}{\ref[content_06_de_l_hospital][verallgemeinerter Mittelwertsatz]{thm:verallg_mws}}
\end{itemize}
%
\lang{de}{Mitternachtsformel}
\begin{itemize}
\item \lang{de}{\ref[content_05_loesen_gleichungen_und_lgs][Mitternachtsformel]{rem:mitternachtsformel}}
\end{itemize}
%
\lang{de}{Monotonie}
\begin{itemize}
\item \lang{de}{\ref[content_22_extremstellen][Monotonie]{Monotonie}}
\end{itemize}
%
\lang{de}{Monotonieverhalten einer Funktion}
\begin{itemize}
\item \lang{de}{\ref[content_22_extremstellen][(streng) monoton fallend]{Monotonie}}
\item \lang{de}{\ref[content_22_extremstellen][(streng) monoton wachsend]{Monotonie}}
\item \lang{de}{\ref[content_12_reelle_funktionen_monotonie][Monotonie der Umkehrfunktion]{thm:inversmonoton}}
\item \lang{de}{\ref[content_22_extremstellen][Zusammenhang zur Ableitung der Funktion (Teil 1)]{thm:abl_monotonie},
                \ref[content_03_hoehere_ableitungen][(Teil 3a)]{thm:diffmonotonie}}
\end{itemize}
%
\lang{de}{Multiplikation}
\begin{itemize}
\item \lang{de}{\ref[content_42_matrixaddition][Matrix mit Skalar]{def:matrix_skalar_mult}}
\item \lang{de}{\ref[content_39b_matrizen][Matrix-Vektor]{sec:matrix-vektor-mult}}
\item \lang{de}{\ref[content_08aneu_komplexeZahlen_intro][komplexer Zahlen]{sec:add-mult-kompl}}
\item \lang{de}{\ref[content_02_rechengrundlagen_terme][reeller Zahlen]{def:grundrechenarten}}
\item \lang{de}{\ref[content_03_bruchrechnung][von Brüchen/Bruchtermen]{mul}}
\item \lang{de}{\ref[content_43_matrizenmultiplikation][Matrix-Matrix]{def:matizen_mult}}
\item \lang{de}{\ref[content_29_linearkombination][Vektor mit Skalar (Skalarmultiplikation)]{Vektorraum_R_n}}
\item \lang{de}{\ref[content_31_skalarprodukt][zweier Vektoren im Skalarprodukt]{def:skalarprodukt}}
\item \lang{de}{\ref[content_34_vektorprodukt][zweier Vektoren im Kreuzprodukt (Vektorprodukt)]{def:Vektorprodukt}}
\end{itemize}
%
\lang{de}{Multiplikationstheorem}
\begin{itemize}
\item \lang{de}{\ref[content_16_logarithmen][Multiplikationstheorem für Logarithmen]{functional_eqn_log}}
\end{itemize}
%
\anchor{sec:n}{textbf{N}}
\\
%
\lang{de}{$\N$}
\begin{itemize}
\item \lang{de}{\ref[content_01_zahlenmengen][$\N$ (natürliche Zahlen)]{def:zahlenmengen}}
\end{itemize}
%
\lang{de}{Negation}
\begin{itemize}
\item \lang{de}{\ref[content_04_aussagen_aequivalenzumformungen][Negation]{def:objekte_aussagenlogik}}
%\item \lang{de}{\ref[content_04_aussagen_aequivalenzumformungen][--, Wahrheitstafel]{ex:negation}}
\end{itemize}
%
\lang{de}{Newton-Verfahren}
\begin{itemize}
%\item \lang{de}{\ref[content_05_newtonverfahren][--, Motivation]{newton}}
\item \lang{de}{\ref[content_05_newtonverfahren][Newton-Verfahren]{theo:newton-verfahren}}
\item \lang{de}{\ref[content_05_newtonverfahren][Konvergenzgeschwindigkeit]{sec:konv-geschw}}
\item \lang{de}{\ref[content_05_newtonverfahren][Approximation von Wurzeln]{sec:n-te-wurzeln}}
\end{itemize}
%
\lang{de}{Norm}
\begin{itemize}
\item \lang{de}{\ref[content_52_Abstaende][Norm (Definition, Eigenschaften)]{def:norm_allgemein}}
\item \lang{de}{\ref[content_52_Abstaende][Eins-Norm]{thm:weitere_normen}}
\item \lang{de}{\ref[content_32_laenge_norm][euklidische]{def:euklidische_norm}}
\item \lang{de}{\ref[content_52_Abstaende][Matrixnorm]{rem:operatornorm}}
\item \lang{de}{\ref[content_52_Abstaende][Maximumsnorm]{thm:weitere_normen}}
\item \lang{de}{\ref[content_52_Abstaende][Norm-Abschätzung]{thm:norm-equivalence}}
\item \lang{de}{\ref[content_52_Abstaende][Operatornorm]{rem:operatornorm}}
%\item \lang{de}{\ref[content_32_laenge_norm][-- eines Vektors, Eigenschaften]{thm:eigenschaften-norm}}
\end{itemize}
%
\lang{de}{Normaleneinheitsvektor}
\begin{itemize}
\item \lang{de}{\ref[content_36_normalenformen][einer Ebene]{def:norm_vec_ebene}}
\item \lang{de}{\ref[content_36_normalenformen][einer Geraden]{def:norm_vec_gerade}}
\end{itemize}

\lang{de}{Normalenvektor}
\begin{itemize}
\item \lang{de}{\ref[content_36_normalenformen][einer Ebene]{def:norm_vec_ebene}}
\item \lang{de}{\ref[content_36_normalenformen][einer Geraden]{def:norm_vec_gerade}}
\end{itemize}
%
\lang{de}{Normalform}
\begin{itemize}
\item \lang{de}{\ref[content_05_loesen_gleichungen_und_lgs][einer quadratische Gleichung]{quadratic.definition.2}}
\end{itemize}
%
\lang{de}{Normalenform}
\begin{itemize}
\item \lang{de}{\ref[content_36_normalenformen][einer Ebene]{def:hesse_nf_ebene}}
\item \lang{de}{\ref[content_36_normalenformen][einer Gerade]{def:koord_form_gerade}}
\end{itemize}

%
\lang{de}{Normalparabel}
\begin{itemize}
\item \lang{de}{\ref[content_08_quadratische_funktionen][Normalparabel]{def:parabel}}
\end{itemize}
%
\lang{de}{notwendige Bedingung}
\begin{itemize}
\item \lang{de}{\ref[content_22_extremstellen][für lokale Extrema (Teil 1)]{thm:notw_bedg_lok_extremum},
                \ref[content_03_hoehere_ableitungen][(Teil 3a)]{thm:NB_extrema}}
\item \lang{de}{\ref[content_23_kurvendiskussion][für eine Wendestelle]{thm:notw_hinr_bedg_wendepkt}}
\end{itemize}
%
\lang{de}{Nullmatrix}
\begin{itemize}
\item \lang{de}{Nullmatrix \ref[content_14_quadratische_matrizen][(reell),]{def:wichtige-quadr-mat} \ref[content_07_quadratische_matrizen][(über beliebigen Körpern)]{def:wichtige-quadratische-matrizen}}
\end{itemize}

%
%
\anchor{sec:o}{\textbf{O}}
\\
%
\lang{de}{Obermenge}
\begin{itemize}
\item \lang{de}{\ref[content_01_zahlenmengen][Obermenge]{def:mengenrelationen}}
\end{itemize}
%
\lang{de}{Obersumme}
\begin{itemize}
%\item \lang{de}{\ref[content_07_ober_und_untersumme][Intervall-Zerlegung]{def:intervall-zer}}
\item \lang{de}{\ref[content_07_ober_und_untersumme][Obersumme]{def:Obersum-untersum}}
\end{itemize}
%
\lang{de}{offen}
\begin{itemize}
\item \lang{de}{\ref[content_01_zahlenmengen][offenes Intervall]{def:intervall}}
\item \lang{de}{\ref[content_52_Abstaende][offene Kugel]{rem:offene_Kugeln}}
\item \lang{de}{\ref[content_22_offene_abgeschlossene_teilmengen][offene Menge (in $\R$)]{sec:offene_mengen}}
\item \lang{de}{\ref[content_52_Abstaende][offene  (im $\R^n$)]{def:offen_abgeschlossen}}
\item \lang{de}{\ref[content_52_Abstaende][offener Quader]{ex:quader}}
\end{itemize}



%
\lang{de}{Orthogonalität}
\begin{itemize}
\item \lang{de}{\ref[content_10c_Orthogonalbasen][orthogonale Vektoren]{def:orthogonaleVektoren}}
\item \lang{de}{\ref[content_33_winkel][orthogonale Vektoren (bzgl. Standard-Skalarprodukt)]{def:orthogonal}}
\item \lang{de}{\ref[content_10c_Orthogonalbasen][Orthogonalbasis]{def:orthogonalbasis}}
\item \lang{de}{\ref[content_10c_Orthogonalbasen][Orthonormalbasis]{def:orthogonalbasis}}
\item \lang{de}{\ref[content_10c_Orthogonalbasen][Orthonormalisierungsverfahren]{gram_schmidt}}
\item \lang{de}{\ref[content_10c_Orthogonalbasen][Gram-Schmidt-Verfahren]{gram_schmidt}}
\end{itemize}
%

\anchor{sec:p}{\textbf{P}}
\\
%
\lang{de}{$p$-$q$-Formel}
\begin{itemize}
\item \lang{de}{\ref[content_05_loesen_gleichungen_und_lgs][pq-Formel]{rule:pqFormel}}
\end{itemize}
%
\lang{de}{Parabel}
\begin{itemize}
\item \lang{de}{\ref[content_08_quadratische_funktionen][Parabelfunktion]{def:quadratic_func}}
\item \lang{de}{\ref[content_08_quadratische_funktionen][faktorisierte Form]{rem:factorised_form}}
\item \lang{de}{\ref[content_51_Kurven][Neilsche]{rem:param_polynom_kurven}}
\item \lang{de}{\ref[content_08_quadratische_funktionen][Normalform, Normalparabel]{def:parabel}}
\item \lang{de}{\ref[content_08_quadratische_funktionen][Nullstellen]{sec:nullstellen}}
\item \lang{de}{\ref[content_08_quadratische_funktionen][Scheitelpunktform]{sec:scheitel}}
\end{itemize}
% 
\lang{de}{parallele}
\begin{itemize}
%\item \lang{de}{\ref[content_31_skalarprodukt][-- oder antiparallele Vektoren]{def:kollinear}} %Referenz existiert nicht.
\item \lang{de}{\ref[content_37_schnittpunkte][Ebenen]{sec:6_7_Ebene-Ebene}}
\item \lang{de}{\ref[content_37_schnittpunkte][Geraden]{rule:lage_geraden}}
\item \lang{de}{\ref[content_37_schnittpunkte][Gerade und Ebene]{sec:6_6_Gerade-Ebene}}
\end{itemize}
%
\lang{de}{Parameterform}
\begin{itemize}
\item \lang{de}{\ref[content_36_normalenformen][einer Ebene]{def:norm_form_ebene}}
\item \lang{de}{\ref[content_36_normalenformen][einer Geraden]{def:koord_form_gerade}}
\end{itemize}
%
%
\lang{de}{Parametrisierung}
\begin{itemize}
\item \lang{de}{\ref[content_51_Kurven][Parametrisierung einer Kurve]{rem:parametrisierung}}
\end{itemize}

% 
\lang{de}{Partialbruchzerlegung}
\begin{itemize}
%\item \lang{de}{\link{content_13_partialbruchzerlegung}{Partialbruchzerlegung (Herleitung)}}
\item \lang{de}{\ref[content_13_partialbruchzerlegung][Partialbruchzerlegung]{sec:partbruchzerl}}
\item \lang{de}{\ref[content_13_partialbruchzerlegung][Verfahren zur Bestimmung von Stammfunktionen rationaler Funktionen]{sec:summary}}
\end{itemize}
%
\lang{de}{partielle Integration}
\begin{itemize}
\item \lang{de}{\ref[content_11_partielle_integration][partielle Integration]{sec:part-int}}
\item \lang{de}{\ref[content_11_partielle_integration][mit anschließender Gleichungsauflösung]{sec:part-int-gleich-aufl}}
\end{itemize}
%
\lang{de}{Partialsummenfolge}
\begin{itemize}
\item \lang{de}{\ref[content_24_reihen_und_konvergenz][Partialsummenfolge]{def:reihe}}
\end{itemize}
%
\lang{de}{Pfeilklasse}
\begin{itemize}
\item \lang{de}{\ref[content_27_vektoren][Pfeil, Pfeilklasse]{rem:pfeilklasse}}
\end{itemize}
%
%
\lang{de}{Phasenwinkel}
\begin{itemize}
\item \lang{de}{\ref[content_08bneu_komplexeZahlen_geom][Phasenwinkel (Argument einer komplexen Zahl)]{sec:polar}}
\end{itemize}
%
\lang{de}{Pol, Polstelle}
\begin{itemize}
\item \lang{de}{\ref[content_10_polynomdivision][Polstelle einer gebrochen rationalen Funktion]{def:pole}}
\item\lang{de}{\ref[content_32_grenzwert_gegen_unendlich][Polstellen und Asymptoten]{rule:rat_fct_def_gaps}}
\end{itemize}
%
\lang{de}{Polarkoordinaten}
\begin{itemize}
\item \lang{de}{\ref[content_08bneu_komplexeZahlen_geom][Polarkoordinaten einer komplexen Zahl]{sec:polar}}
\item \lang{de}{\ref[content_36_anwendungen][Polardarstellungn mit komplexer $e$-Funktion]{sec:polardarst}}

\end{itemize}
%
\lang{de}{Polynom, Polynomfunktionen}
\begin{itemize}
\item \lang{de}{\ref[content_09_polynome][Polynom $n$-ten Grades (Definition)]{polynomial}}
\item \lang{de}{\ref[content_11_eigenwerte][charakteristisches]{sec:charakteristischesPolynom}}
\item \lang{de}{\ref[content_36_anwendungen][Fundamentalsatz der Algebra]{sec:fundamentalsatzAlgebra}}
\item \lang{de}{\ref[content_09_polynome][Linearfaktoren]{sec:linearfaktor}}
\item \link{content_08_quadratische_funktionen}{\lang{de}{quadratisches}}
\item \lang{de}{\ref[content_30_elem_funktionen][Stetigkeit]{sec:polynome-und-rationale-funk}}
\item \lang{de}{\ref[content_09_polynome][Verhalten bei Null und bei Unendlich]{rule:polyn_null_unendl}}
\end{itemize}
%
\lang{de}{Polynomdivision}
\begin{itemize}
\item \lang{de}{\ref[content_10_polynomdivision][Polynomdivision]{sec:poly-div-factoring}}
\item \lang{de}{\ref[content_32_grenzwert_gegen_unendlich][Anwendung für rationale Funktionen]{sec:asymptotics_rational}}
\item \lang{de}{\ref[content_09_polynome][Linearfaktoren]{sec:linearfaktor}}
\item \lang{de}{\ref[content_10_polynomdivision][mittels Horner-Schema]{sec:poly-div-mit-horner}}
\end{itemize}

%
\lang{de}{Potenz}
\begin{itemize}
\item \lang{de}{\ref[content_14_potenzregeln][mit ganzzahligem Exponenten]{zeroexp}}
\item \lang{de}{\ref[content_02_rechengrundlagen_terme][mit natürlichem Exponenten]{def:potenz}}
\item \lang{de}{\ref[content_14_potenzregeln][mit rationalem Exponenten]{def:rat_potenz}}
\item \lang{de}{\ref[content_36_anwendungen][komplexer Zahlen]{def:komplexe-potenz}}
\item \lang{de}{\ref[content_14_potenzregeln][Potenzgesetze]{Potenzgesetze}}
\item \lang{de}{\ref[content_16_logarithmen][Potenzregel für Logarithmen]{thm:log-power_rule}}
%\item \lang{de}{\ref[content_16_logarithmen][Potenzgleichung lösen]{rule:potenz_glg}}
\end{itemize}
%
\lang{de}{Potenzgleichung}
\begin{itemize}
\item \lang{de}{\ref[content_16_logarithmen][Potenzgleichung lösen]{rule:potenz_glg}}
\end{itemize}
%
\lang{de}{Potenzieren}
\begin{itemize}
\item \lang{de}{\ref[content_02_rechengrundlagen_terme][Einführung und Definition]{def:potenz}}
\end{itemize}
%
\lang{de}{Potenzreihe}
\begin{itemize}
\item \lang{de}{\ref[content_27_konvergenzradius][Potenzreihe (Definition)]{def:Potenzreihe}}
\item \lang{de}{\ref[content_27_konvergenzradius][Entwicklungspunkt]{def:Potenzreihe}}
\item \lang{de}{\ref[content_27_konvergenzradius][Koeffizienten]{def:Potenzreihe}}
\item \lang{de}{\ref[content_27_konvergenzradius][Konvergenzbereich]{thm:konvergenzbereich}}
\item \lang{de}{\ref[content_27_konvergenzradius][Konvergenzradius]{def:konvergenzradius}}
% \item \lang{de}{\ref[content_28_exponentialreihe][Satz von Cauchy-Hadamard]{thm:Cauchy-Hadamard}
% \item \lang{de}{\ref[content_28_exponentialreihe][Quotientenregel]{thm:quot-regel}
%\item \lang{de}{\ref[content_27_konvergenzradius][Abelsches Lemma]{thm:abelsches-lemma}}
%\item \lang{de}{\ref[content_28_exponentialreihe][Additionstheoreme, Eulersche Identität]{rem:sin_cos_exp}}
\item \lang{de}{\ref[content_28_exponentialreihe][Exponentialreihe]{sec:exp-reihe}}
\item \lang{de}{\ref[content_28_exponentialreihe][Sinus- und Kosinusreihen]{sec:sinus-kosinus}}
\item \lang{de}{\ref[content_28_exponentialreihe][Sinus hyperbolicus und Kosinus hyperbolicus (Reihendarstellung)]{sec:sinh-cosh}}
\item \lang{de}{\ref[content_30_elem_funktionen][Stetigkeit]{thm:ptenzreihen_stetig}}
\end{itemize}
%
\lang{de}{Prämisse}
\begin{itemize}
\item \lang{de}{\ref[content_04_aussagen_aequivalenzumformungen][Prämisse]{def:implikation}}
\end{itemize}
%
\lang{de}{Primfaktorzerlegung}
\begin{itemize}
\item \lang{de}{\ref[content_03_bruchrechnung][Bestimmung des ggT zweier natürlicher Zahlen]{ggT}}
\item \lang{de}{\ref[content_03_bruchrechnung][Bestimmung des kgV zweier natürlicher Zahlen]{kgV}}
\end{itemize}
%
\lang{de}{Produkt}
\begin{itemize}
\item \lang{de}{\ref[content_02_rechengrundlagen_terme][reeller Zahlen]{def:grundrechenarten}}
\item \lang{de}{\ref[content_08aneu_komplexeZahlen_intro][komplexer Zahlen]{sec:add-mult-kompl}}
\item \lang{de}{\ref[content_02_rechengrundlagen_terme][Produktzeichen]{def:produktzeichen}}
\item \lang{de}{\ref[content_34_vektorprodukt][Kreuzprodukt]{def:Vektorprodukt}}
\item \lang{de}{\ref[content_31_skalarprodukt][Skalarprodukt]{def:skalarprodukt}}
\item \lang{de}{\ref[content_34_vektorprodukt][Vektorprodukt]{def:Vektorprodukt}}
\end{itemize}
%
\lang{de}{Produktregel}
\begin{itemize}
\item \lang{de}{\ref[content_20_ableitung_als_tangentensteigung][der Ableitung (Teil 1)]{rule:produktregel},
                \ref[content_02_ableitungsregeln][(Teil 3a)]{rule:produkt_quotient_regel}}
\item \lang{de}{\ref[content_54_Differentiation][der Ableitung (mehrdimensional)]{thm:summen-und-produktregel}}
\end{itemize}
%
\lang{de}{Punkt-vor-Strich-Regel}
\begin{itemize}
\item \lang{de}{\ref[content_02_rechengrundlagen_terme][Punkt-vor-Strich-Regel]{rule:punkt-vor-strich}}
\end{itemize}

%
\lang{de}{Pythagoras}
\begin{itemize}
\item \lang{de}{\ref[content_17_trigonometrie_im_dreieck][Satz des]{satz-des-pythagoras}}
\item \lang{de}{\ref[content_19_allgemeiner_sinus_cosinus][trigonometrischer]{trig-pythagoras}}
\end{itemize}
%
%
\anchor{sec:q}{\textbf{Q}}
\\
%
%
\lang{de}{$\Q$}
\begin{itemize}
\item \lang{de}{\ref[content_01_zahlenmengen][$\Q$ (rationale Zahlen)]{def:zahlenmengen}}
\end{itemize}
%


\lang{de}{Quader}
\begin{itemize}
\item \lang{de}{\ref[content_52_Abstaende][Quader im $\R^n$]{ex:quader}}
\end{itemize}
%
\lang{de}{quadratische Ergänzung}
\begin{itemize}
\item \lang{de}{\ref[content_05_loesen_gleichungen_und_lgs][quadratische Ergänzung]{alg:quadr_erg}}
\end{itemize}

%
\lang{de}{quadratische Funktion}
\begin{itemize}
\item \lang{de}{\ref[content_08_quadratische_funktionen][quadratische Funktion]{sec:quadratic}}
\end{itemize}


\lang{de}{quadratische Gleichungen}
\begin{itemize}
\item \lang{de}{\ref[content_05_loesen_gleichungen_und_lgs][quadratische Gleichungen]{sec:quadratic}}
\item \lang{de}{\ref[content_09neu_komplexeZahlen_hauptsatz][komplexe Lösungen]{sec:komplexe-quadratische-gleichungen}}
\item \lang{de}{\ref[content_05_loesen_gleichungen_und_lgs][Linearfaktorzerlegung]{thm:Satz_von_Vieta}}
\item \lang{de}{\ref[content_05_loesen_gleichungen_und_lgs][ lösen mittels Normalform]{quadratic.definition.2}}
\item \lang{de}{\ref[content_05_loesen_gleichungen_und_lgs][ lösen mittels quadratische Ergänzung]{alg:quadr_erg}}
\item \lang{de}{\ref[content_05_loesen_gleichungen_und_lgs][ lösen mittels $p$-$q$-Formel]{rule:pqFormel}}
\item \lang{de}{\ref[content_05_loesen_gleichungen_und_lgs][ lösen mittels Mitternachtsformel]{rem:mitternachtsformel}}
\item \lang{de}{\ref[content_05_loesen_gleichungen_und_lgs][ lösen mittels Satz von Viëta, Linearfaktorzerlegung]{thm:Satz_von_Vieta}}
\end{itemize}
%
\lang{de}{quadratische Matrix}
\begin{itemize}
\item \lang{de}{quadratische Matrix \ref[content_14_quadratische_matrizen][(reell),]{def:quadr-mat} \ref[content_07_quadratische_matrizen][(über beliebigen Körpern)]{def:quad_matrix}}
\end{itemize}

\lang{de}{Quotientenkriterium}
\begin{itemize}
\\item \lang{de}{\ref[content_25_konvergenz_kriterien][Quotientenkriterium]{thm:quotientenkriterium}}

\end{itemize}
%
\lang{de}{Quotientenregel der Ableitung}
\begin{itemize}
\item \lang{de}{\ref[content_20_ableitung_als_tangentensteigung][eindimensional (Teil 1)]{rule:quotient_regel},
                \ref[content_02_ableitungsregeln][(Teil 3a)]{rule:produkt_quotient_regel}}
\item\lang{de}{\ref[content_54_Differentiation][mehrdimensional]{thm:kettenregel_mehrdim}}
\end{itemize}
%
\lang{de}{Quotientenregel für Konvergenzradius}
\begin{itemize}
\item \lang{de}{\ref[content_27_konvergenzradius][Quotientenregel für Konvergenzradius]{thm:quot-regel}}
\end{itemize}
%
%
\anchor{sec:r}{\textbf{R}}
\\
%
%
\lang{de}{$\R$}
\begin{itemize}
\item \lang{de}{\ref[content_01_zahlenmengen][$\R$ (reelle Zahlen)]{introduction_R}}
\end{itemize}
%
\lang{de}{Radikand}
\begin{itemize}
\item \lang{de}{\ref[content_13_wurzelfunktionen][Radikant]{sec:n-te_wurzel}}
\end{itemize}
%
\lang{de}{Randpunkt}
\begin{itemize}
\item \lang{de}{\ref[content_22_offene_abgeschlossene_teilmengen][Randpunkt]{sec:innere-randpunkte}}
\end{itemize}

%
\lang{de}{Rang}
\begin{itemize}
\item \lang{de}{Rang einer Matrix \ref[content_45_matrixrang][(Teil 1), ]{sec:rang}\ref[content_06_umformungen_rang][ (Teil 3b)]{def:vollerRang}}
\item \lang{de}{Spaltenrang \ref[content_45_matrixrang][(Teil 1); ]{sec:zeilenrang-spaltenrang}\ref[content_06_umformungen_rang][ (Teil 3b)]{def:zeilenSpaltenRang}}
\item \lang{de}{Zeilenrang \ref[content_45_matrixrang][(Teil 1),]{sec:zeilenrang-spaltenrang}\ref[content_06_umformungen_rang][ (Teil 3b)]{def:zeilenSpaltenRang}}
\end{itemize}


%
\lang{de}{rationale Zahlen}
\begin{itemize}
\item \lang{de}{\ref[content_04_koerperaxiome][Körper der rationalen Zahlen]{ex:koerper}}
\item \lang{de}{\ref[content_01_zahlenmengen][Menge der rationalen Zahlen]{def:zahlenmengen}}
\item \lang{de}{\ref[content_13_unabzaehlbarkeit][Abzählbarkeit]{thm:abzcartesisch}}
\end{itemize}

%
\lang{de}{rationale Funktion}
\begin{itemize}
\item \lang{de}{\ref[content_10_polynomdivision][rationale Funktion,]{def:gebr_rat_fkt}} siehe auch gebrochen rationale Funktion
\item \lang{de}{\ref[content_30_elem_funktionen][Stetigkeit]{sec:polynome-und-rationale-funk}}

\end{itemize}
%
\lang{de}{Realteil}
\begin{itemize}
\item \lang{de}{\ref[content_08aneu_komplexeZahlen_intro][Realteil]{def:real-imaginaer-teil}}
\end{itemize}
%
\lang{de}{Rechenregeln/-gesetze}
\begin{itemize}
\item \lang{de}{\link{content_03_bruchrechnung}{für Brüche und Bruchterme}}
\item \lang{de}{\ref[content_19_bestimmte_divergenz][für bestimmt divergente Folgen]{sec:rechenregeln}}
\item \lang{de}{\ref[content_14_konvergenz][für Folgengrenzwerte]{sec:grenzwertregeln}}
\item \lang{de}{Rechenregeln für inverse Matrizen \ref[content_15_inverse_matrix][(reell),]{sec:rechenregeln} \ref[content_08_inverse_matrix][(über beliebigen Körpern)]{sec:rechenregeln}}
\item \lang{de}{\ref[content_08aneu_komplexeZahlen_intro][für komplexe Zahlen]{sec:add-mult-kompll}}
\item \lang{de}{\ref[content_04_koerperaxiome][für Körper]{sec:rechenregeln}}
\item \lang{de}{\ref[content_16_logarithmen][für Logarithmen]{sec:log_rules}}
\item \lang{de}{\ref[content_14_potenzregeln][für Potenzen]{Potenzgesetze}}
\item \lang{de}{\link{content_02_rechengrundlagen_terme}{für reelle Zahlen}}
\item \lang{de}{\ref[content_31_skalarprodukt][für Skalarprodukte]{sec:regeln}}
\item \lang{de}{\ref[content_24_reihen_und_konvergenz][für Reihengrenzwerte]{sec:grenzwertregeln}}
\item \lang{de}{\ref[content_29_linearkombination][für Vektoren]{Vektorraum_R_n}}
\item \lang{de}{\ref[content_34_vektorprodukt][für Vektorprodukte]{sec:rechenregeln}}
\item \lang{de}{\ref[content_13_wurzelfunktionen][für Wurzeln]{allgwurzel_gesetze}}
\item \lang{de}{\ref[content_42_matrixaddition][zur Addition und Skalar-Multiplikation von Matrizen]{rule:matrix_rechenregeln}}
\item \lang{de}{\ref[content_43_matrizenmultiplikation][zur Matrixmultiplikation]{sec:rechenregeln}}
\item \lang{de}{zur Matrix-Vektor-Multiplikation\ref[content_02_matrizenmultiplikation][ (Teil 1), ]{def:matrix-vektor-multiplikation}\ref[content_02_matrizenmultiplikation][ (Teil 3b)]{def:matrix-vektor-multiplikation}}

\end{itemize}
%
\lang{de}{reell-analytische Funktion}r-Multiplikation
\begin{itemize}
\item \lang{de}{\ref[content_04_taylor_polynom][reell-analytische Funktion]{def:reell-analyt-Fktn}}
\end{itemize}
%
\lang{de}{reelle Zahlen}
\begin{itemize}
\item \lang{de}{\ref[content_06_supremum_infimum][beschränkte Mengen, Schranken]{def:beschraenkt}}
\item \lang{de}{\ref[content_01_zahlenmengen][Darstellung auf dem Zahlenstrahl]{introduction_R}}
\item \lang{de}{\ref[content_07_vollstaendigkeit][als vollständiger angeordneter Körper]{sec:R_vollst_Koerper}
}
%\item \lang{de}{\ref[content_06_supremum_infimum][Maximum und Minimum]{rem:max-min}}
\item \lang{de}{\ref[content_02_rechengrundlagen_terme][Rechengesetze]{sec:rechengesetze}}
%\item \lang{de}{\ref[content_07_vollstaendigkeit][Supremum und Infimum]{def:sup-inf}}
%\item \lang{de}{\ref[content_07_vollstaendigkeit][Vollständigkeit]{def:vollstaendigkeit}}
\item \lang{de}{\ref[content_13_unabzaehlbarkeit][Überabzählbarkeit]{sec:R_ueberabz}}
\end{itemize}
%
\lang{de}{Reihe}
\begin{itemize}
\item \lang{de}{\ref[content_24_reihen_und_konvergenz][Reihe (Definition)]{def:reihe}}
\item \lang{de}{\ref[content_25_konvergenz_kriterien][absolut konvergente]{def:abolut-konvergenz}}
\item \lang{de}{\ref[content_26_produkt_von_reihen][Cauchy-Produkt]{def:cauchy-prod}}
%\item \lang{de}{\ref[content_26_produkt_von_reihen][Funktionalgleichung, komplexe Exponentialreihe]{ex:funktionalgleichung-exp}}
\item \lang{de}{\ref[content_24_reihen_und_konvergenz][geometrische]{ex:konvergenz-geo-reihe}}
\item \lang{de}{\ref[content_24_reihen_und_konvergenz][Grenzwertregeln]{sec:grenzwertregeln}}
\item \lang{de}{\ref[content_24_reihen_und_konvergenz][geometrische Reihe]{ex:konvergenz-geo-reihe}}
\item \lang{de}{\ref[content_24_reihen_und_konvergenz][konvergente]{def:reihenkonvergenz}}
\item \lang{de}{\ref[content_24_reihen_und_konvergenz][Leibniz-Kriterium]{thm:leibnizkriterium}}
\item \lang{de}{\ref[content_25_konvergenz_kriterien][Majorantenkriterium]{thm:majoranten-krit}}
\item \lang{de}{\ref[content_25_konvergenz_kriterien][Minorantenkriterium, schwaches]{positive-reihen}}
\item \lang{de}{\ref[content_25_konvergenz_kriterien][Minorantenkriterium]{thm:majoranten-krit}}
\item \lang{de}{\ref[content_25_konvergenz_kriterien][Minorantenkriterium, schwaches]{positive-reihen}}
\item \lang{de}{\ref[content_25_konvergenz_kriterien][Quotientenkriterium]{thm:quotientenkriterium}}
\item \lang{de}{\ref[content_25_konvergenz_kriterien][Wurzelkriterium]{thm:wurzelkriterium}}
\item \lang{de}{\ref[content_26_produkt_von_reihen][Umordnung von Reihen]{sec:reihen-umordnung}}
\end{itemize}
%
\lang{de}{Richtungsableitung}
\begin{itemize}
\item \lang{de}{\ref[content_54_Differentiation][Richtungsableitung]{def:richtungsableitung}}
\end{itemize}

%
\lang{de}{Riemannsche Zwischensumme}
\begin{itemize}
%\item \lang{de}{\ref[content_07_ober_und_untersumme][--, Intervall-Zerlegung]{def:intervall-zer}}
\item \lang{de}{\ref[content_07_ober_und_untersumme][Riemannsche Zwischensumme]{def:riemann-zwisch-sum}}
\end{itemize}
%

\anchor{sec:s}{\textbf{S}}
\\
%
\lang{de}{Sandwich-Lemma}
\begin{itemize}
\item \lang{de}{\ref[content_16_konvergenzkriterien][Sandwich-Lemma]{thm:sandwich}}
\end{itemize}
%
\lang{de}{Sattelpunkt}
\begin{itemize}
\item \lang{de}{\ref[content_23_kurvendiskussion][Sattelpunkt, Sattelstelle]{sec:wendest_sattel}}
\end{itemize}
%
\lang{de}{Satz}
\begin{itemize}
\item \lang{de}{\ref[content_17_trigonometrie_im_dreieck][des Pythagoras]{satz-des-pythagoras}}
\item \lang{de}{\ref[content_19_allgemeiner_sinus_cosinus][des Pythagoras (trigonometrisch)]{trig-pythagoras}}
\item \lang{de}{\ref[content_16_konvergenzkriterien][von Bolzano-Weierstrass]{thm:bolzano-weierstrass}}
\item \lang{de}{\ref[content_27_konvergenzradius][von Cauchy-Hadamard]{thm:Cauchy-Hadamard}}
\item \lang{de}{\ref[content_05_loesen_gleichungen_und_lgs][von Viëta, Linearfaktoren, Linearfaktorzerlegung]{thm:Satz_von_Vieta}}
\item \lang{de}{
\ref[content_55_Lokale_Umkehrbarkeit][über implizite Funktionen]{thm:implizite_fkten}}
\end{itemize}
%
\lang{de}{Scheitelpunkt}
\begin{itemize}
\item \lang{de}{\ref[content_08_quadratische_funktionen][-- einer Parabel]{sec:scheitel}}
\end{itemize}
%
\lang{de}{Schnitt}
\begin{itemize}
\item \lang{de}{\ref[content_37_schnittpunkte][ Ebene - Ebene]{sec:6_7_Ebene-Ebene}}
\item \lang{de}{\ref[content_37_schnittpunkte][Gerade - Ebene]{sec:6_6_Gerade-Ebene}}
\item \lang{de}{\ref[content_37_schnittpunkte][Gerade - Gerade]{sec:6_5_Gerade-Gerade}}
\item \lang{de}{\ref[content_37_schnittpunkte][Punkt - Ebene]{sec:6_4_Punkt-Ebene}}
\item \lang{de}{\ref[content_37_schnittpunkte][ Punkt - Gerade]{sec:6_3_Punkt-Gerade}}
\item \lang{de}{\ref[content_01_zahlenmengen][Schnitt, Schnittmenge]{def:mengenoperationen}} 
\end{itemize}
%
\lang{de}{Schranke}
\begin{itemize}
\item \lang{de}{\ref[content_06_supremum_infimum][Schranke, obere und untere]{def:beschraenkt}}
\end{itemize}
% 
\lang{de}{Sinus}
\begin{itemize}
\item \lang{de}{\ref[content_17_trigonometrie_im_dreieck][Sinus (geometrische Definition)]{sin-cos-im-dreieck}}
\item \lang{de}{\ref[content_19_allgemeiner_sinus_cosinus][Sinus-Funktion]{def_trig1}}
\item \lang{de}{\ref[content_28_exponentialreihe][ Sinus-Funktion, komplexe]{sec:sinus-kosinus}}
\item \lang{de}{\ref[content_19_allgemeiner_sinus_cosinus][Nullstellen]{sin_cos_roots}}
\item \lang{de}{\ref[content_19_allgemeiner_sinus_cosinus][Periodizität]{sin_cos_periodic}}
\item \lang{de}{\ref[content_19_allgemeiner_sinus_cosinus][Pythagoras, trigonometrischer]{trig-pythagoras}}
\item \lang{de}{\ref[content_28_exponentialreihe][Reihendarstellung]{sec:sinus-kosinus}}
\item \lang{de}{\ref[content_19_allgemeiner_sinus_cosinus][Symmetrie]{sin_cos_symmetry}}
\item \lang{de}{\ref[content_19_allgemeiner_sinus_cosinus][Verschiebung]{sin_cos_verschiebung}}
\end{itemize}
%
\lang{de}{Sinus hyperbolicus}
\begin{itemize}
\item \lang{de}{\ref[content_28_exponentialreihe][Sinus hyperbolicus]{sec:sinh-cosh}}
\end{itemize}
%
\lang{de}{Skalarmultiplikation}
\begin{itemize}
\item \lang{de}{\ref[content_27_vektoren][Skalarmultiplikation (Definition) ]{def:scalar_mult}}
%\item \lang{de}{\ref[content_31_skalarprodukt][Kollinearität zweier Vektoren]{def:kollinear}} %??? Was macht das hier?
\end{itemize}
%
\lang{de}{Skalarprodukt}
\begin{itemize}
\item \lang{de}{\ref[content_10c_Orthogonalbasen][Skalarprodukt]{def:skalarprodukt}}
\item \lang{de}{\ref[content_31_skalarprodukt][Standard-Skalarprodukt ]{def:skalarprodukt}}
\item \lang{de}{\ref[content_31_skalarprodukt][ euklidisches Skalarprodukt ]{def:skalarprodukt}}
\item \lang{de}{\ref[content_31_skalarprodukt][Rechenregeln (euklidisches Skalarprodukt, Standard-Skalarprodukt)]{sec:regeln}}
\item \lang{de}{\ref[content_33_winkel][Winkel zwischen zwei Vektoren]{def:winkel}}
\end{itemize}


%
\lang{de}{Spaltenrang}
\begin{itemize}
\item \lang{de}{Spaltenrang \ref[content_45_matrixrang][(Teil 1),]{sec:zeilenrang-spaltenrang}\ref[content_06_umformungen_rang][ (Teil 3b)]{def:zeilenSpaltenRang}}
\end{itemize}
%
\lang{de}{Spatprodukt}
\begin{itemize}
\item \lang{de}{\ref[content_34_vektorprodukt][Spatprodukt (Spatvolumen)]{rem:spatprodukt}}
\end{itemize}
%
\lang{de}{Stammfunktion}
\begin{itemize}
\item \lang{de}{\ref[content_25_stammfunktion][Stammfunktion (Definition) (Teil 1)]{def:stammfkt},
                \ref[content_09_integrierbare_funktionen][(Teil 3a)]{def:stammf}}
\item \lang{de}{\ref[content_25_stammfunktion][elementare Beispiele (Teil 1)]{sec:elementare_Stammfunktionen}}
\item \lang{de}{\ref[content_09_integrierbare_funktionen][wichtige Beispiele (Teil 3a)]{rule:diffregeln}}
\item \lang{de}{\ref[content_13_partialbruchzerlegung][negativer Potenzen linearer Funktionen]{rule:linearrational}}
\item \lang{de}{\ref[content_13_partialbruchzerlegung][rationaler Funktionen mit quadratischen Nennern]{thm:stammfkt_rationale}}
\item \lang{de}{\ref[content_13_partialbruchzerlegung][rationaler Funktionen generell (Partialbruchzerlegung)]{sec:summary}}
\item \lang{de}{\ref[content_25_stammfunktion][Summen- und Faktorregel (Teil 1)]{rule:sum_fakt_regeln},
                \ref[content_11_partielle_integration][(Teil 3a)]{rule:Linear_Stammfkt}}
\end{itemize}
%
\lang{de}{stationäre Stelle, stationärer Punkt}
\begin{itemize}
\item \lang{de}{\ref[content_22_extremstellen][stationäre Stelle einer Funktion]{def:krit_pkt}}
\item \lang{de}{\ref[content_22_extremstellen][stationärer Punkt eines Graphen]{def:krit_pkt}}
\end{itemize}
%
\lang{de}{Steigung}
\begin{itemize}
\item \lang{de}{\ref[content_20_ableitung_als_tangentensteigung][Sekantentensteigung]{steigung_sekante}}
\item \lang{de}{\ref[content_20_ableitung_als_tangentensteigung][Tangentensteigung(Teil 1)]{thm:steigung_tangente},
                \ref[content_01_differenzenquotient][(Teil 3a)]{thm:steig-abl}}
\end{itemize}
%
\lang{de}{Stetigkeit}
\begin{itemize}
\item \lang{de}{$\epsilon$-$\delta$-Kriterium \ref[content_29_stetigkeit_definitionen][(in $\R$), ]{sec:eps-delta}\ref[content_53_Stetigkeit][ (im $\R^n$)]{def:stetigkeit_n-dim}}
\item \lang{de}{Folgenstetigkeit \ref[content_29_stetigkeit_definitionen][(in $\R$), ]{sec:folgenkriterium} \ref[content_53_Stetigkeit][ (im $\R^n$)]{thm:äquivalent-zu-stetig}}
\item \lang{de}{\ref[content_29_stetigkeit_definitionen][links- und rechtsseitige]{sec:einseitige-stetigkeit}}
\item{\ref[content_53_Stetigkeit][Komponentenweise Stetigkeit]{thm:äquivalent-zu-stetig}}
\item \lang{de}{\ref[content_30_elem_funktionen][Kompositionen stetiger Funktionen]{sec:kompositionen}}
\item \lang{de}{\ref[content_33_zwischenwertsatz][Minimum-Maximum-Satz]{thm:minmax}}
\item \lang{de}{\ref[content_33_zwischenwertsatz][Zwischenwertsatz]{thm:zwischenwertsatz}}
\end{itemize}
%






\lang{de}{Streckfaktor}
\begin{itemize}
\item \lang{de}{\ref[content_08_quadratische_funktionen][Streckfaktor einer Parabel]{def:quadratic_func}}
\end{itemize}
%
\lang{de}{Substitutionsregel}
\begin{itemize}
\item \lang{de}{\ref[content_12_substitutionsregel][Substitutionsregel (Teil 3a)]{sec:allg-subst-reg}}
\item \lang{de}{\ref[content_26_flaechen_zwischen_graphen][Substitutionsregel, lineare (Teil 1)]{subst},
                \ref[content_12_substitutionsregel][(Teil 3a)]{{rule:lin-subst}}}
\end{itemize}
%
\lang{de}{Subtraktion}
\begin{itemize}
\item \lang{de}{\ref[content_02_rechengrundlagen_terme][ reeller Zahlen]{def:grundrechenarten}}
\item \lang{de}{\ref[content_03_bruchrechnung][von Brüchen/Bruchtermen]{add}}
\item \lang{de}{\ref[content_27_vektoren][von Vektoren]{Bem_Vektor_Addition}}
\end{itemize}
%
\lang{de}{Summationsindex}
\begin{itemize}
\item \lang{de}{\ref[content_02_rechengrundlagen_terme][Summationsindex]{def:summenzeichen}}
\item \lang{de}{\ref[content_02_rechengrundlagen_terme][Verschiebung des Summationsindex]{Indexverschiebung}}
\end{itemize}
%
\lang{de}{Summe}
\begin{itemize}
\item \lang{de}{\ref[content_02_rechengrundlagen_terme][Summe reeller Zahlen]{def:grundrechenarten}}
\item \lang{de}{\ref[content_02_rechengrundlagen_terme][Summenzeichen]{def:summenzeichen}}
\end{itemize}
%
\lang{de}{Summenregel der Ableitung}
\begin{itemize}
\item \lang{de}{\ref[content_20_ableitung_als_tangentensteigung][eindimensional (Teil 1)]{rule:additiv},
                \ref[content_02_ableitungsregeln][(Teil 3a)]{rule:summenregel}}
\item \lang{de}{\ref[content_54_Differentiation][mehrdimensional]{thm:summen-und-produktregel}}
\end{itemize}
%
%
\lang{de}{Supremum}
\begin{itemize}
\item \lang{de}{\ref[content_07_vollstaendigkeit][Supremum]{def:sup_inf}}
\end{itemize}
%
\lang{de}{Surjektivität}
\begin{itemize}
\item \lang{de}{\ref[content_11_injektiv_surjektiv_bijektiv][surjektiv]{sec:inj-sur-bi}}
\end{itemize}
%
%
\lang{de}{symmetrische Matrix}
\begin{itemize}
\item \lang{de}{symmetrische Matrix (Definition)\ref[content_44_transponierte_matrix][ (Teil 1),]{def:symmetrische_matrix}
\ref[content_12_symmetrische_matrizen][(Teil 3b) ]{def:symmetrischeMatrix}}
\item \lang{de}{\ref[content_12_symmetrische_matrizen][Eigenwerte reeller symmetrischer Matrizen]{sec:eigenwerteRellerSymmetrischerMatrizen}}
\item \lang{de}{\ref[content_12_symmetrische_matrizen][Eigenvektoren reeller symmetrischer Matrizen]{sec:eigenvektorenRellerSymmetrischerMatrizen}}
\item \lang{de}{\ref[content_12_symmetrische_matrizen][Hauptachsentransformation]{thm:hauptachsentransformation}}
\end{itemize}
%
\anchor{sec:t}{\textbf{T}}
\\
%
\lang{de}{Tangens}
\begin{itemize}
\item \lang{de}{\ref[content_17_trigonometrie_im_dreieck][Definition, geometrische ]{sin-cos-im-dreieck}}
\item \lang{de}{\ref[content_19_allgemeiner_sinus_cosinus][Tangens-Funktion]{def:tan_cot}}
\item \lang{de}{\ref[content_19_allgemeiner_sinus_cosinus][Nullstellen]{tan_roots}}
\item \lang{de}{\ref[content_19_allgemeiner_sinus_cosinus][Periodizität]{tan_eigenschaften}}
\item \lang{de}{\ref[content_19_allgemeiner_sinus_cosinus][Symmetrie]{tan_eigenschaften}}
\item \lang{de}{\ref[content_19_allgemeiner_sinus_cosinus][Verschiebung]{tan_eigenschaften}}
\end{itemize}
%
\lang{de}{Tangente}
\begin{itemize}
\item \lang{de}{\ref[content_20_ableitung_als_tangentensteigung][Tangente (Teil 1)]{thm:steigung_tangente},
                \ref[content_01_differenzenquotient][(Teil 3a)]{thm:steig-abl}}
\item \lang{de}{\ref[content_20_ableitung_als_tangentensteigung][Tangentensteigung (Teil 1)]{sec:ableitung},
                \ref[content_01_differenzenquotient][(Teil 3a)]{sec:ableitung}}
\end{itemize}
%
\lang{de}{Tangentialvektor}
\begin{itemize}
\item \lang{de}{\ref[content_51_Kurven][Tangentialvektor]{def:tangentialvektor}}
\end{itemize}
%
\lang{de}{Tautologie}
\begin{itemize}
\item \lang{de}{\ref[content_04_aussagen_aequivalenzumformungen][Tautologie]{def:tautologie_widerspruch}}
\end{itemize}
%
\lang{de}{Taylor}
\begin{itemize}
\item \lang{de}{\ref[content_04_taylor_polynom][Entwicklung, -Näherung]{sec:polynom}}
\item \lang{de}{\ref[content_04_taylor_polynom][Entwicklungsstelle]{def:taylorpoly}}
\item \lang{de}{\ref[content_04_taylor_polynom][Koeffizienten]{def:taylorpoly}}
\item \lang{de}{\ref[content_04_taylor_polynom][Polynom $n$-ter Ordnung]{def:taylorpoly}}
\item \lang{de}{\ref[content_04_taylor_polynom][Polynom (Approximationseigenschaft)]{thm:qual-taylor-formel}}
\item \lang{de}{\ref[content_04_taylor_polynom][Taylor-Formel (qualitative)]{thm:qual-taylor-formel}}
\item \lang{de}{\ref[content_04_taylor_polynom][Reihe]{def:taylorreihe}}
\item \lang{de}{\ref[content_04_taylor_polynom][Restglied]{def:restglied}}
\item \lang{de}{\ref[content_04_taylor_polynom][Lagrange-Restgliedformel]{thm:rg-Lagrange}}
\item \lang{de}{\ref[content_04_taylor_polynom][Restglied-Abschätzung]{rg-abschaetzung}}
%\item \lang{de}{\ref[content_04_taylor_polynom][reell-analytische Funktionen]{def:reell-analyt-Fktn}}
\end{itemize}
%
\lang{de}{Teilfolge}
\begin{itemize}
\item \lang{de}{\ref[content_16_konvergenzkriterien][Teilfolge]{ex:teilfolge-hpunkt}}
\end{itemize}
%
\lang{de}{Teilmenge}
\begin{itemize}
\item \lang{de}{\ref[content_01_zahlenmengen][Teilmenge]{def:mengenrelationen}}
\end{itemize}

%
\lang{de}{Terme, Termumformung}
\begin{itemize}
\item \lang{de}{\ref[content_02_rechengrundlagen_terme][Terme, Termumformungen]{sec:terme}}
\item \lang{de}{\ref[content_02_rechengrundlagen_terme][Klammer- vor Potenz- vor Punkt- vor Strich-Regel]{rule:punkt-vor-strich}}
\item \lang{de}{\ref[content_02_rechengrundlagen_terme][Kommutativ-, Assoziativ- und Distributivgesetz]{rule:rechengesetze}}
\item \lang{de}{\ref[content_02_rechengrundlagen_terme][Ausmultiplizieren, Ausklammern, Faktorisieren]{rule:rechengesetze}}
\item \lang{de}{\link{content_03_bruchrechnung}{Bruchterme}}
\end{itemize}

%
\lang{de}{transponierte Matrix}
\begin{itemize}
\item \lang{de}{\ref[content_03_transponierte][transponierte Matrix]{def:transponierte-matrix}}
\end{itemize}
%
\lang{de}{trigonometrische Funktionen}
\begin{itemize}
\item \lang{de}{\ref[content_28_exponentialreihe][Additionstheoreme und Euler'sche Identität]{rem:sin_cos_exp}}
\item \lang{de}{\ref[content_35_trigonom_funktionen][Arkus-Funktionen]{arcus.funktionen}}
\item \lang{de}{\ref[content_35_trigonom_funktionen][Sinus und Kosinus]{sincos.definition.1}}
\item \lang{de}{\ref[content_28_exponentialreihe][Sinus und Kosinus, komplex]{sec:sinus-kosinus}}
\item \lang{de}{\ref[content_28_exponentialreihe][Sinus hyperbolicus und Kosinus hyperbolicus]{sec:sinh-cosh}}
\item \lang{de}{\ref[content_35_trigonom_funktionen][Tangens und Cotangens]{tan-cot.definition}}
\end{itemize}
%
\anchor{sec:u}{\textbf{U}}
\\
%
%
\lang{de}{Überabzählbarkeit}
\begin{itemize}
\item \lang{de}{\ref[content_13_unabzaehlbarkeit][Überabzählbarkeit]{def:abzaehlbarkeit}}
\end{itemize}
%
\lang{de}{Umkehrabbildung, Umkehrfunktion}
\begin{itemize}
\item \lang{de}{\ref[content_11_injektiv_surjektiv_bijektiv][Umkehrabbildung]{sec:umkehrabbildung}}
\item \lang{de}{\ref[content_02_ableitungsregeln][Ableitung der Umkehrfunktion (eindimensional)]{rule:abl-umkehrfkt}}
\item \lang{de}{
\ref[content_55_Lokale_Umkehrbarkeit][totale Ableitung der Umkehrfunktion (im $\R^n$)]{tot_Abl_Umkehrf}}
\item \lang{de}{\ref[content_12_reelle_funktionen_monotonie][Monotonie der Umkehrfunktion]{thm:inversmonoton}}
\item \lang{de}{\ref[content_11_injektiv_surjektiv_bijektiv][partielle]{def:partielle-umkehrfunktion}}
\item \lang{de}{\ref[content_55_Lokale_Umkehrbarkeit][lokale Umkehrbarkeit (im $\R^n$)]{thm:lokale-umkehrbarkeit}}
\end{itemize}
%
\lang{de}{Umordnung}
\begin{itemize}
\item \lang{de}{\ref[content_26_produkt_von_reihen][Umordnung von Reihen]{sec:reihen-umordnung}}
\end{itemize}
%
\lang{de}{Untersumme}
\begin{itemize}
%\item \lang{de}{\ref[content_07_ober_und_untersumme][Intervall-Zerlegung]{def:intervall-zer}}
\item \lang{de}{\ref[content_07_ober_und_untersumme][Untersumme]{def:Obersum-untersum}}
\end{itemize}
%
\lang{de}{Untervektorraum}
\begin{itemize}
\item \lang{de}{\ref[content_10a_vektorraum][Untervektorraum]{def:Unter_VR}}
\end{itemize}
%
\lang{de}{Urbild}
\begin{itemize}
\item \lang{de}{\ref[content_10_abbildungen_verkettung][Urbild]{def:urbild}}
\end{itemize}
%
% \lang{de}{...}
% \begin{itemize}

% \end{itemize}
%
\anchor{sec:v}{\textbf{V}}
\\

%
\lang{de}{Vektor}
\begin{itemize}
\item \lang{de}{\ref[content_27_vektoren][Vektor (Definition)]{Gegenvektor}}
\item \lang{de}{\ref[content_27_vektoren][Gegenvektor]{Gegenvektor}}
\item \lang{de}{\ref[content_27_vektoren][Komponentendarstellung]{sec:kompon_spalt_darst}}
\item \lang{de}{\ref[content_32_laenge_norm][Länge, Norm, Betrag]{def:euklidische_norm}}
\item \lang{de}{\ref[content_27_vektoren][Spaltenvektor]{sec:kompon_spalt_darst}}
\item \lang{de}{\ref[content_27_vektoren][Nullvektor]{Gegenvektor}}
\item \lang{de}{\ref[content_33_winkel][--, orthogonaler]{def:orthogonal}}
\item \lang{de}{\ref[content_27_vektoren][Ortsvektor]{def:orts_verb_vec}}
\item \lang{de}{\ref[content_27_vektoren][Verbindungsvektor]{def:orts_verb_vec}}

\end{itemize}

%
\lang{de}{Vektorprodukt}
\begin{itemize}
\item \lang{de}{\ref[content_34_vektorprodukt][Vektorprodukt (Kreuzprodukt)]{def:Vektorprodukt}}
\item \lang{de}{\ref[content_34_vektorprodukt][Eigenschaften]{sec:eigenschaften}}
\item \lang{de}{\ref[content_34_vektorprodukt][Rechenregeln]{sec:rechenregeln}}

\end{itemize}
%
\lang{de}{Vektorraum}
\begin{itemize}
\item \lang{de}{\ref[content_10a_vektorraum][$\K$-Vektorraum]{def:Allg_VR}}
\item \lang{de}{\ref[content_29_linearkombination][reeller Vektorraum]{supp:allg-vektorraum}}
\item \lang{de}{\ref[content_10a_vektorraum][Untervektorraum]{def:Unter_VR}}
\end{itemize}
%
\lang{de}{Vektorrechnung im $\R^n$}
\begin{itemize}
\item \lang{de}{\ref[content_27_vektoren][Addition von Vektoren]{def:vec-addition}}
\item \lang{de}{\ref[content_31_skalarprodukt][Multiplikation von Vektoren, Skalarprodukt]{def:skalarprodukt}}
\item \lang{de}{\ref[content_29_linearkombination][Rechenregeln]{Vektorraum_R_n}}
\item \lang{de}{\ref[content_27_vektoren][Subtraktion von Vektoren]{Bem_Vektor_Addition}}
\item \lang{de}{\ref[content_27_vektoren][Verbindungsvektor berechnen]{thm:verbindungsvektor}}
\end{itemize}
%
\lang{de}{Vereinigung}
\begin{itemize}
\item \lang{de}{\ref[content_01_zahlenmengen][Vereinigung, Vereinigungsmenge]{def:mengenoperationen}}
\end{itemize}
%
\lang{de}{Verkettung}
\begin{itemize}
\item \lang{de}{\ref[content_21_kettenregel][Verkettung zweier Funktionen]{addition.definition.1}}
\end{itemize}
%
\lang{de}{Viëta, Satz von}
\begin{itemize}
\item \lang{de}{\ref[content_05_loesen_gleichungen_und_lgs][Satz von Viëta]{thm:Satz_von_Vieta}}
\end{itemize}
%
\lang{de}{vollständige Induktion}
\begin{itemize}
\item \lang{de}{\ref[content_02_vollstaendige_induktion][vollständige Induktion]{sec:induktionsprinzip}}
\end{itemize}
%
\lang{de}{Vollständigkeit}
\begin{itemize}
\item \lang{de}{\ref[content_07_vollstaendigkeit][Vollständigkeit]{def:vollstaendigkeit}}
\end{itemize}
%
\anchor{sec:w}{\textbf{W}}
\\
%
\lang{de}{Wahrheitstafel}
\begin{itemize}
\item \lang{de}{\ref[content_04_aussagen_aequivalenzumformungen][Wahrheitstafel]{rem:wahrheitstafeln}}
\end{itemize}
%
\lang{de}{Wahrheitswert}
\begin{itemize}
\item \lang{de}{\ref[content_04_aussagen_aequivalenzumformungen][Wahrheitswert]{def:objekte_aussagenlogik}}
\end{itemize}
%
\lang{de}{ Wendepunkt, Wendestelle}
\begin{itemize}
\item \lang{de}{\ref[content_23_kurvendiskussion][Wendestelle (Definition)]{sec:wendest_sattel}}
% \item \lang{de}{\ref[content_23_kurvendiskussion][hinreichende Bedingung]{thm:notw_hinr_bedg_wendepkt}}
% \item \lang{de}{\ref[content_23_kurvendiskussion][notwendige Bedingung]{thm:notw_hinr_bedg_wendepkt}}
\end{itemize}
%
%
\lang{de}{Wertemenge}
\begin{itemize}
\item \lang{de}{\ref[content_10_abbildungen_verkettung][Wertemenge]{def:Wertemenge}} 
\end{itemize}
%
\lang{de}{Widerspruch}
\begin{itemize}
\item \lang{de}{\ref[content_04_aussagen_aequivalenzumformungen][Widerspruch in der Aussagenlogik]{def:tautologie_widerspruch}}
\end{itemize}
%
\lang{de}{Winkel}
\begin{itemize}
\item \lang{de}{\ref[content_33_winkel][Winkel zwischen zwei Vektoren]{def:winkel}}
\end{itemize}
%
\lang{de}{Winkelmaß}
\begin{itemize}
\item \lang{de}{\link{content_18_grad_und_bogenmass}{Winkelmaß}}
\end{itemize}

%
\lang{de}{Wurzel}
\begin{itemize}
\item \lang{de}{\ref[content_02_rechengrundlagen_terme][$n$-te Wurzel]{def:wurzel}}
\item \lang{de}{\ref[content_13_wurzelfunktionen][$n$-te Wurzel (Funktion)]{def:nth-root}}
\item \lang{de}{\ref[content_36_anwendungen][komplexe]{def:komplexe-wurzel}}
\item \lang{de}{\ref[content_30_elem_funktionen][Stetigkeit]{sec:wurzel-und-ln}}
\end{itemize}

%
\lang{de}{Wurzelgesetze}
\begin{itemize}
\item \lang{de}{\ref[content_13_wurzelfunktionen][Wurzelgesetze]{allgwurzel_gesetze}}
\end{itemize}
%
\lang{de}{Wurzelkriterium}
\begin{itemize}
\item \lang{de}{\ref[content_25_konvergenz_kriterien][Wurzelkriterium]{thm:wurzelkriterium}}
\end{itemize}
%
\anchor{sec:x}{\textbf{X}}
\\
%
% \lang{de}{...}
% \begin{itemize}

% \end{itemize}
% %
\anchor{sec:y}{\textbf{Y}}
\\
%
% \lang{de}{...}
% \begin{itemize}

% \end{itemize}
% %
\anchor{sec:z}{\textbf{Z}}
\\
%
%
\lang{de}{$\Z$}
\begin{itemize}
\item \lang{de}{\ref[content_01_zahlenmengen][$\Z$ (ganze Zahlen)]{def:zahlenmengen}}
\end{itemize}
%
%
\lang{de}{Zahlenmengen}
\begin{itemize}
\item \lang{de}{\ref[content_01_zahlenmengen][Zahlenmengen]{def:zahlenmengen}}
\end{itemize}
%
\lang{de}{Zahlenstrahl}
\begin{itemize}
\item \lang{de}{\ref[content_01_zahlenmengen][Zahlenstrahl]{introduction_R}}
\end{itemize}
%
\lang{de}{Zeilenrang}
\begin{itemize}
\item \lang{de}{Zeilenrang \ref[content_45_matrixrang][(Teil 1),]{sec:zeilenrang-spaltenrang}\ref[content_06_umformungen_rang][ (Teil 3b)]{def:zeilenSpaltenRang}}
\end{itemize}
%
\lang{de}{Zielbereich}
\begin{itemize}
\item \lang{de}{\ref[content_10_abbildungen_verkettung][Zielbereich]{def:Wertemenge}} 
\end{itemize}
%
\lang{de}{Zwischenwertsatz}
\begin{itemize}
\item \lang{de}{\ref[content_33_zwischenwertsatz][Zwischenwertsatz]{thm:zwischenwertsatz}}
\end{itemize}
%
%
\lang{de}{Zykloide}
\begin{itemize}
\item \lang{de}{\ref[content_51_Kurven][Zykloide]{ex:zykloide}}
\end{itemize}

%%%%%%%%%%%%%%%%%%%%%%%%%%%%%%%%%%%%%%%%%%%%%%%%%%%%%%%%%%%%%%%%%%%%%%%%%%%%%%%%%%%%%%%%%%%%%%%%%%%%%%%%%%%%%%%%%%%%%



\end{content}
