\documentclass{mumie.problem.gwtmathlet}
%$Id$
\begin{metainfo}
  \name{
    \lang{de}{A01: Infimum/Supremum}
    \lang{en}{input numbers}
  }
  \begin{description} 
 This work is licensed under the Creative Commons License Attribution 4.0 International (CC-BY 4.0)   
 https://creativecommons.org/licenses/by/4.0/legalcode 

    \lang{de}{}
    \lang{en}{}
  \end{description}
  \corrector{system/problem/GenericCorrector.meta.xml}
  \begin{components}
    \component{js_lib}{system/problem/GenericMathlet.meta.xml}{gwtmathlet}
  \end{components}
  \begin{links}
  \end{links}
  \creategeneric
\end{metainfo}
\begin{content}
\usepackage{mumie.ombplus}
\usepackage{mumie.genericproblem}

\lang{de}{\title{A01: Infimum/Supremum}}
\lang{en}{\title{Problem 1}}

\begin{block}[annotation]
	Im Ticket-System: \href{http://team.mumie.net/issues/9784}{Ticket 9784}
\end{block}

\begin{problem}

	\begin{question}
		\lang{de}{
			\text{Gegeben sei die folgende Menge \\
				$M:=\left\{  \var{M}   \,|\, \, x>0 \text{ und } \var{a}<x^{2}\leq \var{b} \right\}\,.$\\
				Es sei $c:=\inf(M)$. Geben Sie $c^{2}$ an.
				}
		}
		\explanation{}
		\type{input.number}
		\field{rational}
		

    \precision[false]{3}
    \displayprecision{3}
    \correctorprecision{3}
		
		\begin{variables}
			\randint{xx}{2}{20}
			\randint{y}{2}{20}
			\randadjustIf{xx,y}{xx=y}
			\function[calculate]{a}{1/2*(xx+y-abs(xx-y))}
			\function[calculate]{b}{1/2*(xx+y+abs(xx-y))}
			\function[calculate]{c2}{a}
			
			\function[normalize]{M}{1/2 * x + 1/2*a/x}
		\end{variables}
		
		\begin{answer}
			\text{$c^2=$}
			\solution{c2}
			\explanation{ Für alle $x^2 \in (\var{a},\var{b}]$ gilt\\
			$(x-\sqrt{\var{a}})^{2}\geq 0\,.$\\
			Dies ist äquivalent zu \\
			$x^{2}-2x\sqrt{\var{a}}+\var{a}\geq 0   \Leftrightarrow   2\sqrt{\var{a}}\leq x+\frac{\var{a}}{x}
            \Leftrightarrow   \sqrt{\var{a}}\leq \frac{x}{2}+\frac{\var{a}}{2x}\,.$\\
			Also ist $M$ durch $\sqrt{\var{a}}$ nach unten beschränkt. 
			Tatsächlich gilt hier sogar $\inf(M)=\sqrt{\var{a}}$. 
			Dies wird durch das sogenannte Heron-Verfahren garantiert.}
		\end{answer}
		
	\end{question}
	
\end{problem}

\embedmathlet{gwtmathlet}

\end{content}