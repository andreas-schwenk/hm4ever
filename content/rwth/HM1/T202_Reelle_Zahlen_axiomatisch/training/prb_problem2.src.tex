\documentclass{mumie.problem.gwtmathlet}
%$Id$
\begin{metainfo}
  \name{
    \lang{de}{A02: Infimum/Supremum}
    \lang{en}{input numbers}
  }
  \begin{description} 
 This work is licensed under the Creative Commons License Attribution 4.0 International (CC-BY 4.0)   
 https://creativecommons.org/licenses/by/4.0/legalcode 

    \lang{de}{Die Beschreibung}
    \lang{en}{}
  \end{description}
  \corrector{system/problem/GenericCorrector.meta.xml}
  \begin{components}
    \component{js_lib}{system/problem/GenericMathlet.meta.xml}{gwtmathlet}
  \end{components}
  \begin{links}
  \end{links}
  \creategeneric
\end{metainfo}
\begin{content}
\usepackage{mumie.ombplus}
\usepackage{mumie.genericproblem}

\lang{de}{\title{A02: Infimum/Supremum}}
\lang{en}{\title{Problem 2}}

\begin{block}[annotation]
	Im Ticket-System: \href{http://team.mumie.net/issues/9785}{Ticket 9785}
\end{block}

\begin{problem}

	\begin{question}
		\lang{de}{
			\text{Gegeben sei die folgende Menge\\
			$M:=\left\{\var{a}+\frac{\var{b}}{n}\,|\,n\in\N\right\}\,.$\\
			Es sei $c:=\inf(M), C:=\sup(M)$. Geben Sie $c$ und $C$ an.
				}
		}
		\explanation{}
		\type{input.number}
		\field{rational}
		

    \precision[false]{3}
    \displayprecision{3}
    \correctorprecision{3}
		
		\begin{variables}
			\randint{x}{1}{20}
			\randint{y}{1}{20}
			\randadjustIf{x,y}{x=y}
			\function[calculate]{a}{1/2*(x+y-abs(x-y))}
			\function[calculate]{b}{1/2*(x+y+abs(x-y))}
			\function[calculate]{c}{a}
			\function[calculate]{C}{a+b}
		\end{variables}
		
		\begin{answer}
			\text{$C=$}
			\solution{C}
			\explanation{ Für alle $n\in\N$ gilt $\var{a}+\frac{\var{b}}{n+1}<\var{a}+\frac{\var{b}}{n}$. \\
			Also gilt für alle $n\in\N$ die Ungleichung $\var{a}+\frac{\var{b}}{n}\leq \var{a}+\var{b}$. \\
			Daher ist $C$ sogar das Maximum von $M$.}
		\end{answer}
		\begin{answer}
			\text{$c=$}
			\solution{c}
			\explanation{ Wir behaupten, dass $\var{a}=c$ ist. 
			Zunächst gilt offensichtlich $x\leq c$ für alle $x\in M$. \\
			Angenommen, $c$ ist nicht die größte untere Schranke. 
			Dann gibt es ein $h>\var{a}$ mit der Eigenschaft $x\geq h$ für alle $x\in M$. \\
			Aus der Vorlesung wissen wir, dass es ein $n_{0}\in\N$ gibt mit der Eigenschaft $n_{0}>\frac{\var{b}}{h-\var{a}}$. 
			Damit erhalten wir \\
			$h=\var{a}+(h-\var{a})=\var{a}+\frac{\var{b}}{\var{b}/(h-\var{a})}>\var{a}+\frac{\var{b}}{n_{0}}=:x_{0}\in M$\\
			als Widerspruch. Also ist $\var{a}$ bereits die kleinste untere Schranke.}
		\end{answer}
	\end{question}
	
\end{problem}

\embedmathlet{gwtmathlet}


\end{content}