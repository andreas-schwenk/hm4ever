%$Id:  $
\documentclass{mumie.article}
%$Id$
\begin{metainfo}
  \name{
    \lang{de}{Überblick: Charakterisierung der reellen Zahlen}
    \lang{en}{Overview: Characterisation of the real numbers}
  }
  \begin{description} 
 This work is licensed under the Creative Commons License Attribution 4.0 International (CC-BY 4.0)   
 https://creativecommons.org/licenses/by/4.0/legalcode 

    \lang{de}{Beschreibung}
    \lang{en}{Description}
  \end{description}
  \begin{components}
  \end{components}
  \begin{links}
\link{generic_article}{content/rwth/HM1/T202_Reelle_Zahlen_axiomatisch/g_art_content_07_vollstaendigkeit.meta.xml}{content_07_vollstaendigkeit}
\link{generic_article}{content/rwth/HM1/T202_Reelle_Zahlen_axiomatisch/g_art_content_06_supremum_infimum.meta.xml}{content_06_supremum_infimum}
\link{generic_article}{content/rwth/HM1/T202_Reelle_Zahlen_axiomatisch/g_art_content_05_anordnungsaxiome.meta.xml}{content_05_anordnungsaxiome}
\link{generic_article}{content/rwth/HM1/T202_Reelle_Zahlen_axiomatisch/g_art_content_04_koerperaxiome.meta.xml}{content_04_koerperaxiome}
\end{links}
  \creategeneric
\end{metainfo}
\begin{content}
\begin{block}[annotation]
	Im Ticket-System: \href{https://team.mumie.net/issues/30135}{Ticket 30135}
\end{block}


\begin{block}[annotation]
Im Entstehen: Überblicksseite für Kapitel Charakterisierung der reellen Zahlen
\end{block}

\usepackage{mumie.ombplus}
\ombchapter{1}
\lang{de}{\title{Überblick: Charakterisierung der reellen Zahlen}}
\lang{en}{\title{Overview: Characterisation of the real numbers}}



\begin{block}[info-box]
\lang{de}{\strong{Inhalt}}
\lang{en}{\strong{Contents}}


\lang{de}{
    \begin{enumerate}%[arabic chapter-overview]
   \item[2.1] \link{content_04_koerperaxiome}{Körperaxiome und Rechenregeln}
   \item[2.2] \link{content_05_anordnungsaxiome}{Anordnungsaxiome und Betrag}
   \item[2.3] \link{content_06_supremum_infimum}{Beschränkte Mengen}
   \item[2.4] \link{content_07_vollstaendigkeit}{Supremum, Infimum und Vollständigkeit}
   \end{enumerate}
} %lang

\end{block}

\begin{zusammenfassung}

\lang{de}{Dieses Kapitel fasst die mathematischen Hintergründe für das Rechnen mit reellen Zahlen zusammen.
\\
Die Rechengesetze Kommutativität, Assoziativität und Distributivität für Addition und Multiplikation sind Ihnen aus der Schule bekannt.
Fordert man für einen Zahlenbereich noch die Existenz von neutralen und inversen Elementen, so spricht man von einem Körper. 
Die rationalen Zahlen $\Q$ und die reellen Zahlen $\R$ sind Körper. Mit den komplexen Zahlen $\C$ werden Sie im nächsten Kapitel einen weiteren kennen lernen.

Besonders für die reellen Zahlen ist deren Anordnung, also die Existenz einer $<$-Beziehung: 
Von zwei verschiedenen reellen Zahlen können wir stets entscheiden, welche die kleinere ist.
Dadurch ist es auch sinnvoll, von oberen und unteren Schranken reeller Teilmengen zu sprechen, was auf die Begriffe Infimum, Supremum bzw. Minimum und Maximum führt.
\\Die reellen Zahlen zeichnen sich durch Vollständigkeit aus, d.h. jede beschränkte Teilmenge besitzt ein Infimum und ein Supremum. 
Dies ist eine für die Analysis sehr wichtige Eigenschaft. 
Die rationalen erfüllen sie beispielsweise nicht.
\\
Sie lernen außerdem wichtige Eigenschaften der bereits bekannten Betragsfunktion auf den reellen Zahlen kennen.}
\lang{en}{This chapter summarises the mathematical background used for computing with real numbers.
\\
The calculation laws of commutativity, associativity and distributivity for addition and multiplication will be familiar from school.
If we also require the existence of neutral and inverse elements for a set of numbers, then we are discussing a field.
The rational numbers $\Q$ and the real numbers $\R$ are fields. You will learn of another one with the complex numbers $\C$ in the next chapter.

One thing that is special about the real numbers is their ordering, that is the existence of a $<$ relationship:
If we have two distinct ral numbers, we can always decide which is smaller.
Because of this, it makes sense to talk about upper and lower bounds of subsets of the reals. This also leads us to the concepts of infimum and supremum,
and respectively mimimum and maximum.\\
The real numbers are characterised by completeness, i.e. every non-empty bounded subset has an infimum and a supremum.
This is a very important property in analysis.
The rational numbers, for example, do not fulfil this property.
\\
You will also learn some important properties of the familiar absolute value function on the real numbers.}


\end{zusammenfassung}

\begin{block}[info]\lang{de}{\strong{Lernziele}}
\lang{en}{\strong{Learning Goals}} 
\begin{itemize}[square]
\item \lang{de}{Sie wissen, welche Gesetze für Addition und Multiplikation erfüllt sein müssen, damit ein Zahlenbereich ein Körper ist.}
\lang{en}{Knowing which laws for addition and multiplication must be fulfilled for a set of numbers to be a field.}
\item \lang{de}{Sie erkennen die reellen Zahlen als angeordneten Körper, kennen die Eigenschaften der $<$-Relation.}
\lang{en}{Recognising the real numbers as an ordered field, knowing the properties of the relation $<$.}
\item \lang{de}{Sie kennen die Eigenschaften der Betragsfunktion, insbesondere die Dreiecksungleichung.}
\lang{en}{Knowing the properties of the absolute value function, especially the triangle inequality.}
\item \lang{de}{Sie kennen die Begriffe obere und untere Schranken für Teilmengen von $\R$.}
\lang{en}{Knowing the concepts of upper and lower bounds for subsets of $\R$.}
\item \lang{de}{Sie untersuchen Teilmengen von $\R$ auf Minima und Maxima bzw. Infima und Suprema.}
\lang{en}{Being able to examine subsets of $\R$ for mimima and maxima, respectively infima and suprema.}
\item \lang{de}{Sie kennen den Begriff der Vollständigkeit eines Körpers sowie ein Beispiel.}
\lang{en}{Knowing the concept of completeness of a field, and also an example.}
\end{itemize}
\end{block}




\end{content}
