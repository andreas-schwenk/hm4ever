%$Id:  $
\documentclass{mumie.article}
%$Id$
\begin{metainfo}
  \name{
    \lang{de}{Vollständigkeit}
    \lang{en}{Completeness}
  }
  \begin{description} 
 This work is licensed under the Creative Commons License Attribution 4.0 International (CC-BY 4.0)   
 https://creativecommons.org/licenses/by/4.0/legalcode 

    \lang{de}{Beschreibung}
    \lang{en}{Description}
  \end{description}
  \begin{components}
  \component{generic_image}{content/rwth/HM1/images/g_img_00_video_button_schwarz-blau.meta.xml}{00_video_button_schwarz-blau}
  \end{components}
  \begin{links}
    \link{generic_article}{content/rwth/HM1/T202_Reelle_Zahlen_axiomatisch/g_art_content_05_anordnungsaxiome.meta.xml}{content_05_anordnungsaxiome}
    \link{generic_article}{content/rwth/HM1/T202_Reelle_Zahlen_axiomatisch/g_art_content_06_supremum_infimum.meta.xml}{content_06_supremum_infimum}
    \link{generic_article}{content/rwth/HM1/T201neu_Vollstaendige_Induktion/g_art_content_02_vollstaendige_induktion.meta.xml}{content_02_vollstaendige_induktion}
  \end{links}
  \creategeneric
\end{metainfo}


\begin{content}
\usepackage{mumie.ombplus}
\ombchapter{2}
\ombarticle{4}

\lang{de}{\title{Supremum, Infimum und Vollständigkeit}}
\lang{en}{\title{Supremum, infimum and completeness}}

\begin{block}[annotation]
  Im Ticket-System: \href{http://team.mumie.net/issues/9649}{Ticket 9649}\\
\end{block}

\begin{block}[info-box]
\tableofcontents
\end{block}


\section{\lang{de}{Supremum und Infimum}\lang{en}{Supremum and infimum}}

\lang{de}{Wie im letzten Abschnitt bezeichne $(\K,<)$ einen angeordneten Körper.}
\lang{en}{As in the last section, let $(\K,<)$ denote an ordered field.}


\begin{definition}[Supremum / Infimum] \label{def:sup_inf}
    \lang{de}{Sei $A\subset \K$. Eine Zahl $S\in\K$ hei"st die \emph{\notion{kleinste obere Schranke}} 
    	von $A$ (oder das \emph{\notion{Supremum}} von $A$, geschrieben $S=\sup A$), wenn}
     \lang{en}{Let $A\subset \K$. A number $S\in\K$ is called the \emph{\notion{least upper bound}} 
     of $A$ (or the \emph{\notion{supremum}} of $A$, written $S=\sup A$), if}
    	\[
    	\lang{de}{S\quad\text{eine obere Schranke von}\; A\;\text{ist und}}
     \lang{en}{S\quad\text{is an upper bound of}\; A\;\text{and}}
    	\]
    	\[
    	\lang{de}{S\,\leq\, C\quad \text{ f\"ur jede obere Schranke }\;C\;\text{ von }\;A.}
     \lang{en}{S\,\leq\, C\quad \text{ for every upper bound }\;C\;\text{ of }\;A.}
    	\]
    	
    	\lang{de}{Eine Zahl $s\in\K$ ist die \emph{\notion{gr\"o\"ste untere Schranke} }
    	von $A$ (oder das \emph{\notion{Infimum}} von $A$,  geschrieben $s=\inf A$), wenn}
     \lang{en}{A number $s\in\K$ is the \emph{\notion{greatest lower bound}} of $A$ 
     (or the \emph{\notion{infimum}} of $A$, written $s=\inf A$), if}
    	\[
    	\lang{de}{s\quad\text{eine untere Schranke von}\;A\;\text{ist und}}
     \lang{en}{s\quad\text{is a lower bound of }\;A\;\text{and}}
    	\]
    	\[
    	\lang{de}{c\,\leq\, s\quad \text{ f\"ur jede untere Schranke }\;c\;\text{von}\;A.}
     \lang{en}{c\,\leq\, s\quad \text{ for every lower bound }\;c\;\text{of}\;A.}
    	\]
  \end{definition}

\begin{remark}
\lang{de}{Sei $A$ eine beliebige Teilmenge eines angeordneten Körpers $(\K,<)$.}
\lang{en}{Let $A$ be an arbitrary subset of an ordered field $(\K,<)$.}
\begin{enumerate}
\item \lang{de}{Besitzt $A$ ein Supremum bzw. ein Infimum, so ist dieses eindeutig bestimmt.}
\lang{en}{If $A$ has a supremum (or an infimum), then it is unique.}

    \begin{showhide}[\buttonlabels{\lang{de}{Zeige Begründung }\lang{en}{Show proof }}{\lang{de}{Verstecke Begründung}\lang{en}{Hide proof}}]
    \lang{de}{Seien nämlich $S_1$ und $S_2$ zwei Suprema von $A$. Da $S_2$ eine obere Schranke ist, gilt nach Definition
    des Supremums für $S_1$, dass $S_1\leq S_2$. Da aber auch $S_1$ eine obere Schranke ist, gilt nach Definition
    des Supremums für $S_2$, dass $S_2\leq S_1$. Insgesamt also $S_1=S_2$.\\
    Ganz entsprechend zeigt man dies für das Infimum.}
    \lang{en}{Let $S_1$ and $S_2$ be two suprema of $A$. Since $S_2$ is an upper bound, the definition of the supremum for $S_1$ gives us that
    $S_1\leq S_2$. However, since $S_2$ is also an upper bound, the definition of the supremum for $S_2$ gives us that $S_2\leq S_1$. Therefore $S_1=S_2$.\\
    The same can similarly be shown for the infimum.}
    \end{showhide}

\item \lang{de}{Ist $A$ nach oben unbeschränkt, so besitzt sie kein Supremum. \\
      Ist sie nach unten unbeschränkt, so besitzt sie kein Infimum. \\
      Doch selbst wenn die Mengen nach oben oder nach unten beschränkt sind, 
      müssen das Supremum und das Infimum von Teilmengen nicht unbedingt existieren, 
      wie in den nachfolgenden Beispielen (\ref{ex:suprema-infima}) zu sehen ist.}
      \lang{en}{If $A$ is unbounded above, then it has no supremum. \\
      If it is unbounded below, then it has no infimum. \\
      However, even if the set is bounded above or below, the supremum and infimum of a subset need not necessarily exist, 
      as can be seen in the following examples (\ref{ex:suprema-infima}).}

\lang{de}{Die Existenz von Suprema und Infima ist ein wesentliche Eigenschaft der reellen Zahlen.}
\lang{en}{The existence of suprema and infima is an essential property of the real numbers.}
\item
\lang{de}{Besitzt $A$ ein Maximum $M=\max(A)$, so ist dieses auch gleichzeitig das Supremum.

Besitzt eine Teilmenge $A$ ein Minimum $m=\min(A)$, so ist dieses auch gleichzeitig das Infimum.}
\lang{en}{If $A$ has a maximum $M=\max(A)$, this is also the supremum.

If $A$ has a mimimum $m=\min(A)$, this is also the infimum.}
\item \lang{de}{Besitzt $A$ eine obere Schranke $S$, welche sogar in $A$ liegt, so ist $S=\sup(A)=\max(A)$.

 Besitzt $A$ ein untere Schranke $s$, welche sogar in $A$ liegt, so ist $s=\inf(A)=\min(A)$.}
 \lang{en}{If $A$ has an upper bound $S$ which lies in $A$, then $S=\sup(A)=\max(A)$.
 
 If $A$ has a lower bound $s$ which lies in $A$, then $s=\inf(A)=\min(A)$.}
\end{enumerate}
\end{remark}


\begin{example}\label{ex:suprema-infima}
\lang{de}{Wir betrachten wieder die Teilmengen der rationalen Zahlen $\Q$ wie in den Beispielen zu 
\ref[content_06_supremum_infimum][Maximum und Minimum]{ex:maxmin}.}
\lang{en}{We again consider the subsets of the rational numbers $\Q$ from the examples for 
\ref[content_06_supremum_infimum][maximum and minimum]{ex:maxmin}.}

\begin{enumerate}
  \item \lang{de}{Das Infimum von $\N$ ist gleich dem Minimum $1$. 
      $\N$ ist jedoch nicht nach oben beschränkt und
  besitzt daher auch kein Supremum.
  Dass $\N$ kein Supremum besitzt, kann man auch direkt aus der Definition zur oberen Schranke folgern.}
  \lang{en}{The infimum of $\N$ is equal to the mimimum $1$.
  $\N$ is not bounded above, and therefore it has no supremum.
  We can also show that $\N$ has no supremum directly from the definition of an upper bound.}

    \begin{showhide}[\buttonlabels{\lang{de}{Zeige Begründung }\lang{en}{Show proof }}{\lang{de}{Verstecke Begründung}\lang{en}{Hide proof}}]
    	  \lang{de}{Angenommen $S$ sei eine obere Schranke für $\N$. Dann gilt   
          \ref[content_06_supremum_infimum][definitionsgemäß]{def:beschraenkt},
          dass $\, n \leq S \,$ für alle $n \in \N \,$ gilt. Da mit $n$ aber auch 
          $n+1$ in $\N$ liegt, gilt auch $n+1 \leq S \,$ und folglich 
          $\, n \leq S-1 \, $ für alle $\, n\in \N$.\\
          Dies wiederum bedeutet, dass auch $S-1$ eine obere Schranke für $\N$ ist, die
          kleiner ist als $S$. Da dieser Schluss für jede beliebig gewählte obere Schranke 
          $S$ gilt, besitzt $\N$ daher keine kleinste obere Schranke.}
          \lang{en}{Suppose $S$ is an upper bound for $\N$. Then 
          \ref[content_06_supremum_infimum][by definition]{def:beschraenkt}, 
          we have $\, n \leq S \,$ for all $n \in \N \,$. Since as well as $n$, $n+1$ also lies in $\N$, 
          we have $n+1 \leq S \,$ and thus $\, n \leq S-1 \, $ for all $\, n\in \N$.\\
          This means that $S-1$ is also an upper bound for $\N$, which is smaller than $S$. 
          Since this argument holds for any arbitrarily chosen upper bound $S$, it shows that 
          $\N$ has no least upper bound.}
    \end{showhide}
% 

  \item \lang{de}{Jede endliche nichtleere Teilmenge $A\subset \Q$ besitzt ein Infimum und ein Supremum, 
        nämlich das Minimum bzw. das Maximum von $A$.}
        \lang{en}{Every finite non-empty subset $A\subset \Q$ has an infimum and a supremum, namely 
        the mimimum and the maximum respectively.}
        
  \item \lang{de}{Die Menge $A=\{\frac{1}{n} \in \Q | n\in \N \}$ besitzt das Maximum $1$, 
        welches daher auch Supremum ist. Wir wissen bereits, dass $0$ eine untere Schranke,
        jedoch kein Minimum von $A$ ist. Da $0$ jedoch die größte untere Schranke von $A$ ist, 
        ist sie das Infimum von $A$, also $\inf(A)=0$.}
        \lang{en}{The set $A=\{\frac{1}{n} \in \Q | n\in \N \}$ has the maximum $1$, 
        which is therefore also its supremum. We already know that $0$ is a lower bound, but not the mimimum, of $A$. 
        Since $0$ is the greatest lower bound of $A$, it is the infimum of $A$, so $\inf(A)=0$.}
        
  \item \lang{de}{Die Menge $A=\{x \in \Q | x^2<2\}$ ist zwar beschränkt, besitzt aber in den rationalen
        Zahlen kein Supremum und kein Infimum. }
        \lang{en}{The set $A=\{x \in \Q | x^2<2\}$ is bounded, but it has no supremum and no maximum in the rational numbers.}

    \begin{showhide}[\buttonlabels{\lang{de}{Zeige Begründung }\lang{en}{Show proof }}{\lang{de}{Verstecke Begründung}\lang{en}{Hide proof}}]
        \lang{de}{Da für jede rationale Zahl $x\in \Q$ genau dann $x\in A$ ist, wenn $-x$ in $A$ ist, 
        ist $s\in \Q$ genau dann ein Infimum von $A$, wenn $-s$ ein Supremum von $A$ ist. 
        Es reicht also, zu zeigen, dass $A$ kein Supremum besitzt.\\
        
        Ist nun $q \in \Q$ eine obere Schranke von $A$, dann gilt $\,q\geq 0\,$ (denn 
        $0\in A$) und $q^2\geq 2\,$ (ansonsten wäre $q\in A$ und daher ein Maximum, was wir im vorigen
        Abschnitt jedoch ausgeschlossen hatten). Da die Zahl $\sqrt{2}$ keine rationale Zahl ist, 
        gilt zudem $q^2\neq 2$, daher ist sogar $\, q^2>2 $. \\
        
        Rechnerisch lässt sich nachweisen, dass für jede positive rationale Zahl $q$, deren 
        Quadrat größer als $2$ ist, mit $r:=\frac{q^2+2}{2q}\, $ eine rationale Zahl existiert,
        für die gilt:
        \[0<r=\frac{q^2+2}{2q}<\frac{q^2+q^2}{2q}=q \quad \text{und}
        \quad r^2=\frac{q^4+4q^2+4}{4q^2}=\frac{(q^2-2)^2+8q^2}{4q^2}>2.\]       
        Letzteres impliziert, dass für alle $a\in A$ gilt:  
        \[ r^2>2>a^2 \Rightarrow r > \abs{a} . \]
        Damit ist die Zahl $r$ eine kleinere obere Schranke für $A$ als $q$, daher kann $q$
        kein Supremum von $A$ sein. 
        Da dieser Schluss für jede beliebig gewählte obere Schranke $q$ gilt, gibt
        es demnach keine kleinste obere Schranke und somit nach Definition \ref{def:sup_inf}
        auch kein Supremum für $A$.}
        \lang{en}{Since for every rational number $x\in \Q$, we have $x\in A$ if and only if $-x\in A$, 
        we therefore have that $s\in \Q$ is the infimum of $A$ if and only if $-s$ is the supremum of $A$.
        Thus it suffices to show that $A$ has no supremum.\\
        
        Supposing $q \in \Q$ is an upper bound of $A$, we have $\,q\geq 0\,$ (since 
        $0\in A$) and $q^2\geq 2\,$ (else we would have $q\in A$ and it would be a maximum, which we proved 
        it was not in the previous section). Since the number $\sqrt{2}$ is not a rational number, this implies 
        that $q^2\neq 2$, so we actually have $\, q^2>2 $. \\

        Now performing some computations allows us to find that a rational number $r:=\frac{q^2+2}{2q}\, $ exists, such that:
        \[0<r=\frac{q^2+2}{2q}<\frac{q^2+q^2}{2q}=q \quad \text{and}
        \quad r^2=\frac{q^4+4q^2+4}{4q^2}=\frac{(q^2-2)^2+8q^2}{4q^2}>2.\]    
        The latter implies that for all $a\in A$ we have:
        \[ r^2>2>a^2 \Rightarrow r > \abs{a} . \]
        Therefore the number $r$ is a smaller upper bound for $A$ than $q$, and so $q$ cannot be the supremum of $A$.
        Since this argument holds for any arbitrary choice of upper bound $q$, there cannot be a least upper bound, 
        and so by definition \ref{def:sup_inf} there is no supremum for $A$.}
    \end{showhide}
  \end{enumerate}
\end{example}

\lang{de}{Eine weitere Charakterisierung von Suprema und Infima ist die Folgende:}
\lang{en}{Another characterisation of suprema and infima is the following:}

\begin{theorem}\label{thm:charakterisierung-sup}
\lang{de}{Sei $A\subset \K$. Ein Element $S\in \K$ ist genau dann das Supremum von $A$, wenn die beiden folgenden
Aussagen erfüllt sind:}
\lang{en}{Let $A\subset \K$. An element $S\in \K$ is the supremum of $A$ if and only if it satisfies the following two conditions:}
\begin{enumerate}
\item \lang{de}{Für jedes $x\in A$ gilt $x\leq S$  (d.h. $S$ ist obere Schranke),}
\lang{en}{For every $x\in A$, $x\leq S$  (i.e. $S$ is an upper bound),}
\item \lang{de}{zu jedem positiven $\epsilon\in \K$, existiert ein $x\in A$ mit $S-\epsilon<x$.}
\lang{en}{for every positive $\epsilon\in \K$, there exists $x\in A$ with $S-\epsilon<x$.}
\end{enumerate}

\lang{de}{Ein Element $s\in \K$ ist genau dann das Infimum von $A$, wenn die beiden folgenden
Aussagen erfüllt sind:}
\lang{en}{An element $s\in \K$ is the infimum of $A$ if and only if it satisfies the following two conditions:}
\begin{enumerate}
\item \lang{de}{Für jedes $x\in A$ gilt $s\leq x$  (d.h. $s$ ist untere Schranke),}
\lang{en}{For every $x\in A$, $s\leq x$  (i.e. $s$ is a lower bound),}
\item \lang{de}{zu jedem positiven $\epsilon\in \K$, existiert ein $x\in A$ mit $x<s+\epsilon$.}
\lang{en}{for every positive $\epsilon\in \K$, there exists $x\in A$ with $x<s+\epsilon$.}
\end{enumerate}
\end{theorem}

\lang{de}{
%Video
Im folgenden Video finden Sie auch die Definitionen von Supremum und Infimum und deren Eigenschaften.

\floatright{\href{https://api.stream24.net/vod/getVideo.php?id=10962-2-10895&mode=iframe&speed=true}{\image[75]{00_video_button_schwarz-blau}}}\\
\\
}

\section{\lang{de}{Vollständigkeit}\lang{en}{Completeness}}

\begin{definition}[\lang{de}{Vollständigkeit eines Körpers}\lang{en}{Completeness of a field}]\label{def:vollstaendigkeit}
\lang{de}{Ein angeordneter Körper $(\K,<)$ heißt \notion{vollständig}, wenn jede nach oben beschränkte, 
nicht-leere Teilmenge von $\K$ ein Supremum besitzt.}
\lang{en}{An ordered field $(\K,<)$ is called \notion{complete} if every subset of $\K$ which is non-empty and bounded above has a supremum.}
\end{definition}

\begin{example}
\lang{de}{Der Körper der rationalen Zahlen $\Q$ ist nicht vollständig, denn wie wir  in Beispiel 
\ref{ex:suprema-infima} gezeigt haben, ist die Menge $\,A=\{x \in \Q | x^2<2\}\,$ 
beschränkt und nicht-leer, besitzt aber in den rationalen Zahlen dennoch kein Supremum. }
\lang{en}{The field of rational numbers $\Q$ is not complete, since as we showed in the example \ref{ex:suprema-infima}, 
the set $\,A=\{x \in \Q | x^2<2\}\,$ is bounded and non-empty, but has no supremum in the rational numbers.}
\end{example}

\lang{de}{Dass man für die Definition der Vollständigkeit die Existenz von Suprema verlangt und nicht die Existenz
von Infima, ist willkürlich. Der folgende Satz sagt aus, dass die Existenz von Suprema auch die
Existenz von Infima impliziert und umgekehrt, sodass man den gleichen Begriff von Vollständigkeit bekommt,
wenn man die Existenz von Infima voraussetzen würde.}
\lang{en}{That the definition for completeness requires the existence of suprema and not the existence of infima is an arbitrary choice. 
The following theorem says that the existence of suprema also implies the existence of infima, and vice versa, so that requiring the existence 
of infima would result in the same definition for completeness.}

\begin{theorem}
\lang{de}{Ein angeordneter Körper $(\K,<)$ ist genau dann vollständig, wenn jede nach unten beschränkte, 
nicht-leere Teilmenge von $\K$ ein Infimum besitzt.}
\lang{en}{An ordered field $(\K,<)$ is complete if and only if every subset of $\K$ which is non-empty and bounded below has an infimum.}
\end{theorem}


\begin{proof*}
\begin{showhide}
\lang{de}{Zu einer beliebig gewählten Teilmenge $A\subset \K$ betrachten wir die Menge 
$\,-A=\{ -x | x\in A\}$. \\
Dann gilt:}
\lang{en}{For an arbitrarily chosen subset $A\subset \K$, consider the set $\,-A=\{ -x | x\in A\}$. \\
Then the following hold:}
\begin{itemize}
\item \lang{de}{$c\in \K$ untere Schranke von $A$ $\; \Leftrightarrow \;$ $-c\in \K$ obere Schranke von $-A$,}
\lang{en}{$c\in \K$ is a lower bound of $A$ $\; \Leftrightarrow \;$ $-c\in \K$ is an upper bound of $-A$,}
\item \lang{de}{$c\in \K$ größte untere Schranke von $A$ $\; \Leftrightarrow \;$  $-c\in \K$ kleinste obere Schranke von $-A$,}
\lang{en}{$c\in \K$ is the greatest lower bound of $A$ $\; \Leftrightarrow \;$  $-c\in \K$ is the least upper bound of $-A$,}
\item \lang{de}{$C\in \K$ obere Schranke von $A$ $\; \Leftrightarrow \;$  $-C\in \K$ untere Schranke von $-A$,}
\lang{en}{$C\in \K$ is an upper bound of $A$ $\; \Leftrightarrow \;$  $-C\in \K$ is a lower bound of $-A$,}
\item \lang{de}{$C\in \K$ kleinste obere Schranke von $A$ $\; \Leftrightarrow \;$  $-C\in \K$ größte untere Schranke von $-A$,}
\lang{en}{$C\in \K$ is the least upper bound of $A$ $\; \Leftrightarrow \;$  $-C\in \K$ is the greatest lower bound of $-A$,}
\end{itemize}
\lang{de}{Dies folgt direkt aus der Definition der \ref[content_06_supremum_infimum][Schranken]{def:beschraenkt},
denn ist nämlich zum Beispiel $c$ untere Schranke von $A$ ist, so ist}
\lang{en}{These follow directly from the definition of \ref[content_06_supremum_infimum][bounds]{def:beschraenkt}, 
since for example if $c$ is a lower bound of $A$, then}
\[ c\leq x \quad \text{\lang{de}{f"ur alle }\lang{en}{for all }}\;  x\in A. \]
\lang{de}{Äquivalent dazu ist}\lang{en}{This is equivalent to}
\[ -c\geq -x \quad \text{\lang{de}{f"ur alle }\lang{en}{for all }}\; x\in A, \]
\lang{de}{d.h.}\lang{en}{i.e.}
\[ -c\geq y \quad \text{\lang{de}{f"ur alle }\lang{en}{for all }}\;  y\in -A, \]
\lang{de}{und damit dass $-c$ eine obere Schranke für $-A$ ist.

Setzt man nun zum Beispiel die Existenz von Suprema voraus, dann erhält man analog:\\
Wenn $A$ eine nach unten beschränkte, nicht-leere Teilmenge von $\K$ ist, ist $-A$ 
eine nach oben beschränkte, nicht-leere Teilmenge und besitzt daher eine kleinste obere Schranke $S$.
Dann ist aber $-S$ eine größte untere Schranke von $A$, also ein Infimum von $A$. Jede nach unten beschränkte, 
nicht-leere Teilmenge  besitzt daher ein Infimum.

Entsprechend folgert man die Existenz von Suprema aus der Existenz von Infima.}
\lang{en}{and thus that $-c$ is an upper bound for $-A$.

For example, assuming the existence of suprema, one obtains analogously:\\
If $A$ is a subset of $\K$ which is non-empty and bounded below, then $-A$ is a non-empty subset which is bounded above, 
and thus has a least upper bound $S$. But then $-S$ is a greatest lower bound of $A$, and so the infimum of $A$. Hence every 
subset which is non-empty and bounded below has an infimum.}
\end{showhide}
\end{proof*}




\begin{theorem}
\lang{de}{Die Menge $\R$ der reellen Zahlen ist ein vollständiger, angeordneter Körper, und dies
charakterisiert $\R$ eindeutig, d.h. $\R$ ist "`der einzige"' vollständige, angeordnete Körper.}
\lang{en}{The set $\R$ of real numbers is a comlete ordered field, and this property characterises $\R$, i.e. $\R$ is 'the only' complete ordered field.}
\end{theorem}

\lang{de}{Insbesondere sind alle Aussagen, die in den vorangegangenen Abschnitten über angeordnete Körper und vollständige, angeordnete
Körper gemacht wurden, für den Körper $\R$ der reellen Zahlen gültig.


Ein wichtiges Beispiel, bei dem die Vollständigkeit zum Tragen kommt, ist die Existenz $n$-ter
Wurzeln nicht-negativer reeller Zahlen.}
\lang{en}{In particular, all statements that were made in the previous sections about ordered fields and complete ordered fields
are valid for the field $\R$ of real numbers.

An important example where completeness comes into play is the existence of $n$-th roots of non-negative real numbers.}

\begin{theorem}\label{thm:n-te-wurzel}
\lang{de}{Sei $n\in \N$ eine natürliche Zahl.
Für jede nicht-negative reelle Zahl $x\in \R_+$ gibt es eine eindeutige nicht-negative reelle
Zahl $a\in \R_+$, für welche $a^n=x$ gilt.

Diese Zahl wird die $n$-te Wurzel von $x$ genannt, und mit $\sqrt[n]{x}$ bezeichnet.}
\lang{en}{Let $n\in \N$ be a natural number.
For every non-negative real number $x\in \R_+$, there exists a unique non-negative real number $a\in \R_+$ for which $a^n=x$ holds.

This number is called the $n$-th root of $x$, and is denoted by $\sqrt[n]{x}$.}
\end{theorem}


\begin{proof*}[\lang{de}{Erklärung}\lang{en}{Proof}]
\begin{showhide}
\lang{de}{Für ein beliebiges   
$x \in \R_+ $ betrachten wir die Menge}
\lang{en}{For an arbitrarily chosen   
$x \in \R_+ $ consider the set}

\[ M:=\{ r\in \R_{\geq 0} |\, r^n\leq x\}. \]
\lang{de}{Diese Menge ist nicht-leer, denn wegen 
$\, 0^n=0  < x\, $ ist $0\in M.\;$ Außerdem ist $r\leq 1+x$ für alle
$r\in M.\;$ Wäre nämlich $r>1+x$, dann erhielte man wegen der 
\ref[content_05_anordnungsaxiome][Anordnungsregel zu Potenzen]{rule:potenzen}
$\; r^n>(1+x)^n\, $ und schließlich durch Anwendung der
\ref[content_02_vollstaendige_induktion][Bernoulli-Ungleichung]{rule:bernoulli-ungleichung}
und da $\, n \geq 1$:}
\lang{en}{This set is non-empty, since due to $\, 0^n=0  < x\, $ we have $0\in M.\;$ 
Additionally, we have $r\leq 1+x$ for all $r\in M.\;$. If we had, on the contrary, 
$r>1+x$, then due to the  
\ref[content_05_anordnungsaxiome][rules for inequalities with powers]{rule:potenzen}
$\; r^n>(1+x)^n\, $ and it follows from an application of 
\ref[content_02_vollstaendige_induktion][Bernoulli's inequality]{rule:bernoulli-ungleichung}
that since $\, n \geq 1$:}
\[  r^n>(1+x)^n\geq 1+nx\geq 1+x>x \]
\lang{de}{im Widerspruch zu\, $r\in M$.
Die Menge $M$ ist daher nach oben beschränkt (durch $1+x$) und wegen der Vollständigkeit von $\R$
besitzt $M$ ein Supremum $a:=\sup(M)\in \R$.

Durch Widerspruch zeigt man nun, dass tatsächlich $a^n=x$ gilt.
Dies soll hier aber nicht ausgeführt werden.

Die Eindeutigkeit eines Elements in $\R_+$, dessen $n$-te Potenz gleich $x$ ist, 
ergibt sich wieder aus der \ref[content_05_anordnungsaxiome][Anordnungsregel zu Potenzen]{rule:potenzen}.
Sind nämlich $a$ und $b$ zwei solche Elemente, ist insbesondere $a^n=b^n$ und daher $a=b$.}
\lang{en}{contradicting\, $r\in M$.
The set $M$ is thus bounded above (by $1+x$), and by the completeness of $\R$ it therefore has a supremum $a:=\sup(M)\in \R$.

Using contradiction one can then show that in fact $a^n=x$.
This will not be done here.

The uniqueness of an element in $\R_+$ whose $n$-th power is equal to $x$ follows again from the 
\ref[content_05_anordnungsaxiome][rules for inequalities with powers]{rule:potenzen}.
If $a$ and $b$ are two such elements, we get that $a^n=b^n$ and so $a=b$.}
 \end{showhide}
\end{proof*}

\lang{de}{
%Video
In dem Video wird nach der Definition der Vollständigkeit der Beweis des Satzes \ref{thm:n-te-wurzel} erläutert.

\floatright{\href{https://api.stream24.net/vod/getVideo.php?id=10962-2-10896&mode=iframe&speed=true}{\image[75]{00_video_button_schwarz-blau}}}\\
\\
}

\section{\lang{de}{Charakterisierung von $\R$ als ein vollständiger, angeordneter Körper}
\lang{en}{Characterisation of $\R$ as a complete ordered field}}\label{sec:R-Koerper}\label{sec:R_vollst_Koerper}

\lang{de}{Zuletzt wollen wir aus den Axiomen noch einige Aussagen folgern, die anzeigen, dass ein vollständiger, angeordneter Körper mit unserer
Vorstellung der reellen Zahlen übereinstimmt, d.h. dass die Axiome wirklich die reellen Zahlen charakterisieren.}
\lang{en}{Finally, we wish to derive several propositions from the axioms, which show that a complete ordered field matches our idea of the real numbers, 
i.e. that the axioms really characterise the real numbers.}

\begin{rule}
\lang{de}{Ist $(\K,<)$ ein angeordneter Körper, so enthält er (eine Kopie) der rationalen Zahlen.}
\lang{en}{If $(\K,<)$ is an ordered field, then it contains (a copy of) the rational numbers.}
\end{rule}


\begin{proof*}[\lang{de}{Erklärung}\lang{en}{Proof}]
\begin{showhide}
\lang{de}{Wir können die natürlichen Zahlen $\N$ als Teil von $\K$ auffassen, indem wir $n\in \N$ mit}
\lang{en}{We can view the natural numbers $\N$ as a subset of $\K$ by identifying $n\in \N$ with}
\[ n_{\K}:=\underbrace{1+\cdots +1}_{n-\text{mal}}\in \K \lang{en}{.}\]
\lang{de}{identifizieren. Die Anordnung garantiert, dass keine zwei verschiedenen Zahlen $n,m\in \N$ mit dem gleichen Element in $\K$ identifiziert
werden, denn zunächst gilt in $\K$, dass $1>0$ ist, und daher auch $\underbrace{1+\cdots +1}_{n-\text{mal}}>0$. Damit gilt dann für natürliche
Zahlen $m>n$, 
also $m=n+k$ mit $k\in \N$:}
\lang{en}{The ordering guarantees that no two distinct numbers $n,m\in \N$ are identified with the same element in $\K$, since in $\K$ we have that $1>0$, 
and thus also $\underbrace{1+\cdots +1}_{n-\text{times}}>0$. Hence for natural numbers $m>n$, writing $m=n+k$ with $k\in \N$:}
\[ m_{\K}:=\underbrace{1+\cdots +1}_{m-\text{\lang{de}{mal}\lang{en}{times}}}=(\underbrace{1+\cdots +1}_{n-\text{\lang{de}{mal}\lang{en}{times}}})+(\underbrace{1+\cdots +1}_{k-\text{\lang{de}{mal}\lang{en}{times}}})=n_{\K}+k_{\K}
>n_{\K}. \]
\lang{de}{Damit sind $m_{\K}$ und $n_{\K}$ verschieden.

Anschließend findet man die negativen ganzen Zahlen in $\K$ als additive Inverse der natürlichen Zahlen wieder und die $0$ in $\Z$ identifiziert
man mit $0\in \K$. Zuletzt kann man dann eine rationale Zahl $\frac{p}{q}$ mit dem entsprechenden Element $pq^{-1}$ in $\K$ identifizieren. 

Eine etwas aufwändigere Arbeit ist es, nachzurechnen, dass bei dieser Identifizierung auch die Rechenoperationen und die Anordnung erhalten 
bleiben, dass also zum Beispiel $r+s$ in den rationalen Zahlen mit der Summe der Elemente $r$ und $s$ in $\K$ identifiziert wird.}
\lang{en}{Therefore $m_{\K}$ and $n_{\K}$ are distinct.

Now one can find the negative integers in $\K$ as the additive inverses of the natural numbers, and one identifies the $0$ in $\Z$ with $0\in \K$. 
Finally, one can identify a rational number $\frac{p}{q}$ with the corresponding element $pq^{-1}$ in $\K$.

A somewhat more time consuming task is to check that with this identification, the operations and ordering is preserved. For example, $r+s$ in the rational numbers should 
be identified with the sum of the elements $r$ and $s$ in $\K$.}
 \end{showhide}
\end{proof*}



\lang{de}{Nun sei $(\K,<)$ stets ein vollständiger, angeordneter Körper}
\lang{en}{Now let $(\K,<)$ be a complete ordered field.}

\begin{theorem}\label{thm:N-unbeschraenkt}
\lang{de}{Die natürlichen Zahlen $\N\subset \K$ sind nach oben unbeschränkt.\\
Die ganzen Zahlen $\Z\subset \K$ sind nach oben und nach unten unbeschränkt.

Insbesondere gilt: \\

Für jedes $x\in \K$ gibt es ganze Zahlen $m,n\in \Z\subset \K$ mit $m<x<n$.}
\lang{en}{The natural numbers $\N\subset \K$ are unbounded above.\\
The integers $\Z\subset \K$ are unbounded above and below.

In particular: \\

For every $x\in \K$ there are integers $m,n\in \Z\subset \K$ with $m<x<n$.}
\end{theorem}


\begin{proof*}[\lang{de}{Erklärung}\lang{en}{Proof}]
\begin{showhide}
\lang{de}{Wäre $\N$ nach oben beschränkt, dann besäße $\N$ nach Definition \ref{def:vollstaendigkeit} 
ein Supremum $S\in \K$, da $\K$ vollständig ist. 
Wir hatten aber in Beispiel \ref{ex:suprema-infima} gezeigt,
dass $\N$ kein Supremum besitzt. Also ist $\N$ nach oben unbeschränkt.

Da $\N$ nach oben unbeschränkt ist, ist $-\N=\{ -n | n\in \N\}$ nach unten unbeschränkt. Damit ist
$\Z=-\N \cup \{0\} \cup \N$ weder nach unten noch nach oben beschränkt.}
\lang{en}{If $\N$ were bounded above, then $\N$ would have, by definition \ref{def:vollstaendigkeit}, 
a supremum $S\in \K$, since $\K$ is complete.
However, we showed in example \ref{ex:suprema-infima} that $\N$ has no supremum. Therefore $\N$ is unbounded above.

Since $\N$ is unbounded above, $-\N=\{ -n | n\in \N\}$ is unbounded below. 
Therefore $\Z=-\N \cup \{0\} \cup \N$ is not bounded above, nor below.}
 \end{showhide}
\end{proof*}


\lang{de}{Aus obiger Aussage folgert man}
\lang{en}{From the above proposition we obtain}
\begin{theorem}\label{thm:Q-dicht}
\begin{enumerate}
\item \lang{de}{Sind $a,b\in \K$ mit $a>0$ und $b>0$, so gibt es eine natürliche Zahl $n\in \N$, so dass $na>b$.}
\lang{en}{If $a,b\in \K$ with $a>0$ and $b>0$, then there is a natural number $n\in \N$ such that $na>b$.}
\item \lang{de}{Ist $a\in \K$ und gilt $0\leq a<\frac{1}{n}$ für jede natürliche Zahl $n\in \N$, so ist $a=0$.}
\lang{en}{If $a\in \K$ and $0\leq a<\frac{1}{n}$ for all natural numbers $n\in \N$, then $a=0$.}
\item \lang{de}{Sind $r,s\in \K$ mit $r<s$, so gibt es eine rationale Zahl $q$ mit $r<q<s$.}
\lang{en}{If $r,s\in \K$ with $r<s$, then there is a rational number $q$ with $r<q<s$.}
\end{enumerate}
\end{theorem}


\begin{proof*}[\lang{de}{Erklärung}\lang{en}{Proof}]
\begin{showhide}
\begin{enumerate}
\item[$\text{\lang{de}{zu}\lang{en}{for}}\, 1.:$] \lang{de}{Gäbe es kein solches $n$, so wäre $na\leq b$ für alle $n\in \N$. Da $a>0$ ist, 
      ist das gleichbedeutend zu $n\leq \frac{b}{a}$ für alle $n\in \N$. Damit wäre aber
      $\frac{b}{a}$ eine obere Schranke für $\N$ im Widerspruch zur Unbeschränktheit 
      gemäß Theorem \ref{thm:N-unbeschraenkt}.}
      \lang{en}{If there were no such $n$, then we would have $na\leq b$ for all $n\in \N$. Since $a>0$, this is equivalent to 
      $n\leq \frac{b}{a}$ for all $n\in \N$. But then $\frac{b}{a}$ would be an upper bound for $\N$, in contradiction to theorem
      \ref{thm:N-unbeschraenkt} which says that $\N$ is unbounded above.}

\item[$\text{\lang{de}{zu}\lang{en}{for}}\, 2.:$] \lang{de}{$a<\frac{1}{n}$ für alle $n\in \N$, ist gleichbedeutend zu $na<1$ 
        für alle $n\in \N\,$ (da $n>0$). Wäre $a>0$, wäre dies (mit $b=1>0$) ein Widerspruch
        zum ersten Teil. Also ist $a=0$.}
        \lang{en}{$a<\frac{1}{n}$ for all $n\in \N$ is equivalent to $na<1$ for all $n\in \N\,$ (as $n>0$). 
        If $a>0$, then this (with $b=1>0$) contradicts the first part of the theorem. So $a=0$.}

\item[$\text{\lang{de}{zu}\lang{en}{for}}\, 3.:$] \lang{de}{Wähle zunächst $z\in \Z$ mit $z<r$. Eine solche Zahl $z \in \Z$ 
        existiert, da $\Z$ nach unten unbeschränkt ist. Da$\, s-r>0$, existiert nach $1.$ 
        eine natürliche Zahl $n\in \N$ mit \[n(s-r)>1 \quad (*).\] 
        Die Menge $\, M=\{ i\in \Nzero\,|\, i>n(r-z) \}$ ist nicht-leer (da $\N$ nach oben 
        unbeschränkt ist) und besitzt ein kleinstes Element, welches größer $0$ ist, da $r-z>0$. 
        Sei also
        $k=\min(M)\in M\subset \N$.\\ 
        
        Da $k$ das Minimum ist, gilt $\; k-1\leq n(r-z) \Leftrightarrow k \leq n(r-z)+1$, \\
        und damit
        \[ n(r-z)<k\leq n(r-z)+1 = n(s-z)+\underbrace{1-n(s-r)}_{<0\, wg.\, (*)} < n(s-z), \]
        insbesondere 
        \[ n(r-z)<k<n(s-z). \]
        Multiplikation mit $\frac{1}{n}>0$ und Addition von $z$ liefert schließlich
        \[ r<\frac{k}{n}+z < s, \]
        und somit mit $q:=\frac{k}{n}+z$ eine rationale Zahl mit der gewünschten Eigenschaft.}
        \lang{en}{Begin by choosing $z\in \Z$ with $z<r$. Such a number $z \in \Z$ exists because $\Z$ is unbounded below. 
        Since $\, s-r>0$, by $1.$ we have that a natural number $n\in \N$ exists with \[n(s-r)>1 \quad (*).\]
        The set $\, M=\{ i\in \Nzero\,|\, i>n(r-z) \}$ is non-empty (since $\N$ is unbounded above) and has a smallest element, 
        which is greater than $0$ since $r-z>0$. 
        Hence let $k=\min(M)\in M\subset \N$.\\ 

        Since $k$ is the mimimum, we have $\; k-1\leq n(r-z) \Leftrightarrow k \leq n(r-z)+1$, \\
        and so
        \[ n(r-z)<k\leq n(r-z)+1 = n(s-z)+\underbrace{1-n(s-r)}_{<0\, by\, (*)} < n(s-z), \]
        in particular 
        \[ n(r-z)<k<n(s-z). \]
        Multiplication by $\frac{1}{n}>0$ and addition of $z$ gives us
        \[ r<\frac{k}{n}+z < s, \]
        and indeed $q:=\frac{k}{n}+z$ is a rational number with the required property.}
\end{enumerate}
 \end{showhide}
\end{proof*}


\begin{theorem}\label{thm:dedekind-schnitt}
\lang{de}{Für jedes Element $r\in \K$ gilt:}\lang{en}{For every element $r\in \K$ we have:}
\[   r=\sup \{ q\in \Q\,|\, q< r \} \]
\lang{de}{und}\lang{en}{and}
\[   r=\inf \{ q\in \Q\,|\, q > r \}. \]
\end{theorem}



\begin{proof*}[\lang{de}{Erklärung}\lang{en}{Proof}]
     \begin{showhide}
\lang{de}{Für $r\in \K$ sei nun $L=\{ q\in \Q\,|\, q< r\}$ und $R=\{ q\in \Q\,|\, q >r \}$.\\
Per Definition der Menge $L$ ist $r$ eine obere Schranke für $L$, d.h. insbesondere ist $L$ nach oben beschränkt. Außerdem gibt es 
sogar eine ganze Zahl, die kleiner als $r$ ist, weshalb $L$ nicht leer ist. Da $\K$ vollständig ist,
besitzt $L$ somit ein Supremum $S=\sup(L)$. Da $r$ eine obere Schranke ist, ist auch $S\leq r$.\\
Angenommen $S<r$, dann gibt es nach Satz \ref{thm:Q-dicht}(3.) eine rationale Zahl $q$ mit $S<q<r$. Dann ist aber $q\in L$ und $S<q$, d.h. $S$
ist keine obere Schranke für $L$. Die Annahme ist daher falsch und somit gilt: $r=S=\sup(L)$.

Ganz entsprechend zeigt man, dass $r=\inf(R)$ gilt. }
\lang{en}{For $r\in \K$ let $L=\{ q\in \Q\,|\, q< r\}$ and $R=\{ q\in \Q\,|\, q >r \}$.\\
By definition of the set $L$, $r$ is a upper bound for $L$, i.e. in particular, $L$ is bounded above. 
Additionally, there is an integer which is smaller than $r$, so $L$ is not empty. Because $\K$ is complete, 
$L$ has a supremum $S=\sup(L)$. Because $r$ is an upper bound, we have $S\leq r$.\\
Assuming $S<r$, by theorem \ref{thm:Q-dicht}(3.) there is a rational number $q$ with $S<q<r$. But then $q\in L$ and $S<q$, i.e. $S$ 
is not an upper bound for $L$. This assumption is therefore false, and so we have $r=S=\sup(L)$.

A similar proof shows that $r=\inf(R)$.}
 \end{showhide}
\end{proof*}


\lang{de}{Der letzte Satz besagt zwei Dinge: Zum einen ist jedes Element in $\K$ das Supremum einer Menge rationaler Zahlen und auch das Infimum
einer Menge rationaler Zahlen, und zum anderen ist jedes Element $r$ in $\K$ nach Satz \ref{thm:charakterisierung-sup}
eindeutig dadurch festgelegt, welche rationalen Zahlen kleiner
als $r$ sind, und welche größer.}
\lang{en}{The final theorem says two things. For one, every element in $\K$ is the supremum of a set of rational numbers, and also the infimum 
of a set of rational numbers. Secondly, every element $r$ in $\K$ is, by theorem \ref{thm:charakterisierung-sup}, uniquely determined by which rational 
numbers are smaller than $r$, and which are greater.}

\begin{remark}
\lang{de}{Satz \ref{thm:dedekind-schnitt}
entspricht anschaulich gesehen genau dem Füllen der "`Lücken"' auf der Zahlengeraden.
Formal hängt dies eng mit einer Konstruktion der reellen Zahlen aus den rationalen Zahlen zusammen (vgl.
\href{http://de.wikipedia.org/wiki/Dedekindscher_Schnitt}{\emph{Dedekindscher Schnitt} auf Wikipedia}).}
\lang{en}{Theorem \ref{thm:dedekind-schnitt} corresponds intuitively to filling in the 'gaps' on the number line. 
Formally this relates closely with a construction of the real numbers from the rational numbers (see 
\href{http://en.wikipedia.org/wiki/Dedekind_cut}{\emph{Dedekind cuts} on Wikipedia}).}
\end{remark}
\lang{de}{
%Video
Im folgenden Video werden die oben genannten Folgerungen aus der Vollständigkeit der reellen Zahlen ebenfalls erörtert.

\floatright{\href{https://api.stream24.net/vod/getVideo.php?id=10962-2-10897&mode=iframe&speed=true}{\image[75]{00_video_button_schwarz-blau}}}\\
}


\end{content}