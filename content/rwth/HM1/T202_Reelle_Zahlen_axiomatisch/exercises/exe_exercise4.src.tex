\documentclass{mumie.element.exercise}
%$Id$
\begin{metainfo}
  \name{
    \lang{de}{Ü04: Beschränkte Mengen}
    \lang{en}{}
  }
  \begin{description} 
 This work is licensed under the Creative Commons License Attribution 4.0 International (CC-BY 4.0)   
 https://creativecommons.org/licenses/by/4.0/legalcode 

    \lang{de}{}
    \lang{en}{}
  \end{description}
  \begin{components}
  \end{components}
  \begin{links}
  \end{links}
  \creategeneric
\end{metainfo}
\begin{content}
\title{
	\lang{de}{Ü04: Beschränkte Mengen}
	\lang{en}{Exercise 4}
}


\begin{block}[annotation]
	Im Ticket-System: \href{http://team.mumie.net/issues/9782}{Ticket 9782}
\end{block}
 

\lang{de}{\begin{enumerate}[(a)]
 \item a) Untersuchen Sie, ob die Menge 
\[M:=\{x\in\Q\,|\,x^{3}\leq 8\}\]
beschränkt ist. Bestimmen Sie gegebenenfalls Maximum und Minimum.
\item b) Wir betrachten die Menge $A=\{x\in\Q\,|\,x^{2}<2\}$. Es sei $x\in A$ mit $x>0$. Zeigen Sie, dass dann die Zahl $y:=\frac{4x}{x^{2}+2}$ ebenfalls in $A$ liegt und größer als $x$ ist. 
\end{enumerate}}


\begin{tabs*}[\initialtab{0}\class{exercise}]

  \tab{
  \lang{de}{Lösung a)}}
  
  
    \lang{de}{Wir zeigen als erstes, dass $2$ das Maximum der Menge ist. Wegen $2^{3}=8$ gehört $2$ offenbar zu $M$. Für jedes $x>2$ gilt 
\[x^{3}>2^{3}=8\,.\]
Dies zeigt, dass es kein $x\in\Q$ und $x>2$ gibt mit  $x\in M$. Ist $0\leq x \leq 2$, so folgern wir $0\leq x^{3}\leq 8$ und 
jedes $x<0$ erfüllt $x^{3}<0\leq 8$. Dies zeigt, dass $M$ durch $2$ nach oben beschränkt ist und $2$ sogar das Maximum von $M$ ist,
 da $2\in M$. 

Andererseits gehört jedes $x\leq 0$ zu $M$. Daher ist $M$ nicht nach unten beschränkt.}
     
    
  

  \tab{
  \lang{de}{Lösung b)}}
  \begin{incremental}[\initialsteps{1}]
    \step 
  \lang{de}{
   Die Zahl $x$ erfüllt die Gleichung $x^{2}<2$. Weil $x>0$ ist, können wir die Ungleichung mit $x^{-1}$ multiplizieren und erhalten daraus $x<\frac{2}{x}$. Daraus folgt 
\[0<x+\frac{2}{x}<\frac{2}{x}+\frac{2}{x}=\frac{4}{x}\,.\]
Also erhalten wir schließlich
\[y=\frac{4x}{x^{2}+2}=\frac{4}{x+\frac{2}{x}}>\frac{4}{\frac{4}{x}}=x\,.\]
Beachte, dass sich das Ordnungszeichen umdreht, wenn man zum Kehrwert eines Bruchs übergeht. Daher ist $x$ kleiner als $y$.}
\step
\lang{de}{ Wir zeigen noch, dass $y$ tatsächlich in $A$ liegt. Dies ist gleichbedeutend damit, dass $y^{2}<2$ ist. nun gilt
\[y^{2}<2\iff \frac{16x^{2}}{(x^{2}+2)^{2}}<2\,.\]
Weil der Nenner der linken Seite positiv ist, bleibt das Ordnungszeichen nach Multiplikation mit  $(x^{2}+2)^{2}$ erhalten.
Daher gilt
\[\frac{16x^{2}}{(x^{2}+2)^{2}}<2\iff 16x^{2}<2(x^{2}+2)^{2}\iff 0<2(x^{4}-4x^{2}+4)=2(x^{2}-2)^{2}\,,\]
wobei wir im letzten Schritt die binomsiche Formel verwendet haben. Weil nach Voraussetzung $x^{2}<2$ war, ist $x^2-2\neq 0$ und somit ist
die letzte Ungleichung richtig. Damit haben wir schließlich bewiesen, dass $y$ in  $A$ liegt und größer als $x$ ist. }
\end{incremental}

  
\end{tabs*}


\end{content}