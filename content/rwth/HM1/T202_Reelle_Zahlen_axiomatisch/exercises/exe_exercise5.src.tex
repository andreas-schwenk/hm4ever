\documentclass{mumie.element.exercise}
%$Id$
\begin{metainfo}
  \name{
    \lang{de}{Ü05: Beschränkte Mengen, Infimum/Supremum}
    \lang{en}{}
  }
  \begin{description} 
 This work is licensed under the Creative Commons License Attribution 4.0 International (CC-BY 4.0)   
 https://creativecommons.org/licenses/by/4.0/legalcode 

    \lang{de}{}
    \lang{en}{}
  \end{description}
  \begin{components}
  \end{components}
  \begin{links}
  \end{links}
  \creategeneric
\end{metainfo}
\begin{content}
\title{
	\lang{de}{Ü05: Beschränkte Mengen, Infimum/Supremum}
	\lang{en}{Exercise 5}
}


\begin{block}[annotation]
	Im Ticket-System: \href{http://team.mumie.net/issues/9783}{Ticket 9783}
\end{block}
 

\lang{de}{Untersuchen Sie die folgenden Mengen auf Beschränktheit. Geben Sie gegebenenfalls Supremum und Infimum an.
\begin{enumerate}[(a)]

 \item a) $M=\left\{a\in\R\,|\,\text{Es existiert ein}\,n\in\N \, \text{mit}\,\,n<a<n+1\right\}$ 
 \item b) $M=\left\{(-1)^{n}(1+\frac{1}{n})\,|\,n\in\N\right\}\subset\R$
 \item c) $M=\left\{x\in\Q\,|\,1<x<5\right\}\subset\R$
\end{enumerate} }


\begin{tabs*}[\initialtab{0}\class{exercise}]

  \tab{
  \lang{de}{Lösung a)}}
  
  \begin{incremental}[\initialsteps{1}]
    \step 
    \lang{de}{Es ist $M = \lbrace x\in\mathbb{R}\,\vert\, x > 1, x\notin\mathbb{N}\rbrace$.}
     \step
     \lang{de}{$\underline{\text{Infimum:}}$ Nach Definition ist $1$ eine untere Schranke der Menge $M$. Angenommen, es gibt eine weitere untere Schranke $s$ von $M$ mit $s > 1$. Dann ist 
$
1+\frac{1}{2}\geq\min\lbrace\frac{s+1}{2},1+\frac{1}{2}\rbrace > \min\lbrace\frac{1+1}{2}=1,1+\frac{1}{2}\rbrace = 1,
$
d.h. $\min\lbrace\frac{s+1}{2},1+\frac{1}{2}\rbrace\in M$, aber 
$
\min\lbrace\frac{s+1}{2},1+\frac{1}{2}\rbrace \leq \frac{s+1}{2} < \frac{s+s}{2} = s
$
im Widerspruch dazu, dass $s$ eine untere Schranke von $M$ ist.}
\step
\lang{de}{$\underline{\text{Supremum:}}$ Es gibt keine obere Schranke, also auch kein Supremum. W\"are $s\in\R$ eine obere Schranke, so g\"abe es  ein $n_0\in\mathbb{N}$ mit $s < n_0 < n_0 + \frac{1}{2}\in M$. Widerspruch.}

    
  \end{incremental}

  \tab{
  \lang{de}{Lösung b)}}
  
  \lang{de}{
  Sei $a_n = (-1)^n (1+\frac{1}{n})$. $\; a_n$ ist positiv f\"ur gerade $n\,$ und negativ f\"ur ungerade $n.\;$ 
  F\"ur gerade $n\,$ ist
$ \;
a_n \leq a_2 =\frac{3}{2}, \;
$
f\"ur ungerade $n$ ist $\; a_n\geq a_1 = -2$. Da die beiden Schranken in der Menge selbst liegen, ist $\inf M = -2$, $\sup M = 3/2$.}

  \tab{
  \lang{de}{Lösung c)}
  }
  
    
  \lang{de}{
  Die Menge ist beschr\"ankt, da $\vert x \vert \leq 5$ f\"ur alle $x\in M$. Nach Definition der oberen/unteren Schranke gilt in diesem 
  Fall $\inf M=1$ und $\sup M=5$.}
  

\end{tabs*}


\end{content}