\documentclass{mumie.element.exercise}
%$Id$
\begin{metainfo}
  \name{
    \lang{de}{Ü02: Anordnungsaxiome}
    \lang{en}{}
  }
  \begin{description} 
 This work is licensed under the Creative Commons License Attribution 4.0 International (CC-BY 4.0)   
 https://creativecommons.org/licenses/by/4.0/legalcode 

    \lang{de}{}
    \lang{en}{}
  \end{description}
  \begin{components}
  \end{components}
  \begin{links}
  \end{links}
  \creategeneric
\end{metainfo}
\begin{content}
\title{
	\lang{de}{Ü02: Anordnungsaxiome}
	\lang{en}{Exercise 2}
}


\begin{block}[annotation]
	Im Ticket-System: \href{http://team.mumie.net/issues/9780}{Ticket 9780}
\end{block}
 

\lang{de}{Beweisen Sie die folgenden Regeln unter alleiniger Verwendung der Anordnungsaxiome. Es seien dabei stets $a,b\in\Q$.}
\begin{enumerate}[(a)]
 \item a) Es gilt $a<b$ genau dann, wenn $-a>-b$. 
 \item b) Es gilt $ab>0$ genau dann, wenn $a>0$ und $b>0$ ist, oder wenn $a<0$ und $b<0$ ist.
 \item c) Ist  $a\neq 0$ so gilt $a^{2}>0$.
 \item d) Es gilt $a>0$ genau dann, wenn $\frac{1}{a}>0$ ist.  
\end{enumerate}


\begin{tabs*}[\initialtab{0}\class{exercise}]

  \tab{
  \lang{de}{Lösung a)}}
  
 
    \lang{de}{Wegen (O3), (A3) und (A4) impliziert $a<b$ die Ungleichung
\begin{align*}
   -b=-b+(a+(-a))&=&&a+((-a)+(-b))&\\
   &<&&b+((-a)+(-b))&\\
  &=&&-a+(b+(-b))=-a\,.&
  \end{align*}
  Hiermit wurde gezeigt, dass $a<b\,\Rightarrow\, -b<-a$.
Wegen der Beziehung $-(-x)=x$, die für alle $x\in\Q$ gilt, impliziert dieselbe Argumentation auch die andere Richtung,
 d. h. $-a>-b\,\Rightarrow\, a<b$.}
     


  \tab{
  \lang{de}{Lösung b)}}
  \begin{incremental}[\initialsteps{1}]
    \step 
  \lang{de}{
   Es sei $a>0$ und $b>0$ vorausgesetzt. Dann erhalten wir aus (O4)
\[0=0\cdot b <a\cdot b\,.\]
Nun sei $a<0$ und $b<0$. Dann gilt $-a>0$ und $-b>0$, und wir sind wieder im vorherigen Fall. 
Damit erhalten wir $0<(-a)\cdot (-b)=a\cdot b$, wobei wir die Rechenregel $(-a)\cdot (-b)=a\cdot b$ benutzt haben. }
\step
\lang{de}{ Nun gelte umgekehrt $0<a b$.  Wir müssen ausschliessen, dass $a,b$ 
unterschiedliche Vorzeichen haben. Der Fall $a<0$ und $0<b$ führt auf $0<-a$ und $0<b$ gemäß Teil (a). 
Also gilt in diesem Fall $0<(-a)\cdot b$. Indem wir erneut Teil a) anwenden, erhalten wir die Ungleichung $0>ab$. 
Der Fall $0<a$ und $b<0$ führt auf denselben Widerspruch. Also bleiben nur die Fälle ($a>0$ und $b>0$)  oder ($a<0$ und $b<0$) übrig. }
\end{incremental}

  \tab{
  \lang{de}{Lösung c)}
  }
  
  \lang{de}{
  Aus der Bedingung $a\neq 0$ folgt $a<0$ oder $a>0$. Nun erhalten wir aus Teil b) direkt $a^{2}=a\cdot a>0$.}
  

   \tab{
  \lang{de}{Lösung d)}
  }
  \begin{incremental}[\initialsteps{1}]
    \step 
  \lang{de}{
  Wir führen den Beweis indirekt und nehmen an, dass $a>0$ und $a^{-1}=\frac{1}{a}<0$ gilt.
}
\step
\lang{de}{Dann gilt nach a) $-a^{-1}>0$. Wenn wir jetzt Teil b) anwenden, erhalten wir 
\[0<a\cdot (-a^{-1})=-(a\cdot a^{-1})=-1\,.\]

Aus Teil c) folgt aber direkt $0<1^{2}=1$. Nach Teil a) sind die Aussagen $0<1$ und $0<-1$ jedoch widersprüchlich.\\
Den Fall $a<0$ und $0<a^{-1}$ behandelt man analog. Daher ist die Annahme, 
dass $a$ und $a^{-1}$ veschiedene Vorzeichen haben, falsch. }
\end{incremental}
\end{tabs*}


\end{content}