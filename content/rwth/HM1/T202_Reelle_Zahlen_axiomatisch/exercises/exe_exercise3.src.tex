\documentclass{mumie.element.exercise}
%$Id$
\begin{metainfo}
  \name{
    \lang{de}{Ü03: Anordnungsaxiome}
    \lang{en}{}
  }
  \begin{description} 
 This work is licensed under the Creative Commons License Attribution 4.0 International (CC-BY 4.0)   
 https://creativecommons.org/licenses/by/4.0/legalcode 

    \lang{de}{}
    \lang{en}{}
  \end{description}
  \begin{components}
  \end{components}
  \begin{links}
  \end{links}
  \creategeneric
\end{metainfo}
\begin{content}
\title{
	\lang{de}{Ü03: Anordnungsaxiome}
	\lang{en}{Exercise 3}
}


\begin{block}[annotation]
	Im Ticket-System: \href{http://team.mumie.net/issues/9781}{Ticket 9781}
\end{block}
 

\lang{de}{Es seien $a,b$ reelle Zahlen. Wir definieren das Minimum und Maximum wie folgt:
\[\min(a,b):=\begin{cases}
              a&\text{ falls } a<b\\
	      b &\text{ sonst}
             \end{cases}\,,\,\max(a,b):=\begin{cases}
              b&\text{ falls } a<b\\
	      a &\text{ sonst}
             \end{cases}\,.
\]
Beweisen Sie, dass die folgenden beiden Identitäten für alle $a,b\in\R$ gelten:
\begin{enumerate}[(i)]
 \item $\min(a,b)=\frac{(a+b)-|a-b|}{2}$
 \item $\max(a,b)=\frac{(a+b)+|a-b|}{2}$
\end{enumerate}}


\begin{tabs*}[\initialtab{0}\class{exercise}]

  \tab{
  \lang{de}{Lösung}}
  
  \begin{incremental}[\initialsteps{1}]
    \step 
    \lang{de}{\textbf{Anschauliche Begründung:}\\

Die Zahl $M:=\frac{a+b}{2}$ ist der Mittelwert der Zahlen $a$ und $b$. Auf der reellen Achse liegt $M$ genau in der
Mitte von $a$ und $b$. Der Abstand der Zahlen $a,b$ wird beschrieben durch den Term $d:=|a-b|=|b-a|$. Offenbar erhält
man das Maximum von $a$ und $b$, indem man den halben Abstand $\frac{d}{2}$ zu $M$ addiert, und das Minimum durch Subtraktion des 
halben Abstandes $\frac{d}{2}$ von $M$. Damit haben wir gezeigt, dass
\[\max(a,b)=M+d=\frac{a+b}{2}+\frac{|a-b|}{2}\quad\text{ und }\quad \min(a,b)=M-d=\frac{a+b}{2}-\frac{|a-b|}{2}\,.\]}
     \step
     \lang{de}{\textbf{Formale Begründung:}\\


Wir nehmen ohne Einschränkung $a< b$ an, denn die Formeln sind im Fall $a=b$ offensichtlich richtig und der Fall $b<a$ 
kann durch Vertauschen der Rollen von $a$ und $b$ auf diesen Fall zurückgeführt werden wegen der Beziehungen 
\[\min(a,b)=\min(b,a)\quad\text{ und }\quad \max(a,b)=\max(b,a)\,.\] }
\step
\lang{de}{
Nach Voraussetzung gilt $|b-a|=b-a$. Man rechnet nun nach, dass
\[\min(a,b)=a=\frac{(a+a)+(b-b)}{2}=\frac{(a+b)-(b-a)}{2}=\frac{(a+b)-|b-a|}{2}\]
und
\[\max(a,b)=b=\frac{(b+b)+(a-a)}{2}=\frac{(a+b)+(b-a)}{2}=\frac{(a+b)+|b-a|}{2}\]
gelten. Damit ist die Behauptung auch hier bewiesen.}
    
  \end{incremental}

  
\end{tabs*}


\end{content}