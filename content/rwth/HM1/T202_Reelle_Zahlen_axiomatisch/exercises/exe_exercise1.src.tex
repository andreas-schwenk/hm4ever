\documentclass{mumie.element.exercise}
%$Id$
\begin{metainfo}
  \name{
    \lang{de}{Ü01: Körperaxiome und Rechenregeln}
    \lang{en}{}
  }
  \begin{description} 
 This work is licensed under the Creative Commons License Attribution 4.0 International (CC-BY 4.0)   
 https://creativecommons.org/licenses/by/4.0/legalcode 

    \lang{de}{}
    \lang{en}{}
  \end{description}
  \begin{components}
  \end{components}
  \begin{links}
\link{generic_article}{content/rwth/HM1/T104_weitere_elementare_Funktionen/g_art_content_14_potenzregeln.meta.xml}{content_14_potenzregeln}
\end{links}
  \creategeneric
\end{metainfo}
\begin{content}
\title{
	\lang{de}{Ü01: Körperaxiome und Rechenregeln}
	\lang{en}{Exercise 1}
}


\begin{block}[annotation]
	Im Ticket-System: \href{http://team.mumie.net/issues/9779}{Ticket 9779}
\end{block}
 

\lang{de}{Beweisen Sie die folgenden Rechengesetze in $\Q$ allein unter der Benutzung der Körperaxiome:}
\begin{enumerate}[(a)]
 \item a) Für alle $a\in\Q$ gilt $0\cdot a =0$.
 \item b) Für alle $a,b\in\Q$ gilt $-(a\cdot b) = (-a)\cdot b$ und $(-a)\cdot (-b)=a\cdot b$.
 \item c) Für $a,b\in\Q$ mit $0\neq a,b$ gilt $(a\cdot b)^{-1}=a^{-1}\cdot b^{-1}$.
 \item d) Für $a,b,c,d\in\Q$ mit $0\neq b,d $ gilt $\frac{a}{b}+\frac{c}{d}=\frac{ad+bc}{bd}$.
\end{enumerate}


\begin{tabs*}[\initialtab{0}\class{exercise}]

  \tab{
  \lang{de}{Lösung a)}}
  

    \lang{de}{Es sei $a\in\Q$. Nach dem Axiom (DG) gilt zunächst
\[0\cdot a \underset{\text{(A3)}}{=} (0+0)\cdot a = 0\cdot a +0\cdot a\,.\]
Nach Axiom (A4) existiert für das Element $0\cdot a\in\Q$ ein additives inverses Element $-(0\cdot a)\in\Q$ mit der Eigenschaft $(0\cdot a)+(-(0\cdot a))=0$. Addiert man nun $-(0\cdot a)$ zu der letzten Gleichung, so erhält man mit Hilfe der Körperaxiome
\[0\underset{\text{(A4)}}{=}(0\cdot a)+(-(0\cdot a))=(0\cdot a +0\cdot a)+(-(0\cdot a))\underset{\text{(A2)}}{=}0\cdot a +\underbrace{(0\cdot a+(-(0\cdot a))}_{=0}\underset{\text{(A4)}}{=}0\cdot a\,.\]
Damit ist die Behauptung gezeigt.}
     
    


  \tab{
  \lang{de}{Lösung b)}}
  \begin{incremental}[\initialsteps{1}]
    \step 
  \lang{de}{
  Die Aussage besagt anders ausgedrückt, dass das additive inverse Element von $(a\cdot b)$ gleich $(-a)\cdot b$ ist. 
  Wir müssen also zeigen, dass für alle $a,b\in\Q$ die Gleichung $(a\cdot b) + (-a)\cdot b=0$ gilt. Nun verwenden wir das Distributivgesetz (DG). Beachte hierbei, dass man bei der Formulierung des Distributivgesetzes die Reihenfolge der beiden Faktoren vertauschen kann wegen (M1). Also gilt:
 \[(a\cdot b) + (-a)\cdot b\underset{\text{(DG)}}{=}(a+(-a))\cdot b\underset{\text{(A4)}}{=}0\cdot b=0\,,\]
wobei wir im letzten Schritt (a) benutzt haben.}
\step
\lang{de}{ Um die zweite Behauptung zu beweisen, stellen wir fest, dass für jedes $x\in\Q$ gilt $-(-x)=x$. Dies ist lediglich eine Umformulierung von (A4). Unter zweimaliger Verwendung der bereits bewiesenen Aussage erhalten wir damit  
\[(-a)\cdot (-b)=-(a\cdot (-b))\underset{\text{(M1)}}{=}-((-b)\cdot a)=-(-(b\cdot a))\underset{\text{(M1)}}{=}-(-(a\cdot b))=a\cdot b\,,\]
was zu zeigen war.}
\end{incremental}

  \tab{
  \lang{de}{Lösung c)}
  }
  \begin{incremental}[\initialsteps{1}]
    \step 
  \lang{de}{
  Die Aussage lautet anders ausgedrückt $(a\cdot b)\cdot (a^{-1}\cdot b^{-1})=1$. Dies wollen wir nun verifizieren.}
  \step
  \lang{de}{ An dieser Stelle ist zu beachten, dass die Gleichung nur gilt, falls $a,b\neq 0$ sind. In der Tat gilt nun
\begin{align*}
(a\cdot b)\cdot (a^{-1}\cdot b^{-1})&\underset{\text{(M2)}}{=}&&a\cdot (b\cdot (a^{-1}\cdot b^{-1}))\underset{\text{(M1)}}{=}a\cdot (b\cdot (b^{-1}\cdot a^{-1}))&\\
&\underset{\text{(M2)}}{=}&&a\cdot ((b\cdot (b^{-1})\cdot a^{-1})\underset{\text{(M4)}}{=}a\cdot \underbrace{(1\cdot a^{-1})}_{=a^{-1}}=a\cdot a^{-1}&\\
&\underset{\text{(M4)}}{=}&&1\,.&   
  \end{align*}}
  \end{incremental}

   \tab{
  \lang{de}{Lösung d)}
  }
  \begin{incremental}[\initialsteps{1}]
    \step 
  \lang{de}{
  Wir stellen zunächst fest, dass unter der Voraussetzung $b,d\neq 0$ die Elemente $b^{-1}=\frac{1}{b},d^{-1}=\frac{1}{d}\in\Q$ 
  existieren 
  (s. Definition \ref[content_14_potenzregeln]["`negative Potenzen"']{def:rat_potenz})
  mit der Eigenschaft
\[\frac{b}{b}=b\cdot \frac{1}{b}=1\quad\text{ und }\quad\frac{d}{d}=d\cdot \frac{1}{d}=1\,.\]
Mit Hilfe von (M3) erhalten wir damit
\[\frac{a}{b}+\frac{c}{d}=\frac{a}{b}\cdot 1+\frac{c}{d}\cdot 1=\frac{a}{b}\cdot\frac{d}{d}+\frac{c}{d}\cdot \frac{b}{b}\,.\]
}
\step
\lang{de}{Gemäß (M1) und (M2) und Teil (c) können wir die letzten beiden Ausdrücke umschreiben zu
 \[\frac{a}{b}\cdot\frac{d}{d}=\frac{a\cdot d}{b\cdot d}\quad\text{ und }\quad\frac{c}{d}\cdot\frac{b}{b}=\frac{c\cdot b}{d\cdot b}=\frac{b\cdot c}{b\cdot d}\,.\]
Damit ergibt sich
\begin{align*}
\frac{a}{b}+\frac{c}{d}&\;=\;& \frac{a\cdot d}{b\cdot d}+\frac{b\cdot c}{b\cdot d}&\\
&\;=\;&(a\cdot d)\cdot (b\cdot d)^{-1}+(b\cdot c)\cdot (b\cdot d)^{-1}& \vert\text{per Definition}\\
&\;=\;&((a\cdot d)+(b\cdot c))\cdot (b\cdot d)^{-1}& \vert\text{(DG)}\\
&\;=\;&\frac{ad+bc}{bd}\,,&  \vert\text{per Definition} 
  \end{align*}
was zu zeigen war. }
\end{incremental}
\end{tabs*}

\end{content}