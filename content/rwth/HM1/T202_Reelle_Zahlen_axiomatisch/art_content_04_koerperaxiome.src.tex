%$Id:  $
\documentclass{mumie.article}
%$Id$
\begin{metainfo}
  \name{
    \lang{de}{Körperaxiome und Rechenregeln}
    \lang{en}{definition of fields}
  }
  \begin{description} 
 This work is licensed under the Creative Commons License Attribution 4.0 International (CC-BY 4.0)   
 https://creativecommons.org/licenses/by/4.0/legalcode 

    \lang{de}{Beschreibung}
    \lang{en}{Description}
  \end{description}
  \begin{components}
    \component{generic_image}{content/rwth/HM1/images/g_img_00_video_button_schwarz-blau.meta.xml}{00_video_button_schwarz-blau}
  \end{components}
  \begin{links}
    \link{generic_article}{content/rwth/HM1/T101neu_Elementare_Rechengrundlagen/g_art_content_02_rechengrundlagen_terme.meta.xml}{content_02_rechengrundlagen_terme}
    \link{generic_article}{content/rwth/HM1/T101neu_Elementare_Rechengrundlagen/g_art_content_01_zahlenmengen.meta.xml}{content_01_zahlenmengen}
    \link{generic_article}{content/rwth/HM1/T202_Reelle_Zahlen_axiomatisch/g_art_content_05_anordnungsaxiome.meta.xml}{anordnung}
    \link{generic_article}{content/rwth/HM1/T202_Reelle_Zahlen_axiomatisch/g_art_content_07_vollstaendigkeit.meta.xml}{vollst}
  \end{links}
  \creategeneric
\end{metainfo}


\begin{content}

\usepackage{mumie.ombplus}
\ombchapter{2}
\ombarticle{1}

\lang{de}{\title{Körperaxiome und Rechenregeln}}
\lang{en}{\title{Field axioms and calculation rules}}

\begin{block}[annotation]
  Im Ticket-System: \href{http://team.mumie.net/issues/9638}{Ticket 9638}\\
\end{block}

\begin{block}[info-box]
\tableofcontents
\end{block}


\lang{de}{Wir sind vertraut mit den reellen Zahlen, die wir als endliche oder unendliche Dezimalzahlen und als Punkte
auf der Zahlengeraden kennen. Es gibt zwei grundsätzliche Möglichkeiten, $\R$ einzuführen. 

Einerseits kann man $\R$ konstruktiv aus $\Q$ gewinnen, indem man, wie im 
\ref[content_01_zahlenmengen][Grundlagenkapitel]{introduction_R} gezeigt, anschaulich die "`Lücken"' 
auf der Zahlengeraden auffüllt.  
Andererseits gibt es die axiomatische Beschreibung der reellen Zahlen $\R$, die wir im Folgenden
ausführen möchten.
Dabei wird $\R$ als eine Menge beschrieben, die gewissen Axiomen genügt, und man kann feststellen,
dass $\R$ durch diese Axiome eindeutig festgelegt wird. Dies wird allerdings hier
nicht stringent nachgewiesen.

Die Axiome sind in drei Teile aufgeteilt. 
Der erste Teil besteht aus Gesetzen, die sich lediglich mit der Addition und der 
Multiplikation beschäftigen. Dieser Teil wird nachfolgend in den Abschnitten 
\ref{sec:axiome} und \ref{sec:rechenregeln} behandelt. Der zweite Teil besteht aus 
Regeln zur Anordnung der Elemente (den \link{anordnung}{Anordnungsaxiomen}) und 
der dritte Teil behandelt die "`Lücken"' über entsprechende \link{vollst}{Vollständigkeitsaxiome}.

Viele der in diesem Kapitel beschreibenen Axiome sind als 
\ref[content_02_rechengrundlagen_terme][elementare Rechnenregeln für $\R\,$]{sec:grundrechenarten} 
bereits bekannt aus dem Grundlagenkapitel. Im Rahmen dieses Kapitels werden sie allgemeiner für 
sogenannte \emph{"`Körper"'} beschrieben. 
}
\lang{en}{We are familiar with the real numbers, which we know as terminating or non-terminating decimals
and as points on the number line. There are two basic ways to introduce $\R$.

On the one hand, $\R$ can be constructed from $\Q$, by filling in the 'gaps' on the number line as shown in 
the \ref[content_01_zahlenmengen][chapter on foundations]{introduction_R}. On the other hand, there exists an axiomatic 
description of the real numbers $\R$, which we will establish in this chapter. $\R$ will be described as a set which satisfies 
certain axioms, and one can then check that $\R$ is uniquely determined by these axioms. However, this is not rigorously proven here.

The axioms are split into three parts. The first part consists of rules that only deal with addition and multiplication. 
This part will be handled in sections \ref{sec:axiome} und \ref{sec:rechenregeln}. The second part consists of rules for the ordering of elements
(the \link{anordnung}{order axioms}), and the third part deals with the 'gaps' via the \link{vollst}{completeness axioms}.

Many of the axioms described in this chapter will already be familiar as \ref[content_02_rechengrundlagen_terme][elementary rules for calculation in $\R\,$]{sec:grundrechenarten}
from the chapter on foundations. In the context of this chapter, we will define these more generally for so-called 'fields'.
}


\section{\lang{de}{Körperaxiome}\lang{en}{Field axioms}}\label{sec:axiome}

\begin{definition}[\lang{de}{Körper}\lang{en}{Fields}] \label{def:koerper}
    \lang{en}{Let $\K$ be a set and $a, b, c\in\K.$}
    \lang{de}{Sei $\K$ eine Menge, $a, b, c\in\K.$}
    
    \lang{en}{
    A set $\K$ is called a \notion{field} if two binary operations are defined on it, 
    usually called \emph{addition}, denoted by +, and \emph{multiplication}, 
    denoted by $\cdot$. These binary operations assign any two elements $a,b\in \K$ a unique 
    \emph{sum} $a+b\in \K$ and \emph{product} $a\cdot b\in \K$ such that the following \notion{axioms} are satisfied.
    
    \\
    \emph{Axioms of addition:} \\ \\
    \begin{itemize}
        \item[(A1)] $a+b=b+a$ $\quad$ (commutative law) 
        \item[(A2)] $\left(a+b\right)+c=a+\left(b+c\right)$ $\quad$ 
            (associative law)
        \item[(A3)] There exists a \emph{zero} element $0\in\K,$ such that $a+0=a$ for all $a\in \K$.
             Zero is also called \emph{additive identity} (or neutral element).
        \item[(A4)] For any $a\in\K$ there exists an \emph{additive inverse} 
            (a \emph{negative}) element \\ $-a\in\K$ such that 
            $a+\left(-a\right)=0$.
    \end{itemize}        
     
    \\   
    \emph{Axioms of multiplication:} \\ \\
    \begin{itemize}
        \item[(M1)] $a\cdot b=b\cdot a$ $\quad$ (commutative law)
        \item[(M2)] $a\cdot\left(b\cdot c\right) = \left(a\cdot b\right)\cdot c$ 
            $\quad$ (associative law)
        \item[(M3)] There exists a '\emph{one}' element $1\in\K, 1\neq 0,$\\ such that $a\cdot 1=a$ for any $a\in \K$.
                    One is also called \emph{multiplicative identity}. 
        \item[(M4)] For any $a\neq0$ there exists a \emph{multiplicative 
            inverse} (a \emph{reciprocal}) element $\frac{1}{a}$ (or $a^{-1}$) 
            such that $a\cdot \frac{1}{a}=1$.
    \end{itemize}
    
    \\
    \emph{Compatibility of addition and multiplication:} \\
    \begin{itemize}
        \item[(D)] $\left(a+b\right)\cdot c=a\cdot c+b\cdot c$ $\; \;$ 
            (distributive law)
    \end{itemize}    
    }
    
    \lang{de}{
    Eine Menge $\K$ hei\"st \emph{\notion{K\"orper}}, wenn auf ihr zwei Verkn\"upfungen 
    definiert sind, die gew\"ohnlich als \emph{Addition} (+) und 
    \emph{Multiplikation} ($\cdot$) bezeichnet werden. Diese Verkn\"upfungen ordnen je zwei Elementen $a,b\in \K$ eine
    eindeutige \emph{Summe} $a+b\in \K$ und ein \emph{Produkt} $a\cdot b\in \K$ zu, f\"ur die folgende 
    \notion{Axiome} gelten.\\
    
  
    Axiome der Addition:    
    \begin{itemize}
        \item[\textbf{(A1)}] $a+b=b+a$ $\qquad \qquad \qquad$ \notion{(Kommutativgesetz)}
        \item[\textbf{(A2)}] $\left(a+b\right)+c=a+\left(b+c\right)$ $\quad \,$
                \notion{(Assoziativgesetz)}
        \item[\textbf{(A3)}] Es existiert ein Element \emph{Null} $\; 0\in\K$, so dass $\nowrap{a+0=a}$ für alle $a
            \in \K$. \\ Die Null wird auch \notion{neutrales Element} der Addition genannt. 
        \item[\textbf{(A4)}] Zu jedem $a\in\K$ existiert ein 
                 \notion{inverses Element} der Addition
                (ein \emph{negatives} Element) $-a\in\K$, so dass 
            $a+\left(-a\right)=0$.
    \end{itemize}
    
    \\    
    Axiome der Multiplikation:
    \begin{itemize}
        \item[\textbf{(M1)}] $a\cdot b=b\cdot a$ $\qquad \qquad \quad$ 
                 \notion{(Kommutativgesetz)}
        \item[\textbf{(M2)}] $a\cdot \left(b\cdot c\right) = \left(a\cdot b\right)\cdot c$ 
                $\quad \,$ \notion{(Assoziativgesetz)}
        \item[\textbf{(M3)}] Es gibt ein Element \emph{Eins} $1\in\K,$ mit $1\neq0$
        so dass für alle $a \in \K$  gilt \nowrap{$a\cdot 1=a$.} Die Eins wird auch
         \notion{neutrales Element} der Multiplikation genannt.  
        \item[\textbf{(M4)}] Zu jedem $a\neq0$ existiert ein 
         \notion{inverses Element} der Multiplikation (ein \emph{Kehrwert}) 
        $\frac{1}{a}$ (oder $a^{-1}$) mit $a\cdot \frac{1}{a}=1$.
    \end{itemize}
    
    \\
    Vertr\"aglichkeit von Addition und Multiplikation: \\
    \begin{itemize}
        \item[\textbf{(DG)}] $\left(a+b\right)\cdot c=a\cdot c+b\cdot c$ $\; \;$ 
             \notion{(Distributivgesetz)}
    \end{itemize}
    
    }
  \end{definition}

\begin{example} \label{ex:koerper}
\begin{enumerate}
\item \lang{de}{Die Menge der rationalen Zahlen $\Q$ ist mit der üblichen Addition "`+"' und Multiplikation "`$\cdot$"' 
ein Körper. Ebenso ist die Menge der reellen Zahlen ein Körper, da 
die Axiome (A1), (A2), (M1), (M2) und (DG) genau den aus dem Grundlagenteil bekannten
\ref[content_02_rechengrundlagen_terme][Rechengesetzen für reelle Zahlen]{rule:rechengesetze} entsprechen.
Zudem sind $0$ und $1$ die neutralen Elemente für die Addition bzw. für die Multiplikation in $\R$. Außerdem sei $r$ eine reelle Zahl. Dann ist
$-r$ auch ein Element von $\R$. Wäre $r\neq0$ eine reelle Zahl, gehörte die Inverse davon $\frac{1}{r}$ auch zu den reellen Zahlen.}
\lang{en}{The set of rational numbers $\Q$ equipped with the usual addition '+' and multiplication '$\cdot$' is a field. Likewise the set of real numbers is a field, 
since the axioms (A1), (A2), (M1), (M2) and (D) follow immediately from the familiar \ref[content_02_rechengrundlagen_terme][calculation rules for real numbers]{rule:rechengesetze}. 
Moreover $0$ and $1$ are, respectively, the additive and multiplicative identity elements in $\R$. If $r$ is a real number, then $-r$ is also an element of $\R$, and similarly for a real number $r\neq0$, 
the reciprocal $\frac{1}{r}$ also belongs to the real numbers.}
\item \lang{de}{$(\Z,+,\cdot)$ ist kein Körper, weil z. B. $2$ in $\Z$ kein Inverses bezüglich der Multiplikation besitzt.}
\lang{en}{$(\Z,+,\cdot)$ is not a field, since for example $2$ in $\Z$ has no mupltiplicative inverse in $\Z$.}
\item \lang{de}{Es gibt auch \emph{endliche} Körper, d.h. Körper deren zugrundeliegende Menge endlich viele Elemente besitzt. Das kleinste Beispiel ist die Menge $\mathbb{F}_2 = \{ \bar{0}; \bar{1} \}$ mit den
Verknüpfungen}
\lang{en}{There are also \emph{finite} fields, i.e. fields whose underlying set has finitely many elements. The smallest example is the set $\mathbb{F}_2 = \{ \bar{0}; \bar{1} \}$ with the operations}
\[ \begin{mtable}[\cellaligns{cccc}] \bar{0} + \bar{0} = \bar{0}, &\quad  \bar{0} + \bar{1} = \bar{1},\quad  &
\bar{1} + \bar{0} = \bar{1},& \quad \bar{1} + \bar{1} = \bar{0},\quad  \\
\bar{0} \cdot \bar{0}  = \bar{0}, & \bar{0} \cdot \bar{1} = \bar{0},&
\bar{1} \cdot \bar{0} = \bar{0}, & \bar{1} \cdot \bar{1} = \bar{1}.\end{mtable} \]
\lang{de}{Die Axiome müssen hier nachgerechnet werden. Endliche Körper spielen in der Informatik und in der Algebra
eine Rolle. Für die Analysis sind sie jedoch irrelevant.} 
%Video
\lang{de}{Dieses Beispiel wird im folgenden Video mit leicht veränderten Symbolen 
ausführlicher behandelt.

\floatright{\href{https://api.stream24.net/vod/getVideo.php?id=10962-2-10834&mode=iframe&speed=true}{\image[75]{00_video_button_schwarz-blau}}}\\
\\
}
\lang{en}{The reader can check the axioms. Finite fields play a role in computer science and algebra, but they are not relevant to analysis.}

\end{enumerate}
\end{example}

%Video
\lang{de}{Auch im folgenden Video werden die Körperaxiome besprochen.

\center{\href{https://api.stream24.net/vod/getVideo.php?id=10962-2-10940&mode=iframe&speed=true}{\image[75]{00_video_button_schwarz-blau}}}\\
\\
}
  \begin{remarks}
    \lang{de}{
    \begin{itemize}
        \item Der Punkt wird in Produkten meist weggelassen, d.h., 
            $ab= a\cdot b$.\\
            In Programmiersprachen ist $*$ als Zeichen f\"ur die Multiplikation verbreitet.
        \item $a+\left(-b\right)$ wird zu $a-b$ abgek\"urzt.
        \item $a\cdot \frac{1}{b}$ wird als $\frac{a}{b}$ (oder $a/b$) geschrieben.
		\item Um Klammern zu sparen, wird die "Punkt-vor-Strich-Regel" verwendet, d.h. 
		$a+b\cdot c= a+(b\cdot c)$.
		\item Auch werden die Klammern bei Addition von $3$ oder mehr Termen und bei
		der Multiplikation von $3$ oder mehr Termen weggelassen, da die Reihenfolge der Additionen
		bzw. der Multiplikationen aufgrund der Axiome (A2) bzw. (M2) keine Rolle spielen.
    \end{itemize}    
    }
    
    \lang{en}{
    \begin{itemize}
        \item The dot in a product is frequently omitted, i.e., 
            $ab\equiv a\cdot b$.\\
            In programming languages $*$ is frequently used as multiplication sign.
        \item $a+\left(-b\right)$ is shortened to $a-b$.
        \item $a\cdot \frac{1}{b}$ is written as $\frac{a}{b}$ (or $a/b$).
        \item To save on using brackets, we employ 'order of operations' rules, prioritising multiplication over addition, i.e. 
    		$a+b\cdot c= a+(b\cdot c)$.
    		\item Furthermore, when adding $3$ or more terms, and when multiplying $3$ or more terms, we will omit the brackets, 
      since the associative laws for addition (A2) and multiplication (M2) ensure that rearranging the brackets does not change the result of the operation.
        \end{itemize}
    }
  \end{remarks}

\begin{example}
\lang{de}{Wir betrachten den vollständig geklammerten Ausdruck in den rationalen Zahlen}
\lang{en}{We consider the expression in rational numbers which has explicit brackets in all places}
\[ \textcolor{#0066CC}{(}(1+3)\cdot 2\textcolor{#0066CC}{)} +\textcolor{#CC6600}{(}\textcolor{#00CC00}{(}4-3\textcolor{#00CC00}{)}
+1\textcolor{#CC6600}{)}. \]
\lang{de}{Zunächst kann man wegen (A2) die orangenen Klammern weglassen. Weil $-3$ für $+(-3)$ steht, kann man auch die grünen Klammern
weglassen. Wegen der Punkt-vor-Strich-Regel, kann man auch die blauen Klammern weglassen. Lediglich die schwarzen Klammern
sind wegen der Punkt-vor-Strich-Regel nötig.
Aufgrund des Distributivgesetzes (DG) kann man aber $(1+3)\cdot 2$ durch $(1\cdot 2+3\cdot 2)$ ersetzen und 
erhält dann}
\lang{en}{First of all, by virtue of (A2) we can omit the orange brackets. Because $-3$ stands for $+(-3)$, we can also omit the green brackets. 
Because multiplication takes precedence over addition in the order of operations, we can omit the blue brackets. Only the black brackets are necessary, because of order of operations. 
However, we can apply the distributive law (D) and expand $(1+3)\cdot 2$ to $(1\cdot 2+3\cdot 2)$, which gives}
\[ (1\cdot 2+3\cdot 2)+4+(-3)+1. \]
\lang{de}{Bei diesem Ausdruck man nun auch die letzte Klammer wegen des Assoziativgesetzes (A2) weglassen könnte.}
\lang{en}{In this expression, we can also leave out the final brackets because of associativity of addition (A2).}
\end{example}


\begin{quickcheck}

			\text{\lang{de}{Es seien $a,b$ Elemente eines Körpers $\mathbb{K}$. Welche der 
            folgenden Ausdrücke sind aufgrund der Körperaxiome und Schreibkonventionen gleich dem Ausdruck
            $\; a \cdot (a-b)$?}
            \lang{en}{Let $a,b$ be elements of the field $\mathbb{K}$. Which of the following expressions are, 
            considering the field axioms and notational conventions, the same as the expression $\; a \cdot (a-b)$?}}
\begin{choices}{multiple}
		\begin{choice}
        \text{$a\cdot a+a\cdot (-b)$}
        \solution{true}
      \end{choice}
      \begin{choice}
        \text{$(-b)\cdot a+a$}
        \solution{false}
      \end{choice}
      \begin{choice}
        \text{$(a-b)\cdot a$}
        \solution{true}
      \end{choice}
      \begin{choice}
        \text{$a\cdot a-b$}
        \solution{false}
      \end{choice}
\end{choices}      
      \explanation{
      \lang{de}{$a-b$ ist die Kurzschreibweise für $a+(-b)$, und mit dem Distributivgesetz erhält man dann
      $a \cdot (a+(-b))=a\cdot a+a\cdot (-b)$. \\
      Mit Hilfe des Kommutativgesetzes der Addition und
      der Multiplikation lässt sich letzteres umformen zu $(-b)\cdot a+a\cdot a$, was aber
      im Allgemeinen zu $(-b)\cdot a+a$ verschieden ist.\\
      Den Ausdruck $(a-b)\cdot a$ erhält man direkt aus $a \cdot (a-b)$ durch Anwenden
      des Kommutativgesetzes der Multiplikation mit den Faktoren $a$ bzw. $(a-b)$.\\
      Wegen der Punkt-vor-Strich-Regel können die Klammern nicht einfach weggelassen werden.}
      \lang{en}{$a-b$ is short for $a+(-b)$, and applying the distributive law gives $a \cdot (a+(-b))=a\cdot a+a\cdot (-b)$. \\
      Now commutativity of addition and multiplication lets us rearrange the latter expression to $(-b)\cdot a+a\cdot a$. Note that in general, 
      this is different from $(-b)\cdot a+a$.\\
      The expression $(a-b)\cdot a$ can be obtained directly from $a \cdot (a-b)$  by applying the commutative law for multiplication on the two factors $a$ and $(a-b)$.\\
      Order of operations does not permit us to omit the brackets here.}
      }
\end{quickcheck}



\section{\lang{de}{Rechenregeln in Körpern}\lang{en}{Calculation rules in fields}}\label{sec:rechenregeln}

\lang{de}{Mit Hilfe der Körperaxiome erhält man folgenden Aussagen.}
\lang{en}{With the help of the field axioms, one obtains the following propositions.}

\begin{theorem}
\lang{de}{Sei $\K$ ein Körper. Dann gelten:}
\lang{en}{Let $\K$ be a field. Then:}
\begin{enumerate}
\item \lang{de}{Für alle $a, b\in \K$ gibt es genau ein $x\in \K$ mit $a+x=b$, nämlich $x=b+(-a)=b-a$.\\
  Insbesondere (wenn man $b=a$ wählt) ist das Element $0$ durch (A3) eindeutig bestimmt, 
  und auch das additive Inverse $-a$ ist für jedes $a$ eindeutig. }
  \lang{en}{For all $a, b\in \K$ there is precisely one $x\in \K$ with $a+x=b$, namely $x=b+(-a)=b-a$.\\
  In particular (setting $b=a$) the element $0$ given by (A3) is uniquely defined, and the additive inverse $-a$ is unique for each $a$.}
\item \lang{de}{Für alle $a, b\in \K$ mit $a\neq 0$ gibt es genau ein $y\in \K$ mit $a\cdot y=b$, 
    nämlich $y=b \cdot a^{-1}=\frac{b}{a}$.\\ 
    Insbesondere (wenn man $b=a$ wählt) ist das Element $1$ durch (M3) eindeutig bestimmt, 
    und auch das multiplikative Inverse $a^{-1}$ ist für jedes $a$ eindeutig.}
    \lang{en}{For all $a, b\in \K$ with $a\neq 0$, there is precisely one $y\in \K$ with $a\cdot y=b$, 
    namely $y=b \cdot a^{-1}=\frac{b}{a}$.\\ 
    In particular (setting $b=a$) the element $1$}
\end{enumerate}
\end{theorem}

\begin{proof*}[\lang{de}{Begründung}\lang{en}{Proof}]
\begin{showhide}
\lang{de}{Zunächst erfüllt $x=b-a=b+(-a)$ die Gleichung $a+x=b$, denn}
\lang{en}{To start with, $x=b-a=b+(-a)$ satisfies the equation $a+x=b$, as}
\begin{align*}
 a+ (b+(-a))  &\; = a+((-a)+b) \quad &\text{\lang{de}{nach}\lang{en}{by} (A1)} \\
              &\; = (a+(-a))+b      &\text{\lang{de}{nach}\lang{en}{by} (A2)} \\
              &\; = 0+b 	        &\text{\lang{de}{nach}\lang{en}{by} (A4)} \\
              &\; = b+0 	        &\text{\lang{de}{nach}\lang{en}{by} (A1)} \\
              &\; = b 	            &\text{\lang{de}{nach}\lang{en}{by} (A3)}.
\end{align*}
\lang{de}{Um zu zeigen, dass $b+(-a)$ das einzige Element ist, betrachten wir ein beliebiges Element $z\in \K$ mit $a+z=b$. Dann
folgt}
\lang{en}{To show that $b+(-a)$ is the only such element, consider an arbitrary element $z\in \K$ with $a+z=b$. Then it follows that}
\begin{align*}
 b+(-a)     &\; = (a+z)+(-a)\quad    &  \\
            &\; = (z+a)+(-a)        &\text{\lang{de}{nach}\lang{en}{by} (A1)} \\
            &\; = z+ (a+(-a))       &\text{\lang{de}{nach}\lang{en}{by} (A2)} \\
            &\; = z+  0 	        &\text{\lang{de}{nach}\lang{en}{by} (A4)} \\
            &\; = z 	            &\text{\lang{de}{nach}\lang{en}{by} (A3)}.
\end{align*}
\lang{de}{Der Nachweis, dass $\frac{b}{a}$ das einzige Element ist, das die Gleichung $a\cdot y=b$ erfüllt, geht ganz entsprechend.}
\lang{en}{The proof that $\frac{b}{a}$ is the only element which satisfies $a\cdot y=b$ is similar.}
\end{showhide}
\end{proof*}

\lang{de}{
%Video
In diesem Video finden Sie eine weitere Folgerung aus den Körperaxiomen mit einigen Beispielen:

\floatright{\href{https://api.stream24.net/vod/getVideo.php?id=10962-2-10942&mode=iframe&speed=true}{\image[75]{00_video_button_schwarz-blau}}}\\
\\
}

\lang{de}{Daraus lassen sich alle schon bekannten Rechenregeln für $+$, $\cdot$, $-$ und $:$ herleiten.}
\lang{en}{From this we can derive all the known calculation rules for  $+$, $\cdot$, $-$ and $:$.}

\begin{rule}\label{regelkoerper}
\lang{de}{Sei $\K$ ein Körper. Dann gilt für alle $a, b, c, d \in \K$:}
\lang{en}{Let $\K$ be a field. Then for all $a, b, c, d \in \K$ we have:}
\begin{enumerate}
\item \lang{de}{$-0=0$ und $-(a+b)=(-a)+(-b)=-a-b$, sowie $-(-a)=a$.}\lang{en}{$-0=0$ and $-(a+b)=(-a)+(-b)=-a-b$, and $-(-a)=a$.}
\item $0\cdot b=0$.
\item \lang{de}{$ (-a)\cdot b=-(a\cdot b)= a\cdot (-b)$ und $(-a)\cdot (-b)=a\cdot b$ (Vorzeichenregel),
insbesondere $(-1)\cdot b=-b$.}\lang{en}{$ (-a)\cdot b=-(a\cdot b)= a\cdot (-b)$ and $(-a)\cdot (-b)=a\cdot b$ (sign rule),
in particular $(-1)\cdot b=-b$.}
\item \lang{de}{Aus $ab=0$ folgt, $a=0$ oder $b=0$. }\lang{en}{If we have $ab=0$, then $a=0$ or $b=0$. }
\item \lang{de}{$1^{-1}=1$ und  $(ab)^{-1}=a^{-1}b^{-1}$, falls $a\neq 0$ und $b\neq 0$ sind 
(und damit auch $ab\neq 0$ ist).}\lang{en}{$1^{-1}=1$ and  $(ab)^{-1}=a^{-1}b^{-1}$, if $a\neq 0$ and $b\neq 0$ 
(and thus also $ab\neq 0$).}
\item \lang{de}{Für $a\neq 0$ ist auch $a^{-1}\neq 0$ und es gilt: $(a^{-1})^{-1}=a$.}\lang{en}{For $a\neq 0$ we have $a^{-1}\neq 0$ and: $(a^{-1})^{-1}=a$.}
\item \lang{de}{Sind $b\neq 0$ und $c\neq 0$, so gilt: $\frac{a}{b}=ab^{-1}=(ac)(bc)^{-1}=\frac{ac}{bc}$. 
("`Erweitern und K"urzen"')}\lang{en}{For $b\neq 0$ and $c\neq 0$, we have: $\frac{a}{b}=ab^{-1}=(ac)(bc)^{-1}=\frac{ac}{bc}$. 
("`Expanding and simplifying"'}
\item $\frac{a}{b}\cdot \frac{c}{d}=\frac{ac}{bd}$ \lang{de}{(Multiplikationsregel f"ur Br"uche), wenn}\lang{en}{(Multiplication rules for fractions), for} $b,d\ne 0$
\item $\frac{a}{b}+ \frac{c}{d}=\frac{ad}{bd}+ \frac{bc}{bd}=\frac{ad+bc}{bd}$ \lang{de}{(Additionsregel f"ur Br"uche), wenn}\lang{en}{(Addition rules for fractions), for} $b,d\ne 0$
\item $\frac{a}{b}:\frac{c}{d}=\frac{a}{b}\cdot \frac{d}{c}$ \lang{de}{(Dividieren mittels Multiplizieren mit dem Kehrwert), wenn}\lang{en}{(Dividing by multiplying with the reciprocal), for} $b,c,d\ne 0$
\end{enumerate}
\end{rule}

\lang{de}{
%Video
Im folgenden Video wird die 4. Regel in \ref{regelkoerper} diskutiert und bewiesen.

\floatright{\href{https://api.stream24.net/vod/getVideo.php?id=10962-2-10943&mode=iframe&speed=true}{\image[75]{00_video_button_schwarz-blau}}}\\
\\
}

%Potenzen

\section{Potenzen}

\lang{de}{F"ur die Multiplikation mehrerer gleicher Zahlen wird auch in allgemeinen Körpern die abk"urzende Schreibweise 
mit Potenzen eingef"uhrt, n"amlich:}
\lang{en}{For the repeated multiplication of the same number with itself in a general field, the notation for powers is employed as a shorthand:}
\[  a^n= \underbrace{a\cdots  a}_{n-\text{mal}} \]
\lang{de}{f"ur ein beliebiges Element $a\in \K$ und eine nat"urliche Zahl $n$.\\
Ist $a\neq 0$, so verwendet man des Weiteren noch die Schreibweisen}
\lang{en}{for an element $a\in \K$ and a natural number $n$.\\
For $a\neq 0$, we further define the notation}
\[ a^{-n}=\left(\frac{1}{a}\right)^n \quad \text{und} \quad a^0=1. \]
\lang{de}{Man erh"alt dann f"ur $a,b\neq 0$ und $n,m\in\Z$ die bekannten \emph{Potenzregeln}:}
\lang{en}{For $a,b\neq 0$ and $n,m\in\Z$ one obtains the familiar \emph{rules for exponents.}}
\begin{itemize}
\item $a^n\cdot a^m=a^{n+m}$ \ \lang{de}{und}\lang{en}{and} \ $\frac{a^n}{a^m}=a^{n-m}$
\item $a^n\cdot b^n=(ab)^n$ \ \lang{de}{und}\lang{en}{and} \ $\frac{a^n}{b^n}= (\frac{a}{b})^n$
\item $(a^m)^n=a^{m\cdot n}$
\end{itemize}

\lang{de}{\textbf{Anmerkung:} Potenzen mit rationalen oder gar reellen Exponenten sind in allgemeinen Körpern nicht definiert.}
\lang{en}{\textbf{Remark:} Powers with rational or even real exponents are not defined in general fields.}


\end{content}