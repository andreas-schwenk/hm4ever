%$Id:  $
\documentclass{mumie.article}
%$Id$
\begin{metainfo}
  \name{
    \lang{de}{Anordnungsaxiome und Betrag}
    \lang{en}{Order axioms absolute value}
  }
  \begin{description} 
 This work is licensed under the Creative Commons License Attribution 4.0 International (CC-BY 4.0)   
 https://creativecommons.org/licenses/by/4.0/legalcode 

    \lang{de}{Beschreibung}
    \lang{en}{Description}
  \end{description}
  \begin{components}
    \component{generic_image}{content/rwth/HM1/images/g_img_00_video_button_schwarz-blau.meta.xml}{00_video_button_schwarz-blau}
  \end{components}
  \begin{links}
\link{generic_article}{content/rwth/HM1/T101neu_Elementare_Rechengrundlagen/g_art_content_01_zahlenmengen.meta.xml}{content_01_zahlenmengen}
\link{generic_article}{content/rwth/HM1/T202_Reelle_Zahlen_axiomatisch/g_art_content_04_koerperaxiome.meta.xml}{content_04_koerperaxiome}
\end{links}
  \creategeneric
\end{metainfo}


\begin{content}
\usepackage{mumie.ombplus}
\ombchapter{2}
\ombarticle{2}

\lang{de}{\title{Anordnungsaxiome und Betrag}}
\lang{en}{\title{Order axioms and absolute value}}

\begin{block}[annotation]
  Im Ticket-System: \href{http://team.mumie.net/issues/9647}{Ticket 9647}\\
\end{block}

\begin{block}[info-box]
\tableofcontents
\end{block}


\section{\lang{de}{Anordnungsaxiome}\lang{en}{Order axioms}}

\lang{de}{Auf der Zahlengeraden sind die Zahlen anschaulich geordnet. Größere Zahlen liegen weiter rechts, kleinere weiter links.
Dies dient als Anschauung für den abstrakten Begriff der Anordnung in einem Körper.}
\lang{en}{On the number line, number lines are clearly ordered. Bigger numbers lie more to the right, and smaller ones more to the left.
This serves as a demonstration for the abstract concept of an order on a field.}

\begin{definition}[\lang{de}{Anordnungsaxiome}\lang{en}{Order axioms}]\label{def:axiome}
\lang{de}{Sei $\K$ ein Körper. Eine \notion{\emph{Anordnungsrelation} $\; <\;$} (sprich: \emph{"`kleiner"'}) 
kann auf $\K$ definiert werden, wenn gilt:}
\lang{en}{Let $\K$ be a field. An \notion{\emph{order relation} $\; <\;$} (said: \emph{'less than'}) can be defined on $\K$ if the following axioms hold:}
\begin{enumerate}
\item[\textbf{(O1)}] \lang{de}{F"ur $a,b\in \K$ gilt \textbf{genau eine} der Beziehungen \\
        $a=b \;$ oder $\; a<b \;$ oder $\; b<a .\qquad$  \notion{(Trichotomiegesetz)}}
        \lang{en}{For $a,b\in \K$ exactly one of the relations 
        $a=b \;$ or $\; a<b \;$ or $\; b<a holds.\qquad$  \notion{(Law of trichotomy)}}
\item[\textbf{(O2)}] \lang{de}{Sind $a<b\,$ und $\, b<c$, so gilt $a<c. \quad$ \notion{(Transitivit"at)}}
                    \lang{en}{If $a<b\,$ and $\, b<c$, then $a<c. \quad$ \notion{(Transitivity)}}
\item[\textbf{(O3)}] \lang{de}{Ist $a<b$, so gilt f"ur alle $c\in \K$:   $\; a+c<b+c ,$\\
        (Monotonie bzgl. der Addition).}
        \lang{en}{If $a<b$, then for all $c\in \K$:   $\; a+c<b+c ,$\\
        (Monotony with respect to addition).}
\item[\textbf{(O4)}] \lang{de}{Ist $a<b\,$ und $\, 0<c$, so gilt $\; ac<bc ,$ \\
        (Monotonie bzgl. der Multiplikation mit positiven Elementen).}
        \lang{en}{If $a<b\,$ and $\, 0<c$, then $\; ac<bc ,$ \\
        (Monotony with respect to multiplying by positive elements).}
\end{enumerate}
\lang{de}{Der Körper $\K$ zusammen mit der Anordnung $<$ heißt dann \notion{\emph{angeordneter 
Körper}} und man schreibt $\mathbf{(\K,<)}$.}
\lang{en}{The field $\K$ together with the order $<$ is then called an \notion{\emph{ordered field}} and we write $\mathbf{(\K,<)}$.}
\end{definition}

\begin{remark}\lang{de}{Man verwendet folgende weitere Schreib- und Sprechweisen:}\lang{en}{The following notations and terminology is also used:}
\begin{enumerate}
\item \lang{de}{$a>b \;$ (sprich: \emph{$a$ größer  $b$}), wenn $b<a$,}\lang{en}{$a>b \;$ (said: \emph{$a$ greater than $b$}), if $b<a$,}
\item \lang{de}{$a\leq b \;$ (sprich: \emph{$a$ kleiner gleich $b$}), wenn $a<b$ oder $a=b$,}\lang{en}{$a\leq b \;$ (said: \emph{$a$ less than or equal to $b$}), if $a<b$ or $a=b$,}
\item \lang{de}{$a\geq b\; $ (sprich: \emph{$a$ größer gleich $b$}), wenn $a>b$ oder $a=b$,}\lang{en}{$a\geq b\; $ (said: \emph{$a$ greater than or equal to $b$}), if $a>b$ or $a=b$,}
\item \lang{de}{$a$ heißt \emph{positiv}, wenn $a>0$,}\lang{en}{$a$ is called \emph{positive}, if $a>0$,}
\item \lang{de}{$a$ heißt \emph{negativ}, wenn $a<0$,}\lang{en}{$a$ is called \emph{negative}, if $a<0$,}
\item \lang{de}{$a$ heißt \emph{nicht-positiv}, wenn $a\leq 0$,}\lang{en}{$a$ is called \emph{nonpositive}, if $a\leq 0$,}
\item \lang{de}{$a$ heißt \emph{nicht-negativ}, wenn $a\geq 0$.}\lang{en}{$a$ is called \emph{nonnegative}, if $a\geq 0$.}
\end{enumerate}
\end{remark}

\begin{example}
  \begin{enumerate}
    \item \lang{de}{Das anschauliche Beispiel der rationalen und reellen Zahlen auf der Zahlengeraden 
        bildet einen angeordneten Körper mit der Relation $<$, falls $a < b$ genau dann, 
        wenn $a$ auf der Zahlengeraden weiter links liegt als $b$.}
        \lang{en}{The examples of the rational and real numbers on a number line form an ordered field with the relation $<$, 
        where $a < b$ precisely when $a$ lies further left on the number line than $b$.}


    \item \lang{de}{Für den im vorherigen \ref[content_04_koerperaxiome][Kapitel zu Körperaxiomen]{ex:koerper}
      beschriebenen endlichen Körper $\mathbb{F}_2=\{ \bar{0}; \bar{1}\}\,$ gibt es keine 
      Anordnung, denn angenommen $\mathbb{F}_2$ wäre ein angeordneter Körper, dann müsste
      nach dem Trichotomiegesetz \textbf{(O1)} wegen $\bar{0}\neq \bar{1}$ entweder
      $\bar{0}< \bar{1}$ oder $\bar{1}<\bar{0}$ sein.\\
      }
      \lang{en}{For the finite field $\mathbb{F}_2=\{ \bar{0}; \bar{1}\}\,$ previously described in the \ref[content_04_koerperaxiome][chapter on field axioms]{ex:koerper}, there exists no order. 
      This is true because if we suppose that $\mathbb{F}_2$ were an ordered firld, then the law of trichotomy \textbf{(O1)} states that since $\bar{0}\neq \bar{1}$ we must have 
      either $\bar{0}< \bar{1}$ or $\bar{1}<\bar{0}$.\\
      }
      
      \lang{de}{Aus $\bar{0}< \bar{1}$ würde dann mit $c=\bar{1}$ und dem Axiom \textbf{(O3)} folgen,
      dass $\; \bar{0}+\bar{1}<\bar{1}+\bar{1} \,$ und damit  $\; \bar{1}<\bar{0}\,$ ist, was 
      im Widerspruch zur Prämisse steht, denn es kann nicht $\bar{0}< \bar{1}$ 
      und $\bar{1}<\bar{0}$ gleichzeitig erfüllt sein. \\
      }
      \lang{en}{From $\bar{0}< \bar{1}$ it follows from $c=\bar{1}$ and the Axiom \textbf{(O3)},
      that $\; \bar{0}+\bar{1}<\bar{1}+\bar{1} \,$ and therefore  $\; \bar{1}<\bar{0}\,$, which is in contradition to the assumption, since we cannot have both $\bar{0}< \bar{1}$ 
      and $\bar{1}<\bar{0}$. \\
      }
      
      \lang{de}{Ebenso führt $\bar{1}<\bar{0} \,$ zu $\, \bar{0}< \bar{1}$, womit gezeigt ist, dass 
      der Körper $\mathbb{F}_2$ das Trichotomiegesetz \textbf{(O1)} nicht erfüllt.}
      \lang{en}{Likewise from $\bar{1}<\bar{0} \,$ follows that $\, \bar{0}< \bar{1}$, so we have shown that  
      the field $\mathbb{F}_2$ does not satisfy the law of trichotomy \textbf{(O1)}.}

  \end{enumerate}
\end{example}

\lang{de}{
%Video
Im folgenden Video wird die Anordnungsrelation für die reellen Zahlen besprochen.

\floatright{\href{https://api.stream24.net/vod/getVideo.php?id=10962-2-10945&mode=iframe&speed=true}{\image[75]{00_video_button_schwarz-blau}}}\\
\\
}

\section{\lang{de}{Ungleichungsregeln und Betrag}\lang{en}{Rules for inequalities and absolute value}}


\lang{de}{Aus den Axiomen lassen sich viele von den reellen Zahlen bekannte Regeln herleiten:}
\lang{en}{From the axioms we can derive many of the familiar rules for the real numbers:}
\begin{rule}[\lang{de}{Anordnungsregeln}\lang{en}{Rules for inequalities}]\label{rule:regeln}
\lang{de}{Für einen angeordneten Körper $(\K,<)$ und beliebige $a,b\in \K$ gilt:}\lang{en}{For the ordered field $(\K,<)$ and any $a,b\in \K$ we have:}
\begin{itemize}
\item $a<b \Leftrightarrow 0<b-a$
\item \lang{de}{$a<0 \Leftrightarrow -a>0$, sowie $a>0 \Leftrightarrow -a<0$}\lang{en}{$a<0 \Leftrightarrow -a>0$, and likewise $a>0 \Leftrightarrow -a<0$}
\item $a<b  \Leftrightarrow -a>-b$
\item \lang{de}{Ist $a<b$ und $c<0$, dann gilt $\; ac>bc \quad$ \\
      (\emph{Bei Multiplikation einer Ungleichung mit einer negativen Zahl dreht sich
      das Ungleichheitszeichen um.})}
      \lang{en}{If $a<b$ and $c<0$, then $\; ac>bc \quad$ \\
      (\emph{Multiplying an inequality by a negative number causes the inequality symbol to flip.})}
\item \lang{de}{Ist $a<b$ und $c<d$, dann gilt $\; a+c<b+d$.}\lang{en}{If $a<b$ and $c<d$, then $\; a+c<b+d$.}
\item \lang{de}{$ab>0$ gilt genau dann, wenn $a>0$ und $b>0$ ist, oder wenn $a<0$ und $b<0$ ist.}\lang{en}{$ab>0$ holds precisely when either $a>0$ and $b>0$, or $a<0$ and $b<0$.}
\item \lang{de}{$ab<0$ gilt genau dann, wenn $a>0$ und $b<0$ ist, oder wenn $a<0$ und $b>0$ ist.}\lang{en}{$ab<0$ holds precisely when either $a>0$ and $b<0$, or $a<0$ and $b>0$.}
\item \lang{de}{F"ur alle $a\ne 0$ gilt $a^2>0$  (insbesondere $1=1^2>0$).}\lang{en}{For all $a\ne 0$ we have $a^2>0$ (in particular $1=1^2>0$).)}
\item $a>0 \Leftrightarrow \frac{1}{a}>0$

\end{itemize}
\end{rule}


\lang{de}{Wichtige Regeln ergeben sich auch für Potenzen von nicht-negativen Elementen.}
\lang{en}{Important rules also emerge for powers of non-negative elements.}

\begin{rule}[\lang{de}{Anordnungsregeln für Potenzen}\lang{en}{Rules for inequalities with powers}]\label{rule:potenzen}
\lang{de}{Sei $(\K,<)$ ein angeordneter Körper.}
\lang{en}{Let $(\K,<)$ be an ordered field.}
\begin{enumerate}
\item \lang{de}{Für $a,b\in \K$ mit $a\geq 0$ und $b\geq 0$ und $n\in \N$ gilt:}
\lang{en}{For $a,b\in \K$ with $a\geq 0$ and $b\geq 0$ and $n\in \N$ we have:}
\[ \begin{mtable}
		a<b & \Longleftrightarrow a^n < b^n, \\
	    a=b & \Longleftrightarrow a^n = b^n.
\end{mtable} \]
\item \lang{de}{Für $a\in \K$ mit $a>1$ und $m,n\in \Z$ gilt}
\lang{en}{For $a\in \K$ with $a>1$ and $m,n\in \Z$ we have}
\[   m<n  \Longleftrightarrow a^m<a^n. \]
\end{enumerate}
\end{rule}


\lang{de}{Desweiteren gibt es eine der \ref[content_01_zahlenmengen][Betragsdefinition in $\R$]{def:betrag}
entsprechende Definition für angeordnete Körper:}
\lang{en}{Furthermore, there is a definition for general ordered fields which corresponds to the \ref[content_01_zahlenmengen][definition of absolute value in $\R$]{def:betrag}:}

\begin{definition}[\lang{de}{Betrag}\lang{en}{Absolute value}]\label{def:betrag}
\lang{de}{Sei $(\K,<)$ ein angeordneter Körper und $a\in \K$.
Der \emph{Betrag} von $a$ ist }
\lang{en}{Let $(\K,<)$ be an ordered field and $a\in \K$.
The \emph{absolute value} of $a$ is }
\[
	{\abs{a}} = 
	\begin{cases}
		 a & \text{\lang{de}{falls }\lang{en}{if }}a \geq 0\\
		-a & \text{\lang{de}{falls }\lang{en}{if }}a < 0.
	\end{cases}
\]
\end{definition}

\begin{remark}
\lang{de}{Aus dem zweiten Punkt in Regel \ref{rule:regeln} folgt, dass der Betrag $\abs{a}$ eines Elements 
$a\neq 0$ immer positiv ist, und dass $\abs{a}=0$ genau dann gilt, wenn $a=0$.\\
Im anschaulichen Fall der Zahlengeraden 
(siehe Beispiel \ref[content_01_zahlenmengen][Zahlengerade in $\R$]{ex:betrag})
stimmt $\abs{a}$ mit dem Abstand von $a$ zu $0$ überein.}
\lang{en}{From the second bullet point in rule \ref{rule:regeln} follows that the absolute value $\abs{a}$ of an element 
$a\neq 0$ is always positive, and that $\abs{a}=0$ holds precisely when $a=0$.\\
In the exemplary case of the number line (see example \ref[content_01_zahlenmengen][number line in $\R$]{ex:betrag}) 
$\abs{a}$ coincides with the distance from $a$ to $0$.}

\end{remark}

\lang{de}{Für den Betrag gelten einige Rechenregeln:}
\lang{en}{For the absolute value, we have several calculation rules:}
\begin{rule}[\lang{de}{Anordnungsregeln für Beträge}\lang{en}{Rules for inequalities with absolute values}]\label{rule:anordnungsregeln}
\lang{de}{Für einen angeordneten Körper $(\K,<)$ und $x,y\in \K$ gilt:}\lang{en}{For an ordered field $(\K,<)$ and $x,y\in \K$ we have:}
\begin{enumerate}
  \item
  	$|x|\geq0,\quad |x|=0\Leftrightarrow x=0$
  \item
  	$|xy|=|x|\cdot |y|,\quad |-x|=|x| $
  \item
    \lang{de}{Für $y>0$ gilt:}\lang{en}{For $y>0$ we have:}\\
  	$ |x| < y \quad\Leftrightarrow \quad-y < x < y,$\\
  	 $|x| \leq y \quad\Leftrightarrow \quad-y \leq x \leq y.$    
  \item
  	\lang{de}{$|x+y|\leq |x|+|y|  \qquad$ \notion{(Dreiecksungleichung)} \\
              $|x-y|\leq |x|+|y|$}
  	\lang{en}{$|x+y|\leq |x|+|y|  \qquad$ \notion{(triangle inequality)} \\
              $|x-y|\leq |x|+|y|$}
  \item
	$  {|x| - |y|} \leq \big| {|x|-|y|}\big| \leq {|x-y|}$

\end{enumerate}

\end{rule}

\begin{proof*}[\lang{de}{Begründung}\lang{en}{Proof}]
\begin{showhide}
  \lang{de}{Die ersten beiden Regeln sind aus der Definition des Betrags leicht ersichtlich.}
  \lang{en}{The first two rules are easily seen from the definition of the absolute value. We prove the others.}\\
  \begin{enumerate}
  \setcounter{enumi}{2}
  \item \lang{de}{Für $y>0$ gilt:}\lang{en}{For $y>0$ we have:}\\
  	$ |x| < y \quad\Leftrightarrow$\\
    \[\begin{cases}
		(a) \; x>0: & x< y,\\
		(b) \; x\leq 0: & -x< y \quad\Leftrightarrow\quad x>y.
	\end{cases}\]
    $\Leftrightarrow \quad-y < x < y.$\\
    \lang{de}{In ähnlicher Weise beweist man die Ungleichung $|x| \leq y \quad\Leftrightarrow \quad-y \leq x \leq y$.}
    \lang{en}{In a similar fashion we prove the inequality $|x| \leq y \quad\Leftrightarrow \quad-y \leq x \leq y$.}
  
% Beweis der 4. Regel:    
  \item \lang{de}{Um die Dreiecksungleichung zu beweisen, betrachten wir drei folgende Fälle:}
  \lang{en}{To prove the triangle inequality, consider the following three cases:}
  
  \lang{de}{\emph{1. Fall:} $\; x,y>0$ oder $x,y\leq 0$. Dann gilt $|x+y|=|x|+|y|$, also $|x+y|\leq |x|+|y|$.}
  \lang{en}{\emph{Case 1:} $\; x,y>0$ or $x,y\leq 0$. Then we have $|x+y|=|x|+|y|$, so $|x+y|\leq |x|+|y|$.}
  
  \lang{de}{\emph{2. Fall:} $\; x>0$ und $y\leq 0$. Hier muß man zwei Fälle betrachten:  }
  \lang{en}{\emph{Case 2:} $\; x>0$ and $y\leq 0$. Here one must consider two cases:  }
  \[\begin{cases}
		(a) \; x+y>0: & |x+y|=x+y\leq x\leq x+(-y)=|x|+|y|,\\
		(b) \; x+y\leq 0: & |x+y|=-x-y\leq -y=|y|\leq |x|+|y|.
	\end{cases}\] 

  \lang{de}{\emph{3. Fall:} $\; y>0$ und $x\leq 0$. Dieser Fall kann ähnlich wie der zweite Fall nachgewiesen werden.  }
  \lang{en}{\emph{Case 3:} $\; y>0$ and $x\leq 0$. This case can be proved in a similar manner to the second case.  }

  \lang{de}{Alle drei Fälle liefern als Ergebnis $\,|x+y|\leq |x|+|y|$, womit die Dreiecksungleichung bewiesen ist.}
  \lang{en}{All three cases give as a result $\,|x+y|\leq |x|+|y|$, so the triangle inequality is proved.}

% 2. Teil der Dreiecksungleichung  
  \lang{de}{Als eine Folgerung der Dreiecksungleichung kann man schreiben}
  \lang{en}{As a consequence of the triangle inequality we have that}
  \[|x-y|=|x+(-y)|\leq |x|+|-y|=|x|+|y|.\]

% Beweis der 5. Regel:

  \item \lang{de}{Aufgrund der Dreiecksungleichung gilt: }\lang{en}{By the triangle inequality, we have: }
 
  \[|x-y|+|x|=|y-x|+|x|\geq |y-x+x|=|y|,\]
  \lang{de}{folglich gilt also}\lang{en}{and it follows that}
  \[|x-y|\geq |y|-|x| \; \Leftrightarrow \;  -|x-y|\leq |x|-|y| \qquad (*).\] 
   \lang{de}{Ähnlicherweise gilt auch: }\lang{en}{Similarly it also holds that: }
  \[|x-y|+|y|\geq |x-y+y|=|x| \; \Leftrightarrow \; |x-y|\geq |x|-|y| \quad (**).\]
   \lang{de}{Aus $(*)$ und $(**)$ folgt per Definition \ref{def:betrag}:}\lang{en}{From $(*)$ and $(**)$ definition \ref{def:betrag} gives:}
  \[|x-y|\geq ||x|-|y||.\]
  \lang{de}{Da für jedes $a\in\R$ gilt $|a|\geq a$, die Ungleichung $||x|-|y||\geq |x|-|y|$ ist auch gültig.}
  \lang{en}{Since for all $a\in\R$ we have that $|a|\geq a$, the inequality $||x|-|y||\geq |x|-|y|$ also holds.}
  \end{enumerate}
  \end{showhide}
\end{proof*}

\begin{remark}
  \lang{en}{The triangle inequality is also true for finite sums:}
  \lang{de}{Die Dreiecksungleichung gilt auch f\"ur endliche Summen:}
  \[
  {\left\vert \sum_{k=1}^n x_k \right\vert} \leq \sum_{k=1}^n {|x_k|},\quad x_k\in\K.
  \]
\end{remark}

\lang{de}{
%Video
Im folgenden Video werden sowohl die Ordnungsrelation zusammen mit ihren Eigenschaften als auch Betrag und seine Eigenschaften 
leicht unterschiedlich vorgestellt.

\floatright{\href{https://api.stream24.net/vod/getVideo.php?id=10962-2-10894&mode=iframe&speed=true}{\image[75]{00_video_button_schwarz-blau}}}\\
}

\end{content}