\documentclass{mumie.problem.gwtmathlet}
%$Id$
\begin{metainfo}
  \name{
    \lang{de}{A02: Intervallschreibweise}
    \lang{en}{input interval}
  }
  \begin{description} 
 This work is licensed under the Creative Commons License Attribution 4.0 International (CC-BY 4.0)   
 https://creativecommons.org/licenses/by/4.0/legalcode 

    \lang{de}{Die Beschreibung}
    \lang{en}{description}
  \end{description}
  \corrector{system/problem/GenericCorrector.meta.xml}
  \begin{components}
    \component{js_lib}{system/problem/GenericMathlet.meta.xml}{gwtmathlet}
  \end{components}
  \begin{links}
  \end{links}
  \creategeneric
\end{metainfo}
\begin{content}
\usepackage{mumie.ombplus}
\usepackage{mumie.genericproblem}

\lang{de}{\title{A02: Intervallschreibweise}}
\lang{en}{\title{Problem 2}}

\begin{block}[annotation]
	Im Ticket-System: \href{http://team.mumie.net/issues/9993}{Ticket 9993}
\end{block}

Geben Sie die nachfolgenden Mengen in Intervallschreibweise an. Verwenden Sie
\textit{infty} für $\infty$ und \textit{-infty}~ für $-\infty$, wenn nötig. Auch die 
Intervallgrenzen lassen sich durch Anklicken ändern!
 
\begin{problem}
	\randomquestionpool{1}{2}
	\randomquestionpool{3}{4}
	\randomquestionpool{5}{8}
	\randomquestionpool{9}{10}
	

    \begin{question} %1
     
      \lang{de}{
      	\text{%
        %Geben Sie die nachfolgende Menge in Intervallschreibweise an.\\
      	 $[\var{a};\var{b}) \cup [\var{b};\var{c}) =$\ansref
              }}

      \type{input.interval}
      
      \field{rational}
      \begin{variables}
      \randint{a}{-30}{-15}
      \randint{b}{-14}{15}
      \randint{c}{16}{30}
      
      \end{variables}
     
     \begin{answer}
             \solution{[a;c)} 
             \explanation[edited]{Durch die Vereinigung gehen beide Intervalle nahtlos ineinander über.}
        \end{answer}
%      \type{mc.unique}
%      	\begin{choice}
%     	 	\text{ $[\var{a},\var{c})$ }
%   		 	\solution{true}
% 		\end{choice}
		
% 		\begin{choice}
%     	 	\text{ $(\var{a},\var{c})$ }
%   		 	\solution{false}
% 		\end{choice}
		
% 		\begin{choice}
%     	 	\text{ $(\var{a},\var{c}]$ }
%   		 	\solution{false}
% 		\end{choice}
     
    \end{question}
    
    
    \begin{question} %2
     
      \lang{de}{
      	\text{%
        %Geben Sie die nachfolgende Menge in Intervallschreibweise an.\\
      	 $(\var{d}; \var{dk}) \cap [\var{d}; \var{f} ]=$\ansref
              }}

      \type{input.interval}
      \field{rational}
      \begin{variables}
      \randint{d}{-30}{30}
      \randint{k}{3}{10}
      \randint{l}{1}{2}
      \function[calculate]{dk}{d+k}
      \function[calculate]{f}{d+k-l}
      
      \end{variables}
      
        \begin{answer}
             \solution{(d;f]} 
             \explanation[edited]{Die linke Intervallgrenze ist nur in einem der beiden 
             Intervalle enthalten.}
        \end{answer}

%       \type{mc.unique}
%       \begin{choice}
%     	 	\text{ $[\var{d},\var{dkl})$ }
%   		 	\solution{false}
% 		\end{choice}
     
%      	\begin{choice}
%     	 	\text{ $(\var{d},\var{dkl})$ }
%   		 	\solution{false}
% 		\end{choice}
		
		
% 		\begin{choice}
%     	 	\text{ $(\var{d},\var{dkl}]$ }
%   		 	\solution{true}
% 		\end{choice}
      
%      	\begin{choice}
%     	 	\text{ $[\var{d},\var{dkl}]$ }
%   		 	\solution{false}
% 		\end{choice}


    \end{question}


\begin{question} %3
%f=mx,m>0
     
      \lang{de}{
      	\text{ $\{ x\in \R \vert \:\var{m} x \le \var{t} \}=$\ansref
              }}
      
      \type{input.interval}      
      \field{rational}
      \begin{variables}
      \randint{t}{1}{10}
      \randint{m}{2}{5}
      \function[calculate]{tm}{t/m}
      
      \end{variables}
      
        \begin{answer}
             \solution{(-infinity;tm]} 
        \end{answer}      
     
%         \type{mc.unique}
%      	\begin{choice}
%     	 	\text{ $(- \infty, \var{t} ]$ }
%   		 	\solution{false}
% 		\end{choice}
		
% 		\begin{choice}
%     	 	\text{ $(- \infty, \frac{\var{t}}{\var{m}} )$ }
%   		 	\solution{false}
% 		\end{choice}
		
% 		\begin{choice}
%     	 	\text{ $(- \infty, \frac{\var{t}}{\var{m}} ]$ }
%   		 	\solution{true}
% 		\end{choice}
		
% 		\begin{choice}
%     	 	\text{ $[- \infty, \frac{\var{t}}{\var{m}} ]$ }
%   		 	\solution{false}
% 		\end{choice}
     
    \end{question}
    
    \begin{question} %4
%f=mx,m<0
     
      \lang{de}{
      	\text{ $\{ x\in \R \vert\: \var{m} x \le \var{t} \}=$\ansref
              }}
      
      \type{input.interval}      
      \field{rational}
      \begin{variables}
      \randint{t}{1}{10}
      \randint{m}{-5}{-2}
      \function[calculate]{tm}{t/m}
      
      \end{variables}
      
        \begin{answer}
             \solution{[tm;infinity)} 
             \explanation[edited]{Bei Multiplikation mit $-1$ dreht sich das Vorzeichen
             der Ungleichung um.}
        \end{answer}      
     
     	
		
%       \type{mc.unique}
% 		\begin{choice}
%     	 	\text{ $(\frac{\var{t}}{\var{m}}, \infty)$ }
%   		 	\solution{false}
% 		\end{choice}
		
% 		\begin{choice}
%     	 	\text{ $[\var{t}, \infty)$ }
%   		 	\solution{false}
% 		\end{choice}
		
% 		\begin{choice}
%     	 	\text{ $[\frac{\var{t}}{\var{m}}, \infty]$ }
%   		 	\solution{false}
% 		\end{choice}
		
% 		\begin{choice}
%     	 	\text{ $[\frac{\var{t}}{\var{m}}, \infty)$ }
%   		 	\solution{true}
% 		\end{choice}
     
    \end{question}
    
    
    
    \begin{question} %5
%f=x^2
     
      \lang{de}{
      	\text{ $\{ x\in \R \vert \:x^2 \le \var{t} \}=$\ansref
              }}
      
      \type{input.interval}      
      \field{rational}
      \begin{variables}
      \randint{s}{1}{10}
      \randint{m}{1}{4}
      \function[calculate]{sm}{s/m}
      \function[calculate]{t}{sm^2}
      \function[calculate]{msm}{-sm}
      
      \end{variables}
      
      \begin{answer}
             \solution{[msm;sm]} 
      \end{answer}          
     	
		
%       \type{mc.unique}
% 		\begin{choice}
%     	 	\text{ $ ( - \sqrt{\var{t}}, \sqrt{\var{t}} )$ }
%   		 	\solution{false}
% 		\end{choice}
		
% 			\begin{choice}
%     	 	\text{ $[ -\var{t}, \var{t} ]$ }
%   		 	\solution{false}
% 		\end{choice}
		
% 		\begin{choice}
%     	 	\text{ $[ - \sqrt{\var{t}}, \sqrt{\var{t}} ]$ }
%   		 	\solution{true}
% 		\end{choice}
		
% 	\begin{choice}
%     	 	\text{ $( -\var{t}, \var{t} )$ }
%   		 	\solution{false}
% 		\end{choice}
     
    \end{question}
    
        \begin{question} %6
%f=x^2
     
      \lang{de}{
      	\text{ $\{ x\in \R \vert \:x^2 < \var{t} \}=$\ansref
              }}
      
      \type{input.interval}      
      \field{rational}
      \begin{variables}
      \randint{s}{1}{10}
      \randint{m}{1}{4}
      \function[calculate]{sm}{s/m}
      \function[calculate]{t}{sm^2}
      \function[calculate]{msm}{-sm}
      
      \end{variables}
      
      \begin{answer}
             \solution{(msm;sm)} 
      \end{answer}          
     	
		
     
    \end{question}
    
    \begin{question} %7
%f=|x|
     
      \lang{de}{
      	\text{ $\{ x\in \R \vert |x| \le \var{t} \}=$\ansref
              }}

      \type{input.interval}      
      \field{rational}
      \begin{variables}
      \randint{s}{1}{10}
      \randint{m}{1}{4}
      \function[calculate]{sm}{s/m}
      \function[calculate]{t}{sm^2}
      \function[calculate]{mt}{-t}
      
      \end{variables}
      
      \begin{answer}
             \solution{[mt;t]} 
      \end{answer}              
     	
%       \type{mc.unique}		
% 		\begin{choice}
%     	 	\text{ $ ( - \infty, \var{t} ]$ }
%   		 	\solution{false}
% 		\end{choice}
		
% 			\begin{choice}
%     	 	\text{ $[ -\var{t}, \var{t} ]$ }
%   		 	\solution{true}
% 		\end{choice}
		
% 		\begin{choice}
%     	 	\text{ $( - \infty, \var{t} )$ }
%   		 	\solution{false}
% 		\end{choice}
		
% 	    \begin{choice}
%     	 	\text{ $( -\var{t}, \var{t} )$ }
%   		 	\solution{false}
% 		\end{choice}
     
    \end{question}
    
        \begin{question} %8
%f=|x|
     
      \lang{de}{
      	\text{ $\{ x\in \R \vert\: |x| < \var{t} \}=$\ansref
              }}

      \type{input.interval}      
      \field{rational}
      \begin{variables}
      \randint{s}{1}{10}
      \randint{m}{1}{4}
      \function[calculate]{sm}{s/m}
      \function[calculate]{t}{sm^2}
      \function[calculate]{mt}{-t}
      
      \end{variables}
      
      \begin{answer}
             \solution{(mt;t)} 
      \end{answer}       
    \end{question}
    
    

    \begin{question} %9

     
      \lang{de}{
      	\text{ $ \{ x\in \R \vert \:(\var{auf})^3 < \var{q} \}=$\ansref
              }}
      
      \type{input.interval}      
      \field{rational}
      \begin{variables}
      \randint[Z]{qq}{-3}{3}
      \function[normalize]{q}{qq*qq*qq}
      \randint{n}{-5}{5}
      \function[normalize]{n1}{-n}
      \function[normalize]{auf}{x-n}
      \function[calculate]{ub}{n+qq}      
      \end{variables}
      
      \begin{answer}
             \solution{(-infinity;ub)} 
      \end{answer}       

% 		\type{mc.unique}
%       \begin{choice}
%     	 	\text{ $(- \infty,\var{n}+ \sqrt[3]{\var{q}}) $ }
%   		 	\solution{true}
% 		\end{choice}
		
% 			\begin{choice}
%     	 	\text{ $(- \infty, \var{q}) $ }
%   		 	\solution{false}
% 		\end{choice}
		
% 		\begin{choice}
%     	 	\text{ $(\var{n1}-\sqrt[3]{\var{q}},\var{n}+ \sqrt[3]{\var{q}}) $ }
%   		 	\solution{false}
% 		\end{choice}
		
% 	    \begin{choice}
%     	 	\text{ $[\var{n1}-\sqrt[3]{\var{q}}, \var{n}+\sqrt[3]{\var{q}}] $ }
%   		 	\solution{false}
% 		\end{choice}
     
    \end{question}

    \begin{question} %10

     
      \lang{de}{
      	\text{ $ \{ x\in \R \vert\: (\var{auf})^3 \le \var{q} \}=$\ansref
              }}
      
      \type{input.interval}      
      \field{rational}
      \begin{variables}
      \randint[Z]{qq}{-3}{3}
      \function[normalize]{q}{qq*qq*qq}
      \randint{n}{-5}{5}
      \function[normalize]{n1}{-n}
      \function[normalize]{auf}{x-n}
      \function[calculate]{ub}{n+qq}      
      \end{variables}
      
      \begin{answer}
             \solution{(-infinity;ub]} 
      \end{answer}       

    \end{question}


\end{problem}

\embedmathlet{gwtmathlet}

\end{content}