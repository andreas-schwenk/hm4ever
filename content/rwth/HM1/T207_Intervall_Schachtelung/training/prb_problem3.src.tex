\documentclass{mumie.problem.gwtmathlet}
%$Id$
\begin{metainfo}
  \name{
    \lang{de}{A03: abgeschlossene/offene Mengen}
    \lang{en}{mc unique}
  }
  \begin{description} 
 This work is licensed under the Creative Commons License Attribution 4.0 International (CC-BY 4.0)   
 https://creativecommons.org/licenses/by/4.0/legalcode 

    \lang{de}{Die Beschreibung}
    \lang{en}{}
  \end{description}
  \corrector{system/problem/GenericCorrector.meta.xml}
  \begin{components}
    \component{js_lib}{system/problem/GenericMathlet.meta.xml}{gwtmathlet}
  \end{components}
  \begin{links}
  \end{links}
  \creategeneric
\end{metainfo}
\begin{content}
\usepackage{mumie.ombplus}
\usepackage{mumie.genericproblem}

\lang{de}{\title{A03: abgeschlossene/offene Mengen}}
\lang{en}{\title{Problem 3}}

\begin{block}[annotation]
	Im Ticket-System: \href{http://team.mumie.net/issues/9994}{Ticket 9994}
\end{block}

\begin{problem}
	
	
	\begin{variables}
			\randint{a}{-10}{9}
			\randint{b}{10}{30}
			\randint{c}{-20}{-6}
			\randint{d}{-5}{15}
	\end{variables}
		
	\randomquestionpool{1}{2}
	\randomquestionpool{3}{6}
	\randomquestionpool{7}{8}
    \randomquestionpool{9}{10}
    
    % Pool mit 1&2
	\begin{question}
		\lang{de}{ 
      		\text{Entscheiden Sie, ob die nachfolgenden Mengen offen oder abgeschlossen sind und kreuzen Sie alle korrekten Aussagen an.\\
      		 $(\var{a};\var{b})$
      		}
			}
    	\explanation[edited]{Das Intervall ist an beiden Grenzen offen, deshalb ist die Menge offen. Die Menge ist nicht abgeschlossen, 
        weil z.B. die Folge $x_n=\var{b}-\frac{1}{n}$ einen Grenzwert außerhalb der Menge hat, nämlich den Randpunkt $\var{b}$.
        }	
    	\type{mc.multiple}
    	\begin{choice}
    		\text{offen}
  			\solution{true}
		\end{choice}
		\begin{choice}
  			\text{abgeschlossen}
  			\solution{false}
		\end{choice}
		\begin{choice}
			\text{weder noch}
  			\solution{false}
		\end{choice}
    \end{question}
    
    
    \begin{question}
		\lang{de}{ 
      		\text{Entscheiden Sie, ob die nachfolgenden Mengen offen oder 
            abgeschlossen sind und kreuzen Sie alle korrekten Aussagen an.\\
      		 $ [\var{c};\var{d})$
      		}
			}
    	\explanation[edited]{Das Intervall ist nicht an beiden Grenzen offen bzw.
        abgeschlossen. Deshalb kann die Menge weder abgeschlossen noch offen sein.} 
        \type{mc.multiple}
    	\begin{choice}
    		\text{offen}
  			\solution{false}
		\end{choice}
		\begin{choice}
  			\text{abgeschlossen}
  			\solution{compute}
  			\iscorrect{c}{=}{d}
		\end{choice}
		\begin{choice}
			\text{weder noch}
  			\solution{compute}
  			\iscorrect{c}{!=}{d}
		\end{choice}
    \end{question}
    
    % Pool mit 3-6
    \begin{question}
		\lang{de}{ 
      		\text{
      		 $\mathbb{R}$
      		}
			}
    	\explanation[edited]{$\R$ ist offen, weil jede reelle Zahl eine $\epsilon$-Umgebung besitzt, die ganz in $\R$ liegt. $\R$ ist
        auch abgeschlossen, weil das Komplement (die leere Menge) offen ist.}	
    	\type{mc.multiple}
    	\begin{choice}
    		\text{offen}
  			\solution{true}
		\end{choice}
		\begin{choice}
  			\text{abgeschlossen}
  			\solution{true}
		\end{choice}
		\begin{choice}
			\text{weder noch}
  			\solution{false}
		\end{choice}
    \end{question}
    
    \begin{question}
		\lang{de}{ 
      		\text{
      		 $\mathbb{N}$
      		}
			}
    	\explanation[edited]{$\N$ ist nicht offen, 
        denn kein einziges  Element ist ein innerer Punkt. $\N$ ist abgeschlossen, 
        weil Folgen natürlicher Zahlen nur Grenzwerte in $\N$ haben können.}	
    	\type{mc.multiple}
    	\begin{choice}
    		\text{offen}
  			\solution{false}
		\end{choice}
		\begin{choice}
  			\text{abgeschlossen}
  			\solution{true}
		\end{choice}
		\begin{choice}
			\text{weder noch}
  			\solution{false}
		\end{choice}
    \end{question}
    
    \begin{question}
		\lang{de}{ 
      		\text{
      		 $\mathbb{Z}$
      		}
			}
    	\explanation[edited]{$\Z$ ist nicht offen, denn kein einziges  Element ist ein innerer Punkt. 
        Die ganzen Zahlen sind abgeschlossen, weil Folgen ganzer Zahlen nur Grenzwerte in $\Z$ haben können.}	
    	\type{mc.multiple}
    	\begin{choice}
    		\text{offen}
  			\solution{false}
		\end{choice}
		\begin{choice}
  			\text{abgeschlossen}
  			\solution{true}
		\end{choice}
		\begin{choice}
			\text{weder noch}
  			\solution{false}
		\end{choice}
    \end{question}
    
    \begin{question}
		\lang{de}{ 
      		\text{
      		 $\mathbb{Q}$
      		}
			}
    	\explanation[edited]{$\Q$ ist nicht offen, weil in jeder Umgebung einer rationalen Zahl auch reelle Zahlen liegen.
        $\Q$ ist aber auch nicht abgeschlossen, weil Folgen in $\Q$ durchaus einen 
        Grenzwert in $\R \setminus \Q $ haben können.}	
    	\type{mc.multiple}
    	\begin{choice}
    		\text{offen}
  			\solution{false}
		\end{choice}
		\begin{choice}
  			\text{abgeschlossen}
  			\solution{false}
		\end{choice}
		\begin{choice}
			\text{weder noch}
  			\solution{true}
		\end{choice}
    \end{question}
    
    % Pool mit 7-10
    \begin{question}
		\lang{de}{ 
      		\text{
      		 $\{\var{a}+\frac{1}{n} \vert n \in \N \}$
      		}
			}
    	\explanation[edited]{Betrachten wir die Folge $x_n=\var{a}+\frac{1}{n}$: Der Grenzwert $\lim_{n\to\infty}x_n=\var{a}$ ist  
        nicht in der Menge  enthalten, die Menge kann also keine abgeschlossene Menge sein. Andererseits ist die Menge auch 
        nicht offen, da sie keine inneren Punkte besitzt.}	
    	\type{mc.multiple}
    	\begin{choice}
    		\text{offen}
  			\solution{false}
		\end{choice}
		\begin{choice}
  			\text{abgeschlossen}
  			\solution{false}
		\end{choice}
		\begin{choice}
			\text{weder noch}
  			\solution{true}
		\end{choice}
    \end{question}
    
    \begin{question}
		\lang{de}{ 
      		\text{
      		 $ \R \setminus \left( \{\var{a}\} \cup \{\var{a}+\frac{\var{b}}{n}\vert n\in\N \}  \right) $
      		}
			}
        \explanation[edited]{Für eine offene Menge ist zu zeigen, dass es zu jedem Element der Menge eine 
        $\epsilon$-Umgebung gibt, die in der Menge enthalten ist. Das ist aber kein Problem, da $\var{a}$ herausgenommen ist und die Folge
        gegen $\var{a}$ konvergiert: jedes (ab jetzt feste) Element der Menge (ohne $\var{a}$) 
        hat eine positive Distanz zu $\var{a}$. Wählen wir $\epsilon=$Distanz/2, so gibt es
        höchstens endlich viele Folgenglieder in einer $\epsilon-$Umgebung des Elementes 
        (alle bis auf endlich viele liegen in der $\epsilon-$Umgebung von $\var{a}$ wegen der 
        Konvergenz). Davon hat eines eine kleinste Entfernung zum Element. Wähle eine 
        Umgebung mit Radius kleiner dieser kleinsten Entfernung - und die ist vollständig 
        in der Menge. Also ist das Element ein innerer Punkt. Die Menge ist offen.\\
        Die Menge ist nicht abgeschlossen, da z.B. die Folge $x_n=\var{a}+\frac{1}{n}$ mit $\lim_{n\to\infty}x_n=\var{a}$ 
        einen Grenzwert außerhalb der Menge hat.}	
    	\type{mc.multiple}
    	\begin{choice}
    		\text{offen}
  			\solution{true}
		\end{choice}
		\begin{choice}
  			\text{abgeschlossen}
  			\solution{false}
		\end{choice}
		\begin{choice}
			\text{weder noch}
  			\solution{false}
		\end{choice}
    \end{question}
    
    \begin{question}
		\lang{de}{ 
      		\text{
      		 $\emptyset$
      		}
			}
    	\explanation[edited]{Die leere Menge 
        ist sowohl offen als auch abgeschlossen. Da sie keine Elemente 
        enthält, ist jedes Element ein innerer Punkt; sie ist offen. Auch die Eigenschaft, dass jeder Grenzwert einer
        Zahlenfolge in der leeren Menge zur leeren Menge gehört, ist erfüllt: Es gibt ja gar keine Folgen!
        konvergiert gegen einen Grenzwert in der leeren Menge.}	
        \type{mc.multiple}
    	\begin{choice}
    		\text{offen}
  			\solution{true}
		\end{choice}
		\begin{choice}
  			\text{abgeschlossen}
  			\solution{true}
		\end{choice}
		\begin{choice}
			\text{weder noch}
  			\solution{false}
		\end{choice}
    \end{question}
    
    \begin{question}
		\lang{de}{ 
      		\text{
      		 $ \{ \var{a};\var{b};\var{c}\} $
      		}
			}
    	\explanation[edited]{Dies ist eine abgeschlossene Menge mit drei
        Elementen. Jede Zahlenfolge kann auch nur diese drei Zahlen als 
        Grenzwert haben. Innere Punkte existieren nicht.}	
    	\type{mc.multiple}
    	\begin{choice}
    		\text{offen}
  			\solution{false}
		\end{choice}
		\begin{choice}
  			\text{abgeschlossen}
  			\solution{true}
		\end{choice}
		\begin{choice}
			\text{weder noch}
  			\solution{false}
		\end{choice}
    \end{question}
    
    
	
\end{problem}

\embedmathlet{gwtmathlet}

\end{content}