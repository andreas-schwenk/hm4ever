\documentclass{mumie.problem.gwtmathlet}
%$Id$
\begin{metainfo}
  \name{
    \lang{de}{A01: Mengen und Intervalle}
    \lang{en}{mc yes-no}
  }
  \begin{description} 
 This work is licensed under the Creative Commons License Attribution 4.0 International (CC-BY 4.0)   
 https://creativecommons.org/licenses/by/4.0/legalcode 

    \lang{de}{Beschreibung}
    \lang{en}{description}
  \end{description}
  \corrector{system/problem/GenericCorrector.meta.xml}
  \begin{components}
    \component{js_lib}{system/problem/GenericMathlet.meta.xml}{gwtmathlet}
  \end{components}
  \begin{links}
  \end{links}
  \creategeneric
\end{metainfo}
\begin{content}
\usepackage{mumie.ombplus}
\usepackage{mumie.genericproblem}

\lang{de}{\title{A01: Mengen und Intervalle}}
\lang{en}{\title{Problem 1}}

\begin{block}[annotation]
	Im Ticket-System: \href{http://team.mumie.net/issues/9991}{Ticket 9991}
\end{block}


\begin{problem}

	\begin{variables}
    	\randint{a}{-30}{30}
    	\randint{b}{-30}{30}
    	\randint{c}{-30}{30}
    	\randint{d}{-30}{30}
    	\randint{e}{-30}{30}
    	\randadjustIf{d,e}{d=e}
    	\randint{q}{-30}{30}
    	\randint[Z]{t}{0}{30}
    	\end{variables}
    	
    	
    	
    	\randomquestionpool{1}{2}
    	\randomquestionpool{3}{4}
    	\randomquestionpool{5}{6}
    	\randomquestionpool{7}{9}
    	%%%a
    	%1
    \begin{question} 
     
     	\lang{de}{ 
      	\text{Kreuzen Sie an, ob die Menge ein Intervall ist.
      	\\
		}
    	}
    	\type{mc.yesno}
    	\precision[false]{3}
    	\displayprecision{3}
        \explanation[edited]{}
		\begin{choice}
			\text{ $\{ x\in \mathbb{R} \vert \:x < \var{a}  \}$}
			\solution{true}
		\end{choice}
		
		\end{question}
		
		\begin{question} 
     
     	\lang{de}{ 
      	\text{Kreuzen Sie an, ob die Menge ein Intervall ist.
      	\\ 
		}
    	}
    	\type{mc.yesno}
    	\precision[false]{3}
    	\displayprecision{3}

		\begin{choice}
			\text{ $\{ x\in \mathbb{R} \vert \:x > \var{a}  \}$}
			\solution{true}
		\end{choice}
		
		\end{question}
		
		\begin{question} 
     
     	\lang{de}{ 
      	\text{Kreuzen Sie an, ob die Menge ein Intervall ist.
      	\\ 
		}
    	}
    	\type{mc.yesno}
    	\precision[false]{3}
    	\displayprecision{3}
    	

		\begin{choice}
			\text{ $\{ x\in \mathbb{R} \vert \:x \leq \var{a}  \}$}
			\solution{true}
		\end{choice}
		
		\end{question}
		
		\begin{question} 
     
     	\lang{de}{ 
      	\text{Kreuzen Sie an, ob die Menge ein Intervall ist.
      	\\
		}
    	}
    	\type{mc.yesno}
    	\precision[false]{3}
    	\displayprecision{3}

		\begin{choice}
			\text{ $\{ x\in \mathbb{R} \vert\: x \geq \var{a}  \}$}
			\solution{true}
		\end{choice}

    	\end{question}
    	
    	  %%%b
    %5
    \begin{question} 
     
     \lang{de}{ 
      	\text{Kreuzen Sie an, ob die Menge ein Intervall ist.
		}
    	}
    	\explanation[edited]{Selbst wenn man die Zahlen in aufsteigende Reihenfolge bringt Hier gibt es zwischen den Zahlen Lücken.}
    	\type{mc.yesno}
    	\precision[false]{3}
    	\displayprecision{3}

		\begin{choice}
			\text{$\{ \var{b}, \var{c}, \var{d}, \var{e} \}$}
			\solution{false}
            \end{choice}
    \end{question}
    
        
   	\begin{question} 
     
     \lang{de}{ 
      	\text{ Kreuzen Sie an, ob die Menge ein Intervall ist.
		}
    	}
    	\explanation[edited]{Dies ist das sehr kleine Intervall $[\var{b1};\var{b1}]$.}
    	\type{mc.yesno}
    	\precision[false]{3}
    	\displayprecision{3}

    	\begin{variables}
    	\randint{b1}{-30}{30}
    	\end{variables}

		\begin{choice}
			\text{$\{ \var{b1} \}$}
			\solution{true}   
		\end{choice}
    \end{question}
    
     %%%c
    %7
    \begin{question} 
     
    	\lang{de}{ 
      	\text{ Kreuzen Sie an, ob die Menge ein Intervall ist.
		}
    	}
    	\type{mc.yesno}
    	\precision[false]{3}
    	\displayprecision{3}

		\begin{choice}
			\text{$\{ x\in \mathbb{R} \vert \:x^3 > \var{t} \}$}
			\solution{true}
		\end{choice}
    \end{question}
    %8
	   
    \begin{question} 
     \lang{de}{ 
      	\text{ Kreuzen Sie an, ob die Menge ein Intervall ist.
		}
    	}
    	\explanation[edited]{Die Lösungsmenge besteht aus zwei disjunkten Teilmengen:
            $\{x\vert\:x\geq\var{t}\}$ und $\{x\vert\:\leq\var{t}\}$. Deshalb ist die Lösungsmenge 
            kein Intervall. }
    	\type{mc.yesno}
    	\precision[false]{3}
    	\displayprecision{3}

		\begin{choice}
			\text{$\{ x\in \mathbb{R} \vert \:|x| \geq \var{t} \}$}
			\solution{compute}
			\iscorrect{t}{=}{0}
            \end{choice}
    \end{question}
   %9
   
  
         \begin{question} 
     \lang{de}{ 
      	\text{ Kreuzen Sie an, ob die Menge ein Intervall ist.
		}
    	}
    	
    	\type{mc.yesno}
    	\precision[false]{3}
    	\displayprecision{3}
 
		\begin{choice}
			\text{$\{ x\in \mathbb{R} \vert \:|x| \leq \var{t} \}$}
			\solution{true}
		\end{choice}
    \end{question}
    
    %%%d
   %10 
     \begin{question} 
     \lang{de}{ 
      	\text{ Kreuzen Sie an, ob die Menge ein Intervall ist.
		}
    	}
    	
    	\type{mc.yesno}
    	\precision[false]{3}
    	\displayprecision{3}

		\begin{choice}
			\text{$\{ x\in \mathbb{R} \vert \:x > \var{q} \}$}
			\solution{true}
		\end{choice}
    \end{question}
  %11  
    \begin{question} 
     \lang{de}{ 
      	\text{ Kreuzen Sie an, ob die Menge ein Intervall ist.
		}
    	}
    	
    	\type{mc.yesno}
    	\precision[false]{3}
    	\displayprecision{3}

		\begin{choice}
			\text{$\{ x\in \mathbb{R} \vert\: x \leq \var{q} \}$}
			\solution{true}
		\end{choice}
    \end{question}
    %12
    \begin{question} 
     \lang{de}{ 
      	\text{ Kreuzen Sie an, ob die Menge ein Intervall ist.
		}
    	}
    	
    	\type{mc.yesno}
    	\precision[false]{3}
    	\displayprecision{3}

		\begin{choice}
			\text{$\{ x\in \mathbb{R} \vert \:x \geq \var{q} \}$}
			\solution{true}
		\end{choice}
    \end{question}
	
\end{problem}

\embedmathlet{gwtmathlet}



\end{content}