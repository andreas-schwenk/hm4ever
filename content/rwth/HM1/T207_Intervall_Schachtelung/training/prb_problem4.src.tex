\documentclass{mumie.problem.gwtmathlet}
%$Id$
\begin{metainfo}
  \name{
    \lang{de}{A04: Intervallhalbierungsmethode}
    \lang{en}{input numbers}
  }
  \begin{description} 
 This work is licensed under the Creative Commons License Attribution 4.0 International (CC-BY 4.0)   
 https://creativecommons.org/licenses/by/4.0/legalcode 

    \lang{de}{Die Beschreibung}
    \lang{en}{}
  \end{description}
  \corrector{system/problem/GenericCorrector.meta.xml}
  \begin{components}
    \component{js_lib}{system/problem/GenericMathlet.meta.xml}{gwtmathlet}
  \end{components}
  \begin{links}
  \end{links}
  \creategeneric
\end{metainfo}
\begin{content}
\usepackage{mumie.ombplus}
\usepackage{mumie.genericproblem}

\lang{de}{\title{A04: Intervallhalbierungsmethode}}
\lang{en}{\title{Problem 4}}

\begin{block}[annotation]
	Im Ticket-System: \href{http://team.mumie.net/issues/9995}{Ticket 9995}
\end{block}

\begin{problem}
%	\randomquestionpool{1}{2}
	\begin{question}%n=2
		
		\text{Gegeben sei die Funktion \\
		$ f:\mathbb{R}\to \mathbb{R}$, $\:f(x) = \var{f}. $\\
		Führen Sie die Intervallhalbierungsmethode zur Bestimmung einer Nullstelle von $f$ beginnend mit dem 
        Intervall $[0; \var{b}]$ durch und geben Sie die
        ersten drei erhaltenen Intervalle an: \\
        Intervall nach erster Halbierung: $\; [$ \ansref ; \ansref $]$\\
        Intervall nach zweiter Halbierung: $\; [$ \ansref ; \ansref $]$\\
        Intervall nach dritter Halbierung: $\; [$ \ansref ; \ansref $]$
		}
		
		\type{input.number}
        \field{rational}

        \explanation{Die neuen Intervalle (obere Hälfte oder untere Hälfte) sind stets so zu wählen, dass die 
        Vorzeichen der zugehörigen Funktionswerte  verschieden sind.}
		
		
		\begin{variables}
			\randint{a}{2}{7}
			\randint[Z]{b1}{1}{3}
            \function[calculate]{b}{4*b1}
            \function[calculate]{b2}{2*b1}
            \randint{k}{0}{7}
			% c wird so gewählt, dass f eine Nullstelle zwischen b/8*(8-k-1) uund b/8(8-k) hat.
            % außerdem ist f im Intervall [0;b] monoton fallend.
            \function[calculate]{c}{b1^2*(a*k^2+2*k+1)/4}
			\function[expand,normalize]{f}{a*(x-b)^2-c} 
            \function[calculate]{l1}{floor(k/4)}
            \function[calculate]{l2}{floor(k/2)}
            \function[calculate]{l3}{k}
            
			\function[calculate]{u1}{b-b/2*(l1+1)}
			\function[calculate]{o1}{b-b/2*l1}
			\function[calculate]{u2}{b-b/4*(l2+1)}
			\function[calculate]{o2}{b-b/4*l2}
			\function[calculate]{u3}{b-b/8*(l3+1)}
			\function[calculate]{o3}{b-b/8*l3}
            
            \number{null}{0}
            \substitute[calculate]{f0}{f}{x}{null}
		\end{variables}
		
		\begin{answer}
			\solution{u1}
		\end{answer}
		\begin{answer}
			\solution{o1}
		\end{answer}
		\begin{answer}
			\solution{u2}
		\end{answer}
		\begin{answer}
			\solution{o2}
		\end{answer}
		\begin{answer}
			\solution{u3}
		\end{answer}
		\begin{answer}
			\solution{o3}
		\end{answer}
		
	\end{question}
	
%	\begin{question}%n=3
%		\lang{de}{
%		\text{Gegeben sei die Funktion \\
%		$ f:\mathbb{R}\to \mathbb{R} , f(x) = \var{a3}(\var{auf})^\var{n} - \var{c3}. $\\
%		Führen Sie die Intervallhalbierungsmethode so lange durch, bis sie eine reelle Nullstelle von 
%		$f$ bis auf ein Intervall der Länge $< \frac{1}{4}$ bestimmt haben. 
%		}
%		}
%		
%		\type{input.function}
%		\field{real}
%
%   
%    \displayprecision{2}
%    \correctorprecision{2}
%		
%		
%		\begin{variables}
%			\randint{a}{2}{4}
%			\randint[Z]{b}{-5}{5}
%			\randint{c}{1}{3}
%			\function[normalize]{a3}{a^3}
%			\function[normalize]{c3}{c^3}
%			\number{n}{3}
%			\function[calculate]{g}{1/100*floor(100*(b+c/a))}
%			
%			\function[normalize]{auf}{x-b}
%		\end{variables}
%		
%		\begin{answer}
%			\text{Geben Sie die linke Intervallgrenze an.}
%			\solution{g}
%			\inputAsFunction{x}{ag} 
%           \checkFuncForZero{ag-g}{-10}{10}{100}
%		\end{answer}
%		
%	\end{question}
	
\end{problem}

\embedmathlet{gwtmathlet}

\end{content}