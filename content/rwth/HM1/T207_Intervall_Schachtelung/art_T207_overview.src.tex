%$Id:  $
\documentclass{mumie.article}
%$Id$
\begin{metainfo}
  \name{
    \lang{de}{Überblick: Intervall-Schachtelung}
    \lang{en}{overview: }
  }
  \begin{description} 
 This work is licensed under the Creative Commons License Attribution 4.0 International (CC-BY 4.0)   
 https://creativecommons.org/licenses/by/4.0/legalcode 

    \lang{de}{Beschreibung}
    \lang{en}{}
  \end{description}
  \begin{components}
  \end{components}
  \begin{links}
\link{generic_article}{content/rwth/HM1/T207_Intervall_Schachtelung/g_art_content_23_intervallschachtelung.meta.xml}{content_23_intervallschachtelung}
\link{generic_article}{content/rwth/HM1/T207_Intervall_Schachtelung/g_art_content_22_offene_abgeschlossene_teilmengen.meta.xml}{content_22_offene_abgeschlossene_teilmengen}
\link{generic_article}{content/rwth/HM1/T207_Intervall_Schachtelung/g_art_content_21_intervalle.meta.xml}{content_21_intervalle}
\end{links}
  \creategeneric
\end{metainfo}
\begin{content}
\begin{block}[annotation]
	Im Ticket-System: \href{https://team.mumie.net/issues/30130}{Ticket 30130}
\end{block}



\begin{block}[annotation]
Im Entstehen: Überblicksseite für Kapitel Intervall-Schachtelung
\end{block}

\usepackage{mumie.ombplus}
\ombchapter{1}
\lang{de}{\title{Überblick: Intervall-Schachtelung}}
\lang{en}{\title{}}



\begin{block}[info-box]
\lang{de}{\strong{Inhalt}}
\lang{en}{\strong{Contents}}


\lang{de}{
    \begin{enumerate}%[arabic chapter-overview]
   \item[7.1] \link{content_21_intervalle}{Intervalle}
   \item[7.2] \link{content_22_offene_abgeschlossene_teilmengen}{Offene und abgeschlossene Teilmengen}
   \item[7.3] \link{content_23_intervallschachtelung}{Intervall-Schachtelung}
   \end{enumerate}
} %lang

\end{block}

\begin{zusammenfassung}

\lang{de}{Zunächst betrachten wir erneut Intervalle in den reellen Zahlen. 
Wir führen den Begriff der $\epsilon$-Umgebung ein,  definieren offene und abgeschlossene Teilmengen und erschließen deren Eigenschaften.
Auch Randpunkte und innere Punkte werden definiert.

Anschließend geben wir mit dem Prinzip der Intervallschachtelung (einer Anwendung des Sandwich-Lemmas) eine effektive und praktische Methode an, irrationale Zahlen zu konstruieren bzw. anzunähern.

}


\end{zusammenfassung}

\begin{block}[info]\lang{de}{\strong{Lernziele}}
\lang{en}{\strong{Learning Goals}} 
\begin{itemize}[square]
\item \lang{de}{Sie kennen reelle Intervalle und $\epsilon$-Umgebungen und gebräuchliche Notationen für diese.}
\item \lang{de}{Sie kennen die Begriffe offene und abgeschlossene Mengen und geben Beispiele dafür.}
\item \lang{de}{Sie untersuchen, ob gegebene Mengen offen, abgeschlossen oder nichts von beidem sind.}
\item \lang{de}{Sie bestimmen Randpunkte gegebener Mengen.}
\item \lang{de}{Sie beherrschen das Intervallschachtelungsverfahren und nähern damit Wurzeln reeller Zahlen an.}
\end{itemize}
\end{block}




\end{content}
