%$Id:  $
\documentclass{mumie.article}
%$Id$
\begin{metainfo}
  \name{
    \lang{de}{Offene und abgeschlossene Teilmengen}
    \lang{en}{}
  }
  \begin{description} 
 This work is licensed under the Creative Commons License Attribution 4.0 International (CC-BY 4.0)   
 https://creativecommons.org/licenses/by/4.0/legalcode 

    \lang{de}{Beschreibung}
    \lang{en}{}
  \end{description}
  \begin{components}
    \component{generic_image}{content/rwth/HM1/images/g_img_00_video_button_schwarz-blau.meta.xml}{00_video_button_schwarz-blau}
    
  \end{components}
  \begin{links}
    \link{generic_article}{content/rwth/HM1/T207_Intervall_Schachtelung/g_art_content_21_intervalle.meta.xml}{intervalle}
    \link{generic_article}{content/rwth/HM1/T205_Konvergenz_von_Folgen/g_art_content_14_konvergenz.meta.xml}{konvergenz}
    \link{generic_article}{content/rwth/HM1/T205_Konvergenz_von_Folgen/g_art_content_15_monotone_konvergenz.meta.xml}{monot-konv}
  \end{links}
  \creategeneric
\end{metainfo}
\begin{content}
\usepackage{mumie.ombplus}
\ombchapter{7}
\ombarticle{2}

\lang{de}{\title{Offene und abgeschlossene Teilmengen}}
 
\begin{block}[annotation]
  
  
\end{block}
\begin{block}[annotation]
  Im Ticket-System: \href{http://team.mumie.net/issues/9679}{Ticket 9679}\\
\end{block}

\begin{block}[info-box]
\tableofcontents
\end{block}


\section{Innere Punkte und Randpunkte}\label{sec:innere-randpunkte}

\begin{definition}
Sei $D$ eine Teilmenge der reellen Zahlen $\R$. Eine Zahl $c\in D$ heißt genau dann \notion{innerer Punkt} von $D$, wenn
es eine $\epsilon$-Umgebung von $c$ gibt, die in $D$ enthalten ist, d.h. es gibt $\epsilon>0$ mit
$U_\epsilon(c)\subseteq D$.

Eine Zahl $c\in \R$ heißt genau dann \notion{Randpunkt} von $D$, 
 wenn jede $\epsilon$-Umgebung von $c$ sowohl mit $D$ als auch mit dem Komplement von $D$ nicht-leeren Schnitt hat,
wenn also gilt: Zu jedem 
 $\epsilon>0$ gibt es (mindestens) ein $x_\epsilon\in D$  mit $|x_\epsilon-c|<\epsilon$ und (mindestens) ein
 $y_\epsilon\in \R\setminus D$ mit $|y_\epsilon-c|<\epsilon$ (d.h. mit
$x_\epsilon, y_\epsilon\in U_\epsilon(c)$).
\end{definition}

\begin{example}
Am Beispiel der \link{intervalle}{Intervalle} werden die Begriffe anschaulich:\\
Die Intervallgrenzen sind die Randpunkte des Intervalls.
Für ein abgeschlossenes Intervall $[a;b]$ mit $a<b$ gehören die Randpunkte $a$ und $b$ zum Intervall dazu,
während alle anderen Elemente von $[a;b]$ innere Punkte sind.

Für $c=a$ und für jedes $\epsilon>0$ ist die Zahl $x_\epsilon=a-\frac{\epsilon}{2}\in U_\epsilon(a)$ nicht in $[a;b]$ enthalten. 
Daher ist $a$ ein Randpunkt des Intervalls. Ebenso ist für jedes $\epsilon>0$ die Zahl $x_\epsilon=b+\frac{\epsilon}{2}\in U_\epsilon(b)$ nicht in $[a;b]$ enthalten, weshalb $b$ ein Randpunkt ist.

Für $c\in [a;b]$ mit $a<c<b$ kann man jedoch $\epsilon>0$ wählen als 
\[ \epsilon =\min\{ c-a; b-c \}. \]
dann gilt $a\leq c-\epsilon$ und $b\geq c+\epsilon$ und daher ist für jedes $x\in U_\epsilon(c)$:
\[  a\leq c-\epsilon <x <c+\epsilon\leq b. \]
Jedes $x\in U_\epsilon(c)$ liegt also im Intervall $[a;b]$ (sogar im offenen Intervall $(a; b)$), 
weshalb $U_\epsilon(c)$ in $[a;b]$ enthalten ist.


Die Überlegung für $c\in [a;b]$ mit $a<c<b$, d.h. für $c\in (a; b)$ zeigt auch, dass jedes Element des offenen
Intervalls $(a; b)$ ein innerer Punkt ist. Die Randpunkte $a$ und $b$ des offenen Intervalls gehören nicht zum Intervall dazu.
\end{example}

\begin{tabs*}
\tab{Hinweis zur Notation}
In der Literatur wird oft auch der Rand $\partial D$ einer Menge $D$ definiert als Menge aller Randpunkte.
Dieser ist wie gesehen nicht notwendigerweise eine Teilmenge von $D$. 
\end{tabs*}

\section{Offene Mengen}\label{sec:offene_mengen}

Die offenen Intervalle sind ein Beispiel für die allgemeineren offenen Mengen.

\begin{definition}
Eine Teilmenge $D$ der reellen Zahlen nennt man genau dann \notion{offen}, wenn  jeder Punkt von $D$ ein innerer Punkt ist.
\end{definition}

\begin{example}
\begin{enumerate}
\item Wie oben beschrieben, sind offene Intervalle $(a; b)$ für $a<b$ offene Mengen.
\item Auch die unendlichen Intervalle $(a; \infty)$, $(-\infty; b)$ und $(-\infty; \infty)=\R$ sind offen. 
Für jede reelle Zahl des Intervalls existiert  eine weitere Zahl in deren Umgebung, da keine Randpunkte zum Intervall gehören.
\item Alle anderen Intervalle sind nicht offen, da sie mindestens einen Randpunkt enthalten. Als Beispiel seien die halboffenen
Intervalle wie $[a;b)$ oder $(a;b]$ genannt.
\item Die leere Menge $\emptyset$ ist offen, da sie keine Elemente enthält. Die Bedingung, dass jeder Punkt ein innerer Punkt ist, ist also erfüllt.
\item Die Mengen $\N$, $\Z$ und $\Q$ sind alle nicht offen, weil sie gar keine inneren Punkte besitzen. 
\end{enumerate}
\end{example}

\begin{theorem}\label{thm:eigenschaften_off_mengen}
\begin{enumerate}
\item Die leere Menge $\emptyset$ und $\R$ sind offen.
\item Sind $D_1,\ldots, D_n\subseteq \R$ endlich viele offene Mengen, so ist auch deren Durchschnitt
$D_1\cap D_2\cap \ldots \cap D_n$ eine offene Menge.
\item Für beliebig viele (auch unendlich viele) offene Mengen ist deren Vereinigung wieder eine offene Menge.
\end{enumerate}
\end{theorem}

\begin{block}[warning]
Man kann in 2. nicht auf die Endlichkeit verzichten: So sind zum Beispiel für jedes $n\in \N$ die Intervalle
$(-\frac{1}{n}; \frac{1}{n})$ offen, ihr Durchschnitt ist jedoch
\[  \bigcap_{n\in \N} (-\frac{1}{n}; \frac{1}{n}) 
%=\{ x\in \R \mid x\in (-\frac{1}{n}; \frac{1}{n})\text{ für alle }n\in \N \} 
= \{ x\in \R \mid  {|x|}< \frac{1}{n}\text{ für alle }n\in \N \}
%=\{ x\in \R \mid  {|x|}=0 \} 
=\{0\}. \]
Die Menge $\{0\}$ ist aber nicht offen, denn jede $\epsilon$-Umgebung von $0$ enthält noch Zahlen außer der $0$.
\end{block}

\begin{proof*}{Beweis}
\begin{incremental}{0}
\step
Die leere Menge und $\R$ waren in den Beispielen schon erklärt. 
\step
Um zu sehen, dass für offene Mengen $D_1,\ldots, D_n$ auch deren Durchschnitt $D_1\cap D_2\cap \ldots \cap D_n$  eine offene Menge ist, betrachten wir
einen beliebigen Punkt $c\in D_1\cap D_2\cap \ldots \cap D_n$ und zeigen, dass er ein innerer Punkt ist. 
Da alle $D_j$ offen sind, gibt es $\epsilon_1,\ldots, \epsilon_n>0$ mit $U_{\epsilon_1}(c)\subseteq D_1$, $U_{\epsilon_2}(c)\subseteq D_2$ etc.

Damit ist aber $U_{\epsilon_1}(c)\cap \ldots\cap U_{\epsilon_n}(c)\subseteq D_1\cap D_2\cap \ldots \cap D_n$ und
$U_{\epsilon_1}(c)\cap \ldots\cap U_{\epsilon_n}(c)$ ist nichts anderes als $U_{\epsilon}(c)$ mit
$\epsilon=\min\{ \epsilon_1,\ldots, \epsilon_n\} >0$. \\ Also enthält $D_1\cap D_2\cap \ldots \cap D_n$ eine
$\epsilon$-Umgebung von $c$, weshalb $c$ innerer Punkt des Durchschnitts ist.
\step 
Ist $c$ ein Punkt in der Vereinigung, dann liegt er nach Definition der Vereinigung in einer der offenen Mengen. Dann gibt es aber eine $\epsilon$-Umgebung von $c$, die Teilmenge dieser offenen Menge ist und daher auch Teilmenge der Vereinigung der Mengen. Also ist $c$ ein innerer Punkt der Vereinigung.
\end{incremental}
\end{proof*}

Das folgende Video erinnert nochmal an Intervalle und behandelt die oben genannten Themen:
\floatright{\href{https://api.stream24.net/vod/getVideo.php?id=10962-2-10817&mode=iframe&speed=true}{\image[75]{00_video_button_schwarz-blau}}}\\

\section{Abgeschlossene Mengen}\label{sec:abgeschl-mengen}

\begin{definition}
Eine Teilmenge $M\subseteq \R$ heißt genau dann \notion{abgeschlossen}, wenn für jede konvergente Folge
$(x_n)_{n\in \N}$ mit Werten in $M$ (d.h. mit $x_n\in M$ für alle $n\in \N$) auch ihr Grenzwert
$\lim_{n\to \infty} x_n$ in $M$ liegt.
\end{definition}

\begin{example}
Abgeschlossene Mengen:
\begin{tabs*}[\initialtab{0}]
\tab{abgeschlossene Intervalle $[a;b]$} 
Jedes endliche abgeschlossene Intervall $[a;b]$ ist abgeschlossen in diesem Sinne.
Ist $(x_n)_{n\in \N}$ eine konvergente Folge mit Werten in $[a;b]$, d.h. mit $a\leq x_n\leq b$ für alle
 $n\in \N$,  dann gilt
nach den \ref[konvergenz][Regeln zu Grenzwerten]{rule:grenzwerte-vergleichen} auch
\[  a= \lim_{n\to \infty}  a \leq \lim_{n\to \infty}  x_n \leq  \lim_{n\to \infty} b =b \]
mit den konstanten Folgen $(a)_{n\in \N}$ und $(b)_{n\in \N}$.
\tab{Intervalle $[a;\infty)$ und $(-\infty; b]$}
Auch die unendlichen Intervalle $[a;\infty)$, $(-\infty; b]$ sind abgeschlossen, wie man genauso wie für endliche
zeigt.
% Jede konvergente Folge ist ja beschränkt, weshalb sie Werte in einem 
%abgeschlossenen Intervall $[-c;c]$ hat mit $c\in \R$. 
%Eine konvergente Folge mit Werten in $[a;\infty)$ bzw. $(-\infty; b]$ hat daher insbesondere Werte in einem abgeschlossenen Intervall $[a;\infty)\cap [-c;c]$ bzw. $(-\infty; b]\cap [-c;c]$, weshalb auch ihr Grenzwert in diesem Intervall liegt. 
\tab{$\N$ und $\Z$} Die Menge der natürlichen Zahlen $\N$ und die Menge der ganzen Zahlen $\Z$ sind abgeschlossen.
Ist  $(x_n)_{n\in \N}$ eine konvergente Folge mit Werten in $\N$ oder $\Z$ und Grenzwert $a\in \R$, so gibt es
auch für $\epsilon=\frac{1}{2}$ ein $N_\epsilon\in \N$ mit $|x_n-a|<\frac{1}{2}$ für alle $n\geq N_\epsilon$.\\
Damit ist aber für alle $n,m\geq N_\epsilon$
\[ |x_n-x_m|=|x_n-a+a-x_m|\leq |x_n-a|+|a-x_m|<\frac{1}{2}+\frac{1}{2}=1.\]
Da aber alle $x_n$ ganze Zahlen sind, kann deren Abstand zueinander nur $<1$ sein, wenn sie gleich sind. Also gilt
$x_n=x_m=x_{N_\epsilon}$ für alle $n,m\geq N_\epsilon$, und auch $a=x_{N_\epsilon}$. D.h. $a$ liegt in den ganzen bzw. natürlichen Zahlen.
\tab{endliche Teilmengen}
Jede endliche Teilmenge von $\R$ ist abgeschlossen, wie man ähnlich zum Fall der ganzen Zahlen sieht. Man muss lediglich ein $\epsilon$ wählen, das halb so groß wie der kleinste auftretende Abstand ist.
\end{tabs*}
Teilmengen, die nicht abgeschlossen sind:
\begin{tabs*}[\initialtab{0}]
\tab{offene und halboffene Intervalle}
Intervalle, bei denen die "`Ränder"' nicht zum Intervall gehören (also z.B. $(a;b)$ oder $(a;b]$ etc.), 
sind keine abgeschlossenen Mengen. Ist nämlich der "`untere Rand"' $a$ nicht im Intervall, so kann man ein beliebiges 
$a<c<b$ im Intervall wählen und die Folge $(x_n)_{n\in \N}$ mit 
\[ x_n= a+\frac{c-a}{n} \quad \text{für alle }n\in \N.\]
Dies ist eine Folge mit Werten im Intervall, da insbesondere $(a;c]$ Teilmenge des Intervalls ist und 
sie gegen $a$ konvergiert, da
\[  \lim_{n\to \infty} x_n =\lim_{n\to \infty}  a+\frac{c-a}{n} =a+(c-a)\cdot \lim_{n\to \infty}\frac{1}{n}=a.\]
\tab{$\Q$} 
Die Menge der rationalen Zahlen $\Q$ ist nicht abgeschlossen. Im Abschnitt \link{monot-konv}{Monotone Konvergenz} hatten wir zum Beispiel eine Folge mit Werten in $\Q$, die gegen die irrationale Zahl $\sqrt{2}$ konvergiert.
\end{tabs*}
\end{example}

Obwohl die Definitionen von offenen und abgeschlossenen Mengen sehr unterschiedlich sind, hängen sie doch zusammen.

\begin{theorem}
Eine Teilmenge $D$ der reellen Zahlen ist genau dann offen, wenn ihr Komplement $\complement{D}=\R\setminus D$ 
abgeschlossen ist. 

Eine Teilmenge $M$ der reellen Zahlen ist genau dann abgeschlossen, wenn ihr Komplement 
$\complement{M}=\R\setminus M$ offen ist. 
\end{theorem}


\begin{proof*}[Beweis]
Da das Komplement des Komplements wieder die ursprüngliche Menge selbst ist, sind die beiden Aussagen gleichbedeutend. 
Wir zeigen also nur die erste. 
\begin{incremental}{0}
\step
Sei nun $D$ eine offene Teilmenge und $(a_n) _{n\in\N}$ eine konvergente Folge mit Werten in $\R\setminus D$ und 
Grenzwert $a \in \R$. Wäre $a\in D$, so gäbe es eine Zahl $\epsilon>0$ mit $U_\epsilon(a) \subseteq D$,da $D$ offen ist. Zu diesem $\epsilon$ gibt es jedoch ein $N_\epsilon\in\N$ mit $|a_n-a|<\epsilon$, d. h.  mit $a_n\in U_\epsilon(a)$, für alle $n\geq N_\epsilon$ im Widerspruch zu $a_n\notin D$. Also ist $a\in 
\R\setminus D$, wie zu zeigen war. 
\step
Ist andererseits $D$ keine offene Menge,  dann gibt es einen Randpunkt $a\in D$. D.h. für jedes $\epsilon>0$ ist 
$U_\epsilon(a) $ keine Teilmenge von $D$. Insbesondere gibt es für jedes $n\in\N$ und $\epsilon=\frac{1}{n}$ eine 
Zahl $a_n\in U_{\frac{1}{n}}$ mit $a_n\in \R\setminus D$.\\
Dies bedeutet aber nichts anderes als dass die so erhaltene Folge $(a_n) _{n\in\N}$ eine konvergente Folge mit Werten 
in $\R\setminus D$ und Grenzwert $a \in D$ ist. Also ist $\R\setminus D$ nicht abgeschlossen. 
\end{incremental}
\end{proof*}

Mit Hilfe der de Morganschen Regeln $\complement{(C\cup D)}=\complement{C}\cap \complement{D}$ und
 $\complement{(C\cap D)}=\complement{C}\cup \complement{D}$, welche auch für unendliche Vereinigungen und unendliche 
 Durchschnitte gelten, erhält man nun aus dem Satz über Durchschnitte und Vereinigungen offener Mengen
 einen ähnlichen Satz zu abgeschlossenen Mengen.

\begin{theorem}
\begin{enumerate}
\item Die leere Menge und $\R$ sind abgeschlossen.
\item Sind $D_1,\ldots, D_n\subseteq \R$ endlich viele abgeschlossene Mengen, so ist auch deren Vereinigung
$D_1\cup D_2\cup \ldots \cup D_n$ eine abgeschlossene Menge.
\item Für beliebig viele (auch unendlich viele) abgeschlossene Mengen ist deren Durchschnitt wieder eine 
abgeschlossene Menge.
\end{enumerate}
\end{theorem}

\begin{remark}
\begin{enumerate}
\item Auch hier kann bei 2. nicht auf die Endlichkeit verzichtet werden. Die Vereinigung der unendlich vielen abgeschlossenen Intervalle
$[\frac{1}{n},1]$ für $n\in \N$ ist das linksoffene Intervall $(0,1]$. Es ist nicht abgeschlossen, da 
zum Beispiel die Folge $(\frac{1}{n})_{n\in \N}$ Werte in $(0,1]$ hat, aber den Grenzwert $0$, welcher nicht in $(0,1]$ liegt.
\item Die leere Menge und ganz $\R$ sind also beide abgeschlossen und offen. Es lässt sich zeigen, dass dies die 
einzigen
beiden Teilmengen von $\R$ sind, welche sowohl abgeschlossen als auch offen sind.
\item Es gibt sehr viele Mengen, die weder abgeschlossen noch offen sind, beispielsweise halboffene Intervalle oder die 
Menge der rationalen Zahlen $\Q$.
\end{enumerate}
\end{remark}

Das Thema Abgeschlossenheit wird in folgemdem Video beleuchtet und gibt einen Ausblick auf die 
komplexe Erweiterung:
\floatright{\href{https://api.stream24.net/vod/getVideo.php?id=10962-2-10818&mode=iframe&speed=true}{\image[75]{00_video_button_schwarz-blau}}}\\

\begin{quickcheck}
    \lang{de}{\text{Markieren Sie alle richtigen Antworten:}}
    \begin{choices}{multiple}
        \begin{choice}
            \text{$(3;4)$ ist eine nicht abgeschlossene Menge}
            \solution{true}
        \end{choice}
        
        \begin{choice}
            \text{$\emptyset$ ist eine offene Menge}
            \solution{true}
        \end{choice}
        
        \begin{choice}
            \text{$[-1;7)$ ist eine abgeschlossene Menge}
            \solution{false}
        \end{choice}
        
        \begin{choice}
            \text{$(-\infty;3]$ ist eine abgeschlossene Menge}
            \solution{true}
        \end{choice}
        \begin{choice}
            \text{$\R$ ist eine nicht abgeschlossene Menge}
            \solution{false}
        \end{choice}
    \end{choices}
    \explanation{siehe Bemerkung 2.10 und:\\
    Bedenken Sie: $\R$ enthält alle Grenzwerte und ist damit abgeschlossen,\\
    sowie\\
    $\emptyset$ ist offen, da es keine Randpunkte enthält.}
\end{quickcheck}

\end{content}