%$Id:  $
\documentclass{mumie.article}
%$Id$
\begin{metainfo}
  \name{
    \lang{de}{Intervalle}
    \lang{en}{}
  }
  \begin{description} 
 This work is licensed under the Creative Commons License Attribution 4.0 International (CC-BY 4.0)   
 https://creativecommons.org/licenses/by/4.0/legalcode 

    \lang{de}{Beschreibung}
    \lang{en}{}
  \end{description}
  \begin{components}
    \component{generic_image}{content/rwth/HM1/images/g_img_00_video_button_schwarz-blau.meta.xml}{00_video_button_schwarz-blau}
    \component{generic_image}{content/rwth/HM1/images/g_tkz_T207_Interval_C.meta.xml}{T207_Interval_C}
    \component{generic_image}{content/rwth/HM1/images/g_tkz_T207_Interval_F.meta.xml}{T207_Interval_F}
    \component{generic_image}{content/rwth/HM1/images/g_tkz_T207_Interval_E.meta.xml}{T207_Interval_E}
    \component{generic_image}{content/rwth/HM1/images/g_tkz_T207_Interval_D.meta.xml}{T207_Interval_D}
    \component{generic_image}{content/rwth/HM1/images/g_tkz_T207_Interval_B.meta.xml}{T207_Interval_B}
    \component{generic_image}{content/rwth/HM1/images/g_tkz_T207_Interval_A.meta.xml}{T207_Interval_A}
  \end{components}
  \begin{links}
  \end{links}
  \creategeneric
\end{metainfo}
\begin{content}
\usepackage{mumie.ombplus}
\ombchapter{7}
\ombarticle{1}

\lang{de}{\title{Intervalle}}
 
\begin{block}[annotation]
  
  
\end{block}
\begin{block}[annotation]
  Im Ticket-System: \href{http://team.mumie.net/issues/9678}{Ticket 9678}\\
\end{block}

\begin{block}[info-box]
\tableofcontents
\end{block}


\section{\lang{de}{Endliche Intervalle}\lang{en}{Finite intervals}}\label{section.intervals}

\lang{de}{\emph{Intervalle} sind spezielle Teilmengen der reellen Zahlen,
die im Folgenden erläutert werden. Für die Visualisierung der Intervalle gibt es verschiedene Möglichkeiten.
Die unten Angegebene ist nur eine davon.

%Zur \ref[link_arith][Erinnerung]{kleiner_groesser}: Eine reelle Zahl $a$ ist kleiner als $b$, $\,(\, a<b\,)$, wenn $a$
%auf der Zahlengeraden links von $b$ liegt. F\"ur zwei reelle Zahlen $a$ und $b$ gilt stets genau eine der drei Beziehungen
%$a<b$, $\,a=b$, oder $b<a$. \label{open_interval}

Das \notion{offene Intervall} von $a$ bis $b$, wobei $a<b$, ist die Menge aller reellen Zahlen zwischen $a$ und $b$.
Sie wird mit $(a;b)$ bezeichnet:
\[(a;b)=\{\,x\,|\,a< x< b\,\}.\]
}
\lang{en}{Special subsets or \notion{intervals} of the real numbers use a different notation that will be explained in the following.
There are also many different ways of visualising such intervals.
The examples below 
% and those in \link{link_ineq_graph_sol}{\lang{de}{Kapitel III}\lang{en}{Chapter III}} use only two such methods.
use only one such method.

%\ref[link_arith][\lang{de}{Erinnerung}\lang{en}{Recall}]{kleiner_groesser}: A real number $a$ is less than $b$ $\,(\, a<b\,)$ if $a$
%lies to the left of $b$ on the number line. Given two real numbers $a$ and $b$, exactly one of the following three relationships holds:
%$a<b$, $\,a=b$, or $b<a$. 

The \notion{open interval} from $a$ to $b$, with $a<b$, is the set of all real numbers between $a$ and $b$ and is denoted by $(a,b)$:
\[(a,b)=\{\,x\,|\,a< x< b\,\}.\]
}

\begin{center}
\image{T207_Interval_A}
\end{center}

\begin{example}

\[(-2\lang{de}{;}\lang{en}{,}0)=\{\,x\,|\,-2< x<0\,\}\]
\begin{center}
\image{T207_Interval_B}
\end{center}
\end{example}
\lang{de}{Da Intervalle immer Teilmengen der reellen Zahlen sind, wird anstelle von $\{\,x\in\R\mid a< x< b\,\}$ kürzer 
$\{\,x\mid a< x< b\,\}$ geschrieben.}
\lang{en}{Because intervals are always subsets of the real numbers, the interval $\{\,x\in\R\mid a< x< b\,\}$ can instead be written  
$\{\,x\mid a< x< b\,\}$.}\\\\



\lang{de}{Das \notion{abgeschlossene Intervall} von $a$ bis $b$ ist die Menge aller
    reellen Zahlen zwischen $a$ und $b$ einschließlich der Randpunkte $a$ und $b$,
    also 
\[[a;b]=\{\,x\,|\,a\leq x\leq b\,\}.\]
}
\lang{en}{The \notion{closed interval} from $a$ to $b$ is the set of all real numbers between $a$ and $b$ including the boundary values $a$ und $b$,
    hence 
\[[a,b]=\{\,x\,|\,a\leq x\leq b\,\}.\]
}

\begin{center}
\image{T207_Interval_C}
\end{center}

\begin{example}

\[\bigg[-\frac{1}{2}\lang{de}{;}\lang{en}{,}3\bigg]=\Big\{\,x\,\Big|\,-\frac{1}{2}\leq x\leq 3\,\Big\}\]
\begin{center}
\image{T207_Interval_D}
\end{center}

\end{example}


\lang{de}{Speziell bei abgeschlossenen Mengen ist zusätzlich zu $a<b$ auch $a=b$ zulässig, das Intervall
$[a;a] = \{ a \}$ ist die einpunktige Menge, die nur aus $a$ besteht.}

\lang{en}{In the specific case of closed sets, we allow not only $a<b$ but also $a=b$. The interval
$[a,a] = \{ a \}$ is thus the single-point set comprised of only $a$.}

%begin-cosh
\lang{de}{Au"ser den offenen und abgeschlossenen Intervallen gibt es auch halboffene Intervalle, bei denen genau 
einer der beiden Randpunkte zur Menge geh"ort. Genauer gesagt hat man das 
\notion{rechtsoffene Intervall} von $a$ bis $b$
\[[a;b)=\{\,x\,|\,a \leq x<b\,\},\]
sowie das \notion{linksoffene Intervall} von $a$ bis $b$
\[(a;b]=\{\,x\,|\,a< x\leq b\,\}.\]
}
\lang{en}{In addition to open and closed intervals, there are also half-open intervals which have one of their boundary points as part of the set. 
More precisely, the \notion{right-open interval} from $a$ to $b$ is \[[a,b)=\{\,x\,|\,a \leq x<b\,\},\]
and the \notion{left-open interval} from $a$ to $b$ is \[(a,b]=\{\,x\,|\,a< x\leq b\,\}.\] }


\begin{tabs*}[\initialtab{0}]
\tab{\lang{de}{Beispiel}\lang{en}{Example}}
	\begin{example}
		\lang{de}{Rechtsoffenes Intervall}\lang{en}{Right-open interval}
		$\qquad \big[-\frac{3}{2}\lang{de}{;}\lang{en}{,}\frac{3}{2}\big)=\big\{\,x\,\big|\,-\frac{3}{2}\leq x<\frac{3}{2}\,\big\}$
		\begin{center}
			\image{T207_Interval_E}
		\end{center}
		
		\lang{de}{Linksoffenes Intervall}\lang{en}{Left-open interval}
		$\qquad (-2\lang{de}{;}\lang{en}{,}1]=\{\,x\,|\,-2< x\leq 1\,\}$
		\begin{center}
			\image{T207_Interval_F}
		\end{center}
	\end{example}
\end{tabs*}
%end-cosh


% \begin{tabs*}[\initialtab{0}]
% \tab{\lang{de}{Halboffene Intervalle}\lang{en}{Half-Open Intervals} $[a\lang{de}{;}\lang{en}{,}b)$ \lang{de}{und}\lang{en}{and} 
% $(a\lang{de}{;}\lang{en}{,}b]$}
% \lang{de}{Nimmt man den linken Randpunkt $a$ zur Menge hinzu, erhält man das \notion{rechtsoffene Intervall} von $a$ bis $b$ und
% schreibt daf\"ur $[a;b)$. Es ist also
% \[[a;b)=\{\,x\,|\,a \leq x<b\,\}.\]
% }
% \lang{en}{If we want to include the left boundary value $a$ in the set, we get what we call a \notion{right-open interval} from $a$ to $b$ and
% write $[a;b)$.
% \[[a,b)=\{\,x\,|\,a \leq x<b\,\}\]
% }
% 
% 
% \begin{center}
% \image{intervall_rechtsoffen}
% \end{center}
% 
% \begin{example}
% 
% \[\bigg[-\frac{3}{2}\lang{de}{;}\lang{en}{,}\frac{3}{2}\bigg)=\Big\{\,x\,\Big|\,-\frac{3}{2}\leq x<\frac{3}{2}\,\Big\}\]
% \begin{center}
% \image{intervall_bsp_rechtsoffen}
% \end{center}
% 
% \end{example}
% 
% \lang{de}{Wird anstelle des linken der rechte Randpunkt $b$ zum offenen Intervall hinzugefügt, erhält man das
% \notion{linksoffene Intervall}
% \[(a;b]=\{\,x\,|\,a< x\leq b\,\}.\]
% }
% \lang{en}{Instead of taking the left boundary value, if we take the right boundary value $b$ and add it to the interval, we obtain what we call a
% \notion{left-open interval}.
% \[(a,b]=\{\,x\,|\,a< x\leq b\,\}\]
% }
% \begin{center}
% \image{intervall_linksoffen}
% \end{center}
% 
% \begin{example}
% 
% \[(-2\lang{de}{;}\lang{en}{,}1]=\{\,x\,|\,-2< x\leq 1\,\}\]
% \begin{center}
% \image{intervall_bsp_linksoffen}
% \end{center}
% 
% \end{example}
% \end{tabs*}

\begin{tabs*}[\initialtab{0}]
\tab{\lang{de}{Anmerkung zu weiteren Notationen}\lang{en}{Note regarding other notation}}
\lang{de}{Während die Notation zu abgeschlossenen Intervallen in der Literatur als $[a;b]$ einheitlich ist, gibt
es für rechts-, links- bzw. offene Intervalle auch eine weitere Notation, nämlich anstelle der runden Klammer die nach außen 
geöffnete eckige Klammer, d.h. das rechtsoffene Intervall von $a$ bis $b$ wird auch mit $[a;b[$ bezeichnet, das linksoffene
mit $]a;b]$ und das offene Intervall von $a$ bis $b$ mit $]a;b[$.\\
In diesem Kurs wird aber durchgängig die Notation mit runden Klammern verwendet.}
\lang{en}{While the notation concerning closed intervals in literature is always written $[a,b]$, there is an alternative
notation for right- and left-open intervals. Instead of writting an open interval with a round bracket, a closed bracket pointing outwards can be used.
E.g. The right open interval from $a$ to $b$ can also be written $[a,b[$, and the left-open interval can be written $]a,b]$. The open interval from $a$ to $b$ can also be written
$]a,b[$.\\
In this course, only the round brackets style will be used to denote open intervals.}
\end{tabs*}


\section{Unendliche Intervalle und Durchschnitte}\label{sec:unendl-intervalle}

\lang{de}{Oft werden auch die Zahlenmengen $\{\,x\,|\,x>a\,\}$, $\{\,x\,|\,x\geq a\,\}$, $\{\,x\,|\,x<a\,\}$ und
$\{\,x\,|\,x\leq a\,\}$ für feste reelle Zahlen $a$ benötigt. In Anlehnung an die Intervalle werden diese Mengen
bezeichnet als}

\lang{en}{The sets $\{\;x\,|\,x>a\,\}$, $\{\,x\,|\,x\geq a\,\}$, $\{\,x\,|\,x<a\,\}$ and
$\{\,x\,|\,x\leq a\,\}$ for a given real number $a$ are used very often. Following the interval rules, these sets are denoted by}

\begin{eqnarray}
\{\,x\,|\,x>a\,\} &=& (a\lang{de}{;}\lang{en}{,}\infty) \, , \\
\{\,x\,|\,x\geq a\,\} &=& [a\lang{de}{;}\lang{en}{,}\infty )\, ,\\
\{\,x\,|\,x<a\,\} &=& (-\infty\lang{de}{;}\lang{en}{,}a) \, , \\
\{\,x\,|\,x\leq a\,\} &=& (-\infty\lang{de}{;}\lang{en}{,} a]\, , \\
\R & = & (-\infty\lang{de}{;}\lang{en}{,} \infty )
\end{eqnarray}
\lang{de}{mit oberen \glqq{}Grenzen\grqq{} $\infty$ (sprich: \notion{unendlich}) bzw. unteren 
\glqq{}Grenzen\grqq{} $-\infty$ (sprich: \notion{minus unendlich} oder \notion{negativ unendlich}).
Da $\infty$ und $-\infty$ keine Zahlen sind, steht dort immer eine runde Klammer.}
\lang{en}{with the upper bound $\infty$ (spoken: \notion{infinity}) or the lower bound 
$-\infty$ (spoken: \notion{minus infinity} or \notion{negative infinity}).
Because $\infty$ and $-\infty$ aren't numbers, they will always have round bracket associated with them.}



\begin{theorem}\label{thm:intervallschnitt}
Der  Durchschnitt (Schnittmenge) zweier Intervalle ist wieder ein Intervall oder die leere Menge.

Sind beide Intervalle offen/abgeschlossen/linksoffen/rechtsoffen, so ist auch der Durchschnitt (sofern er nicht leer ist) wieder  offen/abgeschlossen/linksoffen/rechtsoffen.
\end{theorem}

\begin{proof*}[Beweis]
\begin{incremental}{0}
\step
Für die Intervalle $(a;b)$ und $(c;d)$ ist der Durchschnitt $(a;b)\cap (c;d)$ die Menge der Zahlen, die
in beiden Intervallen liegen, also die Menge
\[   \{ x\in \R \mid a<x<b\text{ und }c<x<d \}. \]
Man hat also die Bedingungen $x>a$ und $x>c$, welche zusammengefasst werden können zu $x>\max\{a;c\}$,
und die Bedingungen $x<b$ und $x<d$, welche zusammengefasst werden können zu $x<\min\{b;d\}$.

Ist nun $\max\{a;c\}\geq \min\{b;d\}$, so können nicht beide Bedingungen $x>\max\{a;c\}$ 
und $x<\min\{b;d\}$ gleichzeitig erfüllt sein und der Durchschnitt ist die leere Menge $\emptyset$.

Ist jedoch $\max\{a;c\}< \min\{b;d\}$, so ist der Durchschnitt die Menge
\[  \{ x\in \R \mid \alpha<x<\beta \} =(\alpha; \beta), \]
wobei $\alpha=\max\{a;c\}$ und $\beta=\min\{b;d\}$.

Für abgeschlossene, links- oder rechtsoffene Intervalle geht man genauso vor. Man muss lediglich Vergleichszeichen $<$ oder $>$ durch $\leq$ bzw. $\geq$ ersetzen.
\end{incremental}
\end{proof*}


\begin{example}
\begin{enumerate}
\item Es sind
\begin{eqnarray*}
 (2; 5)\cap (3; 6) &=& (3; 5), \\
 [2; 5]\cap [3; 6] &=& [3; 5], \\
 (2; 5]\cap (3; 6] &=& (3; 5], \\
 [2; 5)\cap [3; 6) &=& [3; 5), \\ &&\\
 (2; 5)\cap [3; 6] &=& [3; 5), \\
 [2; 5]\cap (3; 6) &=& (3; 5], \\
\end{eqnarray*} 
\item $(1; 4)\cap (10; 14)=\emptyset.$
\end{enumerate}
\end{example}


\begin{quickcheck}
\lang{de}{\text{Markieren Sie alle richtigen Antworten:}}
\begin{choices}{multiple}
    \begin{choice}
        \text{$(3;4)$: das Intervall ist offen}
        \solution{true}
    \end{choice}
    \begin{choice}
        \text{$[-5;-3]$: das Intervall ist abgeschlossen}
        \solution{true}
    \end{choice}
    \begin{choice}
        \text{$(3;4]$: das Intervall ist halboffen}
        \solution{true}
    \end{choice}
    \begin{choice}
        \text{$[1;3)$: das Intervall ist linksoffen}
        \solution{false}
    \end{choice}
    \begin{choice}
        \text{$[0;\infty)$: das Intervall ist offen}
        \solution{false}
    \end{choice}
\end{choices} 
\explanation{$\R$ ist abgeschlossen und damit ist jedes Intervall $[a;\infty)$ auch abgeschlossen.}
\end{quickcheck}

\section{Epsilon-Umgebung}\label{sec:epsilon-umg}

Eine Art von Intervallen, die im folgenden Abschnitt und auch bei der Stetigkeit von Funktionen
und bei der Differenzierbarkeit wichtig wird, sind die Epsilon-Umgebungen.

\begin{definition}
Sei $x_0\in \R$ und $\epsilon>0$. Dann ist die \notion{$\epsilon$-Umgebung} von $x_0$, bezeichnet mit
$U_\epsilon(x_0)$ gegeben als
\[  U_\epsilon(x_0) =\{ x\in \R \mid {|x-x_0|}<\epsilon \} =(x_0-\epsilon; x_0+\epsilon). \]
\end{definition}

\begin{example}
Die $\epsilon$-Umgebungen von $x_0=0$ sind die Intervalle
\[ (-\epsilon; \epsilon). \]

Die $1$-Umgebung von $x_0=2$ ist $(2-1; 2+1)=(1;3)$.
\end{example}



\end{content}