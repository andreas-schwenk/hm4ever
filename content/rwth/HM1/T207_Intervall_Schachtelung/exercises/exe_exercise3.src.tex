\documentclass{mumie.element.exercise}
%$Id$
\begin{metainfo}
  \name{
    \lang{de}{Ü03: abgeschlossene/offene Mengen}
    \lang{en}{}
  }
  \begin{description} 
 This work is licensed under the Creative Commons License Attribution 4.0 International (CC-BY 4.0)   
 https://creativecommons.org/licenses/by/4.0/legalcode 

    \lang{de}{Hier die Beschreibung}
    \lang{en}{}
  \end{description}
  \begin{components}
  \end{components}
  \begin{links}
  \end{links}
  \creategeneric
\end{metainfo}
\begin{content}
\title{
\lang{de}{Ü03: abgeschlossene/offene Mengen}
}
\begin{block}[annotation]
  Im Ticket-System: \href{http://team.mumie.net/issues/9981}{Ticket 9981}
\end{block}
 
\lang{de}{Zeigen oder widerlegen Sie:
\begin{enumerate}
 \item[a)]  Seien $O_1, O_2, ... \subset \R$ offene Teilmengen. Dann gilt $\bigcup_{i=1}^\infty O_i $ ist offen.
 \item[b)] Seien $A_1, A_2, ... \subset \R$ abgeschlossene Teilmengen. Dann ist $\bigcap_{i=1}^\infty A_i$ abgeschlossen.
 \item[c)] Seien $A_1, A_2, ... \subset \R$ abgeschlossene Teilmengen. Dann ist $\bigcup_{i=1}^\infty A_i$ abgeschlossen.
\end{enumerate}}

\begin{tabs*}[\initialtab{0}\class{exercise}]

	\tab{
  \lang{de}{Antwort }}
  a) wahr, b) wahr, c) falsch
  
  \tab{
  \lang{de}{Lösung a)}}
  
  	Sei $x \in \bigcup_{i=1}^\infty O_i$, dann existiert mindestens ein $k\in \N$, sodass
	$x\in O_k$ gilt. Da $O_k$ nach Voraussetzung offen ist, existiert also eine Umgebung 	
	$U=U_\epsilon(x) \subset O_k$, $\epsilon>0$. Da insbesondere $O_k \subset \bigcup_{i=1}^\infty O_i$ 
	ist, folgt also auch $U\subset \bigcup_{i=1}^\infty O_i$. Da $x$ beliebig war, folgt die Offenheit.

	\tab{
  \lang{de}{Lösung b)}}
  \begin{incremental}[\initialsteps{1}]
    \step 
    Wir wollen zeigen dass die Aussage stimmt. Statt der Abgeschlossenheit der angegebenen Menge,
	können wir auch die Offenheit ihres Komplements in $\R$ zeigen. 
	\step 
	Dazu nutzen wir, dass nach de Morgan 
	\[ \R\setminus \left( \bigcap_{i=1}^\infty A_i \right) = \bigcup_{i=1}^\infty ( \R \setminus A_i ). \]
	Da die Mengen $A_i$ für jedes $i\in \N$ abgeschlossen sind, sind ihre Komplemente offen 
	und mit Aufgabenteil a) folgt die Offenheit der rechten Seite, als Vereinigung von offenen
	Mengen. 

\end{incremental}
  
  \tab{
  \lang{de}{Lösung c)}}
  Einpunktige Mengen sind nach Definition abgeschlossen, damit ist für jedes $n\in\N$ die 
	Menge $\{ \frac{1}{n} \} $ abgeschlossen. Aber $\bigcup_{n=1}^\infty \{ \frac{1}{n} \} $
	ist nicht abgeschlossen. 
  
  
\end{tabs*}

\end{content}