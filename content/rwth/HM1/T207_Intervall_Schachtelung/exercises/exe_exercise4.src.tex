\documentclass{mumie.element.exercise}
%$Id$
\begin{metainfo}
  \name{
    \lang{de}{Ü04: Randpunkte und innere Punkte}
    \lang{en}{}
  }
  \begin{description} 
 This work is licensed under the Creative Commons License Attribution 4.0 International (CC-BY 4.0)   
 https://creativecommons.org/licenses/by/4.0/legalcode 

    \lang{de}{Hier die Beschreibung}
    \lang{en}{}
  \end{description}
  \begin{components}
  \end{components}
  \begin{links}
  \end{links}
  \creategeneric
\end{metainfo}
\begin{content}
\title{
\lang{de}{Ü04: Randpunkte und innere Punkte}
}
\begin{block}[annotation]
  Im Ticket-System: \href{http://team.mumie.net/issues/9982}{Ticket 9982}
\end{block}
 
\lang{de}{Bestimmen Sie für die nachfolgenden Mengen $M_i$, $i=1,2,3$, die Mengen der Randpunkte  $\partial M_i$ 
und die Mengen $\mathring{M_i}$ der inneren Punkte. 
\begin{enumerate}
\item[a)] $M_1 = (-3;1)$, 
\item[b)] $M_2 = [2;5]$,
\item[c)] $M_3 = \Q$.
\end{enumerate}}


\begin{tabs*}[\initialtab{0}\class{exercise}]

  \tab{
  \lang{de}{Lösung a)}}
  
  $M_1$ ist ein offenes Intervall, damit also als Menge selbst offen und gleich ihrem Inneren,
$\mathring{M_1}=M_1$. Randpunkte sind die Intervallgrenzen, also $\partial M_1=\{-3;1\}$. 
Es gilt weiter $M_1\cap \partial M_1=\emptyset$.
  
	\tab{
  \lang{de}{Lösung b)}}
  Das abgeschlossene Intervall $M_2$ hat die beiden Grenzen $2$ und $5$ als Randpunkte, $\partial M_2=\{2;5\}$.
  Nimmt man diese heraus, so erhält man die Menge $\mathring{M_2}=(2;5)$ der inneren Punkte.
  
  \tab{
  \lang{de}{Lösung c)}}
  $\Q$ enthält keine inneren Punkte, da in jeder Umgebung einer rationalen Zahl stets eine 
  reelle Zahl liegt. (Man findet sogar beliebig viele reelle Zahlen in jeder Umgebung.)
  Damit ist auch jede reelle Zahl ein Randpunkt, $\partial \Q=\R$, während $\mathring{\Q}=\emptyset$.
  
\end{tabs*}


\end{content}