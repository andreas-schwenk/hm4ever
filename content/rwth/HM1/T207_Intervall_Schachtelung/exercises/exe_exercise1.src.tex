\documentclass{mumie.element.exercise}
%$Id$
\begin{metainfo}
  \name{
    \lang{de}{Ü01: Intervallschreibweise}
    \lang{en}{}
  }
  \begin{description} 
 This work is licensed under the Creative Commons License Attribution 4.0 International (CC-BY 4.0)   
 https://creativecommons.org/licenses/by/4.0/legalcode 

    \lang{de}{Hier die Beschreibung}
    \lang{en}{}
  \end{description}
  \begin{components}
  \end{components}
  \begin{links}
  \end{links}
  \creategeneric
\end{metainfo}
\begin{content}
\title{
\lang{de}{Ü01: Intervallschreibweise}
}
\begin{block}[annotation]
  Im Ticket-System: \href{http://team.mumie.net/issues/9979}{Ticket 9979}
\end{block}
 
\lang{de}{Geben Sie die Intervallschreibweise für folgende Mengen an. 
\begin{itemize}
\item[a)] $[0;1] \cup (1;2]$,
\item[b)] $[-2;3] \cap (-2;4)$,
\item[c)] $(-\infty;5] \cap (1;6)$,
\item[d)] $ \{ x\in \R \vert \:x^2< 1 \}$,
\item[e)] $\{x\in\R \vert \:x\le 0 \}$.
\end{itemize}
}

\begin{tabs*}[\initialtab{0}\class{exercise}]

	\tab{
  \lang{de}{Antwort}}
  
   a) $[0,2]$\\
   b) $(-2,3]$\\
   c) $(1,5]$\\
   d) $(-1,1)$\\
   e) $ (-\infty,0]$ 
  
  \tab{
  \lang{de}{Lösung a)}}
  
    \lang{de}{Die beiden angegebenen Intervalle liefern eine disjunkte Zerlegung des Intervalls $[0;2]$, sodass
	die Vereinigung genau wieder dieses Intervall ist.}
  

  \tab{
  \lang{de}{Lösung b)}
  }
  Das Intervall $(-2;4]$ enthält den Punkt $x_0=-2$ nicht, weshalb er im Schnitt nicht vorkommen kann. 
Da der Randpunkt $x_1=3$ im zweiten Intervall enthalten ist gehört dieser Punkt auch zum gesuchten Intervall.
Damit erhalten wir als Lösung $I_b=(-2;3]$. 

  \tab{
  \lang{de}{Lösung c)}
  }
  $(1;5]$: Analoge Begründung wie in b).
    
    \tab{
  \lang{de}{Lösung d)}
  }
  Gesucht ist die Menge aller Punkte deren Quadrat kleiner als Eins ist. Dies ist genau für die Punkte erfüllt, die zwischen
$x_1=-1$ und $x_2=1$ liegen. Also ist das gesuchte Intervall $I_d:=(-1;1)$. 
    
    \tab{
  \lang{de}{Lösung e)}
  }
  Die Menge der reellen Zahlen, die kleiner oder gleich Null sind, entspricht genau dem Intervall
$I_e:=(-\infty; 0]$.
  


\end{tabs*}

\end{content}