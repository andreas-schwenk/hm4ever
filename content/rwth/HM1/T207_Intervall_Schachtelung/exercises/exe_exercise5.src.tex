\documentclass{mumie.element.exercise}
%$Id$
\begin{metainfo}
  \name{
    \lang{de}{Ü05: Intervallschachtelung}
    \lang{en}{}
  }
  \begin{description} 
 This work is licensed under the Creative Commons License Attribution 4.0 International (CC-BY 4.0)   
 https://creativecommons.org/licenses/by/4.0/legalcode 

    \lang{de}{Hier die Beschreibung}
    \lang{en}{}
  \end{description}
  \begin{components}
  \end{components}
  \begin{links}
  \end{links}
  \creategeneric
\end{metainfo}
\begin{content}
\title{
\lang{de}{Ü05: Intervallschachtelung}
}
\begin{block}[annotation]
  Im Ticket-System: \href{http://team.mumie.net/issues/9983}{Ticket 9983}
\end{block}
 
\lang{de}{Bestimmen Sie $\sqrt{30}$ bis auf 3 Nachkommastellen genau.\\ 
Hinweis: Intervallschachtelung mit geeignetem Startintervall. }

\begin{tabs*}[\initialtab{0}\class{exercise}]

  \tab{
  \lang{de}{Antwort }}
  $\sqrt{30} \approx 5,477$
  
  \tab{
  \lang{de}{Lösung }}
  \begin{incremental}[\initialsteps{1}]
    \step 
    \lang{de}{Es ist $5^2 < 30 <6^2$, damit liegt also $\sqrt{30}\in [5;6]$. Es bietet sich also an, das 
	Intervall $[a_1;b_1]:= [5;6]$ zu setzen.}
     
    \step \lang{de}{Wir berechnen nun so lange Potenzen von $30,x$ 
	für $x\in \{ 0;\ldots; 9 \}$ bis wir das Zehntel gefunden haben, in dem $\sqrt{30}$ liegt.
	Nach dieser Berechnung sieht man $5,4^2 < 30 < 5,5^2$. Damit können wir mit der zweiten 
	Nachkommastelle fortfahren und dann iterativ fortsetzen.}
	
	\step \lang{de}{Es ist
\begin{align*}
 5,47^2 < \, &30 < 5,48^2 , \\
 5,477^2 < \, &30 < 5,478^2, \\
 5,4772^2 < \, &30 < 5,4773^2.
\end{align*}
Damit gilt $ 5,4772 < \sqrt{30} < 5,4773 $. \\
Hinweis: Dieses Verfahren benötigt eine Reihe an Nebenrechnungen und ist daher nicht 
besonders effizient.}
  \end{incremental}

  
\end{tabs*}

\end{content}