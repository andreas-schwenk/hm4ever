\documentclass{mumie.element.exercise}
%$Id$
\begin{metainfo}
  \name{
    \lang{de}{Ü06: Intervallhalbierungsverfahren}
    \lang{en}{}
  }
  \begin{description} 
 This work is licensed under the Creative Commons License Attribution 4.0 International (CC-BY 4.0)   
 https://creativecommons.org/licenses/by/4.0/legalcode 

    \lang{de}{Hier die Beschreibung}
    \lang{en}{}
  \end{description}
  \begin{components}
  \end{components}
  \begin{links}
  \end{links}
  \creategeneric
\end{metainfo}
\begin{content}
\title{
\lang{de}{Ü06: Intervallhalbierungsverfahren}
}
\begin{block}[annotation]
  Im Ticket-System: \href{http://team.mumie.net/issues/9984}{Ticket 9984}
\end{block}

\begin{enumerate}
\item 
Bestimmen Sie ein Intervall der Länge $\le \frac{1}{4} $ in dem eine reelle Nullstelle von 
\[ f:\R \to \R , f(x) = (x-1)^3 -2 \]
liegt. Benutzen Sie das Verfahren der Intervallhalbierung (Bisektionsverfahren).
\item
\begin{enumerate}
\item[a)]
Führen Sie für die Funktion $f(x)=x^3+2x-4$ zwei Schritte des Bisektionsverfahren durch. 
Geben Sie ausgehend von $[0;2]$ ein Intervall der Länge 0,5 an, in dem eine Nullstelle liegt.
\item[b)]
Wieviele Schritte muss man machen, um (ausgehend von $[0;2]$) ein Intervall der Länge $10^{-6}$ anzugeben,
in dem eine Nullstelle liegt?\\
Geben Sie die Anzahl formelmäßig und näherungsweise (mit der groben Abschätzung $2^3\approx 10$) an.
\end{enumerate}
\end{enumerate}


\begin{tabs*}[\initialtab{0}\class{exercise}]
  \tab{
    \lang{en}{...}
    \lang{de}{Antworten}
  }
\begin{enumerate}
\item Die Nullstelle liegt im Intervall $[\frac{9}{4}; \frac{10}{4}]$.
\item  \begin{enumerate}
       \item Die Nullstelle liegt im Intervall $[1;1,5]$.
       \item Die Formel für die Schrittanzahl lautet: $n\geq \log_2(2\cdot 10^6)$\\
             Mit $2^3\approx 1000$ sind es mindestens $n=21$ Schritte.
       \end{enumerate}

\end{enumerate}
        
 
  \tab{
    \lang{en}{...}
    \lang{de}{Lösung 1)}
  }
    \begin{incremental}[\initialsteps{1}]
      %\step
       % \lang{en}{...}
        %\lang{de}{Die Nullstelle liegt im Intervall $[\frac{9}{4}, \frac{10}{4}]$.}
      \step
        \lang{en}{...}
        \lang{de}{Wir starten den Algorithmus mit den folgenden Werten:
\begin{align*}
 a= a_1= 2: \qquad f(a) &=f(2) = (2-1)^3-2 = - 1 < 0 \\
 b= b_1 = 3: \qquad f(b) &= f(3) = (3-1)^3-2 = 6 > 0. 
\end{align*}
Wir bestimmen den Intervallmittelpunkt, den zugehörigen Funktionswert und passen dann die Intervallgrenzen an.
Es ist
\[ \frac{a_1+b_1}{2} = \frac{2+3}{2} = \frac{5}{2}. \]
Damit gilt 
\[ f\left( \frac{a_1+b_1}{2} \right) = f(2,5) = (1,5)^3 - 2 = 1,375 >0. \]}
 \step \lang{de}{Nach den Regeln des Algorithmus setzen wir die neuen Intervallgrenzen auf 
\[ a_2=a_1=2, \qquad b_2 = \frac{a_1+b_1}{2} = \frac{5}{2} . \]
Wir wiederholen nun das Prozedere mit den neuen Werten und erhalten
\begin{align*}
 \frac{a_2+b_2}{2} &= \frac{2+2,5}{2} = \frac{9}{4}, \\
 f\left( \frac{9}{4} \right) &= \left( \frac{5}{4} \right)^3 - 2 = \frac{125}{64} - 2 < 0.
\end{align*}
Das nächste Intervall $[a_3;b_3]$ ist dann gegeben durch 
\[ a_3=\frac{a_2+b_2}{2}=\frac{9}{4},\qquad b_3 = b_2 = \frac{5}{2} = \frac{10}{4} . \]
Damit haben wir das gesuchte Intervall gefunden.
}
    \end{incremental}
  \tab{
    \lang{en}{...}
    \lang{de}{Lösungsvideo zu 2)}
  }

        \lang{de}{\youtubevideo[500][300]{746h2lK8brQ}}
 
\end{tabs*}





\end{content}