\documentclass{mumie.element.exercise}
%$Id$
\begin{metainfo}
  \name{
    \lang{de}{Ü02: Mengen und Intervalle}
    \lang{en}{}
  }
  \begin{description} 
 This work is licensed under the Creative Commons License Attribution 4.0 International (CC-BY 4.0)   
 https://creativecommons.org/licenses/by/4.0/legalcode 

    \lang{de}{Hier die Beschreibung}
    \lang{en}{}
  \end{description}
  \begin{components}
  \end{components}
  \begin{links}
  \end{links}
  \creategeneric
\end{metainfo}
\begin{content}
\title{
\lang{de}{Ü02: Mengen und Intervalle}
}
\begin{block}[annotation]
  Im Ticket-System: \href{http://team.mumie.net/issues/9980}{Ticket 9980}
\end{block}
 
\lang{de}{Entscheiden Sie, ob die nachfolgenden Mengen jeweils ein Intervall beschreiben.
\begin{enumerate}
\item[a)] $M_1= \{ x\in\Q \vert \:x < 7 \}$,
\item[b)] $M_2=\{ x\in \R \vert \:x<0 \}$,
\item[c)] $M_3=\{ -1;3;\pi \}$,
\item[d)] $M_4=\{ x \in \R \vert\: x^2 > 7 \}$,
\item[e)] $M_5=\{ x\in \R \vert \:|x| <3 \}$.
\end{enumerate}}

\begin{tabs*}[\initialtab{0}\class{exercise}]

	\tab{
  \lang{de}{Antwort}}
  
  a) nein, b) ja, c) nein, d) nein, e) ja.
  
  \tab{
  \lang{de}{Lösung a)}}
  
    \lang{de}{Intervalle sind Teilmengen der reellen Zahlen, die zwischen zwei vorgegebenen reellen Zahlen
jede reelle Zahl enthalten, $M_1$ enthält aber nur rationale Zahlen.}
  

  \tab{
  \lang{de}{Lösung b)}}
  
  In Intervallschreibweise gilt $M_2= (-\infty;0)$. Damit ist $M_2$ ein Intervall. 

  \tab{
  \lang{de}{Lösung c)}}
  
  Eine Menge, die nur aus einzelnen Punkten besteht, kann kein Intervall beschreiben (außer
eine einpunktige Menge).
    
    \tab{
  \lang{de}{Lösung d)}}
  
  Es handelt sich nicht um ein Intervall, da $M_4$ aus zwei unzusammenhängenden Komponenten besteht:
  $(-\infty;-\sqrt{7})$ und $(\sqrt{7};\infty)$
    
    \tab{
  \lang{de}{Lösung e)}}
  
  Es gilt $M_5 = (-3;3)$, weshalb $M_5$ ein Intervall ist.
  


\end{tabs*}


\end{content}