%$Id:  $
\documentclass{mumie.article}
%$Id$
\begin{metainfo}
  \name{
    \lang{de}{Intervall-Schachtelung}
    \lang{en}{}
  }
  \begin{description} 
 This work is licensed under the Creative Commons License Attribution 4.0 International (CC-BY 4.0)   
 https://creativecommons.org/licenses/by/4.0/legalcode 

    \lang{de}{Beschreibung}
    \lang{en}{}
  \end{description}
  \begin{components}
\component{generic_image}{content/rwth/HM1/images/g_img_00_Videobutton_schwarz.meta.xml}{00_Videobutton_schwarz}
 \component{generic_image}{content/rwth/HM1/images/g_img_00_video_button_schwarz-blau.meta.xml}{00_video_button_schwarz-blau}
\end{components}
  \begin{links}
    \link{generic_article}{content/rwth/HM1/T205_Konvergenz_von_Folgen/g_art_content_15_monotone_konvergenz.meta.xml}{monot-konv}
    \link{generic_article}{content/rwth/HM1/T205_Konvergenz_von_Folgen/g_art_content_16_konvergenzkriterien.meta.xml}{konv-krit}
    \link{generic_article}{content/rwth/HM1/T210_Stetigkeit/g_art_content_29_stetigkeit_definitionen.meta.xml}{stetig}
    \link{generic_article}{content/rwth/HM1/T211_Eigenschaften_stetiger_Funktionen/g_art_content_33_zwischenwertsatz.meta.xml}{zwischenwertsatz}
  \end{links}
  \creategeneric
\end{metainfo}
\begin{content}
\usepackage{mumie.ombplus}
\ombchapter{7}
\ombarticle{3}

\lang{de}{\title{Intervall-Schachtelung}}
 
\begin{block}[annotation]
  Links zu stetigen Funktionen müssen noch eingefügt werden
  
\end{block}
\begin{block}[annotation]
  Im Ticket-System: \href{http://team.mumie.net/issues/9680}{Ticket 9680}\\
\end{block}

\begin{block}[info-box]
\tableofcontents
\end{block}


\section{Intervall-Schachtelung}
Eine Abfolge  von Intervallen mit reellen Intervallgrenzen nennt man eine 
Intervallschachtelung, wenn jedes Intervall ganz im vorhergehenden enthalten ist und
wenn die Intervalllängen gegen 0 gehen. Aufgrund der Vollständigkeit von $\R$ führt eine 
Interallschachtelung auf ein Element von $\R$.

\begin{definition}
Eine Folge abgeschlossener Intervalle $I_n=[a_n;b_n]$, $n\in \N$ heißt genau dann \notion{Intervallschachtelung},
wenn 
\begin{itemize}
\item jedes Intervall das nachfolgende enthält, d.h. $I_{n+1}\subseteq I_n$ für alle $n\in \N$, und
\item die Folge der Intervalllängen $(b_n-a_n)_{n\in \N}$ gegen $0$ konvergiert.
\end{itemize} 
\end{definition}

\begin{example}
\begin{enumerate}
\item 
Die Intervalle $I_n=[0;\frac{1}{n}]$ für $n\in \N$ bilden eine Intervallschachtelung: Wegen $\frac{1}{n+1}<\frac{1}{n}$
ist $[0;\frac{1}{n+1}]\subseteq [0;\frac{1}{n}]$ und für die Intervalllängen gilt:
\[ \lim_{n\to \infty} \frac{1}{n} =0. \]
\item Die Intervalle $I_n=[\frac{1}{n};\frac{2}{n}]$ für $n\in \N$ bilden \textbf{keine} Intervallschachtelung, da die Intervalle nicht ineinander enthalten sind:
$I_1=[1;2], I_2=[\frac{1}{2};1], I_3=[\frac{1}{3};\frac{2}{3}]$.
\item Die Intervalle $I_n=[1- \frac{1}{n};2+ \frac{1}{n}]$ für $n\in \N$ bilden \textbf{keine} Intervallschachtelung, da die Intervalllängen nicht gegen $0$ konvergieren.
\end{enumerate}
\end{example}

\begin{rule}
Eine Folge abgeschlossener Intervalle $I_n=[a_n; b_n]$, $n\in \N$ ist genau dann eine Intervallschachtelung, wenn
für die Folgen $(a_n)_{n\in \N}$ und  $(b_n)_{n\in \N}$ die folgenden Bedingungen gelten
\begin{itemize}
\item Für alle $n\in \N$ ist $a_n\leq b_n$,
\item die Folge $(a_n)_{n\in \N}$ ist monoton wachsend und die Folge $(b_n)_{n\in \N}$ ist monoton fallend,
\item die  Folgen $(a_n)_{n\in \N}$ und $(b_n)_{n\in \N}$ konvergieren gegen denselben Grenzwert.
%\item $\sup \{ a_n | n\in \N \} = \inf \{ b_n | n\in \N \}$.
\end{itemize}
\end{rule}


\begin{proof*}[Beweis (Intervallschachtelung)]
Wenn $I_n=[a_n; b_n]$ ein Intervall darstellt, dann muss $a_n\leq b_n$ gelten.
\begin{incremental}{0}
\step
Für Intervalle $I_{n+1}$ und $I_n$ mit $I_{n+1}\subseteq I_n$ bedeutet das auch, dass $a_{n+1}\geq a_n$ und $b_{n+1}\leq b_n$ gilt.
Die Intervalle sind also alle ineinander enthalten, wenn die Folge $(a_n)_{n\in \N}$ monoton wachsend und die Folge $(b_n)_{n\in \N}$ monoton fallend ist.
\step
Die ersten beiden Punkte implizieren, dass die Folgen $(a_n)_{n\in \N}$ und $(b_n)_{n\in \N}$ konvergieren, denn
für alle $n\in \N$ gilt dann:
\[   a_n\leq b_n\leq b_1\quad \text{sowie}\quad b_n\geq a_n\geq a_1. \]
Das heißt die Folge $(a_n)_{n\in \N}$ ist monoton wachsend und nach oben durch $b_1$ beschränkt, also konvergent 
(vgl. Abschnitt \link{monot-konv}{Monotone Konvergenz}).
Die Folge $(b_n)_{n\in \N}$ ist monoton fallend und nach unten durch $a_1$ beschränkt, also konvergent.

Wegen $\lim_{n\to \infty} (b_n-a_n)= \lim_{n\to \infty} b_n -  \lim_{n\to \infty}  a_n$, ist schließlich die Bedingung, dass die
Intervalllängen gegen $0$ konvergieren äquivalent dazu, dass die Folgen den gleichen Grenzwert haben.
\end{incremental}
\end{proof*}

Mit Intervallverschachtelungen lassen sich beispielsweise die reellen Zahlen definieren.

\begin{theorem}[Intervall-Schachtelungsprinzip]\label{thm:intervallschachtelungsprinzip}
Ist  $I_n=[a_n;b_n]$, $n\in \N$ eine Intervallschachtelung, so gibt es genau eine reelle Zahl $r\in \R$ mit
\[  r\in I_n \quad \text{für alle }n\in \N.\]
Diese Zahl $r$ ergibt sich als
\[  r= \sup\{ a_n |n\in \N\}= \lim_{n\to \infty}  a_n, \]
bzw. 
\[ r = \inf\{ b_n |n\in \N\} = \lim_{n\to \infty} b_n. \]
\end{theorem}


\begin{proof*}
\begin{incremental}{0}
\step
Eine reelle Zahl $s$ liegt genau dann in allen Intervallen $I_n$, wenn $a_n\leq s\leq b_n$ für alle $n\in \N$, wenn also 
\[  \sup\{ a_n |n\in \N\} \leq s \leq  \inf\{ b_n |n\in \N\}\]
gilt. Da aber $(a_n)_{n\in \N}$ eine monoton wachsende Folge und $(b_n)_{n\in \N}$ eine monoton fallende Folge ist,
gilt $\sup\{ a_n |n\in \N\}= \lim_{n\to \infty}  a_n$ und $\inf\{ b_n |n\in \N\} = \lim_{n\to \infty} b_n$. Weil die $I_n$ eine 
Intervallschachtelung bilden, sind nach obiger Regel die Grenzwerte gleich. Also folgt
\[ s= \lim_{n\to \infty}  a_n = \lim_{n\to \infty} b_n.\]
Dieser Grenzwert ist also genau die reelle Zahl, die in allen Intervallen liegt.
\end{incremental}
\end{proof*}

Das folgende Video behandelt die Intervallschachtelungen umfassend und benutzt das Intervallhalbierungsverfahren
, um eine Näherung für $\sqrt{2}$ zu finden:
\floatright{\href{https://api.stream24.net/vod/getVideo.php?id=10962-2-10819&mode=iframe&speed=true}{\image[75]{00_video_button_schwarz-blau}}}\\


\section{Intervallhalbierungsverfahren}\label{sec:intervallhalbierung}

Ein spezielles Beispiel für eine Intervallschachtelung ergibt sich beim Intervallhalbierungsverfahren 
zur Bestimmung von Nullstellen sogenannter \link{stetig}{stetiger Funktionen}. Dieses Verfahren ist auch 
unter dem Namen \notion{Bisektionsverfahren} bekannt.

\begin{theorem}
  Für eine reellwertige und stetige Funktion $f$, definiert auf dem Intervall \nowrap{$I \subseteq \R$,} seien 
  $a,b\in I$ mit $a<b$ und $f(a)<0<f(b)$.

Dann definiert man eine Intervallschachtelung $I_n=[a_n; b_n]$, $n\in \N$ folgendermaßen: 
%Folgen $(a_n)_{n\in \N}$ und $(b_n)_{n\in \N}$,

  \[ a_1 \coloneq a,\; b_1 \coloneq b.\]

Für alle $n\in \N$ definiert man weiter:
   \begin{align*} a_{n+1} &= \frac{a_n + b_n}{2}\, ,\qquad &b_{n+1} = b_n, \quad &\text{falls }
     f(\frac{a_n + b_n}{2}) < 0,\\
a_{n+1} &= a_n\, ,\qquad &b_{n+1} = \frac{a_n + b_n}{2}, \quad &\text{falls }
     f(\frac{a_n + b_n}{2}) > 0,\\
   a_{n+1} &= \frac{a_n + b_n}{2}\, ,\qquad &b_{n+1} = \frac{a_n + b_n}{2}, \quad &\text{falls }
     f(\frac{a_n + b_n}{2}) = 0.
   \end{align*}

Die durch die Intervallschachtelung definierte Stelle \[x_0= \sup\{ a_n |n\in \N\}= \lim_{n\to \infty}  a_n = 
\inf\{ b_n |n\in \N\} = \lim_{n\to \infty} b_n  \]
ist dann eine Nullstelle der Funktion $f$.\\
\floatright{\href{https://www.hm-kompakt.de/video?watch=408}{\image[75]{00_Videobutton_schwarz}}}\\\\
\end{theorem}

\begin{remarks}
  \begin{enumerate}
    \item Normalerweise lässt man im Sonderfall $f(\frac{a_n + b_n}{2}) = 0$ die Folge abbrechen, weil man dann direkt eine Nullstelle gefunden hat.
	\item 
    \lang{de}{Solange der Algorithmus nicht abbricht, gilt für die  Werte der Funktion $f$ an den Rändern der 
    Intervalle $[a_n,b_n]$: \nowrap{$f(a_n) < 0 < f(b_n)$.} 
    Nach dem \ref[zwischenwertsatz][Zwischenwertsatz]{thm:zwischenwertsatz} existiert 
    somit mindestens eine Nullstelle von $f$ in jedem der Intervalle.
    }
    \lang{en}{As long as the algorithm does not terminate the values of the function $f$ at the boundaries of 
    the invervals $[a_n,b_n]$ satisfy the inequalities \nowrap{$f(a_n) < 0 < f(b_n)$.} 
    Due to the intermediate value theorem there is at least one root 
    in each of the intervalls \nowrap{$[a_n,b_n]$.}}  
     \item 
    \lang{en}{The sequence of lengths of the invervals $[a_n,b_n]$ converges to zero exponentially:}
    \lang{de}{Die Folge der Längen der Intervalle $[a_n,b_n]$ konvergiert \emph{exponentiell} gegen Null:}
    \[
    \text{Länge von }[a_n;b_n]\text{ ist } \frac{b-a}{2^{n-1}}\, .
    \]
     \item 
    \lang{en}{If $f$ satisfies $f(a) > 0 > f(b)$ then apply the above algorithm to $\, -f$.}
    % Analogous statements hold for functions with \nowrap{$f(a) > 0 > f(b)$.}
    \lang{de}{Falls $f$ die Ungleichung $f(a) > 0 > f(b)$ erfüllt, dann wende den Algorithmus auf \nowrap{$\, -f$} an.}
   \end{enumerate}

\end{remarks}

\floatright{\href{https://api.stream24.net/vod/getVideo.php?id=10962-2-10820&mode=iframe&speed=true}{\image[75]{00_video_button_schwarz-blau}}}\\


\begin{example}
Wir suchen eine Nullstelle der Funktion
\[ f(x) = (x-2)^2 -3 .
\]

Für die Berechnung der exakten Nullstellen, lässt sich $$f(x)$$ ausmultiplizieren zu
\[f(x)=x^2-4x+1\]
Mit der pq-Formel folgt direkt
\[x_{1,2}=2 \pm \sqrt{3}\]


Wir wissen also, dass $2+\sqrt{3}\approx 3,732$ eine Nullstelle ist, wollen aber die Intervallhalbierung illustrieren. 
\\\\
Wir starten den Algorithmus mit folgenden Werten:
\[ a=a_1 =2,\quad f(a) = f(2) = -3 <0;
\]
\[b=b_1=4, \quad f(b) = f(4) = 1 >0.
\]
Jetzt bestimmen wir den Punkt in der Mitte zwischen $a_1$ und $b_1$ und berechnen den Wert von $f$ dort:
\[ \frac{a_1 + b_1}{2} = \frac{2+4}{2} = 3, \qquad f\left(\frac{a_1 + b_1}{2}\right) = f(3)= -2 <0.
\]
Nach den Regeln des Algorithmus ist das neue Intervall $\,[a_2;\,b_2]\,$ von folgenden Punkten begrenzt.
\[ a_2 = \frac{a_1 + b_1}{2} = 3,\qquad b_2 = b_1 =4,
\]
mit $\,f(a_2)<0<f(b_2)$. Wir wiederholen diesen Schritt mit den neuen Werten.
\[ \frac{a_2 + b_2}{2} = \frac{3+4}{2} = 3,5,\qquad f(3,5) = -\frac{3}{4} <0,
\]
\[ a_3 = \frac{a_2 + b_2}{2} = \frac{7}{2} = 3,5,\qquad b_3 = b_2 = 4.
\]
Der nächste Schritt ist
\[ \frac{a_3 + b_3}{2} = \frac{3,5+4}{2} = 3,75, \qquad f(3,75) = \frac{1}{16} > 0,
\]
\[ a_4 = a_3 = 3,5,\qquad b_4 = \frac{a_3 + b_3}{2}= \frac{15}{4} = 3,75.
\]
Das Anfangsintervall $\,[a_1;\,b_1] = [2;\,4]\,$ hat Länge $\textcolor{#0066CC}{2}$. Nach $\textcolor{#CC6600}{3}$ Schritten
wissen wir, dass eine Nullstelle im Intervall $\,[a_4;\,b_4] = [3,5;\,3,75]\,$ der Länge $\textcolor{#0066CC}{2}/ 2^{\textcolor{#CC6600}{3}} = 1/4$
liegt, (vgl. mit dem berechneten Wert von 3,73).\\
Der Algorithmus kann fortgesetzt werden bis die gewünschte Genauigkeit erreicht ist.
\end{example}


Ein weiteres Beispiel, bei dem ein Intervallhalbierungsverfahren angewandt wird, ist für den im Abschnitt
\link{konv-krit}{Weitere Konvergenzkriterien} angegebenen Satz.

\begin{theorem}
Ist $(x_n)_{n\in \N}$ eine beschränkte Folge, so gibt es eine Teilfolge $(x_{n_k})_{k\in \N}$, die konvergiert. 
\end{theorem}

\begin{proof*}[Beweis (Satz von Bolzano-Weierstraß)]
Beweisskizze:
Nach Voraussetzung existieren $a,b\in \R$ mit $a\leq x_n\leq b$ für alle $n\in \N$. Wir unterteilen nun das Intervall $[a;b]$ sukzessive 
durch Halbierung in Teilintervalle und machen stets mit einem Teilintervall weiter, in dem unendlich viele Folgenglieder liegen. So entsteht eine
Intervallschachtelung. Die Teilfolge $x_{n_k}$, die entsteht, wenn man bei jeder Intervallteilung
das nächste Folgenglied auswählt, das im ausgewählten Teilintervall liegt, konvergiert gegen den Punkt $x*$, der in
allen ausgewählten Teilintervallen liegt.\\
\begin{incremental}{0}
\step
Sei $a$ eine untere Schranke für die Folge $(x_n)_{n\in \N}$ und $b$ eine obere Schranke, d.h.
$x_n\in [a;b]$ für alle $n\in \N$. Wir definieren nun induktiv eine Intervallschachtelung $I_k=[a_k; b_k]$, $k\in \N$ folgendermaßen: 

Zunächst setzen wir $[a_1;b_1]=[a;b]$. Für $k\geq 1$ setzen wir 
\[ [a_{k+1};b_{k+1}] = [ a_k; \frac{a_k + b_k}{2} ], \]
falls es unendlich viele $n$ gibt mit $a_k\leq x_n\leq \frac{a_k + b_k}{2} $. Zusätzlich setzten wir
\[ [a_{k+1};b_{k+1}] = [ \frac{a_k + b_k}{2}; b_k],\]
falls im Intervall $[ a_k; \frac{a_k + b_k}{2} ]$ nur endlich viele $x_n$ liegen.

Auf diese Weise ist sichergestellt, dass in allen Intervallen $[a_{k+1};b_{k+1}]$ unendlich viele Folgenglieder $x_n$ liegen. 
Wir definieren nun die Indizes $n_k$ ($k\in \N$) für die Teilfolge induktiv durch $n_1=1$ und
\[  n_{k+1}= \min \{ n>n_k \mid x_n\in  [a_{k+1};b_{k+1}] \} \]
für alle $k\geq 1$.
Da jedes Intervall $[a_{k+1};b_{k+1}]$ unendlich viele $x_n$ enthält, ist garantiert, dass es stets ein $n>n_k$ gibt mit $x_n\in [a_{k+1};b_{k+1}]$. Insbesondere ist nach
Konstruktionstets $x_{n_{k+1}}\in [a_{k+1};b_{k+1}]$ (und natürlich auch $x_{n_1}=x_1\in [a_1;b_1]$).
\step
Zuletzt ist noch nachzuweisen, dass die so definierte Teilfolge $(x_{n_k})_{k\in \N}$ tatsächlich konvergiert.

Die Intervallschachtelung definiert eine reelle Zahl $x^*$ durch
\[   x^*=\lim_{k\to \infty}  a_k = \lim_{k\to \infty} b_k. \]
Da für alle $k\in \N$ nach Konstruktion der Teilfolge 
\[  a_k\leq x_{n_k}\leq b_k \]
gilt, folgt aus dem \ref[konv-krit][Sandwich-Lemma]{thm:sandwich}, dass die Teilfolge 
$(x_{n_k})_{k\in \N}$ konvergent ist, und es gilt:
\[ \lim_{k\to \infty} x_{n_k} =x^*. \]
\end{incremental}
\end{proof*}

Das folgende Video zeigt den Beweis nicht nur für reelle Folgen, sondern auch für komplexe Folgen 
sowie eine kleine Anekdote.

\floatright{\href{https://api.stream24.net/vod/getVideo.php?id=10962-2-10821&mode=iframe&speed=true}{\image[75]{00_video_button_schwarz-blau}}}\\


\begin{quickcheck}
    
    \type{input.number}
        \begin{variables}
            \number{nu}{1,414}
            \number{no}{1,415}
        \end{variables}
    \text{Bestimmen Sie mittels einer einfachen Intervallschachtelung (Quadrieren von 
    Dezimalzahlen mit immer mehr Nachkommastellen) die dritte
    Nachkommastelle von $\sqrt{2}$.\\
    Starten Sie mit dem Intervall für die erste Nachkommastelle:
    $[1,4;1,5]$. \\
    Das dritte Intervall lautet:[\ansref;\ansref]
    }
    \begin{answer}
        \solution{nu}
    \end{answer}
    \begin{answer}
        \solution{no}
    \end{answer}
    \explanation{$1,414^2=1,9993$ und $1,415^2=2.0022$\\
    Eine Folge, die sehr schnell gegen $\sqrt{2}$ konvergiert ist $a_n=\frac{1}{2}(a_n+\frac{2}{a_n})$.
    }
\end{quickcheck}
\end{content}