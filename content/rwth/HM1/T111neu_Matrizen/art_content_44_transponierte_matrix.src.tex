%$Id:  $
\documentclass{mumie.article}
%$Id$
\begin{metainfo}
  \name{
    \lang{de}{Transponierte Matrix}
    \lang{en}{Matrix transposition and symmetric matrices}
  }
  \begin{description} 
 This work is licensed under the Creative Commons License Attribution 4.0 International (CC-BY 4.0)   
 https://creativecommons.org/licenses/by/4.0/legalcode 

    \lang{de}{Beschreibung}
    \lang{en}{Description}
  \end{description}
  \begin{components}
    \component{generic_image}{content/rwth/HM1/images/g_img_00_video_button_schwarz-blau.meta.xml}{00_video_button_schwarz-blau}
    \component{generic_image}{content/rwth/HM1/images/g_img_00_Videobutton_schwarz.meta.xml}{00_Videobutton_schwarz}
    \component{js_lib}{system/media/mathlets/GWTGenericVisualization.meta.xml}{mathlet1}
  \end{components}
  \begin{links}
    \link{generic_article}{content/rwth/HM1/T401_Matrizenrechnung/g_art_content_03_transponierte.meta.xml}{content_03_transponierte}
    \link{generic_article}{content/rwth/HM1/T109_Skalar-_und_Vektorprodukt/g_art_content_31_skalarprodukt.meta.xml}{skalarprodukt}
    \link{generic_article}{content/rwth/HM1/T111neu_Matrizen/g_art_content_43_matrizenmultiplikation.meta.xml}{matrix-mult}
    \link{generic_article}{content/rwth/HM1/T112neu_Lineare_Gleichungssysteme/g_art_content_41_gauss_verfahren.meta.xml}{gauss-verfahren}
  \end{links}
  \creategeneric
\end{metainfo}
\begin{content}
\begin{block}[annotation]
	Im Ticket-System: \href{https://team.mumie.net/issues/21333}{Ticket 21333}
\end{block}
\begin{block}[annotation]
Copy of \href{http://team.mumie.net/issues/9061}{Ticket 9061}: content/rwth/HM1/T112_Rechnen_mit_Matrizen/art_content_44_transponierte_matrix.src.tex
\end{block}

\usepackage{mumie.ombplus}
\ombchapter{11}
\ombarticle{5}
\usepackage{mumie.genericvisualization}

\begin{visualizationwrapper}

\title{\lang{de}{Transponierte Matrix, Symmetrische Matrix}
       \lang{en}{Matrix transposition and symmetric matrices}}
 
%\begin{block}[annotation]
%  übungsinhalt
%  
%\end{block}
% \begin{block}[annotation]
%   Im Ticket-System: \href{http://team.mumie.net/issues/9061}{Ticket 9061}\\
% \end{block}

\begin{block}[info-box]
\tableofcontents
\end{block}

\lang{de}{
In diesem Kapitel wird mit dem \textit{Transponieren einer Matrix} eine weitere
Matrizenoperation eingeführt.
Wie die transponierte Matrix zur Charakterisierung von Matrizen beitragen kann,
wird exemplarisch anhand der \textit{symmetrischen} Matrix gezeigt.
}
\lang{en}{
In this section we will introduce the \textit{transpose of a matrix}, another matrix operation. The 
transpose can contribute to the characterisation of matrices by naturally following into the 
definition of \textit{symmetric} matrices.
}



\section{\lang{de}{Transponierte Matrix}\lang{en}{Matrix transposition}}

\lang{de}{
Das Transponieren ist eine Operation, bei der wir die Einträge an der Diagonalen spiegeln.
Die Zeilen der Ursprungsmatrix werden zu den Spalten der Transponierten.
Formal aufgeschrieben bedeutet das:
}
\lang{en}{
Transposition is an operation performed on a single matrix, in which the entries are reflected in the 
\emph{'leading diagonal'}. The rows of a matrix are the columns of its transpose. Formally:
}

\begin{definition}[\lang{de}{Transponierte Matrix}\lang{en}{Matrix transpose}]\label{def:transp_matrix}

\lang{de}{
Es sei $A=\left( a_{ij} \right)_{1 \leq i \leq m, 1 \leq j \leq n}$ eine $(m\times n)$-Matrix:
%Sei $A=\left(a_{ij}\right) \in M(m,n;\R)$ eine Matrix
}
\lang{en}{
Let $A=\left( a_{ij} \right)_{1 \leq i \leq m, 1 \leq j \leq n}$ be a $(m\times n)$-matrix:
}
\begin{equation*}
A =
\begin{pmatrix}
a_{11} & a_{12} & \cdots & a_{1n} \\
a_{21} & a_{22} & \cdots & a_{2n} \\
\vdots & \vdots & \ddots & \vdots \\
a_{m1} & a_{m2} & \cdots & a_{mn}
\end{pmatrix}.
\end{equation*}
\lang{de}{
Die folgende $(n\times m)$-Matrix $A^T$ heißt dann \textbf{Transponierte} der Matrix $A$:
%=\left( a_{ji} \right)_{1 \leq j \leq n, 1 \leq i \leq m}$
%, dann hei"st die Matrix
}
\lang{en}{
The following $(n\times m)$-matrix $A^T$ is then called the \textbf{transpose} of the matrix $A$:
}
\begin{equation*}
A^{T} := 
%(a_{ij})^T_{1 \leq i \leq m, 1 \leq j \leq n} :=
\begin{pmatrix}
a_{11} & a_{21} & \cdots & a_{m1} \\
a_{12} & a_{22} & \cdots & a_{m2} \\
\vdots & \vdots & \ddots & \vdots \\
a_{1n} & a_{2n} & \cdots & a_{mn}
\end{pmatrix}. % = (a_{ji}) \in M(n,m;\R)
\end{equation*}

\lang{de}{
Die Spalten von $A^T$ sind also die Zeilen der Matrix $A$ und die Zeilen von $A^T$ sind die Spalten 
von $A$.\\
\floatright{\href{https://www.hm-kompakt.de/video?watch=824}{\image[75]{00_Videobutton_schwarz}}}\\\\
}
\lang{en}{
The columns of $A^T$ are the rows of $A$ and the rows of $A^T$ are the columns of $A$.\\
}
\end{definition}

\lang{de}{In der Literatur ist auch die Notation $A^{tr}$ für die transponierte Matrix verbreitet.}
\lang{en}{In the literature, the notation $A^{tr}$ is also used for the transpose of a matrix $A$.}

\begin{example}
\begin{enumerate}
\item \lang{de}{Die transponierte Matrix zu}
      \lang{en}{The transpose of}
      $\begin{pmatrix} 
       \textcolor{#CC6600}{1} & \textcolor{#CC6600}{2} & \textcolor{#CC6600}{3}  \\ 
       \textcolor{#0066CC}{4} & \textcolor{#0066CC}{5} & \textcolor{#0066CC}{6} 
       \end{pmatrix} $
      \lang{de}{ist}
      \lang{en}{is}
      $\begin{pmatrix} \textcolor{#CC6600}{1} & \textcolor{#0066CC}{4} \\ 
                       \textcolor{#CC6600}{2} & \textcolor{#0066CC}{5} \\ 
                       \textcolor{#CC6600}{3} & \textcolor{#0066CC}{6} \end{pmatrix} .$ \\
      \lang{de}{Die transponierte Matrix zu}
      \lang{en}{The transpose of}
      $ \begin{pmatrix} 1 & 4 \\ 
                        2 & 5 \\ 
                        3 & 6 \end{pmatrix} $ 
      \lang{de}{ist}\lang{en}{is} 
      $  \begin{pmatrix} 1 & 2 & 3  \\ 
                         4 & 5 & 6 \end{pmatrix}$.
\item \lang{de}{
      Die Transponierte eines Spaltenvektors (also einer $(m\times 1)$-Matrix) ist ein Zeilenvektor, 
      also eine $(1\times m)$-Matrix, und umgekehrt:
      }
      \lang{en}{
      The transpose of a column vector (an $(m\times 1)$-matrix) is a row vector (an 
      $(1\times m)$-matrix), and conversely
      }
\[  \begin{pmatrix} 1 \\ -3 \\ 2 \end{pmatrix}^T =\begin{pmatrix} 1 & -3 & 2 \end{pmatrix}. \]
\end{enumerate}
\end{example}

\begin{remark}
\begin{enumerate}
\item \lang{de}{
      Es ist leicht zu sehen, dass die Transponierte der Transponierten einer jeden Matrix wieder die 
      Matrix selbst ist, d.\,h. für jede Matrix $A$ ist
      }
      \lang{en}{
      It is easy to see that the transpose of the transpose of a matrix is always just the matrix 
      itself again, that is for every matrix $A$ we have
      }
      \[ (A^{T})^{T} =A. \]
\item \lang{de}{
      Mit Hilfe der transponierten Matrix lässt sich das \link{skalarprodukt}{Skalarprodukt} 
      $(\bullet)$ zweier Spaltenvektoren im $\R^n$ auch als \link{matrix-mult}{Matrizenprodukt} 
      $(\cdot)$ schreiben:
      }
      \lang{en}{
      Using the matrix transpose we can write the \link{skalarprodukt}{scalar product} $(\bullet)$ 
      of two column vectors in $\R^n$ as a \link{matrix-mult}{product of matrices}: 
      }
      \[\begin{pmatrix} v_1 \\ v_2\\ \vdots \\ v_n  \end{pmatrix}\bullet 
        \begin{pmatrix} w_1 \\ w_2\\ \vdots \\ w_n  \end{pmatrix} = 
        \begin{pmatrix} v_1 \\ v_2\\ \vdots \\ v_n  \end{pmatrix}^T \cdot 
        \begin{pmatrix} w_1 \\ w_2\\ \vdots \\ w_n  \end{pmatrix}, \]
      \lang{de}{denn Letzteres ist}
      \lang{en}{as the right-hand side is}
      \[\begin{pmatrix} v_1 \\ v_2\\ \vdots \\ v_n  \end{pmatrix}^T \cdot 
        \begin{pmatrix} w_1 \\ w_2\\ \vdots \\ w_n  \end{pmatrix} = 
        \begin{pmatrix} v_1 & v_2& \cdots & v_n  \end{pmatrix}\cdot 
        \begin{pmatrix} w_1 \\ w_2\\ \vdots \\ w_n  \end{pmatrix} = v_1w_1+\ldots+v_nw_n. \]
\end{enumerate}
\end{remark}

\begin{block}[warning]
\lang{de}{Achtung: Es gilt}
\lang{en}{Warning: in general we have}
\[
    v^T \cdot w \quad \neq \quad v \cdot w^T.
\]
\end{block}

\begin{example}
    \begin{tabs*}
        \tab{\lang{de}{Zu Bemerkung 1}\lang{en}{For remark 1}}
            \[
                A = 
                \begin{pmatrix}
                    1 & 2 \\
                    3 & 4
                \end{pmatrix},
                \quad
                A^T =
                \begin{pmatrix}
                    1 & 3 \\
                    2 & 4
                \end{pmatrix},
            \]
            \[
                (A^T)^T = 
                \begin{pmatrix}
                    1 & 3 \\
                    2 & 4
                \end{pmatrix}
                ^T
                =
                \begin{pmatrix}
                    1 & 2 \\
                    3 & 4
                \end{pmatrix}
                =
                A.
            \]
        \tab{\lang{de}{Zu Bemerkung 2}\lang{en}{For remark 2}}
        \lang{de}{Für}
        \lang{en}{For}
            \[
                v =
                \begin{pmatrix}
                    6 \\
                    2
                \end{pmatrix}
                ,\quad w =
                \begin{pmatrix}
                    1 \\
                    3
                \end{pmatrix}
            \]
           \lang{de}{gilt}
           \lang{en}{we have}
            \[
                v \bullet w = v^T \cdot w =
                \begin{pmatrix}
                    6 & 2
                \end{pmatrix}
                \cdot
                \begin{pmatrix}
                    1 \\
                    3
                \end{pmatrix}
                = 6 \cdot 1 + 2 \cdot 3 = 12.
            \]
        \tab{\lang{de}{Zur Warnung}\lang{en}{For warning}}
        %Achtung:
        \lang{de}{Für}
        \lang{en}{For}
            \[
                v =
                \begin{pmatrix}
                    6 \\
                    2
                \end{pmatrix}
                ,\quad w =
                \begin{pmatrix}
                    1 \\
                    3
                \end{pmatrix}
            \]
           \lang{de}{gilt}
           \lang{en}{we have}
            \[
                v \cdot w^T =
                \begin{pmatrix}
                    6 \\
                    2
                \end{pmatrix}
                \cdot
                \begin{pmatrix}
                    1 & 3
                \end{pmatrix}
                =
                \begin{pmatrix}
                    6 & 18 \\
                    2 & 6
                \end{pmatrix}
                \neq
                v \bullet w.
            \]
    \end{tabs*}
\end{example}

\lang{de}{
Rechenregeln für transponierte Matrizen werden im Kursteil \link{content_03_transponierte}{3b} behandelt.
}
\lang{en}{
We will cover the rules for calculating with transposed matrices in the later section 
\link{content_03_transponierte}{3b}.
}



\section{\lang{de}{Symmetrische Matrix}\lang{en}{Symmetric matrices}}

\lang{de}{Die folgende Definition legt fest, wann eine Matrix \textit{symmetrisch} ist.}
\lang{en}{Now we use the transpose to define whether a matrix is \textit{symmetric} or not.}

\begin{definition}\label{def:symmetrische_matrix}
\lang{de}{
Eine quadratische Matrix $A\in M(n;\R)$ heißt \notion{symmetrisch}, wenn sie gleich ihrer 
transponierten Matrix ist, also im Fall
}
\lang{en}{
A square matrix $A\in M(n;\R)$ is called \notion{symmetric} if it is equal to its own transpose, i.e.
}
\[ A^T=A.\]
\end{definition}

\begin{example}
\begin{enumerate}
\item \lang{de}{
      Beim Transponieren von $(2\times 2)$-Matrizen wird lediglich der Eintrag an der Stelle 
      $(1, 2)$ mit dem Eintrag an der Stelle $(2, 1)$ vertauscht. Die $(2\times 2)$-Matrix 
      $A=\left( \begin{smallmatrix}2 & 1 \\ 1 & -1\end{smallmatrix}\right) \in M(2;\R)$ ist also 
      symmetrisch, da die beiden Einträge gleich sind. Die Matrix 
      $A=\left( \begin{smallmatrix}5 & -1 \\ 2 & 0\end{smallmatrix}\right) \in M(2;\R)$ hingegen ist 
      nicht symmetrisch.
      }
      \lang{en}{
      When transposing $(2\times 2)$-matrices we only need to swap the entries at $(1, 2)$ and at 
      $(2, 1)$. The $(2\times 2)$-matrix 
      $A=\left( \begin{smallmatrix}2 & 1 \\ 1 & -1\end{smallmatrix}\right) \in M(2;\R)$ is hence 
      symmetric, as these two entries are equal. On the other hand, the matrix 
      $A=\left( \begin{smallmatrix}5 & -1 \\ 2 & 0\end{smallmatrix}\right) \in M(2;\R)$ is not 
      symmetric.
      }
\item \lang{de}{Für die $(3\times 3)$-Matrix }
      \lang{en}{For the $(3\times 3)$-matrix }
      $A=\left(\begin{smallmatrix}2&0 & 1 \\ 0& 1 & -1\\ 0&1 & 3\end{smallmatrix}\right) $ 
      \lang{de}{gilt}
      \lang{en}{we have}
      \[A^T=\begin{pmatrix} 2&0 & 1 \\ 0& 1 & -1\\ 0&1 & 3\end{pmatrix}^T = 
        \begin{pmatrix} 2&0&0 \\ 0 &1&1\\ 1&-1&3\end{pmatrix}. \]
      \lang{de}{Also ist $A^T\neq A$ und daher $A$ nicht symmetrisch.}
      \lang{en}{Thus $A^T\neq A$ and $A$ is not symmetric.}
\item \lang{de}{Jede Diagonalmatrix}
      \lang{en}{Every diagonal matrix}
      \[\begin{pmatrix} a_{11} & 0 & 0 & ... & 0 \\ 0 & a_{22} & 0 & ... & 0 \\ 
        0 & 0 & a_{33} & ... & 0\\ &&& ... &0 \\ 0 & 0 & 0 & ... & a_{nn} \end{pmatrix}\]
      \lang{de}{
      ist symmetrisch, da beim Transponieren die Diagonaleinträge $a_{ii}$ fest bleiben und 
      außerhalb der Diagonalen nur Nullen stehen.
      }
      \lang{en}{
      is symmetric, as the diagonal entries $a_{ii}$ always remain unchanged by composition, and all 
      other entries are equal to zero.
      }
\end{enumerate}
\end{example}



\end{visualizationwrapper}


\end{content}