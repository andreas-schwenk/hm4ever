\documentclass{mumie.element.exercise}
%$Id$
\begin{metainfo}
  \name{
    \lang{de}{Ü03: Matrixmultiplikation}
    \lang{en}{Exercise 3}
  }
  \begin{description} 
 This work is licensed under the Creative Commons License Attribution 4.0 International (CC-BY 4.0)   
 https://creativecommons.org/licenses/by/4.0/legalcode 

    \lang{de}{Rechnen mit Matrizen}
    \lang{en}{}
  \end{description}
  \begin{components}
  \end{components}
  \begin{links}
  \end{links}
  \creategeneric
\end{metainfo}
\begin{content}
\begin{block}[annotation]
	Im Ticket-System: \href{https://team.mumie.net/issues/21339}{Ticket 21339}
\end{block}
\begin{block}[annotation]
Copy of \href{http://team.mumie.net/issues/9665}{Ticket 9665}: content/rwth/HM1/T112_Rechnen_mit_Matrizen/exercises/exe_exercise3.src.tex
\end{block}

\usepackage{mumie.ombplus}

\title{
  \lang{de}{Ü03: Matrixmultiplikation}
}

\begin{block}[annotation]
  Rechnen mit Matrizen
     
\end{block}



\lang{de}{ 
Bestimmen Sie: \\
\begin{enumerate}
 \item[(a)]
\[
  \begin{pmatrix}
  1 & 2 \\ 
  4 & -6
 \end{pmatrix}
\cdot
 \begin{pmatrix}
  2 & -3 \\ 
  1 & 7
 \end{pmatrix}
\]
 \item[(b)]
\[
  \begin{pmatrix}
  1 & 0 & -3\\ 
  -2 & 5 & 4\\
  0 & 7 & 9
 \end{pmatrix}
\cdot
 \begin{pmatrix}
  0 & 2 & 5\\ 
  3 & 3 & -2\\
  1 & 0 & 4
 \end{pmatrix}
\]
 \item[(c)]
\[
  \begin{pmatrix}
  1 & 0 \\ 
  -4 & 6 \\
  2 & 8 
 \end{pmatrix}
\cdot
 \begin{pmatrix}
  7 & -2 & -3\\ 
  0 & 6 & -5
 \end{pmatrix}
\]
 \item[(d)]
\[
  \begin{pmatrix}
  1 & 4 & 5
 \end{pmatrix}
\cdot
 \begin{pmatrix}
  2\\ 
  1 \\
  4
 \end{pmatrix}
\]
 \item[(e)]
\[
  \begin{pmatrix}
  2 & 4 & 5 \\
  1 & 5 & -2
 \end{pmatrix}
\cdot
 \begin{pmatrix}
  1 & 6 & 8 \\
  -3 & 2 & -5
 \end{pmatrix}
\]
\item[(f)]
Es sei\\
\[A=\begin{pmatrix} 2 & 1 \\ -1 & 1 \end{pmatrix} \; , \;
B=\begin{pmatrix} 1 & 3 & 0 \\ 5 & -1 & 2 \end{pmatrix} \; , \;
C=\begin{pmatrix} 3 & 0 \\ 2 & 1 \\ -1 & 2 \end{pmatrix} \; , \;
D=\begin{pmatrix} 1 & -1 & 0 & 2 \\ 3 & 0 & 1 & -1 \\ 1 & 1 & 2 & 0\end{pmatrix} \; .\]
Welche Matrixprodukte kann man mit diesen Matrizen bilden? Welche Dimensionen haben die Produkte?
Berechnen Sie die Produkte.
\end{enumerate}
}

\begin{tabs*}[\initialtab{0}\class{exercise}]
  \tab{\lang{de}{Lösung (a)}}
  \lang{de}{\[
  \begin{pmatrix}
  1 & 2 \\ 
  4 & -6
 \end{pmatrix}
\cdot
 \begin{pmatrix}
  2 & -3 \\ 
  1 & 7
 \end{pmatrix}
=
  \begin{pmatrix}
  1\cdot 2 +2\cdot 1 & 1\cdot (-3) +2\cdot 7 \\ 
  4\cdot 2 +(-6)\cdot 1 & 4\cdot (-3) +(-6)\cdot 7
 \end{pmatrix}
=
  \begin{pmatrix}
  4 & 11 \\ 
  2 & -54
 \end{pmatrix}
\]}

 \tab{\lang{de}{Lösung (b)}}
  \lang{de}{\begin{align*}
  &\begin{pmatrix}
  1 & 0 & -3\\ 
  -2 & 5 & 4\\
  0 & 7 & 9
 \end{pmatrix}
\cdot
 \begin{pmatrix}
  0 & 2 & 5\\ 
  3 & 3 & -2\\
  1 & 0 & 4
 \end{pmatrix}\\
=&
 \begin{pmatrix}
  1\cdot 0 + 0 \cdot 3+(-3) \cdot 1& 1\cdot 2 + 0 \cdot 3+ (-3) \cdot 0& 1\cdot 5 + 0 \cdot (-2)+(-3) \cdot 4\\ 
  -2 \cdot 0+ 5 \cdot  3+ 4\cdot 1& -2 \cdot 2+ 5 \cdot 3+ 4\cdot 0 & -2 \cdot 5+ 5 \cdot (-2)+ 4\cdot 4\\
  0 \cdot 0+ 7 \cdot 3+ 9 \cdot 1 & 0 \cdot 2 + 7 \cdot 3 + 9 \cdot 0 & 0 \cdot 5 + 7 \cdot (-2)+ 9 \cdot 4
 \end{pmatrix}\\
=&
 \begin{pmatrix}
  -3 & 2 & -7\\ 
  19 & 11 & -4\\
  30 & 21 & 22
 \end{pmatrix}
\end{align*}}

 \tab{\lang{de}{Lösung (c)}}
  \lang{de}{\begin{align*}
  \begin{pmatrix}
  1 & 0 \\ 
  -4 & 6 \\
  2 & 8 
 \end{pmatrix}
\cdot
 \begin{pmatrix}
  7 & -2 & -3\\ 
  0 & 6 & -5
 \end{pmatrix}
&=  
 \begin{pmatrix}
  1 \cdot 7 + 0 \cdot  0 & 1 \cdot (-2) + 0 \cdot 6 & 1 \cdot (-3) + 0 \cdot (-5)\\ 
  -4 \cdot 7 + 6 \cdot 0 & -4 \cdot (-2) + 6 \cdot 6 & -4 \cdot (-3) + 6 \cdot (-5)\\
  2 \cdot 7 + 8 \cdot 0 & 2 \cdot (-2) + 8 \cdot 6 & 2 \cdot (-3) + 8 \cdot (-5)
 \end{pmatrix}\\
&=
 \begin{pmatrix}
  7 & -2 & -3\\ 
  -28 & 44 & -18\\
  14 & 44 & -46
 \end{pmatrix}
\end{align*}
  }
   \tab{\lang{de}{Lösung (d)}}
  \lang{de}{\[
  \begin{pmatrix}
  1 & 4 & 5
 \end{pmatrix}
\cdot
 \begin{pmatrix}
  2\\ 
  1 \\
  4
 \end{pmatrix}
=(1\cdot 2 +4 \cdot 1 + 5\cdot 4)= (26)
\]}
  
   \tab{\lang{de}{Lösung (e)}}
  \lang{de}{
    Für die Multiplikation zweier Matrizen muss die Spaltenanzahl der ersten Matrix mit der Zeilenanzahl der zweiten Matrix übereinstimmen.
    Im Beispiel hat die erste Matrix 3 Spalten und die zweite Matrix 2 Zeilen.
    Die Aufgabe ist nicht lösbar.
}

     \tab{\lang{de}{Lösungsvideo (f)}}
  \youtubevideo[500][300]{WT_nRhJzga0}\\

\end{tabs*}
\end{content}