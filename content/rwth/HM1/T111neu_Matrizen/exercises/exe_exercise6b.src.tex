\documentclass{mumie.element.exercise}
%$Id$
\begin{metainfo}
  \name{
    \lang{de}{Ü06: Transponierte}
    \lang{en}{Exercise 6}
  }
  \begin{description} 
 This work is licensed under the Creative Commons License Attribution 4.0 International (CC-BY 4.0)   
 https://creativecommons.org/licenses/by/4.0/legalcode 

    \lang{de}{Rechnen mit Matrizen}
    \lang{en}{}
  \end{description}
  \begin{components}
  \end{components}
  \begin{links}
  \end{links}
  \creategeneric
\end{metainfo}
\begin{content}
\begin{block}[annotation]
	Im Ticket-System: \href{https://team.mumie.net/issues/28452}{Ticket 28452}
\end{block}
\usepackage{mumie.ombplus}

\title{
  \lang{de}{Ü06: Transponierte}
}

\begin{block}[annotation]
  Rechnen mit Matrizen
     
\end{block}


\lang{de}{ 
\begin{enumerate}
 \item[(a)]
 Berechnen Sie $M_1=A \cdot A^T$ und $M_2=A^T \cdot A$ zu 
$ A=
 \begin{pmatrix}
  1 & -2 & 0\\
  1 & 3 & 1
 \end{pmatrix}.
$
 \item[(b)]
 Überlegen Sie sich, dass man zu $A \in \mathbb{R}^{m \times n}$ 
 stets die Produkte $A \cdot A^T$ und $A^T \cdot A$ bilden kann.
 Welche Dimensionen ergeben sich?
 \item[(c)] 
Die Produkte $M_1$ und $M_2$ aus (a) sind symmetrisch bzgl. der Hauptdiagonalen, also $M_1^T=M_1$ 
und $ M_2^T=M_2$. Ist das Zufall?

\end{enumerate}}

\begin{tabs*}[\initialtab{0}\class{exercise}]
  \tab{\lang{de}{Lösungsvideo}}
  \youtubevideo[500][300]{WQlRMQ08w-k}\\

\end{tabs*}
\end{content}

