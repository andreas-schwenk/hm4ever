\documentclass{mumie.element.exercise}
%$Id$
\begin{metainfo}
  \name{
    \lang{de}{Ü02: Matrixaddition}
    \lang{en}{Exercise 2}
  }
  \begin{description} 
 This work is licensed under the Creative Commons License Attribution 4.0 International (CC-BY 4.0)   
 https://creativecommons.org/licenses/by/4.0/legalcode 

    \lang{de}{Rechnen mit Matrizen}
    \lang{en}{}
  \end{description}
  \begin{components}
  \end{components}
  \begin{links}
  \end{links}
  \creategeneric
\end{metainfo}
\begin{content}
\begin{block}[annotation]
	Im Ticket-System: \href{https://team.mumie.net/issues/21341}{Ticket 21341}
\end{block}
\begin{block}[annotation]
Copy of \href{http://team.mumie.net/issues/9663}{Ticket 9663}: content/rwth/HM1/T112_Rechnen_mit_Matrizen/exercises/exe_exercise1.src.tex
\end{block}

\usepackage{mumie.ombplus}

\title{
  \lang{de}{Ü02: Matrixaddition}
}

\begin{block}[annotation]
  Rechnen mit Matrizen
     
\end{block}



\lang{de}{ 
Bestimmen Sie folgende Summen von Matrizen. \\
\begin{enumerate}
 \item[(a)] 
\[
 \begin{pmatrix}
  1 & 2 \\ 
  4 & -6
 \end{pmatrix}
+
 \begin{pmatrix}
  2 & -3 \\ 
  1 & 7
 \end{pmatrix}
\]
 \item[(b)]
\[
 \begin{pmatrix}
  -4 & 7 \\ 
  0 & -2
 \end{pmatrix}
+2
 \begin{pmatrix}
  5 & 0 \\ 
  1 & 3
 \end{pmatrix}
\]
 \item[(c)]
\[
 \begin{pmatrix}
  1 & 2 \\ 
  -3 & 8
 \end{pmatrix}
 +
 \begin{pmatrix}
  1 & -2 & 1 \\ 
  7 & -3 & -2
 \end{pmatrix}
\]
 \item[(d)]
\[
 \begin{pmatrix}
  1 & 0 & -3\\ 
  -2 & 5 & 4\\
  0 & 7 & 9
 \end{pmatrix}
+
 \begin{pmatrix}
  0 & 2 & 5\\ 
  3 & 3 & -2\\
  1 & 0 & 4
 \end{pmatrix}
\]
 \item[(e)]
\[
 \begin{pmatrix}
  0 & 0 & 0\\ 
  -2 & 5 & 6\\
  5 & 10 & 3
 \end{pmatrix}
- \frac{1}{2} \begin{pmatrix}
  -4 & 6 & 2\\ 
  -8 & -14 & 0\\
  -16 & 0 & 2
 \end{pmatrix}
\]
\end{enumerate}}

\begin{tabs*}[\initialtab{0}\class{exercise}]

  \tab{\lang{de}{Lösung (a)}}
  \lang{de}{\[
  \begin{pmatrix}
  1 & 2 \\ 
  4 & -6
 \end{pmatrix}
+
 \begin{pmatrix}
  2 & -3 \\ 
  1 & 7
 \end{pmatrix}
=
  \begin{pmatrix}
  1+2 & 2-3 \\ 
  4+1 & -6+7
 \end{pmatrix}
=
  \begin{pmatrix}
  3 & -1 \\ 
  5 & 1
 \end{pmatrix}
\]}

 \tab{\lang{de}{Lösung (b)}}
  \lang{de}{\[
 \begin{pmatrix}
  -4 & 7 \\ 
  0 & -2
 \end{pmatrix}
+2
 \begin{pmatrix}
  5 & 0 \\ 
  1 & 3
 \end{pmatrix}
=
 \begin{pmatrix}
  -4 & 7 \\ 
  0 & -2
 \end{pmatrix}
+
 \begin{pmatrix}
  10 & 0 \\ 
  2 & 6
 \end{pmatrix}
=
  \begin{pmatrix}
  -4+10 & 7+0 \\ 
  0+2 & -2+6
 \end{pmatrix}
=
  \begin{pmatrix}
  6 & 7 \\ 
  2 & 4
 \end{pmatrix}
\]}

 \tab{\lang{de}{Lösung (c)}}
  \lang{de}{
    Die Summe zweier Matrizen kann nur bestimmt werden, wenn beide Matrizen die gleiche Zeilenanzahl und die gleiche Spaltenanzahl besitzen.
    Dies ist hier nicht gegeben.
    Die Aufgabe ist nicht lösbar.
  }


  \tab{\lang{de}{Lösung (d)}}
  \lang{de}{\[
 \begin{pmatrix}
  1 & 0 & -3\\ 
  -2 & 5 & 4\\
  0 & 7 & 9
 \end{pmatrix}
+
 \begin{pmatrix}
  0 & 2 & 5\\ 
  3 & 3 & -2\\
  1 & 0 & 4
 \end{pmatrix}
=
 \begin{pmatrix}
  1+0 & 0+2 & -3+5\\ 
  -2+3 & 5+3 & 4-2\\
  0+1 & 7+0 & 9+4
 \end{pmatrix}
=
 \begin{pmatrix}
  1 & 2 & 2\\ 
  1 & 8 & 2\\
  1 & 7 & 13
 \end{pmatrix}
\]}

 \tab{\lang{de}{Lösung (e)}}
  \lang{de}{
\begin{align*}\begin{pmatrix}
  0 & 0 & 0\\ 
  -2 & 5 & 6\\
  5 & 10 & 3
 \end{pmatrix}
- \frac{1}{2} \cdot
 \begin{pmatrix}
  -4 & 6 & 2\\ 
  -8 & -14 & 0\\
  -16 & 0 & 2
 \end{pmatrix}
&=
 \begin{pmatrix}
  0 & 0 & 0\\ 
  -2 & 5 & 6\\
  5 & 10 & 3
 \end{pmatrix}
-
 \begin{pmatrix}
  -2 & 3 & 1\\ 
  -4 & -7 & 0\\
  -8 & 0 & 1
 \end{pmatrix}\\
&= \begin{pmatrix}
  0-(-2) & 0-3 & 0-1\\ 
  -2-(-4) & 5-(-7) & 6-0\\
  5-(-8) & 10-0 & 3-1
 \end{pmatrix}\\
&= \begin{pmatrix}
  2 & -3 & -1\\ 
  2 & 12 & 6\\
  13 & 10 & 2
 \end{pmatrix}
\end{align*}}


\end{tabs*}
\end{content}