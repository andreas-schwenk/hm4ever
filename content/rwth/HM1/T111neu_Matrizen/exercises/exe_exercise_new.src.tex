\documentclass{mumie.element.exercise}
%$Id$
\begin{metainfo}
  \name{
   \lang{de}{Ü01: Matrixeinträge}
   \lang{en}{Exercise 1}
 }
 \begin{description} 
 This work is licensed under the Creative Commons License Attribution 4.0 International (CC-BY 4.0)   
 https://creativecommons.org/licenses/by/4.0/legalcode 

   \lang{de}{Hier die Beschreibung}
   \lang{en}{Description here}
 \end{description}
 \begin{components}
 \end{components}
 \begin{links}
 \end{links}
 \creategeneric
 \end{metainfo}
 \begin{content}
\begin{block}[annotation]
	Im Ticket-System: \href{https://team.mumie.net/issues/21342}{Ticket 21342}
\end{block}
\begin{block}[annotation]
Copy of \href{https://team.mumie.net/issues/15905}{Ticket 15905}: content/rwth/HM1/T111_Matrizen,_lineare_Gleichungssysteme/exercises/exe_exercise_new.src.tex
\end{block}

\title{
  \lang{de}{Ü01: Matrixeinträge}
  \lang{en}{Exercise 1}
}
\begin{block}[annotation]
	Matrizen, Lineare Gleichungen: Übungen
\end{block}
% \begin{block}[annotation]
%   Im Ticket-System: \href{http://team.mumie.net/issues/15903}{Ticket 15903}
% \end{block}

   \lang{de}{Geben Sie die folgenden Matrizen $A$ explizit an.}
   \lang{en}{Subtitle}
 \begin{table}[\class{items}]
   \nowrap{(a) $\ A=(a_{ij})\in M(3, 3;\R)$ mit $a_{ij}=i\cdot j$}\\
   \nowrap{(b) $\ A=(a_{ij})\in M(2, 3;\R)$ mit $a_{ij}=i - j$}\\
   \nowrap{(c) $\ A=(a_{ij})\in M(3, 4;\R)$ mit $a_{ij}=(i+1)\cdot (j-2)$}
 \end{table}
 \begin{tabs*}[\initialtab{0}\class{exercise}]
   \tab{
   \lang{de}{Antwort}
   \lang{en}{Answer}
  }
  
 \begin{table}[\class{items}]
   \nowrap{(a) $\ \begin{pmatrix} 1 & 2 & 3 \\ 2 & 4 & 6 \\ 3 & 6 & 9\end{pmatrix}$}\\
   \nowrap{(b) $\ \begin{pmatrix} 0 & -1 & -2 \\ 1 & 0 & -1\end{pmatrix}$}\\
   \nowrap{(c) $\ \begin{pmatrix} -2 & 0 & 2 & 4\\ -3 & 0 & 3 & 6\\ -4 & 0 & 4 & 8\end{pmatrix}$}
 \end{table}
  \tab{
   \lang{de}{Lösung (a)}
   \lang{en}{Solution (a)}
  }
  Die Einträge der Matrix $A$ sind durch die Formel $a_{ij}=i\cdot j$ bestimmt.
  Dabei steht jeweils in der $i$-ten Zeile und $j$-ten Spalte der Matrix der Eintrag $a_{ij}$.
  Die Größe der Matrix ist, wie in der Aufgabenstellung angegeben, $3 \times 3$. Nun berechnet man explizit die Einträge, indem
  man in die Formel die verschiedenen Werte $1$, $2$ etc. für $i$ und $j$ einsetzt, und erhält für die einzelnen Einträge:
  \[  a_{11}=1\cdot 1=1,\quad a_{12}=1\cdot 2=2, \quad a_{13}=1\cdot 3=3 \quad \text{etc}.\]
  Also ist die Matrix 
  \[ A= \begin{pmatrix} 1 & 2 & 3 \\ 2 & 4 & 6 \\ 3 & 6 & 9\end{pmatrix}. \]
\tab{
   \lang{de}{Lösung (b)}
   \lang{en}{Solution (b)}
  }
  Die Einträge der Matrix $A$ sind durch die Formel $a_{ij}=i-j$ bestimmt.
  Dabei steht jeweils in der $i$-ten Zeile und $j$-ten Spalte der Matrix der Eintrag $a_{ij}$.
  Die Matrix soll $2$ Zeilen und $3$ Spalten, wie in der Aufgabenstellung angegeben, haben. Nun berechnet man explizit die Einträge, indem
  man in die Formel die verschiedenen Werte $1$, $2$ etc. für $i$ und $j$ einsetzt, und erhält für die einzelnen Einträge:
  \[  a_{11}=1- 1=0, \quad a_{12}=1- 2=-1, \quad a_{13}=1- 3=-2 \quad \text{etc}.\]
  Also ist die Matrix 
  \[ A= \begin{pmatrix} 0 & -1 & -2 \\ 1 & 0 & -1\end{pmatrix}.\]
  \tab{
   \lang{de}{Lösung (c)}
   \lang{en}{Solution (c)}
  }
  Die Einträge der Matrix $A$ sind durch die Formel $a_{ij}=(i+1)\cdot (j-2)$ bestimmt.
  Dabei steht jeweils in der $i$-ten Zeile und $j$-ten Spalte der Matrix der Eintrag $a_{ij}$.
  Die Größe der Matrix ist in der Aufgabenstellung angegeben. Nun berechnet man explizit die Einträge, indem
  man in die Formel die verschiedenen Werte $1$, $2$ etc. für $i$ und $j$ einsetzt, und erhält für die einzelnen Einträge:
  \[  a_{11}=(1+1)\cdot (1-2)=-2, \quad a_{12}=(1+1)\cdot (2-2)=0  \quad \text{etc}.\]
  Also ist die Matrix 
  \[ A= \begin{pmatrix} -2 & 0 & 2 & 4\\ -3 & 0 & 3 & 6\\ -4 & 0 & 4 & 8\end{pmatrix}.\]
\end{tabs*}
 \end{content}
