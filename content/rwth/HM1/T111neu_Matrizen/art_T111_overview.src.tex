
%$Id:  $
\documentclass{mumie.article}
%$Id$
\begin{metainfo}
  \name{
    \lang{de}{Überblick: Matrizen}
    \lang{en}{Overview: Matrices}
  }
  \begin{description} 
 This work is licensed under the Creative Commons License Attribution 4.0 International (CC-BY 4.0)   
 https://creativecommons.org/licenses/by/4.0/legalcode 

    \lang{de}{Beschreibung}
    \lang{en}{Description}
  \end{description}
  \begin{components}
  \end{components}
  \begin{links}
  \link{generic_article}{content/rwth/HM1/T111neu_Matrizen/g_art_content_44_transponierte_matrix.meta.xml}{content_44_transponierte_matrix}
    \link{generic_article}{content/rwth/HM1/T111neu_Matrizen/g_art_content_43_matrizenmultiplikation.meta.xml}{content_43_matrizenmultiplikation}
    \link{generic_article}{content/rwth/HM1/T111neu_Matrizen/g_art_content_42_matrixaddition.meta.xml}{content_42_matrixaddition}
    \link{generic_article}{content/rwth/HM1/T111neu_Matrizen/g_art_content_39b_matrizen.meta.xml}{content_39b_matrizen}
    \link{generic_article}{content/rwth/HM1/T111neu_Matrizen/g_art_content_39_matrizen.meta.xml}{content_39_matrizen}
  \end{links}
  \creategeneric
\end{metainfo}
\begin{content}
\begin{block}[annotation]
	Im Ticket-System: \href{https://team.mumie.net/issues/30138}{Ticket 30138}
\end{block}
\begin{block}[annotation]
Copy of : /home/mumie/checkin/content/rwth/HM1/T112neu_Lineare_Gleichungssysteme/art_T112_overview.src.tex
\end{block}



\begin{block}[annotation]
Im Entstehen: Überblicksseite für Kapitel Matrizen
\end{block}

\usepackage{mumie.ombplus}
\ombchapter{1}
\title{\lang{de}{Überblick: Matrizen}\lang{en}{Overview: Matrices}}



\begin{block}[info-box]
\lang{de}{\strong{Inhalt}}
\lang{en}{\strong{Contents}}


\lang{de}{
    \begin{enumerate}%[arabic chapter-overview]
   \item[11.1] \link{content_39_matrizen}{Matrizen}
   \item[11.2] \link{content_42_matrixaddition}{Addition und Skalarmultiplikation von Matrizen}
   \item[11.3] \link{content_39b_matrizen}{Matrix-Vektor-Multiplikation}
   \item[11.4] \link{content_43_matrizenmultiplikation}{Matrizen-Multiplikation}
   \item[11.5] \link{content_44_transponierte_matrix}{Transponierte Matrix, symmetrische Matrix}
     \end{enumerate}
}
\lang{en}{
    \begin{enumerate}%[arabic chapter-overview]
   \item[11.1] \link{content_39_matrizen}{Matrices}
   \item[11.2] \link{content_42_matrixaddition}{Addition and scalar multiplication of matrices}
   \item[11.3] \link{content_39b_matrizen}{Matrix-vector multiplication}
   \item[11.4] \link{content_43_matrizenmultiplikation}{Matrix multiplication}
   \item[11.5] \link{content_44_transponierte_matrix}{Matrix transposition and symmetric matrices}
     \end{enumerate}
} %lang

\end{block}

\begin{zusammenfassung}
\lang{de}{
Wir lernen Matrizen als Zahlenschemata kennen und begreifen sie als Werkzeug, um verschiedene 
Problemstellungen einfacher zu lösen.
\\\\
Nach einer formalen Einführung führen wir Basisoperationen durch. Die Operanden sind dabei Skalare, 
Vektoren und Matrizen. Wir behalten bei der Ausführung stets die Rechenregeln im Auge.
\\\\
Zu den Operationen gehören die Addition zweier Matrizen, die Multiplikation von Skalaren mit 
Matrizen, sowie das Matrix-Vektor-Produkt. Schließlich bestimmen wir das Ergebnis der 
Matrixmultiplikation durch Skalarprodukte, und erhalten mit dem Falkschema zusätzlich eine visuelle 
Merkregel.
\\\\
Mit der Transponieren einer Matrix wird eine weitere Matrizenoperation eingeführt. Wie die 
transponierte Matrix zur Charakterisierung von Matrizen beitragen kann, wird exemplarisch anhand der 
symmetrischen Matrizen gezeigt.
}
\lang{en}{
In this chapter we introduce matrices as objects represented by tables/rectangular arrays with 
entries in the real numbers, and have important applications.
\\\\
We then provide an overview of the most important operations, which are used to combine scalars, 
matrices and vectors in various ways.
\\\\
The operations covered are addition between matrices, multiplication of matrices by scalars, and 
the matrix-vector product. We generalise the latter to matrix multiplication, and provide a method 
for evaluating the product of matrices.
\\\\
Finally, transposition is an operation performed on a single matrix. We can use this operation in 
the characterisation of matrices by defining symmetric matrices.
}
\end{zusammenfassung}

\begin{block}[info]\strong{\lang{de}{Lernziele}\lang{en}{Learning Goals}}
\begin{itemize}[square]
\item \lang{de}{Sie führen Basisoperationen auf Matrizen durch.}
      \lang{en}{Being able to evaluate basic matrix operations.}
\item \lang{de}{
      Sie addieren und multiplizieren Matrizen und überprüfen dabei die Kompatibilität der Operanden.
      }
      \lang{en}{
      Being able to add and multiply matrices and check that the operations are well-defined.
      }
\item \lang{de}{Sie berechnen Terme die Skalare und Matrizen als Operanden enthalten.}
      \lang{en}{Being able to evaluate expressions containing scalars and matrices.}
\item \lang{de}{Sie transponieren Matrizen.}
      \lang{en}{Being able to transpose matrices.}
\end{itemize}
\end{block}



\end{content}
