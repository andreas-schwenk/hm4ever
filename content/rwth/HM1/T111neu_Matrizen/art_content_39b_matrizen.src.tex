
%$Id:  $
\documentclass{mumie.article}
%$Id$
\begin{metainfo}
  \name{
    \lang{de}{Matrix-Vektor-Multiplikation}
    \lang{en}{Matrix-vector multiplication}
  }
  \begin{description} 
 This work is licensed under the Creative Commons License Attribution 4.0 International (CC-BY 4.0)   
 https://creativecommons.org/licenses/by/4.0/legalcode 

    \lang{de}{Beschreibung}
    \lang{en}{Description}
  \end{description}
  \begin{components}
    \component{generic_image}{content/rwth/HM1/images/g_img_00_Videobutton_schwarz.meta.xml}{00_Videobutton_schwarz}
   \component{generic_image}{content/rwth/HM1/images/g_img_00_video_button_schwarz-blau.meta.xml}{00_video_button_schwarz-blau}
   \component{js_lib}{system/media/mathlets/GWTGenericVisualization.meta.xml}{mathlet1}
  \end{components}
  \begin{links}
    \link{generic_article}{content/rwth/HM1/T111neu_Matrizen/g_art_content_43_matrizenmultiplikation.meta.xml}{content_43_matrizenmultiplikation}
    \link{generic_article}{content/rwth/HM1/T108_Vektorrechnung/g_art_content_27_vektoren.meta.xml}{vektorrechnung}
  \end{links}
  \creategeneric
\end{metainfo}
\begin{content}
\begin{block}[annotation]
	Im Ticket-System: \href{https://team.mumie.net/issues/21336}{Ticket 21336}
\end{block}
\begin{block}[annotation]
Copy of \href{https://team.mumie.net/issues/21314}{Ticket 21314}: content/rwth/HM1/T111_Matrizen,_lineare_Gleichungssysteme/art_content_39b_matrizen.src.tex
\end{block}

\usepackage{mumie.ombplus}
\ombchapter{11}
\ombarticle{3}
\usepackage{mumie.genericvisualization}

\begin{visualizationwrapper}

\title{\lang{de}{Matrix-Vektor-Multiplikation}\lang{en}{Matrix-vector multiplication}}
 
\begin{block}[annotation]
  übungsinhalt
  
\end{block}

\begin{block}[info-box]
\tableofcontents
\end{block}

\section{\lang{de}{Matrix-Vektor-Multiplikation}
         \lang{en}{Matrix-vector multiplication}}\label{sec:matrix-vektor-mult}

\lang{de}{
In diesem Abschnitt wird die Multiplikation einer Matrix mit einem Vektor behandelt.
Dies wird im \link{content_43_matrizenmultiplikation}{nächsten} Abschnitt
durch die Multiplikation zweier Matrizen verallgemeinert.
}
\lang{en}{
In this section we introduce multiplication of a matrix with a vector. This is generalised in a 
\link{content_43_matrizenmultiplikation}{following section} where multiplication between matrices 
is introduced.
}


\begin{definition}
\lang{de}{
Eine $(m\times n)$-Matrix $A=(a_{ij})$ kann mit einem Spaltenvektor 
$x=\left(  \begin{smallmatrix}    x_1 \\      x_2\\   \vdots\\ x_n  \end{smallmatrix} \right)$ 
der L"ange $n$ wie folgt multipliziert werden. Das Ergebnis ist dann ein Spaltenvektor der L"ange $m$:
}
\lang{en}{
A $(m\times n)$-matrix $A=(a_{ij})$ can be multiplied as follows with a column vector 
$x=\left(  \begin{smallmatrix}    x_1 \\      x_2\\   \vdots\\ x_n  \end{smallmatrix} \right)$ 
with $n$ entries. The result is a column vector with $m$ entries.
}

\[ A\cdot x=
\begin{pmatrix}
a_{11} & a_{12} & \cdots & a_{1n} \\
a_{21} & a_{22} & \cdots & a_{2n} \\
\vdots & \vdots & \ddots & \vdots \\
a_{m1} & a_{m2} & \cdots & a_{mn}
\end{pmatrix} \cdot  \begin{pmatrix}
                                     x_1 \\      x_2\\ \vdots\\  \vdots\\  x_n  \end{pmatrix} 
                                     = \begin{pmatrix}
                                     a_{11} x_1 + a_{12} x_2 + \cdots + a_{1n} x_n \\
                                     a_{21} x_1 + a_{22} x_2 + \cdots + a_{2n} x_n \\
                                     \vdots \\
                                     a_{m1} x_1 + a_{m2} x_2 + \cdots + a_{mn} x_n
                                     \end{pmatrix}
\]
\lang{de}{
Wie bei Multiplikationen üblich wird der Mal-Punkt oft weggelassen und $Ax$ statt $A\cdot x$ geschrieben.\\
\floatright{\href{https://api.stream24.net/vod/getVideo.php?id=10962-2-11010&mode=iframe&speed=true}{\image[75]{00_video_button_schwarz-blau}}}
\floatright{\href{https://www.hm-kompakt.de/video?watch=801}{\image[75]{00_Videobutton_schwarz}}}\\\\
}
\lang{en}{
As is often done with multiplication, we conventionally omit the multiplication symbol and write 
$Ax$ instead of $A\cdot x$.
}
\end{definition}

\begin{block}[warning]
\lang{de}{
Das Produkt einer Matrix mit einem Spaltenvektor ist nur definiert, wenn die Länge des Spaltenvektors 
mit der Anzahl der Spalten der Matrix übereinstimmt!
}
\lang{en}{
The product of a matrix with a column vector is only defined if the column vector has as many entries 
as the matrix has columns.
}
\end{block}


\begin{example}
\lang{de}{Wir betrachten}
\lang{en}{Consider}
\[ A= \begin{pmatrix} 2 & -3 & 1 \\ 0 & \frac{4}{3} & 5 \end{pmatrix} \quad 
\text{\lang{de}{und}\lang{en}{and}} 
\quad x=\begin{pmatrix} 1 \\ 3 \\ -2\end{pmatrix}.\]
\lang{de}{
$A$ ist eine $(2\times 3)$-Matrix und $x$ ist ein Spaltenvektor der Länge $3$. Seine Länge stimmt 
also mit der Spaltenzahl von $A$ überein, weshalb man das Produkt der Matrix $A$ mit dem 
Spaltenvektor $x$ bilden kann. Das Ergebnis ist der Spaltenvektor
}
\lang{en}{
$A$ is a $(2\times 3)$-matrix and $x$ is a column vector with $3$ entries. As the matrix has $3$ 
columns, the product of $A$ and $x$ is defined, giving
}
\[ A\cdot x= \begin{pmatrix} 2 & -3 & 1 \\ 0 & \frac{4}{3} & 5 \end{pmatrix} \cdot \begin{pmatrix} 1 \\ 3 \\ -2\end{pmatrix}
=\begin{pmatrix} 2\cdot 1+(-3)\cdot 3+1\cdot (-2) \\ 0\cdot 1+\frac{4}{3}\cdot 3 +5\cdot (-2)\end{pmatrix} =\begin{pmatrix} -9\\ -6\end{pmatrix}.
\]
\end{example}
\begin{quickcheck}

      \precision{3}
      \field{real}
      \begin{variables}
           \randint[Z]{a}{2}{4}
           \randint[Z]{b}{2}{5}
           \randint[Z]{m}{2}{4}
           \randint[Z]{n}{5}{6}
           \randint[Z]{k}{7}{8}
           \randint[Z]{l}{2}{4}
           \randint[Z]{o}{5}{6}
           \randint[Z]{p}{7}{8}
           \randint[Z]{l1}{3}{3}
           \function[expand]{l2}{o*a+l*b}
           var
       \end{variables}
      \text{\lang{de}{
      Für 
      $A=\begin{pmatrix} \var{n} & \var{m} \\ \var{o} &\var{l}\\ \var{k} & \var{p} \end{pmatrix}$ und 
      $x=\begin{pmatrix} \var{a} \\ \var{b} \end{pmatrix}$ hat der resultierende Vektor $Ax$ genau 
      \ansref Einträge. Der zweite Eintrag dieses Vektors lautet \ansref.
      }
      \lang{en}{
      Let 
      $A=\begin{pmatrix} \var{n} & \var{m} \\ \var{o} &\var{l}\\ \var{k} & \var{p} \end{pmatrix}$ and 
      $x=\begin{pmatrix} \var{a} \\ \var{b} \end{pmatrix}$. The vector $Ax$ has exactly \ansref 
      entries. The second entry of this vector is \ansref.
      }}
%      \explanation{}
      \begin{answer}
          \type{input.number}
            \solution{l1}
      \end{answer}
      \begin{answer}
       \type{input.function}
            \solution{l2}
      \end{answer}
\end{quickcheck}


\section{\lang{de}{Rechenregeln}
         \lang{en}{Rules for matrix-vector multiplication}}\label{sec:rechenregeln}

\lang{de}{
Die Matrix-Vektor-Multiplikation erfüllt die Rechenregeln, die man von Multiplikationen erwartet.
}
\lang{en}{
Matrix-vector multiplication satisfies the expected rules for a multiplication operation.
}

\begin{rule}
\lang{de}{
Mit der üblichen \ref[vektorrechnung][Addition und Skalarmultiplikation]{sec:rechnen} von 
Spaltenvektoren gelten für alle $(m\times n)$-Matrizen $A$, Spaltenvektoren $x$ und $y$ der Länge 
$n$ und reelle Zahlen $r\in \R$:
}
\lang{en}{
With the standard \ref[vektorrechnung][addition and scalar multiplication]{sec:rechnen} of 
column vectors, for all $(m\times n)$-matrices $A$, column vectors $x$ and $y$ with $n$ entries and 
real numbers $r\in \R$:
}
\begin{enumerate}
\item $\quad A\cdot (x+y)=Ax+Ay$,
\item $\quad A\cdot (rx)=r(Ax)$.
\end{enumerate}
\lang{de}{
\floatright{\href{https://api.stream24.net/vod/getVideo.php?id=10962-2-11011&mode=iframe&speed=true}{\image[75]{00_video_button_schwarz-blau}}}
\floatright{\href{https://www.hm-kompakt.de/video?watch=804}{\image[75]{00_Videobutton_schwarz}}}\\\\
}
\lang{en}{}
\end{rule}

\begin{example}
\lang{de}{Wir betrachten}
\lang{en}{Consider}
\[ A= \begin{pmatrix} 2 & -3 & 1 \\ 0 & \frac{4}{3} & 5 \end{pmatrix}, \quad x=\begin{pmatrix} 1 \\ 3 \\ -2\end{pmatrix}\quad
 \text{\lang{de}{und}\lang{en}{and}} \quad y=\begin{pmatrix} 2 \\ -3 \\ 1\end{pmatrix}.\]
 \lang{de}{Dann sind}
 \lang{en}{Then}
 \begin{eqnarray*} A(x+y) &=& \begin{pmatrix} 2 & -3 & 1 \\ 0 & \frac{4}{3} & 5 \end{pmatrix}\cdot \begin{pmatrix} 1+2 \\ 3-3 \\ -2+1\end{pmatrix}\\
 &=& \begin{pmatrix} 2\cdot 3+(-3)\cdot 0+1\cdot (-1) \\ 0\cdot 3+\frac{4}{3} \cdot 0+5\cdot (-1) \end{pmatrix} =\begin{pmatrix} 5\\ -5 \end{pmatrix}
 \end{eqnarray*}
 \lang{de}{und}
 \lang{en}{and}
  \begin{eqnarray*} Ax+Ay &=& \begin{pmatrix} 2\cdot 1 + (-3)\cdot 3 + 1\cdot (-2) \\ 0\cdot 1 + \frac{4}{3} \cdot 3 + 5\cdot (-2) \end{pmatrix}
   +\begin{pmatrix} 2\cdot 2 + (-3)\cdot (-3) + 1\cdot 1 \\ 0\cdot 2 + \frac{4}{3}\cdot (-3)+ 5\cdot 1 \end{pmatrix} \\
 &=& \begin{pmatrix} -9\\ -6 \end{pmatrix}+ \begin{pmatrix} 14 \\ 1\end{pmatrix} = \begin{pmatrix} 5\\ -5 \end{pmatrix}.
 \end{eqnarray*}
 \end{example}
\end{visualizationwrapper}
\end{content}

