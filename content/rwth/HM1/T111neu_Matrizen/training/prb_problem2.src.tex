\documentclass{mumie.problem.gwtmathlet}
%$Id$
\begin{metainfo}
  \name{
    \lang{de}{A02: Matrix-Vektor-Multiplikation}
    \lang{en}{}
  }
  \begin{description} 
 This work is licensed under the Creative Commons License Attribution 4.0 International (CC-BY 4.0)   
 https://creativecommons.org/licenses/by/4.0/legalcode 

    \lang{de}{Beschreibung}
    \lang{en}{}
  \end{description}
  \corrector{system/problem/GenericCorrector.meta.xml}
  \begin{components}
    \component{js_lib}{system/problem/GenericMathlet.meta.xml}{gwtmathlet}
  \end{components}
  \begin{links}
  \end{links}
  \creategeneric
\end{metainfo}
\begin{content}
\begin{block}[annotation]
	Im Ticket-System: \href{https://team.mumie.net/issues/21346}{Ticket 21346}
\end{block}
\begin{block}[annotation]
Copy of \href{http://team.mumie.net/issues/9581}{Ticket 9581}: content/rwth/HM1/T111_Matrizen,_lineare_Gleichungssysteme/training/prb_problem4.src.tex
\end{block}

\usepackage{mumie.genericproblem}

\lang{de}{
	\title{A02: Matrix-Vektor-Multiplikation}
}
\lang{en}{
	\title{Problem 2}
}





\begin{problem}


\begin{question}

	
\begin{variables}
	\randint[Z]{aa}{-8}{8}
	\randint[Z]{ab}{-8}{8}
	\randint[Z]{ac}{-8}{8}
	\randint[Z]{ad}{-8}{8}
	
	\randint[Z]{xa}{-8}{8}
	\randint[Z]{xb}{-8}{8}
	
	\function[calculate]{sa}{aa*xa+ab*xb}
	\function[calculate]{sb}{ac*xa+ad*xb}
	
\end{variables}
\lang{de}{\explanation{
        Führen Sie eine Matrix-Vektor-Multiplikation durch:
        Da der Vektor $x$ genau so viele Einträge hat,
        wie die Matrix $A$ Spalten besitzt,
        ist die Multiplikation erlaubt.
        Der Zielvektor besteht aus zwei Einträgen,
        wobei für Eintrag $b_i$ ($i \in \{1, 2\}$) jeweils das Skalarprodukt
        aus der $i$-ten Zeile der Matrix und dem Vektor
        $x$ bestimmt wird.
    }}

	\type{input.number}
	\field{real} 
	\precision{1}
    \lang{de}{
	    \text{
	    Es sei $A=\begin{pmatrix}
           \var{aa}&\var{ab}\\\var{ac}&\var{ad}
          \end{pmatrix}$ und $x=\begin{pmatrix}\var{xa}\\\var{xb}\end{pmatrix}$.\\
	    Bestimmen Sie $b=\begin{pmatrix}b_{1}\\b_{2}\end{pmatrix}$, sodass $A \cdot x = b$ erfüllt ist.
    }}
    
    \begin{answer}
	    \text{$b_{1} =$}
	    \solution{sa}
    \end{answer}
        \begin{answer}
	    \text{$b_{2} =$}
	    \solution{sb}
    \end{answer}
    
    
\end{question}
    

\end{problem}


\embedmathlet{gwtmathlet}

\end{content}