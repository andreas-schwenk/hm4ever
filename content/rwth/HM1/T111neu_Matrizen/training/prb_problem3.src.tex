\documentclass{mumie.problem.gwtmathlet}
%$Id$
\begin{metainfo}
  \name{
    \lang{de}{A03: Matrixmultiplikation}
    \lang{en}{}
  }
  \begin{description} 
 This work is licensed under the Creative Commons License Attribution 4.0 International (CC-BY 4.0)   
 https://creativecommons.org/licenses/by/4.0/legalcode 

    \lang{de}{Beschreibung}
    \lang{en}{}
  \end{description}
  \corrector{system/problem/GenericCorrector.meta.xml}
  \begin{components}
    \component{js_lib}{system/problem/GenericMathlet.meta.xml}{gwtmathlet}
  \end{components}
  \begin{links}
  \end{links}
  \creategeneric
\end{metainfo}
\begin{content}
\begin{block}[annotation]
	Im Ticket-System: \href{https://team.mumie.net/issues/21343}{Ticket 21343}
\end{block}
\begin{block}[annotation]
Copy of \href{http://team.mumie.net/issues/9670}{Ticket 9670}: content/rwth/HM1/T112_Rechnen_mit_Matrizen/training/prb_problem3.src.tex
\end{block}

\usepackage{mumie.genericproblem}

\lang{de}{
	\title{A03: Matrixmultiplikation}
}




\begin{problem}

    
\begin{question}

	\begin{variables}
		\randint[Z]{a}{-9}{9}
		\randint[Z]{b}{-9}{9}
		\randint[Z]{c}{-9}{9}
		\randint[Z]{d}{-9}{9}
		
		\randint[Z]{ee}{-9}{9}
		\randint[Z]{f}{-9}{9}
		\randint[Z]{g}{-9}{9}
		\randint[Z]{h}{-9}{9}
		
			\matrix[calculate]{aa}{
  			a & b \\ 
  			c & d
      	}
      	
      	\matrix[calculate]{bb}{
  			ee & f \\ 
  			g & h
      	}
		
		\matrix[calculate]{m}{
		a*ee+b*g & a*f+b*h \\
		c*ee+d*g & c*f+d*h
      } 
	\end{variables}

	\type{input.matrix}
    \field{rational}
    
    \lang{de}{
	    \text{
	    Bestimmen Sie die folgende Matrixmultiplikation:\\
$\var{aa}\cdot\var{bb}$}}
    
    \begin{answer}
	    \solution{m}
      \format{2}{2}
      \explanation{
            Die Matrixmultiplikation zweier quadratischer Matrizen,
            also Matrizen mit gleicher Zeilen- und Spaltenanzahl ist 
            immer erlaubt. Die Ergebnismatrix hat dann ebensoviele
            Zeilen und Spalten.
            Man bestimmt die Ergebnismatrix, indem man für den
            $(i,j)$-ten Eintrag das Skalarprodukt aus der
            $i$-ten Zeile der ersten Matrix und der $j$-ten
            Spalte der zweiten Matrix bestimmt.
      }
	\end{answer}
    
\end{question}

\begin{question}

	\begin{variables}
		\randint[Z]{x11}{-5}{5}
		\randint[Z]{x12}{-5}{5}
		\randint[Z]{x13}{-5}{5}
		\randint[Z]{x21}{-5}{5}
		\randint[Z]{x22}{-5}{5}
		\randint[Z]{x23}{-5}{5}
		\randint[Z]{x31}{-5}{5}
		\randint[Z]{x32}{-5}{5}
		\randint[Z]{x33}{-5}{5}
	
		\randint[Z]{y11}{-5}{5}
		\randint[Z]{y12}{-5}{5}
		\randint[Z]{y13}{-5}{5}
		\randint[Z]{y21}{-5}{5}
		\randint[Z]{y22}{-5}{5}
		\randint[Z]{y23}{-5}{5}
		\randint[Z]{y31}{-5}{5}
		\randint[Z]{y32}{-5}{5}
		\randint[Z]{y33}{-5}{5}
		
			\matrix[calculate]{a}{
  			x11 & x12 & x13 \\ 
  			x21 & x22 & x23 \\
  			x31 & x32 & x33
      	}
      		\matrix[calculate]{b}{
  			y11 & y12 & y13 \\ 
  			y21 & y22 & y23 \\
  			y31 & y32 & y33
      	}
      		\matrix[calculate]{c}{
  			x11*y11 +x12*y21 + x13*y31 & x11*y12 +x12*y22+x13*y32 & x11*y13 +x12*y23+x13*y33\\ 
  x21*y11 +x22*y21 + x23*y31 & x21*y12 +x22*y22 + x23*y32 & x21*y13 +x22*y23 + x23*y33\\
  x31*y11 +x32*y21 + x33*y31 & x31*y12 +x32*y22 + x33*y32 & x31*y13 +x32*y23 + x33*y33
      	}
	\end{variables}

	\type{input.matrix}
    \field{rational}
    
    \lang{de}{
	    \text{Bestimmen Sie die folgende Matrixmultiplikation:\\
$\var{a}\cdot\var{b}$}}
    
    \begin{answer}
	    \solution{c}
      \format{3}{3}
      \explanation{
            Die Matrixmultiplikation zweier quadratischer Matrizen,
            also Matrizen mit gleicher Zeilen- und Spaltenanzahl ist 
            immer erlaubt. Die Ergebnismatrix hat dann ebensoviele
            Zeilen und Spalten.
            Man bestimmt die Ergebnismatrix, indem man für den
            $(i,j)$-ten Eintrag das Skalarprodukt aus der
            $i$-ten Zeile der ersten Matrix und der $j$-ten
            Spalte der zweiten Matrix bestimmt.
      }
	\end{answer}
    
\end{question}

\begin{question}

	\begin{variables}
		\randint[Z]{x11}{-5}{5}
		\randint[Z]{x12}{-5}{5}
		\randint[Z]{x21}{-5}{5}
		\randint[Z]{x22}{-5}{5}
		\randint[Z]{x31}{-5}{5}
		\randint[Z]{x32}{-5}{5}
	
		\randint[Z]{y11}{-5}{5}
		\randint[Z]{y12}{-5}{5}
		\randint[Z]{y13}{-5}{5}
		\randint[Z]{y21}{-5}{5}
		\randint[Z]{y22}{-5}{5}
		\randint[Z]{y23}{-5}{5}
		
			\matrix[calculate]{a}{
  			x11 & x12  \\ 
  			x21 & x22  \\
  			x31 & x32 
      	}
      		\matrix[calculate]{b}{
  			y11 & y12 & y13 \\ 
  			y21 & y22 & y23
      	}
      		\matrix[calculate]{c}{
  			x11*y11 +x12*y21  & x11*y12 +x12*y22 & x11*y13 +x12*y23\\ 
  x21*y11 +x22*y21 & x21*y12 +x22*y22 & x21*y13 +x22*y23\\
  x31*y11 +x32*y21 & x31*y12 +x32*y22 & x31*y13 +x32*y23	}
	\end{variables}

	\type{input.matrix}
    \field{rational}
    
    \lang{de}{
	    \text{Bestimmen Sie die folgende Matrixmultiplikation:\\
$\var{a}\cdot\var{b}$}}
    
    \begin{answer}
	    \solution{c}
        \explanation{
            Die Matrixmultiplikation ist dann erlaubt,
            wenn die Anzahl der Spalten der ersten Matrix
            mit der Anzahl der Zeilen der zweiten Matrix
            übereinstimmt. Dies ist hier gegeben.
            Nun muss das Format bestimmt werden:
            Die Anzahl der Zeilen der Ergebnismatrix
            entspricht der Anzahl der Zeilen der ersten Matrix.
            Die Anzahl der Spalten der Ergebnismatrix
            entspricht der Anzahl der Zeilen der zweiten Matrix.
            Man bestimmt die Komponenten der Ergebnismatrix,
            indem man für den $(i,j)$-ten Eintrag das Skalarprodukt 
            aus der $i$-ten Zeile der ersten Matrix und der $j$-ten
            Spalte der zweiten Matrix bestimmt.
        }
	\end{answer}
    
\end{question}

\end{problem}


\embedmathlet{gwtmathlet}

\end{content}