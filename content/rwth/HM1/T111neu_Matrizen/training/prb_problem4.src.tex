\documentclass{mumie.problem.gwtmathlet}
%$Id$
\begin{metainfo}
  \name{
    \lang{de}{A04: Matrizenoperation}
    \lang{en}{}
  }
  \begin{description} 
 This work is licensed under the Creative Commons License Attribution 4.0 International (CC-BY 4.0)   
 https://creativecommons.org/licenses/by/4.0/legalcode 

    \lang{de}{Beschreibung}
    \lang{en}{}
  \end{description}
  \corrector{system/problem/GenericCorrector.meta.xml}
  \begin{components}
    \component{js_lib}{system/problem/GenericMathlet.meta.xml}{gwtmathlet}
  \end{components}
  \begin{links}
  \end{links}
  \creategeneric
\end{metainfo}
\begin{content}
\usepackage{mumie.genericproblem}

\lang{de}{
	\title{A04: Matrizenoperation}
}
\begin{block}[annotation]
	Im Ticket-System: \href{https://team.mumie.net/issues/21347}{Ticket 21347}
\end{block}
\begin{block}[annotation]
Copy of \href{http://team.mumie.net/issues/9671}{Ticket 9671}: content/rwth/HM1/T112_Rechnen_mit_Matrizen/training/prb_problem4.src.tex
\end{block}



\begin{problem}

\begin{question}

	\begin{variables}
		\randint{a}{-3}{3}
		\randint{b}{-3}{3}
		\randint{c}{-3}{3}
		\randint{d}{-3}{3}
		
		\randint{ee}{-3}{3}
		\randint{f}{-3}{3}
		\randint{g}{-3}{3}
		\randint{h}{-3}{3}
		
		\randint{u}{-3}{3}
		\randint{v}{-3}{3}
		\randint{w}{-3}{3}
		\randint{x}{-3}{3}
		
		\randint{y}{-3}{3}
		\randint{z}{-3}{3}
		
		\randint{r}{2}{4}
		
		\matrix[calculate]{aa}{
  			a & b \\ 
  			c & d
      	}
      	
      	\matrix[calculate]{bb}{
  			ee & f \\ 
  			g & h
      	}
		
		\matrix[calculate]{zz}{
			u & v\\
			w & x
		}
			
		\matrix[calculate]{uu}{
			y\\
			z
		}	
		
		\matrix[calculate]{m}{
  			((a+ee*r)*u+(b+f*r)*w)*y+((a+ee*r)*v+(b+f*r)*x)*z\\
  			((c+g*r)*u+(d+h*r)*w)*y+((c+g*r)*v+(d+h*r)*x)*z
      	} 
		
		
		
	\end{variables}

	\type{input.matrix}
	
    \field{rational}
    
    \lang{de}{
	    \text{Bestimmen Sie:\\$
\left(\var{aa}+\var{r}\cdot\var{bb}\right)\cdot\var{zz}\cdot\var{uu}$}}
    
    \begin{answer}
	    \solution{m}
      
      \explanation{
            In dieser Aufgabe werden die Operationen Matrixmultiplikation,
            Matrixaddition und die Skalarmultiplikation durchgeführt.
            Zunächst berechnet man den Term in den Klammern
            also die skalare Multiplikation, sowie die Summe.
            Man erhält eine $2 \times 2$-Matrix.
            Danach kann die Matrixmultiplikation mit den verbleibenden
            beiden Matrizen nacheinander durchgeführt werden.
            Man erhält zunächst eine $2 \times 2$-Matrix und als
            Ergebnis eine eine $2 \times 1$-Matrix.
            Da die Matrixmultiplikation assoziativ ist, 
            darf man alternativ auch zunächst die beiden hinteren
            Matrizen miteinander multiplizieren, denn: 
            $(A \cdot B) \cdot C = A \cdot (B \cdot C)$.
            Man beachte jedoch stets, dass die Matrixmultiplikation nicht 
            kommutativ ist. Im Allgemeinen gilt: $A \cdot B \neq B \cdot A$.
      }
	\end{answer}
    
\end{question}

\end{problem}



\embedmathlet{gwtmathlet}

\end{content}

