\documentclass{mumie.problem.gwtmathlet}
%$Id$
\begin{metainfo}
  \name{
    \lang{de}{A01: Matrixaddition}
    \lang{en}{}
  }
  \begin{description} 
 This work is licensed under the Creative Commons License Attribution 4.0 International (CC-BY 4.0)   
 https://creativecommons.org/licenses/by/4.0/legalcode 

    \lang{de}{Beschreibung}
    \lang{en}{}
  \end{description}
  \corrector{system/problem/GenericCorrector.meta.xml}
  \begin{components}
    \component{js_lib}{system/problem/GenericMathlet.meta.xml}{gwtmathlet}
  \end{components}
  \begin{links}
  \end{links}
  \creategeneric
\end{metainfo}
\begin{content}
\begin{block}[annotation]
	Im Ticket-System: \href{https://team.mumie.net/issues/21344}{Ticket 21344}
\end{block}
\begin{block}[annotation]
Copy of \href{http://team.mumie.net/issues/9668}{Ticket 9668}: content/rwth/HM1/T112_Rechnen_mit_Matrizen/training/prb_problem1.src.tex
\end{block}

\usepackage{mumie.genericproblem}

\lang{de}{
	\title{A01: Matrixaddition}
}


\lang{de}{Bestimmen Sie die folgenden Summen von Matrizen:}

\begin{problem}
    
\begin{question}

	\begin{variables}
		\randint[Z]{a}{-9}{9}
		\randint[Z]{b}{-9}{9}
		\randint[Z]{c}{-9}{9}
		\randint[Z]{d}{-9}{9}
		
		\randint[Z]{ee}{-9}{9}
		\randint[Z]{f}{-9}{9}
		\randint[Z]{g}{-9}{9}
		\randint[Z]{h}{-9}{9}
		
			\matrix[calculate]{aa}{
  			a & b \\ 
  			c & d
      	}
      	
      	\matrix[calculate]{bb}{
  			ee & f \\ 
  			g & h
      	}
		
		\matrix[calculate]{m}{
  a+ee & b+f \\ 
  c+g & d+h
      } 
	\end{variables}

	\type{input.matrix}
	\displayprecision{3}
    \correctorprecision{2}
    \field{integer}
    
    \lang{de}{
	    \text{
	    
$\var{aa}+\var{bb}$}}
    
    \begin{answer}
	    \solution{m}
      \format{2}{2}
      \explanation{
              Beide Matrizen haben dasselbe Format, 
            also die gleiche Anzahl an Zeilen und Spalten. 
            Folglich ist die Matrizenaddition erlaubt. 
            Die Ergebnismatrix hat ebenso dieses Format.
            Man bestimmt die Ergebnismatrix, 
            indem man die Summe komponentenweise bestimmt:
            Der $(i,j)$-te Koeffizient der Summe ist
            gleich der Summe der $(i,j)$-ten Koeffizienten
            der beiden zu addierenden Matrizen.} 
      
	\end{answer}
    
\end{question}



\begin{question}

	\begin{variables}
		\randint[Z]{x11}{-9}{9}
		\randint[Z]{x12}{-9}{9}
		\randint[Z]{x13}{-9}{9}
		\randint[Z]{x21}{-9}{9}
		\randint[Z]{x22}{-9}{9}
		\randint[Z]{x23}{-9}{9}
		\randint[Z]{x31}{-9}{9}
		\randint[Z]{x32}{-9}{9}
		\randint[Z]{x33}{-9}{9}
	
		\randint[Z]{y11}{-9}{9}
		\randint[Z]{y12}{-9}{9}
		\randint[Z]{y13}{-9}{9}
		\randint[Z]{y21}{-9}{9}
		\randint[Z]{y22}{-9}{9}
		\randint[Z]{y23}{-9}{9}
		\randint[Z]{y31}{-9}{9}
		\randint[Z]{y32}{-9}{9}
		\randint[Z]{y33}{-9}{9}
		
			\matrix[calculate]{a}{
  			x11 & x12 & x13 \\ 
  			x21 & x22 & x23 \\
  			x31 & x32 & x33
      	}
      		\matrix[calculate]{b}{
  			y11 & y12 & y13 \\ 
  			y21 & y22 & y23 \\
  			y31 & y32 & y33
      	}
      		\matrix[calculate]{c}{
  			x11+y11 & x12+y12 & x13+y13 \\ 
  			x21+y21 & x22+y22 & x23+y23 \\
  			x31+y31 & x32+y32 & x33+y33
      	}
	\end{variables}

	\type{input.matrix}
	\displayprecision{3}
    \correctorprecision{2}
    \field{integer}
    
    \lang{de}{
	    \text{
	
$\var{a}+\var{b}$}}
    
    \begin{answer}
	    \solution{c}
      \format{3}{3}
      \explanation{
            Beide Matrizen haben dasselbe Format, 
            also die gleiche Anzahl an Zeilen und Spalten. 
            Folglich ist die Matrizenaddition erlaubt. 
            Die Ergebnismatrix hat ebenso dieses Format.
            Man bestimmt die Ergebnismatrix, 
            indem man die Summe komponentenweise bestimmt:
            Der $(i,j)$-te Koeffizient der Summe ist
            gleich der Summe der $(i,j)$-ten Koeffizienten
            der beiden zu addierenden Matrizen.
      }
	\end{answer}
    
\end{question}


\begin{question}

	\begin{variables}
		\randint[Z]{a}{-9}{9}
		\randint[Z]{b}{-9}{9}
		\randint[Z]{c}{-9}{9}
		\randint[Z]{d}{-9}{9}
		
		\randint[Z]{ee}{-9}{9}
		\randint[Z]{f}{-9}{9}
		\randint[Z]{g}{-9}{9}
		\randint[Z]{h}{-9}{9}
		
		
		\randint[Z]{r}{-9}{-1}
		
		\matrix[calculate]{aa}{
  			a & b \\ 
  			c & d
      	}
      	
      	\matrix[calculate]{bb}{
  			ee & f \\ 
  			g & h
      	}
		
		\matrix[calculate]{m}{
  			a+r*ee & b+r*f \\ 
  			c+r*g & d+r*h
      	} 
	\end{variables}

	\type{input.matrix}
	\displayprecision{3}
    \correctorprecision{2}
    \field{integer}
    
    \lang{de}{
	    \text{$\var{aa}+(\var{r})\cdot\var{bb}$}}
    
    \begin{answer}
	    \solution{m}
      \format{2}{2}
      \explanation{
            Zunächst wendet man die Skalarmultiplikation
            für die zweite Matrix an, indem man jede Komponente
            dieser Matrix mit dem Skalar multipliziert.
            Nun verbleibt die Addition zweier Matrizen:
            Beide Matrizen haben dasselbe Format, 
            also die gleiche Anzahl an Zeilen und Spalten. 
            Folglich ist die Matrizenaddition erlaubt. 
            Die Ergebnismatrix hat ebenso dieses Format.
            Man bestimmt die Ergebnismatrix, 
            indem man die Summe komponentenweise bestimmt:
            Der $(i,j)$-te Koeffizient der Summe ist
            gleich der Summe der $(i,j)$-ten Koeffizienten
            der beiden zu addierenden Matrizen.
      }
	\end{answer}
    
\end{question}



\end{problem}


\embedmathlet{gwtmathlet}

\end{content}