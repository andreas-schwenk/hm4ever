%$Id:  $
\documentclass{mumie.article}
%$Id$
\begin{metainfo}
  \name{
    \lang{de}{Definition einer Matrix}
    \lang{en}{Definition of a matrix}
  }
  \begin{description} 
 This work is licensed under the Creative Commons License Attribution 4.0 International (CC-BY 4.0)   
 https://creativecommons.org/licenses/by/4.0/legalcode 

    \lang{de}{Beschreibung}
    \lang{en}{Description}
  \end{description}
  \begin{components}
    \component{generic_image}{content/rwth/HM1/images/g_img_00_Videobutton_schwarz.meta.xml}{00_Videobutton_schwarz}
    \component{generic_image}{content/rwth/HM1/images/g_img_00_video_button_schwarz-blau.meta.xml}{00_video_button_schwarz-blau}
    \component{js_lib}{system/media/mathlets/GWTGenericVisualization.meta.xml}{mathlet1}
  \end{components}
  \begin{links}
    \link{generic_article}{content/rwth/HM1/T108_Vektorrechnung/g_art_content_27_vektoren.meta.xml}{vektorrechnung}
  \end{links}
  \creategeneric
\end{metainfo}
\begin{content}
\begin{block}[annotation]
	Im Ticket-System: \href{https://team.mumie.net/issues/21335}{Ticket 21335}
\end{block}
\begin{block}[annotation]
Copy of \href{http://team.mumie.net/issues/9056}{Ticket 9056}: content/rwth/HM1/T111_Matrizen,_lineare_Gleichungssysteme/art_content_39_matrizen.src.tex
\end{block}

\usepackage{mumie.ombplus}
\ombchapter{11}
\ombarticle{1}
\usepackage{mumie.genericvisualization}

\begin{visualizationwrapper}

\title{\lang{de}{Definition einer Matrix}\lang{en}{Definition of a matrix}}
 
\begin{block}[annotation]
  übungsinhalt
  
\end{block}
\begin{block}[annotation]
  Im Ticket-System: \href{http://team.mumie.net/issues/9056}{Ticket 9056}\\
\end{block}

\lang{de}{
Eine Matrix beschreibt ein Zahlenschema, anhand dessen sich verschiedene Problemstellungen einfacher 
lösen lassen. 
\\\\
In diesem Teilkapitel werden Matrizen formal eingeführt. In den nachfolgenden Kapiteln werden 
grundlegende Rechenoperationen mit Matrizen durchgeführt.
}
\lang{en}{
A matrix is a mathematical object represented as a table/rectangular array of numbers, whose 
properties make it well-suited as a tool for solving certain problems.
\\\\
In this chapter we formally introduce matrices, their properties and matrix operations.
}

\begin{definition}[\lang{de}{Matrix}\lang{en}{Matrix}]
\lang{de}{
Sind $m, n \in \N$ natürliche Zahlen und sind f\"ur alle $i, j \in \N$ mit $1 \leq i \leq m$ und 
$1 \leq j \leq n$ Zahlen $a_{ij} \in \R$ gegeben, so nennt man das Zahlenschema
}
\lang{en}{
Let $m, n \in \N$ be natural numbers and let $a_{ij} \in \R$ be real numbers for all 
$i, j \in \N$ with $1 \leq i \leq m$ and $1 \leq j \leq n$. We then call the table
}
\[ 
\begin{pmatrix}
a_{11} & a_{12} & a_{13} & \cdots & a_{1n} \\
a_{21} & a_{22} & a_{23} & \cdots & a_{2n} \\
\vdots  & \vdots  & \vdots  & \ddots & \vdots  \\
a_{m1} & a_{m2} & a_{m3} & \cdots & a_{mn}  
\end{pmatrix} = \left( a_{ij} \right)_{1 \leq i \leq m, 1 \leq j \leq n} =: A 
\]
\lang{de}{
eine Matrix \"uber $\R$ mit $m$ \textbf{Zeilen} und $n$ \textbf{Spalten}, oder kurz eine
\emph{$ \left( m \times n \right)$-Matrix} \"uber $\R$. 
\\\\
Die Einträge der Matrix nennt man die \textbf{Koeffizienten} der Matrix, genauer ist
$a_{ij}$ der Koeffizient der Matrix der $i$-ten Zeile und $j$-ten Spalte, oder kurz der 
\textbf{$(i,j)$-te Koeffizient} der Matrix.
\\\\
Die Menge aller reellen $(m\times n)$-Matrizen wird mit $M(m,n;\R)$ bezeichnet.\\
\floatright{\href{https://api.stream24.net/vod/getVideo.php?id=10962-2-11009&mode=iframe&speed=true}{\image[75]{00_video_button_schwarz-blau}}}
\floatright{\href{https://www.hm-kompakt.de/video?watch=800}{\image[75]{00_Videobutton_schwarz}}}\\\\
}
\lang{en}{
a matrix over $\R$ with $m$ \textbf{rows} and $n$ \textbf{columns}, or an 
\emph{$ \left( m \times n \right)$-matrix} over $\R$.
\\\\
The entries of the matrix are also called its \textbf{elements}, and $a_{ij}$ is the entry in the 
$i$th row and $j$th column of the matrix, or the \textbf{$(i,j)$-th entry}.
\\\\
The set of all real $(m\times n)$-matrices is denoted by $M(m,n;\R)$.\\
}
\end{definition}

\begin{remark}
\begin{enumerate}
\item \lang{de}{
      In obiger Darstellung sind die Koeffizienten $a$ mit zwei Indizes $i$ und $j$ versehen. Um 
      Verwechslungen bei großen Matrizen zu vermeiden, sollte man die Indizes dort mit einem Komma 
      trennen: $a_{i,j}$. Beispielsweise wird sonst nicht deutlich, ob $a_{1,12}$ oder $a_{11,2}$ 
      gemeint ist, was ohne Komma beides $a_{112}$ wäre.\\
      Wenn es jedoch keine Verwechslungsmöglichkeit gibt, wird auf das Komma verzichtet.
      }
      \lang{en}{
      The above notation uses two indices $i$ and $j$ to access the coefficients $a$. To avoid 
      confusion in large matrices with non-single-digit indices, these indices can be seperated 
      by a comma: $a_{i,j}$. For example, instead of writing the ambiguous $a_{112}$, we should 
      write $a_{1,12}$ or $a_{11,2}$.\\
      If there is any chance of ambiguity, such a comma is used.
      }
\item \lang{de}{
      Für die Menge der reellen $(m\times n)$-Matrizen gibt es verschiedene Bezeichnungen. Die 
      gebräuchlichsten anderen Bezeichnungen sind
      }
      \lang{en}{
      The set of real $(m\times n)$-matrices has various notations. The most commonly used are
      }
      \[\R^{(m \times n)}, \quad \text{Mat}(m,n;\R) \quad 
        \text{\lang{de}{und}\lang{en}{and}} \quad  M_{m,n}(\R). \]
\end{enumerate}
\end{remark}

\begin{remark}\label{rem:matrix}
\begin{enumerate}
\item \lang{de}{
      Eine $(1\times n)$-Matrix (eine Matrix mit nur einer Zeile) nennt man auch 
      \textbf{Zeilenvektor der L"ange $n$}.
      }
      \lang{en}{
      A $(1\times n)$-matrix (a matrix with only one row) is also called a \textbf{row vector with 
      $n$ entries}.
      }
\item \lang{de}{
      Eine $(m\times 1)$-Matrix (eine Matrix mit nur einer Spalte) nennt man auch 
      \textbf{Spaltenvektor der L"ange $m$}.
      }
      \lang{en}{
      A $(m\times 1)$-matrix (a matrix with only one column) is also called a \textbf{column vector 
      with $n$ entries}
      }
\item \lang{de}{
      Eine $(n\times n)$-Matrix (eine Matrix mit gleich vielen Zeilen wie Spalten) nennt man auch 
      eine \textbf{quadratische} Matrix der Größe $n$.
      }
      \lang{en}{
      A $(n\times n)$-matrix (a matrix with the same amount of rows as columns) is also called a 
      \textbf{square} matrix of order $n$.
      }
\end{enumerate}
\end{remark}

\begin{example}
\lang{de}{Die Matrix}
\lang{en}{The matrix}
\[ A= \begin{pmatrix} 2 & -3 & 1 \\ 0 & 4/3 & 5 \end{pmatrix} \]
\lang{de}{
ist eine $(2\times 3)$-Matrix.\\
Der Koeffizient an der Stelle $(1, 3)$ lautet $a_{13}=1$.\\
Die erste Zeile der Matrix lautet $\begin{pmatrix} 2 & -3 & 1 \end{pmatrix}$ und die erste Spalte 
$\begin{pmatrix} 2\\ 0  \end{pmatrix}$.
}
\lang{en}{
is a $(2\times 3)$-matrix.\\
The entry at $(1, 3)$ is $a_{13}=1$.\\
The first row of the matrix is $\begin{pmatrix} 2 & -3 & 1 \end{pmatrix}$ and the first column is 
$\begin{pmatrix} 2\\ 0  \end{pmatrix}$.
}
\end{example}

\begin{quickcheck}
    \type{input.number}
      \precision{3}
      \field{real}
      \begin{variables}
           \randint[Z]{a}{3}{3}
           \randint[Z]{b}{2}{2}
           \randint[Z]{m}{2}{4}
           \randint[Z]{n}{5}{6}
           \randint[Z]{k}{7}{8}
           \randint[Z]{l}{2}{4}
           \randint[Z]{o}{5}{6}
           \randint[Z]{p}{7}{8}
       \end{variables}
      \text{\lang{de}{
      Bei der Matrix 
      $A=\begin{pmatrix} \var{m} & \var{n} \\ \var{k} &\var{l}\\ \var{o} & \var{p} \end{pmatrix}$
      handelt es sich um eine\\
      $\Big($\ansref $\times$ \ansref$\Big)$-Matrix.\\
      Der Koeffizient an der Stelle $(2, 1)$ lautet $a_{21}=$\ansref.
      }
      \lang{en}{
      The matrix 
      $A=\begin{pmatrix} \var{m} & \var{n} \\ \var{k} &\var{l}\\ \var{o} & \var{p} \end{pmatrix}$ 
      is a\\
      $\Big($\ansref $\times$ \ansref$\Big)$-matrix.\\
      The entry at $(2, 1)$ is $a_{21}=$\ansref.
      }}
%      \explanation{}
      \begin{answer}
            \solution{a}
      \end{answer}
       \begin{answer}
            \solution{b}
      \end{answer}
      \begin{answer}
            \solution{k}
      \end{answer}
\end{quickcheck}


\end{visualizationwrapper}


\end{content}