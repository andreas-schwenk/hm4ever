%$Id:  $
\documentclass{mumie.article}
%$Id$
\begin{metainfo}
  \name{
    \lang{de}{Matrixmultiplikation}
    \lang{en}{Matrix multiplication}
  }
  \begin{description} 
 This work is licensed under the Creative Commons License Attribution 4.0 International (CC-BY 4.0)   
 https://creativecommons.org/licenses/by/4.0/legalcode 

    \lang{de}{Beschreibung}
    \lang{en}{Description}
  \end{description}
  \begin{components}
    \component{generic_image}{content/rwth/HM1/images/g_tkz_T111_FalksScheme.meta.xml}{T111_FalksScheme}
    \component{generic_image}{content/rwth/HM1/images/g_tkz_T111_MatrixProduct.meta.xml}{T111_MatrixProduct}
    \component{generic_image}{content/rwth/HM1/images/g_img_00_Videobutton_schwarz.meta.xml}{00_Videobutton_schwarz}
    \component{generic_image}{content/rwth/HM1/images/g_img_00_video_button_schwarz-blau.meta.xml}{00_video_button_schwarz-blau}
    \component{generic_image}{content/rwth/HM1/images/g_img_T112_Falk.meta.xml}{T112_Falk}
    \component{js_lib}{system/media/mathlets/GWTGenericVisualization.meta.xml}{mathlet1}
  \end{components}
  \begin{links}
    \link{generic_article}{content/rwth/HM1/T403a_Vektorraum/g_art_content_10b_lineare_abb.meta.xml}{content_10b_lineare_abb}
    \link{generic_article}{content/rwth/HM1/T111neu_Matrizen/g_art_content_42_matrixaddition.meta.xml}{matrixaddition}
    \link{generic_article}{content/rwth/HM1/T111neu_Matrizen/g_art_content_39b_matrizen.meta.xml}{matrizen}
  \end{links}
  \creategeneric
\end{metainfo}
\begin{content}
\begin{block}[annotation]
	Im Ticket-System: \href{https://team.mumie.net/issues/21334}{Ticket 21334}
\end{block}
\begin{block}[annotation]
Copy of \href{http://team.mumie.net/issues/9060}{Ticket 9060}: content/rwth/HM1/T112_Rechnen_mit_Matrizen/art_content_43_matrizenmultiplikation.src.tex
\end{block}

\usepackage{mumie.ombplus}
\ombchapter{11}
\ombarticle{4}
\usepackage{mumie.genericvisualization}

\begin{visualizationwrapper}

\title{\lang{de}{Matrixmultiplikation}\lang{en}{Matrix multiplication}}
 
% \begin{block}[annotation]
%   Im Ticket-System: \href{http://team.mumie.net/issues/9060}{Ticket 9060}\\
% \end{block}

\begin{block}[info-box]
\tableofcontents
\end{block}

\section{\lang{de}{Multiplikation zweier Matrizen}\lang{en}{Multiplication of two matrices}}

%Abgesehen von der \link{matrixaddition}{Addition zweier Matrizen und der skalaren Multiplikation einer reellen Zahl mit einer Matrix}, gibt es auch noch die

\lang{de}{
Die Multiplikation von Matrizen ist eine Verallgemeinerung 
der \ref[matrizen][Matrix-Vektor-Multiplikation]{sec:matrix-vektor-mult}
aus dem letzten Teilkapitel.
}
\lang{en}{
Multiplication of matrices is a generalisation of the 
\ref[matrizen][matrix-vektor multiplication]{sec:matrix-vektor-mult} 
defined in a previous section of this chapter.
}

\begin{example}
%   Praktische Beispiele zur Rechnung mit Matrizen.
  \begin{tabs*}[\initialtab{0}]
    \tab{\lang{de}{Rohstoffkalkulation}\lang{en}{Calculating raw material requirements}}
  \lang{de}{
  Ein Unternehmen fertigt die \textit{Endprodukte} $P_1$, $P_2$ und $P_3$.
  Es soll ermittelt werden, welcher Bedarf an Rohstoffen je Endprodukt besteht.
  }
  \lang{en}{
  A company produces \textit{end products} $P_1$, $P_2$ and $P_3$.
  The company wants to find how many of each raw material it neeeds per product.
  }
    \begin{itemize}
        \item \lang{de}{
              Jedes Endprodukt setzt sich aus den \textit{Zwischenprodukten} $Z_1$, $Z_2$ und $Z_3$ 
              zusammen.
              }
              \lang{en}{
              Each end product is assembled from the \textit{components} $Z_1$, $Z_2$ and $Z_3$.
              }
        \item \lang{de}{
              Die Zwischenprodukte bestehen aus den \textit{Rohstoffen} $R_1$, $R_2$, $R_3$ und $R_4$.
              }
              \lang{en}{
              The components are manufactured from the \textit{raw materials} $R_1$, $R_2$, $R_3$ and 
              $R_4$.
              }
    \end{itemize}
\lang{de}{
Die folgende Tabelle 1 gibt den Rohstoffbedarf für jedes Zwischenprodukt in der Einheit Kilogramm an:
}
\lang{en}{
The following table 1 gives the raw material requirements for each component in kilograms.
}
\begin{center}
\begin{table}[\cellaligns{ccc}]
    \head
      & $Z_1$ & $Z_2$ & $Z_3$
    \body
  $R_1$ & 1 & 2 & 5 \\
  $R_2$ & 3 & 6 & 4 \\
  $R_3$ & 0 & 1 & 3 \\
  $R_4$ & 2 & 1 & 0
\end{table}
\end{center}
\lang{de}{
Weiterhin liegt eine Tabelle 2 vor, welche die Stückzahl der Zwischenprodukte je Endprodukt auflistet:
}
\lang{en}{
Table 2 gives the amount of each component required for each end product.
}
\begin{table}[\cellaligns{ccc}]
    \head
      & $P_1$ & $P_2$ & $P_3$
    \body
  $Z_1$ & 3 & 0 & 5 \\
  $Z_2$ & 1 & 3 & 4 \\
  $Z_3$ & 6 & 1 & 7
\end{table}
\lang{de}{
Zur Ermittlung der Rohstoffmenge $R_1$ für das Endprodukt $P_1$ in Kilogramm ist die erste Zeile von 
Tabelle 1 sowie die erste Spalte von Tabelle 2 relevant:
}
\lang{en}{
To determine the amount of the raw material $R_1$ is needed for the end product $P_1$, the first row 
of table 1 and the first column of table 2 are relevant.
}
\begin{itemize}
    \item \lang{de}{
          Durch Rechnung erhält man den Bedarf von $1 \cdot 3 + 2 \cdot 1 + 5 \cdot 6 = 35$ Kilogramm.
          }
          \lang{en}{
          A simple calculation yields a requirement of $1 \cdot 3 + 2 \cdot 1 + 5 \cdot 6 = 35$ 
          kilograms.
          }
\end{itemize}

\lang{de}{
\textit{In diesem Kapitel lernen Sie Probleme dieser Art strukturiert zu lösen.}
\\\\
Es seien $A \in M(4, 3;\R)$ und $B \in M(3, 3;\R)$ Matrizen, deren Koeffizienten den Einträgen aus 
den oben gezeigten Tabellen entsprechen:
}
\lang{en}{
\textit{In this chapter we learn to solve problems similar to this one.}
\\\\
Let $A \in M(4, 3;\R)$ and $B \in M(3, 3;\R)$ be matrices whose entries correspond to the entries of 
the above tables:
}

\begin{equation*}
A = 
	\begin{pmatrix}
		1 & 2 & 5 \\
		3 & 6 & 4 \\
		0 & 1 & 3 \\
		2 & 1 & 0
	\end{pmatrix},
 ~~~~~ B=
	\begin{pmatrix}
		3 & 0 & 5 \\
		1 & 3 & 4 \\
		6 & 1 & 7
	\end{pmatrix}.
\end{equation*}

\lang{de}{
Dann gibt die folgende Matrix $C \in M(4, 3;\R)$ den Rohstoffbedarf für jedes Endprodukt an, 
wobei die Spalten den Endprodukten und die Zeilen den Rohstoffen zugeordnet werden können.
\\\\
\textbf{Rechnung:}
}
\lang{en}{
Then the following matrix $C \in M(4, 3;\R)$ gives the raw material requirements for each end 
product, where the columns represent the end products and the rows represent the raw materials.
\\\\
\textbf{Calculation:}
}


\begin{eqnarray*}
	C = A \cdot B
	&=&
	\begin{pmatrix}
		1 & 2 & 5 \\
		3 & 6 & 4 \\
		0 & 1 & 3 \\
		2 & 1 & 0
	\end{pmatrix}
	\cdot
	\begin{pmatrix}
		3 & 0 & 5 \\
		1 & 3 & 4 \\
		6 & 1 & 7
	\end{pmatrix} 
    \\
	&=&
	\begin{pmatrix}
		1 \cdot 3 + 2 \cdot 1 + 5 \cdot 6 & 1 \cdot 0 + 2 \cdot 3 + 5 \cdot 1 & 1 \cdot 5 + 2 \cdot 4 + 5 \cdot 7 \\
		3 \cdot 3 + 6 \cdot 1 + 4 \cdot 6 & 3 \cdot 0 + 6 \cdot 3 + 4 \cdot 1 & 3 \cdot 5 + 6 \cdot 4 + 4 \cdot 7 \\
		0 \cdot 3 + 1 \cdot 1 + 3 \cdot 6 & 0 \cdot 0 + 1 \cdot 3 + 3 \cdot 1 & 0 \cdot 5 + 1 \cdot 4 + 3 \cdot 7 \\
		2 \cdot 3 + 1 \cdot 1 + 0 \cdot 6 & 2 \cdot 0 + 1 \cdot 3 + 0 \cdot 1 & 2 \cdot 5 + 1 \cdot 4 + 0 \cdot 7 
	\end{pmatrix}
    \\
	&=&
    \begin{pmatrix}
   35 &  11  & 48 \\
   39 &  22  & 67 \\
   19 &   6  & 25 \\
    7 &   3  & 14
	\end{pmatrix}
\end{eqnarray*}

% DAS FOLGENDE BEISPIEL WIRD GGF SPÄTER INTEGRIERT

%    \tab{Künstliche Neuronale Netze}

%\textbf{Künstliche Neuronale Netze (KNN)} finden heute weite Einsatzgebiete. 
%Sie werden zum Beispiel zur Klassifikation von Objekten in Bildern eingesetzt. 

%Vereinfacht betrachtet, approximiert ein KNN eine unbekannte Funktion, 
%die durch eine Menge von Parametern, den sogenannten Gewichten, definiert wird.
%Die Gewinnung dieser Gewichte geschieht durch einen Trainingsprozess, der hier nicht weiter betrachtetet wird.

%Strukturell gesehen besteht ein neuronales Netz aus einer endlichen Menge von Neuronen. 
%Jedes dieser Neuronen berechnet aus mehreren Eingaben eine Ausgabe. 
%Dabei wird eine bestimmte Berechnungsvorschrift angewendet. 
%Die berechnete Ausgabe eines Neurons dient wiederum als Eingabe für andere Neuronen.

%Wir bezeichnen nun den Eingabevektor eines Neurons mit $x$.
%Bei $m$ Eingängen ist $x \in \R^m$.
%Ebenfalls liegt für jeden Eingang $x_i$ weiterhin ein Gewicht $w_i$ vor, also $w \in \R^m$.
%In vielen Fällen wird der Ausgang $y$ des Neurons wie folgt berechnet:
%\begin{equation}
%	y = \max\{0, \sum_{i=1}^{m} x_i \cdot w_i \}
%\end{equation}
%TODO: (Wir vernachlässigen hier, dass überlicherweise ein zusätzlicher, statischer Eingang ) 

%In der folgenden Abbildung ist ein einfaches künstliches neuronales Netz, 
%bestehend aus 6 Neuronen, dargestellt.
%Wendet man die oben dargestellte Rechnung für jedes Neuron (von links nach rechts) an, 
%so erhält man die Gesamtausgabe des Netzes.

%TODO: ABBDILUNG

%Ein KNN besteht in der Praxis aus einer sehr großen Anzahl an Neuronen,
%sodass der insgesamt anfallende Rechenaufwand immens ist.
%Spezielle Computerprozessoren arbeiten sehr effizient bei der Berechnung von \textbf{Matrizenmultiplikationen}.
%Um dies auszunutzen, muss die Problemstellung entsprechend aufbereitet werden.

%XXX TODO

    \end{tabs*}
\end{example}




\begin{definition}\label{def:matizen_mult}
\lang{de}{
Eine $(m\times n)$-Matrix $A=(a_{ij})_{1 \leq i \leq m, 1 \leq j \leq n}$ kann mit einer 
$(n\times k)$-Matrix $B=(b_{ij})_{1 \leq i \leq n, 1 \leq j \leq k}$ wie folgt multipliziert werden. 
Das Ergebnis ist dann eine $(m\times k)$-Matrix:% $C=(c_{ij})_{1 \leq i \leq m, 1 \leq j \leq k}$:
}
\lang{en}{
An $(m\times n)$-matrix $A=(a_{ij})_{1 \leq i \leq m, 1 \leq j \leq n}$ can be multiplied with an 
$(n\times k)$-Matrix $B=(b_{ij})_{1 \leq i \leq n, 1 \leq j \leq k}$ as follows. The result is then 
a $(m\times k)$-matrix:
}
\[A \cdot B = 
  \left( \sum_{\ell=1}^n a_{i\ell} \cdot b_{\ell j} \right)_{1 \leq i \leq m, 1 \leq j \leq k} \]

\lang{de}{Ausführliche Schreibweise:}
\lang{en}{More explicitly:}

\begin{eqnarray*}
 A\cdot B &=&
\begin{pmatrix}
a_{11} & a_{12} & \cdots & a_{1n} \\
a_{21} & a_{22} & \cdots & a_{2n} \\
\vdots & \vdots & \ddots & \vdots \\
a_{m1} & a_{m2} & \cdots & a_{mn}
\end{pmatrix} \cdot  \begin{pmatrix}
b_{11} & b_{12} & \cdots & b_{1k} \\
b_{21} & b_{22} & \cdots & b_{2k} \\
\vdots & \vdots & \ddots & \vdots \\
b_{n1} & b_{n2} & \cdots & b_{nk}
\end{pmatrix} \\ && \\
&=& \begin{pmatrix}
\sum_{j=1}^n a_{1j}b_{j1}\, & \sum_{j=1}^n a_{1j}b_{j2}\, & \cdots & \,\sum_{j=1}^n a_{1j}b_{jk} \\
\sum_{j=1}^n a_{2j}b_{j1}\, & \sum_{j=1}^n a_{2j}b_{j2}\, & \cdots & \,\sum_{j=1}^n a_{2j}b_{jk} \\
\vdots & \vdots & \ddots & \vdots \\
\sum_{j=1}^n a_{mj}b_{j1}\, & \sum_{j=1}^n a_{mj}b_{j2}\, & \cdots &\, \sum_{j=1}^n a_{mj}b_{jk}
\end{pmatrix}
%                                      = \begin{pmatrix}
%                                      a_{11} b_{11} + a_{12} b_{21} + \cdots + a_{1n} b_{n1} &  a_{11} b_{12} + a_{12} b_{22} + \cdots + a_{1n} b_{n2} & \cdots &
%                                      a_{11} b_{1k} + a_{12} b_{2k} + \cdots + a_{1n} b_{nk}
%                                      \\
%                                      a_{21} b_{11} + a_{22} b_{21} + \cdots + a_{2n} b_{n1} & & & \vdots \\
%                                      \vdots  & & & \vdots \\
%                                      a_{m1} b_{11} + a_{m2} b_{21} + \cdots + a_{mn} b_{n1} & \cdots  & \cdots & a_{m1} b_{1k} + a_{m2} b_{2k} + \cdots + a_{mn} b_{nk}
%                                      \end{pmatrix}.
\end{eqnarray*}

\lang{de}{
Wie bei Multiplikationen üblich wird der Mal-Punkt auch oft weggelassen und $AB$ statt $A\cdot B$ geschrieben.\\
\floatright{\href{https://www.hm-kompakt.de/video?watch=819}{\image[75]{00_Videobutton_schwarz}}}\\\\
}
\lang{en}{
Again, we often omit the multiplication symbol and write $AB$ instead of $A\cdot B$.
}
\end{definition}


\begin{quickcheck}
    \type{input.number}
      \precision{3}
      \field{real}
      \begin{variables}
           \randint[Z]{m}{2}{4}
           \randint[Z]{n}{5}{6}
           \randint[Z]{k}{7}{8}
       \end{variables}
      \text{\lang{de}{
            Wird eine $(\var{m}\times \var{n})$-Matrix
            mit einer $(\var{n}\times \var{k})$-Matrix multipliziert,\\
            so ist das Ergebnis eine \ansref $\times$ \ansref-Matrix.
            }
            \lang{en}{
            If a $(\var{m}\times \var{n})$-matrix
            is multiplied with a $(\var{n}\times \var{k})$-matrix,\\
            the result is a \ansref $\times$ \ansref-matrix.
            }}
%      \explanation{}
      \begin{answer}
            \solution{m}
      \end{answer}
       \begin{answer}
            \solution{k}
      \end{answer}
\end{quickcheck}


\begin{block}[warning]
\begin{enumerate}
\item \lang{de}{
      Das Produkt $AB$ zweier Matrizen $A$ und $B$ ist nur definiert, wenn die Anzahl der Spalten von 
      $A$ mit der Anzahl der Zeilen von $B$ übereinstimmt.
      }
      \lang{en}{
      The product $AB$ of two matrices $A$ and $B$ is only defined if $A$ has as many columns as $B$ 
      has rows.
      }

\begin{center}
\image{T111_MatrixProduct}
\end{center}

\item \lang{de}{
      Die Matrizenmultiplikation ist nicht kommutativ, d.\,h. im Allgemeinen gilt $AB\ne BA$, selbst 
      wenn beide Seiten definiert sind.
      }
      \lang{en}{
      Matrix multiplication is not commutative, that is, we do not have $AB = BA$ in general, even 
      if both sides are defined.
      }
\end{enumerate}
\end{block}

\begin{remark}\label{rem:falk_schema}
\lang{de}{
Die Multiplikation zweier Matrizen kann über das \textbf{Falksche Schema} veranschaulicht werden:
}
\lang{en}{
Multiplication of two matrices is illustrated by the following table: 
}

\begin{center}
\image{T111_FalksScheme}
\end{center}

\lang{de}{
\floatright{\href{https://www.hm-kompakt.de/video?watch=820}{\image[75]{00_Videobutton_schwarz}}}\\\\
}
\lang{en}{}

\end{remark}

\begin{example}
\begin{tabs*}[\initialtab{0}]
\tab{\lang{de}{Produkt einer $(2\times 3)$-Matrix und einer $(3\times 2)$-Matrix}
     \lang{en}{Product of a $(2\times 3)$-matrix with a $(3\times 2)$-matrix}} 
\lang{de}{F\"ur}
\lang{en}{The product $A\cdot B$ of the matrices}
\[A = \begin{pmatrix}
1 & 3 & 5 \\ 2 & 4 & 6
\end{pmatrix} \quad
\text{\lang{de}{und}\lang{en}{and}} \quad
B = \begin{pmatrix}
12 & 11\\ 10 & 9 \\8 & 7
\end{pmatrix} 
\]
\lang{de}{
ist das Produkt $A\cdot B$ definiert, da $A$ drei Spalten hat und $B$ genau so viele Zeilen. 
Das Produkt einer $(2\times 3)$-Matrix mit einer $(3\times 2)$-Matrix ergibt dann eine $(2\times 2)$-Matrix und es ist
}
\lang{en}{
is defined, as $A$ has three columns and $B$ has three rows. The product of a 
$(2\times 3)$-matrix with a $(3\times 2)$-matrix yields a $(2\times 2)$-matrix,
}
\begin{eqnarray*} AB &=& \begin{pmatrix} 1\cdot 12+3\cdot 10+5\cdot 8 & 1\cdot 11+3\cdot 9+5\cdot 7 \\
  2\cdot 12+4\cdot 10+6\cdot 8 & 2\cdot 11+4\cdot 9+6\cdot 7  \end{pmatrix}\\
  &=&  \begin{pmatrix}
12+30+40 & 11+27+35 \\ 24+40+48 & 22+36+42
\end{pmatrix}
 =  \begin{pmatrix}
82 & 73 \\ 112 & 100
\end{pmatrix}.\end{eqnarray*}
\end{tabs*}
\begin{tabs*}[\initialtab{0}]
\tab{\lang{de}{Produkt einer $(3\times 1)$-Matrix und einer $(1\times 2)$-Matrix}
     \lang{en}{Product of a $(3\times 1)$-matrix with a $(1\times 2)$-matrix}} 
\lang{de}{F\"ur}
\lang{en}{The product $B\cdot C$ of the matrices}
\[B=\begin{pmatrix} 1 \\ 3 \\ -2\end{pmatrix}\quad
  \text{\lang{de}{und}\lang{en}{and}} \quad C=\begin{pmatrix} 2 & -3 \end{pmatrix} \]
\lang{de}{
ist das Produkt $B\cdot C$ definiert, da $B$ eine Spalte hat und $C$ genau so viele Zeilen. Das 
Produkt ist dann eine $(3\times 2)$-Matrix, da $B$ drei Zeilen hat und $C$ zwei Spalten hat. Es ist
}
\lang{en}{
is defined, as $B$ has one column and $C$ has one row. The product is a 
$(3\times 2)$-matrix, as $B$ has three rows and $C$ has two columns. It is
}
 \[ B\cdot C= \begin{pmatrix} 1\cdot 2 & 1\cdot (-3) \\ 3\cdot 2 & 3\cdot (-3)\\ -2\cdot 2 & -2\cdot (-3) \end{pmatrix}
 = \begin{pmatrix} 2 & -3 \\ 6 & -9\\ -4 & 6 \end{pmatrix}.\]
\end{tabs*}
\begin{tabs*}[\initialtab{0}]
\tab{\lang{de}{$AB$ und $BA$ definiert, aber verschieden groß}
     \lang{en}{$AB$ and $BA$ defined but different sizes}} 
\lang{de}{F\"ur die beiden Matrizen}
\lang{en}{The product $A\cdot B$ of the matrices}
\[A = \begin{pmatrix}
  1 & 3 & 5 \\ 2 & 4 & 6
  \end{pmatrix} \quad
  \text{\lang{de}{und}\lang{en}{and}} \quad
  B = \begin{pmatrix}
  12 & 11\\ 10 & 9 \\8 & 7
  \end{pmatrix}\]
\lang{de}{
ist auch das Produkt $B\cdot A$ definiert, da $B$ zwei Spalten hat und $A$ genau so viele Zeilen. Das 
Ergebnis ist dann aber die $(3\times 3)$-Matrix
}
\lang{en}{
is defined, and also the product $B\cdot A$, as $B$ has two columns and $A$ has two rows. The result 
of $A\cdot C$ is the $(2\times 2)$-matrix shown two examples earlier, whereas the result of 
$B\cdot A$ is the $(3\times 3)$-matrix
}
\[ BA= \begin{pmatrix}12+22 & 36+44 & 60+66\\ 10+18 & 30+36 & 50+54 \\ 8+14 & 24+28 & 40+42 \end{pmatrix} 
=\begin{pmatrix}  
34 & 80 & 126 \\ 28 & 66 & 104 \\ 22 & 52 & 82\end{pmatrix}. \]
\end{tabs*}
\begin{tabs*}[\initialtab{0}]
\tab{\lang{de}{$AB$ nicht definiert, aber $BA$ definiert}
     \lang{en}{$AB$ not defined, but $BA$ defined}} 
\lang{de}{F\"ur}
\lang{en}{The product $A\cdot B$ of the matrices}
\[A = \begin{pmatrix}
1 & 3 & 5 \\ 2 & 4 & 6
\end{pmatrix} \quad
 \text{\lang{de}{und}\lang{en}{and}} \quad B = \begin{pmatrix}
12 & 11\\ 10 & 9 \\8 & 7 \\ 6& 5
\end{pmatrix} 
\]
\lang{de}{
ist das Produkt $A\cdot B$ nicht definiert, da $A$ nur drei Spalten hat, $B$ jedoch vier Zeilen.
\\\\
Das Produkt $B\cdot A$ ist jedoch definiert, da $B$ zwei Spalten hat und $A$ zwei Zeilen, und es gilt
}
\lang{en}{
is not defined, as $A$ only has three columns where $B$ has four rows.
\\\\
However, the product $B\cdot A$ is defined, as $B$ has two columns and $A$ has two rows. We have
}
\begin{eqnarray*} B\cdot A &=& \begin{pmatrix}
12 & 11\\ 10 & 9 \\8 & 7 \\ 6& 5
\end{pmatrix} \cdot \begin{pmatrix}
1 & 3 & 5 \\ 2 & 4 & 6
\end{pmatrix}\\
&=& \begin{pmatrix}
12+22 & 36+44 & 60+66\\ 10+18 & 30+36 & 50+54 \\ 8+14 & 24+28 & 40+42 \\ 6+10 & 18+20 & 30+30
\end{pmatrix} =\begin{pmatrix}  
34 & 80 & 126 \\ 28 & 66 & 104 \\ 22 & 52 & 82 \\ 16 & 38 & 60
\end{pmatrix} \in M(4, 3;\R). \end{eqnarray*}
\end{tabs*}
\begin{tabs*}[\initialtab{0}]
\tab{\lang{de}{Quadratische Matrizen $A,B$ mit $AB\neq BA$}
     \lang{en}{Square matrices $A,B$ with $AB\neq BA$}}
\lang{de}{
Selbst für \emph{quadratische} Matrizen (d.\,h. mit gleicher Zeilenzahl wie Spaltenzahl) sind $AB$ 
und $BA$ meist verschieden. Zum Beispiel gilt f\"ur
}
\lang{en}{
Even for \emph{square} matrices (which have the same number of rows and columns), $AB$ and $BA$ are 
usually distinct. For example, for
}
$A = \begin{pmatrix}
1 & 2 \\ 0& 1
\end{pmatrix} \, $ 
\lang{de}{und}\lang{en}{and} 
$ \,B = \begin{pmatrix}
2 & 0\\ 1&1 \end{pmatrix} \in M(2, 2;\R): 
$
\[ AB=\begin{pmatrix}
1 & 2 \\ 0& 1
\end{pmatrix}\cdot \begin{pmatrix}
2 & 0\\ 1&1 \end{pmatrix}=\begin{pmatrix} 4&2\\ 1&1\end{pmatrix}, \]
\lang{de}{aber}
\lang{en}{but}
\[ BA=\begin{pmatrix}
2 & 0\\ 1&1 \end{pmatrix}\cdot\begin{pmatrix}
1 & 2 \\ 0& 1
\end{pmatrix}=\begin{pmatrix} 2&4\\ 1&3\end{pmatrix}. \]
\end{tabs*}
\end{example}


\begin{remark}
\lang{de}{
Die Matrix-Vektor-Multiplikation wurde im \link{matrizen}{letzten Kapitel} eingeführt. Die $j$-te 
Spalte des Matrizenprodukts $A\cdot B$ ist nichts anderes als das Matrix-Vektor-Produkt von $A$ mit 
der $j$-ten Spalte von $B$:
}
\lang{en}{
Matrix-vector multiplication was introduced \link{matrizen}{in a previous section}. The $j$th column 
of the matrix product $A\cdot B$ is nothing more than the matrix-vector product of $A$ with the 
$j$th column of $B$:
}

\begin{itemize}
    \item \lang{de}{
          Eine Matrix $A \in {M}(m,n;\R)$ besitze die Spaltenvektoren 
          $a_1, a_2, \ldots, a_n \in \R^m$.
          }
          \lang{en}{
          A matrix $A \in {M}(m,n;\R)$ 'contains' the column vectors
          $a_1, a_2, \ldots, a_n \in \R^m$.
          }
    \[
        A =
        \begin{pmatrix}
            | & | & & | \\
            a_1 & a_2 & \cdots & a_n \\
            | & | & & |
        \end{pmatrix}
    \]
          \lang{de}{
          Ein Spaltenvektor $x \in \R^n$ habe die Komponenten $x_1, x_2, \ldots, x_n$. Dann ist
          }
          \lang{en}{
          Consider a column vector $x \in \R^n$ with components $x_1, x_2, \ldots, x_n$. Then
          }
    \[
        A \cdot x = x_1 a_1 + x_2 a_2 + \cdots + x_n a_n \in \R^m
    \] 
          \lang{de}{
          eine Linearkombination der Spaltenvektoren von $A$. Die Gewichte der einzelnen Vektoren 
          sind durch den Vektor $x$ gegeben.
          \\\\
          \textit{Ausblick: $\, f(x)=A \cdot x$ definiert eine lineare Abbildung $\, f$ von 
          $\, \R^n$ nach $\, \R^m$. Lineare Abbildungen über einem Körper $\, \K$ werden im Kursteil 
          \link{content_10b_lineare_abb}{3b} (Lineare Algebra) vertieft.}
          }
          \lang{en}{
          is a linear combination of the column vectors in $A$. The 'weights' (coefficient) of each 
          vector are given by the coefficients of $x$.
          \\\\
          \textit{Preview: $\, f(x)=A \cdot x$ defines a linear map $\, f$ from $\, \R^n$ to 
          $\, \R^m$. Linear maps over a field $\, \K$ are covered in section 
          \link{content_10b_lineare_abb}{3b}}
          }

          

    \item \lang{de}{
          Eine Matrix $B \in {M}(n,k;\R)$ besitze die Spaltenvektoren 
          $b_1, b_2, \ldots, b_k \in \R^n$, also
          }
          \lang{en}{
          A matrix $B \in {M}(n,k;\R)$ contains the column vectors $b_1, b_2, \ldots, b_k \in \R^n$, 
          so
          }
    \[
      B=\begin{pmatrix}
      | & | & & | \\
      b_1 & b_2 & \cdots & b_k \\
      | & | & & | 
      \end{pmatrix}.
    \]
          \lang{de}{Dann ist}
          \lang{en}{Hence}
    \[
        A \cdot B =
        \begin{pmatrix}
        | & | & & | \\
            A \cdot b_1 & A \cdot b_2 & \cdots & A \cdot b_k \\
            | & | & & |
        \end{pmatrix}.
    \]
\end{itemize}

\end{remark}

\begin{example}
  %\begin{tabs*}[\initialtab{0}]
  
   % \tab{Beispiel zur Bemerkung}

   \lang{de}{Wir verdeutlichen die Bemerkung mit einem Beispiel. Wir möchten}
   \lang{en}{We clarify the remark with an example. Suppose we want to calculate}
   \[
   \begin{pmatrix}
              1 & 2 \\
              3 & 4
          \end{pmatrix}
          \cdot
          \begin{pmatrix}
              \textcolor{#CC6600}{5} & \textcolor{#0066CC}{6} \\
              \textcolor{#CC6600}{7} & \textcolor{#0066CC}{8}
          \end{pmatrix}
   \]
   \lang{de}{
   berechnen.
   \\\\
   Die erste Spalte des Matrizenprodukts können wir durch
   }
   \lang{en}{
   The first column of the matrix product can be calculated to be
   }
   \[
   \begin{pmatrix}
                  1 & 2 \\
                  3 & 4
              \end{pmatrix}
              \cdot
              \textcolor{#CC6600}{\begin{pmatrix}
                  5 \\
                  7
              \end{pmatrix}} = \textcolor{#CC6600}{\begin{pmatrix}
            1 \cdot 5 + 2 \cdot 7 \\
            3 \cdot 5 + 4 \cdot 7 
        \end{pmatrix}}= \textcolor{#CC6600}{\begin{pmatrix}
            19 \\ 43
        \end{pmatrix}}
   \]
   \lang{de}{berechnen und die zweite Spalte ist durch}
   \lang{en}{and the second column to be}
   \[
    \begin{pmatrix}
                  1 & 2 \\
                  3 & 4
              \end{pmatrix}
              \cdot\textcolor{#0066CC}{\begin{pmatrix}
                  6 \\
                  8
              \end{pmatrix}}
               = \textcolor{#0066CC}{\begin{pmatrix}
              1 \cdot 6 + 2 \cdot 8 \\
              3 \cdot 6 + 4 \cdot 8 
        \end{pmatrix}}= \textcolor{#0066CC}{\begin{pmatrix}
             22 \\ 50
        \end{pmatrix}}
   \]
   \lang{de}{
   gegeben.
   \\\\
   Also ist
   }
   \lang{en}{
   Therefore
   }
   \[
   \begin{pmatrix}
              1 & 2 \\
              3 & 4
          \end{pmatrix}
          \cdot
          \begin{pmatrix}
              5 & 6 \\
              7 & 8
          \end{pmatrix} = 
          \begin{pmatrix}
             \textcolor{#CC6600}{19} & \textcolor{#0066CC}{22} \\
            \textcolor{#CC6600}{43} & \textcolor{#0066CC}{50}
          \end{pmatrix}.
   \]
  
%       \begin{eqnarray*}
%           \begin{pmatrix}
%               1 & 2 \\
%               3 & 4
%           \end{pmatrix}
%           \cdot
%           \begin{pmatrix}
%               5 & 6 \\
%               7 & 8
%           \end{pmatrix}
%           &=&
%           \begin{pmatrix}
%               \begin{pmatrix}
%                   1 & 2 \\
%                   3 & 4
%               \end{pmatrix}
%               \cdot
%               \begin{pmatrix}
%                   5 \\
%                   7
%               \end{pmatrix}
%               &
%               \begin{pmatrix}
%                   1 & 2 \\
%                   3 & 4
%               \end{pmatrix}
%               \cdot
%               \begin{pmatrix}
%                   6 \\
%                   8
%               \end{pmatrix}
%           \end{pmatrix}
%         \\
%         &&
%         \begin{pmatrix}
%             1 \cdot 5 + 2 \cdot 7 & 1 \cdot 6 + 2 \cdot 8 \\
%             3 \cdot 5 + 4 \cdot 7 & 3 \cdot 6 + 4 \cdot 8 
%         \end{pmatrix}
%         =
%         \begin{pmatrix}
%             19 & 22 \\
%             43 & 50
%         \end{pmatrix}
%       \end{eqnarray*}
      
    %\end{tabs*}
\end{example}


\section{\lang{de}{Rechenregeln}\lang{en}{Rules for matrix multiplication}}\label{sec:rechenregeln}

\lang{de}{Abgesehen von der Vertauschung gelten für die Matrixmultiplikation die üblichen Regeln.}
\lang{en}{Besides not being able to swap matrices, the usual rules for multiplication hold.}

\begin{rule}\label{rule:rechenregeln}
\begin{enumerate}
\item \lang{de}{
      \textbf{(Assoziativregel)} Für alle $A\in M(m,n;\R)$, $B \in M(n,k;\R)$ und $C \in M(k,l;\R)$ 
      gilt
      }
      \lang{en}{
      \textbf{(Associativity)} For all $A\in M(m,n;\R)$, $B \in M(n,k;\R)$ und $C \in M(k,l;\R)$ 
      we have
      }
      \[ A \cdot (B\cdot C) = (A\cdot B)\cdot C \in M(m,l;\R). \]
\item \lang{de}{
      \textbf{(Distributivregeln)} Für alle $A,D \in M(m,n;\R)$ und $B,C \in M(n,k;\R)$  gelten
      }
      \lang{en}{
      \textbf{(Distributivity over addition)} For all $A,D \in M(m,n;\R)$ and $B,C \in M(n,k;\R)$ 
      we have
      }
      \[ A\cdot (B+C) =AB+ AC \quad \text{und}\quad (A+D)\cdot B=AB+DB. \]
\item \lang{de}{
      \textbf{(Verträglichkeit mit Skalaren)} Für alle $A\in M(m,n;\R)$, $B \in M(n,k;\R)$ und 
      reelle Zahlen $r\in \R$ gilt
      }
      \lang{en}{
      \textbf{(Compatibility with scalar multiplication)} For all $A\in M(m,n;\R)$, $B \in M(n,k;\R)$ 
      and real numbers $r\in \R$ we have
      }
      \[ r(AB)=(rA)\cdot B= A\cdot (rB). \]
\end{enumerate}
\lang{de}{
\floatright{\href{https://www.hm-kompakt.de/video?watch=823}{\image[75]{00_Videobutton_schwarz}}}\\\\
}
\lang{en}{}
\end{rule}

\begin{block}[warning]
\lang{de}{
Bei der letzten Regel muss die Reihenfolge von $A$ und $B$ stets gleich bleiben. Lediglich der Skalar 
$r$ darf mit $A$ vertauscht werden.
}
\lang{en}{
When applying the final rule, the order of $A$ and $B$ must remain the same. Only the scalar $r$ may 
be swapped with $A$.
}
\end{block}

\begin{quickcheck}
  \field{rational}
  \type{input.number}
  \begin{variables}
  \end{variables}
      \text{\lang{de}{
      Welche der folgenden Aussagen sind für alle Matrizen $A, B, C \in \mathbb{M}(n,n;\mathbb{R})$ 
      wahr?\\
      }
      \lang{en}{
      Which of the following statements are true for all matrices 
      $A, B, C \in \mathbb{M}(n,n;\mathbb{R})$?\\
      }}
      \begin{choices}{multiple}
          \begin{choice}
            \text{$(A \cdot B) \cdot C = A \cdot (B \cdot C)$}
            \solution{true}
          \end{choice}
          \begin{choice}
            \text{$(A + B) \cdot C = AC + BC$}
            \solution{true}
          \end{choice}
          \begin{choice}
            \text{$A \cdot B \cdot C = A \cdot C \cdot B$}
            \solution{false}
          \end{choice}
       \end{choices}
  \explanation{\lang{de}{Achtung: die Matrizenmultiplikation ist \textbf{nicht} kommutativ!}
               \lang{en}{Warning: matrix multiplication is \textbf{not} commutative!}}
\end{quickcheck}


\begin{example}
\begin{tabs*}
\tab{$A (B C) = (A B) C$}
\lang{de}{Wir betrachten}
\lang{en}{Consider}
\[A=\begin{pmatrix} 2 & -3 & 1 \\ 0 & \frac{4}{3} & 5 \end{pmatrix}, \quad 
  B=\begin{pmatrix} 1 \\ 3 \\ -2\end{pmatrix}\quad
  \text{\lang{de}{und}\lang{en}{and}} \quad 
  C=\begin{pmatrix} 2 & -3 \end{pmatrix}.\]
\lang{de}{Dann sind}
\lang{en}{Then we have}
 \begin{eqnarray*} A(BC)&=& \begin{pmatrix} 2 & -3 & 1 \\ 0 & \frac{4}{3} & 5 \end{pmatrix}\cdot \Big( \begin{pmatrix} 1 \\ 3 \\ -2\end{pmatrix} \cdot
 \begin{pmatrix} 2 & -3 \end{pmatrix} \Big) \\
 &=& \begin{pmatrix} 2 & -3 & 1 \\ 0 & \frac{4}{3} & 5 \end{pmatrix}\cdot \begin{pmatrix} 2 & -3 \\ 6 & -9\\ -4 & 6 \end{pmatrix} \\
 &=& \begin{pmatrix} -18 & 27 \\ -12 & 18 \end{pmatrix}
 \end{eqnarray*}
\lang{de}{und}\lang{en}{and}
 \begin{eqnarray*} (AB)C &=& \Big( \begin{pmatrix} 2 & -3 & 1 \\ 0 & \frac{4}{3} & 5 \end{pmatrix}\cdot \begin{pmatrix} 1 \\ 3 \\ -2\end{pmatrix}\Big) \cdot
 \begin{pmatrix} 2 & -3 \end{pmatrix}  \\
 &=& \begin{pmatrix} 2\cdot 1 + (-3)\cdot 3 + 1\cdot (-2) \\ 0\cdot 1 + \frac{4}{3} \cdot 3 + 5\cdot (-2) \end{pmatrix}\cdot
 \begin{pmatrix} 2 & -3 \end{pmatrix} = \begin{pmatrix} -9\\ -6 \end{pmatrix} \cdot
 \begin{pmatrix} 2 & -3 \end{pmatrix} \\
 &=& \begin{pmatrix} -18 & 27 \\ -12 & 18 \end{pmatrix}.
 \end{eqnarray*}
\tab{$A(B+C)=AB+AC$}
\lang{de}{Wir betrachten}
\lang{en}{Consider}
\[ A= \begin{pmatrix} 2 & -3 & 1 \\ 0 & \frac{4}{3} & 5 \end{pmatrix}, \quad B=\begin{pmatrix} 1 \\ 3 \\ -2\end{pmatrix}\quad
 \text{und} \quad C=\begin{pmatrix} 2 \\ -3 \\ 1\end{pmatrix}.\]
\lang{de}{Dann sind}
\lang{en}{Then we have}
 \begin{eqnarray*} A(B+C) &=& \begin{pmatrix} 2 & -3 & 1 \\ 0 & \frac{4}{3} & 5 \end{pmatrix}\cdot \begin{pmatrix} 1+2 \\ 3-3 \\ -2+1\end{pmatrix}\\
 &=& \begin{pmatrix} 2\cdot 3+(-3)\cdot 0+1\cdot (-1) \\ 0\cdot 3+\frac{4}{3} \cdot 0+5\cdot (-1) \end{pmatrix} =\begin{pmatrix} 5\\ -5 \end{pmatrix}
 \end{eqnarray*}
\lang{de}{und}\lang{en}{and}
  \begin{eqnarray*} AB+AC &=& \begin{pmatrix} 2\cdot 1 + (-3)\cdot 3 + 1\cdot (-2) \\ 0\cdot 1 + \frac{4}{3} \cdot 3 + 5\cdot (-2) \end{pmatrix}
   +\begin{pmatrix} 2\cdot 2 + (-3)\cdot (-3) + 1\cdot 1 \\ 0\cdot 2 + \frac{4}{3}\cdot (-3)+ 5\cdot 1 \end{pmatrix} \\
 &=& \begin{pmatrix} -9\\ -6 \end{pmatrix}+ \begin{pmatrix} 14 \\ 1\end{pmatrix} = \begin{pmatrix} 5\\ -5 \end{pmatrix}.
 \end{eqnarray*}
 \end{tabs*}
 \end{example}


\end{visualizationwrapper}


\end{content}