\documentclass{mumie.article}
%$Id$
\begin{metainfo}
  \name{
    \lang{de}{Cramersche Regel}
    \lang{en}{}
  }
  \begin{description} 
 This work is licensed under the Creative Commons License Attribution 4.0 International (CC-BY 4.0)   
 https://creativecommons.org/licenses/by/4.0/legalcode 

    \lang{de}{Beschreibung}
    \lang{en}{}
  \end{description}
  \begin{components}
    \component{generic_image}{content/rwth/HM1/images/g_img_00_video_button_schwarz-blau.meta.xml}{00_video_button_schwarz-blau}
  \end{components}
  \begin{links}
\link{generic_article}{content/rwth/HM1/T306_Reelle_Quadratische_Matrizen/g_art_content_15_inverse_matrix.meta.xml}{content_15_inverse_matrix}
\link{generic_article}{content/rwth/HM1/T306_Reelle_Quadratische_Matrizen/g_art_content_16_determinante.meta.xml}{content_16_determinante}
\end{links}
  \creategeneric
\end{metainfo}
\begin{content}
\usepackage{mumie.ombplus}
\ombchapter{5}
\ombarticle{4}

\title{Cramersche Regel}

\begin{block}[annotation]

  
\end{block}
\begin{block}[annotation]
  Im Ticket-System: \href{http://team.mumie.net/issues/11620}{Ticket 11620}\\
\end{block}

\begin{block}[info-box]
\tableofcontents
\end{block}


\section{Lineare Gleichungssysteme mit invertierbarer Koeffizientenmatrix}\label{sec:cramsche-regel}

Die Aussagen über Invertierbarkeit von Matrizen aus den vorigen Abschnitten sollen nun
verwendet werden, um Aussagen über Lösungsmengen linearer Gleichungssysteme zu machen.

Wir betrachten also im Folgenden das lineare Gleichungssystem $Ax=b$ mit quadratischer Koeffizientenmatrix $A$, d.h. mit gleich vielen Variablen wie Gleichungen.

\begin{theorem}\label{thm:lgs-mit-quadrat-matrix}
Sei ein lineares Gleichungssystem $A  x =b$ mit $A \in M(n;\R)$, $b \in \R^n$, gegeben. Dann sind \"aquivalent 
\begin{enumerate}
 \item Das LGS hat genau eine L\"osung.
 \item Die Matrix $A$ ist invertierbar.
 \item Es gilt $\det(A) \neq 0 $.
 \item Die Matrix $A$ hat vollen Rang.
\end{enumerate}
\end{theorem}

\begin{proof*}
\begin{showhide}
Wir hatten schon gesehen, dass $A$ genau dann invertierbar ist, wenn sie vollen Rang (also Rang $n$) hat, 
und dass auch $\det(A) \neq 0 $ genau dann gilt, wenn $A$ vollen Rang hat.

Hat die Matrix $A$ nun vollen Rang, dann erhält man die reduzierte Stufenform der Begleitmatrix $(A\mid b)$
als $(E_n\mid \tilde{b})$ und man erhält eine eindeutige Lösung des LGS, nämlich
\[ x=\tilde{b}=A^{-1} \cdot b.\]
Hat die Matrix $A$ keinen vollen Rang, dann besitzt die linke Seite  in der reduzierten Stufenform der Begleitmatrix weniger als 
$n$ Stufen und  enthält daher mindestens eine Nullzeile. 
Damit ist die Lösungsmenge entweder leer (wenn der entsprechende Eintrag der rechten Seite nicht $0$ ist), 
oder es gibt Lösungen und die Variablen, zu denen keine Stufen gehören, können als freie Parameter gewählt werden. 
Damit gibt es aber mehrere Lösungen.

Das LGS hat also genau dann genau eine Lösung, wenn die Matrix $A$ vollen Rang hat.
\end{showhide}
\end{proof*}
\begin{example}
Benötigt man zu einer bestimmten invertierbaren Matrix $A\in M(n;\R)$ die Lösungen des LGS $A\cdot x=b$ zu mindestens $n$ 
verschiedenen Werten von $b$, dann ist es sinnvoll, die Inverse von $A$ zu berechnen. 
\\
Sollen etwa zu 
$A=\begin{pmatrix} 2&5\\1&3\end{pmatrix}$ die Lösungen der LGS  $A\cdot x=b_j$ für $b_1=\begin{pmatrix}1\\1\end{pmatrix}$, $b_2=\begin{pmatrix}1\\-1\end{pmatrix}$,
$b_3=\begin{pmatrix}2\\1\end{pmatrix}$ und $b_4=\begin{pmatrix}5\\-2\end{pmatrix}$ bestimmt werden,
dann erhält man mit $A^{-1}=\begin{pmatrix}3&-5\\-1&2\end{pmatrix}$ dazu die 
Lösungen $x_1=A^{-1}\cdot b_1=\begin{pmatrix}-2\\1\end{pmatrix}$, $x_2=A^{-1}\cdot b_2=\begin{pmatrix}8\\-3\end{pmatrix}$, 
$x_3=A^{-1}\cdot b_3=\begin{pmatrix}1\\0\end{pmatrix}$ und $x_4=A^{-1}\cdot b_4=\begin{pmatrix}25\\-9\end{pmatrix}$ durch einfache Matrix-Vektor-Multiplikation, 
ohne vier LGS lösen zu müssen.
\end{example}


Um die eindeutige Lösung des LGS $A \cdot x=b$ im Fall einer invertierbaren Matrix zu berechnen, gibt es noch eine Formel, 
die sogenannte \emph{Cramersche Regel}.
Für größere Gleichungssysteme (d.h. $n\geq 4$) ist sie im Allgemeinen unpraktisch,
weil der Rechenaufwand der Cramerschen Regel deutlich höher ist als der des Gaußalgorithmus.
Für kleine Gleichungssysteme (d.h. $n=2$ oder $3$) lässt sich die Lösung damit recht flott berechnen. 
\\
Im Fall $n=2$  ist die Benutzung der Formel für die inverse Matrix (wie im obigen Beispiel)
etwa ebenso aufwändig wie die Cramerschen Regel.
\\
Im Fall $n=3$ bedeutet die Cramersche Regel schon mehr Rechenaufwand als der Gauß-Algorithmus. 
Dass man sie mit Papier und Bleistift  trotzdem verwendet, hat einen bestechenden Grund: Bei der Cramerschen Regel  
wird erst im allerletzten Schritt eine Division durchgeführt, ansonsten nur Multiplikationen und Additionen.
Beim Gauß-Algorithmus  muss im Allgemeinen schon im ersten Schritt auch eine Division durchgeführt werden, 
was beim Kopfrechnen für viele Studierende eine Fehlerquelle darstellt.
\\
Übrigens liegt genau in dieser einen Division im letzten Schritt der Grund verborgen, weshalb Mathematiker diese Regel (in der Theoriebildung, nicht praktisch!)
verwenden: Die auftretenden Nenner sind allesamt Teiler der Determinante von $A$. 


\begin{rule}[Cramersche Regel] \label{cramersche_regel}
Sei $A \in M(n;\R)$ invertierbar, $b \in \R^n$, und $A \cdot x=b$ das zugeh\"orige LGS. 
Schreibt man $x=\left(\begin{smallmatrix}
x_1\\ \vdots \\ x_n
\end{smallmatrix}\right)$ und bezeichnet f\"ur alle $k \in \{ 1, \ldots ,n \}$ mit 
$A_{k|b}$ die Matrix, die aus $A$ entsteht, indem man die $k$-te Spalte von $A$ durch $b$ ersetzt, so ist die eindeutig bestimmte L\"osung $x$ von $A \cdot x =b$
gegeben durch
\[
x_k = \frac{\det(A_{k|b})}{\det(A)} \text{ f\"ur alle } k \in {\{ 1, \ldots ,n \} }.
\]
\end{rule}


\begin{example}\label{ex:cramer_LGS}
Wir betrachten das lineare Gleichungssystem 
\[ \begin{mtable}[\cellaligns{ccrcrcrcr}]
\text{(I)}&&x_{1}&+&2x_{2}&+&x_{3}&=&4\\
\text{(II)}&&x_{1}&-&x_{2}&+&\frac{3}{2}x_{3}&=&-7\\
\text{(III)}&\qquad-&4x_{1}&+&2x_{2}&&&=&-2.
\end{mtable} \]
\begin{incremental}[\initialsteps{0}]
\step
Es hat als Koeffizientenmatrix
$ A=\begin{pmatrix} 1 & 2 & 1 \\ 1 & -1 & \frac{3}{2} \\ -4 & 2 & 0\end{pmatrix} $
und als rechte Seite $b=\begin{pmatrix} 4 \\ -7 \\ -2 \end{pmatrix} $.
Zunächst einmal berechnet man mit der Regel von Sarrus
\begin{eqnarray*} \det(A)&=& 1\cdot (-1)\cdot 0+2\cdot\frac{3}{2}\cdot (-4)+1\cdot 1\cdot 2-(-4)\cdot (-1)\cdot 1
-2\cdot \frac{3}{2}\cdot 1-0\cdot 1\cdot 2\\
&=& 0-12+2-4-3-0=-17.\end{eqnarray*}
Die Determinante ist also ungleich $0$ und daher besitzt das LGS eine eindeutige Lösung, die wir im
Folgenden mit der Cramerschen Regel berechnen werden. 
\step
Dazu benötigen wir die Matrizen $A_{1|b}$, $A_{2|b}$
und $A_{3|b}$ bzw. ihre Determinanten. Die Matrix $A_{1|b}$ erhält man aus der Matrix $A$, indem man die
erste Spalte durch den Spaltenvektor $b$ ersetzt. Also gilt:
\[
\det(A_{1|b}) = \det \Big(\begin{pmatrix} 4 & 2 & 1 \\ -7 & -1 & \frac{3}{2} \\ -2 & 2 & 0\end{pmatrix}\Big)
= 0+(-6)+(-14)-2-12-0=-34 \]
und daher 
\[ x_1=\frac{\det(A_{1|b})}{\det(A)}=\frac{-34}{-17}=2.\]
Entsprechend berechnen wir
\begin{eqnarray*}
\det(A_{2|b}) &=& \det \Big(\begin{pmatrix} 1 & 4 & 1 \\ 1 & -7 & \frac{3}{2} \\ -4 & -2 & 0\end{pmatrix}\Big)
= 0+(-24)+(-2)-28-(-3)-0=-51, \\
\det(A_{3|b}) &=& \det \Big(\begin{pmatrix} 1 & 2 & 4 \\ 1 & -1 & -7 \\ -4 & 2 & -2\end{pmatrix}\Big)
= 2+56+8-16-(-14)-(-4)=68.
\end{eqnarray*}
Damit erhält man weiter
\[  x_2=\frac{-51}{-17}=3\quad \text{und}\quad x_3=\frac{68}{-17}=-4.\]
\step
Die Lösungsmenge besteht also aus der eindeutigen Lösung
\[ \left( \begin{smallmatrix}x_1\\ x_2\\ x_3\end{smallmatrix}\right)
=\left( \begin{smallmatrix}2\\ 3\\ -4\end{smallmatrix}\right). \]
\end{incremental}
\end{example}


\begin{quickcheck}
\text{Es sei $A=\begin{pmatrix}2&-1&1\\0&1&0\\0&0&3\end{pmatrix}$. Welcher der folgenden Ausdrücke
bestimmt die zweite Komponente der Lösung $x=\begin{pmatrix}x_1\\x_2\\x_3\end{pmatrix}$ des LGS $A\cdot x=\begin{pmatrix}4\\4\\4\end{pmatrix}$?}
\begin{choices}{unique}
\begin{choice}
\text{$x_2=\frac{1}{6}\det\Big( \left( \begin{smallmatrix} 2&4&1\\0&4&0\\0&4&3\end{smallmatrix}\right)\Big)$}
\solution{true}
\end{choice}
\begin{choice}
\text{$x_2=\frac{1}{6}\det\Big( \left( \begin{smallmatrix} 2&1&4\\0&0&4\\0&3&4\end{smallmatrix}\right)\Big)$}
\solution{false}
\end{choice}
\begin{choice}
\text{$x_2=\frac{1}{6}\det\Big( \left( \begin{smallmatrix} 2&-1&1&4\\0&1&0&4\\0&0&3&4\end{smallmatrix}\right)\Big)$}
\solution{false}
\end{choice}
\end{choices}
\explanation{Der letzte Ausdruck ist gar nicht wohldefiniert, denn Determinanten kann man nur von quadratischen Matrizen bilden.
Beim zweiten Ausdruck ist zwar die zweite Spalte gestrichen, aber der Vektor $b$ wurde nur hinten angehängt. Er gehört hingegebn wie im ersten Ausdruck in die zweite Spalte.}
\end{quickcheck}




\section{Berechnung der inversen Matrix mit Cramerscher Regel}

Da man zu einer invertierbaren Matrix $A$ die Spalten der inversen Matrix $A^{-1}$ als
Lösungen von Gleichungssystemen bekommt, lässt sich auch die inverse Matrix mit Hilfe der
Cramerschen Regel berechnen.


\begin{rule}[Cramersche Regel und inverse Matrizen]\label{rule:inverse_via_cramer}
Sei $A \in M(n;\R)$ invertierbar. Mit Hilfe der Cramerschen Regel l\"asst sich die inverse Matrix berechnen, indem man f\"ur $b$ in der Gleichung $Ax=b$
der Reihe nach die Vektoren 
\[
e_1 := \begin{pmatrix} 1 \\ 0 \\ 0 \\ \vdots \\ 0 \\ 0 \end{pmatrix} , \ e_2 := \begin{pmatrix} 0 \\ 1 \\ 0 \\ \vdots \\ 0 \\ 0 \end{pmatrix}, \ \cdots \ , e_m := \begin{pmatrix} 0 \\ 0 \\ 0 \\ \vdots \\ 0 \\ 1 \end{pmatrix} \in \R^n,
\]
einsetzt und die erhaltenen L\"osungen $x$ (abh\"angig von $e_i$, $1 \leq i \leq n$) nebeneinander in eine Matrix schreibt.

Explizit ergibt sich dabei der $(i,j)$-Eintrag von $A^{-1}$ als
\[   (-1)^{i+j} \frac{\det(A_{ji})}{\det(A)}, \]
wobei $A_{ji}$ die $(n-1)\times (n-1)$-Matrix ist, die man aus $A$ durch Streichen der $j$-ten Zeile und $i$-ten Spalte erhält.
\end{rule}

\begin{proof*}
Wir benutzen die Cramersche Regel und berechnen die auftretenden Determinanten durch Laplace-Entwicklung.
\begin{incremental}[\initialsteps{0}]
\step
Mit der Cramerschen Regel für die Lösung eines linearen Gleichungssystems erhält man für den 
$(i,j)$-Eintrag von $A^{-1}$ zunächst den Wert
\[  \frac{\det(A_{i|e_j})}{\det(A)}. \]
\step
Die Matrix $A_{i|e_j}$ entsteht aus $A$, indem man die $i$-te \emph{Spalte} durch $e_j$ ersetzt.
Entwickeln wir ihre Determinante mit dem 
\ref[content_16_determinante][Laplaceschen Entwicklungssatz]{rule:Laplace}  nach der $i$-ten Spalte,
so ist darin nur der $j$-te Summand von null verschieden. Er besteht aus dem Produkt des Vorzeichen $(-1)^{i+j}$ mit
der Determinante der Matrix, die aus $A$ durch Streichen der $i$-ten \emph{Spalte} und $j$-ten \emph{Zeile} entsteht,
\[\det(A_{i|e_j})=(-1)^{i+j}\det(A_{ji}).\]
% Vertauscht man nun aber zur Berechnung von $\det(A_{i|e_j})$ die $i$-te Zeile nacheinander mit den darüberliegenden
% und dann die $j$-te Spalte nacheinander mit den links davon liegenden, so erhält man eine Block-Dreiecksmatrix der Form
% \[  \begin{pmatrix} 1 & * \\ 0 & B\end{pmatrix},\]
% wobei $*$ irgendeine $1\times (n-1)$-Matrix (=Zeilenvektor) ist und $0$ die $(n-1)\times 1$-Nullmatrix. Die Matrix $B$
% ist jedoch nichts anderes als die Matrix $A_{ji}$, die man aus $A$ durch Streichen der $j$-ten Zeile und 
% $i$-ten Spalte erhält.
% \step
% Insgesamt wurden $i-1$ Zeilenvertauschungen und $j-1$ Spaltenvertauschungen vorgenommen. Also ist
% \[  \det(A_{i|e_j})=(-1)^{i-1+j-1}\cdot \det\big( \begin{pmatrix} 1 & * \\ 0 & A_{ji}\end{pmatrix} \big)
% = (-1)^{i+j}\cdot 1\cdot \det(A_{ji}), \]
% wie behauptet.
\end{incremental}
\end{proof*}




\begin{example}
Wir betrachten wieder die Matrix $ A=\begin{pmatrix} 1 & 2 & 1 \\ 1 & -1 & \frac{3}{2} \\ -4 & 2 & 0\end{pmatrix} $
aus  Beispiel \ref{ex:cramer_LGS}. Deren Determinante war $\det(A)=-17\neq 0$. Die Matrix ist also invertierbar und wir
können mit der Cramerschen Regel die inverse Matrix berechnen.
\begin{incremental}[\initialsteps{0}]
\step
Dazu müssen wir also die Determinanten der Teilmatrizen berechnen, die man erhält, wenn man eine Zeile und eine Spalte
streicht. Für den $(1,1)$-Eintrag von $A^{-1}$ benötigen wir also
\[ \det(A_{11})=\det\big( \begin{pmatrix}-1 & \frac{3}{2} \\ 2 & 0 \end{pmatrix} \big)
=-1\cdot 0-2\cdot \frac{3}{2}=-3,\]
und erhalten den Eintrag als
\[ (-1)^{1+1}\cdot \frac{\det(A_{11})}{\det(A)}=1\cdot \frac{-3}{-17}=\frac{3}{17}.\]
\step
Für den $(1,2)$-Eintrag benötigen wir die Matrix $A_{21}$, bei der die zweite Zeile und die erste Spalte von $A$
gestrichen wurden, also $A_{21}=\left(\begin{smallmatrix}2 & 1 \\ 2 & 0\end{smallmatrix}\right)$.
Dann erhalten wir als $(1,2)$-Eintrag den Wert
\[ (-1)^{1+2}\cdot \frac{\det(A_{21})}{\det(A)}=-1\cdot \frac{-2}{-17}=-\frac{2}{17}. \]
\step
Entsprechend berechnet man die anderen Einträge und erhält
\begin{eqnarray*} A^{-1} &=&\frac{1}{\det(A)}\cdot
\begin{pmatrix} \det(A_{11}) & -\det(A_{21}) &\det(A_{31})\\
-\det(A_{12}) &\det(A_{22}) & -\det(A_{32})\\
\det(A_{13}) & -\det(A_{23}) &\det(A_{33})
  \end{pmatrix}\\
&=& -\frac{1}{17}\cdot  \begin{pmatrix} -3 & 2 & 4\\ -6 & 4 & -\frac{1}{2}\\ -2 & -10 & -3
  \end{pmatrix}=\frac{1}{17}\cdot \begin{pmatrix} 3 & -2 & -4\\ 6 & -4 & \frac{1}{2}\\ 2 & 10 & 3
  \end{pmatrix}. \end{eqnarray*}
\end{incremental}
\end{example}  
  
\begin{example}
Mit Hilfe der Cramerschen Regel erhalten wir die  \ref[content_15_inverse_matrix][uns bekannte Formel]{rule:inverse-2x2}  für die inverse Matrix einer $2 \times 2$ Matrix.
\begin{incremental}[\initialsteps{0}]
\step
Gegeben sei also die Matrix $A= \begin{pmatrix}a &  b \\ c & d\end{pmatrix}$ mit Einträgen $a,b,c,d \in \R$.
Wir setzen natürlich voraus, dass $A$ invertierbar ist, d.h. $\det(A) \neq 0$.
\step
Da wir beim Streichen einer Zeile und Spalte eine $1 \times 1$ Matrix, also eine Zahl, erhalten,
liefert die Cramersche Regel
\[
A^{-1} = \begin{pmatrix}\frac{d}{\det(A)} & - \frac{b}{\det(A)} \\ -\frac{c}{\det(A)} & \frac{a}{\det(A)}\end{pmatrix}
= \frac{1}{\det(A)} \cdot \begin{pmatrix} d & -b \\ -c & a\end{pmatrix} = \frac{1}{ad-bc}\begin{pmatrix}d & -b \\ -c & a\end{pmatrix}.
\]
\end{incremental}
\end{example}


Das folgende Video fasst das vorliegende Kapitel zusammen.
\floatright{\href{https://api.stream24.net/vod/getVideo.php?id=10962-2-10881&mode=iframe&speed=true}{\image[75]{00_video_button_schwarz-blau}}}\\



% \section{Lineare Gleichungssysteme mit invertierbarer Koeffizientenmatrix}

% Die Aussagen über Invertierbarkeit von Matrizen aus den vorigen Abschnitten sollen nun
% verwendet werden, um Aussagen über Lösungsmengen linearer Gleichungssysteme zu machen.

% Wir betrachten also im Folgenden lineare Gleichungssystem $Ax=b$ mit quadratischer Koeffizientenmatrix $A$, d.h. mit gleich vielen Variablen wie Gleichungen.

% \begin{theorem}\label{thm:lgs-mit-quadrat-matrix}
% Sei ein lineares Gleichungssystem $A \cdot x =b$ mit $A \in M(n;\R)$, $b \in \R^n$, gegeben. Dann sind \"aquivalent 
% \begin{enumerate}
%  \item Das LGS hat genau eine L\"osung.
%  \item Die Matrix $A$ ist invertierbar.
%  \item Es gilt $\det(A) \neq 0 $.
% \end{enumerate}
% \end{theorem}


% %\begin{block}[explanation]
% \begin{tabs}
% \tab{Erklärung}
% Wir hatten schon gesehen, dass $A$ genau dann invertierbar ist, wenn sie vollen Rang (also Rang $n$) hat, und dass auch $\det(A) \neq 0 $ genau dann gilt, wenn $A$ vollen Rang hat.

% Hat die Matrix $A$ nun vollen Rang, dann ist in der reduzierten Stufenform die linke Seite die 
% Einheitsmatrix $E_n$ und man erhält eine eindeutige Lösung des LGS, nämlich
% \[ x=A^{-1} \cdot b.\]
% Hat die Matrix $A$ keinen vollen Rang, dann besitzt die linke Seite  in der reduzierten Stufenform weniger als 
% $n$ Stufen und  enthält daher mindestens eine Nullzeile. 
% Damit ist die Lösungsmenge entweder leer (wenn die entsprechende rechte Seite nicht $0$ ist), 
% oder es gibt Lösungen und die Variablen, zu der keine Stufe gehören, können als freie Parameter gewählt werden. 
% Damit gibt es aber mehrere Lösungen.

% Das LGS hat also genau dann genau eine Lösung, wenn die Matrix $A$ vollen Rang hat.
% \tab{Erklärung}
% Wir hatten schon gesehen, dass $A$ genau dann invertierbar ist, wenn sie vollen Rang (also Rang $n$) hat, und dass auch $\det(A) \neq 0 $ genau dann gilt, wenn $A$ vollen Rang hat.

% Hat die Matrix $A$ nun vollen Rang, dann ist in der reduzierten Stufenform die linke Seite die 
% Einheitsmatrix $E_n$ und man erhält eine eindeutige Lösung des LGS, nämlich
% \[ x=A^{-1} \cdot b.\]
% Hat die Matrix $A$ keinen vollen Rang, dann besitzt die linke Seite  in der reduzierten Stufenform weniger als 
% $n$ Stufen und  enthält daher mindestens eine Nullzeile. 

% \end{tabs}
% %\end{block}


% Um die eindeutige Lösung des LGS $A \cdot x=b$ im Fall einer invertierbaren Matrix zu berechnen, gibt es auch eine Formel, die sogenannte \emph{Cramersche Regel}.
% Für kleine Gleichungssysteme (d.h. $n=2$ oder $3$) lässt sich die Lösung damit schnell berechnen. Für größere Gleichungssysteme (d.h. große $n$) ist sie im Allgemeinen unpraktisch,
% da mehrere Determinanten berechnet werden müssen.


% \begin{rule}[Cramersche Regel] \label{cramersche_regel}
% Sei $A \in M(n;\R)$ invertierbar, $b \in \R^n$, und $A \cdot x=b$ das zugeh\"orige LGS. 
% Schreibt man $x=\left(\begin{smallmatrix}
% x_1\\ \vdots \\ x_n
% \end{smallmatrix}\right)$ und bezeichnet f\"ur alle $k \in \{ 1, \ldots ,n \}$ mit 
% $A_{k|b}$ die Matrix, die aus $A$ entsteht, indem man die $k$-te Spalte von $A$ durch $b$ ersetzt, so ist die eindeutig bestimmte L\"osung $x$ von $A \cdot x =b$
% gegeben durch
% \[
% x_k = \frac{\det(A_{k|b})}{\det(A)} \text{ f\"ur alle } k \in {\{ 1, \ldots ,n \} }.
% \]
% \end{rule}

% \begin{example}
% Wir betrachten das lineare Gleichungssystem 
% \[ \begin{mtable}[\cellaligns{ccrcrcrcr}]
% \text{(I)}&&x_{1}&+&2x_{2}&+&x_{3}&=&4\\
% \text{(II)}&&x_{1}&-&x_{2}&+&\frac{3}{2}x_{3}&=&-7\\
% \text{(III)}&\qquad-&4x_{1}&+&2x_{2}&&&=&-2.
% \end{mtable} \]
% Es hat als Koeffizientenmatrix
% $ A=\begin{pmatrix} 1 & 2 & 1 \\ 1 & -1 & \frac{3}{2} \\ -4 & 2 & 0\end{pmatrix} $
% und als rechte Seite $b=\begin{pmatrix} 4 \\ -7 \\ -2 \end{pmatrix} $.
% Zunächst einmal berechnet man mit der Regel von Sarrus
% \begin{eqnarray*} \det(A)&=& 1\cdot (-1)\cdot 0+2\cdot\frac{3}{2}\cdot (-4)+1\cdot 1\cdot 2-(-4)\cdot (-1)\cdot 1
% -2\cdot \frac{3}{2}\cdot 1-0\cdot 1\cdot 2\\
% &=& 0-12+2-4-3-0=-17.\end{eqnarray*}
% Die Determinante ist also ungleich $0$ und daher besitzt das LGS eine eindeutige Lösung, die wir im
% folgenden mit der Cramerschen Regel berechnen werden. Dazu benötigen wir die Matrizen $A_{1|b}$, $A_{2|b}$
% und $A_{3|b}$ bzw. ihre Determinanten. Die Matrix $A_{1|b}$ erhält man aus der Matrix $A$, indem man die
% erste Spalte durch den Spaltenvektor $b$ ersetzt. Also gilt:
% \[
% \det(A_{1|b}) = \det \Big(\begin{pmatrix} 4 & 2 & 1 \\ -7 & -1 & \frac{3}{2} \\ -2 & 2 & 0\end{pmatrix}\Big)
% = 0+(-6)+(-14)-2-12-0=-34 \]
% und daher 
% \[ x_1=\frac{\det(A_{1|b})}{\det(A)}=\frac{-34}{-17}=2.\]
% Entsprechend berechnen wir
% \begin{eqnarray*}
% \det(A_{2|b}) &=& \det \Big(\begin{pmatrix} 1 & 4 & 1 \\ 1 & -7 & \frac{3}{2} \\ -4 & -2 & 0\end{pmatrix}\Big)
% = 0+(-24)+(-2)-28-(-3)-0=-51, \\
% \det(A_{3|b}) &=& \det \Big(\begin{pmatrix} 1 & 2 & 4 \\ 1 & -1 & -7 \\ -4 & 2 & -2\end{pmatrix}\Big)
% = 2+56+8-16-(-14)-(-4)=68.
% \end{eqnarray*}
% Damit erhält man weiter
% \[  x_2=\frac{-51}{-17}=3\quad \text{und}\quad x_3=\frac{68}{-17}=-4.\]
% Die Lösungsmenge besteht also aus der eindeutigen Lösung
% \[ \left( \begin{smallmatrix}x_1\\ x_2\\ x_3\end{smallmatrix}\right)
% =\left( \begin{smallmatrix}2\\ 3\\ -4\end{smallmatrix}\right). \]
% \end{example}


% \section{Berechnung der inversen Matrix mit Cramerscher Regel}

% Da man zu einer invertierbaren Matrix $A$ die Spalten der inversen Matrix $A^{-1}$ als
% Lösungen von Gleichungssystemen bekommt, lässt sich auch die inverse Matrix mit Hilfe der
% Cramerschen Regel berechnen.


% \begin{rule}[Cramersche Regel und inverse Matrizen]
% Sei $A \in M(n;\R)$ invertierbar. Mit Hilfe der Cramer'schen Regel l\"asst sich die inverse Matrix berechnen, indem man f\"ur $b$ in der Gleichung $Ax=b$
% der Reihe nach die Vektoren 
% \[
% e_1 := \begin{pmatrix} 1 \\ 0 \\ 0 \\ \vdots \\ 0 \\ 0 \end{pmatrix} , \ e_2 := \begin{pmatrix} 0 \\ 1 \\ 0 \\ \vdots \\ 0 \\ 0 \end{pmatrix}, \ \cdots \ , e_m := \begin{pmatrix} 0 \\ 0 \\ 0 \\ \vdots \\ 0 \\ 1 \end{pmatrix} \in \R^n,
% \]
% einsetzt und die erhaltenen L\"osungen $x$ (abh\"angig von $e_i$, $1 \leq i \leq n$) nebeneinander in eine Matrix schreibt.

% Expliziter ergibt sich der $(i,j)$-Eintrag von $A^{-1}$ als
% \[   (-1)^{i+j} \frac{\det(A_{ji})}{\det(A)}, \]
% wobei $A_{ji}$ die $(n-1)\times (n-1)$-Matrix ist, die man aus $A$ durch Streichen der $j$-ten Zeile und $i$-ten Spalte erhält.
% \end{rule}


% %\begin{block}[explanation]
% \begin{tabs*}[\initialtab{1}]
% \tab{Erklärung}
% Mit der Cramerschen Regel für die Lösung eines linearen Gleichungssystems erhält man für den 
% $(i,j)$-Eintrag von $A^{-1}$ zunächst den Wert
% \[  \frac{\det(A_{i|e_j})}{\det(A)}. \]
% Vertauscht man nun aber zur Berechnung von $\det(A_{i|e_j})$ die $i$-te Zeile nacheinander mit den darüberliegenden
% und dann die $j$-te Spalte nacheinander mit den links davon liegenden, so erhält man eine Block-Dreiecksmatrix der Form
% \[  \begin{pmatrix} 1 & * \\ 0 & B\end{pmatrix},\]
% wobei $*$ irgendeine $1\times (n-1)$-Matrix (=Zeilenvektor) ist und $0$ die $(n-1)\times 1$-Nullmatrix. Die Matrix $B$
% ist jedoch nichts anderes als die Matrix $A_{ji}$, die man aus $A$ durch Streichen der $j$-ten Zeile und 
% $i$-ten Spalte erhält.

% Insgesamt wurden $i-1$ Zeilenvertauschungen und $j-1$ Spaltenvertauschungen vorgenommen. Also ist
% \[  \det(A_{i|e_j})=(-1)^{i-1+j-1}\cdot \det\big( \begin{pmatrix} 1 & * \\ 0 & A_{ji}\end{pmatrix} \big)
% = (-1)^{i+j}\cdot 1\cdot \det(A_{ji}), \]
% wie behauptet.
% \end{tabs*}
% %\end{block}

% \begin{example}
% Wir betrachten wieder die Matrix $ A=\begin{pmatrix} 1 & 2 & 1 \\ 1 & -1 & \frac{3}{2} \\ -4 & 2 & 0\end{pmatrix} $
% aus obigem Beispiel. Deren Determinante war $\det(A)=-17\neq 0$. Die Matrix ist also invertierbar und wir
% können mit der Cramerschen Regel die inverse Matrix berechnen.

% Dazu müssen wir also die Determinanten der Teilmatrizen berechnen, die man erhält, wenn man eine Zeile und eine Spalte
% streicht. Für den $(1,1)$-Eintrag von $A^{-1}$ benötigen wir also
% \[ \det(A_{11})=\det\big( \begin{pmatrix}-1 & \frac{3}{2} \\ 2 & 0 \end{pmatrix} \big)
% =1\cdot 0-2\cdot \frac{3}{2}=-3,\]
% und erhalten den Eintrag als
% \[ (-1)^{1+1}\cdot \frac{\det(A_{11})}{\det(A)}=1\cdot \frac{-3}{-17}=\frac{3}{17}.\]
% Für den $(1,2)$-Eintrag benötigen wir die Matrix $A_{21}$, bei der die zweite Zeile und die erste Spalte von $A$
% gestrichen wurden, also $A_{21}=\left(\begin{smallmatrix}2 & 1 \\ 2 & 0\end{smallmatrix}\right)$.
% Dann erhalten wir als $(1,2)$-Eintrag den Wert
% \[ (-1)^{1+2}\cdot \frac{\det(A_{21})}{\det(A)}=-1\cdot \frac{-2}{-17}=-\frac{2}{17}. \]
% Entsprechend berechnet man die anderen Einträge und erhält
% \begin{eqnarray*} A^{-1} &=&\frac{1}{\det(A)}\cdot
% \begin{pmatrix} \det(A_{11}) & -\det(A_{21}) &\det(A_{31})\\
% -\det(A_{12}) &\det(A_{22}) & -\det(A_{32})\\
% \det(A_{13}) & -\det(A_{23}) &\det(A_{33})
%   \end{pmatrix}\\
% &=& -\frac{1}{17}\cdot  \begin{pmatrix} -3 & 2 & 4\\ -6 & 4 & -\frac{1}{2}\\ -2 & -10 & -3
%   \end{pmatrix}=\frac{1}{17}\cdot \begin{pmatrix} 3 & -2 & -4\\ 6 & -4 & \frac{1}{2}\\ 2 & 10 & 3
%   \end{pmatrix}. \end{eqnarray*}
% \end{example}


\end{content}