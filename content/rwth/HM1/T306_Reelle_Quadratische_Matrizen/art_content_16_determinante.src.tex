%$Id:  $
\documentclass{mumie.article}
%$Id$
\begin{metainfo}
  \name{
    \lang{de}{Determinante}
    \lang{en}{}
  }
  \begin{description} 
 This work is licensed under the Creative Commons License Attribution 4.0 International (CC-BY 4.0)   
 https://creativecommons.org/licenses/by/4.0/legalcode 

    \lang{de}{Beschreibung}
    \lang{en}{}
  \end{description}
  \begin{components}
    \component{generic_image}{content/rwth/HM1/images/g_tkz_T306_Sarrus.meta.xml}{T306_Sarrus}
    \component{generic_image}{content/rwth/HM1/images/g_img_00_video_button_schwarz-blau.meta.xml}{00_video_button_schwarz-blau}
  \end{components}
  \begin{links}
    \link{generic_article}{content/rwth/HM1/T306_Reelle_Quadratische_Matrizen/g_art_content_16_determinante.meta.xml}{content_16_determinante}
    \link{generic_article}{content/rwth/HM1/T306_Reelle_Quadratische_Matrizen/g_art_content_14_quadratische_matrizen.meta.xml}{content_14_quadratische_matrizen}
    \link{generic_article}{content/rwth/HM1/T306_Reelle_Quadratische_Matrizen/g_art_content_15_inverse_matrix.meta.xml}{inverse-matrix}
  \end{links}
  \creategeneric
\end{metainfo}
\begin{content}
\usepackage{mumie.ombplus}
\ombchapter{5}
\ombarticle{3}
\usepackage{mumie.genericvisualization}

\begin{visualizationwrapper}

\title{Determinante}

\begin{block}[annotation]
 
  
\end{block}
\begin{block}[annotation]
  Im Ticket-System: \href{http://team.mumie.net/issues/11619}{Ticket 11619}\\
\end{block}

\begin{block}[info-box]
\tableofcontents
\end{block}

Die Determinante einer quadratischen Matrix $A\in M(n;\R)$ gilt als Kenngröße für die Invertierbarkeit von $A$. 
Es gibt mehrere äquivalente Möglichkeiten, die Determinante zu definieren. Hier soll eine Definition verwendet werden, die zugleich
eine Möglichkeit der Berechnung liefert.

\section{Definition der Determinante}

\begin{definition}\label{def:determinante}
Die \notion{Determinante} einer Matrix $A\in M(n;\R)$, bezeichnet mit $\det(A)$, ist eine reelle Zahl, welche durch folgende Eigenschaften charakterisiert ist:
\begin{enumerate}
\item Ist $A \in M(n;\R) $ eine obere Dreiecksmatrix oder eine untere Dreiecksmatrix mit Diagonaleintr"agen $a_{11}, a_{22}, \ldots, a_{nn}$, so ist
\[ \det(A)=a_{11}\cdot a_{22} \cdots a_{nn} \]
das Produkt der Diagonalelemente.
 \item Erh\"alt man $B \in M(n;\R)$ aus $A$ durch Addition des $r$-fachen der $j$-ten Zeile zur $i$-ten Zeile ($i \neq j$), so gilt f\"ur die Determinanten:
       \[
       \det(A)=\det(B).
       \]
 \item Erh\"alt man $B$ aus $A$ durch Multiplikation der $i$-ten Zeile mit 
 $r \in \R \setminus \{ 0 \}$, so gilt
       \[
       \det(A) = \frac{1}{r} \cdot  \det(B).
       \]
 \item Erh\"alt man $B$ aus $A$ durch Vertauschen der $i$-ten und $j$-ten Zeile, so gilt
       \[
       \det(A)=- \det(B).
       \]
\end{enumerate}

\floatright{\href{https://api.stream24.net/vod/getVideo.php?id=10962-2-11380&mode=iframe&speed=true}{\image[75]{00_video_button_schwarz-blau}}}\\

\end{definition}

\begin{quickcheck}
\type{input.number}
  \field{complex}
  \displayprecision{3}
  \correctorprecision{4}
 
  \begin{variables}
   \function[calculate]{a}{6}
   \function[calculate]{b}{-1}
  \end{variables}
\text{Bestimmen Sie die Determinanten von $A=\begin{pmatrix}1&0&0\\2&2&0\\1&0&3\end{pmatrix}$ und 
$B=\begin{pmatrix}0&1\\1&0\end{pmatrix}$.

Antwort: $\det(A)=$\ansref und $\det(B)=$\ansref.}
 \begin{answer}
    \solution{a}
  \end{answer}
   \begin{answer}
    \solution{b}
   \end{answer}
   \explanation{$A$ ist eine untere Dreiecksmatrix. 
   Die Determinante berechnet sich also als Produkt der Diagonalglieder $\det(A)=1\cdot 2\cdot 3=6$. 
   Vertauscht man in $B$ die zwei Zeilen, so erhält man die Einheitsmatrix $E_2$,
   deren Determinante gleich $1$ ist. 
   Die Determinante von $B$ erhält noch das Vorzeichen, das durch die Vertauschung aufgenommen wird: $\det(B)=-1$.}
\end{quickcheck}

\begin{remark}
\begin{itemize}
\item Da man mittels Zeilenumformungen jede quadratische Matrix auf obere Dreiecksgestalt bringen kann (z.B. mit dem Gauß-Verfahren), ist somit für jede Matrix $A$ ein Wert $\det(A)$ definiert. Auch zeigt diese Definition, wie man die Determinante berechnet, nämlich durch Zeilenumformungen.
\item Genau genommen müsste man noch nachprüfen, dass dieses Verfahren auch tatsächlich für jede Matrix $A$ einen eindeutigen Wert $\det(A)$ liefert, und nicht davon abhängt, über welche Zeilenumformungen man zu einer oberen Dreiecksmatrix kommt. Darauf soll hier aber verzichtet werden. 
\end{itemize}
\end{remark}

\begin{example}
\begin{enumerate}
\item Die Einheitsmatrix $E_n$ hat als Determinante
\[ \det(E_n)=1\cdot 1\cdots 1=1. \]
\item Wir betrachten die reelle $(3\times 3)$-Matrix 
\[ A= \begin{pmatrix}
-2 & 4 & 6 \\ 1 & -1 & 1 \\ 2 & -3 & -4 \end{pmatrix}. \]
Mit Gauß-Verfahren erhalten wir die Umformungen
\begin{eqnarray*}
&& \begin{pmatrix} -2 & 4 & 6 \\ 1 & -1 & 1  \\ 2 & -3 & -4  \end{pmatrix} 
\begin{matrix} / \cdot (-1/2) \\ \phantom{1}\\ \phantom{1} \end{matrix} \rightsquigarrow 
\begin{pmatrix} 1 & -2 & -3  \\ 1 & -1 & 1  \\ 2 & -3 & -4  \end{pmatrix} 
\begin{matrix} \phantom{1} \\ / - 1\cdot \text{(I)}\\ / -2\cdot \text{(I)} \end{matrix} \rightsquigarrow  \\ &&\\
& \rightsquigarrow & \begin{pmatrix} 1 & -2 & -3  \\ 0 & 1 & 4  \\ 0 & 1 & 2 \end{pmatrix} 
\begin{matrix} \phantom{1} \\  \phantom{1}\\ / -1\cdot \text{(II)} \end{matrix} \rightsquigarrow 
\begin{pmatrix} 1 & -2 & -3  \\ 0 & 1 & 4  \\ 0 & 0 & -2 \end{pmatrix} .
\end{eqnarray*}
Damit gilt unter Verwendung der Eigenschaften aus Definition \ref{def:determinante}
\begin{eqnarray*} \det(A) 
&=& (-2)\cdot \det\Big(\left( \begin{smallmatrix} 1 & -2 & -3  \\ 1 & -1 & 1  \\ 2 & -3 & -4  \end{smallmatrix}\right)\Big) 
=-2\cdot \det\Big(\left(  \begin{smallmatrix} 1 & -2 & -3  \\ 0 & 1 & 4  \\ 0 & 1 & 2 \end{smallmatrix} \right)\Big)\\
&=& -2\cdot \det\Big(\left(  \begin{smallmatrix} 1 & -2 & -3  \\ 0 & 1 & 4  \\ 0 & 0 & -2 \end{smallmatrix} \right)\Big)
=-2\cdot \big( 1\cdot 1\cdot (-2)\big)=4.
\end{eqnarray*}
\end{enumerate}
\end{example}

\begin{example}
Die Determinanten der \ref[content_14_quadratische_matrizen][Elementarmatrizen]{ex:einfache_bsp_quad_matrizen} lassen sich leicht berechnen:
\begin{incremental}
\step
%\begin{itemize}
% \item 
Die Matrix $A_{ij}(r)$, $r \in \R$, $i\neq j$, deren Diagonalelemente alle $1$ sind, deren Eintrag an der Stelle $(i,j)$ gleich $r$ ist und die sonst nur Nulleinträge hat, ist insbesondere eine untere oder obere Dreiecksmatrix. Daher ist
       \[
       \det(A_{ij}(r))=1\cdot 1\cdots 1=1.
       \] 
\step
% \item 
Die Matrix $M_{i}(r)$, $r \in \R \setminus \lbrace 0 \rbrace $, $i\in\{ 1,\ldots,n\}$,  deren $i$-tes Diagonalelement gleich $r$ ist, alle anderen Diagonalelemente gleich $1$, und die sonst nur Nulleinträge hat, ist sogar eine Diagonalmatrix. Daher ist:
       \[
       \det(M_{i}(r))=1\cdots 1\cdot r\cdot 1\cdots 1=r.
       \]
\step
% \item 
Die Matrix $V_{ij}$ unterscheidet sich von der Einheitsmatrix $E_n$ nur dadurch, dass die $i$-te und die $j$-te Zeile vertauscht sind. Daher ist 
       \[
       \det(V_{ij})=-\det(E_n)=-1.
       \]
\end{incremental}
%\end{itemize}

\end{example}


Ein wichtiger Grund, wofür Determinanten verwendet werden, ist der folgende.

\begin{theorem}
Eine quadratische Matrix $A\in M(n;\R)$ ist genau dann invertierbar, wenn ihre Determinante von $0$ verschieden ist.
\end{theorem}

\begin{proof*}[Beweis des Satzes]
\begin{showhide}
Durch Anwenden elementarer Zeilenumformungen kann sich zwar der Wert der Determinante ändern, jedoch nicht die Eigenschaft, ob sie $0$ ist oder nicht.
Denn dabei wird nur mit Elementen ungleich $0$ multipliziert.
Bezeichnen wir mit $S$ die reduzierte Zeilenstufenform von $A$, dann gilt also:
\begin{eqnarray*}
\det(A)=0 &\Leftrightarrow & \det(S)=0 \\
&\Leftrightarrow & S\text{ besitzt weniger als $n$ Stufen} \quad (\text{da $S$ obere Dreiecksmatrix})\\
&\Leftrightarrow & S\text{ hat Rang kleiner als }n\quad (\text{da Rang($S$)=Anzahl der Stufen})\\
&\Leftrightarrow & A\text{ hat Rang kleiner als }n\quad (\text{da Rang($A$)=Rang($S$)})\\
&\Leftrightarrow & A\text{ ist nicht invertierbar}.
\end{eqnarray*}
\end{showhide}
\end{proof*}

%\floatright{\href{https://api.stream24.net/vod/getVideo.php?id=10962-2-10879&mode=iframe&speed=true}{\image[75]{00_video_button_schwarz-blau}}}\\

Das folgende Video zeigt zunächst den Beweis zu Satz 5.3.5. Im Anschluss wird ein Beispiel vorgerechnet:
\floatright{\href{https://api.stream24.net/vod/getVideo.php?id=10962-2-11381&mode=iframe&speed=true}{\image[75]{00_video_button_schwarz-blau}}}\\


\section{Determinanten kleiner Matrizen}\label{sec:determinanten-kleine-matrizen}

Für Matrizen kleiner Größe gibt es einfache Formeln, die Determinante zu berechnen.

\begin{rule}
\begin{enumerate}
\item Für $(1\times 1)$-Matrizen $A=(a)\in M(1;\R)$ gilt nach Definition $\det(A)=a\in \R$.
\item Für $(2\times 2)$-Matrizen $A= \begin{pmatrix}
a & b \\ c& d \end{pmatrix} \in M(2;\R) $ ist
\[ \det(A)=ad-bc.\]
\item Für $(3\times 3)$-Matrizen $A=\left( \begin{smallmatrix}
a_{11} & a_{12} &a_{13} \\ a_{21}& a_{22} &a_{23} \\ a_{31}& a_{32}& a_{33}
\end{smallmatrix}\right)$ ist
\begin{eqnarray*}
\det(A) &=& \ a_{11} a_{22} a_{33} + a_{12} a_{23} a_{31} + a_{13} a_{21}a_{32} \\
                   & & - a_{13} a_{22} a_{31} - a_{11} a_{23} a_{32} - a_{12} a_{21} a_{33}.
\end{eqnarray*}
\end{enumerate}
\end{rule}

\begin{remark}
\begin{enumerate}
\item Den Ausdruck $ad-bc$ für $(2\times 2)$-Matrizen hatten wir im 
\ref[inverse-matrix][vorigen Abschnitt]{rule:inverse-2x2} bei der Berechnung der inversen Matrix als Kriterium 
für die Invertierbarkeit entdeckt. 
\item Die Formel für $(3\times 3)$-Matrizen wird auch Regel von Sarrus genannt. Sie lässt sich mit dem folgenden Schema 
leicht merken:
\begin{center}
\image{T306_Sarrus}
\end{center}

Erg\"anze die ersten beiden Spalten der Matrix $A$ noch einmal zur Matrix $A$ als 4. und 5. Spalte. Dann addiere man die 
Produkte der Diagonalen von links oben nach rechts unten und subtrahiere
die Produkte der Diagonalen von links unten nach rechts oben.
\end{enumerate}
\end{remark}

\begin{example}
Wir betrachten wieder die reelle $(3\times 3)$-Matrix 
\[ A= \begin{pmatrix}
-2 & 4 & 6 \\ 1 & -1 & 1 \\ 2 & -3 & -4 \end{pmatrix}. \]
Schreibt man die ersten zwei Spalten nochmal daneben, erhält man das Schema
\[ \begin{matrix}
-2 & 4 & 6 &: &-2 & 4\\ 1 & -1 & 1&: &1 & -1 \\ 2 & -3 & -4&: &2 & -3
\end{matrix}. \]
Damit berechnen wir also
\begin{eqnarray*}
 \det(A)&=&(-2)\cdot (-1)\cdot (-4)+4\cdot 1\cdot 2
+ 6\cdot 1\cdot (-3) \\ 
&& - 2\cdot (-1)\cdot 6- (-3)\cdot 1\cdot (-2)
- (-4)\cdot 1\cdot 4 \\
&=& -8+8-18+12-6+16=4.
\end{eqnarray*}
\end{example}


\begin{block}[warning]
Die Regel von Sarrus gilt ausschließlich für $(3\times 3)$-Matrizen. Für größere Matrizen ist
eine derartige Regel im Allgemeinen falsch!
\\
Für $(2\times 2)$-Matrizen sieht man das sofort ein: Eine übertragene Sarrus-Regel würde hier
den Term $ad+bc-bc-ad=0$ liefern.
\end{block}



\section{Rechenregeln für Determinanten}\label{sec:rechenregeln-determinante}
Auch Determinanten von quadratischen Matrizen beliebigen Ausmaßes sind gut berechenbar. Wir geben hier nur zwei Möglichkeiten an.
\begin{rule}[Berechnung von allgemeinen Determinanten]\label{rule:Laplace}
Die Determinante einer Matrix $A=(a_{ij})_{ij}\in M(n;\R)$ kann wie folgt berechnet werden.
\begin{itemize}
\item[(i)] Man wendet die \ref[content_16_determinante][Definition der Determinante]{def:determinante} an. 
Durch Multiplikation mit Elementarmatrizen bringt man $A$ also auf eine Stufenform $S$, die eine (obere) Dreiecksmatrix ist.
Deren Determinante kann man daher direkt ablesen, die der Elementarmatrizen kennt man.
Bezeichnen $M_1,\ldots,M_r$ die verwendeten Elementarmatrizen, dann gilt
\[\det(A)=\det(M_1)^{-1}\cdot\ldots\cdot\det(M_r)^{-1}\cdot\det(S).\]
\item[(ii)]
Es sei $A_{ij}\in M(n-1;\R)$ die Matrix, die aus $A$ entsteht, wenn man die $i$-te Zeile und $j$-te Spalte streicht.
\notion{Laplace-Entwicklung nach der $i$-ten Zeile:}
\[\det(A)=\sum_{j=1}^n(-1)^{i+j}a_{ij}\cdot\det(A_{ij}).\]
\notion{Laplace-Entwicklung nach der $j$-ten Spalte:}
\[\det(A)=\sum_{i=1}^n(-1)^{i+j}a_{ij}\cdot\det(A_{ij}).\]
\end{itemize}
\end{rule}

Im nachfolgenden Video wird die Laplace-Entwicklung erklärt und durch ein Beispiel veranschaulicht:

\floatright{\href{https://api.stream24.net/vod/getVideo.php?id=10962-2-10882&mode=iframe&speed=true}{\image[75]{00_video_button_schwarz-blau}}}\\

\begin{quickcheck}
\type{input.number}
  \field{rational}
  \displayprecision{3}
  \correctorprecision{4}
 
  \begin{variables}
   \function{a}{-6}
   \end{variables}
%\text{Es sei $A=\begin{pmatrix}1&2&3\\4&5&6\\-7&8&9\end{pmatrix}$. Dann ist $\det(A_{21})=$\ansref.}
\text{Es sei $A=\begin{pmatrix}1&2&3\\4&5&6\\-7&8&9\end{pmatrix}$. Dann ist $\det(A_{21})=$\ansref.}
 \begin{answer}
    \solution{a}
  \end{answer}
   \explanation{$A_{21}$ entsteht aus $A$ durch streichen der zweiten Zeile und ersten Spalte. 
   Also ist $A_{21}=\begin{pmatrix}2&3\\8&9\end{pmatrix}$ und $\det(A_{21})=2\cdot 9-8\cdot 3=-6$.}
\end{quickcheck}
\begin{remark}
\begin{itemize}
\item
Bei der Laplace-Entwicklung wird also die Berechnung der Determinante zurückgeführt auf die Berechnung von
$n$  Determinanten $\det(A_{ij})$ kleinerer Größe, sogenannten Unterdeterminanten von $A$.
\item Die Laplace-Entwicklung eignet sich besonders gut bei dünnbesetzten Zeilen oder Spalten, also solchen bei denen viele Einträge gleich null sind.
Ist nämlich ein $a_{ij}=0$, dann taucht der entsprechende Term in der Summe nicht auf. Man braucht dann $\det(A_{ij})$ gar nicht erst zu berechnen, was den Rechenaufwand erheblich reduziert.
\item Wenn die Matrix sehr voll besetzt ist, dann ist Methode (i) meist schneller. Man braucht die Matrix dazu nicht einmal in reduzierte Stufenform zu bringen, irgendeine Dreiecksform reicht aus.
\item
Es gibt allgemeinere Versionen der Laplace-Entwicklung und noch viele andere  Möglichkeiten, Determinanten zu berechnen.
Der tiefere mathematische Grund dafür liegt in den faszinierenden Eigenschaften, die die Determinante hat.
Deren Diskussion geht aber weit über das hinaus, was in der praktischen Anwendung benötigt wird. 
Deshalb verzichten wir hier ebenso darauf, wie auf einen Beweis der Laplace-Entwicklung nach der $i$-ten Zeile oder $j$-ten Spalte.
\end{itemize}
\end{remark}
\begin{example}
\begin{incremental}[\initialsteps{1}]
\step Wir berechnen die Determinante von $A=\begin{pmatrix} -1&0&0&2\\2&3&0&1\\1&5&0&0\\0&0&3&2\end{pmatrix}$
mit Hilfe der Laplace-Entwicklung. 
\step
Am dünnbesetztesten ist hier die dritte Spalte, also entwickeln wir nach dieser:
\[\det(A)=\sum_{i=1}^4(-1)^{i+3}a_{i3}\det(A_{i3}),\]
worin alle $a_{i3}$ bis auf $a_{43}$ gleich null sind.
Es ist also
\[\det(A)=(-1)^{4+3}\cdot 3\cdot\det(A_{43})=(-3)\cdot\det\Big(\left(\begin{smallmatrix}
-1&0&2\\2&3&1\\1&5&0\end{smallmatrix}\right)\Big).\]
\step
Diese Determinante könnten wir mit der Regel von Sarrus berechnen, aber wir entwickeln sie hier weiter nach der ersten Zeile:
 \begin{align*}
 \det\Big(\left(\begin{smallmatrix}
-1&0&2\\2&3&1\\1&5&0\end{smallmatrix}\right)\Big)&=(-1)^{1+1}\cdot(-1)\cdot\det\Big(\left(\begin{smallmatrix}3&1\\5&0\end{smallmatrix}\right)\Big)+0
 +(-1)^{1+3}\cdot 2\cdot\det\Big(\left(\begin{smallmatrix}2&3\\1&5\end{smallmatrix}\right)\Big)\\
 &=(-1)(-5)+2(10-3)=19,
 \end{align*}
 wobei wir die Determinantenregel für $(2\times 2)$-Matrizen benutzt haben.
Insgesamt finden wir also 
\[\det(A)=(-3)\cdot 19=-57.\]
\end{incremental}
\end{example}

Für Determinanten gelten weitere wichtige Rechenregeln
\begin{rule}\label{rule:rechenregeln}
\begin{itemize}
 \item[(i)] \notion{Determinantenmultiplikationssatz:} Seien $A, B \in M(n;\R)$. Dann gilt
       \[
       \det(A \cdot B)=\det(A) \cdot \det(B).
       \]
 \item[(ii)] Ist $A \in M(n;\R)$ mit $\det(A) \neq 0$ (d.h. $A$ ist invertierbar), dann gilt f\"ur die Determinante der inversen Matrix $A^{-1}$:
       \[
       \det(A^{-1})= \frac{1}{\det(A)}.
       \]
 \item[(iii)] Die in der Definition angegebenen Gleichungen für die Determinanten nach Zeilenumformungen gelten genauso auch für Spaltenumformungen.
 \item[(iv)] Kann man $A$ in der Blockmatrix-Form
       \[
       A= \begin{pmatrix} B & C \\  0_{r,s} & D \end{pmatrix} \text{ oder } A=\begin{pmatrix} B & 0_{s,r} \\ C & D \end{pmatrix}
       \]
       schreiben, wobei $B \in M(s;\R)$, $D \in M(r;\R)$, f\"ur $r$, $s \in \N$, dann berechnet sich die Determinante von $A$ wie folgt:
       \[
       \det(A)= \det(B) \cdot \det(D).
       \]
\item[(v)] Ist $A^T\in M(n;\R)$ die transponierte Matrix zu $A$, so gilt  \[
       \det(A^T)= \det(A).
       \]
\end{itemize}
\end{rule}
%% Der folgende Beweis ist mit den Kenntnissen des Vertiefungsteils 3a allein nicht verständlich.
% \begin{proof*}
% \begin{showhide}
% \begin{itemize}
% \item[(i)]
% Wir benutzen, dass jede invertierbare Matrix als Produkt von Elementarmatrizen geschrieben werden kann.
% Ist $A$ also invertierbar, so reicht es zu zeigen, 
% dass für jede Matrix $A\in M(n;\R)$ und jede Elementarmatrix $M\in M(n;\R)$ gilt
% \[\det(M\cdot A)=\det(M)\cdot\det(A).\]
% Das ist allerdings klar nach der \ref[content_09_determinante][Definition der Determinante]{def:determinante}.
% Ist in dem Produkt $A\cdot B$ eine der Matrizen nicht invertierbar, hat also nicht vollen Rang, dann gilt das auch für das Produkt.
% Dann ist 
% \[\det(A\cdot B)=0=\det(A)\cdot\det(B).\]
% \item[(ii)]
% Wir wenden den Determinantenmultiplikationssatz (i) an auf die Identität $A\cdot A^{-1}=E_n$ und erhalten
% \[\det(A)\cdot\det(A^{-1})=\det(E_n)=1.\]
% Weil $\det(A)\neq 0$, ist das gleichbedeutend zu
% \[\det(A^{-1})=\frac{1}{\det(A)}.\]
% \item[(iii)] Die Spaltenumformungen erhalten wir durch Multiplikation mit Elementarmatrizen von rechts (statt von links wie bei Zeilenumformungen).
% Somit liefert der Determinantenmultiplikationssatz (i) die Behauptung. 
% \item[(iv)]
% Um die Determinante dieser Blockmatrix zu berechnen, bringt man sie auf Stufenform. Die dazu nötigen elementaren Zeilenumformungen lassen sich aufteilen in die für den
% oberen linken Block und die für den unteren rechten Block. Die dadurch entstehende Stufenform ist eine Dreiecksmatrix, und für diese gilt die Formel aus
% (iv). Also gilt sie auch für die Ausgangsmatrix.
% \item[(v)] Wieder benutzen wir, dass sich $A$ entweder als Produkt von Elementarmatrizen schreiben lässt oder nicht vollen Rang hat.
% Im zweiten Fall hat auch $A^T$ nicht vollen Rang (Spaltenrang$=$Zeilenrang). Es gilt also $\det(A)=0=\det(A^T)$.
% Im ersten Fall sei $A=M_1\cdots M_r$ mit Elementarmatrizen $M_j$, $j=1,\ldots,r$.
% Für die Elementarmatrizen gilt aber offensichtlich $\det(M_j^T)=\det(M_j)$. Also gilt
% \[\det(A^T)=\det(M_r^T\cdots M_1^T)=\det(M_r^T)\cdots\det(M_1^T)=
% \det(M_1)\cdots\det(M_r)=\det(M_1\cdots M_r)=\det(A),\]
% wobei wir die \ref[content_03_transponierte][Rechenregel für transponierte Matrizen]{sec:rechenregeln} und den Determinantenmultiplikationssatz (i) verwendet haben.
% \end{itemize}
% \end{showhide}
% \end{proof*}
% %%
\begin{quickcheck}
\type{input.number}
  \field{rational}
  \displayprecision{3}
  \correctorprecision{4}
 
  \begin{variables}
   \function{a}{5}
   \function{b}{3}
   \function{d}{1/5}
   \function{c}{15}
   \function{f}{5/3}
  \end{variables}
\text{Für zwei Matrizen $A,B\in M(3;\R)$ sei $\det(A)=5$ und $\det(B)=3$.
Dann ist $\det(A\cdot B)=$\ansref, $\det A^{-1}=$\ansref, $\det(B^T)=$\ansref und $\det(B^{-1}A^T)=$\ansref.}
 \begin{answer}
    \solution{c}
  \end{answer}
   \begin{answer}
    \solution{d}
   \end{answer}
   \begin{answer}
    \solution{b}
   \end{answer}
   \begin{answer}
    \solution{f}
   \end{answer}
   \end{quickcheck}

Das folgende Video zeigt die Berechnung von $2 \times2$ - Matrizen,
die Regel von Sarrus,
sowie den Determinantenmultiplikationssatz.
\floatright{\href{https://api.stream24.net/vod/getVideo.php?id=10962-2-10880&mode=iframe&speed=true}{\image[75]{00_video_button_schwarz-blau}}}\\


% Die Determinante einer quadratischen Matrix $A\in M(n;\R)$ gilt als Kenngröße für die Invertierbarkeit von $A$. 
% Es gibt mehrere äquivalente Möglichkeiten, die Determinante zu definieren. Hier soll eine Definition verwendet werden, die zugleich
% eine Möglichkeit der Berechnung liefert.

% \section{Definition der Determinante}

% \begin{definition}
% Die \notion{Determinante} einer Matrix $A\in M(n;\R)$, bezeichnet mit $\det(A)$, ist eine reelle Zahl, welche 
% durch folgende Eigenschaften charakterisiert ist:
% \begin{enumerate}
% \item Ist $A \in M(n;\R) $ eine obere Dreiecksmatrix oder eine untere Dreiecksmatrix mit Diagonaleintr"agen $a_{11}, a_{22}, \ldots, a_{nn}$, so ist
% \[ \det(A)=a_{11}\cdot a_{22} \cdots a_{nn} \]
% das Produkt der Diagonalelemente.
%  \item Erh\"alt man $B \in M(n;\R)$ aus $A$ durch Addition des $r$-fachen der $j$-ten Zeile zur $i$-ten Zeile ($i \neq j$), so gilt f\"ur die Determinanten:
%        \[
%        \det(A)=\det(B).
%        \]
%  \item Erh\"alt man $B$ aus $A$ durch Multiplikation der $i$-ten Zeile mit 
%  $r \in \R \setminus \{ 0 \}$, so gilt
%        \[
%        \det(A) = \frac{1}{r} \cdot  \det(B).
%        \]
%  \item Erh\"alt man $B$ aus $A$ durch Vertauschen der $i$-ten und $j$-ten Zeile, so gilt
%        \[
%        \det(A)=- \det(B).
%        \]
% \end{enumerate}
% \end{definition}

% \begin{remark}
% \begin{itemize}
% \item Da man mittels Zeilenumformungen jede quadratische Matrix auf obere Dreiecksgestalt bringen kann (z.B. mit dem Gauß-Verfahren), ist somit für jede Matrix $A$ ein Wert $\det(A)$ definiert. Auch zeigt diese Definition, wie man die Determinante berechnet, nämlich durch Zeilenumformungen.
% \item Genau genommen müsste man noch nachprüfen, dass dieses Verfahren auch tatsächlich für jede Matrix $A$ einen eindeutigen Wert $\det(A)$ liefert, und nicht davon abhängt, über welche Zeilenumformungen man zu einer oberen Dreiecksmatrix kommt. Darauf soll hier aber verzichtet werden. 
% \end{itemize}
% \end{remark}

% \begin{example}
% \begin{enumerate}
% \item Die Einheitsmatrix $E_n$ hat als Determinante
% \[ \det(E_n)=1\cdot 1\cdots 1=1. \]
% \item Wir betrachten die reelle $(3\times 3)$-Matrix 
% \[ A= \begin{pmatrix}
% -2 & 4 & 6 \\ 1 & -1 & 1 \\ 2 & -3 & -4 \end{pmatrix}. \]
% Mit Gauß-Verfahren erhalten wir die Umformungen:
% \begin{eqnarray*}
% && \begin{pmatrix} -2 & 4 & 6 \\ 1 & -1 & 1  \\ 2 & -3 & -4  \end{pmatrix} 
% \begin{matrix} / \cdot (-1/2) \\ \phantom{1}\\ \phantom{1} \end{matrix} \rightsquigarrow 
% \begin{pmatrix} 1 & -2 & -3  \\ 1 & -1 & 1  \\ 2 & -3 & -4  \end{pmatrix} 
% \begin{matrix} \phantom{1} \\ / - 1\cdot \text{(I)}\\ / -2\cdot \text{(I)} \end{matrix} \rightsquigarrow  \\ &&\\
% & \rightsquigarrow & \begin{pmatrix} 1 & -2 & -3  \\ 0 & 1 & 4  \\ 0 & 1 & 2 \end{pmatrix} 
% \begin{matrix} \phantom{1} \\  \phantom{1}\\ / -1\cdot \text{(II)} \end{matrix} \rightsquigarrow 
% \begin{pmatrix} 1 & -2 & -3  \\ 0 & 1 & 4  \\ 0 & 0 & -2 \end{pmatrix} 
% \end{eqnarray*}
% Damit gilt:
% \begin{eqnarray*} \det(A) 
% &=& (-2)\cdot \det\left( \begin{pmatrix} 1 & -2 & -3  \\ 1 & -1 & 1  \\ 2 & -3 & -4  \end{pmatrix}\right) 
% =-2\cdot \det\left(  \begin{pmatrix} 1 & -2 & -3  \\ 0 & 1 & 4  \\ 0 & 1 & 2 \end{pmatrix} \right)\\
% &=& -2\cdot \det\left(  \begin{pmatrix} 1 & -2 & -3  \\ 0 & 1 & 4  \\ 0 & 0 & -2 \end{pmatrix} \right)
% =-2\cdot \big( 1\cdot 1\cdot (-2)\big)=4.
% \end{eqnarray*}
% \end{enumerate}
% \end{example}


% Ein wichtiger Grund, wofür Determinanten verwendet werden, ist der folgende.

% \begin{theorem}
% Eine quadratische Matrix $A\in M(n;\R)$ ist genau dann invertierbar, wenn ihre Determinante von $0$ verschieden 
% ist.
% \end{theorem}

% \begin{block}[explanation]
% Bei Zeilenumformungen kann sich zwar der Wert der Determinante ändern, jedoch nicht die Eigenschaft, ob sie $0$ ist oder nicht, da nur mit Elementen ungleich $0$ multipliziert wird.
% Bezeichnen wir mit $S$ die reduzierte Zeilenstufenform von $A$, so gilt also:
% \begin{eqnarray*}
% \det(A)=0 &\Leftrightarrow & \det(S)=0 \\
% &\Leftrightarrow & S\text{ besitzt weniger als $n$ Stufen} \quad (\text{da $S$ obere Dreiecksmatrix})\\
% &\Leftrightarrow & S\text{ hat Rang kleiner als }n\quad (\text{da Rang($S$)=Anzahl der Stufen})\\
% &\Leftrightarrow & A\text{ hat Rang kleiner als }n\quad (\text{da Rang($A$)=Rang($S$)})\\
% &\Leftrightarrow & A\text{ ist nicht invertierbar}.
% \end{eqnarray*}
% \end{block}


% \section{Determinanten kleiner Matrizen}\label{sec:determinanten-kleine-matrizen}

% Für Matrizen kleiner Größe gibt es einfache Formeln, die Determinante zu berechnen.

% \begin{rule}
% \begin{enumerate}
% \item Für $(1\times 1)$-Matrizen $A=(a)\in M(1;\R)$ gilt nach Definition $\det(A)=a\in \R$.
% \item Für $(2\times 2)$-Matrizen $A= \begin{pmatrix}
% a & b \\ c& d \end{pmatrix} \in M(2;\R) $ ist
% \[ \det(A)=ad-bc.\]
% \item Für $(3\times 3)$-Matrizen $A=\left( \begin{smallmatrix}
% a_{11} & a_{12} &a_{13} \\ a_{21}& a_{22} &a_{23} \\ a_{31}& a_{32}& a_{33}
% \end{smallmatrix}\right)$ ist
% \begin{eqnarray*}
% \det(A) &=& \ a_{11} a_{22} a_{33} + a_{12} a_{23} a_{31} + a_{13} a_{21}a_{32} \\
%                    & & - a_{13} a_{22} a_{31} - a_{11} a_{23} a_{32} - a_{12} a_{21} a_{33}.
% \end{eqnarray*}
% \end{enumerate}
% \end{rule}

% \begin{remark}
% \begin{enumerate}
% \item Den Ausdruck $ad-bc$ für $(2\times 2)$-Matrizen hatten wir im 
% \ref[inverse-matrix][vorigen Abschnitt]{rule:inverse-2x2} bei der Berechnung der inversen Matrix als Kriterium 
% für die Invertierbarkeit entdeckt. 
% \item Die Formel für $(3\times 3)$-Matrizen wird auch Sarrus-Regel genannt. Sie lässt sich mit dem folgenden Schema 
% leicht merken:
% \begin{center}
% \image[300]{sarrus}
% \end{center}

% Erg\"anze die ersten beiden Spalten der Matrix $A$ noch einmal zur Matrix $A$ als 4. und 5. Spalte. Dann addiere man die 
% Produkte der Diagonalen von links oben nach rechts unten und subtrahiere
% die Produkte der Diagonalen von links unten nach rechts oben.
% \end{enumerate}
% \end{remark}

% \begin{example}
% Wir betrachten wieder die reelle $(3\times 3)$-Matrix 
% \[ A= \begin{pmatrix}
% -2 & 4 & 6 \\ 1 & -1 & 1 \\ 2 & -3 & -4 \end{pmatrix}. \]
% Schreibt man die ersten zwei Spalten nochmal daneben, erhält man das Schema
% \[ \begin{matrix}
% -2 & 4 & 6 &: &-2 & 4\\ 1 & -1 & 1&: &1 & -1 \\ 2 & -3 & -4&: &2 & -3
% \end{matrix}. \]
% Damit berechnen wir also:
% \begin{eqnarray*}
%  \det(A)&=&(-2)\cdot (-1)\cdot (-4)+4\cdot 1\cdot 2
% + 6\cdot 1\cdot (-3) \\ 
% && - 2\cdot (-1)\cdot 6- (-3)\cdot 1\cdot (-2)
% - (-4)\cdot 1\cdot 4 \\
% &=& -8+8-18+12-6+16=4
% \end{eqnarray*}
% \end{example}


% \begin{block}[warning]
% Die Sarrus-Regel gilt ausschließlich für $(3\times 3)$-Matrizen. Für größere Matrizen ist
% eine derartige Regel im Allgemeinen falsch!

% Zwar gibt es auch für größere Matrizen eine Formel für deren Determinante, jedoch werden diese komplizierter (Stichwort \emph{Leibnizformel}).
% \end{block}


% \section{Rechenregeln für Determinanten}

% \begin{rule}\label{rule:rechenregeln}
% \begin{itemize}
%  \item Seien $A, B \in M(n;\R)$. Dann gilt
%        \[
%        \det(A \cdot B)=\det(A) \cdot \det(B).
%        \]
%  \item Ist $A \in M(n;\R)$ mit $\det(A) \neq 0$ (d.h. $A$ ist invertierbar), dann gilt f\"ur die Determinante der inversen Matrix $A^{-1}$:
%        \[
%        \det(A^{-1})= \frac{1}{\det(A)}.
%        \]
%  \item Die in der Definition angegebenen Gleichungen für die Determinanten nach Zeilenumformungen gelten genauso auch für Spaltenumformungen.
%  \item Kann man $A$ in der Blockmatrix-Form
%        \[
%        A= \begin{pmatrix} B & C \\  0_{r,s} & D \end{pmatrix} \text{ oder } A=\begin{pmatrix} B & 0_{s,r} \\ C & D \end{pmatrix}
%        \]
%        schreiben, wobei $B \in M(s;\R)$, $D \in M(r;\R)$, f\"ur $r$, $s \in \N$, dann berechnet sich die Determinante von $A$ wie folgt:
%        \[
%        \det(A)= \det(B) \det(D).
%        \]
%        \item Ist $A^T\in M(n;\R)$ die transponierte Matrix zu $A$, so gilt  \[
%        \det(A^T)= \det(A).
%        \]
% \end{itemize}
% \end{rule}

\end{visualizationwrapper}

\end{content}