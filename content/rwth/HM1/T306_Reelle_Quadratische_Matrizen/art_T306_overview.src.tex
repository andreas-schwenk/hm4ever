%$Id:  $
\documentclass{mumie.article}
%$Id$
\begin{metainfo}
  \name{
    \lang{de}{Überblick: Quadratische Matrizen und Determinanten}
    \lang{en}{overview: }
  }
  \begin{description} 
 This work is licensed under the Creative Commons License Attribution 4.0 International (CC-BY 4.0)   
 https://creativecommons.org/licenses/by/4.0/legalcode 

    \lang{de}{Beschreibung}
    \lang{en}{}
  \end{description}
  \begin{components}
  \end{components}
  \begin{links}
\link{generic_article}{content/rwth/HM1/T306_Reelle_Quadratische_Matrizen/g_art_content_17_cramersche_regel.meta.xml}{content_17_cramersche_regel}
\link{generic_article}{content/rwth/HM1/T306_Reelle_Quadratische_Matrizen/g_art_content_16_determinante.meta.xml}{content_16_determinante}
\link{generic_article}{content/rwth/HM1/T306_Reelle_Quadratische_Matrizen/g_art_content_15_inverse_matrix.meta.xml}{content_15_inverse_matrix}
\link{generic_article}{content/rwth/HM1/T306_Reelle_Quadratische_Matrizen/g_art_content_14_quadratische_matrizen.meta.xml}{content_14_quadratische_matrizen}
\end{links}
  \creategeneric
\end{metainfo}
\begin{content}
\begin{block}[annotation]
	Im Ticket-System: \href{https://team.mumie.net/issues/30120}{Ticket 30120}
\end{block}




\begin{block}[annotation]
Im Entstehen: Überblicksseite für Kapitel Quadratische Matrizen und Determinanten
\end{block}

\usepackage{mumie.ombplus}
\ombchapter{1}
\lang{de}{\title{Überblick: Quadratische Matrizen und Determinanten}}
\lang{en}{\title{}}



\begin{block}[info-box]
\lang{de}{\strong{Inhalt}}
\lang{en}{\strong{Contents}}


\lang{de}{
    \begin{enumerate}%[arabic chapter-overview]
   \item[5.1] \link{content_14_quadratische_matrizen}{Quadratische Matrizen}
   \item[5.2] \link{content_15_inverse_matrix}{Invertierbare Matrizen und Inverse}
   \item[5.3] \link{content_16_determinante}{Determinante}
   \item[5.4] \link{content_17_cramersche_regel}{Cramersche Regel}
   \end{enumerate}
} %lang

\end{block}

\begin{zusammenfassung}

\lang{de}{Zum Abschluss des Vertiefungsteils Analysis beschäftigen wir uns etwas überraschend mit quadratischen Matrizen.
Dieses Kapitel dient der Vorbereitung auf die Höhere Mathematik 2, 
in der mehrdimensionale Analysis behandelt wird, und die unter anderen mit den quadratischen Matrizen Elemente der 
lineare Algebra benötigt.

Wir interessieren uns zunächst für quadratische Matrizen, also Matrizen deren Zeilenzahl und Spaltenzahl gleich ist. Zunächst definieren wir Bezeichnungen für wichtige quadratische Matrizen.

Nach Einführung einiger Rechenregeln geht es um die Invertierbarkeit von Matrizen. Findet man stets eine Matrix zu einer gegebenen Matrix, sodass das Produkt der Einheitsmatrix entspricht? Die bereits erworbenen Kenntnisse über lineare Gleichungssysteme werden uns für die Berechnung dienlich sein. Mit der Determinante einer Matrix erhalten wir eine weitere Kenngröße für die Invertierbarkeit.

}
\end{zusammenfassung}

%\begin{zusammenfassung}
%\lang{de}{
%Im letzten Kapitel dieses Kursteils wollen wir noch einen Ausflug in die lineare Algebra unternehmen und einige der im Grundlagenteil erworbenen Kenntnisse ausbauen.
%
%Wir interessieren uns zunächst für quadratische Matrizen, also Matrizen deren Zeilenzahl und Spaltenzahl gleich ist. Zunächst definieren wir Bezeichnungen für wichtige quadratische Matrizen.
%
%Nach Einführung einiger Rechenregeln geht es um die Invertierbarkeit von Matrizen. Findet man stets eine Matrix zu einer gegebenen Matrix, sodass das Produkt der Einheitsmatrix entspricht? Die bereits erworbenen Kenntnisse über lineare Gleichungssysteme werden uns für die Berechnung dienlich sein. Mit der Determinante einer Matrix erhalten wir eine weitere Kenngröße für die Invertierbarkeit.
%}
%\end{zusammenfassung}

\begin{block}[info]\lang{de}{\strong{Lernziele}}
\lang{en}{\strong{Learning Goals}} 
\begin{itemize}[square]
\item \lang{de}{Sie ordnen gegebenen Matrizen die korrekten Bezeichner zu.}
\item \lang{de}{Sie wenden Basisoperationen zur Verknüpfung von Matrizen an.}
\item \lang{de}{Sie untersuchen quadratische Matrizen auf Invertierbarkeit und berechnen schließlich die Inverse.}
\item \lang{de}{Sie wenden Rechenregeln für inverse Matrizen an.}
\item \lang{de}{Sie berechnen die Determinante einer Matrix und treffen Aussagen über die Invertierbarkeit.}
\item \lang{de}{Sie lösen lineare Gleichungssysteme mit einem alternativen Verfahren; der Cramerschen Regel.}
\end{itemize}
\end{block}




\end{content}
