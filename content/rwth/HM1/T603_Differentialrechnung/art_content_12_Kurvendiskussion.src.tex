%$Id:  $
\documentclass{mumie.article}
%$Id$
\begin{metainfo}
  \name{
    \lang{de}{Kurvendiskussion}
    \lang{en}{}
  }
  \begin{description} 
 This work is licensed under the Creative Commons License Attribution 4.0 International (CC-BY 4.0)   
 https://creativecommons.org/licenses/by/4.0/legalcode 

    \lang{de}{Beschreibung}
    \lang{en}{}
  \end{description}
  \begin{components}
    \component{generic_image}{content/rwth/HM1/images/g_tkz_T603_Example_D.meta.xml}{T603_Example_D}
    \component{generic_image}{content/rwth/HM1/images/g_tkz_T603_Example_C.meta.xml}{T603_Example_C}
    \component{generic_image}{content/rwth/HM1/images/g_tkz_T603_Example_B.meta.xml}{T603_Example_B}
    \component{generic_image}{content/rwth/HM1/images/g_tkz_T603_Example_A.meta.xml}{T603_Example_A}
    \component{js_lib}{system/media/mathlets/GWTGenericVisualization.meta.xml}{mathlet1}
  \end{components}
  \begin{links}
    \link{generic_article}{content/rwth/HM1/T104_weitere_elementare_Funktionen/g_art_content_14_potenzregeln.meta.xml}{power-rules}
  \end{links}
  \creategeneric
\end{metainfo}
\begin{content}
\usepackage{mumie.ombplus}
\ombchapter{3}
\ombarticle{4}
\usepackage{mumie.genericvisualization}

\begin{visualizationwrapper}

\lang{de}{\title{Kurvendiskussion}}
\lang{en}{\title{Analyzing functions}}

\begin{block}[annotation]
Im Ticket-System: \href{https://team.mumie.net/issues/22700}{Ticket 22700}\end{block}

\begin{block}[info-box]
\tableofcontents
\end{block}


\section{
\lang{de}{Extrempunkte und Wendepunkte}
\lang{en}{Extrema and inflection points}
}
\lang{de}{
Im letzten Kapitel haben wir bereits gesehen, dass das Monotonieverhalten einer Funktion, also ob sie steigt oder fällt, 
mit Hilfe der 1. Ableitung  beschrieben werden kann. Steigen die Funktionswerte, so gilt $f'(x) \geq 0$. Bei fallenden 
Funktionswerten gilt hingegen $f'(x) \leq 0$. Von besonderer Bedeutung sind die Punkte, an denen sich dieses Verhalten 
ändert. Typischerweise sind dies die \emph{Extrempunkte} des Graphen.
}
\lang{en}{
In the last chapter, we saw that the monotonicity behavior of a function, i.e.
whether it increases or decreases, is governed by its first derivative.
If the function is increasing, then $f'(x) \geq 0$; for decreasing functions,
on the other hand, $f'(x) \leq 0$. The points where this behavior changes are especially important.
These are typically the \emph{extrema} of the graph.
}

  \lang{de}{	\begin{definition}[Extrempunkt] 
  Es sei $f$ eine auf einem Intervall $I$ definierte stetige Funktion.
  
  Die Stelle $x_0 \in I$ nennen wir ...
		\begin{itemize}
		  \item[] ... \notion{globale Maximalstelle} von $f$, falls $f(x) \leq f(x_0)$ für alle $x \in I$ gilt. 
		  \item[] ... \notion{lokale Maximalstelle} von $f$, falls $f(x) \leq f(x_0)$ zumindest für alle $x$ in einem kleinen offenen Intervall um $x_0$ gilt.
		  \item[] ... \notion {globale Minimalstelle} von $f$, falls $f(x) \geq f(x_0)$ für alle $x \in I$ gilt.
		  \item[] ...  \notion{lokale Minimalstelle} von $f$, falls $f(x) \geq f(x_0)$ zumindest für alle $x$ in einem kleinen offenen Intervall um $x_0$ gilt.
          \end{itemize}

		  Eine lokale/globale Maximal- oder Minimalstelle bezeichnet man
		        auch als lokale/globale \notion{Extremalstelle}. 
         
          Ist $x_0$ eine Extremalstelle, so nennen wir den Punkt $(x_0; f(x_0))$ \notion{Extrempunkt}. Ein Extrempunkt an einer Minimalstelle heißt \notion{Tiefpunkt}, ein Extrempunkt an einer Maximalstelle heißt \notion{Hochpunkt}.
	 \end{definition}}

  \lang{en}{	\begin{definition}[Extrema] 
  Let $f$ be a continuous function defined on an interval $I$.
  
  A point $x_0 \in I$ is called a...
  
		\begin{itemize}
		  \item[] ... \notion{global maximum point} of $f$ if $f(x) \leq f(x_0)$ for all $x \in I$. 
		  \item[] ... \notion{local maximum point} of $f$ if $f(x) \leq f(x_0)$ for at least all $x$ in a small open interval containing $x_0$.
		  \item[] ... \notion {global mimimum point} of $f$ if $f(x) \geq f(x_0)$ for all $x \in I$.
		  \item[] ...  \notion{local minimum point} of $f$ if $f(x) \geq f(x_0)$ for at least all $x$ in a small open interval containing $x_0$.
          \end{itemize}

      A local (global) maximum or minimum point is also called a local (global) \notion{extremum} or \notion{extreme point}.

      Both $x_0$ and the point $(x_0, f(x_0))$ are referred to as \notion{extrema}.
	 \end{definition}}
\lang{de}{Diese Definition gilt auch für Funktionen, die gar keine Ableitung besitzen. Wir werden uns aber meist mit differenzierbaren Funktionen 
befassen. Für solche Funktionen haben wir das folgende Kriterium, um Extremalstellen aufzuspüren.}
\lang{en}{
This definition remains valid for functions that do not have a derivative at all.
However, we will usually be discussing differentiable functions. For differentiable functions,
we have the following criterion to detect extrema.
}
\begin{theorem}[
\lang{de}{Hinreichendes Kriterium für Extremalstellen I}
\lang{en}{Sufficient criterion for extrema I}]
	\lang{de}{
    Es sei $f$ differenzierbar. Ist $f'(x_0)=0$ und wechselt $f'$ in $x_0$ das Vorzeichen, so ist 
	 $x_0$ eine Extremalstelle. 
     
     Ist dies der Fall, dann gilt: 
     \begin{itemize}
     \item $x_0$ ist eine Maximalstelle, wenn sich das Vorzeichen von $+$ nach $-$ ändert. 
     \item $x_0$ ist eine Minimalstelle, wenn sich das Vorzeichen von $-$ nach $+$ ändert.
     \end{itemize}
     }
  \lang{en}{
    Let $f$ be differentiable. If $f'(x_0)=0$ and $f'$ changes sign in $x_0$,
    then $x_0$ is an extremum.

    In this case,
    \begin{itemize}
     \item $x_0$ is a maximum point if the sign changes from $+$ to $-$. 
     \item $x_0$ is a minimum point if the sign changes from $-$ to $+$. 
     \end{itemize}
  }
\end{theorem}
\lang{de}{
Meist benutzt man jedoch das folgende Kriterium, weil es leichter nachzuprüfen ist.
}
\lang{en}{
The following criterion is used more often because its conditions are easier to verify.
}
\begin{theorem}[
\lang{de}{Hinreichendes Kriterium für Extremalstellen II}
\lang{en}{Sufficient criterion for extrema II}]
	\lang{de}{
    Es sei $f$ zweimal differenzierbar. 
    \begin{itemize}
    \item  Ist $f'(x_0)=0$ und $f''(x_0) < 0$, so ist  $x_0$ eine Maximalstelle. 
    \item  Ist $f'(x_0)=0$ und $f''(x_0) > 0$, so ist  $x_0$ eine Minimalstelle. 
     \end{itemize}
}
\lang{en}{
    Suppose $f$ is twice differentiable.
    \begin{itemize}
    \item  If $f'(x_0)=0$ and $f''(x_0) < 0$ then  $x_0$ is a maximum point. 
    \item  If $f'(x_0)=0$ and $f''(x_0) > 0$ then  $x_0$ is a minimum point. 
     \end{itemize}
}
\end{theorem}
 \begin{block}[warning]
\lang{de}{Eine Stelle $x_0$ mit $f'(x_0) = 0$ wird auch \notion{stationäre} Stelle genannt. Extremalstellen können nur an stationären Stellen liegen. 
Also: Gilt $f'(x_0) \neq 0$, liegt keine Extremalstelle vor. }
\lang{en}{
A point $x_0$ for which $f'(x_0) = 0$ is also called a \notion{stationary} point.
Extrema can only occur at stationary points. That is: if $f'(x_0) \neq 0$, then $x_0$
is not an extremum.
}

\lang{de}{
Gilt jedoch $f'(x) = 0 $ und $f''(x) = 0$, dann kann trotzdem eine Extremalstelle vorliegen (muss aber nicht).  
}
\lang{en}{
On the other hand, extrema can (but do not have to) occur even if $f'(x) = 0 $ and $f''(x) = 0$.
}
 \end{block}

\begin{example}

\lang{de}{Wir wollen die gewinnmaximierende Menge $x$ bei gegebener Preis-Absatz-Funktion $p(x)$ und gegebener Kostenfunktion $K(x)$ berechnen.
Dazu sei 
\begin{align*}
p(x) &\ =\ & -2x + 400, \\
K(x) &\ =\ & 4x + 10000.
\end{align*}
}
\lang{en}{
Given a fixed price-demand function $p(x)$ and a cost function $K(x)$, we want to
find the quantity $x$ that maximizes profit. Suppose
\begin{align*}
p(x) &\ =\ & -2x + 400, \\
K(x) &\ =\ & 4x + 10000.
\end{align*}
}

\lang{de}{Hierzu stellen wir zuerst die Gewinnfunktion $G(x)$ auf: \\
Gewinn wird als Erlös minus Kosten definiert, wobei der Erlös $E(x)$ mit der Preis-Absatzfunktion mal der Menge berechnet wird, also konkret \\
\begin{align*}
E(x) &\ =\ & p(x) \cdot x \\
  &\ =\ & -2x^2 + 400x,\\

G(x) &\ =\ & E(x) - K(x)\\
  &\ =\ & -2x^2 + 396x - 10000.\\
\end{align*}
}
\lang{en}{
First we determine the profit function $G(x)$. \\
Profit is defined as revenue minus costs, where the revenue $E(x)$ is the product
of price and quantity, i.e.
\begin{align*}
E(x) &\ =\ & p(x) \cdot x \\
  &\ =\ & -2x^2 + 400x,\\

G(x) &\ =\ & E(x) - K(x)\\
  &\ =\ & -2x^2 + 396x - 10000.\\
\end{align*}
}

\lang{de}{Zur Berechnung der gewinnmaximierenden Menge berechnen wir nun die Extremstellen der Gewinnfunktion mit Hilfe des hinreichenden Kriteriums. Die 
ersten beiden Ableitungen sind\\
\begin{align*}
G'(x) &\ =\ & -4x + 396,\\
G''(x) &\ =\ & -4.
\end{align*}
}
\lang{en}{
To compute the quantity that maximizes profit, we will find the extrema of the
profit function using the above sufficient criterion. The first two derivatives are
\begin{align*}
G'(x) &\ =\ & -4x + 396,\\
G''(x) &\ =\ & -4.
\end{align*}
}

\lang{de}{Die erste Ableitung der Gewinnfunktion muss jetzt gleich Null gesetzt werden: \\
\begin{align*}
& G'(x) &\ =\ & 0 \\
 &\Leftrightarrow \ 0 &\ =\ & -4x + 396\\
 &\Leftrightarrow \ 4x &\ =\ & 396\\
 &\Leftrightarrow \ x &\ =\ & 99.
\end{align*}
}

\lang{en}{The first derivative of the profit function should be set equal to zero:
\begin{align*}
& G'(x) &\ =\ & 0 \\
 &\Leftrightarrow \ 0 &\ =\ & -4x + 396\\
 &\Leftrightarrow \ 4x &\ =\ & 396\\
 &\Leftrightarrow \ x &\ =\ & 99.
\end{align*}
}

\lang{de}{Der Kandidat für eine Extremalstelle ist also $x = 99$. Dieser Wert muss nun in die zweite Ableitung der Gewinnfunktion 
eingesetzt werden: 

\begin{align*}
G''(99) &\ =\ & -4 <0. \\
\end{align*}
}

\lang{en}{
So the candidate for an extremum is $x=99$. Now we need to evaluate the second
derivative of the profit function at this point:

\begin{align*}
G''(99) &\ =\ & -4 <0. \\
\end{align*}
}

\lang{de}{Die zweite Ableitung an der Stelle $x=99$ ist negativ, d.\,h. es liegt ein lokales Maximum vor. \\

Die maximale Gewinn liegt somit bei $G(99) = 9602$. 

Das Unternehmen erzielt den maximalen Gewinn von $9602$ € bei $99$ verkauften Gütern. 
}

\lang{en}{
The second derivative at $x=99$ is negative, so this is a local maximum.

Therefore, the maximal profit is $G(99) = 9602$.

The company earns the maximal profit of $9602$ € if it sells $99$ units.
}

\end{example}


 \begin{quickcheckcontainer}
\randomquickcheckpool{1}{2}
\begin{quickcheck}
		\field{rational}
		\type{input.number}
		\begin{variables}
			\number{k}{1}
%			\randint{k}{1}{2}  % Vorzeichen des höchsten Koeff (-1)^k
			\randint{l}{1}{2}  % Vorzeichen der Extremstelle: (-1)^(l+1)
			\function[calculate]{l1}{2-l}  % Dirac-Funktionen
			\function[calculate]{l2}{l-1}
			
			\randint{a}{1}{3}
			\randint{b}{1}{4}
			\randint{c}{-4}{4}
		    \function[normalize]{f}{(-1)^k*(a*x^4+(-1)^l*b*x^3+c)}
		    \number{n1}{0}
			\function[calculate]{n2}{(-1)^(l+1)*3*b/(4*a)}
			
			\function[calculate]{ns1}{l1*n1+l2*n2}
			\function[calculate]{ns2}{l1*n2+l2*n1}
			
			\function[expand,normalize]{df}{(-1)^k*(4*a*x^3+3*(-1)^l*b*x^2)}
			\function[expand,normalize]{lin}{4*a*x+3*(-1)^l*b}
			\function[normalize]{x2}{(-1)^k*x^2}
% 			\function[calculate]{vz}{(-1)^(l+1)}
% 			\function[calculate]{zae}{3*b}
% 			\function[calculate]{nen}{4*a}
		\end{variables}

      \lang{de}{
			\text{Bestimmen Sie die möglichen Extremalstellen der Funktion $f(x)=\var{f}$, indem Sie
			die Nullstellen ihrer Ableitungsfunktion bestimmen.
            
			Die möglichen Extremalstellen sind (in aufsteigender Reihenfolge): \ansref und \ansref.\\
			Von diesen beiden Stellen können wir mit unseren Kriterien nur die Stelle \ansref 
            als Extremalstelle identifizieren.\\
			Hat $f$ dort (1) ein Maximum oder (2) ein Minimum? \ansref}}

      \lang{en}{
      \text{Determine the potential extrema of the function $f(x)=\var{f}$
      by finding the zeros of its derivative.
      
      The potential extrema (in increasing order) are: \ansref und \ansref.\\
      Using our criteria, we are only able to show that the point \ansref is an extremum.\\
      Does $f$ have (1) a maximum or (2) a minimum there? \ansref
      }
      }
		
		\begin{answer}
			\solution{ns1}
		\end{answer}
		\begin{answer}
			\solution{ns2}
		\end{answer}
		\begin{answer}
			\solution{n2}
		\end{answer}
		\begin{answer}
			\solution{k}
		\end{answer}
    \lang{de}{
		\explanation{Die Ableitung von $f$ ist $f'(x)=\var{df}=\var{x2}\cdot (\var{lin})$.\\
		Die Funktion hat also die möglichen Extremalstellen $x=0$ und $x=\var{n2}$.\\
		Bei $x=\var{n2}$ wechselt die Ableitung ihr Vorzeichen von positiv nach negativ, die
		Funktion $f$ hat dort also ein Maximum. Um $x=0$ bleibt das Vorzeichen der Ableitung
		das gleiche. Hier können wir das Kriterium also nicht anwenden. (Tatsächlich liegt hier  
        keine Extremalstelle vor, wie man etwa durch Grenzwertbetrachtungen nachweisen kann.)
		}}
    \lang{en}{
    \explanation{The derivative of $f$ is $f'(x)=\var{df}=\var{x2}\cdot (\var{lin})$.\\
    This function has $x=0$ and $x=\var{n2}$ as possible extrema.
    At $x=\var{n2}$, the derivative changes its sign from positive to negative,
    and therefore $f$ has a maximum there. The sign of the derivative stays the same
    near $x=0$, so we cannot apply the criterion there.
    (In fact, $x=0$ is not an extremum, as one can show by considering limits.)}
    }
	\end{quickcheck}
	
	\begin{quickcheck}
		\field{rational}
		\type{input.number}
		\begin{variables}
			\number{k}{2}
%			\randint{k}{1}{2}  % Vorzeichen des höchsten Koeff (-1)^k
			\randint{l}{1}{2}  % Vorzeichen der Extremstelle: (-1)^(l+1)
			\function[calculate]{l1}{2-l}  % Dirac-Funktionen
			\function[calculate]{l2}{l-1}
			
			\randint{a}{1}{3}
			\randint{b}{1}{4}
			\randint{c}{-4}{4}
		    \function[normalize]{f}{(-1)^k*(a*x^4+(-1)^l*b*x^3+c)}
		    \number{n1}{0}
			\function[calculate]{n2}{(-1)^(l+1)*3*b/(4*a)}
			
			\function[calculate]{ns1}{l1*n1+l2*n2}
			\function[calculate]{ns2}{l1*n2+l2*n1}
			
			\function[expand,normalize]{df}{(-1)^k*(4*a*x^3+3*(-1)^l*b*x^2)}
			\function[expand,normalize]{lin}{4*a*x+3*(-1)^l*b}
			\function[normalize]{x2}{(-1)^k*x^2}
% 			\function[calculate]{vz}{(-1)^(l+1)}
% 			\function[calculate]{zae}{3*b}
% 			\function[calculate]{nen}{4*a}
		\end{variables}
		
			\lang{de}{
			\text{Bestimmen Sie die möglichen Extremalstellen der Funktion $f(x)=\var{f}$, indem Sie
			die Nullstellen ihrer Ableitungsfunktion bestimmen.
            
			Die möglichen Extremalstellen sind (in aufsteigender Reihenfolge): \ansref und \ansref.\\
			Von diesen beiden Stellen können wir mit unseren Kriterien nur die Stelle \ansref 
            als Extremalstelle identifizieren.\\
			Hat $f$ dort (1) ein Maximum oder (2) ein Minimum? \ansref}}

      \lang{en}{
      \text{Determine the potential extrema of the function $f(x)=\var{f}$
      by finding the zeros of its derivative.
      
      The potential extrema (in increasing order) are: \ansref und \ansref.\\
      Using our criteria, we are only able to show that the point \ansref is an extremum.\\
      Does $f$ have (1) a maximum or (2) a minimum there? \ansref
      }
      }
		
		\begin{answer}
			\solution{ns1}
		\end{answer}
		\begin{answer}
			\solution{ns2}
		\end{answer}
		\begin{answer}
			\solution{n2}
		\end{answer}
		\begin{answer}
			\solution{k}
		\end{answer}
    \lang{de}{
		\explanation{Die Ableitung von $f$ ist $f'(x)=\var{df}=\var{x2}\cdot (\var{lin})$.\\
		Die Funktion hat also die möglichen Extremalstellen $x=0$ und $x=\var{n2}$.\\
		Bei $x=\var{n2}$ wechselt die Ableitung ihr Vorzeichen von negativ nach positiv, die
		Funktion $f$ hat dort also ein Minimum. Um $x=0$ bleibt das Vorzeichen der Ableitung
		das gleiche. Hier können wir das Kriterium also nicht anwenden.	(Tatsächlich liegt hier  
        keine Extremalstelle vor, wie man etwa durch Grenzwertbetrachtungen nachweisen kann.)			
		}}
  \lang{en}{
    \explanation{The derivative of $f$ is $f'(x)=\var{df}=\var{x2}\cdot (\var{lin})$.\\
    This function has $x=0$ and $x=\var{n2}$ as possible extrema.
    At $x=\var{n2}$, the derivative changes its sign from negative to positive,
    and therefore $f$ has a minimum there. The sign of the derivative stays the same
    near $x=0$, so we cannot apply the criterion there.
    (In fact, $x=0$ is not an extremum, as one can show by considering limits.)}
    }
	\end{quickcheck}		
\end{quickcheckcontainer}     

\lang{de}{
Im letzten Kapitel haben wir auch gesehen, dass die Krümmung des Funktionsgraphen mit Hilfe der 2. Ableitung 
ermittelt werden kann. Für $f''(x) >0$ ist der Graph linksgekrümmt bzw. konvex, für $f''(x) <0$ entsprechend rechtsgekrümmt 
bzw. konkav. Die Punkte, an denen sich die Krümmung ändert, sind für uns besonders interessant:
}
\lang{en}{
We also saw in the last chapter that the convexity of the graph of a function
can be described using the second derivative. The graph is concave up (or convex)
where $f''(x) > 0$ and concave down (or simply concave) where $f''(x) < 0$.
The points at which this behavior changes are especially important:
}

\begin{definition}[
\lang{de}{Wendepunkt}
\lang{en}{Inflection point}]

\lang{de}{Es sei $f$ eine zweimal differenzierbare Funktion. }
\lang{en}{Let $f$ be a twice differentiable function.}

\lang{de}{Einen Punkt $P$, an dem sich die Krümmung des Graphen von $f$ ändert, nennen wir \notion{Wendepunkt} von $f$.
Die x-Koordinate eines solchen Punktes $P$ nennen wir \emph{Wendestelle} von $f$.}
\lang{en}{
A point $P$ at which the concavity of the graph of $f$ changes direction
is called an \notion{inflection point} of $f$.
}

\lang{de}{Beträgt die Steigung an einer Wendestelle $0$, so nennen wir den zugehörigen Wendepunkt auch \notion{Sattelpunkt}.
}
\lang{en}{
An inflection point at which the slope is $0$ is also called a \notion{saddle point}.
}
\end{definition}

\begin{example}
	\lang{de}{Der Graph der Funktion $f(x)=x^4-2x^3+2x+1$ beschreibt eine Linkskurve f\"{u}r $x<0$, eine Rechtskurve f\"{u}r $0<x<1$ und wieder eine 
	Linkskurve f\"{u}r $x>1$. Die Wendestellen liegen damit bei $x=0$ und $x=1$.}
   \lang{en}{
    The graph of the function $f(x)=x^4-2x^3+2x+1$ is concave up for $x < 0$, down for $0<x<1$, and up again for $x>1$.
    The inflection points are therefore at $x=0$ and $x=1$.
   }
	
	\begin{center}
    \image{T603_Example_A}
    \end{center}
    
	\lang{de}{Die Ableitung berechnet sich zu }
  \lang{en}{The derivative is}
	$f'(x)=4x^3-6x^2+2$.  
	
	\begin{center}
    \image{T603_Example_B}
    \end{center}
	
	\lang{de}{Man sieht an den Nullstellen von $f'$, wo die Steigung von $f$ den Wert $0$ besitzt.  Dies sind die Orte, wo ein 
    Extrempunkt von $f$ liegen könnte. Bei der ersten Nullstelle besitzt $f$ ein lokales Minimum, bei der zweiten Nullstelle $x=1$ von $f'$ ist dies nicht der Fall. Hier 
    ändert die Funktion $f'$ auch nicht ihr Vorzeichen.}
  \lang{en}{
  The zeros of $f'$ are the points where the slope of $f$ is zero. These are the potential
  extrema of $f$. The first zero corresponds to a local minimum. This is not the case for the second zero in $x=1$,
  where the function $f'$ does not change its sign.
  }
	\begin{center}
    \image{T603_Example_C}
    \end{center}
	
	\lang{de}{Die zweite Ableitung hat die Gleichung $f''(x)=12x^2-12x=
	12x(x-1)$. Die Funktion $f''(x)$ ist positiv f\"{u}r $x<0$ und $x>1$ ($f$ linksgekrümmt) und negativ f\"{u}r $0<x<1$ ($f$ rechtsgekrümmt). An den Stellen $x=0$ und 
    $x=1$ ändert sich also die Krümmung, weshalb wir hier Wendestellen gefunden haben. Weil auch $f'(1)=0$ gilt, besitzt $f$ an der Stelle $x=1$ 
    sogar einen Sattelpunkt.}
  \lang{en}{
  The second derivative is $f''(x)=12x^2-12x=	12x(x-1)$.
  The function $f''(x)$ is positive for $x<0$ and $x>1$ (where $f$ is concave up) and negative for $0<x<1$ (where $f$ is concave down).
  The direction changes at $x=0$ and $x=1$, which is why we found inflection points there.
  Since $f'(1)=0$, the point at $x=1$ is actually a saddle point.
  }
    \end{example}
\lang{de}{Diese Beobachtung nutzen wir, um ein Kriterium für das Vorliegen von Wendestellen zu erhalten.
}
\lang{en}{
We will use this observation to obtain a criterion for the existence of inflection points.
}

\begin{theorem}[
\lang{de}{Hinreichendes Kriterium für Wendestellen I}
\lang{en}{Sufficient criterion for inflection points I}]
	\lang{de}{
    Es sei $f$ zweimal differenzierbar. Ist $f''(x_0)=0$ und wechselt $f''$ in $x_0$ das Vorzeichen, so ist 
	 $x_0$ eine Wendestelle. 
     
     Ist dies der Fall, dann gilt:
     \begin{itemize}
     \item An $x_0$ wechselt die Krümmung von rechts nach links (bzw. von konkav zu konvex), wenn sich das Vorzeichen von $f''$ von $-$ nach $+$ ändert.
     
     \item An $x_0$ wechselt die Krümmung von links nach rechts (bzw. von konvex zu konkav), wenn sich das Vorzeichen von $f''$ von $+$ nach $-$ ändert.
     \end{itemize}
     }
  \lang{en}{
    Let $f$ be twice differentiable. Suppose $f''(x_0)=0$ and that $f''$ changes sign in $x_0$.
    Then $f$ has an inflection point at $x_0$, and:
    \begin{itemize}
      \item If the sign of $f$ changes from $-$ to $+$, then the direction changes
      from concave down to concave up in $x_0$.
     
     \item If the sign of $f$ changes from $+$ to $-$, then the direction changes
      from concave up to concave down in $x_0$.\end{itemize}
  }
\end{theorem}
\lang{de}{
In der Praxis benutzt man häufig das folgende Kriterium, weil es leichter nachzuprüfen ist.
}
\lang{en}{
In practice, the following criterion is used often because its conditions
are easier to verify.
}
\begin{theorem}[
\lang{de}{Hinreichendes Kriterium für Wendestellen II}
\lang{en}{Sufficient criterion for inflection points II}]
	\lang{de}{
    Es sei $f$ dreimal differenzierbar. 
    \begin{itemize}
    \item Ist $f''(x_0)=0$ und $f'''(x_0) > 0$, so ist 
	 $x_0$ eine Wendestelle mit Krümmungswechsel von rechts nach links (bzw. von konkav zu konvex).
     \item Ist $f''(x_0)=0$ und $f'''(x_0) < 0$, so ist 
	 $x_0$ eine Wendestelle mit Krümmungswechsel von links nach rechts (bzw. von konvex zu konkav).
    \end{itemize}
     }
  \lang{en}{
    Let $f$ be three times differentiable.
     \begin{itemize}
    \item If $f''(x_0)=0$ and $f'''(x_0) > 0$, then
	 $x_0$ is an inflection point in which the graph of $f$ changes from concave down to concave up.
  \item If $f''(x_0)=0$ and $f'''(x_0) < 0$, then
	 $x_0$ is an inflection point in which the graph of $f$ changes from concave up to concave down.
  \item Is\end{itemize}
  }
\end{theorem}
\lang{de}{
Wir sehen, dass die Kriterien für Wendestellen und Extremalstellen sehr ähnlich lauten. Wir können Wendestellen auch
als die Extremalstellen der ersten Ableitung auffassen und nur mit dem Kriterium für Extremalstellen arbeiten. 
}
\lang{en}{
Observe that the criteria for inflection points and extrema are very similar.
It is possible to view inflection points as the extrema of the first derivative and to
work only with the criteria for extrema.
}
 \begin{block}[warning]

\lang{de}{Gilt $f''(x_0) \neq 0$, liegt keine Wendestelle vor.

Gilt jedoch $f''(x) = 0$ und $f'''(x) = 0$, so kann eine Wendestelle vorliegen (muss aber nicht).  
}
\lang{en}{
If $f''(x_0) \neq 0$, then $x_0$ is not an inflection point.

On the other hand, if $f''(x)$ and $f'''(x) = 0$, then $x$ can still
be an inflection point (although it might not be one).
}
 
 \end{block}

\begin{quickcheck}
		\field{rational}
		\type{input.number}
		\begin{variables}
			\number{c}{12}   % schönere Zahlen mit c=12 (auf jeden Fall muss c>0 sein.)
			\randint{ns1}{-2}{0}
			\randint{ns2}{1}{2}
			\randint{a}{-3}{3}
			\function[normalize]{dfak}{c*(x-ns1)*(x-ns2)}
			\function[expand,normalize]{d}{c*(x-ns1)*(x-ns2)}
			\function[normalize]{dd}{(c/3)*x^3-(c/2)*(ns1+ns2)*x^2+c*ns1*ns2*x+a}
			\function[normalize]{d3}{(c/12)*x^4-(c/6)*(ns1+ns2)*x^3+(c/2)*ns1*ns2*x^2+a*x}
		
		\end{variables}

      \lang{de}{
			\text{Bestimmen Sie die Wendestellen der Funktion $f(x)=\var{d3}$.
            
			Die Wendestellen (in aufsteigender Reihenfolge) sind \ansref und \ansref.}
      }
      \lang{en}{
      \text{Determine the inflection points of the function $f(x)=\var{d3}$.
      
      The inflection points (in increasing order) are \ansref and \ansref.}
      }
		
		\begin{answer}
			\solution{ns1}
		\end{answer}
		\begin{answer}
			\solution{ns2}
		\end{answer}
    \lang{de}{
		\explanation{Kandidaten für Wendestellen sind die Stellen $x_0$, an denen die zweite Ableitung gleich $0$ ist. Wegen
		$f'(x)=\var{dd}$ und $f''(x)=\var{d}=\var{dfak}$ sind die Kandidaten die Stellen $\var{ns1}$ und $\var{ns2}$. Nun muss noch $f'''(\var{ns1})\neq 0$ und $f'''(\var{ns2})\neq 0$ geprüft werden.}
  	}
     \lang{en}{
    \explanation{The candidate inflection points are the points $x_0$ at which the second derivative is $0$.
    Since $f'(x)=\var{dd}$ and $f''(x)=\var{d}=\var{dfak}$, the candidate points are $\var{ns1}$ and $\var{ns2}$. Now we need to check whether $f'''(\var{ns1})\neq 0$ and $f'''(\var{ns2})\neq 0$.
    }
     }
 \end{quickcheck}


\section{
\lang{de}{Kurvendiskussion}
\lang{en}{Analyzing functions}
}
\lang{de}{Bei einer Kurvendiskussion wird eine Funktion $f$ auf ihre wesentlichen Merkmale \"{u}berpr\"{u}ft. Was zu einer Kurvendiskussion \glqq gehört\grqq, variiert
von Autor zu Autor. Die folgenden Aspekte sind jedoch am weitesten verbreitet.}
\lang{en}{
To analyze a function $f$ means to investigate the significant features of its graph.
What exactly that entails varies from author to author, but the aspects below are the most common.
}

\begin{block}[info]
\lang{de}{
\textbf{Schritte bei der Kurvendiskussion:}
}
\lang{en}{
\textbf{Steps when analyzing a function:}
}
\begin{enumerate}
\item \lang{de}{\textbf{Festlegung des Definitionsbereichs}\\
        Gemeint ist hier der maximal mögliche Definitionsbereich, der für das gegebene 
        Problem sinnvoll ist. (Negative Produktionsmengen sind beispielsweise nicht sinnvoll.)
      }
      \lang{en}{\textbf{Determining the domain}\\
      Here we mean the largest possible domain that makes sense in the context
      of the given problem. (For example, negative quantities of production are not reasonable.)
      }

\item \lang{de}{\textbf{Untersuchung des asymptotischen Verhaltens}\\
        Wenn der Definitionsbereich unbegrenzt ist, meinen wir mit asymptotischem Verhalten die Grenzwerte
        $\lim_{x \to \pm \infty} f(x)$. Ist der Definitionsbereich in eine oder beide Richtungen begrenzt, 
        berechnen wir den Funktionswert am Rand.
        }
      \lang{en}{\textbf{Describing the asymptotic behavior}\\
      If the domain is unbounded then the asymptotic behavior means the limits
      $\lim_{x \to \pm \infty} f(x)$. If the domain is bounded in one or both directions
      then we determine the value of the function at the boundary.
      }

\item \lang{de}{\textbf{Bestimmung der Nullstellen}\\
        Zwischen je zwei Nullstellen kann auch noch interessant sein, ob der Funktionsgraph oberhalb oder 
        unterhalb der x-Achse verläuft.}
      \lang{en}{\textbf{Finding the zeros}\\
      It is also often useful to know whether the graph of the function
      lies above or below the x-axis between any two zeros.}
        

\item \lang{de}{\textbf{Bestimmung der Extremalstellen}\\
        Mit Hilfe der Extremalstellen kann auch angegeben werden, auf welchen Teilintervallen des Definitionsbereichs
        die Funktion steigt oder fällt. }
      \lang{en}{\textbf{Finding the extrema}\\
      Using the extrema, one can also determine the subintervals of the domain on which
      the function is increasing or decreasing.
      }
        

\item \lang{de}{\textbf{Bestimmung der Wendestellen}\\
        Zusätzlich kann auch angegeben werden, wie sich das Krümmungsverhalten an den Wendestellen ändert.
        }
        \lang{en}{\textbf{Finding the inflection points}\\
        In addition, one can determine whether the graph of the function is
        concave up or down between any two inflection points.}
\end{enumerate}
\end{block}

\begin{example}
	\lang{de}{Es sei $f(x)=x^4+4x^3+4x^2$ die zu untersuchende Funktion. 
    
    Der Definitionsbereich ist $D_f = \R$, da die Funktion für alle reellen Zahlen einen Funktionswert liefert. 
    
    Das asymptotische Verhalten berechnen wir über 
    \begin{align*}
     & \lim_{x \to \infty} x^4+4x^3+4x^2 = \lim_{x \to \infty} x^4 (1+\frac{4}{x}+\frac{4}{x^2}) = \infty, \\
     & \lim_{x \to -\infty} x^4+4x^3+4x^2 = \lim_{x \to -\infty} x^4 (1+\frac{4}{x}+\frac{4}{x^2}) = \infty.
    \end{align*}
    }
  \lang{en}{
  Consider the function $f(x)=x^4+4x^3+4x^2$.

  Its domain is $D_f = \R$ because $f$ is defined for all real numbers.

  The asymptotic behavior is described by
  \begin{align*}
     & \lim_{x \to \infty} x^4+4x^3+4x^2 = \lim_{x \to \infty} x^4 (1+\frac{4}{x}+\frac{4}{x^2}) = \infty, \\
     & \lim_{x \to -\infty} x^4+4x^3+4x^2 = \lim_{x \to -\infty} x^4 (1+\frac{4}{x}+\frac{4}{x^2}) = \infty.
    \end{align*}
  }
    
    \lang{de}{
    Die Nullstellen von $f$ berechnen wir, indem wir $x^2$ ausklammern. Dann erhalten wir die Gleichung 
    \[
    x^2 \cdot (x^2+4x+4) = 0,
    \]
    und sehen damit, dass $x_0=0$ eine (doppelte) Nullstelle ist. Die anderen Nullstellen berechnen wir mit der pq-Formel
    und erhalten
    \[
    x_{1} = - \frac{4}{2} \pm \sqrt{\left(\frac{4}{2}\right)^2 - 4} = -2 \pm 0 = -2,
    \]
    also liegt an $x_1 = -2$ wieder eine doppelte Nullstelle. An den Nullstellen ändert sich das Vorzeichen der Funktionswerte 
    nicht. (Für große oder besonders negative $x$ wissen wir aus dem asymptotischen Verhalten, dass die Funktionswerte positiv sind. Wenn wir 
    z.\,B. $x=-1$ in die Funktion einsetzen, erhalten wir wieder einen positiven Funktionswert zwischen den Nullstellen, also 
    kann die Funktion wegen Stetigkeit keine negativen Werte annehmen.)
    }
    \lang{en}{
    We will find the zeros of $f$ by factoring out $x^2$. This leaves us with the equation
    \[
    x^2 \cdot (x^2+4x+4) = 0,
    \]
    from which one can see that $x_0=0$ is a (double) zero. The other zeros can be found with the quadratic formula:
    \[
    x_{1} = - \frac{4}{2} \pm \sqrt{\left(\frac{4}{2}\right)^2 - 4} = -2 \pm 0 = -2,
    \]
    so $x_1 = -2$ is another double zero. The sign of $f$ does not change near either zero.
    (For large or highly negative values of $x$, the asymptotic behavior shows that the function is positive.
    Plugging in a point that lies between the zeros, e.g. $x=-1$, also yields a positive value of $f$.
    Since $f$ is continuous, this shows that it is never negative.)
    }

    \lang{de}{
    Wir berechnen die Ableitung, um anschließend Extremal- und Wendestellen zu suchen: 
    \begin{align*}
     f'(x) &= 4x^3+12x^2+8x=4x(x^2+3x+2)=4x(x+1)(x+2), \\
     f''(x) &= 12x^2+24x+8, \\
     f'''(x) &= 24x.
    \end{align*}
    Die Funktion $f'$ hat die Nullstellen $x_2=0$, $x_3=-1$ und $x_4=-2$. An diesen können Extrempunkte liegen. Es gilt 
    $f''(0) = 8 > 0$, weshalb an $x_2 = 0$ ein lokales Minimum vorliegt. Wegen $f''(-1) = -4 < 0$ und $f''(-2) = 8 >0$ liegt an 
    $x_3 = -1$ ein lokales Maximum und an $x_4 = -2$ ein lokales Minimum. 
    
    Da es sich bei den Minimalstellen auch um Nullstellen von $f$ handelt und wir schon
    ermittelt haben, dass die Funktion $f$ keine negativen Werte annimmt, handelt es sich bei den Minimalstellen sogar um globale Minimalstellen, an denen 
    das globale Minimum $f(0) = f(-2) = 0$ angenommen wird. Ein globales Maximum existiert nicht, da wir schon aus dem asymptotischen Verhalten wissen, dass
    die Funktionswerte von $f$ über alle Schranken hinweg steigen werden.
    
    Die Nullstellen von $f''$ berechnen sich nach der pq-Formel zu 
	$x_5=-1-\frac{1}{\sqrt{3}}$ und $x_6=-1+\frac{1}{\sqrt{3}}$. Einsetzen in $f'''$ liefert uns $f'''(x_5) <0$ und $f'''(x_6) >0$, womit wir zwei 
    Wendestellen nachgewiesen haben. An $x_5$ wechselt die Krümmung von links nach rechts, an $x_6$ ist es genau umgekehrt. 
    
    Mit Hilfe dieser Informationen können wir den Verlauf des Funktionsgraphen jetzt auch zeichnen:
	}
   \lang{en}{
    We will compute derivatives in order to determine the extrema and inflection points later:
    \begin{align*}
     f'(x) &= 4x^3+12x^2+8x=4x(x^2+3x+2)=4x(x+1)(x+2), \\
     f''(x) &= 12x^2+24x+8, \\
     f'''(x) &= 24x.
    \end{align*}
    $f'$ has zeros $x_2=0$, $x_3=1$ and $x_4=-2$ and this is where extrema may occur.
    Since $f''(0) = 8 > 0$, there is a local minimum at $x_2 = 0$. Since $f''(-1) = -4 < 0$ and $f''(-2) = 8 > 0$,
    $x_3 = -1$ is a local maximum and $x_4 = -2$ is a local minimum.

    The minima are also zeros of $f$, and we saw above that $f$ never takes on negative values.
    Therefore, the minima are actually global minima of $f$, and the minimal value of $f$ is $f(0) = f(-2) = 0$.
    There is no global maximum, because we know from its asymptotic behavior that the
    value of $f$ grows beyond any fixed bound.

    The zeros of $f''$ can be shown with the quadratic formula to be
    $x_5=-1-\frac{1}{\sqrt{3}}$ and $x_6=-1+\frac{1}{\sqrt{3}}$. When we substitute these into $f'''$, we find
    $f'''(x_5) <0$ and $f'''(x_6) >0$, so both $x_5$ and $x_6$ are inflection points.
    At $x_5$, the graph changes from concave up to down and at $x_6$ it changes back to concave up.

    We can now use this information to sketch a graph of the function:
   }
    \begin{center}
    \image{T603_Example_D}
    \end{center}
\end{example} 


\end{visualizationwrapper}


\end{content}