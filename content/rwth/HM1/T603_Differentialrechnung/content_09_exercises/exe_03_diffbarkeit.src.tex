\documentclass{mumie.element.exercise}
%$Id$
\begin{metainfo}
  \name{
    \lang{de}{Ü03: rechts- und linksseitige Ableitung}
    \lang{en}{Exercise 3: right- and left-hand derivatives}
  }
  \begin{description} 
 This work is licensed under the Creative Commons License Attribution 4.0 International (CC-BY 4.0)   
 https://creativecommons.org/licenses/by/4.0/legalcode 

    \lang{de}{}
    \lang{en}{}
  \end{description}
  \begin{components}
  \end{components}
  \begin{links}
  \end{links}
  \creategeneric
\end{metainfo}
\begin{content}
\title{\lang{de}{Ü03: rechts- und linksseitige Ableitung}
    \lang{en}{Exercise 3: right- and left-hand derivatives}}
\begin{block}[annotation]
	Im Ticket-System: \href{https://team.mumie.net/issues/24012}{Ticket 24012}
\end{block}
\begin{block}[annotation]
Kopie: hm4mint/T301_Differenzierbarkeit/exercise 2

Im Ticket-System: \href{http://team.mumie.net/issues/10275}{Ticket 10275}
\end{block}

\usepackage{mumie.ombplus}


%######################################################FRAGE_TEXT
\lang{de}{ 
Ist die Funktion $f:\R\to \R$ mit
$f(x)=\begin{cases}x^2-x, & \text{für }x<1\\ \ln(x), & \text{für }x\geq 1 \end{cases}$\\
an der Stelle $x_0=1$ differenzierbar?\\
}

\lang{en}{ 
Is the function $f:\R\to \R$,
$f(x)=\begin{cases}x^2-x, & \text{if }x<1\\ \ln(x), & \text{if }x\geq 1 \end{cases}$\\
differentiable at the point $x_0=1$?\\
}

%##################################################ANTWORTEN_TEXT
\begin{tabs*}[\initialtab{0}\class{exercise}]

  %++++++++++++++++++++++++++++++++++++++++++START_TAB_X
  \tab{\lang{de}{    Lösung  }  \lang{en}{Solution}}
  \begin{incremental}[\initialsteps{1}]
  
  	 %----------------------------------START_STEP_X
    \step 
    \lang{de}{   Um zu sehen, ob $f$ bei $x_0$ differenzierbar ist, müssen wir untersuchen, ob der Grenzwert
\[ \lim_{x\to x_0} \frac{f(x)-f(x_0)}{x-x_0} \]
existiert.    }
    \lang{en}{To decide whether $f$ is differentiable at $x_0$, we must determine whether the limit
    \[ \lim_{x\to x_0} \frac{f(x)-f(x_0)}{x-x_0} \]
    exists.}
  	 %------------------------------------END_STEP_X
  	 
  	 %----------------------------------START_STEP_X
    \step 
    \lang{de}{   Da $f$ links und rechts von der Stelle durch verschiedene Vorschriften gegeben ist, betrachten wir
zunächst den linksseitigen Grenzwert
\[ \lim_{x\nearrow x_0} \frac{f(x)-f(x_0)}{x-x_0} \]
und den rechtsseitigen Grenzwert
\[ \lim_{x\searrow x_0} \frac{f(x)-f(x_0)}{x-x_0} \]
getrennt.
   }
   \lang{en}{   Since $f$ is given by different definitions to the left and the right of $x_0$, we have to 
   consider the left-hand limit
\[ \lim_{x\nearrow x_0} \frac{f(x)-f(x_0)}{x-x_0} \]
and the right-hand limit
\[ \lim_{x\searrow x_0} \frac{f(x)-f(x_0)}{x-x_0} \]
separately.
   }
  	 %------------------------------------END_STEP_X
  	 %----------------------------------START_STEP_X
    \step 
    \lang{de}{   Für den linksseitigen Grenzwert gilt wegen $f(x_0)=\ln(1)=0$ und $f(x)=x^2-x$ für $x<1$:
\[ \lim_{x\nearrow x_0} \frac{f(x)-f(x_0)}{x-x_0} =\lim_{x\nearrow 1} \frac{x^2-x-0}{x-1}
 =\lim_{x\nearrow 1} \frac{x(x-1)}{x-1}=\lim_{x\nearrow 1} x =1.
 \]  }
 \lang{en}{   For the left-hand limit, we use $f(x_0)=\ln(1)=0$ and $f(x)=x^2-x$ and obtain, for $x<1$:
\[ \lim_{x\nearrow x_0} \frac{f(x)-f(x_0)}{x-x_0} =\lim_{x\nearrow 1} \frac{x^2-x-0}{x-1}
 =\lim_{x\nearrow 1} \frac{x(x-1)}{x-1}=\lim_{x\nearrow 1} x =1.
 \]  }
  	 %------------------------------------END_STEP_X
 
  	 %----------------------------------START_STEP_X
    \step 
    \lang{de}{      Für den rechtsseitigen Grenzwert gilt wegen $f(x_0)=\ln(1)=0$ und $f(x)=\ln(x)$ für $x>1$:
\[ \lim_{x\searrow x_0} \frac{f(x)-f(x_0)}{x-x_0} =\lim_{x\searrow 1} \frac{\ln(x)-\ln(1)}{x-1} =
\ln'(1)=\frac{1}{1}=1. \]}
\lang{en}{      For the right-hand limit, we use $f(x_0)=\ln(1)=0$ and $f(x)=\ln(x)$ and obtain, for $x>1$:
\[ \lim_{x\searrow x_0} \frac{f(x)-f(x_0)}{x-x_0} =\lim_{x\searrow 1} \frac{\ln(x)-\ln(1)}{x-1} =
\ln'(1)=\frac{1}{1}=1. \]}
  	 %------------------------------------END_STEP_X
  	 
  	 
  	 %----------------------------------START_STEP_X
    \step 
    \lang{de}{   
Da beide Grenzwerte existieren und gleich sind, ist $f$ in $x_0=1$ differenzierbar und $f'(1)=1$.}
\lang{en}{Both limits exist and equal $1$. Therefore, $f$ is differentiable at $x_0=1$ and $f'(1)=1$.}
  	 %------------------------------------END_STEP_X
  	 
  \end{incremental}
  %++++++++++++++++++++++++++++++++++++++++++++END_TAB_X
  
  


  \tab{\lang{de}{Video: ähnliche Übungsaufgabe} \lang{en}{Video: similar exercise}}
  \youtubevideo[500][300]{Klq3BUoDC9k}\\

\end{tabs*}
\end{content}