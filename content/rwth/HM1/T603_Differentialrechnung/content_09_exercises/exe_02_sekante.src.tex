\documentclass{mumie.element.exercise}
%$Id$
\begin{metainfo}
  \name{
    \lang{de}{Ü02: Übergang Sekante/Tangente}
    \lang{en}{Exercise 2: Secant and tangent}
  }
  \begin{description} 
 This work is licensed under the Creative Commons License Attribution 4.0 International (CC-BY 4.0)   
 https://creativecommons.org/licenses/by/4.0/legalcode 

    \lang{de}{Hier die Beschreibung}
    \lang{en}{}
  \end{description}
  \begin{components}
  \end{components}
  \begin{links}
  \end{links}
  \creategeneric
\end{metainfo}
\begin{content}
\title{
  \lang{de}{Ü02: Übergang Sekante/Tangente}
  \lang{en}{Exercise 2: Secant and tangent}
}


\begin{block}[annotation]
	Im Ticket-System: \href{https://team.mumie.net/issues/24011}{Ticket 24011}
\end{block}

\begin{block}[annotation]
Kopie: hm4mint/T106_Differentialrechnung/exercise 2

Im Ticket-System: \href{http://team.mumie.net/issues/9102}{Ticket 9102}
\end{block}

\lang{de}{
\begin{table}[\class{items}]
a) Gegeben sei die Funktion $f(x)=x^{2}$. Berechnen Sie die Steigung der Sekante
\[\Delta(h):=\frac{f(x+h)-f(x)}{h}\,.\]
Was erhalten Sie für $\Delta(h)$, wenn $h$ gegen 0 geht? Vergleichen Sie ihr Ergebnis mit den Ableitungsformeln.\\
b) Wir betrachten nun eine Funktion $f$, welche die Eigenschaft $f(-x)=f(x)$ für alle $x\in\R$ hat. Man sagt in diesem Fall, dass $f$ symmetrisch zur $y$-Achse ist.
Begründen Sie anschaulich anhand des Steigungsdreiecks, dass dann $f'(-x)=-f'(x)$ gilt für alle $x\in\R$.
\end{table}
}

\lang{en}{
\begin{table}[\class{items}]
a) Given the function $f(x)=x^{2}$, determine the slope of the secant
\[\Delta(h):=\frac{f(x+h)-f(x)}{h}\,.\]
What happens to $\Delta(h)$ as $h$ tends to 0? Compare your result to the derivative.\\
b) Now consider a function $f$ with the property $f(-x)=f(x)$ for all $x\in\R$. Such a function $f$ is called symmetric with respect to the $y$-axis.
Use a slope triangle to explain graphically why $f'(-x)=-f'(x)$ for all $x\in\R$.
\end{table}
}

\begin{tabs*}[\initialtab{0}\class{exercise}]
  \tab{
  \lang{de}{Lösung a)}
  \lang{en}{Solution a)}
  }
  
  \begin{incremental}[\initialsteps{1}]
    \step \lang{de}{Es sei $x\in\R$ und $h\in\R\backslash\{0\}$. Dann gilt nach der ersten binomischen Formel
\[\Delta(h)=\frac{(x+h)^{2}-x^{2}}{h}=\frac{x^{2}+2hx+h^{2}-x^{2}}{h}=\frac{2hx+h^{2}}{h}=2x+h\,.\]}
\lang{en}{Let $x\in\R$ and $h\in\R\backslash\{0\}$. By the first binomial formula,
\[\Delta(h)=\frac{(x+h)^{2}-x^{2}}{h}=\frac{x^{2}+2hx+h^{2}-x^{2}}{h}=\frac{2hx+h^{2}}{h}=2x+h\,.\]}
	\step \lang{de}{Für $h\to 0$ geht also $\Delta(h)$ gegen den Ausdruck $2x$. Dies entspricht genau der Formel für die Ableitung der Potenzfunktion  $x\mapsto x^{n}$ im Falle $n=2$.}
 \lang{en}{Therefore, $\Delta(h)$ tends to $2x$ in the limit $h\to 0$. This matches the formula for the derivative of the power function $x \mapsto x^{n}$ when $n=2$. }
  \end{incremental}
  
  \tab{
  \lang{de}{Lösung b)}
  \lang{en}{Solution b)}
  }
  
  \begin{incremental}[\initialsteps{1}]
    \step \lang{de}{Es sei $x_{0}\in\R$ und $h\neq 0$. Das Steigungsdreieck $\Delta_{1}$ durch die Punkte 
\[(-x_{0};f(-x_{0})),(-x_{0}+h;f(-x_{0})),(-x_{0}+h;f(-x_{0}+h))\] 
soll mit dem Steigungsdreieck $\Delta_{2}$ durch die Punkte  
\[(x_{0};f(x_{0})),(x_{0}+h;f(x_{0})),(x_{0}+h;f(x_{0}+h))\] 
verglichen werden. Wegen der Achsensymmetrie von $f$ lassen sich die Eckpunkte von $\Delta_{1}$ schreiben als 
\[(-x_{0};f(x_{0})),(-x_{0}+h;f(x_{0})),(-x_{0}+h;f(x_{0}-h))\,.\]}
\lang{en}{Let $x_{0}\in\R$ and $h\neq 0$. We are supposed to compare the slope triangle $\Delta_{1}$ through the points
\[(-x_{0};f(-x_{0})),(-x_{0}+h;f(-x_{0})),(-x_{0}+h;f(-x_{0}+h))\] 
with the slope triangle $\Delta_{2}$ through the points  
\[(x_{0};f(x_{0})),(x_{0}+h;f(x_{0})),(x_{0}+h;f(x_{0}+h))\,.\] 
Since $f$ is axially symmetric, we can write the vertices of $\Delta_{1}$ in the form
\[(-x_{0};f(x_{0})),(-x_{0}+h;f(x_{0})),(-x_{0}+h;f(x_{0}-h))\,.\]}
	\step \lang{de}{Damit ergibt sich für $f$ an der Stelle $-x_{0}$ die Steigung
\[\frac{f(x_{0}-h)-f(x_{0})}{h}=-\frac{f(x_{0}+h')-f(x_{0})}{h'}\,,\]
wenn man $h'=-h$ substituiert.}
\lang{en}{By substituting $h'=-h$, we obtain the slope of $f$ at the point $-x_{0}$:
\[\frac{f(x_{0}-h)-f(x_{0})}{h}=-\frac{f(x_{0}+h')-f(x_{0})}{h'}\,.\]}
	\step \lang{de}{Man beachte, dass mit $h$ auch $h'$ gegen 0 strebt. Daran sieht man, dass $f'(-x_{0})=-f'(x_{0})$ gilt.}
 \lang{en}{Note that $h'$ also tends to $0$ as $h$ does. This shows that $f'(-x_{0})=-f'(x_{0})$.}
   \end{incremental}

\end{tabs*}


\end{content}