%$Id:  $
\documentclass{mumie.article}
%$Id$
\begin{metainfo}
  \name{
    \lang{de}{Eigenschaften und höhere Ableitungen}
    \lang{en}{Properties and higher derivatives}
  }
  \begin{description} 
 This work is licensed under the Creative Commons License Attribution 4.0 International (CC-BY 4.0)   
 https://creativecommons.org/licenses/by/4.0/legalcode 

    \lang{de}{Beschreibung}
    \lang{en}{}
  \end{description}
  \begin{components}
    \component{generic_image}{content/rwth/HM1/images/g_tkz_T603_ParabolaTangents.meta.xml}{T603_ParabolaTangents}
  \end{components}
  \begin{links}
    \link{generic_article}{content/rwth/HM1/T303_Approximationen/g_art_content_06_de_l_hospital.meta.xml}{content_06_de_l_hospital}
    \link{generic_article}{content/rwth/HM1/T603_Differentialrechnung/g_art_content_09_Ableitungsbegriff.meta.xml}{content_09_Ableitungsbegriff}
    \link{generic_article}{content/rwth/HM1/T301_Differenzierbarkeit/g_art_content_03_hoehere_ableitungen.meta.xml}{content_03_hoehere_ableitungen}
    \link{generic_article}{content/rwth/HM1/T106_Differentialrechnung/g_art_content_22_extremstellen.meta.xml}{extremstellen}
    \link{generic_article}{content/rwth/HM1/T207_Intervall_Schachtelung/g_art_content_23_intervallschachtelung.meta.xml}{intervallschachtelung}
    \link{generic_article}{content/rwth/HM1/T211_Eigenschaften_stetiger_Funktionen/g_art_content_33_zwischenwertsatz.meta.xml}{zwischenwertsatz}
    \link{generic_article}{content/rwth/HM1/T106_Differentialrechnung/g_art_content_23_kurvendiskussion.meta.xml}{kurvendiskussion}
    \link{generic_article}{content/rwth/HM1/T211_Eigenschaften_stetiger_Funktionen/g_art_content_33_zwischenwertsatz.meta.xml}{saetze-stetig}
  \end{links}
  \creategeneric
\end{metainfo}
\begin{content}
\usepackage{mumie.ombplus}
\ombchapter{3}
\ombarticle{3}

\lang{de}{\title{Wichtige Aussagen über die Ableitung}}
\lang{en}{\title{Important properties of the derivative}}

\begin{block}[annotation]
  
  
\end{block}
\begin{block}[annotation]
 Im Ticket-System: \href{https://team.mumie.net/issues/22699}{Ticket 22699}

\end{block}


\begin{block}[info-box]
\tableofcontents
\end{block}

\lang{de}{Nachdem wir wissen, was die Ableitung ist (Grenzwert des Differenzenquotienten), was sie anschaulich bedeutet (Steigung der
Tangente) und wie man sie ausrechnet (Ableitungsregeln), sammeln wir hier einige nützliche Sätze über die Ableitung, die wir 
im nächsten Kapitel zur Kurvendiskussion verwenden werden. 
}
\lang{en}{
Now that we know what the derivative is
(a limit of difference quotients),
what it means intuitively
(the slope of the tangent line)
and how to compute it (differentiation rules),
we will formulate several useful properties of the
derivative that will be used in the next chapter on
the analysis of curves.
}


\section{
\lang{de}{Stetigkeit und Mittelwertsatz}
\lang{en}{Continuity and the mean value theorem}
}

\lang{de}{In diesem Abschnitt behandeln wir zwei eher theoretische Sätze. Wir beginnen mit dem Zusammenhang zwischen Stetigkeit und 
Differenzierbarkeit. 
}
\lang{en}{
In this section, we will discuss two theorems of a
rather theoretical nature. We will begin with the
relationship between continuity and differentiability.
}


\begin{theorem}
\lang{de}{Es sei $f$ eine Funktion, die auf einem offenen Intervall $I$ differenzierbar ist (d.\,h. $f'$ existiert). 
}
\lang{en}{
Let $f$ be a function that is differentiable on an open
interval $I$ (i.e. $f'$ exists).
}

\lang{de}{In diesem Fall ist $f$ auch stetig auf $I$ (d.\,h. der Funktionsgraph macht keine Sprünge).
}
\lang{en}{Then $f$ is also continuous on $I$
(i.e. its graph has no jumps).}
\end{theorem}

\begin{proof*}
\lang{de}{Der Beweis findet sich im \ref[content_03_hoehere_ableitungen][Hauptkurs]{thm:diffstetig}.
}
\lang{en}{The proof can be found in
the \ref[content_03_hoehere_ableitungen][main course]{thm:diffstetig}.
}
\end{proof*}

\lang{de}{Jede differenzierbare Funktion ist also stetig, aber umgekehrt ist nicht jede stetige Funktion differenzierbar. Die Betragsfunktion 
$f(x) = |x|$ besitzt etwa an der Stelle $x=0$ keine Ableitung, wie wir \ref[content_09_Ableitungsbegriff][bereits gesehen haben]{ex:nichtdiffbar}. 
}
\lang{en}{
While every differentiable function is continuous,
not every continuous function is differentiable.
The absolute value function $f(x) = |x|$, for example,
does not have a derivative at $x=0$, \ref[content_09_Ableitungsbegriff][as we saw earlier]{ex:nichtdiffbar}.
}


\lang{de}{Das folgende Resultat, der \emph{Mittelwertsatz}, ist innerhalb der Analysis von großer Bedeutung, etwa bei der Beweisführung von Aussagen, 
da er einen Zusammenhang zwischen Funktionswerten und Ableitung herstellt.  
Auch wenn wir ihn nicht benutzen werden, soll der Satz hier zumindest erwähnt werden.
}
\lang{en}{
The following result, the \emph{mean value theorem}, plays
an important role in analysis (for example, in proofs)
because it provides a connection between the values of a function
and its derivative. The theorem is worth stating here,
even though we will not have to use it.
}


\begin{theorem}[
\lang{de}{Mittelwertsatz}
\lang{en}{Mean value theorem}]\label{thm:mittelwertsatz}
  \lang{de}{Es sei $f$ eine auf einem Intervall $I \subset \R$ differenzierbare Funktion und es seien $a, b \in I$ zwei verschiedene reelle
  Zahlen. 
  
  Dann existiert eine Stelle $\xi \in \R$ zwischen $a$ und $b$, für die die Gleichung
  \[ 
  \frac{f(b)-f(a)}{b-a} = f'(\xi )
  \]
  erfüllt ist.}
  \lang{en}{Let $f$ be a differentiable function on an interval $I \subseteq \R$,
  and let $a, b \in I$ be two distinct real numbers.

  Then there exists a point $\xi \in \R$ between $a$ and $b$ that satisfies the equation
  \[ 
  \frac{f(b)-f(a)}{b-a} = f'(\xi ).
  \]
  }
\end{theorem}
\begin{proof*}
\lang{de}{Der Beweis findet sich im \ref[content_03_hoehere_ableitungen][Hauptkurs]{thm:mittelwertsatz}.}
\lang{en}{The proof can be found in the \ref[content_03_hoehere_ableitungen][main course]{thm:mittelwertsatz}.}
\end{proof*}
\lang{de}{Anschaulich bedeutet der Mittelwertsatz, dass es bei differenzierbaren Funktionen zu jeder Sekante durch zwei Punkte auch eine Tangente mit gleicher Steigung 
gibt.}
\lang{en}{
In graphical terms, the mean value theorem states that for any
secant through two points on the graph of a differentiable function,
there exists a tangent with the same slope.
}

\section{
\lang{de}{Höhere Ableitungen}
\lang{en}{Higher derivatives}
}

\lang{de}{Für eine differenzierbare Funktion $f: I\to \R$ haben wir die Ableitung auch wieder als Funktion $f': I\to \R$ aufgefasst. Wenn diese
differenzierbar ist, kann man auch diese Funktion ableiten und erhält eine \emph{höhere Ableitung}.
}
\lang{en}{
Given a differentiable function $f: I\to \R$, we have
interpreted its derivative as another function $f' : I\to \R$.
If the latter is differentiable, we can take its derivative
to obtain a \emph{higher derivative} of $f$.
}


\begin{definition}[Höhere Ableitungen]
  	\lang{de}{
  		Es sei $f$ eine differenzierbare Funktion mit Ableitungsfunktion $f'$. }
    \lang{en}{
      Let $f$ be a differentiable function with derivative $f'$.
    }
\begin{itemize}        
        \item \lang{de}{Ist die Funktion $f'$ ebenfalls differenzierbar,
		so hei{\ss}t $f$ \notion{zweimal differenzierbar}. Man schreibt dann statt $(f')'$ kurz $f''$ und nennt $f''$
        die \notion{zweite Ableitung}. }
        \lang{en}{If $f'$ is itself a differentiable function,
        then we call $f$ \notion{twice differentiable}. We write
        $f''$ instead of $(f')'$ and call $f''$ the \notion{second derivative}.}
	
    
\item \lang{de}{Genau so werden die dritte, vierte, $\ldots$, $n$-te Ableitung von $f$ als Ableitungen
der zweiten, dritten, $\ldots$, $(n-1)$-ten Ableitung von $f$ definiert, sofern diese existieren und differenzierbar sind. 
Man sagt dann, dass $f$ \notion{dreimal, viermal,  $\ldots$, $n$-mal differenzierbar} ist. }
\lang{en}{
Similarly, the third, fourth, $\ldots$, $n$th derivatives of $f$ are defined
as the derivatives of the second, third,$\ldots$, $(n-1)$th derivatives of $f$,
assuming that these exist and are differentiable. In this case,
we say that $f$ is \notion{three times, four times, $\ldots$, $n$ times differentiable}.
}

\item 
\lang{de}{Die Ableitung $f'$ wird auch als \notion{erste Ableitung} von $f$ bezeichnet. Die $n$-te 
 Ableitungsfunktion von $f$ wird auch mit $f^{(n)}$ bezeichnet, um zu viele Striche zu vermeiden.
}
\lang{en}{The derivative $f'$ is also called the \notion{first derivative}
of $f$. The $n$th derivative of $f$ is also denoted $f^{(n)}$
in order to avoid an excess of prime symbols.}

\item 
\lang{de}{Existiert die $n$-te Ableitungsfunktion von $f$ und ist diese stetig, so sagt man, dass $f$ \notion{$n$-mal stetig differenzierbar} ist. 
}
\lang{en}{
If the $n$th derivative of $f$ exists and defines a continuous
function, then we call $f$ \notion{$n$ times continuously differentiable}.
}
\end{itemize}	
    
\end{definition}

\lang{de}{Wenn eine Funktion etwa $2$-mal stetig differenzierbar ist, dann ist die Funktion $f''$ stetig. Aber auch $f'$ und $f$  
sind stetig, da sie differenzierbare Funktionen sind und damit stetig. 
}
\lang{en}{
If a function is twice continuously differentiable, for example,
then the function $f''$ is continuous. $f'$ and $f$
are also continuous because they are differentiable.
}

\begin{example}
\begin{enumerate}
\item \lang{de}{Für $f(x)=x^3+2x^2+5$ gelten 
\[f'(x)=3x^2+4x \quad \text{und} \quad f''(x)= (f')'(x) = 6x+4.
\]
Dann sind 
\[ f^{(3)}(x)=(f'')'(x)=6\quad \text{und}\quad f^{(4)}(x)=(f^{(3)})'(x)=0. \]
Alle höheren Ableitungen $f^{(n)}(x)$ mit $n>4$ sind dann aber auch konstant $0$.
}
\lang{en}{For $f(x)=x^3+2x^2+5$,
\[f'(x)=3x^2+4x \quad \text{and} \quad f''(x)= (f')'(x) = 6x+4.
\]
Then
\[ f^{(3)}(x)=(f'')'(x)=6\quad \text{and}\quad f^{(4)}(x)=(f^{(3)})'(x)=0. \]
All higher derivatives $f^{(n)}(x)$, $n>4$ will also be the constant $0$.
}


\item \lang{de}{Für $f(x)=\sin(x)$ ist $f'(x)=\cos(x)$ und  $f''(x)=-\sin(x)=-f(x)$. Damit sind
\[ f^{(3)}(x)=(f'')'(x)=-f'(x)=-\cos(x) \quad \text{und}\quad f^{(4)}(x)=(f^{(3)})'(x)=-f''(x)=f(x). \]
Die vierte Ableitung ist also die Funktion selbst und es geht von vorne los, weshalb wir generell für $k>0$ die Beziehung
\[  f^{(k+4)}=f^{(k)} \]
erhalten.}
\lang{en}{
For $f(x)=\sin(x)$, we have $f'(x)=\cos(x)$ and
$f''(x)=-\sin(x)=-f(x)$. Therefore,
\[ f^{(3)}(x)=(f'')'(x)=-f'(x)=-\cos(x) \quad \text{and}\quad f^{(4)}(x)=(f^{(3)})'(x)=-f''(x)=f(x). \]
The fourth derivative is just $f$ itself and the pattern of derivatives repeats,
such that for $k>0$ we obtain
\[  f^{(k+4)}=f^{(k)}. \]
}

\item 
\lang{de}{Für $f(x)=e^{2x}$ sind $f'(x)=2e^{2x}$ und $f''(x)=4e^{2x}$. Man sieht hier schon ein Muster für die höheren Ableitungen, nämlich
\[  f^{(n)}(x)=2^n\cdot e^{2x}. \]
}
\lang{en}{
For $f(x)=e^{2x}$, we have $f'(x)=2e^{2x}$ and $f''(x)= 4e^{2x}$.
The pattern that the higher derivatives follow is already visible:
\[  f^{(n)}(x)=2^n\cdot e^{2x}. \]
}
\end{enumerate}
\end{example}

   \begin{quickcheck}
		\field{rational}
		\type{input.function}
		\begin{variables}
			\randint[Z]{a}{-3}{5}
		    \function[normalize]{f}{(x-a)/(x+a)}
            \function[normalize]{df}{(2*a)/(x+a)^2}
            \derivative[normalize]{ddf}{df}{x}
            \derivative[normalize]{df3}{ddf}{x}
		\end{variables}

    \lang{de}{
		\text{Die höheren Ableitungen der Funktion $f(x)=\var{f}$ sind\\ $f'(x)= $\ansref, \\
        $f''(x)= $\ansref, \\ $f'''(x)= $\ansref.}
		}
    \lang{en}{
    \text{The higher derivatives of the function $f(x)=\var{f}$ are\\
        $f'(x)= $\ansref, $f''(x)= $\ansref, \\ $f'''(x)= $\ansref.}
    }
  
		\begin{answer}
			\solution{df}
			\checkAsFunction{x}{-1}{1}{10}
		\end{answer}
		\begin{answer}
			\solution{ddf}
			\checkAsFunction{x}{-1}{1}{10}
		\end{answer}
		\begin{answer}
			\solution{df3}
			\checkAsFunction{x}{-1}{1}{10}
		\end{answer}
		\lang{de}{\explanation{Die nächsthöhere Ableitung ist stets die Ableitung der vorhergehenden
        Ableitung. Sie können die Ableitung mit Quotientenregel oder Kettenregel berechnen.
        }}
    \lang{en}{
    \explanation{The next higher derivative is always the derivative
    of the previous function. The derivatives can be calculated with the quotient rule or chain rule.}
    }
	\end{quickcheck}
    

\section{
\lang{de}{Monotonie und Krümmung}
\lang{en}{Monotonicity and convexity}
}

\lang{de}{Nun interessieren wir uns daf\"{u}r, in welchen Bereichen eine Funktion $f$ w\"{a}chst oder f\"{a}llt. Dies ist z.\,B. 
interessant, wenn man nach Minimal- und Maximalstellen sucht. Im nächsten Kapitel werden wir ausführlich behandeln, wie wir solche 
Stellen finden können. Zur Vorbereitung formulieren wir hier den Zusammenhang 
zwischen Ableitung und Monotonie bzw. Krümmung des Funktionsgraphen.
}
\lang{en}{
We will now consider the regions in which a function
is increasing or decreasing. This is useful when
looking for maxima and minima, among other things.
We will describe how to find these points in more detail
in the next chapter. Here we will formulate the connection
between the derivative and monotonicity or convexity
of a function's graph.
}

\lang{de}{Indem wir uns vorstellen, dass wir von links nach rechts entlang des Funktionsgraphen Fahrrad fahren, können wir die Krümmung des Funktionsgraphen dadurch 
beschreiben, in welche Richtung wir lenken müssen (links oder rechts). 
}
\lang{en}{
If we imagine riding a bicycle from left to right along
the graph of the function, then the convexity of the graph
determines the direction that we have to turn
(left or right).
}

\begin{theorem}
\lang{de}{Ist $f$ eine auf einem Intervall $I$ differenzierbare Funktion, so gelten:
}
\lang{en}{
For a differentiable function $f$ on an interval $I$, the following hold:
}
\begin{itemize}
\item 
\lang{de}{$f'(x) \geq 0$ für alle $x \in I \quad \Leftrightarrow \quad f$ ist monoton wachsend.
}
\lang{en}{$f'(x) \geq 0$ for all $x \in I \quad \Leftrightarrow \quad f$ is monotonically increasing}
\item 
\lang{de}{$f'(x) \leq 0$ für alle $x \in I \quad \Leftrightarrow \quad f$ ist monoton fallend.}
\lang{en}{$f'(x) \leq 0$ for all $x \in I \quad \Leftrightarrow \quad f$ is monotonically decreasing.}
\item 
\lang{de}{$f'(x) \textcolor{#CC6600}{>} 0$ für alle $x \in I \quad \textcolor{#CC6600}{\Rightarrow} \quad f$ ist \textcolor{#CC6600}{streng} monoton wachsend.}
\lang{en}{$f'(x) \textcolor{#CC6600}{>} 0$ for all $x \in I \quad \textcolor{#CC6600}{\Rightarrow} \quad f$ is \textcolor{#CC6600}{strictly} monotonically increasing.}
\item 
\lang{de}{$f'(x) \textcolor{#CC6600}{<} 0$ für alle $x \in I \quad \textcolor{#CC6600}{\Rightarrow} \quad f$ ist \textcolor{#CC6600}{streng} monoton fallend.}
\lang{en}{$f'(x) \textcolor{#CC6600}{<} 0$ for all $x \in I \quad \textcolor{#CC6600}{\Rightarrow} \quad f$ is \textcolor{#CC6600}{strictly} monotonically decreasing.}
\end{itemize}
\lang{de}{Ist $f$ zweimal auf $I$ differenzierbar, dann gelten außerdem:}
\lang{en}{If in addition $f$ is twice differentiable on $I$, then:}
\begin{itemize}
\item 
\lang{de}{$f''(x) \geq 0$ für alle $x \in I \quad \Leftrightarrow \quad f$ ist konvex (linksgekrümmt).}
\lang{en}{$f''(x) \geq 0$ for all $x \in I \quad \Leftrightarrow \quad f$ is convex (convex down; concave up).}
\item 
\lang{de}{$f''(x) \leq 0$ für alle $x \in I \quad \Leftrightarrow \quad f$ ist konkav (rechtsgekrümmt).}
\lang{en}{$f''(x) \leq 0$ for all $x \in I \quad \Leftrightarrow \quad f$ is concave (convex up; concave down).}
\end{itemize}
\end{theorem}
\lang{de}{Die rot markierten Implikationspfeile zeigen an, dass die Folgerung nur in eine Richtung gültig ist. 
Eine streng monoton wachsende Funktion kann durchaus Punkte haben, an denen $f'(x) =0$ gilt. Dies gilt z.\,B.
für $f(x) = x^3$ an der Stelle $0$.}
\lang{en}{
The arrows in red indicate that the implication is only true
in one direction. A strictly monotonically increasing function
can have points at which $f'(x) = 0$. For example, this
is the case for $f(x) = x^3$ at $x=0$.
}
\begin{proof*}
\lang{de}{Der Beweis der Monotonieaussagen beruht auf dem Mittelwertsatz und findet sich im 
\ref[content_03_hoehere_ableitungen][Hauptkurs]{thm:diffmonotonie}. Die Aussagen zur Krümmung folgen 
aus den Aussagen zur Monotonie, indem man $f$ durch $f'$ ersetzt. 
}
\lang{en}{
The proof for the claims related to monotonicity relies
on the mean value theorem and can be foun in the 
\ref[content_03_hoehere_ableitungen][main course]{thm:diffmonotonie}.
The claims involving convexity follow from those involving
monotonicity after replacing $f$ by $f'$.
}
\end{proof*}

\begin{example}
	\begin{center}
    \image{T603_ParabolaTangents}
    \end{center}
	\lang{de}{Die Funktion $f(x)=x^2$ ist monoton fallend auf $(-\infty;0]$ und monoton steigend auf $[0;\infty)$. 
    An jeder Stelle $x_0 \leq 0$ hat also die Tangente eine Steigung $f'(x_0)\leq 0$, und an jeder Stelle $x_1 \geq 0$ ist die Steigung
    $f'(x_1) \geq 0$.}
  \lang{en}{
    The function $f(x)=x^2$ is monotonically decreasing
    on $(-\infty, 0]$ and monotonically increasing
    on $[0, \infty)$. Therefore, the tangent at any point
    $x_0 \le 0$ has slope $f'(x_0) \le 0$, and at any point
    $x_1 \ge 0$ its slope is $f'(x_1) \ge 0$.
  }
    
\lang{de}{Im Bild sind zwei Stellen $x_0$ und $x_1$ eingezeichnet. 
    Da die Tangente in $x_0 <0$ sogar eine negative Steigung hat, ist $f'(x) <0$ f\"{u}r $x\in (-\infty;0)$, 
	und damit ist $f$ auf $(-\infty;0)$ streng monoton fallend. Analog gilt $f'(x)>0$ auf $(0;\infty)$ und $f$ ist dort streng monoton steigend.
    }
  \lang{en}{
    Two such points $x_0$ and $x_1$ are displayed
    in the above image. Since the tangent at $x_0$ has
    a strictly negative slope, we have $f'(x) < 0$ for
    $x \in (-\infty, 0)$ and therefore $f$ is strictly
    monotonically decreasing there. Similarly,
    $f'(x) > 0$ on $(0, \infty)$, so $f$ is strictly
    monotonically increasing there.
  }
    \lang{de}{Der Graph der Funktion ist linksgekrümmt bzw. konvex, daher ist $f''(x) \geq 0$ für alle $x \in \R$. Und tatsächlich gilt ja $f''(x)=2 >0$. 
    }
    \lang{en}{
    The graph of the function is convex, so $f''(x) \ge 0$
    for all $x \in \R$. Indeed, $f''(x) = 2 > 0$.
    }
\end{example}

\begin{remark}
\lang{de}{Das Krümmungsverhalten allein erlaubt keinen Rückschluss darauf, ob eine Funktion wächst oder fällt.
}
\lang{en}{
Convexity or concavity of a function by itself is
not enough to draw any conclusions as to whether it
is increasing or decreasing.
}

\lang{de}{Eine linksgekrümmte, monoton wachsende Funktion wird immer schneller wachsen. Man bezeichnet solche Funktionen auch als \emph{progressiv wachsend}.}
\lang{en}{
A convex, monotonically increasing function will increase
more and more steeply. These functions undergo \emph{accelerating growth}.
}

\lang{de}{Eine rechtsgekrümmte, monoton wachsende Funktion wird hingegen immer langsamer wachsen und flacher verlaufen. Man bezeichnet solche Funktionen auch als \emph{degressiv wachsend}.
}
\lang{en}{
A concave, monotonically increasing function will increase
more and more slowly and its graph will become flatter.
These functions undergo \emph{decelerating growth}.
}

\lang{de}{Analog bezeichnet man linksgekrümmte, monoton fallende Funktionen als \emph{degressiv fallend} und rechtsgekrümmte, monoton fallende Funktionen als 
\emph{progressiv fallend}.}
\end{remark}


\section{
\lang{de}{Elastizität, Taylor-Entwicklung und Regeln von L'Hospital}
\lang{en}{Elasticity, Taylor expansions and L'Hôpital's rule}
}
\lang{de}{
In diesem Abschnitt befassen wir uns mit Aussagen, die eine gewisse Näherug an gesuchte 
Werte oder sogar ganze Funktionen erlauben. 
}
\lang{en}{
In this section, we will introduce results that allow
certain values or even entire functions to be approximated.
}

\lang{de}{Wenn wir eine Funktion $f$ haben und die \emph{relative} Änderung des Funktionswerts suchen, ist dies einfach der Bruch
}
\lang{en}{
If we have a function $f$ and we want to find the
\emph{relative} change of its values then this is
simply the quotient
}

\[
{\frac{f(x+h)-f(x)}{f(x)}}.
\]

\lang{de}{Im Zähler steht die Änderung des Funktionswerts und im Nenner der Funktionswert als Referenzwert. 
Die relative Änderung des Funktionswerts pro relativer Änderung der Eingangsgröße (die Variable $x$) ist dann 
}
\lang{en}{
The change in value of the function appears in the
numerator and the value of the function itself appears
in the denominator. The relative change in value
with respect to the relative change in the input
(the variable $x$) is then
}
\[
\frac{\frac{f(x+h)-f(x)}{f(x)}}{\frac{h}{x}} = \frac{x}{f(x)} \cdot \frac{f(x+h) - f(x)}{h} \underset{ {h \to 0}}{\longrightarrow} \frac{xf'(x)}{f(x)},
\]
\lang{de}{vorausgesetzt, die Funktion $f$ ist differenzierbar und $f(x) \neq 0$. Dies führt zur folgenden Definition. 
}
\lang{en}{
assuming the function $f$ is differentiable and $f(x) \ne 0$.
This leads to the following definition.
}

\begin{definition}
\lang{de}{Ist $f: I \rightarrow \R$ eine differenzierbare Funktion, dann definieren wir für alle $x \in I$ mit $f(x) \neq 0$ den Wert
\[
\epsilon_{f}(x) = \frac{x f'(x)}{f(x)}
\]
und bezeichnen ihn als die \notion{Elastizität} von $f$ an der Stelle $x$. 
}
\lang{en}{
Let $f : I \rightarrow \R$ be a differentiable function. For any $x \in I$
with $f(x) \ne 0$, we define
\[
\epsilon_{f}(x) = \frac{x f'(x)}{f(x)}
\]
and call it the \notion{elasticity} of $f$ at the point $x$.
}

\lang{de}{Gilt $|\epsilon_f(x)| < 1$, so sagen wir, dass $f$ an $x$ \notion{unelastisch} ist. 
}
\lang{en}{
If $|\epsilon_f(x)| < 1$, then we call $f$ \notion{inelastic} in $x$.
}
\lang{de}{
Im Fall $|\epsilon_f(x)| > 1$ nennen wir $f$ \notion{elastisch} an $x$. 
}
\lang{en}{
If $|\epsilon_f(x)| < 1$, then we call $f$ \notion{elastic} in $x$.
}
\end{definition}

\lang{de}{
Für den Fall $|\epsilon_f(x)|=1$ existieren verschiedene Bezeichnungen, z.\,B. \emph{einheitselastisch}, \emph{proportional} oder \emph{fließend}.
}
\lang{en}{
When $|\epsilon_f(x)|=1$, $f$ is called \notion{unitary elastic} or \notion{unit elastic} in $x$.
}

\begin{rule}[\lang{de}{Wichtige Faustregel} \lang{en}{Important rule of thumb}]
\lang{de}{Die Elastizität $\epsilon_f(x)$ gibt ungefähr die Änderung des Wertes $f(x)$ in Prozent an, wenn sich die Eingangsgröße $x$ um 1\% erhöht. 

Bei elastischen Funktionen ist die relative Änderung der Funktionswerte 
größer als die relative Änderung der Eingangswerte. Bei elastischen Funktionen 
können kleine Ursachen
große Wirkung haben.}

\lang{en}{
The elasticity $\varepsilon_f(x)$ roughly describes the percent change of $f(x)$
when the input variable $x$ is increased by 1\%.

For elastic functions, the relative change in the value of $f$ is greater than the
relative change of $x$. Small changes in the input of an elastic function can have a large impact.
}
\end{rule}

\begin{example}
\begin{tabs*}[\initialtab{0}]
\tab{$x^{n}$}
\lang{de}{Für die Funktion $f$ mit $f(x) = x^{-2}$ können wir die Elastizität für $x \neq 0$ berechnen und erhalten 
\[
\epsilon_f(x) = \frac{x \cdot (-2) \cdot  x^{-2-1}}{x^{-2}} = \frac{-2 \not{x^{-2}}}{\not{x^{-2}}} = -2.
\]
Wenn sich $x$ um 1\% erhöht, wird der Wert von $f(x)$ ungefähr um 2\% fallen.}
\lang{en}{
Let $f$ be the function $f(x) = x^{-2}$. For $x \ne 0$, we can compute the
elasticity and we find
\[
\epsilon_f(x) = \frac{x \cdot (-2) \cdot  x^{-2-1}}{x^{-2}} = \frac{-2 \not{x^{-2}}}{\not{x^{-2}}} = -2.
\]
When $x$ increases by 1\%, the value of $f(x)$ falls by approximately 2\%.
}

\lang{de}{Wir betrachten nun allgemeiner die Funktion $f$ mit $f(x) = x^n$. Dann ist 
\[
\epsilon_f(x) = \frac{x \cdot n \cdot  x^{n-1}}{x^n} = \frac{n \not{x^{n}}}{\not{x^{n}}} = n.
\]
Die Potenzfunktionen $x^n$ haben also konstante Elastizitäten.}
\lang{en}{
More generally, consider the function $f$ with $f(x) = x^n$. Then
\[
\epsilon_f(x) = \frac{x \cdot n \cdot  x^{n-1}}{x^n} = \frac{n \not{x^{n}}}{\not{x^{n}}} = n.
\]
So the power functions $x^n$ have constant elasticity.
}

%\tab{$e^{ax}$}
%Für die Funktion $f$ mit $f(X) = e^{ax}$ berechnet sich die Elastizität in einem Punkt durch 
%\[
%\epsilon_f(x) = \frac{x \cdot a \cdot  e^{ax}}{e^{ax}} = ax.
%\]

\tab{\lang{de}{Elastische Nachfrage} \lang{en}{Elastic demand}}
\lang{de}{Gegeben ist die Preis-Absatz-Funktion $x(p) = 60-3p$, die für $p \in [0; 20]$ definiert ist. Wir berechnen 
für diese Funktion die Elastizität an einer Stelle $p$:
\[
\epsilon_x(p) = \frac{p \cdot x'(p)}{x(p)} =  \frac{-3p}{60-3p} = -\frac{p}{20-p}.
\]}
\lang{en}{
Suppose we are given the price-demand function $x(p)=60-3p$, defined for $p \in [0,20]$.
We will compute the elasticity of this function at the point $p$:
\[
\epsilon_x(p) = \frac{p \cdot x'(p)}{x(p)} =  \frac{-3p}{60-3p} = -\frac{p}{20-p}.
\]
}

\lang{de}{Wir sind nun daran interessiert, wann $x(p)$ elastisch ist und wann nicht. Dazu bestimmen wir die Werte für $p$, für die 
$\epsilon_x(p) = 1$ oder $\epsilon_x(p) = -1$ gilt. An diesen Stellen kann die 
Funktion $x(p)$ von unelastisch zu elastisch oder umgekehrt wechseln. }
\lang{en}{
We would like to know when $x(p)$ is elastic and when it is not.
Therefore, we will determine the values of $p$ for which $\epsilon_x(p) = 1$
or $\epsilon_x(p) = -1.$ At these points, the function $x(p)$ may change
from inelastic to elastic or vice versa.
}

\lang{de}{Im Fall $\epsilon_x(p) = 1$ müsste
\[
-p =  20 -p  \quad \Leftrightarrow \quad 0 = 20
\]
gelten, was ein Widerspruch ist. Folglich gibt es hier auch keine Lösung für $p$. }
\lang{en}{
For $\epsilon_x(p) = 1$, we would need
\[
-p =  20 -p  \quad \Leftrightarrow \quad 0 = 20,
\]
which is impossible. Therefore, there is no solution for $p$.
}

\lang{de}{Im Fall $\epsilon_x(p) = -1$ erhalten wir die Gleichung
\[
-p =  -1 \cdot (20 -p) \quad \Leftrightarrow \quad -p = -20 + p \quad \Leftrightarrow \quad -2p = -20,
\]
was die eindeutige Lösung $p = 10$ besitzt. }
\lang{en}{
In the case $\epsilon_x(p) = -1$, we get the equation
\[
-p =  -1 \cdot (20 -p) \quad \Leftrightarrow \quad -p = -20 + p \quad \Leftrightarrow \quad -2p = -20,
\]
which has the unique solution $p = 10$.
}

\lang{de}{Wir betrachten nun noch einmal die Elastizitätsfunktion, die wir oben ausgerechnet haben:
\[
\epsilon_x(p) = -\frac{p}{20-p}.
\]
Wenn $p< 10$ ist, dann ist der Zähler oben kleiner als der Nenner (der immer noch eine
zweistellige Zahl sein wird). Es gilt also $|\epsilon_x(p)| < 1$ und die Funktion $x$ ist an $p$ 
unelastisch.}
\lang{en}{
Now consider again the elasticity function that we computed earlier:
\[
\epsilon_x(p) = -\frac{p}{20-p}.
\]
If $p < 10$, the numerator above is smaller than the denominator (which will be a double-digit number),
so $|\epsilon_x(p)| < 1$ and the function $x$ is inelastic in $p$.
}

\lang{de}{Wenn $p >10$ ist, dann verhält es sich genau umgekehrt und es gilt $|\epsilon_x(p)| > 1$, 
die Funktion $x$ ist also an $p$  elastisch. }
\lang{en}{
If $p > 10$, then the opposite is true and $|\epsilon_x(p)| >1$, so the function $x$
is elastic in $p$.
}
%\tab{Amoroso-Robinson}

\end{tabs*}


\end{example}

\lang{de}{Komplizierte Funktionen können häufig auch gut durch Polynom-Funktionen angenähert werden, zumindest in der Nähe einer gewählten 
Stelle $x_0$. Ein Sonderfall hiervon ist die 
\emph{Linearisierung}, die eine Funktion durch eine Geradengleichung approximiert.
}
\lang{en}{
Complicated functions can often be described well by polynomials, at least in a
neighborhood of a chosen point $x_0$. A special case of this is \emph{linearization},
in which a function is approximated by the equation of a line.
}

\begin{definition}[\lang{de}{Linearisierung und Taylor-Polynom}
\lang{en}{Linearization and Taylor polynomials}]
\lang{de}{Es sei $f: I \rightarrow \R$ eine $n$-mal differenzierbare Funktion ($n \in \N$) und es sei $x_0 \in I$ eine reelle Zahl.
}
\lang{en}{
Let $f : I \rightarrow \R$ be an $n$-times differentiable function ($n \in \N$)
and let $x_0 \in I$ be a real number.
}
\begin{itemize}
\item \lang{de}{Das Polynom
  \[ T_n(x)\coloneq\sum_{k=0}^{n} \frac{f^{(k)}(x_0)}{k!}(x-x_0)^k \] 
heißt \notion{Taylor-Polynom $n$-ter Ordnung} von $f$ zur \notion{Entwicklungsstelle} $x_0\in I$.
  }
  \lang{en}{The polynomial
  \[ T_n(x)\coloneq\sum_{k=0}^{n} \frac{f^{(k)}(x_0)}{k!}(x-x_0)^k \] 
  is called the \notion{Taylor polynomial of order $n$} of $f$ about the \notion{base point} $x_0 \in I$.
  }
  
  \lang{de}{Dabei bezeichnet $f^{(k)}(x_0)$ die $k$-te Ableitung von $f$ an der Entwicklungsstelle $x_0$ für $0 \leq k \leq n$.}
  \lang{en}{Here, $f^{(k)}(x_0)$ stands for the $k$-th derivative of $f$ in the base point $x_0$ for $0 \le k \le n$.}
\item \lang{de}{Die Geradengleichung $g(x) = f(x_0) + f'(x_0) \cdot (x- x_0)$, die identisch mit dem Taylor-Polynom $T_1(x)$ ist, 
wird auch als \notion{Linearisierung} von $f$ an $x_0$ bezeichnet. Sie ist die Tangente an den Funktionsgraphen an der 
Stelle $x_0$.}
\lang{en}{
The equation of the line $g(x) = f(x_0) + f'(x_0) \cdot (x - x_0)$, which coincides with
the Taylor polynomial $T_1(x)$, is also known as the \notion{linearization} of $f$ in $x_0$.
This is the tangent to the graph of the function at $x_0$.
}
\end{itemize}
\end{definition}
\lang{de}{Wenn für $n$ immer größere natürliche Zahlen gewählt werden, erhält man schließlich eine \emph{Taylor-Reihe}, die in vielen Fällen 
(aber nicht bei jeder Funktion) gegen die ursprüngliche Funktion konvergiert. }
\lang{en}{
Choosing arbitrarily large natural numbers for $n$ leads to a \emph{Taylor series},
which in many (but not all) cases converges to the original function.
}

\begin{example}\label{ex:taylorpoly-exp}\label{ex:taylorpoly-geom-reihe}\label{ex:taylorpoly-ln}
\begin{tabs*}[\initialtab{0}]
\tab{
\lang{de}{Polynom}
\lang{en}{Polynomials}
}

\lang{de}{Wir bestimmen die Linearisierung der Funktion $f:\R\to\R$, $f(x)=x^2$ zu den zwei Entwicklungsstellen $x_0=0$ und $x_0=1$. Wir bezeichnen dieses
als $T_{0,1}$ und $T_{1,1}$. }
\lang{en}{
We will determine the linearizations of the function $f:\R\to\R$, $f(x)=x^2$
at the two base points $x_0 = 0$ and $x_0 = 1$. These will be labeled $T_{0,1}$
and $T_{1,1}$.
}

\lang{de}{Für $x_0=0$ finden wir
\[T_{0,1}(x)=f(0)+f'(0)\cdot(x-0)=0+0\cdot x=0,\]
also die Nullfunktion.}
\lang{en}{
At $x_0=0$, we find
\[T_{0,1}(x)=f(0)+f'(0)\cdot(x-0)=0+0\cdot x=0,\]
i.e. the zero function.
}
\lang{de}{Für $x_0=1$ erhalten wir
\[T_{1,1}(x)=f(1)+f'(1)\cdot(x-1)=1+2\cdot (x-1)\neq T_{0,1}(x).\]}
\lang{en}{
At $x_0=1$, we find
\[T_{1,1}(x)=f(1)+f'(1)\cdot(x-1)=1+2\cdot (x-1)\neq T_{0,1}(x).\]
}
\tab{\lang{de}{Natürliche Exponentialfunktion}
\lang{en}{The exponential function}}%\label{ex:taylorpoly-exp}
\lang{de}{Wir betrachten die Exponentialfunktion $f:\R\to\R$, $f(x)=e^x$ an der Entwicklungsstelle $x_0=0$.
Da $f' =f$ beim Ableiten unverändert bleibt und $f(0)=e^0=1$ ist, erhalten wir das 
Taylor-Polynom $n$-ter Ordnung an der Stelle $x_0=0$ zu
\[ T_n(x)=\sum_{k=0}^n \frac{1}{k!}x^k .\]}
\lang{en}{
We will consider the exponential function $f:\R\to\R$, $f(x)=e^x$
at the base point $x_0 = 0$. Since $f' = f$ is unchanged by differentiating
and $f(0)=e^0=1$, the Taylor polynomial of order $n$ about the point $x_0 = 0$ is
\[ T_n(x)=\sum_{k=0}^n \frac{1}{k!}x^k .\]
}
\lang{de}{Für $n \to \infty$ erhält man hier übrigens wieder die Reihendarstellung der Exponentialfunktion.}
\lang{en}{
In the limit $n \to \infty$ we obtain the series expansion of the exponential function.
}
\tab{
\lang{de}{Gebrochenrationale Funktion}
\lang{en}{Rational functions}}%\label{ex:taylorpoly-geom-reihe}
\lang{de}{Wir betrachten die Funktion $f:\R\setminus\{1\}\to\R$, $f(x)=\frac{1}{1-x}$ an der Stelle $x_0=0$.
Schreibt man $f(x)$ als $f(x)=(1-x)^{-1}$, erhält man mit der Kettenregel für die ersten Ableitungen
\begin{eqnarray*}
 f'(x) &=& (-1)\cdot (-1)\cdot (1-x)^{-2}=(1-x)^{-2}, \\
f''(x) &=&  (-1)\cdot (-2)\cdot (1-x)^{-3}=2 (1-x)^{-3} ,\\
f'''(x) &=& 2\cdot (-1)\cdot (-3)\cdot (1-x)^{-4}=3!\cdot  (1-x)^{-4} ,
\end{eqnarray*}
und induktiv dann allgemein
\[  f^{(k)}(x)= k! (1-x)^{-k-1}=\frac{k!}{(1-x)^{k+1}} .\]
Somit ist $f^{(k)}(0)=\frac{k!}{(1-0)^{k+1}}=k!$ und das Taylor-Polynom $n$-ter Ordnung zum Entwicklungspunkt $x_0=0$ ist
\[ T_n(x)=\sum_{k=0}^n \frac{k!}{k!}x^k=\sum_{k=0}^n x^k .\]
Für $n \to \infty$ erhält man hier übrigens die geometrische Reihe. }
\lang{en}{
Consider the function $f:\R\setminus\{1\}\to\R$, $f(x)=\frac{1}{1-x}$ and the base point $x_0=0$.
Writing $f(x)=(1-x)^{-1}$ and using the chain rule yields the first few derivatives,
\begin{eqnarray*}
 f'(x) &=& (-1)\cdot (-1)\cdot (1-x)^{-2}=(1-x)^{-2}, \\
f''(x) &=&  (-1)\cdot (-2)\cdot (1-x)^{-3}=2 (1-x)^{-3} ,\\
f'''(x) &=& 2\cdot (-1)\cdot (-3)\cdot (1-x)^{-4}=3!\cdot  (1-x)^{-4} ,
\end{eqnarray*}
and inductively, the derivative of general order,
\[  f^{(k)}(x)= k! (1-x)^{-k-1}=\frac{k!}{(1-x)^{k+1}} .\]
Therefore, $f^{(k)}(0)=\frac{k!}{(1-0)^{k+1}}=k!$, and the Taylor polynomial
of order $n$ at the base point $x_0 = 0$ is
\[ T_n(x)=\sum_{k=0}^n \frac{k!}{k!}x^k=\sum_{k=0}^n x^k .\]
In the limit $n \to \infty$ we obtain the geometric series.
}
\end{tabs*}
\end{example}

\begin{quickcheck}
		\field{rational}
		\type{input.function}
		\begin{variables}
			\function{b}{6}
		    \function{f}{1/(x-5)}
            \function{df1}{-1/(x-5)^2}
            \function[normalize]{df2}{2/(x-5)^3}
            \function{fb}{1}
            \function{df1b}{-1}
            \function{df2b}{2}
            \function{p0}{1}
            \function{p1}{1+(x-1)}
            \function{p2}{1+(x-1)+(x-1)^2}
		\end{variables}
		\lang{de}{
		\text{Bestimmen Sie die Taylor-Polynome nullter, erster und zweiter Ordnung der Funktion $f(x)=\var{f}$ um die Stelle $x_0=\var{b}$.\\ 
        Die Ableitungen der Funktion $f(x)=\var{f}$ sind\\ $f'(x)= $\ansref und $\quad f''(x)= $\ansref. \\
        Daraus ergeben sich die Werte $f(\var{b})=$\ansref, $f'(\var{b})=$\ansref und $\quad f''(\var{b})=$\ansref.\\
        Die Taylor-Polynome sind daher\\
        $T_0(x)=$ \ansref,\\                
        $T_1(x)=$ \ansref,\\                
        $T_2(x)=$ \ansref.                  
        }}
    \lang{en}{
    \text{Determine the Taylor polynomials of order $0$, $1$ and $2$
        for the function $f(x)=\var{f}$ at the base point $x_0=\var{b}$.\\
        The derivatives of the function $f(x)=\var{f}$ are\\ $f'(x)= $\ansref and $\quad f''(x)= $\ansref. \\
        This leads to the values $f(\var{b})=$\ansref, $f'(\var{b})=$\ansref and $\quad f''(\var{b})=$\ansref.\\
        The Taylor polynomials are therefore\\
        $T_0(x)=$ \ansref,\\                
        $T_1(x)=$ \ansref,\\                
        $T_2(x)=$ \ansref.   
        }
    }
		
		\begin{answer}
			\solution{df1}
			\checkAsFunction{x}{-1}{1}{10}
		\end{answer}
		\begin{answer}
			\solution{df2}
			\checkAsFunction{x}{-1}{1}{10}
		\end{answer}
		\begin{answer}
			\solution{fb}
			\checkAsFunction{x}{-1}{1}{10}
		\end{answer}
		\begin{answer}
			\solution{df1b}
			\checkAsFunction{x}{-1}{1}{10}
		\end{answer}
        \begin{answer}
			\solution{df2b}
			\checkAsFunction{x}{-1}{1}{10}
		\end{answer}
		\begin{answer}
			\solution{p0}
			\checkAsFunction{x}{-1}{1}{10}
		\end{answer}
		\begin{answer}
			\solution{p1}
			\checkAsFunction{x}{-1}{1}{10}
		\end{answer}
		\begin{answer}
			\solution{p2}
			\checkAsFunction{x}{-1}{1}{10}
		\end{answer}
    \lang{de}{
		\explanation{Benutzen Sie  die Formel für die Taylor-Polynome.\\
               }}
    \lang{en}{
    \explanation{Use the formula for Taylor polynomials.\\ }
    }
\end{quickcheck}

\lang{de}{Zum Abschluss behandeln wir die \emph{Regeln von de L'Hospital}, mit deren Hilfe Grenzwerte von Brüchen  
 $\lim_{x\to x_0}\;\frac{f(x)}{g(x)}$ berechnet werden, bei denen Zähler und Nenner beide gegen $0$ konvergieren oder beide 
 unbeschränkt wachsen (\glqq gegen $\infty$ gehen\grqq ). Ausdrücke wie $\;$ \glqq $\frac{0}{0}$\grqq{} $\,$ oder $\,$ \glqq $\frac{\infty}{\infty}$\grqq{}
 $\,$ sind nicht definiert, daher werden wir versuchen diese umzuformen. }
\lang{en}{
Finally, we will discuss \emph{L'Hôpital's rule}, which is used to calculate
limits of quotients of the form $\lim_{x\to x_0}\;\frac{f(x)}{g(x)}$ in which the
numerator and denominator both tend to $0$ or both grow unboundedly
(or "tend to $\infty$"). Expressions such as $\;$ "$\frac{0}{0}$" $\,$ or $\,$  "$\frac{\infty}{\infty}$"
 $\,$ are undefined so we will try to rearrange them.
}
 
\begin{theorem}[
\lang{de}{Regeln von de L'Hospital}
\lang{en}{L'Hôpital's rule}]\label{thm:lhospital}
\lang{de}{Es seien $f$ und $g$ zwei Funktionen, die auf einem offenen Intervall $I = (a;b)$ differenzierbar sind, und es gelte $g'(x) \neq 0$ für alle $x \in I$.
Weiter sei $x_0 \in I$ oder $x_0 = a$ oder $x_0 = b$. }
\lang{en}{
Let $f$ and $g$ be two functions that are differentiable on an open interval
$I = (a, b)$, and suppose $g'(x)\ne 0$ for all $x \in I$.
Let $x_0 \in I$ or $x_0 = a$ or $x_0 = b$.
}

\lang{de}{Außerdem sei \notion{eine} der folgenden beiden Bedingungen erfüllt:
\begin{itemize}
\item $ \lim_{x \to x_0} \, f(x) =  \lim_{x \to x_0} \, g(x) = 0,$
\item $  \lim_{x \to x_0} \, f(x) = \pm \infty \ \  \text{sowie} \ \lim_{x \to x_0} \, g(x) = \pm \infty.$
\end{itemize}
}
\lang{en}{
In addition, suppose that \notion{one} of the conditions below is satisfied:
\begin{itemize}
\item $ \lim_{x \to x_0} \, f(x) =  \lim_{x \to x_0} \, g(x) = 0,$
\item $  \lim_{x \to x_0} \, f(x) = \pm \infty \ \  \text{and} \ \lim_{x \to x_0} \, g(x) = \pm \infty.$
\end{itemize}
}
\lang{de}{
Unter diesen Voraussetzungen gilt dann 
\[
\lim_{x \to x_0} \, \frac{f'(x)}{g'(x)} \in \R \quad \Longrightarrow \quad \lim_{x \to x_0} \,  \frac{f(x)}{g(x)} =  \lim_{x \to x_0} \, \frac{f'(x)}{g'(x)}
\]
oder anders ausgedrückt: Wenn der Grenzwert für $\frac{f'(x)}{g'(x)}$ existiert, dann existiert er auch für $\frac{f(x)}{g(x)}$ und beide Grenzwerte stimmen
überein.}
\lang{en}{
Under these assumptions,
\[
\lim_{x \to x_0} \, \frac{f'(x)}{g'(x)} \in \R \quad \Longrightarrow \quad \lim_{x \to x_0} \,  \frac{f(x)}{g(x)} =  \lim_{x \to x_0} \, \frac{f'(x)}{g'(x)}.
\]
In other words: if the limit of $\frac{f'(x)}{g'(x)}$ exists, then the limit
of $\frac{f(x)}{g(x)}$ also exists and the two limits coincide.
}
\end{theorem}

\lang{de}{Es ist erlaubt, $a = -\infty$ und/oder $b = \infty$ zu wählen. }
\lang{en}{
It is possible to choose $a = -\infty$ and/or $b = \infty$.
}
\begin{proof*}
\lang{de}{Eine ausführliche Besprechung der Regeln von de L'Hospital findet sich im \ref[content_06_de_l_hospital][gleichnamigen Kapitel]{} des Hauptkurses. 
Der Beweis benutzt eine Variante des Mittelwertsatzes. Es ist auch möglich, die Regeln aus der 
Taylor-Entwicklung herzuleiten.}
\lang{en}{
A detailed discussion of L'Hôpital's rule can be found in the main HM4Mint course
\ref[content_06_de_l_hospital][in the chapter of the same name]{}.
The proof uses a variant of the mean value theorem. It is also possible
to prove L'Hôpital's rule using Taylor expansions.
}
\end{proof*}
\lang{de}{Ergibt $\lim_{x\to x_0}\;\frac{f'(x)}{g'(x)}$ wieder einen Ausdruck wie $\;$ \glqq$0/0$\grqq\ $\,$ oder $\,$ \glqq$\infty/\infty$\grqq\ $\,$,
so kann man oft die Regel von de L'Hospital erneut (oder sogar mehrfach) anwenden und erhält eventuell einen Grenzwert
$\lim_{x\to x_0}\;\frac{f''(x)}{g''(x)}$. Jeder Schritt erhält dann erst nachträglich seine Legitimation, wenn im letzten Schritt die Existenz des Grenzwerts gezeigt wurde.
Dabei ist es wichtig, in jedem Schritt die Voraussetzungen neu zu prüfen. Einem solchen Fall begegnen wir auch in den folgenden Beispielen. 
}
\lang{en}{
If $\lim_{x\to x_0}\;\frac{f'(x)}{g'(x)}$ also leads to an expression
such as  $\;$ "$0/0$" $\,$ or $\,$ "$\infty/\infty$" $\,$
then it is often possible to apply L'Hôpital's rule again (even multiple times)
to potentially obtain the limit $\lim_{x\to x_0}\;\frac{f''(x)}{g''(x)}$.
These steps are only justified if the existence of the limit can be shown in the last step.
It is important to check that the conditions are satisfied in each step.
We will see a case like this in the examples below.
}

\begin{example}
\begin{tabs*}[\initialtab{0}]
\tab{
\lang{de}{\glqq $\frac{0}{0}$ \grqq{}}
\lang{en}{"$\frac{0}{0}$"}
}
\lang{de}{Um den Grenzwert $\lim_{x\to 0} \;\frac{\tan(x)}{x} $ zu berechnen, bemerken wir zum einen
 $\lim_{x\to 0} \tan(x)=0=\lim_{x\to 0} x$ und zum anderen, dass $f(x)=\tan(x)$ und $g(x)=x$ differenzierbar sind
 mit $f'(x)=1+\tan(x)^2$ und $g'(x)=1\neq 0$. Also können wir die Regel von de L'Hospital anwenden und erhalten
\[ \lim_{x\to 0} \;\frac{\tan(x)}{x} =\lim_{x\to 0} \frac{1+\tan(x)^2}{1} =1. \]}
\lang{en}{
To compute the limit $\lim_{x\to 0} \;\frac{\tan(x)}{x} $, we first observe
that $\lim_{x\to 0} \tan(x)=0=\lim_{x\to 0} x$ and that
$f(x)=\tan(x)$ and $g(x)=x$ are both differentiable
 with $f'(x)=1+\tan(x)^2$ and $g'(x)=1\neq 0$.
 Therefore, L'Hôpital's rule applies, and we find
 \[ \lim_{x\to 0} \;\frac{\tan(x)}{x} =\lim_{x\to 0} \frac{1+\tan(x)^2}{1} =1. \]
}


\tab{
\lang{de}{\glqq $\frac{\infty}{\infty}$\grqq{}}
\lang{en}{"$\frac{\infty}{\infty}$"}
}
\lang{de}{Berechnung von $\lim_{x\to \infty} \;\frac{\ln(x)}{x}$:\\
}
\lang{en}{
Computing $\lim_{x\to \infty} \;\frac{\ln(x)}{x}$:\\
}

\lang{de}{In diesem Fall gelten $\lim_{x\to \infty} \ln(x)=\infty$ und $\lim_{x\to \infty} x=\infty$ 
sowie $(\ln(x))'=\frac{1}{x}$. Der Nenner $x$ hat eine konstante Ableitung. Also können wir die
Regel von de L'Hospital anwenden und erhalten
  \[ \lim_{x\to \infty} \;\frac{\ln(x)}{x} =\lim_{x\to \infty} \;\frac{\frac{1}{x}}{1} =0.\]
}
\lang{en}{
In this case, $\lim_{x\to \infty} \ln(x)=\infty$ and $\lim_{x\to \infty} x=\infty$.
We have $(\ln(x))'=\frac{1}{x}$, and the denominator $x$ has a constant derivative.
 We can apply L'Hôpital's rule to find
 \[ \lim_{x\to \infty} \;\frac{\ln(x)}{x} =\lim_{x\to \infty} \;\frac{\frac{1}{x}}{1} =0.\]
}

\tab{
\lang{de}{\glqq $\frac{0}{0}$\grqq{} (mehrfach)}
\lang{en}{"$\frac{0}{0}$" (multiple)}
}
\lang{de}{Um den Grenzwert $\lim_{x\to\infty}\,\frac{x^n}{e^x}$ zu berechnen, muss man  die Regel von de L'Hospital $n$-mal anwenden.
}
\lang{en}{
To compute the limit $\lim_{x\to\infty}\,\frac{x^n}{e^x}$, we will need to apply
L'Hôpital's rule $n$ times.
}
\lang{de}{
Dazu überlegt man sich, dass $f(x)=x^n$ beliebig oft stetig differenzierbar ist mit $\lim_{x\to\infty}f^{(k)}(x)=\infty$ für $k<n$ 
und dass $g(x)=e^x$ ebenfalls beliebig oft stetig differenzierbar ist mit $g^{(k)}(x)=g(x)\neq 0$ und $\lim_{x\to\infty}g^{(k)}(x)=\infty$.
}
\lang{en}{
First, observe that $f(x)=x^n$ is infinitely differentible 
with $\lim_{x\to\infty}f^{(k)}(x)=\infty$ for $k<n$,
and that $g(x)=e^x$ is also infinitely differentiable with
$g^{(k)}(x)=g(x)\neq 0$ and $\lim_{x\to\infty}g^{(k)}(x)=\infty$.
}
\lang{de}{
Also gilt
\[ \lim_{x\to\infty}\,\frac{x^n}{e^x} =\lim_{x\to\infty}\,\frac{nx^{n-1}}{e^x}=\ldots =\lim_{x\to\infty}\,\frac{n! x^0}{e^x}=0.\]
}
\lang{en}{
Therefore,
\[ \lim_{x\to\infty}\,\frac{x^n}{e^x} =\lim_{x\to\infty}\,\frac{nx^{n-1}}{e^x}=\ldots =\lim_{x\to\infty}\,\frac{n! x^0}{e^x}=0.\]
}
\lang{de}{
Dies zeigt, dass die Exponentialfunktion schneller wächst als jede Potenz von $x$.}
\lang{en}{
This shows that the exponential function grows faster than any power of $x$.
}
\tab{ 
\lang{de}{\glqq $0\cdot\infty$\grqq{}}
\lang{en}{"$0 \cdot \infty$"}
}
\lang{de}{Wir wollen nun den Grenzwert $\lim_{x \to \infty} x\cdot e^{-x}$ ermitteln. }
\lang{en}{
We would now like to determine the limit $\lim_{x \to \infty} x\cdot e^{-x}$.
}
\lang{de}{Hier schreibt man den Term besser um zu
\[   x\cdot e^{-x}=\frac{x}{e^x} \]
und berechnet
\[ \lim_{x\to \infty} x\cdot e^{-x}=\lim_{x\to \infty} \frac{x}{e^x} =\lim_{x\to \infty} \frac{1}{e^x} = 0.\]
}
\lang{en}{
It is better to rewrite this expression in the form
\[   x\cdot e^{-x}=\frac{x}{e^x},\]
so that
\[ \lim_{x\to \infty} x\cdot e^{-x}=\lim_{x\to \infty} \frac{x}{e^x} =\lim_{x\to \infty} \frac{1}{e^x} = 0.\]
}
\lang{de}{Würde man jedoch in der Absicht, de L'Hospital anzuwenden, den Term  als $\frac{e^{-x}}{x^{-1}}$ schreiben, 
 erhielte man den Quotienten der Ableitung als  $\frac{-e^{-x}}{-x^{-2}}$. 
 Dessen Grenzwert kann wieder nicht direkt berechnet werden.}
\lang{en}{
If we instead write the expression as $\frac{e^{-x}}{x^{-1}}$ and attempt to apply
L'Hôpital's rule, however, then we find that the quotient of the derivatives is $\frac{-e^{-x}}{-x^{-2}}$.
This is another limit that cannot be computed directly.
}

 
\tab{
\lang{de}{"`$\infty - \infty$"'}
\lang{en}{"$\infty - \infty$"}
}
\lang{de}{Würde man den Grenzwert $\lim_{x\to 0} \;\left(\frac{1}{e^x-1} - \frac{1}{x}\right) $ summandenweise berechnen wollen, erhielte
man einen Ausdruck "`$\infty - \infty$"', welcher nicht definiert ist. }
\lang{en}{
If we try to compute the limit $\lim_{x\to 0} \;\left(\frac{1}{e^x-1} - \frac{1}{x}\right) $ termwise
then we obtain the expression $\infty - \infty$, which is undefined.
}

\lang{de}{Auch in diesem Fall formt man den Term um, um als Grenzwert einen Ausdruck der Form
\glqq$0/0$\grqq\ oder \glqq$\infty/\infty$\grqq\ zu bekommen, auf welchen man wieder die Regel von de L'Hospital anwenden kann (hier am Ende
sogar zweimal):
  \[ \lim_{x\to 0} \;\left(\frac{1}{e^x-1} - \frac{1}{x}\right) =  \lim_{x\to 0} \frac{x-(e^x-1)}{x\,e^x - x}
  =  \lim_{x\to 0} \frac{1-e^x}{e^x+x\,e^x -1} = \lim_{x\to 0} \;\frac{-e^x}{2e^x+x\,e^x} =\frac{-1}{2}.
  \]
}
\lang{en}{
We rewrite the expression again to obtain something of the form "$0/0$" or "$\infty/\infty$"
in the limit, such that L'Hôpital's rule can be applied (in fact, we use it twice):
\[ \lim_{x\to 0} \;\left(\frac{1}{e^x-1} - \frac{1}{x}\right) =  \lim_{x\to 0} \frac{x-(e^x-1)}{x\,e^x - x}
  =  \lim_{x\to 0} \frac{1-e^x}{e^x+x\,e^x -1} = \lim_{x\to 0} \;\frac{-e^x}{2e^x+x\,e^x} =\frac{-1}{2}.
  \]
}
  
\tab{
\lang{de}{\glqq $1^\infty$\grqq{}}
\lang{en}{"$1^\infty$"}
}

\lang{de}{
 Auch wenn Potenzen vorkommen, helfen die Regeln von L'Hospital oft, um die Grenzwerte zu berechnen. 
Es gibt hier im Wesentlichen drei Fälle, nämlich Grenzwertausdrücke der Form 
 $\,$ \glqq$1^\infty$\grqq, \glqq$0^0$\grqq{} oder \glqq$\infty^0$\grqq.
 \\
}
\lang{en}{
Even when exponents are involved, L'Hôpital's rule can often be used to
compute limits. There are essentially three cases: limit expressions of the form
"$1^\infty$", "$0^0$" and "$\infty^0$".
\\
}
  \lang{de}{Bei dem Grenzwert $ \lim_{x\to \infty} \;\left(1+\frac{1}{x}\right)^x$ würde man einen Ausdruck \glqq$1^\infty$\grqq{} bekommen,
  wenn man die Grenzwerte von Basis und Exponent separat berechnet. }
  \lang{en}{
  The limit $ \lim_{x\to \infty} \;\left(1+\frac{1}{x}\right)^x$ would lead to
  the expression "$1^\infty$" if the limits of the base and exponent were
  computed separately.
  }
    
  \lang{de}{In solchen Fällen berechnen wir zunächst den Logarithmus des Terms, also
  \[ \ln\left( \big(1+\frac{1}{x}\big)^x\right) =x\cdot \ln \big(1+\frac{1}{x}\big) = \frac{\ln \big(1+\frac{1}{x}\big)}{1/x}, \]
  und dessen Grenzwert mit der Regel von de L'Hospital (Voraussetzungen prüfen!):
   \[  \lim_{x\to \infty}\frac{\ln \big(1+\frac{1}{x}\big)}{1/x} = \lim_{x\to \infty}\frac{-\frac{1}{x^2}\cdot \frac{1}{1+1/x}}{-\frac{1}{x^2}} = \lim_{x\to \infty} \frac{1}{1+1/x} =1.\]
  }
  \lang{en}{
  In these cases, we first take the logarithm of the expression:
  \[ \ln\left( \big(1+\frac{1}{x}\big)^x\right) =x\cdot \ln \big(1+\frac{1}{x}\big) = \frac{\ln \big(1+\frac{1}{x}\big)}{1/x}, \]
  and compute its limit using L'Hôpital's rule (checking that the conditions hold!):
  \[  \lim_{x\to \infty}\frac{\ln \big(1+\frac{1}{x}\big)}{1/x} = \lim_{x\to \infty}\frac{-\frac{1}{x^2}\cdot \frac{1}{1+1/x}}{-\frac{1}{x^2}} = \lim_{x\to \infty} \frac{1}{1+1/x} =1.\]
 
  }
	\lang{de}{Da die Exponentialfunktion stetig ist, gilt dann
	\[    \lim_{x\to \infty} \;\left(1+\frac{1}{x}\right)^x= \lim_{x\to \infty} \exp\left(\ln\left( \big(1+\frac{1}{x}\big)^x\right) \right)
	=  \exp\left( \lim_{x\to \infty}  \ln\left( \big(1+\frac{1}{x}\big)^x\right) \right)=\exp(1)=e.\]	
  }
  \lang{en}{
  Since the exponential function is continuous, we have
  \[    \lim_{x\to \infty} \;\left(1+\frac{1}{x}\right)^x= \lim_{x\to \infty} \exp\left(\ln\left( \big(1+\frac{1}{x}\big)^x\right) \right)
	=  \exp\left( \lim_{x\to \infty}  \ln\left( \big(1+\frac{1}{x}\big)^x\right) \right)=\exp(1)=e.\]	
  }
\end{tabs*}
\end{example}
\begin{quickcheck}
\lang{de}{
\text{Welche der folgenden Grenzwertprobleme bieten das Potential, mit der Regel von de l'Hospital berechnet zu werden?}
}
\lang{en}{
\text{Which of the following limit problems could potentially be solved with L'Hôpital's rule?}
}
\begin{choices}{multiple}

        \begin{choice}
            \text{$\lim_{x\to 0}\frac{x^4}{\cos x}$}
			\solution{false}
         \end{choice}
                    
        \begin{choice}
            \text{$\lim_{x\to \infty}\frac{e^x}{e^x-x}$}
			\solution{true}
		\end{choice} 
        \begin{choice}
            \text{$\lim_{x\to \infty}\frac{x^3}{\cos x}$}
			\solution{false}
        \end{choice} 
        \begin{choice}
            \text{$\lim_{x\to 0}\frac{x^3}{\sin x}$}
			\solution{true}
		\end{choice}
        \begin{choice}
            \text{$\lim_{x\to 0}\frac{\ln \frac{1}{x}}{\frac{1}{x}}$}
			\solution{true}
		\end{choice}
\end{choices}
\lang{de}{
\explanation{Den ersten Grenzwert berechnet man direkt $\lim_{x\to 0}\frac{x^4}{\cos x}=\frac{0}{1}=1$.\\
Bei $\lim_{x\to \infty}\frac{e^x}{e^x-x}$  liegt der Fall \glqq$\frac{\infty}{\infty}$\grqq{} vor.\\
Weil der Grenzwert $\lim_{x\to\infty}\cos x$ nicht exisitiert, kann man für $\lim_{x\to \infty}\frac{x^3}{\cos x}$
keine de L'Hospitalsche Regel anwenden.\\
Bei $\lim_{x\to 0}\frac{x^3}{\sin x}$ liegt der Fall \glqq$\frac{0}{0}$\grqq{} vor, und bei
$\lim_{x\to 0}\frac{\ln \frac{1}{x}}{\frac{1}{x}}$ der Fall \glqq$(-1)\cdot\frac{\infty}{\infty}$\grqq{}.}
}
\lang{en}{
\explanation{The first limit can be calculated directly: $\lim_{x\to 0}\frac{x^4}{\cos x}=\frac{0}{1}=1$.\\
$\lim_{x\to \infty}\frac{e^x}{e^x-x}$ is of the form "$\frac{\infty}{\infty}$". \\
Since the limit $\lim_{x\to\infty}\cos x$ does not exist, L'Hôpital's rule
cannot be applied to $\lim_{x\to \infty}\frac{x^3}{\cos x}$.\\
$\lim_{x\to 0}\frac{x^3}{\sin x}$ is of the form "$\frac{0}{0}$", and
$\lim_{x\to 0}\frac{\ln \frac{1}{x}}{\frac{1}{x}}$ is of the form "$(-1)\cdot\frac{\infty}{\infty}$".
}
}
\end{quickcheck}

 
\end{content}