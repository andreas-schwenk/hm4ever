\documentclass{mumie.element.exercise}
%$Id$
\begin{metainfo}
  \name{
    \lang{de}{Ü05: L'Hospital}
    \lang{en}{Exercise 5: L'Hôpital's rule}
  }
  \begin{description} 
 This work is licensed under the Creative Commons License Attribution 4.0 International (CC-BY 4.0)   
 https://creativecommons.org/licenses/by/4.0/legalcode 

    \lang{de}{}
    \lang{en}{}
  \end{description}
  \begin{components}
  \end{components}
  \begin{links}
  \end{links}
  \creategeneric
\end{metainfo}
\begin{content}
\title{\lang{de}{Ü05: L'Hospital}
    \lang{en}{Exercise 5: L'Hôpital's rule}}
\begin{block}[annotation]
	Im Ticket-System: \href{https://team.mumie.net/issues/24026}{Ticket 24026}
\end{block}
\usepackage{mumie.ombplus}



\begin{block}[annotation]
Kopie: hm4mint/T303_Approximationen/exercise 6

Im Ticket-System: \href{http://team.mumie.net/issues/10497}{Ticket 10497}
\end{block}

%######################################################FRAGE_TEXT
\lang{de}{ 
Bestimmen Sie mit Hilfe der Regel von de l'Hospital die folgenden Grenzwerte:}
\lang{en}{Use L'Hôpital's rule to compute the following limits:}
\begin{enumerate}[a)]
\item a) $\lim_{x\to 2} \frac{x^2-4x+4}{x^4-16}$
\item b) $\lim_{x\to 0} \frac{e^x-x-1}{\cos(x)-1}$
\item c) $\lim_{x\nearrow 1} \frac{-\ln(1-x)}{\tan(\frac{\pi}{2}x)}$
\end{enumerate}


%##################################################ANTWORTEN_TEXT
\begin{tabs*}[\initialtab{0}\class{exercise}]

\tab{\lang{de}{   Antwort    } \lang{en}{Answer}}
  \begin{incremental}[\initialsteps{1}]
  
  	 %----------------------------------START_STEP_X
    \step 
    \lang{de}{ Die Grenzwerte sind:}
    \lang{en}{The limits are:}
\begin{enumerate}[a)]
\item a) $\lim_{x\to 2} \frac{x^2-4x+4}{x^4-16}=0$
\item b) $\lim_{x\to 0} \frac{e^x-x-1}{\cos(x)-1}=-1$
\item c) $\lim_{x\nearrow 1} \frac{-\ln(1-x)}{\tan(\frac{\pi}{2}x)}=0$
\end{enumerate}

  	 %------------------------------------END_STEP_X
 
  \end{incremental}

  %++++++++++++++++++++++++++++++++++++++++++START_TAB_X
  \tab{\lang{de}{    Lösung a)    }  \lang{en}{Solution a)}}
  \begin{incremental}[\initialsteps{1}]
  
  	 %----------------------------------START_STEP_X
    \step 
    \lang{de}{   Zähler wie Nenner sind als Polynome beliebig oft differenzierbar.
    Die Grenzwerte des Zählers und des Nenners sind beide gleich $0$, da
\[ \lim_{x\to 2}{(x^2-4x+4)} =2^2-4\cdot 2+4=0 \]
und
\[ \lim_{x\to 2}{(x^4-16)}=2^4-16=0.\]
Also lässt sich die Regel von de l'Hospital anwenden, womit der gesuchte Grenzwert
gleich dem Grenzwert des Quotienten der Ableitungen ist. Wegen
$(x^2-4x+4)'=2x-4$ und $(x^4-16)'=4x^3$ ist also
\[\lim_{x\to 2} \frac{x^2-4x+4}{x^4-16}=\lim_{x\to 2} \frac{2x-4}{4x^3}=\frac{2\cdot 2-4}{4\cdot 2^3}
=\frac{0}{32}=0.\]

\textit{Bemerkung:} In diesem Beispiel wäre man auch ohne die Regel von de l'Hospital zum Ziel gekommen, wenn
man Zähler und Nenner jeweils durch den Linearfaktor $x-2$ geteilt hätte.
    }
\lang{en}{   The numerator and denominator are both polynomials and therefore infinitely differentiable.
    The limits of the numerator and denominator are both $0$, because
\[ \lim_{x\to 2}{(x^2-4x+4)} =2^2-4\cdot 2+4=0 \]
and
\[ \lim_{x\to 2}{(x^4-16)}=2^4-16=0.\]
Therefore we can use L'Hôpital's rule to write the original limit as the limit of
the quotient of the derivatives. Since
$(x^2-4x+4)'=2x-4$ and $(x^4-16)'=4x^3$, we find
\[\lim_{x\to 2} \frac{x^2-4x+4}{x^4-16}=\lim_{x\to 2} \frac{2x-4}{4x^3}=\frac{2\cdot 2-4}{4\cdot 2^3}
=\frac{0}{32}=0.\]

\textit{Remark:} In this example, we could have avoided the use of
L'Hôpital's rule by dividing the numerator and the denominator
by the linear factor $x-2$.}

    
  	 %------------------------------------END_STEP_X
 
  \end{incremental}
  %++++++++++++++++++++++++++++++++++++++++++++END_TAB_X

  %++++++++++++++++++++++++++++++++++++++++++START_TAB_X
  \tab{\lang{de}{    Lösung b)    } \lang{en}{Solution b)}}
  \begin{incremental}[\initialsteps{1}]
  
  	 %----------------------------------START_STEP_X
    \step 
    \lang{de}{  Die Grenzwerte des Zählers und des Nenners sind beide gleich $0$, da
\[ \lim_{x\to 0}{(e^x-x-1)}=1-0-1=0\quad \text{und}\quad \lim_{x\to 0} (\cos(x)-1)=1-1=0.\]
Der Grenzwert des Bruchs lässt sich also nicht direkt bestimmen. Um die Regel von de l'Hospital
anzuwenden, müssen zunächst die Ableitungen der beliebig oft differenzierbaren Funktionen bestimmt werden:
\[ (e^x-x-1)'=e^x-1\quad \text{und}\quad (\cos(x)-1)'=-\sin(x). \]
Nun sind aber
\[ \lim_{x\to 0}{(e^x-1)}=1-1=0\quad \text{und}\quad \lim_{x\to 0} (-\sin(x))=0,\]
weshalb man erneut die Ableitungen bestimmt:
\[ (e^x-1)'=e^x\quad \text{und}\quad (-\sin(x))'=-\cos(x). \]
Dann gilt
\[ \lim_{x\to 0} \frac{e^x}{-\cos(x)}=\frac{e^0}{-\cos(0)}=\frac{1}{-1}=-1.\]
Nach der Regel von de l'Hospital ist also
\[ \lim_{x\to 0} \frac{e^x-x-1}{\cos(x)-1}=\lim_{x\to 0} \frac{e^x-1}{-\sin(x)}
=\lim_{x\to 0} \frac{e^x}{-\cos(x)}=-1.\]
 }
 \lang{en}{  The limits of the numerator and the denominator are both $0$, because
\[ \lim_{x\to 0}{(e^x-x-1)}=1-0-1=0\quad \text{and}\quad \lim_{x\to 0} (\cos(x)-1)=1-1=0.\]
The limit of the quotient cannot be determined directly. To apply L'Hôpital's rule,
we first determine the derivatives of the infinitely differentiable functions:
\[ (e^x-x-1)'=e^x-1\quad \text{and}\quad (\cos(x)-1)'=-\sin(x). \]
Here,
\[ \lim_{x\to 0}{(e^x-1)}=1-1=0\quad \text{and}\quad \lim_{x\to 0} (-\sin(x))=0,\]
so we compute derivatives again:
\[ (e^x-1)'=e^x\quad \text{and}\quad (-\sin(x))'=-\cos(x). \]
Now
\[ \lim_{x\to 0} \frac{e^x}{-\cos(x)}=\frac{e^0}{-\cos(0)}=\frac{1}{-1}=-1.\]
By L'Hôpital's rule,
\[ \lim_{x\to 0} \frac{e^x-x-1}{\cos(x)-1}=\lim_{x\to 0} \frac{e^x-1}{-\sin(x)}
=\lim_{x\to 0} \frac{e^x}{-\cos(x)}=-1.\]
 }
  	 %------------------------------------END_STEP_X
 
  \end{incremental}
  %++++++++++++++++++++++++++++++++++++++++++++END_TAB_X

  %++++++++++++++++++++++++++++++++++++++++++START_TAB_X
  \tab{\lang{de}{    Lösung c)    } \lang{en}{Solution c)}}
  \begin{incremental}[\initialsteps{1}]
  
  	 %----------------------------------START_STEP_X
    \step 
    \lang{de}{ Wegen $\lim_{x\nearrow 1} (-\ln(1-x))=\lim_{y\searrow 0} (-\ln(y))=\infty$ und 
$\lim_{x\nearrow 1} \tan(\frac{\pi}{2}x)=\infty$ lässt sich die Regel von de l'Hospital anwenden,
denn beide Funktionen sind beliebig oft differenzierbar.

Wir berechnen also zunächst die Ableitungen des Zählers $f(x)=-\ln(1-x)$ und des Nenners $g(x)=\tan(\frac{\pi}{2}x)$.
\[ f'(x)=-\frac{1}{1-x}\cdot (-1)=\frac{1}{1-x}, \]
nach der Kettenregel, und ebenfalls mit der Kettenregel und der Formel für die Ableitung des Tangens ist
\[ g'(x)=(1/\cos(\frac{\pi}{2}x)^2)\cdot \frac{\pi}{2}.\]

Also ist 
\[ \frac{f'(x)}{g'(x)}=\frac{1/(1-x)}{\frac{\pi}{2}/\cos(\frac{\pi}{2}x)^2}
=\frac{\cos(\frac{\pi}{2}x)^2}{\frac{\pi}{2}(1-x)}. \]

Wegen $\lim_{x\nearrow 1} \cos(\frac{\pi}{2}x)^2=0$ und $\lim_{x\nearrow 1} \frac{\pi}{2}(1-x)=0$,
lässt sich der Grenzwert $\lim_{x\nearrow 1} \frac{f'(x)}{g'(x)}$ auch nicht direkt berechnen, und wir bestimmen
wieder die Ableitungen von $f'(x)=\cos(\frac{\pi}{2}x)^2$ und $g'(x)=\frac{\pi}{2}(1-x)$.
\[ f''(x)=2 \cos(\frac{\pi}{2}x)\cdot (-\sin(\frac{\pi}{2}x))\cdot \frac{\pi}{2}\]
und
\[ g''(x)=-\frac{\pi}{2}. \]
Damit gilt also
\begin{eqnarray*}
 \lim_{x\nearrow 1} \frac{-\ln(1-x)}{\tan(\frac{\pi}{2}x)} &=& \lim_{x\nearrow 1} \frac{f'(x)}{g'(x)}
=\lim_{x\nearrow 1} \frac{\cos(\frac{\pi}{2}x)^2}{\frac{\pi}{2}(1-x)}\\
&=&\lim_{x\nearrow 1} \frac{f''(x)}{g''(x)}
=\lim_{x\nearrow 1} \frac{2 \cos(\frac{\pi}{2}x)\cdot (-\sin(\frac{\pi}{2}x))\cdot \frac{\pi}{2}}{-\frac{\pi}{2}}\\
&=& \lim_{x\nearrow 1} 2 \cos(\frac{\pi}{2}x)\cdot \sin(\frac{\pi}{2}x)=2\cdot 0\cdot 1=0.
\end{eqnarray*}
 }
 \lang{en}{ Since $\lim_{x\nearrow 1} (-\ln(1-x))=\lim_{y\searrow 0} (-\ln(y))=\infty$ and 
$\lim_{x\nearrow 1} \tan(\frac{\pi}{2}x)=\infty$ and both functions are differentiable infinitely often,
we can apply L'Hôpital's rule. 

First we find the derivatives of the numerator $f(x)=-\ln(1-x)$ and the denominator $g(x)=\tan(\frac{\pi}{2}x)$.
\[ f'(x)=-\frac{1}{1-x}\cdot (-1)=\frac{1}{1-x}, \]
by the chain rule, and using the chain rule and the formula for the derivative of $\tan$,
\[ g'(x)=(1/\cos(\frac{\pi}{2}x)^2)\cdot \frac{\pi}{2}.\]

Therefore,
\[ \frac{f'(x)}{g'(x)}=\frac{1/(1-x)}{\frac{\pi}{2}/\cos(\frac{\pi}{2}x)^2}
=\frac{\cos(\frac{\pi}{2}x)^2}{\frac{\pi}{2}(1-x)}. \]

Since $\lim_{x\nearrow 1} \cos(\frac{\pi}{2}x)^2=0$ and $\lim_{x\nearrow 1} \frac{\pi}{2}(1-x)=0$,
the limit $\lim_{x\nearrow 1} \frac{f'(x)}{g'(x)}$ cannot be computed directly, and we have to find the derivatives
of $f'(x)=\cos(\frac{\pi}{2}x)^2$ and $g'(x)=\frac{\pi}{2}(1-x)$.
\[ f''(x)=2 \cos(\frac{\pi}{2}x)\cdot (-\sin(\frac{\pi}{2}x))\cdot \frac{\pi}{2}\]
und
\[ g''(x)=-\frac{\pi}{2}. \]
Altogether,
\begin{eqnarray*}
 \lim_{x\nearrow 1} \frac{-\ln(1-x)}{\tan(\frac{\pi}{2}x)} &=& \lim_{x\nearrow 1} \frac{f'(x)}{g'(x)}
=\lim_{x\nearrow 1} \frac{\cos(\frac{\pi}{2}x)^2}{\frac{\pi}{2}(1-x)}\\
&=&\lim_{x\nearrow 1} \frac{f''(x)}{g''(x)}
=\lim_{x\nearrow 1} \frac{2 \cos(\frac{\pi}{2}x)\cdot (-\sin(\frac{\pi}{2}x))\cdot \frac{\pi}{2}}{-\frac{\pi}{2}}\\
&=& \lim_{x\nearrow 1} 2 \cos(\frac{\pi}{2}x)\cdot \sin(\frac{\pi}{2}x)=2\cdot 0\cdot 1=0.
\end{eqnarray*}
 }
  	 %------------------------------------END_STEP_X
 
  \end{incremental}
  %++++++++++++++++++++++++++++++++++++++++++++END_TAB_X

%#############################################################ENDE

 \tab{\lang{de}{Video: ähnliche Übungsaufgabe} \lang{en}{Video: similar exercise}}
  \youtubevideo[500][300]{DjTg8lCVUmI}\\

\end{tabs*}
\end{content}