\documentclass{mumie.element.exercise}
%$Id$
\begin{metainfo}
  \name{
    \lang{de}{Ü02: Krümmung}
    \lang{en}{Exercise 2: Curvature}
  }
  \begin{description} 
 This work is licensed under the Creative Commons License Attribution 4.0 International (CC-BY 4.0)   
 https://creativecommons.org/licenses/by/4.0/legalcode 

    \lang{de}{Hier die Beschreibung}
    \lang{en}{}
  \end{description}
  \begin{components}
  \end{components}
  \begin{links}
  \end{links}
  \creategeneric
\end{metainfo}
\begin{content}

\title{
  \lang{de}{Ü02: Krümmung}
  \lang{en}{Exercise 2: Curvature}
}

\begin{block}[annotation]
	Im Ticket-System: \href{https://team.mumie.net/issues/24023}{Ticket 24023}
\end{block}



\begin{block}[annotation]
Kopie: hm4mint/T106_Differentialrechnung/exercise 8

Im Ticket-System: \href{http://team.mumie.net/issues/9108}{Ticket 9108}
\end{block}

\lang{de}{Bestimmen Sie die zweite Ableitung der Funktion $f(x)=e^x(x^2-4x+5)$ und \"{u}berpr\"{u}fen Sie, wo $f''(x)$ positiv, negativ oder $0$ ist. Wo ist 
der Graph von $f$ rechtsgekr\"{u}mmt, wo ist er linksgekr\"{u}mmt?}
\lang{en}{Determine the second derivative of the function $f(x)=e^x(x^2-4x+5)$ and find where $f''(x)$ is positive, negative or $0$. Where is the graph of $f$ convex and where is it concave?}

\begin{tabs*}[\initialtab{0}\class{exercise}]
  \tab{
  \lang{de}{Antwort}
  \lang{en}{Answer}
  }
  \lang{de}{$f''(x)=e^x(x^2-1)$. $f''(x)=0$ f\"{u}r $x=1$ und $x=-1$. 
  Auf $(-1;1)$ ist $f''(x)$ negativ, auf $(-\infty;-1)$ und $(1;\infty)$ ist $f''(x)$ positiv. Also ist der Graph von $f$ auf $(-1;1)$ rechtsgekr\"{u}mmt 
  und auf $(-\infty;-1)$ sowie auf $(1;\infty)$ linksgekr\"{u}mmt.}
  \lang{en}{$f''(x)=e^x(x^2-1)$. $f''(x)=0$ for $x=1$ and $x=-1$.
  $f''(x)$ is negative in the interval $(-1,1)$ and positive in the intervals $(-\infty,-1)$ and $(1,\infty)$. So, the graph of $f$ is concave in $(-1,1)$ and convex in $(-\infty,-1)$ and $(1,\infty)$. }
 
  \tab{
  \lang{de}{L\"{o}sung}
  \lang{en}{Solution}
  }
  
  \begin{incremental}[\initialsteps{1}]
    
\step \lang{de}{Wir berechnen $f'(x)$ mit der Produktregel:}
\lang{en}{We calculate $f'(x)$ with the product rule.}
\begin{eqnarray*}
f'(x)&=&
e^x(x^2-4x+5)+e^x(2x-4)\\
&=&e^x(x^2-4x+5+2x-4)\\
&=&e^x(x^2-2x+1).
\end{eqnarray*}
\step \lang{de}{Wir berechnen $f''(x)$ als Ableitung von $f'(x)$ mit der Produktregel:}
\lang{en}{We calculate $f''(x)$ as the derivative of $f'(x)$ with the product rule.}
\begin{eqnarray*}
f''(x)&=&e^x(x^2-2x+1)+e^x(2x-2)\\
&=&e^x(x^2-2x+1+2x-2)\\
&=&e^x(x^2-1).
\end{eqnarray*} 
\step \lang{de}{Da $e^x>0$ ist f\"{u}r alle $x\in\R$, m\"{u}ssen wir nur den Term $x^2-1$ betrachten.   }
\lang{en}{Since $e^x>0$ for all $x\in\R$, we only need to consider the term $x^2-1$. }

\step \lang{de}{$x^2-1=0$, falls $x=-1$ oder $x=1$ ist. Also ist $f''(x)=0$ f\"{u}r $x=1$ und $x=-1$.}
\lang{en}{$x^2-1=0$ if $x=-1$ or $x=1$. Therefore, $f''(x)=0$ for $x=1$ and $x=-1$.}
 
  \step \lang{de}{$x^2-1>0$ f\"{u}r $x\in (-\infty;-1)$ oder $x\in (1;\infty)$. Also ist $f''(x)$ auf $(-\infty;-1)$ und $(1;\infty)$ positiv. 
  Damit ist der Graph von $f$ auf $(-\infty;-1)$ sowie auf $(1;\infty)$ linksgekr\"{u}mmt.}
  \lang{en}{$x^2-1>0$ for $x\in (-\infty,-1)$ or $x\in (1,\infty)$. Therefore, $f''(x)$ is positive in $(-\infty,-1)$ and $(1,\infty)$. Therefore, the graph of $f$ in $(-\infty,-1)$ and $(1,\infty)$ is convex.}
    
 \step \lang{de}{$x^2-1<0$ f\"{u}r $x\in (-1;1)$. Also ist $f''(x)$ auf $(-1;1)$ negativ, und der Graph von $f$ auf $(-1;1)$ rechtsgekr\"{u}mmt.}
 \lang{en}{$x^2-1<0$ for $x\in (-1,1)$. Therefore, $f''(x)$ is negative in $(-1,1)$ and the graph of $f$ in $(-1,1)$ is concave.}
   
  
  \end{incremental}


\end{tabs*}



\end{content}