\documentclass{mumie.element.exercise}
%$Id$
\begin{metainfo}
  \name{
    \lang{de}{Ü01: Monotonie}
    \lang{en}{Exercise 1: monotonicity}
  }
  \begin{description} 
 This work is licensed under the Creative Commons License Attribution 4.0 International (CC-BY 4.0)   
 https://creativecommons.org/licenses/by/4.0/legalcode 

    \lang{de}{Hier die Beschreibung}
    \lang{en}{}
  \end{description}
  \begin{components}
  \end{components}
  \begin{links}
  \end{links}
  \creategeneric
\end{metainfo}
\begin{content}
\title{\lang{de}{Ü01: Monotonie}
    \lang{en}{Exercise 1: monotonicity}}
\begin{block}[annotation]
	Im Ticket-System: \href{https://team.mumie.net/issues/24022}{Ticket 24022}
\end{block}



\begin{block}[annotation]
Kopie: hm4mint/T016_Differentialrechnung/exercise 7 (ohne stationäre Stelle)

Im Ticket-System: \href{http://team.mumie.net/issues/9106}{Ticket 9106}
\end{block}

\lang{de}{Gegeben sei die Funktion $f(x)=x\ln x$ f\"{u}r $x>0$. Wo steigt $f$ monoton, wo f\"{a}llt $f$ monoton?}

\lang{en}{Let $f(x)=x\ln x$ for $x>0$. Where is $f$ monotonically increasing, and where is $f$ monotonically decreasing?}

\begin{tabs*}[\initialtab{0}\class{exercise}]
  \tab{
  \lang{de}{Antwort}
  \lang{en}{Answer}
  }
\lang{de}{$f$ ist auf dem Intervall $(0;e^{-1})$ streng monoton fallend und auf dem Intervall 
$(e^{-1};\infty)$ streng monoton steigend.}
\lang{en}{$f$ is strictly decreasing in the interval $(0;e^{-1})$ and strictly increasing
in the interval $(e^{-1};\infty)$.}

  \tab{
  \lang{de}{L\"{o}sung}
  \lang{en}{Solution}
  }
  \begin{incremental}[\initialsteps{1}]
  \step \lang{de}{Wir bemerken, dass $f$ differenzierbar ist und wollen die Monotoniekriterien, die sich aus dem 
  Vorzeichen der Ableitung ergeben, anwenden. Wir berechnen $f'(x)$ mit der Produktregel,}
  \lang{en}{We observe that $f$ is differentiable and use the characterization of monotonicity
  in terms of the sign of the derivative. We compute $f'(x)$ using the product rule:}
   \step \[f'(x)=\ln x+x\frac{1}{x}=\ln x+1.\]
   \step \lang{de}{Da $\ln x+1<0$ f\"{u}r $x\in (0; e^{-1})$, ist $f$ dort monoton fallend, und sogar streng monoton fallend.}
   \lang{en}{Since $\ln x+1<0$ for $x \in (0; e^{-1})$, we see that $f$ is decreasing in that interval, and indeed is
   strictly monotonically decreasing.}
   
   \step \lang{de}{Da $\ln x+1>0$ f\"{u}r $x\in (e^{-1};\infty)$, ist $f$ dort monoton steigend, und sogar streng monoton steigend.}
   \lang{en}{Since $\ln x+1>0$ for $x \in (e^{-1};\infty)$, we see that $f$ is increasing in that interval, and indeed is
   strictly monotonically increasing.}

   
  
  \end{incremental}


\end{tabs*}

\end{content}