\documentclass{mumie.element.exercise}
%$Id$
\begin{metainfo}
  \name{
    \lang{de}{Ü03: n-te Ableitung}
    \lang{en}{}
  }
  \begin{description} 
 This work is licensed under the Creative Commons License Attribution 4.0 International (CC-BY 4.0)   
 https://creativecommons.org/licenses/by/4.0/legalcode 

    \lang{de}{}
    \lang{en}{}
  \end{description}
  \begin{components}
  \end{components}
  \begin{links}
  \end{links}
  \creategeneric
\end{metainfo}
\begin{content}
\title{\lang{de}{Ü03: n-te Ableitung}
    \lang{en}{Exercise 3: higher derivatives}}
\begin{block}[annotation]
	Im Ticket-System: \href{https://team.mumie.net/issues/24024}{Ticket 24024}
\end{block}
\usepackage{mumie.ombplus}


\begin{block}[annotation]
Kopie: hm4mint/T301_Differenzierbarkeit/exercise 8

Im Ticket-System: \href{http://team.mumie.net/issues/10281}{Ticket 10281}
\end{block}

%######################################################FRAGE_TEXT
\lang{de}{  
Bestimmen Sie die erste, zweite und dritte Ableitung der folgenden Funktionen. Versuchen Sie
auch, jeweils eine Formel für die $n$-te Ableitung zu finden.}
\lang{en}{
Find the first, second and third derivatives of the following functions. Try to find
a formula for the $n$-th derivative.
}

\[ \text{a) }  h(x)=xe^{-x}\qquad\qquad   \text{ b) } g(x)=\frac{1}{x} \]  

%##################################################ANTWORTEN_TEXT
\begin{tabs*}[\initialtab{0}\class{exercise}]

  %++++++++++++++++++++++++++++++++++++++++++START_TAB_X
  \tab{\lang{de}{    Lösung a)    } \lang{en}{Solution a)}}
  \begin{incremental}[\initialsteps{1}]
  
  	 %----------------------------------START_STEP_X
    \step 
    \lang{de}{   Mit der Kettenregel und Produktregel erhält man als Ableitungen:}
    \lang{en}{Using the chain rule and product rule, we find the derivatives:}
\begin{eqnarray*}
h'(x) &=& 1\cdot e^{-x}+x\cdot ((-1)\cdot e^{-x})=1\cdot e^{-x}-xe^{-x}=(1-x)e^{-x}, \\
h''(x) &=& -1\cdot e^{-x}+ (1-x)\cdot ((-1)\cdot e^{-x})=(x-2)e^{-x},\\
h'''(x) &=& 1\cdot e^{-x}+(x-2)\cdot ((-1)\cdot e^{-x})=(3-x)e^{-x}.
\end{eqnarray*}    
  	 %------------------------------------END_STEP_X
  	 
  	 %----------------------------------START_STEP_X
    \step 
    \lang{de}{   Für die ersten drei Ableitungen (also für $n=1$, $2$ und $3$ gilt die Formel
\[  h^{(n)}(x)=(-1)^n(x-n)e^{-x}. \]
Dass diese Formel auch für alle $n>3$ gilt, erhält man induktiv:    }
\lang{en}{   The first three derivatives ($n=1$, $2$ and $3$) are given by the formula
\[  h^{(n)}(x)=(-1)^n(x-n)e^{-x}. \]
We will show this formula continues to hold for $n>3$ by induction:    }
  	 %------------------------------------END_STEP_X
  	 
  	 %----------------------------------START_STEP_X
    \step 
    \lang{de}{   Wenn sie nämlich für ein $n$ gilt, so folgt:
\begin{eqnarray*}
h^{(n+1)}(x) &=& (h^{(n)})'(x) \\
&=& ((-1)^n(x-n)e^{-x})'\quad \text{nach Voraussetzung}\\
&=& (-1)^n\left( 1\cdot e^{-x}+(x-n)\cdot ((-1)\cdot e^{-x})\right) \\
&=& (-1)^{n+1}\left( -1\cdot e^{-x}+(x-n)\cdot e^{-x}\right) \\
&=&(-1)^{n+1}(x-n-1)e^{-x}
\end{eqnarray*}    
}
\lang{en}{ If the formula is valid for a given $n$, then
\begin{eqnarray*}
h^{(n+1)}(x) &=& (h^{(n)})'(x) \\
&=& ((-1)^n(x-n)e^{-x})'\quad \text{by assumption}\\
&=& (-1)^n\left( 1\cdot e^{-x}+(x-n)\cdot ((-1)\cdot e^{-x})\right) \\
&=& (-1)^{n+1}\left( -1\cdot e^{-x}+(x-n)\cdot e^{-x}\right) \\
&=&(-1)^{n+1}(x-n-1)e^{-x}
\end{eqnarray*}    }
  	 %------------------------------------END_STEP_X
 
  \end{incremental}
  %++++++++++++++++++++++++++++++++++++++++++++END_TAB_X

  %++++++++++++++++++++++++++++++++++++++++++START_TAB_X
  \tab{\lang{de}{    Lösung b)    } \lang{en}{Solution b)}}
  \begin{incremental}[\initialsteps{1}]
  
  	 %----------------------------------START_STEP_X
    \step 
    \lang{de}{   Wegen $g(x)=\frac{1}{x}=x^{-1}$ erhält man mit der Formel für Potenzen:}
    \lang{en}{Since $g(x) = \frac{1}{x} = x^{-1}$, we use the power rule to find: }
\begin{eqnarray*}
g'(x)&=& (-1)x^{-2}, \\
g''(x)&=& (-1)(-2)x^{-3}=2x^{-3}, \\
g'''(x)&=& 2(-3)x^{-4}.
\end{eqnarray*}    
  	 %------------------------------------END_STEP_X
  	 
  	 %----------------------------------START_STEP_X
    \step 
    \lang{de}{   
Für $n\in \{1;2;3\}$ gilt damit
\[  g^{(n)}(x) = (-1)^n\cdot n!\cdot x^{-n-1}. \]
Dass diese Formel auch für alle $n>3$ gilt, erhält man induktiv:    }
\lang{en}{   
Therefore, for $n\in \{1;2;3\}$ we have
\[  g^{(n)}(x) = (-1)^n\cdot n!\cdot x^{-n-1}. \]
We will show that this formula holds for $n>3$ by induction:    }
  	 %------------------------------------END_STEP_X
  	 
  	 %----------------------------------START_STEP_X
    \step 
    \lang{de}{   Wenn sie nämlich für ein $n$ gilt, so folgt:
\begin{eqnarray*}
g^{(n+1)}(x) &=& (g^{(n)})'(x) \\
&=& \left((-1)^n n!x^{-n-1}\right)'\quad \text{nach Voraussetzung}\\
&=& (-1)^n n! \left((-n-1)x^{-n-2}\right) \\
&=& (-1)^{n+1} n! (n+1)x^{-n-2} \\
&=&  (-1)^{n+1} (n+1)! x^{-n-2}
\end{eqnarray*}    }
\lang{en}{  If the formula is valid for a given $n$, then
\begin{eqnarray*}
g^{(n+1)}(x) &=& (g^{(n)})'(x) \\
&=& \left((-1)^n n!x^{-n-1}\right)'\quad \text{by assumption}\\
&=& (-1)^n n! \left((-n-1)x^{-n-2}\right) \\
&=& (-1)^{n+1} n! (n+1)x^{-n-2} \\
&=&  (-1)^{n+1} (n+1)! x^{-n-2}
\end{eqnarray*}    }
  	 %------------------------------------END_STEP_X
 
  \end{incremental}
  %++++++++++++++++++++++++++++++++++++++++++++END_TAB_X


%#############################################################ENDE

    \tab{\lang{de}{Video: ähnliche Übungsaufgabe} \lang{en}{Video: similar exercise}}
  \youtubevideo[500][300]{pcadMVcIocU}\\

\end{tabs*}
\end{content}