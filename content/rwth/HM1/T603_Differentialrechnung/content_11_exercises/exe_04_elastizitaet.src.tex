\documentclass{mumie.element.exercise}
%$Id$
\begin{metainfo}
  \name{
    \lang{en}{Exercise 4: Elasticity}
    \lang{de}{Ü04: Elastizität}
    \lang{zh}{...}
    \lang{fr}{...}
  }
  \begin{description} 
 This work is licensed under the Creative Commons License Attribution 4.0 International (CC-BY 4.0)   
 https://creativecommons.org/licenses/by/4.0/legalcode 

    \lang{en}{...}
    \lang{de}{...}
    \lang{zh}{...}
    \lang{fr}{...}
  \end{description}
  \begin{components}
  \end{components}
  \begin{links}
  \end{links}
  \creategeneric
\end{metainfo}
\begin{content}
\title{\lang{en}{Exercise 4: Elasticity}
    \lang{de}{Ü04: Elastizität}
    \lang{zh}{...}
    \lang{fr}{...}}
\begin{block}[annotation]
	Im Ticket-System: \href{https://team.mumie.net/issues/24025}{Ticket 24025}
\end{block}


\lang{de}{Berechnen Sie die Elastizität $\epsilon_f(x)$ für die Nachfragefunktion $f(x)=-0,2x+400$ an den Stellen
$x=800, 900, 1000, \ldots , 1500$ und interpretieren Sie die Ergebnisse. }

\lang{en}{Compute the elasticity $\epsilon_f(x)$ for the demand function $f(x)=-0.2x+400$ at the points
$x=800, 900, 1000, \ldots , 1500$ and interpret the results. }


  \begin{tabs*}[\initialtab{0}\class{exercise}]
    \tab{
      \lang{en}{Solution}
      \lang{de}{Lösung}
      \lang{zh}{...}
      \lang{fr}{...}
    }
    
    \begin{incremental}[\initialsteps{1}]
      \step
        \lang{en}{\begin{table}[\cellaligns{rcll}]
  \head[\cellaligns{llll}]
  x   & $\epsilon_f(x)$  & $|\epsilon_f(x)|$ & Elasticity
  \body
    800  & $\frac{-0.2\cdot 800}{-0.2\cdot 800+400}$ & $\frac{2}{3}<1$&inelastic \\
    900  & $\frac{-0.2\cdot 900}{-0.2\cdot 900+400}$     & $\frac{9}{11}<1$&inelastic \\
    1000 & $\frac{-0.2\cdot 1000}{-0.2\cdot 1000+400}$   & 1=1&proportional\\
    1500 & $\frac{-0.2\cdot 1500}{-0.2\cdot 1500+400}$   &  3>1&elastic
\end{table}}
        \lang{de}{\begin{table}[\cellaligns{rcll}]
  \head[\cellaligns{llll}]
  x   & $\epsilon_f(x)$  & $|\epsilon_f(x)|$ & Elastizität
  \body
    800  & $\frac{-0,2\cdot 800}{-0,2\cdot 800+400}$ & $\frac{2}{3}<1$&unelastisch \\
    900  & $\frac{-0,2\cdot 900}{-0,2\cdot 900+400}$     & $\frac{9}{11}<1$&unelastisch \\
    1000 & $\frac{-0,2\cdot 1000}{-0,2\cdot 1000+400}$   & 1=1&proportional\\
    1500 & $\frac{-0,2\cdot 1500}{-0,2\cdot 1500+400}$   &  3>1&elastisch
\end{table}
}
        \lang{zh}{...}
        \lang{fr}{...}
      
    \end{incremental}
    \tab{
      \lang{en}{Equations}
      \lang{de}{Gleichungen}
      \lang{zh}{...}
      \lang{fr}{...}
    }
    
    \begin{incremental}[\initialsteps{1}]
      \step
        \lang{en}{\begin{eqnarray*}
\epsilon_f(x)&=&\frac{x\cdot f'(x)}{f(x)}=\frac{x\cdot \frac{\Delta f}{\Delta x}}{f}
=\frac{\frac{x}{\Delta x}}{\frac{f}{\Delta f}}\\
&=&\frac{\frac{\Delta f}{f}}{\frac{\Delta x}{x}}\\
\end{eqnarray*}
The elasticity $\epsilon_f(x)$ of the function f at a point x is the ratio of the change
in f (demand) to the change in x (price). If $|\epsilon_f(x)|>1$, then we call
the demand elastic:
\begin{eqnarray*}
|\epsilon_f(x)|&>&1\\
\Leftrightarrow \ \frac{\Delta f}{f}&>&\frac{\Delta x}{x}
\end{eqnarray*}
The change in demand is greater than the change in price.
}
        \lang{de}{\begin{eqnarray*}
\epsilon_f(x)&=&\frac{x\cdot f'(x)}{f(x)}=\frac{x\cdot \frac{\Delta f}{\Delta x}}{f}
=\frac{\frac{x}{\Delta x}}{\frac{f}{\Delta f}}\\
&=&\frac{\frac{\Delta f}{f}}{\frac{\Delta x}{x}}\\
\end{eqnarray*}
Die Elastizität $\epsilon_f(x)$ der Funktion f an einer Stelle x ist also die relative Änderung 
von f (Nachfrage) bezogen auf eine relative Änderung von x (Preis). Ist $|\epsilon_f(x)|>1$, dann spricht
man von elastisch:
\begin{eqnarray*}
|\epsilon_f(x)|&>&1\\
\Leftrightarrow \ \frac{\Delta f}{f}&>&\frac{\Delta x}{x}
\end{eqnarray*}
Die relative Änderung in der Nachfrage ist größer als die relative Preisänderung.
}
        \lang{zh}{...}
        \lang{fr}{...}
      
     \end{incremental}
 \tab{
      \lang{en}{Calculations}
      \lang{de}{Rechnung}
      \lang{zh}{...}
      \lang{fr}{...}
    }
  \begin{incremental}[\initialsteps{1}]
      \step
        \lang{en}{Now consider $f(x)=-0.2x+400$ with $f'(x)=-0.2$. Then
\[\epsilon_f(x)=\frac{-0.2x}{-0.2x+400}\]}
        \lang{de}{ Wir betrachten nun $f(x)=-0,2x+400$ mit $f'(x)=-0,2$. Dann ist
\[\epsilon_f(x)=\frac{-0,2x}{-0,2x+400}\]
}
        \lang{zh}{...}
        \lang{fr}{...}
     \step  
     \lang{en}{\begin{table}[\cellaligns{rcll}]
  \head[\cellaligns{llll}]
  x   & $\epsilon_f(x)$  & $|\epsilon_f(x)|$ & Elasticity
  \body
    800  & $\frac{-0.2\cdot 800}{-0.2\cdot 800+400}$ & $\frac{2}{3}<1$&inelastic \\
    900  & $\frac{-0.2\cdot 900}{-0.2\cdot 900+400}$     & $\frac{9}{11}<1$&inelastic \\
    1000 & $\frac{-0.2\cdot 1000}{-0.2\cdot 1000+400}$   & 1=1&proportional\\
    1500 & $\frac{-0.2\cdot 1500}{-0.2\cdot 1500+400}$   &  3>1&elastic
\end{table}}
        \lang{de}{\begin{table}[\cellaligns{rcll}]
  \head[\cellaligns{llll}]
  x   & $\epsilon_f(x)$  & $|\epsilon_f(x)|$ & Elastizität
  \body
    800  & $\frac{-0,2\cdot 800}{-0,2\cdot 800+400}$ & $\frac{2}{3}<1$&unelastisch \\
    900  & $\frac{-0,2\cdot 900}{-0,2\cdot 900+400}$     & $\frac{9}{11}<1$&unelastisch \\
    1000 & $\frac{-0,2\cdot 1000}{-0,2\cdot 1000+400}$   & 1=1&proportional\\
    1500 & $\frac{-0,2\cdot 1500}{-0,2\cdot 1500+400}$   &  3>1&elastisch
\end{table}}

\lang{en}{Note that the slope of the demand function is constant everywhere ($f'=-0.2$).
When the price is low, the demand is high and the relative change in demand is small.
Here, demand is inelastic, i.e. it reacts slowly to changes in price.
As the price increases, the relative change in demand increases, 
and we enter the elastic region. }

\lang{de}{Wir berücksichtigen immer, dass die Steigung der Nachfragefunktion überall konstant ist ($f'=-0,2$).
Bei kleineren Preisen ist die Nachfrage groß, so dass die relative Nachfrageänderung klein ist.
Dort ist die Nachfrage unelastisch, will sagen, die Nachfrage spricht nicht gut auf Preisänderungen an.
Mit zunehmendem Preis sinkt die Nachfrage, die relative Nachfrageänderung wird größer, 
wir kommen in den elastischen Bereich. }

    \end{incremental}
  \end{tabs*}




\end{content}

