\documentclass{mumie.element.exercise}
%$Id$
\begin{metainfo}
  \name{
    \lang{en}{Exercise 2: Cost minimization}
    \lang{de}{Ü02: Kostenminimierung}
    \lang{zh}{...}
    \lang{fr}{...}
  }
  \begin{description} 
 This work is licensed under the Creative Commons License Attribution 4.0 International (CC-BY 4.0)   
 https://creativecommons.org/licenses/by/4.0/legalcode 

    \lang{en}{...}
    \lang{de}{...}
    \lang{zh}{...}
    \lang{fr}{...}
  \end{description}
  \begin{components}
  \end{components}
  \begin{links}
  \end{links}
  \creategeneric
\end{metainfo}
\begin{content}
\title{\lang{en}{Exercise 2: Cost minimization}
    \lang{de}{Ü02: Kostenminimierung}
    \lang{zh}{...}
    \lang{fr}{...}}
\begin{block}[annotation]
	Im Ticket-System: \href{https://team.mumie.net/issues/24033}{Ticket 24033}
\end{block}

\lang{de}{
Die Lagerhaltungskosten einer Firma für ein bestimmtes Produkt 
sollen minimiert werden. Pro Woche besteht ein Bedarf von 400 Stück. 
Pro Lieferung entstehen Kosten von 100 \euro, unabhängig von der 
gelieferten Anzahl. Die Firma hat 2 \euro Kosten pro Stück und Woche.
Für jedes Stück wird dieselbe (durchschnittliche) Lagerzeit angenommen.

Bestimmen Sie die bestellte Stückzahl je Lieferung.}

\lang{en}{
A company seeks to minimze the cost of storing a particular product.
It requires 400 units per week. Each delivery of the product entails
a cost of 100 \euro regardless of the number of delivered units.
Storing the product costs 2 \euro per unit. Each unit is assumed to
be stored (on average) for the same length of time.

Determine how many units should be ordered in each delivery.
}

  \begin{tabs*}[\initialtab{0}\class{exercise}]
    \tab{
      \lang{en}{Solution}
      \lang{de}{Lösung}
      \lang{zh}{...}
      \lang{fr}{...}
    }
    \begin{incremental}[\initialsteps{1}]
      \step
        \lang{en}{200 units should be ordered in each delivery.}
        \lang{de}{Es sind 200 Stück je Lieferung zu bestellen.}
        \lang{zh}{...}
        \lang{fr}{...}
      \step
        \lang{en}{Data: \\
        \begin{table}
        Demand per week: & 400 units\\
        Weekly costs per unit: & 2\euro\\
        Cost per delivery:& 100\euro\\
        Average storage time: & m weeks
        \end{table}
        }
        \lang{de}{Angaben: \\
        \begin{table}
        Bedarf pro Woche: & 400 Stück\\
        Kosten pro Woche und Stück: & 2\euro\\
        Kosten je Lieferung:& 100\euro\\
        mittlere Lagerzeit: & m Wochen
        \end{table}
        }
        \lang{zh}{...}
        \lang{fr}{...}
      \step
      \lang{de}{Lieferkosten:\\
      Wenn jede Lieferung $x$ Stück enthält, kostet die Lieferung pro Stück $\frac{100}{x}$. In dem Zeitraum von 
      m Wochen werden $400\cdot m$ Stück benötigt.\\
      $K_{\text{Liefer}}=\frac{100}{x}\cdot 400\cdot m$\\
      Mit diesem alleinigen Kriterium sollte die Menge des Gutes möglichst groß gewählt werden. Mit zunehmendem Lagerbestand
      wachsen allerdings auch die Lagerkosten, so dass hier ein Optimum gesucht werden muss.
      Lagerhaltungskosten:\\
      Betrachten wir einen Zeitraum zwischen zwei Lieferungen (m Wochen), so ist der Bestand zu Beginn x Stück, 
      am Ende ist das Lager jedoch leer. D.\,h. der mittlere Bestand beträgt genau $\frac{x}{2}$ Stück.\\
      $K_{\text{Lager}}=\frac{x}{2}\cdot 2\cdot m$\\ }
       \lang{en}{Delivery costs:\\
       If each delivery contains $x$ units, then the delivery cost per unit is $\frac{100}{x}$.
       Over $m$ weeks, the demand is $400 \cdot m$ units.
      $K_{\text{Delivery}}=\frac{100}{x}\cdot 400\cdot m$\\
      If this were the only cost, then the number of units per delivery would be chosen as large as possible.
      However, increasing the inventory incurs storage costs, so we have to look for an optimum value.
      Consider the time between two deliveries (m weeks). In the beginning, the inventory is x units,
      while at the end the inventory is empty. The average inventory is therefore $\frac{x}{2}$ units. \\
      $K_{\text{Storage}} = \frac{x}{2} \cdot 2 \cdot m.$\\ }
      
      \step
      \lang{de}{Die Gesamtkosten sind also:
      \[K(x)=\frac{40.000\cdot m}{x}+x\cdot m\]
      Wir suchen das Minimum der Kostenfunktion, d.\,h. wir bilden 
      $K'(x)$ und $K''(x)$:
      \begin{align*}
      K'(x)&=&\frac{-40.000\cdot m}{x^2}+m\\
      K''(x)&=&\frac{80.000\cdot m}{x^3}
      \end{align*}
      $K'(x)=0$ liefert:
      \begin{align*}
      -\frac{40.000\cdot m}{x^2}+m&=&0\\
      -40.000+x^2&=&0\\
      x^2&=&40.000\\
      x_{1/2}&=&\pm 200
      \end{align*}
      Die negative Lösung schließen wir aus, wir berechnen noch $K''(200)>0$, 
      d.\,h. bei $x=200$ liegt ein lokles Minimum vor. Wegen $\lim_{x\searrow 0}K(x)=\infty$ 
      sowie $\lim_{x\to\infty}K(x)=\infty$
      handelt es sich um ein globale Minimalstelle.
      
      Also werden pro Lieferung 200
      Stück geliefert.}
      \lang{en}{The total costs are:
      \[K(x)=\frac{40,000\cdot m}{x}+x\cdot m.\]
      We want to minimize the cost function, so we consider
      $K'(x)$ und $K''(x)$:
      \begin{align*}
      K'(x)&=&\frac{-40,000\cdot m}{x^2}+m\\
      K''(x)&=&\frac{80,000\cdot m}{x^3}
      \end{align*}
      $K'(x)=0$ yields:
      \begin{align*}
      -\frac{40,000\cdot m}{x^2}+m&=&0\\
      -40,000+x^2&=&0\\
      x^2&=&40,000\\
      x_{1/2}&=&\pm 200
      \end{align*}
      The negative solution can be eliminated. We calculate $K''(200)>0$ and conclude
      that $K(x)$ has a local minimum at $x=200$. Since $\lim_{x\searrow 0}K(x)=\infty$ 
      and $\lim_{x\to\infty}K(x)=\infty$,
      this is a global minimum.
      
      Therefore, each delivery should contain 200 units.}
    \end{incremental}
    
    
  \end{tabs*}





\end{content}

