\documentclass{mumie.element.exercise}
%$Id$
\begin{metainfo}
  \name{
    \lang{en}{Exercise 4: Extrema}
    \lang{de}{Ü04: Extremstellen}
  }
  \begin{description} 
 This work is licensed under the Creative Commons License Attribution 4.0 International (CC-BY 4.0)   
 https://creativecommons.org/licenses/by/4.0/legalcode 

    \lang{en}{...}
    \lang{de}{...}
  \end{description}
  \begin{components}
  \end{components}
  \begin{links}
  \end{links}
  \creategeneric
\end{metainfo}
\begin{content}
\title{\lang{en}{Exercise 4: Extrema}
    \lang{de}{Ü04: Extremstellen}}
\begin{block}[annotation]
	Im Ticket-System: \href{https://team.mumie.net/issues/24073}{Ticket 24073}
\end{block}



 \begin{block}[annotation]
  Kopie: hm4mint/T106_Differentialrechnung/exercise 11 
  
  Im Ticket-System: \href{http://team.mumie.net/issues/9461}{Ticket 9461}
\end{block}


\lang{de}{Berechnen Sie die lokalen Extremalstellen der Funktion $f(x)=x^2e^{-2x}$. 
Benutzen Sie das Kriterium f\"{u}r die zweite Ableitung von $f$, um zu \"{u}berpr\"{u}fen, ob in den station\"{a}ren Stellen lokale Extrema vorliegen. }
\lang{en}{Calculate the local extrema of the function $f(x)=x^2e^{-2x}$. Use the second derivative to determine whether the stationary points are local extrema.}

\begin{tabs*}[\initialtab{0}\class{exercise}]
  \tab{
  \lang{de}{Antwort}
  \lang{en}{Answer}
  }
  \lang{de}{In $1$ liegt ein lokales Maximum vor, in $0$ liegt ein lokales Minimum vor.}
  \lang{en}{There is a local maximum at $x=1$ and a local minimum at $x=0$.}
 
  \tab{
  \lang{de}{L\"{o}sung}
  \lang{en}{Solution}
  }
  
  \begin{incremental}[\initialsteps{1}]
    \step \lang{de}{ Wir suchen zuerst die station\"{a}ren Stellen. Dazu berechnen wir $f'(x)$:}
    \lang{en}{First, we find the critical points. Therefore, we calculate $f'(x)$:}
\[f'(x)=2xe^{-2x}-2x^2e^{-2x}=2xe^{-2x}(1-x).\]
\step \lang{de}{Wir setzen $f'(x)=0$, also}
\lang{en}{We set $f'(x)=0$, so}
\[2xe^{-2x}(1-x)=0.\] 
\step \lang{de}{Da $e^{-2x}\neq 0$ ist f\"{u}r alle $x$, muss $x=0$ oder $x=1$ gelten. Nur in $x_0:=0$ und in $x_1:=1$ 
k\"{o}nnen also lokale Extrema vorliegen. Wir m\"{u}ssen nun noch \"{u}berpr\"{u}fen, ob dort tats\"{a}chlich lokale Extrema vorliegen, 
und falls dies der Fall sein sollte, welcher Art sie sind.}
\lang{en}{Since $e^{-2x}\neq 0$ for all values of $x$, either $x=0$ or $x=1$. So, local extrema can only exist at $x_0:=0$ and $x_1:=1$.
Now we must verify whether local extrema do in fact exist at these points and, if so, which type they are.}

\step \lang{de}{F\"{u}r die zweite Ableitung erhalten wir mit der Produktregel:}
\lang{en}{The second derivative is obtained using the product rule:}

 
\[f''(x)=2e^{-2x}(1-x)-4xe^{-2x}(1-x)-2xe^{-2x}=2e^{-2x}(2x^2-4x+1).\]
\lang{de}{ Wir setzen $x_0=0$ in $f''$ ein und erhalten}
\lang{en}{We substitute $x_0=0$ into $f''$ and get }
\[f''(0)=2>0.\] \lang{de}{Damit liegt in $x_0$ ein lokales Minimum vor. }
\lang{en}{Therefore, there is a local minimum at $x=0$.}
\step \lang{de}{In $x_1=1$ ist}
\lang{en}{With $x_1=1$, we get}

    \[f''(1)=-2e^{-2}<0.\] \lang{de}{Damit liegt in $x_1$ ein lokales Maximum vor. }
    \lang{en}{Therefore, there is a local maximum at $x=1$.}

   
  
  \end{incremental}


\end{tabs*}

\end{content}

