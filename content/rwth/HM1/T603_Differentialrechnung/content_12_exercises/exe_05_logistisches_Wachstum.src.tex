\documentclass{mumie.element.exercise}
%$Id$
\begin{metainfo}
  \name{
    \lang{en}{Exercise 5: logistic growth}
    \lang{de}{Ü05: logistisches Wachstum}
    \lang{zh}{...}
    \lang{fr}{...}
  }
  \begin{description} 
 This work is licensed under the Creative Commons License Attribution 4.0 International (CC-BY 4.0)   
 https://creativecommons.org/licenses/by/4.0/legalcode 

    \lang{en}{...}
    \lang{de}{...}
    \lang{zh}{...}
    \lang{fr}{...}
  \end{description}
  \begin{components}
\component{generic_image}{content/rwth/HM1/images/g_tkz_T603_logistisch.meta.xml}{T603_logistisch}

\end{components}
  \begin{links}
  \end{links}
  \creategeneric
\end{metainfo}
\begin{content}
\title{\lang{en}{Exercise 5: logistic growth}
    \lang{de}{Ü05: logistisches Wachstum}
    \lang{zh}{...}
    \lang{fr}{...}}
\begin{block}[annotation]
	Im Ticket-System: \href{https://team.mumie.net/issues/24037}{Ticket 24037}
\end{block}
\begin{block}[annotation]
Warum der Graph einen Schlenker hat, weiss ich nicht
\end{block}

\lang{de}{Diskutieren Sie die Funktion $f(x)=\frac{a}{1+be^{-cx}}$ für $x>0$ und $a,b,c,>0$:}
\lang{en}{Discuss the function $f(x) = \frac{a}{1+be^{-cx}}$ for $x > 0$ and $a,b,c > 0$.}
 

   \begin{tabs*}[\initialtab{0}\class{exercise}]

    \tab{
      \lang{en}{Analysis}
      \lang{de}{Kurvendiskussion}
      \lang{zh}{...}
      \lang{fr}{...}
    }
    \begin{incremental}[\initialsteps{1}]
      \step
        \lang{en}{We are given the function:
\[f(x)=\frac{a}{1+be^{-cx}}\]
We will consider this function for $x>0$ and $a,b,c>0$.
\begin{eqnarray*}
f'(x)&=&a\frac{-1}{(1+be^{-cx})^2}\cdot (-cbe^{-cx})\\
&=&\frac{abc\cdot e^{-cx}\cdot (e^{cx})^2}{(1+be^{-cx})^2\cdot (e^{cx})^2}\\
&=&\frac{abc\cdot e^{cx}}{(b+e^{cx})^2}
\end{eqnarray*}
Applying the quotient rule to $f=\frac{u}{v}$ yields $f'=\frac{u'v-uv'}{v^2}$.

\begin{eqnarray*}
f''(x)&=&abc\frac{ce^{cx}(b+e^{cx})^2-e^{cx}2(b+e^{cx})ce^{cx}}{(b+e^{cx})^4}\\
&=&abc\cdot ce^{cx}\frac{b+e^{cx}-2e^{cx}}{(b+e^{cx})^3}\\
&=&abc\cdot ce^{cx}\frac{b-e^{cx}}{(b+e^{cx})^3}
\end{eqnarray*}
\notion{Discussion of the graph:}\\
\begin{itemize}
\item[Zeros]\\
The function is nowhere zero, since $f(x)\neq 0$.
\item[Extrema]\\
The function has no extreme values, since $f'(x)\neq 0$.\\
\item[Monotonicity]\\
The function is strictly monotonically increasing, since $f'(x)>0$.
\item[Inflection points]\\
The function has inflection points: $f''(x_w)=0\iff b-e^{cx_w}=0 
\iff x_w=\frac{\ln(b)}{c}$.\\

For $x<x_w$, we have $f''(x)>0$, so the function increases progressively. (Convex)\\
For $x>x_w$, we have $f''(c)<0$, so the function increases degressively. (Concave).
\item[Limits]\\
$f(0)=\frac{a}{1+b}$ and $\lim_{x\to\infty} f(x) =a$\\
These are lower and upper bounds because the function is monotonic.
\end{itemize}}
        \lang{de}{Wir betrachten die Funktion:
\[f(x)=\frac{a}{1+be^{-cx}}\]
Wir betrachten die Funktion für $x>0$ und $a,b,c>0$.
\begin{eqnarray*}
f'(x)&=&a\frac{-1}{(1+be^{-cx})^2}\cdot (-cbe^{-cx})\\
&=&\frac{abc\cdot e^{-cx}\cdot (e^{cx})^2}{(1+be^{-cx})^2\cdot (e^{cx})^2}\\
&=&\frac{abc\cdot e^{cx}}{(b+e^{cx})^2}
\end{eqnarray*}
Mit der Quotientenregel für $f=\frac{u}{v}$ ist $f'=\frac{u'v-uv'}{v^2}$.

\begin{eqnarray*}
f''(x)&=&abc\frac{ce^{cx}(b+e^{cx})^2-e^{cx}2(b+e^{cx})ce^{cx}}{(b+e^{cx})^4}\\
&=&abc\cdot ce^{cx}\frac{b+e^{cx}-2e^{cx}}{(b+e^{cx})^3}\\
&=&abc\cdot ce^{cx}\frac{b-e^{cx}}{(b+e^{cx})^3}
\end{eqnarray*}
\notion{Kurvendiskussion:}\\
\begin{itemize}
\item[Nullstellen]\\
Die Funktion hat keine Nullstellen,da $f(x)\neq 0$.
\item[Extrema]\\
Die Funktion hat keine Extremstellen, da $f'(x)\neq 0$.\\
\item[Monotonie]\\
Die Funktion ist streng monoton steigend, da $f'(x)>0$.
\item[Wendepunkte]\\
Die Funktion hat Wendestellen für $f''(x_w)=0\iff b-e^{cx_w}=0 
\iff x_w=\frac{\ln(b)}{c}$.\\

Für $x<x_w$ ist $f''(x)>0$, d.\,h. die Funktion ist progressiv wachsend (Linkskrümmung),\\
für $x>x_w$ ist $f''(c)<0$, d.\,h. die Funktion ist degressiv wachsend (Rechtskrümmung).
\item[Grenzwerte]\\
$f(0)=\frac{a}{1+b}$ und $\lim_{x\to\infty} f(x) =a$\\
Wegen der Monotonie der Funktion sind das untere und obere Schranke der Funktion.
\end{itemize}
}
        \lang{zh}{...}
        \lang{fr}{...}
      
    \end{incremental}
    
    
    
        \tab{
      \lang{en}{Explanation}
      \lang{de}{Erläuterung}
      \lang{zh}{...}
      \lang{fr}{...}
    }
    \begin{incremental}[\initialsteps{1}]
      \step
        \lang{en}{In logistic growth, the growth over time follows an S-shaped curve.
        We have the relationship (differential equation)
        \[f'(t)=k\cdot f(t)(G-f(t))\]
        The proportionality between the rate of change and $f(t)$ itself leads to \notion{exponential growth}
        in the beginning and \notion{bounded growth} as time increases.
        Solving the differential equation yields
        \[f(t)=\frac{a\cdot G}{a+(G-a)e^{-Gkt}}\]
        for $a, G > 0$.
        The function is bounded from below by $a=f(0)$ and from above by
        $G=\lim_{t\to\infty}f(t)$.

        Example: Algae growth on a lake
        
        A lake has a surface area of $G=300m^2$. In the beginning, $a=2m^2$ of it is
        covered by algae. The constant of proportionality is $k=0.003$ per month.
        Therefore, the expression for the area covered in $m^2$ is
        \[f(t)=\frac{600}{2+298e^{-0.9\cdot t}}\text{   ,t in months}\]
        \begin{table}
        t &0&1&2&3&4&5&6&7&8&9&10&11&12\\
        f(t)&2&4.9&11.7&27.2&59.2&113.0&179.3&235.6&270.0&287.0&294.7&297.8&299.1
        \end{table}
        \image{T603_logistisch}
        }
        \lang{de}{Beim logistischen Wachstum wird mit der Zeit eine S-förmige Wachstumskurve durchlaufen. 
Es gilt der Zusammenhang (Differentialgleichung):
\[f'(t)=k\cdot f(t)(G-f(t))\]
Die Proportionalität der Änderungsrate zu $f(t)$ führt anfänglich zu \notion{exponentiellem Wachstum}
und für große Zeiten zu \notion{begrenztem Wachstum}. Wird die Differentialgleichung gelöst,
gilt für $a,G>0$:
\[f(t)=\frac{a\cdot G}{a+(G-a)e^{-Gkt}}\]
Die Funktionswerte sind nach unten beschränkt durch $a=f(0)$ und nach oben von
$G=\lim_{t\to\infty}f(t)$.

Beispiel: Algenausbreitung auf einem See

Der See hat eine Fläche von $G=300m^2$. Anfänglich bedecken die Algen eine 
Fläche von $a=2m^2$.
Die Proportionalitätskonstante ist $k=0,003$ pro Monat. Somit lautet die 
Funktionsvorschrift für die Fläche in $m^2$:

\[f(t)=\frac{600}{2+298e^{-0,9\cdot t}}\text{   ,t in Monaten}\]
\begin{table}
t &0&1&2&3&4&5&6&7&8&9&10&11&12\\
f(t)&2&4,9&11,7&27,2&59,2&113,0&179,3&235,6&270,0&287,0&294,7&297,8&299,1
\end{table}
\image{T603_logistisch}

}
        \lang{zh}{...}
        \lang{fr}{...}
      
    \end{incremental}
  \end{tabs*}




\end{content}

