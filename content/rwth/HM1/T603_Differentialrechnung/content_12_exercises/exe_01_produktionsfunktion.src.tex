\documentclass{mumie.element.exercise}
%$Id$
\begin{metainfo}
  \name{
    \lang{en}{Exercise 1: Production function}
    \lang{de}{Ü01: Produktionsfunktion}
    \lang{zh}{...}
    \lang{fr}{...}
  }
  \begin{description} 
 This work is licensed under the Creative Commons License Attribution 4.0 International (CC-BY 4.0)   
 https://creativecommons.org/licenses/by/4.0/legalcode 

    \lang{en}{...}
    \lang{de}{...}
    \lang{zh}{...}
    \lang{fr}{...}
  \end{description}
  \begin{components}
\component{generic_image}{content/rwth/HM1/images/g_tkz_T603_12_Exercise01.meta.xml}{T603_12_Exercise01}
\end{components}
  \begin{links}
  \end{links}
  \creategeneric
\end{metainfo}
\begin{content}
\title{
\lang{en}{Exercise 1: Production function}
    \lang{de}{Ü01: Produktionsfunktion}
    \lang{zh}{...}
    \lang{fr}{...}
}
\begin{block}[annotation]
	Im Ticket-System: \href{https://team.mumie.net/issues/24032}{Ticket 24032}
\end{block}

\lang{de}{
Gegeben sei die Produktionsfunktion $y=f(x)=-0,6\cdot x^3+36\cdot x^2 +73,8\cdot x$ 
\begin{center}
\image{T603_12_Exercise01}
\end{center}

($x=$ INPUT in Mengeneinheiten, $y=$ OUTPUT in Mengeneinheiten) mit maximal möglichem INPUT von 32 
Mengeneinheiten. 

Führen Sie eine Kurvendiskussion durch und interpretieren Sie die Ergebnisse bzgl. (maximalem) Output, 
Grenzproduktivität und Wachstumsverhalten.}

\lang{en}{
Suppose we are given the production function $y=f(x)=-0.6\cdot x^3+36\cdot x^2 +73.8\cdot x$ 
\begin{center}
\image{T603_12_Exercise01}
\end{center}

($x=$ INPUT in units, $y=$ OUTPUT in units) with a maximum possible INPUT of 32 units. 

Analyze the curve and discuss the (maximal) output, 
marginal product and growth.}

  \begin{tabs*}[\initialtab{0}\class{exercise}]
    \tab{
      \lang{en}{Solution}
      \lang{de}{Lösung}
      \lang{zh}{...}
      \lang{fr}{...}
    }
    \begin{incremental}[\initialsteps{1}]
      \step
        \lang{en}{x>0:\\
        Maximal output: $f(32)=19564.8$.
        
        Positive marginal product for $0<x<32$.
        
        Progressive growth for $0<x<20$ and degressive growth for $20<x<32$.
        
        }
        \lang{de}{x>0:\\
        maximaler Output: $f(32)=19564,8$.
        
        positive Grenzproduktivität für $0<x<32$.
        
        progressives Wachstum für $0<x<20$ und degressives Wachstum für $20<x<32$.
        
        }
        \lang{zh}{...}
        \lang{fr}{...}
      
    \end{incremental}
    \tab{
      \lang{en}{Analysis}
      \lang{de}{Kurvendiskussion}
      \lang{zh}{...}
      \lang{fr}{...}
    }
    \begin{incremental}[\initialsteps{1}]
      \step
        \lang{en}{
\begin{itemize}
            \item[(i)]  
            Domain: $\cal{D}=\R$. However, only the domain $[0; 32]$ is economically reasonable.
            \item[(ii)]
            Zeros: 
            \begin{eqnarray*}
                f(x)&=&0\\
                \Leftrightarrow -0.6\cdot x^3+36\cdot x^2+73.8\cdot x&=&0\\
                \Leftrightarrow -0.6x\cdot (x^2-60\ x-123)&=&0\\
                \Leftrightarrow x &=&30 \pm \sqrt{30^2+123}\text{ or } x=0\\
                \Leftrightarrow x&=&-1.98\text{ or }x=61.98\text{ or }x=0
            \end{eqnarray*}
            [The zeros at $-1.98$ and $0$ are indistinguishable on the graph due to
            the way the axes are scaled.]
            
            Only the zero at $x = 0$ is contained in the interval $[0; 32]$.
            \item[(iii)]
            Critical points:
            
            $f'(x)=-1.8\cdot x^2+72\cdot x+73.8$\\
            $f''(x)=-3.6x+72$\\
            $f'''(x)=-3.6$
            
            Candidates for extremal points: $f'(x_e)=0$
            \begin{eqnarray*}
                f'(x)&=&0\\
             \Leftrightarrow   -1.8\cdot x^2+72\cdot x+73.8&=&0\\
                \Leftrightarrow -1.8(x^2-40x-41)&=&0\\
                \Leftrightarrow x &=&20\pm 21\\
                \Leftrightarrow x &=&-1\text{ or }x =41\\
            \end{eqnarray*}
            Since $f''(-1)>0$ and $f''(41)<0$, there is a local minimum at $x_{e1}=-1$ and a local maximum at $x_{e2}=41$.
            Both points lie outside of the interval $[0;32]$. Since the function is continuous,
            we can conclude that it is monotonically increasing in the interval $[0; 32]$. 
            \item[(iv)]
            Inflection points:
            
            Candidate inflection points: $f''(x_w)=0$
            \begin{eqnarray*}
            f''(x)&=&0\\
           \Leftrightarrow -3.6x+72&=&0\\
            \Leftrightarrow x&=&20
            \end{eqnarray*}
            Since $f'''(20)\neq 0$, there is an inflection point at $x_w=20$. At this point, the growth turns from progressive to degressive.
            \item[(v)]
            Limits:
            
            $\lim_{x\to+\infty} f(x) =-\infty$\\
            $\lim_{x\to-\infty} f(x) =+\infty$\\
            
            (The limits do not have any economic interpretation in this example,
            as the function only models production in the interval $[0;32].$)
        \end{itemize}
        
        }
        \lang{de}{
        \begin{itemize}
            \item[(i)]  
            Definitionsbereich: $\cal{D}=\R$, ökonomisch sinnvoll ist jedoch nur das Intervall $[0; 32]$.
            \item[(ii)]
            Nullstellen: 
            \begin{eqnarray*}
                f(x)&=&0\\
                \Leftrightarrow -0,6\cdot x^3+36\cdot x^2+73,8\cdot x&=&0\\
                \Leftrightarrow -0,6x\cdot (x^2-60\ x-123)&=&0\\
                \Leftrightarrow x &=&30 \pm \sqrt{30^2+123}\text{ oder } x=0\\
                \Leftrightarrow x&=&-1,98\text{ oder }x=61,98\text{ oder }x=0
            \end{eqnarray*}
            [Aufgrund der Achsenskalierung sind die Nullstellen bei $1,98$ und bei $0$ im Graphen
            nicht zu unterscheiden.]
            
            Nur die Nullstelle an $x = 0$ liegt im Intervall $[0; 32]$.
            \item[(iii)]
            Extremstellen:
            
            $f'(x)=-1,8\cdot x^2+72\cdot x+73,8$\\
            $f''(x)=-3,6x+72$\\
            $f'''(x)=-3,6$
            
            Kandidaten für Extremstellen: $f'(x_e)=0$
            \begin{eqnarray*}
                f'(x)&=&0\\
             \Leftrightarrow   -1,8\cdot x^2+72\cdot x+73,8&=&0\\
                \Leftrightarrow -1,8(x^2-40x-41)&=&0\\
                \Leftrightarrow x &=&20\pm 21\\
                \Leftrightarrow x &=&-1\text{ oder }x =41\\
            \end{eqnarray*}
            Da $f''(-1)>0$ und $f''(41)<0$, ist bei $x_{e1}=-1$ ein lokales Minimum und bei $x_{e2}=41$ ein lokales Maximum. Beide 
            Werte liegen außerhalb des Intervalls $[0; 32]$, aber da die Funktion stetig ist, können wir schließen, dass die 
            Funktion auf dem Intervall $[0; 32]$ monoton wachsend ist. 
            \item[(iv)]
            Wendestellen:
            
            Kandidaten für Wendestellen: $f''(x_w)=0$
            \begin{eqnarray*}
            f''(x)&=&0\\
           \Leftrightarrow -3,6x+72&=&0\\
            \Leftrightarrow x&=&20
            \end{eqnarray*}
            Da $f'''(20)\neq 0$, ist bei $x_w=20$ eine Wendestelle. Hier wechselt das Wachstum von progressiv zu degressiv.
            \item[(v)]
            Grenzwerte:
            
            $\lim_{x\to+\infty} f(x) =-\infty$\\
            $\lim_{x\to-\infty} f(x) =+\infty$\\
            
            (Die Grenzwerte besitzen in diesem Fall keine ökonomische Interpretation, da die Funktion nur auf dem Intervall $[0; 32]$
            als Modell für die Produktivität dient.)
        \end{itemize}
        
        }
        \lang{zh}{...}
        \lang{fr}{...}

    \end{incremental}
    
   \tab{
      \lang{en}{Interpretation}
      \lang{de}{Interpretation}
      \lang{zh}{...}
      \lang{fr}{...}
    }
    \begin{incremental}[\initialsteps{1}]
      \step
        \lang{en}{
        We only consider $x>0$, as the input must be positive.\\
        The function has a local maximum at $x=41$. However, since the maximum possible input is $x=32$,
        the \notion{maximum possible output} is $y=f(32)=19564.8$. 
        
        The \notion{marginal product} is the increase in return corresponding to
        the increase of input by one unit. Assuming that the return is proportional to the output,
        the marginal product is proportional to $f'(x)$, i.e. the change in output relative to the change in input.
        
        The first derivative is positive in $(0;41)$, so the marginal product is positive.
        
        In the interval $(0;20)$, we have $f''(x)>0$, so the first derivative is increasing 
        and the \notion{growth} is progressive. For $x>20$, however, the first derivative is decreasing
        and the growth is degressive.
        }
        \lang{de}{
        Wir betrachten nun nur $x>0$, da der Input positiv ist.\\
        Die Funktion erreicht ein lokales Maximum bei $x=41$. Da der maximale Input aber $x=32$ beträgt,
        ist der \notion{maximal mögliche Output} $y=f(32)=19564,8$. 
        
        Die \notion{Grenzproduktivität} ist der Zuwachs des Ertrages, der durch den Einsatz einer weiteren Einheit des 
        Produktionsfaktors erzielt wird. Unter der Annahme, dass der Ertrag proportional zum Output ist, ist die
        Grenzproduktivität proportional zu $f'(x)$, also die Änderung des Outputs mit dem Input.
        
        Da die erste Ableitung auf dem Intervall $(0;41)$ positiv ist, ist auch die Grenzproduktivität positiv.
        
        Im Intervall $(0;20)$ ist $f''(x)>0$, d.\,h. die erste Ableitung wird immer größer, 
        das \notion{Wachstum} ist also progressiv, wohingegen sich der Effekt für $x>20$ umkehrt 
        und das Wachstum degressiv ist.
        }
        \lang{zh}{...}
        \lang{fr}{...}

    \end{incremental} 
  \end{tabs*}


\end{content}

