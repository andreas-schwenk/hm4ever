%$Id:  $
\documentclass{mumie.article}
%$Id$
\begin{metainfo}
  \name{
    \lang{de}{Extremstellen}
    \lang{en}{}
  }
  \begin{description} 
 This work is licensed under the Creative Commons License Attribution 4.0 International (CC-BY 4.0)   
 https://creativecommons.org/licenses/by/4.0/legalcode 

    \lang{de}{Beschreibung}
    \lang{en}{}
  \end{description}
  \begin{components}
    \component{js_lib}{system/media/mathlets/GWTGenericVisualization.meta.xml}{mathlet1}
    \component{generic_image}{content/rwth/HM1/images/g_img_hm-DefMenge-Wurzel.meta.xml}{1.2_Wurzel}
    \component{generic_image}{content/rwth/HM1/images/g_img_hm-DefMenge-Parabel.meta.xml}{1.2_Parabel}
    \component{generic_image}{content/rwth/HM1/images/g_img_hm-DefMenge-Hyperbel.meta.xml}{1.2_Hyperbel}
    \component{generic_image}{content/rwth/HM1/images/g_img_hm-extrem1.meta.xml}{image11}
    \component{generic_image}{content/rwth/HM1/images/g_img_hm-monotonie1.meta.xml}{image10}
    \component{generic_image}{content/rwth/HM1/images/g_img_hm-linFun.meta.xml}{image12}
    \component{generic_image}{content/rwth/HM1/images/g_img_hm-LinFunktion.meta.xml}{image2}
    \component{generic_image}{content/rwth/HM1/images/g_img_hm-parabel.meta.xml}{image3}
    \component{generic_image}{content/rwth/HM1/images/g_img_hm-wendest-sin.meta.xml}{image16}
    \component{generic_image}{content/rwth/HM1/images/g_img_Monotonie1.meta.xml}{image1}
    \component{generic_image}{content/rwth/HM1/images/g_img_Extrema2.meta.xml}{image-extrema2}
    \component{generic_image}{content/rwth/HM1/images/g_img_Extrema.meta.xml}{image-extrema1}
  \end{components}
  \begin{links}
    \link{generic_article}{content/rwth/HM1/T104_weitere_elementare_Funktionen/g_art_content_14_potenzregeln.meta.xml}{power-rules}
    \link{generic_article}{content/rwth/HM1/T102neu_Einfache_Reelle_Funktionen/g_art_content_06_funktionsbegriff_und_lineare_funktionen.meta.xml}{link9}
  \end{links}
  \creategeneric
\end{metainfo}
\begin{content}
\begin{block}[annotation]
	Im Ticket-System: \href{https://team.mumie.net/issues/22705}{Ticket 22705}
\end{block}
\begin{block}[annotation]
Copy of : content/rwth/HM1/T603_Differentialrechnung/art_copy_Extremstellen.src.tex
\end{block}

\begin{block}[annotation]
Copy of : content/rwth/HM1/T603_Differentialrechnung/art_content_Extremstellen.src.tex
\end{block}

\begin{block}[annotation]
Copy of : content/rwth/HM1/T603_Differentialrechnung/art_content_13_Extremstellen.src.tex
\end{block}

\usepackage{mumie.ombplus}
\ombchapter{6}
\ombarticle{3}
\usepackage{mumie.genericvisualization}

\begin{visualizationwrapper}

\lang{de}{\title{Extremstellen}}
 
\begin{block}[annotation]
  Kopie von T106_Differentialrechnung content_22_extremstellen 
  
\end{block}
\begin{block}[annotation]
  Im Ticket-System:
\end{block}

\begin{block}[info-box]
\tableofcontents
\end{block}

\section{\lang{de}{Monotonie}\lang{en}{Monotonicity}\label{Monotonie} }


  \lang{de}{\begin{definition}[monoton, streng monoton]
		 Eine Funktion ist auf einem Intervall im Definitionsbereich \emph{monoton wachsend},
		 wenn für ein beliebiges Paar $x_1, x_2$ von Argumenten im Intervall, für die die Ungleichung $x_1 \leq x_2$ gilt, 
		 die Ungleichung $f(x_1)\leq f(x_2)$ für die zugehörigen Funktionswerte folgt.\\
		
		 Sie heißt \emph{streng monoton wachsend}, falls aus der strengeren Ungleichung $x_1 < x_2$ 
		 die strengere Ungleichung $f(x_1) < f(x_2)$ folgt.\\
		 
		 Eine Funktion ist auf einem Intervall im Definitionsbereich \emph{monoton fallend}, 
		 wenn die Ungleichung $x_1 \leq x_2$ für ein beliebiges Paar von Argumenten aus dem Intervall 
		 die entsprechende Ungleichung $f(x_1)\geq f(x_2)$ für die zugehörigen Werte nach sich zieht.\\
		 
		 Sie heißt \emph{streng monoton fallend}, falls die strengere Folgerung $x_1<x_2 \Rightarrow f(x_1)> f(x_2)$ gilt.
		 
		 Eine Funktion heißt auf einem Intervall \emph{monoton}, wenn sie dort monoton wachsend oder fallend ist. 
	 \end{definition} }
	 
  \lang{en}{\begin{definition}[Monotonic, Strictly monotonic]
		 A function is said to be \emph{monotonically increasing} on an interval in the domain if for any given pair of arguments $x_1, x_2$ in that interval such that $x_1 \leq x_2$,
		 the inequality $f(x_1)\leq f(x_2)$ is satisfied.
		 
		 A function is said to be \emph{strictly monotonically increasing} if, for the strict inequality $x_1 < x_2$ the strict inequality $f(x_1) < f(x_2)$ is satisfied. Note here that
		 strict is meant in an inequality sense: strictly less than vs. less than or equal to.
		 
		 A function is said to be \emph{monotonically decreasing} on an interval in the domain if for any given pair of arguments $x_1, x_2$ in that interval such that $x_1 \leq x_2$,
		 the inequality $f(x_1)\geq f(x_2)$ is satisfied.
		 
		 A function is said to be \emph{strictly monotonically decreasing} if, for the strict inequality $x_1 < x_2$ the strict inequality $f(x_1) > f(x_2)$ is satisfied. Note here as well that
		 strict is meant in an inequality sense: strictly greater than vs. greater than or equal to.

		A function is called \emph{monotone} or \emph{monotonic} on an interval if it is either monotonically increasing or monotonically decreasing on the interval.
	 \end{definition} }

% \begin{example}
% \begin{tabs*}[\initialtab{1}\class{example}]
% \end{tabs*}
% \end{example}
 
 
%  \begin{remark}
%   \lang{de}{Funktionen haben in der Regel nicht ein einheitliches Steigungsverhalten auf 
%  		dem ganzen Definitionsbereich, sondern sind nur abschnittsweise monoton.}
%  
%  \lang{en}{Functions, in general, do not have one unified slope behaviour on their entire domain. They are normally only
%  		monotonic on specific intervals.}
%  
%  \end{remark}
%  

 \begin{example}
\begin{tabs*}[\initialtab{1}\class{example}]
\tab{$f(x) = x^2 $}
   \lang{de}{Die Funktion $f(x) = x^2 $ mit einer Parabel als Graph ist f"ur $x$-Werte $\leq 0$ (also auf dem Intervall $(-\infty; 0]$) 
   streng monoton fallend und f"ur $x$-Werte $\geq 0$ (also auf dem Intervall $(0;\infty)$) streng monoton steigend. }
	 
	\lang{en}{The function $f(x) = x^2 $ (whose graph is a parabola) is strictly monotonically decreasing for negative $x$-values and
		strictly monotonically increasing for positive $x$-values. }
	
\begin{center}
\image{image3}
\end{center}

\tab{$f(x)= \sin(x)$}

    \lang{de}{Betrachte die Sinusfunktion $f(x)= \sin(x)$ mit Definitionsbereich $D_f = \left[ -2\pi; 2\pi \right]$.}
    \lang{en}{Consider $f(x) = \sin(x)$ on the domain $D_f = \left[ -2\pi, 2\pi \right]$.}
	\\
    \lang{de}{$f(x)= \sin(x)$ ist streng monoton fallend auf den Intervallen \\ $\left[ -\frac{3}{2}\pi; -\frac{1}{2}\pi  \right]$, $\left[ \frac{1}{2}\pi; \frac{3}{2}\pi  \right]$}
    \lang{en}{$f(x)= \sin(x)$ is strictly monotonically decreasing on the intervals $\left[ -\frac{3}{2}\pi, -\frac{1}{2}\pi  \right]$, $\left[ \frac{1}{2}\pi, \frac{3}{2}\pi  \right]$}
    \\
    \lang{de}{$f(x)= \sin(x)$ ist streng monoton steigend auf den Intervallen \\ $\left[-2\pi; -\frac{3}{2}\pi \right]$, $\left[ -\frac{1}{2}\pi; \frac{1}{2}\pi  \right]$ und $\left[ \frac{3}{2}\pi; 2\pi  \right]$. }
    \lang{en}{$f(x)= \sin(x)$ is strictly monotonically increasing on the intervals $\left[-2\pi, -\frac{3}{2}\pi \right]$, $\left[ -\frac{1}{2}\pi, \frac{1}{2}\pi  \right]$, and $\left[ \frac{3}{2}\pi, 2\pi  \right]$. }
    \\
 \image{image16}
\tab{$f(x) = a x + b$}

	\lang{de}{\ref[link9][Der Funktionsgraph einer linearen Funktion]{geraden.example.1}$\;f(x) = a x + b$ 
	    ist eine Gerade mit der Steigung $a$ und $y$-Achsenabschnitt
		$b$ (Schnittpunkt mit der $y$-Achse). Die linearen Funktionen $f(x) = ax+b$ sind auf ganz $\R$ streng monoton steigend, 
		wenn die Steigung größer Null ist, d.h. $a > 0$, und auf ganz $\R$
		streng monoton fallend, wenn die Steigung negativ ist, d.h. $a<0$. Für $a=0$ sind sie konstant und damit 
		sowohl monoton steigend wie auch monoton fallend. }
    
    \lang{en}{\ref[link9][The graph of a linear function]{geraden.example.1} $\;f(x)=a x + b$ is a line with slope $a$ and $y$-intercept $b$ (the intersection point with the $y$-axis).
    	Linear functions $f(x) = ax+b$ are strictly monotonically increasing if the slope is greater than zero ($a>0$) and strictly monotonically decreasing if the slope is negative ($a<0$). 
    	If $a=0$ linear functions are constant and hence both monotonically increasing and monotonically decreasing. }
\begin{center}
\image{image2}
\end{center}

\tab{$f(x) = a x^3$}\label{ex:cubic}
    \lang{de}{	Eine Funktion der Form $f(x) = a x^3$ (Polynom 3. Grades) ist streng monoton steigend auf ganz $\R$, wenn der Koeffizient 
		$a$ positiv ist, $a > 0$, und streng monoton fallend auf ganz $\R$,
		 wenn $a < 0$ ist.}
	
	 \lang{en}{A function of the form $f(x) = ax^3$ (a third degree polynomial) is strictly monotonically increasing if the 
	 coefficient $a$ is positive ($a>0$) and strictly monotonically decreasing if $a<0$.}
	 
\begin{center}
\image{image10}
\end{center}

\tab{$f(x) = \sqrt{x}$}
    \lang{de}{Die Quadratwurzelfunktion $f(x) = \sqrt{x}$ ist auf ihrem ganzen Definitionsbereich (also auf $[0;\infty)$) 
    streng monoton steigend.}
    
     \lang{en}{The square root $f(x) = \sqrt{x}$ is strictly monotonically increasing.}
\begin{center}
\image{1.2_Wurzel}
\end{center}

\end{tabs*}
\end{example}
 

\section{
\lang{de}{Monotonie und Ableitung}
\lang{en}{Monotonicity and Derivation}}

\lang{de}{	Bei der graphischen Darstellung von Funktionen gibt die Tangente an den Funktionsgraphen ersten Aufschluss "uber das
	Steigungsverhalten: Die Funktion ist monoton steigend (fallend) in Bereichen, wo sie
	eine steigende (fallende) Tangente hat.}
	
\lang{en}{ When representing functions graphically, the tangent of the function gives a first glimpse of the behaviour of the slope:
	a function is strictly monotonically increasing (decreasing) in regions where it has a growing (decreasing) tangent.} 


\begin{theorem}
Ist $f$ eine auf einem Intervall $I$ differenzierbare Funktion, so gelten:
\begin{align*}
    \lang{de}{&\underline{f'(x)\;\text{ auf}\;I}&& \underline{\text{Monotonieverhalten von}\;f\;\text{auf}\;I}}
    \lang{en}{&\underline{f'(x)\;\text{ on}\;I}&& \underline{\text{Monotonicity of}\;f\;\text{on}\;I}}
    \\
    \lang{de}{&f'(x)\geq 0 &\Leftrightarrow &f\;\text{monoton wachsend}}
    \lang{en}{&f'(x)\geq 0 &\Leftrightarrow &f\;\text{monotonically increasing}}
    \\
    \lang{de}{&f'(x)\leq 0 &\Leftrightarrow &f\;\text{monoton fallend}}
    \lang{en}{&f'(x)\leq 0 &\Leftrightarrow &f\;\text{monotonically decreasing}}
    \\
    \lang{de}{\text{und}&&&}
    \lang{en}{\text{and}&&&}
    \\
    \lang{de}{&f'(x)> 0 &\Rightarrow &f\;\text{streng monoton wachsend}}
    \lang{en}{&f'(x)> 0 &\Rightarrow &f\;\text{strictly monotonically increasing}}
    \\
    \lang{de}{&f'(x)< 0 &\Rightarrow &f\;\text{streng monoton fallend}}
    \lang{en}{&f'(x)< 0 &\Rightarrow &f\;\text{strictly monotonically decreasing}}
    \\
\end{align*}
\end{theorem}


\begin{example}
	\image{image1}
	\\
	\lang{de}{Die Funktion $f(x)=x^2$ ist monoton fallend auf $(-\infty;0]$ und monoton steigend auf $[0;\infty)$. An einer Stelle $x_0$ mit 
	$x_0\leq 0$ hat also die Tangente eine Steigung $f'(x_0)\leq 0$, an einer Stelle $x_1$ mit 
	$x_1\geq 0$ hat die Tangente eine positive Steigung $f'(x_1)\geq 0$.\\
	Da die Tangente in $x_0 <0$ sogar eine negative Steigung hat, ist $f'(x) <0$ f\"{u}r $x\in (-\infty;0)$, 
	und damit ist $f$ auf $(-\infty;0)$ streng monoton fallend. 
	Analog gilt $f'(x)>0$ auf $(0;\infty)$ und $f$ ist dort streng monoton steigend. }
	\lang{en}{The function $f(x)=x^2$ is monotonically decreasing on $(-\infty,0]$ and monotonically increasing on $[0,\infty)$. The slope of the tangent at any point $x_0$ with 
	$x_0\leq 0$ is $f'(x_0)\leq 0$. The slope of the tangent at any point $x_1$ with $x_1\geq 0$ is positive: $f'(x_1)\geq 0$.\\
	Because the tangent at $x_0<0$ has a negative slope, $f'(x) <0$ for $x\in (-\infty,0)$, and hence $f$ is strictly monotonically decreasing on $(-\infty,0)$. Analogous results
	hold for $f'(x)>0$ on $(0,\infty)$: $f$ is strictly monotonically increasing on $(0,\infty)$.}
\end{example}

\begin{block}[warning]
Die Funktion $f$ kann streng monoton wachsen bzw. streng monoton fallen, auch wenn an manchen Stellen die Ableitung gleich Null ist, 
wie am \lref{ex:cubic}{Beispiel $f(x)=ax^3$} (mit $a>0$ bzw. mit $a<0$) zu sehen ist.
\end{block}



% 
% \lang{de}{Verschwindet die Ableitung einer Funktion $f$ an einer Stelle $x_0$, so ist die Tangente an den Graphen
% in diesem Punkt waagerecht. Eine solche Stelle nennt man auch eine station\"{a}re Stelle.}
% \lang{en}{If the first derivative of a function $f$ vanishes at a point $x_0$, then the tangent
% of the graph at that point is horizontal.
% Points on the graph of a function where the the tangent line is horizontal are called critical or stationary points.}

\begin{definition}
	\label{stat}
	\lang{de}{Eine Stelle $x$, an der $f'(x)=0$ gilt, hei{\ss}t \textit{station\"{a}re} Stelle von $f$. 
	Den dazugeh\"{o}rigen Punkt $(x;f(x))$ auf dem Graphen nennt man einen station\"{a}ren Punkt.}
	\lang{en}{Any point at which $f'(x)=0$ is called a \textit{critical point} or a \textit{stationary point} of $f$. We say that $f$ has
	a critical point at $x$ or at the point $(x,f(x))$.}
\end{definition}

\begin{example}
  	\lang{de}{Die Stelle $0$ ist eine station\"{a}re Stelle der Funktion $f(x)=x^3$, denn $f'(x)=3x^2$ und $f'(0)=0$. 
	Die Funktion $g(x)=x^3-3x$ hat die Ableitung $g'(x)=3x^2-3$, also sind wegen $g'(1)=0=g'(-1)$ die Stellen $x=1$ und $x=-1$ station\"{a}re Stellen
	von $g$.}
 	\lang{en}{The function $f(x)=x^3$ has a critical point at $x=0$, since $f'(x)=3x^2$ and $f'(0)=0$.
 	The function $g(x)=x^3-3x$ has a first derivative of $g'(x)=3x^2-3$, and since $g'(1)=0=g'(-1)$, 
 	$g$ has critical points at $x=1$ and $x=-1$.}
\end{example}


\begin{quickcheck}
		\field{rational}
		\type{input.number}
		\begin{variables}
			\number{c}{1}   % einfacher zu rechnen mit c=1 (auf jeden Fall muss c>0 sein.)
			\randint{ns1}{-2}{0}
			\randint{ns2}{1}{2}
			\randint{a}{-3}{3}
			\function[normalize]{dfak}{c*(x-ns1)*(x-ns2)}
			\function[expand,normalize]{d}{c*(x-ns1)*(x-ns2)}
			\function[normalize]{dd}{(c/3)*x^3-(c/2)*(ns1+ns2)*x^2+c*ns1*ns2*x+a}
		
		\end{variables}
		
			\text{Bestimmen Sie mit Hilfe der Ableitung den Bereich, in dem die Funktion $f(x)=\var{dd}$ monoton fällt.\\
			Sie fällt im Bereich \ansref$\leq x\leq $\ansref.}
		
		\begin{answer}
			\solution{ns1}
		\end{answer}
		\begin{answer}
			\solution{ns2}
		\end{answer}
		\explanation{Die Funktion fällt in dem Bereich, indem ihre Ableitung $\leq 0$ ist. Wegen
		$f'(x)=\var{d}=\var{dfak}$ ist dies der Bereich $\var{ns1}\leq x\leq \var{ns2}$.}
	\end{quickcheck}


	\begin{genericGWTVisualization}[550][1000]{mathlet1}
		\begin{variables}
			\number[editable]{x0}{real}{1}
			\point{P}{rational}{var(x0),0}
			%\pointOnCurve{P}{rational}{0}{1}  % verschiebbarer Punkt auf der x-Achse, initial p=(1;0)
			%\number{x0}{rational}{var(P)[x]}
			\number{xc}{rational}{var(x0)}  % nicht editierbare Kopie von x0.
			\function{f0}{rational}{x^5+x^4-2*x^3-2*x^2+x}
			\function{f1}{rational}{5*x^4+4*x^3-6*x^2-4*x+1}
			\number{y0}{rational}{var(x0)^5+var(x0)^4-2*var(x0)^3-2*var(x0)^2+var(x0)}
			\number{y1}{rational}{5*var(x0)^4+4*var(x0)^3-6*var(x0)^2-4*var(x0)+1}
			\number{y2}{rational}{20*var(x0)^3+12*var(x0)^2-12*var(x0)-4}
			\number{y3}{rational}{60*var(x0)^2+24*var(x0)-12}
			\point{Pc}{rational}{var(P)[x],var(P)[y]}   % Kopien von P zum anzeigen in verschiedenen canvas.
			\point{Pcc}{rational}{var(P)[x],var(P)[y]}
			\point{P0}{rational}{var(x0),var(y0)}
			\point{P1}{rational}{var(x0),var(y1)}
			\point{P2}{rational}{var(x0),var(y2)}
%% ---- Tangente an Graph(f) am Punkt P0; funktioniert nicht mit \parametricFunction wegen Variablen in der Definition.
%			\parametricFunction{Ta}{rational}{var(xc)+t, var(y0)+t*var(y1),-0.3,0.3,60}
			\point{Q1}{rational}{var(xc)-1/3, var(y0)-1/3*var(y1)}
			\point{Q2}{rational}{var(xc)+1/3, var(y0)+1/3*var(y1)}
			\segment{Ta}{rational}{var(Q1),var(Q2)}
		\end{variables}
		
		\color{f0}{BLUE}
		\color{P0}{DARK_BLUE}
		\color{f1}{CYAN}
%		\color{P1}{DARK_CYAN}
		\color{P1}{DARK_RED}  
 		\color{Ta}{DARK_RED}
% 		\label{f0}{@2d[$\textcolor{BLUE}{f(x)}$]}
% 		\label{f1}{@2d[$\textcolor{CYAN}{f'(x)}$]}
% 		\label{f2}{@2d[$\textcolor{GREEN}{f''(x)}$]}

		\text{Um $x=\var{x0}$ \IFELSE{|var(y1)|<0.02}{\IFELSE{var(y2)=0}{ist der Graph der Funktion 
		$f(x)=\var{f0}$ streng monoton \IFELSE{var(y3)<0}{fallend}{steigend}. Hat jedoch im Punkt 
		$(\var{xc}; \var{y0})$ eine waagrechte Tangente.}{geht 
		der Graph der Funktion $f(x)=\var{f0}$ vom \IFELSE{var(y2)<0}{Steigen}{Fallen} ins 
		\IFELSE{var(y2)<0}{Fallen}{Steigen} über und hat im Punkt $(\var{xc}; \var{y0})$ eine waagrechte 
		Tangente.}}{ist der Graph der Funktion $f(x)=\var{f0}$ streng
		monoton \IFELSE{var(y1)<0}{fallend}{steigend}.}
		}
		
		\begin{canvas}
			\updateOnDrag[false]
			\plotSize{250,400}
			\plotLeft{-2.5}
			\plotRight{2.5}
			\plot[coordinateSystem]{P, Ta, f0, P0}
		\end{canvas}
		\text{\IFELSE{|var(y1)|<0.02}{\IFELSE{var(y2)=0}{Dass der Graph streng monoton \IFELSE{var(y3)<0}{fällt}{steigt},
		ist gleichbedeutend dazu, dass die Ableitungsfunktion $f'(x)=\var{f1}$ um $x=\var{xc}$ 
		\IFELSE{var(y3)<0}{kleiner oder gleich $0$ ist}{größer oder gleich $0$ ist.} Wegen der waagrechten Tangente
		im Punkt $(\var{xc}; \var{y0})$ ist die Ableitung an der Stelle $x=\var{xc}$ sogar gleich $0$, also $f'(\var{xc})=0$.}
		{Dass der Graph der Funktion $f(x)=\var{f0}$ vom \IFELSE{var(y2)<0}{Steigen}{Fallen} ins 
		\IFELSE{var(y2)<0}{Fallen}{Steigen} übergeht, ist gleichbedeutend
		dazu, dass die Ableitungsfunktion $f'(x)=\var{f1}$ links von $x=\var{xc}$ \IFELSE{var(y2)<0}{positiv}{negativ} ist,
		und rechts von $x=\var{xc}$ \IFELSE{var(y2)<0}{negativ}{positiv}. 
		Insbesondere ist die Ableitung an der Stelle $x=\var{xc}$ sogar gleich $0$, also $f'(\var{xc})=0$.}}
		{Dass  der Graph der Funktion $f(x)=\var{f0}$ streng monoton \IFELSE{var(y1)<0}{fällt}{steigt}, ist gleichbedeutend 
		dazu, dass die Ableitungsfunktion $f'(x)=\var{f1}$ um $x=\var{xc}$ 
		\IFELSE{var(y1)<0}{kleiner}{größer} oder gleich $0$ ist. Hier ist sie sogar echt 
		\IFELSE{var(y1)<0}{kleiner}{größer} als $0$.}
		}
		
		\begin{canvas}
			\updateOnDrag[false]
			\plotSize{250,400}
			\plotLeft{-2.5}
			\plotRight{2.5}
			\plot[coordinateSystem]{Pc, f1, P1}
		\end{canvas}
	    	\end{genericGWTVisualization}



\section{\lang{de}{Extremal-, Minimal-, Maximalstellen}\lang{en}{Extrema, Minima, Maxima}\label{Extrema}}

 
  \lang{de}{	\begin{definition}[Maximalstellen, lokale Maximalstellen, Minimalstellen, lokale Minimalstellen] 
		\begin{itemize}
		  \item[(a)]
		    	 Das Argument $x_\max$ im Definitionsbereich $D_f$ heißt
				\emph{globale Maximalstelle} oder kurz \emph{Maximalstelle} der Funktion $f$, falls alle 
				Funktionswerte kleiner oder gleich dem Wert der Funktion an der Stelle $x_\max$ sind:
				$f(x_\max) \geq f(x)$ für alle $x \in D_f$.\\
		  \item[(b)] 
		    	$x_\max$ heißt \textit{lokale Maximalstelle} von $f$, falls 
				\emph{in einer Umgebung} von $x_\max$ alle Funktionswerte kleiner oder gleich 
				dem Wert der Funktion an der Stelle $x_\max$ sind 
			    oder genauer, wenn $x_\max$ zu einem offenen Intervall $I$ gehört, 
			    so dass $f(x_\max) \geq f(x)$ für alle $x \in D_f \cap I$.\\ 
		  \item[(c)] 
			    Das Argument $x_\min$ im Definitionsbereich $D_f$ heißt
				\emph{globale Minimalstelle} oder kurz \emph{Minimalstelle} der Funktion $f$, falls alle 
				Funktionswerte größer oder gleich dem Wert der Funktion an der Stelle $x_\min$ sind:
				$f(x_\min) \leq f(x)$ für alle $x \in D_f$,\\
		  \item[(d)] 
			  	$x_\min$ heißt \textit{lokale Minimalstelle} von $f$, falls 
				\emph{in einer Umgebung} von $x_\min$ alle Funktionswerte größer oder gleich 
				dem Wert der Funktion an der Stelle $x_\min$ sind 
		    	oder genauer, wenn $x_\min$ zu einem offenen Intervall $I$ gehört, 
		    	so dass $f(x_\min) \leq f(x)$ für alle $x \in D_f \cap I$.\\ 
		  \item[(e)] 
		  		Eine lokale (globale) Maximal- oder Minimalstelle bezeichnet man
		        auch als lokale (globale) \emph{Extremalstelle}.\\
		  \item[(f)] 
		  		Die Werte der Funktion $f$ an einer lokalen (globalen) Maximalstelle heißen \emph{lokales (globales) Maximum}.
		        Analog sind die Begriffe  \emph{lokales (globales) Minumum} und \emph{lokales (globales) Extremum} definiert.
		  \end{itemize}
	 \end{definition}}
	 
\lang{en}{
	\begin{definition}[Maxima, Local Maxima, Minima, Local Minima] 
		\begin{itemize}
		  \item[(a)]
		    	A function $f$ is said to have a \emph{global maximum} (plural: maxima) at $x_\max \in D_f$
				if all of the values of the function are less than or equal 
				to the value of the function at the point $x=x_\max$:
				$f(x_\max) \geq f(x)$ for all $x \in D_f$.\\
		  \item[(b)] 
		    	A function $f$ is said to have a \textit{local maximum} at $x_\max \in D_f$ if, 
				\emph{in a neighbourhood} around $x_\max$, the values of the function are less than or equal to
				the value of the function at the point $x=x_\max$. More precisely, $x_\max$ is a local maximum
			    if $x_\max$ is an element of an open interval $I$, 
			    such that $f(x_\max) \geq f(x)$ for all $x \in D_f \cap I$.\\ 
		  \item[(c)] 
			    A function $f$ is said to have a \emph{global minimum} (plural: minima) at $x_\min \in D_f$
				if all of the values of the function are greater than or equal 
				to the value of the function at the point $x=x_\min$:
				$f(x_\min) \leq f(x)$ for all $x \in D_f$.\\
		  \item[(d)] 
			  	A function $f$ is said to have a \textit{local minimum} at $x_\min \in D_f$ if, 
				\emph{in a neighbourhood} around $x_\min$, the values of the function are greater than or equal to
				the value of the function at the point $x=x_\min$. More precisely, $x_\min$ is a local minimum
			    if $x_\min$ is an element of an open interval $I$, 
			    such that $f(x_\min) \leq f(x)$ for all $x \in D_f \cap I$.\\ 
		  \item[(e)] 
		  		Local and global maxima and minima are also called extrema (singular: extremum).
		  \item[(f)] 
			  If $f$ has a global or local maximum at $x_\max$, the value $f(x_\max)$ is called the global \textit{maximum} or a local \textit{maximum} or more explicitely the \textit{maximal value} of $f$.  
			  If $f$ has a global or local minimum at $x_\min$, the value $f(x_\min)$ is called the global \textit{minimum} or a local \textit{minimum} or more explicitely the \textit{minimal value} of $f$.  
		\end{itemize}
	\end{definition}
}

 \begin{example}
    \lang{de}{
	\begin{tabs*}[\initialtab{1}\class{example}]
        
        \tab{Beispiel 1:}
        
        Das globale Minimum der Funktion $f(x) = x^2 + 2$ ist $y=2$. Dies ist der
			 Funktionswert an der Minimalstelle $x_{\min}=0$. Jeder andere Funktionswert
			 ist gr"o"ser, da er noch den Summanden $x^2 > 0$ enth"alt.\\
			 Die Funktion besitzt kein Maximum, da f"ur gro"se $x$-Werte auch die
			 Funktionswerte beliebig gro"s werden.
        
		\begin{center}
		\image{image11}
		\end{center}
	
        \tab{Beispiel 2:}
        
        
        Die Funktion $f(x) = \sqrt{x}$ besitzt das globale Minimum $y=0$,
			 das ist der Funktionswert an der Minimalstelle $x_{\min}=0$ dieser Funktion. Die
			 Funktion hat kein Maximum, da die Quadratwurzelwerte f"ur 
			 steigendes $x$ beliebig gro"s werden.
			 
		
		\begin{center}
		\image{1.2_Wurzel}
		\end{center}
	
        \tab{Beispiel 3:}
        
        Lineare Funktionen $f(x) = mx + b$ haben keine Extremwerte (weder
			 Maxima noch Minima), es sei denn, die Steigung ist $m=0$. Dann ist
			 die Funktion konstant, und $b$ ist zugleich Maximum und Minimum der
			 Funktion.
			 
			 
		\begin{center}
		\image{image12}
		\end{center} 
	\end{tabs*}}
    
    \lang{en}{
    \begin{tabs*}[\initialtab{1}\class{example}]
    \tab{Example 1:}
        The global minimum of the function $f(x) = x^2 + 2$ is $y=2$. This is the value
        of the function at the point $x=x_{\min}=0$. The function is greater than $2$ at every other $x$ value, because there is still a summand of $x^2>0$ in the function.\\
        The function does not have a maximum, since the bigger the $x$-value that is taken, the bigger the function gets.
            
        \begin{center}
		\image{image11}
		\end{center}
        
        \tab{Example 2:}
        
        The function $f(x) = \sqrt{x}$ has a global minimum of $y=0$, which is the value
		of the function at the point $x=x_{\min}=0$. The function does not have a maximum,
		since the bigger the argument $x$ becomes, the bigger the value of the function gets.
        
        \begin{center}
		\image{1.2_Wurzel}
		\end{center}
        
        \tab{Example 3:}
        
        Linear functions $f(x) = mx+b$ have no extreme values (neither minima nor maxima) unless the slope is equal to zero ($m=0$). If the
			slope is zero, the function is a constant, and $b$ is both the maximum and the minimum of the function.
        
        \begin{center}
		\image{image12}
		\end{center} 
	\end{tabs*}}
\end{example}

% \section{
% \lang{de}{Globale und lokale Extrema}
% \lang{en}{Global and Local Extrema}
% }
% \lang{de}{In den Anwendungen ist man oft an m\"{o}glichst gro{\ss}en
% oder kleinen Werten einer Funktion interessiert. Dabei unterscheidet
% man, ob zum Vergleich alle Werte der Variablen $x$ herangezogen werden
% (globale Extrema) oder nur benachbarte Werte von $x$ (lokale Extrema).}
% \lang{en}{In applications, we are often interested in the largest possible or smallest
% possible values of a function. Here, we differentiate whether the value is being compared to all of the values of the $x$ variable (a global extrema) or only neighbouring values of $x$
% (local extrema).}
% \begin{definition}\textit{
%   \lang{de}{Globale Extrema:}
%   \lang{en}{Global Extrema:}
%   }\\
% 	\lang{de}{Wenn an einer Stelle $x_1$ gilt \[f(x_1)\geq f(x)\]
% 	für alle $x$ im Definitionsbereich von $f$, dann heißt $x_1$ eine
% 	\textit{globale Maximalstelle} und der Funktionswert $f(x_1)$ ein
% 	\textit{globales Maximum} von $f$. }
% 	\lang{en}{If \[f(x_1)\geq f(x)\] at a given point $x_1$ for all $x$ in the domain of $f$, then $f$ is said to have a \textit{global maximum} at $x_1$. The \textit{global maximum}
% 	of $f$ is the value of the function at that point, $f(x_1)$.} 
%   	\\
%   	\lang{de}{Entsprechend definiert man eine \textit{globale Minimalstelle} $x_2$ und ein 
% 	\textit{globales Minimum} $f(x_2)$, falls $f(x_2) \leq f(x)$ für alle
% 	$x$ im Definitionsbereich von $f$ gilt.\\\\
% 	Globale Maxima und Minima werden zusammen als \textit{globale
% 	Extrema} bezeichnet.}
% 	\lang{en}{In a similar fashion, we say that $f$ has a \textit{global minimum} at $x_2$ if $f(x_2) \leq f(x)$ for all $x$ in the domain of $f$. The value $f(x_2)$ is 
% 	called the \textit{global minimum} of $f$.}
% \end{definition}
% 
% \lang{de}{Es h\"{a}ngt von der Funktion ab, ob sie ein globales Maximum und/oder
% Minimum hat. $f(x)=x^2,\, x\in\R$ hat ein globales Minimum Null bei
% $x=0$ aber kein globales Maximum, $x=0$ ist die einzige globale
% Minimalstelle. $g(x)=\cos(x),\, x\in\R$ hat das globale Maximum $1$ und
% das globale Minimum $-1$ mit unendlich vielen globalen Maximalstellen
% $x_k = k\cdot 2\pi,\, k\in\Z$ und Minimalstellen $x_l =\pi + l\cdot 2\pi,\,
% l\in\Z$. Die Funktion $h(x)=x, \, x\in\R$ hat keine globalen Extrema.
% 
% Bei lokalen Extrema lassen wir nur Werte der Variablen $x$ im
% Definitionsbereich von $f$ zu, deren Abstand von $x_1$ eine (kleine)
% Schranke $\delta >0$ nicht \"{u}berschreitet: $x_1 -\delta < x <x_1+\delta$.}
% \lang{en}{Whether a function has a global maximum and/or minimum depends on the function.
% The function $f(x)=x^2,\, x\in\R$ has a global minimum of zero at $x=0$, but does not have a global maximum. $x^2$ reaches its global minimum at only one point.
% The function $g(x)=\cos(x),\, x\in\R$ has a global minimum of $1$ and a global minimum of $-1$.
% The minima exist at points $x_l=\pi + l\cdot 2\pi,\,l\in\Z$ and the maxima exist at points $x_k = k\cdot 2\pi,\, k\in\Z$. Note as well that $\cos(x)$ has
% infinitely many minima and maxima.
% The function $h(x)=x, \, x\in\R$ has no global extrema.
% 
% At local extrema, we only allow values of the variable $x$ in the domain of $f$ that are in a neighbourhood
% of the local extrema. If $f$ has a local extremum at $x_1$, then we let $\delta > 0$ be a small 
% positive difference, and examine the function in the region $x_1 -\delta < x < x_1 + \delta$.}
% 
% \begin{definition}\textit{
%   \lang{de}{Lokale Extrema:}
%   \lang{en}{Local Extrema:}
%   }\\
% 	\lang{de}{Wenn es zu einer Stelle $x_1$ ein $\delta>0$ gibt, so dass f\"{u}r $x$ im
% 	Definitionsbereich von $f$ gilt 
% 	\[
% 	f(x_1)\geq f(x)\quad\text{wenn}\quad x_1 -\delta < x <x_1+\delta, 
% 	\]
% 	dann ist $x_1$ eine \textit{lokale Maximalstelle} und $f(x_1)$ ein
% 	\textit{lokales Maximum} von $f$. \\
% 	Entsprechend geht man f\"{u}r \textit{lokale Minimalstellen} und \textit{lokale
% 	Minima} vor.}
% 	\lang{en}{If at a given point $x_1$ there is a $\delta>0$ such that for all $x$
% 	in the domain of $f$
% 	\[
% 	f(x_1)\geq f(x)\quad\text{for}\quad x_1 -\delta <x<x_1+\delta,
% 	\]
% 	then $f$ has a \textit{local maximum} of $f(x_1)$ at $x_1$.
% 	\textit{Local minima} can be defined in a similar fashion.}
% 	
%   	\\
% 	\image{image-extrema1}
% \end{definition}
% 
% \lang{de}{Jedes globale Extremum ist auch ein lokales Extremum. Die Umkehrung
% gilt im Allgemeinen nicht, wie das Bild illustriert.
% 
% Wenn aus dem Zusammenhang klar ist, ob lokale oder globale Extrema
% gemeint sind, wird das Adjektiv lokal oder global oft weggelassen.}
% \lang{en}{Every global extremum is also a local extremum. The converse is in general not true, as the image above illustrates.
% 
% If it is clear from the context whether local or global extrema are being referred to, 
% the adjectives "local" and "global" are often left out.}

\section{
\lang{de}{Die notwendige Bedingung f\"{u}r lokale Extrema}
\lang{en}{The Necessary Condition for Local Extrema}
}

\begin{block}[info]
	\lang{de}{Wir setzen jetzt voraus, dass $f$ auf einem \textit{offenen Intervall}
	$I$ definiert ist (d.h. $I=\R$, $I=(a;b)$, $I= (a;\infty)$ oder
	$I=(-\infty;b)$) und dass $f$ auf $I$ \textit{differenzierbar} ist.\\
	Polynome, Sinus, Kosinus, Exponentialfunktion und viele weitere
	Funktionen erf\"{u}llen diese Voraussetzung, sie sind sogar alle zweimal
	differenzierbar.\\
	Damit steht uns die Ableitung als effizientes Hilfsmittel zur
	Verf\"{u}gung. F\"{u}r Intervalle mit Randpunkten (z.B. $[a;b]$) siehe Bemerkung am Ende des Unterabschnitts 
	\"{u}ber die 
	hinreichende Bedingung f\"{u}r lokale Extrema.}
	\lang{en}{We require in advance that $f$ is defined on an \textit{open interval} $I$, (i.e. $I=\R$, $I=(a,b)$, $I= (a,\infty)$ or
	$I=(-\infty,b)$) and that $f$ is \textit{differentiable} on $I$.\\
	Polynomials, sine, cosine, exponential functions, and many further functions fulfill these requirements, and the examples given
	are even twice differentiable.\\
	With the above conditions fulfilled, the derivative is a very convenient method for writing down the necessary conditions for local extrema.
	For intervals with boundary points (e.g. $[a,b]$) see Note at the end of the section below on the 
	sufficient conditions for local extrema.}
\end{block}

\lang{de}{Wenn $x_1\in I$ eine lokale Maximalstelle ist, dann gilt $f(x_1
+h)\leq f(x_1)$ f\"{u}r alle $h$ mit $-\delta < h < \delta$. Daraus
folgt, dass die Funktion links von der Maximalstelle ($h<0$) steigt und
rechts davon ($h>0$) fällt.}
\lang{en}{If $f$ has a local maximum at $x_1\in I$, then $f(x_1
+h)\leq f(x_1) $ for all $h$ where $-\delta < h < \delta$. From this, it follows
that for the function increases on the left ($h<0$) of this point and descreases on the right ($h>0$).}
% \[ \frac{f(x_1+h)-f(x_1)}{h} \geq 0, \quad h<0, \quad\text{\lang{de}{und} \lang{en}{and} }\]
% \[\frac{f(x_1+h)-f(x_1)}{h} \leq 0, \quad h>0.
% \phantom{\quad\text{und}} \]
% \lang{de}{Dann gilt auch:}
% \lang{en}{In addition:}
% \[\lim_{h\rightarrow 0-}\frac{f(x+h)-f(x)}{h}\geq 0\;\;\;\text{\lang{de}{und} \lang{en}{and} }\;\;\;\lim_{h\rightarrow 0+}\frac{f(x+h)-f(x)}{h}\leq 0.\]
% \lang{de}{Der Grenzwert $f'(x_1)$ muss daher sowohl $f'(x_1)\geq 0$
% als auch  $f'(x_1)\leq 0$ erf\"{u}llen, also ist $f'(x_1)=0$. Die Stelle
% $x_1$ ist eine station\"{a}re Stelle.\\
% Geometrisch sind die Sekanten links von der Maximalstelle
% steigend oder waagerecht und rechts davon fallend oder waagerecht.
\lang{de}{
Dann kann die Tangente an der Maximalstelle nur waagerecht sein.\\
An einer Minimalstelle ist es (mit vertauschten Rollen von rechts und
links) entsprechend. \\
Die Extremalstellen differenzierbarer Funktionen sind
\textit{station\"{a}re Stellen}, wie sie oben
definiert wurden.}
\lang{en}{
% The limit $f'(x_1)$ satisfies both $f'(x_1)\geq 0$ and $f'(x_1)\leq 0$, from which we can conclude that $f'(x_1)=0$. 
% The point $x_1$ is a critical point.\\
% Geometrically, the secants to the left of a maximum are increasing or horizontal, and to the right of that point are 
% either decreasing or horizontal. Similarly, the secants to the left of a minimum are decreasing or horizontal, and to the right of that point 
% are either increasing or horizontal.  
Hence at minima and maxima the tangent lines must be horizontal.\\
If a function $f$ has an extremum at a point $x$, then those points are \textit{critical points} of the function.}

\begin{theorem}\textit{
  \lang{de}{Notwendige Bedingung:}
  \lang{en}{Necessary Condition:}
  }\\
  \lang{de}{Sei $f$ auf $(a;b)$ differenzierbar. In einer lokalen Extremalstelle $x\in (a;b)$ von $f$ gilt }
  \lang{en}{Let $f$ be differentiable on $(a,b)$. In order for $f$ to have a local extremum at a point $x\in (a,b)$, at that point:}
  \[f'(x)=0.\]
\end{theorem}

\begin{example}
	\begin{enumerate}
		
	\item 
		\lang{de}{Die Funktion $f(x)=3x^2-12x+7$ mir ihrer ersten Ableitung $f'(x)=6x-12$ wird auf lokale Extremstellen untersucht. 
        Die notwendige Bedingung $f'(x)=0$ liefert eine stationäre Stelle $x=2$.
        }
    \item 
		\lang{de}{F\"{u}r die Funktion $f(x)=x^3$ mit $f'(x) = 3x^2$ ist $x=0$ eine
		station\"{a}re Stelle, aber keine Extremalstelle, denn $f(x)>0$ f\"{u}r alle
		$x>0$ und $f(x)<0$ für alle  $x<0$. Die station\"{a}ren Stellen - die
		Nullstellen der Ableitung - sind nur Kandidaten für Extremalstellen. Ob
		eine station\"{a}re Stelle eine Extremalstelle ist und ob sie dann eine
		Minimal- oder Maximalstelle ist, erfordert weitere Untersuchungen.}
        \lang{en}{If $f(x)=x^3$, then $f'(x)=3x^2$ and $x=0$ is a critical point, but there is no extremum there, since $f(x)>0$ for all $x>0$ and $f(x)<0$ for all $x<0$. 
		The critical points - the roots of the derivative - are only candidates for points at which extrema can exist. Whether or not
		a critical point is a point at which the function has an extremum, including whether it is a minimum or maximum, needs to be investigated further.
		}
   
      
		
	\end{enumerate}
\end{example}

 \begin{quickcheckcontainer}
\randomquickcheckpool{1}{1}
\begin{quickcheck}
		\field{rational}
		\type{input.number}
		\begin{variables}
			\randint{k}{1}{2}  % Vorzeichen des höchsten Koeff (-1)^k
			\randint{l}{1}{2}  % Vorzeichen der Extremstelle: (-1)^(l+1)
			\function[calculate]{l1}{2-l}  % Dirac-Funktionen
			\function[calculate]{l2}{l-1}
			
			\randint{a}{1}{3}
			\randint{b}{1}{4}
			\randint{c}{-4}{4}
		    \function[expand,normalize]{f}{(-1)^k*(a*x^4+(-1)^l*b*x^3+c)}
		    \number{n1}{0}
			\function[calculate]{n2}{(-1)^(l+1)*3*b/(4*a)}
			\function[calculate]{ns1}{l1*n1+l2*n2}
			\function[calculate]{ns2}{l1*n2+l2*n1}

		    \function[expand,normalize]{df}{(-1)^k*(4*a*x^3+3*(-1)^l*b*x^2)}
			\function[expand,normalize]{lin}{4*a*x+3*(-1)^l*b}
			\function[normalize]{x2}{(-1)^k*x^2}
			\function[calculate]{vz}{(-1)^(l+1)}
			\function[calculate]{zae}{3*b}
			\function[calculate]{nen}{4*a}

		\end{variables}
		
			\text{Bestimmen Sie mögliche Extremstellen der Funktion $f(x)=\var{f}$, indem Sie
			die Nullstellen ihrer Ableitungsfunktion bestimmen.\\
			Die möglichen Extremstellen sind (in aufsteigender Reihenfolge): \ansref und \ansref.}
		
		\begin{answer}
			\solution{ns1}
		\end{answer}
		\begin{answer}
			\solution{ns2}
		\end{answer}
		\explanation{Die Ableitung von $f$ ist $f'(x)=\var{df}=\var{x2}\cdot (\var{lin})$.\\
		Die Ableitung hat also die doppelte Nullstelle $x=0$ und die Nullstelle $x=\var{vz}\frac{\var{zae}}{\var{nen}}=\var{n2}$.		
		}
	\end{quickcheck}
\end{quickcheckcontainer}


\section{
\lang{de}{Hinreichende Bedingungen f\"{u}r lokale Extrema}
\lang{en}{Sufficient Conditions for Local Extrema}
}
\lang{de}{Oben haben wir gesehen, wie das Vorzeichen von $f'$
      das Monotonieverhalten von $f$ beeinflusst.  Angenommen, $f'$
      wechselt in $x_1$ das Vorzeichen von positiv zu negativ. Dann
      steigt $f$ links von $x_1$ und f\"{a}llt rechts von $x_1$.
      Folglich muss in $x_1$ ein lokales Maximum vorliegen. Ist die Ableitung links von $x_2$ negativ und rechts von $x_2$ positiv, so ist
      $x_2$ eine Minimalstelle. Es gen\"{u}gt wieder, nur benachbarte $x$
      mit $x_1 -\delta < x <x_1+\delta$ etc. für ein kleines
      $\delta>0$ zu betrachten.   }
\lang{en}{In the paragraph above we saw how the sign of $f'$ influences the monotonicity of $f$. Assume
	$f'$ changes sign at $x_1$ from positive to negative. Then $f$ increases to the left of $x_1$ and decreases to the right of $x_1$.
	It follows that $f$ needs to have a local maximum at $x_1$. If the derivative to the left of $x_2$ is negative and positive to the right
	of $x_2$, then $f$ has a minimum at $x_2$.  When doing this analysis, it suffices to only look at neighbouring $x$-values, i.e. $x_1 -\delta < x <x_1+\delta$
	for some small $\delta>0$. 
	}

\begin{theorem}\textit{
  \lang{de}{Hinreichende Bedingung f\"{u}r lokale Minima und Maxima:}
  \lang{en}{A Sufficient Condition for Local Minima and Maxima:}
  }\\
  \lang{de}{Sei $f$ auf dem offenen Intervall $I$ differenzierbar und
    $f'(x_0)=0$.
    Wenn f\"{u}r ein $\delta>0$ gilt:\\\\
    $f'(x)>0$ für $x_0-\delta < x < x_0$ und $f'(x)<0$ f\"{u}r $x_0<x<x_0
    +\delta$, dann ist $x_0$ eine \emph{Maximalstelle} von $f$,\\
    $f'(x)<0$ f\"{u}r $x_0-\delta < x < x_0$ und $f'(x)>0$ f\"{u}r $x_0<x<x_0
    +\delta$, dann ist $x_0$ eine \emph{Minimalstelle} von $f$.}
    \lang{en}{Let $f$ be differentiable on an open interval $I$, let $f'(x_0)=0$, and let $\delta>0$.
    If $f'(x)>0$ for $x_0-\delta < x < x_0$ and $f'(x)<0$ for $x_0<x<x_0
    +\delta$, then $f$ has a \emph{maximum} at $x_0$. \\
    If $f'(x)<0$ for $x_0-\delta < x < x_0$ and $f'(x)>0$ for $x_0<x<x_0
    +\delta$ then $f$ has a \emph{minimum} at $x_0$.}
\end{theorem}

\begin{example}
  	\lang{de}{Sei $f(x)=e^{-x^2}$. Dann ist $f'(x)=-2xe^{-x^2}$. Da $e^{-x^2}>0$ f\"{u}r alle $x\in\R$ ist, ist $0$ die einzige Nullstelle von $f'(x)$. 
	In $0$ wechselt $f'$ zudem das Vorzeichen von positiv zu negativ. Also liegt in $0$ eine Maximalstelle von $f$ vor.}
	\lang{en}{Let $f(x)=e^{-x^2}$, and hence $f'(x)=-2xe^{-x^2}$. Because $e^{-x^2}>0$ for all $x\in\R$, $x=0$ is the only root of $f'(x)$.
	At $x=0$ the sign of $f'$ changes from positive to negative, hence $f$ has a maximum at $x=0$.}
\end{example}


\lang{de}{Ein Wechsel der
      Ableitung nahe bei einer station\"{a}ren Stelle von positiven zu
      negativen Werten bedeutet, dass $f'(x)$ dort streng monoton
      fallend ist. Das ist sicher dann der Fall, wenn die Ableitung
      der Ableitung, also die zweite Ableitung von $f$ negativ ist.\\
      Wir setzen deshalb jetzt voraus, dass $f$ auf dem Intervall $I$
      nicht nur differenzierbar, sondern zweimal differenzierbar
      ist. F\"{u}r fast alle unserer Beispielfunktionen ist das
      erf\"{u}llt.}
      \lang{en}{A change in the derivative near a critical point from positive values to negative values means that 
      $f'(x)$ is strictly monotonically decreasing there. That is certainly the case when the derivative of the derivative (the second derivative) is negative.\\
      We now require that $f$ is not only differentiable on an interval $I$, 
      but twice differentiable. For almost all of the example functions, this condition is satisfied.}

\begin{theorem}\textit{
  \lang{de}{Hinreichende Bedingung f\"{u}r lokale Minima und Maxima:}
  \lang{en}{A Sufficient Condition for Local Minima and Maxima:}
  }\\
	\lang{de}{ Sei $f$ auf dem offenen Intervall $I$ zweimal differenzierbar und $f'(x_0)=0$. \\
	Ist $f''(x_0)<0$, so ist $x_0$ eine lokale Maximalstelle von $f$.\\
	Ist $f''(x_0)>0$, so ist $x_0$ eine lokale Minimalstelle von $f$.\\ }
	\lang{en}{Let $f$ be twice differentiable on the open interval $I$ and let $f'(x_0)=0$.\\
	If $f''(x_0)<0$ then $f$ has a local maximum at $x_0$.\\
	If $f''(x_0)>0$ then $f$ has a local minimum at $x_0$.\\ }
\end{theorem}
\\

\begin{tabs*}[\initialtab{0}\class{exercise}]
    \tab{
  \lang{de}{Erg\"{a}nzende Bemerkung}
  \lang{en}{Note}}
	  \lang{de}{Die Bedingungen in den letzten beiden Sätzen 7.5 und 7.7 sind
	\textit{hinreichend} für das Vorliegen von lokalen Maxima bzw.
	Minima, d.h.
	die Existenz lokaler Extrema folgt aus diesen Voraussetzungen. Diese
	Bedingungen sind jedoch \textit{nicht notwendig}: es gibt Funktionen, die lokale Extrema haben, obwohl
	eine der Voraussetzungen nicht erf\"{u}llt ist (z.B. die Funktion
	$f(x)=x^4$, die bei $x=0$ ein Minimum hat, obwohl $f''(0) =0$ nicht
	positiv ist).\\ Andererseits ist f\"{u}r ein Extremum einer in einem offenen
	Intervall (keine Randpunkte!)
	differenzierbaren Funktion \textit{notwendig}, dass die Ableitung eine
	Nullstelle hat (Satz 7.3). Wenn $f'(x_0)\neq 0$, ist $x_0$ sicher keine
	Extremstelle.}
	\lang{en}{The conditions in theorems 7.5 and 7.7 are \textit{sufficient} conditions for a function to have
	a local maximum or minimum, i.e. the existence of local extrema follows from these requirements. These conditions are
	however \textit{not necessary}: there are functions that have local extrema that do not satisfy these conditions. For example,
	the function $f(x)=x^4$ has a minimum at $x=0$, even though $f''(0)=0$ is not positive.\\
	On the other hand, it is \textit{necessary} for a function to have an extremum 
	at a given point $x_0$ in an open interval that the derivative of the function has a root there (Theorem 7.3). If $f'(x_0)\neq 0$, then the function
	definitely has no extremum at $x_0$. This result is not true for closed intervals!}
\end{tabs*}

\begin{example}
	\begin{enumerate}	
    \item 
    	\lang{de}{
	    	Sei $f(x)=-2x^3+3x^2$. Dann ist $f'(x)=-6x^2+6x=-6x(x-1)$. Station\"{a}re Stellen sind also $x=0$ und $x=1$. $f'$ besitzt einen Vorzeichenwechsel in 
			$0$ von negativ zu positiv und einen Vorzeichenwechsel in $1$ von positiv zu negativ. 
			Also liegt nach Satz 7.5 in $0$ ein lokales Minimum und in $1$ ein lokales Maximum vor. }
		\lang{en}{Let $f(x)=-2x^3+3x^2$, and hence $f'(x)=-6x^2+6x=-6x(x-1)$. Note that $f$ has critical points at $x=0$ and $x=1$. $f'$ changes sign at
		$x=0$ from negative to positive, and changes sign at $x=1$ from positive to negative. According to Theorem 7.5, $f$ has a local minimum at $x=0$
		and a local maximum at $x=1$.}

	\item 
		\lang{de}{Sei wieder $f(x)=-2x^3+3x^2$ mit $f'(x)=-6x^2+6x=-6x(x-1)$ und den station\"{a}ren Stellen $0$ und $1$. Nun soll Satz 7.7 zur Untersuchung auf Extremalstellen 
		angewendet werden. 
		Es ist $f''(x)=-12x+6$, damit 
		$f''(0)=6>0$ und $f''(1)=-6<0$. Nach Satz 7.7 liegt also in $0$ ein lokales Minimum und in $1$ ein lokales Maximum vor.}
		\lang{en}{Let $f(x)=-2x^3+3x^2$ again, where $f'(x)=-6x^2+6x=-6x(x-1)$ with critical points $x=0$ and $x=1$. This time, Theorem 7.7 will be used to investigate 
		the points at which $f$ has an extrema.
		Note that $f''(x)=-12x+6$, and with that: $f''(0)=6>0$ and $f''(1)=-6<0$. As per Theorem 7.7, $f$ has a local minimum at $x=0$ and a local maximum at $x=1$.}

	\item 
		\lang{de}{Sei $f(x)=1-(x+2)^4$. Dann ist $f'(x)=-4(x+2)^3$. Die einzige station\"{a}re Stelle ist $x=-2$. Offensichtlich wechselt $f'$ in $-2$ das Vorzeichen von 
		positiv zu negativ. Damit liegt in $-2$ ein lokales Maximum vor.\\
		Da $f''(x)=-12(x+2)^2$, ist $f''(-2)=0$. Mit Hilfe von Satz 7.7 kann man also bei dieser Funktion keine Aussage \"{u}ber Extremalstellen treffen.\\
		Eine M\"{o}glichkeit, sofort und ohne Differenzialrechnung die Extrema dieser Funktion zu bestimmen, ist die Betrachtung der Funktionswerte. Es ist $f(-2)=1$ 
		und $f(x)\leq 1$ f\"{u}r alle $x\in\R$. Damit liegt in $-2$ sogar ein globales Maximum vor.}
		\lang{en}{Let $f(x)=1-(x+2)^4$ and hence $f'(x)=-4(x+2)^3$. The only critical point is at $x=-2$. Obviously $f'$ changes sign at $x=-2$ from positive to negative, hence
		$f$ has a local maximum at $x=-2$.\\
		Because $f''(x)=-12(x+2)^2$, $f''(-2)=0$. According to Theorem $7.7$, nothing can be said about this function's extrema based off of this information.\\
		One way of finding the extrema of this function without using differential calculus is to consider the values of the function near the point $x=-2$. $f(-2)=1$
		and $f(x)\leq 1$ for all $x\in\R$, hence $f$ has a global maximum at $x=-2$.}
	\end{enumerate}
\end{example}

\lang{de}{In folgender Tabelle sind hinreichende Bedingungen f\"{u}r lokale Extrema zusammengefasst:}
\lang{en}{The following table has a summarized list of sufficient conditions for local extrema:}

\begin{block}[info]
\begin{align}
  f' \;\text{   
    \lang{de}{wechselt in}  
    \lang{en}{changes sign at} }
    \; x_1 \;\text{
    \lang{de}{das Vorzeichen von pos. zu neg.}
    \lang{en}{from positive to negative}
    }&\Rightarrow &\text{
    \lang{de}{lokales Maximum in}
    \lang{en}{local maximum at}
    }\; x_1\\
  f' \;\text{ 
   \lang{de}{wechselt in} 
    \lang{en}{changes sign at} }
    \; x_2 \;\text{
    \lang{de}{das Vorzeichen von neg. zu pos.}
    \lang{en}{from negative to positive}}
    &\Rightarrow &\text{
    \lang{de}{lokales Minimum in}
    \lang{en}{local minimum at}}
    \; x_2\\
\end{align}

  \lang{de}{oder alternativ}
  \lang{en}{or alternatively}
\begin{align}
  f'(x_1)=0 \;\text{  \lang{de}{und} \lang{en}{and} }\; f''(x_1)<0 &\Rightarrow& \text{ \lang{de}{lokales Maximum in} \lang{en}{local maximum at} }\; x_1\\
  f'(x_2)=0 \;\text{  \lang{de}{und} \lang{en}{and} }\; f''(x_2)>0 &\Rightarrow &\text{ \lang{de}{lokales Minimum in} \lang{en}{local minimum at} }\; x_2.\\
\end{align}
\end{block}

 \begin{quickcheckcontainer}
\randomquickcheckpool{1}{2}
\begin{quickcheck}
		\field{rational}
		\type{input.number}
		\begin{variables}
			\number{k}{1}
%			\randint{k}{1}{2}  % Vorzeichen des höchsten Koeff (-1)^k
			\randint{l}{1}{2}  % Vorzeichen der Extremstelle: (-1)^(l+1)
			\function[calculate]{l1}{2-l}  % Dirac-Funktionen
			\function[calculate]{l2}{l-1}
			
			\randint{a}{1}{3}
			\randint{b}{1}{4}
			\randint{c}{-4}{4}
		    \function[normalize]{f}{(-1)^k*(a*x^4+(-1)^l*b*x^3+c)}
		    \number{n1}{0}
			\function[calculate]{n2}{(-1)^(l+1)*3*b/(4*a)}
			
			\function[calculate]{ns1}{l1*n1+l2*n2}
			\function[calculate]{ns2}{l1*n2+l2*n1}
			
			\function[expand,normalize]{df}{(-1)^k*(4*a*x^3+3*(-1)^l*b*x^2)}
			\function[expand,normalize]{lin}{4*a*x+3*(-1)^l*b}
			\function[normalize]{x2}{(-1)^k*x^2}
% 			\function[calculate]{vz}{(-1)^(l+1)}
% 			\function[calculate]{zae}{3*b}
% 			\function[calculate]{nen}{4*a}
		\end{variables}
		
			\text{Bestimmen Sie mögliche Extremstellen der Funktion $f(x)=\var{f}$, indem Sie
			die Nullstellen ihrer Ableitungsfunktion bestimmen.\\
			Die möglichen Extremstellen sind (in aufsteigender Reihenfolge): \ansref und \ansref.\\
			Von diesen beiden Stellen, ist nur die Stelle \ansref tatsächlich eine Extremstelle.\\
			Hat $f$ dort (1) ein Maximum oder (2) ein Minimum? \ansref}
		
		\begin{answer}
			\solution{ns1}
		\end{answer}
		\begin{answer}
			\solution{ns2}
		\end{answer}
		\begin{answer}
			\solution{n2}
		\end{answer}
		\begin{answer}
			\solution{k}
		\end{answer}
		\explanation{Die Ableitung von $f$ ist $f'(x)=\var{df}=\var{x2}\cdot (\var{lin})$.\\
		Die Ableitung hat also die möglichen Extremstellen $x=0$ und $x=\var{n2}$.\\
		Bei $x=\var{n2}$ wechselt die Ableitung ihr Vorzeichen von positiv nach negativ, die
		Funktion $f$ hat dort also ein Maximum. Um $x=0$ bleibt das Vorzeichen der Ableitung
		das gleiche. Dort hat $f$ also kein Extremum.
		}
	\end{quickcheck}
	
	\begin{quickcheck}
		\field{rational}
		\type{input.number}
		\begin{variables}
			\number{k}{2}
%			\randint{k}{1}{2}  % Vorzeichen des höchsten Koeff (-1)^k
			\randint{l}{1}{2}  % Vorzeichen der Extremstelle: (-1)^(l+1)
			\function[calculate]{l1}{2-l}  % Dirac-Funktionen
			\function[calculate]{l2}{l-1}
			
			\randint{a}{1}{3}
			\randint{b}{1}{4}
			\randint{c}{-4}{4}
		    \function[normalize]{f}{(-1)^k*(a*x^4+(-1)^l*b*x^3+c)}
		    \number{n1}{0}
			\function[calculate]{n2}{(-1)^(l+1)*3*b/(4*a)}
			
			\function[calculate]{ns1}{l1*n1+l2*n2}
			\function[calculate]{ns2}{l1*n2+l2*n1}
			
			\function[expand,normalize]{df}{(-1)^k*(4*a*x^3+3*(-1)^l*b*x^2)}
			\function[expand,normalize]{lin}{4*a*x+3*(-1)^l*b}
			\function[normalize]{x2}{(-1)^k*x^2}
% 			\function[calculate]{vz}{(-1)^(l+1)}
% 			\function[calculate]{zae}{3*b}
% 			\function[calculate]{nen}{4*a}
		\end{variables}
		
			\text{Bestimmen Sie mögliche Extremstellen der Funktion $f(x)=\var{f}$, indem Sie
			die Nullstellen ihrer Ableitungsfunktion bestimmen.\\
			Die möglichen Extremstellen sind (in aufsteigender Reihenfolge): \ansref und \ansref.\\
			Von diesen beiden Stellen, ist nur die Stelle \ansref tatsächlich eine Extremstelle.\\
			Hat $f$ dort (1) ein Maximum oder (2) ein Minimum? \ansref}
		
		\begin{answer}
			\solution{ns1}
		\end{answer}
		\begin{answer}
			\solution{ns2}
		\end{answer}
		\begin{answer}
			\solution{n2}
		\end{answer}
		\begin{answer}
			\solution{k}
		\end{answer}
		\explanation{Die Ableitung von $f$ ist $f'(x)=\var{df}=\var{x2}\cdot (\var{lin})$.\\
		Die Ableitung hat also die möglichen Extremstellen $x=0$ und $x=\var{n2}$.\\
		Bei $x=\var{n2}$ wechselt die Ableitung ihr Vorzeichen von negativ nach positiv, die
		Funktion $f$ hat dort also ein Minimum. Um $x=0$ bleibt das Vorzeichen der Ableitung
		das gleiche. Dort hat $f$ also kein Extremum.				
		}
	\end{quickcheck}		
\end{quickcheckcontainer}



\begin{tabs*}[\initialtab{0}\class{exercise}]
\tab{
 \lang{de}{Erg\"{a}nzende Bemerkung zu Extrema auf Intervallen mit Randpunkten}
  \lang{en}{Expanded Note on Extrema on Intervals with Boundary Points}
  }
 \label{rand}
  \lang{de}{  Mitunter m\"{u}ssen auch Extrema einer Funktion $f$ bestimmt werden, die auf einem Intervall mit Randpunkten definiert ist. 
  Wie man dann vorzugehen hat, soll an folgendem Beispiel erl\"{a}utert werden.\\
  Sei $f(x)=x^2$ auf dem Intervall $[-1;2]$. Um alle lokalen und globalen Extremalstellen zu bestimmen, bestimmt man zun\"{a}chst alle lokalen 
  Extremalstellen. Ein lokales Extremum kann in $-1$, $2$ oder dem Intervall $(-1;2)$ liegen. Da das Intervall $(-1;2)$ offen ist, kann man dort alle 
  lokalen Extrema mit Hilfe der ersten Ableitung bestimmen. Aus $f'(x)=2x$ folgt die einzige station\"{a}re Stelle $x=0$. Aus $f''(x)=2$ folgt $f''(0)=2>0$, 
  und damit liegt in $0$ eine lokale Minimalstelle von $f$ auf dem Intervall $(-1;2)$ und damit auch auf dem Intervall $[-1;2]$ vor.\\
  Nun muss noch untersucht werden, ob an den Stellen $-1$ und $2$ lokale Extrema vorliegen. Da $f$ auf $[-1;0)$ streng monoton fallend ist, 
  liegt in $-1$ ein lokales Maximum vor. Da $f$ auf $(0;2]$ streng monoton wachsend ist, liegt in $2$ ein lokales Maximum vor. 
   Hier kann man also mit dem Monotonieverhalten von $f$ auf das Vorliegen von Extremalstellen schlie{\ss}en - 
   da $-1$ und $2$ jeweils an einem Randpunkt des Definitionsbereiches von $f$ liegen, kann man hier auch gar nicht mit den oben erlernten 
   Methoden argumentieren.\\
   Will man nun noch bestimmen, was das globale Maximum und was das globale Minimum ist, so bildet man die Funktionswerte an den entsprechenden 
   lokalen Extremalstellen 
   und vergleicht diese dann miteinander. Diese Vorgehensweise f\"{u}hrt hier zum Erfolg, da auf abgeschlossenen Intervallen die globalen Extrema stets 
   angenommen werden, und da zudem jedes globale Extremum ein lokales Extremum ist.\\
   Lokale Maximalstellen sind $-1$ und $2$. $f$ hat dort die Funktionswerte $f(-1)=1$ und $f(2)=4$. Da $f(2)=4>1=f(-1)$, liegt in $2$ das globale 
   Maximum $4$ vor.\\
   Da es nur die lokale Minimalstelle $0$ gibt, ist $0$ auch die globale Minimalstelle, und das globale Minimum ist $f(0)=0$.\\
   Ver\"{a}ndert man die Aufgabenstellung leicht, indem man die Funktion $f(x)=x^2$ auf dem Intervall $(-\infty;2]$ betrachtet, so muss man etwas anders 
   argumentieren. Die einzige station\"{a}re Stelle von $f$ auf $(-\infty;2)$ ist wieder die $0$, und dort liegt ein lokales Minimum vor. Man macht sich leicht 
   klar, dass dieses auch das globale Minimum sein muss, denn $f(x)$ nimmt stets Werte $\geq 0$ an, und $f(0)=0$.\\
   In $2$ liegt wieder ein lokales Maximum vor, da die Funktion auf $(0;2]$ streng monoton steigt. Dies ist jedoch kein globales Maximum, da z.B. $f(-3)=9>4
   =f(2)$ ist. Die Funktion hat auf diesem Definitionsbereich kein globales Maximum, da $\lim_{x\rightarrow-\infty}f(x)=\infty$. }
   \lang{en}{Sometimes the extrema of a function $f$ need to be determined on an interval that includes its boundary points.
   How to proceed in such a case will be examined below.\\
   Let $f(x)=x^2$ on the interval $[-1,2]$. In order to find all the local and global extrema, we first need to find all of the local extrema.
   A local extremum can be at $-1$, $2$, or on the interval $(-1,2)$. Because the interval $(-1,2)$ is open, the local extrema on that interval
   can be found with the help of the first derivative. From the fact that $f'(x)=2x$, the only critical point is $x=0$. Because $f''(x)=2$, it follows
   that $f''(0)=2>0$, and hence there is a local minimum of $f$ at $x=0$. Now, we need to determine whether $f$ has
   local extrema at the points $x=-1$ and $x=2$. Because $f$ is strictly monotonically decreasing on $[-1,0)$, there is a local maximum at $-1$. Because
   $f$ is strictly monotonically increasing on $(0,2]$, $f$ has a local maximum at $x=2$. Here, we can also use the monotonicity of $f$
   to prove the existence of local extrema - because $x=-1$ and $x=2$ are the boundary points of the domain of $f$, we cannot argue using the above method.\\
   If we now want to find what the global maximum and global minimum are, then we compare the value of the function at each of the corresponding
   local extrema. This method is successful only because on closed intervals, the global extrema will always exist; every global extremum is also a local extremum.\\
   The points at $x=-1$ and $x=2$ are both local maxima; $f$ takes on the values $f(-1)=1$ and $f(2)=4$ there. Because $f(2)=4>1=f(-1)$, $f$ has its global maximum of $4$
   at $x=2$.\\
   Because there is only one local minimum, namely at $x=0$, this is also the point where $f$ has its global minimum, and the global minimum is $f(0)=0$.\\
   If we change the original problem slightly by considering the function $f(x)=x^2$ on the interval $(-\infty,2]$, then we need to argue these points differently.
   The only critical point of $f$ on $(-\infty,2)$ is $x=0$, and there is a local minimum there. It can easily be seen that this is also the global minimum, because
   $f(x)$ is always $\geq 0$, and $f(0)=0$.\\
   At $x=2$ there is a local maximum, because the function is strictly monotonically increasing on $(0,2]$. This point is not the global maximum; take for example
   the value of the function at $x=-3$: $f(-3)=9>4=f(2)$. The function has no global maximum on this new domain, since $\lim_{x\rightarrow-\infty}f(x)=\infty$.}
\end{tabs*}


% \begin{quickcheckcontainer}
% \randomquickcheckpool{1}{1}
% \begin{quickcheck}
% 		\field{rational}
% 		\type{input.number}
% 		\begin{variables}
% 			\randint[Z]{a}{-5}{5}
% 			\randint[Z]{b}{1}{4}
% 			\randint{c}{-4}{4}
% 			\randint[Z]{d}{1}{4}
% 		    \function[normalize]{f}{(a/b)*x+c/d}
% 			\function[calculate]{ns}{-(c*b)/(a*d)}
% 		\end{variables}
% 		
% 			\text{Die Nullstelle der linearen Funktion $f(x)=\var{f}$ ist \ansref.}
% 		
% 		\begin{answer}
% 			\solution{ns}
% 		\end{answer}
% 	\end{quickcheck}
% \end{quickcheckcontainer}
% 
% 
% 	\begin{genericGWTVisualization}[550][1000]{mathlet1}
% 		\begin{variables}
% 			\randint{randomA}{1}{2}
% 
% 			\point[editable]{P}{rational}{var(randomA),var(randomA)}
% 		\end{variables}
% 		\color{P}{BLUE}
% 		\label{P}{$\textcolor{BLUE}{P}$}
% 
% 		\begin{canvas}
% 			\plotSize{300}
% 			\plotLeft{-3}
% 			\plotRight{3}
% 			\plot[coordinateSystem]{P}
% 		\end{canvas}
% 		\text{Der Punkt hat die Koordinaten $(\var{P}[x],\var{P}[y])$.}
% 	    	\end{genericGWTVisualization}

\end{visualizationwrapper}


\end{content}

