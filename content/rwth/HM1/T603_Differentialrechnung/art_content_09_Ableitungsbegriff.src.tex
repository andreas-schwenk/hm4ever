%$Id:  $
\documentclass{mumie.article}
%$Id$
\begin{metainfo}
  \name{
    \lang{de}{Der Ableitungsbegriff}
    \lang{en}{}
  }
  \begin{description} 
 This work is licensed under the Creative Commons License Attribution 4.0 International (CC-BY 4.0)   
 https://creativecommons.org/licenses/by/4.0/legalcode 

    \lang{de}{Beschreibung}
    \lang{en}{}
  \end{description}
  \begin{components}
    \component{generic_image}{content/rwth/HM1/images/g_tkz_T603_AbsoluteValue.meta.xml}{T603_AbsoluteValue}
    \component{generic_image}{content/rwth/HM1/images/g_tkz_T603_Tangent_B.meta.xml}{T603_Tangent_B}
    \component{generic_image}{content/rwth/HM1/images/g_tkz_T603_Tangent_A.meta.xml}{T603_Tangent_A}
    \component{js_lib}{system/media/mathlets/GWTGenericVisualization.meta.xml}{mathlet}
  \end{components}
  \begin{links}
    \link{generic_article}{content/rwth/HM1/T601_GrundlagenWiWi/g_art_content_04_Funktionsbegriff.meta.xml}{content_04_Funktionsbegriff}
    \link{generic_article}{content/rwth/HM1/T601_GrundlagenWiWi/g_art_content_03_Folgen_Reihen.meta.xml}{content_03_Folgen_Reihen}
    \link{generic_article}{content/rwth/HM1/T601_GrundlagenWiWi/g_art_content_05_Wichtige_Funktionen.meta.xml}{content_05_Wichtige_Funktionen}
    \link{generic_article}{content/rwth/HM1/T603_Differentialrechnung/g_art_content_09_Ableitungsbegriff.meta.xml}{content_09_Ableitungsbegriff}
    \link{generic_article}{content/rwth/HM1/T301_Differenzierbarkeit/g_art_content_01_differenzenquotient.meta.xml}{content_01_differenzenquotient}
    \link{generic_article}{content/rwth/HM1/T210_Stetigkeit/g_art_content_31_grenzwerte_von_funktionen.meta.xml}{grenzw-funk}
    \link{generic_article}{content/rwth/HM1/T209_Potenzreihen/g_art_content_28_exponentialreihe.meta.xml}{expreihe}
  \end{links}
  \creategeneric
\end{metainfo}
\begin{content}
\usepackage{mumie.ombplus}
\ombchapter{3}
\ombarticle{1}
\usepackage{mumie.genericvisualization}

\begin{visualizationwrapper}

\title{
\lang{de}{Der Ableitungsbegriff}
\lang{en}{The derivative}
}
\begin{block}[annotation]
  
  
\end{block}
\begin{block}[annotation]
Im Ticket-System: \href{https://team.mumie.net/issues/21578}{Ticket 21578}\end{block}

\begin{block}[info-box]
\tableofcontents
\end{block}

\section{
	\lang{de}{Definition der Ableitung}
	\lang{en}{Definition of the derivative}
}

\lang{de}{Sehr häufig haben wir es mit der Situation zu tun, dass wir zwei Größen betrachten und der Wert der 
einen Größe von der anderen abhängt. Wie wir schon in Kapitel 1 gesehen haben, können wir solche 
Zusammenhänge gut durch Funktionen $y = f(x)$ darstellen. Besonders wichtig ist häufig die Frage, 
wie sich der Wert von Größe $y$ ändert, ob der Wert beispielsweise in der Tendenz fällt oder wächst.}
\lang{en}{
We often find ourselves in the situation where we have to consider two variables, one of whose
values depends on the other's. As we saw in Chapter 1, relationships of this type can be
modeled by functions $y = f(x)$. Often, a particularly important question is how the value of
$y$ changes; for example, whether it tends to increase or decrease.
}

\lang{de}{Solche Änderungsraten kennen Sie aus den Naturwissenschaften (z.\,B. Beschleunigung als Änderung der Geschwindigkeit), 
aber auch in den Wirtschaftswissenschaften sind sie unerlässlich, um ökonomische 
Sachverhalte korrekt zu verstehen. }
\lang{en}{
The concept of rate of change is well-known from the natural sciences (e.g. acceleration
as the change of velocity), but it is also indispensible for understanding
economical issues correctly.
}

\begin{example}[\lang{de}{Grenzkosten} \lang{en}{Marginal cost}]\label{ex:grenzkosten}
\lang{de}{Wir bezeichnen mit $K(x)$ die Produktionskosten für $x$ Einheiten eines Gutes und fassen $K$ als Funktion auf. 
Wir nehmen an, dass aktuell $x_0$ Einheiten produziert werden und dass die Produktion um $h$ Einheiten erhöht werden soll. 
Der Kostenzuwachs ist dann $K(x_0+h) - K(x_0)$, also die Kosten für die Produktion von $x_0+h$ Einheiten abzüglich der 
bisherigen Kosten. Diese Mehrkosten verteilen sich auf $h$ Einheiten, also 
ist der Kostenzuwachs pro zusätzlich produzierter Einheit }
\lang{en}{
Let $K(x)$ be the cost of producing $x$ unit of a good. We view $K$ as a function. Suppose
$x_0$ units are currently being produced, but the production then increases by $h$ units.
The increase in cost is then $K(x_0+h)-K(x_0)$, that is, the cost of producing $x_0+h$ units
minus the cost of producing $x_0$ units. These additional costs are distributed over $h$ units,
so the increase in cost per additionally produced unit is
}
\[
\frac{K(x_0+h)-K(x_0)}{h}.
\]
\lang{de}{Bei den \emph{Grenzkosten} handelt es sich um den Kostenzuwachs, der durch die Mehrproduktion einer Einheit des Gutes entsteht. Ist $h$ eine 
\glqq große\grqq Zahl, etwa $h=1000$, dann können wir über die Grenzkosten nicht viel aussagen. Es ist z.\,B. möglich, dass die Produktion 
von $700$ weiteren Einheiten recht kostengünstig möglich war, für die nächsten $300$ Einheiten jedoch teure Maschinen angeschafft werden mussten. 
Der oben berechnete Kostenzuwachs pro zusätzlich produzierter Einheit ist lediglich ein Mittelwert, die \emph{mittlere Änderungsrate} der 
Kostenfunktion. }
\lang{en}{
The \emph{marginal cost} is the increase of cost that arises by producing an additional unit
of the good. If $h$ is "large", say $h=1000$, then the quotient above does not say much
about the marginal cost. For example, it is possible that producing an additional 700 units
is very cost-efficient but the remaining 300 units require expensive machines to be purchased.
The cost growth calculated above is merely an average,
the \emph{average rate of change} of the cost function.
}

\lang{de}{Um die Grenzkosten zu beziffern, sollte $h$ also möglichst klein sein. Hohe Grenzkosten bei $x_0$ äußern sich dadurch, dass die Kostenfunktion 
$K$ bei $x_0$ stark ansteigt.}
\lang{en}{
To quantify the marginal cost, we want $h$ to be as small as possible.
A high marginal cost in $x_0$ is reflected in a strong increase of the cost function $K$ in $x_0$.
}
\end{example}


\lang{de}{Unser erstes Ziel besteht darin, die Steigung einer Funktion an einer bestimmten Stelle messbar zu machen.
Den Begriff \glqq Steigung\grqq im mathematischen Zusammenhang kennen wir bisher
nur von \ref[content_05_Wichtige_Funktionen][Geraden]{sec:linear}, daher werden wir Geraden als Hilfsmittel benutzen. }
\lang{en}{
Our first goal will be to measure the slope of a function at a given point.
The notion of \emph{slope} in a mathematical context has only been introduced for
\ref[content_05_Wichtige_Funktionen][lines]{sec:linear}, so we will reduce our problem to lines.
}

\lang{de}{Eine \emph{Tangente} an einen Funktionsgraphen ist eine Gerade, die sich an den Graphen \glqq anschmiegt\grqq, man sagt \emph{in einem Punkt berührt}. Die Steigung 
der Tangente werden wir als Steigung des Funktionsgraphen verwenden. Das Problem dabei ist, dass wir nicht wissen, wie man die Tangente 
berechnen kann, da wir dafür \emph{zwei} Punkte bräuchten. Die Lösung des Problems besteht darin, dass man die Tangente durch andere Geraden annähert, die man immerhin 
berechnen könnte. }
\lang{en}{
A \emph{tangent} to the graph of a function is a line that "hugs" the graph;
we say that it \emph{touches it in only one point}. The slope of the tangent will be taken
to be the slope of the graph. The problem here is that we do not know how to compute the tangent,
since we need \emph{two} points to determine it. The solution is to approximate the
tangent by other lines that can be computed.
}

\lang{de}{Um rechnerisch zu verstehen, was das bedeutet, betrachten wir zunächst die Gerade durch die beiden Punkte $P_0=(x_0;f(x_0))$ und 
$Q = (x_0+h; f(x_0+h))$. Eine solche Gerade, die durch (mindestens) zwei Punkte einer Kurve verläuft, nennt man auch \emph{Sekante}. 
}
\lang{en}{
To understand what this means numerically, first consider the line through the two points
$P_0 = (x_0, f(x_0))$ and $Q = (x_0+h, f(x_0+h))$. Any such line that goes through (at least)
two points of the curve is called a \emph{secant}.
}

\begin{center}
\image{T603_Tangent_A}
\end{center}

\lang{de}{Die Steigung der Sekante durch die beiden Punkte können wir (wie im \ref[content_05_Wichtige_Funktionen][ersten Teil des Kurses]{rule:geradengleichung}) berechnen: Sie ist
\[m = \frac{f(x_0+h)-f(x_0)}{(x_0+h)-x_0}=\frac{f(x_0+h)-f(x_0)}{h}.\]
Weil $m$ ein Quotient (d.\,h. ein Bruch) von Differenzen ist, hat sich für die Formel von $m$ auch die Bezeichnung \emph{Differenzenquotient} 
durchgesetzt. Die Formel für die Steigung ist übrigens die gleiche Formel, die wir in Beispiel \ref{ex:grenzkosten} zur Näherung der Grenzkosten gefunden haben.}
\lang{en}{
The slope of the secant line through the two points can be calculated (as in the \ref[content_05_Wichtige_Funktionen][first part of the course]{rule:geradengleichung}):
it is
\[m = \frac{f(x_0+h)-f(x_0)}{(x_0+h)-x_0}=\frac{f(x_0+h)-f(x_0)}{h}.\]
Since $m$ is a quotient (fraction) of differences, the formula for $m$ is commonly called a
\emph{difference quotient}. Incidentally, the formula for the slope is the same formula
that we found in Example \ref{ex:grenzkosten} for approximating marginal cost.
}

\lang{de}{Wir bewegen den Punkt $Q$ nun langsam auf den Punkt $P_0$ zu. Das erreichen wir, indem wir für $h$ verschiedene Werte $h_1, h_2, \ldots$ wählen, die immer näher an $0$ liegen. 
Im folgenden Bild sehen wir, dass sich dann die jeweiligen
Sekanten immer weiter der Tangente an den Graphen n\"{a}hern. Auch die Steigungen der Sekanten n\"{a}hern sich der Steigung der Tangente an.}
\lang{en}{
Now, we slowly move the point $Q$ towards the point $P_0$. This is done by substituting values
$h_1,h_2,...$ for $h$ that get closer and closer to zero. The image below shows how the
secant lines then approach the tangent to the graph more and more closely.
The slopes of the secant lines also approach the slope of the tangent.
}

\begin{center}
\image{T603_Tangent_B}
\end{center}

\lang{de}{Wenn diese Annäherung beliebig gut geht, die Sekantensteigung sich für $h \to 0$ also immer besser der Tangentensteigung annähert, werden wir die 
Tangentensteigung als \glqq Ableitung\grqq bezeichnen. Mathematisch präzise:}
\lang{en}{
If these approximations become arbitrarily close, such that the slope of the secant
approximates the slope of the tangent arbitrarily well, then we call the slope of the
tangent the "derivative". Stated rigorously:
}

\begin{definition}
\lang{de}{Es sei $I$ ein offenes Intervall, $f \colon I \to \R$ eine Funktion und $x_0 \in I$.}
\lang{en}{Let $I$ be an open interval, $f\colon I \to \R$ a function and $x_0 \in I$.}
\begin{itemize}
\item \lang{de}{Wenn der folgende \ref[content_03_Folgen_Reihen][Grenzwert]{def:folgenkonvergent} existiert ($\pm \infty$ nicht erlaubt!), 
definieren wir 
\[f'(x_0):= \lim_{h\to 0} \frac{f(x_0+h)-f(x_0)}{h} \quad \left( =\lim_{x\to x_0}\frac{f(x)-f(x_0)}{x-x_0} \right)\]
und nennen den Wert $f'(x_0)$ die \notion{Ableitung von $f$ in $x_0$}. Wir sagen auch, dass die Funktion $f$ 
\notion{in $x_0$ differenzierbar} ist. }
\lang{en}{
If the \ref[content_03_Folgen_Reihen][limit]{def:folgenkonvergent} below exists ($\pm \infty$ are not allowed),
then we define
\[f'(x_0):= \lim_{h\to 0} \frac{f(x_0+h)-f(x_0)}{h} \quad \left( =\lim_{x\to x_0}\frac{f(x)-f(x_0)}{x-x_0} \right)\]
and call the value $f'(x_0)$ the \notion{derivative of $f$ in $x_0$}.
We say that the function $f$ is \notion{differentiable in $x_0$}.
}

\item \lang{de}{Ist $f$ in jedem $x \in I$ differenzierbar, so nennen wir $f$ \notion{differenzierbar} und bezeichnen mit  
$f'$ die Funktion, die an $x \in I$ den Wert $f'(x)$ besitzt. Wir nennen $f'$ 
\notion{Ableitungsfunktion} oder auch \notion{erste Ableitung} von $f$.}
\lang{en}{
If $f$ is differentiable in every point $x \in I$, then we call $f$ \notion{differentiable}
and write $f'$ for the function whose value in $x \in I$ is $f'(x)$.
WE call $f'$ the \notion{derivative}, or \notion{first derivative}, of $f$.
}
\end{itemize}
\end{definition}
\begin{remark}
\lang{de}{Eine andere Schreibweise für die Ableitungsfunktion $f'(x)$ ist $\frac{df}{dx}$. 
Ein Vorteil dieser Schreibweise ist, dass der Name der Variable immer klar ist. 
Für $f'(x_0)$ wird  $\frac{df}{dx}(x_0)$ oder auch $\frac{df}{dx}|_{x=x_0}$ geschrieben. }
\lang{en}{
Another way to write the derivative $f'(x)$ is $\frac{df}{dx}$. One advantage of this
notation is that the name of the variable is always clear.
To write $f'(x_0)$, one uses $\frac{df}{dx}(x_0)$ or $\frac{df}{dx}|_{x=x_0}$.
}
\end{remark}
\lang{de}{Mit der Deutung der Ableitung als Steigung der Tangente an den Funktionsgraphen können wir auch direkt die
ganze Tangentengleichung bestimmen: }
\lang{en}{
Using the interpretation of the derivative as the slope of the tangent to the graph,
we can compute the entire equation of the tangent line:
}
 \begin{theorem}
\lang{de}{Die Tangente an den Graphen einer differenzierbaren Funktion $f$ im Punkt $(x_0;f(x_0))$ ist
gegeben durch 
\[T(x)=f'(x_0)\cdot(x-x_0)+f(x_0).\]
}
\lang{en}{
The tangent line to the graph of a differentiable function $f$ at the point $(x_0;f(x_0))$
is given by \[T(x)=f'(x_0)\cdot(x-x_0)+f(x_0).\]
}
\end{theorem}
 
\begin{quickcheck}
		\field{rational}
		\type{input.number}
		\begin{variables}
			\randint[Z]{a}{-2}{-1}
			\randint[Z]{b}{-1}{1}
			\randint{c}{4}{4}
			\function[normalize]{f}{x^3+a*x^2+b*x+c}
			\randint[Z]{p}{1}{2}
			\function[calculate]{x0}{p}
			\function[calculate]{m}{3*x0^2+2*a*x0+b}
            \function[calculate]{fx0}{x0^3+a*x0^2+b*x0+c}
            \function[calculate]{n}{fx0-x0*m}
		\end{variables}
        
      \lang{de}{
			\text{     
            In der folgenden Visualisierung können Sie sich für ganzrationale Funktionen vom Grad 3 die Werte der
            Ableitung an einer Stelle anzeigen lassen.  }
            }
      \lang{en}{
      \text{
            The visualization below shows the value of the derivative at a point for
            polynomials of degree 3.
      }
      }
            
        \begin{genericGWTVisualization}[550][1000]{mathlet}
		\begin{variables}
			\number[editable]{a}{rational}{1}
			\number[editable]{b}{rational}{1}
			\number[editable]{c}{rational}{1}
			\number[editable]{x0}{rational}{0}
			\function{f}{rational}{x^3+var(a)*x^2+var(b)*x+var(c)}
			\number{y0}{rational}{var(x0)^3+var(a)*var(x0)^2+var(b)*var(x0)+var(c)}
			\number{m}{rational}{3*var(x0)^2+2*var(a)*var(x0)+var(b)}
			\number{q0}{rational}{var(y0)-var(m)*var(x0)}
			\point{P}{rational}{var(x0),var(y0)}
			\point{Q}{rational}{var(x0)+1,var(y0)}
			\point{R}{rational}{var(x0)+1,var(y0)+var(m)}
			\line{T}{rational}{var(P),var(R)}
			\segment{s1}{rational}{var(P),var(Q)}
			\segment{s2}{rational}{var(Q),var(R)}
		\end{variables}
% 		\color{P}{BLUE}
% 		\label{P}{$\textcolor{BLUE}{P}$}
		\color{s1}{#00CC00}
		\color{s2}{#00CC00}
		\color{T}{#0066CC}
%		\label{s1}{$\textcolor{BLUE}{1}$}
%		\label{s2}{$\textcolor{BLUE}{\var{m}}$}
		\begin{canvas}
			\plotSize{300}
			\plotLeft{-1}
			\plotRight{3}
			\plot[coordinateSystem]{f,T,P, s1,s2}
		\end{canvas}
		\lang{de}{\text{Die Funktion  $f$ mit $f(x)=x^3+\var{a}\cdot x^2+\var{b}\cdot x+\var{c}$ hat an der Stelle $x_0=\var{x0}$
		die \textcolor{#0066CC}{Ableitung} \textcolor{BLUE}{$f'(x_0)=\var{m}$}.}
    }
    \lang{en}{\text{The function $f$ given by $f(x)=x^3+\var{a}\cdot x^2+\var{b}\cdot x+\var{c}$
    has\textcolor{#0066CC}{derivative} \textcolor{BLUE}{$f'(x_0)=\var{m}$} in the point $x_0=\var{x0}$.
    }}
	    	\end{genericGWTVisualization}             
            
     \lang{de}{\text{       Wir betrachten nun die Funktion $f(x)=\var{f}$ an der Stelle $x_0=\var{x0}$. 
			Bestimmen Sie mit Hilfe der Visualisierung die Geradengleichung der Tangente an den Graphen von $f$ an der Stelle $x_0 = \var{x0}$.\\            
			$T(x)= $ \ansref  $ \cdot x +  $ \ansref . }}
     \lang{en}{\text{ Consider the function $f(x)=\var{f}$.
     Using the visualization, determine the equation of the tangent line to the graph of $f$
     in the point $x_0 = \var{x0}$.}}
		
		\begin{answer}
			\solution{m}
		\end{answer}
        \begin{answer}
			\solution{n}
		\end{answer}
        
		\lang{de}{\explanation{Der Ableitungswert $f'(\var{x0}) = \var{m}$ ist die Steigung der gesuchten Tangente. Außerdem ist $f(\var{x0}) = \var{fx0}$. Eingesetzt in die Gleichung für die Tangente 
        ergibt sich 
        \begin{align*}
        T(x) &= \var{m} \cdot (x - \var{x0}) + \var{fx0} =  \var{m} \cdot x + \left(\var{m} \cdot (-\var{x0}) + \var{fx0}\right) \\
           & = \var{m} \cdot x + \var{n} .
        \end{align*}
        }}
    \lang{en}{\explanation{The derivative $f'(\var{x0}) = \var{m}$ is the slope of the
    tangent that we are looking for. Also, $f(\var{x0}) = \var{fx0}$. Substituting this into the equation of the tangent line yields
        \begin{align*}
        T(x) &= \var{m} \cdot (x - \var{x0}) + \var{fx0} =  \var{m} \cdot x + \left(\var{m} \cdot (-\var{x0}) + \var{fx0}\right) \\
           & = \var{m} \cdot x + \var{n} .
        \end{align*}
        }}
	\end{quickcheck}
%%%%%%%%%%%%%%%%%%%%%%%%%%%%%%%%%%%%

\section{
	\lang{de}{Einseitige Differenzierbarkeit}
	\lang{en}{One-sided differentiability}
    }

\lang{de}{In der Definition der Ableitung bedeutet $h \to 0$, dass $h$ sich auf jede erdenkliche Weise dem Wert $0$ nähern können muss. Das heißt auch,
dass $h$ negativ sein kann. Im \ref[content_04_Funktionsbegriff][Kapitel zum Funktionsbegriff]{def:funktionsgrenzwert} haben wir auch die Schreibweisen $h \searrow 0$ und $h \nearrow 0$
kennengelernt, die ausdrücken, dass sich der Wert $h$ nur von oben bzw. unten dem Wert $0$ annähert. Damit erhalten wir folgende Variante der
Ableitung:}
\lang{en}{
The limit $h \to 0$ in the definition of the derivative means that $h$ is allowed to approach $0$ in
any possible way. The \ref[content_04_Funktionsbegriff][chapter on functions]{def:funktionsgrenzwert} introduced
the notation $h \searrow 0$ and $h \nearrow 0$, which express the fact that $h$ can only approach $0$ from above
or from below, respectively. This leads to the following variant of the derivative:
}

\begin{definition}
\lang{de}{Es sei $I$ ein offenes Intervall, $x_0 \in I$ und $f \colon I\to \R$ eine Funktion.}
\lang{en}{Let $I$ be an open interval, $x_0 \in I$ and let $f \colon I\to \R$ be a function.}
\begin{itemize}
\item \lang{de}{Die \notion{rechtsseitige Ableitung} von $f$ an $x_0$ ist der Grenzwert 
\[
f^'_r(x_0) := \lim_{h\searrow 0} \frac{f(x_0+h)-f(x_0)}{h} , 
\]
vorausgesetzt dieser Grenzwert existiert. Ist dies der Fall, dann heißt $f$ \notion{rechtsseitig differenzierbar} an $x_0$.}
\lang{en}{The \notion{right-hand derivative} of $f$ in $x_0$ is the limit
\[
f^'_r(x_0) := \lim_{h\searrow 0} \frac{f(x_0+h)-f(x_0)}{h} , 
\]
if this limit exists. In this case, $f$ is called \notion{right differentiable} in $x_0$.}

\item \lang{de}{Entsprechend ist die \notion{linksseitige Ableitung} von $f$ an $x_0$ der Grenzwert 
\[
f^'_l(x_0) := \lim_{h\nearrow 0} \frac{f(x_0+h)-f(x_0)}{h} , 
\]
falls dieser Grenzert existiert. Entsprechend heißt dann $f$ \notion{linksseitig differenzierbar} an $x_0$.}
\lang{en}{
Similarly, the \notion{left-hand derivative} of $f$ in $x_0$ is the limit
\[
f^'_r(x_0) := \lim_{h\searrow 0} \frac{f(x_0+h)-f(x_0)}{h} , 
\]
if this limit exists. In this case, $f$ is called \notion{left differentiable} in $x_0$.}
\end{itemize}
\end{definition}
\lang{de}{Es kann hilfreich sein, zunächst die rechts- und linksseitige Ableitung auszurechnen, etwa wenn die Funktion $f$ für  $x < x_0$ 
anders definiert ist als für $x>x_0$.  Mit Hilfe von $f^'_r(x_0)$ und $f^'_l(x_0)$ kann man entscheiden, ob $f$ überhaupt eine Ableitung 
besitzt:}
\lang{en}{
It is sometimes useful to compute the right- and left-hand derivatives separately; for example,
if the function $f$ is defined differently for $x < x_0$ than for $x > x_0$.
$f^'_r(x_0)$ and $f^'_l(x_0)$ can then be used to decide whether $f$ has a derivative at all:
}

\begin{theorem}
\lang{de}{Es sei $f$ eine an $x_0$ links- und rechtsseitig differenzierbare Funktion. 

Die Funktion $f$ ist genau dann an $x_0$ differenzierbar, wenn $f^'_l(x_0) = f^'_r(x_0)$ gilt. In diesem Fall ist  $f'(x_0) = f^'_l(x_0) = f^'_r(x_0)$.}

\lang{en}{
Let $f$ be a function that is both left and right differentiable in $x_0$.

$f$ is differentiable in $x_0$ if and only if $f^'_l(x_0) = f^'_r(x_0)$. In that case, $f'(x_0) = f^'_l(x_0) = f^'_r(x_0)$.
}
\end{theorem}

\lang{de}{Unterschiedliche Werte für rechts- und linksseitige Ableitung bedeuten anschaulich, dass der Funktionsgraph einen \glqq Knick\grqq hat. Solche 
Knickstellen sind typisch für stetige Funktionen, die nicht differenzierbar sind. }

\lang{en}{
Intuitively, a function having different values for its right-hand and left-hand derivatives means that its graph
has a cusp. Cusps are typical for continuous functions that are not differentiable.
}

\begin{example}[für nicht-differenzierbar]\label{ex:nichtdiffbar}
\begin{center}
\image{T603_AbsoluteValue}
\end{center}

\lang{de}{Wir betrachten die Betragsfunktion, welche ja gegeben ist durch eine Funktionsvorschrift mit Fallunterscheidung}
\lang{en}{
Consider the absolute value, given by the following assignment involving a case distinction:
}
\[
	{\abs{x}} = 
	\begin{cases}
		\quad x, & \text{falls }x \geq 0,\\
		\ -x, & \text{falls }x < 0.
	\end{cases}
\]
\lang{de}{Für $x<0$ ist der Funktionsgraph eine Gerade mit Steigung $-1$, für $x>0$ eine Gerade mit Steigung $1$. An der Stelle $x_0 = 0$ gilt }
\lang{en}{
For $x<0$, the graph is a line of slope $-1$, and for $x>0$, the graph is a line of slope $1$. In the point $x_0=0$, we have
}

\begin{align*}
f^'_r(x_0) &=  \lim_{h\searrow 0} \frac{|0+h| - |0|}{h}  =  \lim_{h\searrow 0} \frac{h}{h} = \lim_{h\searrow 0} 1 = 1,   \\
f^'_l(x_0) &= \lim_{h\nearrow 0} \frac{|0+h|-|0|}{h}  \lim_{h\nearrow 0} \frac{-h}{h} =  \lim_{h\nearrow 0} -1 = -1,
\end{align*}
\lang{de}{denn $h$ ist im ersten Fall positiv und im zweiten Fall negativ (nähert sich von unten an $0$). Da sich die beiden Werte unterscheiden, 
ist die Betragsfunktion an $x_0=0$ somit nicht differenzierbar. }
\lang{en}{
because $h$ is positive in the first case and negative in the second (as it approaches $0$ from below).
Since these two values are different, the absolute value function is not differentiable in $x_0=0$.
}
\end{example}

%%%%%%%%%%%%%%%%%%%%%%%




\section{
\lang{de}{Ableitungen elementarer Funktionen}
\lang{en}{Derivatives of elementary functions}
}\label{sec:abl-elem-funk}
%Teilweise von OMB+-Kap. VII.3
\lang{de}{Nachdem wir nun wissen, wie die Ableitung $f'$ einer Funktion $f$ mathematisch definiert ist, n\"{a}mlich als der 
Grenzwert des Differenzenquotienten, wollen wir für einige Funktionen die Ableitung berechnen.}
\lang{en}{
Now that we know how the derivative $f'$ of a function $f$ is defined mathematically as
a limit of difference quotients, we want to compute the derivatives of some functions.
} 
\begin{example}
\begin{tabs*}[\initialtab{0}]
\tab{$f(x)=3x+7$}
\lang{de}{Wir betrachten die lineare Funktion $f(x)=3x+7$. Wir berechnen die Ableitung an einer beliebigen Stelle $x$:}
\lang{en}{
Consider the linear function $f(x)=3x+7$. We will calculate the derivative at an arbitrary point $x$:
}
\begin{align*}
f'(x)&=\lim_{h\rightarrow 0}\frac{f(x+h)-f(x)}{h}=\lim_{h\rightarrow 0}\frac{3(x+h)+7-(3x+7)}{h} \\ &=\lim_{h\rightarrow 0}\frac{3h}{h}=
\lim_{h\rightarrow 0} 3=3.
\end{align*}
\lang{de}{Die Ableitungsfunktion einer linearen Funktion mit Steigung $3$ ist also die konstante Funktion $f'(x)=3$.
Dies stimmt mit der Anschauung auch überein, dass der Funktionsgraph von $f$ eine Gerade mit Steigung $3$ ist. }
\lang{en}{
The derivative of a linear function of slope $3$ is therefore the constant function $f'(x)=3$.
This matches our intuition, because the graph of $f$ is a line of slope $3$.
}
\tab{$f(x)=x^2$}
\lang{de}{Um die Ableitung der Potenzfunktion $f(x)=x^2$ zu bestimmen, benötigen wir die 
binomische Formel}
\lang{en}{
To determine the derivative of the power function $f(x)=x^2$, we will need the binomial formula
}
\[    (a+b)^2=a^2+2ab+b^2. \]
\lang{de}{Wir berechnen die Ableitung wieder an einer (beliebigen) Stelle $x$:}
\lang{en}{We will compute the derivative in an (arbitrary) point $x$:}
		\begin{align*}
		f'(x)&=\lim_{h\rightarrow 0}\frac{f(x+h)-f(x)}{h}=\lim_{h\rightarrow 0}\frac{(x+h)^2-x^2}{h}\\
		&=\lim_{h\rightarrow 0}\frac{x^2+2xh+h^2-x^2}{h}=\lim_{h\rightarrow 0}\frac{2xh+h^2}{h}\\
		&=\lim_{h\rightarrow 0}2x+h=2x.
		\end{align*}


\tab{$f(x)=\frac{1}{x}$}
\lang{de}{Die Funktion $f(x)=\frac{1}{x}$ ist für $x\in\R\setminus\{0\}$ definiert. Es sei $x\neq 0$ beliebig und $h$ \glqq gen\"{u}gend klein\grqq 
(d.\,h. $|h| < |x|$, damit $x \pm h \neq 0$ nicht passieren kann).
Dann ist}
\lang{en}{
The function $f(x)=\frac{1}{x}$ is only defined for $x\in\R\setminus\{0\}$.
Let $x \ne 0$ be arbitrary and let $h$ be "small enough" (that is, $|h| < |x|$,
so that $x\pm h$ cannot be zero). Then
}
\begin{eqnarray*}
\frac{f(x+h)-f(x)}{h}&=&\frac{\frac{1}{x+h}-\frac{1}{x}}{h} = \frac{1}{h} \cdot \left(\frac{1}{x+h} - \frac{1}{x} \right)
=\frac{1}{h}\cdot \frac{x-(x+h)}{(x+h)x}\\&=&\frac{1}{h}\cdot\frac{-h}{(x+h)x}=-\frac{1}{(x+h)x} \ ,
\end{eqnarray*}
\lang{de}{also} \lang{en}{and therefore}
\[f'(x)=\lim_{h\rightarrow 0}-\frac{1}{(x+h)x}=-\frac{1}{x^2}.\]
\end{tabs*}
\end{example}

\begin{remark}
\lang{de}{
Lust auf kompliziertere Beispiele? Die gibt es \ref[content_01_differenzenquotient][hier]{ex:abl-elem-funk}.
}
\lang{en}{
Would you like to try more complicated examples?
There are some \ref[content_01_differenzenquotient][here]{ex:abl-elem-funk}.
}


\lang{de}{In den allermeisten Fällen wird man Ableitungen mit Hilfe von \notion{Ableitungsregeln} berechnen, die wir im nächsten 
Kapitel behandeln werden. }
\lang{en}{
In most cases, derivatives are computed using the \notion{differentiation rules}
that will be treated in the next chapter.
}

\lang{de}{Bei Funktionen mit Fallunterscheidungen bleibt einem jedoch häufig nichts anderes übrig, als die 
Ableitung an den kritischen Stellen wie oben mit der Definition der Ableitung zu berechnen.  }
\lang{en}{
However, for functions that involve case distinctions, we often have no choice but to
resort to the definition to compute derivatives in the critical points, as above.
}
\end{remark}

\lang{de}{Unser bisheriges Wissen über die Ableitung können wir auch schon in wirtschaftswissenschaftlichem Kontext nutzen: 
}
\lang{en}{
What we have learned about the derivative so far can already be applied to an economic context:
}

\begin{quickcheck}
	\field{rational}
	\type{input.function}
	\begin{variables}
		\function[normalize]{k}{3x^2-5x+9}
		\function[normalize]{dk}{6x-5}
        \function[normalize]{k10}{6*10-5}
	\end{variables}

  \lang{de}{
	\text{
        Gegeben ist die Kostenfunktion $K(x)=\var{k}$ (Kosten in €). Bestimmen Sie mit Hilfe des Differenzenquotienten die Grenzkostenfunktion $K'(x)$.
        Wie hoch sind die Grenzkosten bei $10$ Mengeneinheiten? \\         
        Die Grenzkostenfunktion ist $K'(x)= $\ansref . \\ Bei $10$ Mengeneinheiten betragen die Grenzkosten $K'(10) = $ \ansref .        
     }
     }
  \lang{en}{
  \text{
        We are given the cost function $K(x)=\var{k}$ (in €). Use difference quotients to
        determine the marginal cost function $K'(x)$. What is the marginal cost at $10$ units?\\
        The marginal cost function is $K'(x)= $\ansref . \\
        The marginal cost at $10$ units is $K'(10) = $ \ansref .
  }
  }
		
	\begin{answer}
		\solution{dk}            
	\end{answer}
    \begin{answer}
        \solution{k10}
    \end{answer}
	
    \lang{de}{\explanation{
     Die Grenzkostenfunktion ist die Ableitung der Kostenfunktion und berechnet sich durch   
        \begin{align*}
	    	K'(x)&=\lim_{h\rightarrow 0}\frac{k(x+h)-k(x)}{h}\\
            &=\lim_{h\rightarrow 0}\frac{3(x+h)^2-5(x+h)+9-(3x^2-5x+9)}{h}\\
	    	&=\lim_{h\rightarrow 0}\frac{3(x^2+2xh+h^2)-5x-5h+9-3x^2+5x-9}{h}\\
            &=\lim_{h\rightarrow 0}\frac{3x^2+6xh+3h^2-5h-3x^2}{h}\\
            &=\lim_{h\rightarrow 0}\frac{6xh+3h^2-5h}{h}\\
		    &=\lim_{h\rightarrow 0}6x+3h-5=6x-5.
		\end{align*}
        Die Grenzkosten bei $10$ Mengeneinheiten erhalten wir durch Einsetzen: $K'(10) = 6 \cdot 10 -5 = 55$ €/Mengeneinheit.
		}}
    \lang{en}{\explanation{
     The marginal cost function is the derivative of the cost function and is computed as follows:   
        \begin{align*}
	    	K'(x)&=\lim_{h\rightarrow 0}\frac{k(x+h)-k(x)}{h}\\
            &=\lim_{h\rightarrow 0}\frac{3(x+h)^2-5(x+h)+9-(3x^2-5x+9)}{h}\\
	    	&=\lim_{h\rightarrow 0}\frac{3(x^2+2xh+h^2)-5x-5h+9-3x^2+5x-9}{h}\\
            &=\lim_{h\rightarrow 0}\frac{3x^2+6xh+3h^2-5h-3x^2}{h}\\
            &=\lim_{h\rightarrow 0}\frac{6xh+3h^2-5h}{h}\\
		    &=\lim_{h\rightarrow 0}6x+3h-5=6x-5.
		\end{align*}
        The marginal cost at $10$ units is obtained by setting $x=10$: $K'(10) = 6 \cdot 10 -5 = 55$ €/unit.
		}}
\end{quickcheck}


\lang{de}{In der folgenden Tabelle sind die wichtigsten Ableitungen zusammengefasst. Im nächsten Kapitel werden wir einige dieser
Regeln auch begründen und weitere Regeln behandeln, mit denen zusammengesetzte Funktionen abgeleitet werden können. 
}
\lang{en}{
The table below contains the most important derivatives. In the next chapter, we will
justify some of these rules and consider further rules that can be used to differentiate
compositions of functions.
}
\begin{rule}\label{rule:diffregeln}
\begin{align*}
\underline{\text{\lang{de}{Funktion}\lang{en}{Function}}\; f(x)}&\hspace{20pt}&  \underline{\text{\lang{de}{Ableitung}\lang{en}{Derivative}}\; f'(x)}&\hspace{20pt}&\underline{\text{\lang{de}{Bedingung an }\lang{en}{Conditions on }}\, x}\\
c \;(c\in\R) &&0&&\\
x^n\;( n\in\N)&&nx^{n-1}&&\\
x^n\;(n\in\Z, n<0)&&nx^{n-1}&& x\neq 0\\
x^r\;( r\in\R)&&rx^{r-1}&& x>0\\
\sqrt{x}=x^{1/2} &&\frac{1}{2\sqrt{x}}=\frac{1}{2}x^{-1/2}&& x>0\\
e^x&&e^x&&\\
\ln(x)&&\frac{1}{x}&&x>0\\
a^x \;(a>0)&&\ln a\cdot a^x&&\\
\sin(x)&&\cos(x)&&\\
\cos(x)&&-\sin(x)&&\\
\tan(x)&& \frac{1}{\cos(x)^2} = 1 + \tan(x)^2 && \cos(x) \neq 0 
\end{align*}
\end{rule}

\begin{quickcheck}
		\field{rational}
		\type{input.function}
		\begin{variables}
			\randint[Z]{n}{2}{9}
		    \function[normalize]{f}{1/x^n}
			\function[normalize]{df}{-n/x^(n+1)}
		\end{variables}
		
			\lang{de}{\text{
            Mit den Funktionen aus der obigen Tabelle kennen Sie schon einige wichtige
            Ableitungen. Zeigen Sie hier zum Ende des Kapitels, dass Sie die Tabelle 
            auch benutzen können: \\ \\            
            Die Ableitungsfunktion der Funktion $f(x)=\var{f}$ ist $f'(x)= $\ansref.
            }}
      \lang{en}{
        \text{Now that you have the functions in the table above, you already know
        several important derivatives. Now end the chapter by showing that you can
        use the table: \\ \\
        The derivative of $f(x)=\var{f}$ is $f'(x)= $\ansref.}
      }
		
		\begin{answer}
			\solution{df}
            \checkAsFunction{x}{-1}{1}{10}
		\end{answer}
		\lang{de}{\explanation{Es gilt $\var{f} = x^{-\var{n}}$. Dann kann man die Regel für ganzzahlige Potenzen (dritte Zeile) anwenden.
		}}
    \lang{en}{\explanation{Since $\var{f} = x^{-\var{n}}$, we can use the rule for differentiating integral powers (row 3).
		}}
	\end{quickcheck}



\end{visualizationwrapper}


\end{content}