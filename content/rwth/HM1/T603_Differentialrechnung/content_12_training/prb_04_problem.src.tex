\documentclass{mumie.problem.gwtmathlet}
%$Id$
\begin{metainfo}
  \name{
    \lang{en}{...}
    \lang{de}{A04: Extremstellen II}
  }
  \begin{description} 
 This work is licensed under the Creative Commons License Attribution 4.0 International (CC-BY 4.0)   
 https://creativecommons.org/licenses/by/4.0/legalcode 

    \lang{en}{...}
    \lang{de}{...}
  \end{description}
  \corrector{system/problem/GenericCorrector.meta.xml}
  \begin{components}
    \component{js_lib}{system/problem/GenericMathlet.meta.xml}{gwtmathlet}
  \end{components}
  \begin{links}
  \end{links}
  \creategeneric
\end{metainfo}
\begin{content}
\lang{de}{\title{A03: Extremstellen II}}
\lang{en}{\title{A03: Extreme points II}}
\begin{block}[annotation]
	Im Ticket-System: \href{https://team.mumie.net/issues/24071}{Ticket 24071}
\end{block}


\usepackage{mumie.ombplus}
\usepackage{mumie.genericproblem}




\begin{block}[annotation]
Kopie: hm4mint/T106_Differentialrechnung/training 12  

Im Ticket-System: \href{http://team.mumie.net/issues/9477}{Ticket 9477}
\end{block}


\begin{problem}

\begin{variables}
	\randint[Z]{s}{-1}{1}
	\randint[Z]{t}{-1}{1}
    \randint{a}{2}{6}
	\randint{p}{1}{10}
	\randint{c}{1}{10}
	\randint{n}{2}{6}
	\randint{q}{1}{10}
	\function[calculate]{g}{s*a} 
	\function[calculate]{b}{p*a}
	\function[calculate]{l}{t*n}
	\function[calculate]{m}{q*n}
	\function[calculate]{erg}{-1-s*p}
	\function[calculate]{ergzwei}{-t*q/2}
	\function{f}{(g*x+b)*e^x} 
	\function{h}{-e^(l*x^2+m*x+c)}
\end{variables}

%Frage 1 von 4
\begin{question}
	\lang{de}{
		\text{Gegeben ist die Funktion $f(x)=\var{f}$. Wo hat $f$ eine station\"{a}re Stelle?}
	    \explanation{Für die Berechnung müssen wir die Nullstellen der ersten Ableitung von $f$ berechnen. Da die vorgegebene Funktion sogar beliebig oft differnzierbar ist, 
         kann dafür die Produktregel verwendet werden. Setzen wir hier $f'(x) = 0$ und lösen nach $x$ auf, so erhalten wir die stationären Stellen.}
	}
	\lang{en}{
		\text{Let $f(x)=\var{f}$. Where does $f$ have a stationary point?}
    \explanation{We need to find the zeros of the first derivative of $f$. Since $f$ is differentiable infinitely often,
    we can use the product rule. By setting $f'(x)=0$ and solving for $x$, we find the stationary points.}
	}
	\type{input.number}
	\begin{answer}
		\lang{de}{\text{ Die station\"{a}re Stelle von $f$ ist $x_0=$}}
   		\lang{en}{\text{The stationary point of $f$ is at $x_0=$}}
		\solution{erg}
	\end{answer}		
\end{question}
	 
%Frage 2 von 4	 
\begin{question}
	 \lang{de}{
		\text{Liegt in der stationären Stelle $x_0$ von $f$ ein Extremum vor? Falls ja, welcher Art ist es?}
		\explanation{Um zu überprüfen, ob ein Extremum vorliegt, bilden wir die zweite Ableitung von $f$. 
        Anschließend ermitteln wir, ob diese ausgewertet in den stationären Punkten negativ oder positiv ist. Gilt $f''(x_0) >0$, so liegt eine Minimalstelle vor. Ist hingegen $f''(x_0) <0$, so 
        haben wir eine Maximalstelle gefunden.}
	}
	\lang{en}{
		\text{Is there an extremum at the stationary point $x_0$ of $f$? If so, what type of extremum is it?}
	}
	\type{mc.unique}
	\begin{choice}
    	\lang{de}{\text{In $x_0$ liegt kein Extremum vor.}}
		\lang{en}{\text{There is no extremum at $x_0$.}}
  		\solution{false}
	\end{choice}
	\begin{choice}
   		\lang{de}{\text{In $x_0$ liegt ein lokales Minimum vor.}}
   		\lang{en}{\text{There is a local minimum at $x_0$.}}	
   		\solution{compute}
 		\iscorrect{s}{=}{1}
 	\end{choice}
 	\begin{choice}
 		\lang{de}{\text{In $x_0$ liegt ein lokales Maximum vor.}}
   		\lang{en}{\text{There is a local maximum at $x_0$.}}	
   		\solution{compute}
 		\iscorrect{s}{=}{-1}
 	\end{choice}
\end{question}

%Frage 3 von 4
\begin{question}
	\lang{de}{
		\text{Gegeben ist die Funktion $h(x)=\var{h}$. Wo hat $h$ eine station\"{a}re Stelle?}
		 \explanation{Für die Berechnung müssen wir die Nullstellen der ersten Ableitung von $f$ berechnen. Da die vorgegebene Funktion sogar beliebig oft differnzierbar ist, 
         kann dafür die Produktregel verwendet werden. Setzen wir hier $f'(x) = 0$ und lösen nach $x$ auf, so erhalten wir die stationären Stellen.}
	}
	\lang{en}{
		\text{Let $h(x)=\var{h}$. Where does $h$ have a stationary point?}
	}
	\type{input.number}
	\begin{answer}
		\lang{de}{\text{ Die station\"{a}re Stelle von $h$ ist $x_0=$}}
   		\lang{en}{\text{The stationary point of $h$ is $x_0=$}}
		\solution{ergzwei}
	\end{answer}
\end{question}
	 
%Frage 4 von 4
\begin{question}
	\lang{de}{
		\text{Liegt in der stationären Stelle $x_0$ von $h$ ein Extremum vor? Falls ja, welcher Art ist es?}
		\explanation{Um zu überprüfen, ob ein Extremum vorliegt, bilden wir die zweite Ableitung von $f$. 
        Anschließend ermitteln wir, ob diese ausgewertet in den stationären Punkten negativ oder positiv ist. Gilt $f''(x_0) >0$, so liegt eine Minimalstelle vor. Ist hingegen $f''(x_0) <0$, so 
        haben wir eine Maximalstelle gefunden.}
	}
	\lang{en}{
		\text{Is there an extremum at the stationary point $x_0$ of $h$? If so, what type of extremum is it?}
  	\explanation{To test whether there is an extremum, we take the second derivative of $f$.
     Then we determine whether it is negative or positive when evaluated in the stationary point. If $f''(x_0) > 0$, then we have a minimum.
     On the other hand, if $f''(x_0) < 0$, then we have found a maximum.
 }
 }
    \type{mc.unique}
	\begin{choice}
    	\lang{de}{\text{In $x_0$ liegt kein Extremum vor.}}
		\lang{en}{\text{There is no extremum at $x_0$.}} 
  		\solution{false}
	\end{choice}
	\begin{choice}
   		\lang{de}{\text{In $x_0$ liegt ein lokales Minimum vor.}}
   		\lang{en}{\text{There is a local minimum at $x_0$.}}	
   		\solution{compute}
 		\iscorrect{t}{=}{-1}
 	\end{choice}
 	\begin{choice}
 		\lang{de}{\text{In $x_0$ liegt ein lokales Maximum vor.}}
   		\lang{en}{\text{There is a local maximum at $x_0$.}}	
   		\solution{compute}
 		\iscorrect{t}{=}{+1}
 	\end{choice}
\end{question}

\end{problem}



\embedmathlet{gwtmathlet}

\end{content}
