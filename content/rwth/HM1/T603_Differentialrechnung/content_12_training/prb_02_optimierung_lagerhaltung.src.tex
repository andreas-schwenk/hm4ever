\documentclass{mumie.problem.gwtmathlet}
%$Id$
\begin{metainfo}
  \name{
    \lang{en}{A02: Economic Order Quantity }
    \lang{de}{A02: Andler'sche Losgrößenformel}
    \lang{zh}{...}
    \lang{fr}{...}
  }
  \begin{description} 
 This work is licensed under the Creative Commons License Attribution 4.0 International (CC-BY 4.0)   
 https://creativecommons.org/licenses/by/4.0/legalcode 

    \lang{en}{...}
    \lang{de}{...}
    \lang{zh}{...}
    \lang{fr}{...}
  \end{description}
  \corrector{system/problem/GenericCorrector.meta.xml}
  \begin{components}
    \component{js_lib}{system/problem/GenericMathlet.meta.xml}{gwtmathlet}
  \end{components}
  \begin{links}
  \end{links}
  \creategeneric
\end{metainfo}
\begin{content}
\lang{de}{\title{A02: Andler'sche Losgrößenformel}}
\lang{en}{\title{A02: Economic Order Quantity}}
\begin{block}[annotation]
	Im Ticket-System: \href{https://team.mumie.net/issues/24035}{Ticket 24035}
\end{block}


\usepackage{mumie.genericproblem}


     \begin{problem}
     \begin{variables}
        \drawFromSet{a}{1,2,3,4,5}
        \drawFromSet{c}{100,400,900}
        \function[calculate]{s}{c*a*5}
        \function{b}{a}
        \function{sol}{(2000*s/b)^0.5}
     \end{variables}
          \begin{question}
          \type{input.number}
          \lang{de}{\text{In der Massenproduktion eines Produktes soll die Menge x
          des Gutes festgelegt werden, die zwischen zwei Umrüstungen der Anlage 
          gefertigt wird. \\
          
          Bei jeder Umrüstung entstehen Kosten, die von der Menge des Gutes 
          unabhängig sind, sogenannte Rüstkosten von $R=1000$\euro.
          Für Lagerung und entgangene Kapitalerträge entstehen Bestandskosten
          von $b=\var{b}$\euro pro Mengen- und Zeiteinheit. (Es wird mit einem mittleren
          Bestand von $\frac{b}{2}$ gerechnet, da das Lager zwischen voll und leer schwankt.) Pro Zeiteinheit werden
          $s=\var{s}$ Teile des Gutes gefertigt und abgesetzt. 
          
          Als Modell für die Kostenfunktion wählen wir
          $
          K(x) = \frac{R}{x} s + \frac{b}{2}x.
          $
          
          Berechnen Sie die Menge x des Gutes, so dass die Kosten minimiert werden:\\
          $x=$\ansref
          }}
           \lang{en}{\text{In the context of the mass production of a product,
           the quantity x of the good that is produced between two retoolings of the plant
            needs to be determined. \\
          
          Each retooling incurs costs that are independent of the quantity of the good:
          so-called setup costs of $R=1000$\euro.
          Storage and lost capital gains incur inventory costs
          of $b=\var{b}$\euro per unit of quantity and time. (This is calculated assuming an average
          stock of $\frac{b}{2}$, as the stock fluctuates between full and empty). Per unit of time,
          $s=\var{s}$ units of the good are produced and sold. 
          
          As a model for the cost function, we choose
          $
          K(x) = \frac{R}{x} s + \frac{b}{2}x.
          $
          
          Calculate the quantity x of the good that minimizes the cost:\\
          $x=$\ansref
          }}
               \begin{answer}
               \solution{sol}
               \end{answer}
          \lang{de}{\explanation[edited]{Nutzen Sie die Ableitung von $K(x)=\frac{s}{x}R+\frac{x}{2}b; x>0$, um eine
          Minimalstelle zu finden.}}
          \lang{en}{\explanation[edited]{Use the derivative of $K(x)=\frac{s}{x}R+\frac{x}{2}b; x>0$ to find a
          minimum.}}
          
          \end{question} 
          
     \end{problem}

\embedmathlet{gwtmathlet}

\end{content}
