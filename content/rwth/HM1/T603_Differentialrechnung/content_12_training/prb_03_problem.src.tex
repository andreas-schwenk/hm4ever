\documentclass{mumie.problem.gwtmathlet}
%$Id$
\begin{metainfo}
  \name{
    \lang{en}{...}
    \lang{de}{A03: Extremstellen I}
  }
  \begin{description} 
 This work is licensed under the Creative Commons License Attribution 4.0 International (CC-BY 4.0)   
 https://creativecommons.org/licenses/by/4.0/legalcode 

    \lang{en}{...}
    \lang{de}{...}
  \end{description}
  \corrector{system/problem/GenericCorrector.meta.xml}
  \begin{components}
    \component{js_lib}{system/problem/GenericMathlet.meta.xml}{mathlet}
  \end{components}
  \begin{links}
  \end{links}
  \creategeneric
\end{metainfo}
\begin{content}
\lang{de}{\title{A03: Extremstellen I}}
\lang{en}{\title{A03: Extreme points I}}
\begin{block}[annotation]
	Im Ticket-System: \href{https://team.mumie.net/issues/24070}{Ticket 24070}
\end{block}
\usepackage{mumie.genericproblem}
\usepackage{mumie.ombplus}

\begin{block}[annotation]
Kopie: hm4mint/T106_Differentialrechnung/training 10 

Im Ticket-System: \href{http://team.mumie.net/issues/9475}{Ticket 9475}
\end{block}


\begin{problem}
  	
\randomquestionpool{1}{1}
\randomquestionpool{2}{2}
\randomquestionpool{3}{4}
\randomquestionpool{5}{5}
  	
\begin{variables}
	\randint[Z]{s}{-1}{1}
    \randint{p}{2}{8}
	\randint{q}{1}{9}
	\randint[Z]{r}{-9}{9}
	\function[calculate]{v}{s*p}
	\function[calculate]{w}{2*p*q}
	\function[calculate]{ext}{-s*q}
	\function{f}{v * x^2  + w * x + r} 
\end{variables}

%Frage 1 von 5
\begin{question}
	\lang{de}{
		\text{Gegeben ist die Funktion $f(x)=\var{f}$. Wo hat $f$ eine station\"{a}re Stelle?}
	    \explanation{Zur Bestimmung der stationären Stellen müssen die Nullstellen der ersten 
        Ableitung bestimmt werden. In dieser Aufgabe muss also ein Polynom abgeleitet werden.}
	}
	\lang{en}{
		\text{Given the function $f(x)=\var{f}$, where does $f$ have a critical point?}
	}
	\type{input.number}
	\begin{answer}
  		\lang{de}{\text{$f$ hat eine stationäre Stelle in $x_0$ = }}
   		\lang{en}{\text{$f$ has a critical point at $x_0$ =}}
		\solution{ext}
	\end{answer}
\end{question}

%Frage 2 von 5
\begin{question}
    \lang{de}{
		\text{Liegt in dieser stationären Stelle $x_0$ ein Extremum vor? Falls ja, welcher Art ist es?}
		\explanation{Um zu prüfen, ob die stationären Stellen Extrema sind kann geprüft werden, ob die erste Ableitung
        in den Punkten das Vorzeichen wechselt. Eine andere Möglichkeit wäre es zu prüfen, ob die zweite Ableitung 
        in den stationären Stellen poisitiv oder negativ ist.}
	}
	\lang{en}{
		\text{Is the above critical point $x_0$ an extremum? If so, what type of extremum is it?}
	}
    \type{mc.unique}
	\begin{choice}
		\lang{de}{\text{In $x_0$ liegt kein Extremum vor.}}
		\lang{en}{\text{$x_0$ is not an extremum.}}
  		\solution{false}
	\end{choice}
	\begin{choice}
   		\lang{de}{\text{In $x_0$ liegt ein lokales Minimum vor.}}
   		\lang{en}{\text{$x_0$ is a local minimum.}}
   		\solution{compute}
 		\iscorrect{s}{=}{1}
 	\end{choice}
 	\begin{choice}
 		\lang{de}{\text{In $x_0$ liegt ein lokales Maximum vor.}}
   		\lang{en}{\text{$x_0$ is a local maximum.}}	
   		\solution{compute}
 		\iscorrect{s}{=}{-1}
 	\end{choice}
\end{question}
	
%Frage 3 von 5
\begin{question}
	\lang{de}{
		\text{Gegeben $f(x) = \frac{1}{3}x^3-\frac{\var{vfe}}{2}x^2+\var{vfz}x+\var{idc}$. Wo hat $f$ ein lokales Maximum, wo ein lokales
		 Minimum?}
		\explanation{Zur Bestimmung der stationären Stellen müssen die Nullstellen der ersten 
        Ableitung bestimmt werden. In dieser Aufgabe muss also ein Polynom abgeleitet werden.
        Um zu prüfen, ob die stationären Stellen Extrema sind kann geprüft werden, ob die erste Ableitung
        in den Punkten das Vorzeichen wechselt. Eine andere Möglichkeit wäre es zu prüfen, ob die zweite Ableitung 
        in den stationären Stellen poisitiv oder negativ ist.}
	}
	\lang{en}{
		\text{Let $f(x) = \frac{1}{3}x^3-\frac{\var{vfe}}{2}x^2+\var{vfz}x+\var{idc}$. Where does $f$ have a local maximum? A local minimum?}
	}
	\begin{variables}
        \randint{a}{1}{2}
		\randint[Z]{b}{3}{4}
 		\randint{c}{1}{9}
		\function[calculate]{ida}{a}
		\function[calculate]{idb}{b}
		\function[calculate]{idc}{c}
		\function[calculate]{eins}{2*a}
		\function[calculate]{zwei}{2*b+1}
		\function[calculate]{vfe}{2*a+2*b+1}
		\function[calculate]{vfz}{2*a*(2*b+1)}
	\end{variables}
	\type{input.number}
	\begin{answer}
    	\lang{de}{\text{ $f$ hat ein lokales Maximum in:}}
       	\lang{en}{\text{$f$ has a local maximum at $x=$}}
		\solution{eins}
	\end{answer}
	\begin{answer}
      	\lang{de}{\text{ $f$ hat ein lokales Minimum in:}}
       	\lang{en}{\text{$f$ has a local minimum at $x=$}}
		\solution{zwei}			
	\end{answer}
\end{question}
	
%Frage 4 von 5
\begin{question}
	\lang{de}{
		\text{Gegeben $f(x) = -\frac{1}{3}x^3+\frac{\var{vfe}}{2}x^2-\var{vfz}x-\var{idc}$. Wo hat $f$ ein lokales Maximum, wo ein lokales
		 Minimum?}
		\explanation{Zur Bestimmung der stationären Stellen müssen die Nullstellen der ersten 
        Ableitung bestimmt werden. In dieser Aufgabe muss also ein Polynom abgeleitet werden.
        Um zu prüfen, ob die stationären Stellen Extrema sind kann geprüft werden, ob die erste Ableitung
        in den Punkten das Vorzeichen wechselt. Eine andere Möglichkeit wäre es zu prüfen, ob die zweite Ableitung 
        in den stationären Stellen poisitiv oder negativ ist.}
	}
	\lang{en}{
		\text{Let $f(x) = -\frac{1}{3}x^3+\frac{\var{vfe}}{2}x^2-\var{vfz}x-\var{idc}$. Where does $f$ have a local maximum? A local minimum?}
	}
	\begin{variables}
        \randint{a}{1}{2}
		\randint[Z]{b}{3}{4}
 		\randint{c}{1}{9}
		\function[calculate]{ida}{a}
		\function[calculate]{idb}{b}
		\function[calculate]{idc}{c}
		\function[calculate]{eins}{2*a}
		\function[calculate]{zwei}{2*b+1}
		\function[calculate]{vfe}{2*a+2*b+1}
		\function[calculate]{vfz}{2*a*(2*b+1)}
	\end{variables}
	\type{input.number}
	\begin{answer}
  		\lang{de}{\text{ $f$ hat ein lokales Maximum in:}}
   		\lang{en}{\text{$f$ has a local maximum at $x=$}}
		\solution{zwei}			
	\end{answer}
	\begin{answer}
  		\lang{de}{\text{ $f$ hat ein lokales Minimum in:}}
   		\lang{en}{\text{$f$ has a local minimum at $x=$}}
		\solution{eins}		
	\end{answer}
\end{question}
	
%Frage 5 von 5
\begin{question}
	\lang{de}{
		\text{Gegeben sei $f(x)=\var{vfe}x^4-\var{vfz}x+\var{idc}$. Wo hat $f$ ein lokales Minimum?}
		\explanation{Zur Bestimmung der stationären Stellen müssen die Nullstellen der ersten 
        Ableitung bestimmt werden. In dieser Aufgabe muss also ein Polynom abgeleitet werden.
        Um zu prüfen, ob die stationären Stellen Extrema sind kann geprüft werden, ob die erste Ableitung
        in den Punkten das Vorzeichen wechselt. Eine andere Möglichkeit wäre es zu prüfen, ob die zweite Ableitung 
        in den stationären Stellen poisitiv oder negativ ist.}
	}
	\lang{en}{
	 	\text{Let $f(x)=\var{vfe}x^4-\var{vfz}x+\var{idc}$. Where does $f$ have a local minimum?}
	}
	\begin{variables}
        \randint{a}{1}{3}
		\randint[Z]{b}{2}{4}
 		\randint{c}{1}{9}
		\function[calculate]{idc}{c}
		\function[calculate]{vfe}{2*a^3}
		\function[calculate]{vfz}{b^3}
		\function[calculate]{loes}{0.5*b*a^(-1)}
	\end{variables}
	\type{input.number}
	\begin{answer}
  		\lang{de}{\text{ $f$ hat ein lokales Minimum in:}}
   		\lang{en}{\text{$f$ has a local minimum at $x=$}}
		\solution{loes}				
	\end{answer}
\end{question}
	
	
\end{problem}

\embedmathlet{mathlet}
\end{content}

