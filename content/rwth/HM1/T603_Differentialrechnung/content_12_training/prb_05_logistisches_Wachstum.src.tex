\documentclass{mumie.problem.gwtmathlet}
%$Id$
\begin{metainfo}
  \name{
    \lang{en}{...}
    \lang{de}{A05: logistisches Wachstum}
    \lang{zh}{...}
    \lang{fr}{...}
  }
  \begin{description} 
 This work is licensed under the Creative Commons License Attribution 4.0 International (CC-BY 4.0)   
 https://creativecommons.org/licenses/by/4.0/legalcode 

    \lang{en}{...}
    \lang{de}{...}
    \lang{zh}{...}
    \lang{fr}{...}
  \end{description}
  \corrector{system/problem/GenericCorrector.meta.xml}
  \begin{components}
    \component{js_lib}{system/problem/GenericMathlet.meta.xml}{gwtmathlet}
  \end{components}
  \begin{links}
  \end{links}
  \creategeneric
\end{metainfo}
\begin{content}
\lang{de}{\title{A05: logistisches Wachstum}}
\lang{en}{\title{A05: logistic growth }}
\begin{block}[annotation]
	Im Ticket-System: \href{https://team.mumie.net/issues/24036}{Ticket 24036}
\end{block}


\usepackage{mumie.genericproblem}

\embedmathlet{gwtmathlet}

     \begin{problem}

        \begin{variables}
            \randint{a}{8}{10}
            \randint{b}{2}{5}
            \randint{c}{2}{3}
            \function{f}{a/(1+b*e^(-c*x))}
            \function{xw}{ln(b)/c}
            \derivative{f_1}{f}{x}
            \function[normalize]{ff_1}{a*b*c*e^(c*x)/(b+e^(c*x))^2}
            \function[normalize]{f_2}{a*b*c^2*e^(c*x)(b-e^(c*x))/(b+e^(c*x))^3}
            \function[normalize]{m}{a/(1+b)}
            \function{M}{a}
        \end{variables}
           
          \begin{question}
            \lang{de}{\text{ Die logistische Funktion $f(x)=\var{f}$ beschreibe den kumulierten Absatz y 
eines Produktes in Abhängigkeit von der Zeit x (Dabei sei $x>0$).\\
$f(x)$ gibt also an, welche Menge insgesamt bis zur Zeit x insgesamt verkauft wurde. 
            \\Wir betrachten die Funktion $f(x)$ für $x>0$:}}
            \lang{en}{\text{Suppose the logistic function $f(x)=\var{f}$ describes the cumulative sales y 
of a product as a function of time x (where $x>0$).\\
That is, $f(x)$ indicates the total quantity sold by time x. 
            \\We will consider the function $f(x)$ for $x>0$:}}
            \lang{de}{\explanation{Der Zähler von f(x) hat keine Nullstellen.\\
            Die erste Ableitung ist: $f'(x)=\var{f_1}=\var{ff_1}$.}}
             \lang{en}{\explanation{The numerator of f(x) has no zeros.\\
            The first derivative is: $f'(x)=\var{f_1}=\var{ff_1}$.}}
            \type{mc.yesno}
            
            \begin{choice}
                \lang{de}{\text{Die Funktion hat Nullstellen.}}
                \lang{en}{\text{The function has zeros.}}
                \solution{false}
            \end{choice}
            \begin{choice}
                \lang{de}{\text{Die Funktion hat Extremstellen.}}
                \lang{en}{\text{The function has extrema.}}
                \solution{false}
            \end{choice}
          \end{question}
          
         \begin{question}
            \lang{de}{\text{Bestimmen Sie die Wendestelle $x_w=$\ansref}}
            \lang{en}{\text{Determine the inflection point $x_w=$\ansref}}
            \type{input.function}
            \lang{de}{\explanation{Die zweite Ableitung ist: $f''(x)=\var{f_2}$. Sie kann mit Hilfe
            der Quotientenregel aus der ersten Ableitung errechnet werden. Die Wendestelle ergibt
            sich durch Bestimmen der Nullstelle des Zählers.}}
           \lang{en}{\explanation{The second derivative is: $f''(x)=\var{f_2}$. This can be calculated 
            by the quotient rule using the first derivative. The point of inflection is found
            by determining the zero of the numerator.}}
            \begin{answer}
                \solution{xw}
                \checkAsFunction{x}{1}{3}{1}
            \end{answer}
         \end{question}
         
         \begin{question}
            \lang{de}{\text{Wir betrachten $f(x)=\var{f}$ für $x<x_w$:}}
           \lang{en}{\text{We will consider $f(x)=\var{f}$ for $x<x_w$:}}
            \lang{de}{\explanation{Für $x<x_w$ ist die Funktion linksgekrümmt, also progressiv steigend
            und somit $f''(x)>0$.}}
            \lang{en}{\explanation{The function is concave up for $x<x_w$, so its rate of change is increasing
            and $f''(x)>0$.}}
            
            \type{mc.yesno}
            
            \begin{choice}
                \lang{de}{\text{Die Funktion ist progressiv steigend.}}
                \lang{en}{\text{The rate of change of the function is increasing.}}
                \solution{true}
            \end{choice}
            \begin{choice}
                \lang{de}{\text{Die Funktion ist rechtsgekrümmt.}}
                \lang{en}{\text{The function is concave down.}}
                \solution{false}
            \end{choice}
            \begin{choice}
                \text{$f''(x)<0$}
                \solution{false}
            \end{choice}
         \end{question}
         
         \begin{question}
            \lang{de}{\text{Berechnen Sie die untere und obere Schranke der Funktion:\\
            $f(0)=$\ansref\\
            $\lim_{x\to\infty}f(x)=$\ansref
            }}
            \lang{en}{\text{Calculate the lower and upper bounds of the function:\\
            $f(0)=$\ansref\\
            $\lim_{x\to\infty}f(x)=$\ansref
            }}
            \type{input.function}
            \explanation{$e^0=1$
                 \lang{de}{\text{und}}
                 \lang{en}{\text{and}}
                $\lim_{x\to\infty}e^{-x}=0$.}
            \begin{answer}
                \solution{m}
                \checkAsFunction{x}{1}{3}{1}
            \end{answer}
            \begin{answer}
                \solution{M}
                \checkAsFunction{x}{1}{3}{1}
            \end{answer}
         \end{question}
     \end{problem}


\end{content}
