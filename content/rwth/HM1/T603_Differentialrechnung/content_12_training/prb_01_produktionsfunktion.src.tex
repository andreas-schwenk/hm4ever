\documentclass{mumie.problem.gwtmathlet}
%$Id$
\begin{metainfo}
  \name{
    \lang{en}{...}
    \lang{de}{A01: Produktionsfunktion}
    \lang{zh}{...}
    \lang{fr}{...}
  }
  \begin{description} 
 This work is licensed under the Creative Commons License Attribution 4.0 International (CC-BY 4.0)   
 https://creativecommons.org/licenses/by/4.0/legalcode 

    \lang{en}{...}
    \lang{de}{...}
    \lang{zh}{...}
    \lang{fr}{...}
  \end{description}
  \corrector{system/problem/GenericCorrector.meta.xml}
  \begin{components}
    \component{js_lib}{system/problem/GenericMathlet.meta.xml}{gwtmathlet}
  \end{components}
  \begin{links}
  \end{links}
  \creategeneric
\end{metainfo}
\begin{content}
\lang{de}{\title{A01: Produktionsfunktion}}
\lang{en}{\title{A01: Production function}}
\begin{block}[annotation]
	Im Ticket-System: \href{https://team.mumie.net/issues/24034}{Ticket 24034}
\end{block}

\usepackage{mumie.genericproblem}

  \begin{problem}
     \begin{variables}
        \drawFromSet{a}{1,2,3,4,5} 
        \function[expand, normalize]{f}{a*(-0,1*x^3+6*x^2+12,3*x)}
        \derivative{f_1}{f}{x}
        \derivative{f_2}{f_1}{x}
        \number{m}{41}
        \number{w}{20}
        \function[calculate,2]{a2}{-a*0,3}
        \function[calculate]{M}{a*(-0,1*36^3+6*36^2+12,3*36)}
        
     \end{variables}
          \begin{question}
          \lang{de}{\text{Gegeben sei die Produktionsfunktion $y=f(x)=\var{f}$ 
          \\($x>0$, $x$=INPUT
          in Mengeneinheiten, $y$=OUTPUT in Mengeneinheiten) mit maximal möglichem INPUT 
          von 36 Mengeneinheiten.
          
          Bestimmen Sie den maximalen Output, die Grenzproduktivität und das Wachstumsverhalten.}}
             \lang{en}{\text{Given the production function $y=f(x)=\var{f}$ 
          \\($x>0$, $x$=INPUT
          in units of quantity, $y$=OUTPUT in units of quantity) with maximum possible INPUT 
          of 36 units.
          
          Determine the maximum output, marginal productivity and growth behaviour.}}
          
          \type{input.generic}
          
               \begin{answer}
                 \type{input.function}
                 \text{$f'(x)=$}
                 \solution{f_1}
                 \checkAsFunction{x}{-10}{10}{10}
                 \explanation[edited]{
                 \lang{de}{\text{Die Ableitung ist:}}
                 \lang{en}{\text{The derivative is :}}
                 $f'(x)=\var{f_1}$.}
               \end{answer}
               
               \begin{answer}
                    \type{input.function}
                    \text{$f''(x)=$}
                    \solution{f_2}
                    \checkAsFunction{x}{-10}{10}{10}
               \end{answer}       
               
               \begin{answer}
                 \type{input.number}
                 \lang{de}{\text{lokales Maximum von $f$ liegt bei $x_e=$}}
                 \lang{en}{\text{local maximum of $f$ at $x_e=$}}
                 \solution{m}
                 \lang{de}{\explanation[edited]{Bei der pq-Formel muss man daran denken durch $\var{a2}$ zu teilen.}}
                 \lang{en}{\explanation[edited]{Remember to divide by $\var{a2}$ in the quadratic formula.}}
               \end{answer} 
               
               \begin{answer}
                    \type{input.number}
                    \lang{de}{\text{maximaler Output beträgt $y_{max}=$}}
                    \lang{en}{\text{maximum output is $y_{max}=$}}
                    \solution{M}
                    \lang{de}{\explanation[edited]{x ist höchstens 36.}}
                    \lang{en}{\explanation[edited]{x is at most 36.}}
               \end{answer}              
               
               \begin{answer}
                    \type{mc.yesno}
                    \begin{choice}
                        \lang{de}{\text{Die Grenzproduktivität ist für $x<36$ positiv.}}
                        \lang{en}{\text{Marginal productivity is positive for $x<36$.}}
                        \solution{true}
                    \end{choice}
                    \begin{choice}
                        \lang{de}{\text{Die Grenzproduktivität ist für $x>36$ positiv.}}
                        \lang{en}{\text{Marginal productivity is positive for $x>36$.}}
                        \solution{false}
                    \end{choice}
                    \lang{de}{\explanation{Die Grenzproduktivität als Zuwachs des Ertrages, der durch den Einsatz einer weiteren
                    Einheit eines Produktionsfaktors erzielt wird, ist dort positiv, wo die Funktion streng monoton steigend ist.
                    Wir müssen hier noch beachten, dass die Menge durch 36 nach oben beschränkt ist.}}
                    \lang{en}{\explanation{Marginal productivity, being the increase in output that is achieved through the use of a further unit of a production factor,
                    is positive where the function is strictly monotonically increasing.
                    We still have to note here that the quantity is bounded from above by 36.}}
               \end{answer}
               
               \begin{answer}
                    \type{input.number}
                    \lang{de}{\text{Wendestelle bei $x_w=$}}
                    \lang{en}{\text{Inflection point at $x_w=$}}
                    \solution{w}
                    %\explanation[edited]{...}
               \end{answer}
               
               \begin{answer}
                    \type{mc.yesno}
                    \begin{choice}
                        \lang{de}{\text{Das Wachstum ist für $x<20$ progressiv.}}
                        \lang{en}{\text{Growth increases in rate for $x<20$.}}
                        \solution{true}
                    \end{choice}
                    \begin{choice}
                        \lang{de}{\text{Das Wachstum ist für $x>20$ progressiv.}}
                        \lang{en}{\text{Growth increases in rate for $x>20$.}}
                        \solution{false}
                    \end{choice}
                    \lang{de}{\explanation{In diesem Fall wird die Steigung bis zur Wendestelle immer größer, 
                    d.\,h. $f''>0$.}}
                    \lang{en}{\explanation{In this case, the slope increases up to the inflection point, 
                    i.e. $f''>0$.}}
               \end{answer}
               
               
          \end{question}     
     \end{problem}



\embedmathlet{gwtmathlet}

\end{content}
