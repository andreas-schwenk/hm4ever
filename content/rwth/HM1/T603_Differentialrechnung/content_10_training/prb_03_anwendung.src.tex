\documentclass{mumie.problem.gwtmathlet}
%$Id$
\begin{metainfo}
  \name{
    \lang{en}{...}
    \lang{de}{A03: Grenzkosten vs. Grenzproduktivität}
  }
  \begin{description} 
 This work is licensed under the Creative Commons License Attribution 4.0 International (CC-BY 4.0)   
 https://creativecommons.org/licenses/by/4.0/legalcode 

    \lang{en}{...}
    \lang{de}{...}
  \end{description}
  \corrector{system/problem/GenericCorrector.meta.xml}
  \begin{components}
    \component{js_lib}{system/problem/GenericMathlet.meta.xml}{gwtmathlet}
  \end{components}
  \begin{links}
  \end{links}
  \creategeneric
\end{metainfo}
\begin{content}
\lang{de}{\title{A03: Grenzkosten vs. Grenzproduktivität}}
\lang{en}{\title{A03: Marginal costs vs. marginal productivity}}
\begin{block}[annotation]
	Im Ticket-System: \href{https://team.mumie.net/issues/24005}{Ticket 24005}
\end{block}
\usepackage{mumie.genericproblem}


     \begin{problem}
        \begin{variables}
            \drawFromSet{a}{2,3,4}
            
            \function[normalize]{p}{a*(30-2*x)}
            \function{dk}{k/x}
            \function{gk}{6*a}
            \function{er}{p*x}
            \derivative{ger}{er}{x}
            \number{n}{6}
            \string{k}{qx,xq,q*x,x*q}
            \string{kder}{qg'(y),g'(y)q,qg',g'q,q*g',g'*q,q*g'(y),g'(y)*q}
            \string{gder}{f'(x)}
            \string{kder2}{q/f'(x),q/f'}
        \end{variables}
          \begin{question}
          \type{input.text}
          \lang{de}{\text{Für eine Unternehmung gelte die Produktionsfunktion $y=f(x)$. Hierbei ist
          y der Output, x der Input eines Produktionsfaktors. Der Faktorpreis (Preis pro Einheit des 
          Produktionsfaktors) sei 
          konstant und betrage $q$.\\
          Ziel dieser Aufgabe ist, in mehreren Schritten den folgenden Zusammenhang herzuleiten:\\
          $\text{Grenzkosten}=\frac{\text{Faktorpreis}}{\text{Grenzproduktivität}}$
          }}
          \lang{en}{\text{Suppose a company's productivity is described by the production function $y=f(x)$. Here
          y is the output and x the input of a production factor. Let the factor price (price per unit of the 
          factor of production) be the constant $q$.\\
          The aim of this task is to derive the following relationship in several steps:\\
          $\text{marginal costs}=\frac{\text{factor price}}{\text{marginal productivity}}$
          }}
               \begin{answer}
                    \lang{de}{
                    \text{Die Kostenfunktion lautet: $K(x)$=}
                    }
                    \lang{en}{
                    \text{The cost function is $K(x)$=}
                    }
                    \solution{k}
                    \inputAsString{A1}
                    \checkStringsForRelation[,]{equalIgnoreCaseString(A1,k)} 
               \end{answer}
               \explanation{$K(x)=qx$}
          \end{question} 
         
          
          \begin{question}
          \type{input.text}
          \lang{de}{\text{Die Menge $x$ lässt sich aus der Umkehrfunktion zu $y=f(x)$ berechnen: 
          $x=f^{-1}(y)=g(y)$.
          Bestimmen Sie nun die Grenzkostenfunktion in Abhängigkeit von $y$: 
          }}
          \lang{en}{\text{The quantity $x$ can be calculated from the inverse function of $y=f(x)$: \\
          $x=f^{-1}(y)=g(y)$. \\
          Now determine the marginal cost function as a function of $y$: 
          }}
               \begin{answer}
                    \text{$K'(y)=$}
                    \solution{kder}
                    \inputAsString{A2}
                    \checkStringsForRelation[,]{equalIgnoreCaseString(A2,kder)} 
               \end{answer}
              
               \explanation{$K'(y)=qg'(y)$}
         \end{question}
         
            \begin{question}
            \type{input.text}
            \lang{de}{\text{Für die Ableitung der Umkehrfunktion gilt:\\ $g'(y(x))=1/$\ansref}}
            \lang{en}{\text{The derivative of the inverse function satisfies\\ $g'(y(x))=1/$\ansref}}
            \begin{answer}
                    %\text{$g'(y(x))=$}
                    \solution{gder}
                    \inputAsString{A3}
                    \checkStringsForRelation[,]{equalIgnoreCaseString(A3,gder)} 
               \end{answer}
            \explanation{$g'(y(x))=\frac{1}{f'(x)}$}
         \end{question}
         
         \begin{question}
            \type{input.text}
            \lang{de}{\text{Das Ergebnis aus c) setzen wir nun in b) ein und erhalten den 
            gewünschten Zusammenhang:}}
            \lang{en}{\text{Substitute the result from c) into b) to obtain the desired relationship:}}
            \begin{answer}
                    \text{$K'(y(x))=$}
                    \solution{kder2}
                    \inputAsString{A4}
                    \checkStringsForRelation[,]{equalIgnoreCaseString(A4,kder2)} 
               \end{answer}
            \explanation{$K'(y(x))=\frac{q}{f'(x)}$}
         \end{question}
         
     \end{problem}

\embedmathlet{gwtmathlet}

\end{content}
