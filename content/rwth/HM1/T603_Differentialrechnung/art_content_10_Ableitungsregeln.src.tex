%$Id:  $
\documentclass{mumie.article}
%$Id$
\begin{metainfo}
  \name{
    \lang{de}{Berechnung der Ableitung}
    \lang{en}{}
  }
  \begin{description} 
 This work is licensed under the Creative Commons License Attribution 4.0 International (CC-BY 4.0)   
 https://creativecommons.org/licenses/by/4.0/legalcode 

    \lang{de}{Beschreibung}
    \lang{en}{}
  \end{description}
  \begin{components}
    \component{js_lib}{system/media/mathlets/GWTGenericVisualization.meta.xml}{mathlet1}
  \end{components}
  \begin{links}
    \link{generic_article}{content/rwth/HM1/T601_GrundlagenWiWi/g_art_content_04_Funktionsbegriff.meta.xml}{content_04_Funktionsbegriff}
    \link{generic_article}{content/rwth/HM1/T301_Differenzierbarkeit/g_art_content_02_ableitungsregeln.meta.xml}{content_02_ableitungsregeln}
    \link{generic_article}{content/rwth/HM1/T301_Differenzierbarkeit/g_art_content_01_differenzenquotient.meta.xml}{diffbarkeit}
    \link{generic_article}{content/rwth/HM1/T210_Stetigkeit/g_art_content_31_grenzwerte_von_funktionen.meta.xml}{grenzw-funk}
    \link{generic_article}{content/rwth/HM1/T204_Abbildungen_und_Funktionen/g_art_content_11_injektiv_surjektiv_bijektiv.meta.xml}{umkehrfkt}
  \end{links}
  \creategeneric
\end{metainfo}
\begin{content}

\begin{block}[annotation]
	Im Ticket-System: \href{https://team.mumie.net/issues/22698}{Ticket 22698}
\end{block}


\usepackage{mumie.ombplus}
\ombchapter{3}
\ombarticle{2}
\usepackage{mumie.genericvisualization}

\begin{visualizationwrapper}

\title{
\lang{de}{Berechnung der Ableitung}
\lang{en}{Computing derivatives}
}
 

\begin{block}[info-box]
\tableofcontents
\end{block}

\section{
\lang{de}{Ableitungsregeln}
\lang{en}{Differentiation rules}
}

\lang{de}{In diesem Kapitel beschäftigen wir uns mit der Frage, wie man die Ableitung von Funktionen konkret berechnen kann. 
Dazu werden wir für einige Funktionen die Ableitung angeben und Regeln formulieren, wie man Summen, Produkte, Quotienten, etc. dieser 
Funktionen ableiten kann. Dadurch werden wir bereits eine große Anzahl an Funktionen ableiten können, ohne dass wir 
den komplizierten Rechenweg mit dem Differenzenquotienten aus dem vorherigen Kapitel wählen müssen. 
}
\lang{en}{
In this chapter, we will consider the problem of actually
computing the derivative of a given function.
We will first find the derivatives of a number of functions
and then give rules for computing the derivatives of their
sums, products, quotients, etc.
This will allow us to calculate the derivatives
of a wide variety of functions without using the more
difficult method of difference quotients from the previous chapter.
}


\lang{de}{Nichtsdestotrotz gilt aber auch: Wenn wir es mit einer Funktion zu tun haben, bei der wir keine Ableitungsregel anwenden können, 
bleibt nur der Grenzwert des Differenzenquotienten, also die Definition der Ableitung, um zumindest einen Näherungswert für 
die Ableitung zu erhalten. }
\lang{en}{
Nevertheless: when we are faced with a function to which none
of the rules of differentiation can be applied, the limit of
difference quotients, i.e. the definition of the derivative,
will remain the only option to compute the derivative or at least
an approximation to it.
}

\lang{de}{Zur Erinnerung sind hier noch einmal die Ableitungen der Funktionen dargestellt, die wir nun als \glqq bekannt\grqq\ ansehen wollen.
}
\lang{en}{
Here are the derivatives of a number
of functions that we have seen before
and that we will assume to be known from now on:
}

%\begin{rule}
\begin{align*}
\underline{\text{\lang{de}{Funktion}\lang{en}{Function}}\; f(x)}&\hspace{20pt}&  \underline{\text{\lang{de}{Ableitung}\lang{en}{Derivative}}\; f'(x)}&\hspace{20pt}&\underline{\text{\lang{de}{Bedingung an }\lang{en}{Conditions on }}\, x}\\
c \;(c\in\R) &&0&&\\
x^n\;( n\in\N)&&nx^{n-1}&&\\
x^n\;(n\in\Z, n<0)&&nx^{n-1}&& x\neq 0\\
x^r\;( r\in\R)&&rx^{r-1}&& x>0\\
\sqrt{x}=x^{1/2} &&\frac{1}{2\sqrt{x}}=\frac{1}{2}x^{-1/2}&& x>0\\
e^x&&e^x&&\\
\ln(x)&&\frac{1}{x}&&x>0\\
a^x \;(a>0)&&\ln a\cdot a^x&&\\
\sin(x)&&\cos(x)&&\\
\cos(x)&&-\sin(x)&&\\
\tan(x)&& \frac{1}{\cos(x)^2} = 1 + \tan(x)^2 && \cos(x) \neq 0 
\end{align*}
%\end{rule}

\lang{de}{Die ersten Ableitungsregeln behandeln den Fall, dass Funktionen addiert werden oder eine Funktion mit einer Zahl multipliziert wird. }
\lang{en}{
The first differentiation rules will involve adding functions or multiplying a function by a constant.
}

\begin{rule}[\lang{de}{Summen- und Faktorregel} \lang{en}{Addition and scalar multiplication rule}] \label{rule:summenregel}

\lang{de}{Es seien $f$ und $g$ zwei Funktionen, die auf einem offenen Intervall $I$ differenzierbar sind, und es sei $c\in \R$ eine reelle Zahl. 
}
\lang{en}{
Suppose $f$ and $g$ are functions that are differentiable on
an open interval $I$, and let $c \in \R$ be a real number.
}


\lang{de}{Dann sind auch die Funktionen
$f+g$, $f-g$ und $c \cdot f$ auf $I$ differenzierbar und für alle $x\in I$ gelten
\begin{align*}
(f+g)'(x) &= f'(x) + g'(x),  &\quad & \text{(Summenregel)} \\
(f-g)'(x) &= f'(x) - g'(x), && \\
(c \cdot f)'(x) &= c\cdot  f'(x).  &\quad & \text{(Faktorregel)}
\end{align*}
}
\lang{en}{
Then the functions $f+g$, $f-g$ and $c \cdot f$
are also differentiable on $I$, and for any $x \in I$,
\begin{align*}
(f+g)'(x) &= f'(x) + g'(x),  &\quad & \text{(Addition rule)} \\
(f-g)'(x) &= f'(x) - g'(x), && \\
(c \cdot f)'(x) &= c\cdot  f'(x).  &\quad & \text{(Scalar multiplication rule)}
\end{align*}
}

\lang{de}{
Damit gilt auch
\[
\left( c_1 \cdot f_1 + c_2 \cdot f_2 + \ldots + c_n \cdot f_n \right)'(x) = c_1 \cdot f_1'(x) + c_2 \cdot f_2'(x) + \ldots + c_n \cdot f_n'(x)
\]
für Summen von differenzierbaren Funktionen.
}

\lang{en}{
In particular, for the sum of differentiable functions,
we have 
\[
\left( c_1 \cdot f_1 + c_2 \cdot f_2 + \ldots + c_n \cdot f_n \right)'(x) = c_1 \cdot f_1'(x) + c_2 \cdot f_2'(x) + \ldots + c_n \cdot f_n'(x).
\]
}


\end{rule}

\lang{de}{Zu beachten ist, dass die Regel nicht nur aussagt, wie die Ableitung berechnet wird, sondern auch, dass die Ableitung 
überhaupt existiert. }

\lang{en}{
Note that the above rule does not only describe how the
derivative is computed; it also claims that the derivative
exists in the first place.
}

\begin{proof*}
\lang{de}{
Ein Beweis der Summen- und Faktorregel findet sich im \ref[content_02_ableitungsregeln][Hauptkurs HM4MINT]{rule:summenregel}.
}
\lang{en}{
A proof of the addition and scalar multiplication rules
can be found in the \ref[content_02_ableitungsregeln][main HM4MINT course]{rule:summenregel}.
}
\end{proof*}

\begin{example}
\begin{tabs*}[\initialtab{0}]

\tab{\lang{de}{Gerade} \lang{en}{Lines}}
\lang{de}{
Wir berechnen die Ableitung einer allgemeinen Gerade $f(x) = mx + b$ mit festen Zahlen $m$ und $b$. 
Nach der Tabelle oben ist die Ableitung von $x^1$ das gleiche wie $1 \cdot x^0 = 1$ und die Ableitung von $b$ ist $0$. Mit 
Summen- und Faktorregel folgt daher 
\[
f'(x) = m \cdot 1 + 0 = m.
\]
Die Ableitung einer Gerade ist also genau ihre Steigung. 
}
\lang{en}{
Let us consider the derivative of the equation defining a general
line, $f(x) = mx + b$, for fixed numbers $m$ and $b$.
According to the above table, the derivative of $x^1$ is simply $1 \cdot x^0 = 1$ and the
derivative of $b$ is $0$. Using the rule for addition and scalar multiplication,
\[
f'(x) = m \cdot 1 + 0 = m.
\]
In other words, the derivative of the equation of a line is its slope.
}


\tab{\lang{de}{Polynom} \lang{en}{Polynomial}}
\lang{de}{Wir berechnen die Ableitung der Funktion $f(x)=x^3+2x+5$. Diese Funktion ist die Summe der 
Funktionen $x^3$ und $2x$ und der konstanten Funktion $5$. Die Ableitung von $x^3$ ist nach der Tabelle oben $3x^2$, die
Ableitung von $2x$ ist $2$ und die Ableitung der konstanten Funktion $5$ ist $0$. Mit der Summenregel erhalten wir
}
\lang{en}{
We will compute the derivative of the
function $f(x) = x^3 + 2x + 5$.
This function is the sum of the functions
$x^3$ and $2x$ and the constant function $5$.
According to the table above, the derivative
of $x^3$ is $3x^2$, the derivative of $2x$
is $2$, and the derivative of the constant
function $5$ is $0$. Using the addition rule,
we find
}
\[f'(x)=3x^2+2+0=3x^2+2.\]

\tab{\lang{de}{Grenzgewinn} \lang{en}{Marginal profit}}
\lang{de}{Bei der Produktion von $x$ Einheiten eines Gutes steht der Erlös $E(x) = \sqrt{x} + x$ Kosten 
in Höhe von $K(x) = 0,5 \cdot x$ gegenüber. Die Gewinnfunktion erhalten wir als Differenz
\[
G(x) = E(x) - K(x) = \sqrt{x} + x - 0,5x = \sqrt{x} + 0,5x = x^{\frac{1}{2}} + 0,5x .
\]}
\lang{en}{
The production of $x$ units of a good yields
a revenue of $E(x) = \sqrt{x} + x$ but
entails costs of $K(x) = 0.5 \cdot x$.
The profit function is obtained as the difference
\[
G(x) = E(x) - K(x) = \sqrt{x} + x - 0.5x = \sqrt{x} + 0.5x = x^{\frac{1}{2}} + 0.5x .
\]
}


\lang{de}{Der \emph{Grenzgewinn} gibt den zusätzlichen Gewinn an, wenn eine weitere Einheit 
des Gutes verkauft wird. Bei einem steilen Anstieg der Gewinnfunktion ist der Grenzgewinn hoch, 
bei einem flachen Verlauf der Gewinnfunktion ist der Grenzgewinn niedrig. 
Mathematisch können wir den Grenzgewinn als Ableitung der Gewinnfunktion berechnen. 
}
\lang{en}{The \emph{marginal profit} indicates
the additional profit obtained by
selling one more unit of the good.
The marginal profit is high when the
profit function increases steeply and is
low when the profit function is flat.
We can calculate the marginal profit
mathematically as the derivative of the
profit function.
}


\lang{de}{In diesem Fall können wir die Grenzgewinnfunktion mit Hilfe der Summen- und Faktorregel und 
der obigen Tabelle berechnen:
\[
G'(x) = \frac{1}{2} x^{-\frac{1}{2}} + 0,5 = \frac{1}{2 \sqrt{x}} + 0,5.
\]}
\lang{en}{
In this case, we can compute the marginal
profit function using the addition and
scalar multiplication rules and the table above:
\[
G'(x) = \frac{1}{2} x^{-\frac{1}{2}} + 0.5 = \frac{1}{2 \sqrt{x}} + 0.5.
\]
}

\end{tabs*}
\end{example}



\begin{quickcheckcontainer}
\randomquickcheckpool{1}{2}
\begin{quickcheck}
		\field{rational}
		\type{input.function}
		\begin{variables}
			\randint[Z]{n}{-9}{9}
			\randint[Z]{a}{2}{9}
			\randint[Z]{b}{-6}{9}
		    \function[normalize]{f}{a*sin(x)+b*x^n}
			\function[normalize]{ff}{a*cos(x)+n*b*x^(n-1)}
            \function[normalize]{g}{n*x^(n-1)}
		\end{variables}

    \lang{de}{
		\text{Die Ableitungsfunktion der Funktion $f(x)=\var{f}$ ist $f'(x)= $\ansref.}
    }
    \lang{en}{
    \text{The derivative of the function $f(x)=\var{f}$ is\\ $f'(x)= $\ansref.}
    }
    
		\begin{answer}
			\solution{ff}
			\checkAsFunction{x}{-1}{1}{10}
		\end{answer}

    \lang{de}{
		\explanation{Die Ableitungsfunktion von $\sin(x)$ ist $\cos(x)$ und die von $x^{\var{n}}$
        ist $\var{g}$. Dann muss man die Summen- und Faktorregel anwenden.
		}
    \lang{en}{
    \explanation{The derivative of $\sin(x)$ is $\cos(x)$
    and the derivative of $x^{\var{n}}$ is $\var{g}$.
    Now use the rules for addition and scalar multiplication.}
    }
    }
	\end{quickcheck}
    
   \begin{quickcheck}
		\field{rational}
		\type{input.function}
		\begin{variables}
			\randint[Z]{n}{-9}{9}
			\randint[Z]{a}{2}{9}
			\randint[Z]{b}{-6}{9}
		    \function[normalize]{f}{a*cos(x)+b*x^n}
			\function[normalize]{ff}{-a*sin(x)+n*b*x^(n-1)}
            \function[normalize]{g}{n*x^(n-1)}
		\end{variables}

    \lang{de}{
		\text{Die Ableitungsfunktion der Funktion $f(x)=\var{f}$ ist $f'(x)= $\ansref.}
		}
    \lang{en}{
    \text{The derivative of the function $f(x)=\var{f}$ is $f'(x)= $\ansref.}
    }
  
		\begin{answer}
			\solution{ff}
			\checkAsFunction{x}{-1}{1}{10}
		\end{answer}
    \lang{de}{
		\explanation{Die Ableitungsfunktion von $\cos(x)$ ist $-\sin(x)$ und die von $x^{\var{n}}$
        ist $\var{g}$. Dann muss man die Summen- und Faktorregel anwenden.
        }
    }
    \lang{en}{
    \explanation{The derivative of $\cos(x)$ is $-\sin(x)$
    and the derivative of $x^{\var{n}}$ is $\var{g}$.
    Now use the rules for addition and scalar multiplication.}
    }
	\end{quickcheck}
    
    
\end{quickcheckcontainer} 

\lang{de}{Wie Produkte und Quotienten differenzierbarer Funktionen abgeleitet werden, beschreiben die \emph{Produktregel} und die \emph{Quotientenregel}.}
\lang{en}{The \emph{product rule} and \emph{quotient rule} describe
how to differentiate products and quotients of differentiable functions.}
\begin{rule}[
\lang{de}{Produkt- und Quotientenregel}
\lang{en}{Product and quotient rules}]
\lang{de}{Es seien $f$ und $g$ zwei Funktionen, die auf einem offenen Intervall $I$ differenzierbar sind. 
}
\lang{en}{Let $f$ and $g$ be two functions
that are differentiable on an open interval $I$.}

\lang{de}{Dann ist auch das Produkt $f\cdot g$ auf $I$ differenzierbar und f\"ur alle $x\in I$ gilt 
 \[(f\cdot g)'(x)=f'(x)\cdot g(x)+f(x)\cdot g'(x).\]
}
\lang{en}{
Then the product $f \cdot g$ is also differentiable on $I$,
and for every $x \in I$,
 \[(f\cdot g)'(x)=f'(x)\cdot g(x)+f(x)\cdot g'(x).\]
}

\lang{de}{Ist außerdem $g(x) \neq 0$ für alle $x \in I$, dann ist auch der Quotient $\frac{f}{g}$ auf $I$ differenzierbar mit
\[ \left(\frac{f}{g}\right)^'(x)= \frac{f'(x)\cdot g(x)-f(x)\cdot g'(x)}{g(x)^2} \]
für alle $x \in I$. }
\lang{en}{
If in addition $g(x) \ne 0$ for all $x \in I$,
then the quotient $\frac{f}{g}$ is also
differentiable on $I$ and
\[ \left(\frac{f}{g}\right)^'(x)= \frac{f'(x)\cdot g(x)-f(x)\cdot g'(x)}{g(x)^2} \]
for all $x \in I$.
}

\end{rule}
\lang{de}{Die Bedingung $g(x) \neq 0$ wird nur gebraucht, um sicherzustellen, dass man nicht durch $0$ teilt. 
}
\lang{en}{The condition $g(x) \neq 0$ is
only necessary to ensure that we never divide by $0$.}

\begin{proof*}
\lang{de}{Ein Beweis der Produkt- und der Quotientenregel findet sich im \ref[content_02_ableitungsregeln][Hauptkurs HM4MINT]{rule:produkt_quotient_regel}.
}
\lang{en}{A proof of the product and quotient
rules can be found in the \ref[content_02_ableitungsregeln][main HM4MINT course]{rule:produkt_quotient_regel}}.
\end{proof*}

 \begin{example}%\textit{Beispiel:}\\
 \begin{tabs*}[\initialtab{0}]
\tab{$x^2\cdot \sin(x)$}
\lang{de}{Wir berechnen als erstes Beispiel die Ableitung der Funktion $f(x)={x^2}\cdot \sin x$. Diese ist das Produkt der differenzierbaren Funktionen 
$g(x)={x^2}$ und $h(x)=\sin(x)$. Die Ableitung von $g(x)$ ist $2x$, die Ableitung von $h(x)$ ist $\cos(x)$. Dann ist 
\[f'(x)=g'(x)\cdot h(x)+g(x)\cdot h'(x)=2x\cdot\sin(x)+{x^2}\cdot \cos(x).\]
}
\lang{en}{
As a first example, we will compute the derivative of the function
$f(x) = x^2 \cdot \sin(x)$. This is the product of the
differentiable functions $g(x)={x^2}$ and $h(x)=\sin(x)$.
The derivative of $g(x)$ is $2x$ and the derivative
of $h(x)$ is $\cos(x)$. Therefore,
\[f'(x)=g'(x)\cdot h(x)+g(x)\cdot h'(x)=2x\cdot\sin(x)+{x^2}\cdot \cos(x).\]
}


\tab{\lang{de}{Potenz $x^n$}
\lang{en}{Powers $x^n$}}
\lang{de}{Für die Funktion $f$ mit $f(x) = x^2$ können wir einerseits oben nachsehen, dass die Ableitung 
durch $f'(x) = 2x$ gegeben ist. Wir können aber auch mit der Produktregel die Ableitung berechnen und 
brauchen dann nur zu wissen, dass $x$ abgeleitet zu $1$ wird. Dann folgt mit Produktregel:
\[
f(x) = x \cdot x, \quad f'(x) = 1 \cdot x  + x \cdot 1 = 2x.
\]}
\lang{en}{Let $f$ be the function $f(x) = x^2$.
On one hand, we can see from above that its derivative
is given by $f'(x) = 2x$. However, we can
also compute the derivative using the product
rule, using only the fact that the derivative
of $x$ is $1$. The product rule yields
\[
f(x) = x \cdot x, \quad f'(x) = 1 \cdot x  + x \cdot 1 = 2x.
\]}


\lang{de}{Für die Funktion $g(x) = \frac{1}{x}$ können wir die Ableitung aus der obigen Tabelle ablesen, indem wir
die Funktion umformen zu $g(x) = x^{-1}$. Wir können aber auch die Ableitung mit der Quotientenregel bestimmen.
Es gilt dann: 
\[
g'(x) = \frac{0 \cdot x - 1 \cdot 1}{x^2} = \frac{-1}{x^2}.
\]}
\lang{en}{
Let $g$ be the function $g(x) = \frac{1}{x}$.
We can find its derivative from the above table
by rewriting it in the form $g(x) = x^{-1}$.
However, we can also compute its derivative
using the quotient rule. We have:
\[
g'(x) = \frac{0 \cdot x - 1 \cdot 1}{x^2} = \frac{-1}{x^2}.
\]
}

\lang{de}{
Dieses Beispiel soll demonstrieren, dass es häufig unterschiedliche Rechenwege gibt, die aber 
alle auf das gleiche Ergebnis führen. }
\lang{en}{
This example demonstrates that
there can often be different ways to compute a derivative,
all of which lead to the same solution.
}

\tab{\lang{de}{Grenzerlös}
\lang{en}{Marginal revenue}}
\lang{de}{Gegeben ist die Preis-Absatz-Funktion $p(x) = \ln(x) + 10$, die zugehörige Erlösfunktion ist
\[
E(x) = p(x) \cdot x = (\ln(x) + 10) \cdot x = x \ln(x) + 10x.
\]}
\lang{en}{
Suppose we are given the inverse demand
function $p(x) = \ln(x) + 10$,
with associated revenue function
\[
E(x) = p(x) \cdot x = (\ln(x) + 10) \cdot x = x \ln(x) + 10x.
\]
}

\lang{de}{Der \emph{Grenzerlös} gibt den zusätzlichen Erlös an, wenn eine weitere Einheit 
des Gutes verkauft wird. 
Mathematisch können wir den Grenzerlös als Ableitung der Erlösfunktion berechnen. Mit Hilfe der 
Produktregel erhalten wir: }
\lang{en}{
The \emph{marginal revenue} indicates the
additional revenue obtained from selling
one additional unit of the good.
We can compute the marginal revenue
mathematically as the derivative of the
revenue function. Using the product rule,
we find:
}
\[
E'(x) = 1 \cdot \ln(x) + x \cdot \frac{1}{x} + 10 = \ln(x) + 11.
\]
\end{tabs*}
\end{example}

\begin{quickcheckcontainer}
\randomquickcheckpool{1}{1}
\begin{quickcheck}
		\field{rational}
		\type{input.function}
		\begin{variables}
			\randint[Z]{a}{-2}{2}
			\randint[Z]{b}{1}{4}
			\randint{l}{1}{4}
			\function[calculate]{l1}{l-1}
			
			\randint{k}{1}{4}	% Zufallsvariable zum Vertauschen:
			\function[calculate]{d1}{-(k-2)*(k-3)*(k-4)/6}  % "Dirac"-funktionen
			\function[calculate]{d2}{(k-1)*(k-3)*(k-4)/2}
			\function[calculate]{d3}{-(k-1)*(k-2)*(k-4)/2}
			\function[calculate]{d4}{(k-1)*(k-2)*(k-3)/6}
			
			\function[normalize]{f1}{sin(x)}
			\function[normalize]{f2}{cos(x)}
			\function[normalize]{f3}{e^x}
			\function[normalize]{f4}{ln(x)}

			\function[expand,normalize]{g}{d1*f1+d2*f2+d3*f3+d4*f4}						
			\function[expand,normalize]{dg}{d1*f2+d2*(-f1)+d3*f3+d4*(1/x)}

			\function[normalize]{p}{a*x^l+b}
			\function[normalize]{dp}{l*a*x^(l-1)}
		    \function[normalize]{f}{p*g}
			\function[normalize]{df}{dp*g+p*dg}
		\end{variables}

      \lang{de}{
			\text{Die Ableitungsfunktion der
			Funktion $f(x)=\var{f}$ ist $f'(x)=$ \ansref.}
      }
      \lang{en}{
      \text{The derivative of the function
      $f(x)=\var{f}$ is $f'(x)=$ \ansref.}
      }
		\begin{answer}
			\solution{df}
			\checkAsFunction{x}{-2}{2}{20}
		\end{answer}
    \lang{de}{
		\explanation{Nach der Produktregel gilt $f'(x)=(\var{p})'\cdot \var{g}+(\var{p})\cdot (\var{g})'$. 
		}
    \lang{en}{
    \explanation{By the product rule, $f'(x)=(\var{p})'\cdot \var{g}+(\var{p})\cdot (\var{g})'$. }
    }
    }
		
	\end{quickcheck}
\end{quickcheckcontainer}



\begin{quickcheck}
		\field{rational}
		\type{input.function}
		\begin{variables}
			\randint[Z]{n}{-9}{9}
			\randint[Z]{a}{-9}{9}
			\randint[Z]{b}{-6}{9}
			\randint[Z]{c}{-6}{9}
			\randint[Z]{d}{-6}{9}
		    \function[normalize]{f}{(x^2+a*x+b)/(c*x+d)}            
			\derivative[normalize]{df}{f}{x}
            \function[normalize]{dfz}{(2*x+a)*(c*x+d)-(x^2+a*x+b)*c}
            \function[normalize]{dfn}{(c*x+d)^2}
		\end{variables}

      \lang{de}{
			\text{Die Ableitungsfunktion der
			Funktion $f(x)=\var{f}$ ist $f'(x)=$ \ansref.}
  		}
      \lang{en}{
      \text{The derivative of the function
      $f(x)=\var{f}$ is $f'(x)=$ \ansref.}
      }
    \begin{answer}
			\solution{df}
			\checkAsFunction{x}{-1}{1}{10}
		\end{answer}
    \lang{de}{
		\explanation{Nach der Quotientenregel ist die Ableitungsfunktion gegeben durch
        $\frac{\var{dfz}}{\var{dfn}}$.
        }
    }
    \lang{en}{
    \explanation{By the quotient rule, the derivative
    is given by $\frac{\var{dfz}}{\var{dfn}}$.}
    }
	\end{quickcheck}



\lang{de}{Ein sehr wichtiger Fall betrifft Funktionen, die ineinander eingesetzt sind, also \ref[content_04_Funktionsbegriff][verkettete]{kompositionv} Funktionen. 
Die Ableitung solcher Funktionen lässt sich mit der 
\emph{Kettenregel} berechnen.}
\lang{en}{
Functions that have been substituted into one another,
i.e. \ref[content_04_Funktionsbegriff][compositions]{kompositionv} of functions,
are a very important family of examples.
The derivative of such a function can be computed
with the \emph{chain rule}.
}

\begin{rule}[
\lang{de}{Kettenregel}
\lang{en}{Chain rule}]

\lang{de}{Es seien $g$ und $h$ Funktionen mit der Eigenschaft, dass $h$ differenzierbar an $x_0 \in \R$ und $g$ differenzierbar an $h(x_0) \in \R$ ist. 
}
\lang{en}{
Let $g$ and $h$ be functions with the property
that $h$ is differentiable in $x_0 \in \R$
and $g$ is differentiable in $h(x_0) \in \R$.
}

\lang{de}{
Dann ist auch die Verkettung $f = g \circ h$, $f(x) = g(h(x))$, an der Stelle $x_0$ differenzierbar und es gilt
\[f'(x_0)=g'(h(x_0))\cdot h'(x_0).\]
}
\lang{en}{
Then the composition $f = g \circ h$,
$f(x) = g(h(x))$ is also differentiable
in $x_0$, and \[f'(x_0)=g'(h(x_0))\cdot h'(x_0).\]
}

\end{rule}
\lang{de}{Die Kettenregel wird auch häufig mit dem Slogan \notion{ \glqq innere Ableitung mal äußere Ableitung\grqq} beschrieben. Die äußere Funktion ist $g$, 
in die der Wert der inneren Funktion $h$ eingesetzt wird.}
\lang{en}{
The chain rule can be described by the catchphrase
"inner derivative times outer derivative".
The outer function is $g$, and the value of the
inner function $h$ is substituted into it.
}

\begin{proof*}
\lang{de}{Ein Beweis der Kettenregel findet sich im \ref[content_02_ableitungsregeln][Hauptkurs HM4MINT]{rule:kettenregel}.
}
\lang{en}{A proof of the chain rule can
be found in \ref[content_02_ableitungsregeln][the main HM4MINT course]{rule:kettenregel}.
}

\end{proof*}


\begin{example}
\begin{tabs*}[\initialtab{0}]
\tab{$\sin(x^2)$}
\lang{de}{Wir berechnen als Beispiel die Ableitung der Funktion $f(x)=\sin(x^2)$. Setzen wir als \"{a}u{\ss}ere Funktion $g(y)=\sin(y)$ und als innere Funktion 
$h(x)=x^2$, so ist $g(h(x))=\sin(x^2)$. Wir k\"{o}nnen also die Kettenregel benutzen. Zun\"{a}chst ist $g'(y)=\cos(y)$. 
F\"{u}r $y=h(x)=x^2$ ergibt das 
\[g'(h(x))=\cos(h(x))=\cos(x^2).\] Nun m\"{u}ssen wir noch die Ableitung von $h$ berechnen. Wir erhalten
\[h'(x)=2x.\]
Nach der Kettenregel ist dann die Ableitung von $f(x) = g(h(x))$ gegeben durch
\[(g\circ h)'(x)=g'(h(x))\cdot h'(x)=\cos(x^2)\cdot 2x=2x\cos (x^2).\]
Wenn man ein wenig Erfahrung beim Ableiten von Funktionen gesammelt hat, braucht man nat\"{u}rlich nicht mehr diese vielen kleinen Einzelschritte 
zu gehen! Die Rechnung kann man in etwa auf die letzte Zeile des Beispieles reduzieren.}
\lang{en}{
As an example, we will compute the derivative of the function $f(x)=\sin(x^2)$. We set
$g(y)=\cos(y)$ as the outer function and $h(x)=x^2$ as the inner function,
such that $g(h(x))=\sin(x^2)$, and then we use the chain rule.
First we observe that $g'(y)=-\sin(y)$. For $y=h(x)=x^2$, this implies
\[g'(h(x))=\cos(h(x))=\cos(x^2).\]
We still have to compute the derivative of $h$; we find
\[h'(x)=2x.\]
By the chain rule, the derivative of $f(x) = g(h(x))$ is given by
\[(g\circ h)'(x)=g'(h(x))\cdot h'(x)=\cos(x^2)\cdot 2x=2x\cos (x^2).\]
Once we have accumulated some experience differentiating functions, we will
not need to explain all of the steps in this level of detail!
The computation could be reduced to the last line of the example above.
}

\tab{$2^x$} 
\lang{de}{
F\"ur die Exponentialfunktion $f(x)=2^x$ können wir die Ableitung mit Hilfe der Kettenregel 
bestimmen. Dazu schreiben wir 
\[
f(x) = 2^x = e^{\ln(2^x)} = e^{x \cdot \ln(2)},
\] 
wobei wir ein Logarithmus-Gesetz benutzt haben und die Tatsache, dass $e$-Funktion und natürlicher Logarithmus sich gegenseitig aufheben. 
Für $g(x)=e^x$ als \"au"sere Funktion 
und $h(x)=x\cdot \ln(2)$ als innere Funktion ist
\[ f'(x)=g'(h(x))\cdot h'(x)=e^{h(x)}\cdot \ln(2)=e^{x\cdot \ln(2)}\cdot \ln(2) =2^x\cdot \ln(2). \]
}
\lang{en}{
We can compute the derivative of the
exponential function $f(x)=2^x$ using
the chain rule. First we write
\[
f(x) = 2^x = e^{\ln(2^x)} = e^{x \cdot \ln(2)},
\] 
using a logarithm rule and the fact that the
exp function and natural logarithm cancel each other out.
With $g(x) = e^x$ as the outer function and
$h(x) = x\cdot \ln(2)$ as the inner function,
we have
\[ f'(x)=g'(h(x))\cdot h'(x)=e^{h(x)}\cdot \ln(2)=e^{x\cdot \ln(2)}\cdot \ln(2) =2^x\cdot \ln(2).\]
}
\end{tabs*}
\end{example}



\begin{quickcheck}
		\field{rational}
		\type{input.function}
		\begin{variables}
			\randint{k}{1}{3}	% Zufallsvariable f\"ur Auswahl:
			\function[calculate]{d1}{-(k-2)*(k-3)*(k-4)/6}  % "Dirac"-funktionen
			\function[calculate]{d2}{(k-1)*(k-3)*(k-4)/2}
			\function[calculate]{d3}{-(k-1)*(k-2)*(k-4)/2}

			\randint{l}{1}{2}   %Zufallsvariable f\"ur Auswahl
			\function[calculate]{l1}{-(l-2)}  % "Dirac"-funktionen
			\function[calculate]{l2}{(l-1)}

			\function{h0}{x^2+1}
			\function[normalize]{h}{l1*x+l2*h0}

			\function[normalize]{f1}{sin(h)}
			\function[normalize]{f2}{cos(h)}
			\function[normalize]{f3}{exp(h)}
			\function[expand,normalize]{f}{d1*f1+d2*f2+d3*f3}
			\function[normalize]{df}{d1*f2+d2*(-f1)+d3*f3}  % Ableitung von f an der Stelle h

			% g=h0°f (falls l1=1) oder g=f°h0 (falls l2=1) mit f in fi
			\function[normalize]{g}{l2*f+l1*(f^2+1)}
			\function[expand,normalize]{dg}{l2*df*2*x+l1*(2*df*f)}
            \function[normalize]{ss}{l2*h0+l1*f}            
		\end{variables}

      \lang{de}{
			\text{Bestimmen Sie mit der Kettenregel die Ableitung der Funktion $f(x)=\var{g}$.\\
				Die Ableitung ist $f'(x)=$ \ansref.}
      }
      \lang{en}{
      \text{Using the chain rule, find the derivative of the function $f(x)=\var{g}$.\\
      The derivative is $f'(x)=$ \ansref.}
      }

     \begin{answer}
          \solution{dg}
%		  \allowForInput[false]{)}
          \checkAsFunction{x}{-2}{2}{20}
      \end{answer}
      \lang{de}{
      \explanation{Wählen Sie als innere Funktion die Funktion $h(x)=\var{ss}$.}
      }
      \lang{en}{
      \explanation{Use $h(x)=\var{ss}$ as the inner function.}
      }
\end{quickcheck}


%\begin{example}
%Mit Hilfe der Ableitung des Logarithmus (s.u.) und der Kettenregel lassen sich auch die Ableitungen reeller Potenzfunktionen $f(x)=x^r$ bestimmen.
%Hierzu beachte man, dass die reelle Potenz $x^r$ definiert ist als $\exp(r\cdot \ln(x))$. Damit gilt f\"ur die Ableitung
%\begin{eqnarray*}
%f'(x) &=& \exp'(r\ln(x))\cdot r\cdot \ln'(x) \\
%&=&  \exp(r\ln(x))\cdot r\cdot \frac{1}{x}\\ 
%&=& x^r\cdot r\cdot x^{-1}=rx^{r-1}
%\end{eqnarray*}
%\end{example}


\section{
\lang{de}{Regeln für Potenzreihen und Umkehrfunktionen}
\lang{en}{Rules for power series and inverse functions}
}

\lang{de}{Für Funktionen, die durch Potenzreihen gegeben sind, ist die Ableitung durch
summandenweises Ableiten gegeben. Dies können wir allerdings im Rahmen dieses Kurses nicht beweisen 
(die Summenregel oben gilt zunächst nur für Summen mit endlich vielen Summanden). Formal lautet die Aussage für Reihen wie folgt:
}
\lang{en}{
The derivative of a function that is given as a power series
can be found by differentiating each summand.
We will not be able to prove this in our course.
(The summation rule above only holds for sums with
finitely many summands). The formal statement for
series is as follows:
}

\begin{rule}[
\lang{de}{Ableitung von Potenzreihen}
\lang{en}{Differentiating power series}
]
\lang{de}{Ist $I$ ein offenes Intervall und $f:I\to \R$ eine Funktion, die auf $I$
 durch eine Potenzreihe der Form
\[  f(x)= \sum_{n=0}^\infty c_n(x-x_0)^n \]
gegeben ist, so ist $f$ auf $I$ differenzierbar und die Ableitung von $f$ ist gegeben durch
\[  f'(x)= \sum_{n=1}^\infty n\cdot c_n(x-x_0)^{n-1}.  \]
}
\lang{en}{
Let $I$ be an open interval and let $f:I\to \R$ be a function that
is represented on $I$ by a power series of the form
\[  f(x)= \sum_{n=0}^\infty c_n(x-x_0)^n. \]
Then $f$ is differentiable on $I$ with derivative
\[  f'(x)= \sum_{n=1}^\infty n\cdot c_n(x-x_0)^{n-1}.  \]
}
\end{rule}

\begin{example}[
\lang{de}{$e$-Funktion}
\lang{en}{Exponential function}]\label{ex:abl-umkehrfunktion}
\lang{de}{Wir kennen bereits die Reihendarstellung \[e^x = \sum_{n=0}^\infty \frac{1}{n!} \, x^n\]
der $e$-Funktion. Der Summand $\frac{1}{n!} x^n$ hat die Ableitung
\[
\frac{n}{n!} \, x^{n-1} = \frac{n}{1 \cdot 2 \cdots (n-1) \cdot n} \, x^{n-1} = \frac{1}{1 \cdot 2 \cdots (n-1)} \, x^{n-1} = \frac{1}{(n-1)!} \, x^{n-1}
\]
und nach obiger Regel erhalten wir nun
\[
\left( e^x \right)' \,=\, \sum_{n=1}^\infty \frac{1}{(n-1)!} \, x^{n-1} \,=\, \sum_{m=0}^\infty \frac{1}{m!} \, x^m \,=\, e^x,
\]
wobei im letzten Schritt der Summenindex $m=n-1$ eingesetzt wurde. 
}
\lang{en}{
We already know the series representation
\[e^x = \sum_{n=0}^\infty \frac{1}{n!} \, x^n\]
of the exponential function. Each summand $\frac{1}{n!}x^n$
has the derivative 
\[
\frac{n}{n!} \, x^{n-1} = \frac{n}{1 \cdot 2 \cdots (n-1) \cdot n} \, x^{n-1} = \frac{1}{1 \cdot 2 \cdots (n-1)} \, x^{n-1} = \frac{1}{(n-1)!} \, x^{n-1}.
\]
Therefore, the above rule yields
\[
\left( e^x \right)' \,=\, \sum_{n=1}^\infty \frac{1}{(n-1)!} \, x^{n-1} \,=\, \sum_{m=0}^\infty \frac{1}{m!} \, x^m \,=\, e^x.
\]
In the last step, we substituted $m=n-1$ in the index.
}

\lang{de}{Damit haben wir nun auch selbst nachrechnen können, dass die Ableitung der $e$-Funktion wieder die $e$-Funktion ist. 
}
\lang{en}{
We are thus able to verify that the derivative of $e^x$ is again $e^x$.
}

\end{example}

\lang{de}{Wenn wir es mit einer differenzierbaren Funktion zu tun haben, die eine \ref[content_04_Funktionsbegriff][Umkehrfunktion]{def:functionprops}  besitzt, dann können 
wir mit Hilfe der Kettenregel eine Formel f\"ur die Ableitung der Umkehrfunktion herleiten.
}
\lang{en}{
When we are given a differentiable function that admits
an \ref[content_04_Funktionsbegriff][inverse function]{def:functionprops},
the chain rule leads to a formula for the derivative of the inverse function.
}


\begin{rule}[
\lang{de}{Ableitung der Umkehrfunktion}
\lang{en}{Derivative of an inverse function}
]
\lang{de}{Ist $f$ umkehrbar und differenzierbar in $x_0$ und ist außerdem $f'(x_0)\neq 0$, so ist auch die Umkehrfunktion $f^{-1}$ differenzierbar in $y_0=f(x_0)$ und es gilt
}
\lang{en}{
Let $f$ be invertible and differentiable in $x_0$ and suppose $f'(x_0) \ne 0$.
Then the inverse function $f^{-1}$ is differentiable in $y_0 = f(x_0)$, and
}
\[  (f^{-1})'(y_0)=\frac{1}{f'(f^{-1}(y_0))}. \]
\end{rule}

\begin{proof*}
\lang{de}{Nach Definition der Umkehrfunktion ist $x=f^{-1}(f(x))$. Wenn wir beide Seiten der Gleichung ableiten und $x_0$ für $x$ einsetzen, 
ergibt sich mit der Kettenregel 
\[ 1= (f^{-1})'(f(x_0))\cdot f'(x_0). \]
Teile dann beide Seiten der Gleichung durch $f'(x_0)$ und ersetze $f(x_0)$ durch $y_0$ bzw. $x_0$ durch $f^{-1}(y_0)$:
\[ (f^{-1})'(y_0)=\frac{1}{f'(f^{-1}(y_0))}. \]
Den technischen Beweis, dass die Umkehrfunktion wirklich bei $y_0$ differenzierbar ist, lassen wir hier weg.
}
\lang{en}{
By definition, $x = f^{-1}(f(x))$. Differentiating both sides of
this equation and evaluating it at $x_0$ yields
\[ 1= (f^{-1})'(f(x_0))\cdot f'(x_0) \]
with the chain rule. Now we divide both sides of the
equation by $f'(x_0)$ and substitute $y_0$ for $f(x_0)$
and $f^{-1}(y_0)$ for $x_0$ to obtain
\[ (f^{-1})'(y_0)=\frac{1}{f'(f^{-1}(y_0))}. \]
The technical proof that the inverse function actually
exists is omitted here.
}
\end{proof*}



\begin{example}\label{ex:abl-umkehrfunktion}
\begin{tabs*}[\initialtab{0}]
\tab{$\sqrt[n]{x}$}
\lang{de}{Die $n$-te Wurzelfunktion $g(x)=\sqrt[n]{x}=x^{\frac{1}{n}}$ ist die Umkehrfunktion der $n$-ten Potenzfunktion $f(x)=x^n$.
F\"ur $x>0$ ist letztere differenzierbar mit $f'(x)=nx^{n-1}\neq 0$. Also ist auch die Wurzelfunktion f\"ur alle $x>0$ differenzierbar und es gilt:
}
\lang{en}{
The $n$th root function $g(x) = \sqrt[n]{x} = x^{\frac{1}{n}}$
is the inverse function of the $n$th power $f(x) = x^n$.
The latter is differentiable for $x > 0$ with $f'(x) = nx^{n-1} \ne 0$.
Therefore, the $n$th root function is also differentiable for $x > 0$,
and
}
\begin{eqnarray*}
g'(x) &=& \frac{1}{f'(g(x))}=\frac{1}{n(g(x))^{n-1}} \\
&=& \frac{1}{n\cdot \sqrt[n]{x}^{n-1}}=\frac{1}{n}\cdot x^{- \frac{n-1}{n}}\\
&=& \frac{1}{n}\cdot x^{\frac{1}{n}-1}.
\end{eqnarray*} 
\tab{$\ln(x)$}
\lang{de}{Der nat\"urliche Logarithmus $g(x)=\ln(x)$ ist die Umkehrfunktion der Exponentialfunktion $f(x)= \exp(x)=e^x$. Wegen $\exp'(x)=\exp(x)>0$ f\"ur alle $x\in \R$ ist
auch der nat\"urliche Logarithmus auf seinem gesamten Definitionsbereich $\R^+ =\{ x\in \R\,|\, x>0\}$ differenzierbar und es gilt}
\lang{en}{
The natural logarithm $g(x)=\ln(x)$ is the inverse function of the exponential function $f(x) = \exp(x) = e^x$.
Since $\exp'(x) = \exp(x) > 0$ for all $x \in \R$, the natural logarithm is differentiable
on its entire domain $\R^+ =\{ x\in \R\,|\, x>0\}$, and
}
\[ \left(\ln(x)\right)' =  \frac{1}{f'(g(x))}= \frac{1}{\exp(\ln(x))}=\frac{1}{x}, \]
\lang{de}{wie es auch in der Tabelle oben steht.}
\lang{en}{as in the table above.}
\end{tabs*}
\end{example}

\begin{quickcheck}
		\field{rational}
		\type{input.function}
		\begin{variables}
			\randint[Z]{a0}{1}{5}
            \number{a1}{1}
            \function[calculate]{a}{1+a0/a1}
            \function[calculate]{b}{a+1/a}
		    \function[normalize]{f}{x+1/x}            
			\derivative[normalize]{df}{f}{x}
            \function[calculate]{dgb}{a^2/(a^2-1)}
		\end{variables}

    \lang{de}{
		\text{Die Funktion $f:(1,\infty)\to \R$ mit $f(x)=x+\frac{1}{x}$ besitzt eine reelle
        Umkehrfunktion $g:(2,\infty)\to \R$.
        
        Bestimmen Sie die Ableitung von $g$ an der 
        Stelle $y=f(\var{a})=\var{b}$.\\ Die Ableitungsfunktion von $f$ ist $f'(x)=$ \ansref.\\
          Damit ist $g'(\var{b})=$\ansref. }
          }
    \lang{en}{
    \text{The function $f:(1,\infty)\to \R$, $f(x) = x + \frac{1}{x}$ admits a real inverse function
    $g:(2,\infty)\to \R$.
    
    Determine the derivative of $g$ in the point $y=f(\var{a})=\var{b}$.\\
    The derivative of $f$ is $f'(x)=$ \ansref.\\
    Therefore, $g'(\var{b})=$\ansref.}
    }
		\begin{answer}
			\solution{df}
			\checkAsFunction{x}{-1}{1}{10}
		\end{answer}
		\begin{answer}
			\solution{dgb}
			\checkAsFunction{x}{-1}{1}{10}
		\end{answer}
    \lang{de}{
		\explanation{Nach der Regel für die Ableitung der Umkehrfunktion ist
        $g'(\var{b})=\frac{1}{f'(\var{a})}=\frac{1}{1-\frac{1}{\var{a}^2}}=\frac{\var{a}^2}{\var{a}^2-1}$.
        }
        }
    \lang{en}{
    \explanation{By the rule for the derivative of an inverse function,
        $g'(\var{b})=\frac{1}{f'(\var{a})}=\frac{1}{1-\frac{1}{\var{a}^2}}=\frac{\var{a}^2}{\var{a}^2-1}$.
    }
    }
	\end{quickcheck}
    
    




\end{visualizationwrapper}


\end{content}