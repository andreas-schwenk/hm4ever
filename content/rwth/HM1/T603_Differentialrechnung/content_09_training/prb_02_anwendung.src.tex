\documentclass{mumie.problem.gwtmathlet}
%$Id$
\begin{metainfo}
  \name{
    \lang{en}{...}
    \lang{de}{A02: Kosten, Erlös und Gewinn}
  }
  \begin{description} 
 This work is licensed under the Creative Commons License Attribution 4.0 International (CC-BY 4.0)   
 https://creativecommons.org/licenses/by/4.0/legalcode 

    \lang{en}{...}
    \lang{de}{...}
  \end{description}
  \corrector{system/problem/GenericCorrector.meta.xml}
  \begin{components}
    \component{js_lib}{system/problem/GenericMathlet.meta.xml}{gwtmathlet}
  \end{components}
  \begin{links}
  \end{links}
  \creategeneric
\end{metainfo}
\begin{content}
\lang{de}{\title{A02: Kosten, Erlös und Gewinn}}
\lang{en}{\title{A02: Costs, Revenue and Profit}}

\begin{block}[annotation]
	Im Ticket-System: \href{https://team.mumie.net/issues/24069}{Ticket 24069}
\end{block}
\usepackage{mumie.genericproblem}

\embedmathlet{gwtmathlet}

 \begin{problem}
        \begin{variables}
            \drawFromSet{a}{2,3,4}
            \function[normalize]{k}{a*(6*x+40)}
            \function[normalize]{p}{a*(30-2*x)}
            \function{dk}{k/x}
            \function{gk}{6*a}
            \function{er}{p*x}
            \derivative{ger}{er}{x}
            \number{n}{6}
        \end{variables}
          \begin{question}
          \type{input.function}
          \lang{de}{\text{Für eine Einproduktunternehmung ist die Kostenfunktion $K(x)=\var{k}$ und
          die Preis-Absatz-Funktion $p(x)=\var{p}$.\\
          Berechnen Sie
          \begin{itemize}
          \item
          die Grenzkostenfunktion $K'(x)=$ \ansref und\\
          \item
          die Durchschnittskostenfunktion $k(x)= \frac{K(x)}{x} =$ \ansref
          \end{itemize}
          }}
          \lang{en}{\text{For a one-product firm, the cost function is $K(x)=\var{k}$ and
          the price-sales function is $p(x)=\var{p}$.\\
          Calculate
          \begin{itemize}
          \item
          the marginal cost function $K'(x)=$ \ansref and\\
          \item
          the average cost function $k(x)= \frac{K(x)}{x} =$ \ansref
          \end{itemize}
          }}
               \begin{answer}
                    \solution{gk}
                    \checkAsFunction{x}{-10}{10}{100}
               \end{answer}
               \begin{answer}
                    \solution{dk}
                    \checkAsFunction{x}{-10}{10}{100}
               \end{answer}
               \explanation{$K'(x)=\frac{d}{dx}K(x)$ 
                \lang{de}{\text{und}}
               \lang{en}{\text{and}}
               $k(x)=\frac{K(x)}{x}$.}
          \end{question} 
          
          \begin{question}
          \type{input.function}
          \lang{de}{\text{
          Berechnen Sie
          \begin{itemize}
          \item
          die Erlösfunktion $E(x)=$ \ansref und\\
          \item
          die Grenzerlösfunktion $E'(x)=$ \ansref
          \end{itemize}
          }}
          \lang{en}{\text{
          Calculate
          \begin{itemize}
          \item
          the revenue function $E(x)=$ \ansref and\\
          \item
          the marginal revenue function $E'(x)=$ \ansref
          \end{itemize}
          }}
               \begin{answer}
                    \solution{er}
                    \checkAsFunction{x}{-10}{10}{100}
               \end{answer}
               \begin{answer}
                    \solution{ger}
                    \checkAsFunction{x}{-10}{10}{100}
               \end{answer}
               \explanation{$E(x)=x\cdot p(x)$ 
               \lang{de}{\text{und}}
               \lang{en}{\text{and}}
               $E'(x)=\frac{d}{dx}E(x)$ }
         \end{question}
         
         \begin{question}
            \type{input.generic}
            \lang{de}{\text{Bestimmen Sie den Bereich positiver Gewinne $I$ und die gewinnmaximale 
            Ausbringungsmenge $x_{max}$:}}
            \lang{en}{\text{Determine the range of positive profits $I$ and the profit maximizing
            output quantity $x_{max}$:}}
              \begin{answer}
                    \type{input.interval}
                    \text{I=}
                    \solution{(2;10)}
              \end{answer} 
              
              \begin{answer}
                    \type{input.number}
                    \text{$x_{max}$=}
                    \solution{n}
              \end{answer} 
              \lang{de}{\explanation{$G(x)=E(x)-K(x)$. Die Nullstellen dieser quadratischen Funktion liegen
              bei $x_1=2$ und $x_2=10$.}}
              \lang{en}{\explanation{$G(x)=E(x)-K(x)$. The zeros of this quadratic function lie
              at $x_1=2$ and $x_2=10$.}}
         \end{question}
          
     \end{problem}


\end{content}
