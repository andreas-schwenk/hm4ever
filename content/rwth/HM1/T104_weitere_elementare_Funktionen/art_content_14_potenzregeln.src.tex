%$Id:  $
\documentclass{mumie.article}
%$Id$
\begin{metainfo}
  \name{
    \lang{de}{Potenzgesetze}
    \lang{en}{Power laws}
  }
  \begin{description} 
 This work is licensed under the Creative Commons License Attribution 4.0 International (CC-BY 4.0)   
 https://creativecommons.org/licenses/by/4.0/legalcode 

    \lang{de}{Beschreibung}
    \lang{en}{Description}
  \end{description}
  \begin{components}
\component{generic_image}{content/rwth/HM1/images/g_img_00_Videobutton_schwarz.meta.xml}{00_Videobutton_schwarz}
\end{components}
  \begin{links}
    \link{generic_article}{content/rwth/HM1/T210_Stetigkeit/g_art_content_29_stetigkeit_definitionen.meta.xml}{content_29_stetigkeit_definitionen}
    \link{generic_article}{content/rwth/HM1/T104_weitere_elementare_Funktionen/g_art_content_13_wurzelfunktionen.meta.xml}{link-wurzel}
    \link{generic_article}{content/rwth/HM1/T104_weitere_elementare_Funktionen/g_art_content_15_exponentialfunktionen.meta.xml}{link-exponential}
  \end{links}
  \creategeneric
\end{metainfo}
\begin{content}
\usepackage{mumie.ombplus}
\ombchapter{4}
\ombarticle{2}
\usepackage{mumie.genericvisualization}

\begin{visualizationwrapper}

\title{\lang{de}{Potenzgesetze}\lang{en}{Power laws}}

\begin{block}[annotation]
 
  
\end{block}
\begin{block}[annotation]
  Im Ticket-System: \href{http://team.mumie.net/issues/9010}{Ticket 9010}\\
\end{block}

\begin{block}[info-box]
\tableofcontents
\end{block}

\section{\lang{de}{Allgemeine Potenzen}\lang{en}{General exponents}} \label{sec:potenzen}

\lang{de}{
Für beliebige reelle Zahlen $a$ und natürliche Zahlen $n$ wurde die Potenz $a^n$ definiert als 
Kurzschreibweise für das $n$-fache Produkt von $a$ mit sich selbst:
}
\lang{en}{
For any real number $a$ and natural number $n$ we define the power $a^n$ as the product of $a$ with 
itself $n$ times:
}
 
\begin{definition}[\lang{de}{$n\text{-te}$ Potenz, Basis und Exponent}
                   \lang{en}{The n-th power, base, and exponent}] \label{def:potenzen}

\lang{de}{
Das $n$-fache Produkt einer reellen Zahl $a$ mit sich selbst nennt man die \notion{$n$-te Potenz} von $a$,
geschrieben
}
\lang{en}{
The $n$-fold product of a real number $a$ with itself is called the \emph{$n$-th power} of $a$,
written
}
\[a^n=\underbrace{a\cdot a\cdot \ldots \cdot a}_{\text{\lang{de}{$n$-mal}\lang{en}{$n$-times}}}.\]
\lang{de}{
Insbesondere ist also
}
\lang{en}{
In particular,
}
\[ a^1=a .\]
\lang{de}{
Man nennt $a$  die \notion{Basis} und $n$ den \notion{Exponenten} von $a^n$.
}
\lang{en}{
We call $a$ the \emph{base} and $n$ the \emph{exponent} of $a^n$.
}

\end{definition}

\lang{de}{
Diese Definition kann man ganz einfach auf auf ganzzahlige Exponenten erweitern, sofern die Basis $a$ nicht $0$ ist:
}
\lang{en}{
This definition can easily be extended to allow for integer exponents, as long as the base $a$ is not 
equal to $0$.
}

\begin{definition}\label{zeroexp}
\lang{de}{ F\"ur $a\in\mathbb{R}$ mit $a\neq 0$ und $n\in\mathbb{N}$ gilt}
\lang{en}{Given $a\in\mathbb{R}$ with $a\neq 0$ and $n\in\mathbb{N}$, then}
\[ \quad a^{-n}=\frac{1}{a^n}=\big(\frac{1}{a}\big)^n. \]
\lang{de}{F\"ur alle $a\in \mathbb{R}$ setzen wir $a^{0}= 1$.}
\lang{en}{For all $a\in \mathbb{R}$ we set $a^{0}= 1$.}
\end{definition}

\lang{de}{
Es ist klar, dass der Ausdruck $a^{-n}$ für $a=0$ nicht sinnvoll ist, denn man würde durch Null teilen.
Die Definition $a^0=1$ gilt hingegen auch für $a=0$. Wir werden später sehen, dass dies eine sinnvolle Konvention ist.
\\
Für Zahlen $a\geq 0$ wurde auch die \link{link-wurzel}{$n$-te Wurzel} von $a$ definiert. 
Mit deren Hilfe lassen sich auch rationale Potenzen von $a$ definieren und zwar so, dass sie sowohl die bereits bekannten
\lref{Potenzgesetze}{Potenzgesetze} als auch die \ref[link-wurzel][Wurzelgesetze]{allgwurzel_gesetze} verallgemeinern:
}
\lang{en}{
It is clear that the expression $a^{-n}$ does not make sense for $a=0$, as we cannot divide by zero. 
The definition $a^0=1$ however also holds for $a=0$. We will see later that this convention makes 
sense.
\\
For numbers $a\geq 0$ we also define the \link{link-wurzel}{$n$th root} of $a$. 
Using this we can also define rational powers of $a$ in a way that allows the generalisation of both 
the \lref{Potenzgesetze}{power laws} and the \ref[link-wurzel][laws of roots]{allgwurzel_gesetze}.
}


\begin{definition} \label{def:rat_potenz}
	\lang{de}{F\"ur $a>0$ und jede rationale Zahl $\frac{p}{q}$ (mit $p,q\in \Z$ und $q>0$) ist}
  \lang{en}{For $a>0$ and any rational number $\frac{p}{q}$ (with $p,q\in \Z$ and $q>0$), we have}
	\[ a^{\frac{p}{q}}=\sqrt[q]{a^p}. \]
	\lang{de}{Speziell f"ur $\frac{p}{q}=\frac{1}{n}$ mit $n\in\N$ ist also}
  \lang{en}{A special case is when $\frac{p}{q}=\frac{1}{n}$ with $n\in\N$, which results in}
	\[a^{\frac{1}{n}}=\sqrt[n]{a}.\]
  \lang{de}{Für $\frac{p}{q}>0$ gilt auch}
  \lang{en}{In addition, given $\frac{p}{q}>0$ then}
  $0^{\frac{p}{q}}=0$.
\end{definition}

\begin{example}
	\[7^{\frac{1}{2}}=\sqrt{7},\qquad\left(\frac{1}{27}\right)^{\frac{1}{3}}=\sqrt[3]{\frac{1}{27}}=\frac{1}{3},
	\qquad 8^{\frac{2}{3}}=\sqrt[3]{8^2}=\sqrt[3]{64}=4,\qquad\left(\frac{3}{5}\right)^{\frac{1}{7}}=\sqrt[7]{\frac{3}{5}}.\]
\end{example}

\lang{de}{
Für negative Zahlen sind Potenzen mit nicht ganzzahligen Exponenten nicht definiert. Daher sind 
Ausdr\"ucke wie 
}
\lang{en}{
Non-integer exponents of negative numbers are not defined. Hence expressions such as 
}
\[(-8)^{\frac{2}{3}}\qquad\text{\lang{de}{ oder }\lang{en}{ or }}\qquad
\left(-\frac{2}{3}\right)^{\frac{1}{4}}\]
\lang{de}{sinnlos.}
\lang{en}{are meaningless.}

\begin{quickcheck}
		\field{rational}
		\type{input.number}
		\begin{variables}
%			\randint[Z]{a}{-5}{5}
			\randint[Z]{p}{-4}{4}
			\randint[Z]{q}{2}{4}
			\randadjustIf{p,q}{p = q}
			\randint[Z]{c}{1}{4}
			\randint[Z]{d}{1}{4}
			\randadjustIf{c,d}{c = d}
		    \function[calculate]{a}{c/d}
		    \function[calculate]{f}{a^q}
		    \function[calculate]{r}{p/q}
		    \function[calculate]{l}{a^p}
		\end{variables}
		
			\text{\lang{de}{Es gilt: $\big( \var{f} \big)^{\var{r}}=$ \ansref.}
            \lang{en}{We have: $\big( \var{f} \big)^{\var{r}}=$ \ansref.}}
		
		\begin{answer}
			\solution{l}
		\end{answer}
		
	\end{quickcheck}
	
	

\section{\lang{de}{Potenzgesetze}\lang{en}{Power laws}}

\begin{theorem}[\lang{de}{Potenzgesetze}
                \lang{en}{Power laws}]\label{Potenzgesetze}
\lang{de}{
F\"ur $a,b\in\mathbb{R}$ mit $a,b>0$ und $r,s\in\mathbb{Q}$ gilt:
	\begin{center}
		$a^r\cdot a^s=a^{r+s}$\\
		\textit{(Potenzen mit gleichen Basen werden multipliziert, indem man die Exponenten addiert.)}\\ \\
		$\displaystyle \frac{a^r}{a^s}=a^{r-s}$\\
		\textit{(Potenzen mit gleichen Basen werden dividiert, indem man die Exponenten subtrahiert.)}\\ \\ \\
		$a^r\cdot b^r=(a\cdot b)^{r}$\\
		\textit{(Potenzen mit gleichen Exponenten werden multipliziert, indem man die Basen multipliziert.)}\\ \\ 
		$\displaystyle \frac{a^r}{b^r}=\big(\frac{a}{b}\big)^{r}$\\
		\textit{(Potenzen mit gleichen Exponenten werden dividiert, indem man die Basen dividiert.)}\\ \\ \\
		$(a^r)^s=a^{r\cdot s}$\\
		\textit{(Eine Potenz wird potenziert, indem man die Exponenten multipliziert.)}\\ 
	\end{center}
	\floatright{\href{https://www.hm-kompakt.de/video?watch=140}{\image[75]{00_Videobutton_schwarz}}}\\~
}
\lang{en}{
  Given $a,b\in\mathbb{R}$ with $a,b>0$ and $r,s\in\mathbb{Q}$, the following rules hold:
	\begin{center}
		$a^r\cdot a^s=a^{r+s}$\\
		\textit{(Powers with the same bases can be multiplied by adding their exponents.)}\\ \\
		$\displaystyle \frac{a^r}{a^s}=a^{r-s}$\\
		\textit{(Powers with the same bases can be divided by subtracting their exponents.)}\\ \\ \\
		$a^r\cdot b^r=(a\cdot b)^{r}$\\
		\textit{(Powers with the same exponents can be multiplied by multiplying their bases.)}\\ \\ 
		$\displaystyle \frac{a^r}{b^r}=\big(\frac{a}{b}\big)^{r}$\\
		\textit{(Powers with the same exponents can be divided by dividing their bases.)}\\ \\ \\
		$(a^r)^s=a^{r\cdot s}$\\
		\textit{(Powers can be raised to other powers by multiplying their exponents.)}\\ \\
	\end{center}
}
\end{theorem}

\lang{de}{Aufgrund dieser Rechenregeln gilt insbesondere:}
\lang{en}{As a result of these rules, the following is a particularly important case:}
\[\sqrt[q]{a^p}=(a^p)^{\frac{1}{q}}=a^{p\cdot \frac{1}{q}}=a^{\frac{p}{q}}=a^{\frac{1}{q}\cdot p}=
\big(a^{\frac{1}{q}}\big)^p=(\sqrt[q]{a})^p.\]
\lang{de}{
Es spielt also keine Rolle, ob erst die $p$-te Potenz von $a$ gebildet wird und dann die $q$-te Wurzel gezogen, oder umgekehrt.
}
\lang{en}{
That is, it makes no difference whether the $p$th power of $a$ is taken first and then the 
\nowrap{$q$th root}, or vice versa.
}

\begin{rule}
\lang{de}{F\"ur alle positiven Zahlen $a\in\mathbb{R}$ und alle rationalen Exponenten $\frac{p}{q}\in\mathbb{Q}$ gilt:}
\lang{en}{For all positive numbers $a\in\mathbb{R}$ and all rational exponents $\frac{p}{q}\in\mathbb{Q}$ the following holds:}
\[
	a^{\frac{p}{q}}=\sqrt[q]{a^p}=(\sqrt[q]{a})^p.
\]
\end{rule}



\begin{example}
	\begin{enumerate}
    \item $\sqrt[4]{4}^2=\sqrt[4]{4^2}=\sqrt[4]{16}=2$,
    \item $\sqrt[3]{0,09}=\sqrt[3]{\frac{9}{100}}=
\sqrt[3]{\left(\frac{3}{10}\right)^2}=\left(\sqrt[3]{\frac{3}{10}}\right)^2$,
		\item $\displaystyle 256^{\frac{5}{8}}=\left(256^{\frac{1}{8}}\right)^5=\left(\sqrt[8]{256}\right)^5=2^5=32$,
		\item $\displaystyle\left(\frac{27}{125}\right)^{\frac{-2}{3}}
			   =\left(\sqrt[3]{\frac{27}{125}}\right)^{-2}=\left(\frac{\sqrt[3]{27}}{\sqrt[3]{125}}\right)^{-2}
			   =\left(\frac{3}{5}\right)^{-2}=\left(\frac{5}{3}\right)^{2}=\frac{25}{9}$,
		\item $\left(\sqrt[6]{x+y}\right)^4=\left((x+y)^{\frac{1}{6}}\right)^4=(x+y)^{\frac{1}{6}\cdot 4}
		=(x+y)^{\frac{4}{6}}=(x+y)^{\frac{2}{3}}$,
		\item $\displaystyle\left(2a+b\right)^{-\frac{4}{3}}=\left(\sqrt[3]{2a+b}\right)^{-4}
		=\frac{1}{\sqrt[3]{2a+b}^4}$,
		\item $\displaystyle
			   \left(a^2\cdot a^{\frac{2}{3}}\right)^{\frac{1}{4}}=\left(a^{\frac{8}{3}}\right)^{\frac{1}{4}}
			   =a^{\frac{8}{3}\cdot\frac{1}{4}}=a^{\frac{2}{3}}=\sqrt[3]{a}^2=\sqrt[3]{a^2}$.
	\end{enumerate}
\end{example}

\begin{block}[warning]
\lang{de}{
Die Potenzgesetze sind nur gültig, solange alle Ausdr"ucke definiert sind: \\
Sie gelten nur dann für alle Basen $a,b\ne 0$, wenn alle auftretenden
Exponenten ganzzahlig sind.\\
Sie gelten hingegen für beliebige rationale Exponenten, 
wenn die Basen auf positive Werte $a,b>0$  eingeschränkt werden. \\
In speziellen Fällen gelten sie auch für Basen Null.\\\\
In den bestechenden Potenzgesetzen liegt auch ein Grund, weshalb man darauf verzichtet $n$-te Wurzeln 
für ungerade $n$ aus negativen Zahlen $a$ zu ziehen, 
obwohl $a$ eine Lösung der Gleichung $x^n=a$ wäre. Das führte nämlich zum Beispiel zum folgenden 
Widerspruch:
}
\lang{en}{
The power rules are only valid as long as all expressions are well defined. \\
That is, they are valid for all bases $a,b\ne 0$ for all integer exponents, and are valid for a 
rational exponent if the bases are restricted to $a,b>0$. \\
In special cases, they are also valid for powers with base $0$. \\\\
There is a reason that we do not allow taking $n$th roots of negative bases $a$, even though for odd 
$n$, $a$ is a solution to the equation $x^n=a$. If we allowed this, we could construct a 
contradiction:
}
\[
1=(1)^{\frac{1}{6}}=((-1)^2)^{\frac{1}{6}}=(-1)^{\frac{1}{3}}=\sqrt[3]{-1}=-1. 
\text{ (\lang{de}{Wid.}\lang{en}{contradiction})}
\]
\end{block}

\begin{quickcheck}
		\field{rational}
		\type{input.function}
		\begin{variables}
			\randint[Z]{b}{2}{3}
			\randint[Z]{n}{1}{3}
		    \function[calculate]{n1}{n+1}
		    \function[calculate]{l}{b^(n+1)}
		    \function[calculate]{k}{b^n}
		    \function[calculate]{q}{n/(n+1)}
		    \function{f}{x/k}
		\end{variables}
		
			\text{\lang{de}{Vereinfachen Sie für $x>0$ den Ausdruck:}
            \lang{en}{For $x>0$ the expression }
            $\big(\frac{x}{\var{l}}\big)^{\var{q}}\cdot \sqrt[\var{n1}]{x}$.\\
      			\lang{de}{Es ist }\lang{en}{simplifies to }
            $\big(\frac{x}{\var{l}}\big)^{\var{q}}\cdot \sqrt[\var{n1}]{x}=$\ansref.
			}
		
		\begin{answer}
			\solution{f}
			\checkAsFunction{x}{-10}{10}{100}
		\end{answer}
		
	\end{quickcheck}
	


\begin{remark}
\lang{de}{
Als abschlie{\ss}ende Bemerkung wollen wir noch festhalten, dass sich Potenzen $a^x$ auch f\"ur reelle Exponenten $x\in \R$
(und Basis $a>0$) erkl\"aren lassen.
Ausdr\"ucke wie $a^{\sqrt{2}}$ oder $4^{\pi}$ 
sind daher sinnvoll und wohldefiniert.\\
Auch mit reellen
Exponenten $x,y \in \R$ und $a>0$ gelten dann die Potenzgesetze
$a^x \cdot a^y = a^{x+y}$, $\frac{a^x}{a^y} = a^{x-y}$,
$(a^x)^y = a^{x \cdot y}$ und $a^x \cdot b^x = (ab)^x$,
falls auch $b>0$ ist.
\\
Die präzise Einführung von $a^x$ für allgemeine reelle
Exponenten $x \in \R$  benötigt jedoch den Begriff der
\ref[content_29_stetigkeit_definitionen][Stetigkeit]{sec:eps-delta} (siehe Teil 2 des Kurses).
}
\lang{en}{As a final remark we note that powers $a^x$ are also valid for real exponents $x\in \R$ 
(for bases $a>0$). 
Expressions like $a^{\sqrt{2}}$ or $4^{\pi}$ are meaningful and well defined.\\
The power rules are also valid for real exponents $x,y \in \R$ and $a>0$;
$a^x \cdot a^y = a^{x+y}$, $\frac{a^x}{a^y} = a^{x-y}$,
$(a^x)^y = a^{x \cdot y}$ and $a^x \cdot b^x = (ab)^x$,
where $b>0$.
\\
The formal introduction of $a^x$ for general real exponents
$x \in \R$ requires a principle called 
\ref[content_29_stetigkeit_definitionen][continuity]{sec:eps-delta}, covered in the second part of 
this course).
}\\

\end{remark}



% 	\begin{genericGWTVisualization}[550][1000]{mathlet1}
% 		\begin{variables}
% 			\randint{randomA}{1}{2}
% 			
% 			\point[editable]{P}{rational}{var(randomA),var(randomA)}
% 		\end{variables}
% 		\color{P}{BLUE}
% 		\label{P}{$\textcolor{BLUE}{P}$}
% 
% 		\begin{canvas}
% 			\plotSize{300}
% 			\plotLeft{-3}
% 			\plotRight{3}
% 			\plot[coordinateSystem]{P}
% 		\end{canvas}
% 		\text{Der Punkt hat die Koordinaten $(\var{P}[x],\var{P}[y])$.}
% 	    	\end{genericGWTVisualization}

\end{visualizationwrapper}

\end{content}