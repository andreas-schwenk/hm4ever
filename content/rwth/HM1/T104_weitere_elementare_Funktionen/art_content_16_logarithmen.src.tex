%$Id:  $
\documentclass{mumie.article}
%$Id$
\begin{metainfo}
  \name{
    \lang{de}{Logarithmen}
    \lang{en}{Logarithms}
  }
  \begin{description} 
 This work is licensed under the Creative Commons License Attribution 4.0 International (CC-BY 4.0)   
 https://creativecommons.org/licenses/by/4.0/legalcode 

    \lang{de}{Beschreibung}
    \lang{en}{Description}
  \end{description}
  \begin{components}
    \component{generic_image}{content/rwth/HM1/images/g_tkz_T104_Logarithms_B.meta.xml}{T104_Logarithms_B}
    \component{generic_image}{content/rwth/HM1/images/g_tkz_T104_Logarithms_A.meta.xml}{T104_Logarithms_A}
    \component{generic_image}{content/rwth/HM1/images/g_tkz_T104_Logarithms_C.meta.xml}{T104_Logarithms_C}
    \component{generic_image}{content/rwth/HM1/images/g_img_00_Videobutton_schwarz.meta.xml}{00_Videobutton_schwarz}
     \component{generic_image}{content/rwth/HM1/images/g_img_00_video_button_schwarz-blau.meta.xml}{00_video_button_schwarz-blau}
  \end{components}
  \begin{links}
    \link{generic_article}{content/rwth/HM1/T104_weitere_elementare_Funktionen/g_art_content_15_exponentialfunktionen.meta.xml}{link2}
  \end{links}
  \creategeneric
\end{metainfo}
\begin{content}
\usepackage{mumie.ombplus}
\ombchapter{4}
\ombarticle{4}
\usepackage{mumie.genericvisualization}

\begin{visualizationwrapper}

\title{\lang{de}{Logarithmen}\lang{en}{Logarithms}}

\begin{block}[annotation]
  Definition Logarithmus, Basis $e$ und $10$, Rechenregeln, Transformationsgesetz;
  Logarithmusfunktion, Definitions- und Wertemenge, Umkehrbarkeit, Exponentialfunktion als Umkehrfunktion,
  Exponentialgleichungen.  
\end{block}
\begin{block}[annotation]
  Im Ticket-System: \href{http://team.mumie.net/issues/9012}{Ticket 9012}\\
\end{block}

\begin{block}[info-box]
\tableofcontents
\end{block}


\section{\lang{de}{Einf"uhrung }\lang{en}{Introduction to logarithms}}

\lang{de}{
Die Frage, wie zu einer positiven Zahl $a$ und reellen Zahlen $x$ die Potenzen 
$u = a^x$ zu definieren sind, hat uns zu den \link{link2}{Exponentialfunktionen} 
geführt. Nun stellen wir die Frage umgekehrt: Gegeben sind zwei positive Zahlen $a$ und 
$u$, wie lösen wir die Gleichung $u=a^x$ nach $x$ auf? 
Das führt uns zu  den Logarithmen. 
\\ 
Dazu benutzen wir die \ref[link2][fundamentalen Eigenschaften aller Exponentialfunktionen]{ex:eigenschaften_exp}:
Für jede positive Basis $a$  (mit $a\neq 1$) ist die Funktion $a^x$ streng monoton. 
Das heißt, dass es höchstens eine Lösung $x$ der Gleichung $u=a^x$ geben kann.
Die Wertemenge der Funktion $a^x$ ist gleich der Menge der positiven Zahlen. 
Damit wissen wir, der Wert $u>0$ wird von $a^x$ wirklich angenommen.\\
Zusammen heißt das: Es gibt genau eine Lösung $x$ der Gleichung $u=a^x$. Diese Lösung nennen wir den Logarithmus $x = \log_a(u)$ von $u$ zur Basis $a$.
\\
% Sei $a\neq 1$ eine positive reelle Zahl. Der
% Logarithmus kann definiert werden, weil es für jedes $u > 0$ jeweils \lref{uniqueness}{genau einen} 
% Exponenten $x$ gibt, sodass die Gleichung $u = a^x$ gilt. Für dieses $x$ schreiben wir dann 
% $x = \log_a(u)$ und nennen diesen Ausdruck den Logarithmus von $u$ zur Basis $a$.
%\\
% Dass es jeweils genau einen Exponenten $x$ gibt mit diesen Eigenschaften, folgt wiederum aus zwei 
% \ref[link2][fundamentalen Eigenschaften aller Exponentialfunktionen]{ex:eigenschaften_exp} mit positiver Basis $a$ und $a\neq 1$: 
% Sie sind streng monoton und ihre Wertemengen sind gleich der Menge der positiven Zahlen.\\ 
\\
Für $u \leq 0$ gibt es keinen Exponenten $x$, so dass die Gleichung $u = a^x$ erfüllt 
ist, denn alle Potenzen einer positiven Zahl sind positiv.
}
\lang{en}{
The question of how a positive number $a$ can be raised to a real exponent $x$ was 
answered in the previous section where \link{link2}{exponential functions} such as 
$u = a^x$ were introduced. Now we ask the inverse question: given two positive numbers 
$a$ and $u$, how do we rearrange the equation $u=a^x$ for $x$? 
This brings us to logarithms.
\\
We use
\ref[link2][fundamental properties of exponential functions]{ex:eigenschaften_exp}: 
for every positive base $a$ (with $a\neq 1$), the function $a^x$ is strictly 
monotonous. This means that there is no more than one solution to the equation $u=a^x$. 
The image of the function $a^x$ is the set of positive real numbers. Hence there exists 
at least one solution $x$ to $u=a^x$ for all $u>0$. \\
Combining these two statements, there exists exactly one solution $x$ to the equation 
$u=a^x$. We call this solution the logarithm $x = \log_a(u)$ of $u$ to base $a$.
\\\\
For $u \leq 0$ there exists no exponent $x$ satisfying the equation $u = a^x$, as the 
power of a positive number is always positive.
}

\begin{definition}[\lang{de}{Logarithmus}\lang{en}{Logarithms}]\label{ln}
    \lang{de}{
    Zu vorgegebener Basis $a>0$ und $u>0$ ist der \notion{\emph{Logarithmus $x$ zur 
    Basis $a$ von $u$}} die eindeutige Lösung der Gleichung $a^x = u$. 
    Wir schreiben dafür $x=\log_a(u)$.
    \\\\
    Die folgenden oft benutzten Logarithmen erhalten besondere Namen: \\ 
     {Den \emph{Logarithmus zur Basis $e$}} nennen wir \notion{\emph{natürlichen 
     Logarithmus}} und schreiben dafür $\ln$. \\
     {Den \emph{Logarithmus zur Basis $10$}} nennen wir 
     \notion{\emph{Zehnerlogarithmus}} oder \notion{\emph{dekadischen Logarithmus}} 
     und schreiben dafür $\lg$. \\
     {Den \emph{Logarithmus zur Basis $2$}} nennen wir \notion{\emph{binären 
     Logarithmus}} und schreiben dafür $\text{lb}$.
    }
    \lang{en}{
    For a given base $a>0$ and $u>0$, the \notion{\emph{logarithm of $u$ to base 
    $a$}} is the unique solution $x$ to the equation $a^x = u$. 
    We denote it as $x=\log_a(u)$.
    \\\\
    The following commonly used logarithms have special names: \\
     {The \emph{logarithm to base $e$}} is called the \notion{\emph{natural logarithm}} 
     and denoted as $\ln$. \\
     {The \emph{logarithm to base $10$}} is called the \notion{\emph{common logarithm}} 
     or the \notion{\emph{decadic logarithm}} or the \notion{\emph{decimal logarithm}} 
     and denoted as $\lg$. \\
     {The \emph{logarithm to base $2$}} is called the \notion{\emph{binary logarithm}} 
     and denoted as $\text{lb}$.
    }    
    
\end{definition}

\lang{de}{Es gilt also }
\lang{en}{Hence we have }

  	\begin{align*}	
  		\ln(u) = x\quad \Leftrightarrow \quad e^x= u.
  	\end{align*}

\begin{remark}\label{logid}\label{bem:exp_ln}
	\lang{de}{
  Unmittelbar aus der Definition des Logarithmus folgt für alle $x\in\R$ \\ 
  }
	\lang{en}{
  The following identities follow directly from the definition of the logarithm:\\ 
  }
    \[\quad \ln(e^x)=x,\]
  \lang{de}{sowie für alle $u>0$}\lang{en}{and for all $u>0$}
    \[ e^{\ln(u)}=u.\]
  \lang{de}{
  Dasselbe gilt für alle anderen Basen.
  \\
  Für $x=0$ finden wir insbesondere $\log_a(1)=\log_a(a^0)=0$. Weil $a^x=1$ genau eine 
  Lösung hat, ist $\log_a(u)=0$ genau für $u=1$ erfüllt.
  }
  \lang{en}{
  The same holds for bases other than $e$.
  \\
  For $x=0$ we always have $\log_a(1)=\log_a(a^0)=0$. As $a^x=1$ has exactly one 
  solution, $\log_a(u)=0$ is satisfied precisely by $u=1$.
  } %The reason it is satisfied precisely by u=1 should be 'as a^0=1' - Niccolo
\end{remark}

\begin{example}
	\begin{eqnarray*}
        e^{-1} = \frac{1}{e} &\Leftrightarrow & \ln\left(\frac{1}{e} \right) = -1, \\
		10^4 = 10\;\!000 &\Leftrightarrow & \lg(10\;\!000) = 4,\\
		10^{-3} = \frac{1}{1\;\!000}  &\Leftrightarrow & \lg\left(\frac{1}{1\;\!000}\right) = -3,\\
		e^{-\frac{1}{2}} = \frac{1}{\sqrt{e}} &\Leftrightarrow & \ln\left(\frac{1}{\sqrt{e}}\right)= -\frac{1}{2} .
	\end{eqnarray*}
\end{example}

\begin{quickcheckcontainer}
\randomquickcheckpool{1}{2}

\begin{quickcheck}
		\field{rational}
		\type{input.number}
		\begin{variables}
			\randint{a}{1}{4}
			\randint{k}{1}{2}
			\function[calculate]{n}{k*a+1}
			\function[calculate]{b}{10^a}
			\function[calculate]{c}{a/n}
		\end{variables}
		
			\text{\lang{de}{Bestimmen Sie}
            \lang{en}{Calculate }$\lg(\sqrt[\var{n}]{\var{b}})=$\ansref.}
		
		\begin{answer}
			\solution{c}
		\end{answer}
		\explanation{
    \lang{de}{Es ist}\lang{en}{As}
    $\sqrt[\var{n}]{\var{b}}=10^{\var{c}}$, 
    \lang{de}{und daher}\lang{en}{we have} 
		$\lg(\sqrt[\var{n}]{\var{b}})=\var{c}$.
    }
	\end{quickcheck}
	
\begin{quickcheck}
		\field{rational}
		\type{input.number}
		\begin{variables}
			\randint{n}{2}{4}
			\randint{k}{1}{2}
			\function[calculate]{a}{k*n+1}
			\function[calculate]{b}{10^a}
			\function[calculate]{c}{a/n}
		\end{variables}
		
			\text{\lang{de}{Bestimmen Sie}
            \lang{en}{Calculate} 
            $\lg(\sqrt[\var{n}]{\var{b}})=$\ansref.}
		
		\begin{answer}
			\solution{c}
		\end{answer}
		\explanation{\lang{de}{Es ist}\lang{en}{As} 
    $\sqrt[\var{n}]{\var{b}}=10^{\var{c}}$, 
    \lang{de}{und daher}\lang{en}{we have} 
		$\lg(\sqrt[\var{n}]{\var{b}})=\var{c}$.}

	\end{quickcheck}
\end{quickcheckcontainer}


\lang{de}{
Das folgende Video behandelt den natürlichen Logarithmus, dabei werden auch Stetigkeit, Differenzierbarkeit und die Stammfunktion behandelt. Diese Themen 
werden aber erst später eingeführt.
\\
\floatright{\href{https://api.stream24.net/vod/getVideo.php?id=10962-2-10863&mode=iframe&speed=true}{\image[75]{00_video_button_schwarz-blau}}}
\\
\\
Das folgende Video verdeutlicht den Umgang mit dem Logarithmus zu einer bestimmten Basis $a$:
\\
\floatright{
\href{https://www.hm-kompakt.de/video?watch=152}{\image[75]{00_Videobutton_schwarz}}}\\\\
}
\lang{en}{
\\
}

\section{\lang{de}{Rechenregeln und Transformationsformel}
         \lang{en}{Rules for calculating with logarithms}}\label{sec:log_rules}

\lang{de}{
	Es gibt eine Reihe einfacher Rechenregeln für die Logarithmen, die unmittelbar aus den 
 	\ref[link2][Potenzgesetzen für Exponentialfunktionen]{functional_eqn_exp} folgen. 
 	Von der Diskussion der Exponentialfunktionen wissen Sie, dass sich bei der 
  Multiplikation zweier Potenzen mit der gleichen Basis $a>0$ die Exponenten addieren. 
 	Daraus ergeben sich die folgenden beiden Anwendungen. 
\\
  Weil (für beliebige reelle Zahlen $x$ und $y$) gilt $\log_a(a^x)=x$ und 
  $\log_a (a^y)=y$, ist 
}
\lang{en}{
	There are a series of rules for calculating with logarithms that follow directly from 
  the \ref[link2][power rules for exponential functions]{functional_eqn_exp}. 
	From the discussion with exponential functions, we know that to multiply powers 
  with the same	base $a>0$ we simply add their exponents. TFrom this we derive the 
  following results about logarithms:
}
    \[
    x+y= \log_a(a^x)+\log_a(a^y).
    \]
    \lang{de}{Andererseits gilt aber auch}\lang{en}{Similarly we have}
    \[x+y= \log_a(a^{x+y})=\log_a(a^x\cdot a^y).
    \]
    \lang{de}{Indem wir diese beiden Gleichungen vergleichen, folgt}
    \lang{en}{equating the above two equations yields}
    \[
    \log_a(a^x)+\log_a(a^y)=\log_a(a^x\cdot a^y).
    \]
    \lang{de}{Wir machen dasselbe Spiel mit $x-y= \log_a(a^x)-\log_a(a^y)$ und }
    \lang{en}{We can do the same with $x-y= \log_a(a^x)-\log_a(a^y)$ and }
    $
    x-y=\log_a(a^{x-y})=\log_a\left(\frac{a^x}{a^y}\right)
    $
    \lang{de}{und erhalten}\lang{en}{to obtain}
    \[
    \log_a(a^x)-\log_a(a^y)=\log_a\left(\frac{a^x}{a^y}\right).
    \]


% \begin{align*}
% 	\lg(10^3) &=& 3\\
% 	\lg(10^5) &=& 5\\
% 	\lg(10^3\cdot 10^5) = \lg(10^{3 + 5}) &=& 3+5 = \lg(10^3) + \lg(10^5)\\
%  	\lg\left(\frac{10^3}{10^5}\right) = \lg\left(10^{3-5}\right) &=& 3-5 = \lg(10^3) - \lg(10^5)
% \end{align*}

\lang{de}{
So, wie die \ref[link2][Exponentialfunktionen]{functional_eqn_exp} Addition in Multiplikation überführen,
überführen umgekehrt Logarithmen Multiplikation in Addition. Das gilt natürlich ebenso für Division und Subtraktion.
Denn Division ist lediglich die Multiplikation mit dem Kehrwert und Subtraktion die Addition des Negativen.
\\\\
Jetzt erinnern wir uns noch daran, dass beliebige positiven Zahlen $u$ und $v$ als 
Potenzen $u=a^x$ and $v=a^y$ geschrieben werden können mit (sogar eindeutig bestimmten) 
reellen Zahlen $x$ und $y$. Damit können wir unsere Beobachtungen von eben 
zusammenfassen: 
}
\lang{en}{
	Logarithms connect multiplication with addition by mapping the product of two 
  arguments to the sum of their logarithms.
	This is just the reverse of the relationship between addition and multiplication in 
	\ref[link2][exponential functions]{functional_eqn_exp}, where the sum of the 
  arguments becomes the product of the logarithms. 
	The same relationship exists between division and subtraction.
  \\\\
  Now recall that given $a>0$, any positive numbers $u$ and $v$ can be written as 
  powers $u=a^x$ and $v=a^y$ for some unique $x$ and $y$. Hence we can rephrase our 
  identities from above: 
}
\\

\begin{theorem}[\lang{de}{Multiplikationstheorem}
                \lang{en}{Product theorem}]\label{functional_eqn_log}
	\lang{de}{Für alle $u,v>0$ und alle Basen $a\neq 1$, $a>0$, gilt:}
	\lang{en}{For $u,v>0$, we have:}\\
    \begin{itemize}
    \item[(a)]
    $\log_a(u\cdot v)=\log_a(u)+\log_a(v)$,
    \item[(b)]
    $\log_a\left(\frac{u}{v}\right)=\log_a(u)-\log_a(v)$.
    \end{itemize}
% 	\begin{align*}
% 		\lang{de}{
% 		\text{a)}&\quad\ln(u \cdot v)&=& \ln(u) + \ln(v),\\
% 		\text{b)}& \quad\;\ln\left(\frac{u}{v}\right)&=& \ln(u) - \ln(v), \quad \text{insbesondere gilt}\\
% 			     & \quad\;\ln\left(\frac{1}{v}\right)&=& - \ln(v).\\
% 		\text{c)}&\quad\lg(u \cdot v)&=& \lg(u) + \lg(v),\\
% 		\text{d)}& \quad\;\lg\left(\frac{u}{v}\right)&=& \lg(u) - \lg(v), \quad \text{insbesondere gilt}\\
% 			     & \quad\;\lg\left(\frac{1}{v}\right)&=& - \lg(v).}
% 		\lang{en}{
% 		\text{a)}&\quad\ln(u \cdot v)&=& \ln(u) + \ln(v),\\
% 		\text{b)}& \quad\;\ln\left(\frac{u}{v}\right)&=& \ln(u) - \ln(v), \quad \text{and in particular}\\
% 			     & \quad\;\ln\left(\frac{1}{v}\right)&=& - \ln(v).\\
% 		\text{c)}&\quad\lg(u \cdot v)&=& \lg(u) + \lg(v),\\
% 		\text{d)}& \quad\;\lg\left(\frac{u}{v}\right)&=& \lg(u) - \lg(v), \quad \text{and in particular}\\
% 			     & \quad\;\lg\left(\frac{1}{v}\right)&=& - \lg(v).}
% 	\end{align*}
\lang{de}{\floatright{\href{https://www.hm-kompakt.de/video?watch=153}{\image[75]{00_Videobutton_schwarz}}}}
\\	
\end{theorem}

% \begin{proof*}[\lang{de}{Beweis Multiplikationstheorem} \lang{en}{Proof Product Theorem}]
% \begin{showhide}
% \begin{tabs*}[\initialtab{0}\class{example}]
% 		\tab{\lang{de}{Beweis a)}\lang{en}{Proof of a)}}
% 		\begin{align*}
% 			&\ln(u) &=& s \;\Leftrightarrow\; e^s = u& \\
% 			\text{\lang{de}{und}\lang{en}{and}}\quad &&&&\\
% 			&\ln(v) &=& t \;\Leftrightarrow\; e^t = v &\\
% 			&&&& \\
% 			&\ln(u \cdot v) &=& \ln (e^s \cdot e^t) &\\
% 			&&=& \ln (e^{s + t}) = s+t& \\
% 			&&&&\\
% 			& &=& \ln(u) + \ln(v)  &
% 		\end{align*}
		
% 		\tab{\lang{de}{Beweis b)}\lang{en}{Proof of b)}}
% 		\begin{align*}
% 			&\ln(u) &=& s \;\Leftrightarrow\; e^s = u& \\
% 			\text{\lang{de}{und}\lang{en}{and}}\quad &&&& \\
% 			&\ln(v) &=& t \;\Leftrightarrow\; e^t = v &\\
% 			&&&& \\
% 			&\ln\left(\frac{u}{v}\right) &=& \ln \left(\frac{e^s}{e^t}\right)& \\
% 			&&=& \ln (e^{s - t}) = s-t& \\
% 			&&&&\\
% 			& &=& \ln(u) - \ln(v)  &
% 		\end{align*}
% 	\end{tabs*}
%     \end{showhide}
% \end{proof*}

 

\begin{example}
%  \textbf{\lang{de}{Multiplikationstheorem:}\lang{en}{Product Theorem}}\\
%  \\
\begin{tabs*}[\initialtab{1}\class{example}]
\tab{\lang{de}{Beispiel a)}\lang{en}{Example a)}}
\[
\lg(20) + \lg(0\lang{de}{,}\lang{en}{.}5) = \lg (20\cdot 0\lang{de}{,}\lang{en}{.}5) = \lg(10)=1.
\]
\tab{\lang{de}{Beispiel b)}\lang{en}{Example b)}}
\[
\ln (2 e^5) = \ln 2 + \ln e^5 = \ln (2) + 5 
\]
\tab{\lang{de}{Beispiel c)}\lang{en}{Example c)}}\lang{de}{Für $-5<x<5$ gilt}\lang{en}{Given $-5<x<5$:}
\begin{eqnarray*}
	\lg (5-x) + \lg(5+x) 
	& =& \lg\left((5-x)\cdot(5+x)\right)\\
	& =& \lg(25-x^2).
\end{eqnarray*}
\tab{\lang{de}{Beispiel d)}\lang{en}{Example d)}}
\lang{de}{Für $x>3$ gilt}\lang{en}{For $x>3$:}
\begin{eqnarray*}
	\ln (x^2 -6x +9) - \ln (x-3) 
	&=& \ln\left( \frac{x^2 -6x +9}{x-3}\right)\\
	&=&  \ln\left( \frac{(x-3)^2}{x-3}\right) \\
	&=& \ln (x-3).
\end{eqnarray*}

 \end{tabs*}
 \end{example}

\lang{de}{
Aus der \ref[link2][Potenzregel]{power_exp_fcn} $(a^x)^y=a^{x\cdot y}$ leiten wir die 
zweite Rechenregel für Logarithmen ab:
}
\lang{en}{
The rule for \ref[link2][taking powers of an exponential function]{power_exp_fcn} 
$(a^x)^y=a^{x\cdot y}$ carries over into a rule for logarithms which connects taking 
powers with multiplication.
}

\begin{theorem}[\lang{de}{Potenzregel}\lang{en}{Power Rules}]\label{thm:log-power_rule}
	\lang{de}{
  Für $u>0$ und alle reellen Zahlen $t$, sowie für alle Basen $a\neq 1$ ($a>0$) gilt
  }
	\lang{en}{
  The following rules hold for $u>0$, any real number $t$ and for any natural number 
  $n$:
  }    
    \[
    \log_a (u^t)=t\cdot \log_a(u).
    \]
\end{theorem}
\lang{de}{
Diese Regel kann man insbesondere für $t=\frac{1}{n}$ für eine natürliche Zahl $n$ anwenden. Dann erhält man 
}
\lang{en}{
In particular we can apply these rules to $t=\frac{1}{n}$ for a natural number $n$ to 
get 
}
\[\log_a(\sqrt[n]{u})=\log_a(u^{\frac{1}{n}})=\frac{1}{n}\log_a(u).
\]
%     und alle natürlichen Zahlen $n$ gelten die Regeln:}

% 	\begin{align*}
% 		a)& \ln(u^t) &=& t\, \ln(u) \\
% 		b)& \ln(\sqrt[n]{u}) &=& \frac{1}{n} \cdot \ln(u) \\
% 		c)& \lg(u^t) &=& t\, \lg(u) \\
% 		d)& \lg(\sqrt[n]{u}) &=& \frac{1}{n} \cdot \lg(u)
% 	\end{align*}
	
	

\begin{proof*}[\lang{de}{Beweis Potenzregel}\lang{en}{Proof of the power rule}]
\lang{de}{
Wir drücken $u$ aus durch $u=a^s$ mit passenden $s\in\R$ und benutzen $(a^{s})^t=a^{s\cdot t}$ sowie die Umkehreigenschaft \ref{bem:exp_ln}:
}
\lang{en}{
We express $u$ as $u=a^s$ for some $s\in\R$ and use $(a^{s})^t=a^{s\cdot t}$ and 
remark \ref{bem:exp_ln}:
}
\[
\log_a(u^t)=\log_a((a^s)^t)=\log_a(a^{s\cdot t})=s\cdot t=t\cdot \log_a(u).
\]
\end{proof*}

  \begin{example}
\begin{tabs*}[\initialtab{1}\class{example}]
\tab{\lang{de}{Beispiel a)}\lang{en}{Example a)}}
 \begin{eqnarray*} 
\lg \left( \frac{1}{1000^{3}}\right) &=& \lg (1000^{-3}) \\
&=& -3 \lg (1000) \\
&=& -3 \lg (10^3) \\
&=& -3 \cdot 3 \\
&=& -9
\end{eqnarray*}
\tab{\lang{de}{Beispiel b)}\lang{en}{Example b)}}
 \[
 \ln \left( \sqrt{e}^{\;5} \right) =\ln \left( e^{\frac{5}{2}} \right) 
= \frac{5}{2}\ln (e)= \frac{5}{2}
\]
\end{tabs*}
\end{example}
 
 \begin{quickcheck}
		\field{rational}
		\type{input.number}
		\begin{variables}
			\randint{k}{2}{5}
			\randint{m}{2}{5}
			\randint{n}{2}{5}
			\function[calculate]{k2}{k^2}
			\function[calculate]{c}{2*n-m}
		\end{variables}
		
			\text{\lang{de}{Vereinfachen Sie den Ausdruck }\lang{en}{Simplify the expression }
      			$2\cdot \ln(\var{k}e^\var{n})-\ln(\var{k2}e^\var{m})=$\ansref.}
		
		\begin{answer}
			\solution{c}
		\end{answer}
		\explanation{\lang{de}{Mit obigen Rechenregeln ist }
                 \lang{en}{Using the rules from this section: }\\
		\begin{align*}
		2\cdot \ln(\var{k}e^\var{n})-\ln(\var{k2}e^\var{m}) 
		&=2\cdot \big( \ln(\var{k})+\ln(e^\var{n})\big)-\ln (\var{k}^2)-\ln(e^\var{m}) \\
		&=2\cdot \ln(\var{k}) + 2\cdot \var{n} - 2\cdot \ln(\var{k}) - \var{m}\\
		&=2\cdot \var{n} -\var{m} = \var{c}
		\end{align*}
		}
	\end{quickcheck}
 
%\section{\lang{de}{Die Transformationsformel}\lang{en}{The Change of Base Formula}}
\lang{de}{
Logarithmen zu verschiedenen Basen können ineinander umgerechnet werden:
}
\lang{en}{
The natural logarithm can be transformed into the logarithm with base $10$ and vice 
versa by the following rule for the change of base:
}
 
\begin{theorem}[\lang{de}{Transformationsformel}
                \lang{en}{Change of base}]\label{change-base}
\lang{de}{
Für alle $u>0$ und beliebige Basen $a$ und $b$ (ungleich $1$ und größer $0$) gilt 
}
\lang{en}{
Given $u>0$ and some bases $a$ and $b$ (not equal to $1$ and larger than $0$), we have 
}
\[
\log_a(u)=\frac{\log_b(u)}{\log_b(a)}.
\]

\lang{de}{
Insbesondere ist kann jeder Logarithmus durch den natürlichen ausgedrückt werden,
}
\lang{en}{In particular:}
\[
\log_a(u)=\frac{\ln(u)}{\ln(a)}.
\]
\end{theorem}
\begin{proof*}[\lang{de}{Beweis Transformationsformel}
               \lang{en}{Proof of change of base formula}]
\lang{de}{
Nach der Potenzregel für Logarithmen \ref{thm:log-power_rule} ist 
$\log_a(u)\cdot \log_b(a)=\log_b(a^{\log_a(u)})$. 
Wegen \ref{bem:exp_ln} ist $a^{\log_a(u)}=u$. Also gilt
}
\lang{en}{
By the power rule for logarithms \ref{thm:log-power_rule}, 
$\log_a(u)\cdot \log_b(a)=\log_b(a^{\log_a(u)})$. 
By remark \ref{bem:exp_ln}, $a^{\log_a(u)}=u$. Hence
}
\[
\log_a(u)\cdot \log_b(a)=\log_b(u).
\]
\lang{de}{
Diese Gleichung dürfen wir durch $\log_b(a)$ teilen (denn $a\neq 1$). Damit erhalten wir die Behauptung.
}
\lang{en}{
This equation can be divided by $\log_b(a)$ on both sides (as $a\neq 1$) to complete 
the proof.
}
% \begin{align*}
% c) &\ln(10) &=& \frac{1}{\lg(e)}.
% \end{align*}

% \begin{tabs*}[\initialtab{0}\class{example}]
% \tab{\lang{de}{Beweis a)}\lang{en}{Proof of a)}}
%  \lang{de}{Multiplikation mit $\lg(e)$ ergibt die äquivalente Gleichung:}
%  \lang{en}{Multiplication with $\lg(e)$ gives us the equivalent equation:} 

%  \begin{eqnarray*}
% \ln(u)\,\lg(e) &=& \lg(u),
%  \end{eqnarray*}
%  \lang{de}{die linke Seite ist nach der Potenzregel und Bemerkung \ref{logid}:}
%  \lang{en}{By the power rules and remark \ref{logid} this is:}
%  \begin{eqnarray*}
%   \ln(u)\,\lg(e) &=& \lg(e^{\ln(u)})\\
%    &=& \lg(u).
%  \end{eqnarray*}

%  \tab{\lang{de}{Beweis c)}\lang{en}{Proof of c)}}
%  \lang{de}{Setze $u=10$ in Gleichung a).}
%  \lang{en}{In equation a) set $u=10$.}

% \end{tabs*}
\end{proof*}
 
\begin{example}
 \begin{tabs*}[\initialtab{1}\class{example}] 
 \tab{\lang{de}{Beispiel a)}\lang{en}{Example a)}}
 
 \[
\lg (\text{e}^x) =\frac{\ln(e^x)}{\ln(10)} = \frac{x}{\ln(10)}
\]
\tab{\lang{de}{Beispiel b)}\lang{en}{Example b)}}
\lang{de}{Für $x>-2$ gilt}\lang{en}{For all $x>-2$:} 
 
\begin{eqnarray*} 
\ln (x+2) + \lg (x+2) &=& \ln(x+2) + \frac{\ln(x+2)}{\ln(10)}  \\
&=& \ln(x+2) \left(1 + \frac{1}{\ln(10)}\right)
\end{eqnarray*}
\end{tabs*}
 \end{example} 

\begin{quickcheck}
		\field{real}
		\displayprecision{3}
  		\correctorprecision{3}
		\type{input.number}
		\begin{variables}
			\randint{k}{-3}{3}
			\function[calculate]{b}{10^k}
%			\function[calculate]{a}{2*n-m}
		\end{variables}
		
			\text{\lang{de}{
       Bestimmen Sie $a$ so, dass $\ln(\var{b})=\frac{a}{\lg(e)}$ gilt.\\ $a=$\ansref.
       }
       \lang{en}{
       Determine $a$ such that $\ln(\var{b})=\frac{a}{\lg(e)}$.\\ $a=$\ansref.}
       }
		
		\begin{answer}
			\solution{k}
		\end{answer}
		\explanation{\lang{de}{Nach dem Transformationsgesetz ist}
                 \lang{en}{By the change of base formula,} 
                 $\ln(\var{b})=\frac{\lg(\var{b})}{\lg(e)}$. 
                 \lang{de}{Also ist}\lang{en}{Thus} 
                 $a= \lg(\var{b})=\lg(10^{\var{k}})=\var{k}$. }
	\end{quickcheck}
    
    
    
    
    

\section{\lang{de}{Logarithmusfunktionen}\lang{en}{Logarithmic functions}}\label{sec:log_fkt}

\lang{de}{
Wir haben bereits gesehen, dass die Logarithmen $\log_a(x)$ für jede positive reelle Zahl $x$ wohldefiniert sind.
Wir können also Logarithmusfunktionen definieren. 
Wir benutzen hier anders als bisher die gewöhnliche Funktionsvariable $x$ statt $u$.
Ebenso wissen wir, dass  jeder Wert $y\in\R$ als Logarithmus darstellbar ist, nämlich $y=\log_a(a^y)$.
}
\lang{en}{
As per Definition \ref{ln}, the natural logarithm $\ln(x)$ of $x > 0$ is the number
$s$ such that $e^s = x$. Since there is exactly one such number $s$, we can write 
$s = \ln(x)$ as a function of the variable $x$. This leads us to the following definition:
}

\begin{definition}
\lang{de}{
Für jede Zahl $a>0$ ($a\neq 1$) heißt die Funktion  
\[f:\R^{+}\to\R,\quad x\mapsto f(x)=\log_a(x)\]
die \notion{Logarithmusfunktion} zur Basis $a$. \\
Sie nimmt jeden Wert in $\R$ an.
}
\lang{en}{For any $a>0$ ($a\neq 1$) we call the function
\[f:\R^{+}\to\R,\quad x\mapsto f(x)=\log_a(x)\]
the \emph{logarithm function to base $a$}.
\\\\
This notation means that the domain of the natural logarithm $f$ is the set of positive 
real  numbers, $D_f_{\ln} = R^{+} $, its codomain is the set of real numbers 
$W_f_{\ln} = R$, and it maps each $x$ in the domain to $f(x) = \ln(x)\,$.
}
\end{definition}

% \lang{de}{Entsprechendes gilt für den Zehnerlogarithmus:}
% \lang{en}{Correspondingly we have for the logarithm with base $10$:}

% \begin{definition}
% \lang{de}{Die Funktion mit der Funktionsgleichung
% \begin{center}
% $f(x) = \lg(x)\, ,\quad x > 0,$
% \end{center}
% heißt \emph{Zehner-Logarithmusfunktion}.
% Ihr Definitionsbereich ist die Menge der positiven reellen Zahlen, 
% $D_f_{\lg} = \mathbb{R}^+ $, und die Wertemenge umfasst alle 
% reellen  Zahlen, $W_f_{\lg} = \mathbb{R}$.}

% \lang{en}{The function with the equation
% \begin{center}
% $f(x) = \lg(x)\, ,\quad x > 0,$
% \end{center}
% is called \emph{the logarithm} with base $10$.
% The domain of the log function with base is $10$ is again the set of positive real  numbers,
% $D_f_{\lg} = \mathbb{R}^+ $ and its range is all real numbers $W_f_{\lg} = \mathbb{R}$.}
% \end{definition}

\lang{de}{
Die Graphen der natürlichen Logarithmusfunktion und der Zehner-Logarithmusfunktion 
sehen so aus:
}
\lang{en}{
The graphs of the natural logarithm and common logarithm functions look as follows:
}

\begin{center}
\image{T104_Logarithms_A}
\end{center}

\lang{de}{
Die Bemerkung \ref{bem:exp_ln} liefert uns wichtige Eigenschaften von 
Logarithmusfunktionen: Die Logarithmusfunktion $x\mapsto \log_a(x)$ ist die 
\emph{Umkehrfunktion} der Exponentialfunktion $x\mapsto a^x$.
Dies ist die elegante Formulierung davon, dass wir alle Rechengesetzte für Logarithmen 
aus denen der Exponentialfunktionen gefolgert haben. Jede Logarithmusfunktion hat eine 
einzige Nullstelle, nämlich $x=1$.\\
Die Funktionsgraphen sehen wie folgt aus:
}
\lang{en}{
Remark \ref{bem:exp_ln} tells us that the natural logarithm function is the 
\emph{inverse function} of the exponential function $f(x)=e^x$. The logarithm function 
to base $10$ is the \emph{inverse function} of $f(x)=10^x$. This is the reason why the 
rules for powers transferred to rules for logarithms so easily. \\
The graphs of the functions are shown below, note the single root at $x=1$:
}
\begin{center}
\image{T104_Logarithms_B}
\end{center}
\lang{de}{
In der folgenden Tabelle fassen wir  wichtige Eigenschaften des natürlichen Logarithmus zusammen.
Sie gelten genauso für jede Logarithmusfunktion $\log_a(x)$ mit Basis $a>1$. 
Dies gilt wegen $\log_a(x)=\frac{\ln(x)}{\ln(a)}$ und $\ln(a)>0$ für $a>1$.
}
\lang{en}{
In the following table we present some important properties of natural logarithms. 
They also apply to every logarithm function $\log_a(x)$ with base $a>1$, because 
$\log_a(x)=\frac{\ln(x)}{\ln(a)}$ and $\ln(a)>0$ for $a>1$.
}
\begin{table}
  \head
  $f(x) = \ln (x)$ & $ $ 
  \body
  $D_f = \mathbb{R}^+$ & 
    \lang{de}{Definitionsbereich ist die Menge der positiven Zahlen.}
    \lang{en}{The domain is the set of all positive numbers.} \\
  $W_f = \mathbb{R}$ & 
    \lang{de}{Wertemenge ist die Menge der reellen Zahlen.}
    \lang{en}{The range is the set of all real numbers.}  \\
  $f(1) = 0$& 
    $x=1$ \lang{de}{ist die einzige Nullstelle.}
          \lang{en}{is the only zero.}\\
  $f(x_1) < f(x_2)\,$ \lang{de}{für}\lang{en}{for} $x_1 < x_2$& 
    \lang{de}{Die Funktion ist streng monoton wachsend.}
    \lang{en}{The functions are strictly monotonically increasing.}\\
  \lang{de}{Die Steigung von $f$ nimmt ständig ab.}
  \lang{en}{The slope is decreasing.} & 
    \lang{de}{Rechtskurve}
    \lang{en}{The graph is concave down.} \\
  \lang{de}{für $ x \rightarrow 0$ gilt $\;f(x) \rightarrow -\infty$}
  \lang{en}{$\;f(x) \rightarrow -\infty$ as $ x \rightarrow 0$} & 
    \lang{de}{y-Achse ist Asymptote.}
    \lang{en}{The y-axis is an asymptote.}
\end{table}

\begin{example}[\lang{de}{Wertetabellen (gerundet)}
                \lang{en}{Table of values - with rounded values}]
\begin{tabs*}[\initialtab{0}\class{example}]
 
\tab{$ \ln $}
 \begin{center}
  \lang{de}{
   \begin{table}
  \body[\cellaligns{cccccccc}]
 $x$& $\frac{1}{8}$ & $\frac{1}{4}$ & $\frac{1}{2}$ & $1$ & $2$ & $3$ & $4$\\
  $\;\ln (x)\;$ & $\;-2,1\;$ & $\;-1,4\;$ & $\;-0,7\;$ & $\;0\;$ & $\;0,7\;$ & $\;1,1\;$ & $\;1,4\;$ \\
  \end{table}
  }
  \lang{en}{
   \begin{table}
  \body[\cellaligns{cccccccc}]
 $x$& $\frac{1}{8}$ & $\frac{1}{4}$ & $\frac{1}{2}$ & $1$ & $2$ & $3$ & $4$\\
  $\;\ln (x)\;$ & $\;-2.1\;$ & $\;-1.4\;$ & $\;-0.7\;$ & $\;0\;$ & $\;0.7\;$ & $\;1.1\;$ & $\;1.4\;$ \\
  \end{table}
  }
\end{center}
\tab{$ \lg$ }
 \begin{center}
  \lang{de}{
   \begin{table}
  \body[\cellaligns{cccccccc}]
 $x$& $\frac{1}{8}$ & $\frac{1}{4}$ & $\frac{1}{2}$ & $1$ & $2$ & $3$ & $4$\\
 $\;\lg (x)\;$ & $\;-0,9\;$ & $\;-0,6\;$& $\;-0,3\;$ & $\;0\;$ & $\;0,3\;$ & $\;0,48\;$ & $\;0,6\;$
   \end{table}
  }
  \lang{en}{
   \begin{table}
  \body[\cellaligns{cccccccc}]
 $x$& $\frac{1}{8}$ & $\frac{1}{4}$ & $\frac{1}{2}$ & $1$ & $2$ & $3$ & $4$\\
 $\;\lg (x)\;$ & $\;-0.9\;$ & $\;-0.6\;$& $\;-0.3\;$ & $\;0\;$ & $\;0.3\;$ & $\;0.48\;$ & $\;0.6\;$
   \end{table}
  }
\end{center}
\end{tabs*}
\end{example}
\lang{de}{
Die Graphen von Logarithmusfunktionen $\log_a(x)$ zu Basen $0<a<1$ entstehen aus solchen zu Basen 
$>1$ durch Spiegelung an der $x$-Achse.
Wegen \ref{functional_eqn_log} gilt nämlich $\ln(a)=-\ln(\frac{1}{a})$, und somit folgt
}
\lang{en}{
The graph of a logarithm function $\log_a(x)$ to a base $0<a<1$ is like for a base $a>1$, 
but reflected in the $x$ axis. This is thanks to theorem \ref{functional_eqn_log}, by which 
$\ln(a)=-\ln(\frac{1}{a})$ and hence
}
\[
\log_a(x)=\frac{\ln(x)}{\ln(a)}=\frac{\ln(x)}{-\ln(\frac{1}{a})}=-\log_{\frac{1}{a}}(x).
\]
\begin{center}
\image{T104_Logarithms_C}
\end{center}
\lang{de}{
Zum Abschluss fassen wir die korrespondierenden Regeln für Exponentialfunktionen und Logarithmusfunktionen zusammen:
}
\lang{en}{
To summarize, we present the rules for exponential functions alongside the corresponding rules 
for logarithmic functions:
}
\begin{rule}[\lang{de}{Zusammenfassung}\lang{en}{Summary}]
\begin{table}
  \head
  \lang{de}{Eigenschaft}\lang{en}{Property} & 
    \lang{de}{Für}\lang{en}{For} $x,y\in\R$ & 
      \lang{de}{Für}\lang{en}{For} $u,v>0$
  \body
  $\:\mathbb{R}$ & 
    \lang{de}{Definitionsbereich von $e^x$}\lang{en}{Domain of $e^x$} &  
      \lang{de}{Wertebereich von $\ln(u)$}\lang{en}{Image of $ln(u)$}\\
  $\:\R_{>0} $ 
    & \lang{de}{Wertebereich von $e^x$}\lang{en}{Image of $e^x$} & 
      \lang{de}{Definitionsbereich von $\ln(u)$}\lang{en}{Domain of $\ln{u}$}\\
  \lang{de}{Umkehrfunktionen}\lang{en}{Inverse function} & 
    $\:\ln(e^x)=x$ & 
      $\:e^{\ln(u)}=u$\\
  \lang{de}{spezielle Werte}\lang{en}{Special values} & 
    $\:e^0=1$ & 
      $\:\ln(1)=0$\\
  \lang{de}{Multiplikation/Addition}\lang{en}{Multiplication/addition} & 
    $\:e^x\cdot e^y=e^{x+y}\quad$ & 
      $\:\ln(u)+\ln(v)=\ln(u\cdot v)$\\
  \lang{de}{Kehrwert/Negatives}\lang{en}{Reciprical/negatives} & 
    $\:e^{-x}=\frac{1}{e^x}$ & 
      $\:-\ln(u)=\ln(\frac{1}{u})$\\
  \lang{de}{Potenzen}\lang{en}{Powers} & 
    $\:(e^x)^y=e^{x\cdot y}$ & 
      $\:\ln(u^t)=t\cdot \ln(u)\:$ für $t\in\R$\\
  \lang{de}{Transformation für $a>0$}\lang{en}{Change to natural base $a>0$}, $a\neq 1$ & 
    $\:a^x=e^{x\ln(a)}$ & 
      $\:\log_a(u)=\frac{\ln(u)}{\ln(a)}$
  \end{table}
\end{rule}

\section{\lang{de}{Exponential- und Logarithmusgleichungen}
         \lang{en}{Exponential and logarithmic equations}}\label{sec:exp_log_equation}
\anchor{uniqueness}{}

 \lang{de}{
 Die Definition der Potenzen und Logarithmen bzw. der zugehörigen Funktionen liefert direkt einen 
 Weg, um Gleichungen mit Potenzen bzw. Logarithmen zu lösen. Sowohl die Exponentialfunktionen als 
 auch die Logarithmusfunktionen sind streng monoton wachsend oder fallend. Daher gelten für alle 
 $a>0,\; a\ne 1$ die wichtigen Äquivalenzen
 }
 \lang{en}{
 The definition of powers and logarithms and their respective functions give us a direct way to 
 solve equations involving powers and logarithms. Both exponential functions and logarithmic 
 functions are strictly monotonically increasing or decreasing. From this we get some important 
 equivalencies: for all $a>0,\; a\ne 1$,
 }
 
 \begin{align*}
	a^{x_1} &=& a^{x_2} \qquad&\Leftrightarrow&\qquad x_1=x_2 \quad\text{\lang{de}{für}\lang{en}{for}}\; x_1,x_2\in\R,\\
	\log_a(x_1)&=&\log_a(x_2) \qquad&\Leftrightarrow&\qquad x_1=x_2 \quad\text{\lang{de}{für}\lang{en}{for}}\; x_1,x_2 >0.
 \end{align*}
  
\lang{de}{
Die Rechenregeln für Logarithmen und  Potenzen helfen, Potenz- und Logarithmusgleichungen 
umzuformen, um die Gleichungen zu vereinfachen und zu lösen:\\
}
\lang{en}{
The rules for calculating with logarithms and powers help us to transform, simplify and solve 
exponential and logarithmic equations.\\
}
 
\begin{rule}[\lang{de}{Lösen von Potenzgleichungen}\lang{en}{Solving exponential equations}]\label{rule:potenz_glg}
	\lang{de}{Für $u > 0$ und jede Basis $a>0$, $a\neq 1$, gilt:}
	\lang{en}{For $a,u > 0,\; a\neq 1$, we can rearrange an exponential equation as follows:}
    \[
    a^x=u \:\Leftrightarrow\: \ln(a^x)=\ln(u) \:\Leftrightarrow\: x\ln(a)=\ln(u) 
    \:\Leftrightarrow\: x = \frac{\ln(u)}{\ln(a)}.
    \]
    \lang{de}{
    Statt des natürlichen Logarithmus zur Basis $e$ kann hier auch jede andere Basis $b$ benutzt 
    werden (\lref{change-base}{Transformationsformel}!).
    }
    \lang{en}{
    Instead of the natural logarithm (base $e$), the above holds for any other base $b$ 
    (\lref{change-base}{change of base}!). 
    }
% 	\begin{eqnarray*}
% 		\text{a)}\quad &&a^x = u \qquad | \qquad \ln\\
% 		&\Leftrightarrow& \ln (a^x) = \ln(u)\\
% 		&\Leftrightarrow& x\ln(a) = \ln(u)\\
% 		&\Leftrightarrow& x = \frac{\ln(u)}{\ln(a)} \,,\\
% 		\text{b)}\quad &&a^x = u \qquad | \qquad \lg\\
% 		&\Leftrightarrow& \lg (a^x) = \lg(u)\\
% 		&\Leftrightarrow& x\lg(a) = \lg(u)\\
% 		&\Leftrightarrow& x = \frac{\lg(u)}{\lg(a)} \,.
% 	\end{eqnarray*}
\end{rule}
% \lang{en}{The equality $\;\frac{\ln(u)}{\ln(a)}= \frac{\lg(u)}{\lg(a)}\;$ 
% also follows from the \lref{change-base}{change of base rule}.}
% \lang{de}{Die Gleichung $\;\frac{\ln(u)}{\ln(a)}= \frac{\lg(u)}{\lg(a)}\;$ 
% folgt auch aus dem \lref{change-base}{Transformationsformel}.}
\begin{rule}[\lang{de}{Lösen von Logarithmusgleichungen}\lang{en}{Logarithmic equations}]\label{rule:log_glg}
 \lang{de}{
 Für $x > 0$, $r\in\R$, und jede Basis $a>0$, $a\neq 1$, gilt:
 }
 \lang{en}{
 For $a,x > 0$, $a\neq 1$ and $r\in\R$, we can rearrange a logarithmic equation as follows:
 }
 \[
 \log_a(x)=r \:\Leftrightarrow\: a^{\log_a(x)}=a^r \:\Leftrightarrow\: x=a^r.
 \]
%  \begin{align*}
%  \text{a)}&&&\ln (x) = u \qquad | \qquad  e^{(\cdot)}\\
%  &&\Leftrightarrow& e^{\ln(x)}= e^u\\
%  &&\Leftrightarrow& x = e^u .\\
%  \text{b)}&&&\lg (x) = u \qquad | \qquad  10^{(\cdot)}\\
%  &&\Leftrightarrow& 10^{\lg (x)}= 10^u\\
%  &&\Leftrightarrow& x = 10^u .
% \end{align*}
\end{rule}
\lang{de}{
Auch Logarithmusgleichungen kann man mit Hilfe beliebiger Basen lösen, z.B. mit $e$:
}
\lang{en}{
Logarithmic equations can also be solved in different bases, for instance to base $e$:
}
\[
\log_a(x)=r\:\Leftrightarrow \:x=a^r=(e^{\ln(a)})^r=e^{r\cdot\ln(a)}.
\]

\begin{example}[\lang{de}{Exponentialgleichungen:}\lang{en}{Exponential Equations}]\label{rule:exp_glg}
 \begin{tabs*}[\initialtab{1}\class{example}] 
 \tab{\lang{de}{Beispiel a)}\lang{en}{Example a)}}
 \begin{eqnarray*}
 e^{2x + 3} = 4 &\Leftrightarrow& \ln ( e^{2x + 3}) = \ln 4 \\
 &\Leftrightarrow& 2x + 3 = \ln 4 \\
  &\Leftrightarrow& x = \frac{1}{2} (\ln(4) -3 ).
 \end{eqnarray*}
 \tab{\lang{de}{Beispiel b)}
 \lang{en}{Example b)}}
 \lang{de}{Für Parameter $b>0$, $b\neq 1$ und $a$:}
 \lang{en}{For $b>0,\;b\neq 1$:}
 \begin{eqnarray*}
 \sqrt[4]{b^{x-a}} = \sqrt[5]{b^{x+a}} &\Leftrightarrow& b^{\frac{x - a}{4}} = b^{\frac{x + a}{5}}\\
 &\Leftrightarrow& \ln (b^{\frac{x - a}{4}}) = \ln (b^{\frac{x + a}{5}})\\
  &\Leftrightarrow& \frac{x - a}{4} \cdot\ln(b) = \frac{x + a}{5}\cdot\ln(b)\\
  &\Leftrightarrow& 5 (x - a) = 4 (x + a)\\
  &\Leftrightarrow& 5x -5a = 4x + 4a \\
  &\Leftrightarrow& x = 9a\,.
 \end{eqnarray*}
 \lang{de}{Die Lösung ist also unabhängig von $b$.}
 \lang{en}{The solution is in fact independent from $b$.}
  \tab{\lang{de}{Beispiel c)}\lang{en}{Example c)}}
 \begin{eqnarray*}
 \left( 2^{3 - x} \right)^{2-x} = 1  &\Leftrightarrow& 2^{(3-x)(2-x)} = 1 \\
 &\Leftrightarrow& \ln \left(2^{(3-x)(2-x)}\right) = \ln (1) \\
 &\Leftrightarrow& (3-x)(2-x)\ln(2)=0 \\
 &\Leftrightarrow& x = 3\;\;\; \text{\lang{de}{oder}\lang{en}{or}} \;\;\; x = 2.
 \end{eqnarray*}
 \end{tabs*}
\end{example}

\begin{example}[\lang{de}{Logarithmusgleichungen}\lang{en}{Logarithmic Equations}]
 	\begin{tabs*}[\initialtab{1}\class{example}] 
 	\tab{\lang{de}{Beispiel a)}\lang{en}{Example a)}}
	 	\lang{de}{Für $x>-2$ gilt:}\lang{en}{For $x>-2$:} 
	 	\begin{eqnarray*}
	 	\ln \frac{1}{2+x} = 0 &\Leftrightarrow& \text{e}^{ \ln \frac{1}{2+x}} = \text{e}^0 \\
	 								    &\Leftrightarrow& \frac{1}{2+x} = 1 \\
	 									&\Leftrightarrow& 2 + x = 1 \\ 
	 									&\Leftrightarrow& x = -1.
	 	\end{eqnarray*}
	 	
 	\tab{\lang{de}{Beispiel b)}\lang{en}{Example b)}}
 		\lang{de}{Für $x>0$ gilt:}\lang{en}{For $x>0$:} 
		\begin{eqnarray*}
			\lg (x^3) + 2 \lg (x^2) = 21  
			&\Leftrightarrow& \lg (x^3) +  \lg \left((x^2)^2\right) = 21\\
			&\Leftrightarrow& \lg (x^3\cdot x^4) = 21\\
			&\Leftrightarrow& \lg (x^7) = 21\\
			&\Leftrightarrow& 10^{ \lg (x^7)} = 10^{21}\\ 
			&\Leftrightarrow& x^7 = 10^{21}\\
			&\Leftrightarrow& \sqrt[7]{x^7} = \sqrt[7]{10^{21}}\\
			&\Leftrightarrow& x = 10^{\frac{21}{7}}\\
			&\Leftrightarrow& x = 10^{3} .
		\end{eqnarray*}
		
	\tab{\lang{de}{Beispiel c)}\lang{en}{Example c)}}
	  	\lang{de}{Für $x>0$ gilt:}\lang{en}{For $x>0$:} 
	 	\begin{eqnarray*}
	 		5^{\lg x} = 2 \cdot 3^{\lg x} &\Leftrightarrow& \frac{ 5^{\lg x}}{3^{\lg x}} = 2 \\
	  		&\Leftrightarrow& \left(\frac{5}{3} \right)^{\lg x} = 2 \\
	  		&\Leftrightarrow&  \ln \left(\left(\frac{5}{3} \right)^{\lg x}\right) = \ln 2\\
	   		&\Leftrightarrow& \lg x \cdot \ln \left(\frac{5}{3} \right) = \ln 2\\
	    	&\Leftrightarrow& \lg x =  \frac{\ln 2}{\ln \left(\frac{5}{3} \right)}\\
	     	&\Leftrightarrow& \lg x =  \frac{\ln 2}{\left( \ln 5 - \ln 3 \right)}\\
	    	&\Leftrightarrow& 10^{\lg x} =  10^{\frac{\ln 2}{ \ln 5 - \ln 3}}\\
	    	&\Leftrightarrow& x = 10^{\frac{\ln 2}{\ln 5 - \ln 3}}\,.
	 \end{eqnarray*}
 	\end{tabs*}
\end{example}


\begin{quickcheck}
		\field{real}
		\displayprecision{2}
 		\correctorprecision{2}
		\type{input.number}
		\begin{variables}
			\randint[Z]{c}{-2}{4}
			\randint[Z]{k}{-3}{3}
		    \function[normalize]{f}{x-k}
			\function[calculate]{ls}{k+10^c}
		\end{variables}

			\text{\lang{de}{
      Bestimmen Sie die Lösung der Gleichung $\lg(\var{f})=\var{c}$.\\ Die Lösung ist \ansref.
      }
      \lang{en}{
      The solution of the equation $\lg(\var{f})=\var{c}$ is\\ \ansref.
      }}

		\begin{answer}
			\solution{ls}
		\end{answer}
		\explanation{\lang{de}{Umformen ergibt:}\lang{en}{Rearranging the equation gives:}
		\begin{align*}
			& \lg(\var{f}) &=\var{c} &\quad &\\
			\Leftrightarrow & \var{f} & =10^\var{c}  &\quad & 
      (\text{\lang{de}{und }\lang{en}{and }}\var{f}>0) \\
			\Leftrightarrow & x & =\var{k}+ 10^\var{c}= \var{ls}.&&
		\end{align*} }
	\end{quickcheck}

% 	\begin{genericGWTVisualization}[550][1000]{mathlet1}
% 		\begin{variables}
% 			\randint{randomA}{1}{2}
% 			
% 			\point[editable]{P}{rational}{var(randomA),var(randomA)}
% 		\end{variables}
% 		\color{P}{BLUE}
% 		\label{P}{$\textcolor{BLUE}{P}$}
% 
% 		\begin{canvas}
% 			\plotSize{300}
% 			\plotLeft{-3}
% 			\plotRight{3}
% 			\plot[coordinateSystem]{P}
% 		\end{canvas}
% 		\text{Der Punkt hat die Koordinaten $(\var{P}[x],\var{P}[y])$.}
% 	    	\end{genericGWTVisualization}

\end{visualizationwrapper}

\end{content}