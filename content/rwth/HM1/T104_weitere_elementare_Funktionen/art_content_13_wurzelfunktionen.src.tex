%$Id:  $
\documentclass{mumie.article}
%$Id$
\begin{metainfo}
  \name{
    \lang{de}{Wurzelfunktionen}
    \lang{en}{$n$th roots}
  }
  \begin{description} 
 This work is licensed under the Creative Commons License Attribution 4.0 International (CC-BY 4.0)   
 https://creativecommons.org/licenses/by/4.0/legalcode 

    \lang{de}{Beschreibung}
    \lang{en}{Description}
  \end{description}
  \begin{components}
    \component{generic_image}{content/rwth/HM1/images/g_img_00_Videobutton_schwarz.meta.xml}{00_Videobutton_schwarz}
    \component{js_lib}{system/media/mathlets/GWTGenericVisualization.meta.xml}{mathlet1}
  \end{components}
  \begin{links}
  \end{links}
  \creategeneric
\end{metainfo}
\begin{content}
\usepackage{mumie.ombplus}
\ombchapter{4}
\ombarticle{1}
\usepackage{mumie.genericvisualization}

\begin{visualizationwrapper}

\title{\lang{de}{Wurzelfunktionen}\lang{en}{$n$th roots}}

\begin{block}[annotation]
 
  
\end{block}
\begin{block}[annotation]
  Im Ticket-System: \href{http://team.mumie.net/issues/9009}{Ticket 9009}\\
\end{block}

\begin{block}[info-box]
\tableofcontents
\end{block}

\section{\lang{de}{{}n-te Wurzeln}\lang{en}{The $n$th root}} \label{sec:n-te_wurzel}

\lang{de}{
Wie im Fall der quadratischen Gleichung $x^2=a$ mit $a\geq 0$ hat auch die allgemeinere 
Gleichung $x^n=a$ mit $a\geq 0$ und $n\in\N$
genau eine reelle Lösung größer oder gleich Null, welche \nowrap{$n$-te Wurzel} von $a$ 
(sprich n-te Wurzel von $a$) genannt wird.
}
\lang{en}{
Like the quadratic equation $x^2=a$ with $a\geq 0$, the more general equation $x^n=a$ with $a\geq 0$ 
and $n\in\N$ has exactly one real solution greater than or equal to zero, which is called the \nowrap{$n$th root} of $a$ (pronounced as it is written).
}

\begin{definition}[\lang{de}{$n\text{-te}$ Wurzel}
                   \lang{en}{The $n$th root}]\label{def_allgwurzel}
	\lang{de}{
  Für $a\geq0$ und $n\in\N$ wird die \notion{\emph{$n$-te Wurzel}} $\sqrt[n]{a}$ aus $a$ definiert 
  als eindeutige Lösung der Gleichung
  }
  \lang{en}{
  Given $a\geq0$ and $n\in\mathbb{N}$ the \nowrap{$n$th root} $\sqrt[n]{a}$ of $a$ is defined as the 
  unique real solution greater than or equal to zero to the equation 
  }
	\[x^n=a,\]
  \lang{de}{die größer gleich Null ist. Speziell gilt: }
  \lang{en}{that is greater than or equal to zero. In particular, }
	\[\sqrt[1]{a}=a.\]
  \lang{de}{
	Man nennt $a$ \notion{\emph{Radikand}} (oder  \notion{\emph{Wurzelbasis}}) und $n$ 
  \notion{\emph{Wurzelexponent}}.
  }
  \lang{en}{
	We call $a$ the \emph{"radicand"} (sometimes called the argument of the radical) and $n$ the 
  \emph{"index"}.
  }
\end{definition} 

\begin{example} 
\lang{de}{
\[\sqrt[3]{0,027}=0,3\,,\qquad\sqrt[4]{\frac{16}{625}}=\frac{2}{5}\,,\qquad\sqrt[6]{4096}=4\,, 
\qquad\sqrt[3]{\frac{1}{125}}=\frac{\sqrt[3]{1}}{\sqrt[3]{125}}= \frac{1}{5}.\]
}
\lang{en}{
\[\sqrt[3]{0.027}=0.3\,,\qquad\sqrt[4]{\frac{16}{625}}=\frac{2}{5}\,,\qquad\sqrt[6]{4096}=4\,, 
\qquad\sqrt[3]{\frac{1}{125}}=\frac{\sqrt[3]{1}}{\sqrt[3]{125}}= \frac{1}{5}.\]
}
\end{example}

\begin{block}[warning]
\lang{de}{
Für negative Radikanden sind Wurzeln nicht definiert! 
Das ist eine Konvention, die sich im Folgenden als äußerst sinnvoll erweisen wird. 
Daher sind Ausdr\"ucke wie
}
\lang{en}{
Roots of negative numbers are not defined in the real numbers! 
This is a convention which will make sense later. 
Hence expressions such as 
}
\[\sqrt[3]{-8}\qquad\text{\lang{de}{ oder }\lang{en}{ or }}\qquad\sqrt[4]{-\frac{2}{3}}\]
\lang{de}{
sinnlos, obwohl zum Beispiel $-2$ eine Lösung der Gleichung $x^3=-8$ ist.
}
\lang{en}{
are meaningless, even though $-2$ is in fact a solution of the equation $x^3=-8$.
}
\end{block}

\lang{de}{
Für das Rechnen mit Wurzeln erhält man folgende Gesetze.
}
\lang{en}{
The following laws must be kept in mind when manipulating expressions containing roots.
}

\begin{theorem}[\lang{de}{Wurzelgesetze}\lang{en}{Laws of roots}]\label{allgwurzel_gesetze}

\lang{de}{Für alle $a,b,c\in\R$ mit $a,b\geq0,\; c>0$ und für alle $m,n\in\mathbb{N}$ gilt}
\lang{en}{For all $a,b,c\in\R$ with $a,b\geq0,\; c>0$ and $m,n\in\mathbb{N}$, we have}
\begin{eqnarray*}
\sqrt[n]{ab} &=& \sqrt[n]{a}\cdot \sqrt[n]{b},\\
\sqrt[n]{\frac{a}{c}} &=& \frac{\sqrt[n]{a}}{\sqrt[n]{c}}\, ,\\
\sqrt[n]{\sqrt[m]{a}} &=& \sqrt[n\cdot m]{a}\,.
\end{eqnarray*}

\end{theorem}

\begin{quickcheck}
		\field{rational}
		\type{input.number}
		\begin{variables}
%			\randint[Z]{a}{-5}{5}
			\randint[Z]{b}{2}{5}
			\randint[Z]{c}{1}{4}
			\randint[Z]{d}{1}{4}
			\randadjustIf{c,d}{c = d}
		    \function[calculate]{q}{c/d}
		    \function[calculate]{f}{q^b}
		\end{variables}
		
			\text{\lang{de}{Die $\var{b}$-te Wurzel von $\var{f}$ ist \ansref.}
            \lang{en}{The $\var{b}$th root of $\var{f}$ is \ansref.}}
		
		\begin{answer}
			\solution{q}
		\end{answer}
	\end{quickcheck}


\section{\lang{de}{Wurzelfunktionen}\lang{en}{The $n$th root function}}\label{sec:wurzel_fkt}

\begin{definition}[\lang{de}{Wurzelfunktion}\lang{en}{$n$th root function}]\label{def:nth-root}
\lang{de}{
Für jede natürliche Zahl $n$, ist die $n$-te Wurzelfunktion die Abbildung 
}
\lang{en}{
For every natural number $n$, the $n$th root function is given by 
}
\[ [0,\infty)\to \R, x\mapsto \sqrt[n]{x}, \]
\lang{de}{
die jeder Zahl $x$ größer oder gleich Null ihre $n$-te Wurzel zuordnet, welche per Definition der
Wurzel auch wieder größer oder gleich Null ist.\\
\floatright{\href{https://www.hm-kompakt.de/video?watch=150}{\image[75]{00_Videobutton_schwarz}}}\\\\
}
\lang{en}{
which maps every real $x\geq0$ to its $n$th root, by definition also greater than or equal to zero.
}
\end{definition}

\begin{remark}
\begin{enumerate}
\item \lang{de}{
Der maximale Definitionsbereich der $n$-ten Wurzelfunktion ist also die Menge der nicht negativen 
reellen Zahlen $[0,\infty)$, die Wertemenge ist ebenso $[0,\infty)$.
}
\lang{en}{
The maximal domain of the $n$th root function is therefore the set of non-negative real numbers 
$[0,\infty)$, and likewise the image of the function is clearly $[0,\infty)$.
}
\item \lang{de}{
Die $n$-te Wurzelfunktion ist eine \emph{reelle Umkehrfunktion} zur
(eingeschränkten) $n$-ten Potenzfunktion $f:[0,\infty)\to \R, x\mapsto x^n$, denn 
\[ \sqrt[n]{x^n}=x=(\sqrt[n]{x})^n \quad \text{für }x\geq 0. \]
Da die $n$-te Wurzelfunktion die reelle Umkehrfunktion ist, erhält man ihren Funktionsgraphen, indem man
den Graphen der Potenzfunktion $f:[0,\infty)\to \R, x\mapsto x^n$, an der Winkelhalbierenden $y=x$
spiegelt.
}
\lang{en}{
The $n$th root function is a \emph{inverse function} to the (restricted) $n$th power function 
$f:[0,\infty)\to \R, x\mapsto x^n$, as 
\[ \sqrt[n]{x^n}=x=(\sqrt[n]{x})^n \quad \text{for }x\geq 0. \]
As the $n$th root function is the inverse of the $n$th power function, we can obtain its graph by 
simply reflecting the graph of the (restricted) power function $f:[0,\infty)\to \R, x\mapsto x^n$ 
in the diagonal line $y=x$.
}
\item \lang{de}{
Für ungerade $n$, z.B. $n=3$ oder $n=5$, ist aber die Funktion $g:\R\to\R$, $x\mapsto x^n$, sogar als Ganzes umkehrbar. 
Die Umkehrfunktion wird gegeben durch $h:\R\to\R$,
}
\lang{en}{
For odd $n$, e.g. $n=3$ or $n=5$, the function $g:\R\to\R$, $x\mapsto x^n$ is even invertible on its 
whole domain. 
The inverse function is given by $h:\R\to\R$,
}
$
x\mapsto \begin{cases}\sqrt[n]{x}& x\geq 0,\\
                        -\sqrt[n]{-x}& x<0
          \end{cases}
$.
\end{enumerate}
\end{remark}

\lang{de}{
	\begin{genericGWTVisualization}[550][1000]{mathlet1}
		\begin{variables}
% 			\number[editable]{n}{integer}{3}
%			\function{f}{real}{sqrt(x)}
% 			\function{p1}{real}{x}
% 			\function{p2}{real}{x^2}
% 			\function{p3}{real}{x^3}
% 			\function{p4}{real}{x^4}			
			\parametricFunction{p1}{real}{t, t, 0.01, 3, 1000}
			\parametricFunction{p2}{real}{t, t^2, 0.01, 3, 1000}
			\parametricFunction{p3}{real}{t, t^3, 0.01, 3, 1000}
			\parametricFunction{p4}{real}{t, t^4, 0.01, 3, 1000}
			\parametricFunction{f2}{real}{t^2, t, 0.01, 3, 1000}
			\parametricFunction{f3}{real}{t^3, t, 0.01, 3, 1000}
			\parametricFunction{f4}{real}{t^4, t, 0.01, 3, 1000}
		\end{variables}
		\color[0.5]{p1}{GRAY}
 		\color[0.1]{p2}{#0066CC}
 		\color[0.1]{p3}{#00CC00}
  		\color[0.1]{p4}{#CC6600}
		\color{f2}{#0066CC}
		\color{f3}{#00CC00}
		\color{f4}{#CC6600}
% 		\label{p2}{$\textcolor{BLUE}{x^2}$}
% 		\label{p3}{$\textcolor{BLUE}{x^3}$}
% 		\label{p4}{$\textcolor{BLUE}{x^4}$}
		\begin{canvas}
			\plotSize{450}
			\plotLeft{-1}
			\plotRight{5}
			\plot[coordinateSystem]{p1,p2,p3,p4,f2,f3,f4}
		\end{canvas}
		\text{Hier sehen Sie die Graphen der
		Wurzelfunktionen $\textcolor{#0066CC}{f_2(x)=\sqrt{x}}$,
		$\textcolor{#00CC00}{f_3(x)=\sqrt[3]{x}}$ und $\textcolor{#CC6600}{f_4(x)=\sqrt[4]{x}}$, sowie die 
		zugehörenden Potenzfunktionen $\textcolor{#0066CC}{x^2}$, $\textcolor{#00CC00}{x^3}$ und 
		$\textcolor{#CC6600}{x^4}$ im Bereich $x\geq 0$.}
	\end{genericGWTVisualization}
}
\lang{en}{
	\begin{genericGWTVisualization}[550][1000]{mathlet1}
		\begin{variables}
% 			\number[editable]{n}{integer}{3}
%			\function{f}{real}{sqrt(x)}
% 			\function{p1}{real}{x}
% 			\function{p2}{real}{x^2}
% 			\function{p3}{real}{x^3}
% 			\function{p4}{real}{x^4}			
			\parametricFunction{p1}{real}{t, t, 0.01, 3, 1000}
			\parametricFunction{p2}{real}{t, t^2, 0.01, 3, 1000}
			\parametricFunction{p3}{real}{t, t^3, 0.01, 3, 1000}
			\parametricFunction{p4}{real}{t, t^4, 0.01, 3, 1000}
			\parametricFunction{f2}{real}{t^2, t, 0.01, 3, 1000}
			\parametricFunction{f3}{real}{t^3, t, 0.01, 3, 1000}
			\parametricFunction{f4}{real}{t^4, t, 0.01, 3, 1000}
		\end{variables}
		\color[0.5]{p1}{GRAY}
 		\color[0.1]{p2}{#0066CC}
 		\color[0.1]{p3}{#00CC00}
  		\color[0.1]{p4}{#CC6600}
		\color{f2}{#0066CC}
		\color{f3}{#00CC00}
		\color{f4}{#CC6600}
% 		\label{p2}{$\textcolor{BLUE}{x^2}$}
% 		\label{p3}{$\textcolor{BLUE}{x^3}$}
% 		\label{p4}{$\textcolor{BLUE}{x^4}$}
		\begin{canvas}
			\plotSize{450}
			\plotLeft{-1}
			\plotRight{5}
			\plot[coordinateSystem]{p1,p2,p3,p4,f2,f3,f4}
		\end{canvas}
		\text{Here we see the graphs of the $n$th root functions for $n=2,3,4$, 
    $\textcolor{#0066CC}{f_2(x)=\sqrt{x}}$, $\textcolor{#00CC00}{f_3(x)=\sqrt[3]{x}}$ and 
    $\textcolor{#CC6600}{f_4(x)=\sqrt[4]{x}}$, and the corresponding $n$th power functions 
    $\textcolor{#0066CC}{x^2}$, $\textcolor{#00CC00}{x^3}$ and 
		$\textcolor{#CC6600}{x^4}$ with domain $x \geq 0$.}
	\end{genericGWTVisualization}
}







\begin{quickcheck}
		\field{rational}
		\type{input.number}
		\begin{variables}
			\randint[Z]{b}{1}{5}
		    \function[calculate]{a2}{b^2}
		    \function[calculate]{a}{-b}
		\end{variables}

			\text{\lang{de}{Der maximale Definitionsbereich der Funktion $f(x)=\sqrt{\var{a2}-x^2}$ ist }
            \lang{en}{The maximal domain of the function $f(x)=\sqrt{\var{a2}-x^2}$ is } \\
			$D_f=\{ x\in \R \,|\, $\ansref$\leq x\leq $\ansref $\}$.}

		\begin{answer}
			\solution{a}
		\end{answer}
		\begin{answer}
			\solution{b}
		\end{answer}
		\explanation{\lang{de}{
    Der maximale Definitionsbereich besteht aus den Werten von $x$, für die der Ausdruck
		definiert ist, für die also $\var{a2}-x^2\geq 0$ gilt.
    }
    \lang{en}{
    The maximal domain consists of those values of $x$ for which the function is defined, so those 
    for which $\var{a2}-x^2\geq 0$.
    }}
	\end{quickcheck}
\begin{quickcheck}
		\field{rational}
		\type{input.number}
		\begin{variables}
			\randint[Z]{a}{1}{5}
			\randint[Z]{b}{1}{5}
		    \function[calculate]{na}{-a}
		\end{variables}

			\text{\lang{de}{Der maximale Definitionsbereich der Funktion 
            $f(x)=\sqrt{x+\var{a}}-\sqrt{\var{b}-x}$ ist }
            \lang{en}{The maximal domain of the function 
            $f(x)=\sqrt{x+\var{a}}-\sqrt{\var{b}-x}$ is }\\
			$D_f=\{ x\in \R \,|\, $\ansref$\leq x\leq $\ansref $\}$.}

		\begin{answer}
			\solution{na}
		\end{answer}
		\begin{answer}
			\solution{b}
		\end{answer}
		\explanation{\lang{de}{
    Der maximale Definitionsbereich besteht aus den Werten von $x$, für die der gesamte Ausdruck
		definiert ist, für die also $x+\var{a}\geq 0$ ist und $\var{b}-x\geq 0$ ist.
    }
    \lang{en}{
    The maximal domain consists of those values of $x$ for which both terms of the function are 
    defined, so those for which $x+\var{a}\geq 0$ and $\var{b}-x\geq 0$.
    }}
\end{quickcheck}


\end{visualizationwrapper}

\end{content}