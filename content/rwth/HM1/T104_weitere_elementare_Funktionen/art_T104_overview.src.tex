
%$Id:  $
\documentclass{mumie.article}
%$Id$
\begin{metainfo}
  \name{
    \lang{de}{Überblick: Weitere elementare Funktionen}
    \lang{en}{Overview: Further elementary functions}
  }
  \begin{description} 
 This work is licensed under the Creative Commons License Attribution 4.0 International (CC-BY 4.0)   
 https://creativecommons.org/licenses/by/4.0/legalcode 

    \lang{de}{Beschreibung}
    \lang{en}{Description}
  \end{description}
  \begin{components}
  \end{components}
  \begin{links}
\link{generic_article}{content/rwth/HM1/T104_weitere_elementare_Funktionen/g_art_content_16_logarithmen.meta.xml}{content_16_logarithmen}
\link{generic_article}{content/rwth/HM1/T104_weitere_elementare_Funktionen/g_art_content_15_exponentialfunktionen.meta.xml}{content_15_exponentialfunktionen}
\link{generic_article}{content/rwth/HM1/T104_weitere_elementare_Funktionen/g_art_content_14_potenzregeln.meta.xml}{content_14_potenzregeln}
\link{generic_article}{content/rwth/HM1/T104_weitere_elementare_Funktionen/g_art_content_13_wurzelfunktionen.meta.xml}{content_13_wurzelfunktionen}
\end{links}
  \creategeneric
\end{metainfo}
\begin{content}
\begin{block}[annotation]
	Im Ticket-System: \href{https://team.mumie.net/issues/30145}{Ticket 30145}
\end{block}


\begin{block}[annotation]
Im Entstehen: Überblicksseite für Kapitel Weitere elementare Funktionen
\end{block}

\usepackage{mumie.ombplus}
\ombchapter{1}
\lang{de}{\title{Überblick: Weitere elementare Funktionen}}
\lang{en}{\title{Overview: Further elementary functions}}



\begin{block}[info-box]
\lang{de}{\strong{Inhalt}}
\lang{en}{\strong{Contents}}


\lang{de}{
    \begin{enumerate}%[arabic chapter-overview]
   \item[4.1] \link{content_13_wurzelfunktionen}{Wurzelfunktionen}
   \item[4.2] \link{content_14_potenzregeln}{Potenzgesetze}
   \item[4.3] \link{content_15_exponentialfunktionen}{Exponentialfunktionen}
   \item[4.4] \link{content_16_logarithmen}{Logarithmen}
     \end{enumerate}
}
\lang{en}{
    \begin{enumerate}%[arabic chapter-overview]
   \item[4.1] \link{content_13_wurzelfunktionen}{$n$th roots}
   \item[4.2] \link{content_14_potenzregeln}{Power laws}
   \item[4.3] \link{content_15_exponentialfunktionen}{Exponential functions}
   \item[4.4] \link{content_16_logarithmen}{Logarithms}
     \end{enumerate}
}%lang

\end{block}

\begin{zusammenfassung}

\lang{de}{
Dieses Kapitel versammelt bekannte und wichtige Funktionen und stellt deren Eigenschaften vor.
Diese elementaren Funktionen bilden (zusammen mit den Polynomen, rationalen Funktionen und trigonometrischen Funktionen) unseren Hauptvorrat an elementaren Funktionen.
Durch Zusammensetzungen dieser Funktionen gelangen wir zu beliebig vielen weiteren.
\\
Die $n$-ten Wurzeln sind im Wesentlichen die Umkehrfunktionen des Potenzierens, also der bei Potenzfunktionen $x^n$. Sie führen zu den Potenzgesetzen mit rationalen Koeffizienten für positive Basen.
\\
Während bei Potenzfunktionen $x^n$ die Funktionsvariable $x$ ist, lassen Exponentialfunktionen $a^x$ die Variable im Exponenten zu.
Exponentialfunktionen zu verschiedenen Basen haben universelle Eigenschaften, unter anderem ihren Wertebereich,  
ihr Monotonieverhalten und, aus den Potenzgesetzen folgend, ihre Funktionalgleichung.
\\
Die Umkehrfunktionen der Exponentialfunktionen sind die Logarithmusfunktionen. 
Auch ihre Eigenschaften werden hier dargestellt und denen der Exponentialfunktionen gegenübergestellt.
Daraus ergeben sich wichtige Rechenregeln für Logarithmen.
}
\lang{en}{
This chapter introduces some commonly used and important functions and their properties. These 
elementary functions (alongside polynomials, rational functions and trigonometric functions) can 
also be combined to create arbitrarily many new and interesting functions
\\
The $n$th roots are in essence the inverse functions of $f(x)=x^n$. Having introduced these, it 
makes sense to define powers with rational exponents and their associated power laws (for positive 
bases).
\\
The variable $x^n$ is the base in a power function $x^n$. Exponential functions on the other hand 
have the variable as the exponent. Exponential functions have useful properties such as 
monotonicity and the rules for calculating with them are a useful application of the power laws.
\\
The inverse function of an exponential function is a logarithmic function. 
We also introduce the properties of logarithms and how they compare to exponential functions. As 
with exponents, we derive rules for calculating with logarithms.
}

\end{zusammenfassung}

\begin{block}[info]\lang{de}{\strong{Lernziele}}
\lang{en}{\strong{Learning Goals}} 
\begin{itemize}[square]
\item \lang{de}{Sie kennen Definitions- und Wertebereich der $n$-ten Wurzelfunktion.}
      \lang{en}{Knowing the definition, domain and image of the $n$th root function.}
\item \lang{de}{
      Sie kennen die Voraussetzungen für die Gültigkeit der verschiedenen Potenzgesetze und wenden 
      diese sicher an.
      }
      \lang{en}{
      Knowing the prerequisites for the different power laws and being able to apply these correctly.
      }
\item \lang{de}{
      Sie kennen die Eigenschaften von Exponential- und Logarithmusfunktionen und deren Bezüge 
      untereinander.
      }
      \lang{en}{
      Understanding the properties of exponential and logarithmic functions and the relationship 
      between them.
      }
\item \lang{de}{
      Sie drücken jede Exponentialfunktion durch die natürliche aus. Ebenso stellen Sie jede 
      Logarithmusfunktion durch den natürlichen Logarithmus dar.
      }
      \lang{en}{
      Being able to express every exponential function in terms of the natural exponential function. 
      Likewise, being able to express every logarithmic function in terms of the natural logarithm.
      }
\item \lang{de}{
      Sie vereinfachen und berechnen auch komplizierte Terme mit Hilfe der Logarithmus- und 
      Exponentialgesetze.
      }
      \lang{en}{
      Being able to simplify and compute even complex expressions using the rules for calculating 
      with exponents and logarithms.
      }
\end{itemize}
\end{block}




\end{content}
