%$Id:  $
\documentclass{mumie.article}
%$Id$
\begin{metainfo}
  \name{
    \lang{de}{Exponentialfunktionen}
    \lang{en}{Exponential functions}
  }
  \begin{description} 
 This work is licensed under the Creative Commons License Attribution 4.0 International (CC-BY 4.0)   
 https://creativecommons.org/licenses/by/4.0/legalcode 

    \lang{de}{Beschreibung}
    \lang{en}{Description}
  \end{description}
  \begin{components}
    \component{generic_image}{content/rwth/HM1/images/g_tkz_T104_Exponentials_B.meta.xml}{T104_Exponentials_B}
    \component{generic_image}{content/rwth/HM1/images/g_tkz_T104_Exponentials_A.meta.xml}{T104_Exponentials_A}
    \component{generic_image}{content/rwth/HM1/images/g_img_00_Videobutton_schwarz.meta.xml}{00_Videobutton_schwarz}
    \component{generic_image}{content/rwth/HM1/images/g_img_00_video_button_schwarz-blau.meta.xml}{00_video_button_schwarz-blau}
  \end{components}
  \begin{links}
    \link{generic_article}{content/rwth/HM1/T104_weitere_elementare_Funktionen/g_art_content_15_exponentialfunktionen.meta.xml}{content_15_exponentialfunktionen}
    \link{generic_article}{content/rwth/HM1/T104_weitere_elementare_Funktionen/g_art_content_14_potenzregeln.meta.xml}{powers-roots}
    \link{generic_article}{content/rwth/HM1/T104_weitere_elementare_Funktionen/g_art_content_16_logarithmen.meta.xml}{link2}
    \link{generic_article}{content/rwth/HM1/T106_Differentialrechnung/g_art_content_20_ableitung_als_tangentensteigung.meta.xml}{link-diff}
    \link{generic_article}{content/rwth/HM1/T107_Integralrechnung/g_art_content_25_stammfunktion.meta.xml}{link-int}
  \end{links}
  \creategeneric
\end{metainfo}
\begin{content}
\usepackage{mumie.ombplus}
\ombchapter{4}
\ombarticle{3}
\usepackage{mumie.genericvisualization}

\begin{visualizationwrapper}

\lang{de}{\title{Exponentialfunktionen}}
\lang{en}{\title{Exponential functions}}

\begin{block}[annotation]
 $a^x$, Definition, Funktionsgraphen, Definitions- und Wertebereich, Umkehrbarkeit. $e$
\end{block}

\begin{block}[annotation]
  Im Ticket-System: \href{http://team.mumie.net/issues/9011}{Ticket 9011}\\
\end{block}

\begin{block}[info-box]
\tableofcontents
\end{block}

\section{\lang{de}{Einf"uhrung in Exponentialfunktionen}
         \lang{en}{Introduction to exponential functions}}\label{sec:exp_fkt}
\lang{de}{
Die Funktionen $2^x, 10^x, e^x$ (wobei $e$ die \lref{eulernum}{Eulersche Zahl} ist) sind Beispiele 
von Exponentialfunktionen, also Funktionen, bei denen die Variable im Exponenten steht.
}
\lang{en}{
The functions $2^x, 10^x, e^x$ (where $e$ is \lref{eulernum}{Euler's number}) are examples of 
'exponential functions', whose variable is found in the exponent.
}
\begin{definition}[\lang{de}{Exponentialfunktion}\lang{en}{Exponential function}]
   \lang{de}{Die Funktionen}
   \lang{en}{Functions of the form} 
     \begin{equation*}
    f(x) = a^x, \quad a > 0
      \end{equation*}
    \lang{de}{
    heißen \notion{Exponentialfunktionen}. Sie sind für alle reellen Zahlen $x$ definiert: 
    $D_f = \mathbb{R}$. Die Zahl $a>0$ heißt \emph{Basis}.\\
    \floatright{\href{https://www.hm-kompakt.de/video?watch=141}{\image[75]{00_Videobutton_schwarz}}}\\~
    }
    \lang{en}{
    are called \emph{exponential functions}. They are defined for all real numbers: 
    $D_f = \mathbb{R}$. The number $a>0$ is called the \emph{base}.
    } 
\end{definition}
\lang{de}{
Für \emph{rationale Exponenten} $x$ ist $a^x$ bereits im Abschnitt 
\ref[powers-roots][Potenzgesetze]{potenz_rational} eingeführt worden. Um $a^x$ für beliebige reelle 
Zahlen ordentlich zu definieren, sind Konzepte nötig (z.B. die Stetigkeit), die erst im nächsten Teil 
dieses Kurses behandelt werden. Wir begnügen uns deshalb hier mit der folgenden Beobachtung: Kann 
eine reelle Zahl $x$ beliebig gut durch eine rationale Zahl $\frac{m}{n}$ angenähert werden, dann ist 
auch $a^{\frac{m}{n}}$ eine beliebig gute Näherung für $a^x$. Dies genügt für alle praktischen Zwecke.
}
\lang{en}{
We have introduced $a^x$ for \emph{rational exponents} $x$ in the section on 
\ref[powers-roots][general exponents]{potenz_rational}. To define $a^x$ for any real $x$, we require 
concepts (such as continuity) that are only covered in the second part of this course. Therefore we 
content ourselves here with making the following observation: if a real number $x$ can be 
approximated arbitrarily well by a rational number $\frac{m}{n}$, then $a^{\frac{m}{n}}$ can be an 
arbitrarily good approximation for $a^x$. This suffices for our practical needs.
}

\begin{theorem}
\lang{de}{
Die Wertemenge jeder Exponentialfunktion (mit Ausnahme von  $a = 1$) ist die Menge 
$\R_{>0}$ der positiven reellen Zahlen, wie die Grafiken im untenstehenden Beispiel nahelegen. \\
Im  Fall $a=1$ ist $f(x)=1$ für alle $x$, die Wertemenge ist nun also $\{1\}$.
}\label{uninteresting_de}
\lang{en}{
The image of every exponential function with $x\in\R$ (except when $a = 1$) is the set of positive 
real numbers $\R^+$; this can be seen in the plot below. 
The uninteresting case is when $a=1$, which implies $f(x)=1$ for all $x$.
}\label{uninteresting_en}
\end{theorem}
%Da $a>0$ ist und entsprechend $f(x) = a^x$ immer positiv ist, sind die Werte jeder Exponentialfunktion positiv.
%Zusätzlich gilt, dass 

\lang{de}{
Alle Exponentialfunktionen haben die Eigenschaft $f(0)=a^0=1$. Der Verlauf der Exponentialfunktionen
ist von der Basis $a$ abhängig und kann in zwei Fälle unterteilt werden, $0 < a < 1$ und $a > 1$. Im nächsten Beispiel 
ist der Zusammenhang von Basiswert und Verlauf der Exponentialfunktionen ersichtlich.
}
\lang{en}{
All exponential functions have the property $f(0)=a^0=1$. The gradient of the exponential function depends on the basis $a$, and can be divided into two cases:
$0 < a < 1$ and $a > 1$. In the following example we'll see the relationship between the base and gradient of an exponential function more clearly.
}

\begin{example}\label{ex:exp-examples_A}
\lang{de}{
  \begin{table}[\cellaligns{cccccccc}]
\head
  $x$ & $\quad-3 \quad$ &$\quad -2  \quad$& $\quad -1 \quad$ &$\quad 0 \quad$ & $\quad 1 \quad$& $\quad 2 \quad$ &$\quad 3 \quad$ 
  \body
  $2^x$ & $\frac{1}{8}$ & $\frac{1}{4}$ & $\frac{1}{2}$& $1$ & $2$& $4$ & $8$\\
  $\left( \frac{1}{2} \right)^x = 2^{-x}$& $8$ & $4$ & $2$ & $1$ & $\frac{1}{2}$ & $\frac{1}{4}$ & $\frac{1}{8}$ \\
  $3^x$ & $\frac{1}{27}$ & $\frac{1}{9}$ & $\frac{1}{3}$& $1$ & $3$& $9$ & $27$\\
  $\left( \frac{1}{3} \right)^x = 3^{-x}$& $27$ & $9$ & $3$ & $1$ & $\frac{1}{3}$ & $\frac{1}{9}$ & $\frac{1}{27}$\\
   $10^x$& $0,001$& $0,01$ &$0,1$& $1$ & $10$& $100$ & $1000$\\
  $\left( \frac{1}{10} \right)^x = 10^{-x}$ & $1000$ & $100$ & $10$ & $1$ &$0,1$ & $0,01$ & $0,001$
%   \lang{en}{\head
%   $x$ & $\quad-3 \quad$ &$\quad -2  \quad$& $\quad -1 \quad$ &$\quad 0 \quad$ & $\quad 1 \quad$& $\quad 2 \quad$ &$\quad 3 \quad$ 
%   \body
%   $2^x$ & $\frac{1}{8}$ & $\frac{1}{4}$ & $\frac{1}{2}$& $1$ & $2$& $4$ & $8$\\
%   $\left( \frac{1}{2} \right)^x = 2^{-x}$& $8$ & $4$ & $2$ & $1$ & $\frac{1}{2}$ & $\frac{1}{4}$ & $\frac{1}{8}$ \\
%   $3^x$ & $\frac{1}{27}$ & $\frac{1}{9}$ & $\frac{1}{3}$& $1$ & $3$& $9$ & $27$\\
%   $\left( \frac{1}{3} \right)^x = 3^{-x}$& $27$ & $9$ & $3$ & $1$ & $\frac{1}{3}$ & $\frac{1}{9}$ & $\frac{1}{27}$\\
%    $10^x$& $0.001$& $0.01$ &$0.1$& $1$ & $10$& $100$ & $1000$\\
%   $\left( \frac{1}{10} \right)^x = 10^{-x}$ & $1000$ & $100$ & $10$ & $1$ &$0.1$ & $0.01$ & $0.001$
%   }
  \end{table}
}
\lang{en}{
  \begin{table}[\cellaligns{cccccccc}]
\head
  $x$ & $\quad-3 \quad$ &$\quad -2  \quad$& $\quad -1 \quad$ &$\quad 0 \quad$ & $\quad 1 \quad$& $\quad 2 \quad$ &$\quad 3 \quad$ 
  \body
  $2^x$ & $\frac{1}{8}$ & $\frac{1}{4}$ & $\frac{1}{2}$& $1$ & $2$& $4$ & $8$\\
  $\left( \frac{1}{2} \right)^x = 2^{-x}$& $8$ & $4$ & $2$ & $1$ & $\frac{1}{2}$ & $\frac{1}{4}$ & $\frac{1}{8}$ \\
  $3^x$ & $\frac{1}{27}$ & $\frac{1}{9}$ & $\frac{1}{3}$& $1$ & $3$& $9$ & $27$\\
  $\left( \frac{1}{3} \right)^x = 3^{-x}$& $27$ & $9$ & $3$ & $1$ & $\frac{1}{3}$ & $\frac{1}{9}$ & $\frac{1}{27}$\\
   $10^x$& $0.001$& $0.01$ &$0,1$& $1$ & $10$& $100$ & $1000$\\
  $\left( \frac{1}{10} \right)^x = 10^{-x}$ & $1000$ & $100$ & $10$ & $1$ &$0.1$ & $0.01$ & $0.001$
%   \lang{en}{\head
%   $x$ & $\quad-3 \quad$ &$\quad -2  \quad$& $\quad -1 \quad$ &$\quad 0 \quad$ & $\quad 1 \quad$& $\quad 2 \quad$ &$\quad 3 \quad$ 
%   \body
%   $2^x$ & $\frac{1}{8}$ & $\frac{1}{4}$ & $\frac{1}{2}$& $1$ & $2$& $4$ & $8$\\
%   $\left( \frac{1}{2} \right)^x = 2^{-x}$& $8$ & $4$ & $2$ & $1$ & $\frac{1}{2}$ & $\frac{1}{4}$ & $\frac{1}{8}$ \\
%   $3^x$ & $\frac{1}{27}$ & $\frac{1}{9}$ & $\frac{1}{3}$& $1$ & $3$& $9$ & $27$\\
%   $\left( \frac{1}{3} \right)^x = 3^{-x}$& $27$ & $9$ & $3$ & $1$ & $\frac{1}{3}$ & $\frac{1}{9}$ & $\frac{1}{27}$\\
%    $10^x$& $0.001$& $0.01$ &$0.1$& $1$ & $10$& $100$ & $1000$\\
%   $\left( \frac{1}{10} \right)^x = 10^{-x}$ & $1000$ & $100$ & $10$ & $1$ &$0.1$ & $0.01$ & $0.001$
%   }
  \end{table}
}

 \begin{center}
\image{T104_Exponentials_A}
  \end{center}

\end{example}
\lang{de}{
Wir fassen einige wichtige Eigenschaften von Exponentialfunktionen zusammen. 
(Einige Begriffe wie die Monotonie werden an späterer Stelle ausführlich besprochen, sind Ihnen aber wahrscheinlich aus der Schule bekannt.)
}
\lang{en}{
We summarise some important properties of exponential functions. 
Certain definitions like monotone functions are only introduced later, but are included for 
completeness, and some students may know these already.
}

\begin{example}\label{ex:eigenschaften_exp}
   \begin{tabs*}[\initialtab{1}\class{example}]
 \tab{$f(x) = a^x, \; a > 1$ }
   \begin{table}
  \head
  $f(x) = a^x$ & $a > 1, \;\; a \in \mathbb{R}$ 
  \body
  $D_f = \mathbb{R}$ & 
    \lang{de}{Definitionsbereich ist die Menge aller reellen Zahlen.}
    \lang{en}{The domain is the set of all real numbers.} \\
  $W_f = \mathbb{R}^+$  & 
    \lang{de}{Wertemenge ist die Menge der positiven reellen Zahlen.}
    \lang{en}{The image is the set of all positive real numbers.} \\
  $f(x) \neq 0$& 
    \lang{de}{Die Funktionen haben keine Nullstellen.}
    \lang{en}{The functions have no zeros.}\\
  $f(x_1) < f(x_2)$ \lang{de}{für}\lang{en}{for} $x_1 < x_2$&  
    \lang{de}{Die Funktionen sind streng monoton wachsend.}
    \lang{en}{The functions are strictly monotonically increasing.}\\
  \lang{de}{Die Steigung von $f$ nimmt ständig zu.}
  \lang{en}{The slope is increasing}  & 
    \lang{de}{Der Graph beschreibt eine Linkskurve.}
    \lang{en}{The graph is concave upward.} \\
  $\lim_{x \to -\infty}f(x) = 0, \lim_{x \to \infty}f(x) = \infty$& 
    \lang{de}{Die Funktionen streben von $0$ nach $\infty$.}
    \lang{en}{The functions tend towards $0$ as $x$ decreases, and to $\infty$ as $x$ increases.}\\
  \end{table}
 \tab{$f(x) = a^{x}, \; 0 < a < 1$}
\begin{table}
  \head
  $f(x) = a^{x} = \left(\frac{1}{a}\right)^{-x}$ & $ 0 < a < 1, \;\; a \in \mathbb{R}$ 
  \body
  $D_f = \mathbb{R}$ & 
    \lang{de}{Definitionsbereich ist die Menge aller reellen Zahlen.}
    \lang{en}{The domain is the set of all real numbers.} \\
  $W_f = \mathbb{R}^+$ & 
    \lang{de}{Wertemenge ist die Menge der positiven reellen Zahlen.}
    \lang{en}{The image is the set of all positive real numbers.} \\
  $f(x) \neq 0$& 
    \lang{de}{Die Funktionen haben keine Nullstellen.}
    \lang{en}{The functions have no zeros.}\\
  $f(x_1) > f(x_2)$ \lang{de}{für}\lang{en}{for} $x_1 < x_2$& 
    \lang{de}{Die Funktionen sind streng monoton fallend.}
    \lang{en}{The functions are strictly monotonically decreasing.}\\
  \lang{de}{Die Steigung von $f$ nimmt ständig zu.}
  \lang{en}{The slope is increasing} & 
    \lang{de}{Der Graph beschreibt eine Linkskurve.}
    \lang{en}{The graph is concave upward.}  \\
  $\lim_{x \to -\infty}f(x) = \infty, \lim_{x \to \infty}f(x) = 0$& 
    \lang{de}{Die Funktionen streben von $\infty$ nach $0$.}
    \lang{en}{The functions tend towards $\infty$ as $x$ decreases, and to $0$ as $x$ increases.}\\
  \end{table}
 \end{tabs*}
\end{example}

\begin{quickcheck}
		\field{rational}
		\type{input.number}
		\begin{variables}
			\randint{a1}{1}{4}
			\randint{a2}{1}{4}		
			\function[calculate]{a}{a1/a2} % a=1 ausschließen???
			\randint[Z]{xp}{-1}{1} % sollte -4 .. 4 sein
			\function[calculate]{yp}{a^xp}
		\end{variables}
		
			\text{\lang{de}{
            Für welches $a>0$ geht der Graph der Exponentialfunktion $f(x)=a^x$ durch den Punkt 
            $P=(\var{xp};\var{yp})$?\\
      			Für $a=$\ansref.
            }
            \lang{en}{
            For which $a>0$ does the graph of the exponential function $f(x)=a^x$ contain the point 
            $P=(\var{xp};\var{yp})$?\\
            For $a=$\ansref.
            }}
		
		\begin{answer}
			\solution{a}
		\end{answer}
		\explanation{\lang{de}{
                 Wenn der Punkt $P=(\var{xp};\var{yp})$ auf dem Graphen liegt, gilt: 
                 $f(\var{xp})=\var{yp}$, also $a^\var{xp}=\var{yp}$.
                 }
                 \lang{en}{
                 If the point $P=(\var{xp};\var{yp})$ lies on the graph, we have 
                 $f(\var{xp})=\var{yp}$, so $a^\var{xp}=\var{yp}$.
                 }}
	\end{quickcheck}


\lang{de}{
Aus den \ref[powers-roots][Potenzgesetzen]{Potenzgesetze} folgt eine wichtige Eigenschaft
jeder Exponentialfunktion: Die Addition der Argumente  führt zur Multiplikation der
Funktionswerte.  Exponentialfunktionen verwandeln also Addition in Multiplikation!
}
\lang{en}{
Every exponential function has the property that the image of the sum is equal to the product of the 
images of two numbers. In this way the exponential function creates a connection between addition and 
multiplication.
}

\begin{theorem}[\lang{de}{Potenzgesetz für Exponentialfunktionen}
                \lang{en}{Power Rules for Exponential Functions}]\label{functional_eqn_exp}
\lang{de}{Jede Exponentialfunktion $a^x$ mit $a>0$ erfüllt für alle $x_1,\,x_2\in\R$}
\lang{en}{For every exponential function $a^x$ with $a>0$ and $x_1,\,x_2\in\R$,}
\[
	a^{x_1 + x_2} = a^{x_1}\cdot a^{x_2}.
\]
\lang{de}{Insbesondere gilt}
\lang{en}{In particular,}
\begin{equation*} 
a^{-x} = \frac{1}{a^{x}}.
\end{equation*}
\end{theorem}
\lang{de}{
Dieser Satz folgt unmittelbar aus den \ref[powers-roots][Potenzgesetzen]{Potenzgesetze}, 
die nicht nur für rationale Potenzen sondern auch für alle reellen Potenzen gelten.
\\
Wird das Argument einer Exponentialfunktion um einen festen Wert $\Delta x$ erhöht (oder bei 
$\Delta x<0$ verringert), dann wird der Funktionswert mit dem festen Faktor $a^{\Delta x}$ 
multipliziert:
}
\lang{en}{
This follows directly from the previously established \ref[powers-roots][power laws]{Potenzgesetze}, 
and holds for real exponents as well as rational exponents.
\\
If the argument of an exponential function is increased by a fixed value $\Delta x$ (or decreased 
when $\Delta x<0$), then its image is multiplied by a fixed factor $a^{\Delta x}$:
}

\begin{equation*} f(x+\Delta x) = f(\Delta x)\cdot f(x) =  a^{\Delta x}\cdot f(x). \label{factor_shift}
\end{equation*}

\lang{de}{
Diese Eigenschaft ist wichtig für \lref{applications_exp}{Anwendungen der Exponentialfunktionen}
in der mathematischen Modellbildung.
}
\lang{en}{
This property is important for the \lref{applications_exp}{applications of exponential functions} in 
mathematical modelling.
}

\begin{quickcheck}
		\field{rational}
		\type{input.number}
		\begin{variables}
			\randint{a}{2}{5}
			\randint{k}{1}{3}
			\function[calculate]{c}{a^k}
		\end{variables}
		
			\text{\lang{de}{
      Wir betrachten die Exponentialfunktion $f(x)=\var{a}^{x}$. Bestimmen Sie den Faktor $c$ so, dass
			$f(x+\var{k})=c\cdot f(x)$ für alle reellen Zahlen $x$ gilt. \\
			Es ist $c=$\ansref.
      }
      \lang{en}{
      We consider the exponential function $f(x)=\var{a}^{x}$. The factor $c$ satisfying 
      $f(x+\var{k})=c\cdot f(x)$ for all real numbers $x$ is:\\
      $c=$\ansref.
      }}
		
		\begin{answer}
			\solution{c}
		\end{answer}
		\explanation{\lang{de}{Es gilt}\lang{en}{We have} 
    $f(x+\var{k})=\var{a}^{x+\var{k}}=\var{a}^\var{k}\cdot 
    \var{a}^x=\var{a}^\var{k}\cdot f(x)=\var{c}\cdot f(x)$.}
	\end{quickcheck}


% Die Umkehrung,
% dass die Multiplikation der Argumente zur Addition der Funktionswerte führt, ist eine charakteristische
% \ref[link2][Eigenschaft der Logarithmusfunktionen]{functional_eqn_log}.
% 
% Auch die Multiplikation und Potenzierung werden durch die Exponentialfunktion verbunden.
% Setzt man in der Regel $x=x_1=x_2$, so erhält man $f(2x)=f(x+x)=\left(f(x)\,\right)^2$. Das
% gilt auch für andere Faktoren als nur $c=2$:}
% \lang{en}{This property is important for \lref{applications_exp}{applications of the exponential function}
% in building up mathematical models. The converse, that multiplication of arguments leads to the addition of function values, is a characteristic
% \ref[link2][property of logarithms]{functional_eqn_log}.
% 
% Multiplication and exponentiation are also connected via the exponential function.
% If we use the above rule and let $x=x_1=x_2$, we get $f(2x)=f(x+x)=\left(f(x)\,\right)^2$. This is
% true for other factors, not only $c=2$:}
% \begin{rule}\label{power_exp_fcn}
% \lang{de}{Jede Exponentialfunktion $f(x)=a^x$ mit $a>0$ erfüllt für alle $x,\,c\in\R$}
% \lang{en}{For every exponential function $f(x)=a^x$ with $a>0$ and $x,\,c\in\R$:}
% \begin{equation*} 
% \lang{de}{f(c\cdot x) = \left(f(x)\,\right)^c= \left(a^c\right)^x.}
% \lang{en}{f(c\cdot x) = \left(f(x)\,\right)^c= \left(a^c\right)^x}
% \end{equation*}
% \end{rule}
% \lang{de}{Dass die Transformation $\,f(x)\longrightarrow f(cx)$ den Graphen horizontal streckt oder staucht und evtl. spiegelt, wird 
% im \ref[transformations][Abschnitt über Transformationen]{horizontale_Skalierung_exp} erläutert und visualisiert.}
% \lang{en}{The transformation $\,f(x)\longrightarrow f(cx)$ leads to a stretching or shrinking of the graph (or even a mirroring), and will be dealt with
% in further detail in the section on \ref[transformations][transformations]{horizontale_Skalierung_exp}.}

\section{\lang{de}{Die natürliche Exponentialfunktion}
         \lang{en}{The natural exponential function}}\label{sec:nat-exp-fct}
\begin{definition}[\lang{de}{natürliche Exponentialfunktion}
                   \lang{en}{Natural exponential function}]\label{eulernum}
   \lang{de}{Die Exponentialfunktion mit der Basis $e$ heißt \notion{natürliche Exponentialfunktion}}
   \lang{en}{The exponential function with base $e$ is called the \emph{natural exponential function}}  \\
     \begin{equation*}
    f(x) = e^x = \exp(x),
      \end{equation*}
      \lang{de}{wobei die \notion{Eulersche Zahl}}
      \lang{en}{where \emph{Euler's Number}}
     \begin{equation*} e \approx 2\lang{de}{,}\lang{en}{.}718281828459\end{equation*}
     \lang{de}{eine irrationale Zahl ist.}
     \lang{en}{is an irrational number.}
\end{definition}

\lang{de}{
  Die natürliche Exponentialfunktion ist eine der wichtigsten
  Funktionen in Mathematik und Naturwissenschaften. Wachstumsvorgänge
  und Zerfallsprozesse in der Natur werden durch sie dargestellt. Sie
  hat die ganz besondere Eigenschaft, dass sie mit ihrer Ableitung
  (siehe Abschnitt \link{link-diff}{Ableitung und Ableitungsformeln}) übereinstimmt, das heißt, dass $(e^x)' = e^x$
  für alle $x \in \mathbb{R}$ gilt. 
  Das vereinfacht viele Berechnungen und
  Umformungen zum Beispiel beim Differenzieren und
  Integrieren (siehe Abschnitt \link{link-int}{Stammfunktion}).
}
\lang{en}{
  The natural exponential function is one of the most important
  functions in mathematics and the natural sciences. Growth and decay
  in nature can be represented via the natural exponential function. The function has the
  special property that its derivative is the function itself (see the section on 
  \link{link-diff}{differentiation and rules of differentiation}), i.e. $(e^x)' = e^x$
  for all $x \in \mathbb{R}$. 
  This tends to simplify many calculations and transformations, 
  most notably when there is differentiation and integration involved 
  (see the section on \link{link-int}{integration}).
}
  
\begin{remark}
   \lang{de}{Jede Exponentialfunktion kann durch die natürliche Exponentialfunktion dargestellt werden:\\
   }
   \lang{en}{Every exponential function can be represented using the natural exponential function:\\
   }
     \begin{equation*}
    f(x) = a^x = e^{bx},
      \end{equation*}
   \lang{de}{
   wobei $a = e^b$. Für $a>1$ ist $b > 0$ und für $0 < a < 1 $ ist $b < 0$.
   }
   \lang{en}{
   where $a = e^b$. For $a>1$ we have $b>0$ and for $0<a<1$ we have $b<0$.\\
   For this reason, whenever we refer to \emph{the} exponential function, we almost always mean the 
   natural exponential function. \emph{An} exponential function can refer to an exponential function 
   with any base.
   }
\end{remark}
\lang{de}{
Der Parameter $b$ kann als $b=\ln(a)$ berechnet werden, was  im Abschnitt über 
\link[_blank]{link2}{Logarithmusfunktionen} behandelt wird.
}
\lang{en}{
We can actually write the parameter $b$ as $b=\ln(a)$, but this will be dealt with in 
the section on 
\link[_blank]{link2}{logarithms}.
}
\begin{example}
\begin{center}
\image{T104_Exponentials_B}
\end{center}
\lang{de}{
Hier sehen Sie die Graphen einiger Exponentialfunktionen. Vergleichen Sie mit Beispiel 
\ref{ex:exp-examples_A}: Auch optisch ist klar, dass z.B. die Funktion $2^x$ dort nicht mit der hier 
dargestellten Funktion $e^{2x}=(e^2)^x$ übereinstimmt. Erstere wächst deutlich langsamer, denn 
$2<\!<e^2\approx 7,39$.
}
\lang{en}{
Above are the graphs of several exponential functions. Even visually comparing then with example 
\ref{ex:exp-examples_A} quickly makes it clear that e.g. the function $2^x$ is distinctly 
different from $e^{2x}=(e^2)^x$. The prior grows substantially slower, as $2<\!<e^2\approx 7,39$.
}

\end{example}
 
\begin{quickcheck}
		\field{rational}
		\type{input.number}
		\begin{variables}
			\randrat[Z]{c}{1}{4}{1}{4}
			\randint[Z]{xp}{-4}{4}
			\function[calculate]{yp}{c*xp}
		\end{variables}
		
			\text{\lang{de}{
      Für welches $c$ geht der Graph der Exponentialfunktion $f(x)=e^{cx}$ durch den Punkt 
			$P=(\var{xp};e^\var{yp})$?\\
			Für $c=$\ansref.
      }
      \lang{en}{
      The graph of the exponential function $f(x)=e^{cx}$ contains the point 
      $P=(\var{xp};e^\var{yp})$ if \\
      $c=$\ansref.
      }}
		
		\begin{answer}
			\solution{c}
		\end{answer}
		\explanation{\lang{de}{
    Wenn der Punkt $P=(\var{xp};e^\var{yp})$ auf dem Graphen liegt, gilt: $f(\var{xp})=e^\var{yp}$, 
		also $e^{\var{xp}c}=e^\var{yp}$.
    }
    \lang{en}{
    If the point $P=(\var{xp};e^\var{yp})$ lies on the graph, we have $f(\var{xp})=e^\var{yp}$, so 
    $e^{\var{xp}c}=e^\var{yp}$.
    }}
	\end{quickcheck}

 \lang{de}{
  Nachfolgendes Video behandelt auch die Exponentialfunktionen und ergänzt obigen Text, dabei wird 
  auch auf die Stetigkeit und Differenzierbarkeit eingegangen, was erst später besprochen wird.
	 
 \floatright{\href{https://api.stream24.net/vod/getVideo.php?id=10962-2-10799&mode=iframe&speed=true}{\image[75]{00_video_button_schwarz-blau}}}\\
 }
 \lang{en}{
 \\
 }
 
\section{\lang{de}{Anwendungen der Exponentialfunktionen}
         \lang{en}{Applications of exponential functions}}\label{applications_exp}

\lang{de}{
Bei der Beschreibung von Prozessen in den Natur-, Ingenieur- und Wirtschaftswissenschaften 
durch mathematische Modelle ("`mathematische Modellbildung"') treten sehr oft Exponentialfunktionen 
auf, insbesondere, um den zeitlichen Verlauf wiederzugeben. Die unabhängige Variable wird dann meist 
mit $t$ für die Zeit (lateinisch tempus) bezeichnet. Die Maßeinheit (z.B. Sekunden $s$, Stunden $h$, 
Tage $d$ oder Jahre $a$) muss jeweils angegeben werden.
}
\lang{en}{
In describing processes in the natural, engineering, and economic sciences via mathematical
modelling we often encounter the exponential function, especially when dealing with growth over time. 
The independent variable is most often denoted by $t$ for time (originally from the Latin 
\emph{tempus}). The units (e.g. seconds (s), hours (h), days (d) or years (a)) always need to be 
specified.
}

\begin{enumerate}
\item \lang{de}{
In einer typischen Bakterienkultur verdoppelt sich die Anzahl der Bakterien etwa alle $20$ Minuten
(solange genügend Platz und Nährlösung vorhanden ist). In einer Stunde hat sich die Anzahl dreimal
verdoppelt, also verachtfacht. Wenn die Kultur am Anfang $N_0$ Bakterien hatte, und wenn die Zeit von 
da an in Stunden gemessen wird, dann hat die Kultur zu einer späteren Zeit $t>0$
}
\lang{en}{
In a typical bacterial culture, the number of bacteria doubles itself every $20$ minutes 
(as long as there's enough room and nutrition to sustain the growth).
In one hour, the number of bacteria doubles three times, that is, grows 8-fold. If the culture 
starts with $N_0$ bacteria and the time from then on is measured in hours, then the culture $t>0$ 
hours later has:
}
\begin{equation*}N(t) = N_0\cdot 8^t\;\text{\lang{de}{Bakterien, $\quad t\,$ in Stunden.}
                                            \lang{en}{bacteria, $\quad t\,$ in hours}}
\end{equation*}
\lang{de}{
Wenn z.B. am Anfang $N_0=7$ Bakterien in der Kultur waren, dann erwartet man nach einer Stunde
$N(1) = 7\cdot 8^1 =56$ Bakterien, nach $3$ Stunden
$N(3) = 7\cdot 8^3 = 3.584$ Bakterien und nach 6 Stunden $N(6) = 7\cdot 8^6 \approx 1,8$ Millionen 
Bakterien. Sie sehen daran die Grenzen eines einfachen Modells für exponentielles Wachstum. Die 
Modelle sind Näherungen und sie sind nur für ein begrenztes Zeitintervall eine gute Beschreibung. 
Wenn die Population zu groß wird, muss auch der begrenzte Platz oder Nahrungsvorrat berücksichtigt 
werden, der bei kleinen Populationen noch keine Rolle spielt.
}
\lang{en}{
For example, if at the beginning we have $N_0=7$ bacteria in the culture, then after an hour we 
expect there to be $N(1) = 7\cdot 8^1 =56$ bacteria, after $3$ hours $N(3) = 7\cdot 8^3 = 3,584$ 
bacteria, and after 6 hours $N(6) = 7\cdot 8^6 \approx 1.8$ million bacteria. Here we can see the 
limits of the simple model of exponential growth. The models are approximations, and are only valid 
for a limited time interval. If the population gets too large, the amount of space available and 
the amount of food come into play, which normally don't play a role with small populations.
}
\item \lang{de}{
Auch das Wachstum eines Kapitals mit festem Zinssatz gemäß der
Zinseszinsformel ist ein Beispiel für exponentielles Wachstum. Das verzinste Kapital lässt sich berechnen mit folgender Formel:
   \begin{equation*}
   K_{neu} = K_{0}  \cdot  (1 + \frac{p}{100} )^{h}
   \end{equation*}
Wobei $K_{0}$ für das Startkapital steht, $p$ für den Zinssatz und $h$ für die Laufzeit der 
Verzinsung.\\
Angenommen wir legen 1000\euro{} für 15 Jahre zu 5,0 \% Zinsen an. Eingesetzt in die Formel erhalten 
wir
$ K_{neu} = 1000 \cdot (1 +  \frac{5}{100} )^{15} \approx 2079$\euro{}
}
\lang{en}{
The growth of capital with a fixed interest rate according to the interest formula is an example of 
exponential growth, modelled by the formula
   \begin{equation*}
   K_{new} = K_{0}  \cdot  (1 + \frac{p}{100} )^{h}
   \end{equation*}
where $K_{0}$ is the initial capitak, $p$ is the interest rate and $h$ is the time variable. \\
If we deposited $1000$\euro{} and waited $15$ years at a $5.0$\% interest rate, by the formula we 
would have
$ K_{new} = 1000 \cdot (1 +  \frac{5}{100} )^{15} \approx 2079$\euro{}
}

\item \lang{de}{
Ein aufgeladener elektrischer Kondensator mit Anfangsspannung $U_0 = 20\,\text{V}$ wird 
über einen Widerstand entladen. Nach einer Sekunde
hat er die Spannung $18\,\text{V} = (\frac{9}{10})\cdot 20\,\text{V}$. 
Dann wird der zeitliche Verlauf der Spannung (in Volt) durch
    \begin{equation*}
    U(t) = U_0\cdot \left(\frac{9}{10}\right)^t,\qquad t\;\text{ in Sekunden (s)}
    \end{equation*}
beschrieben. Nach $2,5$ Sekunden  ist die Spannung auf 
$U(2,5)=20\,\text{V}\cdot (\frac{9}{10})^{2,5} \approx 15,4\,\text{V}$ \\
abgefallen, nach $10\,\text{s}$ auf 
$U(10)=20\,\text{V}\cdot (\frac{9}{10})^{10} \approx 7\,\text{V}$ \\
und nach einer Minute auf 
$U(60)=20\,\text{V}\cdot (\frac{9}{10})^{60} \approx 0,04\,\text{V}$.
}
\lang{en}{
A charged capacitor with an initial voltage of $U_0 = 20\,\text{V}$ is released through a resistor. After one second, the voltage drops to 
$18\,\text{V} = (\frac{9}{10})\cdot 20\,\text{V}$. The voltage can then be given (in Volts) via the following equation:
    \begin{equation*}
    U(t) = U_0\cdot \left(\frac{9}{10}\right)^t,\qquad t\;\text{ in seconds (s)}
    \end{equation*}
After $2.5$ seconds the voltage has dropped to 
$U(2.5)=20\,\text{V}\cdot (\frac{9}{10})^{2,5} \approx 15.4\,\text{V}$, \\
after $10$ seconds it has dropped to 
$U(10)=20\,\text{V}\cdot (\frac{9}{10})^{10} \approx 7\,\text{V}$ \\
and after a minute it has dropped to 
$U(60)=20\,\text{V}\cdot (\frac{9}{10})^{60} \approx 0.04\,\text{V}$.
}
\item \lang{de}{
Ein anderes Beispiel für exponentielles Abklingen ist der radioaktive Zerfall. 
Eine radioaktive Substanz zerfällt nach dem Zerfallsgesetz
\[
n(t) = n_0 \cdot e^{-\lambda t}, \qquad t \geq 0.
\]
Dabei ist $n_0$ das zum Zeitpunkt $t = 0$ vorhandene radioaktive Material und 
$\lambda > 0$ die sogenannte Zerfallskonstante. 
Die Halbwertszeit gibt die Zeitdauer an, nach der die Hälfte der Atomkerne einer 
radioaktiven Substanz zerfallen sind. Sie lässt sich über die Formel
\[
t_{1/2} = \frac{1}{\lambda} \cdot \ln(2)
\]
berechnen. 
Dabei bezeichnet $\ln$ den natürlichen Logarithmus, siehe dazu \ref[link2][hier]{ln}.
Das radioaktive Isotop Radon 220 hat beispielsweise die Zerfallskonstante 
$ \lambda = 1,247 \cdot 10^{-2} \cdot s^{-1}$. Damit ergibt sich die Halbwertszeit 
von Radon 220 zu
}
\lang{en}{
Another example for exponential decay is radioactive decay, which can be modelled by 
the formula 
\[
n(t) = n_0 \cdot e^{-\lambda t}, \qquad t \geq 0,
\]
where $n_0$ is the level of radiactivity at the time $t=0$, and $\lambda > 0$ is the 
so-called \emph{decay constant}. 
The \emph{half-life} is the time required for the radioactivity of a sample to decay 
to half its original value, and can be calculated using the formula 
\[
t_{1/2} = \frac{1}{\lambda} \cdot \ln(2). 
\]
Here $\ln$ is the natural logarithm, introduced \ref[link2][here]{ln}. 
For example, the radioactive isotope \emph{radon 220} has a decay constant 
$ \lambda = 1,247 \cdot 10^{-2} \cdot s^{-1}$, making its half-life
}
\[
t_{1/2} = \frac{\ln(2)}{1,247 \cdot 10^{-2} \cdot s^{-1}}  = 
\frac{\ln(2) \cdot s}{1,247 \cdot 10^{-2}} \approx 55,8 \ s.
\]




\end{enumerate}

\end{visualizationwrapper}

\end{content}