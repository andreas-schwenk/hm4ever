\documentclass{mumie.problem.gwtmathlet}
%$Id$
\begin{metainfo}
  \name{
    \lang{de}{A03: Potenzgesetze}
    \lang{en}{}
  }
  \begin{description} 
 This work is licensed under the Creative Commons License Attribution 4.0 International (CC-BY 4.0)   
 https://creativecommons.org/licenses/by/4.0/legalcode 

    \lang{de}{Rechnen mit Potenzen}
    \lang{en}{}
  \end{description}
  \corrector{system/problem/GenericCorrector.meta.xml}
  \begin{components}
    \component{js_lib}{system/problem/GenericMathlet.meta.xml}{gwtmathlet}
  \end{components}
  \begin{links}
  \end{links}
  \creategeneric
\end{metainfo}
\begin{content}
\usepackage{mumie.genericproblem}
\usepackage{mumie.ombplus}

\lang{de}{
	\title{A03: Potenzgesetze}
}


\begin{block}[annotation]
  Im Ticket-System: \href{http://team.mumie.net/issues/9291}{Ticket 9291}
\end{block}


\begin{problem}
	\begin{question}
      \text{Berechne:}
      \explanation{Einfach ausrechnen}


      \begin{variables}
                   \randint[Z]{n}{1}{5}
                   \randint[Z]{m}{2}{3}
                   \randint[Z]{b}{-4}{-1}
                   \randint[Z]{c}{2}{4}
                   \randint[Z]{a}{1}{5}
                   \randint[Z]{z}{-3}{-1}
                   
                   \function[calculate]{fa}{n^m}
                   \function[calculate]{fb}{b^c}
                   \function[calculate]{fc}{a^z}
                   
       \end{variables}

      \type{input.number}
      \field{rational}

      \begin{answer}
            \text{$\var{n}^{\var{m}}=$ }
            \solution{fa}
      \end{answer}
      
      \begin{answer}
            \text{$(\var{b})^{\var{c}}=$ }
            \solution{fb}
      \end{answer}
      \begin{answer}
            \text{$\var{a}^{\var{z}}=$ }
            \solution{fc}
      \end{answer}
	\end{question}
	
	
		\begin{question}
      \text{Verwenden Sie die Potenzregeln, um die folgenden Ausdrücke als \textit{eine} Potenz 
      der Form $a^b$ mit $b>0$ zu schreiben:\\
      $\var{fa}=a^b$ mit $\, a=$\ansref, $b=$\ansref \\
      $\var{fb}=a^b$ mit $\, a=$\ansref, $b=$\ansref \\
      $\var{fc}=a^b$ mit $\, a=$\ansref, $b=$\ansref \\
      $\var{fd}=a^b$ mit $\, a=$\ansref, $b=$\ansref \\
      }
      \explanation{}

      \begin{variables}
                   \randint[Z]{n}{2}{5}
                   \randint[Z]{m}{2}{3}
                   \randint[Z]{b}{-4}{4}
                   \randint[Z]{c}{2}{4}
                   \randint[Z]{a}{2}{5}
                   \randint[Z]{z}{-3}{-1}
                   \randint[Z]{x}{2}{7}
                   \randint[Z]{d}{2}{7}
                   
                   \randadjustIf{m}{m = a}
                   \randadjustIf{d,x}{d = x OR d=4 OR x=4 }
                   \function[calculate]{aa}{n}
                   \function[calculate]{ab}{m+a}
                   \function[calculate]{ba}{d*x}
                   \function[calculate]{bb}{n}
                   \function[calculate]{ca}{d/x}
                   \function[calculate]{cb}{a}
                   \function[calculate]{da}{x}
                   \function[calculate]{db}{d*n}
%                   \function[calculate]{ea}{x}
%                   \function[calculate]{eb}{a}
                   
                   \function{fa}{n^m * n^a}
                   \function{fb}{d^n * x^n}
                   \function{fc}{d^a * x^(-a)}
                   \function{fd}{(x^d)^n}
%                   \function{fe}{x^(-a)*(x^2)^a}
                   
       \end{variables}

      \type{input.number}
      \field{rational}

      \begin{answer}
            \solution{aa}
      \end{answer}
      \begin{answer}
            \solution{ab}
      \end{answer}
      
      
      \begin{answer}
            \solution{ba}
      \end{answer}
      \begin{answer}
            \solution{bb}
      \end{answer}
      
     
      \begin{answer}
            \solution{ca}
      \end{answer}
      \begin{answer}
            \solution{cb}
      \end{answer}
      
      
      \begin{answer}
            \solution{da}
      \end{answer}
      \begin{answer}
            \solution{db}
      \end{answer}
      
      
      
%      \begin{answer}
%            \solution{ea}
%      \end{answer}
%      \begin{answer}
%            \solution{eb}
%      \end{answer}
           
      
	\end{question}
	%TODO AUFGABE 2
	
	
\end{problem}
\embedmathlet{gwtmathlet}


\end{content}