\documentclass{mumie.problem.gwtmathlet}
%$Id$
\begin{metainfo}
  \name{
    \lang{de}{A01: Potenzgesetze}
    \lang{en}{problem_1}
  }
  \begin{description} 
 This work is licensed under the Creative Commons License Attribution 4.0 International (CC-BY 4.0)   
 https://creativecommons.org/licenses/by/4.0/legalcode 

    \lang{de}{Vergleichen von Zahlen und einfache Addition}
    \lang{en}{}
  \end{description}
  \corrector{system/problem/GenericCorrector.meta.xml}
  \begin{components}
    \component{js_lib}{system/problem/GenericMathlet.meta.xml}{gwtmathlet}
  \end{components}
  \begin{links}
  \end{links}
  \creategeneric
\end{metainfo}
\begin{content}
\usepackage{mumie.ombplus}
\usepackage{mumie.genericproblem}


\lang{de}{
	\title{A01: Potenzgesetze}
}
\lang{en}{
	\title{Problem 1}
}

\begin{block}[annotation]
  Im Ticket-System: \href{http://team.mumie.net/issues/9285}{Ticket 9285}
\end{block}

\begin{problem}
	\begin{question}
      \lang{de}{\text{Schreiben Sie den Ausdruck $\frac{\var{a}\cdot \var{a}^{\frac{2}{\var{n}}}}{\var{a}^{\frac{1}{\var{n}}}}$ 
             in der Form
            $\var{a}^{\frac{x}{y}}$, wobei $\frac{x}{y}$ ein vollständig gekürzter Bruch ist.}}
      \lang{en}{\text{Write the expression $\frac{\var{a}\cdot \var{a}^{\frac{2}{\var{n}}}}{\var{a}^{\frac{1}{\var{n}}}}$ 
             in the form
            $\var{a}^{\frac{x}{y}}$, where $\frac{x}{y}$ is a simplified fraction.}}
      \explanation{}
      \type{input.number}
      \field{rational}

      \begin{variables}
                   \randint[Z]{a}{2}{11}
                   \randint[Z]{n}{2}{12}
                   \function[calculate]{b}{n+1}
                   
       \end{variables}

      \begin{answer}
            \text{$x=$ }
            \solution{b}
      \end{answer}
       \begin{answer}
            \text{$y=$ }
            \solution{n}
      \end{answer}
	\end{question}
	
	\begin{question}
      \lang{de}{\text{Berechnen Sie $x=\sqrt{(-\var{a})^\var{b}}$ und $y=\sqrt[3]{(-\var{a})^\var{c}}$.}}
      \lang{en}{\text{Calculate $x=\sqrt{(-\var{a})^\var{b}}$ and $y=\sqrt[3]{(-\var{a})^\var{c}}$.}}
      \explanation{}

      \type{input.number}
      \precision{3}
      \field{real}

      \begin{variables}
                   \randint[Z]{a}{2}{7}
                   \randint[Z]{n}{1}{4}
                   \randint[Z]{m}{1}{2}
                   \function[calculate]{b}{2*n}
                   \function[calculate]{c}{6*m}
                   \function[calculate]{x}{a^n}
                   \function[calculate]{y}{a^(2*m)}
       \end{variables}

      \begin{answer}
            \text{$x=$ }
            \solution{x}
      \end{answer}
       \begin{answer}
            \text{$y=$ }
            \solution{y}
      \end{answer}
	\end{question}
	
\end{problem}
\embedmathlet{gwtmathlet}



\end{content}