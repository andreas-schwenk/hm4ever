\documentclass{mumie.problem.gwtmathlet}
%$Id$
\begin{metainfo}
  \name{
    \lang{de}{A10: Logarithmusgleichungen}
    \lang{en}{}
  }
  \begin{description} 
 This work is licensed under the Creative Commons License Attribution 4.0 International (CC-BY 4.0)   
 https://creativecommons.org/licenses/by/4.0/legalcode 

    \lang{en}{}
    \lang{de}{}
  \end{description}
  \corrector{system/problem/GenericCorrector.meta.xml}
  \begin{components}
    \component{js_lib}{system/problem/GenericMathlet.meta.xml}{gwtmathlet}
  \end{components}
  \begin{links}
  \end{links}
  \creategeneric
  \begin{taxonomy}
        \difficulty{0}
        \usage{}
        \objectives{apply,remember}
        \topic{analysis1/equations/one_variable}
  \end{taxonomy}
\end{metainfo}

\begin{content}
\begin{block}[annotation]
	Im Ticket-System: \href{https://team.mumie.net/issues/30084}{Ticket 30084}
\end{block}
\usepackage{mumie.ombplus}
\usepackage{mumie.genericproblem}
\begin{block}[annotation]
Frage 2 und 3:	Copy of \href{https://team.mumie.net/issues/28387}{Ticket 28387} von Hoever/Hensgens FH Aachen.
\end{block}


\lang{de}{
	\title{A10: Logarithmusgleichungen}
}


\begin{problem}

    	\begin{question}
		\begin{variables}
			\randint{a}{0}{2}
			\randint{b}{1}{10}
            \randadjustIf{b}{a=b OR a+1=b}
            \randint{basis}{2}{99}
            %Nullstellen von pol
			\function[calculate]{C}{a+1}
            \function[calculate]{A}{a}
            \function[calculate]{B}{-b}
            \function[expand,normalize]{pol}{(x-A)*(x+b)*(x-C)+1}
           \end{variables}

	    \type{input.number}
	    \field{integer} 
        \lang{de}{
	    \text{Geben Sie die Lösungsmenge $\mathbb{L}$ der Gleichung  $~~\log_{\var{basis}}(\var{pol})=0$ an.\\\\
        $\quad\mathbb{L}=$\ansref}
	   }
       \begin{answer}
            \type{input.finite-number-set}
	        \solution{C,A,B}
            \lang{de}{\explanation{Formen Sie die Gleichung zunächst in eine Polynomgleichung um.}}
        \end{answer}
	\end{question}
    
	\begin{question}
		\begin{variables}
			\randint{a}{2}{4}
			\randint{b}{2}{10}
			\function[calculate]{c}{2*a+1}
			\function{loes1}{b*b}
		\end{variables}

	    \type{input.number}
	    \field{integer} 
        \lang{de}{
	    \text{Für welches $x$ gilt $~~\log_{\var{b}}x^\var{a}+\log_{\var{b}}\sqrt{x}~=~\var{c}~~$?}
	   }
       \begin{answer}
	        \text{$\quad x =$}
	        \solution{loes1}
            \lang{de}{\explanation{Benutzen Sie Logarithmusgesetze.}}
        \end{answer}
	\end{question}

	\begin{question}
		\begin{variables}
			\randint{a}{2}{6}
			\randint{a1}{2}{3}
			\randint{a2}{2}{4}
			\function[calculate]{a4}{2-a1-a2} % Vorfaktor vor log soll 2 sein
			\function[calculate]{a4neg}{-a4}  % Fuer schoene Darstellung
            \randint{expo}{1}{2} % Die log sollen gleich -expo sein
			\function[calculate]{b}{-2*expo+2*a1+a2-a4}
			\function[calculate]{loes}{1/(a^expo)}
			\function[calculate]{eda}{1/a}
			\function[calculate]{ahoch2}{a*a}
		\end{variables}

	    \type{input.number}
	    \field{rational} 
        \lang{de}{
	    \text{Für welches $x$ gilt $~~\var{a1}\log_{\var{a}}(\var{ahoch2}x)+\var{a2}\log_{\var{a}}(\var{a}x)-\var{a4neg}\log_{\var{a}}(\var{eda}x)~=~\var{b}~~$?}
	   }
       \begin{answer}
	        \text{$\quad x =$}
	        \solution{loes}
            \lang{de}{\explanation{Benutzen Sie Logarithmusgesetze.}}
        \end{answer}
	\end{question}

\end{problem}


\embedmathlet{gwtmathlet}

\end{content}



