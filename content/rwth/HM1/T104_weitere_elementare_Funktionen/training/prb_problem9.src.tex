\documentclass{mumie.problem.gwtmathlet}
%$Id$
\begin{metainfo}
  \name{
    \lang{en}{Simplification of logarithmic terms}
    \lang{de}{A09: Rechnen mit Logarithmen}
    \lang{zh}{...}
    \lang{fr}{...}
  }
  \begin{description} 
 This work is licensed under the Creative Commons License Attribution 4.0 International (CC-BY 4.0)   
 https://creativecommons.org/licenses/by/4.0/legalcode 

    \lang{en}{...}
    \lang{de}{...}
    \lang{zh}{...}
    \lang{fr}{...}
  \end{description}
  \corrector{system/problem/GenericCorrector.meta.xml}
  \begin{components}
    \component{js_lib}{system/problem/GenericMathlet.meta.xml}{gwtmathlet}
  \end{components}
  \begin{links}
  \end{links}
  \creategeneric
  \begin{taxonomy}
        \difficulty{1}
        \usage{HM4Mint}
        \objectives{apply}
        \topic{analysis1/elementary_functions/real_logarithmic_function}
  \end{taxonomy}
\end{metainfo}
\begin{content}
\begin{block}[annotation]
	Im Ticket-System: \href{https://team.mumie.net/issues/30101}{Ticket 30101}
\end{block}
\begin{block}[annotation]
Copy of : /home/mumie/checkin/content/playground/TimJacobs/prb_H4MintLogExp.src.tex
\end{block}

\usepackage{mumie.genericproblem}
\lang{de}{
	\title{A09: Rechnen mit Logarithmen}
}
\embedmathlet{gwtmathlet}
\begin{problem}
    \begin{question}
 
      \begin{variables}
        \randint{a}{2}{99}
        \randint{b}{2}{99}
        \randadjustIf{b}{a = b}
        \randint{c}{2}{99}
        \randadjustIf{c}{c=a OR c=b}
        \randint{d}{2}{99}
        \randadjustIf{d}{d=c OR d=b OR d=a }
        \function{f1}{log(d)}
        \function{f2}{ln(d)}
      \end{variables}
 
      \type{input.function}
      \field{real}
      \lang{en}{\text{Simplify as much as possible:}}
      \lang{de}{\text{Vereinfachen Sie soweit wie möglich:}}     
      
 
      \begin{answer}
          \text{$\log_{\var{a}}\left(\var{b}^{\log_{\var{b}}( \var{c})} \right) \cdot \log_{\var{c}}\left(\var{d}^{\ln(\var{a})} \right) = $}
          \inputAsFunction{x}{g}
          \solution{f2}
          \checkStringsForRelation{equal(g,f1) OR equal(g,f2)}
          \explanation{Benutzen Sie die Logarithmusgesetze. Geben Sie Ihr Ergebnis in der Form $\ln(a)$ ein.} 
      \end{answer}
 
  \end{question}
  
  \begin{question} 
    \begin{variables}
        \randint{a}{2}{3}
        \randint{c}{4}{10}
        \randint{d}{4}{10}
        \randadjustIf{d}{c=d}
        \function[calculate]{b}{c*d}
        \function{f}{a}
        \function[calculate]{CA}{c^a}
    \end{variables}
    \lang{en}{\text{Simplify as much as possible:}}
    \lang{de}{\text{Vereinfachen Sie soweit wie möglich:}}
    \type{input.function}
     \field{real}
    \begin{answer}
        \text{$\log_{\var{b}}(e) \cdot \ln(\var{CA}) + \var{a}  \cdot \ln(\var{d})\cdot \log_{\var{b}}(e)=$}
         \solution{f}
         \inputAsFunction{x}{g}
         \checkStringsForRelation{equal(g,f)}
         \explanation{Benutzen Sie die Logarithmusgesetze. }
    \end{answer}
  \end{question}
  
  \begin{question}
    \begin{variables}
        \randint{c}{2}{5}
        \randint{r}{2}{4}
        \randint{a}{2}{10}
        \randadjustIf{a,c,r}{c^r>=a^2}
        \function[calculate]{XX}{a^2-c^r}
        \function{b}{sqrt(XX)}
        \function{f}{r}
    \end{variables}
    \type{input.function}
      \field{real}
      \lang{en}{\text{Simplify as much as possible:}}
      \lang{de}{\text{Vereinfachen Sie soweit wie möglich:}}     
      \explanation{} 
      \begin{answer}
        \text{$\log_{\var{c}}(\var{a}-\var{b}) + \log_{\var{c}}(\var{a}+\var{b})=$}
        \solution{f}
        \inputAsFunction{x}{g}
        \checkStringsForRelation{equal(g,f)}
         \explanation{Benutzen Sie die Logarithmusgesetze. }
      \end{answer}
  \end{question}
\end{problem}

\end{content}
