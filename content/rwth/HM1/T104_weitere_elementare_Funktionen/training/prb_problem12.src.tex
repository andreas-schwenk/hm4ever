\documentclass{mumie.problem.gwtmathlet}
%$Id$
\begin{metainfo}
  \name{
    \lang{en}{Problem 12}
    \lang{de}{A12: Logarithmusfunktion}
  }
  \begin{description} 
 This work is licensed under the Creative Commons License Attribution 4.0 International (CC-BY 4.0)   
 https://creativecommons.org/licenses/by/4.0/legalcode 

    \lang{en}{...}
    \lang{de}{...}
  \end{description}
  \corrector{system/problem/GenericCorrector.meta.xml}
  \begin{components}
    \component{js_lib}{system/problem/GenericMathlet.meta.xml}{gwtmathlet}
  \end{components}
  \begin{links}
  \end{links}
  \creategeneric
  \begin{taxonomy}
        \difficulty{2}
        \usage{FH Aachen, Mathematische Grundlagen}
        \objectives{analyze,apply,understand}
        \topic{analysis1/functions/graph_of_function}
  \end{taxonomy}
\end{metainfo}
\begin{content}
\begin{block}[annotation]
	Im Ticket-System: \href{https://team.mumie.net/issues/30083}{Ticket 30083}
\end{block}
\begin{block}[annotation]
Copy of : /home/mumie/checkin/content/rwth/HM1/T104_weitere_elementare_Funktionen/training/prb_problem11.src.tex
\end{block}

\begin{block}[annotation]
Copy of : /home/mumie/checkin/content/pool/fh_aachen/MGI/Logarithmusfunktionen/prb_Log_A7.2.src.tex
\end{block}


\usepackage{mumie.genericproblem}
\lang{de}{\title{A12: Logarithmusfunktion}}
\begin{visualizationwrapper}


\begin{genericJSXVisualization}
	\begin{variables}
        \parametricFunction{k}{real}{t,0*t,-20,20,1000} %geht nicht mit infinity als Grenzen
        \pointOnCurve{P}{real}{k}{0.5}
        \point[editable]{A}{real}{-0.5,0}
        \point[editable]{B}{real}{-0.5,1}
        \line[editable]{l}{real}{var(A), var(B)}
        \function{g}{real}{log((x-A[x])/(P[x]-A[x]))}
        \number{b0}{real}{A[x]}
        \number{c0}{real}{P[x]-A[x]}
        \number{px}{real}{P[x]}
	\end{variables}
\color{g}{RED}
\label{l}{l}
%\label{g}{g}
\label{P}{P}
\answer{b0}{1,1}
\answer{px}{1,2}

	\begin{canvas}
    \snapToGrid{0.1,0.1}
        \plotSize{500,500}
        \plotLeft{-5.5}
        \plotRight{5.5}
        \plot[coordinateSystem,showPointCoords]{P,l,g}
	\end{canvas}

\end{genericJSXVisualization}

\end{visualizationwrapper}

     \begin{problem} 
          \begin{question}
          \type{input.function}
          \field{real}
          \begin{variables}
                \randint{b}{-3}{3}
                \randint[Z]{c}{-3}{3}
                \randadjustIf{b,c}{|b+c|=6}
                \function{f}{ln((x-b)/c)}
                \function[calculate]{px}{b+c}
           \end{variables}
          \lang{de}{
               \text{Verändern Sie die Kurve in der Grafik, sodass der Graph der Funktion $f(x)=\var{f}$ zu sehen ist.\\
                Beweglich sind dabei der Punkt $P$ auf der $x$-Achse und die senkrechte Asymptote.}}        
          \begin{answer}
            \type{graphics.number}
            \solution{b}
          \end{answer}
          \begin{answer}
            \type{graphics.number}
            \solution{px}
          \end{answer}
         \explanation[equal(ans_1,b)]{Die Asymptote ist richtig gewählt.}
         \explanation[edited AND NOT [equal(ans_1,b)]]{Die Asymptote ist falsch.}
         \explanation[equal(ans_2,px)]{Der Punkt $P$ ist richtig gewählt.}
         \explanation[edited AND NOT [equal(ans_2,px)]]{Der Punkt $P$ ist falsch.}
          \end{question}
     \end{problem}
     

\embedmathlet{gwtmathlet}

\end{content}
