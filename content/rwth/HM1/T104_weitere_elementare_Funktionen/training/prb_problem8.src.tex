\documentclass{mumie.problem.gwtmathlet}
%$Id$
\begin{metainfo}
  \name{
    \lang{de}{A08: Rechnen mit Potenzen}
    \lang{en}{Problem 8}
  }
  \begin{description} 
 This work is licensed under the Creative Commons License Attribution 4.0 International (CC-BY 4.0)   
 https://creativecommons.org/licenses/by/4.0/legalcode 

    \lang{de}{Lösen von Gleichungen mit Potenzen}
    \lang{en}{}
  \end{description}
  \corrector{system/problem/GenericCorrector.meta.xml}
  \begin{components}
    \component{js_lib}{system/problem/GenericMathlet.meta.xml}{gwtmathlet}
  \end{components}
  \begin{links}
  \end{links}
  \creategeneric
\end{metainfo}
\begin{content}
\usepackage{mumie.genericproblem}
\usepackage{mumie.ombplus}
\lang{de}{
	\title{A08: Rechnen mit Potenzen}
}


\begin{block}[annotation]
  Im Ticket-System: \href{http://team.mumie.net/issues/9290}{Ticket 9290}
\end{block}



\begin{problem}
	\begin{question}
      \lang{de}{\text{Bestimmen Sie $m$ und $n$ so, dass gilt: $(\var{a}x^{\var{b}})^\var{c}=m{x}^{n}$. }}
      \lang{en}{\text{Find $m$ and $n$ so that: $(\var{a}x^{\var{b}})^\var{c}=m{x}^{n}$. }}
      \explanation{}

      \type{input.number}
      \precision{3}
      \field{real}

      \begin{variables}
                   \randint[Z]{a}{2}{20}
                   \randint[Z]{b}{2}{20}
                   \randint[Z]{c}{2}{3}
                   \randint[Z]{d}{2}{20}
                   \function[calculate]{ee}{a^c}
                   \function[calculate]{f}{b*c}
       \end{variables}

      \begin{answer}
            \text{$m=$ }
            \solution{ee}
      \end{answer}
      
      \begin{answer}
            \text{$n=$ }
            \solution{f}
      \end{answer}
	\end{question}
	
	\begin{question}
      \lang{de}{\text{Für welches $x$ ist
      $\var{a}^{\var{b}}\cdot \var{a}^{-4}\cdot \var{a}^{\var{c}}:\var{a}^{\var{n}}=\var{a}^{x}$? }}
      \lang{en}{\text{For what value of $x$ is $\var{a}^{\var{b}}\cdot \var{a}^{-4}\cdot \var{a}^{\var{c}}/\var{a}^{\var{n}}=\var{a}^{x}$? }}
      \explanation{}

      \type{input.number}
      \precision{3}
      \field{real}

      \begin{variables}
                   \randint[Z]{a}{2}{20}
                   \randint[Z]{n}{2000}{3000}
                   \function[calculate]{b}{2*n-2}
                   \function[calculate]{c}{n+3}
                   \function[calculate]{d}{b-1}
       \end{variables}

      \begin{answer}
            \lang{de}{\text{Antwort: }}
            \lang{en}{\text{Answer: }}
            \solution{d}
      \end{answer}
	\end{question}
	
	\begin{question}
      \lang{de}{\text{Bestimmen Sie $x$ und $y$, so dass für alle positiven $z$ gilt: 
      $\quad z^{\var{b1}}\cdot z^{\var{b}}+ z^{\var{c1}}:z^{\var{c}}=y \, z^{x}$. }}
      \lang{en}{\text{Find $x$ and $y$ so that for all positive $z$:  
      $\quad z^{\var{b1}}\cdot z^{\var{b}}+ z^{\var{c1}}/z^{\var{c}}=y \, z^{x}$. }}
      \explanation{}

      \type{input.number}
      \precision{3}
      \field{real}

      \begin{variables}
                   \randint[Z]{a}{2}{20}
                   \randint[Z]{b}{3}{11}
                   \randint[Z]{c}{3}{11}
                   \function[calculate]{b1}{-b+3}
                   \function[calculate]{c1}{c+3}
                   \function[calculate]{x}{3}
                   \function[calculate]{y}{2}
                          \end{variables}

      \begin{answer}
            \text{$x=$ }
            \solution{x}
      \end{answer}
      \begin{answer}
            \text{$y=$ }
            \solution{y}
      \end{answer}
	\end{question}
\end{problem}
\embedmathlet{gwtmathlet}



\end{content}