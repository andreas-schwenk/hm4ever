\documentclass{mumie.problem.gwtmathlet}
%$Id$
\begin{metainfo}
  \name{
    \lang{de}{A07: Rechnen mit Potenzen}
    \lang{en}{Problem 7}
  }
  \begin{description} 
 This work is licensed under the Creative Commons License Attribution 4.0 International (CC-BY 4.0)   
 https://creativecommons.org/licenses/by/4.0/legalcode 

    \lang{de}{Rechnen mit Potenzen}
    \lang{en}{}
  \end{description}
  \corrector{system/problem/GenericCorrector.meta.xml}
  \begin{components}
    \component{js_lib}{system/problem/GenericMathlet.meta.xml}{gwtmathlet}
  \end{components}
  \begin{links}
  \end{links}
  \creategeneric
\end{metainfo}
\begin{content}
\usepackage{mumie.genericproblem}
\usepackage{mumie.ombplus}

\lang{de}{
	\title{A07: Rechnen mit Potenzen}
}


\begin{block}[annotation]
  Im Ticket-System: \href{http://team.mumie.net/issues/9289}{Ticket 9289}
\end{block}



\begin{problem}
	\begin{question}
      \lang{de}{\text{Bestimmen Sie $m$ und $n$ so, dass gilt: $\var{ee}x^{\var{b}}:\var{c}x^{\var{d}}=m{x}^{n}$. }}
      \lang{en}{\text{Find $m$ and $n$ so that: $\var{ee}x^{\var{b}}/\var{c}x^{\var{d}}=m{x}^{n}$. }}
      \explanation{}

      \type{input.number}
      \precision{3}
      \field{real}

      \begin{variables}
                   \randint[Z]{a}{2}{20}
                   \randint[Z]{b}{2}{20}
                   \randint[Z]{c}{2}{20}
                   \randint[Z]{d}{2}{20}
                   \function[calculate]{ee}{a*c}
                   \function[calculate]{f}{b-d}
       \end{variables}

      \begin{answer}
            \text{$m=$ }
            \solution{a}
      \end{answer}
      
      \begin{answer}
            \text{$n=$ }
            \solution{f}
      \end{answer}
	\end{question}
	
	\begin{question}
      \lang{de}{\text{W\"{a}hlen Sie alle richtigen Antworten aus.}}
      \lang{en}{\text{Select all of the correct answers.}}
      \explanation{}

      \permutechoices{1}{4}
      \type{mc.multiple}
      \field{real}

      \begin{variables}
            \randint{a}{1}{11}
            \randint{b}{2}{12}
             \randint{m}{2}{35}
            \randint{c}{2}{4}
             \function[calculate]{d}{-c}
              \function[calculate]{ee}{b^c}
                \randadjustIf{a,b}{a=b} % see Part 9 http://team.mumie.net/projects/support/wiki/GenericTexProblems 
      \end{variables}
      \begin{choice}
            \text{$(\var{a} \cdot \var{b})^{\var{c}}=\var{a}^{\var{c}}\cdot \var{b}^{\var{c}}$}
            \solution{true}
      \end{choice}

      \begin{choice}
            \text{$\var{b}^{-\var{c}} =\frac{1}{\var{ee}}$}
            \solution{true}
      \end{choice}
            \begin{choice}
            \text{$\var{b}^{\var{c}}=\var{ee}$}
            \solution{true}
      \end{choice}
      \begin{choice}
            \text{$\var{m}^1=\var{m}$}
            \solution{true}
      \end{choice}
      \begin{choice}
            \text{$\var{m}^0=0$}
            \solution{false}
      \end{choice}
\end{question}

\begin{question}
      \lang{de}{\text{W\"{a}hlen Sie alle richtigen Antworten aus.}}
      \lang{en}{\text{Select all of the correct answers.}}
      \explanation{}

      \permutechoices{1}{4}
      \type{mc.multiple}
      \field{real}

      \begin{variables}
            \randint{a}{1}{11}
            \randint{b}{2}{12}
             \randint{m}{2}{35}
            \randint{c}{2}{4}
             \function[calculate]{d}{-c}
                \randadjustIf{a,b}{a=b} % see Part 9 http://team.mumie.net/projects/support/wiki/GenericTexProblems 
      \end{variables}
      \begin{choice}
            \text{$(1 \cdot \var{b})^{\var{c}}=1\cdot \var{b}^{\var{c}}$}
            \solution{true}
      \end{choice}

     \begin{choice}
            \text{$(0 \cdot \var{b})^{\var{c}}=0^{\var{c}}\cdot \var{b}$}
            \solution{true}
      \end{choice}
           
      \begin{choice}
            \text{$\var{m}^0=\var{m}$}
            \solution{false}
      \end{choice}
      \begin{choice}
            \text{$\var{m}^1=1$}
            \solution{false}
      \end{choice}
\end{question}
\end{problem}
\embedmathlet{gwtmathlet}



\end{content}