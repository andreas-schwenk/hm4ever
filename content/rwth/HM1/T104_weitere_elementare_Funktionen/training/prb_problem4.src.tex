\documentclass{mumie.problem.gwtmathlet}
%$Id$
\begin{metainfo}
  \name{
    \lang{de}{A04: Potenzgesetze}
    \lang{en}{}
  }
  \begin{description} 
 This work is licensed under the Creative Commons License Attribution 4.0 International (CC-BY 4.0)   
 https://creativecommons.org/licenses/by/4.0/legalcode 

    \lang{de}{Rechnen mit Potenzen}
    \lang{en}{}
  \end{description}
  \corrector{system/problem/GenericCorrector.meta.xml}
  \begin{components}
    \component{js_lib}{system/problem/GenericMathlet.meta.xml}{gwtmathlet}
  \end{components}
  \begin{links}
  \end{links}
  \creategeneric
\end{metainfo}
\begin{content}
\usepackage{mumie.genericproblem}
\usepackage{mumie.ombplus}

\lang{de}{
	\title{A04: Potenzgesetze}
}

\begin{block}[annotation]
  Im Ticket-System: \href{http://team.mumie.net/issues/9414}{Ticket 9414}
\end{block}


\begin{problem}
	\begin{question}
      \text{Schreiben Sie die folgenden Ausdrücke als Potenzen $a^q$ mit rationalen
      Zahlen $a$ und $q\geq 0$. Berechnen Sie keine Potenzen, sondern verwenden Sie dazu lediglich
      die Rechenregeln für Potenzen und Wurzeln.\\
      $\var{fa}=a^q$ mit $a=$\ansref, $q=$\ansref,\\
      $\var{fb}=a^q$ mit $a=$\ansref, $q=$\ansref,\\
      $\var{fc}=a^q$ mit $a=$\ansref, $q=$\ansref.\\ }
      \explanation{Haben Sie eine Lösung mit $q\geq 0$ gewählt?}


      \begin{variables}
                   \randint{b}{-4}{-1}
                   \randint[Z]{a}{2}{5}
                   \randint[Z]{x}{2}{7}
                   \randint[Z]{f}{2}{10}
                   \randint[Z]{g}{2}{7}
                   \randadjustIf{a,b}{a = -b}
                   
                   \function[calculate]{saa}{1/x}
                   \function[calculate]{saq}{-b/a}
                   
                   \function[calculate]{sba}{1/f}
                   \function[calculate]{sbq}{1/2}
                   
                   \function[calculate]{sca}{a}
                   \function[calculate]{scq}{(g-2)/(2*g)}
                  
                   \function[normalize]{q1}{b/a}
                   \function{fa}{x^(q1)}
                   \function{fb}{sqrt(f)/f}
                   \function{fc}{a/(sqrt(a)*a^(1/g))}
                   
                   
       \end{variables}

      \type{input.number}
      \field{rational}

      \begin{answer}
            \solution{saa}
      \end{answer}

      \begin{answer}
            \solution{saq}
      \end{answer}


      \begin{answer}
            \solution{sba}
      \end{answer}

      \begin{answer}
            \solution{sbq}
      \end{answer}


      \begin{answer}
            \solution{sca}
      \end{answer}

      \begin{answer}
            \solution{scq}
      \end{answer}

      
	\end{question}
	
	
\end{problem}
\embedmathlet{gwtmathlet}


\end{content}