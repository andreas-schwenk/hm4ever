\documentclass{mumie.element.exercise}
%$Id$
\begin{metainfo}
  \name{
    \lang{de}{Ü04: Exponentialfunktion}
    \lang{en}{Exercise 4}
  }
  \begin{description} 
 This work is licensed under the Creative Commons License Attribution 4.0 International (CC-BY 4.0)   
 https://creativecommons.org/licenses/by/4.0/legalcode 

    \lang{de}{Hier die Beschreibung}
    \lang{en}{}
  \end{description}
  \begin{components}
  \end{components}
  \begin{links}
  \end{links}
  \creategeneric
\end{metainfo}
\begin{content}
\title{
	\lang{de}{Ü04: Exponentialfunktion}
	\lang{en}{Exercise 4} 														
}

\begin{block}[annotation]
	Übungsaufgabe zur Bestimmung von Funktionsgleichungen exponentiellen Typs					
\end{block}
\begin{block}[annotation]
	Im Ticket-System: \href{http://team.mumie.net/issues/9099}{Ticket 9099}   
\end{block}

\lang{de}{Bestimmen Sie die Gleichung der Exponentialfunktion vom Typ $f(x) = c \cdot a^x,\; a>0$, die jeweils durch die folgenden beiden Punkte $P=\left(x; f(x)\right)$ verläuft:}
\lang{en}{Find the equation of the exponential function of the form $f(x) = c \cdot a^x,\; a>0$ that goes through the following points:}
							
\begin{table}[\class{items}]
	\nowrap{a) 
	\lang{de}{$(2; \, 1)$ und $(3; \, 6)$}
	\lang{en}{$(2, \, 1)$ and $(3, \, 6)$}} 
	& 
	\nowrap{b) 
	\lang{de}{$(1; \, 5)$ und $(4; \, 40)$}
	\lang{en}{$(1, \, 5)$ and $(4, \, 40)$}}
	& 
	\nowrap{c) 
	\lang{de}{$(1; \, -\frac{2}{3})$ und $(-1; \, -6)$}
	\lang{en}{$(1, \, -\frac{2}{3})$ and $(-1, \, -6)$}
	} 
\end{table}

\begin{tabs*}[\initialtab{0}\class{exercise}]								
\tab{\lang{de}{Antwort}\lang{en}{Answer}}
	\begin{table}[\class{items}]
		\nowrap{a) $f(x) = \frac{1}{36} \cdot 6^x$} & \nowrap{b) $f(x) = \frac{5}{2} \cdot 2^x$} 
		& \nowrap{c) $f(x) = -2 \cdot \left(\frac{1}{3}\right)^x$ }  
	\end{table}
 

\tab{\lang{de}{L"osung a)}\lang{en}{Solution a)}}
	\begin{incremental}[\initialsteps{1}] 										
	\step 
		\lang{de}{Nach Voraussetzung müssen die Gleichungen }
		\lang{en}{As per the requirements, the following two equations need to be satisfied:}
		\begin{align*}
			\lang{de}{1 = f(2) = c \cdot a^2, \\
			6 = f(3) = c \cdot a^3}
			\lang{en}{1 = f(2) = c \cdot a^2 \\
			6 = f(3) = c \cdot a^3}
		\end{align*}
		\lang{de}{erfüllt sein.}
	\step
		\lang{de}{Wir multiplizieren die erste Gleichung mit $a \neq 0$ und erhalten}
		\lang{en}{Multiply the first equation by $a\neq 0$ and get}
		\begin{align*}
			\lang{de}{a = c \cdot a^3 , \\
			6 = c \cdot a^3 ,}
			\lang{en}{a = c \cdot a^3  \\
			6 = c \cdot a^3 }
		\end{align*}
		\lang{de}{also }
		\lang{en}{and hence }
		\[
		 \lang{de}{a = c \cdot a^3 = 6.}
		 \lang{en}{a = c \cdot a^3 = 6}
		\]
	\step 
		\lang{de}{Um $c$ zu bestimmen, lösen wir die erste Gleichung nach $c$ auf, also $c = a^{-2} = 6^{-2} = \frac{1}{36}$.\\
		Insgesamt ist die Lösung damit $f(x) = \frac{1}{36} \cdot 6^x$.}
		\lang{en}{In order to determine $c$, we can solve the first equation for $c$, i.e. $c = a^{-2} = 6^{-2} = \frac{1}{36}$.\\
		The full solution is $f(x) = \frac{1}{36} \cdot 6^x$.}
	\end{incremental}

\tab{\lang{de}{L"osung b)}\lang{en}{Solution b)}}
	\begin{incremental}[\initialsteps{1}]
	\step
	    \lang{de}{Nach Voraussetzung müssen die Gleichungen }
	    \lang{en}{As per the requirements, the following two equations need to be satisfied:}
		\begin{align*}
			\lang{de}{5 = f(1) = c \cdot a^1 , \\
			40 = f(4) = c \cdot a^4}
			\lang{en}{5 = f(1) = c \cdot a^1  \\
			40 = f(4) = c \cdot a^4}
		\end{align*}
	    \lang{de}{erfüllt sein.}
	    \lang{en}{}
	\step
	    \lang{de}{Wir multiplizieren die erste Gleichung mit $a^3 \neq 0$ und erhalten }
	    \lang{en}{Multiply the first equation by $a^3 \neq 0$ and get}
		\begin{align*}
			\lang{de}{5a^3 = c \cdot a^4 , \\
			40 = c \cdot a^4 ,}
			\lang{en}{5a^3 = c \cdot a^4  \\
			40 = c \cdot a^4 }
		\end{align*}
	    \lang{de}{also }
	    \lang{en}{and hence }
		\[
		 \lang{de}{5 a^3  = c \cdot a^4 = 40.}
		 \lang{en}{5 a^3  = c \cdot a^4 = 40}
		\]
	    \lang{de}{Dies ist äquivalent zu $a^3 = 8$, also $a=2$.}
	    \lang{en}{This is equivalent to $a^3 = 8$, i.e. $a=2$.}
	\step
	    \lang{de}{Um $c$ zu bestimmen, lösen wir die erste Gleichung nach $c$ auf, also $c = 5 \cdot a^{-1}$. 
		Mit dem Ergebnis $a=2$ folgt $c=5 \cdot a^{-1}= \frac{5}{2}$.\\
		Es ergibt sich die Lösung $f(x) = \frac{5}{2} \cdot 2^x$.}
		\lang{en}{In order to determine $c$, we can solve the first equation for $c$, i.e. $c = 5 \cdot a^{-1}$.
		Using the result that $a=2$, it follows that $c=5 \cdot a^{-1}= \frac{5}{2}$.\\
		The full solution is $f(x) = \frac{5}{2} \cdot 2^x$.}
	\end{incremental}
	
	
\tab{\lang{de}{L"osung c)}\lang{en}{Solution c)}}
	\begin{incremental}[\initialsteps{1}] 										
	\step 
	    \lang{de}{Nach Voraussetzung müssen die Gleichungen }
	    \lang{en}{As per the requirements, the following two equations need to be satisfied:}
		\begin{align*}
			\lang{de}{-\frac{2}{3} = f(1) = c \cdot a, \\
			-6 = f(-1) = c \cdot a^{-1}}
			\lang{en}{-\frac{2}{3} = f(1) = c \cdot a \\
			-6 = f(-1) = c \cdot a^{-1}}
		\end{align*}
	    \lang{de}{erfüllt sein.}
	    \lang{en}{}
	\step
	    \lang{de}{Wir multiplizieren die zweite Gleichung mit $a^2 \neq 0$ und erhalten}
	    \lang{en}{Multiply the second equation by $a^2 \neq 0$ and get}
		\begin{align*}
			\lang{de}{-\frac{2}{3} = c \cdot a, \\
			-6 a^2 = c \cdot a ,}
			\lang{en}{-\frac{2}{3} = c \cdot a \\
			-6 a^2 = c \cdot a }
		\end{align*}
	    \lang{de}{also }
	    \lang{en}{and hence }
		\[
		 \lang{de}{-\frac{2}{3} = c \cdot a = -6 a^2,}
		 \lang{en}{-\frac{2}{3} = c \cdot a = -6 a^2}
		\]
	    \lang{de}{woraus folgt}
	    \lang{en}{from which it follows that}
		\[ 
		\lang{de}{a^2 = \frac{\frac{2}{3}}{6} = \frac{1}{9},\qquad a= \frac{1}{3}>0.}
		\lang{en}{a^2 = \frac{\frac{2}{3}}{6} = \frac{1}{9},\qquad a= \frac{1}{3}>0}
		\]
	\step 
	    \lang{de}{Um $c$ zu bestimmen, lösen wir die zweite Gleichung nach $c$ auf, also $c = -6 a = \frac{-6}{3} = -2$.\\
		Insgesamt ist die Lösung damit $f(x) = -2 \cdot \left(\frac{1}{3}\right)^x$.}
		\lang{en}{In order to determine $c$, we can solve the second equation for $c$, i.e. $c = -6 a = \frac{-6}{3} = -2$.\\
		The full solution is $f(x) =  -2 \cdot \left(\frac{1}{3}\right)^x$.}
	\end{incremental}
	
\end{tabs*}  
\end{content}