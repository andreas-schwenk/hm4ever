\documentclass{mumie.element.exercise}
%$Id$
\begin{metainfo}
  \name{
    \lang{de}{Ü08: Logarithmus}
    \lang{en}{Exercise 8}
  }
  \begin{description} 
 This work is licensed under the Creative Commons License Attribution 4.0 International (CC-BY 4.0)   
 https://creativecommons.org/licenses/by/4.0/legalcode 

    \lang{de}{Hier die Beschreibung}
    \lang{en}{}
  \end{description}
  \begin{components}
  \end{components}
  \begin{links}
  \end{links}
  \creategeneric
\end{metainfo}
\begin{content}
\title{
	\lang{de}{Ü08: Logarithmus}
}

\begin{block}[annotation]
	Definitionsbereiche von Funktionen	
\end{block}
\begin{block}[annotation]
	Im Ticket-System: \href{http://team.mumie.net/issues/9284}{Ticket 9284}
\end{block}

Lösen Sie folgende Gleichungen nach den Variablen auf:
\begin{enumerate}[a)]
 \item a) $\log_c 8 = 3$
 \item b) $\log_2 z = 4$
 \item c) $\log_5(b^2) + \log_5(b) = 6$
 \item d) $ \log_2(8x) + \log_2(4x) + \log_2(\frac{x}{2} ) = 1 $
 \item e) $ 3\log_{10}(x) + \log_{10}(\sqrt{x}) = 7 $
 \item f) $ \log_3(\sqrt{a}) - \log_3(\sqrt[3]{a}) = \frac{1}{3} $
 \item g) $ \log_3(x) - \log_9(x) =1 $
 \item h) $ \log_a(4) + \log_a(9) = 2 $
  \item \qquad
\end{enumerate}



Berechnen Sie die folgenden Ausdrücke mit Hilfe der Logarithmen-Gesetze.\\
\begin{enumerate}[a)]
 \item i) $\ln(x^2-16)-\ln(x+4)$, für $x>4$
 \item j) $2\ln(\sqrt{e^3})$
 \item k) $\lg\left(\frac{10^2}{10000} \right)$
  \item \qquad
\end{enumerate}

\begin{tabs*}[\initialtab{0}\class{exercise}]
\tab{\lang{de}{Antwort}\lang{en}{Answer}}
a) $c=2$\\
b) $z = 16$\\
c) $b = 25$\\
d) $x = \frac{1}{2}$\\
e) $x = 100 $\\
f) $ a = 9 $ \\
g) $ x= 9$ \\
h) $ c = 6$\\
i) $\ln(x-4)$\\
j) $3$\\
k) $-2$

  \tab{\lang{de}{Lösungsvideo a) - h)}}	
    \youtubevideo[500][300]{LMibdscwpGU}\\


\tab{\lang{de}{Lösung i)}\lang{en}{Solution i)}}
	\begin{incremental}[\initialsteps{1}]
    \step 
	 Mit Hilfe der $3.$ binomischen Formel sehen wir ein, dass $x^2-16=(x+4)(x-4)$ ist.
    \step 
     Dann berechnen wir wie folgt für $x>4$
\begin{align*}
 \ln(x^2-16) - \ln(x+4) = \ln \left( \frac{x^2-16}{x+4} \right) 
 = \ln \left( \frac{(x-4)(x+4)}{x+4} \right) = \ln(x-4).
\end{align*} 
  	\end{incremental}

\tab{\lang{de}{Lösung j)}\lang{en}{Solution j)}}
    \begin{incremental}[\initialsteps{1}]
    \step
	Wir benutzen die Definition der Wurzel und erhalten damit
\begin{align*}
 2\ln(\sqrt{e^3}) = 2 \ln( e^{\frac{3}{2}} ) = 2 \cdot \frac{3}{2} \ln(e) = 3.
\end{align*}
	
    \end{incremental}

\tab{\lang{de}{Lösung k)}\lang{en}{Solution k)}}
    \begin{incremental}[\initialsteps{1}]
    \step
	   Da $10^4=10000$ gilt mit Hilfe der Rechenregeln für den Logarithmus zur Basis 10
\begin{align*}
 \lg\left(\frac{10^2}{10000} \right) &= \lg(10^2)- \lg(10^4) \\
 &= 2 \lg(10) - 4 \lg(10) =2-4 =-2.
\end{align*}

Alternativ rechnet man
\begin{align*}
\lg\left(\frac{10^2}{10000} \right)=\lg\left(\frac{10^2}{10^4}\right) = \lg(10^{-2}) = -2.
\end{align*}
    \end{incremental}
    
    
    
    
\end{tabs*}


\end{content}