\documentclass{mumie.element.exercise}
%$Id$
\begin{metainfo}
  \name{
    \lang{de}{Ü05: Exponentialgraphen}
    \lang{en}{Exercise 5}
  }
  \begin{description} 
 This work is licensed under the Creative Commons License Attribution 4.0 International (CC-BY 4.0)   
 https://creativecommons.org/licenses/by/4.0/legalcode 

    \lang{de}{Hier die Beschreibung}
    \lang{en}{}
  \end{description}
  \begin{components}
    \component{generic_image}{content/rwth/HM1/images/g_tkz_T104_Exercise05.meta.xml}{T104_Exercise05}
  \end{components}
  \begin{links}
  \end{links}
  \creategeneric
\end{metainfo}
\begin{content}
\title{
	\lang{de}{Ü05: Exponentialgraphen}
	\lang{en}{Exercise 5} 														
}

\begin{block}[annotation]
	TODO: Verlauf der Expononentialfunktionen
\end{block}
\begin{block}[annotation]
	Im Ticket-System: \href{http://team.mumie.net/issues/9100}{Ticket 9100}
\end{block}

\lang{de}{
a) Sizzieren Sie die Funktionen

\begin{table}[\class{items}]
  \nowrap{$f(x) = 3^x$ } &  \nowrap{ $g(x) = 2^{-x} = (\frac{1}{2})^{x} $ }
\end{table}

b) Wie groß muss ihr Papier sein, damit sie bei 1 cm als Längeneinheit die Funktion $f$
im Intervall $[-5, 5]$ bzw. im Intervall $ [-10,10] $ zeichnen können?
}

\lang{de}{In der Abbildung \lref{Exponentialfunktionen_6_4}{Abb 1} sind in
orangener, grüner, blauer und violetter Farbe die Graphen von Exponentialfunktionen
dargestellt.}
\lang{en}{In \lref{Exponentialfunktionen_6_4}{Figure 1} there are four graphs of exponential functions in orange, violet, blue, and green.}
\\
\label{Exponentialfunktionen_6_4}

\begin{figure} 
	\image{T104_Exercise05}
	\caption{
	\lang{de}{Abb 1: Exponentialfunktionen }
	\lang{en}{Figure 1: Exponential functions}
	}
\end{figure}

\begin{table}[\class{items}]
	\nowrap{ $g(x)= \exp(x)$}\\
	\nowrap{ $h(x)= \exp(-x)$}\\ 
	\nowrap{ $j(x)= 5^x$}\\ 
	\nowrap{ $k(x)= 9^{-x}$}\\ 
	\lang{de}{Ordnen Sie die vier Graphen jeweils der zugehörigen Funktionsvorschrift zu.}
	\lang{en}{Match each of the four graphs to their corresponding function.}
	\\
\end{table}

\begin{tabs*}[\initialtab{0}\class{exercise}]

 \tab{\lang{de}{Lösungsvideos zu a) und b)}}	
   \youtubevideo[500][300]{wAmrpA74T_Q}\\

\tab{\lang{de}{Zuordnung der Graphen} \lang{en}{Labels of the graphs}}
	\begin{table}[\class{items}]
	    a) \textcolor{#CC6600}{$h(x)$}: 
	    \lang{de}{Der orangene Graph gehört zur Funktion $h(x)= \exp(-x)$.}
	    \lang{en}{The orange graph belongs to the function $h(x)= \exp(-x)$.}
	    \\
	    b) \textcolor{#00CC00}{$k(x)$}:
	    \lang{de}{Der grüne Graph gehört zur Funktion $k(x)= 9^{-x}$.}
	    \lang{en}{The green graph belongs to the function $k(x)= 9^{-x}$.}
	    \\
        c) \textcolor{#0066CC}{$j(x)$}:
	    \lang{de}{Der blaue Graph gehört zur Funktion $j(x)= 5^x$.}
	    \lang{en}{The blue graph belongs to the function $j(x)= 5^x$.}
	    \\
	    d) \textcolor{#CC00CC}{$g(x)$}:
	    \lang{de}{Der violette Graph gehört zur Funktion $g(x)= \exp(x)$.}
	    \lang{en}{The violet graph belongs to the function $g(x)= \exp(x)$.}
	    
	\end{table}

\tab{\lang{de}{Orangener Graph}\lang{en}{Orange Graph}}
	\begin{incremental}[\initialsteps{1}]
	\step 
	    \lang{de}{Die Funktion $h(x)= \exp(-x)$ ist die gespiegelte Exponentialfunktion.}
	    \lang{en}{The function $h(x)= \exp(-x)$ is a mirrored exponential function.}
	\step 
	    \lang{de}{Sie verläuft streng monoton fallend.}
	    \lang{en}{It is strictly monotonically decreasing.}
	\step 
	    \lang{de}{In diesem Fall gilt $h(0)=1$ und hier $h(-1)=e\approx 2,7181$. Dies zeigt, dass der orangene Graph mit der Funktion $h(x)$ übereinstimmt.}
	    \lang{en}{For all exponential functions $h(0)=1$, and in this case $h(-1)=e\approx 2.7181$.}  
	\end{incremental}

\tab{\lang{de}{Grüner Graph}\lang{en}{Green Graph}}
	\begin{incremental}[\initialsteps{1}]
	\step 
	    \lang{de}{Die Funktion $k(x)= 9^{-x}$ entspricht dem grünen Graphen.}
	    \lang{en}{The function $k(x)= 9^{-x}$ corresponds to the green graph.}
	\step 
	    \lang{de}{Sie verläuft streng monoton fallend.}
	    \lang{en}{It is strictly monotonically decreasing.}
	\step 
	    \lang{de}{In diesem Fall gilt $k(0)=1$ und hier $k(-1)=9$. Dies zeigt, dass der grüne Graph mit der Funktion $k(x)$ übereinstimmt.}
	    \lang{en}{For all exponential functions $k(0)=1$, and in this case $k(-1)=9$.}
	\end{incremental}

\tab{\lang{de}{Blauer Graph}\lang{en}{Blue Graph}}
	\begin{incremental}[\initialsteps{1}]
	\step 
	    \lang{de}{Die Funktion $j(x)= 5^x$ entspricht dem blauen Graphen.}
	    \lang{en}{The function $j(x)= 5^x$ corresponds to the blue graph.}
	\step 
	    \lang{de}{Sie verläuft streng monoton steigend.}
	    \lang{en}{It is strictly monotonically increasing.}
	\step 
	    \lang{de}{In diesem Fall gilt $j(0)=1$ und hier $j(1)=5$.}
	    \lang{en}{For all exponential functions $j(0)=1$, and in this case $j(1)=5$. Dies zeigt, dass der blaue Graph mit der Funktion $j(x)$ übereinstimmt.}
	\end{incremental}

\tab{\lang{de}{Violetter Graph}\lang{en}{Violet Graph}}
	\begin{incremental}[\initialsteps{1}]
	\step 
	    \lang{de}{Die Funktion $g(x)= \exp(x)$ ist die klassische Exponentialfunktion.}
	    \lang{en}{The function $g(x)= \exp(x)$ is the classic exponential function.}
	\step 
	    \lang{de}{Sie verläuft streng monoton steigend.}
	    \lang{en}{It is strictly monotonically increasing.}
	\step 
	    \lang{de}{In diesem Fall gilt $g(0)=1$ und hier $g(1)=e\approx 2,7181$. Dies zeigt, dass der violette Graph mit der Funktion $g(x)$ übereinstimmt.}
	    \lang{en}{For all exponential functions $g(0)=1$, and in this case $g(1)=e\approx 2.7181$.}
	\end{incremental}

  


\end{tabs*}

\end{content}