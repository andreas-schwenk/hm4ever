\documentclass{mumie.element.exercise}
%$Id$
\begin{metainfo}
  \name{
    \lang{de}{Ü03: Potenzgesetze}
    \lang{en}{Exercise 3}
  }
  \begin{description} 
 This work is licensed under the Creative Commons License Attribution 4.0 International (CC-BY 4.0)   
 https://creativecommons.org/licenses/by/4.0/legalcode 

    \lang{de}{Berechnung von Potenzen}
    \lang{en}{}
  \end{description}
  \begin{components}
  \end{components}
  \begin{links}
  \end{links}
  \creategeneric
\end{metainfo}
\begin{content}
\usepackage{mumie.ombplus}

\title{
  \lang{de}{Ü03: Potenzgesetze}
  \lang{en}{Exercise 3}
}

\begin{block}[annotation]
  Berechnung von Potenzen
     
\end{block}
\begin{block}[annotation]
  Im Ticket-System: \href{http://team.mumie.net/issues/9098}{Ticket 9098}
\end{block}

\lang{de}{Welche der folgenden Aussagen sind richtig?
\begin{table}[\class{items}]
  \nowrap{a) $2^3 \cdot 2^3 = 4^3$ } &  \nowrap{b) $(\frac{1}{2})^4 = (-2)^4$ }\\
    \nowrap{c) $\frac{6^10}{6^2} = 6^5$ } &  \nowrap{d) $3^4 \cdot 3^5 = 3^20$ }\\
   \nowrap{e) $ 5^3\cdot 5^3 = 5^9 $} & \nowrap{f) $ 4^3 \cdot 5^3 = 20^3 $} \\
   \nowrap{g) $ \frac{6^3 }{2^3} = 3^3 $} & \nowrap{h) $ (3^4)^5 = (3^5)^4 $}
\end{table}
}

\lang{de}{Berechnen Sie:}
\lang{en}{Calculate:}
\begin{table}[\class{items}]
  \nowrap{i) $2^{12}\cdot 2^{-11}$ } &  \nowrap{j) $4^{3}\cdot 16^{-2}\cdot \left(\frac{1}{2}\right)^{-2}$ }\\
    \nowrap{k) $\frac{3^{11}}{3^{-5}}\cdot (3^3)^{-5}$ } &  \nowrap{l) $\frac{3^{(3^3)}}{(3^3)^3}$ }
\end{table}

\begin{tabs*}[\initialtab{0}\class{exercise}]
  \tab{
  \lang{de}{Antwort}
  \lang{en}{Answer}
  }
\begin{table}[\class{items}]

    \nowrap{i) $2$} & \nowrap{j) $1$} \\
    \nowrap{k) $3$} &  \nowrap{l) $3^18$}
  \end{table}
  
    \tab{\lang{de}{Lösungsvideos a) - h)}}	
    \youtubevideo[500][300]{Jg2GbKPWfjA}\\

  
  \tab{
  \lang{de}{Lösung i)}
  \lang{en}{Solution i)}}
  
  \begin{incremental}[\initialsteps{1}]
    \step 
    \lang{de}{Das Produkt zweier Potenzen mit gleicher Basis lässt sich zu einer Potenz vereinfachen, also}
    \lang{en}{The product of two powers with the same base can be simplified to one power:}
    \begin{align*}2^{12}\cdot 2^{-11}=2^{12-11}=2^1=2\,.
    \end{align*}
     
  \end{incremental}
  
   \tab{
  \lang{de}{Lösung j)}
  \lang{en}{Solution j)}}
  
  \begin{incremental}[\initialsteps{1}]
    \step 
    \lang{de}{Wir schreiben zuerst die drei Faktoren als Potenzen mit der gemeinsamen Basis $2$ und erhalten}
    \lang{en}{First we write the three factors as powers with a common base $2$, and obtain}
    \begin{equation*}\textcolor{#0066CC}{4^{3}} = (2^2)^3 = 2^{2\cdot 3} = \textcolor{#0066CC}{2^6}\,, \quad \textcolor{#CC6600}{16^{-2}} = (2^4)^{-2} = \textcolor{#CC6600}{2^{-8}}\,, \quad \textcolor{#00CC00}{\left(\frac{1}{2}\right)^{-2}} = (2^{-1})^{-2} = 2^{(-1)(-2)} = \textcolor{#00CC00}{2^2}\,.
    \end{equation*}
    \lang{de}{Das Produkt kann nun vereinfacht werden:}
    \lang{en}{The product can now be simplified:}
    \begin{equation*}
    	\textcolor{#0066CC}{4^{3}} \cdot \textcolor{#CC6600}{16^{-2}} \cdot \textcolor{#00CC00}{\left(\frac{1}{2}\right)^{-2}} = \textcolor{#0066CC}{2^6} \cdot \textcolor{#CC6600}{2^{-8}} \cdot \textcolor{#00CC00}{2^2} = 2^{6-8+2} = 2^0 = 1\,.
    \end{equation*}
     
  \end{incremental}

 \tab{
  \lang{de}{Lösung k)}
  \lang{en}{Solution k)}}
  
  \begin{incremental}[\initialsteps{1}]
    \step 
    \lang{de}{Erneut vereinfachen wir den Term durch die Anwendung der Rechenregeln für Potenzen. Wir erhalten}
    \lang{en}{Again, we simplify the expression by applying the rules for powers:}
   
   \begin{equation*}\frac{3^{11}}{3^{-5}}\cdot \textcolor{#CC6600}{(3^3)^{-5}} = \frac{3^{11}}{3^{-5}}\cdot \textcolor{#CC6600}{3^{3\cdot (-5)}}
   =\frac{3^{11}}{3^{-5}}\cdot \textcolor{#CC6600}{3^{-15}} = \frac{\textcolor{#0066CC}{3^{11}\cdot 3^{-15}}}{3^{-5}}=\frac{\textcolor{#0066CC}{3^{-4}}}{3^{-5}} = 3^{-4-(-5)} = 3^1 = 3\,.
    \end{equation*}
  \end{incremental}
  
  \tab{
  \lang{de}{Lösung l)}
  \lang{en}{Solution l)}}
  
  \begin{incremental}[\initialsteps{1}]
    \step 
    \lang{de}{Wir berechnen}
    \lang{en}{We calculate:}
    \begin{equation*}
    	\frac{3^{(\textcolor{#0066CC}{3^3})}}{\textcolor{#CC6600}{(3^3)^3}} = \frac{3^{\textcolor{#0066CC}{27}}}{\textcolor{#CC6600}{3^{3\cdot 3}}} = \frac{3^27}{3^9} = 3^18\,.
\end{equation*}
    ($3^{18} = 387\,420\,489$)
  \end{incremental}
  
 
\end{tabs*}
\end{content}