\documentclass{mumie.element.exercise}
%$Id$
\begin{metainfo}
  \name{
    \lang{de}{Ü07: Potenzgesetze}
    \lang{en}{Exercise 7}
  }
  \begin{description} 
 This work is licensed under the Creative Commons License Attribution 4.0 International (CC-BY 4.0)   
 https://creativecommons.org/licenses/by/4.0/legalcode 

    \lang{de}{Umformen von Wurzeln und Potenzen}
    \lang{en}{}
  \end{description}
  \begin{components}
  \end{components}
  \begin{links}
  \end{links}
  \creategeneric
\end{metainfo}
\begin{content}
\usepackage{mumie.ombplus}

\title{
  \lang{de}{Ü07: Potenzgesetze}
}

\begin{block}[annotation]
  Umformen von Wurzeln und Potenzen
     
\end{block}
\begin{block}[annotation]
  Im Ticket-System: \href{http://team.mumie.net/issues/9283}{Ticket 9283}
\end{block}

\lang{de}{Schreiben Sie die folgenden Ausdr\"{u}cke als Potenzen:}
\lang{en}{Rewrite the following expressions as powers:}
\begin{table}[\class{items}]
  \nowrap{a) $\sqrt{2}$ } &  \nowrap{b) $\sqrt{7^5}$ }\\
    \nowrap{c) $(\sqrt[3]{3})^4$ } &  \nowrap{d) $\sqrt{\frac{2}{3}} \cdot \left(\frac{2}{3}\right)^4$ }
\end{table}

\begin{tabs*}[\initialtab{0}\class{exercise}]
  \tab{
  \lang{de}{Antwort}
  \lang{en}{Answer}
  }
\begin{table}[\class{items}]

    \nowrap{a) $2^{\frac{1}{2}}$} & \nowrap{b) $7^{\frac{5}{2}}$} \\
    \nowrap{c) $3^{\frac{4}{3}}$} &  \nowrap{d) $\left(\frac{2}{3}\right)^{\frac{9}{2}}$}
  \end{table}
  
  \tab{
  \lang{de}{Lösung a)}
  \lang{en}{Solution a)}}
  
  \begin{incremental}[\initialsteps{1}]
    \step 
    \lang{de}{Nach Definition ist}
    \lang{en}{By definition}
    \[
    	\sqrt{2} = 2^{\frac{1}{2}}\,.
    \]
    
  \end{incremental}

 \tab{
  \lang{de}{Lösung b)}
  \lang{en}{Solution b)}}
  
  \begin{incremental}[\initialsteps{1}]
    \step 
    \lang{de}{Der Ausdruck $\sqrt{7^5}$ kann als $(7^5)^{\frac{1}{2}}$ 
    geschrieben werden. Wir erhalten mit den Rechenregeln
    f\"{u}r Potenzen: $7^{5\cdot \frac{1}{2}}=7^{\frac{5}{2}}$.}
    \lang{en}{The expression $\sqrt{7^5}$ can be written as $(7^5)^{\frac{1}{2}}$.
    Using the rules for powers we obtain: $7^{5\cdot \frac{1}{2}}=7^{\frac{5}{2}}$.}
    
    
  \end{incremental}
  
  \tab{
  \lang{de}{Lösung c)}
  \lang{en}{Solution c)}}
  
  \begin{incremental}[\initialsteps{1}]
    \step 
    \lang{de}{$\sqrt[3]{3}$ ist per Definition $3^{\frac{1}{3}}$. Also ist $(\sqrt[3]{3})^4
    =\left(3^{\frac{1}{3}}\right)^4=3^{\frac{4}{3}}\,.$}
    \lang{en}{$\sqrt[3]{3}$ is by definition $3^{\frac{1}{3}}$. Therefore $(\sqrt[3]{3})^4
    =\left(3^{\frac{1}{3}}\right)^4=3^{\frac{4}{3}}\,.$}
    
     
  \end{incremental}
  
  \tab{
  \lang{de}{Lösung d)}
  \lang{en}{Solution d)}}
  
  \begin{incremental}[\initialsteps{1}]
    \step
    \lang{de}{Wir schreiben $\sqrt{\frac{2}{3}}$ als $\left(\frac{2}{3}\right)^{\frac{1}{2}}$ und berechnen}
    \lang{en}{We rewrite $\sqrt{\frac{2}{3}}$ as $\left(\frac{2}{3}\right)^{\frac{1}{2}}$ and calculate}
    
    \begin{equation*}
    	\sqrt{\frac{2}{3}} \cdot \left(\frac{2}{3}\right)^4 = \left(\frac{2}{3}\right)^{\textcolor{#0066CC}{\frac{1}{2}}} \cdot \left(\frac{2}{3}\right)^{\textcolor{#0066CC}{4}}
    	= \left(\frac{2}{3}\right)^{\textcolor{#0066CC}{\frac{1}{2}+4}} = \left(\frac{2}{3}\right)^{\frac{1}{2}+\frac{8}{2}} = \left(\frac{2}{3}\right)^{\frac{9}{2}}\,.
    \end{equation*}
     
  \end{incremental}

\end{tabs*}

\end{content}