\documentclass{mumie.element.exercise}
%$Id$
\begin{metainfo}
  \name{
    \lang{de}{Ü02: Potenzgesetze}
    \lang{en}{Exercise 2}
  }
  \begin{description} 
 This work is licensed under the Creative Commons License Attribution 4.0 International (CC-BY 4.0)   
 https://creativecommons.org/licenses/by/4.0/legalcode 

    \lang{de}{Darstellung von Zahlen als Zweierpotenzen}
    \lang{en}{}
  \end{description}
  \begin{components}
  \end{components}
  \begin{links}
  \end{links}
  \creategeneric
\end{metainfo}
\begin{content}
\usepackage{mumie.ombplus}

\title{
  \lang{de}{Ü02: Potenzgesetze}
  \lang{en}{Exercise 2}
}

\begin{block}[annotation]
  Darstellung von Zahlen als Zweierpotenzen
     
\end{block}
\begin{block}[annotation]
  Im Ticket-System: \href{http://team.mumie.net/issues/9097}{Ticket 9097}
\end{block}

 \lang{de}{Geben Sie Zahlen a,b,c > 0 an, für die gilt:
 
 \nowrap{a) $a^b \cdot a^c = a^{b \cdot c}$}
 \nowrap{b) $ (a^b)^c = a^{(b^c)} $}}
 

\lang{de}{Schreiben Sie folgenden Ausdr\"{u}cke als Potenz mit der Basis $2$:
 }

\lang{en}{Write the following expressions as powers with base $2$:}
\lang{de}{
\begin{table}[\class{items}]
  \nowrap{c) $32\cdot 64$ } &  \nowrap{d) $0{,\!}125$ }\\
    \nowrap{e) $\frac{2^9}{8^2}$ } &  \nowrap{f) $1$ }
\end{table}}
\lang{en}{
\begin{table}[\class{items}]
  \nowrap{c) $32\cdot 64$ } &  \nowrap{d) $0{.\!}125$ }\\
    \nowrap{e) $\frac{2^9}{8^2}$ } &  \nowrap{f) $1$ }
\end{table}}

\begin{tabs*}[\initialtab{0}\class{exercise}]
  \tab{
  \lang{de}{Antwort}\lang{en}{Answer}
  }
\begin{table}[\class{items}]

    \nowrap{c) $2^11$} & \nowrap{d) $2^{-3}$} \\
    \nowrap{e) $2^{3}$} &  \nowrap{f) $2^0$}
  \end{table}
  
    \tab{\lang{de}{Lösungsvideos a) und b)}}	
    \youtubevideo[500][300]{guYFGrHQbc0}\\
  
  \tab{
  \lang{de}{Lösung c)}
  \lang{en}{Solution c)}}
  
  \begin{incremental}[\initialsteps{1}]
    \step 
    \lang{de}{Wir schreiben die einzelnen Faktoren des Produkts als Zweierpotenzen, bevor wir sie multiplizieren. Dadurch ergibt sich}
    \lang{en}{We write the individual factors of the multiplication as powers of two, then we multiply them. Thus we obtain}
    \begin{align*}32 \cdot 64 = 2^5 \cdot 2^6 = 2^{11}\,.
    \end{align*}
  \end{incremental}
  
   \tab{
  \lang{de}{Lösung d)}
  \lang{en}{Soution d)}}
  
  \begin{incremental}[\initialsteps{1}]
    \step 
    \lang{de}{Es ist \[0{,\!}125 = \frac{125}{1000} = \frac{\textcolor{#0066CC}{125}}{\textcolor{#0066CC}{125} \cdot 8} = \frac{1}{8} = \frac{1}{2^3} = 2^{-3}\,.\]}
    \lang{en}{It is \[0.125 = \frac{125}{1000} = \frac{\textcolor{#0066CC}{125}}{\textcolor{#0066CC}{125} \cdot 8} = \frac{1}{8} = \frac{1}{2^3} = 2^{-3}\,.\]}
  \end{incremental}

 \tab{
  \lang{de}{Lösung e)}
  \lang{en}{Solution e)}}
  \begin{incremental}[\initialsteps{1}]
    \step 
    \lang{de}{Mithilfe der Rechenregeln f\"{u}r Potenzen k\"{o}nnen wir den Ausdruck umschreiben:}
    \lang{en}{With the help of the rules for powers, we can rewrite the expression:}
    \begin{equation*}
		\frac{2^9}{\textcolor{#0066CC}{8^2}} = \frac{2^9}{(\textcolor{#0066CC}{2^3})^2} = \frac{2^9}{2^{3\cdot 2}} = \frac{2^9}{2^6} = 2^9 \cdot \textcolor{#CC6600}{\frac{1}{2^6}} = 2^9 \cdot \textcolor{#CC6600}{2^{-6}} = 2^{9-6}=2^3\,.
    \end{equation*}
     
  \end{incremental}
  
  \tab{
  \lang{de}{Lösung f)}
  \lang{en}{Solution f)}}
  \begin{incremental}[\initialsteps{1}]
    \step 
    \lang{de}{Es gilt $x^0=1$ für alle reellen Zahlen $x$. Eine Potenz mit dem Exponenten $0$ ergibt also immer $1$. Die Anwort lautet daher $1=2^0$.}
    \lang{en}{For all real numbers $x^0=1$, an exponential with exponent $0$ is always equal to $1$. Thus the answer is $1=2^0$.}
   
  \end{incremental}
 


\end{tabs*}
\end{content}