\documentclass{mumie.element.exercise}
%$Id$
\begin{metainfo}
  \name{
    \lang{de}{Ü06: Exponentialgleichung}
    \lang{en}{Exercise 6}
  }
  \begin{description} 
 This work is licensed under the Creative Commons License Attribution 4.0 International (CC-BY 4.0)   
 https://creativecommons.org/licenses/by/4.0/legalcode 

    \lang{de}{Hier die Beschreibung}
    \lang{en}{}
  \end{description}
  \begin{components}
  \end{components}
  \begin{links}
  \end{links}
  \creategeneric
\end{metainfo}
\begin{content}
\title{
	\lang{de}{Ü06: Exponentialgleichung}
	\lang{en}{Exercise 6} 														
}

\begin{block}[annotation]
    Übungsaufgabe zur L"osung von Exponentialgleichungen					 
\end{block}
\begin{block}[annotation]
	Im Ticket-System: \href{http://team.mumie.net/issues/9101}{Ticket 9101}   
\end{block}

\lang{de}{Lösen Sie die folgenden Gleichungen nach $x$ auf:}
\lang{en}{Solve the following equations for $x$.}

\begin{table}[\class{items}]
	& \nowrap{a) $2^{x}=8$} \\
	& \nowrap{b) $e^{x}=5$} \\
	& \nowrap{c)  $2^{x^{2}}=4^{x}-2^{x^{2}}$} \\    
\end{table}

\begin{tabs*}[\initialtab{0}\class{exercise}]								
\tab{\lang{de}{Antwort}\lang{en}{Answer}}
	\begin{table}[\class{items}]
		& \nowrap{a) $x=3$} \\
		& \nowrap{b) $x= \ln 5$} \\
		& \nowrap{c)  $x=1$} \\   
	\end{table}
 
\tab{\lang{de}{L"osung a)}\lang{en}{Solution a)}}
	%begin-cosh 
% 	\begin{incremental}[\initialsteps{1}] 										
% 	\step
% 	 	\lang{de}{Laut Definition ist}
% 		\lang{en}{By definition:}
% 		\[
% 		\lang{de}{x=\log_{2} 8= \log_{2} 2^{3}=3\,.}
% 		\lang{en}{x=\log_{2} 8= \log_{2} 2^{3}=3\,}
% 		\]
% 	\end{incremental}     
	\begin{incremental}[\initialsteps{1}] 										
	\step
		\lang{de}{Wende den natürlichen Logarithmus an. Nach den Rechenregeln für Logarithmen ist:}
		\lang{en}{Use the natural logarithm. Using the rules for calculating with logarithms:}
		\begin{eqnarray*}
		\lang{de}{
		& & 2^x = 8 \\
		&\iff& \ln(2^x) = \ln(2^3) \\
		&\iff& x \ln(2) = 3 \ln(2)  \\
		&\iff& x = 3\,.}
		\lang{en}{
		& & 2^x = 8 \\
		&\iff& \ln(2^x) = \ln(2^3) \\
		&\iff& x \ln(2) = 3 \ln(2)  \\
		&\iff& x = 3\,}
		\end{eqnarray*}
	\end{incremental}     
%end-cosh

\tab{\lang{de}{L"osung b)}\lang{en}{Solution b)}}
	\begin{incremental}[\initialsteps{1}] 										
%begin-cosh
%	\step 
%		\lang{de}{Der natürliche Logarithmus von $u>0$ ist definiert durch $\ln u=\log_{e}u.$}
%		\lang{en}{The natural logarithm of $u>0$ is defined by $\ln u=\log_{e}u.$}
	\step
		\lang{de}{%Wendet man dies auf die Gleichung $e^{x}=5$ an, so erhält man 
		Anwenden des natürlichen Logarithmus ergibt direkt}
		\lang{en}{Using the natural logarithm gives:}
		\[  
		\lang{de}{\ln (e^{x})=\ln (5) \Leftrightarrow x=\ln (5)\,.}
		\lang{en}{\ln (e^{x})=\ln (5) \Leftrightarrow x=\ln (5)\,}
		\]
%end-cosh
	\end{incremental}     

\tab{\lang{de}{L"osung c)}\lang{en}{Solution c)}}
	\begin{incremental}[\initialsteps{1}] 										
	\step
	    \lang{de}{Man forme zuerst die Gleichung wie folgt um}
	    \lang{en}{First, transform the equation slightly:}
	\step
		\begin{eqnarray*}
			\lang{de}{& & 2^{x^{2}}=4^{x}-2^{x^{2}} \\
			& \Leftrightarrow & 2^{x^{2}}+2^{x^{2}}= 4^{x} \\
			& \Leftrightarrow & 2 \cdot 2^{x^{2}} = (2^2)^{x} \\
			& \Leftrightarrow & 2 \cdot 2^{x^{2}} = 2^{2x} \\
			& \Leftrightarrow & 2 \cdot 2^{x^{2}} : 2^{2x} = 1 \\
			& \Leftrightarrow & 2^1 \cdot 2^{x^{2}}\cdot 2^{-2x}=1\\
			& \Leftrightarrow & 2^{x^{2}-2x+1}=1. }
			\lang{en}{& & 2^{x^{2}}=4^{x}-2^{x^{2}} \\
			& \Leftrightarrow & 2^{x^{2}}+2^{x^{2}}= 4^{x} \\
			& \Leftrightarrow & 2 \cdot 2^{x^{2}} = (2^2)^{x} \\
			& \Leftrightarrow & 2 \cdot 2^{x^{2}} = 2^{2x} \\
			& \Leftrightarrow & 2 \cdot 2^{x^{2}} / 2^{2x} = 1 \\
			& \Leftrightarrow & 2^1 \cdot 2^{x^{2}}\cdot 2^{-2x}=1\\
			& \Leftrightarrow & 2^{x^{2}-2x+1}=1}
		\end{eqnarray*} 
	\step
	    \lang{de}{Mit den Rechengesetzen für die Exponentialfunktion erhält man somit}
	    \lang{en}{Using the rules for calculating with exponential functions, we get:}
		\[
		\lang{de}{2^{x^{2}-2x+1}=2^0 \Leftrightarrow x^{2}-2x+1 = 0 \Leftrightarrow (x-1)^{2}=0 \Leftrightarrow x=1,}
		\lang{en}{2^{x^{2}-2x+1}=2^0 \Leftrightarrow x^{2}-2x+1 = 0 \Leftrightarrow (x-1)^{2}=0 \Leftrightarrow x=1}
		\]
	    \lang{de}{wobei hier die zweite binomische Formel verwendet wurde.}
	    \lang{en}{which was obtained by using the second binomial formula.}
	\step
	    \lang{de}{Somit lautet die Lösung der Gleichung $x=1$.}
	    \lang{en}{The solution of the equation is $x=1$.} 
	\end{incremental}     

\end{tabs*}  
 
\end{content}