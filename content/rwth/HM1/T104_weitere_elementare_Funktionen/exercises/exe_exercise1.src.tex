\documentclass{mumie.element.exercise}
%$Id$
\begin{metainfo}
  \name{
    \lang{de}{Ü01: Wurzelgleichungen}
    \lang{en}{Exercise 1}
  }
  \begin{description} 
 This work is licensed under the Creative Commons License Attribution 4.0 International (CC-BY 4.0)   
 https://creativecommons.org/licenses/by/4.0/legalcode 

    \lang{de}{Hier die Beschreibung}
    \lang{en}{}
  \end{description}
  \begin{components}
  \end{components}
  \begin{links}
  \end{links}
  \creategeneric
\end{metainfo}
\begin{content}
\title{
	\lang{de}{Ü01: Wurzelgleichungen}
}

\begin{block}[annotation]
	Definitionsbereiche von Funktionen
\end{block}
\begin{block}[annotation]
	Im Ticket-System: \href{http://team.mumie.net/issues/9094}{Ticket 9094}
\end{block}

Bestimmen Sie die Lösungen folgender Gleichungen:

\begin{enumerate}[a)]
\item a) $\sqrt{2 + 3x} = 2$.
\item b) $\sqrt{x-2} = \frac{1}{3}x$.
\item c) $ \sqrt{1-x} = x-2  $.
\item d) $ \sqrt{32-16x} = x-5 $.
\item e) $ \sqrt{x+2} = x  $.
\item f) $ \sqrt{8-4x} = x-3  $.
\end{enumerate}

Betrachten Sie die Funktion 
\[f: D_f \to \R ; \, f(x)=\sqrt{3x+4}.\]
\begin{enumerate}[a)]
 \item g) Bestimmen Sie den maximalen Definitionsbereich $D_f$ von $f$. 
 \item h) Hat die Funktion $f$ eine Nullstelle, wenn ja wo?
 \item i) Betrachten Sie die Gleichung $f(x)=x$ und geben Sie alle Lösungen an.
\end{enumerate}
\\

\begin{tabs*}[\initialtab{0}\class{exercise}]
\tab{\lang{de}{Antwort}\lang{en}{Answer}}
	\\
    a) $\mathbb{L} = \{\frac{2}{3} \}$\\
    b) $\mathbb{L} = \{3; 6\}$\\
    c) $\mathbb{L} = \{\}$\\
    d) $\mathbb{L} = \{\}$\\
    e) $\mathbb{L} = \{2\}$\\
    f) $\mathbb{L} = \{\}$\\
    g) $D_f= \{ x\in \R ~\vert~ x\ge -\frac{4}{3} \}$\\
    h) ja, $x_0= -\frac{4}{3}$\\
    i) $\mathbb{L} = \{4\}$\\
    
         \tab{\lang{de}{Lösungsvideos a) - f)}}	
    \youtubevideo[500][300]{8dwopEhKXh4}\\
    
    
\tab{\lang{de}{Lösung g)}\lang{en}{Solution a)}}
	\begin{incremental}[\initialsteps{1}]
    \step 
	   Nach Definition wird durch die Wurzelfunktion eine Abbildung definiert auf $[0,\infty)$. Wir 
dürfen also unter der Wurzel nur Werte größer oder gleich $0$ einsetzen. 

    \step
    Dazu berechnen wir
$3x+4\ge 0$ was äquivalent ist zu $x\ge -\frac{4}{3}$.
    \step
     Damit gilt also für den maximalen 
Definitionsbereich $D_f= \{x\in \R ~\vert~ x\ge -\frac{4}{3}\}$. 
  	\end{incremental}

\tab{\lang{de}{Lösung h)}\lang{en}{Solution b)}}
    \begin{incremental}[\initialsteps{1}]
    \step
	     Es ist $\sqrt{x}=0$ genau dann, wenn $x=0$ gilt. 
	     
    \step
    Also gilt
\[\sqrt{3x+4}=0 \Leftrightarrow 3x+4=0 \Leftrightarrow x=-\frac{4}{3}.  \]
	
    \end{incremental}

\tab{\lang{de}{Lösung i)}\lang{en}{Solution c)}}
    \begin{incremental}[\initialsteps{1}]
    \step
	    Die Gleichung $\sqrt{3x+4}=x$ wird durch Quadrieren beider Seiten zu $3x+4=x^2$, was 
gleichbedeutend zum Nullstellenproblem $x^2-3x-4=0$ ist. 

    \step
    Mit Hilfe der $p-q-$Formel 
erhält man die Lösungen $x=4$ oder $x=-1$. Da beim Quadrieren Lösungen dazu kommen können,
müssen wir die beiden Lösungen noch testen. 

    \step
    Eine Probe ergibt, dass für $x=4$ genau
\[\sqrt{3\cdot 4+4}= \sqrt{16}= 4,  \]

    \step
wie gefordert gilt, während für $x=-1$ der Widerspruch 
\[\sqrt{3\cdot (-1) +4} = \sqrt{1} =1 \neq -1\] entsteht. Also erfüllt nur $x=4$ die 
ursprüngliche Gleichung und somit ist $\mathbb{L} = \{4\}$.
    \end{incremental}
    
 
 
    
\end{tabs*}


\end{content}