%$Id:  $
\documentclass{mumie.article}
%$Id$
\begin{metainfo}
  \name{
    \lang{de}{Überblick: Orientierung  im $n$-dimensionalen Raum}
    \lang{en}{overview: }
  }
  \begin{description} 
 This work is licensed under the Creative Commons License Attribution 4.0 International (CC-BY 4.0)   
 https://creativecommons.org/licenses/by/4.0/legalcode 

    \lang{de}{Beschreibung}
    \lang{en}{}
  \end{description}
  \begin{components}
  \end{components}
  \begin{links}
\link{generic_article}{content/rwth/HM1/T501_Orientierung_im_n-dim_Raum/g_art_content_52_Abstaende.meta.xml}{content_52_Abstaende}
\link{generic_article}{content/rwth/HM1/T501_Orientierung_im_n-dim_Raum/g_art_content_51_Kurven.meta.xml}{content_51_Kurven}
\end{links}
  \creategeneric
\end{metainfo}
\begin{content}
\begin{block}[annotation]
	Im Ticket-System: \href{https://team.mumie.net/issues/30096}{Ticket 30096}
\end{block}
\begin{block}[annotation]
Im Entstehen: Überblicksseite für Kapitel Orientierung im $n$-dimensionalen Raum
\end{block}

\usepackage{mumie.ombplus}
\ombchapter{1}
\lang{de}{\title{Überblick: Orientierung im $n$-dimensionalen Raum}}
\lang{en}{\title{}}



\begin{block}[info-box]
\lang{de}{\strong{Inhalt}}
\lang{en}{\strong{Contents}}


\lang{de}{
    \begin{enumerate}%[arabic chapter-overview]
   \item[1.1] \link{content_51_Kurven}{Kurven im $n$-dimensionalen Raum}
   \item[1.2] \link{content_52_Abstaende}{Abstände, Normen, Konvergenz}
  \end{enumerate}
} %lang

\end{block}

\begin{zusammenfassung}

\lang{de}{Wir beginnen damit, den $n$-dimensionalen reellen Raum $\R^n$ durch Spaziergänge zu erkunden. 
Zu jeder Zeit $t$ geben wir den Ort $\phi(t)\in\R^n$ an, an dem wir uns befinden. Das führt uns zu dem Begriff der parametrisierten Kurve.
Ändern wir unser Tempo oder kehren wir gar um, dann ändert sich zwar der Ort-Zeit-Zusammenhang, die sogenannte Parametrisierung unseres Spaziergangs,
aber nicht die von uns hinterlassene Spur im Raum.

Weiter interessieren wir uns für die Entfernung, z.B. von unserem Zielpunkt. 
Es gibt mehrere Möglichkeiten, diesen Abstand zu messen. Die gängigsten benutzen dazu Normen, von denen es wiederum viele gibt.
Während wir bezüglich verschiedener Messweisen unterschiedliche Abstände zu unserem Zielpunkt haben können, so gilt doch:
Nähern wir uns in einer Messweise immer mehr unserem Ziel an, dann tun wir das auch in jeder anderen Messweise.
\\
Mathematisch vornehm ausgedrückt studieren wir verschiedene Normen und ihre Eigenschaften und zeigen, dass der Konvergenzbegriff im $\R^n$
unabhängig von der Wahl der Norm ist.
}


\end{zusammenfassung}

\begin{block}[info]\lang{de}{\strong{Lernziele}}
\lang{en}{\strong{Learning Goals}} 
\begin{itemize}[square]
\item \lang{de}{Sie kennen die Begriffe parametrisierte Kurve, Spur, Umparametrisierung und können sie in einfachen Situationen berechnen.}
\item \lang{de}{Sie kennen den Begriff der Norm, deren Eigenschaften und die wichtigsten Beispiele.}
\item \lang{de}{Sie können verschiedene Normen berechnen.}
\item \lang{de}{Sie kennen den Begriff der Folgenkonvergenz im $\R^n$ und prüfen Folgen auf Konvergenz.}
\item \lang{de}{Sie wissen, was die Aussage \emph{\glqq Im $\:\R^n$ sind alle Normen äquivalent.\grqq{}$\:$} bedeutet.}
\item \lang{de}{Sie kennen und erkennen offene und abgeschlossene Teilmengen. }
\end{itemize}

\end{block}




\end{content}
