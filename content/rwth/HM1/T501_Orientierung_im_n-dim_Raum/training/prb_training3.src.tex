\documentclass{mumie.problem.gwtmathlet}
%$Id$
\begin{metainfo}
  \name{
    \lang{de}{A03: offen/abgeschlossen}
    \lang{en}{Problem 3}
  }
  \begin{description} 
 This work is licensed under the Creative Commons License Attribution 4.0 International (CC-BY 4.0)   
 https://creativecommons.org/licenses/by/4.0/legalcode 

    \lang{de}{Beschreibung}
    \lang{en}{}
  \end{description}
  \corrector{system/problem/GenericCorrector.meta.xml}
  \begin{components}
    \component{js_lib}{system/problem/GenericMathlet.meta.xml}{mathlet}
  \end{components}
  \creategeneric
\end{metainfo}
\begin{content}
\begin{block}[annotation]
	Im Ticket-System: \href{https://team.mumie.net/issues/23985}{Ticket 23985}
\end{block}

\usepackage{mumie.genericproblem}
\lang{de}{\title{A03: offen/abgeschlossen}}
\lang{en}{\title{Problem 3}}


\begin{problem}
        
    \randomquestionpool{1}{3}
    \randomquestionpool{4}{5}
    \randomquestionpool{6}{7}
     
        
    \begin{question}   
           
      \begin{variables}
			\randint{a}{2}{4}
			\randint{b}{5}{25}
            \randint{c}{-20}{-2}
            \randint{d}{-1}{12}
            \randint{e}{-5}{7}
            \randint{f}{8}{19}
	  \end{variables}

        \type{mc.multiple}
        \text{Die Menge $(\var{c};\var{d})\cup (\var{e};\var{f})\times (\var{c};\var{f})$ ist ...}
            \explanation{$(\var{e};\var{f})\times (\var{c};\var{f})$ ist eine offene Quader und die Vereinigung
            zwei offener Mengen ist offen.}            
		\begin{choice}
			\text{offen}
			\solution{true}
        \end{choice}
		\begin{choice}
			\text{abgeschlossen}
			\solution{false}
		\end{choice}
        \begin{choice}
			\text{weder offen noch abgeschlossen}
			\solution{false}
		\end{choice}		
    \end{question}
        
    \begin{question}     
           
      \begin{variables}
			\randint{a}{2}{4}
			\randint{b}{5}{25}
            \randint{c}{-20}{-2}
            \randint{d}{-1}{12}
            \randint{e}{-5}{7}
            \randint{f}{8}{19}
	  \end{variables}

        \type{mc.multiple}
        \text{Die Menge $(\var{a};\var{b})\times (\var{c};\var{d})\cap (\var{e};\var{f})\times (0;1)$ ist ...}
            \explanation{$(\var{a};\var{b})\times (\var{c};\var{d})$ und $(\var{e};\var{f})\times (0;1)$ sind offene Quader und die Vereinigung zwei offener Mengen ist offen.}            
		\begin{choice}
			\text{offen}
			\solution{true}
        \end{choice}
		\begin{choice}
			\text{abgeschlossen}
			\solution{false}
		\end{choice}
        \begin{choice}
			\text{weder offen noch abgeschlossen}
			\solution{false}
		\end{choice}		
    \end{question}
        
    \begin{question}      
           
      \begin{variables}
			\randint{a}{2}{4}
			\randint{b}{5}{25}
            \randint{c}{-20}{-2}
            \randint{d}{-1}{12}
            \randint{e}{-5}{7}
            \randint{f}{8}{19}
	  \end{variables}

        \type{mc.multiple}
        \text{Die Menge $(\var{c};\var{d})\cup (\var{e};\var{f}]\times (\var{c};\var{f})$ ist ...}
            \explanation{$(\var{e};\var{f}]$ ist weder offen noch abgesclossen. So sind $(\var{e};\var{f}]\times (\var{c};\var{f})$
            und $(\var{c};\var{d})\cup (\var{e};\var{f}]\times (\var{c};\var{f})$ auch.}            
		\begin{choice}
			\text{offen}
			\solution{false}
        \end{choice}
		\begin{choice}
			\text{abgeschlossen}
			\solution{false}
		\end{choice}
        \begin{choice}
			\text{weder offen noch abgeschlossen}
			\solution{true}
		\end{choice}		
    \end{question}
        
    \begin{question}        
        \type{mc.multiple}
        \text{Die Vereinigung der Mengen\\ 
        ${U_{\frac{1}{2}}}(0;1;2)\cup {U_{\frac{1}{2}}}(1;2;3) \cup
        {U_{\frac{1}{2}}}(2;3;4)\cup\ldots\cup{U_{\frac{1}{2}}}(n;n+1;n+2)\cup\ldots$\\ 
        ist in $\R^3$ ...}
            \explanation{Beachten Sie, dass Beliebige Vereinigungen offener Mengen offen sind.}            
		\begin{choice}
			\text{offen}
			\solution{true}
        \end{choice}
		\begin{choice}
			\text{abgeschlossen}
			\solution{false}
		\end{choice}
        \begin{choice}
			\text{weder offen noch abgeschlossen}
			\solution{false}
		\end{choice}		
    \end{question}
	
    \begin{question}
       \type{mc.multiple}	
       \text{Die Menge $\Q^{k}$ ist ...}
            \explanation{Beachten Sie, dass $\Q$ weder offen noch abgeschlossen ist.}            
		\begin{choice}
			\text{offen}
			\solution{false}
        \end{choice}
		\begin{choice}
			\text{abgeschlossen}
			\solution{false}
		\end{choice}
        \begin{choice}
			\text{weder offen noch abgeschlossen}
			\solution{true}
		\end{choice}		
    \end{question}
        
    \begin{question}        
           
      \begin{variables}
			\randint{a}{2}{4}
	  \end{variables}

        \type{mc.multiple}
        \text{Die Menge $A:=\{(x,y)\in\R^2 | \: y-x^{\var{a}}=0\}$ ist ...}
            \explanation{$A$ ist abgeschlossen, weil $\R^2\setminus A$ offen ist. Der Grund dafür ist, dass
            für ein belibieges $x\in\R^2\setminus A$ ein $r$ existiert, nämlich\\
            $r:=\inf\{\Vert x-a\Vert$ $\vert$ $a\in A,x\in\R^2\setminus A\}$,\\
            mit $U_{\frac{1}{2}r}(x)\subset\R^2\setminus A$. Dass $r$ positiv ist, kann man wie folgt zeigen:\\
            Angenommen $r=0$. Dann für ein $x\in\R^2\setminus A$ und jedes $n\in\N$ kann man ein $a^{(n)}\in A$ finden,
            so dass $\Vert x-a^{(n)}\Vert\leq\frac{1}{n}$ wäre. D.h. aber, dass die Folge $(a^{(n)})_{n\in\N}\subset A$
            nach $x=(x_1,x_2)$ konvengiert. Daraus aber folgt, dass $(a^{(n)})_{n\in\N}$ komponentenweise konvergiert. 
            D.h. $x_1 =\lim_{n\to\infty}a_{1}^{(n)}$ und $x_2 =\lim_{n\to\infty}a_{2}^{(n)}=\lim_{n\to\infty}
            (a_{1}^{(n)})^{\var{a}}=x_{1}^{\var{a}}$. Daraus wiederum folgt, dass $x\in A$.  
            Das ist aber im Widerspruch mit der Annahme, dass $x\in\R^2\setminus A$.\\
            Im nächsten Abschnitt werden wir ein eleganteres Argument dafür kennenlernen, dass die Menge $A$ abgeschlossen ist.}
		\begin{choice}
			\text{offen}
			\solution{false}
        \end{choice}
		\begin{choice}
			\text{abgeschlossen}
			\solution{true}
		\end{choice}
        \begin{choice}
			\text{weder offen noch abgeschlossen}
			\solution{false}
		\end{choice}		
    \end{question}
        
    \begin{question}        
           
      \begin{variables}
			\randint{a}{2}{4}
            \randint{b}{2}{3}
	  \end{variables}

        \type{mc.multiple}
        \text{Die Menge $A:=\{(x,y)\in\R^2 | \:y-\var{b}x^{\var{a}}=0\}$ ist ...}
            \explanation{$A$ ist abgeschlossen, weil $\R^2\setminus A$ offen ist. Der Grund dafür ist, dass
            für ein belibieges $x\in\R^2\setminus A$ ein $r$ existiert, nämlich\\
            $r:=\inf\{\Vert x-a\Vert$ $\vert$ $a\in A,x\in\R^2\setminus A\}$,\\
            mit $U_{\frac{1}{2}r}(x)\subset\R^2\setminus A$. Dass $r$ positiv ist, kann man wie folgt zeigen:\\
            Angenommen $r=0$. Dann für ein $x\in\R^2\setminus A$ und jedes $n\in\N$ kann man ein $a^{(n)}\in A$ finden,
            so dass $\Vert x-a^{(n)}\Vert\leq\frac{1}{n}$ wäre. D.h. aber, dass die Folge $(a^{(n)})_{n\in\N}\subset A$
            nach $x=(x_1,x_2)$ konvengiert. Daraus aber folgt, dass $(a^{(n)})_{n\in\N}$ komponentenweise konvergiert. 
            D.h. $x_1 =\lim_{n\to\infty}a_{1}^{(n)}$ und $x_2 =\lim_{n\to\infty}a_{2}^{(n)}=\lim_{n\to\infty}
            \var{b}(a_{1}^{(n)})^{\var{a}}=\var{b}x_{1}^{\var{a}}$. Daraus wiederum folgt, dass $x\in A$. 
            Das ist aber im Widerspruch mit der Annahme, dass $x\in\R^2\setminus A$.\\
            Im nächsten Abschnitt werden wir ein eleganteres Argument dafür kennenlernen, dass die Menge $A$ abgeschlossen ist.}
		\begin{choice}
			\text{offen}
			\solution{false}
        \end{choice}
		\begin{choice}
			\text{abgeschlossen}
			\solution{true}
		\end{choice}
        \begin{choice}
			\text{weder offen noch abgeschlossen}
			\solution{false}
		\end{choice}		
    \end{question}	
	
	
\end{problem}

\embedmathlet{mathlet}
\end{content}