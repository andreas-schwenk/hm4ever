\documentclass{mumie.problem.gwtmathlet}
%$Id$
\begin{metainfo}
  \name{
    \lang{en}{Training 1}
    \lang{de}{A01: Kurven}
  }
  \begin{description} 
 This work is licensed under the Creative Commons License Attribution 4.0 International (CC-BY 4.0)   
 https://creativecommons.org/licenses/by/4.0/legalcode 

    \lang{en}{...}
    \lang{de}{...}
  \end{description}
  \corrector{system/problem/GenericCorrector.meta.xml}
  \begin{components}
    \component{js_lib}{system/problem/GenericMathlet.meta.xml}{gwtmathlet}
  \end{components}
  \begin{links}
  \end{links}
  \creategeneric
\end{metainfo}
\begin{content}
\begin{block}[annotation]
	Im Ticket-System: \href{https://team.mumie.net/issues/23027}{Ticket 23027}
\end{block}
\usepackage{mumie.genericproblem}
\title{
    \lang{en}{Training 1}
    \lang{de}{A01: Kurven}
  }
  \begin{problem}

%QS Änderung: Zerlegung in Einzelne Teilaufgaben (zur besseren Zuordnung der Erklärungen)
% \begin{question}
    \begin{variables}
    %Koeffizienten:
    \randint[Z]{a}{2}{10}
    \randint[Z]{b}{-10}{-2}
    \randint[Z]{c}{-10}{10}
    \randint[Z]{d}{-10}{10}
    %Kurve
    \function{phix}{a*sin(t)}
    \function{phiy}{b*cos(t)}
    %Kurve doppelt so schnell
	\function{doppx}{a*sin(2*t)}
    \function{doppy}{b*cos(2*t)}
    %Kurve rückwärts
    \function{rueckx}{a*sin(2*pi-t)}
    \function{ruecky}{b*cos(2*pi-t)}    
    %Kurve verschoben
    \function{shiftx}{a*sin(t)+c}
    \function{shifty}{b*cos(t)+d}
    \end{variables}  
%    \type{input.generic}
%     \field{real}

\begin{question} % Teilaufg. (a)
    \type{input.generic}
     \field{real}
 
    \lang{de}{\text{Gegeben ist die Kurve $\phi:[0;2\pi]\to\R^2$, 
    $t\mapsto\begin{pmatrix}\var{a}\sin(t)\\\var{b}\cos(t)\end{pmatrix}$.\\
    \\
    Geben Sie die Parametrisierung der mit doppelter Geschwindigkeit durchlaufenen Kurve $\phi$ an:\\
    \\
         
    
    \begin{table}[\class{layout}]
      $\phi_{dopp}$:\ansref $\to\R^2$, $\: t\mapsto$ & 
      \begin{table}[\class{pmatrix}]  \ansref\\ \ansref \end{table}  \end{table} 
    }}
    
%neue Erklärung   
  \explanation[NOT [equal(ans_2,doppx)] OR NOT[equal(ans_3,doppy)]]{Wird die Kurve mit doppelter Geschwindikeit durchlaufen, 
               muss in der ursprünglichen Kurve $\; t\mapsto 2t$ substituiert werden.
               } 

   \begin{answer}  % Intervall
   \type{input.interval}
% NEU (2.11.2021): "pi" kann jetzt als Intervallgrenze eingegeben werden und wird auch als "pi" angezeigt   
   \allowForInput{pi}
  %\allowIntervalUnionsForInput
  \solution{[0;pi]}
%QS neue Erklärung   
  \explanation[answerEqual(\(0;pi\)) OR answerEqual(\(0;pi\]) OR answerEqual(\[0;pi\))]{Der Definitionsbereich von $\phi_{dopp}$ ist wie der ursprüngliche abgeschlossen.}
  \explanation[NOT[answerEqual(\[0;pi\])] AND NOT[answerEqual(\(0;pi\))] AND NOT[answerEqual(\(0;pi\])] AND NOT[answerEqual(\[0;pi\))]]{Püfen Sie die Intervallgrenzen der Kurve $\phi_{dopp}$. Durch die höhere Geschwindigkeit verkürzt sich das Zeitintervall.}   
    \end{answer}
    
    \begin{answer}  % Parametrisierung
    \type{input.function}
    \solution{doppx}
    \inputAsFunction{t}{Dx}
			\checkFuncForZero{Dx-doppx}{-1}{1}{10}  
            \explanation{Die erste Komponente von $\phi_{dopp}$ ist nicht korrekt.}
    \end{answer}
    \begin{answer} %(b)
    \type{input.function}
    \solution{doppy}
    \inputAsFunction{t}{Dy}
    			\checkFuncForZero{Dy-doppy}{-1}{1}{10}
                \explanation{Die zweite Komponente von $\phi_{dopp}$ ist nicht korrekt.}
    \end{answer}
    
\end{question}    

\begin{question} % Teilaufg. (b)
    \lang{de}{\text{Geben Sie die Parametrisierung der rückwärts durchlaufenen Kurve $\phi$ an: \\
\\    
    
      \begin{table}[\class{layout}]
      $\phi_{rueck}$:\ansref $\to\R^2$, $t\mapsto $ & 
      \begin{table}[\class{pmatrix}]  \ansref\\ \ansref \end{table}  \end{table} 
    }}

%neue Erklärung   
  \explanation[NOT [equal(ans_2,rueckx)] OR NOT[equal(ans_3,ruecky)]]{Wird die Kurve rückwärts durchlaufen, erhält man die neue Kurve aus  
               der ursprünglichen Kurve $\phi$ durch Substitution von $\; t\mapsto 2\pi-t$.
               }                        
   \begin{answer} % Intervall
   \type{input.interval}
% NEU (10.11.2021): Intervalle auch hier eingeben:    
   \allowForInput{pi}
  \solution{[0;2*pi]}
  \explanation{Püfen Sie den Definitionsbereich der Kurve $\phi_{rueck}$. Das Zeitintervall bleibt unverändert.}   
    \end{answer}
    
    \begin{answer} 
    \type{input.function}
    \solution{rueckx}
    \inputAsFunction{t}{Rx}
			\checkFuncForZero{Rx-rueckx}{-1}{1}{10}
            \explanation{Die erste Komponente von $\phi_{rueck}$ ist nicht korrekt.}
    \end{answer}
    \begin{answer}
    \type{input.function}
    \solution{ruecky}
    \inputAsFunction{t}{Ry}
			\checkFuncForZero{Ry-ruecky}{-1}{1}{10}
            \explanation{Die zweite Komponente von $\phi_{rueck}$ ist nicht korrekt.}
    \end{answer}
\end{question}    

\begin{question} % Teilaufg. (c)
    \lang{de}{\text{Geben Sie die um $\left(\begin{smallmatrix}\var{c}\\\var{d}\end{smallmatrix}\right)$ 
    verschobene Kurve $\phi$ an:\\
\\    
    \begin{table}[\class{layout}]
     $\phi_{transl}$: \ansref $\to\R^2$, $t\mapsto$ & 
      \begin{table}[\class{pmatrix}]  \ansref\\ \ansref \end{table}  \end{table} 
    }}

%neue Erklärung
  \explanation[NOT [equal(ans_2,shiftx)] OR NOT[equal(ans_3,shifty)]]{Die verschobene Kurve $\phi_{transl}$ erhält man durch Addition des Verschiebungsvektors
               zum Vektor $\phi(t)$. }                
    \begin{answer} 
    \type{input.interval}
% NEU (10.11.2021): Intervalle auch hier eingeben:    
   \allowForInput{pi}
  \solution{[0;2*pi]}
  \explanation{Püfen Sie den Definitionsbereich der Kurve $\phi_{transl}$. Das Zeitintervall bleibt bei der räumlichen Verschiebung unverändert.}   
    \end{answer}
    \begin{answer}
    \type{input.function}
    \solution{shiftx}
    \inputAsFunction{t}{Sx}
			\checkFuncForZero{Sx-shiftx}{-1}{1}{10}
            \explanation{Die erste Komponente von $\phi_{transl}$ ist nicht korrekt.}
    \end{answer}
    
    \begin{answer}
    \type{input.function}
    \solution{shifty}
    \inputAsFunction{t}{Sy}
			\checkFuncForZero{Sy-shifty}{-1}{1}{10}
            \explanation{Die zweite Komponente von $\phi_{transl}$ ist nicht korrekt.}
    \end{answer}
\end{question}

\end{problem}
\embedmathlet{gwtmathlet}

\end{content}
