%$Id:  $
\documentclass{mumie.article}
%$Id$
\begin{metainfo}
  \name{
    \lang{en}{...}
    \lang{de}{Kurven im $n$-dimensionalen Raum}
   }
  \begin{description} 
 This work is licensed under the Creative Commons License Attribution 4.0 International (CC-BY 4.0)   
 https://creativecommons.org/licenses/by/4.0/legalcode 

    \lang{en}{...}
    \lang{de}{...}
  \end{description}
  \begin{components}
\component{generic_image}{content/rwth/HM1/images/g_img_00_video_button_schwarz-blau.meta.xml}{00_video_button_schwarz-blau}
\component{generic_image}{content/rwth/HM1/images/g_tkz_T501_Cycloid.meta.xml}{T501_Cycloid}
\component{generic_image}{content/rwth/HM1/images/g_tkz_T501_Loop_B.meta.xml}{T501_Loop_B}
\component{generic_image}{content/rwth/HM1/images/g_tkz_T501_SemicubicalParabola_B.meta.xml}{T501_SemicubicalParabola_B}
\component{generic_image}{content/rwth/HM1/images/g_tkz_T501_Helix.meta.xml}{T501_Helix}
\component{generic_image}{content/rwth/HM1/images/g_tkz_T501_TwistedCubic.meta.xml}{T501_TwistedCubic}
\component{generic_image}{content/rwth/HM1/images/g_tkz_T501_Loop_A.meta.xml}{T501_Loop_A}
\component{generic_image}{content/rwth/HM1/images/g_tkz_T501_SemicubicalParabola_A.meta.xml}{T501_SemicubicalParabola_A}
\component{generic_image}{content/rwth/HM1/images/g_tkz_T501_Ellipse.meta.xml}{T501_Ellipse}
\component{generic_image}{content/rwth/HM1/images/g_tkz_T501_Parabola.meta.xml}{T501_Parabola}
\component{generic_image}{content/rwth/HM1/images/g_tkz_T501_Circle.meta.xml}{T501_Circle}
\component{generic_image}{content/rwth/HM1/images/g_tkz_T501_Segment.meta.xml}{T501_Segment}
\end{components}
  \begin{links}
\link{generic_article}{content/rwth/HM1/T404_Eigenwerte,_Eigenvektoren/g_art_content_12_symmetrische_matrizen.meta.xml}{content_12_symmetrische_matrizen}
\link{generic_article}{content/rwth/HM1/T110_Geraden,_Ebenen/g_art_content_36_normalenformen.meta.xml}{content_36_normalenformen}
\end{links}
  \creategeneric
\end{metainfo}


\begin{content}
\begin{block}[annotation]
	Im Ticket-System: \href{https://team.mumie.net/issues/21403}{Ticket 21403}
\end{block}
\usepackage{mumie.ombplus}

\ombchapter{1}
\ombarticle{1}
\usepackage{mumie.genericvisualization}
\begin{visualizationwrapper}
\title{Kurven im $n$-dimensionalen Raum}
\begin{block}[info-box]
\tableofcontents
\end{block}
Ein Wort vorab zur Notation: Inzwischen haben Sie einen Lern- und Wissensstand erreicht, der es Ihnen ermöglicht,
Schreibweisen zu verstehen und zu interpretieren, die vorher nicht vollständig erklärt wurden.
So schreiben wir zum Beispiel oft kompakt $f(x_1,x_2,\ldots,x_n)$ für den Funktionsterm einer Funktion $f:\R^n\to M$, wo
wir streng genommen $f(\left(\begin{smallmatrix}x_1\\\vdots\\x_n\end{smallmatrix}\right))$ schreiben müssten.
Oft trennen wir Einträge in Tupeln der Einfachheit halber durch Kommata, nicht durch Semikola, wenn keine Verwechslungsgefahr besteht, etc..
\section{Kurven im $\R^n$}
% Eingangsbeispiel einfügen
%
%Ich hätte hier so gerne eine Fliege, deren Ortsvektor über die Bildschirm summt.
%
\begin{definition}[Kurven]\label{def:kurve}
Es sei $n\in\N$, $n\geq 2$, eine natürliche Zahl und es sei $I\subset \R$ ein nichtausgeartetes Intervall.
(Das heißt, das Intervall $I$ enthält zumindest ein kleines Teilintervall $(a;b)$ mit $a<b$.)
Weiter seien $\varphi_1,\ldots,\varphi_n: I\to\R$ Funktionen.
\begin{enumerate}
\item[(i)]
Die Abbildung
\begin{equation*}
\varphi: I\to \R^n,\quad t\mapsto \begin{pmatrix}\varphi_1(t)\\\vdots\\\varphi_n(t)\end{pmatrix},
\end{equation*}
heißt \notion{Kurve} im $\R^n$.
\\
Ist $I=[a;b]$ ein abgeschlossenes Intervall endlicher Länge, dann heißt $\varphi(a)$ Anfangspunkt und $\varphi(b)$ Endpunkt der Kurve.
\item[(ii)]
Die Kurve $\varphi$ heißt \notion{stetig im Punkt} $t_0\in I$, wenn alle $\varphi_1,\ldots,\varphi_n$ in $t_0\in I$ sind.
Ebenso heißt $\varphi$ \notion{stetig} (auf ganz $I$), wenn alle $\varphi_1,\ldots,\varphi_n$ stetig sind.
\item[(iii)]
Analog heißt $\varphi$ \notion{ (stetig) differenzierbar in} $t_0\in I$, wenn alle $\varphi_1,\ldots,\varphi_n$ (stetig) differenzierbar in $t_0$ sind.
Und die Kurve $\varphi$ heißt \notion{(stetig) differenzierbar} (auf ganz $I$), wenn alle $\varphi_1,\ldots,\varphi_n$ (stetig) differenzierbar auf $I$ sind.
\item[(iv)]
Ist $\varphi:I\to\R^n$ eine in $t_0\in I$ differenzierbare Kurve, dann nennt man ihre Ableitung
\[
\varphi'(t_0)=\begin{pmatrix}\varphi_1'(t_0)\\\vdots\\\varphi_n'(t_0)\end{pmatrix}
\]
den \notion{Tangentialvektor} \label{def:tangentialvektor} an die Kurve im Punkt $\varphi(t_0)$.
\item[(v)]
Das Bild $\varphi(I)=\{\varphi(t)\mid t\in I\}\subset\R^n$ heißt \notion{Spur} der Kurve $\varphi$.
\end{enumerate}
\end{definition}
%
% Einfache Beispiele
%
\begin{example}[Einfache Kurven]\label{ex1:kurven}
\begin{tabs*}[\initialtab{0}]
\tab{Strecke}
Es seien $u,v\in\R^n$  zwei Punkte im $\R^n$. Dann wird durch $\varphi:[0;1]\to\R^n$, $t\mapsto(1-t)u+tv$, 
eine stetig differenzierbare Kurve gegeben, deren Spur die Strecke zwischen $u$ und $v$ ist. Der Anfangspunkt ist $\varphi(0)=u$, der Endpunkt ist $\varphi(1)=v$.
Der Tangentialvektor $\varphi'(t)=v-u$ ist konstant und gibt an, in welcher Richtung diese Strecke durchlaufen wird.
\begin{center}
\image{T501_Segment}
\end{center}
\tab{Parabel}
Die Kurve $\varphi:\R\to\R^2, t\mapsto \begin{pmatrix}t\\t^2\end{pmatrix}$ hat die Spur $\varphi(\R)=\{\begin{pmatrix}x\\y\end{pmatrix}\in\R^2\mid y=x^2\}$.
Sie beschreibt also eine Parabel. Die Kurve ist stetig differenzierbar mit Tangentialvektor $\varphi'(t)=\begin{pmatrix}1\\2t\end{pmatrix}$. 
Das entspricht dem Richtungsvektor der Tangenten $T: \begin{pmatrix}t\\t^2\end{pmatrix}+\lambda\begin{pmatrix}1\\2t\end{pmatrix}$
an den Graphen der Funktion $\R\to\R$, $x\mapsto x^2$ (in $x=t$).
\begin{center}
\image{T501_Parabola}
\end{center}
\tab{Graph einer Funktion}
Allgemeiner: Ist $f:I\to\R$ eine reelle Funktion auf einem Intervall $I$, dann kann man den Graphen $\text{Graph}(f)=\{(x;f(x))^T\mid x\in I\}\subset\R^2$
auffassen als Kurve $\varphi:I\to\R^2$, $\varphi(t)\mapsto\begin{pmatrix} t\\f(t)\end{pmatrix}$.
Ist die Funktion $f$ (stetig) differenzierbar, dann gilt das auch für die Kurve $\varphi$.
Der Tangentialvektor $\varphi'(t)=\begin{pmatrix}1\\f'(t)\end{pmatrix}$  gibt die Richtung der Tangenten an den Graphen an.
\tab{Kreislinie}
Die Kurve $\varphi:[0;2\pi]\to\R^2$, $t\mapsto\begin{pmatrix} \cos t\\\sin t\end{pmatrix}$ hat als Spur die Einheitskreislinie
$\varphi([0;2\pi])=\{\begin{pmatrix}x_1\\x_2\end{pmatrix}\mid x_1^2+x_2^2=1\}$ Anfangs- und Endpunkt ist $\varphi(0)=\begin{pmatrix}1\\0\end{pmatrix}=\varphi(2\pi)$.
Die Kurve ist stetig differenzierbar mit Tangentenvektor $\varphi'(t)=\begin{pmatrix} -\sin t\\\cos t\end{pmatrix}$. Er beschreibt die Richtung der Tangenten
$T: \begin{pmatrix} \cos t\\\sin t\end{pmatrix}+\lambda \begin{pmatrix} -\sin t\\\cos t\end{pmatrix}$ an die Kreislinie und steht senkrecht auf dem Ortvektor $\varphi(t)$.
\begin{center}
\image{T501_Circle}
\end{center}
\end{tabs*}
\end{example}

%Video
\floatright{\href{https://api.stream24.net/vod/getVideo.php?id=10962-2-10835&mode=iframe&speed=true}{\image[75]{00_video_button_schwarz-blau}}}\\
\\

%%%
%%% Quickcheck
\begin{quickcheck}
\text{Welche der folgenden Kurven hat als Spur die Strecke zwischen $\begin{pmatrix} 0\\2 \end{pmatrix}$ und $\begin{pmatrix} 2\\3 \end{pmatrix}$?}
\begin{choices}{unique}
 \begin{choice}
      \text{$\varphi:[0;1]\to\R^2$, $t\mapsto \begin{pmatrix} 2t^2\\2+t \end{pmatrix}$}
      \solution{false}
    \end{choice}
    \begin{choice}
      \text{$\psi:[0;1]\to\R^2$, $t\mapsto \begin{pmatrix} 2t\\2+t \end{pmatrix}$}
      \solution{true}
    \end{choice}
\begin{choice}
      \text{$\eta:[0;1]\to\R^2$, $t\mapsto \begin{pmatrix} 2t\\2+2t \end{pmatrix}$}
      \solution{false}
    \end{choice}

  \end{choices}
  \explanation{Die Kurve $\varphi$ hat zwar den richtigen Anfangs- und Endpunkt, 
  aber der Tangentialvektor $\varphi'(t)=\begin{pmatrix} 4t\\1 \end{pmatrix}$ hat keine konstante Richtung, 
  also beschreibt $\varphi$ keine Strecke.
  Die Kurve $\eta$ ist eine Strecke mit Anfangspunkt $\begin{pmatrix} 0\\2 \end{pmatrix}$, 
  aber Endpunkt $\begin{pmatrix} 2\\4 \end{pmatrix}$.
  Die Kurve $\psi$ hingegen ist eine Strecke mit passendem Anfangs- und Endpunkt.}
\end{quickcheck}
%%%
\begin{remark}[Parametrisierung]\label{rem:parametrisierung}
\begin{enumerate}
\item[(i)]
An den obigen Beispielen sieht man, dass sich die Anforderung oft genau andersherum stellt: 
Es ist eine Linie $L$ im $\R^n$ gegeben, und man möchte diese
als Kurve darstellen. Man möchte also ein Intervall $I\subset\R$ und eine Kurve $\varphi:I\to\R^n$ so angeben, dass
die Spur von $\varphi$ genau Linie $L$ ist $\varphi(I)=L$.
Dann hat man die Linie \notion{parametrisiert}, d.h. jedem $y\in L$ wurde (mindestens) ein $t\in I$ zugeordnet mit $\varphi(t)=y$.
\item[(ii)]
Eine solche Parametrisierung ist niemals eindeutig, d.h. \emph{wenn} es eine Kurve gibt, welche die Spur $L$ hat, dann gibt es beliebig viele weitere solche.
Es sei etwa $\varphi:[a;b]\to\R^n$, $t\mapsto \varphi(t)$, eine Kurve mit Spur $L=\varphi([a;b])$.
Dann ist 
\begin{itemize}
\item
$\psi_1:[a;b]\to\R^n$, $t\mapsto \psi_1(t)=\varphi(a+b-t)$, die rückwärts durchlaufene Kurve,
\item
$\psi_2:[a;\frac{a+b}{2}]\to\R^n$, $t\mapsto \psi_2(t)=\varphi(2t-a)$, die in doppelter Geschwindigkeit durchlaufene Kurve,
\item
$\psi_3:[a;2b-a]\to\R^n$, $t\mapsto \psi_3(t)=\varphi(\frac{t+a}{2})$, die in halber Geschwindigkeit durchlaufene Kurve,
\item
$\psi_4:[a;2b-a]\to\R^n$, $t\mapsto \psi_4(t)=\varphi(t)$ für $t\in[a;b]$ 
und $\psi_4(t)=\varphi(2b-t)$ für $t\in[b;2b-a]$, die hin und zurück durchlaufene Kurve,
\item
$\psi_5:[0;1]\to\R^n$, $t\mapsto\psi_5(t)=\varphi(a+(b-a)t)$, eine auf dem Standardintervall $[0;1]$ parametrisierte Kurve.
\end{itemize}
und für alle $j=1,\ldots,5$ hat die Kurve $\psi_j$ die Spur $L$. Alle diese Kurven parametrisieren somit $L$.
\end{enumerate}
%
% Hier wäre eine Animation mit der Kreislinie wunderschön!
%
\end{remark}
\begin{quickcheck}
\text{Welche der folgenden Kurven parametrisieren die Strecke zwischen $\begin{pmatrix} 0\\2 \end{pmatrix}$ und $\begin{pmatrix} 2\\3 \end{pmatrix}$?}
\begin{choices}{multiple}
 \begin{choice}
      \text{$\psi_0:[0;1]\to\R^2$, $t\mapsto \begin{pmatrix} 2t\\2+t \end{pmatrix}$}
      \solution{true}
    \end{choice}
    \begin{choice}
      \text{$\psi_1:[0;1]\to\R^2$, $t\mapsto \begin{pmatrix} 2-2t\\3-t \end{pmatrix}$}
      \solution{true}
    \end{choice}
    \begin{choice}
      \text{$\psi_2:[0;1]\to\R^2$, $t\mapsto \begin{pmatrix} 2+2t\\3+t \end{pmatrix}$}
      \solution{false}
    \end{choice}
    \begin{choice}
      \text{$\psi_3:[0;2]\to\R^2$, $t\mapsto \begin{pmatrix} t\\2-t \end{pmatrix}$}
      \solution{false}
    \end{choice}
    \begin{choice}
      \text{$\psi_4:[1;2]\to\R^2$, $t\mapsto \begin{pmatrix} 2t-2\\1+t \end{pmatrix}$}
      \solution{true}
    \end{choice}
      \begin{choice}
      \text{$\psi_5:[0;2]\to\R^2$, $t\mapsto \begin{pmatrix} t\\2+\frac{t}{2} \end{pmatrix}$}
      \solution{true}
    \end{choice}
\end{choices}
\end{quickcheck}
\begin{remark}\label{rem:param_polynom_kurven}
Besonders häufig werden Spuren von Kurven durch polynomiale Gleichungen \label{rem:polyGl} in den Koordinaten beschrieben.
So ist die Einheitskreislinie die Lösungsmenge der Gleichung $x_1^2+x_2^2=1$. Dann ist die Parametrisierung eine gute Möglichkeit, 
deren Eigenschaften mit analytischen Methoden (Stetigkeit, Differenzierbarkeit, Länge,...) zu studieren.
\begin{tabs*}[\initialtab{0}]
\tab{Geraden}
Ein prominentes Beispiel kennen Sie bereits:
Die Umrechnung von \ref[content_36_normalenformen][Normal- in Parameterform von Geraden]{sec:normalform_geraden} im $\R^2$.
\tab{Ellipsen}
Ellipsen werden beschrieben als Lösungsmenge von Gleichungen der Form $\alpha x_1^2+\beta x_2^2 =\gamma$ mit
positiven Konstanten $\alpha,\beta,\gamma\in\R_{>0}$.  Indem wir $a=\sqrt{\frac{\gamma}{\alpha}}$ und 
$b=\sqrt{\frac{\gamma}{\beta}}$ setzen,
erhalten wir die äquivalente Gleichung $\frac{x_1^2}{a^2}+\frac{x_2^2}{b^2}=1$. 
Das ist die Kreisgleichung in $\frac{x_1}{a}$ und $\frac{x_2}{b}$ statt $x$ und $y$. 
Wir bedienen uns der Parametrisierung des Einheitskreises aus Beispiel~\ref{ex1:kurven} und erhalten daraus
eine Parametrisierung der Ellipse
\[\varphi:[0;2\pi]\to\R^2, \quad t\mapsto \begin{pmatrix}a\cos t\\b\sin t\end{pmatrix}.\]
\begin{center}
\image{T501_Ellipse}
\end{center}

Allgemeiner lässt sich jeder \ref[content_12_symmetrische_matrizen][Kegelschnitt]{rem:kegelschnitte}  parametrisieren.
\tab{Neilsche Parabel}
Die Lösungsmenge der Gleichung $y^2=x^3$ ist die Spur der Kurve $\varphi:\R\to\R^2$, 
$t\mapsto \begin{pmatrix} t^2\\t^3\end{pmatrix}$.
\begin{center}
\image{T501_SemicubicalParabola_A}
\end{center}
\tab{Schleife}
Die Lösungsmenge Gleichung $y^2=x^2(x+1)$ läßt sich parametrisieren durch die Kurve
$\varphi:\R\to\R^2$, $t\mapsto \begin{pmatrix}t^2-1\\t(t^2-1)\end{pmatrix}$.
\begin{center}
\image{T501_Loop_A}
\end{center}
\end{tabs*}
\end{remark}
%%
%%Weitere Beispiele -> ToDo: ausformulieren
Bisher haben wir uns auf ebene Kurven beschränkt, aber natürlich ist der Kurvenbegriff in belieber Dimension sinnvoll. Hier einige anschauliche Beispiele im $\R^3$.
\begin{example}[Einige dreidimensionale Kurven]
\begin{tabs*}[\initialtab{0}]
\tab{Getwistete kubische Parabel}
\begin{center}
\[\phi:\R\to\R^3,\quad t\mapsto \begin{pmatrix}t\\t^2\\t^3\end{pmatrix}.\]
\image{T501_TwistedCubic}
\end{center}
\tab{Helix}
\[\psi:\R\to\R^3, \quad t\mapsto \begin{pmatrix} \cos t\\\sin t\\t\end{pmatrix}.\]
\begin{center}
\image{T501_Helix}
\end{center}
\end{tabs*}
\end{example}

%Video
\floatright{\href{https://api.stream24.net/vod/getVideo.php?id=10962-2-10836&mode=iframe&speed=true}{\image[75]{00_video_button_schwarz-blau}}}\\
\\
%
%
\begin{remark}[Tangentialvektoren]
Der Tangentialvektor einer differenzierbaren Kurve gibt im  Großen und Ganzen die Richtung der Tangenten an die Spur der Kurve an.
Allerdings muss man ein wenig aufpassen:
\begin{enumerate}
\item[(a)]
Weil die Kurve $\varphi: I\to \R^n$ nicht injektiv ist, kann es in einem Punkt der Spur mehr als einen Tangentialvektor und mehr als eine Tangente geben.
\begin{tabs*}[\initialtab{0}]
\tab{Doppelpunkt}
So wird z.B. für $\varphi:\R\to\R^2$, $t\mapsto \begin{pmatrix}t^2-1\\t(t^2-1)\end{pmatrix}$, der Ursprung angenommen für $t_1=1$ und $t_2=-1$. 
Dazu gibt es zwei Tangentialvektoren $\varphi'(t_1)=\begin{pmatrix}2\\2\end{pmatrix}$ und $\varphi'(t_2)=\begin{pmatrix}-2\\2\end{pmatrix}$, 
denen zwei sinnvolle Tangenten an die Kurve im Ursprung entsprechen.
\begin{center}
\image{T501_Loop_B}
\end{center}
\end{tabs*}
\item[(b)]
Der Tangentialvektor kann in einem Punkt gleich null sein. Dann läßt sich keine  Tangente an die Kurve legen.
\begin{tabs*}[\initialtab{0}]
\tab{Spitze}
Die \notion{Neilsche Parabel}
$\varphi:\R\to\R^2$, $t\mapsto \begin{pmatrix} t^2\\t^3\end{pmatrix},$
hat die Richtungsableitung $\varphi'(t)=\begin{pmatrix} 2t\\3t^2\end{pmatrix}$. Diese wird null in $t=0$.
Die Spur der Kurve hat hier eine  sogenannte Spitze  im Ursprung $\varphi(0)$. 
Jede beliebige Gerade durch den Ursprung wäre eine mögliche Tangente, was keinen sinnvollen Tangentenbegriff ermöglicht.
\begin{center}
\image{T501_SemicubicalParabola_B}
\end{center}
\end{tabs*}
\end{enumerate}

%Video
\center{\href{https://api.stream24.net/vod/getVideo.php?id=10962-2-10837&mode=iframe&speed=true}{\image[75]{00_video_button_schwarz-blau}}}\\

\end{remark} 

Zum Abschluss bestimmen wir die Bahn  eines festen Punkts auf einem rollenden Reifen.
\begin{example}[Zykloide]\label{ex:zykloide}
%
% Bekomme ich wohl eine Zykloiden-Animation hin?
%
Wir werden die Bahn eines Punktes auf einem rollenden Reifen beschreiben.
Dazu nehmen wir an, dass der Reifen von Radius $r$ mit konstanter Geschwindigkeit $v$ entlang der $x$-Achse rollt.
Der Ortsvektor des Reifenmittelpunkts ist demnach $M(t)=\begin{pmatrix}v\cdot t\\r\end{pmatrix}$.
\\
Es soll uns reichen, die Bahn eines Punktes während einer Umdrehung des Reifens zu beschreiben. Zur Zeit $t=0$ sei dieser Punkt im Ursprung $O=\begin{pmatrix}0\\0\end{pmatrix}$.
\begin{center}
\image{T501_Cycloid}
\end{center}
Wenn der Ball ein bisschen weitergerollt ist, ist die vom Reifenmittelpunkt zurückgelegte Strecke genauso lang, wie der Kreisbogen, der bis dahin abgerollt ist.
Wenn wir den zurückgelegten Winkel mit $\theta$ bezeichnen, gilt somit
\[v\cdot t = r\cdot\theta \qquad\text{oder}\qquad \theta=\frac{v}{r}\cdot t.\]
Mit Hilfe der Winkelfunktionen erhalten wir außerdem die Koordinaten von $P$ in Abhängigkeit von $\theta$
\[
P(\theta)=r\cdot\begin{pmatrix}\theta-\sin\theta\\1-\cos\theta\end{pmatrix}.\]
Schließlich gilt für den Zeitpunkt $t_1$, zu dem der Punkt am Reifen zum ersten Mal wieder unten ist, dass
\[2\pi r=vt_1 \qquad\text{oder}\qquad t_1=\frac{2\pi r}{v}.\]
%Wenn der Ball ein bisschen weitergerollt ist, ist der Punkt dann in $P=\begin{pmatrix}\overline{OT}-\overline{PQ}\\
%\overline{MT}-\overline{MQ}\end{pmatrix}$.
%Dabei ist das Streckenstück $\overline{OT}$ genauso lang, wie der Kreisbogen von $P$ bis $T$, der bis dahin abgerollt ist.
%Den geben wir in Abhängigkeit vom Winkel $\theta$ an: $\overline{OT}=r\cdot\theta$. 
%So erhalten wir mit Hilfe der Winkelfunktionen $P$ in Abhängigkeit von $\theta$
%\[
%P(\theta)=r\cdot\begin{pmatrix}\theta-\sin\theta\\1-\cos\theta\end{pmatrix}.
%\]
%Dabei können wir sogar noch $\theta$ durch die Zeit $t$ ausdrücken, denn $\overline{OT}$ ist auch die $x$-Komponente des Mittelpunkts $M(t)$, also
%$r\cdot\theta=v\cdot t$ oder $\theta=\frac{v}{r}\cdot t$. Und der Zeitpunkt $t_1$, zu dem der Punkt am Reifen zum ersten Mal wieder unten ist, erfüllt
%$2\pi r=vt_1$.
Zusammenfassend finden wir somit folgende Kurve als erste Zykloide
\[
P:[0;\frac{2\pi r}{v}]\to\R^2, \quad t\mapsto P(t)=
\begin{pmatrix}v\cdot t-r\sin(\frac{v}{r}\cdot t)\\r-r\cos(\frac{v}{r}\cdot t)\end{pmatrix}.
\]
\notion{Bemerkung:} Keplers Leistung für sein erstes Gesetz (\glqq Die Planeten kreisen auf Elipsenbahnen, in deren gemeinsamem Brennpunkt die Sonne steht.\grqq{})
ging genau in die umgekehrte Richtung unseres Beispiels. Solchen Zykloiden -- und den komplizierten Epizykeln der Planeten am Himmelgewölbe -- liegt eine einfache Kreis- bzw. 
Ellipsenbahn um einen Fokus zugrunde. Kepler hat diese erkannt und so genau berechnet, dass er seine Gesetze aufstellen konnte.
\end{example} 

%Video
\floatright{\href{https://api.stream24.net/vod/getVideo.php?id=10962-2-10838&mode=iframe&speed=true}{\image[75]{00_video_button_schwarz-blau}}}\\
	

\end{visualizationwrapper}
%%%
%%%
\end{content}
