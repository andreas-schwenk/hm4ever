\documentclass{mumie.element.exercise}
%$Id$
\begin{metainfo}
  \name{
    \lang{en}{Exercise 3}
    \lang{de}{Ü03: Kurven}
  }
  \begin{description} 
 This work is licensed under the Creative Commons License Attribution 4.0 International (CC-BY 4.0)   
 https://creativecommons.org/licenses/by/4.0/legalcode 

    \lang{en}{...}
    \lang{de}{...}
  \end{description}
  \begin{components}
  \end{components}
  \begin{links}
  \end{links}
  \creategeneric
\end{metainfo}
\begin{content}
\begin{block}[annotation]
	Im Ticket-System: \href{https://team.mumie.net/issues/23020}{Ticket 23020}
\end{block}

\title{
 \lang{en}{Exercise 3}
    \lang{de}{Ü03: Kurven}
}
Es sei $\phi:[0;2\pi]\to\R^3,\: t\mapsto\left(\begin{smallmatrix}t\\\cos(t)\\t^2\end{smallmatrix}\right)$,
eine parametrisierte Kurve im $\R^3$.
\begin{enumerate}
\item[(a)]
Geben Sie die Parametrisierung der Kurve an, wenn diese mit doppelter Geschwindigkeit durchlaufen wird.
\item[(b)]
Geben Sie die Parametrisierung der Kurve an, wenn diese mit halber Geschwindigkeit durchlaufen wird.
\end{enumerate}
\begin{tabs*}[\initialtab{0}\class{exercise}]
    \tab{
      Lösung (a)
    }
    Wird die Kurve mit doppelter Geschwindigkeit durchlaufen, so braucht man nur halb so lang.
    Der Definitionsbereich der neuen Kurve $\phi_{dopp}$ darf also lediglich $[0;\pi]$ sein. 
    Zum Zeitpunkt $t$ sind wir nun bereits dort, wo wir ursprünglich erst zum Zeitpunkt $2t$ gewesen wären.
    Also ist die neue Parametrisierung
    \[\phi_{dopp}:[0;\pi]\to \R^3,\quad t\mapsto\begin{pmatrix}2t\\\cos(2t)\\4t^2\end{pmatrix}\:.\]
    \tab{
    Lösung (b)
    }
    Wird die Geschwindigkeit halbiert, so verdoppelt sich das Zeitintervall, das wir für den Durchlauf benötigen, auf $[0;4\pi]$.
    Zum Zeitpunkt $t$ sind wir nun erst dort, wo wir ursprünglich bereits zu Zeit $\frac{t}{2}$ waren.
    Daher ist die neue Parametrisierung
    \[\phi_{halb}:[0;4\pi]\to\R^3, \quad t\mapsto\begin{pmatrix}\frac{t}{2}\\\cos(\frac{t}{2})\\\frac{t^2}{4}\end{pmatrix}\:.\]
\end{tabs*}
\end{content}

