\documentclass{mumie.element.exercise}
%$Id$
\begin{metainfo}
  \name{
    \lang{en}{Exercise 8}
    \lang{de}{Ü08: abgeschlossen}
  }
  \begin{description} 
 This work is licensed under the Creative Commons License Attribution 4.0 International (CC-BY 4.0)   
 https://creativecommons.org/licenses/by/4.0/legalcode 

    \lang{en}{...}
    \lang{de}{...}
  \end{description}
  
  \begin{components}

  \end{components}
  
  \begin{links}
  \end{links}
  \creategeneric
\end{metainfo}


\begin{content}
\begin{block}[annotation]
	Im Ticket-System: \href{https://team.mumie.net/issues/23983}{Ticket 23983}
\end{block}

\begin{block}[annotation]

\end{block}

\title{
  \lang{en}{Exercise 8}
    \lang{de}{Ü08: abgeschlossen}}
    
Zeigen Sie, dass 

  \begin{table}[\class{items}]
    \nowrap{a) $\overline{U_{r_{0}}(x_{0})}:=\{x\in\R^{k}\text{ }|\text{ }||x-x_{0}||\leq r_{0}\}$ und}
    \nowrap{b) $B:=\{b_{n}| n\geq1\}\cup\{b\}$, wenn $(b_{n})_{n}\subset\R^{k}$ eine konvergente Folge und 
    $b:=\lim_{n\to\infty} b_{n}$ ist,}
 
  \end{table}


abgeschlossen sind.


  \begin{tabs*}[\initialtab{0}\class{exercise}]
    
    \tab{
      Lösung a)
    }
    \begin{incremental}[\initialsteps{1}]
      \step 
    Um zu beweisen, dass $\overline{U_{r_{0}}(x_{0})}$ abgeschlossen ist, müssen wir zeigen, dass 
    $A:=\R^{k}\setminus\overline{U_{r_{0}}(x_{0})}$ offen ist. D.h. wir müssen zeigen, dass für jedes 
    $a\in A$ eine positive reelle Zahl $r$ existiert, so dass $U_{r}(a)\subset A$.
      \step 
      Für ein beliebiges $a\in A$ wähle $r:=\frac{1}{2}(||x_{0}-a||-r_{0})$. Weil $a\notin\overline{U_{r_{0}}(x_{0})}$,
      ist $r$ eine positive reelle Zahl. Dann gilt 
      \[U_{r}(a)\cap\overline{U_{r_{0}}(x_{0})}=\emptyset.\]
      \step
      Daraus folgt \[U_{r}(a)\subset\R^{k}\setminus\overline{U_{r_{0}}(x_{0})}.\]
      Damit haben wir jedoch gezeigt, dass $\R^{k}\setminus\overline{U_{r_{0}}(x_{0})}$ offen und infolgedessen
      $\overline{U_{r_{0}}(x_{0})}$ abgeschlossen ist.
    \end{incremental}
    
      \tab{
      Lösung b)
    }
    \begin{incremental}[\initialsteps{1}]
      \step 
    Um zu beweisen, dass $B$ abgeschlossen ist, müssen wir zeigen, dass $\R^{k}\setminus B$ offen ist. 
    D.h. wir müssen zeigen, dass für jedes $x\in\R^{k}\setminus B$  eine positive reelle Zahl $r_{0}$ existiert, 
    so dass $U_{r_{0}}(x)\subset\R^{k}\setminus B$.
    \step
    Für ein beliebiges $x\in\R^{k}\setminus B$ wähle $l:=\frac{1}{3}||x-b||$. Da $||x-b||$ eine positive reelle Zahl ist, 
    ist auch $l$ eine solche. Weil $(b_{n})_{n}$ konvergent ist , gibt es für dieses $l$ ein $N\in\N$, so dass für jedes 
    $n\geq N$ gilt $b_{n}\in U_{l}(b)$. 
    \step
    Man kann also schreiben $U_{l}(b)\cap U_{l}(x)=\emptyset$. Daraus folgt, dass $B\cap U_{l}(x)=\{c_{1},c_{2},\ldots,c_{j}\}$
    für ein $j\leq N-1$, wo $c_{i}\in (b_{n})_{n}$ und $c_{i}\notin U_{l}(b)$ für $1\leq i\leq j$.
    \step
    Setze 
    
    \[r:=\text{min}_\limits{1\leq i\leq j} {\Vert x-c_{i}\Vert}.\] 
    
    Da $x\neq b_{n}$ für jedes $n\in\N$, gilt $x\neq c_{i}$ für jedes 
    $1\leq i\leq j$ und somit ist $r$ eine positive reelle Zahl. Für $r_{0}:=\frac{1}{2}r$ gilt aber 
    $U_{r_{0}}(x)\cap B=\emptyset$. Folglich haben wir $U_{r_{0}}(x)\subset\R^{k}\setminus B$. Damit wurde gezeigt, dass
    $\R^{k}\setminus B$ offen und infolgedessen $B$ abgeschlossen ist.
    \end{incremental}

 \end{tabs*}   
\end{content}

