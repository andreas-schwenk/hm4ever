\documentclass{mumie.element.exercise}
%$Id$
\begin{metainfo}
  \name{
    \lang{en}{Exercise 2}
    \lang{de}{Ü02: Kurven}
  }
  \begin{description} 
 This work is licensed under the Creative Commons License Attribution 4.0 International (CC-BY 4.0)   
 https://creativecommons.org/licenses/by/4.0/legalcode 

    \lang{en}{...}
    \lang{de}{...}
  \end{description}
  \begin{components}
\component{generic_image}{content/rwth/HM1/images/g_tkz_T501_Exercise02.meta.xml}{T501_Exercise02}
\component{generic_image}{content/rwth/HM1/images/g_tkz_T501_Exercise01.meta.xml}{T501_Exercise01}
\component{generic_image}{content/rwth/HM1/images/g_img_T501_ueb_1.meta.xml}{T501_ueb_1}
\end{components}
  \begin{links}
  \end{links}
  \creategeneric
\end{metainfo}
\begin{content}
\begin{block}[annotation]
	Im Ticket-System: \href{https://team.mumie.net/issues/23007}{Ticket 23007}
\end{block}

\title{
  \lang{en}{Exercise 2}
    \lang{de}{Ü02: Kurven}}
Gegeben sei die Kurve
\[\phi:[0,2\pi]\to\R^2,\quad \phi(t)=\begin{pmatrix} \sin(t)\\-2\cdot\sin(\frac{t}{2})\end{pmatrix}\:.\]
\begin{enumerate}
\item[(a)] Skizzieren Sie die Spur der Kurve in der Ebene.
\item[(b)] Bestimmen Sie die Tangentengleichung an $\phi$ zum Zeitpunkt $t_0=\frac{\pi}{3}$ und 
zeichnen Sie die Tangente ein.
\end{enumerate}
  \begin{tabs*}[\initialtab{0}\class{exercise}]
    \tab{
      Lösung (a)
    }
    \begin{center}
    \image{T501_Exercise02}
    \end{center}
    \tab{
    Lösung (b)
    }
Der Punkt auf der Kurve zum Zeitpunkt $t_0=\frac{\pi}{3}$ ist
\[\phi(\frac{\pi}{3})=\begin{pmatrix} \sin(\frac{\pi}{3})\\-2\cdot\sin(\frac{\pi}{6})\end{pmatrix}=
\begin{pmatrix} \frac{\sqrt{3}}{2}\\-1\end{pmatrix}\:.\]
Weil beide Komponenten von $\phi$ differenzierbar sind, ist die Kurve differenzierbar mit
\[\phi'(t)=\begin{pmatrix}\cos(t)\\-\cos(\frac{t}{2})\end{pmatrix}\:.\]
Daraus ergibt sich der Tangentialvektor in $t_0$ zu
\[\phi'(t_0)=\begin{pmatrix}\cos(\frac{\pi}{3})\\-\cos(\frac{\pi}{6})\end{pmatrix}=
\begin{pmatrix}  \frac{1}{2}\\-\frac{\sqrt{3}}{2}\end{pmatrix}\:.\]
Die Tangentengleichung zum Zeitpunkt $t_0$ in Parameterform ist also
\[T=T_{t_0}: \begin{pmatrix}x\\y\end{pmatrix}=\phi(t_0)+\lambda\cdot\phi'(t_0)=\begin{pmatrix} \frac{\sqrt{3}}{2}\\-1\end{pmatrix}+\lambda\cdot
\begin{pmatrix}  \frac{1}{2}\\-\frac{\sqrt{3}}{2}\end{pmatrix},\quad \text{mit } \lambda\in\R.\]
 Wir können die Tangentengleichung auch in Normalform $y=mx+b$ angeben. Dazu bestimmen wir zwei Punkte
 auf der Tangenten, etwa $\begin{pmatrix}x_0\\y_0\end{pmatrix}=\begin{pmatrix} \frac{\sqrt{3}}{2}\\-1\end{pmatrix}$ zu $\lambda=0$
 und $\begin{pmatrix}x_2\\y_2\end{pmatrix}=\begin{pmatrix} \frac{\sqrt{3}}{2}+1\\-1-\sqrt{3}\end{pmatrix}$ zu $\lambda=2$,
 und setzen diese in die Normalform ein. So erhalten wir das folgende lineare Gleichungssystem
 \begin{align*}
 -1&=\:\frac{\sqrt{3}}{2}m+b,\\
 -1-\sqrt{3}&=\:(\frac{\sqrt{3}}{2}+1)m+b,
 \end{align*}
 aus dem wir $m=-\sqrt{3}$ und $b=\frac{1}{2}$ und schließlich auch die Tangentengleichung in Normalform 
 eindeutig bestimmen, also
 \[T:\quad y=-\sqrt{3} \cdot x+\frac{1}{2}.\]
 Die Tangente haben wir in die Abbildung der Lösung zu Teil (a) eingezeichnet.
 \end{tabs*}   
\end{content}

