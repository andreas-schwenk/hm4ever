\documentclass{mumie.element.exercise}
%$Id$
\begin{metainfo}
  \name{
    \lang{en}{Exercise 6}
    \lang{de}{Ü06: Matrizennorm}
  }
  \begin{description} 
 This work is licensed under the Creative Commons License Attribution 4.0 International (CC-BY 4.0)   
 https://creativecommons.org/licenses/by/4.0/legalcode 

    \lang{en}{...}
    \lang{de}{...}
  \end{description}
  
  \begin{components}

  \end{components}
  
  \begin{links}
  \end{links}
  \creategeneric
\end{metainfo}


\begin{content}
\begin{block}[annotation]
	Im Ticket-System: \href{https://team.mumie.net/issues/23980}{Ticket 23980}
\end{block}


\title{
  \lang{en}{Exercise 6}
    \lang{de}{Ü06: Matrizennorm}}
    
Sei $A=(a_{ij})_{1\leq i\leq m,1\leq j\leq n}$ eine Matrix über $\R$. Zeigen Sie, dass

\begin{enumerate}
\item[(a)] ${\Vert A\Vert_{1}}=\text{max}_\limits{j=1,\ldots,n}{\sum_{i=1}^{m}{\vert a_{ij}\vert}}$. 
(Deswegen bezeichnet man $\Vert A\Vert_{1}$ als \emph{Spaltensummennorm} von $A$.)
\item[(b)] ${\Vert A\Vert_{\infty}}=\text{max}_\limits{i=1,\ldots,m}{\sum_{j=1}^{n}{\vert a_{ij}\vert}}$. 
(Deswegen bezeichnet man $\Vert A\Vert_{\infty}$ als \emph{Zeilensummennorm} von $A$.)
\item[(c)] Für $A:=\begin{pmatrix}
  5 & -1 & 2\\
  -1 & 7 & -2\\
  2 & -3 & -3  
 \end{pmatrix}$ und $B:=\begin{pmatrix}
  -5 & -1 & 2\\
  3 & -3 & -3\\
  4 & -\frac{3}{2} & 3  
 \end{pmatrix}$ berechnen Sie die Maximumsnorm und 1-Norm von $A$ und $B$. 
\end{enumerate}

  \begin{tabs*}[\initialtab{0}\class{exercise}]
    
    \tab{
      Lösung a)
    }
    Seien $\R^n$ und $\R^m$ mit der 1-Norm gegeben. Per Definition haben wir
    \[{\Vert A\Vert_{1}}= \text{sup}_\limits{x\in\R^n\text{ mit }\Vert x\Vert_{1}=1}{\Vert A.x\Vert_{1}}.\]
    Da wir in endlich-dimensionalen Räumen ($\R^n$) arbeiten, können wir in der obigen Gleichung \emph{max} 
    anstelle von \emph{sup} verwenden. D.h.
    \[{\Vert A\Vert_{1}}= \text{max}_\limits{\Vert x\Vert_{1}=1}{\Vert A.x\Vert_{1}}
    =\text{max}_\limits{\Vert x\Vert_{1}=1}{\sum_{i=1}^{m}{\vert \sum_{j=1}^{n} a_{ij}x_{j}\vert}}.\]
    
    Da $\vert\sum_{j=1}^{n} a_{ij}x_{j}\vert$ für festes $i$ für einen der Einheitsvektoren 
    \[{\vec {e}}_{1}={\begin{pmatrix}1\\0\\0\\\vdots \\0\end{pmatrix}},\;
    {\vec {e}}_{2}={\begin{pmatrix}0\\1\\0\\\vdots \\0\end{pmatrix}},\;
    {\vec {e}}_{3}={\begin{pmatrix}0\\0\\1\\\vdots \\0\end{pmatrix}},\;
    \ldots ,
    \;{\vec {e}}_{n}={\begin{pmatrix}0\\0\\0\\\vdots \\1\end{pmatrix}}\]
    maximal wird, gilt
    \[{\Vert A\Vert_{1}}=\text{max}_\limits{\Vert x\Vert_{1}=1}{\sum_{i=1}^{m}{\vert \sum_{j=1}^{n} a_{ij}x_{j}\vert}}
    =\max_{j=1,\ldots,n}{\sum_{i=1}^{m}{\vert a_{ij}\vert}}.\]
    
    \tab{
    Lösung b)
    }
    Seien $\R^n$ und $\R^m$ mit der Maximumsnorm gegeben. Per Definition haben wir
    \[{\Vert A\Vert_{\infty}}= \text{sup}_\limits{x\in\R^3\text{ mit }\Vert x\Vert_{\infty}=1}{\Vert A.x\Vert_{\infty}}.\]
    Da wir in endlich-dimensionalen Räumen ($\R^n$) arbeiten, können wir in der obigen Gleichung \emph{max} 
    anstelle von \emph{sup} verwenden. D.h.
    
    
    \begin{align*}
    {\Vert A\Vert_{\infty}}
    &=\text{max}_\limits{\Vert x\Vert_{\infty}=1}{\Vert A.x\Vert_{\infty}}&&\\
    &=\text{max}_\limits{\Vert x\Vert_{\infty}=1}\text{ }\text{ max}_\limits{i=1,\ldots,m}{\vert \sum_\limits{j=1}^{n} a_{ij}x_{j}\vert}&&\\
    &\leq\text{max}_\limits{\Vert x\Vert_{\infty}=1}\text{ }\text{ max}_\limits{i=1,\ldots,m}{\sum_\limits{j=1}^{n} \vert a_{ij}\vert \vert x_{j}\vert}&\qquad&| Dreiecksungleichung\\
    &=\text{max}_\limits{i=1,\ldots,m}\text{ }\text{ max}_\limits{\Vert x\Vert_{\infty}=1}{\sum_\limits{j=1}^{n} \vert a_{ij}\vert \vert x_{j}\vert}&&\\
    &\leq\text{max}_\limits{i=1,\ldots,m}\text{ }\text{ max}_\limits{\Vert x\Vert_{\infty}=1}{\sum_\limits{j=1}^{n} \vert a_{ij}\vert \Vert x\Vert_{\infty}}&&\\
    &=\text{max}_\limits{i=1,\ldots,m}{\sum_\limits{j=1}^{n} \vert a_{ij}\vert}.&&\\
    \end{align*}
    
    
    Es gilt also
    
    \[{\Vert A\Vert_{\infty}}
    =\text{max}_\limits{\Vert x\Vert_{\infty}=1}{\Vert A.x\Vert_{\infty}}
    \leq\text{max}_\limits{i=1,\ldots,m}{\sum_{j=1}^{n} {\vert a_{ij}\vert}}.\qquad(1)\\\]
    
    
    Sei $k$ die Zeilennummer, für die gilt
    \[\text{max}_\limits{i=1,\ldots,m}{\sum_{j=1}^{n} {\vert a_{ij}\vert}}=\sum_{j=1}^{n} {\vert a_{kj}\vert}.\]
    
    Sei außerdem $v=(v_1,...,v_n)$ in der Weise, dass für jedes $j$ gilt $v_j=1$, wenn $a_{kj}\geq 0$ und $v_j=-1$ wenn 
    $a_{kj}\leq 0$. Es ist klar, dass $\Vert v\Vert_{\infty}=1$. Für jedes $j$ gilt also $a_{kj}v_{j}=|a_{kj}x_{j}|$ und wir erhalten somit für jede Zeile 
    $i\in\{1,\ldots,m\}$:
    \begin{align*}
    \left|\sum_{j=1}^n a_{ij}v_j\right|
    &\leq \sum_{j=1}^n \left|a_{ij}v_j\right|\\
    &=\sum_{j=1}^n \left|a_{ij}\right|\\
    &\leq \sum_{j=1}^n \left|a_{kj}\right|\\
    &=\left|\sum_{j=1}^n a_{kj}v_j \right|.
    \end{align*}
    
    Für $v$ und $k$ gilt also
    
    \[{\Vert A.v\Vert_{\infty}}=\text{max}_\limits{i=1,\ldots,m}\left|\sum_{j=1}^n a_{ij}v_j\right|
    =\left|\sum_{j=1}^n a_{kj}v_j \right|
    =\sum_{j=1}^n \left|a_{kj}\right|
    =\text{max}_\limits{i=1,\ldots,m}\sum_{j=1}^n \left|a_{ij}\right|.\qquad(2)\].

    Aus $(1)$ und $(2)$ können wir schließen, dass
    
    \[{\Vert A\Vert_{\infty}}
    =\text{max}_\limits{\Vert x\Vert_{\infty}=1}{\Vert A.x\Vert_{\infty}}
    =\text{max}_\limits{i=1,\ldots,m}\sum_{j=1}^n \left|a_{ij}\right|.\]
    
    \tab{
    Lösung (c)
    }
    \[{\Vert A\Vert_{1}}=\text{max}_\limits{j=1,\ldots,n}{\sum_{i=1}^{m}{\vert a_{ij}\vert}}
    =\max\{5+1+2;1+7+3;2+2+3\}=\max\{8;11;7\}=11.\]
    \[{\Vert A\Vert_{\infty}}=\text{max}_\limits{i=1,\ldots,m}{\sum_{j=1}^{n}{\vert a_{ij}\vert}}
    =\max\{5+1+2;1+7+2;2+3+3\}=10.\]
    
    \[{\Vert B\Vert_{1}}=\text{max}_\limits{j=1,\ldots,n}{\sum_{i=1}^{m}{\vert b_{ij}\vert}}
    =\max\{12;\frac{11}{2};8\}=12.\]
    \[{\Vert B\Vert_{\infty}}=\text{max}_\limits{i=1,\ldots,m}{\sum_{j=1}^{n}{\vert b_{ij}\vert}}
    =\max\{8;9;\frac{17}{2}\}=9.\]

 \end{tabs*}   
\end{content}

