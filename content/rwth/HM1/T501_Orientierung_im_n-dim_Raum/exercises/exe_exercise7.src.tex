\documentclass{mumie.element.exercise}
%$Id$
\begin{metainfo}
  \name{
    \lang{de}{Ü07: Konvergenz}
    \lang{en}{Exercise 7}
  }
  \begin{description} 
 This work is licensed under the Creative Commons License Attribution 4.0 International (CC-BY 4.0)   
 https://creativecommons.org/licenses/by/4.0/legalcode 

    \lang{de}{}
    \lang{en}{}
  \end{description}
  \begin{components}
  \end{components}
  \begin{links}
  \end{links}
  \creategeneric
\end{metainfo}
\begin{content}
\begin{block}[annotation]
	Im Ticket-System: \href{https://team.mumie.net/issues/23982}{Ticket 23982}

\end{block}


\usepackage{mumie.ombplus}

\title{
  \lang{de}{Ü07: Konvergenz}
  \lang{en}{Exercise 7}
}


  \lang{de}{Sei $(x^{(n)})_{n\in\N}$ in $(\R^{k},\Vert.\Vert)$ eine konvergente Folge mit $\lim_{n\to\infty}x^{(n)}=x$.
  Zeigen Sie, dass $(\Vert x^{(n)}\Vert)_{n\in\N}$ eine konvergente Folge in $\R$ ist mit $\lim_{n\to\infty}{\Vert x^{(n)}\Vert}={\Vert x\Vert}$.}
  
  \begin{tabs*}[\initialtab{0}\class{exercise}]
 
    \tab{Lösung}
    
    \begin{incremental}[\initialsteps{1}]
      \step
      Weil für jedes $a\in\R^{k}$ gilt $||a||\in\R$, ist $(\Vert x^{(n)}\Vert)_{n\in\N}$ eine reelle Folge.      
      \step 
      Nun sei $\epsilon>0$ beliebig. Da $(x^{(n)})_{n\in\N}$ konvergiert, existiert ein $N\in\N$, so dass
      für jedes $n>N$ gilt $||x^{(n)}-x||<\epsilon$.
      Mit der zweiten Dreiecksungleichung haben wir 
      \[\vert\text{ }||x^{(n)}||-||x||\text{ }\vert<||x^{(n)}-x||<\epsilon.\]
      Somit haben wir bewiesen, dass $\lim_{n\to\infty}{\Vert x^{(n)}\Vert}={\Vert x\Vert}$.
      
    \end{incremental}
  \end{tabs*}
\end{content}





