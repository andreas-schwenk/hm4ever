\documentclass{mumie.element.exercise}
%$Id$
\begin{metainfo}
  \name{
    \lang{en}{Exercise 1}
    \lang{de}{Ü01: Kurven}
  }
  \begin{description} 
 This work is licensed under the Creative Commons License Attribution 4.0 International (CC-BY 4.0)   
 https://creativecommons.org/licenses/by/4.0/legalcode 

    \lang{en}{...}
    \lang{de}{...}
  \end{description}
  \begin{components}
\component{generic_image}{content/rwth/HM1/images/g_tkz_T501_Exercise01.meta.xml}{T501_Exercise01}
\end{components}
  \begin{links}
  \end{links}
  \creategeneric
\end{metainfo}
\begin{content}
\begin{block}[annotation]
	Im Ticket-System: \href{https://team.mumie.net/issues/23008}{Ticket 23008}
\end{block}
\title{
 \lang{en}{Exercise 1}
    \lang{de}{Ü01: Kurven}}
 \begin{enumerate}
 \item[(a)]
Skizzieren Sie die Spur der Kurve
\[\phi:[0;3]\to\R^2,\quad \phi(t)=\begin{pmatrix}t^2\\t^2\end{pmatrix}.\]
\item[(b)]
Geben Sie eine weitere parametrisierte Kurve mit derselben Spur an.
\end{enumerate}
 \begin{tabs*}[\initialtab{0}\class{exercise}]
    \tab{
      Lösung (a)
    }
    Die Spur ist das Bild der Kurve, also die Menge
    \[\text{Spur}(\phi)=\{\begin{pmatrix}x\\y\end{pmatrix}=\begin{pmatrix}t^2\\t^2\end{pmatrix}\in\R^2\mid t\in[0;3]\}\subset \R^2.
    \]
    Ein Punkt $\left(\begin{smallmatrix}x\\y\end{smallmatrix}\right)$ liegt also genau dann auf der Spur der Kurve,
    wenn $x=y$ und wenn $x=t^2$ mit $t\in [0;3]$ ist, also wenn $x=y$ und $x\in[0;9]$ erfüllt ist,
    \[\text{Spur}(\phi)=\{\begin{pmatrix}x\\x\end{pmatrix}\mid x\in[0;9]\}\subset\R^2.\]
    \begin{center}
    \image{T501_Exercise01}
    \end{center}
  \tab{
    Lösung (b)
      }
   Hier gibt es viele mögliche Lösungen.
Eine davon ist zum Beispiel
\[\psi:[0;9]\to\R^2,\quad \psi(t)=\begin{pmatrix}t\\t\end{pmatrix},\]
denn 
\[\text{Spur}(\psi)=\{\begin{pmatrix}x\\y\end{pmatrix}=\begin{pmatrix}t\\t\end{pmatrix}\in\R^2\mid t\in [0;9]\}=\text{Spur}(\phi).\]
 \end{tabs*}
\end{content}

