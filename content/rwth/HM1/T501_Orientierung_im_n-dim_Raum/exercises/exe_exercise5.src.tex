\documentclass{mumie.element.exercise}
%$Id$
\begin{metainfo}
  \name{
    \lang{en}{Exercise 5}
    \lang{de}{Ü05: Matrizennorm}
  }
  \begin{description} 
 This work is licensed under the Creative Commons License Attribution 4.0 International (CC-BY 4.0)   
 https://creativecommons.org/licenses/by/4.0/legalcode 

    \lang{en}{...}
    \lang{de}{...}
  \end{description}
  
  \begin{components}

  \end{components}
  
  \begin{links}
  \end{links}
  \creategeneric
\end{metainfo}


\begin{content}
\begin{block}[annotation]
	Im Ticket-System: \href{https://team.mumie.net/issues/23981}{Ticket 23981}
\end{block}


\title{
  \lang{en}{Exercise 5}
    \lang{de}{Ü05: Matrizennorm}}
    
Sei $A:=\begin{pmatrix}
  2 & 3 & 3\\ 
  2 & 3 & 3  
 \end{pmatrix}$. Berechnen Sie $\Vert A\Vert$

\begin{enumerate}
\item[(a)] für den Fall, dass $\R^2$ und $\R^3$ mit der 1-Norm versehen werden.
\item[(b)] für den Fall, dass $\R^2$ und $\R^3$ mit der Maximumsnorm versehen werden.
\end{enumerate}

  \begin{tabs*}[\initialtab{0}\class{exercise}]
    
    \tab{
      Lösung (a)
    }
     Nach der Defintion müssen wir das Folgende berechnen:
     \[{\Vert A \Vert} = \text{sup}_\limits{x\in\R^3\text{ mit }\Vert x\Vert_{1}=1}{\Vert A\cdot x\Vert_{1}}.\]
     
     Sei nun $x\in\R^3$ mit $||x||_{1}= |x_{1}|+|x_{2}|+|x_{3}|=1$. Dann ist
        \[||A\cdot x||_{1}=2|2x_{1}+3x_{2}+3x_{3}|\leq 6||x||_{1}=6.\]

    Es gilt also $||A||\leq 6$. Betrachten Sie nun den Vektor $(0, 0, 1)$. Es gilt $||(0, 0, 1)||_{1}=1$ und
    $||A\cdot (0, 0, 1)^{T}||_{1}=||(3, 3)||_{1}=6$. Da $||A||$ die kleinste obere Schranke von $||A\cdot x||_{1}$ auf
    allen Einheitsvektoren ist, muss also auch gelten $||A||\geq 6$ und somit ist $||A||=6$.

    \tab{
    Lösung (b)
    }
    Nach der Defintion müssen wir das Folgende berechnen:
     \[{\Vert A \Vert} = \text{sup}_\limits{x\in\R^3\text{ mit }\Vert x\Vert_{\infty}=1}{\Vert A\cdot x\Vert_{\infty}}.\]
    Sei $x\in\R^3$ mit $||x||_{\infty}= \max\{|x_{1}|;|x_{2}|;|x_{3}|\}=1$. Dann ist
        \[||A\cdot x||_{\infty}=|2x_{1}+3x_{2}+3x_{3}|\leq 8||x||_{\infty}=8.\]
    Es gilt also $||A||\leq 8$. Betrachten Sie nun den Vektor $(1, 1, 1)$. Es gilt $||(1, 1, 1)||_{\infty}=1$ und
    $||A\cdot (1, 1, 1)^{T}||_{\infty}=||(8, 8)||_{\infty}=8$. Da $||A||$ die kleinste obere Schranke von $||A\cdot x||_{\infty}$ auf
    allen Einheitsvektoren ist, ist also auch $||A||\geq 8$ und somit ist $||A||=8$.

     
     

 \end{tabs*}   
\end{content}

