%$Id:  $
\documentclass{mumie.article}
%$Id$
\begin{metainfo}
  \name{
    \lang{en}{...}
    \lang{de}{Nullstellen von Polynomen}
   }
  \begin{description} 
 This work is licensed under the Creative Commons License Attribution 4.0 International (CC-BY 4.0)   
 https://creativecommons.org/licenses/by/4.0/legalcode 

    \lang{en}{...}
    \lang{de}{...}
  \end{description}
  \begin{components}
  \component{generic_image}{content/rwth/HM1/images/g_tkz_T203_ComplexMultiplication.meta.xml}{T203_ComplexMultiplication}
  \component{generic_image}{content/rwth/HM1/images/g_img_00_Videobutton_schwarz.meta.xml}{00_Videobutton_schwarz}
  \component{generic_image}{content/rwth/HM1/images/g_img_00_video_button_schwarz-blau.meta.xml}{00_video_button_schwarz-blau}
   \component{js_lib}{system/media/mathlets/GWTGenericVisualization.meta.xml}{mathlet1}
  \end{components}
  \begin{links}
    \link{generic_article}{content/rwth/HM1/T103_Polynomfunktionen/g_art_content_10_polynomdivision.meta.xml}{content_10_polynomdivision}
    \link{generic_article}{content/rwth/HM1/T209_Potenzreihen/g_art_content_28_exponentialreihe.meta.xml}{content_28_exponentialreihe}
    \link{generic_article}{content/rwth/HM1/T203_komplexe_Zahlen/g_art_content_09neu_komplexeZahlen_hauptsatz.meta.xml}{content_03_komplexeZahlen_hauptsatz}
    \link{generic_article}{content/rwth/HM1/T103_Polynomfunktionen/g_art_content_09_polynome.meta.xml}{polynome}
    \end{links}
  \creategeneric
\end{metainfo}
\begin{content}
\begin{block}[annotation]
	Im Ticket-System: \href{https://team.mumie.net/issues/21564}{Ticket 21564}
\end{block}
\begin{block}[annotation]
Copy of \href{https://team.mumie.net/issues/21007}{Ticket 21007}: content/rwth/HM1/T203_komplexe_Zahlen/art_content_09_komplexeZahlen_hauptsatz.src.tex
\end{block}



\usepackage{mumie.ombplus}
\ombchapter{3}
\ombarticle{3}

% \lang{de}{\title{Multiplikation, Quadratwurzeln und Fundamentalsatz der Algebra}}
\lang{de}{\title{Nullstellen von Polynomen}}
\lang{en}{\title{Zeros of polynomials}}
\begin{block}[info-box]
\tableofcontents
\end{block}

\usepackage{mumie.genericvisualization}

\begin{visualizationwrapper}

\section{
\lang{de}{Multiplikation in Polarkoordinaten}
\lang{en}{Multiplication in polar coordinates}}\label{sec:multiplikation-polar}

\lang{de}{Wir betrachten die beiden komplexen Zahlen $z,w\in\C$ mit
\[z=|z|\cdot(\cos(\beta)+i\sin(\beta))\ \text{ und } \ w=|w|\cdot(\cos(\alpha)+i\sin(\alpha)). \]
Von den \ref[content_28_exponentialreihe][Additionstheoremen]{rem:sin_cos_exp} für die trigonomerischen Funktionen,
die später aus der komplexen Exponentialfunktion folgen, benötigen wir hier zwei Formeln:
}
\lang{en}{
Consider two complex numbers $z,w\in\C$, with
\[z=|z|\cdot(\cos(\beta)+i\sin(\beta))\ \text{ and } \ w=|w|\cdot(\cos(\alpha)+i\sin(\alpha)). \]
We will need two of the \ref[content_28_exponentialreihe][addition theorems]{rem:sin_cos_exp}
for trigonometric functions (which will follow from the complex exponential):
}
\begin{align*}
\cos(\alpha+\beta)&=&\cos(\alpha)\cos(\beta)-\sin(\alpha)\sin(\beta)\\
\sin(\alpha+\beta)&=&\sin(\alpha)\cos(\beta)+\cos(\alpha)\sin(\beta)
\end{align*}
\lang{de}{
Für das Produkt $z\cdot w$ gilt dann:
}
\lang{en}{
The product $z \cdot w$ is then
}
\begin{eqnarray*}
z\cdot w &=&|z|\cdot(\cos(\beta)+i\sin(\beta))\cdot |w|\cdot(\cos(\alpha)+i\sin(\alpha))\\
&=& |z|\cdot|w|\cdot(\cos(\beta)\cos(\alpha)+i\cos(\beta)\sin(\alpha)+
i\sin(\beta)\cos(\alpha)-\sin(\beta)\sin(\alpha))\\
&=&|z|\cdot|w|\cdot(\cos(\beta)\cos(\alpha)-\sin(\beta)\sin(\alpha)
+i(\cos(\beta)\sin(\alpha)+\sin(\beta)\cos(\alpha)))\\
&=&|z\cdot w|\cdot(\cos(\alpha+\beta)+i \sin(\alpha+\beta))
\end{eqnarray*}
\lang{de}{
Hier wird die Eleganz der Polarkoordinaten deutlich: Die Multiplikation komplexer Zahlen
entspricht einer Addition der Argumente und einer Multiplikation der Beträge.
}
\lang{en}{
Here, the elegance of polar coordinates is clear: multiplication of two complex numbers
corresponds to adding their arguments and multiplying their magnitudes.
}

\begin{rule}[
\lang{de}{Multiplikation in Polardarstellung}
\lang{en}{Multiplying in polar coordinates}]
Bei Multiplikation zweier komplexer Zahlen werden also die \notion{Beträge multipliziert} und
die \notion{Argumente addiert}. 

Genauer: Für das Argument des Produktes $z\cdot w$ gilt
 \[\arg(z\cdot w)=\begin{cases}\arg(z)+\arg(w), & \text{falls }-\pi<\arg(z)+\arg(w)\leq \pi, \\
 \arg(z)+\arg(w)-2\pi, & \text{falls }\arg(z)+\arg(w)>\pi, \\
 \arg(z)+\arg(w)+2\pi,& \text{falls }\arg(z)+\arg(w)\leq -\pi. \\
 \end{cases}\]
Die Terme $-2\pi$ und $+2\pi$ dienen lediglich dazu, dass das Argument wieder im
Hauptbereich $(-\pi,\pi]$ liegt.

\begin{center}
\image{T203_ComplexMultiplication}
\end{center}

Anschaulich bedeutet das, dass die Multiplikation mit einer komplexen Zahl $z\neq 0$ auf der Gaußschen Zahlenebene eine Drehstreckung
 um den Ursprung bewirkt mit Drehwinkel $\arg(z)$ und Streckfaktor $|z|$.
\end{rule}

\begin{example}
	\begin{genericGWTVisualization}[550][800]{mathlet1}
		\begin{variables}
			\point{A}{rational}{1,0}
			\point{B}{rational}{2,-1}
			\point{C}{rational}{3,1}
			\point[editable]{z}{rational}{-1,2}
			\point{zA}{rational}{var(z)[x],var(z)[y]}
			\point{zB}{rational}{var(z)[x]*var(B)[x]-var(z)[y]*var(B)[y],var(z)[x]*var(B)[y]+var(z)[y]*var(B)[x]}
			\point{zC}{rational}{var(z)[x]*var(C)[x]-var(z)[y]*var(C)[y],var(z)[x]*var(C)[y]+var(z)[y]*var(C)[x]}

			\segment{AB}{rational}{var(A),var(B)}
			\segment{BC}{rational}{var(B),var(C)}
			\segment{AC}{rational}{var(A),var(C)}
			
		\end{variables}


		\color{A}{#0066CC}
		\color{B}{#0066CC}
		\color{C}{#0066CC}
%		\label{A}{$\textcolor{BLUE}{A}$}
		\color{z}{#CC6600}
		\color{zA}{#CC6600}
		\color{zB}{#CC6600}
		\color{zC}{#CC6600}
		\label{z}{$\textcolor{#CC6600}{z}$}

		\begin{canvas}
			\updateOnDrag[false]
			\plotSize{400}
			\plotLeft{-7}
			\plotRight{7}
			\plot[coordinateSystem]{A,B,C, z,zA, zB, zC}
		\end{canvas}
		\text{Blau sind die komplexen Zahlen $1$, $2-i$ und $3+i$ dargestellt. Orange dargestellt sind ihre Bilder nach der Multiplikation
		mit $z=\var{z}[x]+\var{z}[y]i$, also die Zahlen\\
		$ (\var{zA}[x]+\var{zA}[y]i)\cdot 1= \var{zA}[x]+\var{zA}[y]i$,\\
		$ (\var{zA}[x]+\var{zA}[y]i)\cdot (2-i)= \var{zB}[x]+\var{zB}[y]i$,\\
		$ (\var{zA}[x]+\var{zA}[y]i)\cdot (3+i)= \var{zC}[x]+\var{zC}[y]i$.}
	    	\end{genericGWTVisualization}

\end{example}

\begin{example}
Wir berechnen das Produkt der komplexen Zahl $z=\sqrt{3}+i$ mit sich selbst:
\[|z|=\sqrt{\sqrt{3}^2+1^2}=\sqrt{3+1}=2 \ \text{ und } \ \tan\phi=\frac{1}{\sqrt{3}}=\frac{\frac{1}{2}}{\frac{\sqrt{3}}{2}}, \text{ also }\]
\[\phi=\frac{\pi}{6}\hat{=}30^\circ\text{, denn } \sin(30^\circ)=\frac{1}{2} \ \text{ und } \ \cos(30^\circ)=\frac{\sqrt{3}}{2}. \]
\[\text{ Damit folgt: } \ z\cdot z=2\cdot 2 \cdot (\cos\left(\frac{\pi}{3}\right)+i\sin\left(\frac{\pi}{3}\right))=4\left(\frac{1}{2}+i\frac{\sqrt{3}}{2}\right)=2+2\sqrt{3}i.\]
Die Probe liefert
\[(\sqrt{3}+i)\cdot(\sqrt{3}+i)=\sqrt{3}^2+\sqrt{3}i+\sqrt{3}i+i^2=3+2\sqrt{3}i-1=2+2\sqrt{3}i.\]
\end{example}


\begin{quickcheck}
\field{complex}
\type{input.function}
    \begin{variables}
        \randint{a}{1}{2}
        \randint{b}{3}{7}
        \function[calculate]{c}{a*b}
        \function{z}{a(cos(pi/6)+i*sin(pi/6))}
        \function{w}{b(cos(pi/6)+i*sin(pi/6))}
        \function{sol}{c(cos(2*pi/6)+i*sin(2*pi/6))}
     \end{variables}
     \text{Berechnen Sie das Produkt der Zahlen $z=\var{z} \ $ und $ \ w=\var{w}$.\\
     $z\cdot w=$\ansref}
     \begin{answer}
        \solution{sol}
     \end{answer}
     \explanation{Die Beträge werden multipliziert, die Argumente addiert.}
\end{quickcheck}

\section{Komplexe Lösungen quadratischer Gleichungen}\label{sec:komplexe-quadratische-gleichungen}


Die Gleichung $x^2+1=0$ hat über den komplexen Zahlen die Lösungsmenge $\mathbb{L}=\{i;-i\}$, denn $(\pm i)^2=-1$.\\ 
Die Gleichung $x^2+a=0$ hat für $a>0$ die Lösungsmenge $\mathbb{L}=\{i\sqrt{a};-i\sqrt{a}\}$, 
denn $(\pm i\sqrt{a})^2=i^2a=-a$. \\ 

Die quadratische Gleichung $x^2+px+q=0$ hat über $\R$ die Lösungen
\[x_1=-\frac{p}{2}+\sqrt{D} \ \text{ und } \ x_2=-\frac{p}{2}-\sqrt{D},\]
falls $D=\left(\frac{p}{2}\right)^2-q\geq 0$ erfüllt ist.

Falls $D<0$ gilt, hat die quadratische Gleichung zwei Lösungen in $\C$, denn für $D$ finden 
wir die Quadratwurzeln $i\sqrt{|D|}$ sowie $-i\sqrt{|D|}$. Weitere Lösungen sind nach dem
\ref[content_03_komplexeZahlen_hauptsatz][Fundamentalsatz der Algebra]{sec:fundamentalsatz-der-algebra} nicht möglich.\\ 
Im Fall, dass die Diskriminante $D$ negativ ist, gilt: $(\pm i\sqrt{|D|})^2=i^2|D|=-1|D|=-1\cdot(-D)=D$.\\
Die Lösungen über $\C$ lauten daher 
\[z_1=-\frac{p}{2}+i\sqrt{|D|} \ \text{  und  } \ z_2=-\frac{p}{2}-i\sqrt{|D|}.\]


\begin{proof}[Beweis Quadratwurzeln]
\begin{showhide}
Wir setzen $z_1$ und $z_2$ in $x^2+px+q$ ein:
\begin{eqnarray*}
z_{1/2}^2+p\cdot z_{1/2}+q &=& \left(-\frac{p}{2}\pm i\sqrt{|D|}\right)^2+p\cdot \left(-\frac{p}{2}\pm i\sqrt{|D|}\right)+q\\
&=&\frac{p^2}{4}\mp2\frac{p}{2}i\sqrt{|D|}+i^2|D|-\frac{p^2}{2}\pm ip\sqrt{|D|}+q\\
&=&-\frac{p^2}{4}-|D|+q=-\frac{p^2}{4}+\left(\left(\frac{p}{2}\right)^2-q \right)+q\\
&=&0
\end{eqnarray*}
\end{showhide}
\end{proof}

\begin{theorem}[komplexe Quadratwurzeln]
Über der Grundmenge $\C$ hat jedes quadratische Polynom $z^2+pz+q$ zwei Nullstellen
$z_1,z_2\in\C$, die gleich sein können. Es gilt außerdem die Linearfaktorzerlegung
\[z^2+pz+q=(z-z_1)\cdot(z-z_2).\]
\end{theorem}

\begin{block}[warning]
Die Potenzgesetze gelten nur für positive, reelle Basen!\\
Bei negativen oder komplexen Basen treten Fehler auf. 

Beispiel: 
\[
-3 = -3^1 = (-3)^{\frac{2}{2}} \neq ((-3)^2)^{\frac{1}{2}} =  \sqrt{(-3)^2}=\sqrt{9}=3
\]  

Dieser Widerspruch liegt in den Argumenten komplexer Zahlen begründet.
\end{block}

\begin{quickcheck}
\field{complex}
\type{input.number}
\displayprecision{2}
\correctorprecision{2}
\begin{variables}
    \drawFromSet{p}{2,4,6}
    \drawFromSet{q}{10,11,12,13}
    \function{tmp1}{-p/2}
    \function{tmp2}{q-(p/2)^2}
    \function[normalize]{sol1}{tmp1+sqrt(tmp2)*i}
    \function[normalize]{sol2}{tmp1-sqrt(tmp2)*i}
\end{variables}
%Leider noch nicht für Quickchecks verfügbar.
%\permuteAnswers{1,2}
\text{Bestimmen Sie die Lösungen der quadratischen Gleichung
\[z^2+\var{p}z+\var{q}=0.\]
Geben Sie Ihre Lösung auf 2 Nachkommastellen gerundet an.\\
(positiver Wurzelterm) $z_1=$\ansref und \\
(negativer Wurzelterm) $z_2=$\ansref
}
\explanation{Die Lösungen sind $z_{1/2}=\var{tmp1}\pm\sqrt{\var{tmp2}}\cdot i$.}

\begin{answer}
    \solution{sol1}
\end{answer}
\begin{answer}
    \solution{sol2}
\end{answer}
\end{quickcheck}

\section{Der Fundamentalsatz der Algebra}\label{sec:fundamentalsatz-der-algebra}

Abschließend gehen wir noch einmal auf den anfangs genannten Grund für die Einführung der komplexen
Zahlen ein, nämlich das Problem, dass gewisse Polynomgleichungen über den reellen Zahlen keine
Lösung haben. In den komplexen Zahlen haben solche Gleichungen jedoch immer (mindestens) eine Lösung. Diese Tatsache
nennt man den \notion{Fundamentalsatz der Algebra}.

\begin{theorem}[Fundamentalsatz der Algebra]
Ist $n\geq 1$ und $p(x)=x^n+a_{n-1}x^{n-1}+\ldots + a_1x+a_0$ ein Polynom mit komplexen Koeffizienten 
(d.\,h. $a_0, a_1, \ldots, a_{n-1}\in \C$), so besitzt $p(x)$ eine Nullstelle, d.\,h. es gibt eine komplexe Zahl $c\in \C$ mit
\[  p(c)=c^n+a_{n-1}c^{n-1}+\ldots + a_1c+a_0=0. \]
\end{theorem}

Aus dem Fundamentalsatz der Algebra erhält man direkt folgende stärkere Aussage über Polynome mit komplexen Koeffizienten: 
\begin{theorem}
Ist $n\geq 1$ und $p(x)=x^n+a_{n-1}x^{n-1}+\ldots + a_1x+a_0$ ein Polynom mit komplexen Koeffizienten 
(d.\,h. $a_0, a_1,\ldots, a_{n-1}\in \C$), dann gibt es komplexe Zahlen $c_1, c_2, \ldots, c_n\in \C$ (welche auch gleich sein können) so,
dass
\[ p(x)=(x-c_1)\cdot (x-c_2)\cdots (x-c_n). \]
\end{theorem}

\begin{proof*}
\begin{incremental} 
\step
Ist nämlich $c_1$ eine Nullstelle von $p(x)$, so kann man von $p$ einen Linearfaktor $x-c_1$ abspalten, d.\,h. ein Polynom $q(x)$ berechnen
mit $p(x)=q(x)\cdot (x-c_1)$. Die Abspaltung von Linearfaktoren kennen wir bereits vom 
Horner-Schema, vgl. \link{content_10_polynomdivision}{Kapitel zu Polynomdivision}. Nun besitzt $q(x)$, wenn sein Grad mindestens 1 ist, nach dem Fundamentalsatz 
der Algebra wiederum eine
Nullstelle, d.\,h. man kann von $q(x)$ wieder einen Linearfaktor $(x-c_2)$ abspalten etc. Dies führt man fort, bis das zuletzt berechnete 
Polynom auch linear ist, also von der Form $ax+b$ mit $a\neq 0$.
\step
Dann ist
\[ p(x)=(x-c_1)\cdot (x-c_2)\cdots (x-c_{m-1})\cdot (ax+b). \]
Multipliziert man die rechte Seite aus und vergleicht mit $p(x)$, so sieht man am höchsten Koeffizienten, dass zum einen $m=n$ sein muss und
zum anderen $a=1$. Setzt man $c_n=-b$, so erhält man schließlich die Form
\[ p(x)=(x-c_1)\cdot (x-c_2)\cdots (x-c_n). \]
\end{incremental}
\end{proof*}

Es sei noch angemerkt, dass man die Aussage der obigen Theoreme auch leicht auf Polynome $p(x) = a_n x^n + ... + a_0$ mit $a_n \neq 0$ erweitern kann,
indem wir die Theoreme auf $\frac{1}{a_n} p(x) = x^n + \frac{1}{a_n} a_{n-1}x^{n-1}+...+\frac{1}{a_n}a_0$ anwenden.

\begin{quickcheck}
\field{real}
\type{input.number}
\begin{variables}
    \randint{a}{1}{5}
    \number{b}{i}
    \number{c}{-i}
    \randint{d}{2}{3}
    \function[expand,normalize]{p}{(x^2+1)(x-a)}
    \function[expand,normalize]{q}{d*(x^2+1)(x-a)}
\end{variables}
\text{Bestimmen Sie die (einzige) reelle Nullstelle des folgenden Polynoms: 
\[p(x)=\var{p}\]
$c_1=$\ansref
}
\begin{answer}
    \solution{a}
\end{answer}
\text{Schreiben Sie das Polynom nun als Produkt von Linearfaktoren
\[p(x)=(x-c_1)(x-c_2)(x-c_3).\]
Geben Sie zuerst die reelle Nullstelle ein.

$p(x)=(x-$\ansref$)(x-$\ansref$)(x-$\ansref$)$
}
\begin{answer}
    \solution{a}
\end{answer}
\begin{answer}
    \solution{b}
\end{answer}
\begin{answer}
    \solution{c}
\end{answer}
\explanation{Die einzige reelle Nullstelle ist $c_1=\var{a}$. Ein Hinweis darauf ist der letzte Term des Polynoms. 
Damit lautet das Polynom dann $p(x)=(x-\var{a})(x-i)(x+i)$.
}
\text{Betrachten Sie nun ein zweites Polynom:
\[q(x)=\var{q}\]
Es hat dieselben Nullstellen wie $p(x)$, 
unterscheidet sich lediglich um den Faktor \ansref vom ersten Polynom.
}
\begin{answer}
    \solution{d}
\end{answer}
\end{quickcheck}

Das Kapitel ist im folgenden Video nochmals dargestellt:
\floatright{\href{https://api.stream24.net/vod/getVideo.php?id=10962-2-10826&mode=iframe&speed=true}{\image[75]{00_video_button_schwarz-blau}}}\\


\end{visualizationwrapper}
\end{content}

