\documentclass{mumie.element.exercise}
%$Id$
\begin{metainfo}
  \name{
    \lang{de}{Ü02: Gauss'sche Zahlenebene}
    \lang{en}{Ex02: Complex plane}
  }
  \begin{description} 
 This work is licensed under the Creative Commons License Attribution 4.0 International (CC-BY 4.0)   
 https://creativecommons.org/licenses/by/4.0/legalcode 

    \lang{de}{Hier die Beschreibung}
    \lang{en}{}
  \end{description}
  \begin{components}
  \end{components}
  \begin{links}
  \end{links}
  \creategeneric
\end{metainfo}
\begin{content}
\title{
\lang{de}{Ü02: Gauss'sche Zahlenebene}
\lang{en}{Ex02: Complex plane}
}
\begin{block}[annotation]
  Im Ticket-System: \href{http://team.mumie.net/issues/9744}{Ticket 9744}
\end{block}

\lang{de}{In welchem Quadranten 
bzw. auf welcher Achse der Gaußschen Zahlenebene liegen die folgenden komplexen Zahlen? 
Berechnen Sie ihre Beträge und Argumente (Phasenwinkel).}
\lang{en}{In which quadrant respectively on which axes of the complex plane are the following
complex numbers?
Calculate their absolute value and their argument (angle $\varphi$).}
\begin{table}[\class{items}]
  \nowrap{a) 
  $\ 3 + 4\,{i}$} & 
  \nowrap{b) 
  $\ -{i} - 1 $} &
  \nowrap{c) 
  $\ {i} \cdot \sqrt{3} - 1$} \\
  \nowrap{d) 
  $\ -42$} &
  \nowrap{e) 
  $\ 7 \, {i}$} &
  \nowrap{f) 
  $\ 0$}
\end{table}

\begin{tabs*}[\initialtab{0}\class{exercise}]
  \tab{
  \lang{de}{Antwort}
  \lang{en}{Answer}
  }
  \lang{de}{\begin{table}[\class{items}]
    \nowrap{a) 
    Erster Quadrant, \quad 
    $r = 5, \quad 
    \varphi = \arccos \left(\frac{3}{5}\right)$} \\ 
    \nowrap{b) 
    Dritter Quadrant, \quad
    $r = \sqrt{2}, \quad 
    \varphi = -\frac{3}{4} \pi$} \\
    \nowrap{c) 
    Zweiter Quadrant, \quad
    $r = 2, \quad 
    \varphi = \frac{2}{3} \pi$}\\ 
    \nowrap{d) Negative reelle Achse, \quad
    $r = 42, \quad 
    \varphi = \pi$} \\
    \nowrap{e) 
    Positive imaginäre Achse, \quad
    $r = 7, \quad 
    \varphi = \frac{\pi}{2}$} \\
    \nowrap{f) 
    Auf beiden Achsen im Ursprung, \quad
    $r = 0, \quad 
    \varphi= \text{unbestimmt}$} 
  \end{table}}
  
  \lang{en}{\begin{table}[\class{items}]
    \nowrap{a) 
    First quadrant, \quad 
    $r = 5, \quad 
    \varphi = \arccos \left(\frac{3}{5}\right)$} \\ 
    \nowrap{b) 
    Third quadrant, \quad
    $r = \sqrt{2}, \quad 
    \varphi = -\frac{3}{4} \pi$} \\
    \nowrap{c) 
    Second quadrant, \quad
    $r = 2, \quad 
    \varphi = \frac{2}{3} \pi$}\\ 
    \nowrap{d) Negative real axis, \quad
    $r = 42, \quad 
    \varphi = \pi$} \\
    \nowrap{e) 
    Positive imaginary axis, \quad
    $r = 7, \quad 
    \varphi = \frac{\pi}{2}$} \\
    \nowrap{f) 
    In the origin of both axes, \quad
    $r = 0, \quad 
    \varphi= \text{undefined}$} 
  \end{table}}

  \tab{
  \lang{de}{Lösung a)}
  \lang{en}{Solution a)}}
  \begin{incremental}[\initialsteps{1}]
    \step 
    \lang{de}{Da der Realteil und der Imaginärteil der Zahl beide positiv sind,
    liegt sie im ersten Quadranten.}
    \lang{en}{Real and imaginary part of the number are both positive,
    so it is in the first quadrant.}
    \step
    \lang{de}{Der Betrag berechnet sich als} 
    \lang{en}{The absolut value is}
    \begin{align*}
      r 
      &= \sqrt{\Re^2(z) + \Im^2(z)} \\
      &= \sqrt{3^2 + 4^2} \\
      &= 5.
    \end{align*}

    \lang{de}{Da der Imaginärteil positiv ist, berechnet sich der Phasenwinkel als} 
    \lang{en}{Since the imaginary part is positive, the argument is} 
    \begin{align*}
      \varphi 
      &= \arccos \left(\frac{\Re(z)}{r}\right) \\
      &= \arccos \left(\frac{3}{5}\right).
    \end{align*}

  \end{incremental}

  \tab{
  \lang{de}{Lösung b)}
  \lang{en}{Solution b)}}
  \begin{incremental}[\initialsteps{1}]
    \step 
    \lang{de}{Da der Realteil und der Imaginärteil der Zahl beide negativ sind,
    liegt sie im dritten Quadranten.}
    \lang{en}{Since real and imaginary part of the number are both negativ,
    it is in the third quadrant.}
    \step
    \lang{de}{Der Betrag berechnet sich als}
    \lang{en}{The absolute value is calculated as}
    \begin{align*}
      r 
      &= \sqrt{\Re^2(z) + \Im^2(z)} \\
      &= \sqrt{(-1)^2 + (-1)^2} \\
      &= \sqrt{2}.
    \end{align*}

    \lang{de}{Da der Imaginärteil negativ ist, 
    berechnet sich der Phasenwinkel als} 
    \lang{en}{Since the imaginary part is negative, the argument is calculated as}
    \begin{align*}
      \varphi 
      = - \arccos \left(\frac{\Re(z)}{r}\right)
      = - \arccos \left(\frac{-1}{\sqrt{2}}\right) = -\frac{3}{4} \pi.
    \end{align*}

  \end{incremental}

  \tab{
  \lang{de}{Lösung c)}
  \lang{en}{Solution c)}}
  \begin{incremental}[\initialsteps{1}]
    \step 
    \lang{de}{Da der Realteil der Zahl negativ und der Imaginärteil positiv ist,
    liegt sie im zweiten Quadranten.}
    \lang{en}{Since the real part of the number is negative and the imaginary part is positive, it is in the
    second quadrant.}
    \step
    \lang{de}{Der Betrag berechnet sich als} 
    \lang{en}{The absolute value is calculated as}
    \begin{align*}
      r 
      &= \sqrt{\Re^2(z) + \Im^2(z)} \\ 
      &= \sqrt{(-1)^2 + \left(\sqrt{3}\right)^2} \\ 
      &= 2.
    \end{align*}

    \lang{de}{Da der Imaginärteil positiv ist,
    berechnet sich der Phasenwinkel als} 
    \lang{en}{Since the imaginary part is positive, the argument is calculated as}
    \begin{align*}
      \varphi 
      = \arccos \left(\frac{\Re(z)}{r}\right) 
      = \arccos \left(\frac{-1}{2}\right) = \frac{2}{3} \pi.
    \end{align*}


  \end{incremental}

  \tab{
  \lang{de}{Lösung d)}
  \lang{en}{Solution d)}}
  \begin{incremental}[\initialsteps{1}]
    \step 
    \lang{de}{Da der Realteil der Zahl negativ und der Imaginärteil gleich 0 ist,
    liegt sie auf der negativen reellen Achse.}
    \lang{en}{Since the real part of the number is negative and the imaginary part is equal to $0$,
    it is on the negative real axis.}
    \step
    \lang{de}{Da der Imaginärteil der Zahl verschwindet, 
    ist ihr Betrag gleich dem Betrag des Realteils:}
    \lang{en}{Since the imaginary part of the number is non-existent, its absolute value is equal to the
    absolute value of the real part:}
    \begin{align*}
      r 
      &= \abs{\Re(z)} \\
      &= \abs{-42} \\
      &= 42.
    \end{align*}

    \lang{de}{Da der Imaginärteil gleich 0 ist und der Realteil negativ, ist der Phasenwinkel gegeben durch $\varphi=\pi$.}
\lang{en}{Since the imaginary part is equal to $0$ and the real part is negative, the argument is given by $\varphi=\pi$.}

  \end{incremental}

  \tab{
  \lang{de}{Lösung e)}
  \lang{en}{Solution e)}}
  \begin{incremental}[\initialsteps{1}]
    \step 
    \lang{de}{Da der Realteil der Zahl gleich 0 und 
    der Imaginärteil positiv ist,
    liegt sie auf der positiven imaginären Achse.}
    \lang{en}{Since the real part of the number is equal to $0$ and the imaginary part is positive, it
    is on the positive imaginary axis.}
    \step
    \lang{de}{Da der Realteil der Zahl verschwindet, 
    ist ihr Betrag gleich dem Betrag des Imaginärteils:} 
    \lang{en}{The real part of the number is non-existent and therefore its absolute value is equal to the absolute
    value of the imaginary part:}
    \begin{align*}
      r 
      &= \abs{\Im(z)} \\
      &= \abs{7} \\
      &= 7.
    \end{align*}

	\step
	\lang{de}{Da der Imaginärteil positiv ist, berechnet sich der Phasenwinkel als}
 \lang{en}{Since the imaginary part is positive, the argument is calculated as}
	\[ \varphi=\arccos\left(\frac{\Re(z)}{r}\right)=\arccos\left(\frac{0}{7}\right)=\frac{1}{2}\pi.\]

  \end{incremental}

  \tab{
  \lang{de}{Lösung f)}}
  \begin{incremental}[\initialsteps{1}]
    \step 
    \lang{de}{Da der Realteil und der Imaginärteil der Zahl beide verschwinden,
    liegt sie sowohl auf der reellen als auch auf der imaginären Achse.}
    \step
    \lang{de}{Der Betrag berechnet sich als} 
    \begin{align*}
      r 
      &= \sqrt{\Re^2(z) + \Im^2(z)} \\
      &= \sqrt{0^2 + 0^2} \\
      &= 0.
    \end{align*}

	\step
    \lang{de}{Nach Skript ist das Argument von $0$ nicht definiert.}


  \end{incremental}

\end{tabs*}

\end{content}