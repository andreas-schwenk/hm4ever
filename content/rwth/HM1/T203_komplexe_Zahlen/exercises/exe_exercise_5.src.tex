\documentclass{mumie.element.exercise}
%$Id$
\begin{metainfo}
  \name{
    \lang{de}{Ü05: Produkte/Quotienten}
    \lang{en}{Ex05: Products, Quotients}
  }
  \begin{description} 
 This work is licensed under the Creative Commons License Attribution 4.0 International (CC-BY 4.0)   
 https://creativecommons.org/licenses/by/4.0/legalcode 

    \lang{de}{}
    \lang{en}{}
  \end{description}
  \begin{components}
  \end{components}
  \begin{links}
  \end{links}
  \creategeneric
\end{metainfo}
\begin{content}
\title{
\lang{de}{Ü05: Produkte/Quotienten}
\lang{en}{Ex05: Products, Quotients}
}
\begin{block}[annotation]
  Im Ticket-System: \href{http://team.mumie.net/issues/9748}{Ticket 9748}
\end{block}

\begin{enumerate}
\item
\lang{de}{Berechnen Sie die folgenden Produkte bzw. Quotienten. Geben Sie die Lösung in der Form $a+b i$ für reelle Zahlen $a$ und $b$ an.}
\lang{en}{Calculate the following products and quotients. Give the solution in the form $a+b i$ for $a,b\in\mathbb{R}$.}
\begin{table}[\class{items}]
  \nowrap{a) 
  $\ (2 + 3 \, {i}) \cdot (4 + 5 \, {i})$} &
  \nowrap{b) 
  $\ \frac{6 - 4 \, {i}}{2 - 8 \, {i}}$}
\end{table}

\item
\begin{enumerate}
\item
\lang{de}{
Berechnen Sie $\frac{1}{z}$ und visualisieren Sie die Ergebnisse zu:}
\lang{en}{Calculate $\frac{1}{z}$ and visualize the results for:}
\begin{table}[\class{items}]
  \nowrap{(i) 
  $\ 3 + 4 \, {i}$} &
  \nowrap{(ii) 
  $\ 1 - \, {i} $} &
  \nowrap{(iii) 
  $\ 2 \, {i}$}
\end{table}

\item
\lang{de}{
Berechnen Sie:}
\lang{en}{Calculate:}
\begin{table}[\class{items}]
  \nowrap{(i) 
  $\ \frac{2+ \, {i}}{3 + 4 \, {i}}$} &
  \nowrap{(ii) 
  $\ \frac{ \, {i}}{1 - 2 \, {i}}$} &
  \nowrap{(iii) 
  $\ \frac{4 + 2 \, {i}}{2 + \, {i}}$} &
  \nowrap{(iv) 
  $\ \frac{1 + 3 \, {i}}{ \, {i}}$}
\end{table}
\end{enumerate}

\end{enumerate}

\begin{tabs*}[\initialtab{0}\class{exercise}]
  \tab{
  \lang{de}{Antworten}
  \lang{en}{Answer}
  }
  \begin{enumerate}
  \item
   \begin{table}[\class{items}]
    \nowrap{a) 
    $\ -7 + 22  i$} \\ 
    \nowrap{b) 
    $\ \frac{11}{17} + \frac{10}{17} i$} 
   \end{table}
  
  \item
  \begin{enumerate}
  \item
    \begin{table}[\class{items}]
    \nowrap{(i) 
    $\ \frac{3}{25} - \frac{4}{25} i $} \\ 
    \nowrap{(ii) 
    $\ \frac{1}{2} + \frac{1}{2} i$} \\ 
    \nowrap{(iii) 
    $\ -\frac{1}{2}  i$} \\ 
    \end{table}
  \item
    \begin{table}[\class{items}]
    \nowrap{(i) 
    $\ \frac{2}{5} - \frac{1}{5} i$} \\ 
    \nowrap{(ii) 
    $\ -\frac{2}{5} + \frac{1}{5}i$} \\ 
    \nowrap{(iii) 
    $\ 2$} \\ 
    \nowrap{(iv) 
    $\ 3- i$} \\ 
    \end{table}
   \end{enumerate}
   
  \end{enumerate}
  
 

  \tab{
  \lang{de}{Lösung 1a)}
  \lang{en}{Solution 1a)}}
  \begin{incremental}[\initialsteps{1}]
    \step 
    \lang{de}{Die beiden komplexen Zahlen werden multipliziert,
    indem jeder Summand der ersten Klammer 
    mit jedem Summanden der zweiten Klammer multipliziert wird:}
    \lang{en}{The two complex numbers are multiplied, by multiplying each summand in the first bracket with
    each summand in the second bracket:}
    \step 
    \begin{equation*}
      (2 + 3 \, {i}) \cdot (4 + 5 \, {i}) = 
      2 \cdot 4 + 
      2 \cdot 5 \, {i} + 
      3 \, {i} \cdot 4 + 
      3 \, {i} \cdot 5 \, {i}.
    \end{equation*}

    \step 
    \lang{de}{Unter Berücksichtigung der Tatsache, 
    dass 
    ${i} \cdot {i} = {i}^2 = -1$ 
    ist, 
    können die Realteile und die Imaginärteile zusammengefasst werden:}
    \lang{en}{With respect of 
    ${i} \cdot {i} = {i}^2 = -1$ , 
    we can combine the real and imaginary parts:}
    \begin{align*}
      2 \cdot 4 + 
      2 \cdot 5 \, {i} + 
      3 \, {i} \cdot 4 + 
      3 \, {i} \cdot 5 \, {i}
      &= 
      (2 \cdot 4 - 3 \cdot 5) + 
      (2 \cdot 5 + 3 \cdot 4) {i} 
      \\ &= 
      -7 + 22 \, {i}.
    \end{align*}

  \end{incremental}


  \tab{
  \lang{de}{Lösung 1 b)}
  \lang{en}{Solution 1b)}}
  \begin{incremental}[\initialsteps{1}]
    \step 
    \lang{de}{Als Erstes kann der Bruch durch $2$ gekürzt werden:} 
    \lang{en}{First of all, reduce the fraction by $2$:}
    \step 
    \begin{equation*}
      \frac{6 - 4 \, {i}}{2 - 8 \, {i}} 
      = 
      \frac{3 - 2 \, {i}}{1 - 4 \, {i}}.
    \end{equation*}

    \step 
    \lang{de}{Im nächsten Schritt erweitert man den Bruch 
    mit dem konjugiert komplexen Nenner und 
    multipliziert Zähler und Nenner 
    unter Berücksichtigung von
    ${i} \cdot {i} = {i}^2 = -1$ 
    aus. 
    Im Nenner kann man die dritte binomische Formel anwenden,
    sodass der Imaginärteil verschwindet:}
    \lang{en}{In the next step we reduce the fraction to higher terms with the complex conjugate of the denominator and
    expand the numerator and denominator with respect of
    ${i} \cdot {i} = {i}^2 = -1$. 
    We utilise the third binomial formular for the dnominator, which erases the imaginary part:}
    \step 
    \begin{align*}
      \frac{3 - 2 \, {i}}{1 - 4 \, {i}}
      &= 
      \frac{3 - 2 \, {i}}{1 - 4 \, {i}}
      \cdot
      \frac{1 + 4 \, {i}}{1 + 4 \, {i}} 
      \\ &= 
      \frac{
      3 \cdot 1 + 
      3 \cdot 4 \, {i}
      -2 \, {i} \cdot 1 
      -2 \, {i} \cdot 4 \, {i}
      }{1^2 + 4^2} 
      \\ &= 
      \frac{11 + 10 \, {i}}{17}. 
    \end{align*}

    \step 
    \lang{de}{Im letzten Schritt kann man dann noch 
    den Realteil und den Imaginärteil des Zählers einzeln
    durch den reellen Nenner teilen:} 
    \lang{en}{In a last step, we divide the real and the imaginary part of the numerator separately by the real denominator:}
    \step 
    \begin{equation*}
      \frac{11 + 10 \, {i}}{17} 
      = 
      \frac{11}{17} 
      +
      \frac{10}{17} \, {i}. 
    \end{equation*}

  \end{incremental}

 \tab{\lang{de}{Lösungsvideo 2)}}	
    \lang{de}{\youtubevideo[500][300]{IKhd_SmRmMk}\\
    Beachten Sie, dass die imaginäre Einheit $i$ im Video durch $j$ repräsentiert wird.}


\end{tabs*}
\end{content}