\documentclass{mumie.element.exercise}
%$Id$
\begin{metainfo}
  \name{
    \lang{de}{Ü01: Realteil, Imaginärteil}
    \lang{en}{Ex01: Real and imaginary part}
  }
  \begin{description} 
 This work is licensed under the Creative Commons License Attribution 4.0 International (CC-BY 4.0)   
 https://creativecommons.org/licenses/by/4.0/legalcode 

    \lang{de}{Hier die Beschreibung}
    \lang{en}{}
  \end{description}
  \begin{components}
  \end{components}
  \begin{links}
  \end{links}
  \creategeneric
\end{metainfo}
\begin{content}
\title{
\lang{de}{Ü01: Realteil, Imaginärteil }
\lang{en}{Ex01: Real and imaginary part}
}
\begin{block}[annotation]
  Im Ticket-System: \href{http://team.mumie.net/issues/9743}{Ticket 9743}
\end{block}
\begin{enumerate}
\item
\lang{de}{
Bestimmen Sie zu jeder der folgenden komplexen Zahlen die konjugiert komplexe Zahl 
und geben Sie an, ob die Zahlen reell, rein imaginär oder lediglich komplex sind.}
\lang{en}{
Determine for each of the following complex numbers the complex conjute and indicate, if the numbers are real, purely 
imaginary or only complex.
}
\begin{table}[\class{items}]
  \nowrap{a) 
  $\ 2 + 3\,{i}$} & 
  \nowrap{b) 
  $\ -42 - 3 \, {i}$} &
  \nowrap{c) 
  $\ {i} \cdot \sqrt{2} + 1$} \\
  \nowrap{d) 
  $\ 666$} &
  \nowrap{e) 
  $\ 5 \, {i}$} &
  \nowrap{f) 
  $\ 0$}
\end{table}
\item
\lang{de}{
Geben Sie $\Re(z)$, $\Im(z)$, $\bar{z}$ und $|z|$ an zu:}
\lang{en}{
Give $\Re(z)$, $\Im(z)$, $\bar{z}$ and $|z|$ for:}
\begin{table}[\class{items}]
  \nowrap{a) 
  $\ 3 - 2\,{i}$} & 
  \nowrap{b) 
  $\ 1 +  \, {i}$} &
  \nowrap{c) 
  $\ 2\,{i} $} \\
  \nowrap{d) 
  $\ -1$} & &  
\end{table}
\end{enumerate}

\begin{tabs*}[\initialtab{0}\class{exercise}]
  \tab{
  \lang{de}{Antworten}
  \lang{en}{Answer}
  }
  \begin{enumerate}
  \item
  \lang{de}{
  \begin{table}[\class{items}]
    
    \nowrap{a) 
    $\ 2 - 3\,{i}$, komplex} & 
    \nowrap{b) 
    $\ -42 + 3 \, {i}$, komplex} \\
    \nowrap{c) 
    $\ -{i} \cdot \sqrt{2} + 1$, komplex} & 
    \nowrap{d) 
    $\ 666$, reell, komplex}\\
    \nowrap{e) 
    $\ -5 \, {i}$, rein imaginär, komplex} &
    \nowrap{f) 
    $\ 0$, reell, rein imaginär, komplex}   
  \end{table}}

  \lang{en}{
  \begin{table}[\class{items}]
    
    \nowrap{a) 
    $\ 2 - 3\,{i}$, complex} & 
    \nowrap{b) 
    $\ -42 + 3 \, {i}$, complex} \\
    \nowrap{c) 
    $\ -{i} \cdot \sqrt{2} + 1$, complex} & 
    \nowrap{d) 
    $\ 666$, real, complex}\\
    \nowrap{e) 
    $\ -5 \, {i}$, purely imaginary, complex} &
    \nowrap{f) 
    $\ 0$, real, purely imaginary, complex}   
  \end{table}}
 
  \item
  \begin{table}[\class{items}]
    
    \nowrap{a) 
    $\ \Re(z)=3,\, \Im(z)=-2,\,\bar{z}=3+2i,\, |z|=\sqrt{13}$} \\
    \nowrap{b) 
    $\ \Re(z)=1,\, \Im(z)=1,\,\bar{z}=1-i,\, |z|=\sqrt{2}$}\\
    \nowrap{c) 
    $\ \Re(z)=0,\, \Im(z)=2,\,\bar{z}=-2i,\, |z|=2$} \\ 
    \nowrap{d) 
    $\ \Re(z)=-1,\, \Im(z)=0,\,\bar{z}=-1,\, |z|=1$}    
  \end{table}
    \end{enumerate}
  \tab{
  \lang{de}{Lösung 1a)}
  \lang{en}{Solution 1a)}}
  \begin{incremental}[\initialsteps{1}]
    \step 
    \lang{de}{Das Vorzeichen des Imaginärteils ($3$) wird umgedreht, 
    so dass die konjugiert komplexe Zahl $2 - 3\,{i}$ lautet.}
    lang{en}{The sign of the imaginary part ($3$) is reversed, 
    so that the complex conjugate is $2 - 3\,{i}$.}
    \step
    \lang{de}{Da die Zahl sowohl einen von Null verschiedenen Realteil ($2$) 
    als auch einen von Null verschiedenen Imaginärteil ($3$) besitzt, ist sie komplex, aber weder reell noch rein imaginär.}
  \lang{en}{Since the number has a non-zero real part ($2$) 
    and also a non-zero imaginary part ($3$), it is complex, but neither real nor purely imaginary.}
  \end{incremental}

  \tab{
  \lang{de}{Lösung 1b)}
  \lang{en}{Solution 1b)}}
  \begin{incremental}[\initialsteps{1}]
    \step 
    \lang{de}{Das Vorzeichen des Imaginärteils ($-3$) wird umgedreht, 
    so dass die konjugiert komplexe Zahl $-42 + 3\,{i}$ lautet.}
    \lang{en}{The sign of the imaginary part ($-3$) is reversed, 
    so that the complex conjugate is $-42 + 3\,{i}$.}
    \step
    \lang{de}{Da die Zahl sowohl einen von Null verschiedenen Realteil ($-42$) 
    als auch einen von Null verschiedenen Imaginärteil ($-3$) besitzt, ist sie komplex, aber weder reell noch rein imaginär.}
 \lang{en}{Since the number has a non-zero real part ($-42$) 
    and also a non-zero imaginary part ($-3$), it is complex, but neither real nor purely imaginary.}
 \end{incremental}

  \tab{
  \lang{de}{Lösung 1c)}
  \lang{en}{Solution 1c)}}
  \begin{incremental}[\initialsteps{1}]
    \step 
    \lang{de}{Das Vorzeichen des Imaginärteils ($\sqrt{2}$) wird umgedreht, 
    so dass die konjugiert komplexe Zahl $-{i} \cdot \sqrt{2} + 1$ lautet.}
    \lang{en}{The sign of the imaginary part ($\sqrt{2}$) is reversed, 
    so that the complex conjugate is $-{i} \cdot \sqrt{2} + 1$.}
    \step
    \lang{de}{Da die Zahl sowohl einen von Null verschiedenen Realteil ($1$) 
    als auch einen von Null verschiedenen Imaginärteil ($\sqrt{2}$) besitzt, ist sie komplex, aber weder reell noch rein imaginär.}
\lang{en}{Since the number hast a non-zero real part ($1$) 
    and also a non-zero imaginary part ($\sqrt{2}$), it is complex, but neither real nor purely imaginary.}  
  \end{incremental}

  \tab{
  \lang{de}{Lösung 1d)}
  \lang{en}{Solution 1d)}}
  \begin{incremental}[\initialsteps{1}]
    \step 
    \lang{de}{Da der Imaginärteil der Zahl $0$ ist, 
    lautet auch die konjugiert komplexe Zahl $666$.}
    \lang{en}{Since the imaginary part of the number is $0$, 
    the complex conjugate is also $666$.}
    \step
    \lang{de}{Da die Zahl keinen von Null verschiedenen Imaginärteil besitzt,
    ist sie reell.
    Da die reellen Zahlen eine Teilmenge der komplexen Zahlen sind, 
    ist die Zahl auch komplex.}
    \lang{en}{Since the imaginary part is equal to $0$, the number is real.
    Since the real numbers are a subset of the complex numbers, the number is also complex.}
  \end{incremental}

  \tab{
  \lang{de}{Lösung 1e)}
  \lang{en}{Solution 1e)}}
  \begin{incremental}[\initialsteps{1}]
    \step 
    \lang{de}{Das Vorzeichen des Imaginärteils ($5$) wird umgedreht, 
    so dass die konjugiert komplexe Zahl $-5 \, {i}$ lautet.}
    \lang{en}{The sign of the imaginary part ($5$) is reversed, 
    so that the complex conjugate is $-5 \, {i}$.}
    \step
    \lang{de}{Da die Zahl keinen von Null verschiedenen Realteil besitzt,
    ist sie rein imaginär. 
    Da die rein imaginären Zahlen eine Teilmenge der komplexen Zahlen sind, 
    ist die Zahl auch komplex.}
    \lang{en}{Since the real part of the number is equal to $0$, it is purely imaginary. 
    Since the purely imaginary numbers are a subset of the complex number, the number is also complex.}
  \end{incremental}

  \tab{
  \lang{de}{Lösung 1f)}
  \lang{en}{Solution 1f)}}
  \begin{incremental}[\initialsteps{1}]
    \step 
    \lang{de}{Da die Zahl keinen von Null verschiedenen Imaginärteil besitzt,
    dessen Vorzeichen umgedreht werden könnte, 
    lautet auch die konjugiert komplexe Zahl $0$.}
    \lang{en}{Since there is no imaginary part of the number, which sign could be reversed, the complex conjugate is also $0$.}
    \step
    \lang{de}{Da die Zahl keinen von Null verschiedenen Imaginärteil besitzt,
    ist sie reell.
    Da die Zahl einen verschwindenden Realteil besitzt,
    ist sie außerdem rein imaginär.
    Da sowohl die reellen Zahlen als auch die imaginären Zahlen 
    Teilmengen der komplexen Zahlen sind, 
    ist die Zahl auch komplex.}
     \lang{en}{Since the number does not have a non-zero imaginary part, it is real.
    But it is also purely imaginary, because the real teal is equal to $0$.
    The real numbers and also the purely imaginary numbers are a subset of the complex numbers, which is why
    the number is also complex.}
  \end{incremental}


  \tab{\lang{de}{Lösungsvideo 2 a)-d)}}	
    \lang{de}{\youtubevideo[500][300]{lf2bhWoBcqk}\\
    Beachten Sie, dass die imaginäre Einheit $i$ im Video durch $j$ repräsentiert wird. Ebenso wird die konjugiert komplexe Zahl $\overline{z}$ mit $z^*$ bezeichnet.}


\end{tabs*}

\end{content}