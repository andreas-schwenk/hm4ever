\documentclass{mumie.element.exercise}
%$Id$
\begin{metainfo}
  \name{
    \lang{de}{Ü06: Potenzen}
    \lang{en}{Ex06: Powers}
  }
  \begin{description} 
 This work is licensed under the Creative Commons License Attribution 4.0 International (CC-BY 4.0)   
 https://creativecommons.org/licenses/by/4.0/legalcode 

    \lang{de}{}
    \lang{en}{}
  \end{description}
  \begin{components}
  \end{components}
  \begin{links}
  \end{links}
  \creategeneric
\end{metainfo}
\begin{content}
\title{
\lang{de}{Ü06: Potenzen}
\lang{en}{Exercise 6}
}
\begin{block}[annotation]
  Im Ticket-System: \href{http://team.mumie.net/issues/9749}{Ticket 9749}
\end{block}

\begin{enumerate}
\item\lang{de}{Berechnen Sie die folgenden Potenzen.}
\lang{de}{Calculate the following powers.}
\begin{table}[\class{items}]
  \nowrap{a) 
  $\ {i}^42$} & 
  \nowrap{b) 
  $\ (3 - 2 \, {i})^2$} &
  \nowrap{c) 
  $\ (1 + {i})^0$} &
 % \nowrap{d) 
 % $\ (4 - 3 \, {i})^4$}
\end{table}

\item \lang{de}{Sei $z=1+i$.} \lang{en}{Let be $z=1+i$.}
\lang{de}{\begin{table}[\class{items}]
  \nowrap{a) 
  Berechnen Sie $|z^n|$, $n=1,2,\ldots,8$.}  
  \nowrap{b)
  Stellen Sie $z,z^2,z^3,\ldots,z^8$ in der Gauß'schen Zahlenebene dar.}
  \end{table}}
\lang{en}{\begin{table}[\class{items}]
  \nowrap{a) 
  Calculate $|z^n|$, $n=1,2,\ldots,8$.}  
  \nowrap{b)
  Visualise $z,z^2,z^3,\ldots,z^8$ in the complex plane.}
  \end{table}}
\end{enumerate}

\begin{tabs*}[\initialtab{0}\class{exercise}]
  \tab{
  \lang{de}{Antworten)}
  \lang{en}{Answer}}
  \begin{enumerate}
  \item
    \begin{table}[\class{items}]
    \nowrap{a) 
    $\ -1$} & 
    \nowrap{b) 
    $\ 5 - 12 \, {i}$} &
    \nowrap{c) 
    $\ 1$}&
   % \nowrap{d) 
   % $\ 625 \cdot e^{-2,574\,{i}}$}
    \end{table}
  \item
  \begin{enumerate}
  \item \lang{de}{Die Längen vergrößern sich jeweils um den Faktor $\sqrt{2}$, d.h. sie verdoppeln sich jeweils 
  nach zwei Schritten:}
  \lang{en}{The lengths increase each by the factor $\sqrt{2}$, i.e. they double each after two steps:}
  \begin{eqnarray*}
  |z|^1&=&\sqrt{2}\\
  |z|^2&=&2\\
  |z|^3&=&2\sqrt{2}\\
  |z|^4&=&4\\
  |z|^5&=&4\sqrt{2}\\
  |z|^6&=&8\\
  |z|^7&=&8\sqrt{2}\\
  |z|^8&=&16\\
  \end{eqnarray*}
  \item $\arg(z)=\frac{\pi}{4}$;\\
  \lang{de}{
  Beim Winkel kommen immer $45^\circ$ hinzu.}
  \lang{en}{
For the angle, $45^\circ$ are always added.}
  \end{enumerate}
  
  \end{enumerate}
  \tab{
  \lang{de}{Lösung 1)}
  \lang{de}{Solution 1)}}
  \begin{incremental}[\initialsteps{1}]
   
    \step
    \lang{de}{\textbf{a)}\\Unter Berücksichtigung von
    ${i} \cdot {i} = {i}^2 = -1$ 
    wird deutlich, dass die Potenzen von ${i}$
    eine Periode von $4$ besitzen, 
    dass also beispielsweise 
    ${i}^5$ 
    wieder den gleichen Wert hat wie
    ${i}^1$:}
    \lang{en}{\textbf{a)}\\With respect of
    ${i} \cdot {i} = {i}^2 = -1$ 
    we see, that the powers of ${i}$
    have a period of $4$, 
    so for example 
    ${i}^5$ 
    has the same value as
    ${i}^1$:}
    \begin{align*}
      {i}^0 
      &= 
      1,
      \\ 
      {i}^1 
      &= 
      {i},
      \\ 
      {i}^2 
      &= 
      -1,
      \\ 
      {i}^3 
      &= 
      {i}^2 \cdot {i} = (-1) \cdot {i} = -{i},
      \\ 
      {i}^4 
      &= 
      {i}^3 \cdot {i} = -{i} \cdot {i} = 1,
      \\
      {i}^5 
      &= 
      {i}^4 \cdot {i} = 1 \cdot {i} = {i}. 
    \end{align*}

    \step 
    \lang{de}{Aus diesem Grund können Vielfache der Periode
    (in diesem Fall 
    $10 \cdot 4 = 40$)
    im Exponenten abgezogen werden:} 
    \lang{en}{For this reason we may subtract multiples of the period
    (in this case 
    $10 \cdot 4 = 40$)
    in the exponent:}
    \begin{align*}
      {i}^42 
      &= 
      {i}^{42 - 40} 
      \\ &= 
      {i}^{2} 
      \\&= 
      -1. 
    \end{align*}
    
    \step 
    \lang{de}{\textbf{b)}\\Quadrate (und vielleicht auch noch Kuben)
    komplexer Zahlen der Form $a + b \, i$ lassen sich 
    unter Berücksichtigung von
    ${i}^2 = -1$ 
    durch Anwendung der binomischen Formeln berechnen:}
    \lang{en}{\textbf{b)}\\Squares (and maybe even cubes)
    of complex numbers of the form $a + b \, i$ can be calculated with respect of ${i}^2 = -1$ 
    by using the binomial formulas:}
    \begin{align*}
      (3 - 2 \, {i})^2 
      &=
      9 - 2 \cdot 3 \cdot 2 \, {i} + (2 \, {i})^2
      \\ &=
      9 - 12 \, {i} - 4
      \\ &=
      5 - 12 \, {i}.
    \end{align*}
    
    \step 
    \lang{de}{\textbf{c)}\\Jede (komplexe) Zahl außer $0$ ergibt, 
    wenn man sie mit $0$ potenziert, 
    den Wert $1$.} 
    \lang{en}{\textbf{c)}\\Each (complex) number except $0$ has, 
    when raised to the power of $0$,
    the value $1$.}


  \end{incremental}

 
  
   \tab{\lang{de}{Lösungsvideo 2)}}	
    \lang{de}{\youtubevideo[500][300]{HQOVJbUOL3c}\\
    Beachten Sie, dass die imaginäre Einheit $i$ im Video durch $j$ repräsentiert wird.}
  
  
 % \tab{
 % \lang{de}{Lösung d)}}
 % \begin{incremental}[\initialsteps{1}]
 %  \step 
 %   \lang{de}{Bei der Berechnung benutzen wir jetzt Polarkoordinaten:}
 %   \begin{align*}
 %     z&=&4-3\, {i}\\
 %     |z|&=&\sqrt{16+9}=5\\
 %     \phi&=&-\arctan\frac{3}{4}\approx-0,643
 %   \end{align*}
 %   Das Argument haben wir im Bogenmaß angegeben.\\
 %   Für $w=z^4$ gilt dann:
 %   \begin{align*}
 %   w&=&\left(5\cdot e^{-0,643\,{i}}\right)^4\\
 %   &=&5^4\cdot \left(e^{-0,643\,{i}}\right)^4\\
 %   &=&625 \cdot e^{4\left(-0,643\,{i}\right)}\\
 %  &=&625\cdot e^{-2,574\,{i}}
 %   \end{align*}
 %  
 % \end{incremental}

\end{tabs*}
\end{content}