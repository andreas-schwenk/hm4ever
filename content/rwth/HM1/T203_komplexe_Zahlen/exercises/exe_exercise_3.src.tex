\documentclass{mumie.element.exercise}
%$Id$
\begin{metainfo}
  \name{
    \lang{de}{Ü03: Aussagen}
    \lang{en}{Ex03: Statements}
  }
  \begin{description} 
 This work is licensed under the Creative Commons License Attribution 4.0 International (CC-BY 4.0)   
 https://creativecommons.org/licenses/by/4.0/legalcode 

    \lang{de}{}
    \lang{en}{}
  \end{description}
  \begin{components}
  \end{components}
  \begin{links}
  \end{links}
  \creategeneric
\end{metainfo}
\begin{content}
\title{
\lang{de}{Ü03: Aussagen}
\lang{en}{Ex03: Statements}
}
\begin{block}[annotation]
  Im Ticket-System: \href{http://team.mumie.net/issues/9745}{Ticket 9745}
\end{block}


\lang{de}{Überprüfen Sie den Wahrheitsgehalt folgender Aussagen.}
\lang{en}{Check if the following statements are true.}
\lang{de}{\begin{table}[\class{items}]
  \nowrap{a) 
  Alle komplexen Zahlen sind reelle Zahlen.} \\ 
  \nowrap{b) 
  Konjugiert komplexe Zahlen besitzen den gleichen Imaginärteil.} \\
  \nowrap{c) 
  Alle Lösungen einer quadratischen Gleichung sind komplexe Zahlen.} \\
  \nowrap{d) 
  Es gilt $\abs{z} = \sqrt{\left(\Re{(z)}\right)^2 + \left({i} \cdot
  \Im{(z)}\right)^2}.$} \\
  \nowrap{e) 
  Die Gleichung $42 \cdot {i} \cdot x = x \cdot 42 \cdot {i}$ ist für jede Zahl $x$ erfüllt.} \\ 
  \nowrap{f) 
  Die Zahl $-3 -\frac{1}{10} \, {i}$ liegt im dritten Quadranten der Gaußschen Zahlenebene.} 
\end{table}}

\lang{en}{\begin{table}[\class{items}]
  \nowrap{a) 
  All complex numbers are real numbers.} \\ 
  \nowrap{b) 
  Complex conjugates have the same imaginary part.} \\
  \nowrap{c) 
  All solutions of a quadratic equation are complex numbers} \\
  \nowrap{d) 
  It holds $\abs{z} = \sqrt{\left(\Re{(z)}\right)^2 + \left({i} \cdot
  \Im{(z)}\right)^2}.$} \\
  \nowrap{e) 
  The equation $42 \cdot {i} \cdot x = x \cdot 42 \cdot {i}$ is fulfilled for every number $x$.} \\ 
  \nowrap{f) 
  The number $-3 -\frac{1}{10} \, {i}$ is in the third quadrant of the complex plane.} 
\end{table}}

\begin{tabs*}[\initialtab{0}\class{exercise}]
  \tab{
  \lang{de}{Antwort}
  \lang{en}{Answer}
  }
  \lang{de}{
  \begin{table}[\class{items}]
    
    \nowrap{a) 
    Unwahr} & 
    \nowrap{b) 
    Unwahr} & 
    \nowrap{c) 
    Wahr} & 
    \nowrap{d) 
    Unwahr} & 
    \nowrap{e) 
    Wahr} & 
    \nowrap{f) 
    Wahr} & 
    
  \end{table}}

  \lang{en}{
  \begin{table}[\class{items}]
    
    \nowrap{a) 
    False} & 
    \nowrap{b) 
    False} & 
    \nowrap{c) 
    True} & 
    \nowrap{d) 
    False} & 
    \nowrap{e) 
    True} & 
    \nowrap{f) 
    True} & 
    
  \end{table}}

  \tab{
  \lang{de}{Lösung a)}
  \lang{en}{Solution a)}}
  \begin{incremental}[\initialsteps{1}]
    \step 
    \lang{de}{Die reellen Zahlen bilden eine Teilmenge der komplexen Zahlen.
    Damit ist jede reelle Zahl auch eine komplexe Zahl,
    aber nicht jede komplexe Zahl ist eine reelle Zahl. Die imaginäre Einheit $i$ ist beispielsweise keine reelle Zahl.}
  \lang{en}{The real numbers are a subset of the complex numbers.#
  Therefore every real number is also a complex number, but not every complex number is a real number.
  For example the imaginary unit $i$ is not real.}
  \end{incremental}

  \tab{
  \lang{de}{Lösung b)}
  \lang{en}{Solution b)}}
  \begin{incremental}[\initialsteps{1}]
    \step 
    \lang{de}{
    Konjugiert komplexe Zahlen besitzen Imaginärteile,
    deren Vorzeichen sich unterscheiden.} 
    \lang{en}{The imaginary part of the complex conjugates has a different sign.}
  \end{incremental}

  \tab{
  \lang{de}{Lösung c)}
  \lang{en}{Solution c)}}
  \begin{incremental}[\initialsteps{1}]
    \step 
    \lang{de}{Die Aussage ist selbst dann wahr,
    wenn die Lösungen reelle Zahlen sind, 
    da auch die reellen Zahlen zu den komplexen Zahlen gehören.}
    \lang{en}{The statement is true, even if the solutions are real numbers, since they are a subset of the complex numbers.}
  \end{incremental}

  \tab{
  \lang{de}{Lösung d)}
  \lang{en}{Solution d)}}
  \begin{incremental}[\initialsteps{1}]
    \step 
    \lang{de}{Vor dem Imaginärteil darf in der Formel 
    keine imaginäre Einheit stehen.}
    \lang{en}{There is no imaginary unit in front of the imaginary part allowed.}
  \end{incremental}

  \tab{
  \lang{de}{Lösung e)}
  \lang{en}{Solution e)}}
  \begin{incremental}[\initialsteps{1}]
    \step 
    \lang{de}{Die Reihenfolge der Multiplikation 
    von Imaginärteil und imaginärer Einheit spielt keine Rolle.}
    \lang{en}{The order of imaginary part and imaginary unit within the multiplication does not matter.}
  \end{incremental}

  \tab{
  \lang{de}{Lösung f)}
  \lang{en}{Solution f)}}
  \begin{incremental}[\initialsteps{1}]
    \step 
    \lang{de}{Da sowohl der Realteil als auch der Imaginärteil der Zahl negativ sind,
    liegt sie im dritten Quadranten.}
    \lang{en}{Since real and imaginary part of the number are negativ, it is in the third quadrant.}
  \end{incremental}

\end{tabs*}

\end{content}