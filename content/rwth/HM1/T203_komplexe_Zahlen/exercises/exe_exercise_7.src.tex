\documentclass{mumie.element.exercise}
%$Id$
\begin{metainfo}
  \name{
    \lang{de}{Ü07: Gleichungen}
    \lang{en}{Ex07: Equations}
  }
  \begin{description} 
 This work is licensed under the Creative Commons License Attribution 4.0 International (CC-BY 4.0)   
 https://creativecommons.org/licenses/by/4.0/legalcode 

    \lang{de}{}
    \lang{en}{}
  \end{description}
  \begin{components}
  \end{components}
  \begin{links}
  \end{links}
  \creategeneric
\end{metainfo}
\begin{content}
\title{
\lang{de}{Ü07: Gleichungen}
\lang{en}{Ex07: Equations}
}
\begin{block}[annotation]
  Im Ticket-System: \href{http://team.mumie.net/issues/9705}{Ticket 9750}
\end{block}

\begin{enumerate}
\item
\lang{de}{Ermitteln Sie alle Lösungen der folgenden Gleichungen.}
\lang{en}{Determine all the solutions for the following equations.}
\begin{table}[\class{items}]
  \nowrap{a) 
  $\ z^2 = 1$} & 
  \nowrap{b) 
  $\ z^2 = -1$} &
  \nowrap{c) 
  $\ z^2 = -81$} &
  \nowrap{d) 
  $\ z^2+4z+29=0$}
\end{table}
\item
\lang{de}{
Zerlegen Sie in Linearfaktoren:}
\lang{en}{Decompose the following expressions into its linear factors: }
\begin{table}[\class{items}]
  \nowrap{a) 
  $\ z^2 +2z +5$} & 
  \nowrap{b) 
  $\ z^3 +iz^2+2i \text{\qquad  (\lang{de}{Tipp: $z=i$ ist Nullstelle} \lang{en}{Tip: $z=i$ is a zero of the equation})}$} 
\end{table}
\end{enumerate}

\begin{tabs*}[\initialtab{0}\class{exercise}]
  \tab{
  \lang{de}{Antworten}
  \lang{en}{Answer}
  }
  \begin{enumerate}
  \item
  \begin{table}[\class{items}]
    \nowrap{a) 
    $\ z_1 = 1, 
    \quad 
    z_2 = -1$} \\ 
    \nowrap{b) 
    $\ z_1 = {i}, 
    \quad 
    z_2 = -{i}$} \\ 
    \nowrap{c) 
    $\ z_1 = 9 \, {i}, 
    \quad 
    z_2 = -9 \, {i}$}\\
    \nowrap{d) 
    $\ z_1 = -2+5\, {i}, 
    \quad 
    z_2 = -2-5\, {i}$}\\
    \end{table}
    \item
    \begin{table}[\class{items}]
    \nowrap{a) 
    $\ z^2+2z+5=(z-(-1+2\,{i}))(z-(-1-2\,{i}))$}\\
    \nowrap{b) 
    $\ z^3+iz^2+2\,{i} = (z-\,{i})(z-(-i+1))(z-(-1-i))$}
    
  \end{table}
\end{enumerate}
  \tab{
  \lang{de}{Lösung 1a)}
  \lang{en}{Solution 1a)}}
  \begin{incremental}[\initialsteps{1}]
    \step 
    \lang{de}{Die quadratische Gleichung hat zwei reelle Lösungen,
    die sich auch ohne komplexe Rechnung einfach ermitteln lassen.} 
    \lang{en}{This quadratic equation has two real solutions,
    which can be easily determined without any complex calcuations.} 
    \step
    \lang{de}{Sowohl $z_1 = 1$ als auch $z_2 = -1$ 
    ergeben quadriert den Wert $1$.}
    \lang{en}{Both $z_1 = 1$ and $z_2 = -1$ 
    squared are equal to $1$.}
  \end{incremental}

  \tab{
  \lang{de}{Lösung 1b)}
  \lang{en}{Solution 1b)}}
  \begin{incremental}[\initialsteps{1}]
    \step 
    \lang{de}{Die quadratische Gleichung besitzt zwei komplexe Lösungen,
    da es keine reellen Zahlen gibt, deren Quadrat negativ ist.}
    \lang{en}{The quadratic equation has two complex solutions, because there exists no real number with a
    negative square.}
    \step 
    \lang{de}{Die erste Lösung 
    $z_1 = {i}$
    folgt direkt aus der Definition der imaginären Einheit (${i}^2 = -1$).}
    \lang{en}{The first solution 
    $z_1 = {i}$
    follows from the definition of the imaginary unit (${i}^2 = -1$).}
    \step 
    \lang{de}{Komplexe Nullstellen treten bei Polynomen mit reellen Koeffizienten
    immer als konjugiert komplexe Paare auf.
    Daher erfüllt die konjugiert komplexe Zahl 
    $z_2 = -{i}$
    die Gleichung ebenfalls:}
    \lang{en}{Complex zeros of a polynomial with real coefficients 
    always appear as pairs of complex conjugates. Therefore the complex conjugate
    $z_2 = -{i}$
    also fulfills the equations:}
    \begin{align*}
      z_2^2 &= z_2 \cdot z_2
      \\ &=
      (-{i}) \cdot (-{i})
      \\ &= 
      {i} \cdot {i}
      \\ &=
      -1.
    \end{align*}

  \end{incremental}

  \tab{
  \lang{de}{Lösung 1c)}
  \lang{en}{Solution 1c)}}
  \begin{incremental}[\initialsteps{1}]
    \step 
    \lang{de}{Die quadratische Gleichung besitzt zwei komplexe Lösungen,
    da es keine reellen Zahlen gibt, deren Quadrat negativ ist.}
    \lang{en}{The quadratic equation has two complex solutions, because there exists no real number with a
    negative square.}
    \step 
    \lang{de}{Die erste Lösung 
    $z_1 = 9 \, {i}$
    ergibt sich aus der Definition der imaginären Einheit (${i}^2 = -1$)
    und der Tatsache, dass $9^2 = 81$ ist.}
    \lang{en}{The first solution 
    $z_1 = 9 \, {i}$
    is a consequence of the definition of the imaginary unit (${i}^2 = -1$)
    and the fact, that $9^2 = 81$ holds.}
    \step 
    \lang{de}{Komplexe Nullstellen treten bei Polynomen mit reellen Koeffizienten
    immer als konjugiert komplexe Paare auf.
    Daher erfüllt die konjugiert komplexe Zahl 
    $z_2 = -9 \, {i}$
    die Gleichung ebenfalls:}
    \lang{en}{Complex zeros of a polynomial with real coefficients 
    always appear as pairs of complex conjugates. Therefore the complex conjugate 
    $z_2 = -9 \, {i}$
    also fulfills the equation:}
    \begin{align*}
      z_2^2 &= z_2 \cdot z_2
      \\ &=
      (-9 \, {i}) \cdot (-9 \, {i})
      \\ &= 
      81 \cdot {i} \cdot {i}
      \\ &=
      -81.
    \end{align*}

  \end{incremental}

  \tab{
  \lang{de}{Lösung 1d)}
  \lang{en}{Solution 1d)}}
  \begin{incremental}[\initialsteps{1}]
    \step 
    \lang{de}{Die quadratische Gleichung besitzt zwei komplexe Lösungen.}
    \lang{en}{The quadratic equation has two complex solutions.}
    \step 
    \lang{de}{Die pq-Formel liefert als Lösungen:}
    \lang{en}{The pq-formula gives the solutions:}
    \begin{align*}
    z_{1/2}&=&-\frac{4}{2}\pm \sqrt{4-29}\\
    &=&-2\pm \sqrt{-25}\\
    &=&-2\pm5\, {i}
    \end{align*}
    \lang{de}{Auch hier gilt wieder: $z_2=\bar{z_1}$.}
    \lang{en}{Hier also holds: $z_2=\bar{z_1}$.}
  \end{incremental}

 \tab{\lang{de}{Lösungsvideo 2)}}	
    \lang{de}{\youtubevideo[500][300]{YW8phs-4NWU}\\
    Beachten Sie, dass die imaginäre Einheit $i$ im Video durch $j$ repräsentiert wird.}


\end{tabs*}
\end{content}