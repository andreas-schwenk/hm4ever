%$Id:  $
\documentclass{mumie.article}
%$Id$
\begin{metainfo}
  \name{
    \lang{de}{Überblick: Komplexe Zahlen}
    \lang{en}{overview: }
  }
  \begin{description} 
 This work is licensed under the Creative Commons License Attribution 4.0 International (CC-BY 4.0)   
 https://creativecommons.org/licenses/by/4.0/legalcode 

    \lang{de}{Beschreibung}
    \lang{en}{}
  \end{description}
  \begin{components}
  \end{components}
  \begin{links}
\link{generic_article}{content/rwth/HM1/T211_Eigenschaften_stetiger_Funktionen/g_art_T211_overview.meta.xml}{T211_overview}
\link{generic_article}{content/rwth/HM1/T203_komplexe_Zahlen/g_art_content_09neu_komplexeZahlen_hauptsatz.meta.xml}{content_09neu_komplexeZahlen_hauptsatz}
\link{generic_article}{content/rwth/HM1/T203_komplexe_Zahlen/g_art_content_08bneu_komplexeZahlen_geom.meta.xml}{content_08bneu_komplexeZahlen_geom}
\link{generic_article}{content/rwth/HM1/T203_komplexe_Zahlen/g_art_content_08aneu_komplexeZahlen_intro.meta.xml}{content_08aneu_komplexeZahlen_intro}
\end{links}
  \creategeneric
\end{metainfo}
\begin{content}
\begin{block}[annotation]
	Im Ticket-System: \href{https://team.mumie.net/issues/30134}{Ticket 30134}
\end{block}




\begin{block}[annotation]
Im Entstehen: Überblicksseite für Kapitel Komplexe Zahlen
\end{block}

\usepackage{mumie.ombplus}
\ombchapter{1}
\lang{de}{\title{Überblick: Komplexe Zahlen}}
\lang{en}{\title{}}



\begin{block}[info-box]
\lang{de}{\strong{Inhalt}}
\lang{en}{\strong{Contents}}


\lang{de}{
    \begin{enumerate}%[arabic chapter-overview]
   \item[3.1] \link{content_08aneu_komplexeZahlen_intro}{Einführung der komplexen Zahlen}
   \item[3.2] \link{content_08bneu_komplexeZahlen_geom}{Geometrische Anschauung}
   \item[3.3] \link{content_09neu_komplexeZahlen_hauptsatz}{Nullstellen von Polynomen}
   \end{enumerate}
} %lang

\end{block}

\begin{zusammenfassung}

\lang{de}{Mit den komplexen Zahlen $\C$ lernen Sie neben $\Q$ und $\R$ einen weiteren Körper kennen.
Die komplexen Zahlen entstehen aus der Notwendigkeit, genügend Lösungen für Polynomgleichungen wie $x^2+1=0$ zu finden.
Sie erweitern den Bereich der reellen Zahlen.

Geometrisch stellt man sich die Menge der komplexen Zahlen als Ebene vor, der Gaußschen Zahlenebene. 
Dort entspricht die Addition komplexer Zahlen der Vektoraddition. Der Betrag einer komplexen Zahl entspricht ihrer euklidischen Länge.
Die geometrische Interpretation der Multiplikation wird anschaulich in Polarkoordinaten beschrieben: Die Längen (die Beträge) werden multipliziert, die Argumente (Winkel mit positiver reeller Achse) werden addiert.


Weil wir die Polarkoordinatenschreibweise $z=r\cdot e^{i\phi}$ erst in \link{T211_overview}{Kapitel 11} einführen, 
steht die Zwischenprüfung zu den komplexen Zahlen am Ende von Teil 2.}


\end{zusammenfassung}

\begin{block}[info]\lang{de}{\strong{Lernziele}}
\lang{en}{\strong{Learning Goals}} 
\begin{itemize}[square]
 \item \lang{de}{Sie können die Bedeutung der imaginären Einheit erklären.}
\item \lang{de}{Sie bestimmen Real- und Imaginärteile komplexer Zahlen.}
\item \lang{de}{Sie additieren, multiplizieren und invertieren komplexe Zahlen in der Normalform $a+ib$.}
\item \lang{de}{Sie kennen den komplexen Absolutbetrag und seine Eigenschaften.}
\item \lang{de}{Sie rechnen komplexe Zahlen von Normalform $a+ib$ in Polarkoordinaten um und umgekehrt.}
\item \lang{de}{Sie interpretieren die Rechenoperationen geometrisch und skizzieren Mengen komplexer Zahlen in einfachen Beispielen.}
\item \lang{de}{Sie lösen jede quadratische Gleichung $x^2+px+q=0$ für reelle Zahlen $p$ und $q$.}
\end{itemize}
 \end{block}




\end{content}
