%$Id:  $
\documentclass{mumie.article}
%$Id$
\begin{metainfo}
  \name{
    \lang{en}{...}
    \lang{de}{Einführung}
   }
  \begin{description} 
 This work is licensed under the Creative Commons License Attribution 4.0 International (CC-BY 4.0)   
 https://creativecommons.org/licenses/by/4.0/legalcode 

    \lang{en}{...}
    \lang{de}{...}
  \end{description}
  \begin{components}
\component{generic_image}{content/rwth/HM1/images/g_tkz_T203_ComplexPlane.meta.xml}{T203_ComplexPlane}
\component{generic_image}{content/rwth/HM1/images/g_img_00_Videobutton_schwarz.meta.xml}{00_Videobutton_schwarz}
\component{generic_image}{content/rwth/HM1/images/g_img_00_video_button_schwarz-blau.meta.xml}{00_video_button_schwarz-blau}
\component{generic_image}{content/rwth/HM1/images/g_img_T203_gauss_ebene.meta.xml}{T203_gauss_ebene}
\end{components}
  \begin{links}
\link{generic_article}{content/rwth/HM1/T202_Reelle_Zahlen_axiomatisch/g_art_content_05_anordnungsaxiome.meta.xml}{content_05_anordnungsaxiome}
\link{generic_article}{content/rwth/HM1/T202_Reelle_Zahlen_axiomatisch/g_art_content_04_koerperaxiome.meta.xml}{content_04_koerperaxiome}
\end{links}
  \creategeneric
\end{metainfo}
\begin{content}
\begin{block}[annotation]
	Im Ticket-System: \href{https://team.mumie.net/issues/21562}{Ticket 21562}
\end{block}
\begin{block}[annotation]
Copy of \href{https://team.mumie.net/issues/20592}{Ticket 20592}: content/rwth/HM1/T203_komplexe_Zahlen/art_content_01_komplexeZahlen_intro.src.tex
\end{block}

\usepackage{mumie.ombplus}
\ombchapter{3}
\ombarticle{1}


\lang{de}{\title{Komplexe Zahlen}}
\lang{en}{\title{Complex numbers}}

\begin{block}[info-box]
\tableofcontents
\end{block}

\section{
\lang{de}{Einführung der komplexen Zahlen}
\lang{en}{Introduction to complex numbers}}\label{sec:gauss-ebene}

\lang{de}{
Die quadratische Gleichung $x^2+1=0$ hat in $\R$ keine Lösung, denn für alle 
$x\in\R$ gilt $x^2\geq0$ und damit (durch Addition von 
$+1$ auf beiden Seiten der Ungleichung) gilt auch $x^2+1\geq1$. \\
Weil wir diese Gleichung aber lösen können wollen, \glqq $i$maginieren\grqq wir uns eine
Lösung und nennen sie $i$.\\
}
\lang{en}{
The quadratic equation $x^2+1=0$ has no solutions in $\R$,
because $x^2 \geq 0$ for all $x\in\R$ and therefore
(after adding $1$ to both sides of the inequality)
$x^2+1\geq 1$. \\
Since we would like to be able to solve the equation, we "$i$magine"
a solution and call it $i$.
}

\begin{definition}\label{def:imaginäre_Einheit}
\lang{de}{Die \notion{imaginäre Einheit} $i$ erfüllt die Gleichung $i^2+1=0$, anders gesagt
\[i^2=-1.\]}
\lang{en}{
The \notion{imaginary unit} $i$ satisfies the equation $i^2+1=0$,
or in other words \[i^2=-1.\]
}
\end{definition}


\lang{de}{Diese Definition ist sinnvoll, wenn wir $i$ im Weiteren behandeln wie jede andere Zahl auch. 
Wenn es $i$ gibt, dann gibt es auch $2i$, $3i$, ..., $-i$, also allgemein jedes Vielfache in $i\R$.
Aber man muss diese Zahlen auch zu reellen Zahlen hinzuaddieren können, $1+i$, $2-i$, etc. So gelangt man zu $\R+i\R$.
Das kann man geometrisch veranschaulichen, indem wir zur Zahlengerade der reellen Zahlen eine weitere Dimension hinzunehmen.
}
\lang{en}{
This definition is reasonable if we treat $i$ like any other number.
If $i$ exists, then $2i, 3i, ...$ and $-i$, and more generally
every multiple in $i\R$, should also exist. But we must also be able to add these numbers
to real numbers: $1+i$, $2-i$, etc. That leads us to $\R+i\R$.
We can visualize this geometrically by adding another dimension
to the number line that represents the real numbers.
}
\begin{center}
\image{T203_ComplexPlane}
\end{center}

\begin{definition}[
\lang{de}{Gaußsche Zahlenebene}
\lang{en}{Complex plane}]\label{def:Gausssche_Ebene}
\lang{de}{Die \notion{Gaußsche Zahlenebene} ist die Ebene der komplexen Zahlen
\[\C=\{a+bi\,|\,a,b\in\R\}.\]
Sie enthält die reellen Zahlen 
\[\R=\{a+bi\,|\,a\in\R,b=0\}\subset\C.\]
}
\lang{en}{
The \notion{complex plane} is the plane of complex numbers
\[\C=\{a+bi\,|\,a,b\in\R\}.\]
It contains the real numbers
\[\R=\{a+bi\,|\,a\in\R,b=0\}\subset\C.\]
}
\end{definition}



\section{
\lang{de}{Addition und Multiplikation komplexer Zahlen}
\lang{en}{Adding and multiplying complex numbers}}\label{sec:add-mult-kompl}

\lang{de}{Wir können komplexe Zahlen addieren und multiplizieren, indem wir die aus $\R$ bekannten Rechenregeln für $\C$ übernehmen:\\
}
\lang{en}{
Complex numbers can be added and multiplied by carrying over
the familiar rules of arithmetic from $\R$ to $\C$:
}
%\[(a+bi)+(c+di)=(a+c)+(bi+di)=(a+c)+(b+d)i\in \C, \]
\begin{eqnarray*}
(a+bi)+(c+di)&=&(a+c)+(bi+di)\\
&=&(a+c)+(b+d)i\in \C ,
\end{eqnarray*}
\begin{eqnarray*}
(a+bi)\cdot(c+di)&=&ac +adi+ bci +bdi^2\\
&=&ac-bd +adi+bci\\
&=&(ac-bd)+(ad+bc)i\in\C .
\end{eqnarray*}

\begin{example}
\begin{enumerate}
\item
$ \ (5-3i)+(-2+7i)=(5-2)+(-3i+7i)=3+4i$
\item
$ \ (-2-2i)+(3+2i)=1+0i=1$
\item
$ \ (2+3i)\cdot(1-4i)=2\cdot 1+2\cdot(-4i)+3i\cdot1+3i\cdot(-4i)=2-8i+3i-12i^2=2-5i+12=14-5i$
\item
$ \ \frac{5+3i}{1-2i}=\frac{(5+3i)(1+2i)}{(1-2i)(1+2i)}=\frac{5+10i+3i+6i^2}{1^2-(2i)^2}
=\frac{-1+13i}{1+4}=\frac{-1}{5}+\frac{13}{5}i$\\
\lang{de}{(Im Nenner kommt die 3. binomische Formel zur Anwendung.)}
\lang{en}{(The 3rd binomial formula was used in the denominator.)}
\end{enumerate}
\end{example}

\begin{definition}\label{def:real-imaginaer-teil}
\lang{de}{
Für eine komplexe Zahl $z=a+bi\in \C$ mit $a,b\in \R$ ist der \notion{Realteil} von $z$, bezeichnet mit $\Re (z)$, gegeben durch
\[ \Re (z)=a\in \R \]
und der \notion{Imagin\"arteil} von $z$, bezeichnet mit $\Im (z)$, gegeben durch
\[ \Im (z)=b\in \R. \]
(Beachte: Auch der Imagin\"arteil einer komplexen Zahl ist 
        eine \emph{reelle} Zahl!)
        
Die zu $z$ \notion{konjugiert komplexe} Zahl, bezeichnet mit $\bar{z}$, ist
\[ \bar{z}=a-bi.  \]
}
\lang{en}{
Let $z=a+bi \in \C$ be a complex number with $a,b\in \R$.
The \notion{real part} of $z$, denoted $\Re(z)$, is
\[ \Re (z)=a\in \R,\]
and the \notion{imaginary part} of $z$, denoted $\Im(z)$, is
\[ \Im (z)=b\in \R. \]
(Note: the imaginary part of a complex number is a \emph{real} number!)
The \notion{complex conjugate} of $z$, denoted $\bar{z}$, is
\[ \bar{z}=a-bi.  \]
}
\end{definition}  

\lang{de}{
Die folgenden Videos erläutern die bisherigen Definitionen und Aussagen anhand von kleinen Beispielen. Beachten Sie dabei, dass in den Videos die imaginäre Einheit $i$ 
durch $j$ repräsentiert wird:
 \floatright{
\href{https://www.hm-kompakt.de/video?watch=200}{\image[75]{00_Videobutton_schwarz}}
\href{https://www.hm-kompakt.de/video?watch=203}{\image[75]{00_Videobutton_schwarz}}
\href{https://www.hm-kompakt.de/video?watch=206}{\image[75]{00_Videobutton_schwarz}}
}
}

\\
\\
\\

\begin{quickcheck}
\field{real}
\type{input.function}
\begin{variables}
    \randint{a}{1}{3}
    \randint[Z]{b}{4}{6}
    \function[expand, normalize]{z}{a+b*i}
    \function[expand, normalize]{zquer}{a-b*i}
\end{variables}
\lang{de}{
\text{Betrachten Sie die Zahl $z=\var{z}$ und bestimmen Sie:\\
$\Re(z)=$\ansref, $\Im(z)=$\ansref, $\bar{z}$=\ansref. 
}
}
\lang{en}{
\text{Consider the number $z=\var{z}$. Determine\\
$\Re(z)=$\ansref, $\Im(z)=$\ansref, $\bar{z}$=\ansref. 
}
}
\begin{answer}
    \solution{a}
\end{answer}
\begin{answer}
    \solution{b}
\end{answer}
\begin{answer}
    \solution{zquer}
\end{answer}
\end{quickcheck}


\begin{remark}
\lang{de}{
Es sei $z = a+bi$ eine komplexe Zahl mit $a,b\in\R$ und $i\in\C$ die imaginäre Einheit. Dann heißt
$z$ \notion{rein imaginär}, falls $a=0$ gilt, also wenn $z=ib$ ein reelles Vielfaches der imaginären Einheit ist.
}
\lang{en}{
Let $z=a+bi$ be a complex number with $a,b\in\R$, where $i\in\C$ is the imaginary unit.
$z$ is called \notion{purely imaginary} if $a=0$; that is,
if $z=ib$ is a real multiple of the imaginary unit.
}
\end{remark}


\begin{quickcheck}
\field{real}
\type{input.function}
\begin{variables}
    \number{a}{1+3i}
    \number{b}{2+4i}
    \function{z}{-10+10i}
    \randint{c}{-3}{-2}
    \randint{d}{2}{4}
    \function[expand,normalize]{z1}{c+3i}
    \function[expand,normalize]{z2}{d-4i}
    \function[expand,normalize]{z3}{c*d+12+(3*d-4*c)*i}
\end{variables}
\lang{de}{
\text{Berechnen Sie das Produkt der folgenden beiden komplexen Zahlen:
\[z_1=1+3i \ \text{  und  } \ z_2=2+4i. \]
$z_1\cdot z_2=$\ansref 
}
}
\lang{en}{
\text{Calculate the product of the following two complex numbers:
\[z_1=1+3i \ \text{  und  } \ z_2=2+4i. \]
$z_1\cdot z_2=$\ansref 
}
}
\begin{answer}
    \solution{z}
\end{answer}
\lang{de}{
\text{Nun müssen Sie auch auf die Vorzeichen achten:\\
$(\var{z1})\cdot (\var{z2})= $\ansref
}
}
\lang{en}{
\text{Here you will have to pay attention to the minus signs:\\
$(\var{z1})\cdot (\var{z2})= $\ansref
}
}
\begin{answer}
    \solution{z3}
\end{answer}
\explanation{$(1+3i)(2+4i)=2+4i+6i+12i^2=2+10i-12=-10+10i$\\
\lang{de}{und}\lang{en}{and} \\
$(\var{z1})\cdot(\var{z2})=\var{z3}.$
}
\end{quickcheck}



\begin{quickcheck}
    \field{real}
        \type{input.function}
        \lang{de}{
        \text{Berechnen Sie $z_1+z_2$ sowie $z_1\cdot z_2$ für die folgenden beiden komplexen Zahlen:}
        }
        \lang{en}{
        \text{Calculate $z_1+z_2$ and $z_1\cdot z_2$ for the following two complex numbers:}
        }
        \begin{variables}
            \randint{a}{-6}{6}
            \randint[Z]{b}{-6}{6}
            \randint{c}{-6}{6}
            \randint[Z]{d}{-6}{6}
            \function[expand,normalize]{z1}{a+b*i}
            \function[expand,normalize]{z2}{c+d*i}
            \function[expand,normalize]{zp}{a+c+(b+d)*i}
            \function[expand,normalize]{zm}{a*c-b*d+(a*d+b*c)*i}
        \end{variables}
        \lang{de}{
        \text{$z_1=\var{z1} \ $ und $ \ z_2=\var{z2}$.\\
            $z_1+z_2=$\ansref und  $z_1\cdot z_2=$\ansref}
        }
        \lang{en}{
        \text{$z_1=\var{z1} \ $ and $ \ z_2=\var{z2}$.\\
            $z_1+z_2=$\ansref and  $z_1\cdot z_2=$\ansref}
        }
            \begin{answer}
                \solution{zp}
            \end{answer}
            \begin{answer}
                \solution{zm}
            \end{answer}
            \explanation{$z_1+z_2=(\var{a}+\var{c})+(\var{b}+\var{d})i=\var{zp} \ $ \lang{de}{und} \lang{en}{and}
            $ \ z_1\cdot z_2=(\var{a})\cdot(\var{c})-(\var{b})\cdot(\var{d})+((\var{a})\cdot(\var{d})+(\var{b})\cdot(\var{c}))i=\var{zm}.$
            }
\end{quickcheck}

\lang{de}{
Im folgenden Video kann der Sachverhalt des Kapitels nochmals angesehen werden:
\floatright{\href{https://api.stream24.net/vod/getVideo.php?id=10962-2-10823&mode=iframe&speed=true}{\image[75]{00_video_button_schwarz-blau}}}\\
}

\section{
\lang{de}{Der Körper der komplexen Zahlen}
\lang{en}{The field of complex numbers}}\label{sec:koerper-komplexe-zahlen}

\lang{de}{Wir haben die uns bekannte Addition \glqq +\grqq  und Multiplikation \glqq $\cdot$\grqq der reellen Zahlen auf die komplexen 
Zahlen ausgedehnt. Kommutativ-, Assoziativ- und Distributivgesetz gelten dann weiterhin, wie sich leicht nachrechnen lässt. Wir gehen jetzt
der Frage nach dem \textit{multiplikativen Inversen} nach:
Gibt es zu einer Zahl $z$ auch eine Zahl $z^{-1}$ so, dass $z\cdot z^{-1}=1$ gilt?
}
\lang{en}{
We have extended the familiar addition "+" and multiplication "$\cdot$" of real numbers to complex numbers.
It is not difficult to verify that the commutative, associative and distributive laws
continue to hold. We will now consider the existence of the \textit{multiplicative inverse}:
for a complex number $z$, does there exist a number $z^{-1}$ such that $z \cdot z^{-1} = 1$?
}

\begin{theorem}[
\lang{de}{multiplikative Inverse}
\lang{en}{multiplicative inverse}]
\begin{enumerate}
\item 
\lang{de}{
Für jede komplexe Zahl $z=a+bi$ mit $a,b\in\R$ ist $z\cdot\bar{z}=a^2+b^2\in\R$ eine nicht negative reelle Zahl. 
Ist $z\neq 0$, dann ist auch $z \cdot\bar{z}\neq 0$.
}
\lang{en}{
For any complex number $z=a+bi$ with $a,b\in \R$, $z\cdot\bar{z}=a^2+b^2\in \R$
is a nonnegative real number. If $z\neq 0$, then $z\cdot\bar{z}\neq 0$ as well.
}
\item 
\lang{de}{
Für jede von Null verschiedene komplexe Zahl $z=a+bi\in\C\setminus\{0\}$ ist ihr multiplikatives Inverses durch $z^{-1}=\frac{\bar{z}}{z\bar{z}}=\frac{a-bi}{a^2+b^2}$
bestimmt.}
\lang{en}{
The multiplicative inverse of a nonzero complex number $z=a+bi \in\C\setminus\{0\}$ is
given by $z^{-1}=\frac{\bar{z}}{z\bar{z}}=\frac{a-bi}{a^2+b^2}$.
}
\end{enumerate}
\end{theorem}

\begin{proof*}
\begin{showhide}
\begin{enumerate}
\item
$z\cdot\bar{z}=(a+bi)(a-bi)=a^2+b^2$\\
\lang{de}{
Nur für $z=0=\bar{z}$ ist $z\cdot\bar{z}=0$.}
\lang{en}{
$z\cdot\bar{z}=0$ only if $z=0=\bar{z}$.
}
\item
\lang{de}{Mit $z=a+bi$ und $z^{-1}= \frac{a-bi}{a^2+b^2}$ ist}
\lang{en}{Setting $z=a+bi$ and $z^{-1}= \frac{a-bi}{a^2+b^2}$, }
\[ z\cdot z^{-1}=(a+bi)\cdot\frac{a-bi}{a^2+b^2}=\frac{a^2+b^2}{a^2+b^2}=1. \]
\end{enumerate}
\end{showhide}
\end{proof*}

\lang{de}{
Da wir die Frage nach dem multiplikativen Inversen mit ja beantworten konnten, können 
wir nun prüfen, ob die komplexen Zahlen ebenso wie die reellen Zahlen einen \link{content_04_koerperaxiome}{Körper} bilden.
}
\lang{en}{
Since we were able to show the existence of multiplicative inverses,
we can now check whether the complex numbers (like the real numbers) form a \link{content_04_koerperaxiome}{field}.
}
\begin{theorem}\label{thm:C_field}
\lang{de}{Die Menge der komplexen Zahlen $\C$ bildet zusammen mit der Addition und Multiplikation
einen \notion{Körper}.
}
\lang{en}{
The set of complex numbers $\C$, together with its addition and multiplication,
is a \notion{field}.
}
\end{theorem}

\begin{proof*}[
\lang{de}{Körperaxiome für $\C$}
\lang{en}{Field axioms for $\C$}]
\lang{de}{
Dass die Menge der komplexen Zahlen $\C$  zusammen mit der Addition und Multiplikation 
einen Körper bildet, bedeutet, dass
die \ref[content_04_koerperaxiome][Axiome (A1)-(A4), (M1)-(M4) und (D)]{sec:axiome} für $\C$ gelten.
Es bedeutet auch, dass alle \ref[content_04_koerperaxiome][Rechenregeln]{sec:rechenregeln}, 
die für Körper gelten, auch für die komplexen Zahlen gelten.
Die Axiome lassen sich mit den Definitionen nachrechnen. 
Wir geben hier nur die in (A3), (M3), (A4) und (M4) geforderten Elemente an.
}
\lang{en}{
For the set of complex numbers $\C$ with its addition and multiplication
to be a field means that $\C$ satisfies the axioms
\ref[content_04_koerperaxiome][(A1)-(A4), (M1)-(M4) and (D)]{sec:axiome}.
It also means that the \ref[content_04_koerperaxiome][rules of arithmetic]{sec:rechenregeln}
that hold in any field also hold for the complex numbers.
The axioms can be verified using the definitions.
Here, we will only describe the elements required by (A3), (M3), (A4) and (M4).
}


\begin{incremental}{0}
\step 
\lang{de}{
Wir betrachten ein Element $z=a+bi\in\C$.\\
Das \textcolor{#CC6600}{Nullelement bezüglich der Addition} in $\C$ ist $\textcolor{#CC6600}{z_0^+=0}=0+0i$, denn $z+z_0^+=(a+bi)+(0+0i)=(a+0)+(bi+0i)
=a+(b+0)i=a+bi=z$.\\
Das \textcolor{#0066CC}{Einselement bezüglich der Multiplikation} ist 
$\textcolor{#0066CC}{z_1^*=1}=1+0i$,
denn $z\cdot z_1^*=z\cdot1=z$.
Das Nullelement in $\C$ ist $0+0i$ und das Einselement ist $1+0i$, welche mit der Null $0$ in $\R$ und mit der Eins $1$ in $\R$
identifiziert werden.\\
}
\lang{en}{
Consider an element $z=a+bi\in\C$.\\
The \textcolor{#CC6600}{additive identity} in $\C$ is $\textcolor{#CC6600}{z_0^+=0}=0+0i$,
because $z+z_0^+=(a+bi)+(0+0i)=(a+0)+(bi+0i)=a+(b+0)i=a+bi=z$.\\
The \textcolor{#0066CC}{multiplicative identity} is $\textcolor{#0066CC}{z_1^*=1}=1+0i$,
because $z\cdot z_1^*=z\cdot1=z$. \\
The zero in $\C$ is $0+0i$ and the unity is $1+0i$, and they are identified with the zero $0$ in $\R$ and the unity $1$ in $\R$.
}
\step
\lang{de}{
Das \textcolor{#CC6600}{inverse Element bezüglich der Addition} in $\C$ ist $\textcolor{#CC6600}{-z=-a-bi}$, denn $z+(-z)=(a+bi)+(-a-bi)=\textcolor{#CC6600}{z_0^+}$.\\
Das \textcolor{#0066CC}{inverse Element bezüglich der Multiplikation} in $\C\setminus \{0\}$ ist
\[  z^{-1}= \frac{\bar{z}}{z \bar{z}}, \]
wie wir im vorherigen Satz bereits nachgerechnet haben. 
}
\lang{en}{
The \textcolor{#CC6600}{additive inverse} in $\C$ is $\textcolor{#CC6600}{-z=-a-bi}$,
because $z+(-z)=(a+bi)+(-a-bi)=\textcolor{#CC6600}{z_0^+}$.\\
The \textcolor{#0066CC}{multiplicative inverse} in $\C\setminus \{0\}$ is
\[  z^{-1}= \frac{\bar{z}}{z \bar{z}}, \]
as we saw in the previous theorem.
}
%Das \textcolor{blue}{inverse Element bezüglich der Multiplikation} in $\C\setminus \{0\}$ ist:
%$\textcolor{blue}{z^{-1}=\frac{\bar{z}}{z\cdot\bar{z}}}$, denn $z\cdot z^{-1}=\textcolor{blue}{1}$.
%\[  z^{-1}= \frac{a-ib}{a^2+b^2}, \]
%denn $(a+ib)\cdot(a-ib)=a^2-(ib)^2=a^2-i^2b^2=a^2-(-1)b^2=a^2+b^2$ und damit ist 
%\[z\cdot z^{-1}=(a+ib)\cdot\frac{a-ib}{a^2+b^2}=\frac{1}{a^2+b^2}(a+ib)(a-ib)=\frac{a^2+b^2}{a^2+b^2}=\textcolor{blue}{1}\]
%Man beachte, dass der Nenner $a^2+b^2$ ungleich $0$ ist, da wir $\C\setminus\{0\}$ betrachten. Dann ist aber 
%mindestens eine der beiden reellen Zahlen $a$ und $b$ ungleich $0$ und damit der Nenner von 0 verschieden.
\end{incremental}
\end{proof*}

% Dann weiter mit Beispielen (lassen sich bestimmt auch noch aus den alten Kapiteln extrahieren), und der Nicht-Anordnung.


\begin{example}
\lang{de}{
Zu $z\in\C$ sollen die konjugiert komplexe Zahl $\bar{z}$, \textcolor{#CC6600}{das inverse Element bzgl. der Addition} $\textcolor{#CC6600}{(-z)}$ und 
\textcolor{#0066CC}{das inverse Element bzgl. der Multiplikation} $\textcolor{#0066CC}{(z^{-1})}$ berechnet werden:
}
\lang{en}{
Given $z\in\C$, we want to compute the complex conjugate $\bar{z}$,
\textcolor{#CC6600}{the additive inverse} $\textcolor{#CC6600}{(-z)}$ and 
\textcolor{#0066CC}{the multiplicative inverse} $\textcolor{#0066CC}{(z^{-1})}$:
}
\begin{enumerate}
\item
\lang{de}{Für $z=1+i$ sind\\
$\bar{z}=1-i$, $\ \textcolor{#CC6600}{-z}=-(1+i)=-1-i \ $ und $ \ \textcolor{#0066CC}{z^{-1}}=\frac{1-i}{(1+i)(1-i)}=\frac{1-i}{1^2-i^2}=\frac{1-i}{2}
=\frac{1}{2}-\frac{i}{2}$.}
\lang{en}{For $z=1+i$, we have\\
$\bar{z}=1-i$, $\ \textcolor{#CC6600}{-z}=-(1+i)=-1-i \ $ and $ \ \textcolor{#0066CC}{z^{-1}}=\frac{1-i}{(1+i)(1-i)}=\frac{1-i}{1^2-i^2}=\frac{1-i}{2}
=\frac{1}{2}-\frac{i}{2}$.
}
\item
\lang{de}{
Für $z=5$ sind\\
$\bar{z}=5$, $\ \textcolor{#CC6600}{-z}=-5 \ $ und $ \ \textcolor{#0066CC}{z^{-1}}=\frac{5}{5\cdot 5}=\frac{1}{5}$ (wie aus $\R$ erwartet).
}
\lang{en}{
For $z=5$, we have\\
$\bar{z}=5$, $\ \textcolor{#CC6600}{-z}=-5 \ $ and $ \ \textcolor{#0066CC}{z^{-1}}=\frac{5}{5\cdot 5}=\frac{1}{5}$ (as in $\R$).
}
\item
\lang{de}{
Für $z=i$ sind\\
$\bar{z}=-i$, $ \ \textcolor{#CC6600}{-z}=-i \ $ und $ \ \textcolor{#0066CC}{z^{-1}}=\frac{-i}{i\cdot (-i)}=-i$ (Probe ist hier einfach mit $i\cdot(-i)=-(-1)=1$).
}
\lang{en}{
For $z=i$, we have\\
$\bar{z}=-i$, $ \ \textcolor{#CC6600}{-z}=-i \ $ and $ \ \textcolor{#0066CC}{z^{-1}}=\frac{-i}{i\cdot (-i)}=-i$ (which is easily checked using $i\cdot(-i)=-(-1)=1$).
}
\end{enumerate}
\end{example}
\lang{de}{
Anders als bei den reellen Zahlen ist es bei komplexen Zahlen nicht möglich, sie mit \glqq $<$\grqq oder \glqq $>$\grqq zu vergleichen. 
}
\lang{en}{
In contrast to the real numbers, it is not possible to compare complex numbers with "$<$" and "$>$".
}
\begin{theorem}
\lang{de}{
Die Menge der komplexen Zahlen $\C$ ist \notion{kein} angeordneter Körper.
}
\lang{en}{
The set of complex numbers $\C$ is \notion{not} an ordered field.
}
\end{theorem}

\begin{proof*}
\lang{de}{
Wäre $\C$ ein angeordneter Körper, dann  müsste  für jedes $z\in\C$ einer der drei folgenden Fälle gelten: $z<0$ oder $z=0$ oder $z>0$. Wir werden
zeigen, dass für die komplexe Zahl $z = i$ alle drei Aussagen zum Widerspruch führen. Damit ist dann gezeigt, dass $\C$ kein angeordneter Körper sein kann.
}
\lang{en}{
If $\C$ were an ordered field, then one of the following three cases would hold for every $z\in\C$:
$z<0$, $z=0$ or $z>0$. We will show that all three cases lead to a contradiction for $z=i$.
This will show that $\C$ could not have been an ordered field.
}
\begin{incremental}{0}
\step
\lang{de}{
Angenommen, es gäbe eine Anordnung \glqq $<$\grqq auf $\C$. Wir dürfen ohne Weiteres annehmen, dass schon $0<1$ gilt. Offensichtlich ist $i \neq 0$,
da $i^2 = -1 \neq 0^2$. Es muss also entweder $0< i$ oder $i<0$ gelten.
Gehen wir nun beide Fälle durch: Ist $0<i$, dann folgt aus den \ref[content_05_anordnungsaxiome][Anordnungsaxiomen]{def:axiome}, dass $0 <i^2$ ist, aber es gilt $i^2=-1<0$.
Wäre $i < 0$, dann folgt aus den Anordnungsaxiomen (nutze $0 < -i$ aus), dass $i\cdot(-i)=-i^2<0$ gilt, aber $0 < 1 = -i^2$.
}
\lang{en}{
Suppose there is an order "$<$" on $\C$. We can assume without loss of generality that $0 < 1$.
Clearly, $i \neq 0$, because $i^2 = -1 \neq 0^2$. Therefore, either $0<i$ or $i<0$.
Now we will consider the two cases. If $0<i$, then the \ref[content_05_anordnungsaxiome][order axioms]{def:axiome}
imply that $0 < i^2$; however, $i^2 = -1 < 0$.
If $i < 0$, then (using $0 < -i$) the order axioms imply that $i \cdot (-i) = -i^2 < 0$; however, $0 < 1 = -i^2$.
}
\end{incremental}
\end{proof*}

\begin{quickcheck}
    \field{rational}
        \type{input.function}
            \lang{de}{
            \text{Bestimmen Sie das \textcolor{#CC6600}{inverse Element bzgl. der Addition} $\textcolor{#CC6600}{(-z)}$ und das 
            \textcolor{#0066CC}{inverse Element bzgl. der Multiplikation} $\textcolor{#0066CC}{(z^{-1})}$ von \\
            \[z=\var{z}.\]}
            }
            \lang{en}{
            \text{Determine the \textcolor{#CC6600}{additive inverse} $\textcolor{#CC6600}{(-z)}$ and the
            \textcolor{#0066CC}{multiplicative inverse} $\textcolor{#0066CC}{(z^{-1})}$ of
            \[z=\var{z}.\]
            }
            }
            \begin{variables}
                \randint{a}{-6}{6}
                \randint[Z]{b}{-6}{6}
                \function[expand, normalize]{z}{a+b*i}
                \function[expand, normalize]{za}{-a-b*i}
                \function[expand, normalize]{zm}{(a-b*i)/(a^2+b^2)}
                \function[expand, normalize]{zq}{a-b*i}   
                \function[expand, normalize]{zb}{a^2+b^2}
                \function[expand, normalize]{mb}{-b}                
            \end{variables}
            \lang{de}{
            \text{Antworten:\\
            $\textcolor{#CC6600}{(-z)}=$\ansref und  $\textcolor{#0066CC}{z^{-1}}=$\ansref}
            }
            \lang{en}{
            \text{Answers:\\
            $\textcolor{#CC6600}{(-z)}=$\ansref and  $\textcolor{#0066CC}{z^{-1}}=$\ansref}
            }
            \begin{answer}
                \solution{za}
            \end{answer}
            \begin{answer}
                \solution{zm}
            \end{answer}
            \lang{de}{
            \explanation{Schreibe $z=a+bi$; dann ist $-z=-(a+bi)=-a-bi=\var{za}$ und $\bar{z}=a-bi=\var{zq}$ 
            und somit $z^{-1}=\frac{\bar{z}}{z\cdot\bar{z}}=\frac{a-bi}{a^2+b^2}=\frac{\var{zq}}{\var{zb}} = \frac{\var{a}}{\var{zb}} + \frac{\var{mb}}{\var{zb}} i.$
            }
            }
            \lang{en}{
            \explanation{Write $z=a+bi$. Then $-z=-(a+bi)=-a-bi=\var{za}$
            and $\bar{z}=a-bi=\var{zq}$ 
            and therefore $z^{-1}=\frac{\bar{z}}{z\cdot\bar{z}}=\frac{a-bi}{a^2+b^2}=\frac{\var{zq}}{\var{zb}} = \frac{\var{a}}{\var{zb}} + \frac{\var{mb}}{\var{zb}} i.$
            }
            }

\end{quickcheck}


\end{content}

