%$Id:  $
\documentclass{mumie.article}
%$Id$
\begin{metainfo}
  \name{
    \lang{en}{...}
    \lang{de}{Geometrische Anschauung}
   }
  \begin{description} 
 This work is licensed under the Creative Commons License Attribution 4.0 International (CC-BY 4.0)   
 https://creativecommons.org/licenses/by/4.0/legalcode 

    \lang{en}{...}
    \lang{de}{...}
  \end{description}
  \begin{components}
\component{generic_image}{content/rwth/HM1/images/g_tkz_T203_ComplexQuadrants.meta.xml}{T203_ComplexQuadrants}
\component{generic_image}{content/rwth/HM1/images/g_tkz_T203_Circle.meta.xml}{T203_Circle}
\component{generic_image}{content/rwth/HM1/images/g_tkz_T203_PolarCoordinates.meta.xml}{T203_PolarCoordinates}
\component{generic_image}{content/rwth/HM1/images/g_tkz_T203_ComplexSum.meta.xml}{T203_ComplexSum}
\component{generic_image}{content/rwth/HM1/images/g_img_00_Videobutton_schwarz.meta.xml}{00_Videobutton_schwarz}
\component{generic_image}{content/rwth/HM1/images/g_img_00_video_button_schwarz-blau.meta.xml}{00_video_button_schwarz-blau}
\component{generic_image}{content/rwth/HM1/images/g_img_T203_zgleichr.meta.xml}{T203_zgleichr}
\component{generic_image}{content/rwth/HM1/images/g_img_komplexe-zahlen-quadranten.meta.xml}{komplexe-zahlen-quadranten}
\component{generic_image}{content/rwth/HM1/images/g_img_komplexe-zahlen-polar.meta.xml}{komplexe-zahlen-polar}
\component{generic_image}{content/rwth/HM1/images/g_img_vk-vektor-add-vw.meta.xml}{vk-vektor-add-vw}
\component{generic_image}{content/rwth/HM1/images/g_img_vk-vektor-add-vw-wv.meta.xml}{vk-vektor-add-vw-wv}
\component{generic_image}{content/rwth/HM1/images/g_img_hm-add-komplex.meta.xml}{hm-add-komplex}
\component{js_lib}{system/media/mathlets/GWTGenericVisualization.meta.xml}{mathlet1}
\end{components}
  \begin{links}
\link{generic_article}{content/rwth/HM1/T203_komplexe_Zahlen/g_art_content_08bneu_komplexeZahlen_geom.meta.xml}{content_08bneu_komplexeZahlen_geom}
\link{generic_article}{content/rwth/HM1/T105_Trigonometrische_Funktionen/g_art_content_18_grad_und_bogenmass.meta.xml}{content_18_grad_und_bogenmass}
\link{generic_article}{content/rwth/HM1/T211_Eigenschaften_stetiger_Funktionen/g_art_content_35_trigonom_funktionen.meta.xml}{content_35_trigonom_funktionen}
\link{generic_article}{content/rwth/HM1/T105_Trigonometrische_Funktionen/g_art_content_17_trigonometrie_im_dreieck.meta.xml}{content_17_trigonometrie_im_dreieck}
\end{links}
  \creategeneric
\end{metainfo}
\begin{content}
\begin{block}[annotation]
	Im Ticket-System: \href{https://team.mumie.net/issues/21563}{Ticket 21563}
\end{block}
\begin{block}[annotation]
Copy of \href{https://team.mumie.net/issues/20694}{Ticket 20694}: content/rwth/HM1/T203_komplexe_Zahlen/art_content_02_komplexeZahlen_geom.src.tex
\end{block}

\usepackage{mumie.ombplus}

\lang{de}{\title{Geometrische Anschauung}}
\lang{en}{\title{Geometric interpretation}}

\ombchapter{3}
\ombarticle{2}

\begin{block}[info-box]
\tableofcontents
\end{block}


\usepackage{mumie.genericvisualization}

\begin{visualizationwrapper}

\section{
\lang{de}{Schreibweise als Zahlenpaare}
\lang{en}{Complex numbers as ordered pairs}
}\label{sec:zahlenpaare}
\lang{de}{
Die komplexen Zahlen $z\in\C$ bilden mit ihrem Realteil und ihrem Imaginärteil die Punkte der
Gaußschen Zahlenebene. Die Ebene wird aber genauso durch den $\R^2$ beschrieben.
Wir können komplexe Zahlen also als Vektoren auffassen: Der Realteil
wird entlang der x-Achse aufgetragen, der Imaginärteil entlang der y-Achse.
\\
Somit kann man sie auch in der Tupel-Schreibweise darstellen:
\[z=a+bi\hat{=}(a;b)\]
Damit entspricht auch die Addition der komplexen Zahlen der Vektoraddition:
\[(a+bi) \ + \ (c+di) = (a+c) + (b+d)i\]
wird zu
\[(a,b) + (c,d) = (a+c, b+d).\]
}
\lang{en}{
The complex numbers $z\in\C$ with their real and imaginary parts make up
the points of the complex plane. However, we can describe the plane equally well as $\R^2$.
Hence, a complex number can be represented as a vector: its real part is plotted
on the x-axis and its imaginary part on the y-axis. \\
This means that complex numbers can be represented in tuple notation:
\[z=a+bi\hat{=}(a;b)\]
Addition of complex numbers corresponds to addition of vectors:
\[(a+bi) \ + \ (c+di) = (a+c) + (b+d)i\]
becomes
\[(a,b) + (c,d) = (a+c, b+d).\]
}
\begin{center}
\image{T203_ComplexSum}
\end{center}

\begin{example}
    \lang{de}{Im folgenden interaktiven Beispiel können Sie sich die Addition zweier komplexer Zahlen
     in der Ebene veranschaulichen. Sie können die Zahlen $z$ und $w$ in der Grafik verändern oder 
     auch darunter eingeben.
	Es werden dann die entsprechende komplexe Zahl $z+w$ eingezeichnet und deren Komponenten berechnet.}
   \lang{en}{
    In the interactive example below, you can visualize the addition of two complex numbers
    in the plane as a so-called Argand diagram. The numbers $z$ and $w$ can be changed by dragging
    in the diagram or by entering them below.
    The corresponding complex number $z+w$ will also be drawn and its components calculated.
   }

	
	\begin{genericGWTVisualization}[600][700]{mathlet1}
      	\lang{de}{\title{Addition}}
      	\lang{en}{\title{Addition}}
	
		\begin{variables}
			\vector[editable]{v}{real}{2,1}
			\vector[editable]{w}{real}{-1,1}
			\vector{vw}{real}{var(v)[x]+var(w)[x],var(v)[y]+var(w)[y]}
			\affine{mw}{real}{var(v),var(w)}
			\affine{mz}{real}{var(w),var(v)}
		\end{variables}
	
		% COLOR:
		\color{v}{#0066CC}
		\color{w}{#0066CC}
		\color{mw}{#00CC00}
		\color{mz}{#00CC00}
		\color{vw}{#CC6600}
	
		%LABEL:
		\label{v}{@2d[$\textcolor{#0066CC}{z}$]}
		\label{w}{@2d[$\textcolor{#0066CC}{w}$]}
		\label{vw}{@2d[$\textcolor{#CC6600}{z+w}$]}
	
		%PLOT
		\begin{canvas}
			\plot[coordinateSystem]{v,w,mw,mz, vw}
		\end{canvas}
		
      	\lang{de}{\text{Zu den komplexen Zahlen $z = \var{v}[x]+\var{v}[y]i$ 
      	und $w = \var{w}[x]+\var{w}[y]i$ ist die Summe gegeben durch}}
       \lang{en}{\text{The sum of the complex numbers $z = \var{v}[x]+\var{v}[y]i$
       and $w = \var{w}[x]+\var{w}[y]i$ is}}
      	\lang{de}{\text{$z+ w = \var{vw}[x]+\var{vw}[y]i$.}}
         \lang{en}{\text{$z+ w = \var{vw}[x]+\var{vw}[y]i$.}}
	
	\end{genericGWTVisualization}
\end{example}

\lang{de}{\\Die Multiplikation der komplexen Zahlen besitzt keine offensichtliche Entsprechung für Vektoren, wir können aber natürlich 
trotzdem die Regeln für die Multiplikation übertragen: Mit $(a+bi)\cdot(c+di)=(ac-bd)+(bc+ad)i$ ergibt sich für die Multiplikation
\[(a,b) \cdot (c,d) =(ac-bd;bc+ad).\]
}
\lang{en}{\\ Multiplication of complex numbers does not have an
obvious equivalent for vectors. Of course, we can carry over the rule for multiplying complex numbers
to tuples: since $(a+bi)\cdot(c+di)=(ac-bd)+(bc+ad)i$, we have
\[(a,b) \cdot (c,d) =(ac-bd;bc+ad).\]
}

\section{
\lang{de}{Betrag}
\lang{en}{Absolute value}
}
\lang{de}{
Ebenso wie jeder Vektor eine Länge hat, nämlich seinen Betrag, ordnen wir auch jeder komplexen Zahl
ihre \glqq Länge\grqq zu:
}
\lang{en}{
Just as every vector has a length, namely its magnitude, we define
the "length" of any complex number:
}

\begin{definition}[
\lang{de}{Betrag}
\lang{en}{Absolute value}]\label{betragkz}\label{def:betragkz}

\lang{de}{Der \notion{(Absolut-)Betrag} einer komplexen Zahl $z=a+bi$ ist die (euklidische) Entfernung 
des Punktes in der Zahlenebene zum Ursprung. Mit dem Satz des Pythagoras ergibt sich
\[{|z|}=\sqrt{a^2+b^2}.\]
Für reelle Zahlen stimmt der komplexe Absolutbetrag mit dem reellen Absolutbetrag überein.
}
\lang{en}{
The \notion{absolute value} of a complex number $z=a+bi$ is the (Euclidean) distance
from the origin of the corresponding point in the complex plane.
Using the Pythagorean theorem, we obtain
\[{|z|}=\sqrt{a^2+b^2}.\]
For real numbers, the complex absolute value agrees with the real absolute value.
}
\end{definition}

\begin{example}
\begin{eqnarray*}
  |3+4i| &=& \sqrt{3^2+4^2}=\sqrt{25}=5, \\
  |2-i| &=& \sqrt{2^2+(-1)^2}=\sqrt{5}, \\
  |i|  &=&  \sqrt{0^2+1^2}=1, \\
  |-5| &=& \sqrt{(-5)^2+0^2}=\sqrt{5^2}=5.
\end{eqnarray*}
\end{example}


\begin{rule}\label{rule:betragsregeln}
\lang{de}{
Der Betrag hat die folgenden Eigenschaften für alle $z,w\in\C$:
}
\lang{en}{
The absolute value has the following properties for all $z,w\in \C$:
}
\begin{enumerate}
\item
\lang{de}{
$\ |z|\geq 0$ und es gilt $|z|=0 \Leftrightarrow z=0$,
}
\lang{en}{
$\ |z|\geq 0$ and $|z|=0 \Leftrightarrow z=0$,
}
\item
 $ \ $ \lang{de}{es gelten die Dreiecksungleichungen} \lang{en}{The triangle inequalities hold:}\\
 $ \ |z+w|\leq |z|+|w|\quad\text{ \lang{de}{und}\lang{en}{and} }\quad \big| {|z|-|w|}\big| \leq |z-w|$,
 \item
$ \ |z\cdot w|=|z|\cdot|w|$,
\item
$ \ |z|^2=z\cdot \bar{z}$.
\end{enumerate}
\end{rule}

\begin{proof*}

\begin{incremental}{0}
\step
\lang{de}{
Regel 1. Teil\\
Mit $z=a+bi$ ist $|z|=\sqrt{a^2+b^2}\geq0$ und \\
$|z|=0\Leftrightarrow a^2+b^2=0\Leftrightarrow a=b=0 
\Leftrightarrow z=0$ }
\lang{en}{
Part 1:\\
If $z=a+bi$, then $|z|=\sqrt{a^2+b^2}\geq0$ and \\
$|z|=0\Leftrightarrow a^2+b^2=0\Leftrightarrow a=b=0 
\Leftrightarrow z=0$ }
\step
\lang{de}{
Regel 2. Teil}
\lang{en}{
Part 2:}
\begin{enumerate}
\item[(i)]
\lang{de}{zu zeigen: $|z+w|\leq |z|+|w|$\\
Da Beträge nicht-negativ sind, sind Quadrieren und Wurzelziehen der Ungleichung 
Äquivalenzumformungen, denn bei positiven Zahlen gibt es genau eine positive Wurzel und das Zeichen $\leq$ ändert sich auch nicht
(weil Quadrieren und Wurzelziehen streng monoton steigende Funktionen sind). So können wir die Ungleichung direkt nachrechnen:}
\lang{en}{
We want to show that $|z+w| \leq |z|+|w|$.\\
Since absolute values are nonnegative, we can take squares and square roots
without changing the inequality: positive numbers have exactly one
positive root, and the sign $\leq$ also remains unchanged
(as the square and square root are strictly monotonically
increasing functions).
}
\begin{eqnarray*}
|z+w|^2&=&(z+w)\cdot(\bar{z+w})=(z+w)\cdot(\bar{z}+\bar{w})\\
&=&z\bar{z}+z\bar{w}+w\bar{z}+w\bar{w}\\
&=&|z|^2+z\bar{w}+\bar{z\bar{w}}+|w|^2\\
&=&|z|^2+2\Re(z\bar{w})+|w|^2\\
&\leq&|z|^2+2|z\bar{w}|+|w|^2\\
&=&|z|^2+2|z||w|+|w|^2\\
&=&(|z|+|w|)^2
\end{eqnarray*}
\item[(ii)]
\lang{de}{zu zeigen: $\big| {|z|-|w|}\big| \leq |z-w|$\\
Es gilt: }
\lang{en}{
We want to show that $\big| {|z|-|w|}\big| \leq |z-w|$.\\
We have:
}
\begin{eqnarray*}
|z|=|(z-w)+w|&\leq&|z-w|+|w|\\
 \Leftrightarrow \quad\quad\quad\quad |z|-|w|&\leq&|z-w|
\end{eqnarray*}
\lang{de}{Genauso wird mit vertauschten Rollen von $z$ und $w$ }
\lang{en}{Similarly, after swapping the roles of $z$ and $w$,}
\begin{eqnarray*}
|w|-|z|&\leq&|w-z|\\
\Leftrightarrow \quad -(|z|-|w|)&\leq&|z-w|\lang{en}{.}
\end{eqnarray*}
\lang{de}{gezeigt. Die beiden letzten Ungleichungen $|z|-|w|\leq|z-w|$ und $-(|z|-|w|)\leq|z-w|$
 lassen sich dann zusammenfassen als $\big| {|z|-|w|}\big| \leq |z-w|$.
 }
 \lang{en}{
The two inequalities $|z|-|w|\leq|z-w|$ and $-(|z|-|w|)\leq|z-w|$
can be combined to $\big| {|z|-|w|}\big| \leq |z-w|$.
 }
%\item[(iii)]
%Gleichheit gilt natürlich, wenn entweder $z=0$ oder $w=0$. Sind beide Zahlen aber von 0 verschieden,
%so folgt aus (i), dass $\Re(z\bar{w})=|z\bar{w}|$ gelten muss.\\
%Realteil und Betrag sind aber nur gleich, falls $z\bar{w}\in\R$, sogar $\in\R>0$, da der 
%Betrag nicht negativ ist. Also gilt:\\
%$\frac{z\cdot\bar{w}}{w\cdot\bar{w}}>0\iff\frac{z}{w}>0\iff 
%\frac{z}{w}=\frac{1}{\lambda}$ mit $\lambda\in\R>0 \iff w=\lambda z$ 
\end{enumerate}
\step
\lang{de}{Regel 3. Teil\\
Mit $z=a+bi$ und $w=c+di$ ist }
\lang{en}{
Part 3. \\
If $z=a+bi$ and $w=c+di$, then
}
\begin{eqnarray*}
|z\cdot w|&=&|(a+bi)(c+di)|=\big|(ac-bd)+(ad+bc)i\big|\\
&=&\sqrt{(ac-bd)^2+(ad+bc)^2}\\
&=&\sqrt{(ac)^2-2abcd+(bd)^2+(ad)^2+2abcd+(bc)^2}\\
&=&\sqrt{a^2c^2+b^2d^2+a^2d^2+b^2c^2}\\
&=&\sqrt{a^2c^2+a^2d^2+b^2c^2+b^2d^2}\\
&=&\sqrt{(a^2+b^2)\cdot (c^2+d^2)}\\
&=&|z|\cdot|w|.
\end{eqnarray*}
\step
\lang{de}{Regel 4. Teil\\
Und mit $z=a+bi$ und $\bar{z}=a-bi$ ist}
\lang{en}{
Part 4. \\
If $z=a+bi$ and $\bar{z}=a-bi$, then
}
\begin{eqnarray*}
z\cdot \bar{z}&=&(a+bi)(a-bi)\\
&=&a^2-(bi)^2\\
&=&a^2+b^2\\
&=&|z|^2.
\end{eqnarray*}
\end{incremental}
\end{proof*}

\lang{de}{
Das folgende Video erläutert den bisherigen Sachverhalt zum \ref[content_08bneu_komplexeZahlen_geom][Betrag]{betragkz} komplexer Zahlen anhand kurzer Beispiele. 
Beachten Sie das die imaginäre Einheit $i$ im Video durch $j$ repräsentiert wird:
%\begin{center}
%\href{https://www.hm-kompakt.de/video?watch=205}{\image[75]{00_Videobutton_schwarz}}
%\end{center}
\floatright{\href{https://www.hm-kompakt.de/video?watch=205}{\image[75]{00_Videobutton_schwarz}}}
}
\\
\\


\begin{quickcheck}
\field{rational}
\type{input.function}
\begin{variables}
    \randint{a}{1}{3}
    \randint[Z]{b}{4}{6}
    \function[expand, normalize]{z}{a+b*i}
    \function[expand, normalize]{zquer}{a-b*i}
    \function[expand, normalize]{zabs2}{a^2+b^2}
    \function[expand, normalize]{zabs2_}{a^2+b^2}
    \function[expand, normalize]{zinv}{(a-b*i)/(a^2+b^2)}
\end{variables}
\lang{de}{
\text{Betrachten Sie die Zahl $z=\var{z}$ und bestimmen Sie:\\
$\Re(z)=$\ansref, $\Im(z)=$\ansref, $\bar{z}$=\ansref. \\
Mit Hilfe der 3. binomischen Formel 
können Sie zügig $z\cdot\bar{z}=$\ansref berechnen.\\
Berechnen Sie nun mit dem Satz des Pythagoras das Quadrat des Betrages: $|z|^2=$\ansref.\\
Berechnen Sie als Letztes: $z^{-1}=$\ansref.
}
}
\lang{en}{
\text{Consider the number $z=\var{z}$. Determine
$\Re(z)=$\ansref, $\Im(z)=$\ansref, $\bar{z}$=\ansref. \\
Use the 3rd binomial formula to quickly find $z\cdot\bar{z}=$\ansref.\\
Then use the Pythagorean theorem to determine the square of the absolute value: $|z|^2=$\ansref.\\
Finally, determine $z^{-1}=$\ansref.
}
}
\begin{answer}
    \solution{a}
\end{answer}
\begin{answer}
    \solution{b}
\end{answer}
\begin{answer}
    \solution{zquer}
\end{answer}
\begin{answer}
    \solution{zabs2}
\end{answer}
\begin{answer}
    \solution{zabs2_}
\end{answer}
\begin{answer}
    \solution{zinv}
\end{answer}
\end{quickcheck}

\lang{de}{
Die folgenden Eigenschaften sind wichtig, wenn man allgemein mit komplexen Zahlen rechnet, bei denen 
man Real- und Imaginärteil nicht explizit angeben kann oder will, zum Beispiel wenn man mit Variablen rechnet, die komplexe Werte annehmen können.
}
\lang{en}{
The following properties are useful for working with complex numbers
for which the real and imaginary parts are not given explicitly.
For example, this includes working with variables that can take 
complex values.
}


\begin{rule}
\lang{de}{
Für komplexe Zahlen $z,w\in \C$ gelten:
}
\lang{en}{
For complex numbers $z,w\in \C$,
}
\begin{enumerate}
\item $ \ \Re (\bar{z})=\Re(z) \ $ \lang{de}{und} \lang{en}{and} $ \ \Im(\bar{z})= - \Im(z)$,
\item $ \ \Re(z)=\frac{1}{2}(z+\bar{z}) \ $ \lang{de}{und} \lang{en}{and} $ \ \Im(z)=\frac{1}{2i}(z-\bar{z})$,
\item $ \ \bar{z+w}=\bar{z}+\bar{w} \ $ \lang{de}{und} \lang{en}{and} $ \ \bar{z\cdot w}=\bar{z}\cdot \bar{w}$, \lang{de}{sowie} \lang{en}{and} $\overline{\bar{z}}=z \ $ \lang{de}{und} \lang{en}{and} 
$ \ \bar{z}^{-1}=\bar{z^{-1}}$ \lang{de}{für} \lang{en}{if} $z\neq 0$,
\item $ \ z=\bar{z} \ \Leftrightarrow \ \Im(z)=0 \ \Leftrightarrow \ z\in \R$,
\item $ \ z\cdot \bar{z}\in \R_{\geq 0}$.
\end{enumerate}
\end{rule}

\begin{proof*}
\lang{de}{
Die aufgeführten Eigenschaften lassen sich mit Hilfe der Definitionen leicht nachrechnen.
}
\lang{en}{
The properties listed above are easy to check using the definitions.
}
\begin{incremental}{0}
\step
\lang{de}{Für $z=x+yi$ mit $x,y\in \R$ ist ja $\bar{z}=x-yi=x+(-y)i$ und daher
\[ \Re(\bar{z})=x=\Re(z), \quad \text{sowie}\quad \Im(\bar{z})= -y= - \Im(z).\]
}
\lang{en}{
If $z=x+iy$ with $x,y\in \R$, then $\bar{z}=x-iy=x+i(-y)$ and therefore
\[ \Re(\bar{z})=x=\Re(z) \quad \text{and}\quad \Im(\bar{z})= -y= - \Im(z).\]
}

\step
\lang{de}{
Weiter ist:}
\lang{en}{Also,}
\begin{eqnarray*} \frac{1}{2}(z+\bar{z}) &=& \frac{1}{2}(x+yi+x-yi)=\frac{1}{2}(2x)=x=\Re(z) \quad
\text{\lang{de}{und}\lang{en}{and}} \\
\frac{1}{2i}(z-\bar{z}) &=& \frac{1}{2i}\big(x+yi-(x-yi)\big)=\frac{1}{2i}(2yi)=y=\Im(z).
\end{eqnarray*}
\lang{de}{Ähnlich erhält man auch die anderen Eigenschaften.}
\lang{en}{The other properties are proved similarly.}

\end{incremental}
\end{proof*}

\lang{de}{
Die bisher angesprochenen Themen werden im folgenden Video nochmal erläutert:
\floatright{\href{https://api.stream24.net/vod/getVideo.php?id=10962-2-10824&mode=iframe&speed=true}{\image[75]{00_video_button_schwarz-blau}}}\\
}

\section{
\lang{de}{Polarkoordinaten}
\lang{en}{Polar coordinates}}\label{sec:polar}

\lang{de}{
Ein Punkt $z=a+bi\in\C\setminus\{0\}$ der komplexen Ebene wird eindeutig festgelegt durch seine Länge $|z|=r$ zusammen mit
seinem Winkel $\phi\in (- \pi;\pi]$. Der Winkel $\phi$ heißt das \notion{Argument} $\phi=\arg(z)$ von $z$. 
$\phi$ wird im \ref[content_18_grad_und_bogenmass][Bogenmaß]{winkelmasse} angegeben!
}

\lang{en}{
Points $z=a+bi\in\C\setminus\{0\}$ in the complex plane are uniquely
determined by their magnitude $|z|=r$ and their
angle $\phi \in (-\pi,\pi]$. The angle $\phi$ is called the \notion{argument} $\phi=\arg(z)$ of $z$.
$\phi$ is measured in \ref[content_18_grad_und_bogenmass][radians]{winkelmasse}!
}

\begin{figure}
\image{T203_PolarCoordinates}
\caption{
  \lang{de}{Komplexe Zahlen als Vektoren in der Gaußschen
  Zahlenebene}
  \lang{en}{Complex numbers as vectors in the complex plane}
  \label{fig:Komplexe-Zahlen-als-Zeiger}}
\end{figure}

\lang{de}{
Aus den Definitionen der \ref[content_17_trigonometrie_im_dreieck][trigonometrischen Funktionen]{sinus} folgt, dass 
\[ \cos(\phi)=\frac{a}{r} \ \text{  und  } \sin(\phi)=\frac{b}{r} \ \text{  und  } \tan(\phi)=\frac{b}{a}.\]
}
\lang{en}{
By definition of the \ref[content_17_trigonometrie_im_dreieck][trigonometric functions]{sinus},
\[ \cos(\phi)=\frac{a}{r} \ \text{  and  } \sin(\phi)=\frac{b}{r} \ \text{  and  } \tan(\phi)=\frac{b}{a}.\]
}


\begin{example}
\lang{de}{Es sei $z=i$. Dann ist die Länge $r=|z|=\sqrt{0^2+1^2}=1$ und das Argument $\arg(z)=\phi=\frac{\pi}{2}$.
}
\lang{en}{
For $z=i$, the magnitude is $r=|z|=\sqrt{0^2+1^2}=1$ and the
argument is $\arg(z)=\phi=\frac{\pi}{2}$.
}
\end{example}

\begin{example}
\lang{de}{
Gesucht sind die Zahlen $z\in\C$, die zum Koordinatenursprung einen festen Abstand $r>0$ haben: $|z|=r$. Das sind alle Zahlen 
$z$ auf der Kreislinie um den Ursprung mit dem Radius $r$. (Für das Argument $\phi$ ergibt sich keine Einschränkung.)
}
\lang{en}{
Suppose we want to find the numbers $z\in\C$ that are at a fixed distance $r>0$
from the origin: $|z|=r$. These are the numbers on the circle of radius $r$ centered
at the origin. (There is no restriction on the argument $\phi$.)
}
\begin{enumerate}
\item \lang{de}{Schwarzer Kreis: $|z|=r=3$.} \lang{en}{Black circle: $|z|=r=3$.}
\item \lang{de}{\textcolor{#0066CC}{Blauer Kreis mit Verschiebung des Kreismittelpunktes: $|z-(2+i)|=r=3$.}}
\lang{en}{\textcolor{#0066CC}{Blue circle with its center shifted: $|z-(2+i)|=r=3$.}}
\end{enumerate}
\begin{center}
\image{T203_Circle}
\end{center}
\end{example}

\lang{de}{
Der Vektor, der zur komplexen Zahl $z=a+bi\neq 0$ gehört, schließt mit der positiven reellen
Achse einen Winkel $\varphi$ ein, den wir \notion{Argument} genannt haben und dessen Wert  
\begin{equation*}
  -\pi < \varphi=\arg(z)  \le \pi \label{eq:hauptwertebereich}
\end{equation*}
erfüllt. Das Argument der komplexen Zahl $0$ ist nicht definiert, denn jeder Winkel $\varphi$ wäre möglich.
}
\lang{en}{
The vector that corresponds to the complex number $z=a+ib\neq 0$
and the positive real axis form an angle $\phi$ that we called the
\notion{argument}. Its value satisfies
\begin{equation*}
  -\pi < \varphi=\arg(z)  \le \pi. \label{eq:hauptwertebereich}
\end{equation*}
The argument of the complex $0$ is undefined because every angle $\phi$ would work.
}

\lang{de}{
Mit dieser Definition des Arguments hat man die Wahl getroffen, den \notion{Phasenwinkel} 
(möglichst) symmetrisch zur reellen Achse zu definieren. 
 Dies ist die   meistens in der Technik getroffene Wahl, ebenso wie in der Mathematik, 
 wo sie z.\,B. bei der Definition des komplexen Logarithmus verwendet wird. 
 Grundsätzlich sind aber andere Phasenbereiche möglich und gebräuchlich, beispielsweise $[0;2\pi)$.
 }
\lang{en}{
This definition of the argument is chosen such that the
\notion{phase angle} is as symmetric as possible with respect
to the real axis. This is the most common convention in engineering.
It is also standard in mathematics, where it appears
in the definition of the complex logarithm (for example).
Other phase ranges, such as $[0,2\pi)$, can be and are used as well, however.
}
 
 %Kathrin Das Folgende finde ich unglücklich, denn eigentlich müsste man erstmal die Quadranten ordentlich definieren. 
 %Und genau auf der reellen Halbachse, die man dabei zuordnen muss, ist der Sprung unlogisch verteilt...
% Ich weiß nicht, weshalb Volker Bach die Quadraten hier gebracht hat, ohne weiter daruaf einzugehen, 
%was dort für die Vorzeichen von Real- und Imaginärteil gilt. da fände ich es sinnvoller. 
% Ich würde sogar dafür pädieren, die Quadranten ganz raus zu lassen -- nur habe ich in meinem Intensiv-Kurs gemerkt, 
%dass die Teilnehmer mit dem Begriff Quadrant nichts anfangen konnten. Ich habe den in der siebten Klass gelernt.
 \lang{de}{Im ersten und zweiten Quadranten ist der Phasenwinkel dabei positiv (oder gleich $0$),
d.\,h. $0 \le \varphi \le \pi$. 
Im dritten und vierten Quadranten ist er negativ, also $-\pi < \varphi < 0$.
}
\lang{en}{
In the first and second quadrants, the phase angle is positive (or $0$),
i.e. $0 \le \phi \le \pi$.
It is negative in the third and fourth quadrants, i.e. $-\pi < \phi < 0$.
}
\begin{figure}
  \image{T203_ComplexQuadrants}
  \caption{
  \label{fig:Hauptwertebereich-des-Phasenwink}}
\end{figure}

\lang{de}{
Ist die komplexe Zahl $z=a+bi$ gegeben, erhält man das Argument folgendermaßen:
}
\lang{en}{
For a given complex number $z=a+bi$, the argument is obtained as follows.
}

\begin{rule}
\lang{de}{
Für eine komplexe Zahl $z=(a;b)=a+bi\in \C\setminus \{0\}$ mit $a,b\in \R$ ist
das Argument $\varphi = \arg(z)$ diejenige reelle Zahl $\varphi\in (-\pi;\pi]$, für welche
\[   \cos(\varphi)= \frac{a}{\sqrt{a^2+b^2}} \quad \text{und}\quad \sin(\varphi)= \frac{b}{\sqrt{a^2+b^2}} \]
gilt.
}
\lang{en}{
Let $z=(a,b)=a+bi \in \C\setminus\{0\}$ with $a,b \in \R$.
The argument $\phi=\arg(z)$ is the real number $\phi\in (-\pi,\pi]$
that satisfies
\[   \cos(\varphi)= \frac{a}{\sqrt{a^2+b^2}} \quad \text{and}\quad \sin(\varphi)= \frac{b}{\sqrt{a^2+b^2}}. \]
}

\lang{de}{
Die komplexe Zahl $z=a+bi$ lässt sich dann auch in der sogenannten \notion{Polardarstellung} schreiben als
\[  z=r\left(\cos(\varphi)+i\sin(\varphi)\right), \]
wobei $r=|z|=\sqrt{a^2+b^2}$ der Betrag von $z$ ist.
}
\lang{en}{
The complex number $z=a+bi$ can then be written in its
\notion{polar representation},
\[  z=r\left(\cos(\phi)+i\sin(\phi)\right), \]
where $r=|z|=\sqrt{a^2+b^2}$ is the absolute value of $z$.
}
\end{rule}  

\begin{remark}
\begin{enumerate}
\item[i)]
\lang{de}{
Das Argument $\varphi$ ist bis auf das Vorzeichen sogar schon durch $\cos(\varphi)= \frac{a}{\sqrt{a^2+b^2}}$
bestimmt. Falls $b\neq 0$, so ist das Vorzeichen von $\varphi$ gleich dem Vorzeichen von $b$.\\
}
\lang{en}{
Up to sign, the argument $\phi$ is already determined by
$\cos(\phi)= \frac{a}{\sqrt{a^2+b^2}}$.
If $b \ne 0$, then the sign of $\phi$ equals the sign of $b$.
}
\lang{de}{
Für $b=0$ (und $a\neq 0)$ ist die Zahl reell und daher $\varphi=0$, falls $a>0,$ bzw.
$\varphi=\pi$, falls $a<0$. Letzteres erhält man auch aus 
\[ \cos(\varphi)= \frac{a}{|a|}= \begin{cases} 1, & \text{ falls }a>0, \\ -1, & \text{ falls }a<0,\end{cases} \]
und der Einschränkung von $\varphi$ auf den Bereich $(-\pi;\pi]$.
}
\lang{en}{
If $b=0$ (and $a\neq 0$), then the number is real. Hence, $\phi=0$
if $a>0$ and $\phi=\pi$ if $a<0$. This also follows from
\[ \cos(\phi)= \frac{a}{|a|}= \begin{cases} 1, & \text{ if }a>0, \\ -1, & \text{ if }a<0,\end{cases} \]
and the fact that $\phi$ is restricted to the range $(-\pi,\pi]$.
}
\item[ii)]
\lang{de}{
Eine oft genutzte Möglichkeit, bequeme $\phi=$arg(z) zu berechnen, fußt auf der Gleichung:
\[\tan\phi=\frac{b}{a}\] 
}
\lang{en}{
One commonly used method to compute $\phi=\arg(z)$ conveniently
is based on the equation \[\tan\phi=\frac{b}{a}.\] 
}
\lang{de}{
Die \ref[content_35_trigonom_funktionen][Arkusfunktionen]{arcus.funktionen} tauchen erst in einem späteren Kapitel auf. Es ist hier ausreichend zu wissen, dass
$\arctan(x)\in\left(-\frac{\pi}{2};+\frac{\pi}{2}\right)$. Da aber $\phi\in(-\pi;\pi]$, muss je nach Quadrant
der Winkel "von Hand" berechnet werden zu:\\
$\phi=\arg(z)=\begin{cases}
\arctan\frac{b}{a},& a>0\\
\arctan\frac{b}{a}-\pi,&a<0, b<0\\
\arctan\frac{b}{a}+\pi,&a<0,b>0\\
\frac{\pi}{2},&a=0, b>0\\
-\frac{\pi}{2},&a=0,b<0.
\end{cases}$
}
\lang{en}{
The \ref[content_35_trigonom_funktionen][inverse trigonometric functions]{arcus.funktionen}
will be introduced in a later chapter. Here, it is enough to know that
$\arctan(x)\in\left(-\frac{\pi}{2},+\frac{\pi}{2}\right)$. However,
since $\phi\in(-\pi,\pi]$, the angle must be computed "by hand"
depending on the quadrant as follows:\\
$\phi=\arg(z)=\begin{cases}
\arctan\frac{b}{a},& a>0\\
\arctan\frac{b}{a}-\pi,&a<0, b<0\\
\arctan\frac{b}{a}+\pi,&a<0,b>0\\
\frac{\pi}{2},&a=0, b>0\\
-\frac{\pi}{2},&a=0,b<0.
\end{cases}$
}
\end{enumerate}
\end{remark}


\begin{example}
	\begin{genericGWTVisualization}[550][800]{mathlet1}
		\begin{variables}
			\point{A}{rational}{0,0}
			\point[editable]{z}{rational}{2,3}
%			\point{C}{rational}{1,0}
			\number{s}{rational}{1+sign(var(z)[y])-sign(var(z)[y])^2}  %1 für var(z)[y]>=0, -1 sonst.
			\segment{oz}{rational}{var(A),var(z)}
			\number{betz2}{rational}{(var(z)[x])^2+(var(z)[y])^2}
			\number{betz}{real}{sqrt(var(betz2))}
			\number{sqbetz2}{operation}{sqrt((var(z)[x])^2+(var(z)[y])^2)}
			\number{argu}{real}{var(s)*arccos(var(z)[x]/var(betz))}
			\number{ug}{real}{(var(argu)-abs(var(argu)) )/2}   % min(var(argu),0)
			\number{og}{real}{(var(argu)+abs(var(argu)) )/2}   % max(var(argu),0)
			\angle{win}{real}{var(A), 0.7, var(ug), var(og)}
			%\number{al}{real}{(arcsin(var(p2))-arcsin(var(p1)))*180/pi}
			
		\end{variables}


		\color{A}{#0066CC}
%		\label{A}{$\textcolor{BLUE}{A}$}
		\color{z}{#CC6600}
		\label{z}{$\textcolor{#CC6600}{z}$}
		\color{oz}{#CC6600}
		\label{oz}{$\textcolor{#CC6600}{|z|=\var{sqbetz2}}$}
		\label{win}{$\phantom{xxx}\varphi$}

		\begin{canvas}
			\updateOnDrag[false]
			\plotSize{400}
			\plotLeft{-4}
			\plotRight{4}
			\plot[coordinateSystem]{A,z,oz,win}
		\end{canvas}
    \lang{de}{
		\text{Für $z=\var{z}[x]+\var{z}[y]i$ ist $|z|=\var{sqbetz2}
		\approx 
		\var{betz}$ und $\varphi=\arg(z)\approx \var{argu}$.}
    }
    \lang{en}{
    \text{If $z=\var{z}[x]+i\var{z}[y]$, then 
    $|z|=\var{sqbetz2}
		\approx 
		\var{betz}$
    and  $\phi=\arg(z)\approx \var{argu}$.}
    }
	    	\end{genericGWTVisualization}
\end{example}            
\end{visualizationwrapper}

\begin{quickcheck}
\field{real}
\type{input.function}
    \begin{variables}
         \function{r}{1}
         \function{phi}{pi}
         \function{wurz}{(-1)^2+0^2}
    \end{variables}
    \lang{de}{
    \text{Stellen Sie die Zahl $z=-1$ in Polarkoordinaten dar:\\
    $\phi=$\ansref und $r=$\ansref
    }
    }
    \lang{en}{
    Express the number $z=-1$ in polar coordinates:\\
    $\phi=$\ansref and $r=$\ansref
    }
    \begin{answer}
        \solution{phi}
    \end{answer}
    \begin{answer}
        \solution{r}
    \end{answer}
    \lang{de}{
    \explanation{$r=\sqrt{\var{wurz}}=1$ und $\phi=\pi$, da $-\pi$ nicht im 
    Wertebereich der Winkel liegt.}   
    }
    \lang{en}{
    \explanation{$r=\sqrt{\var{wurz}}=1$ and $\phi=\pi$, because
    $-\pi$ does not lie in the range of the argument.}
    }
\end{quickcheck}

\begin{quickcheck}
\field{real}
\type{input.number}
    \displayprecision{3}
    \begin{variables}
         \randint{a}{1}{4}
         \randint{b}{1}{4}
         \function[expand,normalize]{z}{a+b*i}
         \function[calculate,3]{r}{sqrt(a^2+b^2)}
         \function[calculate,3]{phi}{arctan(b/a)}
         \function{wurz}{a^2+b^2}
    \end{variables}
    \lang{de}{
    \text{Stellen Sie die Zahl $z=\var{z}$ in Polarkoordinaten dar. Runden Sie bitte 
    auf 3 Nachkommastellen (eine 5 muss aufgerundet werden).\\
    $\phi=$\ansref und $r=$\ansref
    }
    }
    \lang{en}{
    \text{Express the number $z=\var{z}$ in polar coordinates.
    Round your answer to three decimal places. (5 should be rounded up.)\\
    $\phi=$\ansref and $r=$\ansref
    }
    }
    \begin{answer}
        \solution{phi}
    \end{answer}
    \begin{answer}
        \solution{r}
    \end{answer}
\explanation{$r=\sqrt{\var{wurz}}$ und $\tan(\phi)=\frac{\var{b}}{\var{a}}$.
    }    
\end{quickcheck}

\lang{de}{
Polarkoordinaten sind im folgenden Video nochmals zusammengefasst:
\floatright{\href{https://api.stream24.net/vod/getVideo.php?id=10962-2-10825&mode=iframe&speed=true}{\image[75]{00_video_button_schwarz-blau}}}\\
}
\end{content}

