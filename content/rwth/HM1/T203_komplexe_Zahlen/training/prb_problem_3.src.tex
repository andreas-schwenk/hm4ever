\documentclass{mumie.problem.gwtmathlet}
%$Id$
\begin{metainfo}
  \name{
    \lang{de}{A03: Polarkoordinaten}
    \lang{en}{Problem 3}
  }
  \begin{description} 
 This work is licensed under the Creative Commons License Attribution 4.0 International (CC-BY 4.0)   
 https://creativecommons.org/licenses/by/4.0/legalcode 

    \lang{de}{Beschreibung}
    \lang{en}{}
  \end{description}
  \corrector{system/problem/GenericCorrector.meta.xml}
  \begin{components}
    \component{js_lib}{system/problem/GenericMathlet.meta.xml}{mathlet}
  \end{components}
  \begin{links}
  \end{links}
  \creategeneric
\end{metainfo}
\begin{content}
\usepackage{mumie.genericproblem}
\lang{de}{\title{A03: Polarkoordinaten}}
\lang{en}{\title{Problem 3}}
\begin{block}[annotation]
  Im Ticket-System: \href{http://team.mumie.net/issues/9753}{Ticket 9753}
\end{block}

\begin{problem}
  \begin{question}
    \lang{de}{\text{
    Bestimmen Sie den Betrag und das Argument (den Phasenwinkel) von \\
    $\var{z}$. Runden Sie auf 2 Nachkommastellen.
    }}
    \lang{de}{\explanation{
    Der Betrag berechnet sich aus der Wurzel der Summe
    der Quadrate von Real- und Imaginärteil. \\
    Der Phasenwinkel berechnet sich aus dem Arkustangens
    des Quotienten von Imaginärteil und Realteil.
    }}
    \type{input.number}
    \field{real}
    \begin{variables}
      \randint{a}{1}{10} % Erster und vierter Quadrant
      \randint{b}{-10}{10}
      \function{z}{a + b*i}
      \function[calculate]{r}{sqrt(a^2 + b^2)}
      \function[calculate]{ph}{atan(b/a)}
    \end{variables}

    \begin{answer}
      \text{Betrag:}
      \solution{r}
    \end{answer}

    \begin{answer}
      \text{Argument:}
      \solution{ph}
    \end{answer}

  \end{question}

  \begin{question}
    \lang{de}{\text{
    Bestimmen Sie den Betrag und das Argument (den Phasenwinkel) von \\
    $\var{z}$. Runden Sie auf 2 Nachkommastellen.
    }}
    \lang{de}{\explanation{
    Der Betrag berechnet sich aus der Wurzel der Summe
    der Quadrate von Real- und Imaginärteil. \\
    Der Phasenwinkel berechnet sich aus dem Arkustangens
    des Quotienten von Imaginärteil und Realteil. \\
    Da die Zahl im zweiten Quadranten liegt
    (negativer Realteil, positiver Imaginärteil),
    muss zum Phasenwinkel noch $\pi$ addiert werden.
    }}
    \type{input.number}
    \field{real}
    \begin{variables}
      \randint{a}{-10}{-1} % Zweiter Quadrant
      \randint{b}{1}{10}
      \function{z}{a + b*i}
      \function[calculate]{r}{sqrt(a^2 + b^2)}
      \function[calculate]{ph}{atan(b/a) + pi}
    \end{variables}

    \begin{answer}
      \text{Betrag:}
      \solution{r}
    \end{answer}

    \begin{answer}
      \text{Argument:}
      \solution{ph}
    \end{answer}

  \end{question}

  \begin{question}
    \lang{de}{\text{
    Bestimmen Sie den Betrag und das Argument (den Phasenwinkel) von \\
    $\var{z}$. Runden Sie auf 2 Nachkommastellen.
    }}
    \lang{de}{\explanation{
    Der Betrag berechnet sich aus der Wurzel der Summe
    der Quadrate von Real- und Imaginärteil. \\
    Der Phasenwinkel berechnet sich aus dem Arkustangens
    des Quotienten von Imaginärteil und Realteil. \\
    Da die Zahl im dritten Quadranten liegt
    (negativer Realteil, negativer Imaginärteil),
    muss vom Phasenwinkel noch $\pi$ subtrahiert werden.
    }}
    \type{input.number}
    \field{real}
    \begin{variables}
      \randint{a}{-10}{-1} % Dritter Quadrant
      \randint{b}{-10}{-1}
      \function{z}{a + b*i}
      \function[calculate]{r}{sqrt(a^2 + b^2)}
      \function[calculate]{ph}{atan(b/a) - pi}
    \end{variables}

    \begin{answer}
      \text{Betrag:}
      \solution{r}
    \end{answer}

    \begin{answer}
      \text{Argument:}
      \solution{ph}
    \end{answer}

  \end{question}

\end{problem}

\embedmathlet{mathlet}
\end{content}