\documentclass{mumie.problem.gwtmathlet}
%$Id$
\begin{metainfo}
  \name{
    \lang{de}{A01: Aussagen}
    \lang{en}{P01: Statements}
  }
  \begin{description} 
 This work is licensed under the Creative Commons License Attribution 4.0 International (CC-BY 4.0)   
 https://creativecommons.org/licenses/by/4.0/legalcode 

    \lang{de}{Beschreibung}
    \lang{en}{}
  \end{description}
  \corrector{system/problem/GenericCorrector.meta.xml}
  \begin{components}
    \component{js_lib}{system/problem/GenericMathlet.meta.xml}{mathlet}
  \end{components}
  \begin{links}
  \end{links}
  \creategeneric
\end{metainfo}
\begin{content}
\usepackage{mumie.genericproblem}
\lang{de}{\title{A01: Aussagen}}
\lang{en}{\title{P01: Statements}}
\begin{block}[annotation]
  Im Ticket-System: \href{http://team.mumie.net/issues/9751}{Ticket 9751}
\end{block}

\begin{problem}
  \begin{question}
    \lang{de}{\text{
    Kreuzen Sie alle wahren Aussagen an.
    }}
    \lang{en}{\text{
    Tick all the true statements.
    }}
    \lang{de}{\explanation{i ist die imaginäre Einheit mit $i^2=-1$;\\
    Bsp.: $\sqrt{-9}=3i$;\\
    quadratische Gleichungen: Wurzeln aus komplexen Zahlen können gezogen werden.
    }}
    \lang{en}{\explanation{i is the imaginary unit with $i^2=-1$;\\
    Example: $\sqrt{-9}=3i$;\\
    Quadratic equations: Complex numbers can be extract.
    }}
    \type{mc.multiple}
    \permutechoices{1}{10}
    \begin{choice}
      \lang{de}{\text{
      Quadratische Gleichungen besitzen im Körper der komplexen Zahlen
      immer Lösungen.
      }}
      \lang{en}{\text{
      Quadratic equations over the field of the complex numbers always have solutions.
      }}
      \solution{true}
    \end{choice}

    \begin{choice}
      \lang{de}{\text{
      Aus negativen Zahlen kann man im Körper der komplexen Zahlen
      keine Wurzeln ziehen.
      }}
      \lang{en}{\text{
      Negative numbers over the field of the complex numbers cannot be extract.
      }}
      \solution{false}
    \end{choice}

    \begin{choice}
      \lang{de}{\text{
      Die Zahl ${i}^2$ ist eine komplexe Zahl.
      }}
      \lang{en}{\text{
      The number ${i}^2$ is a complex number.
      }}
      \solution{true}
    \end{choice}

    \begin{choice}
      \lang{de}{\text{
      Die Zahl ${i}^2$ ist die imaginäre Einheit.
      }}
      \lang{en}{\text{
      The number ${i}^2$ is the imaginary unit.
      }}
      \solution{false}
    \end{choice}

    \begin{choice}
      \lang{de}{\text{
      Die Zahl ${i}^2$ ist eine reelle Zahl.
      }}
      \lang{en}{\text{
      The number ${i}^2$ is a real number.
      }}
      \solution{true}
    \end{choice}

    \begin{choice}
      \lang{de}{\text{
      Die Zahl ${i}^2$ ist eine rein imaginäre Zahl.
      }}
      \lang{en}{\text{
      The number ${i}^2$ is a purely imaginary number.
      }}
      \solution{false}
    \end{choice}

    \begin{choice}
      \lang{de}{\text{
      Der Realteil von $4 \, {i} + 3$ ist $3$.
      }}
      \lang{en}{\text{
      The real part of $4 \, {i} + 3$ is $3$.
      }}
      \solution{true}
    \end{choice}

    \begin{choice}
      \lang{de}{\text{
      Der Imaginärteil von $666 + 11 \, {i}$ ist $11 \, {i}$.
      }}
      \lang{en}{\text{
      The imaginary part of $666 + 11 \, {i}$ is $11 \, {i}$.
      }}
      \solution{false}
    \end{choice}

    \begin{choice}
      \lang{de}{\text{
      Die Zahl $0 - 4 \, {i}$ ist eine rein imaginäre Zahl.
      }}
      \lang{en}{\text{
      The number $0 - 4 \, {i}$ is a purely imaginary number.
      }}
      \solution{true}
    \end{choice}

    \begin{choice}
      \lang{de}{\text{
      Die Zahl $42 + 0 \, {i}$ ist eine reelle Zahl.
      }}
      \lang{en}{\text{
      The number $42 + 0 \, {i}$ is a real number.
      }}
      \solution{true}
    \end{choice}

  \end{question}

\end{problem}

\embedmathlet{mathlet}
\end{content}