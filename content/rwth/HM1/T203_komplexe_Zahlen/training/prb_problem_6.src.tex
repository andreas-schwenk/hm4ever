\documentclass{mumie.problem.gwtmathlet}
%$Id$
\begin{metainfo}
  \name{
    \lang{de}{A06: Produkte }
    \lang{en}{Problem 6}
  }
  \begin{description} 
 This work is licensed under the Creative Commons License Attribution 4.0 International (CC-BY 4.0)   
 https://creativecommons.org/licenses/by/4.0/legalcode 

    \lang{de}{}
    \lang{en}{}
  \end{description}
  \corrector{system/problem/GenericCorrector.meta.xml}
  \begin{components}
    \component{js_lib}{system/problem/GenericMathlet.meta.xml}{mathlet}
  \end{components}
  \begin{links}
  \end{links}
  \creategeneric
\end{metainfo}
\begin{content}
\usepackage{mumie.genericproblem}
\lang{de}{\title{A06: Produkte}}
\lang{en}{\title{Problem 6}}
\begin{block}[annotation]
  Im Ticket-System: \href{http://team.mumie.net/issues/9756}{Ticket 9756}
\end{block}

\begin{problem}
  \begin{question}
    \lang{de}{\text{
    Berechnen Sie die folgende Potenz: \\
    $z = (\var{z})^2.$ 
    }}
    \lang{de}{\explanation{
    Das Quadrat einer komplexen Zahl kann man berechnen, indem man die Zahl mit sich selbst multipliziert: 
    $z=(\var{a})^2-(\var{b})^2+2\cdot(\var{a})\cdot(\var{b})i$}
    }
    
    \begin{tabs*}[\initialtab{0}]
  \tab{
    \lang{en}{...}
    \lang{de}{...}
  }
  \tab{
    \lang{en}{...}
    \lang{de}{...}
  }
\end{tabs*}

    
    \type{input.number}
    \field{complex}
    \begin{variables}
      \randint{a}{-10}{10}
      \randint{b}{-10}{10}
      \function{z}{a + b*i}
      \function[calculate]{a2}{a*a - b*b}
      \function[calculate]{b2}{2*a*b}
      \function{z2}{a2 + b2*i}
    \end{variables}

    \begin{answer}
      \text{$z =$}
      \solution{z2}
    \end{answer}

  \end{question} 

\begin{question}  
                \field{complex}
                %\displayprecision{1}
                %\correctorprecision{1}
                \type{input.function}
                \text{Berechnen Sie das Produkt der Zahlen $z=\var{z1}$ und $w=\var{w1}$.
                  \\ $z\cdot w=$\ansref}
                \begin{variables}
                        \randint{a}{1}{2}
                        \randint{b}{3}{7}
                        \function[calculate]{c}{a*b}
                        \function{z1}{a*(cos(pi/6)+i*sin(pi/6))}
                        \function{w1}{b*(cos(pi/6)+i*sin(pi/6))}
                        \function[expand,normalize]{sol}{c*(cos(2*pi/6)+i*sin(2*pi/6))}
                        \function[calculate]{f}{z1*w1}
                        \function{f2}{z1*w1}
               \end{variables}          
               \begin{answer}
                    \solution{sol}
                    \inputAsFunction{}{antwort}
                    \checkFuncForZero{sol-antwort}{1}{2}{1}
                    %\solution{f}
                    %\inputAsFunction{}{antwort}
                    %\checkStringsForRelation{count(i,antwort) <= 2}
                    %\checkFuncForZero[1E-2]{f-antwort}{1}{2}{1}
                    
               \end{answer}
               \explanation{Bei Polarkoordinaten addieren sich die Winkel 
               und die Beträge werden multipliziert.}
          \end{question}     

\end{problem}

\embedmathlet{mathlet}
\end{content}