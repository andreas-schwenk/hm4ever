\documentclass{mumie.problem.gwtmathlet}
%$Id$
\begin{metainfo}
  \name{
    \lang{de}{A02: komplex konjugiert}
    \lang{en}{P02: The complex conjugate}
  }
  \begin{description} 
 This work is licensed under the Creative Commons License Attribution 4.0 International (CC-BY 4.0)   
 https://creativecommons.org/licenses/by/4.0/legalcode 

    \lang{de}{Beschreibung}
    \lang{en}{}
  \end{description}
  \corrector{system/problem/GenericCorrector.meta.xml}
  \begin{components}
    \component{js_lib}{system/problem/GenericMathlet.meta.xml}{mathlet}
  \end{components}
  \begin{links}
  \end{links}
  \creategeneric
\end{metainfo}
\begin{content}
\usepackage{mumie.genericproblem}
\title{\lang{de}{A02: komplex konjugiert} \lang{en}{P02: The complex conjugate}}
\begin{block}[annotation]
  Im Ticket-System: \href{http://team.mumie.net/issues/9752}{Ticket 9752}
\end{block}

\begin{problem}
  \begin{question}
    \lang{de}{\text{
    Bestimmen Sie die konjugiert komplexe Zahl $\bar{z}$ zu \\
    $z = \var{z}.$}}
    \lang{en}{\text{
    Determine the complex conjugate $\bar{z}$ of \\
    $z = \var{z}.$}}
    \lang{de}{\explanation{
    Bei der konjugiert komplexen Zahl ändert sich 
    das Vorzeichen des Imaginärteils.
    }}
    \lang{en}{\explanation{The imaginary part of the complex conjugate has the opposite sign.
    }}
    \type{input.number}
    \field{complex}
    \begin{variables}
      \randint{a}{-10}{10}
      \randint{b}{-10}{10}
      \function{z}{a + b*i}
      \function{zquer}{a - b*i}
    \end{variables}

    \begin{answer}
      \text{$\bar{z} = $}
      \solution{zquer}
    \end{answer}

  \end{question}

  \begin{question}
    \lang{de}{\text{
    Bestimmen Sie den Realteil und den Imaginärteil von \\
    $\var{z}.$}}
    \lang{en}{\text{
    Determine the real part and the imaginary part of \\
    $\var{z}.$}}
    \lang{de}{\explanation{Der Realteil ist der Summand, der kein \glqq $i$\grqq besitzt. \\
    Der Imaginärteil ist der Faktor vor dem \glqq $i$\grqq. \\
    In beiden Fällen gehört das Vorzeichen dazu.
    }}
    \lang{en}{\explanation{
    The real part is the summand, that does not have an \glqq $i$\grqq. \\
    The imaginary part is the factor in front of the \glqq $i$\grqq. \\
    For both parts the sign belongs to it.
    }}
    \type{input.number}
    \field{complex}
    \begin{variables}
      \randint{a}{-10}{10}
      \randint{b}{-10}{10}
      \function{z}{b*i + a}
    \end{variables}

    \begin{answer}

      \solution{a}
    \end{answer}

    \begin{answer}
          \lang{de}{\text{Imaginärteil:}} 
    \lang{en}{\text{Imaginary part:}}
      \solution{b}
    \end{answer}

  \end{question}

\end{problem}

\embedmathlet{mathlet}
\end{content}